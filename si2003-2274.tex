\documentclass[12pt,a4paper]{article}

\newcommand\regstitle{The State Pension Credit (Transitional and Miscellaneous Provisions) Amendment Regulations 2003}

\newcommand\regsnumber{2003/2274}

%\opt{newrules}{
\title{\regstitle}
%}

%\opt{2012rules}{
%\title{Child Maintenance and Other Payments Act 2008\\(2012 scheme version)}
%}

\author{S.I.\ 2003 No.\ 2274}

\date{Made
7th September 2003\\
Laid before Parliament
12th September 2003\\
Coming into force
6th October 2003
}

%\opt{oldrules}{\newcommand\versionyear{1993}}
%\opt{newrules}{\newcommand\versionyear{2003}}
%\opt{2012rules}{\newcommand\versionyear{2012}}

\usepackage{csa-regs}

\setlength\headheight{27.61603pt}

\begin{document}

\maketitle

\noindent
The Secretary of State in exercise of the powers conferred by the sections 5(1)($h$), ($hh$)  and (3A), 189(1), (4)($a$)  and (6), and 191 of the Social Security Administration Act 1992\footnote{1992 c.\ 5, section 191 is cited for the definition of “prescribed”.}, sections 10(3) and (6), 79(1) and 84 of the Social Security Act 1998\footnote{1998 c.\ 14, section 84 is cited for the definition of “prescribed”.}, section 115(3) and (4) of the Immigration and Asylum Act 1999\footnote{1999 c.\ 33.} and sections 1(5)($b$), 2(6), 12(2) and (3), 13(1), 15(1)($e$)  and ($j$), 15(2), (6)($b$)  and ($d$), 17(1) and (2)($a$)  of the State Pension Credit Act 2002\footnote{2002 c.\ 16, section 17(1) is cited for the definition of “prescribed”.} and of all other powers enabling him in that behalf, and after agreement by the Social Security Advisory Committee that proposals in respect of these Regulations should not be referred to it\footnote{\emph{See} sections 170 and 173(1)($b$)  of the Social Security Administration Act 1992 (c.\ 5); paragraph 20 of Schedule 2 to the State Pension Credit Act 2002 added that Act to the list of “relevant enactments” in respect of which regulations must normally be referred to the Committee.}, hereby makes the following Regulations: 

{\sloppy

\tableofcontents

}

\bigskip

\setcounter{secnumdepth}{-2}

\subsection[1. Citation and commencement]{Citation and commencement}

1.  These Regulations may be cited as the State Pension Credit (Transitional and Miscellaneous Provisions) Amendment Regulations 2003 and shall come into force on 6th October 2003 immediately after the coming into force of the State Pension Credit (Consequential, Transitional and Miscellaneous Provisions) (No.\ 2) Regulations 2002\footnote{S.I.\ 2002/3197.}.

\subsection[2. Amendment of the State Pension Credit Regulations 2002]{Amendment of the State Pension Credit Regulations 2002}

2.---(1)  The State Pension Credit Regulations 2002\footnote{S.I.\ 2002/1792.} shall be amended in accordance with the following paragraphs.

(2) In regulation 1(2) (definitions) at the appropriate place in the alphabetical order insert—
\begin{quotation}
““adoption leave” means a period of absence from work on ordinary or additional adoption leave in accordance with section 75A or 75B of the Employment Rights Act 1996\footnote{1996 c.\ 18. Sections 75A and 75B were inserted by section 3 of the Employment Act 2002 (c.\ 22).};

“paternity leave” means a period of absence from work on leave in accordance with section 80A or 80B of the Employment Rights Act 1996\footnote{Sections 80A and 80B were inserted by section 1 of the Employment Act 2002.};”.
\end{quotation}

(3) In regulation 2 (persons not in Great Britain) after paragraph ($d$)  add—
\begin{quotation}
“($e$) a person in Great Britain who left the territory of Montserrat after 1st November 1995 because of the effect on that territory of a volcanic eruption.”.
\end{quotation}

(4) In regulation 3(1) (persons temporarily absent from Great Britain) for “person's”, substitute “claimant's”.

(5) In regulation 5 (persons treated as being or not being members of the same household)—
\begin{enumerate}\item[]
($a$) in paragraph (1)($c$)(i)  for the words after “under” until the end, substitute “the provisions of the Mental Health Act 1983\footnote{1983 c.\ 20.}, the Mental Health (Scotland) Act 1984\footnote{1984 c.\ 36.}, or the Criminal Procedure (Scotland) Act 1995\footnote{1995 c.\ 40.}; or”;

($b$) in paragraph (1)($d$)  omit the words “either of the circumstances specified in paragraph (2) or in paragraph (3) of”;

($c$) for paragraph (1)($f$)\footnote{Paragraph (1)($f$) was added by regulation 23($c$) of S.I.\ 2002/3019.} there shall be substituted the following—
\begin{quotation}
“($f$) he is absent from Great Britain—
\begin{enumerate}\item[]
(i) for more than 8 weeks where he is accompanying a young person solely in connection with arrangements made for the treatment of that person for a disease or bodily or mental disablement, and those arrangements relate to treatment outside Great Britain by, or under the supervision of, a person appropriately qualified to carry out the treatment, during the period whilst he is temporarily absent from Great Britain; or

(ii) for more than 4 weeks in all other cases.”;
\end{enumerate}
\end{quotation}

($d$) after paragraph (2) insert—
\begin{quotation}
“(3) In paragraph (1)($f$)  “young person” and “appropriately qualified” shall have the meaning given to them in regulation 3(4).”;
\end{quotation}

($e$) omit paragraph (1)($g$).
\end{enumerate}

(6) In regulation 10(1)(assessed income period) after sub-paragraph ($b$)  add—
\begin{quotation}
“($c$) that—
\begin{enumerate}\item[]
(i) the Secretary of State has sent the claimant the notification required by regulation 32(6)($a$)  of the Claims and Payments Regulations; and

(ii) the claimant has not provided sufficient information to enable the Secretary of State to determine whether there will be any variation in the claimant’s retirement provision throughout the period of 12 months beginning with the day following the day on which the previous assessed income period ends.”.
\end{enumerate}
\end{quotation}

(7) In regulation 15(5) (prescribed income for the purposes of section 15 of the Act)—
\begin{enumerate}\item[]
($a$) at the end of sub-paragraph ($f$)  omit “and”;

($b$) at the end of sub-paragraph ($g$)  add—
\begin{quotation}
“($h$) any income in lieu of that specified in—
\begin{enumerate}\item[]
(i) paragraphs ($a$)  to ($i$)  of section 15(1) of the Act, or

(ii) in this regulation;
\end{enumerate}

($i$) any payment of rent made to a claimant who—
\begin{enumerate}\item[]
(i) owns the freehold or leasehold interest in any property or is a tenant of any property;

(ii) occupies part of that property; and

(iii) has an agreement with another person allowing that person to occupy that property on payment of rent.”.
\end{enumerate}
\end{quotation}
\end{enumerate}

(8) In Schedule I (circumstances in which persons are treated as being severely disabled), in—
\begin{enumerate}\item[]
($a$) paragraph 1(1)($a$)(iii); and

($b$) paragraph 4(2),
\end{enumerate}
for the words “invalid care allowance” substitute “carer’s allowance”.

(9) In Schedule II (Housing Costs)—
\begin{enumerate}\item[]
($a$) in paragraph 2(7) after “maternity leave” insert “, paternity leave or adoption leave,”;

($b$) in paragraph 14(2)—
\begin{enumerate}\item[]
(i) in paragraphs ($a$)  and ($b$)  for “£88$.$00” substitute “£92$.$00”;

(ii) in paragraphs ($b$)  and ($c$)  for “£131$.$00” substitute “£137$.$00”;

(iii) in paragraphs ($c$)  and ($d$)  for “£170$.$00” substitute “£177$.$00”;

(iv) in paragraphs ($d$)  and ($e$)  for “£225$.$00” substitute “£235$.$00”;

(v) in paragraph ($e$)  for “£281$.$00” substitute “£293$.$00”.
\end{enumerate}
\end{enumerate}

(10) In Schedule III (Special Groups) in paragraph 1(8) for “6(5)($b$)(iv)” substitute “6(5)($b$)(v)”.

(11) In Schedule IV (amounts to be disregarded in the calculation of income other than earnings)—
\begin{enumerate}\item[]
($a$) in paragraph 1($c$)  after “widow” insert the words “or widower\footnote{Such payments were extended to widowers by the Naval, Military and Air Forces Etc.\ (Disablement and Death) Service Pensions Amendment Order 2002 (S.I.\ 2002/792).}”;

($b$) in paragraph 4 after the word “widows” insert the words “or widowers”;

($c$) in paragraph 5 after the word “widows” insert the words “or widowers”;

($d$) in paragraph 6(1)($a$)  after “widow” insert the words “or widower”;

($e$) in paragraph 6(1)($b$)  after “widows” insert the words “and widowers”;

($f$) after paragraph 16 add—
\begin{quotation}
“17.  Any special war widows payment made under—
\begin{enumerate}\item[]
($a$) the Naval and Marine Pay and Pensions (Special War Widows Payment) Order 1990 made under section 3 of the Naval and Marine Pay and Pensions Act 1865\footnote{28 \& 29 Vict.\ c.\ 73. Copies of the Order are available from the Ministry of Defence, Veterans Agency, Policy Section, Norcross, Blackpool, \textsc{\lowercase{FY5 3WP}}.};

($b$) the Royal Warrant dated 19th February 1990 amending the Schedule to the Army Pensions Warrant 1977\footnote{Army code no.\ 13045 published by the Stationery Office.};

($c$) the Queen’s Order dated 26th February 1990 made under section 2 of the Air Force (Constitution) Act 1917\footnote{7 \& 8 Geo.\ 5 c.\ 51. Queen’s Regulations for the Royal Air Force are available from the Stationery Office.};

($d$) the Home Guard War Widows Special Payments Regulations 1990 made under section 151 of the Reserve Forces Act 1980\footnote{1980 c.\ 9. Copies of the Regulations are available from the Ministry of Defence at the address given in the footnote above.};

($e$) the Orders dated 19th February 1990 amending Orders made on 12th December 1980 concerning the Ulster Defence Regiment made in each case under section 140 of the Reserve Forces Act 1980\footnote{Army code no.\ 60589 published by the Stationery Office.},
\end{enumerate}
and any analogous payment made by the Secretary of State for Defence to any person who is not a person entitled under the provisions mentioned in sub-paragraphs ($a$)  to ($e$)  of this paragraph.

\medskip

18.  Except in the case of income from capital specified in Part II of Schedule V, any actual income from capital.”.
\end{quotation}
\end{enumerate}

(12) In Schedule V, Part I (capital disregarded for the purposes of calculating income)—
\begin{enumerate}\item[]
($a$) after paragraph 1 add—
\begin{quotation}
“1A.  The dwelling occupied by the claimant as his home but only one home shall be disregarded under this paragraph.”;
\end{quotation}

($b$) in paragraph 9A\footnote{Paragraph 9A was inserted into the principal regulations by regulation 23($o$)(i) of S.I.\ 2002/3019.} omit the words from “for a period” to the end;

($c$) in paragraph 13—
\begin{enumerate}\item[]
(i) in sub-paragraph (1), for the words “where one of the partners” substitute the words “who is”;

(ii) in sub-paragraph (1)($a$), omit the word “is”;

(iii) in sub-paragraph (1)($b$), immediately before the word “was” insert the words “a diagnosed person’s partner or”;

(iv) in sub-paragraph (1)($c$), omit the word “is”;

(v) in sub-paragraph (2), after the word “Where” insert the words “a trust payment is made to”;

(vi) in sub-paragraph (2)($a$), for the words “sub-paragraph (1)($a$)  or ($b$)  applies, it” substitute “a person referred to in sub-paragraph (1)($a$)  or ($b$), that sub-paragraph” and for the words “the partner” substitute “that person”;

(vii) in sub-paragraph (2)($b$), for the words “sub-paragraph (1)($c$)  applies, it” substitute “a person referred to in sub-paragraph (1)($c$), that sub-paragraph”;

(viii) in sub-paragraph (3), for the words “where one of the partners” substitute the words “who is”;

(ix) in sub-paragraph (3)($a$), omit the word “is”;

(x) in sub-paragraph (3)($b$), immediately before the word “was” insert the words “a diagnosed person’s partner or”;

(xi) in sub-paragraph (3)($c$), omit the word “is”;

(xii) in sub-paragraph (4), immediately after the word “Where”, insert the words “a payment referred to in sub-paragraph (3) is made to”;

(xiii) in sub-paragraph (4)($a$), for the words “sub-paragraph (3)($a$)  or ($b$)  applies, it” substitute “a person referred to in sub-paragraph (3)($a$)  or ($b$), that sub-paragraph” and for the words “the partner” substitute “that person”;

(xiv) in sub-paragraph (4)($b$), for the words “sub-paragraph (3)($c$)  applies, it” substitute “a person referred to in sub-paragraph (3)($c$), that sub-paragraph”;

(xv) in sub-paragraph (6), for “Creutzfeld”, wherever it appears, substitute “Creutzfeldt”;
\end{enumerate}

($d$) for paragraph 20(1)($d$)\footnote{Paragraph ($d$) was added by regulation 23($o$)(iii)($aa$) of S.I.\ 2002/3019.} substitute—
\begin{quotation}
“($d$) any payment made by a local authority (including in England a county council), or by the National Assembly for Wales, to or on behalf of the claimant or his partner relating to a service which is provided to develop or sustain the capacity of the claimant or his partner to live independently in his accommodation.”;
\end{quotation}

($e$) in paragraph 20(2)—
\begin{enumerate}\item[]
(i) paragraph ($g$)  shall be omitted;

(ii) for paragraph ($h$)  substitute—
\begin{quotation}
    “an increase of a disablement pension under section 104 of the Contributions and Benefits Act (increase where constant attendance needed), and any further increase of such a pension under section 105 of that Act (increase for exceptionally severe disablement);”; 
\end{quotation}

(iii) in paragraph ($i$)  after “severe disablement” insert “or need for constant attendance,”;

(iv) paragraph ($m$)\footnote{Sub-paragraph ($m$) was inserted by regulation 23($o$)(iii)($bb$) of S.I.\ 2002/3019.} shall be omitted;
\end{enumerate}

($f$) for paragraph 20A\footnote{Paragraph 20A was inserted into the principal regulations by paragraph 12($d$) of the Schedule to S.I.\ 2002/3197.} substitute—
\begin{quotation}
“20A.---(1)  Subject to sub-paragraph (3), any payment of £5,000 or more to which paragraph 20(1)($a$), ($b$)  or ($c$)  applies, which has been made to rectify, or to compensate for, an official error relating to a relevant benefit and has been received by the claimant in full on or after the day on which he became entitled to benefit under these Regulations.

(2) Subject to sub-paragraph (3), the total amount of any payment disregarded under—
\begin{enumerate}\item[]
($a$) paragraph 7(2) of Schedule 10 to the Income Support (General) Regulations 1987\footnote{S.I.\ 1987/1967. The relevant amending instrument is S.I.\ 2002/2380.};

($b$) paragraph 12(2) of Schedule 8 to the Jobseeker’s Allowance Regulations 1996\footnote{S.I.\ 1996/207. The relevant amending instrument is S.I.\ 2002/2380.};

($c$) paragraph 8(2) of Schedule 5 or paragraph 21A of Schedule 5ZA to the Housing Benefit (General) Regulations 1987\footnote{S.I.\ 1987/1971. The relevant amending instruments are S.I.\ 2002/2380 and S.I.\ 2003/325 and 2275.}; or

($d$) paragraph 8(2) of Schedule 5 or paragraph 21A of Schedule 5ZA to the Council Tax Benefit (General) Regulations 1992\footnote{S.I.\ 1992/1814. The relevant amending instruments are S.I.\ 2002/2380 and S.I.\ 2003/325 and 2275.},
\end{enumerate}
where the award during which the disregard last applied in respect of the relevant sum either terminated immediately before the relevant date or is still in existence at that date.

(3) Any disregard which applies under sub-paragraph (1) or (2) shall have effect until the award comes to an end.

(4) In this paragraph—
\begin{enumerate}\item[]
    “the award”, except in sub-paragraph (2), means—
\begin{enumerate}\item[]
    ($a$) 
    the award of State Pension Credit under these Regulations during which the relevant sum or, where it is received in more than one instalment, the first instalment of that sum is received; or

    ($b$) 
    where that award is followed immediately by one or more further awards which begins immediately after the previous award ends, such further awards until the end of the last award, provided that, for such further awards, the claimant—
\begin{enumerate}\item[]
    (i) 
    is the person who received the relevant sum;

    (ii) 
    is the partner of that person; or

    (iii) 
    was the partner of that person at the date of his death;
\end{enumerate}
\end{enumerate}

    “official error”—
\begin{enumerate}\item[]
    ($a$) 
    where the error relates to housing benefit or council tax benefit, has the meaning given by regulation 1(2) of the Housing Benefit and Council Tax Benefit (Decisions and Appeals) Regulations 2001\footnote{S.I.\ 2001/1002.};
    and

    ($b$) 
    where the error relates to any other relevant benefit, has the meaning given by regulation 1(3) of the Social Security and Child Support (Decisions and Appeals) Regulations 1999;
\end{enumerate}

    “the relevant date” means the date on which the claimant became entitled to benefit under the Act;

    “relevant benefit” means any benefit specified in paragraph 20(2); and

    “the relevant sum” means the total payment referred to in sub-paragraph (1) or, as the case may be, the total amount referred to in sub-paragraph (2).”; 
\end{enumerate}
\end{quotation}

($g$) omit paragraph 27.
\end{enumerate}

(13) In Schedule VI (sums disregarded from claimant’s earnings) after paragraph 2A\footnote{Paragraph 2A was inserted by paragraph 13($b$) of the Schedule to S.I.\ 2002/3197.} add—
\begin{quotation}
“2B.  Where only one member of a couple is in employment specified in paragraph 2(2), so much of the earnings of the other member of the couple as would not, in aggregate with the earnings disregarded under paragraph 2, exceed £20.”.
\end{quotation}

\subsection[3. Amendment of the State Pension Credit (Consequential, Transitional and Miscellaneous Provisions) Regulations 2002]{Amendment of the State Pension Credit (Consequential, Transitional and Miscellaneous Provisions) Regulations 2002}

3.  In regulation 36 of the State Pension Credit (Consequential, Transitional and Miscellaneous Provisions) Regulations 2002\footnote{S.I.\ 2002/3019.} (persons entitled to income support immediately before the appointed day)—
\begin{enumerate}\item[]
($a$) at the beginning of paragraph (7), there shall be inserted the words “Notwithstanding the provisions of Schedule 3B of the Decisions and Appeals Regulations,”;

($b$) after paragraph (7) there shall be inserted—
\begin{quotation}
“(7A) Notwithstanding the provisions of paragraph (7), where the relevant change of circumstances is that the transferee becomes a patient\footnote{“Patient” means a person (other than a prisoner) who is regarded as receiving free in-patient care within the meaning of the Social Security (Hospital In-Patients) Regulations 1975 (S.I.\ 1975/555).} again within the same benefit week in which he ceased to be a patient, the superseding decision in respect of becoming a patient again shall take effect from the first day of the benefit week following the benefit week in which the change occurs.”.
\end{quotation}
\end{enumerate}

\subsection[4. Amendment of the Social Security (Claims and Payments) Regulations 1987]{Amendment of the Social Security (Claims and Payments) Regulations 1987}

4.  In regulation 32 of the Social Security (Claims and Payments) Regulations 1987\footnote{S.I.\ 1987/1968.} (information to be given and changes to be notified)—
\begin{enumerate}\item[]
($a$) at the end of paragraph (6)($a$)  delete “and”;

($b$) for paragraph (6)($b$)\footnote{Paragraph (6) was added to regulation 32 by regulation 11 of S.I.\ 2002/3019.} substitute—
\begin{quotation}
“($b$) except to the extent that sub-paragraph ($a$)  applies, changes to an element of the claimant’s retirement provision need not be notified if—
\begin{enumerate}\item[]
(i) an assessed income period is current in his case;

(ii) the time limit set out in sub-paragraph ($c$)  has not expired; or

(iii) the Secretary of State grants, or has granted, such longer period as he considers reasonable under sub-paragraph ($c$)  and that period has not expired; and
\end{enumerate}

($c$) the information and evidence required under sub-paragraph ($a$)  shall be furnished within one month of the date on which the Secretary of State notifies the claimant of the requirement or within such longer period as the Secretary of State considers reasonable.”.
\end{quotation}
\end{enumerate}

\subsection[5. Amendment of the Social Security and Child Support (Decisions and Appeals) Regulations 1999]{Amendment of the Social Security and Child Support (Decisions and Appeals) Regulations 1999}

5.---(1)  The Social Security and Child Support (Decisions and Appeals) Regulations 1999\footnote{S.I.\ 1999/991.} shall be amended in accordance with the following paragraphs.

(2) In regulation 6(2) (supersession of decisions) after sub-paragraph ($l$)\footnote{Sub-paragraph ($l$) was inserted into regulation 6(2) by regulation 17 of S.I.\ 2002/3019.} add—
\begin{quotation}
“($m$) is a relevant decision for the purposes of section 6 of the State Pension Credit Act in a case where—
\begin{enumerate}\item[]
(i) the information and evidence required under regulation 32(6)($a$)  of the Claims and Payments Regulations has not been provided in accordance with the time limits set out in regulation 32(6)($c$)  of those Regulations;

(ii) the Secretary of State was prevented from specifying a new assessed income period under regulation 10(1) of the State Pension Credit Regulations; and

(iii) the information and evidence required under regulation 32(6)($a$)  of the Claims and Payments Regulations has since been provided.”.
\end{enumerate}
\end{quotation}

(3) In regulation 7 (date from which a decision under section 10 takes effect)—
\begin{enumerate}\item[]
($a$) in paragraph (29)\footnote{Paragraph (29) was inserted into regulation 7 by regulation 18 of S.I.\ 2002/3019.} for “A” substitute “Subject to paragraphs (29A) and (29B), a”;

($b$) after paragraph (29) add—
\begin{quotation}
“(29A) A decision to which regulation 6(2)($l$)  applies, where—
\begin{enumerate}\item[]
($a$) the decision is advantageous to the claimant; and

($b$) the information and evidence required under regulation 32(1) of the Claims and Payments Regulations has not been provided within the period allowed under that regulation,
\end{enumerate}
shall take effect from the day the information and evidence required under that regulation is provided if that day is the first day of the claimant’s benefit week, but, if it is not, from the next following such day.

(29B) A decision to which regulation 6(2)($l$)  applies, where—
\begin{enumerate}\item[]
($a$) the decision is disadvantageous to the claimant; and

($b$) the information and evidence required under regulation 32(1) of the Claims and Payments Regulations has not been provided within the period allowed under that regulation,
\end{enumerate}
shall take effect from the day after the period allowed under that regulation expired.

(29C) Except where there is a change of circumstances during the period in which the Secretary of State was prevented from specifying a new assessed income period under regulation 10(1) of the State Pension Credit Regulations, a decision to which regulation 6(2)($m$)  applies shall take effect from the day on which the information and evidence required under regulation 32(6)($a$)  of the Claims and Payments Regulations was provided.”.
\end{quotation}
\end{enumerate}

(4) In Schedule 3B\footnote{Schedule 3B was inserted by regulation 22 of S.I.\ 2002/3019.} (date on which change of circumstances takes effect where claimant entitled to state pension credit) for paragraph 5 substitute—
\begin{quotation}
“5.  In a case where the relevant circumstance is that the claimant ceased to be a patient, if he becomes a patient again in the same benefit week, the superseding decision in respect of ceasing to be a patient shall take effect from the first day of the week in which the change occured.”.
\end{quotation}

\subsection[6. Amendment of the Social Security (Immigration and Asylum) Consequential Amendments Regulations 2000]{Amendment of the Social Security (Immigration and Asylum) Consequential Amendments Regulations 2000}

6.---(1)  Regulation 2 of the Social Security (Immigration and Asylum) Consequential Amendments Regulations 2000\footnote{S.I.\ 2000/636.} shall be amended in accordance with the following paragraphs.

(2) In paragraph (1) (persons not excluded from specified benefits under section 115 of the Immigration and Asylum Act 1999) after “Contributions and Benefits Act,” there shall be inserted the words “or state pension credit under the State Pension Credit Act 2002\footnote{2002 c.\ 15.},”.

(3) After paragraph (4)($b$)  insert—
\begin{quotation}
“($c$) state pension credit under the State Pension Credit Act 2002, a person to whom sub-paragraph ($a$)  would have applied but for the fact that they have attained the qualifying age for the purposes of state pension credit, is a person to whom section 115 of the Act does not apply.”.
\end{quotation}

(4) After paragraph (6) insert—
\begin{quotation}
“(7) For the purposes of entitlement to state pension credit under the State Pension Credit Act 2002, a person to whom paragraph (5) would have applied but for the fact that they have attained the qualifying age for the purposes of state pension credit, is a person to whom section 115 of the Act does not apply.

(8) Where paragraph 1 of Part I of the Schedule to these Regulations applies in respect of entitlement to state pension credit, the period for which a claimant’s state pension credit is to be calculated shall be any period, or the aggregate of any periods, not exceeding 42 days during any one period of leave to which paragraph 1 of Part I of the Schedule to these Regulations applies.”.
\end{quotation}

\bigskip

%Signed 
%by authority of the 
%Secretary of State for Work and Pensions.

{\raggedleft
\emph{Andrew Smith}\\*Secretary of State,\\*Department for Work and Pensions

}

7th September 2003

\small

\part{Explanatory Note}

\renewcommand\parthead{— Explanatory Note}

\subsection*{(This note is not part of the Regulations)}

These Regulations amend the Social Security (Claims and Payments) Regulations 1987 (“the Claims and Payments Regulations”), the Social Security and Child Support (Decisions and Appeals) Regulations 1999 (“the Decisions and Appeals Regulations”), the State Pension Credit Regulations 2002 and the State Pension Credit (Consequential, Transitional and Miscellaneous Provisions) Regulations 2002 in connection with the introduction of state pension credit on 6th October 2003.

Regulation 2 amends the State Pension Credit Regulations 2002. In particular it—
\begin{enumerate}\item[]
    adds a definition of adoption leave and paternity leave;

    makes provision for persons from Montserrat;

    clarifies the provisions under which a person is treated as being or not being a member of the same household as the claimant;

    clarifies provisions in respect of certain categories of income;

    adds adoption leave and paternity leave to the periods in which a person is treated as not being in remunerative work for the purposes of housing costs;

    amends the reference to invalid care allowance to carer’s allowance;

    uprates the amounts deducted in respect of non-dependants for the purposes of housing costs;

    adds new categories of income disregards for war widowers and in respect of special war widows payments;

    provides that actual income from capital will only be taken into account in certain specified circumstances;

    amends a capital disregard in respect of relevant trust payments to persons suffering from variant Creutzfeldt-Jacob disease so that it also applies to single claimants;

    amends a capital disregard in respect of compensation payments received in respect of certain specified benefits;

    allows couples, where one partner is in a special occupation, to benefit from a £20 disregard;

    makes minor and technical amendments. 
\end{enumerate}

Regulation 3 clarifies regulation 36(7) of the State Pension Credit (Consequential, Transitional and Miscellaneous Provisions) Regulations 2002 as an exception to the Decisions and Appeals Regulations.

Regulation 4 amends regulation 32(6) of the Claims and Payments Regulations so as to extend the category of cases to which sub-paragraph ($b$)  applies and inserting sub-paragraph ($c$).

Regulation 5 amends the Decisions and Appeals Regulations so as to provide a new ground for supersession in relation to state pension credit and new dates on which certain state pension credit decisions take effect.

Regulation 6 amends provisions concerning the circumstances in which certain persons who are subject to immigration control are eligible to be awarded state pension credit.

These Regulations do not impose a charge on business. 

\end{document}