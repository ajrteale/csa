\documentclass[12pt,a4paper]{article}

\newcommand\regstitle{The Child Support (Maintenance Calculation Procedure) Regulations 2000}

\newcommand\regsnumber{2001/157}

%\opt{newrules}{
\title{\regstitle}
%}

%\opt{2012rules}{
%\title{Child Maintenance and Other Payments Act 2008\\(2012 scheme version)}
%}

\author{S.I.\ 2001 No.\ 157}

\date{Made
18th January 2001\\
%Laid before Parliament
%27th July 2000\\
Coming into force
in accordance with regulation 1(5)
}

%\opt{oldrules}{\newcommand\versionyear{1993}}
%\opt{newrules}{\newcommand\versionyear{2003}}
%\opt{2012rules}{\newcommand\versionyear{2012}}

\usepackage{csa-regs}

\setlength\headheight{27.57402pt}

\begin{document}

\maketitle

\noindent
Whereas a draft of this instrument was laid before Parliament in accordance with section 52(2) of the Child Support Act 1991\footnote{1991 c.\ 48.} and approved by a resolution of each House of Parliament:

Now, therefore, the Secretary of State for Social Security, in exercise of the powers conferred upon him by sections 3(3), 5(3), 12(4) and (5)($b$), 46(2), (5), (8) and (10), 51, 52(4), 54 and 55 of, and paragraphs 11 and 14 of Schedule 1 to, the Child Support Act 1991\footnote{Sections 5(3), 51 and 54 were amended by and sections 12 and 46 were substituted by the Child Support, Pensions and Social Security Act 2000 (c.\ 19). Section 54 is cited because of the meaning assigned to the word “prescribed”.} and of all other powers enabling him in that behalf, hereby makes the following Regulations: 

{\sloppy

\tableofcontents

}

\bigskip

\setcounter{secnumdepth}{-2}

\section[Part I --- General]{Part I\\*General}

\renewcommand\parthead{--- Part I}

\subsection[1. Citation, commencement and interpretation]{Citation, commencement and interpretation}

1.---(1)  These Regulations may be cited as the Child Support (Maintenance Calculation Procedure) Regulations 2000.

(2) In these Regulations, unless the context otherwise requires—
\begin{enumerate}\item[]
“the Act” means the Child Support Act 1991;

“date of notification to the non-resident parent” means the date on which the non-resident parent is first given notice of a maintenance application;

“effective application” means as provided for in regulation 3;

“date of receipt” means the date on which the information or document is actually received;

“effective date” means the date on which a maintenance calculation takes effect for the purposes of the Act;

“notice of a maintenance application” means notice by the Secretary of State under regulation 5(1) that an application for a maintenance calculation has been made, or treated as made, in relation to which the non-resident parent is named as a parent of the child to whom the application relates;

“Maintenance Calculations and Special Cases Regulations” means the Child Support (Maintenance Calculations and Special Cases) Regulations 2000\footnote{S.I.\ 2001/155.};

“maintenance period” has the same meaning as in section 17(4A) of the Act\footnote{Section 17(4A) was inserted by section 9 of the Child Support, Pensions and Social Security Act 2000.};

“relevant person” means—
\begin{enumerate}\item[]
($a$) 
a person with care;

($b$) 
a non-resident parent;

($c$) 
a parent who is treated as a non-resident parent under regulation 8 of the Maintenance Calculations and Special Cases Regulations;

($d$) 
where the application for a maintenance calculation is made by a child under section 7 of the Act, that child, in respect of whom a maintenance calculation has been applied for, or has been treated as applied for under section 6(3) of the Act, or is or has been in force.
\end{enumerate}
\end{enumerate}

(3) The provisions in Schedule 1 shall have effect to supplement the meaning of “child” in section 55 of the Act.

(4) In these Regulations, unless the context otherwise requires, a reference—
\begin{enumerate}\item[]
($a$) to a numbered Part is to the Part of these Regulations bearing that number;

($b$) to a numbered Schedule is to the Schedule to these Regulations bearing that number;

($c$) to a numbered regulation is to the regulation in these Regulations bearing that number;

($d$) in a regulation or Schedule to a numbered paragraph is to the paragraph in that regulation or Schedule bearing that number; and

($e$) in a paragraph to a lettered or numbered sub-paragraph is to the sub-paragraph in that paragraph bearing that letter or number.
\end{enumerate}

(5) These Regulations shall come into force in relation to a particular case on the day on which the amendments to sections 5, 6, 12, 46, 51
%, 54 and 55 
and 54  % Words substituted (30.4.02) by SI 2002/1204 reg 6(2)
of the Act made by the Child Support, Pensions and Social Security Act 2000 come into force in relation to that type of case.

\amendment{
Words substituted in reg. 1(5) (30.4.02) by the Child Support (Miscellaneous Amendments) Regulations 2002 reg. 6(2).
}

\subsection[2. Documents]{Documents}

2.  Except where otherwise stated, where—
\begin{enumerate}\item[]
($a$) any document is given or sent to the Secretary of State, that document shall be treated as having been so given or sent on the day that it is received by the Secretary of State; and

($b$) any document is given or sent to any other person, that document shall, if sent by post to that person’s last known or notified address, be treated as having been given or sent on the day that it is posted.
\end{enumerate}

\section[Part II --- Applications for a maintenance calculation]{Part II\\*Applications for a maintenance calculation}

\renewcommand\parthead{--- Part II}

\subsection[3. Applications under section 4 or 7 of the Act]{Applications under section 4 or 7 of the Act}

3.---(1)  A person who applies for a maintenance calculation under section 4 or 7 of the Act need not normally do so in writing, but if the Secretary of State directs that the application be made in writing, the application shall be made either by completing and returning, in accordance with the Secretary of State’s instructions, a form provided for that purpose, or in such other written form as the Secretary of State may accept as sufficient in the circumstances of any particular case.

(2) An application for a maintenance calculation is effective if it complies with paragraph (1) and, subject to paragraph (4), is made on the date it is received.

(3) Where an application for a maintenance calculation is not effective the Secretary of State may request the person making the application to provide such additional information or evidence as the Secretary of State may specify and, where the application was made on a form, the Secretary of State may request that the information or evidence be provided on a fresh form.

(4) Where the additional information or evidence requested is received by the Secretary of State within 14 days of the date of his request, or at a later date in circumstances where the Secretary of State is satisfied that the delay was unavoidable, he shall treat the application as made on the date on which the earlier or earliest application would have been treated as made had it been effective.

(5) Where the Secretary of State receives the additional information or evidence requested by him more than 14 days from the date of the request and in circumstances where he is not satisfied that the delay was unavoidable, the Secretary of State shall treat the application as made on the date of receipt of the information or evidence.

(6) Subject to paragraph (7), a person who has made an effective application may amend or withdraw the application at any time before a maintenance calculation is made and such amendment or withdrawal need not be in writing unless, in any particular case, the Secretary of State requires it to be.

(7) No amendment made under paragraph (6) shall relate to any change of circumstances arising after the effective date of a maintenance calculation resulting from an effective application.

\subsection[4. Multiple applications]{Multiple applications}

4.---(1)  The provisions of Schedule 2 shall apply in cases where there is more than one application for a maintenance calculation.

(2) The provisions of paragraphs 1, 2 and 3 of Schedule 2 relating to the treatment of two or more applications as a single application shall apply where no request is received for the Secretary of State to cease acting in relation to all but one of the applications.

(3) Where, under the provisions of paragraph 1, 2 or 3 of Schedule 2, two or more applications are to be treated as a single application, that application shall be treated as an application for a maintenance calculation to be made with respect to all of the qualifying children mentioned in the applications, and the effective date of that maintenance calculation shall be determined by reference to the earlier or earliest application.

\subsection[5. Notice of an application for a maintenance calculation]{Notice of an application for a maintenance calculation}

5.---(1)  Where an effective application has been made under section 4 or 7 of the Act, or 
an application  % Words inserted (21.2.03) by SI 2003/328 reg 7(2)
is treated as made under section 6(3) of the Act, as the case may be, the Secretary of State shall as soon as is reasonably practicable notify, orally or in writing, the non-resident parent and any other relevant persons (other than the person who has made, or is treated as having made, the application) of that application and request such information as he may require to make the maintenance calculation in such form and manner as he may specify in the particular case.

(2) Where the person to whom notice is being given under paragraph (1) is a non-resident parent, that notice shall specify the effective date of the maintenance calculation if one is to be made, and the ability to make a default maintenance decision.

(3) Subject to paragraph (4), a person who has provided information under paragraph (1) may amend the information he has provided at any time before a maintenance calculation is made and such information need not be in writing unless, in any particular case, the Secretary of State requires it to be.

(4) No amendment under paragraph (3) shall relate to any change of circumstances arising after the effective date of any maintenance calculation made in response to the application in relation to which the information was requested.

\amendment{
Words inserted in reg. 5(1)(21.2.03) by the Child Support (Miscellaneous Amendments) Regulations 2003 reg. 7(2).
}

\subsection[6. Death of a qualifying child]{Death of a qualifying child}

6.---(1)  Where the Secretary of State is informed of the death of a qualifying child with respect to whom an application for a maintenance calculation has been made or has been treated as made, he shall—
\begin{enumerate}\item[]
($a$) proceed with the application as if it had not been made with respect to that child if he has not yet made a maintenance calculation;

($b$) treat any maintenance calculation already made by him as not having been made if the relevant persons have not been notified of it and proceed with the application as if it had not been made with respect to that child.
\end{enumerate}

(2) Where all of the qualifying children with respect to whom an application for a maintenance calculation has been made have died, and either the calculation has not been made or the relevant persons have not been notified of it, the Secretary of State shall treat the application as not having been made.

\section[Part III --- Default maintenance decisions]{Part III\\*Default maintenance decisions}

\renewcommand\parthead{--- Part III}

\subsection[7. Default rate]{Default rate}

7.---(1)  Where the Secretary of State makes a default maintenance decision under section 12(1) of the Act (insufficient information to make a maintenance calculation or to make a decision under section 16 or 17 of the Act) the default rate is as set out in paragraph (2).

(2) The default rate for the purposes of section 12(5)($b$)  of the Act shall be—
\begin{enumerate}\item[]
    £30 where there is one qualifying child of the non-resident parent;

    £40 where there are two qualifying children of the non-resident parent;

    £50 where there are three or more qualifying children of the non-resident parent, 
\end{enumerate}
    apportioned, where the non-resident parent has more than one qualifying child and in relation to them there is more than one person with care, as provided in paragraph 6(2) of Part I of Schedule 1 to the Act. 

(3) Subject to paragraph (4), where any apportionment made under this regulation results in a fraction of a penny that fraction shall be treated as a penny if it is either one half or exceeds one half, otherwise it shall be disregarded.

(4) If, in making the apportionment required by this regulation, the effect of the application of paragraph (3) would be such that the aggregate amount of child support maintenance payable by a non-resident parent would be different from the aggregate amount payable before any apportionment, the Secretary of State shall adjust that apportionment so as to eliminate that difference; and that adjustment shall be varied from time to time so as to secure that, taking one week with another and so far as is practicable, each person with care receives the amount which she would have received if no adjustment had been made under this paragraph.

\section[Part IV --- Reduced benefit decisions]{Part IV\\*Reduced benefit decisions}

\renewcommand\parthead{--- Part IV}

\subsection[8. Interpretation of Part IV]{Interpretation of Part IV}

8.---(1)  For the purposes of this Part—
\begin{enumerate}\item[]
    “applicable amount” is to be construed in accordance with Part IV of the Income Support Regulations and regulations 83 to 86 of the Jobseeker’s Allowance Regulations;

    “benefit week”, in relation to income support has the same meaning as in the Income Support Regulations, and in relation to jobseeker’s allowance has the same meaning as in the Jobseeker’s Allowance Regulations;

    “Income Support Regulations” means the Income Support (General) Regulations 1987\footnote{S.I.\ 1987/1967.};

\begin{sloppypar}
    “Jobseeker’s Allowance Regulations” means the Jobseeker’s Allowance Regulations 1996\footnote{S.I.\ 1996/207.};
\end{sloppypar}

    “parent concerned” means the parent with respect to whom a reduced benefit decision is given;

    “reduced benefit decision” has the same meaning as in section 46(10)($b$)  of the Act; and

    “relevant benefit” has the same meaning as in section 46(10)($c$)  of the Act. 
\end{enumerate}

(2) In this Part references to a reduced benefit decision as being “in operation”, “suspended” or “in force” shall be construed as follows—
\begin{enumerate}\item[]
($a$) a reduced benefit decision is “in operation” if, by virtue of that decision, relevant benefit is currently being reduced;

($b$) a reduced benefit decision is “suspended” if—
\begin{enumerate}\item[]
(i) after that decision has been given, relevant benefit ceases to be payable, or becomes payable at one of the rates indicated in regulation 14(4) or, as the case may be, regulation 15(4);

(ii) at the time the reduced benefit decision is given, relevant benefit is payable at one of the rates indicated in regulation 15(4) or, as the case may be, regulation 16(4),
\end{enumerate}
and these Regulations provide for relevant benefit payable from a later date to be reduced by virtue of the same reduced benefit decision; and

($c$) a reduced benefit decision is “in force” if it is either in operation or suspended and cognate terms shall be construed accordingly.
\end{enumerate}

\subsection[9. Period within which reasons are to be given]{Period within which reasons are to be given}

9.  The period specified for the purposes of section 46(2) of the Act (for the parent to supply her reasons) is 4 weeks from the date on which the Secretary of State serves notice under that subsection.

% Reg 9A inserted (30.4.02) by SI 2002/1204 reg 6(3)
\subsection[9A. Period for parent to state if request still stands]{Period for parent to state if request still stands}

9A.  The period to be specified for the purposes of section 46(6) of the Act (period for the parent to state if her request still stands) is 4 weeks from the date on which the Secretary of State serves notice under that subsection.

\amendment{
Reg. 9A inserted (30.4.02) by the Child Support (Miscellaneous Amendments) Regulations 2002 reg. 6(3).
}

\subsection[10. Circumstances in which a reduced benefit decision shall not be given]{Circumstances in which a reduced benefit decision shall not be given}

10.  The Secretary of State shall not give a reduced benefit decision where—
\begin{enumerate}\item[]
($a$) income support is paid to, or in respect of, the parent in question and the applicable amount of the claimant for income support includes one or more of the amounts set out in paragraph 15(3), (4) or (6) of Part IV of Schedule 2 to the Income Support Regulations\footnote{Part IV of Schedule 2 was substituted by S.I.\ 2000/440.}; or

($b$) an income-based jobseeker’s allowance is paid to, or in respect of, the parent in question and the applicable amount of the claimant for an income-based jobseeker’s allowance includes one or more of the amounts set out in paragraph 20(4), (5) or (7) of Schedule 1 to the Jobseeker’s Allowance Regulations\footnote{Paragraph 20 of Schedule 1 was substituted by S.I.\ 2000/440.};
%
% Reg 10(c) added (6.4.03) by SI 2003/328 reg 7(3)
or

    ($c$) 
    an amount prescribed under section 9(5)($c$)  of the Tax Credits Act 2002 (increased elements of child tax credit for children or young persons with a disability) is included in an award of child tax credit payable to the parent in question or a member of that parent’s family living with him.
\end{enumerate}

\amendment{
Reg. 10(c) added (6.4.03) by the Child Support (Miscellaneous Amendments) Regulations 2003 reg. 7(3).
}

\subsection[11. Amount of and period of reduction of relevant benefit under a reduced benefit decision]{Amount of and period of reduction of relevant benefit under a reduced benefit decision}

11.---(1)  The reduction in the amount payable by way of a relevant benefit to, or in respect of, the parent concerned and the period of such reduction by virtue of a reduced benefit decision shall be determined in accordance with paragraphs (2) to (8) below.

(2) Subject to paragraph (6) and regulations 12, 13, 14, and 15, there shall be a reduction for a period of 156 weeks from the day specified in the reduced benefit decision under the provisions of section 46(8) of the Act in respect of each such week equal to—
\[
0.4 \times B
\]
where
\begin{enumerate}\item[]
    B is an amount equal to the weekly amount in relation to the week in question, specified in column (2) of paragraph 1(1)($e$)  of Schedule 2 to the Income Support Regulations. 
\end{enumerate}

(3) Subject to paragraph (4), a reduced benefit decision shall come into operation on the first day of the second benefit week following the date of the reduced benefit decision.

(4) Subject to paragraph (5), where a reduced benefit decision (“the subsequent decision”) is made on a day when a reduced benefit decision (“the earlier decision”) is in force in respect of the same parent, the subsequent decision shall come into operation on the day immediately following the day on which the earlier decision ceased to be in force.

(5) Where the relevant benefit is income support and the provisions of regulation 26(2) of the Social Security (Claims and Payments) Regulations 1987\footnote{S.I.\ 1987/1968; relevant amending instruments are S.I.\ 1988/522, 1989/136 and 1999/3128.} (deferment of payment of different amount of income support) apply, a reduced benefit decision shall come into operation on such later date as may be determined by the Secretary of State in accordance with those provisions.

(6) Where the benefit payable is income support or an income-based jobseeker’s allowance and there is a change in the benefit week whilst a reduced benefit decision is in operation, the period of the reduction specified in paragraph (2) shall be a period greater than 155 weeks but less than 156 weeks and ending on the last day of the last benefit week falling entirely within the period of 156 weeks specified in that paragraph.

(7) Where the weekly amount specified in column (2) of paragraph 1(1)($e$)  of Schedule 2 to the Income Support Regulations changes on a day when a reduced benefit decision is in operation, the amount of the reduction of income support or income-based jobseeker’s allowance shall be changed from the first day of the first benefit week to commence for the parent concerned on or after the day that weekly amount changes.

(8) Only one reduced benefit decision in relation to a parent concerned shall be in force at any one time.

\subsection[12. Modification of reduction under a reduced benefit decision to preserve minimum entitlement to relevant benefit]{Modification of reduction under a reduced benefit decision to preserve minimum entitlement to relevant benefit}

12.  Where in respect of any benefit week the amount of the relevant benefit that would be payable after it has been reduced following a reduced benefit decision would, but for this regulation, be nil or less than the minimum amount of that benefit that is payable as determined—
\begin{enumerate}\item[]
($a$) in the case of income support, by regulation 26(4) of the Social Security (Claims and Payments) Regulations 1987;

($b$) in the case of an income-based jobseeker’s allowance, by regulation 87A of the Jobseeker’s Allowance Regulations\footnote{S.I.\ 1996/207; regulation 87A was inserted by S.I.\ 1996/1517.},
\end{enumerate}
the amount of that reduction shall be decreased to such extent as to raise the amount of that benefit to the minimum amount that is payable.

\subsection[13. Suspension of a reduced benefit decision when relevant benefit ceases to be payable]{Suspension of a reduced benefit decision when relevant benefit ceases to be payable}

13.---(1)  Where relevant benefit ceases to be payable to, or in respect of, the parent concerned at a time when a reduced benefit decision is in operation, that reduced benefit decision shall, subject to paragraph (2), be suspended for a period of 52 weeks from the date the relevant benefit ceases to be payable.

(2) Where a reduced benefit decision has been suspended for a period of 52 weeks and no relevant benefit is payable at the end of that period, it shall cease to be in force.

(3) Where a reduced benefit decision is suspended and relevant benefit again becomes payable to, or in respect of, the parent concerned, the amount payable by way of that benefit shall, subject to regulations 14 and 15, be reduced in accordance with that reduced benefit decision for the balance of the reduction period.

(4) The amount or, as the case may be, the amounts of that reduction to be made during the balance of the reduction period shall be determined in accordance with regulation 11(2).

(5) No reduction in the amount of benefit under paragraph (3) shall be made before the expiry of a period of 14 days from service of the notice specified in paragraph (6), and the provisions of regulation 11(3) shall apply as to the date the reduced benefit decision again comes into operation.

(6) Where relevant benefit again becomes payable to, or in respect of, a parent with respect to whom a reduced benefit decision is suspended, she shall be notified in writing by the Secretary of State that the amount of relevant benefit paid to, or in respect of, her will again be reduced, in accordance with the provisions of paragraph (3), if she falls within section 46(1) of the Act.

\subsection[14. Suspension of a reduced benefit decision when a modified applicable amount is payable (income support)]{Suspension of a reduced benefit decision when a modified applicable amount is payable (income support)}

14.---(1)  Where a reduced benefit decision is given or is in operation at a time when income support is payable to, or in respect of, the parent concerned but her applicable amount falls to be calculated under the provisions mentioned in paragraph (4), that decision shall be suspended for so long as her applicable amount falls to be calculated under the provisions mentioned in that paragraph, or 52 weeks, whichever period is the shorter.

(2) Where a reduced benefit decision is given or is in operation at a time when income support is payable to, or in respect of, the parent concerned, but her applicable amount includes a residential allowance under regulation 17 of, and paragraph 2A of Schedule 2 to, the Income Support Regulations\footnote{Regulation 17 was amended and paragraph 2A added by S.I.\ 1992/3147. Paragraph 2A was amended by S.I.\ 1993/518 and 1219, 1996/599, 1997/2197 and 2000/990.} (applicable amounts for persons in residential care and nursing homes), that decision shall be suspended for as long as her applicable amount includes a residential allowance under that regulation and Schedule 2, or 52 weeks, whichever period is the shorter.

(3) Where a case falls within paragraph (1) or (2) and a reduced benefit decision has been suspended for 52 weeks, it shall cease to be in force.

(4) The provisions of paragraph (1) shall apply where the applicable amount in relation to the parent concerned falls to be calculated under—
\begin{enumerate}\item[]
($a$) regulation 19 of, and Schedule 4 to, the Income Support Regulations (applicable amounts for persons in residential care and nursing homes)\footnote{Regulation 19 was amended by S.I.\ 1988/663, 2022, 1989/1678, 1991/1033, 1992/3147, 1993/2119, 1994/527, 2139, 1996/206, 462. Schedule 4 was amended by S.I.\ 1988/663, 1445, 2022, 
1989/1678, 1991/544, 1559, 1992/468, 1993/2119, 1997/2197 and 2000/440.};

($b$) regulation 21 of, and paragraphs 1 to 3 of Schedule 7 to, the Income Support Regulations (patients)\footnote{Regulation 21 was amended by S.I.\ 1991/1033, 1656, 1992/2155, 3147, 1993/518, 2119, 1994/527, 1807, 2139, 1995/516, 1996/206, 2006, 2431, 2614, 1944, 1998/563, 2000/636, 979. Relevant amendments to Schedule 7 were made by S.I.\ 1990/547, 1996/1803, 1998/563 and 2000/440.};

($c$) regulation 21 of, and paragraphs 10B, 10C and 13 of Schedule 7 to, the Income Support Regulations (persons in local authority or residential accommodation)\footnote{Paragraph 10B was inserted by S.I.\ 1988/663 and amended by S.I.\ 1992/3147 and 2000/440. Paragraph 10C was inserted by S.I.\ 1988/2022 and amended by S.I.\ 1990/547, 1992/3147, 1996/599, 1803 and 2000/440.}.
\end{enumerate}

\subsection[15. Suspension of a reduced benefit decision when a modified applicable amount is payable (income-based jobseeker’s allowance)]{Suspension of a reduced benefit decision when a modified applicable amount is payable (income-based jobseeker’s allowance)}

15.---(1)  Where a reduced benefit decision is given or is in operation at a time when an income-based jobseeker’s allowance is payable to, or in respect of, the parent concerned but her applicable amount falls to be calculated under the provisions mentioned in paragraph (4), that reduced benefit decision shall be suspended for so long as the applicable amount falls to be calculated under those provisions, or 52 weeks, whichever is the shorter.

(2) Where a reduced benefit decision is given or is in operation at a time when an income-based jobseeker’s allowance is payable to, or in respect of, the parent concerned but her applicable amount includes a residential allowance under regulation 83($c$)  of, and paragraph 3 of Schedule 1 to, the Jobseeker’s Allowance Regulations (persons in residential care or nursing homes)\footnote{Paragraph 3 was amended by S.I.\ 1997/2197.}, that reduced benefit decision shall be suspended for so long as the applicable amount includes such a residential allowance, or 52 weeks, whichever is the shorter.

(3) Where a case falls within paragraph (1) or (2) and a reduced benefit decision has been suspended for 52 weeks, it shall cease to be in force.

(4) The provisions of paragraph (1) shall apply where the applicable amount in relation to the parent concerned falls to be calculated under—
\begin{enumerate}\item[]
($a$) regulation 85 of, and paragraph 1 or 2 of Schedule 5 to, the Jobseeker’s Allowance Regulations (patients)\footnote{Regulation 85 was amended by S.I.\ 1996/1516, 1538, 1997/454, 2000/636, 979 and paragraphs 1 and 2 were amended by S.I.\ 1996/1516 and 2000/440.};

($b$) regulation 85 of, and paragraphs 8, 9 or 15 of Schedule 5 to, the Jobseeker’s Allowance Regulations (persons in local authority or residential accommodation)\footnote{Paragraphs 8 and 15 were amended by S.I.\ 1996/1516 and paragraph 9 was amended by S.I.\ 1996/1803.}; or

($c$) regulation 86 of, and Schedule 4 to, the Jobseeker’s Allowance Regulations (applicable amounts for persons in residential care and nursing homes)\footnote{Schedule 4 was amended by S.I.\ 1996/1516, 1999/2860 and 2000/440.}.
\end{enumerate}

\subsection[16. Termination of a reduced benefit decision]{Termination of a reduced benefit decision}

16.  A reduced benefit decision shall cease to be in force—
\begin{enumerate}\item[]
($a$) where the parent concerned—
\begin{enumerate}\item[]
(i) withdraws her request under section 6(5) of the Act;

(ii) complies with her obligation under section 6(7) of the Act; or

(iii) consents to take a scientific test (within the meaning of section 27A of the Act);
\end{enumerate}

($b$) where following written notice under section 46(6)($b$)  of the Act, the parent concerned responds to such notice and the Secretary of State considers there are reasonable grounds;

($c$) subject to regulation 13, where relevant benefit ceases to be payable to, or in respect of, the parent concerned; or

($d$) where a qualifying child with respect to whom a reduced benefit decision is in force applies for a maintenance calculation to be made with respect to him under section 7 of the Act and a calculation is made in response to that application in respect of all the qualifying children in relation to whom the parent concerned falls within section 46(1) of the Act.
\end{enumerate}

\subsection[17. Reduced benefit decisions where there is an additional qualifying child]{Reduced benefit decisions where there is an additional qualifying child}

17.---(1)  Where a reduced benefit decision is in operation, or would be in operation but for the provisions of regulations 14 and 15, and the Secretary of State gives a further reduced benefit decision with respect to the same parent concerned in relation to an additional qualifying child of whom she is a parent with care, the earlier reduced benefit decision shall cease to be in force.

(2) Where a further reduced benefit decision comes into operation in a case falling within paragraph (1), the provisions of regulation 11 shall apply to it.

(3) Where—
\begin{enumerate}\item[]
($a$) a reduced benefit decision (“the earlier decision”) has ceased to be in force by virtue of regulation 13(2); and

($b$) the Secretary of State gives a further reduced benefit decision (“the further decision”) with respect to the same parent concerned where that parent falls within section 46(1) of the Act,
\end{enumerate}
as long as the further decision remains in force, no additional reduced benefit decision shall be brought into force with respect to that parent in relation to one or more children to whom the earlier decision was given.

(4) Where a case falls within paragraph (1) or (3) and the further decision, but for the provisions of this paragraph, would cease to be in force by virtue of the provisions of regulation 16, but the earlier decision would not have ceased to be in force by virtue of the provisions of regulation 16, the further reduced benefit decision shall remain in force for a period calculated in accordance with regulation 11.

(5) In this regulation “additional qualifying child” means a qualifying child of whom the parent concerned is a parent with care and who was either not such a qualifying child at the time the earlier decision was given or had not been born at the time the earlier decision was given.

\subsection[18. Suspension and termination of a reduced benefit decision where the sole qualifying child ceases to be a child or where the parent concerned ceases to be a person with care]{\sloppy Suspension and termination of a reduced benefit decision where the sole qualifying child ceases to be a child or where the parent concerned ceases to be a person with care}

18.---(1)  Where a reduced benefit decision is in operation and—
\begin{enumerate}\item[]
($a$) there is, in relation to that decision, only one qualifying child, and that child ceases to be a child within the meaning of the Act; or

($b$) the parent concerned ceases to be a person with care,
\end{enumerate}
the decision shall be suspended from the last day of the benefit week during the course of which the child ceases to be a child within the meaning of the Act, or the parent concerned ceases to be a person with care, as the case may be.

(2) Where, under the provisions of paragraph (1), a decision has been suspended for a period of 52 weeks and no relevant benefit is payable at that time, it shall cease to be in force.

(3) If during the period specified in paragraph (2) the former child again becomes a child within the meaning of the Act or the parent concerned again becomes a person with care and relevant benefit is payable to, or in respect of, that parent, a reduction in the amount of that benefit shall be made in accordance with the provisions of paragraphs (3) to (6) of regulation 13.

\subsection[19. Notice of termination of a reduced benefit decision]{Notice of termination of a reduced benefit decision}

19.  Where a reduced benefit decision ceases to be in force under the provisions of regulation 16, 17 or 18 the Secretary of State shall serve notice of this on the parent concerned and shall specify the date on which the reduced benefit decision ceases to be in force.

\subsection[20. Rounding provisions]{Rounding provisions}

20.  Where any calculation made under this Part results in a fraction of a penny, that fraction shall be treated as a penny if it exceeds one half and shall otherwise be disregarded.

\section[Part V --- Miscellaneous provisions]{Part V\\*Miscellaneous provisions}

\renewcommand\parthead{--- Part V}

\subsection[21. Persons who are not persons with care]{Persons who are not persons with care}

21.---(1)  For the purposes of the Act the following categories of person shall not be persons with care—
\begin{enumerate}\item[]
($a$) a local authority;

($b$) a person with whom a child who is looked after by a local authority is placed by that authority under the provisions of the Children Act 1989\footnote{1989 c.\ 41.}, except where that person is a parent of such a child and the local authority allow the child to live with that parent under section 23(5) of that Act;

($c$) in Scotland, a family or relative with whom a child is placed by a local authority under the provisions of section 26 of the Children (Scotland) Act 1995\footnote{1995 c.\ 36.}.
\end{enumerate}

(2) In paragraph (1) above—
\begin{enumerate}\item[]
“family” means family other than such family defined in section 93(1) of the Children (Scotland) Act 1995;

\pagebreak[3]

“local authority” means, in relation to England, a county council, a district council, a London borough council, the Common Council of the City of London or the Council of the Isles of Scilly and, in relation to Wales, a county council or a county borough council, and, in relation to Scotland, a council constituted under section 2 of the Local Government etc (Scotland) Act 1994\footnote{1994 c.\ 39.}; and

“a child who is looked after by a local authority” has the same meaning as in section 22 of the Children Act 1989 or section 17(6) of the Children (Scotland) Act 1995 as the case may be.
\end{enumerate}

\subsection[22. Authorisation of representative]{Authorisation of representative}

22.---(1)  A person may authorise a representative, whether or not legally qualified, to receive notices and other documents on his behalf and to act on his behalf in relation to the making of applications and the supply of information under any provisions of the Act or these Regulations.

(2) Where a person has authorised a representative for the purposes of paragraph (1) who is not legally qualified, he shall confirm that authorisation in writing to the Secretary of State.

\section[Part VI --- Notifications following certain decisions]{Part VI\\*Notifications following certain decisions}

\renewcommand\parthead{--- Part VI}

\subsection[23. Notification of a maintenance calculation]{Notification of a maintenance calculation}

23.---(1)  A notification of a maintenance calculation made under section 11 or 12(2) of the Act (interim maintenance decision) shall set out, in relation to the maintenance calculation in question—
\begin{enumerate}\item[]
($a$) the effective date of the maintenance calculation;

($b$) where relevant, the non-resident parent’s net weekly income;

($c$) the number of qualifying children;

($d$) the number of relevant other children;

($e$) the weekly rate;

($f$) the amounts calculated in accordance with Part I of Schedule 1 to the Act and, where there has been agreement to a variation or a variation has otherwise been taken into account, the Child Support (Variations) Regulations 2000\footnote{S.I.\ 2001/156.};

($g$) where the weekly rate is adjusted by apportionment or shared care, or both, the amount calculated in accordance with paragraph 6, 7 or 8, as the case may be, of Part I of Schedule 1 to the Act; and

($h$) where the amount of child support maintenance which the non-resident parent is liable to pay is decreased in accordance with regulation 9 or 11 of the Maintenance Calculations and Special Cases Regulations (care provided in part by local authority and non-resident parent liable to pay maintenance under a maintenance order), the adjustment calculated in accordance with that regulation.
\end{enumerate}

(2) A notification of a maintenance calculation made under section 12(1) of the Act (default maintenance decision) shall set out the effective date of the maintenance calculation, the default rate, the number of qualifying children on which the rate is based, whether any apportionment has been applied under regulation 7 and shall state the nature of the information required to enable a decision under section 11 of the Act to be made by way of section 16 of the Act.

(3) Except where a person gives written permission to the Secretary of State that the information in relation to him, mentioned in sub-paragraphs ($a$)  and ($b$)  below, may be conveyed to other persons, any document given or sent under the provisions of paragraph (1) or (2) shall not contain—
\begin{enumerate}\item[]
($a$) the address of any person other than the recipient of the document in question (other than the address of the office of the officer concerned who is exercising functions of the Secretary of State under the Act) or any other information the use of which could reasonably be expected to lead to any such person being located;

($b$) any other information the use of which could reasonably be expected to lead to any person, other than a qualifying child or a relevant person, being identified.
\end{enumerate}

(4) Where a decision as to a maintenance calculation is made under section 11 or 12 of the Act, a notification under paragraph (1) or (2) shall include information as to the provisions of sections 16, 17 and 20 of the Act.

\subsection[24. Notification when an applicant under section 7 of the Act ceases to be a child]{Notification when an applicant under section 7 of the Act ceases to be a child}

24.  Where a maintenance calculation has been made in response to an application by a child under section 7 of the Act and that child ceases to be a child for the purposes of the Act, the Secretary of State shall immediately notify, so far as that is reasonably practicable—
\begin{enumerate}\item[]
($a$) the other qualifying children who have attained the age of 12 years and the non-resident parent with respect to whom that maintenance calculation was made; and

($b$) the person with care.
\end{enumerate}

\section[Part VII --- Effective dates of maintenance calculations]{Part VII\\*Effective dates of maintenance calculations}

\renewcommand\parthead{--- Part VII}

\subsection[25. Effective dates of maintenance calculations]{Effective dates of maintenance calculations}

25.---(1)  Subject to regulations 26 to 29
and 31%  % Words inserted (21.2.03) by SI 2003/328 reg 7(4)
, where no maintenance calculation is in force with respect to the person with care or the non-resident parent, the effective date of a maintenance calculation following an application made under section 4 or 7 of the Act, or treated as made under section 6(3) of the Act, as the case may be, shall be the date determined in accordance with paragraphs (2) to (4) below.

(2) Where the application for a maintenance calculation is made under section 4 of the Act by a non-resident parent, the effective date of the maintenance calculation shall be the date that an effective application is made or treated as made under regulation 3.

(3) Where the application for a maintenance calculation is—
\begin{enumerate}\item[]
($a$) made under section 4 of the Act by a person with care;

($b$) treated as made under section 6(3) of the Act; or

($c$) made by a child under section 7 of the Act,
\end{enumerate}
the effective date of the maintenance calculation shall be the date of notification to the non-resident parent.

(4) For the purposes of this regulation, where the Secretary of State is satisfied that a non-resident parent has intentionally avoided receipt of a notice of a maintenance application he may determine the date of notification to the non-resident parent as the date on which the notification would have been given to him but for such avoidance.

(5) Where in relation to a decision made under section 11 of the Act a maintenance calculation is made to which paragraph 15 of Schedule 1 to the Act applies, the effective date of the calculation shall be the beginning of the maintenance period in which the change of circumstance to which the calculation relates occurred or is expected to occur.

\amendment{
Words inserted in reg. 25(1) (21.2.03) by the Child Support (Miscellaneous Amendments) Regulations 2003 reg. 7(4).
}

\subsection[26. Effective dates of maintenance calculations—maintenance order and application under section 4 or 7]{Effective dates of maintenance calculations—maintenance order and application under section 4 or 7}

26.---(1)  This regulation applies, subject to regulation 28, where—
\begin{enumerate}\item[]
($a$) no maintenance calculation is in force with respect to the person with care or the non-resident parent;

($b$) an application for a maintenance calculation is made under section 4 or 7 of the Act; and

%($c$) there is a maintenance order in force, made on or after the date prescribed for the purposes of section 4(10)($a$)  of the Act, in relation to the person with care and the non-resident parent and that order has been in force for at least one year prior to the date the application for a maintenance calculation is made.

% Reg 26(1)(c) substituted (30.4.02) by SI 2002/1204 reg 6(4)
($c$) there is a maintenance order which—
\begin{enumerate}\item[]
(i) is in force and was made on or after the date prescribed for the purposes of section 4(10)($a$)  of the Act;

(ii) relates to the person with care, the non-resident parent and all the children to whom the application referred to in sub-paragraph ($b$)  relates; and

(iii) has been in force for at least one year prior to the date of the application referred to in sub-paragraph ($b$).
\end{enumerate}
\end{enumerate}

(2) The effective date of the maintenance calculation shall be two months and two days after the application is made.

\amendment{
Reg. 26(1)(c) substituted (30.4.02) by the Child Support (Miscellaneous Amendments) Regulations 2002 reg. 6(4).
}

\subsection[27. Effective dates of maintenance calculations—maintenance order and application under section 6]{Effective dates of maintenance calculations—maintenance order and application under section 6}

27.---(1)  This regulation applies, subject to regulation 28, where—
\begin{enumerate}\item[]
($a$) the circumstances set out in regulation 26(1)($a$)  apply;

($b$) an application for a maintenance calculation is treated as made under section 6(3) of the Act; and

($c$) there is a maintenance order in force in relation to the person with care% 
%and the non-resident parent
, the non-resident parent and all the children to whom the application referred to in sub-paragraph ($b$)  relates%  % Words substituted (30.4.02) by SI 2002/1204 reg 6(5)
.
\end{enumerate}

(2) The effective date of the maintenance calculation shall be 2 days after the maintenance calculation is made.

\amendment{
Words substituted in reg. 27(1)(c) (30.4.02) by the Child Support (Miscellaneous Amendments) Regulations 2002 reg. 6(5).
}

\subsection[28. Effective dates of maintenance calculations—maintenance order ceases]{Effective dates of maintenance calculations—maintenance order ceases}

28.  Where—
\begin{enumerate}\item[]
($a$) a maintenance calculation is made; and

($b$) there was a maintenance order in force in relation to the person with care and the non-resident parent which ceased to have effect after the date on which the application for the maintenance calculation was made but before the effective date provided for in regulation 
%25 or 26 
26 or 27  % Words substituted (30.4.02) by SI 2002/1204 reg 6(6)
as the case may be,
\end{enumerate}
the effective date of the maintenance calculation shall be the day following that on which the maintenance order ceased to have effect.

\amendment{
Words substituted in reg. 28(b) (30.4.02) by the Child Support (Miscellaneous Amendments) Regulations 2002 reg. 6(6).
}

\subsection[29. Effective dates of maintenance calculations in specified cases]{\sloppy Effective dates of maintenance calculations in specified cases}

29.%
---(1)  % Reg 29 renumbered as reg 29(1) (21.2.03) by SI 2003/328 reg 7(5)
  Where an application for a maintenance calculation is made under section 4 or 7 of the Act, or treated as made under section 6(3) of the Act—
\begin{enumerate}\item[]
($a$) except where the parent with care has made a request under section 6(5) of the Act, where in the period of 8 weeks immediately preceding the date the application is made, or treated as made under regulation 3, there has been in force a maintenance calculation in respect of the same non-resident parent and child but a different person with care, the effective date of the maintenance calculation made in respect of the application shall be %the day following the day 
the date  % Words substituted (21.2.03) by SI 2003/328 reg 7(5)(a)
on which the previous maintenance calculation ceased to have effect;

($b$) where a maintenance calculation (“the existing calculation”) is in force with respect to the person who is the person with care in relation to the application but who is the non-resident parent in relation to the existing calculation, the effective date of the calculation shall be a date not later than 7 days after the date of notification to the non-resident parent which is the day on which a maintenance period in respect of the existing calculation begins;

% Reg 29(c) added (30.4.02) by SI 2002/1204 reg 6(7)
($c$) except where the parent with care has made a request under section 6(5) of the Act, where—
\begin{enumerate}\item[]
(i) in the period of 8 weeks immediately preceding the date the application is made, or treated as made under regulation 3, a maintenance calculation (“the previous maintenance calculation”) has been in force and has ceased to have effect;

(ii) the parent with care in respect of the previous maintenance calculation is the non-resident parent in respect of the application;

(iii) the non-resident parent in respect of the previous maintenance calculation is the parent with care in respect of the application; and

(iv) the application relates to the same qualifying child, or all of the same qualifying children, and no others, as the previous maintenance calculation,
\end{enumerate}
the effective date of the maintenance calculation to which the application relates shall be the date on which the previous maintenance calculation ceased to have effect.
\end{enumerate}

% Reg 29(2), (3) added (21.2.03) by SI 2003/328 reg 7(5)(b)
(2) Where an application is treated as made under section 6(3) of the Act, references in sub-paragraphs ($a$)  and ($c$)  of paragraph (1) to “the date the application is made” shall mean whichever is the later of—
\begin{enumerate}\item[]
($a$) the date of the claim for a prescribed benefit made by or in respect of the parent with care, as determined by regulation 6 of the Social Security (Claims and Payments) Regulations 1987\footnote{S.I.\ 1987/1968. Regulation 6 was amended by S.I.\ 1988/522, 1989/1686, 1990/725 and 2208, 1991/2284 and 2741, 1993/2113 and 2319, 1996/1460 and 2431, 1997/793, 1999/2572 and 3108, 2000/636, 897 and 1982, and 2001/567 and 892.}; and

($b$) the date on which the parent with care or her partner in the claim reports to the Secretary of State (in respect of a claim for a prescribed benefit) or to the Commissioners of Inland Revenue (in respect of a claim for a tax credit) a change of circumstances, which change—
\begin{enumerate}\item[]
(i) relates to an existing claim, in respect of the parent with care, for a prescribed benefit; and

(ii) has the effect that the parent with care is treated as applying for a maintenance calculation under section 6(1) of the Act (whether or not that section already applied to that parent with care).
\end{enumerate}
\end{enumerate}

(3) For the purposes of—
\begin{enumerate}\item[]
($a$) paragraph (1), “ceased to have effect” means ceased to have effect under paragraph 16 of Schedule 1 to the Act\footnote{\emph{See} the Child Support Act 1991 (c.\ 48); paragraph 16 of Schedule 1 was amended by Schedule 9 to the Child Support, Pensions and Social Security Act 2000.}; and

($b$) paragraph (2), “prescribed benefit” means a benefit referred to in section 6(1) of the Act or prescribed in regulations made under that section.
\end{enumerate}

\amendment{
Reg. 29(c) added (30.4.02) by the Child Support (Miscellaneous Amendments) Regulations 2002 reg. 6(7).

Reg. 29 renumbered as reg. 29(1), words substituted in reg. 29(1)(a) and reg. 29(2), (3) added (21.2.03) by the Child Support (Miscellaneous Amendments) Regulations 2003 reg. 7(5).
}

\section[Part VIII --- Revocation, savings and transitional provisions]{Part VIII\\*Revocation, savings and transitional provisions}

\renewcommand\parthead{--- Part VIII}

\subsection[30. Revocation and savings]{Revocation and savings}

30.---(1)  Subject to 
the Child Support (Transitional Provisions) Regulations 2000 and % Words inserted (3.3.03) by SI 2003/347 reg 2(1), (2)(a)
paragraph (2), the Child Support (Maintenance Assessment Procedure) Regulations 1992\footnote{S.I.\ 1992/1813.} shall be revoked with respect to a particular case with effect from the date that these Regulations come into force with respect to that type of case (“the commencement date”).

(2) Subject to 
%regulation 31(2)
regulation 31(1C)($b$)  and (2)%  % Words substituted (21.2.03) by SI 2003/328 reg 7(6)
, where before the commencement date in respect of a particular case—
\begin{enumerate}\item[]
($a$) an application was made and not determined for—
\begin{enumerate}\item[]
(i) a maintenance assessment;

(ii) a departure direction; or

(iii) a revision or supersession of a decision;
\end{enumerate}

($b$) the Secretary of State had begun but not completed a revision or supersession of a decision on his own initiative;

($c$) any time limit provided for in Regulations for making an application for a revision or a departure direction had not expired; or

($d$) any appeal was made but not decided or any time limit for making an appeal had not expired,
\end{enumerate}
the provisions of the Child Support (Maintenance Assessment Procedure) Regulations 1992 shall continue to apply for the purposes of—
\begin{enumerate}\item[]
($aa$) the decision on the application referred to in sub-paragraph ($a$);

($bb$) the revision or supersession referred to in sub-paragraph ($b$);

($cc$) the ability to apply for the revision or the departure direction referred to in sub-paragraph ($c$)  and the decision whether to revise or to give a departure direction following any such application;

($dd$) any appeal outstanding or made during the time limit referred to in sub-paragraph ($d$); or

($ee$) any revision, supersession, appeal or application for a departure direction in relation to a decision, ability to apply or appeal referred to in sub-paragraphs ($aa$)  to ($dd$)  above.
\end{enumerate}

(3) Where immediately before the commencement date in respect of a particular case an interim maintenance assessment was in force, the provisions of the Child Support (Maintenance Assessment Procedure) Regulations 1992 shall continue to apply for the purposes of the decision under section 17 of the Act to make a maintenance assessment calculated in accordance with Part I of Schedule 1 to the 1991 Act before its amendment by the 2000 Act and any revision, supersession or appeal in relation to that decision.

(4) Where after the commencement date a maintenance assessment is revised, cancelled or ceases to have effect from a date which is prior to the commencement date, the Child Support (Maintenance Assessment Procedure) Regulations 1992 shall apply for the purposes of that cancellation or cessation.

(5) Where under regulation 28(1) of the Child Support (Transitional Provisions) Regulations 2000\footnote{S.I.\ 2000/3186.} an application for a maintenance calculation is treated as an application for a maintenance assessment, the provisions of the Child Support (Maintenance Assessment Procedure) Regulations 1992 shall continue to apply for the purposes of the determination of the application and any revision, supersession or appeal in relation to any such assessment made.

(6) For the purposes of this regulation—
\begin{enumerate}\item[]
($a$) “departure direction”, “maintenance assessment” and “interim maintenance assessment” have the same meaning as in section 54 of the Act before its amendment by the 2000 Act;

($b$) “revision or supersession” means a revision or supersession of a decision under section 16 or 17 of the Act before their amendment by the 2000 Act;

($c$) “2000 Act” means the Child Support, Pensions and Social Security Act 2000.
\end{enumerate}

\amendment{
Words substituted in reg. 30(2) (21.2.03) by the Child Support (Miscellaneous Amendments) Regulations 2003 reg. 7(6).

Words inserted in reg. 30(1) (3.3.03) by the Child Support (Transitional Provision) (Miscellaneous Amendments) Regulations 2003 reg. 2(1), (2)(b).
}

\subsection[31. Transitional provision—effective dates and reduced benefit decisions]{Transitional provision—effective dates and reduced benefit decisions}

31.---%(1)  Where a maintenance assessment is in force with respect to a non-resident parent or a parent with care and an application for a maintenance calculation is made to which regulation 29 applies, that regulation shall apply as if references to a maintenance calculation in force were to a maintenance assessment in force.
%
%(2) Where—
%\begin{enumerate}\item[]
%($a$) the application for a maintenance assessment was made before the date prescribed for the purposes of section 4(10)($a$)  of the Act; and
%
%($b$) the effective date of the maintenance assessment, if it were a maintenance assessment to which the Assessment Procedure Regulations applied (“the assessment effective date”) would be later than the effective date provided for in these Regulations,
%\end{enumerate}
%the application shall be treated as an application for a maintenance calculation and the effective date of that maintenance calculation shall be the assessment effective date.
%
% Reg 31(1)--(2) substituted (21.2.03) by SI 2003/328 reg 7(7)(a)
(1) Where a maintenance assessment is, or has been, in force and an application to which regulation 29 applies is made, or is treated as made under section 6(3) of the Act, that regulation shall apply as if in paragraph (1) references to—
\begin{enumerate}\item[]
($a$) a maintenance calculation in force were to a maintenance assessment in force;

($b$) a maintenance calculation having been in force were to a maintenance assessment having been in force; and

($c$) a non-resident parent in sub-paragraph ($a$), the first time it occurs in sub-paragraph ($b$)  and in sub-paragraph ($c$)(iii), were to an absent parent.
\end{enumerate}

(1A) Where regulation 28(7) of the Child Support (Transitional Provisions) Regulations 2000 (linking provisions) applies, the effective date of the maintenance calculation shall be the date which would have been the beginning of the first maintenance period in respect of the conversion decision on or after what, but for this paragraph, would have been the relevant effective date provided for in regulation 25(2) to (4).

(1B) The provisions of Schedule 3 shall apply where—
\begin{enumerate}\item[]
($a$) an effective application for a maintenance assessment has been made under the former Act (“an assessment application”); and

($b$) an effective application for a maintenance calculation is made or an application for a maintenance calculation is treated as made under the Act (“a calculation application”).
\end{enumerate}

(1C) Where the provisions of Schedule 3 apply and, by virtue of regulation 4(3) of the Assessment Procedure Regulations, the relevant date would be—
\begin{enumerate}\item[]
($a$) before the prescribed date, the application to be proceeded with shall be treated as an application for a maintenance assessment;

($b$) on or after the prescribed date, that application shall be treated as an application for a maintenance calculation and the effective date of that maintenance calculation shall be the date which would be the assessment effective date if a maintenance assessment were to be made.
\end{enumerate}

(2) Where—
\begin{enumerate}\item[]
($a$) an application for a maintenance assessment was made before the prescribed date; and

($b$) the assessment effective date of that application would be on or after the prescribed date,
\end{enumerate}
the application shall be treated as an application for a maintenance calculation and the effective date of that maintenance calculation shall be the date which would be the assessment effective date if a maintenance assessment were to be made.

(3) Paragraphs (4) to (7) shall apply where, 
%on or before 
immediately before  % Words substituted (30.4.02) by SI 2002/1204 reg 6(8)
the commencement date, section 6 of the former Act applied to the parent with care.

(4) 
%Where a maintenance assessment was made with an effective date, applying the Assessment Procedure Regulations, or the Maintenance Arrangements and Jurisdiction Regulations, which 
Where the assessment effective date  % Words substituted (21.2.03) by SI 2003/328 reg 7(7)(b)
is before the prescribed date and on or after the commencement date the parent with care notifies the Secretary of State that she is withdrawing her authorisation under subsection (1) of that section, these Regulations shall apply as if the notification were a request not to act under section 6(5) of the Act.

(5) Where a maintenance assessment was not made because section 6(2) of the former Act applied, these Regulations shall apply as if section 6(5) of the Act applied.

(6) Where a maintenance assessment was not made, section 6(2) of the former Act did not apply and a reduced benefit direction was given under section 46(5) of the former Act, these Regulations shall apply as if the reduced benefit direction were a reduced benefit decision made under section 46(5) of the Act, from the same date and with the same effect as the reduced benefit direction.

(7) Where a maintenance assessment was not made, the parent with care failed to comply with a requirement imposed on her under section 6(1) of the former Act and the Secretary of State was in the process of serving a notice or considering reasons given by the parent with care under section 46(2) or (3) of the former Act, these Regulations shall apply as if the Secretary of State was in the process of serving a notice or considering reasons under section 46(2) or (3) of the Act.

(8) For the purposes of this regulation—
\begin{enumerate}\item[]($a$) “2000 Act” means the Child Support, Pensions and Social Security Act 2000;

% Definitions of ``absent parent'', ``assessment effective date'' inserted in reg. 31(8)(a) (21.2.03) by SI 2003/328 reg 7(7)(c)(i)
“absent parent” has the meaning given in section 3(2) of the former Act;

“assessment effective date” means the effective date of the maintenance assessment under regulation 30 or 33(7) of the Assessment Procedure Regulations\footnote{Regulation 30 was amended by S.I.\ 1995/123, 1045 and 3261, 1996/1945 and 1999/1047 and is revoked, with savings, by S.I.\ 2001/157. Regulation 33(7) was inserted by S.I.\ 1995/3261 and is revoked, with savings, by S.I.\ 2001/157.} or regulation 3(5), (7) or (8) of the Maintenance Arrangements and Jurisdiction Regulations\footnote{Regulation 3 was amended by S.I.\ 1995/123, 1045 and 3261 and 1999/1510 (C.\ 43) and is amended by S.I.\ 2001/161. Paragraphs (5) to (8) are omitted, with savings, by S.I.\ 2001/161.}, whichever applied to the maintenance assessment in question or would have applied had the effective date not been determined under regulation 8C or 30A of the Assessment Procedure Regulations;

“Assessment Procedure Regulations” means the Child Support (Maintenance Assessment Procedure) Regulations 1992\footnote{S.I.\ 1992/1813.};

“commencement date” means with respect to a particular case the date these Regulations come into force with respect to that type of case;

“former Act” means the Act before its amendment by the 2000 Act;

“Maintenance Arrangements and Jurisdiction Regulations” means the Child Support (Maintenance Arrangements and Jurisdiction) Regulations 1992\footnote{S.I.\ 1992/2645.};

“maintenance assessment” has the meaning given in the former Act; and

“prescribed date” means the date prescribed for the purposes of section 4(10)($a$)  of the Act;
%
% Definition of ``relevant date'' inserted (21.2.03) by SI 2003/328 reg 7(7)(c)(ii)
and

    “relevant date” means the date which would be the assessment effective date of the application which is to be proceeded with in accordance with Schedule 3, if a maintenance assessment were to be made;

($b$) references in paragraphs (4) to (7) to sections 6(5), 46(5) and 46(2) and (3) of the Act mean those provisions as substituted by the 2000 Act; and

($c$) in the application of the Assessment Procedure Regulations for the purposes of paragraph (4) where, on or after the prescribed date, no maintenance enquiry form, as defined in those Regulations, is given or sent to the absent parent, the Regulations shall be applied as if references in regulation 30—
\begin{enumerate}\item[]
(i) to the date when the maintenance enquiry form was given or sent to the absent parent were to the date of notification to the non-resident parent;

(ii) to the return by the absent parent of the maintenance enquiry form containing his name, address and written confirmation that he is the parent of the child or children in respect of whom the application was made were to the provision of this information by the non-resident parent; and
\end{enumerate}

%\enlargethispage{\baselineskip}
\pagebreak[3]

($d$) in the application of the Maintenance Arrangements and Jurisdiction Regulations for the purposes of paragraph (4), where, on or after the prescribed date no maintenance enquiry form, as defined in the Assessment Procedure Regulations, is given or sent to the absent parent, regulation 3(8) shall be applied as if the reference to the date when the maintenance enquiry form was given or sent were a reference to the date of notification to the non-resident parent.
\end{enumerate}

\amendment{
Words substituted in reg. 31(3) (30.4.02) by the Child Support (Miscellaneous Amendments) Regulations 2002 reg. 6(8).

Words substituted in reg. 31(4), definitions of ``absent parent'', ``assessment effective date'' and ``relevant date'' inserted in reg. 31(8)(a) and reg. 31(1)--(2) substituted (21.2.03) by the Child Support (Miscellaneous Amendments) Regulations 2003 reg. 7(7).
}

\bigskip

Signed 
by authority of the Secretary of State for Social Security.

{\raggedleft
\emph{P.~Hollis}\\*Parliamentary Under-Secretary of State,\\*Department of Social Security

}

18th January 2001

\small

\part[Schedule 1 --- Meaning of “child” for the purposes of the Act]{Schedule 1\\*Meaning of “child” for the purposes of the Act}

\renewcommand\parthead{--- Schedule 1}

\subsection*{Persons of 16 or 17 years of age who are not in full-time non-advanced education}

1.---(1)  Subject to sub-paragraph (3), the conditions which must be satisfied for a person to be a child within section 55(1)($c$)  of the Act are—
\begin{enumerate}\item[]
($a$) the person is registered for work or for training under work-based training for young people or, in Scotland, Skillseekers training with—
\begin{enumerate}\item[]
(i) the Department for Education and Employment;

(ii) the Ministry of Defence;

(iii) in England and Wales, a local education authority within the meaning of the Education Acts 1944 to 1992;

\pagebreak[3]

(iv) in Scotland, an education authority within the meaning of section 135(1) of the Education (Scotland) Act 1980\footnote{1980 c.\ 44.} (interpretation); or

(v) for the purposes of applying Council Regulation (EEC) No.\ 1408/\hspace{0pt}71\footnote{\frenchspacing O.J. No. L149, 5.7.1971; Regulations (EEC) No. 1408/71 and No. 574/72 were restated in amended form in Council Regulation (EEC) No. 2001/83 (O.J. No. L230, 22.8.1983) and further amended by Council Regulations (EEC) Nos. 1660/85 (O.J. No. L160, 20.6.1985); 1661/85 (O.J. No. L160, 20.6.1985) and 3811/86 (O.J. No. L355, 16.12.86); Commission Regulation (EEC) No. 513/86 (O.J. No. L51, 28.2.1986) and Articles 60 and 220 of, and Point I, Part VIII of Annex I to the Act of Accession to the European Communities of Spain and Portugal.}, any corresponding body in another member State;
\end{enumerate}

($b$) the person is not engaged in remunerative work, other than work of a temporary nature that is due to cease before the end of the extension period which applies in the case of that person;

($c$) the extension period which applies in the case of that person has not expired; and

($d$) immediately before the extension period begins, the person is a child for the purposes of the Act without regard to this paragraph.
\end{enumerate}

(2) For the purposes of heads ($b$), ($c$)  and ($d$)  of sub-paragraph (1), the extension period—
\begin{enumerate}\item[]
($a$) begins on the first day of the week in which the person would no longer be a child for the purposes of the Act but for this paragraph; and

($b$) where a person ceases to fall within section 55(1)($a$)  of the Act or within paragraph 5—
\begin{enumerate}\item[]
(i) on or after the first Monday in September, but before the first Monday in January of the following year, ends on the last day of the week which falls immediately before the week which includes the first Monday in January in that year;

(ii) on or after the first Monday in January but before the Monday following Easter Monday in that year, ends on the last day of the week which falls 12 weeks after the week which includes the first Monday in January in that year;

(iii) at any other time of the year, ends on the last day of the week which falls 12 weeks after the week which includes the Monday following Easter Monday in that year.
\end{enumerate}
\end{enumerate}

(3) A person shall not be a child for the purposes of the Act under this paragraph if—
\begin{enumerate}\item[]
($a$) he is engaged in training under work-based training for young people or, in Scotland, Skillseekers training; or

($b$) he is entitled to income support or an income-based jobseeker’s allowance.
\end{enumerate}

\subsection*{Meaning of “advanced education” for the purposes of section 55 of the Act}

2.  For the purposes of section 55 of the Act “advanced education” means education of the following description—
\begin{enumerate}\item[]
($a$) a course in preparation for a degree, a Diploma of Higher Education, a higher national diploma, a higher national diploma or higher national certificate of the Business and Technology Education Council or the Scottish Qualifications Council or a teaching qualification; or

($b$) any other course which is of a standard above that of an ordinary national diploma, a national diploma or a national certificate of the Business and Technology Education Council or the Scottish Qualifications Authority, the advanced level of the General Certificate of Education, a Scottish certificate of education (higher level), a Scottish certificate of sixth year studies, or a Scottish National Qualification at Higher Level.
\end{enumerate}

\subsection*{Circumstances in which education is to be treated as full-time education}

3.  For the purposes of section 55 of the Act education shall be treated as being full-time if it is received by a person attending a course of education at a recognised educational establishment and the time spent receiving instruction or tuition, undertaking supervised study, examination of practical work or taking part in any exercise, experiment or project for which provision is made in the curriculum of the course, exceeds 12 hours per week, so however that in calculating the time spent in pursuit of the course, no account shall be taken of time occupied by meal breaks or spent on unsupervised study, whether undertaken on or off the premises of the educational establishment.

\subsection*{Interruption of full-time education}

4.---(1)  Subject to sub-paragraph (2), in determining whether a person falls within section 55(1)($b$)  of the Act no account shall be taken of a period (whether beginning before or after the person concerned attains age 16) of up to 6 months of any interruption to the extent to which it is accepted that the interruption is attributable to a cause which is reasonable in the particular circumstances of the case; and where the interruption or its continuance is attributable to the illness or disability of mind or body of the person concerned, the period of 6 months may be extended for such further period as the Secretary of State considers reasonable in the particular circumstances of the case.

(2) The provisions of sub-paragraph (1) shall not apply to any period of interruption of a person’s full-time education which is likely to be followed immediately or which is followed immediately by a period during which—
\begin{enumerate}\item[]
($a$) provision is made for the training of that person, and for an allowance to be payable to that person, under work-based training for young people or, in Scotland, Skillseekers training; or

($b$) he is receiving education by virtue of his employment or of any office held by him.
\end{enumerate}

\subsection*{Circumstances in which a person who has ceased to receive full-time education is to be treated as continuing to fall within section 55(1) of the Act}

5.---(1)  Subject to sub-paragraphs (2) and (5), a person who has ceased to receive full-time education (which is not advanced education) shall, if—
\begin{enumerate}\item[]
($a$) he is under the age of 16 when he so ceases, from the date on which he attains that age; or

($b$) he is 16 or over when he so ceases, from the date on which he so ceases,
\end{enumerate}
be treated as continuing to fall within section 55(1) of the Act up to and including the week including the terminal date, or if he attains the age of 19 on or before that date, up to and including the week including the last Monday before he attains that age.

(2) In the case of a person specified in sub-paragraph (1)($a$)  or ($b$)  who had not attained the upper limit of compulsory school age when he ceased to receive full-time education, the terminal date shall be that specified in head ($a$), ($b$)  or ($c$)  of sub-paragraph (3), whichever next follows the date on which he would have attained that age.

(3) In this paragraph the “terminal date” means—
\begin{enumerate}\item[]
($a$) the first Monday in January; or

($b$) the Monday following Easter Monday; or

($c$) the first Monday in September,
\end{enumerate}
whichever first occurs after the date on which the person’s said education ceased.

(4) In this paragraph “compulsory school age” means—
\begin{enumerate}\item[]
($a$) in England and Wales, compulsory school age as determined in accordance with section 9 of the Education Act 1962\footnote{10 \& 11 Eliz.\ 2 c.\ 12 as amended by the Education (School-leaving Dates) Act 1976 (c.\ 5).};

($b$) in Scotland, school age as determined in accordance with sections 31 and 33 of the Education (Scotland) Act 1980\footnote{1980 c.\ 44.}.
\end{enumerate}

(5) A person shall not be treated as continuing to fall within section 55(1) of the Act under this paragraph if he is engaged in remunerative work, other than work of a temporary nature that is due to cease before the terminal date.

(6) Subject to sub-paragraphs (5) and (8), a person whose name was entered as a candidate for any external examination in connection with full-time education (which is not advanced education), which he was receiving at the time, shall so long as his name continued to be so entered before ceasing to receive such education be treated as continuing to fall within section 55(1) of the Act for any week in the period specified in sub-paragraph (7).

(7) Subject to sub-paragraph (8), the period specified for the purposes of sub-paragraph (6) is the period beginning with the date when that person ceased to receive such education ending with—
\begin{enumerate}\item[]
($a$) whichever of the dates in sub-paragraph (3) first occurs after the conclusion of the examination (or the last of them, if there is more than one); or

($b$) the expiry of the week which includes the last Monday before his 19th birthday,
\end{enumerate}
whichever is the earlier.

(8) The period specified in sub-paragraph (7) shall, in the case of a person who had not attained the age of 16 when he so ceased, begin with the date on which he did attain that age.

\subsection*{Interpretation}

6.  In this Schedule—
\begin{enumerate}\item[]
    “Education Acts 1944 to 1992” has the meaning prescribed in section 94(2) of the Further and Higher Education Act 1992\footnote{1992 c. 13.};

    “remunerative work” means work of not less than 24 hours a week—
\begin{enumerate}\item[]
    ($a$) 
    in respect of which payment is made; or

    ($b$) 
    which is done in expectation of payment;
\end{enumerate}

    “week” means a period of 7 days beginning with a Monday;

    “work-based training for young people or, in Scotland, Skillseekers training” means—
\begin{enumerate}\item[]
    ($a$) 
    arrangements made under section 2 of the Employment and Training Act 1973\footnote{1973 c.\ 50; section 2 is substituted by the Employment Act 1988 (c.\ 19), section 25(1).} (functions of the Secretary of State) or section 2 of the Enterprise and New Towns (Scotland) Act 1990\footnote{1990 c.\ 35.};

    ($b$) 
    arrangements made by the Secretary of State for the persons enlisted in Her Majesty’s forces for any special term of service specified in regulations made under section 2 of the Armed Forces Act 1966\footnote{1996 c.\ 45; section 2 was amended by section 2 of the Army Act 1992 (c.\ 39).} (power of Defence Council to make regulations as to engagement of persons in regular forces); or

    ($c$) 
    for the purposes of the application of Council Regulation (EEC) No.\ 1408/71, any corresponding provisions operated in another member State, for purposes which include the training of persons who, at the beginning of their training, are under the age of 18. 
\end{enumerate}
\end{enumerate}

\part[Schedule 2 --- Multiple applications]{Schedule 2\\*Multiple applications}

\renewcommand\parthead{--- Schedule 2}

\subsection*{No maintenance calculation in force: more than one application for a maintenance calculation by the same person under section 4 or 6 or under sections 4 and 6 of the Act}

1.---(1)  Where an effective application is made or treated as made, as the case may be, for a maintenance calculation under section 4 or 6 of the Act and, before that calculation is made, the applicant makes a subsequent effective application under that section with respect to the same non-resident parent or person with care, as the case may be, those applications shall be treated as a single application.

(2) Where an effective application for a maintenance calculation is made, or treated as made, as the case may be, by a person with care—
\begin{enumerate}\item[]
($a$) under section 4 of the Act; or

($b$) under section 6 of the Act,
\end{enumerate}
and, before that maintenance calculation is made, the person with care—
\begin{enumerate}\item[]
(i) in a case falling within head ($a$), is treated as making an application under section 6 of the Act; or

(ii) in a case falling within head ($b$), makes a subsequent effective application under section 4 of the Act,
\end{enumerate}
with respect to the same non-resident parent, those applications shall, if the person with care does not cease to fall within section 6(1) of the Act, be treated as a single application under section 6 of the Act, and shall otherwise be treated as a single application under section 4 of the Act.

\subsection*{No maintenance calculation in force: more than one application by a child under section 7 of the Act}

2.  Where a child makes an effective application for a maintenance calculation under section 7 of the Act and, before that calculation is made, makes a subsequent effective application under that section with respect to the same person with care and non-resident parent, both applications shall be treated as a single application for a maintenance calculation.

\subsection*{No maintenance calculation in force: applications by different persons for a maintenance calculation}

3.---(1)  Where the Secretary of State receives more than one effective application for a maintenance calculation with respect to the same person with care and non-resident parent, he shall, if no maintenance calculation has been made in relation to any of the applications, determine which application he shall proceed with in accordance with sub-paragraphs (2) to (11).

(2) Where an application by a person with care is made under section 4 of the Act or is treated as made under section 6 of the Act, and an application is made by a non-resident parent under section 4 of the Act, the Secretary of State shall proceed with the application of the person with care.

(3) Where there is an application for a maintenance calculation by a qualifying child under section 7 of the Act and a subsequent application is made with respect to that child by a person who is, with respect to that child, a person with care or a non-resident parent, the Secretary of State shall proceed with the application of that person with care or non-resident parent, as the case may be.

(4) Where, in a case falling within sub-paragraph (3), there is made more than one subsequent application, the Secretary of State shall apply the provisions of sub-paragraphs (2), (7), (8), or (10), as is appropriate in the circumstances of the case, to determine which application he shall proceed with.

(5) Where there is an application for a maintenance calculation by more than one qualifying child under section 7 of the Act in relation to the same person with care and non-resident parent, the Secretary of State shall proceed with the application of the elder or, as the case may be, eldest of the qualifying children.

(6) Where there are two non-resident parents in respect of the same qualifying child and an effective application is received from each such person, the Secretary of State shall proceed with both applications, treating them as a single application for a maintenance calculation.

(7) Where an application is treated as having been made by a parent with care under section 6 of the Act and there is an application under section 4 of the Act by another person with care who has parental responsibility for (or, in Scotland, parental rights over) the qualifying child or qualifying children with respect to whom the application under section 6 of the Act was treated as made, the Secretary of State shall proceed with the application under section 6 of the Act by the parent with care.

(8) Where—
\begin{enumerate}\item[]
($a$) more than one person with care makes an application for a maintenance calculation under section 4 of the Act in respect of the same qualifying child or qualifying children (whether or not any of those applications is also in respect of other qualifying children);

($b$) each such person has parental responsibility for (or, in Scotland, parental rights over) that child or children; and

($c$) under the provisions of regulation 8 of the Maintenance Calculations and Special Cases Regulations one of those persons is to be treated as a non-resident parent,
\end{enumerate}
the Secretary of State shall proceed with the application of the person who does not fall to be treated as a non-resident parent under the provisions of regulation 8 of those Regulations.

(9) Where, in a case falling within sub-paragraph (8), there is more than one person who does not fall to be treated as a non-resident parent under the provisions of regulation 8 of those Regulations, the Secretary of State shall apply the provisions of paragraph (10) to determine which application he shall proceed with.

(10) Where—
\begin{enumerate}\item[]
($a$) more than one person with care makes an application for a maintenance calculation under section 4 of the Act in respect of the same qualifying child or qualifying children (whether or not any of those applications is also in respect of other qualifying children); and

($b$) either—
\begin{enumerate}\item[]
(i) none of those persons has parental responsibility for (or, in Scotland, parental rights over) that child or children; or

(ii) the case falls within sub-paragraph (8)($b$)  but the Secretary of State has not been able to determine which application he is to proceed with under the provisions of sub-paragraph (8),
\end{enumerate}
\end{enumerate}
the Secretary of State shall proceed with the application of the principal provider of day to day care, as determined in accordance with sub-paragraph (11).

(11) Where—
\begin{enumerate}\item[]
($a$) the applications are in respect of one qualifying child, the application of that person with care to whom child benefit is paid in respect of that child;

($b$) the applications are in respect of more than one qualifying child, the application of that person with care to whom child benefit is paid in respect of those children;

($c$) the Secretary of State cannot determine which application he is to proceed with under head ($a$)  or ($b$)  the application of that applicant who in the opinion of the Secretary of State is the principal provider of day to day care for the child or children in question.
\end{enumerate}

(12) Subject to sub-paragraph (13), where, in any case falling within sub-paragraphs (2) to (10), the applications are not in respect of identical qualifying children, the application that the Secretary of State is to proceed with as determined by those sub-paragraphs shall be treated as an application with respect to all of the qualifying children with respect to whom the applications were made.

(13) Where the Secretary of State is satisfied that the same person with care does not provide the principal day to day care for all of the qualifying children with respect to whom an application would but for the provisions of this paragraph be made under sub-paragraph (12), he shall make separate maintenance calculations in relation to each person with care providing such principal day to day care.

(14) For the purposes of this paragraph “day to day care” has the same meaning as in the Maintenance Calculations and Special Cases Regulations.

\subsection*{Maintenance calculation in force: subsequent application with respect to the same persons}

4.  Where a maintenance calculation is in force and a subsequent application is made or treated as made, as the case may be, under the same section of the Act for a maintenance calculation with respect to the same person with care, non-resident parent, and qualifying child or qualifying children as those with respect to whom the maintenance calculation in force has been made, that application shall not be proceeded with. 

% Sch 3 inserted (21.2.03) by SI 2003/328 reg 7(8) and Sch
\part[Schedule 3 --- Multiple applications---transitional provisions]{Schedule 3\\*Multiple applications---transitional provisions}

\renewcommand\parthead{--- Schedule 3}

\amendment{
Sch. 3 inserted (21.2.03) by the Child Support (Miscellaneous Amendments) Regulations 2003 reg. 7(8), Sch.
}

\section*{\itshape No maintenance assessment or calculation in force: more than one application for maintenance by the same person under section 4 or 6, or under sections 4 and 6, of the former Act and of the Act.}

1.---(1)  Where an assessment application is made and, before a maintenance assessment under the former Act is made, the applicant makes or is treated as making, as the case may be, a calculation application under section 4 or 6 of the Act, with respect to the same person with care or with respect to a non-resident parent who is the absent parent with respect to the assessment application, as the case may be, those applications shall be treated as a single application.

(2) Where an assessment application is made by a person with care—
\begin{enumerate}\item[]
($a$) under section 4 of the former Act; or

($b$) under section 6(1) of the former Act,
\end{enumerate}
and, before a maintenance assessment under the former Act is made, the person with care—
\begin{enumerate}\item[]
(i) in a case falling within head ($a$), is treated as making a calculation application under section 6(1) of the Act; or

(ii) in a case falling within head ($b$), makes a calculation application under section 4 of the Act,
\end{enumerate}
with respect to a non-resident parent who is the absent parent with respect to the assessment application, those applications shall, if the person with care does not cease to fall within section 6(1) of the Act, be treated as a single application under section 6(1) of the former Act or of the Act, as the case may be, and shall otherwise be treated as a single application under section 4 of the former Act or of the Act, as the case may be.

\section*{\itshape No maintenance assessment or calculation in force: more than one application for maintenance by a child under section 7 of the former Act and of the Act}

2.  Where a child makes an assessment application under section 7 of the former Act and, before a maintenance assessment under the former Act is made, makes a calculation application under section 7 of the Act with respect to the same person with care and a non-resident parent who is the absent parent with respect to the assessment application, both applications shall be treated as a single application.

\section*{\itshape No maintenance assessment or calculation in force: applications by different persons for maintenance}

3.---(1)  Where the Secretary of State receives more than one application for maintenance with respect to the same person with care and absent parent or non-resident parent, as the case may be, he shall, if no maintenance assessment under the former Act or maintenance calculation under the Act, as the case may be, has been made in relation to any of the applications, determine which application he shall proceed with in accordance with sub-paragraphs (2) to (11).

(2) Where an application by a person with care is made under section 4 of the former Act or of the Act, or is made under section 6 of the former Act, or is treated as made under section 6 of the Act, and an application is made by an absent parent or non-resident parent under section 4 of the former Act or of the Act, as the case may be, the Secretary of State shall proceed with the application of the person with care.

(3) Where there is an assessment application by a qualifying child under section 7 of the former Act and a calculation application is made with respect to that child by a person who is, with respect to that child, a person with care or a non-resident parent, the Secretary of State shall proceed with the application of that person with care or non-resident parent, as the case may be.

(4) Where, in a case falling within sub-paragraph (3), there is made more than one subsequent application, the Secretary of State shall apply the provisions of sub-paragraphs (2), (7), (8) or (10), as appropriate in the circumstances of the case, to determine which application he shall proceed with.

(5) Where there is an assessment application and a calculation application by more than one qualifying child under section 7 of the former Act or of the Act, in relation to the same person with care and absent parent or non-resident parent, as the case may be, the Secretary of State shall proceed with the application of the elder or, as the case may be, eldest of the qualifying children.

(6) Where there is one absent parent and one non-resident parent in respect of the same qualifying child and an assessment application and a calculation application is received from each such person respectively, the Secretary of State shall proceed with both applications, treating them as a single application.

(7) Where a parent with care is required to authorise the Secretary of State to recover child support maintenance under section 6 of the former Act and there is a calculation application under section 4 of the Act by another person with care who has parental responsibility for (or, in Scotland, parental rights over) the qualifying child or qualifying children with respect to whom the application was made under section 6 of the former Act, the Secretary of State shall proceed with the assessment application under section 6 of the former Act by the parent with care.

(8) Where—
\begin{enumerate}\item[]
($a$) a person with care makes an assessment application under section 4 of the former Act and a different person with care makes a calculation application under section 4 of the Act and those applications are in respect of the same qualifying child or qualifying children (whether or not any of those applications is also in respect of other qualifying children);

($b$) each such person has parental responsibility for (or, in Scotland, parental rights over) that child or children; and

($c$) under regulation 20 of the Child Support (Maintenance Assessments and Special Cases) Regulations 1992 (“the Maintenance Assessments and Special Cases Regulations”) one of those persons is to be treated as an absent parent or under the provisions of regulation 8 of the Maintenance Calculations and Special Cases Regulations one of those persons is to be treated as a non-resident parent, as the case may be,
\end{enumerate}
the Secretary of State shall proceed with the application of the person who does not fall to be treated as an absent parent under regulation 20 of the Maintenance Assessments and Special Cases Regulations, or as a non-resident parent under regulation 8 of the Maintenance Calculations and Special Cases Regulations, as the case may be.

(9) Where, in a case falling within sub-paragraph (8), there is more than one person who does not fall to be treated as an absent parent under regulation 20 of the Maintenance Assessments and Special Cases Regulations or as a non-resident parent under regulation 8 of the Maintenance Calculations and Special Cases Regulations, as the case may be, the Secretary of State shall apply the provisions of paragraph (10) to determine which application he shall proceed with.

(10) Where—
\begin{enumerate}\item[]
($a$) a person with care makes an assessment application under section 4 of the former Act and a different person with care makes a calculation application under section 4 of the Act and those applications are in respect of the same qualifying child or qualifying children (whether or not any of those applications is also in respect of other qualifying children); and

($b$) either—
\begin{enumerate}\item[]
(i) none of those persons has parental responsibility for (or, in Scotland, parental rights over) that child or children; or

(ii) the case falls within sub-paragraph (8)($b$)  but the Secretary of State has not been able to determine which application he is to proceed with under the provisions of sub-paragraph (8),
\end{enumerate}
\end{enumerate}
the Secretary of State shall proceed with the application of the principal provider of day to day care, as determined in accordance with sub-paragraph (11).

(11) For the purposes of sub-paragraph (10), the application of the principal provider is, where—
\begin{enumerate}\item[]
($a$) the applications are in respect of one qualifying child, the application of that person with care to whom child benefit is paid in respect of that child;

($b$) the applications are in respect of more than one qualifying child, the application of that person with care to whom child benefit is paid in respect of those children;

($c$) the Secretary of State cannot determine which application he is to proceed with under head ($a$)  or ($b$), the application of that applicant who in the opinion of the Secretary of State is the principal provider of day to day care for the child or children in question.
\end{enumerate}

(12) Subject to sub-paragraph (13), where, in any case falling within sub-paragraphs (2) to (10), the applications are not in respect of identical qualifying children, the application that the Secretary of State is to proceed with as determined by those sub-paragraphs shall be treated as an application with respect to all of the qualifying children with respect to whom the applications were made.

(13) Where the Secretary of State is satisfied that the same person with care does not provide the principal day to day care for all of the qualifying children with respect to whom an application would but for the provisions of this paragraph be made under sub-paragraph (12), he shall make separate maintenance assessments under the former Act or maintenance calculations under the Act, as the case may be, in relation to each person with care providing such principal day to day care.

(14) For the purposes of this paragraph “day to day care” has the same meaning as in the Maintenance Assessments and Special Cases Regulations or the Maintenance Calculations and Special Cases Regulations, as the case may be.

\section*{\itshape Maintenance assessment in force: subsequent application with respect to the same persons}

4.  Where—
\begin{enumerate}\item[]
($a$) a maintenance assessment is in force under the former Act;

($b$) a calculation application is made or treated as made under the section of the Act which is the same section as the section of the former Act under which the assessment application was made; and

($c$) the calculation application relates to—
\begin{enumerate}\item[]
(i) the same person with care and qualifying child or qualifying children as the maintenance assessment; and

(ii) a non-resident parent who is the absent parent with respect to the maintenance assessment,
\end{enumerate}
\end{enumerate}
the calculation application shall not be proceeded with.

\section*{\itshape Interpretation}

5.  In this Schedule, “absent parent”, “former Act” and “maintenance assessment” have the meanings given in regulation 31(8)($a$).

\part{Explanatory Note}

\renewcommand\parthead{— Explanatory Note}

\subsection*{(This note is not part of the Regulations)}

These Regulations provide for various procedural matters relating to an application for a maintenance calculation under the Child Support Act 1991 (c.\ 48) (“the Act”), and make provision in respect of effective dates of calculations and of reduced benefit decisions, consequent upon the introduction of changes to the child support system made by the Child Support, Pensions and Social Security Act 2000 (c.\ 19). Subject to savings for transitional purposes these Regulations revoke the Child Support (Maintenance Assessment Procedure) Regulations 1992 (1992/1813). These Regulations come into force at different times for different cases according to the dates on which provisions of the Child Support, Pensions and Social Security Act 2000 which are relevant to these Regulations are commenced for different types of cases.

Regulation 1 contains provisions relating to citation, commencement and interpretation. Schedule 1 contains provisions relating to the interpretation of a “child” for the purposes of the Act.

Regulation 2 contains provisions relating to the service and receipt of documents and regulation 3 sets out the procedures in relation to an application for a maintenance calculation.

Regulation 4 and Schedule 2 provide for multiple applications for a maintenance calculation.

Regulations 5 and 6 provide for notice to be given to the non-resident parent and any other relevant person when an effective application for a maintenance calculation has been made or treated as made by the person with care, and for the procedure on the death of a qualifying child.

Regulation 7 prescribes the default rate, payable when a default maintenance decision is made under section 12(1) of the Act.

Regulation 8 contains provisions relating to interpretation for the purposes of Part IV of these Regulations (reduced benefit decisions).

Regulation 9 prescribes the period within which reasons are to be given by the parent with care for the purposes of section 46(2) of the Act.

Regulations 10 to 20 make provision as to the amount and duration of reduced benefit decisions following a request under section 6(5) of the Act, or a failure to comply with the obligation in section 6(7) of the Act, or a refusal to take a scientific test (within the meaning of section 27A of the Act).

Regulation 21 prescribes persons who are not persons with care for the purposes of the Act and regulation 22 makes provision for the authorisation of representatives. Regulations 23 and 24 set out what is to be notified following decisions by the Secretary of State.

Regulations 25 to 29 prescribe the effective dates of maintenance calculations.

Regulation 30 revokes the Child Support (Maintenance Assessment Procedure) Regulations 1992 with savings for transitional purposes. Regulation 31 makes transitional provision for the effective date of a calculation applied for after the new system comes into force where there is an assessment in force under the previous scheme and where reduced benefit decisions have been made or are being considered when the new system comes into force.

The impact on business of these Regulations was covered in the Regulatory Impact Assessment (RIA) for the Child Support, Pensions and Social Security Act 2000, in accordance with which, and in consequence of which, these Regulations are made. A copy of that RIA has been placed in the libraries of both Houses of Parliament and can be obtained from the Department of Social Security, Regulatory Impact Unit, Adelphi, 1–11 John Adam Street, London, \textsc{\lowercase{WC2N 6HT}}. 

\end{document}