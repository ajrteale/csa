\documentclass[12pt,a4paper]{article}

\newcommand\regstitle{The Child Support (Miscellaneous Amendments) Regulations 1996}

\newcommand\regsnumber{1996/1945}

%\opt{newrules}{
\title{\regstitle}
%}

%\opt{2012rules}{
%\title{Child Maintenance and Other Payments Act 2008\\(2012 scheme version)}
%}

\author{S.I. 1996 No. 1945}

\date{Made 23rd July 1996\\Coming into force:\\
Regulations 1, 4, 7, 8, 9, 12, 20, 21 and 22 and paragraphs (1) and (3) of regulation 18 5th August 1996\\Remainder 7th October 1996}

%\opt{oldrules}{\newcommand\versionyear{1993}}
%\opt{newrules}{\newcommand\versionyear{2003}}
%\opt{2012rules}{\newcommand\versionyear{2012}}

\usepackage{csa-regs}

\setlength\headheight{27.57402pt}

\begin{document}

\maketitle

\noindent
Whereas a draft of this instrument was laid before Parliament in accordance with section 52(2) of the Child Support Act 1991\footnote{\frenchspacing 1991 c. 48.}, and approved by a resolution of each House of Parliament:

 Now, therefore, the Secretary of State for Social Security, in exercise of the powers conferred by sections 14(1), 21(2), 32(1), 42(3), 46(11), 47(1) and (2), 51, 52(4) and 54 of, and paragraphs 5(1) and (2), 6, 8 and 11 of Schedule 1 to, the Child Support Act 1991\footnote{\frenchspacing Section 54 is cited because of the meaning ascribed to the word “prescribed”.}, and of all other powers enabling him in that behalf, hereby makes the following Regulations:

{\sloppy

\tableofcontents

}

\setcounter{secnumdepth}{-2}

\subsection[1. Citation, commencement and interpretation]{Citation, commencement and interpretation}

1.—(1) These Regulations may be cited as the Child Support (Miscellaneous Amendments) Regulations 1996.

(2) This regulation, regulations 4, 7, 8, 9, 12, 20, 21 and 22, and paragraphs (1) and (3) of regulation 18, of these Regulations shall come into force on 5th August 1996. The remaining regulations other, than paragraphs (3) and (6) of regulation 14, shall come into force on 7th October 1996 and those paragraphs shall come into force immediately following the coming into force of regulation 5 of the Social Security and Child Support (Jobseeker’s Allowance) (Consequential Amendments) Regulations 1996\footnote{\frenchspacing S.I. 1996/1345.}.

(3) In these Regulations—
\begin{enumerate}\item[]
“the Appeal Regulations” means the Child Support Appeal Tribunals (Procedure) Regulations 1992\footnote{\frenchspacing S.I. 1992/2641.};

“the Collection and Enforcement Regulations” means the Child Support (Collection and Enforcement) Regulations 1992\footnote{\frenchspacing S.I. 1992/1989. Regulations 8(1) and 11 were amended by S.I. 1995/1045.};

“the Fees Regulations” means the Child Support Fees Regulations 1992\footnote{\frenchspacing S.I. 1992/3094. Regulation 1 was amended by S.I. 1994/227 and regulation 3 by S.I. 1994/227 and S.I. 1995/1045. Regulation 3 is also amended, with effect from 7th October 1996, by S.I. 1996/1345.};

“the Information, Evidence and Disclosure Regulations” means the Child Support (Information, Evidence and Disclosure) Regulations 1992\footnote{\frenchspacing S.I. 1992/1812. Regulation 2 was amended by S.I. 1995/123, S.I. 1995/1045 and S.I. 1995/3261 and regulation 3 by S.I. 1995/1045 and S.I. 1995/3261.};

“the Maintenance Assessment Procedure Regulations” means the Child Support (Maintenance Assessment Procedure) Regulations 1992\footnote{\frenchspacing S.I. 1992/1813. Regulation 31 was substituted, regulations 15A and 31C inserted and regulation 33 amended by S.I. 1995/3261. Regulation 36 was amended by S.I. 1995/1045. Regulations 36, 38, 39 and 47 are amended, with effect from 7th October 1996, by S.I. 1996/1345.};

“the Maintenance Assessments and Special Cases Regulations” means the Child Support (Maintenance Assessments and Special Cases) Regulations 1992\footnote{\frenchspacing S.I. 1992/1815. Regulations 1(2) and 9 were amended by S.I. 1993/913, S.I. 1995/1045 and S.I. 1995/3261. Regulation 11 was amended by S.I. 1994/227, S.I. 1995/1045 and S.I. 1995/3261. Regulations 12 and 16 were amended by S.I. 1995/1045. Relevant amendments to Schedule 1 were made by S.I. 1995/1045. Regulation 1(2) and Schedule 1 are also amended, with effect from 7th October 1996, by S.I. 1996/1345.}.
\end{enumerate}

\subsection[2. Substitution of regulation 17 of the Appeal Regulations]{Substitution of regulation 17 of the Appeal Regulations}

2.—(1) For regulation 17 of the Appeal Regulations (confidentiality) there shall be substituted the following regulation—
\begin{quotation}
\subsection*{“Confidentiality}

17.—(1) No information such as is mentioned in paragraph (2), and which has been provided for the purposes of any proceedings to which these Regulations apply, shall be disclosed if, before the expiry of the period of 21 days specified in paragraph (3), written notification has been received from the person to whom the information relates that he does not consent to such disclosure.

(2) The information referred to in paragraph (1) is—
\begin{enumerate}\item[]
($a$) the address of the person referred to in that paragraph; and

($b$) any other information the use of which could reasonably be expected to lead to that person being located.
\end{enumerate}

(3) Except where the appeal is made under section 46(7) of the Act or is one to which regulation 3(1)($b$) applies, the clerk to the tribunal shall notify the person to whom the information referred to in paragraphs (1) and (2) relates of the provisions of those paragraphs and that disclosure of that information may be made, unless the written notification specified in paragraph (1) is received before the expiry of the period of 21 days, beginning with the date the notification by the clerk to the tribunal was given or sent to that person.”
\end{quotation}

\subsection[3. Amendment of regulation 8 of the Collection and Enforcement Regulations]{Amendment of regulation 8 of the Collection and Enforcement Regulations}

3.  In the definition of “interim maintenance assessment” in paragraph (1) of regulation 8 of the Collection and Enforcement Regulations (interpretation), for the words “regulation 8(1B)” there shall be substituted the words “regulation 8(3)”.

\subsection[4. Amendment of regulation 11 of the Collection and Enforcement Regulations]{Amendment of regulation 11 of the Collection and Enforcement Regulations}

4.  In paragraph (3) of regulation 11 of the Collection and Enforcement Regulations, (protected earnings rate), after the words “interim maintenance assessment”, there shall be inserted the words “, except a Category B interim maintenance assessment,”.

\subsection[5. Amendment of regulation 1 of the Fees Regulations]{Amendment of regulation 1 of the Fees Regulations}

5.  In paragraph (2) of regulation 1 of the Fees Regulations (citation, commencement and interpretation), after the definition of “collection fee” there shall be inserted the following definitions—
\begin{quotation}
““earnings top-up” means the allowance paid by the Secretary of State under the rules specified in the Earnings Top-up Scheme;

“the Earnings Top-up Scheme” means the Earnings Top-up Scheme 1996\footnote{\frenchspacing This Scheme, which applies only in certain parts of Great Britain, is an extra-statutory Scheme, introduced by the Secretary of State for Social Security, having effect on 8th October 1996. Copies of the rules of this Scheme may be obtained from the Customer Services Manager, Earnings Top-up, Norcross, Blackpool FY5 3TA.};”
\end{quotation}

\subsection[6. Amendment of regulation 3 of the Fees Regulations]{Amendment of regulation 3 of the Fees Regulations}

6.  In paragraph (5) of regulation 3 of the Fees Regulations (liability to pay fees), after sub-paragraph ($d$) there shall be added the following sub-paragraph—
\begin{quotation}
“($e$) any person to or in respect of whom earnings top-up is paid.”
\end{quotation}

\subsection[7. Amendment of regulation 2 of the Information, Evidence and Disclosure Regulations]{Amendment of regulation 2 of the Information, Evidence and Disclosure Regulations}

7.—(1) Regulation 2 of the Information, Evidence and Disclosure Regulations (persons under a duty to furnish information or evidence), shall be amended in accordance with the following provisions of this regulation.

(2) After paragraph (1), there shall be inserted the following paragraph—
\begin{quotation}
“(1A) A person falling within paragraph (2)($a$) or ($e$) shall furnish such information or evidence as is required by the Secretary of State and is needed by him to enable a decision to be made in relation to the matters listed in regulation 3(1)($h$) and ($hh$) where the person concerned has that information or evidence in his possession or can reasonably be expected to acquire it.”
\end{quotation}

(3) In sub-paragraph ($e$) of paragraph (2), after the word “sub-\hspace{0pt}paragraphs” there shall be inserted “($aa$), ($ab$),”.

\subsection[8. Amendment of regulation 3 of the Information, Evidence and Disclosure Regulations]{Amendment of regulation 3 of the Information, Evidence and Disclosure Regulations}

8.—(1) Regulation 3 of the Information, Evidence and Disclosure Regulations (purposes for which information or evidence may be required) shall be amended in accordance with the following provisions of this regulation.

(2) After sub-paragraph ($a$) of paragraph (1), there shall be inserted the following sub-paragraphs—
\begin{quotation}
“($aa$) a decision to be made as to whether there is in force a written maintenance agreement made before 5th April 1993, or a maintenance order, in relation to a qualifying child and the person who is at that time the absent parent of that child;

($ab$) a decision to be made as to whether a person with care has parental responsibility for a qualifying child for the purposes of section 5(1) of the Act;”
\end{quotation}

(3) After sub-paragraph ($h$) of paragraph (1), there shall be inserted the following sub-paragraph—
\begin{quotation}
“($hh$) a decision to be made as to whether to take action under section 35(1) or 38(1) of the Act or to apply under section 36(1) of the Act for an order for recovery by means of garnishee proceedings or a charging order;”
\end{quotation}

\subsection[9. Amendment of regulation 15A of the Maintenance Assessment Procedure Regulations]{Amendment of regulation 15A of the Maintenance Assessment Procedure Regulations}

9.  At the end of paragraph (2) of regulation 15A of the Maintenance Assessment Procedure Regulations, (notification of reinstatement of a maintenance assessment), there shall be added “and where the review is carried out under section 19(1)($d$) of the Act, except where that review is of the cancellation of a Category A or Category D interim maintenance assessment, as to the provisions of section 18 of the Act and regulations 24(1) and 31A(8).”.

\subsection[10. Amendment of regulation 31 of the Maintenance Assessment Procedure Regulations]{Amendment of regulation 31 of the Maintenance Assessment Procedure Regulations}

10.  For paragraph (1) of regulation 31 of the Maintenance Assessment Procedure Regulations (effective dates following a review under section 16 or 17 of that Act), there shall be substituted the following paragraph—
\begin{quotation}
“(1) Subject to paragraph (2), where a fresh maintenance assessment is made following a review under section 16 of the Act, the effective date of that assessment shall be 104 weeks after the effective date of the previous assessment disregarding—
\begin{enumerate}\item[]
($a$) any previous assessment made following a review under section 17 of the Act, where, after 22nd January 1996, a child support officer decided, in accordance with section 17(3) of the Act, to proceed with a review;

($b$) any previous assessment made following a review under section 18 or 19 of the Act;

($c$) any interim maintenance assessment made under section 12(1A)($b$) or ($c$) of the Act, except a Category B interim maintenance assessment made under paragraph ($b$) or ($c$) of that section where that interim maintenance assessment is the assessment being reviewed under section 16 of the Act.”
\end{enumerate}
\end{quotation}

\subsection[11. Amendment of regulation 31C of the Maintenance Assessment Procedure Regulations]{Amendment of regulation 31C of the Maintenance Assessment Procedure Regulations}

11.  In paragraph (3) of regulation 31C of the Maintenance Assessment Procedure Regulations (effective dates in specific cases), for the words “that determined” to the end of that paragraph there shall be substituted the words “the correct effective date applicable to the maintenance assessment which is being reviewed”.

\subsection[12. Amendment of regulation 33 of the Maintenance Assessment Procedure Regulations]{Amendment of regulation 33 of the Maintenance Assessment Procedure Regulations}

12.—(1) Regulation 33 of the Maintenance Assessment Procedure Regulations (maintenance periods) shall be amended in acordance with the following provisions of this regulation.

(2) For paragraph (6), there shall be substituted the following paragraph—
\begin{quotation}
“(6) Where a case is to be treated as a special case for the purposes of the Act by virtue of regulation 22 of the Maintenance Assessments and Special Cases Regulations (multiple applications relating to an absent parent) and an application is made by a person with care in relation to an absent parent where—
\begin{enumerate}\item[]
($a$) there is already a maintenance assessment in force in relation to that absent parent and a different person with care; or

($b$) sub-paragraph ($a$) does not apply, but before a maintenance assessment is made in relation to that application, a maintenance assessment is made in relation to that absent parent and a different person with care,
\end{enumerate}
the maintenance periods in relation to an assessment made in response to that application shall coincide with the maintenance periods in relation to the earlier maintenance assessment, except where regulation 3(7) of the Maintenance Arrangements and Jurisdiction Regulations or paragraph (8) applies, and the first such period shall, subject to paragraph (9), commence not later than 7 days after the date of notification to the relevant persons of the later maintenance assessment.”
\end{quotation}

(3) After paragraph (8) there shall be added the following paragraph—
\begin{quotation}
“(9) Where the case is one to which, if paragraphs (6) and (7) did not apply, regulation 30(2)($a$)(i) or ($b$)(i) would apply, and the first maintenance period would, under the provisions of paragraph (6), commence during the 8 week period referred to in sub-paragraph ($a$) or ($b$) of that regulation, the first maintenance period shall commence not later than 7 days after the expiry of that period of 8 weeks.”
\end{quotation}

\subsection[13. Amendment of regulation 35 of the Maintenance Assessment Procedure Regulations]{Amendment of regulation 35 of the Maintenance Assessment Procedure Regulations}

13.—(1) Regulation 35 of the Maintenance Assessment Procedure Regulations (periods for compliance with obligations imposed by section 6 of the Act) shall be amended in accordance with the following provisions of this regulation.

(2) For paragraph (2), there shall be substituted the following paragraph—
\begin{quotation}
“(2) The Secretary of State shall not refer a case to a child support officer prior to the expiry of a period of—
\begin{enumerate}\item[]
($a$) 2 weeks from the date he serves notice under paragraph (1) on the parent in question; or

($b$) 6 weeks from that date, where, before the expiry of 2 weeks from service of that notice, he has received from the parent in question in writing her reasons why she believes that if she were to be required to comply with an obligation imposed by section 6 of the Act, there would be a risk, as a result of that compliance, of her or any child or children living with her suffering harm or undue distress,
\end{enumerate}
and the notice shall contain a statement setting out the provisions of sub-paragraphs ($a$) and ($b$).”
\end{quotation}

(3) In paragraph (3), for the words “the Secretary of State refers a case to a child support officer and the” there shall be substituted the word “a”.

\subsection[14. Amendment of regulation 36 of the Maintenance Assessment Procedure Regulations]{Amendment of regulation 36 of the Maintenance Assessment Procedure Regulations}

14.—(1) Regulation 36 of the Maintenance Assessment Procedure Regulations (amount of and period of reduction of relevant benefit under a reduced benefit direction) shall be amended in accordance with the following provisions of this regulation.

(2) In paragraph (2), for the words “26 weeks” there shall be substutited the words “156 weeks” and for the formula
\[0.2 \times B\] there shall be substituted the formula— \[0.4 \times B.\]

(3) Paragraphs (3) and (9) shall be omitted.

(4) In paragraph (4), after the word “paragraphs” there shall be inserted “(4A),”.

(5) After paragraph (4) there shall be inserted the following paragraph—
\begin{quotation}
“(4A) Subject to paragraphs (5), (5A) and (5B), where a reduced benefit direction (“the subsequent direction”) is made on a day when a reduced benefit direction (“the earlier direction”) is in force in respect of the same parent, the subsequent direction shall come into operation on the day immediately following the day on which the earlier direction ceased to be in force.”.
\end{quotation}

(6) For paragraph (6), there shall be substituted the following paragraph—
\begin{quotation}
“(6) Where the benefit payable is income support or income-based jobseekers allowance and there is a change in the benefit week whilst a direction is in operation, the period of the reduction specified in paragraph (2) shall be a period greater than 155 weeks but less than 156 weeks and ending on the last day of the last benefit week falling entirely within the period of 156 weeks specified in that paragraph.”
\end{quotation}

\subsection[15. Amendment of regulation 38 of the Maintenance Assessment Procedure Regulations]{Amendment of regulation 38 of the Maintenance Assessment Procedure Regulations}

15.  In paragraph (4) of regulation 38 of the Maintenance Assessment Procedure Regulations (suspension of a reduced benefit direction when relevant benefit ceases to be payable), the words “and (3)” shall be omitted.

\subsection[16. Amendment of regulation 39 of the Maintenance Assessment Procedure Regulations]{Amendment of regulation 39 of the Maintenance Assessment Procedure Regulations}

16.  In paragraph (2) of regulation 39 of the Maintenance Assessment Procedure Regulations (reduced benefit direction where family credit or disability working allowance is payable and income support becomes payable), the words “and (3)” shall be omitted.

\subsection[17. Amendment of regulation 47 of the Maintenance Assessment Procedure Regulations]{Amendment of regulation 47 of the Maintenance Assessment Procedure Regulations}

17.—(1) Regulation 47 of the Maintenance Assessment Procedure Regulations (reduced benefit directions where there is an additional qualifying child) shall be amended in accordance with the following provisions of this regulation.

(2) For paragraph (3) there shall be substituted the following paragraph—
\begin{quotation}
“(3) Where—
\begin{enumerate}\item[]
($a$) a direction (“the earlier direction”) has ceased to be in force by virtue of regulation 38(2); and

($b$) a child support officer gives a direction (“the further direction”) with respect to the same parent on account of that parent’s failure to comply with the obligations imposed by section 6 of the Act in relation to an additional qualifying child,
\end{enumerate}
as long as that further direction remains in force, no additional direction shall be brought into force with respect to that parent on account of her failure to comply with the obligations imposed by section 6 of the Act in relation to one or more children in relation to whom the earlier direction was given.”
\end{quotation}

(3) In paragraph (4), for the words “shall be determined in accordance with paragraphs (6) and (7).” there shall be substituted the words “for the extended period shall be determined in accordance with regulation 36(2).”.

(4) In paragraph (5), for the formula
\[(78-F-S) \mathrm{\ weeks}\]
there shall be substituted the formula—
\[(156-F-S) \mathrm{\ weeks.}\]

(5) Paragraphs (6) and (7) shall be omitted.

\subsection[18. Amendment of regulation 1 of the Maintenance Assessments and Special Cases Regulations]{\sloppy Amendment of regulation 1 of the Maintenance Assessments and Special Cases Regulations}

18.—(1) Paragraph (2) of regulation 1 of the Maintenance Assessments and Special Cases Regulations (citation, commencement and interpretation) shall be amended in accordance with the following provisions of this regulation.

(2) After the definition of “earnings”, there shall be inserted the following definitions—
\begin{quotation}
““earnings top-up” means the allowance paid by the Secretary of State under the rules specified in the Earnings Top-up Scheme;

“The Earnings Top-up Scheme” means the Earnings Top-up Scheme 1996\footnote{\frenchspacing This Scheme, which applies only in certain parts of Great Britain, is an extra-statutory Scheme, introduced by the Secretary of State for Social Security, having effect on 8th October 1996. Copies of the rules of this Scheme may be obtained from the Customer Services Manager, Earnings Top-up, Norcross, Blackpool \textsc{\lowercase{FY5 3TA}}.};”.
\end{quotation}

(3) For the definition of “family” there shall be substituted the following definitions—
\begin{quotation}
““family” means—
\begin{enumerate}\item[]
($a$) a married or unmarried couple (including the members of a polygamous marriage);

($b$) a married or unmarried couple (including the members of a polygamous marriage) and any child or children living with them for whom at least one member of that couple has day to day care;

($c$) where a person who is not a member of a married or unmarried couple has day to day care of a child or children, that person and any such child or children;
\end{enumerate}
and for the purposes of this definition a person shall not be treated as having day to day care of a child who is a member of that person’s household where the child in question is being looked after by a local authority within the meaning of section 22 of the Children Act 1989\footnote{\frenchspacing 1989 c. 41.} or, in Scotland, where the child is boarded out with that person by a local authority under the provisions of section 21 of the Social Work (Scotland) Act 1968\footnote{\frenchspacing 1968 c. 49.};”
\end{quotation}

\subsection[19. Amendment of regulation 9 of the Maintenance Assessments and Special Cases Regulations]{\sloppy Amendment of regulation 9 of the Maintenance Assessments and Special Cases Regulations}

19.  In paragraph (1)($b$) of regulation 9 of the Maintenance Assessments and Special Cases Regulations (exempt income), for “18” there shall be substituted “16 and 18”.

\subsection[20. Amendment of regulation 11 of the Maintenance Assessments and Special Cases Regulations]{Amendment of regulation 11 of the Maintenance Assessments and Special Cases Regulations}

20.—(1) Regulation 11 of the Maintenance Assessments and Special Cases Regulations (protected income) shall be amended in accordance with the following provisions of this regulation.

(2) In paragraph (1), for the words “and (6)” there shall be substituted the words “, (6) and (6A)”.

(3) After paragraph (6), there shall be inserted the following paragraph—
\begin{quotation}
“(6A) In a case to which paragraph (6) does not apply, if the application of paragraphs (1) to (5) and of regulation 12(1)($a$) would result in the amount of child support maintenance payable being greater than 30 per centum of the absent parent’s net income calculated in accordance with regulation 7, paragraphs (1) to (5) shall not apply in his case and instead his protected income level shall be 70 per centum of his net income as so calculated.”
\end{quotation}

(4) In paragraph (7), after the words “paragraph (6)” there shall be inserted the words “or (6A)”.

\subsection[21. Amendment of regulation 12 of the Maintenance Assessments and Special Cases Regulations]{Amendment of regulation 12 of the Maintenance Assessments and Special Cases Regulations}

21.  In paragraph (1) of regulation 12 of the Maintenance Assessments and Special Cases Regulations (disposable income), after the words “regulation 11(6)” in sub-paragraphs ($a$) and ($b$), there shall be inserted the words “or (6A)”.

\subsection[22. Substitution of regulation 16 of the Maintenance Assessments and Special Cases Regulations]{Substitution of regulation 16 of the Maintenance Assessments and Special Cases Regulations}

22.  For regulation 16 of the Maintenance Assessments and Special Cases Regulations (weekly amount of housing costs), there shall be substituted the following regulation—
\begin{quotation}
\subsection*{“Weekly amount of housing costs}

16.—(1) Where a parent pays housing costs—
\begin{enumerate}\item[]
($a$) on a weekly basis, the amount of such housing costs shall subject to paragraph (2), be the weekly rate payable at the effective date;

($b$) on a monthly basis, the amount of such housing costs shall subject to paragraph (2), be the monthly rate payable at the effective date, multiplied by 12 and divided by 52;

($c$) by way of rent payable to a housing association, as defined in section 1(1) of the Housing Associations Act 1985\footnote{\frenchspacing 1985 c. 69.} which is registered in accordance with section 5 of that Act, or to a local authority, on a free week basis, that is to say the basis that he pays an amount by way of rent for a given number of weeks in a 52 week period, with a lesser number of weeks in which there is no liability to pay (“free weeks”), the amount of such housing costs shall be the amount which he pays—
\begin{enumerate}\item[]
(i) in the relevant week if it is not a free week; or

(ii) in the last week before the relevant week which is not a free week, if the relevant week is a free week;
\end{enumerate}

($d$) on any other basis, the amount of such housing costs shall, subject to paragraph (2), be the rate payable at the effective date, multiplied by the number of payment periods, or the nearest whole number of payment periods (any fraction of one half being rounded up), falling within a period of 365 days and divided by 52.
\end{enumerate}

(2) Where housing costs consist of payments on a repayment mortgage and the absent parent or parent with care has not provided information or evidence as to the rate of repayment of the capital secured and the interest payable on that mortgage at the effective date and that absent parent or parent with care has provided a statement from the lender, in respect of a period ending not more than 12 months prior to the first day of the relevant week, for the purposes of the calculation of exempt income under regulation 9 and protected income under regulation 11—
\begin{enumerate}\item[]
($a$) if the amount of capital repaid for the period covered by that statement is shown on it, the rate of repayment of capital owing under that mortgage shall be calulated by reference to that amount; and

($b$) if the amount of capital owing and the interest rate applicable at the end of the period covered by that statement are shown on it, the interest payable on that mortgage shall be calculated by reference to that amount and that interest rate.”
\end{enumerate}
\end{quotation}

\subsection[23. Amendment of regulation 19 of the Maintenance Assessments and Special Cases Regulations]{Amendment of regulation 19 of the Maintenance Assessments and Special Cases Regulations}

23.  After sub-paragraph ($c$) of paragraph (2) of regulation 19 of the Maintenance Assessments and Special Cases Regulations (both parents are absent), there shall be added the following sub-paragraph—
\begin{quotation}
“($d$) where the application is made in relation to one absent parent only, the amount of the maintenance requirement applicable in that case shall be one-half of the amount determined in accordance with paragraph 1(2) of Schedule 1 to the Act or, where regulation 23 applies (person caring for children of more than one absent parent), of the amount determined in accordance with paragraphs (2) to (3) of that regulation.”.
\end{quotation}

\subsection[24. Amendment of Schedule 1 to the Maintenance Assessments and Special Cases Regulations]{Amendment of Schedule 1 to the Maintenance Assessments and Special Cases Regulations}

24.—(1) Schedule 1 to the Maintenance Assessments and Special Cases Regulations (calculation of N and M), shall be amended in accordance with the following provisions of this regulation.

(2) For head ($d$) of paragraph 1(1) there shall be substituted the following head—
\begin{quotation}
“($d$) any payments made by the parent’s employer in respect of any expenses not wholly, exclusively and necessarily incurred in the performance of the duties of the employment, including any payment made by the parent’s employer in respect of—
\begin{enumerate}\item[]
(i) travelling expenses incurred by that parent between his home and place of employment; and

(ii) expenses incurred by that parent under arrangements made for the care of a member of his family owing to that parent’s absence from home;”.
\end{enumerate}
\end{quotation}

(3) In sub-paragraph (3)($a$) of paragraph 7, before the words “weekly earnings” there shall be inserted the word “normal”.

(4) After paragraph 14, there shall be inserted the following paragraph—
\begin{quotation}
“14A.—(1) Subject to sub-paragraph (2), the amount of any earnings top-up paid to or in respect of the absent parent or the parent with care.

(2) Subject to sub-paragraphs (3) and (4), where earnings top-up is payable and the amount which is payable has been calculated by reference to the weekly earnings of either the absent parent and another person or the parent with care and another person—
\begin{enumerate}\item[]
($a$) if during the period which is used to calculate his earnings under paragraph 2 or, as the case may be, paragraph 5, the normal weekly earnings of that parent exceed those of the other person, the amount payable by way of earnings top-up shall be treated as the income of that parent;

($b$) if during that period, the normal weekly earnings of that parent equal those of the other person, half of the amount payable by way of earnings top-up shall be treated as the income of that parent;

($c$) if during that period, the normal weekly earnings of that parent are less than those of that other person, the amount payable by way of earnings top-up shall not be treated as the income of that parent.
\end{enumerate}

(3) Where any earnings top-up is in payment and, not later than the effective date, the person, or, if more than one, each of the persons by reference to whose engagement and normal engagement in remunerative work that payment has been calculated is no longer the partner of the person to whom that payment is made, the payment in question shall be treated as the income of the parent in question only where that parent is in receipt of it.

(4) Where earnings top-up is in payment and, not later than the effective date, either or both of the persons by reference to whose engagement and normal engagement in remunerative work that payment has been calculated has ceased to be employed, half of the amount payable by way of earnings top-up shall be treated as the income of the parent in question.”
\end{quotation}

(5) In paragraph 15, after the words “Schedule” there shall be inserted the words “except payments or other amounts which are excluded from the definition of “earnings” by virtue of paragraph 1(2)”.

\subsection[25. Transitional provisions]{Transitional provisions}

25.—(1) The provisions of regulation 33 of the Maintenance Assessment Procedure Regulations in force prior to 5th August 1996 shall continue to apply to any application made prior to that date.

(2) The provisions of regulation 35 of the Maintenance Assessment Procedure Regulations in force prior to 7th October 1996 shall continue to apply to any case where the failure to comply referred to in paragraph (1) of that regulation arose prior to that date.

(3) The provisions of regulation 36 of the Maintenance Assessment Procedure Regulations in force prior to 7th October 1996 shall 
%continue to apply 
apply with the amendments made by regulation 5(6) of the Social Security and Child Support (Jobseeker’s Allowance) (Consequential Amendments) Regulations 1996\footnote{\frenchspacing S.I. 1996/1345.}  % Words substituted (6.10.96) by SI 1996/2378 reg 3(2)
to a parent in respect of whom a reduced benefit direction was given prior to that date.

(4) The provisions of regulation 47 of the Maintenance Assessment Procedure Regulations in force prior to 7th October 1996 shall 
%continue to apply 
apply with the amendments made by regulation 5(11) of the Social Security and Child Support (Jobseeker’s Allowance) (Consequential Amendments) Regulations 1996  % Words substituted (6.10.96) by SI 1996/2378 reg 3(3)
to any reduced benefit direction made prior to that date, and in relation to an earlier direction referred to in paragraph (4) of that regulation, which was in force prior to that date, whether or not the further direction referred to in that paragraph was made after that date.

(5) The provisions of regulation 19 of the Maintenance Assessments and Special Cases Regulations in force prior to 7th October 1996 shall continue to apply to any application made prior to that date 
%and those provisions, as amended by regulation 23, shall not apply to a maintenance assessment in force on that date until it is first reviewed after that date under section 16, 17 or 18 of the Act.
and a decision with respect to a maintenance assessment in force on that date shall not be superseded by a decision under section 17 of the Child Support Act 1991 solely to give effect to the provisions of regulation 19 as amended by regulation 23.  % Words substituted (1.6.99) by SI 1999/1510 reg 43

\amendment{
Words substituted in reg. 25(3), (4) (6.10.96) by the Social Security and Child Support (Jobseeker's Allowance) (Transitional Provisions) (Amendment) Regulations 1996 reg. 3.

Words substituted in reg. 25(5) (1.6.99) by the Social Security Act 1998 (Commencement No. 7 and Consequential and Transitional Provisions) Order 1999 reg. 43.

}

\bigskip

Signed by authority of the Secretary of State for Social Security.

{\raggedleft
\emph{A J B Mitchell}\\*Parliamentary Under-Secretary of State,\\*Department of Social Security

}

23rd July 1996

\bigskip

\small

\part{Explanatory Note}

\renewcommand\parthead{--- Explanatory Note}

\subsection*{(This note is not part of the Regulations)}

These Regulations amend various regulations made under the Child Support Act 1991 (“the Act”).

  For regulation 17 of the Child Support Appeal Tribunal (Procedure) Regulations 1992, there is substituted a new regulation, which allows disclosure to the parties to an appeal of an address or information which might lead to a person being located, where the appeal relates to a reduced benefit direction, and in all other cases, unless that person gives notification that he does not consent to such disclosure (regulation 2).

  Reference to the Earnings Top-up Scheme is inserted into the Child Support Fees Regulations 1992 (regulations 5 and 6) and into Schedule 1 to the Child Support (Maintenance Assessments and Special Cases) Regulations 1992 (regulation 24).

  The Child Support (Information, Evidence and Disclosure) Regulations 1992 are amended to make provision for information to be given to enable the Secretary of State to decide whether a court order for maintenance or a written maintenance agreement made before 5th April 1993 is in force; whether a person with care who has applied for a maintenance assessment has parental responsibility for the child in question; and which means of enforcement of a maintenance assessment available to him would be appropriate in the circumstances of a particular case (regulations 7 and 8).

  The Child Support (Maintenance Assessment Procedure) Regulations 1992 are amended in the following respects—
\begin{itemize}
\item regulation 33 is amended to make provision for maintenance periods to coincide where more than one application for a maintenance assessment is being dealt with at the same time (regulation 12);

\item Part IX is amended to provide that a case may be referred by the Secretary of State to a Child Support Officer two weeks after notification of intention to refer, if the parent with care has not set out in writing her reasons for contending that harm or undue distress might result from compliance with a requirement under section 6 of the Act. It is also amended to allow for a further Reduced Benefit Direction to be issued on the expiry of the previous one if the parent continues to fail to comply with a requirement imposed under that section; and to provide for a reduction for 156 weeks of 40 per cent.\ of the income support personal allowance for a single claimant of 25 or over (regulations 13 to 17).
\end{itemize}

  The Child Support (Maintenance Assessments and Special Cases) Regulations 1992 are amended in the following respects—
\begin{itemize}
\item regulation 11 is amended to ensure that an absent parent is always left with 70 per cent.\ of his net income after deduction of maintenance (regulation 20);

\item regulation 19 is amended to provide for the maintenance requirement to be halved where an application is made in relation to only one parent, where both are absent (regulation 23);

\item Schedule 1 is amended to make it clear that earnings include reimbursement by an employer of travelling expenses between home and work and the expenses of caring for a member of the parent’s family while he is absent from the home (regulation 24).
\end{itemize}

  Other amendments made are of a minor, technical, consequential or procedural nature.

  Copies of the rules of the Earnings Top-up Scheme referred to in regulations 5 and 18, may be obtained from the Customer Services Manager, Earnings Top-up, Norcross, Blackpool \textsc{FY5 3TA}.

  These Regulations do not impose any costs on business.

\end{document}