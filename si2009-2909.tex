\documentclass[12pt,a4paper]{article}

\newcommand\regstitle{The Child Support (Miscellaneous Amendments) (No.~2) Regulations 2009}

\newcommand\regsnumber{2009/2909}

%\opt{newrules}{
\title{\regstitle}
%}

%\opt{2012rules}{
%\title{Child Maintenance and~Other Payments Act 2008\\(2012 scheme version)}
%}

\author{S.I.\ 2009 No.\ 2909}

\date{Made
2nd November 2009\\
Laid before Parliament
9th November 2009\\
Coming into~force\\
for the purpose of regulations 1, 3(1) and (4), 4 and 5\\
10th November 2009\\
for the purpose of regulations 2 and 3(2) and (3)\\
4th December 2009
}

%\opt{oldrules}{\newcommand\versionyear{1993}}
%\opt{newrules}{\newcommand\versionyear{2003}}
%\opt{2012rules}{\newcommand\versionyear{2012}}

\usepackage{csa-regs}

\setlength\headheight{27.61603pt}

%\hbadness=10000

\begin{document}

\maketitle

\noindent
The Secretary of State for Work and Pensions, in exercise of the powers conferred by sections 17(3) and (5), 51(1) and (2)($b$), 52(4), 54 and 55(1)($c$)(ii), (3), (6) and (7) of the Child Support Act 1991\footnote{1991 c.~48. Section 17(3) and (5) was substituted by section~41 of the Social Security Act 1998 (c.~14) (“the 1998 Act”). Section 51(2)($b$)  was amended by section~86(1) of, and paragraph~46($b$)  of Schedule 7 to, the 1998 Act. Section 54 is cited for the meaning given to the word “prescribed”.}, makes the following Regulations: 

{\sloppy

\tableofcontents

}

\bigskip

\setcounter{secnumdepth}{-2}

\subsection[1. Citation, commencement and interpretation]{Citation, commencement and interpretation}

1.---(1)  These Regulations may be cited as the Child Support (Miscellaneous Amendments) (No.~2) Regulations 2009.

(2) Subject to paragraph~(3), these Regulations shall come into force on the day after the day on which they were laid before Parliament.

(3) Regulations 2 and 3(2) and (3) shall come into force on 4th December 2009.

(4) In these Regulations—
\begin{enumerate}\item[]
“the Act” means the Child Support Act 1991;

“the Maintenance Arrangements and Jurisdiction Regulations” means the Child Support (Maintenance Arrangements and Jurisdiction) Regulations 1992\footnote{S.I.~1992/2645.};

“the Maintenance Assessment Procedure Regulations” means the Child Support (Maintenance Assessment Procedure) Regulations 1992\footnote{S.I.~1992/1813, which is revoked with savings, by S.I.~2001/157.};

“the Maintenance Calculation Procedure Regulations” means the Child Support (Maintenance Calculation Procedure) Regulations 2000\footnote{S.I.~2001/157.}.
\end{enumerate}

\subsection[2. Amendment of the Maintenance Arrangements and Jurisdiction Regulations]{Amendment of the Maintenance Arrangements and Jurisdiction Regulations}

2.  Omit regulation~7 of the Maintenance Arrangements and Jurisdiction Regulations (cancellation of a maintenance assessment on grounds of lack of jurisdiction)\footnote{Regulation 7 was amended by S.I.~1993/913 and 1999/1510 and was revoked with savings by S.I.~2001/161.}.

\subsection[3. Amendment of the Maintenance Assessment Procedure Regulations]{Amendment of the Maintenance Assessment Procedure Regulations}

3.---(1)  The Maintenance Assessment Procedure Regulations are amended as follows.

(2) After paragraph~(3) of regulation~20 (supersession of decisions)\footnote{Regulation 20 was substituted by S.I.~1999/1047 and revoked with savings by S.I.~2001/157 (as amended by S.I.~2003/328, 2003/347 and 2004/2415) and S.I.~2000/3186 (as amended by S.I.~2004/2415). Relevant amending instruments are S.I.~2000/1596 and 2005/785.}, insert—
\begin{quotation}
“(3A) For the purposes of paragraph~2 of Schedule 4C to the Act\footnote{Schedule 4C was inserted by section~86(1) of, and paragraph~54 of Schedule 7 to, the 1998 Act.}, the circumstances in which a decision may be superseded under paragraph~(2) or (3) include where the material change of circumstances causes the maintenance assessment to cease by virtue of paragraph~16(1) of Schedule 1 to the Act or where the Commission no longer has jurisdiction by virtue of section~44 of the Act (jurisdiction)\footnote{Some of the words in section~44 were substituted by section~86(1) of, and paragraph~41 of Schedule 7 to, the 1998 Act and section~13(4) of, and paragraphs (1) and (46) of Schedule 3 to, the Child Maintenance and Other Payments Act 2008 (c.~6). Some words in section~44(1) and subsection~(2A) were inserted by section~22(1) to (3) of the Child Support, Pensions and Social Security Act 2000 (c.~19) and some of the words in section~44(2A)($c$)  were substituted by S.I.~2009/1941. There are other substitutions none of which are relevant to these Regulations.}.”.
\end{quotation}

(3) After paragraph~(21) of regulation~23 (date from which a decision is superseded)\footnote{Regulation 23 was substituted by S.I.~1999/1047 and revoked with savings by S.I.~2001/157 (as amended by S.I.~2003/328, 2003/347 and 2004/2415) and S.I.~2000/3186 (as amended by S.I.~2004/2415). There are other amendments none of which are relevant to these Regulations.}, insert—
\begin{quotation}
“(21A) Where a superseding decision is made in a case to which regulation~20(3A) applies and the material circumstance is—
\begin{enumerate}\item[]
($a$) a qualifying child dies or ceases to be a qualifying child;

($b$) the person with care ceases to be a person with care in relation to a qualifying child; or

($c$) the person with care, the absent parent or a qualifying child ceases to be habitually resident in the United Kingdom,
\end{enumerate}
the decision takes effect from the first day of the maintenance period in which the change occurred.”.
\end{quotation}

(4) In Schedule 1 (meaning of “child” for the purposes of the Act)\footnote{Schedule 1 was revoked with savings by S.I.~2001/157 (as amended by S.I.~2003/328, 2003/347 and 2004/2415) and S.I.~2000/3186 (as amended by S.I.~2004/2415). Relevant amending instruments are S.I.~1993/913, 1999/977, 1999/1047 and 2009/396.}—
\begin{enumerate}\item[]
($a$) for paragraph~1 (persons of 16 or 17 years of age who are not in full-time non-advanced education), substitute—
\begin{quotation}
“1.  The conditions which must be satisfied for a person to be a child within section~55(1)($c$)  of the Act are that the person—
\begin{enumerate}\item[]
($a$) is registered for training with a qualifying body; and

($b$) is a person in respect of whom child benefit is payable.
\end{enumerate}

\section*{\itshape Period for which a person is to be treated as continuing to fall within section~55(1) of the Act}

1A.  Where a person (“$P$”) has ceased to fall within section~55(1) of the Act, $P$ is to be treated as continuing to fall within that subsection~for any period during which $P$ is a person in respect of whom child benefit is payable.”;
\end{quotation}

($b$) for paragraph~2 (meaning of “advanced education” for the purposes of section~55 of the Act), substitute—
\begin{quotation}
“2.  For the purposes of section~55 of the Act “advanced education” means education for the purposes of—
\begin{enumerate}\item[]
($a$) a course in preparation for a degree, a diploma of higher education, a higher national diploma or a teaching qualification; or

($b$) any other course which is of a standard above ordinary national diploma, a national diploma or national certificate of Edexcel, a general certificate of education (advanced level) or Scottish national qualifications at higher or advanced higher level.”;
\end{enumerate}
\end{quotation}

($c$) for sub-paragraph~(2) of paragraph~4 (interruption of full-time education), substitute—
\begin{quotation}
“(2) The provisions of sub-paragraph~(1) do not apply to any period of interruption of a person’s full-time education which is followed immediately by a period during which child benefit ceases to be payable in respect of that person.”;
\end{quotation}

($d$) omit paragraph~5 (circumstances in which a person who has ceased to receive full-time education is to be treated as continuing to fall within section~55(1) of the Act);

($e$) for paragraph~6 (interpretation), substitute—
\begin{quotation}
“6.  In this Schedule “qualifying body” has the same meaning as in regulation~5(4) of the Child Benefit (General) Regulations 2006 (extension period: 16 and 17 year olds)\footnote{S.I.~2006/223.}.”.
\end{quotation}
\end{enumerate}

\subsection[4. Amendment of the Maintenance Calculation Procedure Regulations]{Amendment of the Maintenance Calculation Procedure Regulations}

4.---(1)  Schedule 1 to the Maintenance Calculation Procedure Regulations (meaning of “child” for the purposes of the Act)\footnote{Schedule 1 was amended by S.I.~2008/1554.} is amended as follows.

(2) For paragraph~1 (persons of 16 or 17 years of age who are not in full-time non-advanced education), substitute—
\begin{quotation}
“1.  The conditions which must be satisfied for a person to be a child within section~55(1)($c$)  of the Act are that the person—
\begin{enumerate}\item[]
($a$) is registered for training with a qualifying body; and

($b$) is a person in respect of whom child benefit is payable.
\end{enumerate}

\section*{\itshape\sloppy Period for which a person is to be treated as continuing to fall within section~55(1) of the Act}

1A.  Where a person (“$P$”) has ceased to fall within section~55(1) of the Act, $P$ is to be treated as continuing to fall within that subsection~for any period during which $P$ is a person in respect of whom child benefit is payable.”.
\end{quotation}

(3) For paragraph~2 (meaning of “advanced education” for the purposes of section~55 of the Act), substitute—
\begin{quotation}
“2.  For the purposes of section~55 of the Act “advanced education” means education for the purposes of—
\begin{enumerate}\item[]
($a$) a course in preparation for a degree, a diploma of higher education, a higher national diploma or a teaching qualification; or

($b$) any other course which is of a standard above ordinary national diploma, a national diploma or national certificate of Edexcel, a general certificate of education (advanced level) or Scottish national qualifications at higher or advanced higher level.”.
\end{enumerate}
\end{quotation}

(4) For sub-paragraph~(2) of paragraph~4 (interruption of full-time education), substitute—
\begin{quotation}
“(2) The provisions of sub-paragraph~(1) do not apply to any period of interruption of a person’s full-time education which is followed immediately by a period during which child benefit ceases to be payable in respect of that person.”.
\end{quotation}

(5) Omit paragraph~5 (circumstances in which a person who has ceased to receive full-time education is to be treated as continuing to fall within section~55(1) of the Act).

(6) For paragraph~6 (interpretation), substitute—
\begin{quotation}
“6.  In this Schedule “qualifying body” has the same meaning as in regulation~5(4) of the Child Benefit (General) Regulations 2006 (extension period: 16 and 17 year olds)\footnote{S.I.~2006/223.}.”.
\end{quotation}

\subsection[5. Transitional provisions---qualifying child]{Transitional provisions---qualifying child}

5.---(1)  Where the circumstances in paragraph~(2) apply the effective date of—
\begin{enumerate}\item[]
($a$) a maintenance assessment or maintenance calculation made following an application under section~4 or 7 of the Act; or

($b$) a supersession decision made under section~17 of the Act where the relevant change of circumstances is that a person has become a qualifying child by virtue of these Regulations,
\end{enumerate}
is the day on which this regulation~comes into force.

(2) The circumstances are—
\begin{enumerate}\item[]
($a$) before these Regulations came into force there was a maintenance assessment or maintenance calculation in force in relation to the qualifying child to whom the application or supersession relates;

($b$) a person (“$C$”) who was a qualifying child to whom that maintenance assessment or maintenance calculation relates, ceased to be a qualifying child on or after 10th April 2006 by virtue of no longer falling within the provisions of—
\begin{enumerate}\item[]
(i) Schedule 1 to the Maintenance Assessment Procedure Regulations (meaning of “child” for the purposes of the Act); or, as the case may be,

(ii) Schedule 1 to the Maintenance Calculation Procedure Regulations (meaning of “child” for the purposes of the Act); and
\end{enumerate}

($c$) child benefit was payable in respect of $C$ on the day $C$ ceased to be a qualifying child and is payable in respect of $C$ on the day on which this regulation~comes into force.
\end{enumerate}

(3) Where an application under section~4 or 7 of the Act is made in a case to which the circumstances in paragraph~(2) apply in respect of a maintenance assessment, the definition of “the relevant period” in regulation~28(3) of the Child Support (Transitional Provisions) Regulations 2000 (linking provisions)\footnote{S.I.~2000/3186, relevant amending instruments are S.I.~2002/1204 and 2008/2543.}, is modified as follows—
\begin{quotation}
“(3) For the purposes of paragraph~(1) “the relevant period” means the period starting on the day immediately before the day the maintenance assessment ceased to have effect under paragraph~16(1) of Schedule 1 to the Act, to the day that the application referred to in paragraph~(1) is made, in a case where the circumstances of regulation~5(2) of the Child Support (Miscellaneous Amendments) (No.~2) Regulations 2009 (transitional provisions---qualifying child)\footnote{S.I.~2009/2909.} apply.”.
\end{quotation}

\bigskip

\pagebreak[3]

Signed 
by authority of the 
Secretary of State for~Work and~Pensions.
%I concur
%By authority of the Lord Chancellor

{\raggedleft
\emph{Helen Goodman}\\*
%Secretary
%Minister
Parliamentary Under-Secretary 
of State\\*Department 
for~Work and~Pensions

}

2nd November 2009

\small

\part{Explanatory Note}

\renewcommand\parthead{— Explanatory Note}

\subsection*{(This note is not part of the Regulations)}

These Regulations amend the Child Support (Maintenance Arrangements and Jurisdiction) Regulations 1992, the Child Support (Maintenance Assessment Procedure) Regulations 1992 (“the 1992 Regulations”) and the Child Support (Maintenance Calculation Procedure) Regulations 2000 (“the 2000 Regulations”). They also make transitional provisions.

Regulation 2 amends the Child Support (Maintenance Arrangements and Jurisdiction) Regulations 1992. Regulation 7 of those Regulations is omitted in consequence of the changes made by regulations 3(2) and (3).

Regulation 3 amends the 1992 Regulations:
\begin{enumerate}\item[]
    paragraph~(2) inserts a new paragraph~(3A) into regulation~20 of the 1992 Regulations to provide a ground for superseding a decision where a maintenance assessment has ceased by virtue of paragraph~16(1) of Schedule 1 to the Child Support Act 1991 (c.~48) (“the Act”) or where the Commission has no jurisdiction by virtue of section~44 of the Act;

    paragraph~(3) inserts a new paragraph~(21A) into regulation~23 of the 1992 Regulations to provide that the day on which a decision made on the ground in new regulation~20(3A) of the 1992 Regulations takes effect, is the first day of the maintenance period in which the specified material change of circumstances occurred;

    paragraph~(4)($a$)  substitutes paragraph~1 of Schedule 1 to the 1992 Regulations. The new paragraph~1 of that Schedule provides the prescribed conditions for the purposes of section~55(1)($c$)  of the Act and inserts a new paragraph~1A into that Schedule to provide that, where a person ceases to fall within section~55(1) of the Act, a person is to be treated as continuing to fall within that subsection~for any period during which that person is a person in respect of whom child benefit is payable, this is subject to the absolute bar in section~55(8) of the Act;

    paragraph~(4)($b$)  substitutes paragraph~2 of that Schedule prescribing the meaning of “advanced education” for the purposes of section~55 of the Act, the substituted paragraph~mirrors the definition in regulation~1(3) of the Child Benefit (General) Regulations 2006;

    paragraph~(4)($c$)  substitutes paragraph~4(2) of that Schedule to provide that paragraph~4(1) of that Schedule (circumstances in which interruptions in full-time education are not taken into account) does not apply where a period of interruption to a person’s full-time education is followed immediately by a period during which child benefit ceases to be payable in respect of that person;

    paragraph~(4)($d$)  omits paragraph~5 of that Schedule. The provisions are otiose as a result of the amendments made by these Regulations inserting paragraph~1A into that Schedule as equivalent provision is made in respect of child benefit by regulation~7 of the Child Benefit (General) Regulations 2006;

    paragraph~(4)($e$)  substitutes the interpretation provision at paragraph~6 of that Schedule to reflect the amendments made by these Regulations. 
\end{enumerate}

Regulation 4 makes amendments to Schedule 1 to the 2000 Regulations. These amendments have the same effect as those made by regulation~3(4) amending Schedule 1 to the 1992 Regulations.

Regulation 5 makes transitional provision. Paragraphs (1) and (2) provide that the day on which certain maintenance assessments or maintenance calculations made on an application under section~4 or 7 of the Act or on a supersession decision made under section~17 of that Act are to take effect is the day on which regulation~5 comes into force, in specified circumstances. Paragraph (3) modifies regulation~28(3) of the Child Support (Transitional Provisions) Regulations 2000 (linking provisions) where an application made under section~4 or 7 of the Act in a case to which the circumstances in regulation~5(2) of these Regulations apply, to change the meaning of “the relevant period” for the purposes of regulation~28(1) of those Regulations.

A full impact assessment has not been published for this instrument as it has no impact on the private or voluntary sectors. 

\end{document}