\documentclass[12pt,a4paper]{article}

\newcommand\regstitle{The Social Security (Miscellaneous Amendments) Regulations 1998}

\newcommand\regsnumber{1998/563}

%\opt{newrules}{
\title{\regstitle}
%}

%\opt{2012rules}{
%\title{Child Maintenance and Other Payments Act 2008\\(2012 scheme version)}
%}

\author{S.I. 1998 No. 563}

\date{Made 4th March 1998\\Laid before Parliament 11th March 1998\\Coming into force in accordance with regulation 1
}

%\opt{oldrules}{\newcommand\versionyear{1993}}
%\opt{newrules}{\newcommand\versionyear{2003}}
%\opt{2012rules}{\newcommand\versionyear{2012}}

\usepackage{csa-regs}

\setlength\headheight{27.57402pt}

\begin{document}

\maketitle

\noindent
The Secretary of State for Social Security, in exercise of powers conferred upon her by sections 64(1), 68(4)($c$), 70(4), 71(6), 123(1), 124(1), 128(1), 129(1), 130(1)($a$), (2), (4) and (5), 131(1), (3) and (12), 135(1), 136(2), (3) and (5), 137(1), (2)($a$)  and ($i$), 146A, 147(1), 175(1), (4) and (5) of the Social Security Contributions and Benefits Act 1992\footnote{\frenchspacing 1992 c. 4; sections 123, 131, 135 and 137 were amended to have effect with respect to council tax benefit by Schedule 9 to the Local Government Finance Act 1992 (c. 14); section 146A was inserted by the Asylum and Immigration Act 1996 (c. 49); sections 137(1) and 147(1) are interpretation provisions and are cited because of the meaning ascribed to the word “prescribed”.}, sections 5(1)($k$)  and ($p$), 32(8), 75(2), 76(1), 189(1) and (3) and 191 of the Social Security Administration Act 1992\footnote{\frenchspacing 1992 c. 5; section 5(6) was inserted by section 120 of the Housing Act 1996 (c. 52); section 76 was amended to have effect with respect to council tax benefit by Schedule 9 to the Local Government Finance Act 1992 (c. 14); section 191 is an interpretation provision and is cited because of the meaning ascribed to the word “prescribe”.}, sections 4(5), 12(1), (2) and (4), 13(3), 35(1), 36(1), (2) and (4) of, and paragraphs 3 and 11 of Schedule 1 to the Jobseekers Act 1995\footnote{\frenchspacing 1995 c. 18; section 35(1) is an interpretation provision and is cited because of the meaning given to the words “prescribed” and “regulations”.}, sections 10(1), (5)($a$)  and ($b$)  and (7) and 26(1) to (3) of the Child Support Act 1995\footnote{\frenchspacing 1995 c. 34.}, paragraph 7A of Schedule 2 to the Abolition of Domestic Rates (Scotland) Act 1987\footnote{\frenchspacing 1987 c. 47 (“the 1987 Act”); paragraph 7A was inserted by paragraph 36(1) of Schedule 12 to the Local Government Finance Act 1988 (c. 41). The 1987 Act was repealed by Schedule 14 of the Local Government Finance Act 1992 (c. 14) but paragraph 7A of Schedule 2 continues to have effect for the purposes of amending the Community Charges (Deductions from Income Support) (Scotland) Regulations 1989 (S.I. 1989/507) (S.59) by virtue of Article 2 of the Local Government Finance Act 1992 (Recovery of Community Charge) Saving Order 1993 (S.I. 1993/1780). Paragraph 7A of Schedule 2 to the 1987 Act as continued in effect by virtue of that Order, was amended by paragraph 10 of Schedule 2 to the Jobseekers Act 1995 (c. 18).}, section 146(6), of and paragraph 6 of Schedule 4 to, the Local Government Finance Act 1988\footnote{\frenchspacing 1988 c. 41 (“the 1988 Act”); section 146(6) is cited because of the meaning given to the word “prescribed”. Paragraph 6 of Schedule 4 was repealed by Schedule 14 to the Local Government Finance Act 1992 (c. 14), but continues to have effect for the purposes of amending the Community Charges (Deductions from Income Support) (No. 2) Regulations 1990 (S.I. 1990/545) by virtue of Article 2 of the Local Government Finance Act 1992 (Recovery of Community Charge) Saving Order 1993 (S.I. 1993/1780). Paragraph 6 of Schedule 4 to the 1988 Act as continued in effect by virtue of that Order, was amended by paragraph 18 of Schedule 2 to the Jobseekers Act 1995.}, sections 24 and 30 of the Criminal Justice Act 1991\footnote{\frenchspacing 1991 c. 53.} and section 116(1) of, and paragraphs 1 and 6 of Schedule 4 and paragraph 6 of Schedule 8 to, the Local Government Finance Act 1992\footnote{\frenchspacing 1992 c. 14; paragraph 6 of Schedule 4 and paragraph 6 of Schedule 8 were amended by the Jobseekers Act 1995, Schedule 2, paragraphs 75 and 76 respectively. Section 116(1) is an interpretation provision and is cited because of the meaning given to the word “prescribed”.} and of all other powers enabling her in that behalf, after consultation, in respect of those provisions in these Regulations relating to housing benefit and council tax benefit, with organisations appearing to her to be representative of the authorities concerned\footnote{\frenchspacing \emph{See} section 176(1) of the Social Security Administration Act 1992 (c. 5).} and after agreement by the Social Security Advisory Committee that these Regulations should not be referred to it\footnote{\frenchspacing See sections 170 and 173(1)($b$) of the Social Security Administration Act 1992 (c. 5); paragraph 20 of Schedule 3 to the Child Support Act 1995 (c. 34) added that Act to the list of “relevant enactments” in respect of which regulations must normally be referred to the Committee.}, hereby makes the following Regulations: 

\enlargethispage{3.30339pt}


{\sloppy

\tableofcontents

}

\setcounter{secnumdepth}{-2}

\section[Part I --- General]{Part I\\*General}

\renewcommand\parthead{--- Part I}

\subsection[1. Citation, commencement and interpretation]{Citation, commencement and interpretation}

1.---(1)  These Regulations may be cited as the Social Security (Miscellaneous Amendments) Regulations 1998 and this regulation 
%and regulation 16 
and regulations 2, 3, 16, 18(2)($a$)  and ($d$)  to ($g$)  and 18(1) in so far as it relates those sub-paragraphs  % Words substituted (20.3.98) by SI 1998/865 reg 2
shall come into force on 1st April 1998.

\enlargethispage{3.30339pt}

(2) Except in the case of regulation 16, in so far as they amend provisions relating to income support or jobseeker’s allowance, these Regulations shall come into force on 6th April 1998 and in relation to a claimant for either income support or jobseeker’s allowance, these Regulations shall have effect from the first day of the first benefit week to commence for that claimant on or after that date and, in a case to which regulation 8(2)($c$)  and (3) applies, immediately after article 18(8) of, and Schedule 7 to, the Social Security Benefits Up-rating Order 1998\footnote{\frenchspacing S.I. 1998/470.} come into force.

\begin{sloppypar}
(3) In paragraph (2) above, the expressions “benefit week” and “claimant” with respect to income support shall have the same meaning as in regulation 2(1) of the Income Support Regulations\footnote{\frenchspacing The definition of “benefit week” was amended by S.I. 1988/1445.} and with respect to jobseeker’s allowance “benefit week” shall have the same meaning as in regulation 1(3) of the Jobseeker’s Allowance Regulations\footnote{\frenchspacing The definition of “benefit week” was amended by S.I. 1996/1517 and 2538.}.
\end{sloppypar}

(4) In so far as these Regulations amend provisions relating to council tax benefit they shall come into force on 1st April 1998.

(5) In so far as these Regulations amend provisions relating to disability working allowance or family credit, they shall come into force on 7th April 1998 and, in relation to any particular claimant for either of those benefits, these Regulations shall have effect where a claimant has an award of disability working allowance or family credit which is current on 7th April 1998, on the day following the expiration of that award.

(6) In so far as these Regulations amend provisions relating to housing benefit, they shall come into force—
\begin{enumerate}\item[]
($a$) in any case where rent is payable at intervals of a whole number of weeks, on 6th April 1998;

($b$) in any other case, on 1st April 1998.
\end{enumerate}

(7) In these Regulations—
\begin{enumerate}\item[]
“the Child Maintenance Bonus Regulations” means the Social Security (Child Maintenance Bonus) Regulations 1996\footnote{\frenchspacing S.I. 1996/3195; relevant amending instrument S.I. 1997/454.};

“the Council Tax Benefit Regulations” means the Council Tax Benefit (General) Regulations 1992\footnote{\frenchspacing S.I. 1992/1814; relevant amending instruments S.I. 1994/470, 1807, 1995/560, 1996/1944, 2303, 2432 and 1997/2197.};

“the Disability Working Allowance Regulations” means the Disability Working Allowance (General) Regulations 1991\footnote{\frenchspacing S.I. 1991/2887; relevant amending instruments S.I. 1995/2303, 1996/30, and 1997/2197.};

“the Family Credit Regulations” means the Family Credit (General) Regulations 1987\footnote{\frenchspacing S.I. 1987/1973; relevant amending instruments S.I. 1992/573, 1995/2303, 1996/30 and 1997/2197.};

“the Housing Benefit Regulations” means the Housing Benefit (General) Regulations 1987\footnote{\frenchspacing S.I. 1987/1971; relevant amending instruments S.I. 1989/1446, 1990/396 (S.45), 546, 1775, 1991/235, 1992/50, 432, 1993/1249, 1994/470, 1003, 1807, 1995/560, 1644, 2868, 1996/965, 1994, 2432, 1997/852, 1974 and 2197.};

“the Income Support Regulations” means the Income Support (General) Regulations 1987\footnote{\frenchspacing S.I. 1987/1967; relevant amending instruments S.I. 1988/999, 1445, 2022, 1989/1323, 1990/127, 547, 1549, 1991/235, 1175, 2742, 1992/468, 1101, 1993/963, 1249, 1994/527, 1804, 1995/516, 1996/965, 1944, 2431, 1997/65 and 2197.};

“the Jobseeker’s Allowance Regulations” means the Jobseeker’s Allowance Regulations 1996\footnote{\frenchspacing S.I. 1996/207; relevant amending instruments S.I. 1996/1517, 2538 and 1997/65.}.
\end{enumerate}

\amendment{
Words substituted in reg. 1(1) (20.3.98) by the Social Security (Miscellaneous Amendments) (No. 2) Regulations 1998 reg. 2.

\medskip

Reg. 2 (and therefore Pt. II) revoked (3.3.03) by the Social Security (Child Maintenance Premium and Miscellaneous Amendments) Regulations 2000 reg. 4(1)(d).
}

%\section[Part II --- Child maintenance bonus]{Part II\\*Child maintenance bonus}
%
%\renewcommand\parthead{--- Part II}
%
%\subsection[2. Amendments to the Child Maintenance Bonus Regulations]{Amendments to the Child Maintenance Bonus Regulations}
%
%2.---(1)  The Child Maintenance Bonus Regulations shall be amended in accordance with the following paragraphs.
%
%(2) In regulation 1(2) (interpretation) for the definition of “child maintenance” there shall be substituted the following definition—
%\begin{quotation}
%    ““child maintenance” means maintenance in any of the following forms—
%\begin{enumerate}\item[]
%    ($a$) 
%    child support maintenance paid or payable;
%
%    ($b$) 
%    maintenance paid or payable by an absent parent to a person with care of a qualifying child, under an agreement (whether enforceable or not) between them, or by virtue of an order of a court; or
%
%    ($c$) 
%    maintenance deducted from any benefit payable to an absent parent who is liable to maintain a qualifying child,
%\end{enumerate}
%    which, as the case may be, is paid, payable or deducted on or after 1st April 1998, but does not include any maintenance paid or payable in respect of a former partner;”. 
%\end{quotation}
%
%(3) In regulation 3(1) (entitlement to a bonus) for sub-paragraph ($f$)  there shall be substituted the following sub-paragraph—
%\begin{quotation}
%“($f$) the work condition is satisfied within the period of—
%\begin{enumerate}\item[]
%(i) in a case where an applicant with care cares for one child only and that child dies, 12 months immediately following the date of death;
%
%(ii) in a case where the absent parent has—
%\begin{enumerate}\item[]
%($aa$) died;
%
%($bb$) ceased to be habitually resident in the United Kingdom; or
%
%($cc$) has been found not to be the parent of the qualifying child or children,
%\end{enumerate}
%12 weeks immediately following the first date on which any of those events occurs;
%
%(iii) in any other case, 14 days immediately following the day on which the bonus period applying to the applicant comes to an end.”.
%\end{enumerate}
%\end{quotation}
%
%(4) In regulation 4(1) (bonus period) for head (i)  of sub-paragraph ($c$)\footnote{\frenchspacing Sub-paragraph ($c$)(i) was substituted by S.I. 1997/454.} there shall be substituted the following head—
%\begin{quotation}
%“(i) paid or payable to the applicant; or”.
%\end{quotation}
%
%(5) In regulation 10(1) (claiming a bonus)—
%\begin{enumerate}\item[]
%($a$) in sub-paragraph ($b$), the words “($c$)  or” shall be omitted; and
%
%($b$) sub-paragraph ($c$)  shall be omitted.
%\end{enumerate}

%\addtocontents{toc}{\protect\enlargethispage{\baselineskip}}

\section[Part III --- Amendments with respect to income-related benefits and jobseeker's allowance]{Part III\\*Amendments with respect to income-related benefits and jobseeker's allowance}

\renewcommand\parthead{--- Part III}

\subsection[3. Common amendments: Deductions from income support]{Common amendments: Deductions from income support}

3.---(1)  In regulation 1(2) (citation, commencement and interpretation) of each of the Regulations specified in paragraph (2) below, the following definitions shall be inserted in the appropriate places—
\begin{quotation}
    ““contribution-based jobseeker’s allowance”, except in a case to which paragraph ($b$)  of the definition of income-based jobseeker’s allowance applies, means a contribution-based jobseeker’s allowance under Part I of the Jobseekers Act 1995\footnote{\frenchspacing 1995 c. 18.}, but does not include any back to work bonus under section 26 of the Jobseekers Act which is paid as jobseeker’s allowance;”; 

    ““income-based jobseeker’s allowance” means—
\begin{enumerate}\item[]
    ($a$) 
    an income-based jobseeker’s allowance under Part I of the Jobseekers Act 1995; and

    ($b$) 
    in a case where, if there was no entitlement to contribution\hspace{0pt}-based jobseeker’s allowance, there would be entitlement to income-based jobseeker’s allowance at the same rate, contribution-based jobseeker’s allowance,
\end{enumerate}\item[]
    but does not include any back to work bonus under section 26 of the Jobseekers Act which is paid as jobseeker’s allowance;”. 
\end{quotation}

(2) The Regulations specified in this paragraph are—
\begin{enumerate}\item[]
($a$) the Community Charges (Deductions from Income Support) (No.\ 2) Regulations 1990\footnote{\frenchspacing S.I. 1990/545; relevant amending instrument S.I. 1996/2344.};

($b$) the Community Charges (Deductions from Income Support) (Scotland) Regulations 1989\footnote{\frenchspacing S.I. 1989/507; relevant amending instrument S.I. 1996/2344.};

($c$) the Council Tax (Deductions from Income Support) Regulations 1993\footnote{\frenchspacing S.I. 1993/494; relevant amending instrument S.I. 1996/2344.}; and

($d$) the Fines (Deductions from Income Support) Regulations 1992\footnote{\frenchspacing S.I. 1992/2182; relevant amending instrument S.I. 1996/2344.}.
\end{enumerate}

\subsection[4. Common amendments: Disregard of contribution]{Common amendments: Disregard of contribution}

4.---(1)  In each of the regulations specified in paragraph (2) below (students) in the definition of “contribution”\footnote{\frenchspacing The definition of “contribution” was substituted in the regulations referred to in paragraph (2)($a$), ($d$) and ($e$) by S.I. 1996/1944.}, after the words “in respect of the income” there shall be inserted the words “of a student or”.

(2) The regulations specified in this paragraph are—
\begin{enumerate}\item[]
($a$) regulation 38 of the Council Tax Benefit Regulations;

($b$) regulation 41 of the Disability Working Allowance Regulations;

($c$) regulation 37 of the Family Credit Regulations;

($d$) regulation 46 of the Housing Benefit Regulations;

($e$) regulation 61 of the Income Support Regulations; and

($f$) regulation 130 of the Jobseeker’s Allowance Regulations.
\end{enumerate}

(3) In each of the Regulations specified in paragraph (4) below, the following regulation shall be inserted and numbered in accordance with that paragraph—
\begin{quotation}
\subsection*{“Further disregard of student’s income}

Where any part of a student’s income has already been taken into account for the purposes of assessing his entitlement to a grant, the amount taken into account shall be disregarded in assessing that student’s income.”.
\end{quotation}

(4) The Regulations and regulation numbers specified in this paragraph are—
\begin{enumerate}\item[]
($a$) regulation 48A of the Council Tax Benefit Regulations;

($b$) regulation 48A of the Disability Working Allowance Regulations;

($c$) regulation 43A of the Family Credit Regulations;

($d$) regulation 58A of the Housing Benefit Regulations;

($e$) regulation 67A of the Income Support Regulations; and

($f$) regulation 137A of the Jobseeker’s Allowance Regulations.
\end{enumerate}

\subsection[5. Common amendments: Interpretation]{Common amendments: Interpretation}

5.---(1)  In each of the regulations specified in paragraph (2) below (interpretation) there shall be inserted the following definition in the appropriate place—
\begin{quotation}
““the Children Order” means the Children (Northern Ireland) Order 1995\footnote{\frenchspacing S.I. 1995/755 (N.I. 2).};”.
\end{quotation}

(2) The regulations specified in this paragraph are—
\begin{enumerate}\item[]
($a$) regulation 2(1) of the Council Tax Benefit Regulations;

($b$) regulation 2(1) of the Disability Working Allowance Regulations;

($c$) regulation 2(1) of the Family Credit Regulations;

($d$) regulation 2(1) of the Housing Benefit Regulations;

($e$) regulation 2(1) of the Income Support Regulations; and

($f$) regulation 1(3) of the Jobseeker’s Allowance Regulations.
\end{enumerate}

\subsection[6. Common amendments: Notional income]{Common amendments: Notional income}

6.---(1)  In each of the regulations specified in paragraph (2) below (notional income), the word “or” and the following sub-paragraph shall be added bearing the specified letter—
\begin{quotation}
    “rehabilitation allowance made under section 2 of the Employment and Training Act 1973\footnote{\frenchspacing 1973 c. 50, as amended by section 25 of the Employment Act 1988 (c. 19).}.”. 
\end{quotation}

(2) The regulations specified in this paragraph are—
\begin{enumerate}\item[]
($a$) regulation 26(2)($e$)  of the Council Tax Benefit Regulations\footnote{\frenchspacing Paragraph (2)($d$) was added by S.I. 1997/2197.\label{fn:6(2)(a)}};

($b$) regulation 29(2)($e$)  of the Disability Working Allowance Regulations\footref{fn:6(2)(a)};

($c$) regulation 26(2)($e$)  of the Family Credit Regulations\footref{fn:6(2)(a)};

($d$) regulation 35(2)($e$)  of the Housing Benefit Regulations\footref{fn:6(2)(a)};

($e$) regulation 42(2)($j$)  of the Income Support Regulations\footnote{\frenchspacing Paragraph (2)($i$) was added by S.I. 1997/2197.}; and

($f$) regulation 105(2)($i$)  of the Jobseeker’s Allowance Regulations\footnote{\frenchspacing Paragraph (2)($h$) was added by S.I. 1997/2197.}.
\end{enumerate}

\subsection[7. Common amendments: Disregard of income other than earnings]{\sloppy Common amendments: Disregard of income other than earnings}

7.---(1)  In each of the paragraphs of the Schedules to the Regulations specified in paragraph (2) below (sums to be disregarded in the calculation of income other than earnings) the following head shall be added bearing the specified letter—
\begin{quotation}
    “which is a payment made by an authority, as defined in Article 2 of the Children Order, in pursuance of Article 15 of, and paragraph 17 of Schedule 1 to, that Order (contribution by an authority to child’s maintenance);”. 
\end{quotation}

(2) The paragraphs of the Schedules specified in this paragraph and the specified head in those paragraphs are—
\begin{enumerate}\item[]
($a$) head ($c$)  of paragraph 24(1) of Schedule 4 to the Council Tax Benefit Regulations;

($b$) head ($c$)  of paragraph 22(1) of Schedule 3 to the Disability Working Allowance Regulations;

($c$) head ($c$)  of paragraph 22(1) of Schedule 2 to the Family Credit Regulations\footnote{\frenchspacing Paragraph 22(1)($b$) was substituted by S.I. 1992/573.};

($d$) head ($c$)  of paragraph 23(1) of Schedule 4 to the Housing Benefit Regulations\footnote{\frenchspacing Paragraph 23(1)($b$) was substituted by S.I. 1992/432.}; and

($e$) head ($d$)  of paragraph 26(1) of Schedule 7 to the Jobseeker’s Allowance Regulations.
\end{enumerate}

(3) For each of the paragraphs of the Schedules to the Regulations specified in paragraph (4) below (sums to be disregarded in the calculation of income other than earnings), there shall be substituted the following paragraph—
\begin{quotation}
    “Any payment made to the claimant or his partner for a person (“the person concerned”), who is not normally a member of the claimant’s household but is temporarily in his care, by—
\begin{enumerate}\item[]
    ($a$) 
    a health authority;

    ($b$) 
    a local authority;

    ($c$) 
    a voluntary organisation; or

    ($d$) 
    the person concerned pursuant to section 26(3A) of the National Assistance Act 1948\footnote{\frenchspacing 11 \& 12 Geo. 6 c. 29; section 26(3A) was inserted by section 42(4) of the National Health Service and Community Care Act 1990 (c. 19).}.”. 
\end{enumerate}
\end{quotation}

(4) The paragraphs of the Schedules to the Regulations specified in this paragraph are—
\begin{enumerate}\item[]
($a$) paragraph 26 of Schedule 4 to the Council Tax Benefit Regulations;

($b$) paragraph 24 of Schedule 3 to the Disability Working Allowance Regulations;

($c$) paragraph 24 of Schedule 2 to the Family Credit Regulations;

($d$) paragraph 25 of Schedule 4 to the Housing Benefit Regulations;

($e$) paragraph 27 of Schedule 9 to the Income Support Regulations; and

($f$) paragraph 28 of Schedule 7 to the Jobseeker’s Allowance Regulations.
\end{enumerate}

(5) In each of the paragraphs in the Schedules to the Regulations specified in paragraph (6) below for the words “paragraphs 13” there shall be substituted the words “paragraphs 13(1)”.

(6) The paragraphs in the Schedules to the Regulations specified in this paragraph are—
\begin{enumerate}\item[]
($a$) paragraph 34 of Schedule 4 to the Council Tax Benefit Regulations; and

($b$) paragraph 33 of Schedule 4 to the Housing Benefit Regulations.
\end{enumerate}

\subsection[8. Council Tax Benefit, Housing Benefit, Income Support and Jobseeker’s Allowance: Amendments with respect to persons in detention]{Council Tax Benefit, Housing Benefit, Income Support and Jobseeker’s Allowance: Amendments with respect to persons in detention}

8.---(1)  For the words referred to in each of the provisions specified in paragraph (2) below there shall be substituted the words “who is detained in hospital under the provisions of the Mental Health Act 1983\footnote{\frenchspacing 1983 c. 72.}, or, in Scotland, under the provisions of the Mental Health (Scotland) Act 1984\footnote{\frenchspacing 1984 c. 36.} or the Criminal Procedure (Scotland) Act 1995\footnote{\frenchspacing 1995 c. 46.},”.

(2) The words referred to and the provisions specified in this paragraph are—
\begin{enumerate}\item[]
($a$) in the Council Tax Benefit Regulations in regulation 4B (circumstances in which a person is or is not to be treated as occupying a dwelling as his home)\footnote{\frenchspacing Regulation 4B was inserted by S.I. 1995/560.} the words from “who is detained” to “1984”;

\enlargethispage{\baselineskip}

($b$) in the Housing Benefit Regulations—
\begin{enumerate}\item[]
(i) in regulation 5(8A) (circumstances in which a person is to be treated as liable to make payments in respect of a dwelling)\footnote{\frenchspacing Paragraph (8A) was inserted by S.I. 1995/560.} the words from “who is detained” to “1984”;

(ii) in regulation 63(7)($e$)(iii)  (non-dependant deductions)\footnote{\frenchspacing Regulation 63(7)($e$) was added by S.I. 1992/50.} the words from “whose detention is under” to “1984”;
\end{enumerate}

($c$) in the Income Support Regulations—
\begin{enumerate}\item[]
(i) in regulation 21(3) (special cases) in the definition of “prisoner”\footnote{\frenchspacing The definition of “prisoner” was substituted by S.I. 1995/516.}, the words from “whose detention is” to the end of the definition;

(ii) in paragraph 2A of Schedule 7 (applicable amounts in special cases)\footnote{\frenchspacing Paragraph 2A was substituted in Schedule 7 by S.I. 1998/470.} the words from “who is detained” to “1984”; and
\end{enumerate}

($d$) in the Jobseeker’s Allowance Regulations in regulation 85(4) (special cases) in the definition of “prisoner” the words from “whose detention is” to “1984”.
\end{enumerate}

(3) In paragraph 2A of Schedule 7 to the Income Support Regulations for the words “under either of those Acts” there shall be substituted the words “under any of those Acts”.

\subsection[9. Housing Benefit: Maximum rent]{Housing Benefit: Maximum rent}

9.  Regulation 11 of the Housing Benefit Regulations (maximum rent)\footnote{\frenchspacing Regulation 11 was substituted by S.I. 1995/1644.} shall be amended in accordance with the following paragraphs—
\begin{enumerate}\item[]
($a$) in paragraph (8C)\footnote{\frenchspacing Paragraph (8C) was inserted by S.I. 1995/2868.} for the words “paragraph 8(2A)”\footnote{\frenchspacing Words substituted by S.I. 1997/852; relevant amending instrument S.I. 1997/852.} there shall be substituted the words “paragraph 10”;

($b$) in paragraph (13)—
\begin{enumerate}\item[]
(i) in the definition of “property-specific rent” for the words “paragraph 1(2)” there shall be substituted “paragraph 1” and for the words “paragraph 3” there shall be substituted the words “paragraph 3(3)”;

(ii) for the definition of “the Rent Officers order” the following definition shall be substituted and inserted after the definition of “relevant rent”—
\begin{quotation}
““the Rent Officers Order” means the Rent Officers (Housing Benefit Functions) Order 1997\footnote{\frenchspacing S.I. 1997/1984.} or, as the case may be, the Rent Officers (Housing Benefit Functions) Scotland Order 1997\footnote{\frenchspacing S.I. 1997/1995 (S.144).};”;
\end{quotation}

(iii) in the definition of “relevant rent” for the words “paragraph 5” there shall be substituted the words “paragraph 6”; and

(iv) in the definition of “single room rent”\footnote{\frenchspacing This definition was inserted by S.I. 1996/965.} for the words “paragraph 4A” there shall be substituted the words “paragraph 5”.
\end{enumerate}
\end{enumerate}

\subsection[10. Housing Benefit: Requirement to refer to rent officers]{Housing Benefit: Requirement to refer to rent officers}

10.  In paragraph (1A) of regulation 12A of the Housing Benefit Regulations\footnote{\frenchspacing Regulation 12A was inserted by S.I. 1990/546 and paragraph (1A) was inserted by S.I. 1995/2868.} (requirement to refer to rent officers) after sub-paragraph ($b$), the word “and” and the following sub-paragraph shall be added—
\begin{quotation}
“($c$) where the accommodation is supported accommodation within the meaning of paragraph 7 of Schedule 1\footnote{\frenchspacing Paragraph 7 was added by S.I. 1997/1974.} and either—
\begin{enumerate}\item[]
(i) they include charges that are eligible for housing benefit solely by virtue of the exception in paragraph 1($f$)(iii)  of Schedule 1 (charges for general counselling and support in supported accommodation)\footnote{\frenchspacing Paragraph 1($f$) was substituted by S.I. 1994/1003 and head (iii) was added by S.I. 1997/1974.} or by virtue of that exception and the exception in paragraph 1($f$)(ii)—
\begin{enumerate}\item[]
($aa$) that such charges are included; and

($bb$) the value of those charges as determined by the local authority for the purposes of regulation 10 (rent) and Schedule 1; or
\end{enumerate}

(ii) they include charges that are eligible for housing benefit by virtue of the exception in paragraph 1($f$)(iii)  of Schedule 1 or by virtue of that exception and the exception in paragraph 1($f$)(ii)  and those charges are also eligible by virtue of the exception in paragraph 1($f$)(i)  (charges with respect the provision of adequate accommodation)—
\begin{enumerate}\item[]
($aa$) that such charges are included; and

($bb$) the value of those charges as determined by the local authority for the purposes of regulation 10 and Schedule 1.”.
\end{enumerate}
\end{enumerate}
\end{quotation}

\subsection[11. Housing Benefit: Excluded tenancies]{Housing Benefit: Excluded tenancies}

11.  Schedule 1A to the Housing Benefit Regulations\footnote{\frenchspacing Schedule 1A was inserted by S.I. 1990/546; relevant amending instruments S.I. 1991/235, 1995/560 and 1996/965.} (excluded tenancies) shall be amended in accordance with the following paragraphs—
\begin{enumerate}\item[]
($a$) in paragraph 2(3)($c$)\footnote{\frenchspacing Paragraph 2(3)($c$) was amended by S.I. 1995/560.}—
\begin{enumerate}\item[]
(i) after the words “the same as such a term)” there shall be added the words “and that determination was not made under paragraph 1(2), 2(2) or 3(3) of Schedule 1 to the Rent Officers Order”; and

(ii) the word “if” and heads (i)  and (ii)  shall be omitted;
\end{enumerate}

($b$) in paragraph 2(3)($e$)\footnote{\frenchspacing Head ($e$) was added to paragraph 2(3) of Schedule 1A by S.I. 1991/235 and amended by S.I. 1995/560.} for the words from “for the purposes” to the words “(Scotland) Order 1990” there shall be substituted the words “under paragraph 2(2) of Schedule 1 to the Rent Officers Order”;

($c$) in paragraph 2(3)($f$)\footnote{\frenchspacing Head ($f$) was added to paragraph 2(3) of Schedule 1A by S.I. 1996/965.} for the words “paragraph 4A” there shall be substituted the words “paragraph 5”;

($d$) in paragraph 12—
\begin{enumerate}\item[]
(i) after the definition of “rent” the following definition shall be inserted—
\begin{quotation}
““the Rent Officers Order” means the Rent Officers (Housing Benefit Functions) Order 1997 or, as the case may be, the Rent Officers (Housing Benefit Functions) Scotland Order 1997\footnote{\frenchspacing S.I. 1997/1995 (S.144).};”;
\end{quotation}

(ii) the following definitions in paragraph 12 shall be omitted—
\begin{enumerate}\item[]
($aa$) “the Order”;

($bb$) “the relevant provisions”\footnote{\frenchspacing Amended by S.I. 1990/1775.}; and

($cc$) “the Scottish Order”\footnote{\frenchspacing The definition of “the Scottish Order” was amended by S.I. 1989/1446, 1990/396 (S.45), 1993/1249 and 1995/1644.}.
\end{enumerate}
\end{enumerate}
\end{enumerate}

\subsection[12. Income Support and Jobseeker’s Allowance: Treatment of grant awards for former students]{Income Support and Jobseeker’s Allowance: Treatment of grant awards for former students}

12.  In regulation 29(2B) of the Income Support Regulations\footnote{\frenchspacing Paragraph (2B) was inserted in regulation 29 by S.I. 1997/65.} and regulation 94(2B) of the Jobseeker’s Allowance Regulations\footnote{\frenchspacing Paragraph (2B) was inserted in regulation 94 by S.I. 1997/65.} (calculation of earnings derived from employed earner’s employment and income other than earnings) after sub-paragraph ($a$)  the following sub-paragraph shall be inserted—
\begin{quotation}
“($aa$) where the grant is paid in instalments, on the day before the next instalment would have been paid had the claimant remained a student; or”.
\end{quotation}

\subsection[13. Income Support and Jobseeker’s Allowance: Calculation of notional income and income other than earnings]{Income Support and Jobseeker’s Allowance: Calculation of notional income and income other than earnings}

13.---(1)  The Income Support Regulations shall be amended in accordance with the following sub-paragraphs—
\begin{enumerate}\item[]
($a$) in regulation 40(1) (calculation of income other than earnings) for the words “paragraphs (2) to (3A)\footnote{\frenchspacing These words were substituted by S.I. 1990/1549.}” there shall be substituted the words “paragraphs (2) to (3B)\footnote{\frenchspacing Paragraph (3B) was inserted by S.I. 1997/65.}”;

($b$) in regulation 42(7) (notional income) for the words “paragraphs (1) to (4)” there shall be substituted the words “paragraphs (1) to (4A)\footnote{\frenchspacing Paragraph (4A) was inserted by S.I. 1994/527.}”;

($c$) in Schedule 9 (sums to be disregarded in the calculation of income other than earnings)\footnote{\frenchspacing Paragraph 63 was inserted in Schedule 9 by S.I. 1997/2863.} at the end there shall be added the following paragraph—
\begin{quotation}
“64.  Any payment made with respect to a person on account of the provision of after-care under section 117 of the Mental Health Act 1983\footnote{\frenchspacing 1983 c. 20.} or section 8 of the Mental Health (Scotland) Act 1984\footnote{\frenchspacing 1984 c. 30.} or the provision of accommodation or welfare services to which Parts III and IV of the National Health Service and Community Care Act 1990\footnote{\frenchspacing 1990 c. 19.} refer, which falls to be treated as notional income under paragraph (4A) of regulation 42 above (payments made in respect of a person in a residential care or nursing home).”.
\end{quotation}
\end{enumerate}

(2) In Schedule 7 to the Jobseeker’s Allowance Regulations (sums to be disregarded in the calculation of income other than earnings) at the end there shall be added the following paragraph—
\begin{quotation}
“63.  Any payment which falls to be treated as notional income made under paragraph (11) of regulation 105 above (payments made in respect of a person in a residential care or nursing home).”.
\end{quotation}

\subsection[14. Income Support and Jobseeker’s Allowance: Income treated as capital]{\sloppy \textls[25]{Income Support and Jobseeker’s Allowance: Income} treated as capital}

14.---(1)  For regulation 48(8) of the Income Support Regulations (income treated as capital), as it has effect in England and Wales and as it has effect in Scotland\footnote{\frenchspacing Regulation 48(8) as it has effect in England and Wales was substituted by S.I. 1992/468; paragraph (8) as it has effect in Scotland was saved by S.I. 1992/468, Schedule paragraph 11.}, there shall be substituted the following paragraphs—
\begin{quotation}
“(8) Any payment made by a local authority, which represents arrears of payments under—
\begin{enumerate}\item[]
($a$) paragraph 15 of Schedule 1 to the Children Act 1989 (power of a local authority to make contributions to a person with whom a child lives as a result of a residence order); or

($b$) section 34(6) or as the case may be, section 50 of the Children Act 1975\footnote{\frenchspacing 1975 c. 72; as amended by section 64 of the Domestic Proceedings and Magistrates Courts Act 1978 (c. 22).} (payments towards maintenance for children),
\end{enumerate}
shall be treated as capital.

(8A) Any payment made by an authority, as defined in Article 2 of the Children Order\footnote{\frenchspacing \emph{See} regulation 5(1) above.} which represents arrears of payments under Article 15 of, and paragraph 17 of Schedule 1 to, that Order (contribution by an authority to child’s maintenance), shall be treated as capital.”.
\end{quotation}

(2) In regulation 110(8) of the Jobseeker’s Allowance Regulations (income treated as capital), after the words “maintenance of a child)” there shall be inserted the words “or any payment, made by an authority, as defined in Article 2 of the Children Order, which represents arrears of payments under Article 15 of, and paragraph 17 of Schedule 1 to, that Order (contribution by an authority to child’s maintenance),”.

\subsection[15. Income Support: Disregard of income other than earnings]{Income Support: Disregard of income other than earnings}

15.---(1)  Paragraph 7 of Schedule 9 to the Income Regulations (sums to be disregarded in the calculation of income other than earnings) shall be amended in accordance with the following paragraphs—
\begin{enumerate}\item[]
($a$) in sub-paragraph ($a$)  for the words “, 9 or 9A”\footnote{\frenchspacing The words “9 or 9A” were substituted by S.I. 1991/2742; paragraph 9A was omitted by S.I. 1993/2119.} there shall be substituted the words “or 9”;

($b$) in sub-paragraph ($b$)  at the end there shall be added the words “or jobseeker’s allowance”.
\end{enumerate}

(2) For head ($b$)  of paragraph 25(1) of Schedule 9 to the Income Support Regulations (sums to be disregarded in the calculation of income other than earnings) as it has effect in England and Wales\footnote{\frenchspacing Head ($b$) of paragraph 25(1) was substituted with respect to England and Wales by S.I. 1992/468.} and as it has effect in Scotland the following heads shall be substituted—
\begin{quotation}
“($b$) which is a payment made by a local authority in pursuance of section 34(6) or, as the case may be, section 50 of the Children Act 1975 (contributions towards the cost of the accommodation and maintenance of a child);

($c$) which is a payment made by a local authority in pursuance of section 15(1) of, and paragraph 15 of Schedule 1 to, the Children Act 1989 (local authority contribution to a child’s maintenance where the child is living with a person as a result of a residence order);

($d$) which is a payment made by an authority, as defined in Article 2 of the Children Order, in pursuance of Article 15 of, and paragraph 17 of Schedule 1 to, that Order (contribution by an authority to child’s maintenance);”.
\end{quotation}

\subsection[16. Jobseeker’s Allowance: Periods of interruption of employment]{Jobseeker’s Allowance: Periods of interruption of employment}

16.---(1)  Regulation 47A of the Jobseeker’s Allowance Regulations (jobseeking periods: periods of interruption of employment)\footnote{\frenchspacing Regulation 47A was inserted by S.I. 1996/2538 and paragraph ($za$) was inserted by S.I. 1997/2677.} shall be renumbered as regulation 47A(1) and in sub-paragraph ($za$)  of that paragraph the words “and is still current on 1st December 1997” shall be omitted.

(2) After the renumbered paragraph (1) the following paragraph shall be added—
\begin{quotation}
“(2) In paragraph (1) “period of interruption of employment” in relation to a period prior to 7th October 1996 has the same meaning as it had in the Benefits Act by virtue of section 25A of that Act (determination of days for which unemployment benefit is payable)\footnote{\frenchspacing Section 25A was inserted by the Social Security (Incapacity for Work) Act 1994 (c. 18), Schedule 1 and repealed by the Jobseekers Act 1995 (c. 18), Schedule 3 paragraph 5.} as in force on 6th October 1996.”.
\end{quotation}

\subsection[17. Amendment of the Housing Benefit (General) Amendment Regulations 1995: Savings]{Amendment of the Housing Benefit (General) Amendment Regulations 1995: Savings}

17.  In regulation 10(6) of the Housing Benefit (General) Amendment Regulations 1995 (saving provision)\footnote{\frenchspacing S.I. 1995/1644; relevant amending instruments S.I. 1996/462 and 1944.} in the definition of “exempt accommodation”\footnote{\frenchspacing Sub-paragraph (ii) of the definition of “exempt accommodation” was amended by S.I. 1996/1944} for head (ii)  there shall be substituted the following head—
\begin{quotation}
“(ii) provided by a non-metropolitan county council in England within the meaning of section 1 of the Local Government Act 1972\footnote{\frenchspacing 1972 c. 71.}, a housing association, a registered charity or voluntary organisation where care, support or supervision is provided by, or on behalf of, that body to the occupants of that accommodation;”.
\end{quotation}

\section[Part IV --- Amendments with respect to persons from abroad]{\sloppy Part IV\\*\textls[50]{Amendments with respect to persons from} abroad}

\renewcommand\parthead{--- Part IV}

\subsection[18. Common amendments with respect to persons from abroad]{Common amendments with respect to persons from abroad}

18.---(1)  In each of the regulations specified in paragraph (2) below (conditions affecting the entitlement of a person from abroad to the relevant benefits), for the words “to remain in the United Kingdom by the Secretary of State” there shall be substituted the words—
\begin{quotation}
“(i) to enter the United Kingdom by an immigration officer appointed for the purposes of the Immigration Act 1971\footnote{\frenchspacing 1971 c. 77.}; or

(ii) to remain in the United Kingdom by the Secretary of State”.
\end{quotation}

(2) The regulations specified in this paragraph are—
\begin{enumerate}\item[]
($a$) regulation 14B($b$)  of the Child Benefit (General) Regulations 1976\footnote{\frenchspacing S.I. 1976/965; regulation 14B was inserted by S.I. 1996/2327; relevant amending instrument S.I. 1996/2530.};

($b$) regulation 5(1A)($b$)  of the Disability Working Allowance Regulations\footnote{\frenchspacing Paragraph (1A) was inserted in regulation 5 by S.I. 1996/30.};

($c$) regulation 3(1A)($b$)  of the Family Credit Regulations\footnote{\frenchspacing Paragraph (1A) was inserted in regulation 3 by S.I. 1996/30.};

($d$) regulation 2(1A)($b$)  of the Social Security (Attendance Allowance) Regulations 1991\footnote{\frenchspacing S.I. 1991/2740; paragraph (1A) was inserted in regulation 2 by S.I. 1996/30.};

($e$) regulation 2(1A)($b$)  of the Social Security (Disability Living Allowance) Regulations 1991\footnote{\frenchspacing S.I. 1991/2890; paragraph (1A) was inserted in regulation 2 by S.I. 1996/30.};

($f$) regulation 9(1A)($b$)  of the Social Security (Invalid Care Allowance) Regulations 1976\footnote{\frenchspacing S.I. 1976/409; paragraph (1A) was inserted in regulation 9 by S.I. 1996/30.}; and

($g$) regulation 3(1B)($b$)  of the Social Security (Severe Disablement Allowance) Regulations 1984\footnote{\frenchspacing S.I. 1984/1303; paragraph (1B) was inserted in regulation 3 by S.I. 1996/30.}.
\end{enumerate}

(3) In each of the regulations specified in paragraph (4) below (conditions affecting the entitlement of a person from abroad to the relevant benefits) after the words “exceptional leave” there shall be inserted the words “to enter the United Kingdom by an immigration officer within the meaning of the Immigration Act 1971, or”.

(4) The regulations specified in this paragraph are—
\begin{enumerate}\item[]
($a$) regulation 4A(4)($e$)(iii)  of the Council Tax Benefit Regulations\footnote{\frenchspacing Regulation 4A was inserted by S.I. 1994/470 and paragraph (4)($e$) was added by S.I. 1994/1807; relevant amending instruments S.I. 1996/1944 and 2432.};

($b$) regulation 7A(4)($e$)(iii)  of the Housing Benefit Regulations\footnote{\frenchspacing Regulation 7A was inserted by S.I. 1994/470 and paragraph (4)($e$) was added by S.I. 1994/1807; relevant amending instruments S.I. 1996/1944 and 2432.};

($c$) regulation 21(3) of the Income Support Regulations\footnote{\frenchspacing This definition was inserted in regulation 21(3) by S.I. 1994/1807; relevant amending instrument S.I.1996/1944.} in sub-paragraph ($c$)  of the second definition of “person from abroad”; and

($d$) regulation 85(4) of the Jobseeker’s Allowance Regulations\footnote{\frenchspacing Relevant amending instrument S.I. 1996/1516.} in sub-paragraph ($c$)  of the second definition of “person from abroad”.
\end{enumerate}

\subsection[19. Further amendments of the Income Support Regulations and the Jobseeker’s Allowance Regulations]{Further amendments of the Income Support Regulations and the Jobseeker’s Allowance Regulations}

19.---(1)  In regulation 72 of the Income Support Regulations\footnote{\frenchspacing Regulation 72 was amended by S.I. 1988/999 and 2022, 1989/1323, 1990/127, 1991/1175, 1992/1101, 1993/963 and 1249 and 1996/2431.} (assessment of income and capital in urgent cases)—
\begin{enumerate}\item[]
($a$) for sub-paragraph ($a$)  of paragraph (1) there shall be substituted the following sub-paragraph—
\begin{quotation}
“($a$) any income other than—
\begin{enumerate}\item[]
(i) a payment of income or income in kind made under the Macfarlane Trust, the Macfarlane (Special Payments) Trust, the Macfarlane (Special Payments) (No.\ 2) Trust, the Fund, the Eileen Trust or the Independent Living Funds; or

(ii) income to which paragraph 5, 7 (but only to the extent that a concessionary payment would be due under that paragraph for any non-payment of income support under regulation 70 of these Regulations or of jobseeker’s allowance under regulation 147 of the Jobseeker’s Allowance Regulations 1996 (urgent cases)), 31, 39(2), (3) or (4), 40, 42, 52 or 57 of Schedule 9 (disregard of income other than earnings) applies,
\end{enumerate}
possessed or treated as possessed by him, shall be taken into account in full notwithstanding any provision in that Part disregarding the whole or any part of that income;”;
\end{quotation}

($b$) in paragraph (2) after the words “Housing Benefits Act 1982” there shall be inserted the words “or any arrears of benefit due under regulation 70 of these Regulations or regulation 147 of the Jobseeker’s Allowance Regulations 1996 (urgent cases)”.
\end{enumerate}

(2) In regulation 149 of the Jobseeker’s Allowance Regulations (assessment of income and capital in urgent cases)—
\begin{enumerate}\item[]
($a$) for sub-paragraph ($a$)  of paragraph (1) there shall be substituted the following sub-paragraph—
\begin{quotation}
“($a$) any income other than—
\begin{enumerate}\item[]
(i) a payment of income or income in kind made under the Macfarlane Trust, the Macfarlane (Special Payments) Trust, the Macfarlane (Special Payments) (No.\ 2) Trust, the Fund, the Eileen Trust or the Independent Living Funds; or

(ii) income to which paragraph 6, 8 (but only to the extent that a concessionary payment would be due under that paragraph for any non-payment of jobseeker’s allowance under regulation 147 of these Regulations or of income support under regulation 70 of the Income Support Regulations (urgent cases)), 33, 41(2), (3) or (4) or 42 of Schedule 7 (disregard of income other than earnings) applies,
\end{enumerate}
possessed or treated as possessed by him, shall be taken into account in full notwithstanding any provision in that Part disregarding the whole or any part of that income;”;
\end{quotation}

($b$) in paragraph (2) after the words “Housing Benefits Act 1982” there shall be inserted the words “or any arrears of benefit due under regulation 147 of these Regulations or regulation 70 of the Income Support Regulations (urgent cases)”.
\end{enumerate}

\bigskip

Signed by authority of the Secretary of State for Social Security.

{\raggedleft
\emph{Keith Bradley}\\*Parliamentary Under-Secretary of
State,\\*Department of Social Security

}

4th March 1998

\small

\part{Explanatory Note}

\renewcommand\parthead{--- Explanatory Note}

\subsection*{(This note is not part of the Regulations)}

These Regulations amend—
\begin{enumerate}\item[]
($a$) the Social Security (Child Maintenance Bonus) Regulations 1996 (S.I.\ 1996/3195);

($b$) the Community Charges (Deductions from Income Support) (No.\ 2) Regulations 1990 (S.I.\ 1990/545), the Community Charge (Deductions from Income Support) (Scotland) Regulations 1989 (S.I.\ 1989/577), the Council Tax (Deductions from Income Support) Regulations 1993 (S.I.\ 1993/494) and the Fines (Deductions from Income Support) Regulations 1992 (S.I.\ 1992/2182) (collectively referred to below as “the Deductions from Income Support Regulations”);

($c$) the Council Tax Benefit (General) Regulations 1992 (S.I.\ 1992/1814), the Disability Working Allowance (General) Regulations 1991 (S.I.\ 1991/2887), the Family Credit (General) Regulations 1987 (S.I.\ 1987/1973), the Housing Benefit (General) Regulations 1987 (S.I.\ 1987/1971), the Income Support (General) Regulations 1987 (S.I.\ 1987/1967) (collectively referred to below as “the Income-related Benefits Regulations”) and the Jobseeker’s Allowance Regulations 1996 (S.I.\ 1996/207);
\end{enumerate}
in the following respects.

The Social Security (Child Maintenance Bonus) Regulations 1996 are amended with respect to the definition of “child maintenance”, to provide for the periods for which the work condition for entitlement to a bonus will remain satisfied in specified cases and to amend the conditions specifying what comprises a bonus period (regulation 2);

Additional definitions with respect to jobseeker’s allowance are inserted in the Deductions from Income Support Regulations (regulation 3);

The Income-related Benefits Regulations and the Jobseeker’s Allowance Regulations are amended—
\begin{enumerate}\item[]
($a$) with respect to the assessment of a student’s income (regulation 4);

($b$) with the addition of a definition of the Children (Northern Ireland) Order 1995 (regulation 5);

($c$) to exclude Rehabilitation Allowances from notional income (regulation 6);

($d$) with respect to the disregard of income other than earnings (regulation 7).
\end{enumerate}

The Council Tax Benefit, Housing Benefit, Income Support and Jobseeker’s Allowance Regulations are amended with respect to the meaning of persons in detention (regulation 8);

The Housing Benefit Regulations are amended so as to up-date definitions applicable for establishing a person’s maximum rent eligible for housing benefit, with respect to tenancies excluded from benefit and to amend the requirement to refer claims for benefit to a rent officer (regulations 9 to 11);

The Income Support and Jobseeker’s Allowance Regulations are amended with respect to the treatment of instalments of grant awards for former students and the calculation of notional income, income other than earnings and income treated as capital (regulations 12 to 15);

The Jobseeker’s Allowance Regulations are amended by extending the meaning of “period of interruption of employment” which is to form part of jobseeking periods and so as to provide that in relation to linked periods which span 6th October 1996, days of unemployment which form part of a period of interruption of employment, shall be treated as a jobseeking period in certain circumstances where such linked periods had already ended before 1st December 1997 (regulation 16);

A saving provision for housing benefit is amended with respect to the definition of exempt accommodation (regulation 17).

All the Regulations referred to in sub-paragraph ($c$)  of the first paragraph to this Note, together with the Child Benefit (General) Regulations 1976 (S.I.\ 1976/965), the Social Security (Attendance Allowance) Regulations 1991 (S.I.\ 1991/2740), the Social Security (Disability Living Allowance) Regulations 1991 (S.I.\ 1991/2890), the Social Security (Invalid Care Allowance) Regulations 1976 (S.I.\ 1976/409) and the Social Security (Severe Disablement Allowance) Regulations 1984 (S.I.\ 1984/1303) are amended so as to extend the right to benefit of persons granted exceptional leave to enter the United Kingdom by an immigration officer (regulation 18).

The Income Support Regulations and the Jobseeker’s Allowance Regulations are further amended so as to disregard concessionary payments in respect of unpaid income support, income-based jobseeker’s allowance and payments from the social fund, in calculating a person’s entitlement to urgent payments of either of those benefits (regulation 19).

These Regulations do not impose a charge on business. 

\end{document}