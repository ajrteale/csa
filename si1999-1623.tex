\documentclass[12pt,a4paper]{article}

\newcommand\regstitle{The Social Security and Child Support (Decisions and Appeals) Amendment (No.\ 2) Regulations 1999}

\newcommand\regsnumber{1999/1623}

%\opt{newrules}{
\title{\regstitle}
%}

%\opt{2012rules}{
%\title{Child Maintenance and Other Payments Act 2008\\(2012 scheme version)}
%}

\author{S.I. 1999 No. 1623}

\date{Made
9th June 1999\\
Laid before Parliament
14th June 1999\\
Coming into force
5th July 1999}

%\opt{oldrules}{\newcommand\versionyear{1993}}
%\opt{newrules}{\newcommand\versionyear{2003}}
%\opt{2012rules}{\newcommand\versionyear{2012}}

\usepackage{csa-regs}

\setlength\headheight{27.57402pt}

\begin{document}

\maketitle

\noindent
The Secretary of State for Social Security, in exercise of the powers conferred on him by sections 9(1), 10(3) and (6), 21(1)($b$), 79(1), (3), (4) and (6) and 84 of the Social Security Act 1998\footnote{\frenchspacing 1998 c. 14. Section 84 is cited because of the meaning ascribed to the word “prescribed”.}, by this Instrument which is made before the end of the period of six months beginning with the coming into force of sections 9, 10 and 21 of that Act\footnote{\frenchspacing See section 173(5)($a$) of the Social Security Administration Act 1992 (c. 5).}, hereby makes the follow Regulations: 

{\sloppy

\tableofcontents

}

\bigskip

\setcounter{secnumdepth}{-2}

\subsection[1. Citation, commencement and interpretation]{Citation, commencement and interpretation}

1.---(1)  These Regulations may be cited as the Social Security and Child Support (Decisions and Appeals) Amendment (No.\ 2) Regulations 1999 and shall come into force on 5th July 1999.

(2) In these Regulations “the principal Regulations” means the Social Security and Child Support (Decisions and Appeals) Regulations 1999\footnote{\frenchspacing S.I. 1999/991. The Regulations were amended by S.I. 1999/1466.}.

\subsection[2. Amendment of regulation 3 of the principal Regulations]{Amendment of regulation 3 of the principal Regulations}

2.  Regulation 3 of the principal Regulations shall be amended---
\begin{enumerate}\item[]
($a$) by the insertion in paragraph (5)($b$) before the word “where” of the words “except in the case of a disability benefit decision or an incapacity benefit decision where there has been an incapacity determination (whether before or after the decision)”, and

($b$) by the insertion after paragraph (5)($b$) of the following---
\begin{quotation}
“($c$) where the decision is a disability benefit decision, or is an incapacity benefit decision where there has been an incapacity determination (whether before or after the decision), which was made in ignorance or, or was based upon a mistake as to, some material fact in relation to a disability determination embodied in or necessary to the disability benefit decision, or the incapacity determination, and---
\begin{enumerate}\item[]
(i) as a result of that ignorance of or mistake as to that fact the decision was more advantageous to the claimant than it would otherwise have been but for that ignorance or mistake and,

(ii) the Secretary of State is satisfied that at the time the decision was made the claimant or payee knew or could reasonably have been expected at the time the decision was made to know of the fact in question and that it was relevant to the decision,”.
\end{enumerate}
\end{quotation}
\end{enumerate}

\subsection[3. Amendment of regulation 6 of the principal Regulations]{Amendment of regulation 6 of the principal Regulations}

3.  After sub-paragraph ($f$) of regulation 6(2) of the principal Regulations there shall be inserted---
\begin{quotation}
“($g$) is an incapacity benefit decision where there has been an incapacity determination (whether before or after the decision) and where, since the decision was made, the Secretary of State has received medical evidence following an examination in accordance with regulation 8 of the Social Security (Incapacity for Work) (General) Regulations 1995\footnote{\frenchspacing S.I. 1995/311; relevant amending instruments are S.I. 1995/987, 1996/3207 and 1997/1009.} from a doctor referred to in paragraph (1) of that regulation.”.
\end{quotation}

\subsection[4. Amendment of regulation 7 of the principal Regulations]{Amendment of regulation 7 of the principal Regulations}

4.  In regulation 7(2)($c$) of the principal Regulations for head (ii) there shall be substituted–
\begin{quotation}
“(ii) in the case of a disability benefit decision, or an incapacity benefit decision where there has been an incapacity determination (whether before or after the decision), where the Secretary of State is satisfied that in relation to a disability determination embodied in or necessary to the disability benefit decision, or the incapacity determination, the claimant or payee failed to notify an appropriate office of a change of circumstances which regulations under the Administration Act required him to notify, and the claimant or payee, as the case may be, knew or could reasonably have been expected to know that the change of circumstances should have been notified---
\begin{enumerate}\item[]
($aa$) from the date on which the claimant or payee, as the case may be, ought to have notified the change of circumstances, or

($bb$) if more than one change has taken place between the date from which the decision to be superseded took effect and the date of the superseding decision, from the date on which the first change ought to have been notified, or
\end{enumerate}

(iii) in any other case, except in the case of a decision which supersedes a disability benefit decision, or an incapacity benefit decision where there has been an incapacity determination (whether before or after the decision), from the date of the change.”.
\end{quotation}

\subsection[5. Insertion of regulation 7A into the principal Regulations]{Insertion of regulation 7A into the principal Regulations}

5.  After regulation 7 of the principal Regulations there shall be inserted---
\begin{quotation}
\subsection*{“Definitions for the purposes of regulations 3(5)($c$), 6(2)($g$) and 7(2)($c$) and ancillary provisions}

7A.---(1)  For the purposes of regulations 3(5)($c$), 6(2)($g$) and 7(2)($c$)---
\begin{enumerate}\item[]
“disability benefit decision” means a decision to award a relevant benefit embodied in or necessary to which is a disability determination,

“disability determination” means–
\begin{enumerate}\item[]
($a$)
in the case of a decision as to an award of an attendance allowance or a disability living allowance, whether the person satisfies any of the conditions in section 64, 72(1) or 73(1) to (3), as the case may be, of the Contributions and Benefits Act,

($b$)
in the case of a decision as to an award of severe disablement allowance, whether the person is disabled for the purpose of section 68 of the Contributions and Benefits Act, or

($c$)
in the case of a decision as to an award of industrial injuries benefit, whether the existence or extent of any disablement is sufficient for the purposes of section 103 or 108 of the Contributions and Benefits Act or for the benefit to be paid at the rate which was in payment immediately prior to that decision;
\end{enumerate}

“incapacity benefit decision” means a decision to award a relevant benefit embodied in or necessary to which is a determination that a person is or is to be treated as incapable of work under Part XIIA of the Contributions and Benefits Act,

\begin{sloppypar}
“incapacity determination” means a determination whether a person is incapable of work by applying the all work test in regulation 24 of the Social Security (Incapacity for Work) (General) Regulations 1995 or whether a person is to be treated as incapable of work in accordance with regulation 10 (certain persons with a severe condition to be treated as incapable of work) or 27 (exceptional circumstances) of those Regulations, and
\end{sloppypar}

“payee” means a person to whom a benefit referred to in paragraph ($a$), ($b$) or ($c$) of the definition of “disability determination”, or a benefit referred to in the definition of “incapacity benefit decision” is payable.
\end{enumerate}

(2) Where a person’s receipt of or entitlement to a benefit (“the first benefit”) is a condition of his being entitled to any other benefit, allowance or advantage (“a second benefit”) and a decision is revised under regulation 3(5)($c$) or a superseding decision is made under regulation 6(2) to which regulation 7(2)($c$)(ii) applies, the effect of which is that the first benefit ceases to be payable, or becomes payable at a lower rate than was in payment immediately prior to that revision or supersession, a consequent decision as to his entitlement to the second benefit shall take effect from the date of the change in his entitlement to the first benefit.”.
\end{quotation}

\subsection[6. Amendment of regulation 20 of the principal Regulations]{Amendment of regulation 20 of the principal Regulations}

6.  In regulation 20(1) of the principal Regulations---
\begin{enumerate}\item[]
($a$) after the words “regulation 16” there shall be inserted the words “or 17”, and

($b$) after sub-paragraph ($c$) there shall be inserted the following sub-paragraph–
\begin{quotation}
“($d$) in a case to which regulation 17(5) applies, the Secretary of State is satisfied that the benefit suspended is properly payable and the requirements of regulation 17(4) have been satisfied.”.
\end{quotation}
\end{enumerate}

\subsection[7. Amendment of Schedule 4 to the principal Regulations]{Amendment of Schedule 4 to the principal Regulations}

7.  In column 1 of Schedule 4 to the principal Regulations for “1997/995” there shall be substituted “1997/955”. 

\bigskip

Signed 
by authority of the Secretary of State for Social Security.

{\raggedleft
\emph{Hugh Bayley
}\\*Parliamentary Under-Secretary of State,\\*Department of Social Security

}

9th June 1999

\small

\part{Explanatory Note}

\renewcommand\parthead{--- Explanatory Note}

\subsection*{(This note is not part of the Regulations)}

These Regulations amend the Social Security and Child Support (Decisions and Appeals) Regulations 1999 (“the principal Regulations”). Regulation 2 amends regulation 3(5) as regards decisions on disability or incapacity related benefits. Revision can take place on grounds connected with the claimant’s condition for error of law or ignorance of or mistake as to a material fact if the claimant or recipient of the benefit knew or could reasonably be expected to know of the fact and that it was relevant to the original decision.

Regulation 3 inserts paragraph ($g$) into regulation 6(2). This provides that a ground for superseding a decision where there has been a determination as to a person’s incapacity for work by applying the all work test may be receipt of new medical evidence from a doctor approved by the Secretary of State.

Regulation 4 amends regulation 7(2)($c$) as regards superseding decisions. The amendment applies where the supersession is of a decision on a disability or incapacity related benefit on grounds connected with the claimant’s condition and the claimant or recipient of the benefit knew, or could reasonably have been expected to know, of a relevant change of circumstances which he is under a duty to notify. The effective date of the supersession is the date from which the change, or, if more than one, the first relevant change, ought to have been notified. (The duty to notify changes is currently in regulation 32 of the Social Security (Claims and Payments) Regulations S.I.\ 1987/1968).

Regulation 5 inserts new regulation 7A, paragraph (1) of which defines terms for the purposes of regulation 3(2)($c$), 6(2)($g$) and 7(2)($c$) of the principal Regulations.

The revision and supersession rules apply to revised entitlement to benefits where payment of or entitlement to another benefit is a condition and entitlement ceases or reduces as a result of a revision under regulation 3(2)($c$) or a supersession under regulation 6(2) to which regulation 7(2)($c$)(ii) applies (new regulation 7A(2)).

Regulation 6 amends regulation 20 to provide for suspended benefits to be paid when the information requirements of regulation 17 have been satisfied. Regulation 7 corrects an error in the reference to revoked regulations in Schedule 4 to the principal Regulations.

The Regulations do not impose any charge on business. 

\end{document}