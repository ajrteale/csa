\documentclass[12pt,a4paper]{article}

\newcommand\regstitle{The Child Support (Maintenance Assessments and Special Cases) Regulations 1992}

\newcommand\regsnumber{1992/1815}

%\opt{newrules}{
\title{\regstitle}
%}

%\opt{2012rules}{
%\title{Child Maintenance and Other Payments Act 2008\\(2012 scheme version)}
%}

\author{S.I. 1992 No. 1815}

\date{Made 20th July 1992\\Coming into force 5th April 1993}

%\opt{oldrules}{\newcommand\versionyear{1993}}
%\opt{newrules}{\newcommand\versionyear{2003}}
%\opt{2012rules}{\newcommand\versionyear{2012}}

\usepackage{csa-regs}

\begin{document}

\maketitle

\noindent
 Whereas a draft of this instrument was laid before Parliament in accordance with section 52(2) of the Child Support Act 1991\footnote{\frenchspacing 1991 c. 48.} and approved by a resolution of each House of Parliament:

Now, therefore, the Secretary of State for Social Security, in exercise of the powers conferred by sections 42, 43, 51, 52(4) and 54\footnote{\frenchspacing Section 54 is cited because of the meaning ascribed to the word “prescribed”.} of, and paragraphs 1, 2 and 4 to 9 of Schedule 1 to, the Child Support Act 1991 and of all other powers enabling him in that behalf hereby makes the following Regulations: 

{\sloppy

\tableofcontents

}

\setcounter{secnumdepth}{-2}

\section[Part I --- General]{Part I\\*General}

\renewcommand\parthead{--- Part I}

\subsection[1. Citation, commencement and interpretation]{Citation, commencement and interpretation}

1.—(1) These Regulations may be cited as the Child Support (Maintenance Assessments and Special Cases) Regulations 1992 and shall come into force on 5th April 1993.

(2) In these Regulations unless the context otherwise requires—
\begin{enumerate}\item[]
“the Act” means the Child Support Act 1991;

%Definition of ``Child Benefit Rates Regulations'' inserted (7.4.97) by SI 1996/1803 reg 7
“Child Benefit Rates Regulations” means the Child Benefit and Social Security (Fixing and Adjustment of Rates) Regulations 1976\footnote{\frenchspacing S.I. 1976/1267 is amended by S.I. 1977/1328, 1980/110, 1991/502, 1993/965, 1995/559 and 1996/1803.};

“claimant” means a claimant for income support;

“Contributions and Benefits Act” means the Social Security Contributions and Benefits Act 1992\footnote{\frenchspacing 1992 c. 4.};

% Definition of ``Contributions and Benefits (Northern Ireland) Act'' inserted (13.1.97) by SI 1996/3196 reg 10(2)(b)
“Contributions and Benefits (Northern Ireland) Act” means the Social Security Contributions and Benefits (Northern Ireland) Act 1992\footnote{\frenchspacing 1992 c. 7.};

“council tax benefit” has the same meaning as in the Local Government Finance Act 1992\footnote{\frenchspacing 1992 c. 14.};

%Definition of ``couple'' inserted (5.4.93) by SI 1993/913 reg 19(2)($b$)
“couple” means a married or unmarried couple;

“course of advanced education” means---
\begin{enumerate}\item[]
($a$)
a full-time course leading to a postgraduate degree or comparable qualification, a first degree or comparable qualification, a Diploma of Higher Education, a higher national diploma, a higher national diploma or higher national certificate of the Business and 
%Technician 
Technology % Word substituted (5.4.93) by SI 1993/913 reg 19(2)($a$)
Education Council or the Scottish Vocational Education Council or a teaching qualification; or

($b$)
any other full-time course which is a course of a standard above that of an ordinary national diploma, a national diploma or national certificate of the Business and 
%Technician 
Technology % Word substituted (5.4.93) by SI 1993/913 reg 19(2)($a$)
Education Council or the Scottish Vocational Education Council, the advanced level of the General Certificate of Education, a Scottish certificate of education (higher level) or a Scottish certificate of sixth year studies;
\end{enumerate}

“covenant income” means the gross income payable to a student under a Deed of Covenant by a parent;

“day” includes any part of a day;

%“day to day care” means care of not less than 2 nights per week on average during—
%\begin{enumerate}\item[]
%($a$)
%the 12 month period ending with the relevant week; or
%
%($b$)
%such other period, ending with the relevant week, as in the opinion of the child support officer is more representative of the current arrangements for the care of the child in question;
%\end{enumerate}
%and for the purposes of this definition, where a child is a boarder at a boarding school or is an in-patient in a hospital, the person who, but for those circumstances, would otherwise provide day to day care of the child, shall be treated as providing day to day care during the periods in question.

%Definition of ``day to day care'' substituted (18.4.95) by SI 1995/1045 reg 41(2)(i)
“day to day care” means—
\begin{enumerate}\item[]
($a$) care of not less than 104 nights in total during the 12 month period ending with the relevant week; or

($b$) where, in the opinion of the child support officer, a period other than 12 months but ending with the relevant week is more representative of the current arrangements for the care of the child in question, care during that period is not less in total than the number of nights which bears the same ratio to 104 nights as that period bears to 12 months,
\end{enumerate}
and for the purpose of this definition—
\begin{enumerate}\item[]
(i) where a child is a boarder at a boarding school, or is an in-patient in a hospital, the person who, but for those circumstances, would otherwise provide day to day care of the child shall be treated as providing day to day care during the periods in question;

%(ii) “relevant week” shall have the meaning ascribed to it in head (ii) of sub-paragraph ($a$) of the definition of “relevant week” in this paragraph;

% Paras (ii)--(iv) substituted for para (ii) (22.1.96) by SI 1995/3261 reg 40(2)($a$)
(ii) in relation to an application for child support maintenance, “relevant week” shall have the meaning ascribed to it in head (ii) of sub-paragraph ($a$) of the definition of “relevant week” in this paragraph;

(iii) in relation to a review of a maintenance assessment under section 16 of the Act “relevant week” means the period of 7 days immediately preceding whichever is the later of the date on which a request is made to an absent parent or to a person with care for information or evidence under regulation 17(5) of the Maintenance Assessment Procedure Regulations; or

(iv) in relation to a review under section 17, 18(1)($a$), (1)($b$), (2) or (6A) or 19(1)($a$) to ($c$) or (6) of the Act, “relevant week” shall have the meaning ascribed to it in sub-paragraph ($a$), ($c$), ($d$), ($e$) or ($f$), as the case may be, of the definition of “relevant week” in this paragraph;
\end{enumerate}

% Definition of ``Departure Direction and Consequential Amendments Regulations'' inserted in reg. 1(2) (2.12.96) by SI 1996/2907 reg 68(2)
\begin{sloppypar}
“Departure Direction and Consequential Amendments Regulations” means the Child
Support Departure Direction and Consequential Amendments Regulations 1996\footnote{\frenchspacing S.I. 1996/2907.};
\end{sloppypar}

“disability working allowance” has the same meaning as in section 129 of the Contributions and Benefits Act;

“earnings” has the meaning assigned to it by paragraph 1 or 3, as the case may be, of Schedule 1;

% Definitions of ``earnings top-up'' and ``the Earnings Top-up Scheme'' inserted in reg. 1(2) (7.10.96) by SI 1996/1945 reg 18(2)
“earnings top-up” means the allowance paid by the Secretary of State under the rules specified in the Earnings Top-up Scheme;

“The Earnings Top-up Scheme” means the Earnings Top-up Scheme 1996\footnote{\frenchspacing This Scheme, which applies only in certain parts of Great Britain, is an extra-statutory Scheme, introduced by the Secretary of State for Social Security, having effect on 8th October 1996. Copies of the rules of this Scheme may be obtained from the Customer Services Manager, Earnings Top-up, Norcross, Blackpool \textsc{fy5 3ta}.};

“effective date” means the date on which a maintenance assessment takes effect for the purposes of the Act;

“eligible housing costs” shall be construed in accordance with Schedule 3;

“employed earner” has the same meaning as in section 2(1)($a$) of the Contributions and Benefits Act;

%“family” means—
%\begin{enumerate}\item[]
%($a$)
%a married or unmarried couple (including the members of a polygamous marriage) and any child or children living with them for whom at least one member of that couple has day to day care;
%
%($b$)
%where a person who is not a member of a married or unmarried couple has day to day care of a child, that person and any such child or children;
%\end{enumerate}
%and for the purposes of this definition a person shall not be treated as having day to day care of a child who is a member of that person’s household where the child in question is being looked after by a local authority within the meaning of section 22 of the Children Act 1989\footnote{\frenchspacing 1989 c. 41.} or, in Scotland, where the child is boarded out with that person by a local authority under the provisions of section 21 of the Social Work (Scotland) Act 1968\footnote{\frenchspacing 1968 c. 49.};

% Definition of ``family'' substituted (5.8.96) by SI 1996/1945 reg 18(3)
“family” means—
\begin{enumerate}\item[]
($a$) a married or unmarried couple (including the members of a polygamous marriage);

($b$) a married or unmarried couple (including the members of a polygamous marriage) and any child or children living with them for whom at least one member of that couple has day to day care;

($c$) where a person who is not a member of a married or unmarried couple has day to day care of a child or children, that person and any such child or children;
\end{enumerate}
and for the purposes of this definition a person shall not be treated as having day to day care of a child who is a member of that person’s household where the child in question is being looked after by a local authority within the meaning of section 22 of the Children Act 1989\footnote{\frenchspacing 1989 c. 41.} or, in Scotland, where the child is boarded out with that person by a local authority under the provisions of section 21 of the Social Work (Scotland) Act 1968\footnote{\frenchspacing 1968 c. 49.};

% Definition of ``family credit'' inserted (13.1.97) by SI 1996/3196 reg 10(2)(a)
“family credit” has the same meaning as in section 128 of the Contributions and Benefits Act;

“grant” means any kind of educational grant or award and includes any scholarship, exhibition, allowance or bursary but does not include a payment made under section 100 of the Education Act 1944\footnote{\frenchspacing 7 \& 8 Geo. 6 c. 6.} or section 73 of the Education (Scotland) Act 1980\footnote{\frenchspacing 1980 c. 44.};

“grant contribution” means any amount which a Minister of the Crown or an education authority treats as properly payable by another person when assessing the amount of a student’s grant and by which that amount is, as a consequence, reduced;

“home” means—
\begin{enumerate}\item[]
($a$)
the dwelling in which a person and any family of his normally live; or

($b$)
if he or they normally live in more than one home, the principal home of that person and any family of his,
\end{enumerate}
and for the purpose of determining the principal home in which a person normally lives no regard shall be had to residence in a residential care home or a nursing home during a period which does not exceed 52 weeks or, where it appears to the child support officer that the person will return to his principal home after that period has expired, such longer period as that officer considers reasonable to allow for the return of that person to that home;

“housing benefit” has the same meaning as in section 130 of the Contributions and Benefits Act;

“Housing Benefit Regulations” means the Housing Benefit (General) Regulations 1987\footnote{\frenchspacing S.I. 1987/1971; the relevant amending instruments are S.I. 1988/1444, 1989/416 and 1991/503, 2910.};

“Income Support Regulations” means the Income Support (General) Regulations 1987\footnote{\frenchspacing S.I. 1987/1967; the relevant amending instruments are S.I. 1988/663, 1228, 1445, 2022; 1989/534, 1034, 1678; 1990/547, 1168, 1776; 1991/236, 387, 503, 1559.};

%Definitions of ``Independent Living (1993) Fund'' and ``Independent Living (Extension) Fund inserted (5.4.93) by SI 1993/913 reg 19(1)($c$)
“Independent Living (1993) Fund” means the charitable trust of that name established by a deed made between the Secretary of State for Social Security of the one part and Robin Glover Wendt and John Fletcher Shepherd of the other part;

“Independent Living (Extension) Fund” means the charitable trust of that name established by a deed made between the Secretary of State for Social Security of the one part and Robin Glover Wendt and John Fletcher Shepherd of the other part;

%Definition of ``the Jobseekers Act'' inserted (7.10.96) by SI 1996/1345
“the Jobseekers Act” means the Jobseekers Act 1995\footnote{\frenchspacing 1995 c. 18.};

“Maintenance Assessment Procedure Regulations” means the Child Support (Maintenance Assessment Procedure) Regulations 1992\footnote{\frenchspacing S.I. 1992/1813.};

“married couple” means a man and a woman who are married to each other and are members of the same household;

“non-dependant” means a person who is a non-dependant for the purposes of either—
\begin{enumerate}\item[]
($a$)
regulation 3 of the Income Support Regulations; or

($b$)
regulation 3 of the Housing Benefit Regulations,
\end{enumerate}
or who would be a non-dependant for those purposes if another member of the household in which he is living were entitled to income support or housing benefit as the case may be;

“nursing home” has the same meaning as in regulation 19(3) of the Income Support Regulations;

“occupational pension scheme” has the same meaning as in section 66(1) of the Social Security Pensions Act 1975\footnote{\frenchspacing 1975 c. 60.};

“ordinary clothing or footwear” means clothing or footwear for normal daily use, but does not include school uniforms, or clothing or footwear used solely for sporting activities;

“parent with care” means a person who, in respect of the same child or children, is both a parent and a person with care;

“partner” means—
\begin{enumerate}\item[]
($a$)
in relation to a member of a married or unmarried couple who are living together, the other member of that couple;

($b$)
in relation to a member of a polygamous marriage, any other member of that marriage with whom he lives;
\end{enumerate}

“patient” means a person (other than a person who is serving a sentence of imprisonment or detention in a young offender institution within the meaning of the Criminal Justice Act 1982\footnote{\frenchspacing 1982 c. 48.} as amended by the Criminal Justice Act 1988\footnote{\frenchspacing 1988 c. 33.}) who is regarded as receiving free in-patient treatment within the meaning of the Social Security (Hospital In-Patients) Regulations 1975\footnote{\frenchspacing S.I. 1975/555; the relevant amending instruments are S.I. 1977/1693 and 1987/1683.};

“person” does not include a local authority;

“personal pension scheme” has the same meaning as in 
%section 84(1) of the Social Security Act 1986\footnote{\frenchspacing 1986 c. 50.} 
section 1 of the Pensions Schemes Act 1993\footnote{\frenchspacing 1993 c. 48.}  % Words substituted (13.1.97) by SI 1996/3196 reg 10(2)(c)
and, in the case of a self-employed earner, includes a scheme approved by the Inland Revenue under Chapter IV of Part XIV of the Income and Corporation Taxes Act 1988\footnote{\frenchspacing 1988 c. 1.};

“polygamous marriage” means any marriage during the subsistence of which a party to it is married to more than one person and in respect of which any ceremony of marriage took place under the law of a country which at the time of that ceremony permitted polygamy;

“prisoner” means a person who is detained in custody pending trial or sentence upon conviction or under a sentence imposed by a court other than a person whose detention is under the Mental Health Act 1983\footnote{\frenchspacing 1983 c. 20.} or the Mental Health (Scotland) Act 1984\footnote{\frenchspacing 1984 c. 36.};

% Definition of ``profit-related pay'' inserted (13.1.97) by SI 1996/3196 reg 10(2)(d)
“profit-related pay” means any payment by an employer calculated by reference to actual or anticipated profits;

% Definition of ``qualifying transfer'' inserted (18.4.95) by SI 1995/1045 reg 41(2)(ii)
“qualifying transfer” has the meaning assigned to it in Schedule 3A;

“relevant child” means a child of an absent parent or a parent with care who is a member of the same family as that parent;

“relevant Schedule” means Schedule 2 to the Income Support Regulations (income support applicable amounts);

%“relevant week” means—
%\begin{enumerate}\item[]
%($a$)
%in relation to an application for child support maintenance—
%\begin{enumerate}\item[]
%(i)
%in the case of the person making the application, the period of 7 days immediately preceding the date on which the appropriate maintenance assessment application form is submitted to the Secretary of State;
%
%(ii)
%in the case of a person to whom a maintenance assessment enquiry form is given or sent as a result of such application, the period of 7 days immediately preceding the date on which that form is to be treated as given or sent under regulation 1(6)($b$) of the Maintenance Assessment Procedure Regulations;
%\end{enumerate}
%
%($b$)
%in relation to a review of a maintenance assessment under section 16 or 17 of the Act, the period of 7 days immediately preceding the date on which a maintenance assessment review enquiry form given or sent to the person in question is to be treated as having been given or sent under regulation 1(6)($b$) of the Maintenance Assessment Procedure Regulations;
%\end{enumerate}

%Definition of ``relevant week'' substituted (5.4.93) by SI 1993/913 reg 19(2)($d$)
“relevant week” means---
\begin{enumerate}\item[]
($a$) in relation to an application for child support maintenance
or a review under section 18(1)($a$) or 19(1)($a$) of the Act% Words inserted (22.1.96) by SI 1995/3261 reg 40(2)($b$)
---
\begin{enumerate}\item[]
(i) in the case of the person making the application, the period of 7 days immediately preceding the date on which the appropriate maintenance assessment application form (being an effective application within the meaning of regulation 2(4) of the Maintenance Assessment Procedure Regulations) is submitted to the Secretary of State;

(ii) in the case of a person to whom a maintenance assessment enquiry form is given or sent as the result of such an application, the period of 7 days immediately preceding the date on which that form is given to him or, as the case may be, the date on which it is treated as having been sent to him under regulation 1(6)($b$) of the Maintenance Assessment Procedure Regulations;
\end{enumerate}

%($b$) in relation to a review of a maintenance assessment under section 16 or 17 of the Act, the period of 7 days immediately preceding the date on which a request is made for information or evidence under regulation 17(5) or, as the case may be, regulation 19(2) of the Maintenance Assessment Procedure Regulations;

%Para ($b$)--($f$) substituted for para ($b$) (22.1.96) by SI 1995/3261 reg 40(2)($c$)
($b$) in relation to a review of an assessment under section 16 of the Act, the period of 7 days immediately preceding the date on which a request for information or evidence under regulation 17(5) of the Maintenance Assessment Procedure Regulations is made;

($c$) in relation to a review under section 17 of the Act, the period of 7 days immediately preceding the date on which the application for review is received by the Secretary of State;

($d$) in relation to a review under section 18(1)($b$) or 19(1)($b$) of the Act, the period of 7 days immediately preceding the date on which application for the review under section 17 of the Act was received by the Secretary of State;

($e$) in relation to a review under section 18(2), (6A) or 19(1)($c$) of the Act, the relevant week which was applicable for the purposes of the making of the maintenance assessment which is being reviewed; or

($f$) in relation to a review under section 19(6) of the Act, the period of 7 days immediately preceding the date on which, in the circumstances referred to in that sub-section, the child support officer first suspected that it would be appropriate to make one or more fresh assessments;
\end{enumerate}

“residential care home” has the same meaning as in regulation 19(3) of the Income Support Regulations;

“retirement annuity contract” means an annuity contract for the time being approved by the Board of Inland Revenue as having for its main object the provision of a life annuity in old age or the provision of an annuity for a partner or dependant and in respect of which relief from income tax may be given on any premium;

“self-employed earner” has the same meaning as in section 2(1)($b$) of the Contributions and Benefits Act;

“student” means a person, other than a person in receipt of a training allowance, who is aged less than 19 and attending a full-time course of advanced education or who is aged 19 or over and attending a full-time course of study at an educational establishment; and for the purposes of this definition—
\begin{enumerate}\item[]
($a$)
a person who has started on such a course shall be treated as attending it throughout any period of term or vacation within it, until the last day of the course or such earlier date as he abandons it or is dismissed from it;

($b$)
a person on a sandwich course (within the meaning of paragraph 1(1) of Schedule 5 to the 
%Education (Mandatory Awards) Regulations 1988\footnote{\frenchspacing S.I. 1988/1360.}%
Education (Mandatory Awards) (No.\ 2) Regulations 1993\footnote{\frenchspacing S.I. 1993/2914.}%Words substituted (18.4.95) by SI 1995/1045 reg 41(2)(iii)
) shall be treated as attending a full-time course of advanced education or, as the case may be, of study;
\end{enumerate}

“student loan” means a loan which is made to a student pursuant to arrangements made under section 1 of the Education (Student Loans) Act 1990\footnote{\frenchspacing 1990 c. 6; section 1 is amended by the Further and Higher Education (Scotland) Act 1992 (c. 37), Schedule 9.};

%Definition of ``the Independent Living Fund'' omitted (5.4.93) by SI 1993/913 reg 19(2)($e$)
%“the Independent Living Fund” means the charitable trust of that name established out of funds provided by the Secretary of State for the purpose of providing financial assistance to those persons incapacitated by or otherwise suffering from very severe disablement who are in need of such assistance to enable them to live independently;

“training allowance” has the same meaning as in regulation 2 of the Income Support Regulations;

“unmarried couple” means a man and a woman who are not married to each other but are living together as husband and wife;

“weekly council tax” means the annual amount of the council tax in question payable in respect of the year in which the effective date falls, divided by 52;

“year” means a period of 52 weeks;

“youth training” means—
\begin{enumerate}\item[]
($a$)
arrangements made under section 2 of the Employment and Training Act 1973\footnote{\frenchspacing 1973 c. 50; section 2 is substituted by the Employment Act 1988 (c. 19), section 25(1).} or section 2 of the Enterprise and New Towns (Scotland) Act 1990\footnote{\frenchspacing 1990 c. 35.}; or

($b$)
arrangements made by the Secretary of State for persons enlisted in Her Majesty’s forces for any special term of service specified in regulations made under section 2 of the Armed Forces Act 1966\footnote{\frenchspacing 1966 c. 45.} (power of Defence Council to make regulations as to engagement of persons in regular forces);
\end{enumerate}

for purposes which include the training of persons who, at the beginning of their training, are under the age of 18.
\end{enumerate}

%Reg 1(2A) inserted (5.4.93) by SI 1993/913 reg 19(3)
(2A) Where any provision of these Regulations requires the income of a person to be estimated and that or any other provision of these Regulations requires that the amount of such estimated income is to be taken into account for any purpose after deducting from it a sum in respect of income tax or of primary Class 1 contributions under the Contributions and Benefits Act 
or, as the case may be, the Contributions and Benefits (Northern Ireland) Act  % Words inserted (13.1.97) by SI 1996/3196 reg 10(3)(a)
or of contributions paid by that person towards an occupational or personal pension scheme, then
subject to sub-paragraph ($e$)%Words inserted (18.4.95) by SI 1995/1045 reg 41(3)($a$)
---
\begin{enumerate}\item[]
($a$) the amount to be deducted in respect of income tax shall be calculated by applying to that income the rates of income tax applicable at the 
%effective date 
relevant week  % Words substituted (18.4.95) by SI 1995/1045 reg 41(3)($b$)
less only the personal relief to which that person is entitled under Chapter I of Part VII of the Income and Corporation Taxes Act 1988\footnote{\frenchspacing 1988 c. 50; the definition of “lower rate” was added by the Finance Act 1992 (c. 20), s.9(9).} (personal relief); but if the period in respect of which that income is to be estimated is less than a year, the amount of the personal relief deductible under this sub-paragraph shall be calculated on a pro rata basis;

($b$) the amount to be deducted in respect of Class 1 contributions under the Contributions and Benefits Act 
or, as the case may be, the Contributions and Benefits (Northern Ireland) Act  % Words inserted (13.1.97) by SI 1996/3196 reg 10(3)(a)
shall be calculated by applying to that income the appropriate primary percentage applicable in the relevant week; and

($c$) the amount to be deducted in respect of contributions paid by that person towards an occupational 
%or personal %Words omitted (18.4.95) by SI 1995/1045 reg 41(3)($c$)
pension scheme shall be one-half of the sums so 
%paid.
paid; and  % Words substituted (18.4.95) by SI 1995/1045 reg 41(3)($c$)

%Reg 1(2A)($d$), ($e$) inserted (18.4.95) by SI 1995/1045 reg 41(3)($d$)
($d$) the amount to be deducted in respect of contributions towards a personal pension scheme shall be one half of the contributions paid by that person or, where that scheme is intended partly to provide a capital sum to discharge a mortgage secured on that person’s home, 37.5 per centum of those contributions;

($e$) in relation to any bonus or commission which may be included in that person’s income—
\begin{enumerate}\item[]
(i) the amount to be deducted in respect of income tax shall be calculated by applying to the gross amount of that bonus or commission the rate or rates of income tax applicable in the relevant week;

(ii) the amount to be deducted in respect of primary Class 1 contributions under the Contributions and Benefits Act 
or, as the case may be, the Contributions and Benefits (Northern Ireland) Act  % Words inserted (13.1.97) by SI 1996/3196 reg 10(3)(a)
%or under the Social Security Contributions and Benefits (Northern Ireland) Act 1992  % Words omitted (13.1.97) by SI 1996/3196 reg 10(3)(b)
shall be calculated by applying to the gross amount of that bonus or commission the appropriate main primary percentage applicable in the relevant week
but no deduction shall be made in respect of the portion (if any) of the bonus or commission which, if added to estimated income, would cause such income to exceed the upper earnings limit for Class 1 contributions as provided for in section 5(1)($b$) of the Contributions and Benefits Act % Words inserted (22.1.96) by SI 1995/3261 reg 40(3)
or, as the case may be, the Contributions and Benefits (Northern Ireland) Act;  % Words inserted (13.1.97) by SI 1996/3196 reg 10(3)(a)
and

(iii) the amount to be deducted in respect of contributions paid by that person in respect of the gross amount of that bonus or commission towards an occupational pension scheme shall be one half of any sum so paid.
\end{enumerate}
\end{enumerate}

(3) In these Regulations, unless the context otherwise requires, a reference—
\begin{enumerate}\item[]
($a$) to a numbered Part is to the Part of these Regulations bearing that number;

($b$) to a numbered Schedule is to the Schedule to these Regulations bearing that number;

($c$) to a numbered regulation is to the regulation in these Regulations bearing that number;

($d$) in a regulation or Schedule to a numbered paragraph is to the paragraph in that regulation or Schedule bearing that number;

($e$) in a paragraph to a lettered or numbered sub-paragraph is to the sub-paragraph in that paragraph bearing that letter or number.
\end{enumerate}

(4) 
These Regulations are subject to the provisions of Parts VIII and IX of the Departure Direction and Consequential Amendments Regulations and  % Words inserted (2.12.96) by SI 1996/2907 reg 68(3)
the regulations in Part II and the provisions of the Schedules to these Regulations are subject to the regulations relating to special cases in Part III.

\amendment{
\begin{sloppypar}
Words substituted in definition of ``course of advanced education'' in reg. 1(2), definitions of ``couple'', ``Independent Living (1993) Fund'' and ``Independent Living (Extension) Fund'' inserted in reg. 1(2), def\-in\-i\-tion of ``relevant week'' substituted in reg. 1(2), definition of ``the Independent Living Fund'' omitted in reg. 1(2) and reg. 1(2A) inserted (5.4.93) by the Child Support (Miscellaneous Amendments) Regulations 1993 reg. 19.
\end{sloppypar}

Words inserted in reg. 1(2A), words substituted in reg. 1(2A)($a$), ($c$) and in the definition of ``student'' in reg. 1(2), words omitted in reg. 1(2A)($c$), definition of ``qualifying transfer'' inserted in reg. 1(2) and definition of ``day to day care'' substituted in reg. 1(2) and reg. 1(2A)($d$), ($e$) inserted (18.4.95) by the Child Support and Income Support (Amendment) Regulations 1995 reg. 41.

Words inserted in sub-para. ($a$) in the definition of ``relevant week'' in reg. 1(2), words inserted in reg. 1(2A)($e$)(ii), sub-paras. (ii)--(iv) substituted for sub-para. (ii) in the definition of ``day to day care'' in reg. 1(2) and sub-paras. ($b$)--($f$) substituted for sub-para. ($b$) in the definition of ``relevant week'' in reg. 1(2) (22.1.96) by the Child Support (Miscellaneous Amendments) (No. 2) Regulations 1995 reg. 40 (subject to transitional provisions in reg. 57).

Definition of ``family'' substituted in reg. 1(2) (5.8.96) by the Child Support (Miscellaneous Amendments) Regulations 1996 reg. 18(3).

Definition of ``the Jobseekers Act'' inserted in reg. 1(2) (7.10.96) by the Social Security and Child Support (Jobseeker's Allowance) (Consequential Amendments) Regulations 1996 reg. 6(2).

Definitions of ``earnings top-up'' and ``the Earnings Top-up Scheme'' inserted in reg. 1(2) (7.10.96) by the Child Support (Miscellaneous Amendments) Regulations 1996 reg. 18(2).

Definition of ``Child Benefit Rates Regulations'' inserted in reg. 1(2) (7.4.97) by the Child Benefit, Child Support and Social Security (Miscellaneous Amendments) Regulations 1996 reg. 7 (subject to transitional provisions in reg. 49).

Definition of ``Departure Direction and Consequential Amendments Regulations'' inserted in reg. 1(2) and words inserted in reg. 1(4) (2.12.96) by the Child Support Departure Direction and Consequential Amendments Regulations 1996 reg. 68(2), (3).

Words inserted in reg. 1(2A), words substituted in definition of ``personal pension scheme'' in reg. 1(2), words omitted in reg. 1(2A)(e)(ii) and definitions of ``Contributions and Benefits (Northern Ireland) Act'', ``family credit'' and ``profit-related pay'' inserted in reg. 1(2) (13.1.97) by the Child Support (Miscellaneous Amendments) (No. 2) Regulations 1996 reg. 10.
}

\section[Part II --- Calculation or estimation of child support maintenance]{\sloppy Part II\\*Calculation or estimation of child support maintenance}

\renewcommand\parthead{--- Part II}

\subsection[2. Calculation or estimation of amounts]{Calculation or estimation of amounts}

2.—(1) Where any amount falls to be taken into account for the purposes of these Regulations, it shall be calculated or estimated as a weekly amount and, except where the context otherwise requires, any reference to such an amount shall be construed accordingly.

(2) Subject to 
%regulation 13(2), 
regulations 11(6) and (7) and 13(2) and 
  %regulation 8(2C) 
  %regulation 8A(4)  % Words substituted (22.1.96) by SI 1995/3261 reg 41
  regulation 8A(5)  % Words substituted (22.1.96) by SI 1995/3265 reg 3
of the Maintenance Assessment Procedure Regulations,  % Words substituted (18.4.95) by SI 1995/1045 reg 42
where any calculation made under 
the Act or  % Words inserted (18.4.95) by SI 1995/1045 reg 42
these Regulations results in a fraction of a penny that fraction shall be treated as a penny if it is either one half or exceeds one half, otherwise it shall be disregarded.

(3) A child support officer shall calculate the amounts to be taken into account for the purposes of these Regulations by reference, as the case may be, to the dates, weeks, months or other periods specified herein provided that if he becomes aware of a material change of circumstances occurring after such date, week, month or other period but before the effective date, he shall take that change of circumstances into account.

\amendment{
Words inserted and substituted in reg. 2(2) (18.4.95) by the Child Support and Income Support (Amendment) Regulations 1995 reg. 42.

%Words substituted in reg. 2(2) (22.1.96) by the Child Support (Miscellaneous Amendments) (No. 2) Regulations 1995 reg. 41.

Words substituted in reg. 2(2) (22.1.96) by the Child Support (Miscellaneous Amendments) (No. 3) Regulations 1995 reg. 3.
}

\subsection[3. Calculation of AG]{Calculation of AG}

3.—(1) The amounts to be taken into account for the purposes of calculating AG in the formula set out in paragraph 1(2) of Schedule 1 to the Act are—
\begin{enumerate}\item[]
($a$) with respect to each qualifying child, an amount equal to the amount specified in column (2) of paragraph 2 of the relevant Schedule for a person of the same age (income support personal allowance for child or young person);

%($b$) with respect to a person with care of a qualifying child aged less than 16, an amount equal to the amount specified in column (2) of paragraph 1(1)($e$) of the relevant Schedule (income support personal allowance for a single claimant aged not less than 25);

($b$) with respect to a person with care of one or more qualifying children—
\begin{enumerate}\item[]
(i) where one or more of those children is aged less than 11, an amount equal to the amount specified in column (2) of paragraph 1(1)($e$) of the relevant Schedule (income support personal allowance for a single claimant aged not less than 25);

(ii) where none of those children are aged less than 11 but one or more of them is aged less than 14, an amount equal to 75 per centum of the amount specified in head (i) above; and

(iii) where none of those children are aged less than 14 but one or more of them is aged less than 16, an amount equal to 50 per centum of the amount specified in head (i) above;
\end{enumerate} % Reg 3(1)($b$) substituted (7.2.94) by SI 1994/227 reg 4(2)

%($c$) an amount equal to the amount specified in paragraph 3 of the relevant Schedule (income support family premium);

%Reg 3(1)(c) substituted (7.4.97) by SI 1996/1803 reg 8(a)
($c$) an amount equal to—
\begin{enumerate}\item[]
(i) the amount specified in paragraph 3($b$) of the relevant Schedule; or

(ii) where the person with care is a lone parent as defined in regulation 2(1) of the Income Support Regulations, the amount specified in paragraph 3($a$) of the relevant Schedule.
\end{enumerate}

%Reg 3(1)(d) omitted (7.4.97) by SI 1996/1803 reg 8(b)
%($d$) where the person with care of the qualifying child or children has no partner, an amount equal to the amount specified in paragraph 15(1) of the relevant Schedule (income support lone parent premium).
\end{enumerate}

(2) The amounts referred to in paragraph (1) shall be the amounts applicable at the effective date.

\amendment{
Reg. 3(1)($b$) substituted (7.2.94) by the Child Support (Miscellaneous Amendments and Transitional Provisions) Regulations 1994 reg. 4(2) (subject to transitional provisions in reg. 12).

Reg. 3(1)($c$) substituted and reg. 3(1)($d$) omitted (7.4.97) by the Child Benefit, Child Support and Social Security (Miscellaneous Amendments) Regulations 1996 reg. 8 (subject to transitional provisions in reg. 49).
}

\subsection[4. Basic rate of child benefit]{Basic rate of child benefit}

4.  For the purposes of paragraph 1(4) of Schedule 1 to the Act “basic rate” means the rate of child benefit which is specified in 
%regulation 2(1) of the Child Benefit and Social Security (Fixing and Adjustment of Rates) Regulations 1976\footnote{\frenchspacing S.I. 1976/1267; the relevant amending instruments are S.I. 1977/1328, 1991/502, 543, 1595.} (rates of child benefit) 
regulation 2(1)($a$)(i) or 2(1)($b$) of the Child Benefit Rates Regulations (weekly rate for only, elder or eldest child and for other children)  % Words substituted (7.4.97) by SI 1996/1803 reg 9
applicable to the child in question at the effective date.

\amendment{
Words substituted in reg. 4 (7.4.97) by the Child Benefit, Child Support and Social Security (Miscellaneous Amendments) Regulations 1996 reg. 9 (subject to transitional provisions in reg. 49).
}

\subsection[5. The general rule]{The general rule}

5.  For the purposes of paragraph 2(1) of Schedule 1 to the Act—
\begin{enumerate}\item[]
($a$) the value of C, otherwise than in a case where the other parent is the person with care, is nil; and

($b$) the value of P is 0.5.
\end{enumerate}

\subsection[6. The additional element]{The additional element}

6.—%(1) For the purposes of the formula in paragraph 4(1) of Schedule 1 to the Act, the value of R is 0.25.
(1) For the purposes of the formula in paragraph 4(1) of Schedule 1 to the Act, the value of R is—
\begin{enumerate}\item[]
($a$) where the maintenance assessment in question relates to one qualifying child, 0.15;

($b$) where the maintenance assessment in question relates to two qualifying children, 0.20; and

($c$) where the maintenance assessment in question relates to three or more qualifying children, 0.25.
\end{enumerate}% Reg 6(1) substituted (7.2.94) by SI 1994/227 reg 4(3)

(2) For the purposes of the alternative formula in paragraph 4(3) of Schedule 1 to the Act—
\begin{enumerate}\item[]
($a$) the value of Z is 
%3;
1.5;  % Figure substituted (18.4.95) by SI 1995/1015 reg 43

($b$) the amount for the purposes of paragraph ($b$) of the definition of Q is the same as the amount specified in 
%regulation 3(1)($c$) 
regulation 3(1)($c$)(i)  % Words substituted (7.4.97) by SI 1996/1803 reg 10
(income support family premium) in respect of each qualifying child.
\end{enumerate}

\amendment{
Reg. 6(1) substituted (7.2.94) by the Child Support (Miscellaneous Amendments and Transitional Provisions) Regulations 1994 reg. 4(3) (subject to transitional provisions in reg. 12).

Figure substituted in reg. 6(2)($a$) (18.4.95) by the Child Support and Income Support (Amendment) Regulations 1995 reg. 43.

Words substituted in reg. 6(2)($b$) (7.4.97) by the Child Benefit, Child Support and Social Security (Miscellaneous Amendments) Regulations 1996 reg. 10 (subject to transitional provisions in reg. 49).
}

\subsection[7. Net income: calculation or estimation of N]{Net income: calculation or estimation of N}

7.—(1) Subject to the following provisions of this regulation, for the purposes of the formula in paragraph 5(1) of Schedule 1 to the Act, the amount of N (net income of absent parent) shall be the aggregate of the following amounts—
\begin{enumerate}\item[]
($a$) the amount, determined in accordance with Part I of Schedule 1, of any earnings of the absent parent;

($b$) the amount, determined in accordance with Part II of Schedule 1, of any benefit payments under the Contributions and Benefits Act 
or the Jobseekers Act  % Words inserted (7.10.96) by SI 1996/1345 reg 6(6), (7)($a)
paid to or in respect of the absent parent;

($c$) the amount, determined in accordance with Part III of Schedule 1, of any other income of the absent parent;

($d$) the amount, determined in accordance with Part IV of Schedule 1, of any income of a relevant child which is treated as the income of the absent parent;

($e$) any amount, determined in accordance with Part V of Schedule 1, which is treated as the income of the absent parent.
\end{enumerate}

(2) Any amounts referred to in Schedule 2 shall be disregarded.

(3) Where an absent parent’s income consists—
\begin{enumerate}\item[]
($a$) only of a youth training allowance; or

($b$) in the case of a student, only of grant, an amount paid in respect of grant contribution or student loan or any combination thereof; or

($c$) only of prisoner’s pay,
\end{enumerate}
then for the purposes of determining N such income shall be disregarded.

(4) Where a parent and any other person are beneficially entitled to any income but the shares of their respective entitlements are not ascertainable the child support officer shall estimate their respective entitlements having regard to such information as is available but where sufficient information on which to base an estimate is not available the parent and that other person shall be treated as entitled to that income in equal shares.

(5) Where any income normally received at regular intervals has not been received it shall, if it is due to be paid and there are reasonable grounds for believing it will be received, be treated as if it had been received.

\amendment{
Words inserted in reg. 7(1)($b$) (7.10.96) by the Social Security and Child Support (Jobseeker's Allowance) (Consequential Amendments) Regulations 1996 reg. 6(6), (7)($a$).
}

\subsection[8. Net income: calculation or estimation of M]{Net income: calculation or estimation of M}

8.  For the purposes of paragraph 5(2) of Schedule 1 to the Act, the amount of M (net income of the parent with care) shall be calculated in the same way as N is calculated under regulation 7 but as if references to the absent parent were references to the parent with care.

\subsection[9. Exempt income: calculation or estimation of E]{Exempt income: calculation or estimation of E}

9.—(1) For the purposes of paragraph 5(1) of Schedule 1 to the Act, the amount of E (exempt income of absent parent) shall, subject to paragraphs (3) and (4), be the aggregate of the following amounts—
\begin{enumerate}\item[]
($a$) an amount equal to the amount specified in column (2) of paragraph 1(1)($e$) of the relevant Schedule (income support personal allowance for a single claimant aged not less than 25);

($b$) an amount in respect of housing costs determined in accordance with regulations 14 to 
%18;
16 and 18;  % Words substituted in reg 9(1)(b) (7.10.96) by SI 1996/1945 reg 19

% Reg 9(1)($bb$) inserted (18.4.95) by SI 1995/1045 reg 44(2)($a$)
($bb$) where applicable, an amount in respect of a qualifying transfer of property determined in accordance with Schedule 3A;

($c$) where—
\begin{enumerate}\item[]
(i) the absent parent is the parent of a relevant child; and

%(ii) if he were a claimant, the condition in paragraph 8 of the relevant Schedule (income support lone parent premium) would be satisfied but the conditions referred to in sub-paragraph (1)($d$) would not be satisfied;

% Reg 9(1)(c)(ii) substituted (7.4.97) by SI 1996/1803 reg 11(2)(a)(i)
(ii) if he were a claimant, the conditions in paragraph 3($a$) of the relevant Schedule would be satisfied;
\end{enumerate}
an amount equal to the amount specified in 
%column (2) of paragraph 15(1) of that Schedule (income support lone parent premium)
that sub-paragraph;  % Words substituted (7.4.97) by SI 1996/1803 reg 11(2)(a)(ii)

($d$) where, if the parent were a claimant aged less than 60, the conditions in paragraph 11 of the relevant Schedule (income support disability premium) would be satisfied in respect of him, an amount equal to the amount specified in column (2) of paragraph 15(4)($a$) of that Schedule (income support disability premium);

($e$) where—
\begin{enumerate}\item[]
(i) if the parent were a claimant, the conditions in paragraph 13 of the relevant Schedule (income support severe disability premium) would be satisfied, an amount equal to the amount specified in column (2) of paragraph 15(5)($a$) of that Schedule (except that no such amount shall be taken into account in the case of an absent parent in respect of whom an invalid care allowance under section 70 of the Contributions and Benefits Act is payable to some other person);

(ii) if the parent were a claimant, the conditions in paragraph 14ZA of the relevant Schedule (income support carer premium) would be satisfied in respect of him, an amount equal to the amount specified in column (2) of paragraph 15(7) of that Schedule;
\end{enumerate}

($f$) where, if the parent were a claimant, the conditions in paragraph 3 of the relevant Schedule (income support family premium) would be satisfied in respect of a relevant child of that parent, 
but he is not a lone parent as defined in regulation 2(1) of the Income Support Regulations,  % Words inserted (7.4.97) by SI 1996/1803 reg 11(2)(b)(i)
the amount specified in 
sub-paragraph ($b$) of  % Words inserted (7.4.97) by SI 1996/1803 reg 11(2)(b)(ii)
that paragraph or, where those conditions would be satisfied only by virtue of the case being one to which paragraph (2) applies, half that amount;

($g$) in respect of each relevant child—
\begin{enumerate}\item[]
(i) an amount equal to the amount of the personal allowance for that child, specified in column (2) of paragraph 2 of the relevant Schedule (income support personal allowance) or, where paragraph (2) applies, half that amount;

(ii) if the conditions set out in paragraph 14($b$) and ($c$) of the relevant Schedule (income support disabled child premium) are satisfied in respect of that child, an amount equal to the amount specified in column (2) of paragraph 15(6) of the relevant Schedule or, where paragraph (2) applies, half that amount;
\end{enumerate}

($h$) where the absent parent in question or his partner is living in—
\begin{enumerate}\item[]
(i) accommodation provided under Part III of the National Assistance Act 1948\footnote{\frenchspacing 11 \& 12 Geo. 6 c. 29.};

(ii) accommodation provided under paragraphs 1 and 2 of Schedule 8 to the National Health Service Act 1977\footnote{\frenchspacing 1977 c. 49.}; or

(iii) a nursing home or residential care home,
\end{enumerate}
the amount of the fees paid in respect of the occupation of that accommodation or, as the case may be, that home
but where a local authority has determined that the absent parent in question or his partner is entitled to housing benefit in respect of fees for that accommodation or that home, the net amount of such fees after deduction of housing benefit;  % Words inserted (22.1.96) by SI 1995/3261 reg 42

% Reg 9(1)(i) inserted (18.4.95) by SI 1995/1045 reg 44(2)($b$)
($i$) where applicable, an amount in respect of travelling costs determined in accordance with Schedule 3B.
\end{enumerate}

(2) This paragraph applies where—
\begin{enumerate}\item[]
($a$) the absent parent has a partner;

($b$) the absent parent and the partner are parents of the same relevant child; and

($c$) the income of the partner, calculated under regulation 7(1) 
(but excluding the amount mentioned in sub-paragraph ($d$) of that regulation) % Words inserted (5.4.93) by SI 1993/913 reg 20.
as if that partner were an absent parent to whom that regulation applied, exceeds the aggregate of—
\begin{enumerate}\item[]
(i) the amount specified in column 2 of paragraph 1(1)($e$) of the relevant Schedule (income support personal allowance for a single claimant aged not less than 25);

(ii) half the amount of the personal allowance for that child specified in column (2) of paragraph 2 of the relevant Schedule (income support personal allowance);

(iii) half the amount of any income support disabled child premium specified in column (2) of paragraph 15(6) of that Schedule in respect of that child;
and % Word inserted (18.4.95) by SI 1995/1045 reg 44(3)(i)

(iv) half the amount of any income support family premium specified in paragraph 
%3 of the Schedule 
3($b$) of the relevant Schedule  % Words substituted (7.4.97) by SI 1996/1803 reg 11(3)
except where such premium is payable irrespective of that child;%; and
%Reg 9(2)($c$)(v) and preceding ``and'' omitted (18.4.95) by SI 1995/1045 reg 44(3)(ii)
%(v) the amount by which the housing costs of the absent parent, calculated in accordance with these Regulations, have been reduced by an apportionment under regulation 17.

% Reg 9(2)(c)(v) inserted (2.12.96) by SI 1996/2907 reg 68(4)
(v) where a departure direction has been given on the grounds that a case falls
within regulation 27 of the Departure Direction and Consequential Amendments
Regulations (partner’s contribution to housing costs), the amount of the housing
costs which corresponds to the percentage of the housing costs mentioned in
regulation 40(7) of those Regulations.
\end{enumerate}
\end{enumerate}

(3) Where an absent parent does not have day to day care of any relevant child for 7 nights each week but does have day to day care of one or more such children for fewer than 7 nights each week, 
%any amounts 
any amount  % Words substituted (7.4.97) by SI 1996/1803 reg 11(4)(a)
to be taken into account under sub-paragraphs (1)($c$) 
%and ($f$) 
or ($f$)  % Words substituted (7.4.97) by SI 1996/1803 reg 11(4)(b)
shall be reduced so that they bear the same proportion to the amounts referred to in those sub-paragraphs as the average number of nights each week in respect of which such care is provided has to 7.

(4) Where an absent parent has day to day care of a relevant child for fewer than 7 nights each week, any amounts to be taken into account under sub-paragraph (1)($g$) in respect of such a child shall be reduced so that they bear the same proportion to the amounts referred to in that sub-paragraph as the average number of nights each week in respect of which such care is provided has to 7.

(5) The amounts referred to in paragraph (1) are the amounts applicable at the effective date.

\amendment{
Words inserted in reg. 9(2)($c$) (5.4.93) by the Child Support (Miscellaneous Amendments) Regulations 1993 reg. 20.

Word inserted in reg. 9(2)($c$)(iii), reg. 9(1)($bb$), (i) inserted and reg. 9(2)($c$)(v) and preceding word ``and'' omitted (18.4.95) by the Child Support and Income Support (Amendment) Regulations 1995 reg. 44.

Words inserted in reg. 9(1)($h$) (22.1.96) by the Child Support (Miscellaneous Amendments) (No. 2) Regulations 1995 reg. 42 (subject to transitional provisions in reg. 57).

Words inserted in reg. 9(1)($c$), words substituted in reg. 9(1)($c$), (2)($c$)(iv), (3) and reg. 9(1)($c$)(ii) substituted (7.4.97) by the Child Benefit, Child Support and Social Security (Miscellaneous Amendments) Regulations 1996 reg. 11 (subject to transitional provisions in reg. 49).

Words substituted in reg. 9(1)($b$) (7.10.96) by the Child Support (Miscellaneous Amendments) Regulations 1996 reg. 19.

Reg. 9(2)(c)(v) inserted (2.12.96) by the Child Support Departure Direction and Consequential Amendments Regulations 1996 reg. 68(4).
}

\subsection[10. Exempt income: calculation or estimation of F]{Exempt income: calculation or estimation of F}

10.  For the purposes of paragraph 5(2) of Schedule 1 to the Act, the amount of F (exempt income of parent with care) shall be calculated in the same way as E is calculated under regulation 9 but as if references to the absent parent were references to the parent with care
%except that paragraphs (3) and (4) of that regulation shall apply only in a case where the parent with care shares day to day care of the child mentioned in those paragraphs with one or more other persons. %Words inserted (5.4.93) by SI 1993/913 reg 21.
expect that—
\begin{enumerate}\item[]
($a$) sub-paragraph ($bb$) of paragraph (1) of that regulation shall not apply unless at the time of the making of the qualifying transfer the parent with care would have been the absent parent had the Child Support Act 1991 been in force at the date of the making of the transfer; and

($b$) paragraphs (3) and (4) of that regulation shall apply only where the parent with care shares day to day care of the child mentioned in those paragraphs with one or more other persons.
\end{enumerate}  % Words substituted (18.4.95) by SI 1995/1045 reg 45

\amendment{
Words inserted in reg. 10 (5.4.93) by the Child Maintenance (Miscellaneous Amendments) Regulations 1993 reg. 21.

Words substituted in reg. 10 (18.4.95) by the Child Support and Income Support (Amendment) Regulations 1995 reg. 45.
}

% Reg 10A inserted (13.1.97) by SI 1996/3196 reg 11
\subsection[10A. Assessable income: family credit or disability working allowance paid to or in respect of a parent with care or an absent parent]{\sloppy Assessable income: family credit or disability working allowance paid to or in respect of a parent with care or an absent parent}

10A.—(1) Subject to paragraph (2), where family credit or disability working allowance is paid to or in respect of a parent with care or an absent parent, that parent shall, for the purposes of Schedule 1 to the Act, be taken to have no assessable income.

(2) Paragraph (1) shall apply to an absent parent only if—
\begin{enumerate}\item[]
($a$) he is also a parent with care; and

($b$) either—
\begin{enumerate}\item[]
(i) a maintenance assessment in respect of a child in relation to whom he is a parent with care is in force; or

(ii) the child support officer is considering an application for such an assessment to be made.
\end{enumerate}
\end{enumerate}


\amendment{
Reg. 10A inserted (13.1.97) by the Child Support (Miscellaneous Amendments) (No. 2) Regulations 1996 reg. 11.

Under the Child Support (Miscellaneous Amendments) (No. 2) Regulations 1996 reg. 16, a maintenance assessment in force on 13.1.97 shall not be reviewed solely to give effect to reg. 10A, but reg. 10A shall apply in conducting a review of the assessment under s. 16, 17, 18 or 19 of the Act, and the effective date of any fresh assessment affected by reg. 10A shall not be earlier than the first day of the first maintenance period which commences on or after 13.1.97.
}

\subsection[11. Protected income]{Protected income}

11.—(1) For the purposes of paragraph 6 of Schedule 1 to the Act the protected income level of an absent parent shall, 
%subject to paragraphs (3) and (4), 
subject to paragraphs (3), (4)% 
%and (6), % Words substituted (18.4.95) by SI 1995/1045 reg 46(2)($a$)
  , (6) and (6A)  % Words substituted (5.8.96) by SI 1996/1945 reg 20(2)
be the aggregate of the following amounts—
\begin{enumerate}\item[]
($a$) where—
\begin{enumerate}\item[]
(i) the absent parent does not have a partner, an amount equal to the amount specified in column (2) of paragraph 1(1)($e$) of the relevant Schedule (income support personal allowance for a single claimant aged not less than 25 years);

(ii) the absent parent has a partner, an amount equal to the amount specified in column (2) of paragraph 1(3)($c$) of the relevant Schedule (income support personal allowance for a couple where both members are aged not less than 18 years);

(iii) the absent parent is a member of a polygamous marriage, an amount in respect of himself and one of his partners, equal to the amount specified in sub-paragraph (ii) and, in respect of each of his other partners, an amount equal to the difference between the amounts specified in sub-paragraph (ii) and sub-paragraph (i);
\end{enumerate}

($b$) an amount in respect of housing costs determined in accordance with regulations 14, 15, 16 and 18, or, in a case where the absent parent is a non-dependant member of a household who is treated as having no housing costs by 
%regulation 15(10)($a$), 
regulation 15(4),  % Words substituted (22.1.96) by SI 1995/3261 reg 43(2)
the non-dependant amount which would be calculated in respect of him under 
%regulation 15(5);
paragraphs (1), (2) and (9) of regulation 63 of the Housing Benefit Regulations (non-dependant deductions) if he were a non-dependant in respect of whom a calculation were to be made under those paragraphs (disregarding any other provision of that regulation);  % Words substituted (18.4.95) by SI 1995/1045 reg 46(2)($b$)

%($c$) where, if the absent parent were a claimant, the condition in paragraph 8 of the relevant Schedule (income support lone parent premium) would be satisfied but the condition set out in paragraph 11 of that Schedule (income support disability premium) would not be satisfied, an amount equal to the amount specified in column (2) of paragraph 15(1) of that Schedule (income support lone parent premium);

% Reg 11(1)(c) substituted (7.4.97) by SI 1996/1803 reg 12(2)(a)
($c$) where, if the absent parent were a claimant, the conditions in paragraph 3($a$) of the relevant Schedule (income support family premium) would be satisfied, an amount equal to the amount specified in that sub-paragraph;

($d$) where, if the parent were a claimant, the conditions in paragraph 11 of the relevant Schedule (income support disability premium) would be satisfied, an amount equal to the amount specified in column (2) of paragraph 15(4) of that Schedule (income support disability premium);

($e$) where, if the parent were a claimant, the conditions in paragraph 13 or 14ZA of the relevant Schedule (income support severe disability and carer premiums) would be satisfied in respect of either or both premiums, an amount equal to the amount or amounts specified in column (2) of paragraph 15(5) or, as the case may be, (7) of that Schedule in respect of that or those premiums (income support premiums);

($f$) where, if the parent were a claimant, the conditions in paragraph 3 of the relevant Schedule (income support family premium) would be satisfied 
but he is not a lone parent as defined in regulation 2(1) of the Income Support Regulations,  % Words inserted (7.4.97) by SI 1996/1803 reg 12(2)(b)(i)
the amount specified in 
sub-paragraph ($b$) of  % Words inserted (7.4.97) by SI 1996/1803 reg 12(2)(b)(ii)
that paragraph;

($g$) in respect of each child who is a member of the family of the absent parent—
\begin{enumerate}\item[]
(i) an amount equal to the amount of the personal allowance for that child, specified in column (2) of paragraph 2 of the relevant Schedule (income support personal allowance);

(ii) if the conditions set out in paragraphs 14($b$) and ($c$) of the relevant Schedule (income support disabled child premium) are satisfied in respect of that child, an amount equal to the amount specified in column (2) of paragraph 15(6) of the relevant Schedule;
\end{enumerate}

($h$) where, if the parent were a claimant, the conditions specified in Part III of the relevant Schedule would be satisfied by the absent parent in question or any member of his family in relation to any premium not otherwise included in this regulation, an amount equal to the amount specified in Part IV of that Schedule (income support premiums) in respect of that premium;

($i$) where the absent parent in question or his partner is living in—
\begin{enumerate}\item[]
(i) accommodation provided under Part III of the National Assistance Act 1948\footnote{\frenchspacing 11 \& 12 Geo. 6 c. 29.};

(ii) accommodation provided under paragraphs 1 and 2 of Schedule 8 to the National Health Service Act 1977\footnote{\frenchspacing 1977 c. 49.}; or

(iii) a nursing home or residential care home,
\end{enumerate}
the amount of the fees paid in respect of the occupation of that accommodation or, as the case may be, that home
but where housing benefit is paid to the absent parent in question or his partner in respect of fees for that accommodation or that home the net amount of such fees after deduction of housing benefit;  % Words inserted (22.1.96) by SI 1995/3261 reg 43(3)

%($j$) the amount of council tax which the absent parent in question or his partner is liable to pay in respect of the home for which housing costs are included under sub-paragraph ($b$) less any council tax benefit;

% Reg 11(1)($j$) substituted (18.4.95) by SI 1995/1045 reg 46(2)($c$)
($j$) where—
\begin{enumerate}\item[]
(i) the absent parent is, or that absent parent and any partner of his are, the only person or persons resident in, and liable to pay council tax in respect of, the home of which housing costs are included under sub-paragraph ($b$), the amount of weekly council tax for which he is liable in respect of that home, less any applicable council tax benefit;

(ii) where other persons are resident with the absent parent in, and liable to pay council tax in respect of, the home for which housing costs are included under sub-paragraph ($b$), an amount representing the share of the weekly council tax in respect of that home applicable to the absent parent, determined by dividing the total amount of council tax due in that week by the number of persons liable to pay it, less any council tax benefit applicable to that share, provided that, if the absent parent is required to pay and pays more than that share because of default by one or more of those other persons, the amount of the purposes of this regulation shall be the amount of weekly council tax the absent parent pays, less any council tax benefit applicable to such amount;
\end{enumerate}

($k$) an amount of 
%£8·00;
£30.00; % Words substituted (7.2.94) by SI 1994/227 reg 4(4)

% Reg 11(1)($kk$) inserted (18.4.95) by SI 1995/1045 reg 46(2)($d$)
($kk$) an amount in respect of travelling costs determined in accordance with Schedule 3B;

($l$) where the income of—
\begin{enumerate}\item[]
(i) the absent parent in question;

(ii) any partner of his; and

(iii) any child or children for whom an amount is included under sub-paragraph ($g$)(i),
\end{enumerate}
exceeds the sum of the amounts to which reference is made in sub-paragraphs 
%($a$) to ($k$), 
($a$) to ($kk$),  % Words substituted (18.4.95) by SI 1995/1045 reg 46(2)($e$)
%10 per centum 
15 per centum % Words substituted (7.2.94) by SI 1994/227 reg 4(5)
of the excess.
\end{enumerate}

(2) For the purposes of sub-paragraph ($l$) of paragraph (1) “income” shall be calculated—
\begin{enumerate}\item[]
($a$) in respect of the absent parent in question or any partner of his, in the same manner as N (net income of absent parent) is calculated under regulation 7 except—
\begin{enumerate}\item[]
(i) there shall be taken into account the basic rate of any child benefit and any maintenance which in either case is in payment in respect of any member of the family of the absent parent;

(ii) there shall be deducted the amount of any maintenance under a maintenance order which the absent parent or his partner is paying in respect of a child in circumstances where an application for a maintenance assessment could not be made in accordance with the Act in respect of that child; 
%and % Word omitted (18.4.95) by SI 1995/1045 reg 46(3)

% Reg 11(2)($a$)(iii) inserted (18.4.95) by SI 1995/1045 reg 46(4)
(iii) to the extent that it falls under sub-paragraph ($b$), the income of any child in that family shall not be treated as the income of the parent or his partner and Part IV of Schedule 1 shall not apply; 
%and  % Word omitted (22.1.96) by SI 1995/3261 reg 43(4)

% Reg 11(2)($a$)(iv), (v) inserted (22.1.96) by SI 1995/3261 reg 46(5)
(iv) paragraph 27 of Schedule 2 shall apply as though the reference to paragraph 3(2) and (4) of Schedule 3 were omitted;

(v) there shall be deducted the amount of any maintenance which is being paid in respect of a child by the absent parent or his partner under an order requiring such payment made by a court outside Great Britain; and
\end{enumerate}

($b$) in respect of any child in that family, as being the total of 
%that child’s income 
that child’s relevant income (within the meaning of paragraph 23 of Schedule 1), there being disregarded any maintenance in payment to or in respect of him,  % Words substituted (18.4.95) by SI 1995/1045 reg 46(5)
but only to the extent that such income does not exceed the amount included under sub-paragraph ($g$) of paragraph (1) (income support personal allowance for a child and income support disabled child premium) reduced, as the case may be, under paragraph (4).
\end{enumerate}

(3) Where an absent parent does not have day to day care of any child (whether or not a relevant child) for 7 nights each week but does have day to day care of one or more such children for fewer than 7 nights each week, 
%any amounts 
any amount  % Words substituted (7.4.97) by SI 1996/1803 reg 12(3)(a)
to be taken into account under sub-paragraphs ($c$) 
%and ($f$) 
or ($f$)  % Words substituted (7.4.97) by SI 1996/1803 reg 12(3)(b)
of paragraph (1) (%
%income support lone parent premium and % Words omitted (7.4.97) by SI 1996/1803 reg 12(3)(c)
income support family premium) shall be reduced so that they bear the same proportion to the amounts referred to in those sub-paragraphs as the average number of nights each week in respect of which such care is provided has to 7.

(4) Where an absent parent has day to day care of a child (whether or not a relevant child) for fewer than 7 nights each week any amounts in relation to that child to be taken into account under sub-paragraph ($g$) of paragraph (1) (income support personal allowance for child and income support disabled child premium) shall be reduced so that they bear the same proportion to the amounts referred to in that sub-paragraph as the average number of nights in respect of which such care is provided has to 7.

(5) The amounts referred to in paragraph (1) shall be the amounts applicable at the effective date.

% Reg 11(6), (7) inserted (18.4.95) by SI 1995/1045 reg 46(6)
(6) If the application of the above provisions of this regulation would result in the protected income level of an absent parent being less than 70 per centum of his net income, as calculated in accordance with regulation 7, those provisions shall not apply in his case and instead his protected income level shall be 70 per centum of his net income as so calculated.

% Reg 11(6A) inserted (5.8.96) by SI 1996/1945 reg 20(3)
(6A) In a case to which paragraph (6) does not apply, if the application of paragraphs (1) to (5) and of regulation 12(1)($a$) would result in the amount of child support maintenance payable being greater than 30 per centum of the absent parent’s net income calculated in accordance with regulation 7, paragraphs (1) to (5) shall not apply in his case and instead his protected income level shall be 70 per centum of his net income as so calculated.

(7) Where any calculation under paragraph (6) 
or (6A)  % Words inserted (5.8.96) by SI 1996/1945 reg 20(4)
results in a fraction of a penny, that fraction shall be treated as a penny.

\amendment{
Words substituted in reg. 11(1)($k$), ($l$) (7.2.94) by the Child Support (Miscellaneous Amendments and Transitional Provisions) Regulations 1994 reg. 4(4), (5) (subject to transitional provisions in reg. 12).

Words substituted in reg. 11(1), (1)($b$), (1)($l$), (2)($b$), word omitted in reg. 11(2)($a$)(ii), reg. 11(1)($kk$), (2)($a$)(iii), (6), (7) inserted and reg. 11(1)($j$) substituted (18.4.95) by the Child Support and Income Support (Amendment) Regulations 1995 reg. 46.

Words substituted in reg. 11(1)($b$), (i) and reg. 11(2)($a$)(iv), (v) inserted (22.1.96) by the Child Support (Miscellaneous Amendments) (No. 2) Regulations 1995 reg. 43 (subject to transitional provisions in reg. 57).

Words inserted in reg. 11(7), words substituted in reg. 11(1) and reg. 11(6A) inserted (5.8.96) by the Child Support (Miscellaneous Amendments) Regulations 1996 reg. 20.

Words inserted in reg. 11(1)($f$), words substituted and omitted in reg. 11(3) and reg. 11(1)($c$) substituted (7.4.97) by the Child Benefit, Child Support and Social Security (Miscellaneous Amendments) Regulations 1996 reg. 12 (subject to transitional provisions in reg. 49).
}

\subsection[12. Disposable income]{Disposable income}

12.—%(1) For the purposes of paragraph 6(4) of Schedule 1 to the Act (protected income), the disposable income of an absent parent shall be the aggregate of his income and any income of any member of his family calculated in like manner as under regulation 11(2).
%Reg 12(1) substituted (18.4.95) by SI 1995/1045 reg 47
(1) For the purposes of paragraph 6(4) of Schedule 1 to the Act (protected income), the disposable income of an absent parent shall be—
\begin{enumerate}\item[]
($a$) except in a case to which regulation 11(6) 
or (6A)  % Words inserted (5.8.96) by SI 1996/1945 reg 21
applies, the aggregate of his income and any income of any member of his family calculated in like manner as under regulation 11(2); 
%and  % Word omitted (13.1.97) by SI 1996/3196 reg 12(2)

($b$) 
subject to sub-paragraph ($c$),  % Words inserted (13.1.97) by SI 1996/3196 reg 12(3)(a)
in a case to which regulation 11(6) 
or (6A)  % Words inserted (5.8.96) by SI 1996/1945 reg 21
applies, his net income as calculated in accordance with regulation 7%
%.
; and  % Word substituted (13.1.97) by SI 1996/3196 reg 12(3)(b)

% Reg 12(1)(c) inserted (13.1.97) by SI 1996/3196 reg 12(4)
($c$) in a case to which regulation 11(6) applies and the absent parent is paying maintenance under an order of a kind mentioned in regulation 11(2)($a$)(ii) or (v), his net income as calculated in accordance with regulation 7 less the amount of maintenance he is paying under that order.
\end{enumerate}

(2) Subject to paragraph (3), where a maintenance assessment has been made with respect to the absent parent and payment of the amount of that assessment would reduce his disposable income below his protected income level the amount of the assessment shall be reduced by the minimum amount necessary to prevent his disposable income being reduced below his protected income level.

(3) Where the prescribed minimum amount fixed by regulations under paragraph 7 of Schedule 1 to the Act is applicable (such amount being specified in regulation 13) the amount payable under the assessment shall not be reduced to less than the prescribed minimum amount.

\amendment{
Reg. 12(1) substituted (18.4.95) by the Child Support and Income Support (Amendment) Regulations 1995 reg. 47.

Words inserted in reg. 12(1)(a), (b) (5.8.96) by the Child Support (Miscellaneous Amendments) Regulations 1996 reg. 21.

Words inserted and substituted in reg. 12(1)(b), word omitted in reg. 12(1)(a) and reg. 12(1)(c) inserted (13.1.97) by the Child Support (Miscellaneous Amendments) (No. 2) Regulations 1996 reg. 12.

Under the Child Support (Miscellaneous Amendments) (No. 2) Regulations 1996 reg. 16, a maintenance assessment in force on 13.1.97 shall not be reviewed solely to give effect to reg. 12(1)(c), but reg. 12(1)(c) shall apply in conducting a review of the assessment under s. 16, 17, 18 or 19 of the Act, and the effective date of any fresh assessment affected by reg. 12(1)(c) shall not be earlier than the first day of the first maintenance period which commences on or after 13.1.97.
}

\subsection[13. The minimum amount]{The minimum amount}

13.—(1) Subject to regulation 26, for the purposes of paragraph 7(1) of Schedule 1 to the Act the minimum amount shall be 
2 multiplied by  % Words inserted (8.4.96) by SI 1996/481 reg 2(2)
5 per centum of the amount specified in paragraph 1(1)($e$) of the relevant Schedule (income support personal allowance for single claimant aged not less than 25).

(2) Where 
%an amount 
the 5 per centum amount  % Words substituted (8.4.96) by SI 1996/481 reg 2(3)
calculated under paragraph (1) results in a sum other than a multiple of 5 pence, it shall be treated as the sum which is the next higher multiple of 5 pence.

\amendment{
Words inserted in reg. 13(1) and substituted in reg. 13(2) (8.4.96) by the Child Support (Maintenance Assessments and Special Cases) and Social Security (Claims and Payments) Amendment Regulations 1996 reg. 2.
}

\subsection[14. Eligible housing costs]{Eligible housing costs}

14.  Schedule 3 shall have effect for the purpose of determining the costs which are eligible to be taken into account as housing costs for the purposes of these Regulations.

\subsection[15. Amount of housing costs]{Amount of housing costs}

15.—(1) Subject to the provisions of this regulation and 
%regulations 16 to 18, 
regulations 16 and 18,  % Words substituted (18.4.95) by SI 1995/1045 reg 48(2)
a parent’s housing costs shall be the aggregate of the eligible housing costs payable in respect of his home.

(2) Where a local authority has determined that a parent is entitled to housing benefit, the amount of his housing costs shall, subject to paragraphs (4) to (9), be the weekly amount treated as rent under regulations 10 and 69 of the Housing Benefit Regulations (rent and calculation of weekly amounts) less the amount of housing benefit.

(3) Where a parent has eligible housing costs and another person who is not a member of his family is also liable to make payments in respect of the home, the amount of the parent’s housing costs shall be his share of those costs.

%(4) Where one or more non-dependants are members of the parent’s household, there shall be deducted from the amount of any housing costs determined under the preceding paragraphs of this regulation any non-dependant amount or amounts determined in accordance with the provisions of paragraphs (5) to (9).
%
%(5) The non-dependant amount shall be an amount equal to the amount which would be calculated under 
%%paragraph 63 
%paragraphs (1), (2) and (9) of regulation 63 % Words substituted (5.4.93) by SI 1993/913 reg 22.
%of the Housing Benefit Regulations (non-dependant deductions) for the non-dependant in question if he were a non-dependant in respect of whom a calculation were to be made under 
%%that regulation.
%those paragraphs (disregarding any other provision of that regulation). % Words substituted (5.4.93) by SI 1993/913 reg 22.
%
%(6) For the purposes of paragraph (5)—
%\begin{enumerate}\item[]
%($a$) in the case of a couple or, as the case may be, the members of a polygamous marriage—
%\begin{enumerate}\item[]
%(i) regard shall be had to their joint weekly income; and
%
%(ii) only one deduction shall be made at whichever is the higher rate.
%\end{enumerate}
%\end{enumerate}
%
%(7) Where a person is a non-dependant in respect of more than one joint occupier of a dwelling (except where the joint occupiers are a couple or members of a polygamous marriage), the deduction in respect of that non-dependant shall be apportioned between the joint occupiers having regard to the number of joint occupiers and the proportion of the housing costs in respect of the home payable by each of them.
%
%(8) No deduction shall be made in respect of any non-dependants occupying the home of the parent, if the parent or any partner of his is—
%\begin{enumerate}\item[]
%($a$) blind or treated as blind by virtue of paragraph 12 of the relevant Schedule (income support additional condition for the higher pensioner and disability premiums); or
%
%($b$) receiving in respect of himself either—
%\begin{enumerate}\item[]
%(i) attendance allowance under section 64 of the Contributions and Benefits Act; or
%
%(ii) the care component of disability living allowance.
%\end{enumerate}
%\end{enumerate}
%
%(9) No deduction shall be made in respect of a non-dependant—
%\begin{enumerate}\item[]
%($a$) if, although he resides with the parent, it appears to the child support officer that his home is normally elsewhere; or
%
%($b$) if he is in receipt of a training allowance paid in connection with a Youth Training Programme established under section 2 of the Employment and Training Act 1973\footnote{\frenchspacing 1977 c. 49.} or section 2 of the Enterprise and New Towns (Scotland) Act 1990\footnote{\frenchspacing 1973 c. 50, as amended by sections 9 and 11 and Schedule 2, Part II, paragraph 9 and Schedule 3 of the Employment and Training Act 1981 (c. 57).}; or
%
%($c$) if he is a student; or
%
%($d$) if he is aged under 25 and in receipt of income support; or
%
%($e$) if he is not residing with the parent because he is a prisoner or because he has been a patient for a period, or two or more periods separated by not more than 28 days, exceeding 6 weeks.
%\end{enumerate}

% Reg 15(4)--(9) omitted (18.4.95) by SI 1995/1045 reg 48(3)

%(10) A parent shall be treated as having no housing costs where—
%\begin{enumerate}\item[]
%($a$) he is a non-dependant member of a household and is not responsible for meeting housing costs except to another member, or other members, of that household; or
%
%($b$) but for this paragraph, his housing costs would be less than nil.
%\end{enumerate}

% Reg 15(10) substituted (18.4.95) by SI 1995/1045 reg 48(4), renumbered as reg 15(4) by SI 1995/3261 reg 44
%(10) 
(4) 
A parent shall be treated as having no housing costs where he is a non-dependant member of a household and is not responsible for meeting housing costs except to another member, or other members, of that household.

\amendment{
Words substituted in reg. 15(5) (5.4.93) by the Child Support (Miscellaneous Amendments) Regulations 1993 reg. 22.

Words substituted in reg. 15(1), reg. 15(10) substituted and reg. 15(4)--(9) omitted (18.4.95) by the Child Support and Income Support (Amendment) Regulations 1995 reg. 48.

Reg. 15(10) renumbered as reg. 15(4) (22.1.96) by the Child Support (Miscellaneous Amendments) (No. 2) Regulations 1995 reg. 44.
}

%\subsection[16. Weekly amount of housing costs]{Weekly amount of housing costs}
%
%16.  Where a parent pays housing costs—
%\begin{enumerate}\item[]
%($a$) on a weekly basis, the amount of such housing costs shall be the weekly rate payable at the effective date;
%
%($b$) on a monthly basis, the amount of such housing costs shall be the monthly rate payable at the effective date, multiplied by 12 and divided by 52;
%
%% Reg 16($bb$) inserted (18.4.95) by SI 1995/1045 reg 49
%($bb$) by way of rent payable to a housing association, as defined in section 1(1) of the Housing Associations Act 1985\footnote{\frenchspacing 1985 c. 69.}, which is registered in accordance with section 5 of that Act, or to a local authority, on a free week basis, that is to say the basis that he pays an amount by way of rent for a given number of weeks in a 52 week period, with a lesser number of weeks in which there is no liability to pay (“free weeks”), the amount of such housing costs shall be the amount which he pays---
%\begin{enumerate}\item[]
%(i) in the relevant week if it is not a free week; or
%
%(ii) in the last week before the relevant week which is not a free week, if the relevant week is a free week;
%\end{enumerate}
%
%($c$) on any other basis, the amount of such housing costs shall be the rate payable at the effective date, multiplied by the number of payment periods, or the nearest whole number of payment periods (any fraction of one half being rounded up), falling within a period of 365 
%days and divided by 52.
%\end{enumerate}
%
%\amendment{
%Reg. 16($bb$) inserted (18.4.95) by the Child Support and Income Support (Amendment) Regulations 1995 reg. 49.
%}

% Reg 16 substituted (5.8.96) by SI 1996/1945 reg 22
\subsection[16. Weekly amount of housing costs]{Weekly amount of housing costs}

16.—(1) Where a parent pays housing costs—
\begin{enumerate}\item[]
($a$) on a weekly basis, the amount of such housing costs shall subject to paragraph (2), be the weekly rate payable at the effective date;

($b$) on a monthly basis, the amount of such housing costs shall subject to paragraph (2), be the monthly rate payable at the effective date, multiplied by 12 and divided by 52;

($c$) by way of rent payable to a housing association, as defined in section 1(1) of the Housing Associations Act 1985\footnote{\frenchspacing 1985 c. 69.} which is registered in accordance with section 5 of that Act, or to a local authority, on a free week basis, that is to say the basis that he pays an amount by way of rent for a given number of weeks in a 52 week period, with a lesser number of weeks in which there is no liability to pay (“free weeks”), the amount of such housing costs shall be the amount which he pays—
\begin{enumerate}\item[]
(i) in the relevant week if it is not a free week; or

(ii) in the last week before the relevant week which is not a free week, if the relevant week is a free week;
\end{enumerate}

($d$) on any other basis, the amount of such housing costs shall, subject to paragraph (2), be the rate payable at the effective date, multiplied by the number of payment periods, or the nearest whole number of payment periods (any fraction of one half being rounded up), falling within a period of 365 days and divided by 52.
\end{enumerate}

(2) Where housing costs consist of payments on a repayment mortgage and the absent parent or parent with care has not provided information or evidence as to the rate of repayment of the capital secured and the interest payable on that mortgage at the effective date and that absent parent or parent with care has provided a statement from the lender, in respect of a period ending not more than 12 months prior to the first day of the relevant week, for the purposes of the calculation of exempt income under regulation 9 and protected income under regulation 11—
\begin{enumerate}\item[]
($a$) if the amount of capital repaid for the period covered by that statement is shown on it, the rate of repayment of capital owing under that mortgage shall be calulated by reference to that amount; and

($b$) if the amount of capital owing and the interest rate applicable at the end of the period covered by that statement are shown on it, the interest payable on that mortgage shall be calculated by reference to that amount and that interest rate.
\end{enumerate}

\amendment{
Reg. 16 substituted (5.8.96) by the Child Support (Miscellaneous Amendments) Regulations 1996 reg. 22.

\medskip

Reg. 17 revoked (18.4.95) by the Child Support and Income Support (Amendment) Regulations 1995 reg. 50.
}

%\subsection[17. Apportionment of housing costs: exempt income]{Apportionment of housing costs: exempt income}
%
%17.  For the purposes of calculating or estimating exempt income the amount of the housing costs of a parent shall be—
%\begin{enumerate}\item[]
%($a$) where the parent does not have a partner, the whole amount of the housing costs;
%
%($b$) where the parent has a partner, the proportion of the amount of the housing costs calculated by multiplying those costs by—
%\[\frac{0.75 + (\mathrm{A} \times 0.2)}{1.00 + (\mathrm{B} \times 0.2)}\]
%where—
%\begin{enumerate}\item[]
%    A is the number of relevant children (if any);
%
%    B is the number of children in that parent’s family (if any); 
%\end{enumerate}
%
%($c$) where the parent is a member of a polygamous marriage the proportion of the amount of the housing costs calculated by multiplying those costs by—
%\[\frac{0.75 + (\mathrm{A} \times 0.2)}{1.00 + (\mathrm{X} \times 0.25) + (\mathrm{B} \times 0.2)}\]
%where—
%\begin{enumerate}\item[]
%    A and B have the same meanings as in sub-paragraph ($b$); and
%
%    X is the number which is one less than the number of partners. 
%\end{enumerate}
%\end{enumerate}

% Reg 17 revoked (18.4.95) by SI 1995/1045 reg 50

\subsection[18. Excessive housing costs]{Excessive housing costs}

18.—(1) Subject to paragraph (2), the amount of the housing costs of an absent parent which are to be taken into account—
\begin{enumerate}\item[]
($a$) under regulation 9(1)($b$) shall not exceed the greater of £80·00 or half the amount of N as calculated or estimated under regulation 7;

($b$) under regulation 11(1)($b$) shall not exceed the greater of £80·00 or half of the amount calculated in accordance with regulation 11(2).
\end{enumerate}

(2) The restriction imposed by paragraph (1) shall not apply where—
\begin{enumerate}\item[]
($a$) the absent parent in question—
\begin{enumerate}\item[]
(i) has been awarded housing benefit (or is awaiting the outcome of a claim to that benefit);

(ii) has the day to day care of any child; or

(iii) is a person to whom a disability premium under paragraph 11 of the relevant Schedule applies in respect of himself or his partner or would so apply if he were entitled to income support and were aged less than 60;
\end{enumerate}

($b$) the absent parent in question, following a divorce from, or the breakdown of his relationship with, his former partner, remains in the home he occupied with his former partner;

($c$) the absent parent in question has paid the housing costs under the mortgage, charge or agreement in question for a period in excess of 52 weeks before the date of the first application for child support maintenance in relation to a qualifying child of his and there has been no increase in those costs other than an increase in the interest payable under the mortgage or charge or, as the case may be, in the amount payable under the agreement under which the home is held;

($d$) the housing costs in respect of the home in question would not exceed the amount set out in paragraph (1) but for an increase in the interest payable under a mortgage or charge secured on that home or, as the case may be, in the amount payable under any agreement under which it is held; or

($e$) the absent parent is responsible for making payments in respect of housing costs which are higher than they would be otherwise by virtue of the unavailability of his share of the equity of the property formerly occupied with his partner and which remains occupied by that former partner.
\end{enumerate}

\section[Part III --- Special cases]{Part III\\*Special cases}

\renewcommand\parthead{--- Part III}

\subsection[19. Both parents are absent]{Both parents are absent}

19.—(1) Subject to regulation 27, where the circumstances of a case are that each parent of a qualifying child is an absent parent in relation to that child (neither being a person who is treated as an absent parent by regulation 20(2)) that case shall be treated as a special case for the purposes of the Act.

(2) For the purposes of this case—
\begin{enumerate}\item[]
($a$) where the application is made in relation to both absent parents, separate assessments shall be made under Schedule 1 to the Act in respect of each so as to determine the amount of child support maintenance payable by each absent parent;

($b$) subject to paragraph (3), where the application is made in relation to both absent parents, the value of C in each case shall be the assessable income of the other absent parent and where the application is made in relation to only one the value of C in the case of the other shall be nil;

($c$) where the person with care is a body of persons corporate or unincorporate, the value of AG shall 
%not include any amount mentioned in regulation 3(1)($d$) (income support lone parent premium).
include the amount specified in regulation 3(1)($c$)(i) but not the amount specified in regulation 3(1)($c$)(ii) (income support family premium);  % Words substituted (7.4.97) by SI 1996/1803 reg 13

% Reg 19(2)(d) added (7.10.96) by SI 1996/1945 reg 23
($d$) where the application is made in relation to one absent parent only, the amount of the maintenance requirement applicable in that case shall be one-half of the amount determined in accordance with paragraph 1(2) of Schedule 1 to the Act or, where regulation 23 applies (person caring for children of more than one absent parent), of the amount determined in accordance with paragraphs (2) to (3) of that regulation.
\end{enumerate}

(3) Where, for the purposes of paragraph (2)($b$), information regarding the income of the other absent parent has not been submitted to the Secretary of State or to a child support officer within the period specified in regulation 6(1) of the Maintenance Assessment Procedure Regulations then until such information is acquired the value of C shall be nil.

(4) When the information referred to in paragraph (3) is acquired the child support officer shall make a fresh assessment which shall have effect from the effective date in relation to that other absent parent.

\amendment{
Reg. 19(2)($d$) added (7.10.96) by the Child Support (Miscellaneous Amendments) Regulations 1996 reg. 23.

Under the Child Support (Miscellaneous Amendments) Regulations 1996 reg. 25(5) the following version of reg. 19 continues to apply to any application made prior to 7.10.96, and the current version of reg. 19 does not apply to maintenance assessments in force on 7.10.96 until they are first reviewed after 7.10.96 under s. 16, 17 or 18 of the Act:

\begin{quotation}
19.—(1) Subject to regulation 27, where the circumstances of a case are that each parent of a qualifying child is an absent parent in relation to that child (neither being a person who is treated as an absent parent by regulation 20(2)) that case shall be treated as a special case for the purposes of the Act.

(2) For the purposes of this case—
\begin{enumerate}\item[]
($a$) where the application is made in relation to both absent parents, separate assessments shall be made under Schedule 1 to the Act in respect of each so as to determine the amount of child support maintenance payable by each absent parent;

($b$) subject to paragraph (3), where the application is made in relation to both absent parents, the value of C in each case shall be the assessable income of the other absent parent and where the application is made in relation to only one the value of C in the case of the other shall be nil;

($c$) where the person with care is a body of persons corporate or unincorporate, the value of AG shall 
not include any amount mentioned in regulation 3(1)($d$) (income support lone parent premium).
\end{enumerate}

(3) Where, for the purposes of paragraph (2)($b$), information regarding the income of the other absent parent has not been submitted to the Secretary of State or to a child support officer within the period specified in regulation 6(1) of the Maintenance Assessment Procedure Regulations then until such information is acquired the value of C shall be nil.

(4) When the information referred to in paragraph (3) is acquired the child support officer shall make a fresh assessment which shall have effect from the effective date in relation to that other absent parent.
\end{quotation}

Words substituted in reg. 19(2)($c$) (7.4.97) by the Child Benefit, Child Support and Social Security (Miscellaneous Amendments) Regulations 1996 reg. 13 (subject to transitional provisions in reg. 49).
}

\subsection[20. Persons treated as absent parents]{Persons treated as absent parents}

20.—(1) Where the circumstances of a case are that—
\begin{enumerate}\item[]
($a$) two or more persons who do not live in the same household each provide day to day care for the same qualifying child; and

($b$) at least one of those persons is a parent of that child,
\end{enumerate}
that case shall be treated as a special case for the purposes of the Act.

(2) For the purposes of this case a parent who provides day to day care for a child of his in the following circumstances is to be treated as an absent parent for the purposes of the Act and these Regulations—
\begin{enumerate}\item[]
($a$) a parent who provides such care to a lesser extent than the other parent, person or persons who provide such care for the child in question;

($b$) where the persons mentioned in paragraph (1)($a$) include both parents and the circumstances are such that care is provided to the same extent by both but each provides care to a greater or equal extent than any other person who provides such care for that child—
\begin{enumerate}\item[]
(i) the parent who is not in receipt of child benefit for the child in question; or

(ii) if neither parent is in receipt of child benefit for that child, the parent who, in the opinion of the child support officer, will not be the principal provider of day to day care for that child.
\end{enumerate}
\end{enumerate}

(3) Subject to paragraphs (5) and (6), where a parent is treated as an absent parent under paragraph (2) child support maintenance shall be payable by that parent in respect of the child in question and the amount of the child support maintenance so payable shall be calculated in accordance with the formula set out in paragraph (4).

(4) The formula for the purposes of paragraph (3) is—
\[\mathrm{T} = \mathrm{X} - \left\{ (\mathrm{X}+\mathrm{Y}) \times \frac{\mathrm{J}}{7 \times \mathrm{L}} \right\}\]
where—
\begin{enumerate}\item[]
    T is the amount of child support maintenance payable;

    X is the amount of child support maintenance which would be payable by the parent who is treated as an absent parent, assessed under Schedule 1 to the Act as if paragraphs 6 and 7 of that Schedule did not apply, and, where the other parent is an absent parent, as if the value of C was the assessable income of the other parent;

    Y is—
\begin{enumerate}\item[]
    (i)
    the amount of child support maintenance assessed under Schedule 1 to the Act payable by the other parent if he is an absent parent or which would be payable if he were an absent parent, and for the purposes of such calculation the value of C shall be the assessable income of the parent treated as an absent parent under paragraph (2); or,

    (ii)
    if there is no such other parent, shall be nil;
\end{enumerate}

    J is the total of the weekly average number of nights for which day to day care is provided by the person who is treated as the absent parent in respect of each child included in the maintenance assessment and shall be calculated to 2 decimal places;

    L is the number of children who are included in the maintenance assessment in question. 
\end{enumerate}

(5) Where the value of T calculated under the provisions of paragraph (4) is less than zero, no child support maintenance shall be payable.

(6) The liability to pay any amount calculated under paragraph (4) shall be subject to the provision made for protected income and minimum payments under paragraphs 6 and 7 of Schedule 1 to the Act.

\subsection[21. One parent is absent and the other is treated as absent]{One parent is absent and the other is treated as absent}

21.—(1) Where the circumstances of a case are that one parent is an absent parent and the other parent is treated as an absent parent by regulation 20(2), that case shall be treated as a special case for the purposes of the Act.

(2) For the purpose of assessing the child support maintenance payable by an absent parent where this case applies, each reference in Schedule 1 to the Act to a parent who is a person with care shall be treated as a reference to a person who is treated as an absent parent by regulation 20(2).

\subsection[22. Multiple applications relating to an absent parent]{Multiple applications relating to an absent parent}

22.—%(1) Where the circumstances of a case are that—
%\begin{enumerate}\item[]
%($a$) two or more applications for a maintenance assessment have been made which relate to the same absent parent (or to a person who is treated as an absent parent by regulation 20(2)); and
%
%($b$) those applications relate to different children,
%\end{enumerate}
%that case shall be treated as a special case for the purposes of the Act.
%
% Reg 22(1) substituted (22.1.96) by SI 1995/3261 reg 45(2)
(1) Where an application for a maintenance assessment has been made in respect of an absent parent and—
\begin{enumerate}\item[]
($a$) at least one other application for a maintenance assessment has been made in relation to the same absent parent (or a person who is treated as an absent parent by regulation 20(2)) but to different children; or

($b$) at least one maintenance assessment is in force in relation to the same absent parent or a person who is treated as an absent parent by regulation 20(2) but to different children,
\end{enumerate}
that case shall be treated as a special case for the purposes of the Act.

%(2) For the purposes of assessing the amount of child support maintenance payable in respect of each application where paragraph (1) applies, for references to the assessable income of an absent parent in the Act and in these Regulations there shall be substituted references to the amount calculated by the formula—
%\[\mathrm{A} \times \frac{\mathrm{B}}{\mathrm{D}}\]
%where—
%\begin{enumerate}\item[]
%    A is the assessable income of the absent parent;
%
%    B is the maintenance requirement calculated in respect of the application in question;
%
%    D is the sum of the maintenance requirements as calculated for the purposes of each application relating to the absent parent in question. 
%\end{enumerate}

% Reg 22(2) substituted (18.4.95) by SI 1995/1045 reg 51
(2) For the purposes of assessing the amount of child support maintenance payable in respect of each application where 
%paragraph (1) 
paragraph (1)($a$)  % Words substituted (22.1.96) by SI 1995/3261 reg 45(3)
applies
or in respect of the application made in circumstances where paragraph (1)($b$) applies, % Words inserted (22.1.96) by SI 1995/3261 reg 45(3)
for references to the assessable income of an absent parent in the Act and in these Regulations% 
, and subject to paragraph (2ZA),  % Words inserted (2.12.96) by SI 1996/2907 reg 68(5)(a)
there shall be substituted references to the amount calculated by the formula—
\[ \left( (A + T) \times \frac{B}{D}\right) - CS\]
where—
\begin{enumerate}\item[]
A is the absent parent’s assessable income;

T is the sum of the amounts allowable in the calculation or estimation of his exempt income by virtue of Schedule 3A;

B is the maintenance requirement calculated in respect of the application in question;

D is the sum of the maintenance requirements as calculated for the purposes of each assessment relating to the absent parent in question; and

CS is the amount (if any) allowable by virtue of Schedule 3A in calculating or estimating the absent parent’s exempt income in respect of a relevant qualifying transfer of property in respect of the assessment in question.
\end{enumerate}

% Reg 22(2ZA) inserted (2.12.96) by SI 1996/2907 reg 68(5)(b)
(2ZA) Where a case falls within regulation 39(1)($a$) of the Departure Direction
and Consequential Amendment Regulations, for the purposes of assessing the
amount of child support maintenance payable in respect of an application for
child support maintenance before a departure direction in respect of the
maintenance assessment in question is given, for references to the assessable
income of an absent parent in the Act and in these Regulations there shall be
substituted references to the amount calculated by the formula—
\[(A + T) \times \frac{B}{D}\]
where A, T, B and D have the same meanings as in paragraph (2).

% Reg 22(2A) inserted (22.1.96) by SI 1995/3261 reg 45(4)
(2A) Where paragraph (1)($b$) applies, and a maintenance assessment has been made in respect of the application referred to in paragraph (1), each maintenance assessment in force at the time of that assessment shall be reduced using the formula for calculation of assessable income set out in paragraph (2) and each reduction shall take effect on the date specified in regulation 33(7) of the Maintenance Assessment Procedure Regulations.

(3) Where more than one maintenance assessment has been made with respect to the absent parent and payment by him of the aggregate of the amounts of those assessments would reduce his disposable income below his protected income level, the aggregate amount of those assessments shall be reduced (each being reduced by reference to the same proportion as those assessments bear to each other) by the minimum amount necessary to prevent his disposable income being reduced below his protected income level provided that the aggregate amount payable under those assessments shall not be reduced to less than the minimum amount prescribed in regulation 13(1).

%(4) Where the aggregate of the child support maintenance payable by the absent parent is less than the minimum amount prescribed in regulation 13(1), the child support maintenance payable shall be that prescribed minimum amount apportioned between the two or more applications in the same ratio as the maintenance requirements in question bear to each other.
% Reg 22(4) substituted (5.4.93) by SI 1993/913 reg 23.
(4) Where the aggregate of the child support maintenance payable by the absent parent is less than the minimum amount prescribed in regulation 13(1), the child support maintenance payable shall be---
\begin{enumerate}\item[]
($a$) that prescribed minimum amount apportioned between the two or more applications in the same ratio as the maintenance requirements in question bear to each other; or

($b$) where, because of the application of regulation 2(2), such an apportionment produces an aggregate amount which is different from that prescribed minimum amount, that different amount.
\end{enumerate}

(5) Payment of each of the maintenance assessments calculated under this regulation shall satisfy the liability of the absent parent (or a person treated as such) to pay child support maintenance.

\amendment{
Reg. 22(4) substituted (5.4.93) by the Child Support (Miscellaneous Amendments) Regulations 1993 reg. 23.

Reg. 22(2) substituted (18.4.95) by the Child Support and Income Support (Amendment) Regulations reg. 51.

Words inserted and substituted in reg. 22(2), reg. 22(2A) inserted and reg. 22(1) substituted (22.1.96) by the Child Support (Miscellaneous Amendments) (No. 2) Regulations 1995 reg. 45.

Words inserted in reg. 22(2) and reg. 22(2ZA) inserted (2.12.96) by the Child Support Departure Direction and Consequential Amendments Regulations 1996 reg. 68(5).
}

\subsection[23. Person caring for children of more than one absent parent]{Person caring for children of more than one absent parent}

23.—(1) Where the circumstances of a case are that—
\begin{enumerate}\item[]
($a$) a person is a person with care in relation to two or more qualifying children; and

($b$) in relation to at least two of those children there are different persons who are absent parents or persons treated as absent parents by regulation 20(2);
\end{enumerate}
that case shall be treated as a special case for the purposes of the Act.

(2) 
Subject to paragraph (2A), % Words inserted in reg 23(2) (7.2.94) by SI 1994/227 reg 4(6)
in calculating the maintenance requirements for the purposes of this case, for any amount which (but for this paragraph) would have been included under regulation 3(1)($b$)
%, ($c$) or ($d$) 
or ($c$)  % Words substituted (7.4.97) by SI 1996/1803 reg 14
(amounts included in the calculation of AG) there shall be substituted an amount calculated by dividing the amount which would have been so included by the relevant number.

%Reg 23(2A) inserted (7.2.94) by SI 1994/227 reg 4(6)
(2A) In applying the provisions of paragraph (2) to the amount which is to be included in the maintenance requirements under regulation 3(1)($b$)—
\begin{enumerate}\item[]
($a$) first take the amount specified in head (i) of regulation 3(1)($b$) and divide it by the relevant number;

($b$) then apply the provisions of regulation 3(1)($b$) as if the references to the amount specified in column (2) of paragraph 1(1)($e$) of the relevant Schedule were references to the amount which is the product of the calculation required by head ($a$) above, and as if, in relation to an absent parent, the only qualifying children to be included in the assessment were those qualifying children in relation to whom he is the absent parent.
\end{enumerate}

(3) 
%In paragraph (2) 
In paragraph (2) and (2A) % Words substituted in reg 23(3) (7.2.94) by SI 1994/227 reg 4(7)
“the relevant number” means the number equal to the total number of persons who, in relation to those children, are either absent parents or persons treated as absent parents by regulation 20(2) except that where in respect of the same child both parents are persons who are either absent parents or persons who are treated as absent parents under that regulation, they shall count as one person.

(4) Where the circumstances of a case fall within this regulation and the person with care is the parent of any of the children, for C in paragraph 2(1) of Schedule 1 to the Act (the assessable income of that person) there shall be substituted the amount which would be calculated under regulation 22(2) if the references therein to an absent parent were references to a parent with care.

\amendment{Words inserted in reg. 23(2), words substituted in reg. 23(3) and reg. 23(2A) inserted (7.2.94) by the Child Support (Miscellaneous Amendments and Transitional Provisions) Regulations reg. 4(6), (7) (subject to transitional provisions in reg. 12).

Words substituted in reg. 23(2) (7.4.97) by the Child Benefit, Child Support and Social Security (Miscellaneous Amendments) Regulations 1996 reg. 14 (subject to transitional provisions in reg. 49).
}

\subsection[24. Persons with part-time care—not including a person treated as an absent parent]{Persons with part-time care—not including a person treated as an absent parent}

24.—(1) Where the circumstances of a case are that—
\begin{enumerate}\item[]
($a$) two or more persons who do not live in the same household each provide day to day care for the same qualifying child; and

($b$) those persons do not include any parent who is treated as an absent parent of that child by regulation 20(2),
\end{enumerate}
that case shall be treated as a special case for the purposes of the Act.

(2) For the purposes of this case—
\begin{enumerate}\item[]
($a$) the person whose application for a maintenance assessment is being proceeded with shall, subject to paragraph ($b$), be entitled to receive all of the child support maintenance payable under the Act in respect of the child in question;

($b$) on request being made to the Secretary of State by—
\begin{enumerate}\item[]
(i) that person; or

(ii) any other person who is providing day to day care for that child and who intends to continue to provide that care,
\end{enumerate}
the Secretary of State may make arrangements for the payment of any child support maintenance payable under the Act to the persons who provide such care in the same ratio as that in which it appears to the Secretary of State, that each is to provide such care for the child in question;

($c$) before making an arrangement under sub-paragraph ($b$), the Secretary of State shall consider all of the circumstances of the case and in particular the interests of the child, the present arrangements for the day to day care of the child in question and any representations or proposals made by the persons who provide such care for that child.
\end{enumerate}

\subsection[25. Care provided in part by a local authority]{Care provided in part by a local authority}

25.—(1) Where the circumstances of a case are that a local authority and a person each provide day to day care for the same qualifying child, that case shall be treated as a special case for the purposes of the Act.

(2) 
%In a case where this regulation applies—
Subject to paragraph (3), in a case where this regulation applies---  % Words substituted (18.4.95) by SI 1995/1045 reg 52(2)
\begin{enumerate}\item[]
($a$) child support maintenance shall be calculated in respect of that child as if this regulation did not apply;

($b$) the amount so calculated shall be divided by 7 so as to produce a daily amount;

($c$) in respect of each night for which day to day care for that child is provided by a person other than the local authority, the daily amount relating to that period shall be payable by the absent parent (or, as the case may be, by the person treated as an absent parent under regulation 20(2));

($d$) child support maintenance shall not be payable in respect of any night for which the local authority provides day to day care for that qualifying child.
\end{enumerate}

% Reg 25(3) inserted (18.4.95) by SI 1995/1045 reg 52(3)
(3) In a case where more than one qualifying child is included in a child support maintenance assessment application and where this regulation applies to at least one of those children, child support maintenance shall be a calculated by applying the formula—
\[ \mathrm{S} \times \left( \frac{\mathrm{A}}{7 \times \mathrm{B}} \right)  \]
where—
\begin{enumerate}\item[]
S is the total amount of child support maintenance in respect of all qualifying children included in that maintenance assessment application, calculated as if this regulation did not apply;

A is the aggregate of the number of nights of day to day care for all qualifying children included in that maintenance assessment application provided in each week by a person other than the local authority;

B is the number of qualifying children in respect of whom the maintenance assessment application has been made.
\end{enumerate}

\amendment{
Words substituted in reg. 25(2) and reg. 25(3) inserted (18.4.95) by the Child Support and Income Support (Amendment) Regulations 1995 reg. 52.
}

\subsection[26. Cases where child support maintenance is not to be payable]{Cases where child support maintenance is not to be payable}

26.—(1) Where the circumstances of a case are that—
\begin{enumerate}\item[]
($a$) but for this regulation the minimum amount prescribed in regulation 13(1) would apply; and

($b$) any of the following conditions are satisfied—
\begin{enumerate}\item[]
(i) the income of the absent parent includes one or more of the payments or awards specified in Schedule 4 or would include such a payment but for a provision preventing the receipt of that payment by reason of it overlapping with some other benefit payment or would, in the case of the payments referred to in paragraph ($a$)(i) or (iv) of that Schedule, include such a payment if the relevant contribution conditions for entitlement had been satisfied;

(ii) an amount to which regulation 
%11(1)($f$) 
11(1)($c$) or ($f$)  % Word substituted (7.4.97) by SI 1996/1803 reg 15
applies (protected income: income support family premium) is taken into account in calculating or estimating 
under paragraphs (1) to (5) of regulation 11  % Words inserted (18.4.95) by SI 1995/1045 reg 53
the protected income of the absent parent;

(iii) the absent parent is a child within the meaning of section 55 of the Act;

(iv) the absent parent is a prisoner; or

(v) the absent parent is a person in respect of whom N (as calculated or estimated under regulation 7(1)) is less than the minimum amount prescribed by regulation 13(1),
\end{enumerate}
\end{enumerate}
the case shall be treated as a special case for the purposes of the Act.

(2) For the purposes of this case—
\begin{enumerate}\item[]
($a$) the requirement in paragraph 7(2) of Schedule 1 to the Act (minimum amount of child support maintenance fixed by an assessment to be the prescribed minimum amount) shall not apply;

($b$) the amount of the child support maintenance to be fixed by the assessment shall be nil.
\end{enumerate}

\amendment{
Words inserted in reg. 26(1)($b$)(ii) (18.4.95) by the Child Support and Income Support (Amendment) Regulations 1995 reg. 53.

Words substituted in reg. 26(1)($b$)(ii) (7.4.97) by the Child Benefit, Child Support and Social Security (Miscellaneous Amendments) Regulations 1996 reg. 15 (subject to transitional provisions in reg. 49).
}

\subsection[27. Child who is a boarder or an in-patient]{Child who is a boarder or an in-patient}

27.—(1) Where the circumstances of a case are that—
\begin{enumerate}\item[]
($a$) a qualifying child is a boarder at a boarding school or is an in-patient in a hospital; and

($b$) by reason of those circumstances, the person who would otherwise provide day to day care is not doing so,
\end{enumerate}
that case shall be treated as a special case for the purposes of the Act.

(2) For the purposes of this case, section 3(3)($b$) of the Act shall be modified so 
that % Word inserted (5.4.93) by SI 1993/913 reg 24.
for the reference to the person who usually provides day to day care for the child there shall be substituted a reference to the person who would usually be providing such care for that child but for the circumstances specified in paragraph (1).

\amendment{
Word inserted in reg. 27(2) (5.4.93) by the Child Support (Miscellaneous Amendments) Regulations 1993 reg. 24.
}

%Reg 27A inserted (5.4.93) by SI 1993/913 reg 25.
\subsection[27A. Child who is allowed to live with his parent under section 23(5) of the Children Act 1989]{Child who is allowed to live with his parent under section 23(5) of the Children Act 1989}

27A.—(1) Where the circumstances of a case are that a qualifying child who is in the care of a local authority in England and Wales is allowed by the authority to live with a parent of his under section 23(5) of the Children Act 1989\footnote{\frenchspacing 1989 c. 41.}, that case shall be treated as a special case for the purposes of the Act.

(2) For the purposes of this case, section 3(3)($b$) of the Act shall be modified so that for the reference to the person who usually provides day to day care for the child there shall be substituted a reference to the parent of a child whom the local authority allow the child to live with under section 23(5) of the Children Act 1989.

\amendment{
Reg. 27A inserted (5.4.93) by the Child Support (Miscellaneous Amendments) Regulations 1993 reg. 25.
}

\subsection[28. Amount payable where absent parent is in receipt of income support or other prescribed benefit]{Amount payable where absent parent is in receipt of income support or other prescribed benefit}

28.—(1) Where the condition specified in section 43(1)($a$) of the Act is satisfied in relation to an absent parent (assessable income to be nil where income support%
, income-based jobseeker’s allowance  % Words inserted (7.10.96) by SI 1996/1345 reg 6(3)
 or other prescribed benefit is paid), the prescribed conditions for the purposes of section 43(1)($b$) of the Act are that—
\begin{enumerate}\item[]
($a$) the absent parent is aged 18 or over;

($b$) he does not satisfy the conditions in paragraph 
%3 
3($a$) or ($b$)  % Word substituted (7.4.97) by SI 1996/1803 reg 16
of the relevant Schedule (income support family premium)
and does not have day to day care of any child (whether or not a relevant child)% Words inserted (5.4.93) by SI 1993/913 reg 26(1)($a$)
; and

($c$) 
%he does not satisfy the conditions for entitlement to 
his income does not include %Words substituted (5.4.93) by SI 1993/913 reg 26(1)($b$)
one or more of the payments or awards specified in Schedule 4 (other than by reason of a provision preventing receipt of overlapping benefits or by reason of a failure to satisfy the relevant contribution conditions).
\end{enumerate}

(2) For the purposes of section 43(2)($a$) of the Act, the prescribed amount shall be equal to the minimum amount prescribed in regulation 13(1) for the purposes of paragraph 7(1) of Schedule 1 to the Act.

%Reg 28(3)--(5) inserted (5.4.93) by SI 1993/913 reg 26(2)
%Reg 28(3) substituted (26.4.93) by SI 1993/925 reg 2(2)
(3) Subject to paragraph (4), where—
\begin{enumerate}\item[]
($a$) an absent parent is liable under section 43 of the Act and this regulation to make payments in place of payments of child support maintenance with respect to two or more qualifying children in relation to whom there is more than one parent with care; or

($b$) that absent parent and his partner (within the meaning of regulation 2(1) of the Social Security (Claims and Payments) Regulations 1987\footnote{\frenchspacing S.I. 1987/1968.}) are both liable to make such payments,
\end{enumerate}
the prescribed amount mentioned in paragraph (2) shall be apportioned between the persons with care in the same ratio as the maintenance requirements of the qualifying child or children in relation to each of those persons with care bear to each other.

(4) If, in making the apportionment required by paragraph (3), the effect of the application of regulation 2(2) would be such that the aggregate amount payable would be different from the amount prescribed in paragraph (2) the Secretary of State shall adjust that apportionment so as to eliminate that difference; and that adjustment shall be varied from time to time so as to secure that, taking one week with another and so far as is practicable, each person with care receives the amount which she would have received if no adjustment had been made under this paragraph.

(5) The provisions of Schedule 5 shall have effect in relation to cases to which section 43 of the Act and this regulation apply.

\amendment{
Words inserted in reg. 28(1)($b$), words substituted in reg. 28(1)($c$) and reg. 28(3)--(5) inserted (5.4.93) by the Child Support (Miscellaneous Amendments) Regulations 1993 reg. 26.

Reg. 28(3) substituted (26.4.93) by the Child Support (Maintenance Assessments and Special Cases) Amendment Regulations 1993 reg. 2(2).

Words inserted in reg. 28(1) (7.10.96) by the Social Security and Child Support (Jobseeker's Allowance) (Consequential Amendments) Regulations 1996 reg. 6(3).

Word substituted in reg. 28(1)($b$) (7.4.97) by the Child Benefit, Child Support and Social Security (Miscellaneous Amendments) Regulations 1996 reg. 16 (subject to transitional provisions in reg. 49).
}

\bigskip

Signed by authority of the Secretary of State for Social Security.

{\raggedleft
\emph{Alistair Burt}\\*Parliamentary Under-Secretary of State,\\*Department of Social Security

}

20th July 1992

\clearpage

\small

\part*{S C H E D U L E S}

\part[Schedule 1 --- Calculation of N and M]{Schedule 1\\*Calculation of N and M}

\section[Part I --- Earnings]{Part I\\*Earnings}

\subsection[Chapter I --- Earnings of an employed earner]{Chapter I\\*Earnings of an employed earner}

\renewcommand\parthead{--- Schedule 1 Part I Chapter I}

1.—(1) Subject to sub-paragraphs (2) and (3), “earnings” means in the case of employment as an employed earner, any remuneration or profit derived from that employment and includes—
\begin{enumerate}\item[]
($a$) any bonus, commission, 
payment in respect of overtime,  % Words inserted (18.4.95) by SI 1995/1045 reg 54(2)
royalty or fee;

% Para 1(1)(aa) inserted (13.1.97) by SI 1996/3196 reg 13(2)(a)
($aa$) any profit-related pay, whether paid in anticipation of, or following, the calculation of profits;

($b$) any holiday pay except any payable more than 4 weeks after termination of the employment;

($c$) any payment by way of a retainer;

%($d$) any payment made by the parent’s employer in respect of any expenses not wholly, exclusively and necessarily incurred in the performance of the duties of the employment;

% Para 1(1)(d) substituted (7.10.96) by SI 1996/1945 reg 24(2)
($d$) any payments made by the parent’s employer in respect of any expenses not wholly, exclusively and necessarily incurred in the performance of the duties of the employment, including any payment made by the parent’s employer in respect of—
\begin{enumerate}\item[]
(i) travelling expenses incurred by that parent between his home and place of employment; and

(ii) expenses incurred by that parent under arrangements made for the care of a member of his family owing to that parent’s absence from home;
\end{enumerate}

($e$) any award of compensation made under section 68(2) or 71(2)($a$) of the Employment Protection (Consolidation) Act 1978\footnote{\frenchspacing 1978 c. 44.} (remedies and compensation for unfair dismissal);

($f$) any such sum as is referred to in section 112 of the Contributions and Benefits Act (certain sums to be earnings for social security purposes);

($g$) any statutory sick pay under Part I of the Social Security and Housing Benefits Act 1982\footnote{\frenchspacing 1982 c. 24.} or statutory maternity pay under Part V of the Social Security Act 1986\footnote{\frenchspacing 1986 c. 50.};

($h$) any payment in lieu of notice and any compensation in respect of the absence or inadequacy of any such notice but only insofar as such payment or compensation represents loss of income;

($i$) any payment relating to a period of less than a year which is made in respect of the performance of duties as—
\begin{enumerate}\item[]
(i) an auxiliary coastguard in respect of coast rescue activities;

(ii) a part-time fireman in a fire brigade maintained in pursuance of the Fire Services Acts 1947 to 1959\footnote{\frenchspacing 10 \& 11 Geo. 6 c. 41, 14 \& 15 Geo. 6 c. 27, 7 \& 8 Eliz. 2 c. 44.};

(iii) a person engaged part-time in the manning or launching of a lifeboat;

(iv) a member of any territorial or reserve force prescribed in Part I of Schedule 3 to the Social Security (Contributions) Regulations 1979\footnote{\frenchspacing S.I. 1979/591; the relevant amending instrument is S.I. 1980/1975.};
\end{enumerate}

($j$) any payment made by a local authority to a member of that authority in respect of the performance of his duties as a member, other than any expenses wholly, exclusively and necessarily incurred in the performance of those duties.
\end{enumerate}

(2) Earnings shall not include—
\begin{enumerate}\item[]
($a$) any payment in respect of expenses wholly, exclusively and necessarily incurred in the performance of the duties of the employment;

($b$) any occupational pension;

($c$) any payment where—
\begin{enumerate}\item[]
(i) the employment in respect of which it was made has ceased; and

(ii) a period of the same length as the period by reference to which it was calculated has expired since that cessation but prior to the effective date;
\end{enumerate}

($d$) any advance of earnings or any loan made by an employer to an employee;

($e$) any amount received from an employer during a period when the employee has withdrawn his services by reason of a trade dispute;

($f$) any payment in kind;

($g$) where, in any week or other period which falls within the period by reference to which earnings are calculated, earnings are received both in respect of a previous employment and in respect of a subsequent employment, the earnings in respect of the previous employment;

% Para 1(2)(h) inserted (13.1.97) by SI 1996/3196 reg 13(2)(b)
($h$) any tax-exempt allowance made by an employer to an employee.
\end{enumerate}

(3) The earnings to be taken into account for the purposes of calculating N and M shall be gross earnings less—
\begin{enumerate}\item[]
($a$) any amount deducted from those earnings by way of—
\begin{enumerate}\item[]
(i) income tax;

(ii) primary Class 1 contributions under the Contributions and Benefits Act
or under the Social Security Contributions and Benefits (Northern Ireland) Act 1992;  % Words inserted (18.4.95) by SI 1995/1045 reg 54(3)($a$)
and
\end{enumerate}

($b$) one half of any sums paid by the parent towards an 
%occupational or personal pension scheme.
occupational pension scheme;  % Words substituted (18.4.95) by SI 1995/1045 reg 54(3)($b$)

% Para 1(3)($c$) inserted (18.4.95) by SI 1995/1045 reg 54(3)($c$)
($c$) one half of any sums paid by the parent towards a personal pension scheme, or, where that scheme is intended partly to provide a capital sum to discharge a mortgage secured upon the parent’s home, 37.5 per centum of any such sums.
\end{enumerate}

\amendment{
Words inserted in para. 1(1)($a$), (3)($a$)(ii), words substituted in para. 1(3)($b$) and para. 1(3)($c$) inserted (18.4.95) by the Child Support and Income Support (Amendment) Regulations reg. 54(2), (3).

Para. 1(1)($d$) substituted (7.10.96) by the Child Support (Miscellaneous Amendments) Regulations 1996 reg. 24(2).

Para. 1(1)(aa), (2)(h) inserted (13.1.97) by the Child Support (Miscellaneous Amendments) (No. 2) Regulations 1996 reg. 13(2).

Under the Child Support (Miscellaneous Amendments) (No. 2) Regulations 1996 reg. 16, a maintenance assessment in force on 13.1.97 shall not be reviewed solely to give effect to para. 1(1)(aa), (2)(h), but para. 1(1)(aa), (2)(h) shall apply in conducting a review of the assessment under s. 16, 17, 18 or 19 of the Act, and the effective date of any fresh assessment affected by para. 1(1)(aa), (2)(h) shall not be earlier than the first day of the first maintenance period which commences on or after 13.1.97.
}

\medskip

2.—%(1) Subject to sub-paragraphs (2) to (4)—
%\begin{enumerate}\item[]
%($a$) where a person is paid weekly, the amount of those earnings shall be determined by aggregating the amounts received in the 5 weeks ending with the relevant week and dividing by 5;
%
%($b$) where a person is paid monthly, the amount of those earnings shall be determined by aggregating the amounts received in the 2 months ending with the relevant week, multiplying the aggregate by 6 and dividing by 52;
%
%($c$) where a person is paid by reference to some other period, the amount of those earnings shall be determined by aggregating the amounts received in the 3 months ending with the relevant week, multiplying the aggregate by 4 and dividing by 52.
%\end{enumerate}
%
% Para. 2(1) substituted (18.4.95) by SI 1995/1045 reg 54(4)
(1) Subject to sub-paragraphs 
%(2) 
(1A)  % Word substituted (13.1.96) by SI 1996/3196 reg 13(3)(a)
to (4), the amount of the earnings to be taken into account for the purpose of calculating N and M shall be calculated or estimated by reference to the average earnings at the relevant week having regard to such evidence as is available in relation to that person’s earnings during such period as appears appropriate to the child support officer beginning not earlier than eight weeks before the relevant week and ending not later than the date of the assessment and for the purpose of that calculation or estimate he may consider evidence of that person’s cumulative earnings during the period beginning with the start of the year of assessment (within the meaning of section 832 of the Income and Corporation Taxes Act 1988\footnote{\frenchspacing 1988 c. 1.}) in which the relevant week falls and ending with a date no later than the date of the assessment.

% Para 2(1A) inserted (13.1.96) by SI 1996/3196 reg 13(3)(b)
(1A) Subject to sub-paragraph (4), where a person has claimed, or has been paid, family credit or disability working allowance on any day during the period beginning not earlier than eight weeks before the relevant week and ending not later than the date on which the assessment is made, the child support officer may have regard to the amount of earnings taken into account in determining entitlement to those benefits in order to calculate or estimate the amount of earnings to be taken into account for the purposes of calculating N and M, notwithstanding the fact that entitlement to those benefits may have been determined by reference to earnings attributable to a period other than that specified in sub-paragraph (1).

%(2) Where a person’s earnings include a bonus or commission which is paid during the period of 52 weeks ending with the relevant week and is paid separately from, or, in relation to a longer period than, the other earnings with which it is paid, the amount of that bonus or commission shall be determined by aggregating such payments received in the 52 weeks ending with the relevant week and dividing by 52.

% Para 2(2) substituted (13.1.96) by SI 1996/3196 reg 13(3)(c)
(2) Where a person’s earnings during the period of 52 weeks ending with the relevant week include—
\begin{enumerate}\item[]
($a$) a bonus, commission, or payment of profit-related pay made in anticipation of the calculation of profits which is paid separately from or in relation to a longer period than, the other earnings with which it is paid; or

($b$) a payment in respect of profit-related pay made following the calculation of the employer’s profits,
\end{enumerate}
the amount of that bonus, commission or profit-related payment shall be determined for the purposes of the calculation of earnings by aggregating any such payments received in that period and dividing by 52.

(3) Subject to sub-paragraph (4), the amount of any earnings of a student shall be determined by aggregating the amount received in the year ending with the relevant week and dividing by 52 or, where the person in question has been a student for less than a year, by aggregating the amount received in the period starting with his becoming a student and ending with the relevant week and dividing by the number of complete weeks in that period.

% Para 2(3A) inserted (13.1.96) by SI 1996/3196 reg 13(3)(d)
(3A) Where a case is one to which regulation 30A(1) or (3) of the Maintenance Assessment Procedure Regulations applies (effective dates of new maintenance assessments in particular cases), the term “relevant week” shall, for the purpose of this paragraph, mean the period of 7 days immediately preceding the date on which the information or evidence is received which enables a child support officer to make a new maintenance assessment calculated in accordance with the provisions of Part I of Schedule 1 to the Act in respect of that case for a period beginning after the effective date applicable to that case.

(4) Where a calculation would, but for this sub-paragraph, produce an amount which, in the opinion of the child support officer, does not accurately reflect the normal amount of the earnings of the person in question, such earnings, or any part of them, shall be calculated by reference to such other period as may, in the particular case, enable the normal weekly earnings of that person to be determined more accurately and for this purpose the child support officer shall have regard to—
\begin{enumerate}\item[]
($a$) the earnings received, or due to be received, from any employment in which the person in question is engaged, has been engaged or is due to be engaged;

($b$) the duration and pattern, or the expected duration and pattern, of any employment of that person.
\end{enumerate}

\amendment{
Para. 2(1) substituted (18.4.95) by the Child Support and Income Support (Amendment) Regulations 1995 reg. 54(4).

Word substituted in para. 2(1), para. 2(1A), (3A) inserted and para. 2(2) substituted (13.1.97) by the Child Support (Miscellaneous Amendments) (No. 2) Regulations 1996 reg. 13(3).

Under the Child Support (Miscellaneous Amendments) (No. 2) Regulations 1996 reg. 16, a maintenance assessment in force on 13.1.97 shall not be reviewed solely to give effect to para. 2(1A), (3A) or para. 2(2) as substituted by those regulations, but para. 2(1A), (2) as substituted, (3A) shall apply in conducting a review of the assessment under s. 16, 17, 18 or 19 of the Act, and the effective date of any fresh assessment affected by para. 2(1A), (2) as substituted, (3A) shall not be earlier than the first day of the first maintenance period which commences on or after 13.1.97.
}

\subsection[Chapter II --- Earnings of a self-employed earner]{Chapter II\\*Earnings of a self-employed earner}

\renewcommand\parthead{--- Schedule 1 Part I Chapter II}

3.—(1) Subject to sub-paragraphs (2) and (3) and to paragraph 4, “earnings” in the case of employment as a self-employed earner means the gross receipts of the employment including, where an allowance in the form of periodic payments is paid under section 2 of the Employment and Training Act 1973\footnote{\frenchspacing 1973 c. 50; section 2 was amended by sections 9 and 11 of, and Schedule 2, Part II, paragraph 9 and Schedule 3, to the Employment and Training Act 1981 (c. 57).} or section 2 of the Enterprise and New Towns (Scotland) Act 1990\footnote{\frenchspacing 1990 c. 35.} in respect of the relevant week for the purpose of assisting him in carrying on his business, the total of those payments made during the period by reference to which his earnings are determined under paragraph 5.

(2) Earnings shall not include—
\begin{enumerate}\item[]
($a$) any allowance paid under either of those sections in respect of any part of the period by reference to which his earnings are determined under paragraph 5 if no part of that allowance is paid in respect of the relevant week;

($b$) any income consisting of payments received for the provision of board and lodging accommodation unless such payments form the largest element of the recipient’s income.
\end{enumerate}

(3) 
Subject to sub-paragraph (7), % Words inserted (5.4.93) by SI 1993/913 reg 27(1)($a$)
there shall be deducted from the gross receipts referred to in sub-paragraph (1)—
\begin{enumerate}\item[]
($a$) 
except in a case to which paragraph 4 applies, % Words inserted (5.4.93) by SI 1993/913 reg 27(1)($b$)
any expenses which are reasonably incurred and are wholly and exclusively defrayed for the purposes of the earner’s business in the period by reference to which his earnings are determined under paragraph 5(1) or, where paragraph 5(2) applies, any such expenses relevant to the period there mentioned (whether or not defrayed in that period);

($b$) 
except in a case to which paragraph 4 
  or 5(2)  % Words inserted (18.4.95) by SI 1995/1045 reg 54(5)($a$)
applies, % Words inserted (5.4.93) by SI 1993/913 reg 27(1)($b$)
any value added tax paid in the period by reference to which earnings are determined in excess of value added tax received in that period;

($c$) any amount in respect of income tax determined in accordance with sub-paragraph (5);

($d$) any amount in respect of National Insurance contributions determined in accordance with sub-paragraph (6);

($e$) one half of any premium paid in respect of a retirement annuity contract or a personal pension scheme%
, or, where that scheme is intended partly to provide a capital sum to discharge a mortgage or charge secured upon the parent’s home, 37.5 per centum of the contributions payable. % Words inserted (18.4.95) by SI 1995/1045 reg 54(5)($b$)
\end{enumerate}

(4) For the purposes of sub-paragraph (3)($a$)—
\begin{enumerate}\item[]
($a$) such expenses include—
\begin{enumerate}\item[]
(i) repayment of capital on any loan used for the replacement, in the course of business, of equipment or machinery, or the repair of an existing business asset except to the extent that any sum is payable under an insurance policy for its repair;

(ii) any income expended in the repair of an existing business asset except to the extent that any sum is payable under an insurance policy for its repair;

(iii) any payment of interest on a loan taken out for the purposes of the business;
\end{enumerate}

($b$) such expenses do not include—
\begin{enumerate}\item[]
(i) repayment of capital on any other loan taken out for the purposes of the business;

(ii) any capital expenditure;

(iii) the depreciation of any capital asset;

(iv) any sum employed, or intended to be employed, in the setting up or expansion of the business;

(v) any loss incurred before the beginning of the period by reference to which earnings are determined;

(vi) any expenses incurred in providing business entertainment;

(vii) any loss incurred in any other employment in which he is engaged as a self-employed earner.
\end{enumerate}
\end{enumerate}

%(5) For the purposes of sub-paragraph (3)($c$), the amount of income tax to be allowed against earnings shall be calculated 
%on the basis of chargeable earnings and % Words inserted (5.4.93) by SI 1993/913 reg 27(2)
%as if those earnings, less any personal allowance applicable to the earner under Chapter I of Part VII of the Income and Corporation Taxes Act 1988 (Personal Relief) (or where the earnings are determined over a period of less than a year, a proportionate part of such relief), were assessable to income tax at the rates of tax applicable at the effective date.

%% Para 3(5) substituted (18.4.95) by SI 1995/1045 reg 54(6)
%(5) For the purposes of sub-paragraph (3)($c$), the amount of income tax to be allowed against earnings shall be calculated—
%\begin{enumerate}\item[]
%($a$) where the earnings are determined over a period of 12 months on the basis of chargeable earnings and as if those earnings, less any personal allowance applicable to the earner under Chapter I of Part VII of the Income and Corporation Taxes Act 1988 (Personal Relief), were assessable to income tax at the rates of tax applicable at the effective date; or
%
%($b$) where the earnings are determined over a period of less than 12 months on the basis of chargeable earnings and as if those earnings, less a proportionate part of any personal allowance applicable to the earner under Chapter I of Part VII of the Income and Corporation Taxes Act 1988 (Personal Relief), were assessable to income tax at the rates of tax applicable at the effective date, but the amount of the earnings to which each tax rate applies shall be determined on the basis that the ratio of that amount to the full amount of chargeable earnings to which each tax rate applies is the same as the ratio of the proportionate part of the personal allowance to the full personal allowance.
%\end{enumerate}

% Para 3(5) substituted (13.1.97) by SI 1996/3196 reg 13(4)
(5) For the purposes of sub-paragraph (3)($c$), the amount in respect of income tax shall be determined in accordance with the following provisions—
\begin{enumerate}\item[]
($a$) subject to head ($c$), an amount of chargeable earnings equivalent to any personal allowance applicable to the earner by virtue of the provisions of Chapter I of Part VII of the Income and Corporation Taxes Act 1988 (personal reliefs) shall be disregarded;

($b$) an amount equivalent to income tax shall be calculated with respect to taxable earnings at the rates applicable at the effective date;

($c$) the amount to be disregarded by virtue of head ($a$) shall be calculated by reference to the yearly rate applicable at the effective date, that amount being reduced or increased in the same proportion to that which the period represented by the chargeable earnings bears to the period of one year;

($d$) in this sub-paragraph, “taxable earnings” means the chargeable earnings of the earner following the disregard of any applicable personal allowances.”.
\end{enumerate}

(6) For the purposes of sub-paragraph (3)($d$), the amount to be deducted in respect of National Insurance contributions shall be the total of—
\begin{enumerate}\item[]
($a$) the amount of Class 2 contributions (if any) payable under section 11(1) or, as the case may be, 
%(4) 
(3)  % Word substituted (18.4.95) by SI 1995/1045 reg 54(7)
of the Contributions and Benefits Act; and

($b$) the amount of Class 4 contributions (if any) payable under section 15(2) of that Act,
\end{enumerate}
at the rates applicable 
to the chargeable earnings % Words inserted (5.4.93) by SI 1993/913 reg 27(3)
at the effective date.

% Para 3(7), (8) inserted (5.4.93) by SI 1993/913 reg 27(4)
%(7) In the case of a self-employed earner whose employment is carried on in partnership or is that of a share fisherman within the meaning of the Social Security (Mariners' Benefits) Regulations 1975\footnote{\frenchspacing  S.I. 1975/470.}, sub-paragraph (3) shall have effect as though it requires a deduction from the earner’s gross receipts of an amount calculated by---
%\begin{enumerate}\item[]
%($a$) deducting from the gross receipts of the partnership or fishing boat the sums mentioned in heads ($a$) and ($b$) of that sub-paragraph; and
%
%($b$) deducting from the earner’s share of the balance after such deductions the sums mentioned in heads ($c$) to ($e$) of that sub-paragraph;
%\end{enumerate}

% Para 3(7) substituted (18.4.95) by SI 1995/1045 reg 54(8)
(7) In the case of a self-employed earner whose employment is carried on in partnership or is that of a share fisherman within the meaning of the Social Security (Mariners' Benefits) Regulations 1975\footnote{\frenchspacing S.I. 1975/470.}, sub-paragraph (3) shall have effect as though it requires—
\begin{enumerate}\item[]
($a$) a deduction from the earner’s estimated or, where appropriate, actual share of the gross receipts of the partnership or fishing boat, of his share of the sums likely to be deducted or, where appropriate, deducted from those gross receipts under heads ($a$) and ($b$) of that sub-paragraph; and

($b$) a deduction from the amount so calculated of the sums mentioned in heads ($c$) to ($e$) of that sub-paragraph.
\end{enumerate}

(8) In sub-paragraphs (5) and (6) “chargeable earnings” means the gross receipts of the employment less any deductions mentioned in sub-paragraph (3)($a$) and ($b$).

\amendment{
Words inserted in para. 3(3), %(5), 
(6) and para. 3(7), (8) inserted (5.4.93) by the Child Support (Miscellaneous Amendments) Regulations 1993 reg. 27.

Words inserted in para. 3(3)($b$), ($e$), words substituted in para. 3(6) and para. 3%(5), 
(7) substituted (18.4.95) by the Child Support and Income Support (Amendment) Regulations 1995 reg. 54(5)--(8).

Para. 3(5) substituted (13.1.97) by the Child Support (Miscellaneous Amendments) (No. 2) Regulations 1996 reg. 13(4).

Under the Child Support (Miscellaneous Amendments) (No. 2) Regulations 1996 reg. 16, a maintenance assessment in force on 13.1.97 shall not be reviewed solely to give effect to para. 3(5) as substituted by those regulations, but para. 3(5) as substituted shall apply in conducting a review of the assessment under s. 16, 17, 18 or 19 of the Act, and the effective date of any fresh assessment affected by para. 3(5) as substituted shall not be earlier than the first day of the first maintenance period which commences on or after 13.1.97.
}

\medskip

4.  In a case where a person is self-employed as a childminder the amount of earnings referable to that employment shall be one-third of the gross receipts.

\medskip

5.—(1) Subject to sub-paragraphs 
%(2) and (3)—
(2) to (3)---  % Words substituted (18.4.95) by SI 1995/1045 reg 54(9)($a$)
\begin{enumerate}\item[]
($a$) where a person has been a self-employed earner for 52 weeks or more including the relevant week, the amount of his earnings shall be determined by reference to the average of the earnings which he has received in the 52 weeks ending with the relevant week;

($b$) where the person has been a self-employed earner for a period of less than 52 weeks including the relevant week, the amount of his earnings shall be determined by reference to the average of the earnings which he has received during that period.
\end{enumerate}

(2) 
%Where 
Subject to sub-paragraph (2A), where  % Words substituted (18.4.95) by SI 1995/1045 reg 54(9)($b$)(i)
a person who is a self-\hspace{0pt}employed earner provides in respect of the employment a profit and loss account and, where appropriate, a trading account or a balance sheet or both, and the profit and loss account is in respect of a period at least 6 months but not exceeding 15 months and that period terminates within the 
%12 months 
24 months  % Words substituted (18.4.95) by SI 1995/1045 reg 54(9)($b$)(ii)
immediately preceding the effective date, the amount of his earnings shall be determined by reference to the average of the earnings over the period to which the profit and loss account relates and such earnings shall include receipts relevant to that period (whether or not received in that period).

% Para 5(2A) inserted (18.4.95) by SI 1995/1045 reg 54(9)($c$)
(2A) Where the child support officer is satisfied that, in relation to the person referred to in sub-paragraph (2) there is more than one profit and loss account, each in respect of different periods, both or all of which satisfy the conditions mentioned in that sub-paragraph, the provisions of that sub-paragraph shall apply only to the account which relates to the latest such period, unless the officer is satisfied that the latest such account is not available for reasons beyond the control of that person, in which case he may have regard to any such other account which satisfies the requirements of that sub-paragraph.

(3) Where a calculation would, but for this sub-paragraph, produce an amount which, in the opinion of the child support officer, does not accurately reflect the normal amount of the earnings of the person in question, such earnings, or any part of them, shall be calculated by reference to such other period as may, in the particular case, enable the normal weekly earnings of that person to be determined more accurately and for this purpose the child support officer shall have regard to—
\begin{enumerate}\item[]
($a$) the earnings received, or due to be received, from any employment in which the person in question is engaged, or has been engaged or is due to be engaged;

($b$) the duration and pattern, or the expected duration and pattern, of any employment of that person.
\end{enumerate}

(4) In sub-paragraph (2)—
\begin{enumerate}\item[]
($a$) “balance sheet” means a statement of the financial position of the employment disclosing its assets, liabilities and capital at the end of the period in question;

($b$) “profit and loss account” means a financial statement showing net profit or loss of the employment for the period in question; and

($c$) “trading account” means a financial statement showing the revenue from sales, the cost of those sales and the gross profit arising during the period in question.
\end{enumerate}

% Para 5(5) inserted (13.1.97) by SI 1996/3196 reg 13(5)
(5) Subject to sub-paragraph (3), where a person has claimed, or has been paid, family credit or disability working allowance on any day during the period beginning not earlier than eight weeks before the relevant week and ending not later than the date on which the assessment is made, the child support officer may have regard to the amount of earnings taken into account in determining entitlement to those benefits in order to calculate or estimate the amount of earnings to be taken into account for the purposes of calculating N and M, notwithstanding the fact that entitlement to those benefits may have been determined by reference to earnings attributable to a period other than that specified in sub-paragraph (1).

\amendment{
Words inserted in para. 5(2), words substituted in para. 5(1), (2) and para. 5(2A) inserted (18.4.95) by the Child Support and Income Support (Amendment) Regulations 1995 reg. 54(9).

Para. 5(5) inserted (13.1.97) by the Child Support (Miscellaneous Amendments) (No. 2) Regulations 1996 reg. 13(5).

Under the Child Support (Miscellaneous Amendments) (No. 2) Regulations 1996 reg. 16, a maintenance assessment in force on 13.1.97 shall not be reviewed solely to give effect to para. 5(5), but para. 5(5) shall apply in conducting a review of the assessment under s. 16, 17, 18 or 19 of the Act, and the effective date of any fresh assessment affected by para. 5(5) shall not be earlier than the first day of the first maintenance period which commences on or after 13.1.97.
}

\section[Part II --- Benefit payments]{Part II\\*Benefit payments}

\renewcommand\parthead{--- Schedule 1 Part II}

6.—(1) The benefit payments to be taken into account in calculating or estimating N and M shall be determined in accordance with this Part.

(2) “Benefit payments” means any benefit payments under the Contributions and Benefits Act 
or the Jobseekers Act  % Words inserted (7.10.96) by SI 1996/1345 reg 6(6), (7)(b)
except amounts to be disregarded by virtue of Schedule 2.

(3) The amount of any benefit payment to be taken into account shall be determined by reference to the rate of that benefit applicable at the effective date.

\amendment{
Words inserted in para. 6(2) (7.10.96) by the Social Security and Child Support (Jobseeker's Allowance) (Consequential Amendments) Regulations 1996 reg. 6(6), (7)($b$).
}

\medskip

7.—(1) Where a benefit payment under the Contributions and Benefits Act includes an adult or child dependency increase—
\begin{enumerate}\item[]
($a$) if that benefit is payable to a parent, the income of that parent shall be calculated or estimated as if it did not include that amount;

($b$) if that benefit is payable to some other person but includes an amount in respect of the parent, the income of the parent shall be calculated or estimated as if it included that amount.
\end{enumerate}

% Para 7(1A) inserted (7.10.96) by SI 1996/1345 reg 6(4)(a)
\begin{sloppypar}
(1A) For the purposes of sub-paragraph (1), an addition to a contribution\hspace{0pt}-\hspace{0pt}based jobseeker’s allowance under regulation 9(4) of the Jobseeker’s Allowance (Transitional Provisions) Regulations 1995\footnote{\frenchspacing S.I. 1995/3276.} shall be treated as a dependency increase included with a benefit under the Contributions and Benefits Act.
\end{sloppypar}

(2) Subject to sub-paragraph (3), payments to a person by way of family credit shall be treated as the income of the parent who has qualified for them by his engagement in, and normal engagement in, remunerative work.

(3) Subject to sub-paragraphs (4) and (5), where family credit is payable and the amount which is payable has been calculated by reference either to the 
normal  % Word inserted (7.10.96) by SI 1996/1945 reg 24(3)
weekly earnings of the absent parent and another person or the parent with care and another person—
\begin{enumerate}\item[]
($a$) if during the period which is used to calculate his earnings under paragraph 2 or, as the case may be, paragraph 5, the weekly earnings of that parent exceed those of the other person, the amount payable by way of family credit shall be treated as the income of that parent;

($b$) if during that period the normal weekly earnings of that parent equal those of the other person, half of the amount payable by way of family credit shall be treated as the income of that parent; and

($c$) if during that period the normal weekly earnings of that parent are less than those of that other person, the amount payable by way of family credit shall not be treated as the income of that parent.
\end{enumerate}

(4) Where—
\begin{enumerate}\item[]
($a$) family credit (calculated, as the case may be, by reference to the weekly earnings of the absent parent and another person or the parent with care and another person) is in payment; and

($b$) not later than the effective date either or both the persons by reference to whose engagement and normal engagement in remunerative work that payment has been calculated has ceased to be so employed,
\end{enumerate}
half of the amount payable by way of family credit shall be treated as the income of the parent in question.

(5) Where—
\begin{enumerate}\item[]
($a$) family credit is in payment; and

($b$) not later than the effective date the person or, if more than one, each of the persons by reference to whose engagement, and normal engagement, in remunerative work that payment has been calculated is no longer the partner of the person to whom that payment is made,
\end{enumerate}
the payment in question shall only be treated as the income of the parent in question where he is in receipt of it.

%Para 7(6) added (7.4.97) by SI 1996/1803 reg 17(2)
(6) Where child benefit in respect of a relevant child is in payment at the rate specified in regulation 2(1)($a$)(ii) of the Child Benefit Rates Regulations, the difference between that rate and the basic rate applicable to that child, as defined in regulation 4.

\amendment{
Para. 7(1A) inserted (7.10.96) by the Social Security and Child Support (Jobseeker's Allowance) (Consequential Amendments) Regulations 1996 reg. 6(4)($a$).

Word inserted in para. 7(3)($a$) (7.10.96) by the Child Support (Miscellaneous Amendments) Regulations 1996 reg. 24(3).

Para. 7(6) added (7.4.97) by the Child Benefit, Child Support and Social Security (Miscellaneous Amendments) Regulations 1996 reg. 17(2) (subject to transitional provisions in reg. 49).

}

\section[Part III --- Other income]{Part III\\*Other income}

\renewcommand\parthead{--- Schedule 1 Part III}

8.  The amount of the other income to be taken into account in calculating or estimating N and M shall be the aggregate of the following amounts determined in accordance with this Part.

\medskip

9.  Any periodic payment of pension or other benefit under an occupational or personal pension scheme or a retirement annuity contract or other such scheme for the provision of income in retirement.

\medskip

10.  Any payment received on account of the provision of board and lodging which does not come within Part I of this Schedule.

\medskip

11.  Subject to regulation 7(3)($b$) and paragraph 12, any payment to a student of—
\begin{enumerate}\item[]
($a$) grant;

($b$) an amount in respect of grant contribution;

($c$) covenant income except to the extent that it has been taken into account under sub-paragraph ($b$);

($d$) a student loan.
\end{enumerate}

\medskip

12.  The income of a student shall not include any payment—
\begin{enumerate}\item[]
($a$) intended to meet tuition fees or examination fees;

($b$) intended to meet additional expenditure incurred by a disabled student in respect of his attendance on a course;

($c$) intended to meet additional expenditure connected with term time residential study away from the student’s educational establishment;

($d$) on account of the student maintaining a home at a place other than that at which he resides during his course;

($e$) intended to meet the cost of books, and equipment (other than special equipment) or, if not so intended, an amount equal to the amount allowed under regulation 38(2)($f$) of the Family Credit (General) Regulations 1987\footnote{\frenchspacing S.I. 1987/1973; the relevant amending instrument is S.I. 1991/1520. At the date of making these Regulations the amount was £257 per annum.} towards such costs;

($f$) intended to meet travel expenses incurred as a result of his attendance on the course.
\end{enumerate}

\medskip

13.  Any interest, dividend or other income derived from capital.

\medskip

14.  Any maintenance payments in respect of a parent.

\medskip

% Para 14A inserted (7.10.96) by SI 1996/1945 reg 24(4)
14A.—(1) Subject to sub-paragraph (2), the amount of any earnings top-up paid to or in respect of the absent parent or the parent with care.

(2) Subject to sub-paragraphs (3) and (4), where earnings top-up is payable and the amount which is payable has been calculated by reference to the weekly earnings of either the absent parent and another person or the parent with care and another person—
\begin{enumerate}\item[]
($a$) if during the period which is used to calculate his earnings under paragraph 2 or, as the case may be, paragraph 5, the normal weekly earnings of that parent exceed those of the other person, the amount payable by way of earnings top-up shall be treated as the income of that parent;

($b$) if during that period, the normal weekly earnings of that parent equal those of the other person, half of the amount payable by way of earnings top-up shall be treated as the income of that parent;

($c$) if during that period, the normal weekly earnings of that parent are less than those of that other person, the amount payable by way of earnings top-up shall not be treated as the income of that parent.
\end{enumerate}

(3) Where any earnings top-up is in payment and, not later than the effective date, the person, or, if more than one, each of the persons by reference to whose engagement and normal engagement in remunerative work that payment has been calculated is no longer the partner of the person to whom that payment is made, the payment in question shall be treated as the income of the parent in question only where that parent is in receipt of it.

(4) Where earnings top-up is in payment and, not later than the effective date, either or both of the persons by reference to whose engagement and normal engagement in remunerative work that payment has been calculated has ceased to be employed, half of the amount payable by way of earnings top-up shall be treated as the income of the parent in question.

\amendment{
Para. 14A inserted (7.10.96) by the Child Support (Miscellaneous Amendments) Regulations 1996 reg. 24(4).
}


\medskip

15.  Any other payments or other amounts received on a periodical basis which are not otherwise taken into account under Part I, II, IV or V of this Schedule
except payments or other amounts which are excluded from the definition of “earnings” by virtue of paragraph 1(2).  % Words inserted (7.10.96) by SI 1996/1945 reg 24(5)

\amendment{
Words inserted in para. 15 (7.10.96) by the Child Support (Miscellaneous Amendments) Regulations 1996 reg. 24(5).
}

\medskip

16.—(1) Subject to sub-paragraphs (2) to (6) the amount of any income to which this Part applies shall be calculated or estimated—
\begin{enumerate}\item[]
($a$) where it has been received in respect of the whole of the period of 26 weeks which ends at the end of the relevant week, by dividing such income received in that period by 26;

($b$) where it has been received in respect of part of the period of 26 weeks which ends at the end of the relevant week, by dividing such income received in that period by the number of complete weeks in respect of which such income is received and for this purpose income shall be treated as received in respect of a week if it is received in respect of any day in the week in question.
\end{enumerate}

(2) The amount of maintenance payments made in respect of a parent—
\begin{enumerate}\item[]
($a$) where they are payable weekly and have been paid at the same amount in respect of each week in the period of 13 weeks which ends at the end of the relevant week, shall be the amount equal to one of those payments;

($b$) in any other case, shall be the amount calculated by aggregating the total amount of those payments received in the period of 13 weeks which ends at the end of the relevant week and dividing by the number of weeks in that period in respect of which maintenance was due.
\end{enumerate}

(3) In the case of a student—
\begin{enumerate}\item[]
($a$) the amount of any grant and any amount paid in respect of grant contribution shall be calculated by apportioning it equally between the weeks in respect of which it is payable;

($b$) the amount of any covenant income shall be calculated by dividing the amount payable in respect of a year by 52 (or, where such amount is payable in respect of a lesser period, by the number of complete weeks in that period) and, subject to sub-paragraph (4), deducting £5·00;

($c$) the amount of any student loan shall be calculated by apportioning the loan equally between the weeks in respect of which it is payable and, subject to sub-paragraph (4), deducting £10·00.
\end{enumerate}

(4) For the purposes of sub-paragraph (3)—
\begin{enumerate}\item[]
($a$) not more than £500 shall be deducted under sub-paragraph (3)($b$);

($b$) not more than £1000 in total shall be deducted under sub-\hspace{0pt}paragraphs (3)($b$) and ($c$).
\end{enumerate}

(5) Where in respect of the period of 52 weeks which ends at the end of the relevant week a person is in receipt of interest, dividend or other income which has been produced by his capital, the amount of that income shall be calculated by dividing the aggregate of the income so received by 52.

(6) Where a calculation would, but for this sub-paragraph, produce an amount which, in the opinion of the child support officer, does not accurately reflect the normal amount of the other income of the person in question, such income, or any part of it, shall be calculated by reference to such other period as may, in the particular case, enable the other income of that person to be determined more accurately and for this purpose the child support officer shall have regard to the nature and pattern of receipt of such income.

\section[Part IV --- Income of child treated as income of parent]{Part IV\\*Income of child treated as income of parent}

\renewcommand\parthead{--- Schedule 1 Part IV}

17.  The amount of any income of a child which is to be treated as the income of the parent in calculating or estimating N and M shall be the aggregate of the amounts determined in accordance with this Part.

\medskip

18.  Where a child has income which falls within the following paragraphs of this Part and that child is a member of the family of his parent (whether that child is a qualifying child in relation to that parent or not), the relevant income of that child shall be treated as that of his parent.

\medskip

19.  Where child support maintenance is being assessed for the support of only one qualifying child, the relevant income of that child shall be treated as that of the parent with care.

\medskip

20.  Where child support maintenance is being assessed to support more than one qualifying child, the relevant income of each of those children shall be treated as that of the parent with care to the extent that it does not exceed the aggregate of—
\begin{enumerate}\item[]
($a$) the amount determined under—
\begin{enumerate}\item[]
(i) regulation 3(1)($a$) (calculation of AG) in relation to the child in question; and

(ii) the total of any other amounts determined under regulation 3(1)($b$) 
%to ($d$) 
and ($c$)  % Words substituted (7.4.97) by SI 1996/1803 reg 17(3)(a)
which are applicable in the case in question divided by the number of children for whom child support maintenance is being calculated,
\end{enumerate}
less the basic rate of child benefit (within the meaning of regulation 4) for the child in question; and

($b$) 
%three times 
one-and-a-half times  % Words substituted (18.4.95) by SI 1995/1045 reg 54(10)
the total of the amounts calculated under regulation 3(1)($a$) (income support personal allowance for child or young person) in respect of that child and regulation 
%3(1)($c$) 
3(1)($c$)(i)  % Word substituted (7.4.97) by SI 1996/1803 reg 17(3)(b)
(income support family premium).
\end{enumerate}

\amendment{
Words substituted in para. 20($b$) (18.4.95) by the Child Support and Income Support (Amendment) Regulations 1995 reg. 54(10).

Words substituted in para. 20($a$)(ii), ($b$) (7.4.97) by the Child Benefit, Child Support and Social Security (Miscellaneous Amendments) Regulations 1996 reg. 17(3) (subject to transitional provisions in reg. 49).
}

\medskip

21.  Where child support maintenance is not being assessed for the support of the child whose income is being calculated or estimated, the relevant income of that child shall be treated as that of his parent to the extent that it does not exceed the amount determined under regulation 9(1)($g$).

\medskip

22.%
---(1)  % Para 22 renumbered as para 22(1) (7.10.96) by SI 1996/1456 reg 6(4)(b)
  Where a benefit under the Contributions and Benefits Act includes an adult or child dependency increase in respect of a relevant child, the relevant income of that child shall be calculated or estimated as if it included that amount.

(1A) For the purposes of sub-paragraph (1), an addition to a contribution\hspace{0pt}-based jobseeker’s allowance under regulation 9(4) of the Jobseeker’s Allowance (Transitional Provisions) Regulations 1995 shall be treated as a dependency increase included with a benefit under the Contributions and Benefits Act.

\amendment{
Para. 22 renumbered as para. 22(1) and para. 22(1A) inserted (7.10.96) by the Social Security and Child Support (Jobseeker's Allowance) (Consequential Amendments) Regulations 1996 reg. 6(4)($b$).
}

\medskip

23.  For the purposes of this Part, “the relevant income of a child” does not include—
\begin{enumerate}\item[]
($a$) any earnings of the child in question;

($b$) payments by an absent parent in respect of the child for whom maintenance is being assessed;

($c$) where the class of persons who are capable of benefiting from a discretionary trust include the child in question, payments from that trust except in so far as they are made to provide for food, ordinary clothing and footwear, gas, electricity or fuel charges or housing costs; or

($d$) any interest payable on arrears of child support maintenance for that child;

% Para 23($e$) inserted (18.4.95) by SI 1995/1045 reg 54(11)
($e$) the first £10 of any other income of that child.
\end{enumerate}

\amendment{
Para. 23($e$) inserted (18.4.95) by the Child Support and Income Support (Amendment) Regulations 1995 reg. 54(11).
}

\medskip

24.  The amount of the income of a child which is treated as the income of the parent shall be determined in the same way as if such income were the income of the parent.

\section[Part V --- Amounts treated as the income of a parent]{Part V\\*Amounts treated as the income of a parent}

\renewcommand\parthead{--- Schedule 1 Part V}

25.  The amounts which fall to be treated as income of the parent in calculating or estimating N and M shall include amounts to be determined in accordance with this Part.

\medskip

26.  Where a child support officer is satisfied—
\begin{enumerate}\item[]
($a$) that a person has performed a service either—
\begin{enumerate}\item[]
(i) without receiving any remuneration in respect of it; or

(ii) for remuneration which is less than that normally paid for that service;
\end{enumerate}

($b$) that the service in question was for the benefit of—
\begin{enumerate}\item[]
(i) another person who is not a member of the same family as the person in question; or

(ii) a body which is neither a charity nor a voluntary organisation;
\end{enumerate}

($c$) that the service in question was performed for a person who, or as the case may be, a body which was able to pay remuneration at the normal rate for the service in question;

($d$) that the principal purpose of the person undertaking the service without receiving any or adequate remuneration is to reduce his assessable income for the purposes of the Act; and

($e$) that any remuneration foregone would have fallen to be taken into account as earnings,
\end{enumerate}
the value of the remuneration foregone shall be estimated by a child support officer and an amount equal to the value so estimated shall be treated as income of the person who performed those services.

\medskip

27.  Subject to paragraphs 28 to 30, where the child support officer is satisified that, otherwise than in the circumstances set out in paragraph 26, a person has intentionally deprived himself of—
\begin{enumerate}\item[]
($a$) any income or capital which would otherwise be a source of income;

($b$) any income or capital which it would be reasonable to expect would be secured by him,
\end{enumerate}
with a view to reducing the amount of his assessable income, his net income shall include the amount estimated by a child support officer as representing the income which that person would have had if he had not deprived himself of or failed to secure that income, or as the case may be, that capital.

\medskip

28.  No amount shall be treated as income by virtue of paragraph 27 in relation to—
\begin{enumerate}\item[]
%($a$) one parent benefit;

% Para 28(a) substituted (7.4.97) by SI 1996/1803 reg 17(4)
($a$) if the parent satisfies the conditions for payment of the rate of child benefit specified in regulation 2(1)($a$)(ii) of the Child Benefit Rates Regulations, an amount representing the difference between that rate and the basic rate, as defined in regulation 4;

($b$) if the parent is a person to, or in respect of, whom income support is payable, %unemployment benefit;
a contribution-based jobseeker’s allowance;  % Words substituted (7.10.96) by SI 1996/1345 reg 6(4)(c)

($c$) a payment from a discretionary trust or a trust derived from a payment made in consequence of a personal injury.
\end{enumerate}

\amendment{
Words substituted in para. 28($b$) (7.10.96) by the Social Security and Child Support (Jobseeker's Allowance) (Consequential Amendments) Regulations 1996 reg. 6(4)($c$).

Para. 28($a$) substituted (7.4.97) by the Child Benefit, Child Support and Social Security (Miscellaneous Amendments) Regulations 1996 reg. 17(4) (subject to transitional provisions in reg. 49).
}

\medskip

29.  Where an amount is included in the income of a person under paragraph 27 in respect of income which would become available to him on application, the amount included under that paragraph shall be included from the date on which it could be expected to be acquired.

\medskip

30.  Where a child support officer determines under paragraph 27 that a person has deprived himself of capital which would otherwise be a source of income, the amount of that capital shall be reduced at intervals of 52 weeks, starting with the week which falls 52 weeks after the first week in respect of which income from it is included in the calculation of the assessment in question, by an amount equal to the amount which the child support officer estimates would represent the income from that source in the immediately preceding period of 52 weeks.

\medskip

31.  Where a payment is made on behalf of a parent or a relevant child in respect of food, ordinary clothing or footwear, gas, electricity or fuel charges, housing costs or council tax, an amount equal to the amount which the child support officer estimates represents the value of that payment shall be treated as the income of the parent in question except to the extent that such amount is—
\begin{enumerate}\item[]
($a$) disregarded under paragraph 38 of Schedule 2;

($b$) a payment of school fees paid by or on behalf of someone other than the absent parent.
\end{enumerate}

\medskip

32.  Where paragraph 26 applies the amount to be treated as the income of the parent shall be determined as if it were earnings from employment as an employed earner and in a case to which paragraph 27 or 31 applies the amount shall be determined as if it were other income to which Part III of this Schedule applies.

\part[Schedule 2 --- Amounts to be disregarded when calculating or estimating N and M]{Schedule 2\\*Amounts to be disregarded when calculating or estimating N and M}

\renewcommand\parthead{--- Schedule 2}

1.  The amounts referred to in this Schedule are to be disregarded when calculating or estimating N and M (parent’s net income).

\medskip

2.  An amount in respect of income tax applicable to the income in question where not otherwise allowed for under these Regulations.

\medskip

3.  Where a payment is made in a currency other than sterling, an amount equal to any banking charge or commission payable in converting that payment to sterling.

\medskip

4.  Any amount payable in a country outside the United Kingdom where there is a prohibition against the transfer to the United Kingdom of that amount.

\medskip

5.  Any compensation for personal injury and any payments from a trust fund set up for that purpose.

\medskip

6.  Any advance of earnings or any loan made by an employer to an employee.

\medskip

7.  Any payment by way of, or any reduction or discharge of liability resulting from entitlement to, housing benefit or council tax benefit.

\medskip

8.  Any disability living allowance, mobility supplement or any payment intended to compensate for the non-payment of any such allowance or supplement.

\medskip

9.  Any payment which is—
\begin{enumerate}\item[]
($a$) an attendance allowance under section 64 of the Contributions and Benefits Act;

($b$) an increase of disablement pension under section 104 or 105 of that Act (increases where constant attendance needed or for exceptionally severe disablement);

($c$) a payment made under regulations made in exercise of the power conferred by Schedule 8 to that Act (payments for pre-1948 cases);

($d$) an increase of an allowance payable in respect of constant attendance under that Schedule;

($e$) payable by virtue of articles 14, 15, 16, 43 or 44 of the Personal Injuries (Civilians) Scheme 1983\footnote{\frenchspacing S.I. 1983/686.} (allowances for constant attendance and exceptionally severe disablement and severe disablement occupational allowance) or any analogous payment; or

($f$) a payment based on the need for attendance which is paid as part of a war disablement pension.
\end{enumerate}

\medskip

10.  Any payment under section 148 of the Contributions and Benefits Act (pensioners' Christmas bonus).

\medskip

11.  Any social fund payment within the meaning of Part VIII of the Contributions and Benefits Act.

\medskip

12.  Any payment made by the Secretary of State to compensate for the loss (in whole or in part) of entitlement to housing benefit.

\medskip

13.  Any payment made by the Secretary of State to compensate for loss of housing benefit supplement under regulation 19 of the Supplementary Benefit (Requirements) Regulations 1983\footnote{\frenchspacing S.I. 1983/1399.}.

\medskip

14.  Any payment made by the Secretary of State to compensate a person who was entitled to supplementary benefit in respect of a period ending immediately before 11th April 1988 but who did not become entitled to income support in respect of a period beginning with that day.

\medskip

15.  Any concessionary payment made to compensate for the non-payment of income support, 
income-based jobseeker’s allowance,  % Words inserted (7.10.96) by SI 1996/1345 reg 6(5)
disability living allowance, or any payment to which paragraph 9 applies.

\amendment{
Words inserted in para. 15 (7.10.96) by the Social Security and Child Support (Jobseeker's Allowance) (Consequential Amendments) Regulations 1996 reg. 6(5).
}

\medskip

16.  Any payments of child benefit to the extent that they do not exceed the basic rate of that benefit as defined in regulation 4.

\medskip

17.  Any payment made under regulations 9 to 11 or 13 of the Welfare Food Regulations 1988\footnote{\frenchspacing S.I. 1988/536; the relevant amending instrument is S.I. 1990/3.} (payments made in place of milk tokens or the supply of vitamins).

\medskip

18.  Subject to paragraph 20 and to the extent that it does not exceed £10·00—
\begin{enumerate}\item[]
($a$) war disablement pension or war widow’s pension or a payment made to compensate for non-payment of such a pension;

($b$) a pension paid by the government of a country outside Great Britain and which either—
\begin{enumerate}\item[]
(i) is analogous to a war disablement pension; or

(ii) is analogous to a war widow’s pension.
\end{enumerate}
\end{enumerate}

\medskip

19.—(1) Except where sub-paragraph (2) applies and subject to sub-\hspace{0pt}paragraph (3) and paragraphs 20, 38 and 47, 
%£10.00 
up to £20.00  % Word substituted (8.4.96) by SI 1996/481 reg 3(2)
of any charitable or voluntary payment made, or due to be made, at regular intervals.

(2) Subject to sub-paragraph (3) and paragraphs 38 and 47, any charitable or voluntary payment made or due to be made at regular intervals which is intended and used for an item other than food, ordinary clothing or footwear, gas, electricity or fuel charges, housing costs of any member of the family or the payment of council tax.

(3) Sub-paragraphs (1) and (2) shall not apply to a payment which is made by a person for the maintenance of any member of his family or of his former partner or of his children.

(4) For the purposes of sub-paragraph (1) where a number of charitable or voluntary payments fall to be taken into account they shall be treated as though they were one such payment.

\amendment{
Word substituted in para. 19(1) (8.4.96) by the Child Support (Maintenance Assessments and Special Cases) and Social Security (Claims and Payments) Amendment Regulations 1996 reg. 3(2) (subject to transitional provisions in reg. 4).
}

\medskip

20.—(1) Where, but for this paragraph, more than 
%£10·00 
£20.00  % Word substituted (18.4.96) by SI 1996/481 reg 3(3)
would be disregarded under paragraphs 18 and 19(1) in respect of the same week, only 
%£10·00 
£20.00  % Word substituted (18.4.96) by SI 1996/481 reg 3(3)
in aggregate shall be disregarded and where an amount falls to be deducted from the income of a student under paragraph 16(3)($b$) or ($c$) of Schedule 1, that amount shall count as part of the 
%£10·00 
£20.00  % Word substituted (18.4.96) by SI 1996/481 reg 3(3)
disregard allowed under this paragraph.

(2) Where any payment which is due to be paid in one week is paid in another week, sub-paragraph (1) and paragraphs 18 and 19(1) shall have effect as if that payment were received in the week in which it was due.

\amendment{
Words substituted in para. 20(1) (8.4.96) by the Child Support (Maintenance Assessments and Special Cases) and Social Security (Claims and Payments) Amendment Regulations 1996 reg. 3(3)  (subject to transitional provisions in reg. 4).
}

\medskip

21.  In the case of a person participating in arrangements for training made under section 2 of the Employment and Training Act 1973\footnote{\frenchspacing 1973 c. 50; section 2 was substituted by section 25(1) of the Employment Act 1988 (c. 19).} or section 2 of the Enterprise and New Towns (Scotland) Act 1990\footnote{\frenchspacing 1990 c. 39.} (functions in relation to training for employment etc.)\ or attending a course at an employment rehabilitation centre established under section 2 of the 1973 Act—
\begin{enumerate}\item[]
($a$) any travelling expenses reimbursed to the person;

($b$) any living away from home allowance under section 2(2)($d$) of the 1973 Act or section 2(4)($c$) of the 1990 Act;

($c$) any training premium,
\end{enumerate}
but this paragraph, except in so far as it relates to a payment mentioned in sub-paragraph ($a$), ($b$) or ($c$), does not apply to any part of any allowance under section 2(2)($d$) of the 1973 Act or section 2(4)($c$) of the 1990 Act.

\medskip

22.  Where a parent occupies a dwelling as his home and that dwelling is also occupied by a person, other than a non-dependant or a person who is provided with board and lodging accommodation, and that person is contractually liable to make payments in respect of his occupation of the dwelling to the parent, the amount or, as the case may be, the amounts specified in paragraph 19 of Schedule 2 to the Family Credit (General) Regulations 1987\footnote{\frenchspacing S.I. 1987/1973; the relevant amending instrument is S.I. 1991/503.} which apply in his case, or, if he is not in receipt of family credit, the amounts which would have applied if he had been in receipt of that benefit.

\medskip

23.  Where a parent, who is not a self-employed earner, is in receipt of rent or any other money in respect of the use and occupation of property other than his home, that rent or other payment to the extent of any sums which that parent is liable to pay by way of—
\begin{enumerate}\item[]
%($a$) payments which would be treated as housing costs by paragraph 3 of Schedule 3 if that property were his home (exempt income: additional provisions relating to housing costs);
%Para 23($a$) substituted (5.4.93) by SI 1993/913 reg 28
($a$) payments which are to be taken into account as eligible housing costs under sub-paragraphs ($b$), ($c$), ($d$) and ($t$) of paragraph 1 of Schedule 3 (eligible housing costs for the purposes of determining exempt income and protected income) and paragraph 3 of that Schedule (exempt income: additional provisions relating to eligible housing costs);

($b$) council tax payable in respect of that property;

($c$) water and sewerage charges payable in respect of that property.
\end{enumerate}

\amendment{
Para. 23($a$) substituted (5.4.93) by the Child Support (Miscellaneous Amendments) Regulations 1993 reg. 28.
}

\medskip

24.  
%Where a parent provides 
For each week in which a parent provides  % Words substituted (18.4.95) by SI 1995/1045 reg 55(2)
board and lodging accommodation in his home otherwise than as a self-employed earner—
\begin{enumerate}\item[]
($a$) £20.00 of any payment for that accommodation made by%
, on behalf or in respect of  % Words inserted (18.4.95) by SI 1995/1045 reg 55(3)
the person to whom that accommodation is provided; and

($b$) where any such payment exceeds £20.00, 50 per centum of the excess.
\end{enumerate}

\amendment{
Words inserted in para. 24($a$) and words substituted in para. 24 (18.4.95) by the Child Support and Income Support (Amendment) Regulations 1995 reg. 55(2), (3).
}

\medskip

25.  Any payment made to a person in respect of an adopted child who is a member of his family that is made in accordance with any regulations made under section 57A or pursuant to section 57A(6) of the Adoption Act 1976\footnote{\frenchspacing 1976 c. 36. Section 57A was inserted by paragraph 25 of Schedule 10 to the Children Act 1989 (c. 41). The Adoption Allowance Regulations 1991 (S.I. 1991/2030) and the Adoption Allowance (Amendment) Regulations 1991 (S.I. 1991/2130) have been made.} (permitted allowances) or, as the case may be, section 51 of the Adoption (Scotland) Act 1978\footnote{\frenchspacing 1978 c. 28.} (schemes for the payment of allowances to adopters)—
\begin{enumerate}\item[]
($a$) where the child is not a child in respect of whom child support maintenance is being assessed, to the extent that it exceeds 
%the amount referred to in regulation 9(1)($g$)(i), 
the aggregate of the amounts to be taken into account in the calculation of E under regulation 9(1)($g$), % Words substituted (5.4.93) by SI 1993/913 reg 29
reduced, as the case may be, under regulation 9(4);

($b$) in any other case, to the extent that it does not exceed the amount of the income of a child which is treated as that of his parent by virtue of Part IV.
\end{enumerate}

\amendment{
Words substituted in para. 25($a$) (5.4.93) by the Child Support (Miscellaneous Amendments) Regulations 1993 reg. 29.
}

\medskip

26.  Where a local authority makes a payment in respect of the accommodation and maintenance of a child in pursuance of paragraph 15 of Schedule 1 to the Children Act 1989\footnote{\frenchspacing 1989 c. 41.} (local authority contribution to child’s maintenance) to the extent that it exceeds the amount referred to in 
%regulation 9(1)($g$)(i) 
regulation 9(1)($g$) % Words substituted (5.4.93) by SI 1993/913 reg 30
(reduced, as the case may be, under regulation 9(4)).

\amendment{
Words substituted in para. 26 (5.4.93) by the Child Support (Miscellaneous Amendments) Regulations 1993 reg. 30.
}

\medskip

27.  Any payment received under a policy of insurance taken out to insure against the risk of being unable to maintain repayments on a loan taken out to acquire an interest in, or to meet the cost of repairs or improvements to, the parent’s home and used to meet such repayments, to the extent that the payment received under that policy 
%does not in any period exceed 
exceeds  % Words substituted (18.4.95) by SI 1995/1045 reg 55(4)
%the total of—
%\begin{enumerate}\item[]
%($a$) any interest payable on that loan;
%
%($b$) any capital repayable on that loan; and
%
%($c$) any premiums payable on that policy.
%\end{enumerate}
the total of the amount of the payments set out in paragraphs 1($b$), 3(2) and (4) of Schedule 3 as modified, where applicable, by regulation 18.  % Words substituted (22.1.96) by SI 1995/3261 reg 46

\amendment{
Words substituted in para. 27 (18.4.95) by the Child Support and Income Support (Amendment) Regulations 1995 reg. 55(4).

Words substituted in para. 27 (22.1.96) by the Child Support (Miscellaneous Amendments) (No. 2) Regulations 1995 reg. 46 (subject to transitional provisions in reg. 57).
}

\medskip

28.  In the calculation of the income of the parent with care, any maintenance payments made by the absent parent in respect of his qualifying child.

\medskip

29.  Any payment made by a local authority to a person who is caring for a child under section 23(2)($a$) of the Children Act 1989 (provision of accommodation and maintenance by a local authority for children whom the authority is looking after) or, as the case may be, section 21 of the Social Work (Scotland) Act 1968\footnote{\frenchspacing 1968 c. 49.} or by a voluntary organisation under section 59(1)($a$) of the Children Act 1989 (provision of accommodation by voluntary organisations) or by a care authority under regulation 9 of the Boarding Out and Fostering of Children (Scotland) Regulations 1985\footnote{\frenchspacing S.I. 1985/1799.} (provision of accommodation and maintenance for children in care).

\medskip

30.  Any payment made by a health authority, local authority or voluntary organisation in respect of a person who is not normally a member of the household but is temporarily in the care of a member of it.

\medskip

31.  Any payment made by a local authority under section 17 or 24 of the Children Act 1989 or, as the case may be, section 12, 24 or 26 of the Social Work (Scotland) Act 1968 (local authorities' duty to promote welfare of children and powers to grant financial assistance to persons looked after, or in, or formerly in, their care).

\medskip

32.  Any resettlement benefit which is paid to the parent by virtue of regulation 3 of the Social Security (Hospital In-Patients) Amendment (No.\ 2) Regulations 1987\footnote{\frenchspacing S.I. 1987/1683.} (transitional provisions).

\medskip

33.—(1) Any payment or repayment made—
\begin{enumerate}\item[]
($a$) as respects England and Wales, under regulation 3, 5 or 8 of the National Health Service (Travelling Expenses and Remission of Charges) Regulations 1988\footnote{\frenchspacing S.I. 1988/546.} (travelling expenses and health service supplies);

($b$) as respects Scotland, under regulation 3, 5 or 8 of the National Health Service (Travelling Expenses and Remission of Charges) (Scotland) Regulations 1988\footnote{\frenchspacing S.I. 1988/551.} (travelling expenses and health service supplies).
\end{enumerate}

(2) Any payment or repayment made by the Secretary of State for Health, the Secretary of State for Scotland or the Secretary of State for Wales which is analogous to a payment or repayment mentioned in sub-paragraph (1).

\medskip

34.  Any payment made (other than a training allowance), whether by the Secretary of State or any other person, under the Disabled Persons Employment Act 1944\footnote{\frenchspacing 7 \& 8 Geo. 6 c. 10.} or in accordance with arrangements made under section 2 of the Employment and Training Act 1973\footnote{\frenchspacing 1973 c. 50.} to assist disabled persons to obtain or retain employment despite their disability.

\medskip

35.  Any contribution to the expenses of maintaining a household which is made by a non-dependant member of that household.

\medskip

36.  Any sum in respect of a course of study attended by a child payable by virtue of regulations made under section 81 of the Education Act 1944\footnote{\frenchspacing 7 \& 8 Geo. 6 c. 31.} (assistance by means of scholarship or otherwise), or by virtue of section 2(1) of the Education Act 1962\footnote{\frenchspacing 10 \& 11 Eliz. 2 c. 12.} (awards for courses of further education) or section 49 of the Education (Scotland) Act 1980\footnote{\frenchspacing 1980 c. 44; section 49 was amended by the Self Governing Schools (Scotland) Act 1989 (c. 39), Schedule 10.} (power to assist persons to take advantage of educational facilities).

\medskip

37.  Where a person receives income under an annuity purchased with a loan which satisfies the following conditions—
\begin{enumerate}\item[]
($a$) that loan was made as part of a scheme under which not less than 90 per centum of the proceeds of the loan were applied to the purchase by the person to whom it was made of an annuity ending with his life or with the life of the survivor of two or more persons (in this paragraph referred to as “the annuitants”) who include the person to whom the loan was made;

($b$) that the interest on the loan is payable by the person to whom it was made or by one of the annuitants;

($c$) that at the time the loan was made the person to whom it was made or each of the annuitants had attained the age of 65;

($d$) that the loan was secured on a dwelling in Great Britain and the person to whom the loan was made or one of the annuitants owns an estate or interest in that dwelling; and

($e$) that the person to whom the loan was made or one of the annuitants occupies the dwelling on which it was secured as his home at the time the interest is paid,
\end{enumerate}
the amount, calculated on a weekly basis equal to—
\begin{enumerate}\item[]
(i) where, or insofar as, section 26 of the Finance Act 1982\footnote{\frenchspacing 1982 c. 39.} (deduction of tax from certain loan interest) applies to the payments of interest on the loan, the interest which is payable after the deduction of a sum equal to income tax on such payments at the basic rate for the year of assessment in which the payment of interest becomes due;

(ii) in any other case the interest which is payable on the loan without deduction of such a sum.
\end{enumerate}

\medskip

38.  Any payment of the description specified in paragraph 39 of Schedule 9 to the Income Support Regulations\footnote{\frenchspacing Paragraph 39 was substituted by S.I. 1991/1175.} (disregard of payments made under certain trusts and disregard of certain other payments) and any income derived from the investment of such payments.

\medskip

39.  Any payment made to a juror or witness in respect of attendance at court other than compensation for loss of earnings or for loss of a benefit payable under the Contributions and Benefits Act
or the Jobseekers Act.  % Words inserted (7.10.96) by SI 1996/1345 reg 6(6), (7)(c)

\amendment{
Words inserted in para. 39 (7.10.96) by the Social Security and Child Support (Jobseeker's Allowance) (Consequential Amendments) Regulations 1996 reg. 6(6), (7)($c$).
}

\medskip

40.  Any special war widows' payment made under—
\begin{enumerate}\item[]
($a$) the Naval and Marine Pay and Pensions (Special War Widows Payment) Order 1990 made under section 3 of the Naval and Marine Pay and Pensions Act 1865\footnote{\frenchspacing 28 \& 29 Vict. c. 73. Copies of the Order are available from the Ministry of Defence, NPC2, Room 317, Archway Block South, Old Admiralty Buildings, Spring Gardens, London SW1A 2BE.};

($b$) the Royal Warrant dated 19th February 1990 amending the Schedule to the Army Pensions Warrant 1977\footnote{\frenchspacing Army Code No. 13045 published by HMSO.};

($c$) the Queen’s Order dated 26th February 1990 made under section 2 of the Air Force (Constitution) Act 1917\footnote{\frenchspacing 7 \& 8 Geo. 5 c. 51. Queen’s Regulations for the Royal Air Force are available from HMSO.};

($d$) the Home Guard War Widows Special Payments Regulations 1990 made under section 151 of the Reserve Forces Act 1980\footnote{\frenchspacing 1980 c. 9. Copies of the Regulations are available from the Ministry of Defence, NPC2, Room 317, Archway Block South, Old Admiralty Building, Spring Gardens, London SW1A 2BE.};

($e$) the Orders dated 19th February 1990 amending Orders made on 12th December 1980 concerning the Ulster Defence Regiment made in each case under section 140 of the Reserve Forces Act 1980\footnote{\frenchspacing Army Code No. 60589 published by HMSO.},
\end{enumerate}
and any analogous payment by the Secretary of State for Defence to any person who is not a person entitled under the provisions mentioned in sub-paragraphs ($a$) to ($e$).

\medskip

41.  Any payment to a person as holder of the Victoria Cross or the George Cross or any analogous payment.

\medskip

42.  Any payment made either by the Secretary of State for the Home Department or by the Secretary of State for Scotland under a scheme established to assist relatives and other persons to visit persons in custody.

\medskip

43.  Any amount by way of a refund of income tax deducted from profits or emoluments chargeable to income tax under Schedule D or Schedule E.

\medskip

44.  Maintenance payments (whether paid under the Act or otherwise) insofar as they are not treated as income under Part III or IV.

\medskip

45.  Where following a divorce or separation—
\begin{enumerate}\item[]
($a$) capital is divided between the parent and the person who was his partner before the divorce or separation; and

($b$) that capital is intended to be used to acquire a new home for that parent or to acquire furnishings for a home of his,
\end{enumerate}
income derived from the investment of that capital for one year following the date on which that capital became available to the parent.

\medskip

%46.  Payments in kind.
%Para 46 substituted (5.4.93) by SI 1993/913 reg 31
46.  Except in the case of a self-employed earner, payments in kind.

\amendment{
Para. 46 substituted (5.4.93) by the Child Support (Miscellaneous Amendments) Regulations 1993 reg. 31.
}

\medskip

47.  Any payment made by the Joseph Rowntree Memorial Trust from money provided to it by the Secretary of State for Health for the purpose of maintaining a family fund for the benefit of severely handicapped children.

\medskip

48.  Any payment of expenses to a person who is—
\begin{enumerate}\item[]
($a$) engaged by a charitable or voluntary body; or

($b$) a volunteer,
\end{enumerate}
if he otherwise derives no remuneration or profit from the body or person paying those expenses.

\medskip

%Paras 48A, 48B inserted (5.4.93) by SI 1993/913 reg 32.

48A.  Any guardian’s allowance under Part III of the Contributions and Benefits Act. 

\amendment{
Para. 48A inserted (5.4.93) by the Child Support (Miscellaneous Amendments) Regulations 1993 reg. 32.
}

\medskip

48B. Any payment in respect of duties mentioned in paragraph 1(1)($i$) of Chapter I of Part I of Schedule 1 relating to a period of one year or more.

\amendment{
Para. 48B inserted (5.4.93) by the Child Support (Miscellaneous Amendments) Regulations 1993 reg. 32.
}

\medskip

% Para 48C inserted (13.1.97) by SI 1996/3196 reg 14
48C.  Any payment to a person under section 1 of the Community Care (Direct Payments) Act 1996\footnote{\frenchspacing 1996 c. 30.} or section 12B of the Social Work (Scotland) Act 1968\footnote{\frenchspacing 1968 c. 49; section 12B was inserted by section 4 of the Community Care (Direct Payments) Act 1996.} in respect of his securing community care services, as defined in section 46 of the National Health Services and Community Care Act 1990\footnote{\frenchspacing 1990 c. 46.}.

\amendment{
Para. 48C inserted (13.1.97) by the Child Support (Miscellaneous Amendments) (No. 2) Regulations 1996 reg. 14.

Under the Child Support (Miscellaneous Amendments) (No. 2) Regulations 1996 reg. 16, a maintenance assessment in force on 13.1.97 shall not be reviewed solely to give effect to para. 48C, but para. 48C shall apply in conducting a review of the assessment under s. 16, 17, 18 or 19 of the Act, and the effective date of any fresh assessment affected by para. 48C shall not be earlier than the first day of the first maintenance period which commences on or after 13.1.97.
}

\medskip

49.  In this Schedule—
\begin{enumerate}\item[]
“concessionary payment” means a payment made under arrangements made by the Secretary of State with the consent of the Treasury which is charged either to the National Insurance Fund or to a Departmental Expenditure Vote to which payments of benefit under the Contributions and Benefits Act 
or the Jobseekers Act  % Words inserted (7.10.96) by SI 1996/1345 reg 6(6), (7)(c)
are charged;

“health authority” means a health authority established under the National Health Service Act 1977\footnote{\frenchspacing 1977 c. 49.} or the National Health Service (Scotland) Act 1978\footnote{\frenchspacing 1978 c. 29.};

“mobility supplement” has the same meaning as in regulation 2(1) of the Income Support Regulations;

“war disablement pension” and “war widow” have the same meanings as in section 150(2) of the Contributions and Benefits Act.
\end{enumerate}

\amendment{
Words inserted in reg. 49 (7.10.96) by the Social Security and Child Support (Jobseeker's Allowance) (Consequential Amendments) Regulations 1996 reg. 6(6), (7)($c$).
}

\part[Schedule 3 --- Eligible housing costs]{Schedule 3\\*Eligible housing costs}

\renewcommand\parthead{--- Schedule 3}

\subsection*{Eligible housing costs for the purposes of determining exempt income and protected income}

1.  Subject to the following provisions of this Schedule, the following payments in respect of the provision of a home shall be eligible to be taken into account as housing costs for the purposes of these Regulations—
\begin{enumerate}\item[]
($a$) payments of, or by way of, rent;

($b$) mortgage interest payments;

($c$) interest payments under a hire purchase agreement to buy a home;

($d$) interest payments on loans for repairs and improvements to the home%
, including interest on a loan for any service charge imposed to meet the cost of such repairs and improvements; % Words inserted (5.4.93) by SI 1993/913 reg 33($a$)

($e$) payments by way of ground rent or in Scotland, payments by way of feu duty;

($f$) payments under a co-ownership scheme;

($g$) payments in respect of, or in consequence of, the use and occupation of the home;

($h$) where the home is a tent, payments in respect of the tent and the site on which it stands;

($i$) payments in respect of a licence or permission to occupy the home (whether or not board is provided);

($j$) payments by way of mesne profits or, in Scotland, violent profits;

($k$) payments of, or by way of, service charges, the payment of which is a condition on which the right to occupy the home depends;

($l$) payments under or relating to a tenancy or licence of a Crown tenant;

($m$) mooring charges payable for a houseboat;

($n$) where the home is a caravan or a mobile home, payments in respect of the site on which it stands;

($o$) any contribution payable by a parent resident in an almshouse provided by a housing association which is either a charity of which particulars are entered in the register of charities established under section 4 of the Charities Act 1960\footnote{\frenchspacing 8 \& 9 Eliz. 2 c. 58; subsections (8) and (10) of section 4 were amended by section 1(4) and (5) and Schedule 2, Parts I and II of the Education Act 1973 (c. 16).} (register of charities) or an exempt charity within the meaning of that Act, which is a contribution towards the cost of maintaining that association’s almshouses and essential services in them;

($p$) payments under a rental purchase agreement, that is to say an agreement for the purchase of a home under which the whole or part of the purchase price is to be paid in more than one instalment and the completion of the purchase is deferred until the whole or a specified part of the purchase price has been paid;

($q$) where, in Scotland, the home is situated on or pertains to a croft within the meaning of section 3(1) of the Crofters (Scotland) Act 1955\footnote{\frenchspacing 3 \& 4 Eliz. 2 c. 21; section 3(1) was amended by section 14 of the Crofting Reform (Scotland) Act 1976 (c. 21).}, the payment in respect of the croft land;

($r$) where the home is provided by an employer (whether under a condition or term in a contract of service or otherwise), payments to that employer in respect of the home, including payments made by the employer deducting the payment in question from the remuneration of the parent in question;

%Para 1(s) omitted (5.4.93) by SI 1993/913 reg 33($b$)
%(s) payments analogous to those mentioned in this paragraph;

($t$) payments in respect of a loan taken out to pay off another loan but only to the extent that it was incurred 
%for that purpose.
in respect of payments eligible to be taken into account as housing costs by virtue of the other provisions of this Schedule.  % Words substituted in para 1(t) by SI 1996/3196 reg 15(2)
%and only to the extent to which the interest on that other loan would have been met under this paragraph.  % Words omitted in para 1(t) by SI 1995/1045 reg 56(2)
\end{enumerate}

\amendment{
Words inserted in para. 1($d$) and para. 1(s) omitted (5.4.93) by the Child Support (Miscellaneous Amendments) Regulations 1993 reg. 33.

Words omitted in para. 1($t$) (18.4.95) by the Child Support and Income Support (Amendment) Regulations 1995 reg. 56(2).

Words substituted in para. 1(t) (13.1.97) by the Child Support (Miscellaneous Amendments) (No. 2) Regulations 1996 reg. 15(2).

Under the Child Support (Miscellaneous Amendments) (No. 2) Regulations 1996 reg. 16, a maintenance assessment in force on 13.1.97 shall not be reviewed solely to give effect to para. 1(t) as amended by those Regulations, but para. 1(t) as so amended shall apply in conducting a review of the assessment under s. 16, 17, 18 or 19 of the Act, and the effective date of any fresh assessment affected by para. 1(t) as so amended shall not be earlier than the first day of the first maintenance period which commences on or after 13.1.97.
}

\subsection*{Loans for repairs and improvements to the home}

2.  %For the purposes of 
Subject to paragraph 2A (loans for repairs and improvements in transitional cases), for the purposes of  % Words substituted (18.4.95) by SI 1995/1045 reg 56(3)
paragraph 1($d$) “repairs and improvements” means major repairs necessary to maintain the fabric of the home and any of the following measures undertaken with a view to improving its fitness for occupation—
\begin{enumerate}\item[]
($a$) installation of a fixed bath, shower, wash basin or lavatory, and necessary associated plumbing;

($b$) damp proofing measures;

($c$) provision or improvement of ventilation and natural lighting;

($d$) provision of electric lighting and sockets;

($e$) provision or improvement of drainage facilities;

($f$) improvement of the structural condition of the home;

($g$) improvements to the facilities for the storing, preparation and cooking of food;

($h$) provision of heating, including central heating;

($i$) provision of storage facilities for fuel and refuse;

($j$) improvements to the insulation of the home;

($k$) other improvements which the child support officer considers reasonable in the circumstances.
\end{enumerate}

\amendment{
Words substituted in para. 2 (18.4.95) by the Child Support and Income Support (Amendment) Regulations 1995 reg. 56(3).
}

% Para 2A inserted (18.4.95) by SI 1995/1045 reg 56(4)
\subsection*{Loans for repairs and improvements in transitional cases}

2A.  In the case of a loan entered into before the first date upon which a maintenance application or enquiry form is given or sent or treated as given or sent to the relevant person, for the purposes of paragraph 1($d$) “repairs and improvements” means repairs and improvements of any description whatsoever.

\amendment{
Para. 2A inserted (18.4.95) by the Child Support and Income Support (Amendment) Regulations 1995 reg. 56(4).
}

\subsection*{\sloppy Exempt income: additional provisions relating to eligible housing costs}

3.—(1) The additional provisions made by this paragraph shall have effect only for the purpose of calculating or estimating exempt income.

(2) Subject to sub-paragraph (6), where the home of an absent parent or, as the case may be, a parent with care, is subject to a mortgage or charge and that parent makes periodical payments to reduce the capital secured by that mortgage or charge of an amount provided for in accordance with the terms thereof, the amount of those payments shall be eligible to be taken into account as the housing costs of that parent.

% Para 3(2A) inserted (22.1.96) by SI 1995/3261 reg 47(2)
(2A) Where an absent parent or as the case may be a parent with care has entered into a loan for repairs or improvements of a kind referred to in paragraph 1($d$) and that parent makes periodical payments of an amount provided for in accordance with the terms of that loan to reduce the amount of that loan, the amount of those payments shall be eligible to be taken into account as housing costs of that parent.

(3) Subject to sub-paragraph (6), where the home of an absent parent or, as the case may be, a parent with care, is held under an agreement and certain payments made under that agreement are included as housing costs by virtue of paragraph 1 of this Schedule, the weekly amount of any other payments which are made in accordance with that agreement by the parent in order either—
\begin{enumerate}\item[]
($a$) to reduce his liability under that agreement; or

($b$) to acquire the home to which it relates,
\end{enumerate}
shall also be eligible to be taken into account as housing costs.

(4) Where a policy of insurance has been obtained and retained for the purpose of discharging a mortgage or charge on the home of the parent in question, the amount of the premiums paid under that policy shall be eligible to be taken into account as a housing cost
including for the avoidance of doubt such a policy of insurance whose purpose is to secure the payment of monies due under the mortgage or charge in the event of the unemployment, sickness or disability of the insured.  % Words inserted (18.4.95) by SI 1995/1045 reg 56(5)($a$)

% Para 3(4A) inserted (13.1.97) by SI 1996/3196 reg 15(3)(a)
(4A) Where—
\begin{enumerate}\item[]
($a$) an absent parent or parent with care has obtained a loan which constitutes an eligible housing cost falling within sub-paragraph ($d$) or ($t$) of paragraph 1; and

($b$) a policy of insurance has been obtained and retained, the purpose of which is solely to secure the payment of monies due under that loan in the event of the unemployment, sickness or disability of the insured person,
\end{enumerate}
the amount of the premiums payable under that policy shall be eligible to be taken into account as a housing cost.

%(5) Where a policy of insurance has been obtained and retained for the purpose of discharging a mortgage or charge on the home of the parent in question and also for the purpose of accruing profits on the maturity of the policy, the part of the premiums paid under that policy which are necessarily incurred for the purpose of discharging the mortgage or charge shall be eligible to be taken into account as a housing cost; and, where that part cannot be ascertained, 0.0277 per centum of the amount secured by the mortgage or charge shall be deemed to be the part which is eligible to be taken into account as a housing cost.

% Para 3(5) substituted (7.2.94) by SI 1994/227 reg 4(8)
(5) Where a policy of insurance has been obtained and retained for the purpose of discharging a mortgage or charge on the home of the parent in question and also for the purpose of accruing profits on the maturity of the policy, there shall be eligible to be taken into account as a housing cost—
\begin{enumerate}\item[]
($a$) where the sum secured by the mortgage or charge does not exceed £60,000, the whole of the premiums paid under that policy; and

($b$) where the sum secured by the mortgage or charge exceeds £60,000, the part of the premiums paid under that policy which are necessarily incurred for the purpose of discharging the mortgage or charge or, where that part cannot be ascertained, 0.0277 per centum of the amount secured by the mortgage or charge.
\end{enumerate}

% Para 3(5A), (5B) inserted (18.4.95) by SI 1995/1045 reg 56(5)($b$)
(5A) Where a plan within the meaning of regulation 4 of the Personal Equity Plans Regulations 1989\footnote{\frenchspacing S.I. 1989/469; relevant amendments were made by S.I. 1990/678 and 1991/733.} has been obtained and retained for the purpose of discharging a mortgage or charge on the home of the parent in question and also for the purpose of accruing profits upon the realisation of the plan, there shall be eligible to be taken into account as a housing cost—
\begin{enumerate}\item[]
($a$) where the sum secured by the mortgage or charge does not exceed £60,000, the whole of the premiums payable in respect of the plan; and

($b$) where the sum secured by the mortgage or charge exceeds £60,000, that part of the premiums payable in respect of the plan which is necessarily incurred for the purpose of discharging the mortgage or charge or, where that part cannot be ascertained, 0.0277 per centum of the amount secured by the mortgage or charge.
\end{enumerate}

(5B) Where a personal pension plan 
derived from a personal pension scheme  % Words inserted in para 3(5B) by SI 1996/3196 reg 15(3)(b)
has been obtained and retained for the purpose of discharging a mortgage or charge on the home of the parent in question and also for the purpose of securing the payment of a pension to him, there shall be eligible to be taken into account as a housing cost 25 per centum of the contributions payable in respect of that personal pension plan.

(6) For the purposes of sub-paragraphs (2) and (3), housing costs shall not include—
\begin{enumerate}\item[]
($a$) 
%any payment of arrears or payments in excess of those which are required 
any payments in excess of those required  % Words substituted (18.4.95) by SI 1995/1045 reg 56(5)($c$)(i)
to be made under or in respect of a mortgage, charge or agreement to which either of those sub-paragraphs relate;

($b$) payments under any second or subsequent mortgage on the home to the extent that 
%they are attributable to arrears or would otherwise not be eligible 
they would not be eligible  % Words substituted (18.4.95) by SI 1995/1045 reg 56(5)($c$)(ii)
to be taken into account as housing costs;

($c$) premiums payable in respect of any policy of insurance against loss caused by the destruction of or damage to any building or land.
\end{enumerate}

\amendment{
Para. 3(5) substituted (7.2.94) by the Child Support (Miscellaneous Amendments and Transitional Provisions) Regulations 1994 reg. 4(8) (subject to transitional provisions in reg. 12).

Words inserted in para. 3(4), words substituted in para. 3(6)($a$), ($b$) and para. 3(5A), (5B) inserted (18.4.95) by the Child Support and Income Support (Amendment) Regulations 1995 reg. 56(5).

Para. 3(2A) inserted (22.1.96) by the Child Support (Miscellaneous Amendments) (No. 2) Regulations 1995 reg. 47(2) (subject to transitional provisions in reg. 57).

Words inserted in para. 3(5B) and para. 3(4A) inserted (13.1.97) by the Child Support (Miscellaneous Amendments) (No. 2) Regulations 1996 reg. 15(3).

Under the Child Support (Miscellaneous Amendments) (No. 2) Regulations 1996 reg. 16, a maintenance assessment in force on 13.1.97 shall not be reviewed solely to give effect to para. 3(4A) or the words ``derived from a personal pension scheme'' in para. 3(5B), but para. 3(4A) and those words in para. 3(5B) shall apply in conducting a review of the assessment under s. 16, 17, 18 or 19 of the Act, and the effective date of any fresh assessment affected by para. 3(4A) or those words in para. 3(5B) shall not be earlier than the first day of the first maintenance period which commences on or after 13.1.97.
}

\subsection*{Conditions relating to eligible housing costs}

4.—(1) Subject to the following provisions of this paragraph the housing costs referred to in this Schedule shall be included as housing costs only where—
\begin{enumerate}\item[]
%($a$) they are incurred in relation to the parent’s home;

% Para 4(1)(a) substituted (13.1.97) by SI 1996/3196 reg 15(4)(a)
($a$) they are necessarily incurred for the purpose of purchasing, renting or otherwise securing possession of the home for the parent and his family, or for the purpose of carrying out repairs and improvements to that home;

($b$) the parent or, if he is one of a family, he or a member of his family, is responsible for those costs; and

($c$) the liability to meet those costs is to a person other than a member of the same household.
\end{enumerate}

% Para 4(1A) inserted (13.1.97) by SI 1996/3196 reg 15(4)(b)
(1A) For the purposes of sub-paragraph (1)($a$) “repairs and improvements” shall have the meaning given in paragraph 2 of this Schedule.

(2) For the purposes of sub-paragraph (1)($b$) a parent shall be treated as responsible for housing costs where—
\begin{enumerate}\item[]
($a$) because the person liable to meet those costs is not doing so, he has to meet those costs in order to continue to live in the home and either he was formerly the partner of the person liable, or he is some other person whom it is reasonable to treat as liable to meet those costs; or

($b$) he pays a share of those costs in a case where—
\begin{enumerate}\item[]
(i) he is living in a household with other persons;

(ii) those other persons include persons who are not close relatives of his or his partner;

(iii) a person who is not such a close relative is responsible for those costs under the preceding provisions of this paragraph or has an equivalent responsibility for housing expenditure; and

(iv) it is reasonable in the circumstances to treat him as sharing that responsibility.
\end{enumerate}
\end{enumerate}

% Para 4(3), (4) inserted (13.1.97) by SI 1996/3196 reg 15(4)(c)
(3) Subject to sub-paragraph (4), payments on a loan shall constitute an eligible housing cost only if that loan has been obtained for the purposes specified in sub-paragraph (1)($a$).

(4) Where a loan has been obtained only partly for the purposes specified in sub-paragraph (1)($a$), the eligible housing cost shall be limited to that part of the payment attributable to those purposes.

\amendment{
Para. 4(1A), (3), (4) inserted and para. 4(1)(a) substituted (13.1.97) by the Child Support (Miscellaneous Amendments) (No. 2) Regulations 1996 reg. 15(3).

Under the Child Support (Miscellaneous Amendments) (No. 2) Regulations 1996 reg. 16, a maintenance assessment in force on 13.1.97 shall not be reviewed solely to give effect to para. 4(1A), (3), (4) or para. 4(1)(a) as substituted by those Regulations, but para. 4(1A), (3), (4), (1)(a) as so substituted shall apply in conducting a review of the assessment under s. 16, 17, 18 or 19 of the Act, and the effective date of any fresh assessment affected by para. 4(1A), (3), (4), (1)(a) as so substituted shall not be earlier than the first day of the first maintenance period which commences on or after 13.1.97.
}

\subsection*{Accommodation also used for other purposes}

5.  Where amounts are payable in respect of accommodation which consists partly of residential accommodation and partly of other accommodation, only such proportion thereof as is attributable to residential accommodation shall be eligible to be taken into account as housing costs.

\subsection*{Ineligible service and fuel charges}

6.  Housing costs shall not include—
\begin{enumerate}\item[]
%($a$) where the costs are inclusive of ineligible service charges within the meaning of 
%%paragraph 1 
%paragraph 1($a$)(i)  % Words substituted (18.4.95) by SI 1995/1045 reg 56(6)($a$)
%of Schedule 1 to the Housing Benefit (General) Regulations 1987\footnote{\frenchspacing S.I. 1987/1971.} (ineligible service charges), the amounts attributable to those ineligible service charges or, where that amount is not separated from or separately identified within the housing costs to be met under this paragraph, such part of the payments made in respect of those housing costs which are fairly attributable to the provision of those ineligible services having regard to the costs of comparable services;

% Para 6($a$) substituted (22.1.96) by SI 1995/3261 reg 47(3)(i)
($a$) where the costs are inclusive of ineligible service charges within the meaning of paragraph 1($a$)(i) of Schedule 1 to the Housing Benefit (General) Regulations 1987\footnote{\frenchspacing S.I. 1987/1971.} (ineligible service charges), the amounts specified as ineligible in paragraph 1A of that Schedule;

% Para 6(aa) inserted (18.4.95) by SI 1995/1045 reg 56(6)($b$), omitted (22.1.96) by SI 1995/3261 reg 47(3)(ii)
%(aa) where the costs are inclusive of charges, other than those which are not to be included by virtue of sub-paragraph ($a$), that part of those charges which exceeds the greater of the following amounts—
%\begin{enumerate}\item[]
%(i) the total of the charges other than those which are ineligible service charges within the meaning of paragraph 1 of Schedule 1 to the Housing Benefit Regulations (housing costs);
%
%(ii) 25 per centum of the total amount of eligible housing costs;
%\end{enumerate}

($b$) where the costs are inclusive of any of the items mentioned in paragraph 5(2) of Schedule 1 to the Housing Benefit (General) Regulations 1987 (payment in respect of fuel charges), the deductions prescribed in that paragraph unless the parent provides evidence on which the actual or approximate amount of the service charge for fuel may be estimated, in which case the estimated amount; 
%and % Word omitted (22.1.96) by SI 1995/3261 reg 47(3)(iii)

($c$) charges for water, sewerage or allied environmental services and where the amount of such charges is not separately identified, such part of the charges in question as is attributable to those services;
and

% Para 6($d$) inserted (22.1.96) by SI 1995/3261 reg 47(3)(iv)
($d$) where the costs are inclusive of charges, other than those which are not to be included by virtue of sub-paragraphs ($a$) to ($c$), that part of those charges which exceeds the greater of the following amounts—
\begin{enumerate}\item[]
(i) the total of the charges other than those which are ineligible service charges within the meaning of paragraph 1 of Schedule 1 to the Housing Benefit Regulations (housing costs);

(ii) 25 per centum of the total amount of eligible housing costs,
\end{enumerate}
and for the purposes of this sub-paragraph, where the amount of those charges is not separately identifiable, that amount shall be such amount as is reasonably attributable to those charges.
\end{enumerate}

\amendment{
%Words substituted in para. 6($a$) and para. 6(aa) inserted (18.4.95) by the Child Support and Income Support (Amendment) Regulations 1995 reg. 56(6).

Para. 6($d$) inserted, para. 6($a$) substituted and para. 6(aa) omitted (22.1.96) by the Child Support (Miscellaneous Amendments) (No. 2) Regulations 1995 reg. 47(3) (subject to transitional provisions in reg. 57).
}

\subsection*{Interpretation}

7.  In this Schedule except where the context otherwise requires—
\begin{enumerate}\item[]
“close relative” means a parent, parent-in-law, son, son-in-law, daughter, daughter-in-law, step-parent, step-son, step-daughter, brother, sister, or the spouse of any of the preceding persons or, if that person is one of an unmarried couple, the other member of that couple;

“co-ownership scheme” means a scheme under which the dwelling is let by a housing association and the tenant, or his personal representative, will, under the terms of the tenancy agreement or of the agreement under which he became a member of the association, be entitled, on his ceasing to be a member and subject to any conditions stated in either agreement, to a sum calculated by reference directly or indirectly to the value of the dwelling;

“housing association” has the meaning assigned to it by section 1(1) of the Housing Association Act 1985\footnote{\frenchspacing 1985 c. 69.}.
\end{enumerate}

% Sch. 3A inserted (18.4.95) by SI 1995/1045 reg 57 and Sch 1
\part[Schedule 3A --- Amount to be allowed in respect of transfer of property]{Schedule 3A\\*Amount to be allowed in respect of transfer of property}

\renewcommand\parthead{--- Schedule 3A}

\amendment{
Sch. 3A inserted (18.4.95) by the Child Support and Income Support (Amendment) Regulations 1995 reg. 57 and Sch. 1.
}

\subsection*{Interpretation}

1.—(1) In this Schedule—
\begin{enumerate}\item[]
“property” means—
\begin{enumerate}\item[]
($a$) a legal estate or an equitable interest in land; or

($b$) a sum of money which is derived from or represents capital, whether in cash or in the form of a deposit with—
\begin{enumerate}\item[]
(i) the Bank of England;

(ii) an authorised institution or an exempted person within the meaning of the Banking Act 1987\footnote{\frenchspacing 1987 c. 22.};

(iii) a building society incorporated or deemed to be incorporated under the Building Societies Act 1986\footnote{\frenchspacing 1986 c. 53.};
\end{enumerate}

\begin{sloppypar}
($c$) any business asset as defined in sub-paragraph (2) (whether in the form of money or an interest in land or otherwise);
\end{sloppypar}

($d$) any policy of insurance which has been obtained and retained for the purpose of providing a capital sum to discharge a mortgage or charge secured upon an estate or interest in land which is also the subject of the transfer (in this schedule referred to as an endowment policy);
\end{enumerate}

“qualifying transfer” means a transfer of property—
\begin{enumerate}\item[]
($a$) which was made in pursuance of a court order made, or a written maintenance agreement executed, before 5th April 1993;

($b$) which was made between the absent parent and either the parent with care or a relevant child;

($c$) which was made at a time when the absent parent and the parent with care were living separate and apart;

($d$) the effect of which is that the parent with care or a relevant child is beneficially entitled (subject to any mortgage or charge) to the whole of the asset transferred; and 

($e$) which was not made expressly for the purpose only of compensating the parent with care for the loss of any right to apply for or receive periodical payments or a capital sum in respect of herself;
\end{enumerate}

“compensating transfer” means a transfer of property which would be a qualifying transfer (disregarding the requirement of paragraph ($e$) of the definition of “qualifying transfer”) if it were made by the absent parent, but which is made by the parent with care in favour of the absent parent or a relevant child;

“relevant date” means the date of the making of the court order or the execution of the written maintenance agreement in pursuance of which the qualifying transfer was made.
\end{enumerate}

(2) For the purposes of sub-paragraph (1) “business asset” means an asset, whether in the form of money or an interest in land or otherwise which, prior to the date of transfer was use in the course of a trade or business carried on—
\begin{enumerate}\item[]
($a$) by the absent parent as a sole trader;

($b$) by the absent parent in partnership, whether with the parent with care or not;

($c$) by a close company within the meaning of sections 414 and 415 of the Income and Corporation Taxes Act 1988\footnote{\frenchspacing 1986 c. 1.} in which the absent parent was a participator at the date of the transfer.
\end{enumerate}

(3) Where the condition specified in regulation 10($a$) is satisfied this Schedule shall apply as if references—
\begin{enumerate}\item[]
($a$) to the parent with care were references to the absent parent; and

($b$) to the absent parent were references to the parent with care.
\end{enumerate}

\subsection*{\sloppy Evidence to be produced in connection with the allowance for transfers of property}

2.—(1) Where the absent parent produces to the Secretary of State—
\begin{enumerate}\item[]
($a$) contemporaneous evidence in writing of the making of a court order or of the execution of a written maintenance agreement, which requires the relevant person to make a qualifying transfer of property;

($b$) evidence in writing and whether contemporaneous or not as to—
\begin{enumerate}\item[]
(i) the fact of the transfer;

(ii) the value of the property transferred at the relevant date;

(iii) the amount of any mortgage or charge outstanding at the relevant date,
\end{enumerate}
\end{enumerate}
an amount in respect of the relevant value of the transfer determined in accordance with the following provisions of this Schedule shall be allowed in calculating or estimating the exempt income of the absent parent.

(2) Whether the evidence specified in sub-paragraph (1) is not produced within a reasonable time after the Secretary of State has been notified of the wish of the absent parent that a child support officer consider the question, the officer shall determine the question on the basis that the relevant value of the transfer is nil.

\subsection*{Consideration of evidence produced by other parent}

3.  Where an absent parent has notified the Secretary of State that he wishes a child support officer to consider whether an amount should be allowed in respect of the relevant value of a qualifying transfer, the Secretary of State shall give notice to the other parent that he proposes to refer the question to a child support officer for consideration and shall transmit to the child support officer any representations made by the other parent in considering the question.

\subsection*{{Computation of qualifying value—business assets} and land}

4.—(1) Subject to paragraph 6, where the property which is the subject of the transfer by the absent parent is, or includes an estate or interest in land, or a business asset, the qualifying value of that estate, interest or asset shall be determined in accordance with the formula—
\[\mathrm{QV} = \frac{\mathrm{VT} - \mathrm{MC}}{2}\]
where—
\begin{enumerate}\item[]
(i) QV is the qualifying value,

(ii) VT is the value of the estate or interest in land or the value of the asset (as the case may be) calculated at the relevant date, and

(iii) MC is the amount of the principal outstanding at the relevant date under any mortgage or charge on the estate, interest or asset.
\end{enumerate}

(2) For the purposes of sub-paragraph (1) the value of an estate or interest in land is to be determined upon the basis that the parent with care and any relevant child, if in occupation of the land, would quit on completion of the sale.

\subsection*{Computation of qualifying value—cash, deposits and endowment policies}

5.  Subject to paragraph 6, where the property which is the subject of the qualifying transfer is, or includes—
\begin{enumerate}\item[]
(i) a sum of money whether in cash or in the form of a deposit with the Bank of England, and authorised institution or exempted person within the meaning of the Banking Act 1987, or a building society incorporated or deemed to be incorporated under the Building Societies Act 1986, derived from or representing capital; or

(ii) an endowment policy,
\end{enumerate}
the amount of the qualifying value shall be determined by applying the formula—
\[ \mathrm{QV} = \frac{\mathrm{VT}}{2}\]
where—
\begin{enumerate}\item[]
QV is the qualifying value; and

VT is the amount of cash, the balance of the account or the surrender value of the endowment policy on the relevant date.
\end{enumerate}

\subsection*{\sloppy Transfer wholly in lieu of periodical payments for relevant child}

6.  Where the evidence produced in relation to a transfer to, or in respect of, a relevant child, shows expressly that the whole of that transfer was made exclusively in lieu of periodical payments in respect of that child—
\begin{enumerate}\item[]
($a$) in a case to which paragraph 4 applies, for the formula given in that paragraph there shall be substituted the following—
\[\mathrm{QV} = \mathrm{VT} - \mathrm{MC};\]
and

($b$) in a case to which paragraph 5 applies, the qualifying value shall be the value of the transfer.
\end{enumerate}

\subsection*{Multiple transfers to related persons}

7.—(1) Where there has been more than one qualifying transfer from the absent parent—
\begin{enumerate}\item[]
($a$) to the same parent with care;

($b$) to or for the benefit of the same relevant child;

($c$) to or for the benefit of two or more relevant children with respect to all of whom the same persons are respectively the parent with care and the absent parent;
\end{enumerate}
or any combination thereof, the relevant value by reference to which the allowance is to be calculated in accordance with paragraph 10 shall be the aggregate of the qualifying transfers calculated individually in accordance with the preceding paragraphs of this Schedule, less the value of any compensating transfer or where there has been more than one, the aggregate of the values of the compensating transfers so calculated.

(2) Except as provided by sub-paragraph (1), the values of transfers shall not be aggregated for the purposes of this Schedule.

\subsection*{Computation of the value of compensation transfers}

8.  
%The value of 
Subject to paragraph 8A, the value of  % Words substituted (18.12.95) by SI 1995/3261 reg 48(1)
a compensation transfer shall be determined in accordance with paragraph 4 to 7 above, but as if any reference in those paragraphs—
\begin{enumerate}\item[]
($a$) to the absent parent were a reference to the parent with care;

($b$) to the parent with care were a reference to the absent parent; and

($c$) to a qualifying transfer were a reference to a compensating transfer.
\end{enumerate}

\amendment{
Words substituted in para. 8 (18.12.95) by the Child Support (Miscellaneous Amendments) (No. 2) Regulations 1995 reg. 48(1).
}

\medskip

% Para 8A inserted (18.12.95) by SI 1995/3261
8A.—(1) This paragraph applies where—
\begin{enumerate}\item[]
($a$) the property which is the subject of a compensating transfer is or includes cash or deposits as defined in paragraph 5(i);

($b$) that property was acquired by the parent with care after the relevant date;

($c$) the absent parent has no legal interest in that property;

($d$) if that property is or includes cash obtained by a mortgage or charge, that mortgage or charge was executed by the parent with care after the relevant date and was of property to the whole of which she is legally entitled; and

($e$) the effect of the compensating transfer is that the parent with care or a relevant child is beneficially entitled (subject to any mortgage or charge) to the whole of the absent parent’s legal estate in the land which is the subject of the qualifying transfer.
\end{enumerate}

(2) Where sub-paragraph (1) applies, the qualifying value of the compensating transfer shall be the amount of the cash or deposits transferred pursuant to the court order or written maintenance agreement referred to in head ($a$) of the definition of “qualifying transfer” in paragraph 1(1).

\amendment{
Para. 8A inserted (18.12.95) by the Child Support (Miscellaneous Amendments) (No. 2) Regulations 1995 reg. 48(2).
}

\subsection*{Computation of relevant value of a qualifying transfer}

9.  The relevant value of a qualifying transfer shall be calculated by deducting from the qualifying value of the qualifying transfer the qualifying value of any compensating transfer between the same persons as are parties to the qualifying transfer.

\subsection*{Amount to be allowed in respect of a qualifying transfer}

10.  For the purposes of regulation 9(1)($bb$), the amount to be allowed in the computation of E, or in the case where regulation 10($a$) applies, F, shall be—
\begin{enumerate}\item[]
($a$) where the relevant value calculated in accordance with paragraph 9 is less than £5,000, nil;

($b$) where the relevant value calculated in accordance with paragraph 9 is at least £5,000, but less than £10,000, £20.00 per week;

($c$) where the relevant value calculated in accordance with paragraph 9 is at least £10,000, but less than £25,000, £40.00 per week;

($d$) where the relevant value calculated in accordance with paragraph 9 is not less than £25,000, £60.00 per week.
\end{enumerate}

\medskip

11.  This Schedule in its application to Scotland shall have effect as if—
\begin{enumerate}\item[]
($a$) in paragraph 1 for the words “legal estate or equitable interest in land” there were substituted the words “an interest in land within the meaning of section 2(6) of the Conveyancing and Feudal Reform (Scotland) Act 1970\footnote{\frenchspacing 1970 c. 35.}”;

($b$) in paragraph 4 the word “estate” and the words “estate or” in each place where they respectively occur were omitted.
\end{enumerate}

% Sch. 3B inserted (18.4.95) by SI 1995/1045 reg 57 and Sch 2
\part[Schedule 3B --- Amount to be allowed in respect of travelling costs]{Schedule 3B\\*Amount to be allowed in respect of travelling costs}

\renewcommand\parthead{--- Schedule 3B}

\amendment{
Sch. 3B inserted (18.4.95) by the Child Support and Income Support (Amendment) Regulations 1995 reg. 57 and Sch. 2.
}

\subsection*{Interpretation}

1.  In this Schedule—
\begin{enumerate}\item[]
“day” means, in relation to a person who attends at a work place for one period of work which commences before midnight of one day and concludes the following day, the first of those days;

“journey” means a single journey, and “pair of journeys” means two journeys in opposing directions, between the same two places;

“relevant employment” means an employed earner’s employment in which the relevant person is employed and in the course of which he is required to attend at a work place, and “relevant employer” means the employer of the relevant person in that employment;

“relevant person” means—
\begin{enumerate}\item[]
($a$) in the application of the provisions of this Schedule to regulation 9, the absent parent or the parent with care; and

($b$) in the application of the provisions of this Schedule to regulation 11, the absent parent;
\end{enumerate}

“straight-line distance” means the straight-line distance measured in miles and calculated to 2 decimal places, and, where that distance is not a whole number of miles, rounded to the nearest whole number of miles, a distance which exceeds a whole number of miles by 0.50 of a mile being rounded up;

“travelling costs” means the costs of—
\begin{enumerate}\item[]
($a$) purchasing either fuel or a ticket for the purpose of travel;

($b$) contributing to the costs borne by a person other than a relevant employer in providing transport; or

($c$) paying another to provide transport,
\end{enumerate}
which are incurred by the relevant person in travelling between the relevant person’s home and his work place, and where he has more than one relevant employment between any of his work places in those employments;

“work place” means the relevant person’s normal place of employment in a relevant employment, and “deemed work place” means a place which has been selected by the child support officer, pursuant either to paragraph 8(2) or 15(2) for the purpose of calculating the amount to be allowed in respect of the relevant person’s travelling costs.
\end{enumerate}

\subsection*{Computation of amount allowable in respect of travelling costs}

2.  For the purpose of regulation 9 and regulation 11 an amount in respect of the travelling costs of the relevant person shall be determined in accordance with the following provisions of this Schedule if the relevant person—
\begin{enumerate}\item[]
($a$) has travelling costs; and

($b$) provides the information required to enable the amount of the allowance to be determined.
\end{enumerate}

\subsection*{Computation in cases where there is one relevant employment and one work place in that employment}

3.  Subject to paragraphs 21 to 23, where the relevant persons has one relevant employment and is normally required to attend at only one work place in the course of that employment the amount to be allowed in respect of travelling costs shall be determined in accordance with paragraphs 4 to 7 below.

\medskip

4.  There shall be calculated or, if this is impracticable, estimated—
\begin{enumerate}\item[]
($a$) the straight-line distance between the relevant person’s home and his work place;

($b$) the number of journeys between the relevant person’s home and this work place which he makes during a period comprising a whole number of weeks which appears to the child support officer to be representative of his normal pattern of work, there being disregarded any pair of journeys between his work place and his home and where the first journey is from his work place to his home and where the time which elapses between the start of the first journey and the conclusion of the second is not more than two hours.
\end{enumerate}

\medskip

5.  The results of the calculation or estimate produced by sub-paragraph ($a$) of paragraph 4 shall be multiplied by the result of the calculation or estimate required by sub-paragraph ($b$) of that paragraph.

\medskip


6.  The product of the multiplication required by paragraph 5 shall be divided by the number of weeks in the period.

\medskip

7.  Where the result of the division required by paragraph 6 is less than or equal to 150, the amount to be allowed in respect of the relevant person’s travelling costs shall be nil, and where it is greater than 150 the weekly allowance to be made in respect of the relevant person’s travelling costs shall be 10 pence multiplied by the number by which that number exceeds 150.

\subsection*{Computation in cases where there is more than one work place but only one relevant employment}

8.—(1) Subject to sub-paragraph (2) and paragraphs 21 to 23 below, where the relevant person has one relevant employment but attends at more than one work place the amount to be allowed in respect of travelling costs for the purposes of regulations 9 and 11 shall be determined in accordance with paragraphs 9 to 13.

(2) Where it appears that the relevant person works at more than one work place but his pattern of work is not sufficiently regular to enable the calculation of the amounts to be allowed in respect of his travelling costs to be made readily, the child support officer may—
\begin{enumerate}\item[]
($a$) select a place which is either one of the relevant person’s work places or some other place which is connected with the relevant employment; and

($b$) apply the provisions of paragraphs 4 to 7 above to calculate the amount of the allowance to be made in respect of travelling costs upon the basis that the relevant person makes one journey from his home to the deemed work place and one journey from the deemed work place to home on each day on which he attends at a work place in connection with relevant employment,
\end{enumerate}
and the provision of paragraphs 9 to 13 shall not apply.

(3) For the purposes of sub-paragraph (2)($b$) there shall be disregarded any day upon which the relevant person attends at a work place and in order to travel to or from that work place he undertakes a journey in respect of which—
\begin{enumerate}\item[]
($a$) the travelling costs are borne wholly or in part by the relevant employer; or

($b$) the relevant employer provides transport for any part of the journey for the use of the relevant person,
\end{enumerate}
and where he attends at more than one work place on the same day that day shall be disregarded only if the condition specified in this sub-paragraph is satisfied in respect of all the work places at which he attends on that day,

\medskip

9.  There shall be calculated, or if that is impracticable, estimated—
\begin{enumerate}\item[]
($a$) the straight-line distances between the relevant person’s home and each work place; and

($b$) the straight-line distances between each of the relevant person’s work places, other than those between which he does not ordinarily travel.
\end{enumerate}

\medskip

10.  Subject to paragraph 11, there shall be calculated for each pair of places referred to in paragraph 9 the number of journeys which the relevant person makes between them during a period comprising a whole number of weeks which appears to the child support officer to be representative of the normal working pattern of the relevant person.

\medskip

11.  For the purposes of the calculation required by paragraph 10 there shall be disregarded—
\begin{enumerate}\item[]
($a$) any pair of journeys between the same work place and the relevant person’s home where the first journey is from his work place to his home and the time which elapses between the start of the first journey and the conclusion of the second is not more than two hours; and

($b$) any journey in respect of which—
\begin{enumerate}\item[]
(i) the travelling costs are borne wholly or in part by the relevant employer; or

(ii) the relevant employer provides transport for any part of the journey for the use of the relevant person.
\end{enumerate}
\end{enumerate}

\medskip

12.  The result of the calculation of the number of journeys made between each pair of places required by paragraph 10 shall be multiplied by the result of the calculation or estimate of the straight-line distance between them required by paragraph 9.

\medskip

13.  All the products of the multiplications required by paragraph 12 shall be added together and the resulting sum divided by the number of weeks in the period.

\medskip

14.  Where the result of the division required by paragraph 13 is less than or equal to 150, the amount to be allowed in respect of travelling costs shall be nil, and where it is greater than 150, the weekly allowance to be made in respect of the relevant person’s travelling costs shall be 10 pence multiplied by the number by which that number exceeds 150.

\subsection*{Computation in cases where there is more than one relevant employment}

15.—(1) Subject to sub-paragraph (2) and paragraphs 21 to 23, where the relevant person has more than one relevant employment the amount to be allowed in respect of travelling costs for the purposes of regulations 9 and 11 shall be determined in accordance with paragraphs 16 to 20.

(2) Where it appears that in respect of any of his relevant employments, whilst the relevant person works at more than one work place, his pattern or work is not sufficiently regular to enable the calculations of the amount to be allowed in respect of his travelling costs to be made readily, the child support officer—
\begin{enumerate}\item[]
($a$) may select a place which is either one of the relevant person’s work places in that relevant employment or some other place which is connected with that relevant employment;

($b$) may calculate the weekly average distance travelled in the course of his journeys made in connection with the relevant employment upon the basis that—
\begin{enumerate}\item[]
(i) the relevant person makes one journey from his home, or from another work place or deemed work place in another relevant employment, to the deemed work place and one journey from the deemed work place to his home, or to another work place or deemed work place in another relevant employment, on each day on which he attends at a work place in connection with the relevant employment in relation to which the deemed work place has been selected, and

(ii) the distance he travels between those places is the straight-line distance between them; and
\end{enumerate}

($c$) shall disregard any journeys made between work places in the relevant employment in respect of which a deemed work place has been selected.
\end{enumerate}

(3) For the purposes of sub-paragraph (2)($b$) there shall be disregarded any day upon which the relevant person attends at a work place and in order to travel to or from that work place he undertakes a journey in respect of which—
\begin{enumerate}\item[]
($a$) the travelling costs are borne wholly or in part by the relevant employer; or

($b$) the relevant employer provides transport for any part of the journey for the use of the relevant person,
\end{enumerate}
and where in the course of the particular relevant employment he attends at more than one work place on the same day, that day shall be disregarded only if the condition specified in this paragraph is satisfied in respect of all the work places at which he attends on that day in the course of that employment.

\medskip

16.  There shall be calculated, or if that is impracticable, estimated—
\begin{enumerate}\item[]
($a$) the straight-line distances between the relevant person’s home and each work place; and

($b$) the straight-line distances between each of the relevant person’s work places, except—
\begin{enumerate}\item[]
(i) those between which he does not ordinarily travel, and

(ii) those for which a calculation of the distance from the relevant person’s home is not required by virtue of paragraph 15($c$).
\end{enumerate}
\end{enumerate}

\medskip

%17.  There shall be calculated, or if that is impracticable, estimated for each pair of places referred to in paragraph 16 between which straight-line distances are required to be calculated or estimated the number of journeys which the relevant person makes between them during a period comprising a whole number of weeks which appears to the child support officer to be representative of the normal working pattern of the relevant person, thee being disregarded any pair of journeys between the same work place and his home where the first journey is from his work place to his home and the time which elapses between the start of the first journey and the conclusion of the second is not more than two hours.

% Paras 17, 17A substituted for para 17 (22.1.96) by SI 1995/3261 reg 49

17.  Subject to paragraph 17A, there shall be calculated, or if that is impracticable estimated, for each pair of places referred to in paragraph 16 between which straight-line distances are required to be calculated or estimated, the number of journeys which the relevant person makes between them during a period comprising a whole number of weeks which appears to the child support officer to be representative of the normal working pattern of the relevant person.

\amendment{
Para. 17 substituted (22.1.96) by the Child Support (Miscellaneous Amendments) (No. 2) Regulations 1995 reg. 49 (subject to transitional provisions in reg. 57(3)).
}

\medskip

17A.  For the purposes of the calculation required by paragraph 17, there shall be disregarded—
\begin{enumerate}\item[]
($a$) any pair of journeys between the same work place and his home where the first journey is from his work place to his home and the time which elapses between the start of the first journey and the conclusion of the second is not more than two hours; and

($b$) any journey in respect of which—
\begin{enumerate}\item[]
(i) the travelling costs are borne wholly or in part by the relevant employer; or

(ii) the relevant employer provides transport for any part of the journey for the use of the relevant person.
\end{enumerate}
\end{enumerate}

\amendment{
Para. 17A substituted for para. 17 (22.1.96) by the Child Support (Miscellaneous Amendments) (No. 2) Regulations 1995 reg. 49 (subject to transitional provisions in reg. 57(3)).
}

\medskip

18.  The result of the calculation or estimate of the number of journeys made between each pair of places required by paragraph 17 shall be multiplied by the result of the calculation or estimate of the straight-line distance between them required by paragraph 16.

\medskip

19.  All the products of the multiplications required by paragraph 18, shall be added together and the resulting sum divided by the number of weeks in the period.

\medskip

20.  Where the result of the division required by paragraph 19, plus where appropriate the result of the calculation required by paragraph 15 in respect of a relevant employment in which a deemed work place has been selected, is less than or equal to 150 the amount to be allowed in respect of travelling costs shall be nil, and where it is greater than 150, the weekly allowance to be made in respect of the relevant person’s travelling costs shall be 10 pence multiplied by the number by which that number exceeds 150.

\subsection*{Relevant employments in respect of which no amount is to be allowed}

21.—(1) No allowance shall be made in respect of travelling costs in respect of journeys between the relevant person’s home and his work place or between his work place and his home in a particular relevant employment if the condition set out in paragraph 22 or 23 is satisfied in respect of that employment.

(2) The condition mentioned in paragraph 22, or as the case may be 23, is satisfied in relation to a case where the relevant person has more than one work place in a relevant employment only where the employer provides assistance of the kind mentioned in that paragraph in respect of all of the work places to or from which the relevant person travels in the course of that employment, but those journeys in respect of which that assistance is provided shall be disregarded in computing the total distance travelled by the relevant person in the course of the relevant employment.

\medskip

22.  The condition is that the relevant employer provides transport of any description in connection with the employment which is available to the relevant person for any part of the journey between his home and his work place or between his work place or between his work place and his home.

\medskip

23.  The condition is that the relevant employer bears any part of the travelling costs arising from the relevant person travelling between his home and his work place or between his work place and his home in connection with that employment, and for the purposes of this paragraph he does not bear any part of that cost where he does no more than—
\begin{enumerate}\item[]
($a$) make a payment to the relevant person which would fail to be taken into account in determining the amount of the relevant person’s net income;

($b$) make a loan to the relevant person;

($c$) pay to the relevant person an increased amount of remuneration,
\end{enumerate}
to enable the relevant person to meet those costs himself.

\part[Schedule 4 --- Cases where child support maintenance is not to be payable]{Schedule 4\\*Cases where child support maintenance is not to be payable}

\renewcommand\parthead{--- Schedule 4}

The payments and awards specified for the purposes of regulation 26(1)($b$)(i) are—
\begin{enumerate}\item[]
($a$) the following payments under the Contributions and Benefits Act—
\begin{enumerate}\item[]
%(i) sickness benefit under section 31;
%
%(ii) invalidity pension under section 33;
%
%(iii) invalidity pension for widowers under section 34;

% Para ($a$)(i)--(iii) substituted (13.4.95) by SI 1995/1045 reg 58($a$)
(i) incapacity benefit under section 30A\footnote{\frenchspacing Section 30A was inserted by section 1 of, and sections 40 and 41 substituted by paragraphs 8 and 9 of Schedule 1 to, the Social Security (Incapacity for Work) Act 1994 (c. 18).};

(ii) long-term incapacity benefit for widows under section 40;

(iii) long-term incapacity benefit for widowers under section 41;

(iv) maternity allowance under section 35;

%(v) invalidity pension for widows under section 40; % Para ($a$)(v) omitted (13.4.95) by SI 1995/1045 reg 58($b$)

(vi) attendance allowance under section 64;

(vii) severe disablement allowance under section 68;

(viii) invalid care allowance under section 70;

(ix) disability living allowance under section 71;

(x) disablement benefit under section 103;

(xi) disability working allowance under section 129;

(xii) statutory sick pay within the meaning of section 151;

(xiii) statutory maternity pay within the meaning of section 164;
\end{enumerate}

($b$) awards in respect of disablement made under (or under provisions analogous to)—
\begin{enumerate}\item[]
(i) the War Pensions (Coastguards) Scheme 1944\footnote{\frenchspacing S.I. 1944/500.};

(ii) the War Pensions (Naval Auxiliary Personnel) Scheme 1964\footnote{\frenchspacing S.I. 1964/1985.};

(iii) the Pensions (Polish Forces) Scheme 1964\footnote{\frenchspacing S.I. 1964/2007.};

(iv) the War Pensions (Mercantile Marine) Scheme 1964\footnote{\frenchspacing S.I. 1964/2058.};

(v) the Royal Warrant of 21st December 1964 (service in the Home Guard before 1945)\footnote{\frenchspacing Cmnd. 2563.};

(vi) the Order by Her Majesty of 22nd December 1964 concerning pensions and other grants in respect of disablement or death due to service in the Home Guard after 27th April 1952\footnote{\frenchspacing Cmnd. 2564.};

(vii) the Order by Her Majesty (Ulster Defence Regiment) of 4th January 1971\footnote{\frenchspacing Cmnd. 4567.};

(viii) the Personal Injuries (Civilians) Scheme 1983\footnote{\frenchspacing S.I. 1983/686.};

(ix) the Naval, Military and Air Forces Etc.\ (Disablement and Death) Service Pensions Order 1983\footnote{\frenchspacing S.I. 1983/883.}; and
\end{enumerate}

($c$) payments from 
%the Independent Living Fund.
the Independent Living (1993) Fund or the Independent Living (Extension) Fund. % Words substituted (5.4.93) by SI 1993/913 reg 34
\end{enumerate}

\amendment{
Words substituted in head ($c$) (5.4.93) by the Child Support (Miscellaneous Amendments) Regulations 1993 reg. 34.

Para. ($a$)(i)--(iii) and para. ($a$)(v) omitted (13.4.95) by the Child Support and Income Support (Amendment) Regulations 1995 reg. 58.
}

%Sch 5 inserted (5.4.93) by SI 1993/913 reg. 26(3), Sch.

\part[Schedule 5 --- Provisions applying to cases to which section 43 of the Act and regulation 28 apply]{Schedule 5\\*Provisions applying to cases to which section 43 of the Act and regulation 28 apply}

\renewcommand\parthead{--- Schedule 5}

\amendment{
Sch. 5 inserted (5.4.93) by the Child Support (Miscellaneous Amendments) Regulations 1993 reg. 26(3) and Sch.
}
\medskip

%1.  In this Schedule “relevant decision” means a decision of a child support officer given under section 43 of the Act (contribution to maintenance by deduction from benefit) and regulation 28.
% Para 1 substituted (26.4.93) by SI 1993/925 reg 2(3)(i)
1.  In this Schedule—
\begin{enumerate}\item[]
($a$) “relevant decision” means a decision of a child support officer given under section 43 of the Act (contribution to maintenance by deduction from benefit) and regulation 28; and

($b$) “relevant person” has the same meaning as in regulation 1(2) of the Maintenance Assessment Procedure Regulations.
\end{enumerate}

\amendment{
Para. 1 substituted (26.4.93) by the Child Support (Maintenance Assessments and Special Cases) Amendment Regulations 1993 reg. 2(3)(i).
}

\medskip

%2.  A relevant decision may be reviewed by a child support officer, either on application by a relevant person or of his own motion, if it appears to him that the absent parent has at some time after that decision was given satisfied the conditions prescribed by regulation 28(1) or, as the case may be, no longer satisfies those conditions.

% Para 2 substituted (18.4.95) by SI 1995/1045 reg 59(2)
2.  A relevant decision may be reviewed by a child support officer, either on application by a relevant person or of his own motion—
\begin{enumerate}\item[]
($a$) if it appears to him that the absent parent has at some time after that decision was given satisfied the conditions prescribed by regulation 28(1) or, as the case may be, no longer satisfies those conditions; or

($b$) if it appears to him that the relevant decision was wrong in law or was made in ignorance of, or based on a mistake as to, a material fact.
\end{enumerate}

\amendment{
Para. 2 substituted (18.4.95) by the Child Support and Income Support (Amendment) Regulations 1995 reg. 59(2).
}

\medskip

3.  A relevant decision 
made on or before 18th April 1994  % Words inserted (18.4.95) by SI 1995/1045 reg 59(3)
shall be reviewed by a child support officer 
%when 
after  % Word substituted (18.4.95) by SI 1995/1045 reg 59(3)
it has been in force for 52 weeks.

\amendment{
Words inserted and substituted in para. 3 (18.4.95) by the Child Support and Income Support (Amendment) Regulations 1995 reg. 59(3).
}

\medskip

% Para 3A inserted (18.4.95) by SI 1995/1045 reg 59(4)
3A.  A relevant decision made after 18th April 1994 shall be reviewed by a child support officer after it has been in force for 104 weeks.

\amendment{
Para. 3A inserted (18.4.95) by the Child Support and Income Support (Amendment) Regulations 1995 reg. 59(4).
}

\medskip

4.—(1) Before conducting a review under paragraph 6 the child support officer shall---
\begin{enumerate}\item[]
($a$) give 14 days' notice of the proposed review to the relevant persons%
%(within the meaning of regulation 1(2) of the Maintenance Assessment Procedure Regulations) % Words omitted (26.4.93) by SI 1993/925 reg. 2(3)(ii).
; and

($b$) invite representations, either in person or in writing, from the relevant persons on any matter relating to the review and set out the provisions of sub-paragraphs (2) to (4) in relation to such representations.
\end{enumerate}

(2) Subject to sub-paragraph (3), where the child support officer conducting the review does not, within 14 days of the date on which notice of the review was given, receive a request from a relevant person to make representations in person, or receives such a request and arranges for an appointment for such representations to be made but that appointment is not kept, he may complete the review in the absence of such representations from that person.

(3) Where the child support officer conducting the review is satisfied that there was good reason for failure to keep an appointment, he shall provide for a further opportunity for the making of representations by the relevant person concerned before he completes the review.

(4) Where the child support officer conducting the review does not receive written representations from a relevant person within 14 days of the date on which notice of the review was given, he may complete the review in the absence of written representations from that person.

\amendment{
Words omitted in para. 4(1)($a$) (26.4.93) by the Child Support (Maintenance Assessments and Special Cases) Amendment Regulations 1993 reg. 2(3)(ii).
}

\medskip

5.  After completing a review under paragraph 2, 3 or 6, the child support officer shall notify all relevant persons of the result of the review and---
\begin{enumerate}\item[]
($a$) in the case of a review under paragraph 2 or 3, of the right to apply for a further review under paragraph 6; and

($b$) in the case of a review under 
%that paragraph, 
paragraph 6, % Words substituted by SI 1993/925 reg 2(3)(iii)
of the right of appeal under section 20 of the Act as applied by paragraph 8.
\end{enumerate}

\amendment{
Words substituted in para. 5($b$) (26.4.93) by the Child Support (Maintenance Assessments and Special Cases) Amendment Regulations 1993 reg. 2(3)(iii).
}

\medskip

6.  Where a child support officer has made a decision under regulation 28 or paragraph 2 or 3, any relevant person may apply to the Secretary of State for a review of that decision and, subject to the modifications set out in paragraph 7, the provisions of section 18(5) to (7) of the Act shall apply to such a review.

\medskip

7.  The modifications to the provisions of section 18(5) to (7) of the Act referred to in paragraph 6 are---
\begin{enumerate}\item[]
($a$) any reference in those provisions to a maintenance assessment shall be read as a reference to a relevant decision; and

($b$) subsection (6) shall apply as if the reference to the cancellation of an assessment was omitted.
\end{enumerate}

\medskip

%Para 7A inserted (26.4.93) by SI 1993/925 reg. 2(3)(iv).

7A.  If, on a review under paragraph 2, 3, or 6, the relevant decision is revised (“the revised decision”) the revised decision shall have effect—
\begin{enumerate}\item[]
($a$) if the revised decision is that no payments such as are mentioned in section 43 of the Act are to be made, from the date on which the event giving rise to the review occurred; or

($b$) if the revised decision is that such payments are to be made, from the date on which the revised decision is given.
\end{enumerate}

\amendment{
Para. 7A inserted (26.4.93) by the Child Support (Maintenance Assessments and Special Cases) Amendment Regulations 1993 reg. 2(3)(iv).
}

\medskip

8.  The provisions of section 20 of the Act (appeals) shall apply in relation to a review or a refusal to review under paragraph 6.

\medskip

9.  The provisions of paragraphs (1) and (2) of regulation 5 of the Child Support (Collection and Enforcement) Regulations 1992\footnote{\frenchspacing  S.I. 1992/1989.} shall apply to the transmission of payments in place of payments of child support maintenance under section 43 of the Act and regulation 28 as they apply to the transmission of payments of child support maintenance.

\part{Explanatory Note}

\renewcommand\parthead{--- Explanatory Note}

\subsection*{(This note is not part of the Regulations)}

These Regulations provide for various matters relating to the calculation of child support maintenance under the Child Support Act 1991 (“the Act”) and also make provision for special cases under the Act.

Regulation 1 contains interpretation provisions. Regulation 2 contains general provisions regarding calculations under the Act.

Regulation 3 prescribes the amounts which are to be taken into account in the maintenance requirement formula in paragraph 1 of Schedule 1 to the Act. Regulation 4 defines the “basic rate” of child benefit for the purposes of that paragraph.

Regulation 5 prescribes values for the general rule formula in paragraph 2 of Schedule 1 to the Act. Regulation 6 prescribes an amount and a value for the purposes of the additional element formula in paragraph 4 of that Schedule.

Regulation 7 and 8 and Schedules 1 and 2 prescribe the amounts to be taken into account as assessable income for the purposes of paragraph 5 of Schedule 1 to the Act. Regulations 9 and 10 prsecribe the exempt income of the absent parent and the parent with care for the purposes of that paragraph.

Regulation 11 prescribes the protected income level of that absent parent for the purposes of paragraph 6 of Schedule 1 to the Act. Regulation 12 provides for the calculation of the disposable income of the absent parent for the purposes of that paragraph.

Regulation 13 prescribes the minimum amount of child support maintenance for the purposes of paragraph 7 of Schedule 1 to the Act.

Regulations 19 to 27 and Schedule 4 prescribe the circumstances in which cases are to be treated as special cases for the purposes of the Act. These include cases where both parents are absent; where more than one application for child support maintenance is made in relation to the same absent where care arrangements are shared and for child support maintenance not to be payable in certain circumstances.

Regulation 28 makes provision for the amount payable where the absent parent is in receipt of income support or other prescribed benefit. 
\end{document}