\documentclass[12pt,a4paper]{article}

\newcommand\regstitle{The Social Security (Jobcentre Plus Interviews for Partners) Regulations 2003}

\newcommand\regsnumber{2003/1886}

%\opt{newrules}{
\title{\regstitle}
%}

%\opt{2012rules}{
%\title{Child Maintenance and~Other Payments Act 2008\\(2012 scheme version)}
%}

\author{S.I.\ 2003 No.\ 1886}

\date{Made
17th July 2003\\
%Laid before Parliament
%15th October 2008\\
Coming into~force
12th April 2004
}

%\opt{oldrules}{\newcommand\versionyear{1993}}
%\opt{newrules}{\newcommand\versionyear{2003}}
%\opt{2012rules}{\newcommand\versionyear{2012}}

\usepackage{csa-regs}

\setlength\headheight{42.11603pt}

%\hbadness=10000

\begin{document}

\maketitle

\noindent
Whereas a draft of this instrument was laid before Parliament in accordance with section 190(1) of the Social Security Administration Act 1992\footnote{1992 c.~5.} and approved by resolution of each House of Parliament;

Now, therefore, the Secretary of State for Work and Pensions, in exercise of the powers conferred upon him by sections 2AA(1) and (4) to (7), 2B(6),~189(1) and (4) to (6) and 191 of the Social Security Administration Act 1992\footnote{Section~2AA was inserted by section 49 of the Employment Act 2002 (c.~22), and section 2B was inserted by section 57 of the Welfare Reform and Pensions Act 1999 (c.~30) and amended by section 53 of, and paragraphs 8 and 9 of Schedule 7 to, and section 54 of, and Schedule 8 to, the Employment Act 2002; section 191 is an interpretation provision and is cited because of the meaning ascribed to the word “prescribe”. Section~2AA(7) is cited because of the meaning ascribed to the words “specified” and “work-focused interview”.} and of all other powers enabling him in that behalf, after consultation with the Council on Tribunals in accordance with section 8(1) of the Tribunals and Inquiries Act 1992\footnote{1992 c.~53.}, by this instrument, which contains only regulations made by virtue of, or consequential upon, section 2AA of the Social Security Administration Act 1992\footnote{\emph{See} section 173(5)($b$) of the Social Security Administration Act 1992 (c.~5).} and which is made before the end of the period of 6 months beginning with the coming into force of that provision, hereby makes the following Regulations: 

{\sloppy

\tableofcontents

}

\bigskip

\setcounter{secnumdepth}{-2}

\subsection[1. Citation and commencement]{Citation and commencement}

1.  These Regulations may be cited as the Social Security (Jobcentre Plus Interviews for Partners) Regulations 2003 and shall come into force on 12th April 2004.

\subsection[2. Interpretation and application]{Interpretation and application}

2.---(1)  In these Regulations—
\begin{enumerate}\item[]
“the 1998 Act” means the Social Security Act 1998\footnote{1998 c.~14.};

“benefit week” means any period of seven days corresponding to the week in respect of which the relevant specified benefit is due to be paid;

“claimant” means a claimant of a specified benefit who has a partner to whom these Regulations apply;

“interview” means a work-focused interview with a partner which is conducted for any or all of the following purposes—
\begin{enumerate}\item[]
($a$) 
assessing the partner’s prospects for existing or future employment (whether paid or voluntary);

($b$) 
assisting or encouraging the partner to enhance his prospects of such employment;

($c$) 
identifying activities which the partner may undertake to strengthen his existing or future prospects of employment;

($d$) 
identifying current or future employment or training opportunities suitable to the partner’s needs; and

($e$) 
identifying educational opportunities connected with the existing or future employment prospects or needs of the partner;
\end{enumerate}

“officer” means a person who is an officer of, or who is providing services to or exercising functions of, the Secretary of State;

“partner” means a person who is a member of the same couple as the claimant, or, in a case where the claimant has more than one partner, a person who is a partner of the claimant by reason of a polygamous marriage, but only where—
\begin{enumerate}\item[]
($a$) 
the claimant has been awarded a specified benefit at a higher rate referable to that partner; and

($b$) 
both the partner and the claimant have attained the age of 18 but have not attained the age of 60;
\end{enumerate}

“polygamous marriage” means any marriage during the subsistence of which a party to it is married to more than one person and the ceremony of marriage took place under the law of a country which permits polygamy;

“specified benefit” means a benefit to which section 2AA applies.
\end{enumerate}

(2) Regulations 3 to 13 apply to a partner in circumstances where on or after 12th April 2004 the claimant’s award of a specified benefit is being administered from an office of the Department for Work and Pensions which is designated by the Secretary of State as a Jobcentre Plus Office\footnote{A list of offices designated as Jobcentre Plus Offices is available from the Department for Work and Pensions at the following address: Jobcentre Plus Implementation Project, Jobcentre Plus Head Office, Level 1, Steel City House, West Street, Sheffield, \textsc{\lowercase{S1 2GQ}}.} and the claimant has been continuously entitled to the benefit for 26 weeks or longer.

\subsection[3. Requirement for partner to take part in an interview as a condition of a specified benefit continuing to be paid at full amount]{Requirement for partner to take part in an interview as a condition of a specified benefit continuing to be paid at full amount}

3.---(1)  Subject to regulations 5 to 8, a partner to whom these Regulations apply is required to take part in an interview as a condition of the claimant continuing to be paid the full amount of a specified benefit which is payable apart from these Regulations.

(2) Where a requirement to take part in an interview arises under paragraph (1), a requirement to take part in an interview shall also apply to any other specified benefit in payment to the claimant at a higher rate referable to his partner on the date set for the interview and notified to the partner in accordance with regulation~9(1).

\subsection[4. Time when interview is to take place]{Time when interview is to take place}

4.  An officer shall arrange for an interview to take place as soon as reasonably practicable after—
\begin{enumerate}\item[]
($a$) the requirement under regulation~3(1) arises; or

($b$) in a case where regulation~6(1) applies, the time when that requirement is to apply by virtue of regulation~6(2).
\end{enumerate}

\subsection[5. Waiver of requirement to take part in an interview]{Waiver of requirement to take part in an interview}

5.---(1)  A requirement imposed by these Regulations to take part in an interview shall not apply where an officer determines that an interview would not—
\begin{enumerate}\item[]
($a$) be of assistance to the partner concerned; or

($b$) be appropriate in the circumstances.
\end{enumerate}

(2) A partner in relation to whom a requirement to take part in an interview has been waived under paragraph (1) shall be treated for the purposes of regulation~3 as having complied with that requirement.

\subsection[6. Deferment of requirement to take part in an interview]{Deferment of requirement to take part in an interview}

6.---(1)  An officer may determine, in the case of any particular partner, that the requirement to take part in an interview shall be deferred at the time that the requirement to take part in it arises or applies because an interview would not at that time—
\begin{enumerate}\item[]
($a$) be of assistance to the partner concerned; or

($b$) be appropriate in the circumstances.
\end{enumerate}

(2) Where the officer determines in accordance with paragraph (1) that the requirement to take part in an interview shall be deferred, he shall also, when that determination is made, determine the time when the requirement to take part in an interview is to apply in the partner’s case.

(3) Where a requirement to take part in an interview has been deferred in accordance with paragraph (1), then until—
\begin{enumerate}\item[]
($a$) a determination is made under regulation~5(1);

($b$) the partner takes part in an interview; or

($c$) a relevant decision has been made in accordance with regulation~10(3),
\end{enumerate}
the partner shall be treated for the purposes of regulation~3 as having complied with that requirement.

\subsection[7. Exemption]{Exemption}

7.  A partner who, on the day on which the requirement to take part in an interview arises or applies under regulation~3(1) or 6(2), is in receipt of a specified benefit as a claimant in his own right shall be exempt from the requirement to take part in an interview under these Regulations.

\subsection[8. Claims for two or more specified benefits]{Claims for two or more specified benefits}

8.  A partner who would otherwise be required under these Regulations to take part in interviews relating to more than one specified benefit—
\begin{enumerate}\item[]
($a$) is only required to take part in one interview during any period where the claimant is in receipt of two or more specified benefits concurrently; and

($b$) that interview counts for the purposes of each of those benefits.
\end{enumerate}

\subsection[9. The interview]{The interview}

9.---(1)  An officer shall inform a partner who is required to take part in an interview of the date, place and time of the interview.

(2) The officer may determine that an interview is to take place in the partner’s home where it would, in his opinion, be unreasonable to expect the partner to attend elsewhere because the partner’s personal circumstances are such that attending elsewhere would cause him undue inconvenience or endanger his health.

(3) An officer shall conduct the interview.

\subsection[10. Taking part in an interview]{Taking part in an interview}

10.---(1)  The officer shall determine whether a partner has taken part in an interview.

(2) A partner shall be regarded as having taken part in an interview if and only if—
\begin{enumerate}\item[]
($a$) he attends for the interview at the place and time notified to him by the officer; and

($b$) he provides answers (where asked) to questions and appropriate information about—
\begin{enumerate}\item[]
(i) the level to which he has pursued any educational qualifications;

(ii) his employment history;

(iii) any vocational training he has undertaken;

(iv) any skills he has acquired which fit him for employment;

(v) any paid or unpaid employment he is engaged in;

(vi) any medical condition which, in his opinion, puts him at a disadvantage in obtaining employment; and

(vii) any caring or childcare responsibilities he has.
\end{enumerate}
\end{enumerate}

(3) Where an officer determines that a partner has failed to take part in an interview and good cause has not been shown either by the partner or by the claimant for that failure within five working days of the day on which the interview was to take place, a relevant decision shall be made for the purposes of section 2B of the Social Security Administration Act 1992\footnote{1992 c.~5; section 2B was inserted by section 57 of the Welfare Reform and Pensions Act 1999 (c.~30) and amended by section 53 of, and paragraphs 8 and 9 of Schedule 7 to, and section 54 of, and Schedule 8 to, the Employment Act 2002 (c.~22).} and the partner and the claimant shall be notified accordingly.

\subsection[11. Failure to take part in an interview]{Failure to take part in an interview}

11.---(1)  Where a relevant decision has been made in accordance with regulation~10(3), subject to paragraph (11), the specified benefit payable to the claimant in respect of which the requirement for the partner to take part in an interview under regulation~3 arose shall be reduced, either as from the first day of the next benefit week following the day on which the relevant decision was made, or, if that date arises five days or less after the day on which the relevant decision was made, as from the first day of the second benefit week following the date of the relevant decision.

(2) The deduction made to benefit in accordance with paragraph (1) shall be by a sum equal (but subject to paragraphs (3) and (4)) to 20 per cent.\ of the amount applicable on the date the deduction commences in respect of a single claimant for income support aged not less than 25.

(3) Benefit reduced in accordance with paragraph (1) shall not be reduced below ten pence per week.

(4) Where two or more specified benefits are in payment to a claimant, in relation to each of which a requirement for the partner to take part in an interview had arisen under regulation~3, a deduction made in accordance with this regulation~shall be applied, except in a case to which paragraph (5) applies, to those benefits in the following order of priority—
\begin{enumerate}\item[]
($a$) an income-based jobseeker’s allowance;

($b$) income support;

($c$) incapacity benefit;

($d$) severe disablement allowance;

($e$) carer’s allowance.
\end{enumerate}

(5) Where the amount of the reduction is greater than some (but not all) of those benefits, the reduction shall be made against the first benefit in the list in paragraph (4) which is the same as, or greater than, the amount of the reduction.

(6) For the purpose of determining whether a benefit is the same as, or greater than, the amount of the reduction for the purposes of paragraph (5), ten pence shall be added to the amount of the reduction.

(7) In a case where the whole of the reduction cannot be applied against any one benefit because no one benefit is the same as, or greater than, the amount of the reduction, the reduction shall be applied against the first benefit in the list of priorities at paragraph (4) and so on against each benefit in turn until the whole of the reduction is exhausted or, if this is not possible, the whole of those benefits are exhausted, subject in each case to ten pence remaining in payment.

(8) Where the rate of any specified benefit payable to a claimant changes, the rules set out above for a reduction in the benefit payable shall be applied to the new rates and any adjustments to the benefits against which the reductions are made shall take effect from the beginning of the first benefit week to commence for that claimant following the change.

(9) Where the partner of a claimant whose benefit has been reduced in accordance with this regulation~subsequently takes part in an interview, the reduction shall cease to have effect on the first day of the benefit week in which the requirement to take part in an interview was met.

(10) For the purposes of determining the amount of any benefit payable, a claimant shall be treated as receiving the amount of any specified benefit which would have been payable but for a reduction made in accordance with this regulation.

(11) Benefit shall not be reduced in accordance with this regulation~where the partner or the claimant brings new facts to the notice of the Secretary of State within one month of the date on which the decision that the partner failed without good cause to take part in an interview was notified and—
\begin{enumerate}\item[]
($a$) those facts could not reasonably have been brought to the Secretary of State’s notice within five working days of the day on which the interview was to take place; and

($b$) those facts show that he had good cause for his failure to take part in the interview.
\end{enumerate}

\subsection[12. Circumstances where regulation~11 does not apply]{Circumstances where regulation~11 does not apply}

12.  The reduction of benefit to be made under regulation~11 shall not apply as from the date when a partner who failed to take part in an interview ceases to be a partner for the purposes of these Regulations or is no longer a partner to whom these Regulations apply under the provisions within regulation~2(2).

\subsection[13. Good cause]{Good cause}

13.  Matters to be taken into account in determining whether the partner or the claimant has shown good cause for the partner’s failure to take part in an interview include—
\begin{enumerate}\item[]
($a$) that the partner misunderstood the requirement to take part in an interview due to any learning, language or literacy difficulties of the partner or any misleading information given to the partner by the officer;

($b$) that the partner was attending a medical or dental appointment, or accompanying a person for whom the partner had caring responsibilities to such an appointment, and that it would have been unreasonable, in the circumstances, to rearrange the appointment;

($c$) that the partner had difficulties with his normal mode of transport and that no reasonable alternative was available;

($d$) that the established customs and practices of the religion to which the partner belongs prevented him attending on that day or at that time;

($e$) that the partner was attending an interview with an employer with a view to obtaining employment;

($f$) that the partner was actually pursuing employment opportunities as a self-employed earner;

($g$) that the partner, claimant or a dependant or a person for whom the partner provides care suffered an accident, sudden illness or relapse of a chronic condition;

($h$) that he was attending the funeral of a close friend or relative on the day fixed for the interview;

($i$) that a disability from which the partner suffers made it impracticable for him to attend at the time fixed for the interview.
\end{enumerate}

\subsection[14. Appeals]{Appeals}

14.---(1)  This regulation~applies to any relevant decision made under regulation~10(3) or any decision under section 10 of the 1998 Act superseding such a decision.

(2) This regulation~applies whether the decision is as originally made or as revised under section 9 of the 1998 Act.

(3) In the case of a decision to which this regulation~applies, the partner in respect of whom the decision was made and the claimant shall each have a right of appeal under section 12 of the 1998 Act to an appeal tribunal.

\subsection[15. Amendments to the Social Security and Child Support (Decisions and Appeals) Regulations 1999]{Amendments to the Social Security and Child Support (Decisions and Appeals) Regulations 1999}

15.---(1)  The Social Security and Child Support (Decisions and Appeals) Regulations 1999\footnote{S.I.~1999/991; relevant amending instruments are S.I.~1999/2570, 2000/897, 2000/3185 and 2002/1703.} shall be amended in accordance with this regulation.

(2) In paragraph (3) of regulation~1 (citation, commencement and interpretation)—
\begin{enumerate}\item[]
($a$) in paragraph ($b$)  of the definition of “party to the proceedings” before the words “or section 12(2)” there shall be inserted the words “,~section 2B(6) of the Administration Act\footnote{1992 c.~5; section 2B was inserted by section 57 of the Welfare Reform and Pensions Act 1999 (c.~30) and amended by section 53 of, and paragraphs 8 and 9 of Schedule 7 to, and section 54 of, and Schedule 8 to, the Employment Act 2002 (c.~22).}”;

($b$) at the end of the definition of “work-focused interview” there shall be added the words “or under the Social Security (Jobcentre Plus Interviews for Partners) Regulations 2003”.
\end{enumerate}

(3) In paragraph (6A) of regulation~3 after the words “section 2B(2)” there shall be inserted the words “or (2A)”.

(4) In paragraph (2) of regulation~6 at the end of sub-paragraph ($h$)(ii)  there shall be added the words “or, in the case of a partner who was required to take part in a work-focused interview under the Social Security (Jobcentre Plus Interviews for Partners) Regulations 2003, ceased to be a partner for the purposes of those Regulations or is no longer a partner to whom those Regulations apply”.

(5) For paragraph (25) of regulation~7 there shall be substituted the following paragraph—
\begin{quotation}
“(25) In a case where a decision (“the first decision”) has been made that a person failed without good cause to take part in a work-focused interview, the decision under section 10 shall take effect as from—
\begin{enumerate}\item[]
($a$) the first day of the benefit week to commence for that person following the date of the first decision; or

($b$) in a case where a partner has failed without good cause to take part in a work-focused interview under the Social Security (Jobcentre Plus Interviews for Partners) Regulations 2003—
\begin{enumerate}\item[]
(i) the first day of the benefit week to commence for the claimant (as defined in regulation~2(1) of those Regulations) following the date of the first decision; or

(ii) if that date arises five days or less after the day on which the first decision was made, as from the first day of the second benefit week to commence for the claimant following the date of the first decision.”. 
\end{enumerate}
\end{enumerate}
\end{quotation}

\bigskip

Signed 
by authority of the 
Secretary of State for~Work and~Pensions.
%I concur
%By authority of the Lord Chancellor

{\raggedleft
\emph{Des Browne}\\*
%Secretary
Minister
%Parliamentary Under-Secretary 
of State\\*Department 
for~Work and~Pensions
%Ministry of Justice
%Two of the Commissioners of Inland~Revenue

}

17th July 2003


\small

\part{Explanatory Note}

\renewcommand\parthead{— Explanatory Note}

\subsection*{(This note is not part of the Regulations)}

These Regulations impose a requirement on the partners of claimants of certain benefits to take part in a work-focused interview (“an interview”) where the claimant is entitled to that benefit at a higher rate referable to the partner.

Regulation~3 requires those partners to whom the Regulations apply to take part in an interview as a condition of the claimant continuing to be paid the full amount of benefit which would otherwise be payable.

Regulation~4 prescribes the time when the interview is to take place. Regulation~5 provides that the requirement to take part in an interview can be waived where an interview would not be of assistance to the partner or it would not be appropriate in the circumstances of the case and regulation~6 specifies that an interview can be deferred. Regulation~7 prescribes circumstances when a partner is exempted from the requirement to take part in an interview. Regulation~8 specifies when a requirement to take part in two or more interviews is satisfied by the partner taking part in a single interview.

Regulation~9 provides for the partner to be advised of the date, time and place of the interview and provides that an interview can take place in the partner’s home if it is considered that it would be unreasonable to require the partner to attend elsewhere.

Regulation~10 prescribes circumstances as to when a partner is to be regarded as having taken part in an interview and regulation~11 details the consequences of a failure to take part in an interview. Regulation~12 specifies the circumstances where those consequences do not apply and regulation~13 specifies matters to be taken into account in determining whether a partner had good cause for his failure to take part in an interview.

Regulation~14 provides that a decision that a partner has failed to take part in an interview without good cause can be appealed to an appeal tribunal under section 12 of the Social Security Act 1998 (c.~14).

Regulation~15 makes consequential amendments to the Social Security and Child Support (Decisions and Appeals) Regulations 1999 (S.I.1999/991).

These Regulations do not impose a charge on business. 

\end{document}