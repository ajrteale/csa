\documentclass[12pt,a4paper]{article}

\newcommand\regstitle{The Welfare Reform Act 2007 (Commencement No.\ 6 and Consequential Provisions) Order 2008}

\newcommand\regsnumber{2008/787}

%\opt{newrules}{
\title{\regstitle}
%}

%\opt{2012rules}{
%\title{Child Maintenance and Other Payments Act 2008\\(2012 scheme version)}
%}

\author{S.I.\ 2008 No.\ 787 (C.~36)}

\date{Made
%17th March 2008\\
%Laid before Parliament
%4th March 2008\\
%Coming into force
17th March 2008
}

%\opt{oldrules}{\newcommand\versionyear{1993}}
%\opt{newrules}{\newcommand\versionyear{2003}}
%\opt{2012rules}{\newcommand\versionyear{2012}}

\usepackage{csa-regs}

\setlength\headheight{42.11603pt}

\hbadness=10000

\begin{document}

\maketitle

\noindent
The Secretary of State for Work and Pensions makes the following Order in exercise of the powers conferred by sections 68 and 70(2) of the Welfare Reform Act 2007\footnote{2007 c.\ 5}. 

{\sloppy

\tableofcontents

}

\bigskip

\setcounter{secnumdepth}{-2}

\subsection[1. Citation and interpretation]{Citation and interpretation}

1.---(1)  This Order may be cited as the Welfare Reform Act 2007 (Commencement No.\ 6 and Consequential Provisions) Order 2008.

(2) In this Order “the Act” means the Welfare Reform Act 2007.

\subsection[2. Commencement]{Commencement}

2.---(1)  For the purpose only of conferring power to make regulations, the day appointed for the coming into force of the provisions of the Act specified in the Schedule is 18th March 2008.

(2) 1st April 2008 is the day appointed for the coming into force of section~49 of the Act (loss of benefit for commission of benefit offences).

(3) Insofar as they are not already in force, 27th July 2008 is the day appointed for the coming into force of—
\begin{enumerate}\item[]
($a$) paragraphs 10(1), (2) and~(32) and 17(1), (2), (3) and~(6) of Schedule~3 to the Act and section 28(1) of the Act so far as it relates to those paragraphs; and

($b$) section 25(2)($b$)  of, and paragraphs 1, 2, 3($b$), ($c$)  and~($d$)  and 11 of Schedule~4 to, the Act and section 29 of the Act so far as it relates to those paragraphs.
\end{enumerate}

(4) Insofar as they are not already in force, 27th October 2008 is the day appointed for the coming into force of the following provisions of the Act—
\begin{enumerate}\item[]
($a$) sections 1 to 12, 14, 16 to 27, 28(2) and~(3);

($b$) section 28(1) so far as it relates to the provisions of Schedule 3 referred to in sub-paragraph ($f$);

($c$) section 67 so far as it relates to the repeals referred to in sub-paragraph ($g$);

($d$) Schedule~1;

($e$) Schedule 2;

($f$) paragraphs 1 to 4, 6, 7(1), (7) and~(8), 8, 9(1) to (4), (6) to (11) and~(13), 10 to 20 and 22 to 24 of Schedule 3; and

($g$) the repeals set out in Schedule 8 to the Act relating to—
\begin{enumerate}\item[]
(i) sections 6A(3) and 124(1) of the Social Security Contributions and Benefits Act 1992\footnote{1992 c.\ 4.};

(ii) section 73(4) of the Social Security Administration Act 1992\footnote{1992 c.\ 5.};

(iii) paragraphs 19(2) and~(3), 40(2) and 53(2) of Schedule 2 to the Jobseeker’s Act 1995\footnote{1995 c.\ 18.};

(iv) sections 2(2) and 28(3) of, paragraph 6($b$)  of Schedule 2 to, and paragraph 3 of Schedule 3 to, the Social Security Act 1998\footnote{1998 c.\ 14.}; and

(v) section 72(3) of the Welfare Reform and Pensions Act 1999\footnote{1999 c.\ 30.}.
\end{enumerate}
\end{enumerate}

\subsection[3. Consequential provisions]{Consequential provisions}

3.---(1)  The following amendments have effect as from 1st April 2008 (in consequence of the coming into force on that day of section 49 of the Act).

(2) In the Social Security and Child Support (Decisions and Appeals) Regulations 1999\footnote{S.I.\ 1999/991. Paragraph 27 of Schedule 2 was inserted by S.I.\ 2001/4022, regulation~21.}, in paragraph 27 of Schedule 2 (decisions against which no appeal lies), for “3 years” substitute “5 years”.

(3) In the Social Security (Loss of Benefit) Regulations 2001\footnote{S.I.\ 2001/4022. Regulation 2(2) was amended by S.I. 2002/486, regulation~2($d$).}, in regulation~2(2) (disqualification period), for “3 years” substitute “5 years”. 

\bigskip

Signed 
by authority of the 
Secretary of State for Work and Pensions.

{\raggedleft
\emph{Stephen C.\ Timms}\\*
Minister
%Parliamentary Under-Secretary 
of State,\\*Department for Work and Pensions

}

17th March 2008

\small

\part[Schedule --- Provisions brought into force on 18th March 2008 for regulation-making purposes]{Schedule\\*Provisions brought into force on 18th March 2008 for regulation-making purposes}

%\begin{tabulary}{\linewidth}{JJ}
\begin{longtable}{p{236.03096pt}p{129.95471pt}}
\hline
\itshape Provision of the Act 	& \itshape Subject-matter\\
\hline
\endhead
\hline
\endlastfoot
Section 2(1)($a$)  and~($c$)  and~(4)($a$)  and~($c$) 	&Prescribing the amount of a contributory allowance\\
Section 3(1)($c$)  and~(2)($b$)  and~($d$) 	&Deductions from contributory allowance: supplementary\\
Section 3(3) for the purposes of prescribing a payment under paragraph ($b$)  of the definition of “pension payment”	&Deductions from contributory allowance: supplementary\\
Section 4(2)($a$), (3) and~(6)($a$)  and~($c$) 	&Amount of income related allowance\\
Section 5(2) and~(3)	&Advance award of income-related allowance\\
Section 8 except for subsection (4)($c$) &	Limited capability for work\\
Section 9 except for subsection (4)($c$) &	Limited capability for work-related activity\\
Section 11(1), (2)($a$)  to ($g$), (3) to (5), (6)($a$)  and~(7)($c$) 	&Work-focused health-related assessments\\
Section 12(1), (2)($a$)  to ($h$)  and~(3) to (7)	&Work-focused interviews\\
Section 14(1) and~(2)($a$)  and~($b$) 	&Action plans in connection with work-focused interviews\\
Section 16(2)($a$)  and~(4)	&Contracting out\\
Section 17	&Income and capital: general\\
Section 18(1), (2) and~(4)	&Disqualification\\
Section 20(2) to (7)	&Relationship with statutory payments\\
Section 22 insofar as it relates to paragraphs 1 to~7, 8(1), 9, 10, 12 and 14 of Schedule 2	&Supplementary provisions\\
Section 23(1) and~(3)	&Recovery of sums in respect of maintenance\\
Section 24(1) for the purpose of defining “employed”, “employment”, “period of limited capability for work”, “prescribed”, “regulations” and “week”	&Interpretation of Part~I\\
Section 24(2)($b$)  and~(3)	&Interpretation of Part~I\\
Section 25(1) to (5)	&Regulations\\
Section 26(2)	&Parliamentary Control\\
Section 28(1) insofar as it relates to the provisions of Schedule 3 set out in this Schedule	&Consequential amendments relating to Part~I\\
Section 28(2)	&Consequential amendments relating to Part~I\\
Section 29 insofar as it relates to paragraphs 1(1), 2, 3($b$), ($c$)  and~($d$)  of Schedule~4	&Transition relating to Part~I\\
Paragraphs 1(4) and 3(2) of Schedule~1	&Conditions relating to national insurance\\
Paragraphs 4(1)($a$)  and~($c$), (3) and~(4) of Schedule~1	&Conditions relating to youth\\
Paragraphs 6(1)($b$), (2) to (4), (7) and~(8) of Schedule~1	&Employment and Support Allowance: additional conditions: income-related allowance\\
Paragraph 6(5) of Schedule~1 for the purposes of the meaning of “couple”, “education” and “remunerative work”	&Employment and Support Allowance: additional conditions: income-related allowance\\
Paragraphs 1 to~7, 8(1), 9, 10, 12 and 14 of Schedule 2	&Employment and Support Allowance: Supplementary Provisions\\
Paragraphs 7(1) and~(8), 8, 9(1), (3)($b$)  and~(4), 10(1), (2), (3) to (8), (12) and~(32), 12(1) and~(5), 15, 17(1), (3), (6) and~(7), 18, 20, 23(1) to (5) and~(8) and 24 of Schedule 3	&Consequential amendments relating to Part~I\\
Paragraphs 1(1), 2, 3($b$), ($c$)  and~($d$)  of Schedule~4	&Transition relating to Part I\\
%\end{tabulary}
\end{longtable}

\part{Explanatory Note}

\renewcommand\parthead{— Explanatory Note}

\subsection*{(This note is not part of the Order)}

This Order makes further provision for the coming into force of the Welfare Reform Act 2007. In particular, it makes provision for the coming into force of regulation-making powers required to make the Employment and Support Allowance Regulations 2008.

Article 2 provides that the day appointed for the coming into force of the powers listed in Schedule~1 to the Order, for the purposes of conferring power to make regulations is 18th March 2008. It further provides that—

    1st April 2008 is the day appointed for the coming into force of section 49 of the Act (loss of benefit for commission of benefit offences);

    27th July 2008 is the day appointed for the coming into force of section 25(2)($b$)  and section 29 (partially) of the Act, of paragraphs 10(1), (2) and~(32) and 17(1), (2) and~(3) of Schedule 3 and of paragraphs 1, 2, 3($b$), ($c$)  and~($d$)  and 11 of Schedule~4 to the Act, (all of which concern employment and support allowance) insofar as they have not already been commenced; and

    27th October 2008 is the day appointed for the coming into force generally of Part I of and Schedules 1 to 4 to, the Act (which concern employment and support allowance) insofar as they have not already been commenced. 

Article 3 amends the Social Security and Child Support (Decisions and Appeals) Regulations 1999 and the Social Security (Loss of Benefit) Regulations 2001 in consequence of the coming into force of section 49 of the Act.

An impact assessment has not been published for this instrument as it has no direct impact on the costs of businesses, charities and the voluntary sector. 

\end{document}