\documentclass[12pt,a4paper]{article}

\newcommand\regstitle{The Child Maintenance and Other Payments Act 2008 (Commencement No.~14 and Transitional Provisions) and the Welfare Reform Act 2012 (Commencement No.~18 and Transitional and Savings Provisions) Order 2014}

\newcommand\regsnumber{2014/1635}

\title{\regstitle}

\author{S.I.\ 2014 No.\ 1635 (C.~65)}

\date{Made
%23rd June 2014\\
%Laid before Parliament
%23rd June 2014\\
%Coming into force
24th June 2014
}

%\opt{oldrules}{\newcommand\versionyear{1993}}
%\opt{newrules}{\newcommand\versionyear{2003}}
%\opt{2012rules}{\newcommand\versionyear{2012}}

\usepackage{csa-regs}

\setlength\headheight{56.61603pt}

%\hbadness=10000

\begin{document}

\maketitle

%\enlargethispage{\baselineskip}

\noindent
The Secretary of State for Work and Pensions, in exercise of the powers conferred by section 62(3) and (4) of the Child Maintenance and Other Payments Act 2008\footnote{2008 c.~6.} and section 150(3) and (4)($c$)  of the Welfare Reform Act 2012\footnote{2012 c.~5.}, makes the following Order: 

{\sloppy

\tableofcontents

}

\bigskip

\setcounter{secnumdepth}{-2}

\subsection[1. Citation and interpretation]{Citation and interpretation}

1.—(1) This Order may be cited as the Child Maintenance and Other Payments Act 2008 (Commencement No.~14 and Transitional Provisions) and the Welfare Reform Act 2012 (Commencement No.~18 and Transitional and Savings Provisions) Order 2014.

(2) In this Order—
\begin{enumerate}\item[]
“1991 Act” means the Child Support Act 1991\footnote{1991 c.~48.};

“2008 Act” means the Child Maintenance and Other Payments Act 2008;

“maintenance assessment” has the meaning given in section 54 (interpretation) of the 1991 Act before its substitution by section 26 (amendments) of, and paragraph 11(1) and (20)($d$)  of Schedule 3 (amendment of enactments) to, the Child Support, Pensions and Social Security Act 2000\footnote{2000 c.~19.};

“maintenance calculation” has the meaning given in section 54 of the 1991 Act\footnote{The definition of “maintenance calculation” was substituted for the definition of “maintenance assessment” in section 54 of the 1991 Act by paragraph 11(1) and (20)($d$)  of Schedule 3 to the Child Support, Pensions and Social Security Act 2000 (“the 2000 Act”).};

“new calculation rules” means Part I of Schedule 1 (calculation of weekly amount of child support maintenance) to the 1991 Act as amended by paragraph 2 of Schedule 4 (calculation by reference to gross weekly income) to the 2008 Act;

“transition period” has the meaning given in regulation 1(2) (citation, commencement and interpretation) of the Child Support (Ending Liability in Existing Cases and Transition to New Calculation Rules) Regulations 2014\footnote{S.I.~2014/614.}.
\end{enumerate}

(3) In this Order, a reference to an existing case is to a case in which there is—
\begin{enumerate}\item[]
($a$) a maintenance assessment in force;

($b$) a maintenance calculation, made otherwise than in accordance with the new calculation rules, in force;

($c$) an application for a maintenance assessment which has been made but not determined; or

($d$) an application for a maintenance calculation, which falls to be made otherwise than in accordance with the new calculation rules, which has been made but not determined.
\end{enumerate}

\subsection[2. Appointed day for provisions of the 2008 Act]{Appointed day for provisions of the 2008 Act}

2.  Section 19 (transfer of cases to new rules) of, and Schedule 5 (maintenance calculations: transfer of cases to new rules)\footnote{Schedule 5 was amended by section 136(2) of the Welfare Reform Act 2012 and by S.I.~2012/2007.} to, the 2008 Act come into force for all purposes, in so far as those provisions are not already in force, on 30th June 2014.

\subsection[3. Thirteen week linking rule in relation to certain cases]{Thirteen week linking rule in relation to certain cases}

3.—(1) This article has effect from 30th June 2014.

(2) Where—
\begin{enumerate}\item[]
($a$) the Secretary of State ceases acting in an existing case as a result of being requested, on or after 30th June 2014, to cease acting by—
\begin{enumerate}\item[]
(i) the person with care under section 4(5) (child support maintenance) of the 1991 Act\footnote{Section 4(5) was amended by S.I.~2012/2007.}, or

(ii) in Scotland, a child under section 7(6) (right of child in Scotland to apply for assessment) of that Act\footnote{Section 7(6) was amended by S.I.~2012/2007.}; and
\end{enumerate}
($b$) the qualifying child or, if there is more than one qualifying child, all of the qualifying children in relation to the existing case will reach the age of 20 before the end of the transition period,
\end{enumerate}
the non-resident parent is not eligible to make an application under section~4(1) of the 1991 Act\footnote{Section 4(1) was amended by section 1(2)($a$)  of, and paragraph 11(1) and (2) of Schedule 3 to, the 2000 Act and by S.I.~2012/2007.} in relation to a qualifying child referred to in sub-paragraph~($b$)  before the expiry of the period of 13 weeks from the date of cessation of action by the Secretary of State, unless paragraph (3) applies.

(3) Where the non-resident parent becomes the person with care, paragraph (2) does not apply.

(4) For the purposes of paragraph (2), the date of cessation of action by the Secretary of State is—
\begin{enumerate}\item[]
($a$) where there is a maintenance assessment or maintenance calculation in force, the date on which the liability under that assessment or calculation ends as a result of the request to cease acting;

($b$) where there is an application still to be determined, the date notified to the person with care as the date on which the Secretary of State has ceased acting.
\end{enumerate}

(5) In this article—
\begin{enumerate}\item[]
($a$) subject to sub-paragraph ($b$) , “non-resident parent”, “person with care” and “qualifying child” have the meanings given in section 3 (meaning of certain terms) of the 1991 Act\footnote{The term “non-resident parent” was substituted for the term “absent parent” by paragraph 11(1) and (2) of Schedule 3 to the 2000 Act. The definition of “qualifying child” in section 3(1) of the Child Support Act 1991 (c.~48) (“the 1991 Act”) was amended by paragraph 11(1) and (2) of Schedule 3 to the 2000 Act.};

($b$) a reference to a non-resident parent includes reference to a person who is—
\begin{enumerate}\item[]
(i) treated as the non-resident parent for the purposes of the 1991 Act\footnote{A person may be treated as a non-resident parent for the purposes of the 1991 Act under regulation 50(2) of the Child Support Maintenance Calculation Regulations 2012 (S.I.~2012/2677) or regulation 8(2) of the Child Support (Maintenance Calculations and Special Cases) Regulations 2000 (S.I.~2001/155).},

(ii) an absent parent (which has the meaning given in section~54 of the 1991 Act before its substitution by section 26 of, and paragraph~11(1) and (2) of Schedule 3 to, the Child Support, Pensions and Social Security Act 2000), or

(iii) treated as the absent parent for the purposes of the 1991 Act\footnote{A person may be treated as an absent parent for the purposes of the 1991 Act under regulation 20(2) of the Child Support (Maintenance Assessments and Special Cases) Regulations 1992 (S.I.~1992/1815).}.
\end{enumerate}
\end{enumerate}

\subsection[4. Appointed day for section 137 of the Welfare Reform Act 2012]{Appointed day for section 137 of the Welfare Reform Act 2012}

4.  Section 137 (collection of child support maintenance) of the Welfare Reform Act 2012 comes into force on 30th June 2014, subject to the transitional and savings provision in article 5.

\subsection[5. Transitional and savings provision for existing cases]{Transitional and savings provision for existing cases}

5.  Notwithstanding article 4, sections 4, 7 and 29(1) (collection of child support maintenance) of the 1991 Act\footnote{Section 4 was amended by section 18(1) of the Child Support Act 1995 (c.~34), paragraph 19 of Schedule 7, and Schedule 8, to the Social Security Act 1998 (c.~14) (“the 1998 Act”), sections 1(2) and 2(1) to (3) of, and paragraph 11(1) to (3) of Schedule 3 to, the 2000 Act, section 35(1) of, and Schedule 8 to, the Child Maintenance and Other Payments Act 2008 (c.~6) (“the 2008 Act”) and S.I.~2012/2007. Section 7 was amended by paragraph 21 of Schedule 7, and Schedule 8, to the 1998 Act, section 1(2) of, and paragraph 11(1),(2) and (4) of Schedule 3 to, the 2000 Act, section 35(2) of the 2008 Act and S.I.~2012/2007. Section 29(1) was amended by section 1(2)($a$)  of the 2000 Act and paragraph 32(1) and (2) of Schedule 3, and Schedule 8, to the 2008 Act.} shall continue to have effect as they were in force immediately before 30th June 2014, in relation to existing cases (that is, until the end of the transition period).

\subsection[6. Amendment of the Child Maintenance and Other Payments Act 2008 (Commencement No.~10 and Transitional Provisions) Order 2012]{Amendment of the Child Maintenance and Other Payments Act 2008 (Commencement No.~10 and Transitional Provisions) Order 2012}

6.  With effect from 30th June 2014, in the Child Maintenance and Other Payments Act 2008 (Commencement No.~10 and Transitional Provisions) Order 2012\footnote{S.I.~2012/3042. The Order was amended by S.I.~2013/1860.}—
\begin{enumerate}\item[]
($a$) in article 3(1) (cases to which the new calculation rules apply), “any of paragraphs (2) to (4)” shall be read as “paragraph (2)”;

($b$) articles 3(3), (4) and (7) and 5 do not apply.
\end{enumerate}

\subsection[7. Amendment of the Child Maintenance and Other Payments Act 2008 (Commencement No.~11 and Transitional Provisions) Order 2013]{Amendment of the Child Maintenance and Other Payments Act 2008 (Commencement No.~11 and Transitional Provisions) Order 2013}

7.  With effect from 30th June 2014, in the Child Maintenance and Other Payments Act 2008 (Commencement No.~11 and Transitional Provisions) Order 2013\footnote{S.I.~2013/1860.}—
\begin{enumerate}\item[]
($a$) in article 3(1) (cases to which the new calculation rules apply), “any of paragraphs (2) to (4)” shall be read as “paragraph (2)”;

($b$) articles 3(3), (4) and (7) and 5 do not apply. 
\end{enumerate}

\bigskip

\pagebreak[3]

Signed 
by authority of the 
Secretary of State for~Work and~Pensions.
%I concur
%By authority of the Lord Chancellor

{\raggedleft
\emph{Steve Webb}\\*
%Secretary
Minister
%Parliamentary Under Secretary 
of State\\*Department 
for~Work and~Pensions

}

24th June 2014

\small

\part{Explanatory Note}

\renewcommand\parthead{— Explanatory Note}

\subsection*{(This note is not part of the Order)}

Article 2 of this Order brings into force section 19 of, and Schedule 5 to, the Child Maintenance and Other Payments Act 2008 (c.~6) for all remaining purposes on 30th June 2014. Section 19 introduces Schedule 5. Schedule 5 enables the Secretary of State to require the parties in existing cases (that is, cases administered under the 1993 or 2003 schemes of child support maintenance) to choose whether or not to remain in the statutory scheme under the 2012 rules for calculating child support maintenance (“the 2012 scheme”). It enables the Secretary of State to make provision in regulations in relation to requiring the parties to make that choice.

Article 3 makes transitional provision from 30th June 2014 so that where the Secretary of State ceases acting in an existing case following a request to cease acting being made by a person with care or a qualifying child in Scotland and the children in the case will reach the age of 20 by the date specified in the scheme prepared by the Secretary of State, the non-resident parent is not eligible to make an application within 13 weeks of the Secretary of State having ceased acting. This provision does not apply where the non-resident parent becomes the person with care.

Article 4 brings section 137 of the Welfare Reform Act 2012 (c.~5) into force on 30th June 2014, subject to the transitional and savings provisions in article 5. Section 137(2) repeals some of the wording in section 4(2) of the Child Support Act 1991 (c.~48) (“the 1991 Act”) and inserts a new subsection (2A) into section 4 of that Act. The effect of the amendment is that, once a maintenance calculation has been made under the 1991 Act, if the person with care asks the Secretary of State to collect child maintenance, the Secretary of State will only do so if the non-resident parent agrees or the Secretary of State is satisfied that the non-resident parent is otherwise unlikely to make payments. Section 137(3) makes corresponding amendments to section 7 of the 1991 Act, which relates to an application for a maintenance calculation made by a child in Scotland. Section 137(4) makes a consequential amendment to section 29 of the 1991 Act. Article 5 makes transitional and savings provision so that the amendments made to the 1991 Act by section 137 do not apply in relation to cases administered under the 1993 or 2003 schemes of child support maintenance.

Articles 6 and 7 amend the Child Maintenance and Other Payments Act 2008 (Commencement No.~10 and Transitional Provisions) Order 2012 (S.I.~2012/3042) and Child Maintenance and Other Payments Act 2008 (Commencement No.~11 and Transitional Provisions) Order 2013 (S.I.~2013/1860) so that the provisions in those Orders that deal with when and in what circumstances an existing case that is related to a new application will transfer to the 2012 scheme do not apply from 30th June 2014. The Child Support (Ending Liability in Existing Cases and Transition to New Calculation Rules) Regulations 2014 (S.I.~2014/614) make provision for an existing case that is related to a new application from 30th June 2014. 

\end{document}