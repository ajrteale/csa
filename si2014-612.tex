\documentclass[12pt,a4paper]{article}

\newcommand\regstitle{The Child Support Fees Regulations 2014}

\newcommand\regsnumber{2014/612}

\title{\regstitle}

\author{S.I.\ 2014 No.\ 612}

\date{Made
12th March 2014\\
%%%Laid before Parliament
%%%27th June 2013\\
Coming into force
in accordance with regulation 1(3), (4) and (5)
}

%\opt{oldrules}{\newcommand\versionyear{1993}}
%\opt{newrules}{\newcommand\versionyear{2003}}
%\opt{2012rules}{\newcommand\versionyear{2012}}

\usepackage{csa-regs}

%\setlength\headheight{42.11603pt}

%\hbadness=10000

\begin{document}

\maketitle

\enlargethispage{\baselineskip}

\noindent
The Secretary of State for Work and Pensions makes the following Regulations in exercise of the powers conferred by section 5(1)($p$)  and 189(4) of the Social Security Administration Act 1992\footnote{1992 c.~5.}, sections 43(1) and 51(1) and (2)($a$)  of the Child Support Act 1991\footnote{1991 c.~48. Section 43 was substituted by section 21 of the Child Support, Pensions and Social Security Act 2000 (c.~19) (“the 2000 Act”) for certain cases specified in article 3 of the Child Support, Pensions and Social Security Act 2000 (Commencement No.~12) Order 2003 (S.I.~2003/192). Section 43 was further amended by section 139 of the Welfare Reform Act 2012 (c.~5).} and sections 6(1) to (4) and 55(3) and (4) of the Child Maintenance and Other Payments Act 2008\footnote{2008 c.~6. Section 6 was amended by sections 140 and 141 of the Welfare Reform Act 2012.}.

A draft of this instrument was laid before and approved by a resolution of each House of Parliament in accordance with section 52(2) of the Child Support Act 1991 and section 55(5) of the Child Maintenance and Other Payments Act 2008\footnote{Section 55(5) was amended by S.I.~2012/2007.}.

The Social Security Advisory Committee has agreed that proposals in respect of these Regulations, so far as made under section 5(1)($p$)  of the Social Security Administration Act 1992 should not be referred to it. 

{\sloppy

\tableofcontents

}

\bigskip

\setcounter{secnumdepth}{-2}

\section[Part I --- Citation, commencement and interpretation]{Part I\\*Citation, commencement and interpretation}

\renewcommand\parthead{--- Part I}

\subsection[1. Citation, commencement and interpretation]{Citation, commencement and interpretation}

1.—(1) These Regulations may be cited as the Child Support Fees Regulations 2014.

(2) In these Regulations—
\begin{enumerate}\item[]
“the 1991 Act” means the Child Support Act 1991;

“the 2008 Act” means the Child Maintenance and Other Payments Act 2008;

“non-resident parent” has the meaning given in section 3(2) (meaning of certain terms) of the 1991 Act\footnote{The term “non-resident parent” was substituted for the term “absent parent” by section 26 of, and paragraph 11(1) and (2) of Schedule 3 to, the 2000 Act.}.
\end{enumerate}

(3) Subject to paragraphs (4) and (5), these Regulations come into force on the day on which section 137 (collection of child support maintenance) of the Welfare Reform Act 2012\footnote{2012 c.~5.} comes into force.

(4) Regulations 6, 7, 8, 14 and 15 come into force on the day falling six weeks after the day on which section 137 of the Welfare Reform Act 2012 comes into force.

(5) This regulation and regulation 13 come into force—
\begin{enumerate}\item[]
($a$) for the purposes of making regulations, on the day following the day on which these Regulations are made; and

($b$) for all other purposes, on the day on which section 137 of the Welfare Reform Act 2012 comes into force.
\end{enumerate}

\section[Part II --- Application fee]{Part II\\*Application fee}

\renewcommand\parthead{--- Part II}

\subsection[2. Interpretation of this Part]{Interpretation of this Part}

2.  For the purposes of this Part—
\begin{enumerate}\item[]
“application fee” means the fee payable under regulation 3(1);

“application for child support maintenance” means an application for child support maintenance under section 4(1) (child support maintenance) or 7(1) (right of child in Scotland to apply for assessment) of the 1991 Act\footnote{Section 4(1) of the Child Support Act 1991 (c.~48) (“the 1991 Act”) was amended by section 1(2)($a$)  of, and paragraph 11(2) of Schedule 3 to, the 2000 Act. Section 7(1) of the 1991 Act was amended by section 1(2)($a$)  of, and paragraph 11(2) of Schedule 3 to, the 2000 Act and by Schedule 8 to the Child Maintenance and Other Payments Act 2008 Act (c.~6) (“the 2008 Act”).}.
\end{enumerate}

\subsection[3. The application fee]{The application fee}

3.—(1) On making an application for child support maintenance a fee of £20 is payable to the Secretary of State by the person making the application, whether or not a maintenance calculation is made under the 1991 Act as a result of that application.

(2) For the purposes of section 11(1) and (2) (maintenance calculations) of the 1991 Act\footnote{Section 11(1) was substituted by section 1(1) of the 2000 Act.}, the Secretary of State is not obliged to determine an application for child support maintenance until the application fee has been paid or is waived under regulation 4 (waiver of the application fee).

\subsection[4. Waiver of the application fee]{Waiver of the application fee}

4.—(1) The application fee shall be waived where the person making the application for child support maintenance (“$\mathcal{A}$”) satisfies one or both of paragraphs (2) and (3).

(2) $\mathcal{A}$ is under the age of 19 years at the time of making the application for child support maintenance.

(3) $\mathcal{A}$—
\begin{enumerate}\item[]
($a$) is in the opinion of the Secretary of State a victim of domestic violence or abuse\footnote{Guidance on how the Secretary of State will determine if a person is a victim or domestic violence or abuse is available on the \url{www.gov.uk} website. A paper copy of the guidance may be obtained from the Department for Work and Pensions, Child Support, Level 7, Caxton House, Tothill Street, London, \textsc{\lowercase{SW1H~9NA}}.};

($b$) has reported the domestic violence or abuse to an appropriate person; and

($c$) declares to the Secretary of State that $\mathcal{A}$ is a victim of domestic violence or abuse and states to the Secretary of State the appropriate person to whom $\mathcal{A}$ has reported domestic violence or abuse, either—

(i) at the time of making the application for child support maintenance, or

(ii) where the Secretary of State provides $\mathcal{A}$ with a written declaration to complete, in that written declaration (provided payment of the application fee has not been made prior to the date on which that declaration is returned to the Secretary of State).
\end{enumerate}

(4) For the purposes of paragraph (3), “appropriate person” means a person specified in “Guidance on regulation 4(3) of the Child Support Fees Regulations 2014: List of persons to whom an applicant must have reported domestic violence or abuse” published by the Secretary of State in December 2013\footnote{This guidance is available on the \url{www.gov.uk} website. A paper copy of the guidance may be obtained from the Department for Work and Pensions, Child Support, Level 7, Caxton House, Tothill Street, London, \textsc{\lowercase{SW1H~9NA}}.}.

\subsection[5. Repayment of the application fee]{Repayment of the application fee}

5.  An application fee that has been paid must be repaid by the Secretary of State to the person who made the application for child support maintenance where a qualifying child (which has the meaning given in section 3(1) (meaning of certain terms) of the 1991 Act) dies—
\begin{enumerate}\item[]
($a$) after the application for child support maintenance is made; and

($b$) before a maintenance calculation under the 1991 Act is made,
\end{enumerate}
and as a result a maintenance calculation shall not be made.

\section[Part III --- Collection fee]{Part III\\*Collection fee}

\renewcommand\parthead{--- Part III}

\subsection[6. Interpretation of this Part]{Interpretation of this Part}

6.  For the purposes of this Part—
\begin{enumerate}\item[]
“child support maintenance” means child support maintenance calculated under Part~I of Schedule 1 to the 1991 Act\footnote{Part~I of Schedule 1 to the 1991 Act was substituted by section 1(3) of, and Schedule 1 to, the 2000 Act and amended by Schedule 4 to the 2008 Act.} as amended by Schedule~4 to the 2008 Act, which has accrued on or after the date on which this regulation comes into force;

“person in receipt of child support maintenance” means a person to whom child support maintenance is paid, being a person with care (which has the meaning given in section 3(3) of the 1991 Act) or a child who makes an application under section 7(1) (right of child in Scotland to apply for assessment) of the 1991 Act.
\end{enumerate}

\subsection[7. The collection fee]{The collection fee}

7.—(1) A collection fee is payable to the Secretary of State by—
\begin{enumerate}\item[]
($a$) the non-resident parent; and

($b$) the person in receipt of child support maintenance,
\end{enumerate}
in relation to a case where there are arrangements for collection.

(2) The amount of the collection fee payable by a non-resident parent in respect of each day is—
\begin{enumerate}\item[]
($a$) subject to sub-paragraph ($b$), 20\% of the daily amount;

($b$) where there is more than one person in receipt of child support maintenance in relation to that non-resident parent, in respect of each person in receipt of child support maintenance, 20\% of the alternative daily amount.
\end{enumerate}

(3) The amount of the collection fee payable by a person in receipt of child support maintenance is 4\% of any payment of child support maintenance in relation to which there are arrangements for collection, which the Secretary of State has collected and which would otherwise be paid to that person.

(4) In this regulation—
\begin{enumerate}\item[]
“alternative daily amount” means the alternative weekly amount divided by 7;

“alternative weekly amount” means the weekly amount of child support maintenance that the non-resident parent is liable to pay in respect of the person in receipt of child support maintenance in question and in relation to which there are arrangements for collection;

“daily amount” means the weekly amount divided by 7;

“weekly amount” means the weekly amount of child support maintenance that the non-resident parent is liable to pay and in relation to which there are arrangements for collection.
\end{enumerate}

(5) Where a calculation carried out under this regulation results in a fraction of a penny, that is to be treated as a penny if it is either one half or exceeds one half, and otherwise it is to be disregarded.

(6) For the purposes of this regulation, there are arrangements for collection where the Secretary of State is making arrangements to collect child support maintenance under section 29(1) (collection of child support maintenance) of the 1991 Act\footnote{Section 29(1) of the 1991 Act was amended by s.1(2)($a$)  of the 2000 Act and Schedule 8 to the 2008 Act.} and the payments of child support maintenance are transmitted through the Secretary of State.

\subsection[8. Recovery of the collection fee]{Recovery of the collection fee}

8.—(1) Any amount of the collection fee payable by a non-resident parent under regulation 7 (the collection fee) may be recovered by the Secretary of State from any payment made by that non-resident parent to the Secretary of State.

(2) Any amount of the collection fee payable by a person in receipt of child support maintenance under regulation 7 may be recovered by the Secretary of State from any payment of child support maintenance which would otherwise be paid to that person by the Secretary of State.

\section[Part IV --- Enforcement fee]{Part IV\\*Enforcement fee}

\renewcommand\parthead{--- Part IV}

\subsection[9. Interpretation of this Part]{Interpretation of this Part}

9.  For the purposes of this Part—
\begin{enumerate}\item[]
“armed forces” means the naval, military and air forces of the Crown;

“child support maintenance” means child support maintenance calculated under Part~I of Schedule 1 to the 1991 Act as amended by Schedule~4 to the 2008 Act;

“committed to operations” means deployed on an operational tour of duty and includes pre-operational training and leave, rest and recuperation during an operational tour of duty and post-operational leave;

“deduction from earnings order” means an order made under section~31(2) of the 1991 Act\footnote{Section 31(2) of the 1991 Act was amended by section 1(2) of the 2000 Act.} and, with the exception of where it appears in regulation 12(4)($b$)  (waiver of an enforcement fee), includes a deduction from earnings request;

“deduction from earnings request” means a request from the Secretary of State in respect of a non-resident parent, who is a member of the armed forces and who is liable to pay child support maintenance, for a sum to be deducted from that non-resident parent’s pay and appropriated in or towards satisfaction of the non-resident parent’s obligation to pay child support maintenance;

“liability order” means an order made under section 33(3) of the 1991 Act;

“lump sum deduction order” means an order made under section 32E(1) of the 1991 Act\footnote{Section 32E(1) of the 1991 Act was inserted by section 23 of the 2008 Act.};

“regular deduction order” means an order made under section 32A(1) of the 1991 Act\footnote{Section 32A(1) of the 1991 Act was inserted by section 22 of the 2008 Act.}.
\end{enumerate}

\subsection[10. The enforcement fee]{The enforcement fee}

10.  An enforcement fee of an amount set out in column (2) of the table below is payable to the Secretary of State by a non-resident parent when the Secretary of State takes a method of enforcement action specified in column (1) of the table below to secure payment of child support maintenance.

\begin{center}
\noindent
\begin{tabular}{ll}
\hline
Column (1)	& Column (2)\\
\itshape Enforcement Action	&\itshape Fee payable\\
\hline
(i)  Making a deduction from earnings order	&£50\\
(ii)  Making a regular deduction order	&£50\\
(iii) Making a lump sum deduction order	&£200\\
(iv) Making an application for a liability order	&£300\\
\hline
\end{tabular}
\end{center}

\subsection[11. Recovery of an enforcement fee]{Recovery of an enforcement fee}

11.  An enforcement fee payable by a non-resident parent under regulation 10 (the enforcement fee) may be recovered by the Secretary of State from any payment made by that non-resident parent to the Secretary of State.

\subsection[12. Waiver of an enforcement fee]{Waiver of an enforcement fee}

12.—(1) An enforcement fee payable under regulation 10 may be waived in the circumstances specified in paragraphs (2) to (6).

(2) The circumstances specified in this paragraph are where an additional enforcement fee is payable with respect to concurrent or subsequent action of the same type taken by the Secretary of State in circumstances where—
\begin{enumerate}\item[]
($a$) the non-resident parent has more than one employer at the same time and the Secretary of State makes two or more deduction from earnings orders; or

($b$) the non-resident parent holds more than one account with a deposit-taker and the Secretary of State makes more than one regular deduction order or lump sum deduction order simultaneously.
\end{enumerate}

(3) The circumstances specified in this paragraph are where an additional enforcement fee is payable with respect to action taken to make an additional deduction from earnings order or additional regular deduction order in circumstances where—
\begin{enumerate}\item[]
($a$) the non-resident parent has changed employer;

($b$) the non-resident parent has changed their account held with a deposit-taker; or

($c$) the amount being collected under a prior deduction from earnings order or a prior regular deduction order has changed.
\end{enumerate}

(4) The circumstances specified in this paragraph are where—
\begin{enumerate}\item[]
($a$) an application for a liability order is made to a court, but no liability order results from the application;

($b$) a successful appeal or a successful challenge by way of judicial review has been made against the making of a deduction from earnings order, a regular deduction order or a lump sum deduction order; or

($c$) a deduction from earnings order, a regular deduction order or a lump sum deduction order has lapsed or been discharged due to an error or maladministration by the Secretary of State.
\end{enumerate}

(5) The circumstances specified in this paragraph are where a non-resident parent elects to pay child support maintenance by way of a deduction from earnings order.

(6) The circumstances specified in this paragraph are where a deduction from earnings request is made when the non-resident parent to which it relates is committed to operations.

\section[Part V --- Miscellaneous]{Part V\\*Miscellaneous}

\renewcommand\parthead{--- Part V}

\subsection[13. Collection and enforcement of fees]{Collection and enforcement of fees}

13.—(1) Subject to paragraph (2), the provisions of the 1991 Act with respect to—
\begin{enumerate}\item[]
($a$) the collection of child support maintenance;

($b$) the enforcement of any obligation to pay child support maintenance,
\end{enumerate}
shall apply equally to the collection and enforcement of fees payable by virtue of one or both of Parts III and IV of these Regulations.

(2) The following provisions of the 1991 Act do not apply where those provisions would be used solely to enforce payment of a fee payable by virtue of one or both of Parts III and IV—
\begin{enumerate}\item[]
($a$) section 39A (commitment to prison and disqualification from driving)\footnote{Section 39A of the Child Support Act 1991 (c.~48) was inserted by section 16 of the Child Support, Pensions and Social Security Act 2000 (c.~19) (“the 2000 Act”).};

($b$) section 40 (commitment to prison)\footnote{Section 40 of the 1991 Act was amended by sections 16(2), 17(1) and 85 of, and Part~I of Schedule 9 to, the 2000 Act.};

($c$) section 40A (commitment to prison: Scotland)\footnote{Section 40A of the 1991 Act was inserted by section 17 of the 2000 Act.}; and

($d$) section 40B (disqualification from driving: further provisions)\footnote{Section 40B of the 1991 Act was inserted by section 16 of the 2000 Act.}.
\end{enumerate}

\subsection[14. Amendments to the Social Security (Claims and Payments) Regulations 1987]{Amendments to the Social Security (Claims and Payments) Regulations 1987}

14.—(1) Schedule 9B (deductions from benefits in respect of child support maintenance and payment to persons with care) to the Social Security (Claims and Payments) Regulations 1987\footnote{S.I.~1987/1968. Schedule 9B was added by S.I.~2001/18.} is amended as follows.

(2) In paragraph 1 (interpretation), after the definition of “beneficiary” insert—
\begin{quotation}
““fee” means any collection fee under Part III of the Child Support Fees Regulations 2014 which is payable by the non-resident parent,”.
\end{quotation}

(3) In paragraph 2 (deductions)—
\begin{enumerate}\item[]
($a$) in sub-paragraph (1)\footnote{Sub-paragraph (1) was amended by S.I.~2002/3019 and 2008/1554.}—
\begin{enumerate}\item[]
(i) after “an amount equal to the amount of maintenance” insert “and any fee”;

(ii) after “of the liability to pay maintenance” insert “, and retain any amount deducted in discharge of any liability to pay a fee”;
\end{enumerate}

($b$) in sub-paragraph (2) after “A deduction” insert “for maintenance and fees”.
\end{enumerate}

(4) In sub-paragraph (1) of paragraph 3 (arrears)\footnote{Sub-paragraph (1) was amended by S.I.~2002/3019 and 2008/1554.}—
\begin{enumerate}\item[]
($a$) for ``£1'' substitute “£1$.$20”;

($b$) after “the beneficiary’s liability to pay arrears of maintenance” insert “, and retain any amount deducted in discharge of any liability to pay a fee”;
\end{enumerate}

(5) In paragraph 4 (apportionment), after “the amount deducted” insert “in respect of maintenance”.

(6) In sub-paragraph (2) of paragraph 5 (flat rate maintenance)\footnote{Sub-paragraph (2) was amended by S.I.~2013/1654.}—
\begin{enumerate}\item[]
($a$) before the words “may be deducted” insert “and any fee”;

($b$) after “both partners’ liability to pay maintenance” insert “and any fee”;

($c$) after “of the respective liabilities to pay maintenance” insert “or retained in discharge of any liability to pay a fee”.
\end{enumerate}

(7) In sub-paragraph (2) of paragraph 6 (flat rate maintenance (polygamous marriage))\footnote{Sub-paragraph (2) was amended by S.I.~2013/1654.}—
\begin{enumerate}\item[]
($a$) before the words “may be deducted” insert “and any fee”;

($b$) after “all the members’ liability to pay maintenance” insert “and any fee”;

($c$) after “of the respective liabilities to pay maintenance” insert “or retained in discharge of any liability to pay a fee”.
\end{enumerate}

\subsection[15. Amendments to the Universal Credit, Personal Independence Payment, Jobseeker’s Allowance and Employment and Support Allowance (Claims and Payments) Regulations 2013]{Amendments to the Universal Credit, Personal Independence Payment, Jobseeker’s Allowance and Employment and Support Allowance (Claims and Payments) Regulations 2013}

15.—(1) Schedule 7 (deductions from benefit in respect of child support maintenance and payment to persons with care) to the Universal Credit, Personal Independence Payment, Jobseeker’s Allowance and Employment and Support Allowance (Claims and Payments) Regulations 2013\footnote{S.I.~2013/380.} is amended as follows.

(2) In paragraph 1 (interpretation), after the definition of “beneficiary” insert—
\begin{quotation}
““fee” means any collection fee under Part III of the Child Support Fees Regulations 2014 which is payable by the non-resident parent;”
\end{quotation}

(3) In paragraph 2 (deductions)—
\begin{enumerate}\item[]
($a$) in sub-paragraph (1)—
\begin{enumerate}\item[]
(i) after “an amount equal to the amount of maintenance” insert “and any fee”;

(ii) after “of the liability to pay maintenance” insert “, and retain any amount deducted in discharge of any liability to pay a fee”;
\end{enumerate}

($b$) in sub-paragraph (2), after “A deduction” insert “for maintenance and fees”.
\end{enumerate}

(4) In sub-paragraph (1) of paragraph 3 (arrears)—
\begin{enumerate}\item[]
($a$) for “£1” substitute “£1$.$20”;

($b$) after “the beneficiary’s liability to pay arrears of maintenance” insert “, and retain any amount deducted in discharge of any liability to pay a fee”;
\end{enumerate}

(5) In paragraph 4 (apportionment), after “the amount deducted” insert “in respect of maintenance”.

(6) In sub-paragraph (2) of paragraph 5 (flat rate maintenance)—
\begin{enumerate}\item[]
($a$) after “the flat rate of maintenance” insert “and any fee”;

($b$) after “liability of both partners to pay maintenance” insert “and any fee”;

($c$) after “of the respective liabilities to pay maintenance” insert “or retained in discharge of any liability to pay a fee”.
\end{enumerate}

\bigskip

\pagebreak[3]

Signed 
by authority of the 
Secretary of State for~Work and~Pensions.
%I concur
%By authority of the Lord Chancellor

{\raggedleft
\emph{Steve Webb}\\*
%Secretary
Minister
%Parliamentary Under Secretary 
of State\\*Department 
for~Work and~Pensions

}

12th March 2014

\small

\part{Explanatory Note}

\renewcommand\parthead{— Explanatory Note}

\subsection*{(This note is not part of the Regulations)}

These Regulations make provision about the charging of fees by the Secretary of State in connection with the exercise of the Secretary of State’s functions with regards to child support maintenance.

Regulation 3 provides that when an application for child support maintenance is made a fee of £20 is payable by the person who makes the application. The Secretary of State is not obliged to determine an application for child support maintenance until the application fee is paid or is waived under regulation 4.

Regulation 4 provides that the application fee must be waived where the applicant is: under the age of 19 years; or a victim of domestic violence or abuse, has reported that domestic violence or abuse to an appropriate person, declares that he or she is a victim of domestic violence or abuse and states the appropriate person to whom he or she has reported domestic violence or abuse. An appropriate person is a person specified in “Guidance on regulation 4(3) of the Child Support Fees Regulations 2014: List of persons to whom an applicant must have reported domestic violence or abuse” published by the Secretary of State in December 2013. The guidance is available on the \url{www.gov.uk} website. Guidance on how the Secretary of State will determine if a person is a victim or domestic violence or abuse is available on the \url{www.gov.uk} website. A paper copy of either piece of guidance may be obtained, free of charge, from the Department for Work and Pensions, Child Support, Level 7, Caxton House, Tothill Street, London, \textsc{\lowercase{SW1H~9NA}}.

Regulation 5 enables the Secretary of State to repay the application fee where the child in respect of whom the application is made dies before a maintenance calculation is made and as a result of the child’s death a maintenance calculation will not be made.

Regulation 7 provides that a collection fee is payable in a case where arrangements for collection are made by the Secretary of State. A collection fee is payable by both the non-resident parent and the person in receipt of child support maintenance. The collection fee is only payable in a case where child support maintenance is calculated under the 2012 scheme and in respect of child support maintenance that has accrued since the coming into force of regulation 6.

The collection fee payable by a non-resident parent is normally 20\% of the daily amount of child support maintenance that the non-resident parent is liable to pay. The collection fee payable by a person in receipt of child support maintenance is 4\% of the child support maintenance that is collected by the Secretary of State and which would otherwise be paid to that person.

Regulation 8 makes provision for the recovery of the collection fee. The collection fee payable by the non-resident parent may be recovered from any payment made by that non-resident parent to the Secretary of State. The collection fee payable by the person in receipt of child support maintenance may be recovered from any payment of child support maintenance which would be otherwise paid to that person.

Regulation 10 makes provision for the payment of an enforcement fee by a non-resident parent where the Secretary of State makes a deduction from earnings order, a regular deduction order, a lump sum deduction order or an application for a liability order. An enforcement fee is only payable in a case where child support maintenance is calculated under the 2012 scheme.

Regulation 11 allows the Secretary of State to recover the enforcement fee from any payment made by the non-resident parent to the Secretary of State.

Regulation 12 prescribes circumstances in which an enforcement fee payable under regulation 10 may be waived.

Regulation 13 provides that the provisions of the 1991 Act with respect to the collection and enforcement of child support maintenance shall apply to the collection and enforcement of fees payable under these Regulations with certain exceptions.

Regulations 14 and 15 make supplemental amendments to the Social Security (Claims and Payments) Regulations 1987 and the Universal Credit, Personal Independence Payment, Jobseeker’s Allowance and Employment and Support Allowance (Claims and Payments) Regulations 2013. These amendments enable the Secretary of State to make deductions from prescribed benefits to include the collection fee payable under these Regulations by the non-resident parent.

An assessment of the impact of these Regulations on the costs of business and the voluntary sector is available from the Department for Work and Pensions, Child Support, Level 7, Caxton House, Tothill Street, London, \textsc{\lowercase{SW1H 9NA}} and is annexed to the Explanatory Memorandum to these Regulations which is available alongside the instrument on \url{www.legislation.gov.uk}.

\end{document}