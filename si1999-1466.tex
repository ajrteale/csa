\documentclass[12pt,a4paper]{article}

\newcommand\regstitle{The Social Security and Child Support (Decisions and Appeals) (Amendment) Regulations 1999}

\newcommand\regsnumber{1999/1466}

%\opt{newrules}{
\title{\regstitle}
%}

%\opt{2012rules}{
%\title{Child Maintenance and Other Payments Act 2008\\(2012 scheme version)}
%}

\author{S.I. 1999 No. 1466}

\date{Made 24th May 1999\\Coming into force 1st June 1999}

%\opt{oldrules}{\newcommand\versionyear{1993}}
%\opt{newrules}{\newcommand\versionyear{2003}}
%\opt{2012rules}{\newcommand\versionyear{2012}}

\usepackage{csa-regs}

\setlength\headheight{27.57402pt}

\begin{document}

\maketitle

\noindent
Whereas a draft of this Instrument was laid before Parliament in accordance with section 80(1) of the Social Security Act 1998\footnote{\frenchspacing 1998 c. 14.} and approved by a resolution of each House of Parliament;

Now, therefore, the Secretary of State for Social Security, in exercise of powers in sections 7(6) and 79(1), (3), (4) and (7) of the Social Security Act 1998 and of all other powers enabling him in that behalf, by this Instrument, which contains only regulations made by virtue of those provisions of the Social Security Act 1998 and which is made before the end of the period of six months beginning with the coming into force of section 7(6) of that Act\footnote{\frenchspacing See section 173(5)($b$) of the Social Security Administration Act 1992 (c. 5).}, after consultation with the Council on Tribunals in accordance with section 8 of the Tribunals and Inquiries Act 1992\footnote{\frenchspacing 1992 c. 53.}, hereby makes the following Regulations: 

{\sloppy

\tableofcontents

}

\bigskip

\setcounter{secnumdepth}{-2}

\subsection[1. Citation and commencement]{Citation and commencement}

1.  These Regulations may be cited as the Social Security and Child Support (Decisions and Appeals) (Amendment) Regulations 1999 and shall come into force on 1st June 1999.

\subsection[2. Amendment of regulation 36 of the Social Security and Child Support (Decisions and Appeals) Regulations 1999]{Amendment of regulation 36 of the Social Security and Child Support (Decisions and Appeals) Regulations 1999}

2.  In regulation 36 of the Social Security and Child Support (Decisions and Appeals) Regulations 1999\footnote{\frenchspacing S.I. 1999/991.}—
\begin{enumerate}\item[]
($a$) for paragraph (2) there shall be substituted the following paragraph—
\begin{quotation}
“(2) Subject to paragraphs (3) to (5), an appeal tribunal shall consist of a legally qualified panel member and—
\begin{enumerate}\item[]
($a$) a medically qualified panel member where—
\begin{enumerate}\item[]
(i) the issue, or one of the issues, raised on the appeal is whether the all work test is satisfied; or

(ii) the appeal is made under section 11(1)($b$) of the 1997 Act; or
\end{enumerate}

($b$) one medically qualified panel member or two such members or one medically qualified panel member and an additional member drawn from the panel for the purposes described in paragraph (5) below where—
\begin{enumerate}\item[]
(i) the issue, or one of the issues, raised on the appeal relates to either industrial injuries benefit under Part V of the Contributions and Benefits Act or severe disablement allowance under section 68 of that Act; or

(ii) the appeal is made under section 4 of the Vaccine Damage Payments Act.”;
\end{enumerate}
\end{enumerate}
\end{quotation}

($b$) in paragraph (5), for the words “paragraphs (1), (2) or” there shall be substituted the words “paragraph (1), (2)($a$) or”; and

($c$) after paragraph (6) there shall be added the following paragraph—
\begin{quotation}
“(7) In paragraph (2)($a$)(i) above, “all work test” has the meaning it bears in regulation 2(1) of the Social Security (Incapacity for Work) (General) Regulations 1995\footnote{\frenchspacing S.I. 1995/311.}”. 
\end{quotation}
\end{enumerate}

\bigskip

Signed 
by authority of the Secretary of State for Social Security.

{\raggedleft
\emph{Angela Eagle
}\\*Parliamentary Under-Secretary of State,\\*Department of Social Security

}

24th May 1999

\small

\part{Explanatory Note}

\renewcommand\parthead{--- Explanatory Note}

\subsection*{(This note is not part of the Regulations)}

These Regulations amend the Social Security and Child Support (Decisions and Appeals) Regulations 1999 (S.I.\ 1999/991) (“the principal Regulations”).

Regulation 2($a$) amends regulation 36 of the principal Regulations (composition of appeal tribunals constituted under section 7 of the Social Security Act 1998 (c.\ 14)). An appeal tribunal hearing an appeal which is made under section 11(1)($b$) of the Social Security (Recovery of Benefits) Act 1997 (c.\ 27) or which concerns the all work test (as defined in regulation 2(1) of the Social Security (Incapacity for Work) (General) Regulations 1995 (S.I.\ 1995/311)) shall consist of a legally qualified panel member and a medically qualified panel member. The expressions “legally qualified panel member” and “medically qualified panel member” are defined in regulation 1(3) of the principal Regulations by reference to Schedule 3 to those Regulations.

An appeal tribunal hearing an appeal which is made under section 4 of the Vaccine Damage Payments Act 1979 (c.\ 17) or which concerns industrial injuries benefit or severe disablement allowance shall consist of a legally qualified panel member and—
\begin{enumerate}\item[]
($a$) one medically qualified panel member;

($b$) two medically qualified panel members; or

($c$) one medically qualified panel member and a member drawn from the panel constituted under section 6 of the Social Security Act 1998 for the purposes of providing further experience for that additional member or for assisting the President in the monitoring of standards of decision making by panel members.
\end{enumerate}

Regulation 2($b$) makes a minor amendment which is consequential on the amendment made by regulation 2($a$).

These Regulations do not impose a charge on business. 

\end{document}