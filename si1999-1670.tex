\documentclass[12pt,a4paper]{article}

\newcommand\regstitle{The Social Security and Child Support (Decisions and Appeals) Amendment (No.\ 3) Regulations 1999}

\newcommand\regsnumber{1999/1670}

%\opt{newrules}{
\title{\regstitle}
%}

%\opt{2012rules}{
%\title{Child Maintenance and Other Payments Act 2008\\(2012 scheme version)}
%}

\author{S.I. 1999 No. 1623}

\date{Made
14th June 1999\\
Laid before Parliament
14th June 1999\\
Coming into force
5th July 1999}

%\opt{oldrules}{\newcommand\versionyear{1993}}
%\opt{newrules}{\newcommand\versionyear{2003}}
%\opt{2012rules}{\newcommand\versionyear{2012}}

\usepackage{csa-regs}

\setlength\headheight{27.57402pt}

\begin{document}

\maketitle

\noindent
The Secretary of State for Social Security, in exercise of powers conferred by sections 10A, 20(2), 24A and 79(1), (3), (6) and (7) of the Social Security Act 1998\footnote{\frenchspacing 1998 c. 14. Sections 10A and 24A are inserted by, respectively, paragraphs 24 and 33 of Schedule 7 to the Social Security Contributions (Transfer of Functions, etc.) Act 1999 (c. 2); section 20 is amended by paragraph 31 of that Schedule.} and of all other powers enabling him in that behalf, after consultation with the Council on Tribunals in accordance with section 8 of the Tribunals and Inquiries Act 1992\footnote{\frenchspacing 1992 c. 53.}, by this Instrument, which contains only regulations made by virtue of, or consequential upon, that section 20(2) or paragraph 24, 31 or 33 of Schedule 7 to the Social Security Contributions (Transfer of Functions, etc.)\ Act 1999\footnote{\frenchspacing 1999 c. 2.} and which is made before the end of the period of six months beginning with the coming into force of those provisions\footnote{\frenchspacing \emph{See} section 173(5)($b$) of the Social Security Administration Act 1992 (c. 5).}, makes the following Regulations: 

{\sloppy

\tableofcontents

}

\bigskip

\setcounter{secnumdepth}{-2}

\subsection[1. Citation, commencement and interpretation]{Citation, commencement and interpretation}

1.---(1)  These Regulations may be cited as the Social Security and Child Support (Decisions and Appeals) Amendment (No.\ 3) Regulations 1999 and shall come into force on 5th July 1999.

(2) In these Regulations, “the principal Regulations” means the Social Security and Child Support (Decisions and Appeals) Regulations 1999\footnote{\frenchspacing S.I. 1999/991; the relevant amending instrument is S.I. 1999/1662 (C.47).}.

\subsection[2. Amendment of the principal Regulations]{Amendment of the principal Regulations}

2.---(1)  The principal Regulations shall be amended in accordance with the following paragraphs of this regulation.

(2) In regulation 1(3) (interpretation), after the definition of “referral” there shall be added the following definition:—
\begin{quotation}
““the Transfer Act” means the Social Security Contributions (Transfer of Functions, etc.)\ Act 1999;”.
\end{quotation}

(3) After regulation 11 there shall be inserted the following regulation:—
\begin{quotation}
\subsection*{“Issues for decision by officers of Inland Revenue}

11A.---(1)  Where, on consideration of any claim or other matter, it appears to the Secretary of State that an issue arises which, by virtue of section 8 of the Transfer Act, falls to be decided by an officer of the Board, he shall refer that issue to the Board.

(2) Where—
\begin{enumerate}\item[]
($a$) the Secretary of State has decided any claim or other matter on an assumption of facts—
\begin{enumerate}\item[]
(i) as to which there appeared to him to be no dispute, but

(ii) concerning which, had an issue arisen, that issue would have fallen, by virtue of section 8 of the Transfer Act, to be decided by an officer of the Board; and
\end{enumerate}

($b$) an application for revision or an application for supersession is made in relation to the decision of that claim or other matter; and

($c$) it appears to the Secretary of State on consideration of the application that such an issue arises,
\end{enumerate}
he shall refer that issue to the Board.

(3) Pending the final decision of any issue which has been referred to the Board in accordance with paragraph (1) or (2) above, the Secretary of State may—
\begin{enumerate}\item[]
($a$) determine any other issue arising on consideration of the claim or other matter or, as the case may be, of the application,

($b$) seek a preliminary opinion of the Board on the issue referred and decide the claim or other matter or, as the case may be, the application in accordance with that opinion on that issue; or

($c$) defer making any decision on the claim or other matter or, as the case may be, the application.
\end{enumerate}

(4) On receipt by the Secretary of State of the final decision of an issue which has been referred to the Board in accordance with paragraph (1) or (2) above, the Secretary of State shall—
\begin{enumerate}\item[]
($a$) in a case to which paragraph (3)($b$) above applies—
\begin{enumerate}\item[]
(i) consider whether the decision ought to be revised under section 9 or superseded under section 10, and

(ii) if so, revise it, or, as the case may be, make a further decision which supersedes it; or
\end{enumerate}

($b$) in a case to which paragraph (3)($a$) or ($c$) above applies, decide the claim or other matter or, as the case may be, the application,
\end{enumerate}
in accordance with the final decision of the issue so referred.

(5) In paragraphs (3) and (4) above “final decision” means the decision of an officer of the Board under section 8 of the Transfer Act or the determination of any appeal in relation to that decision.”.
\end{quotation}

(4) After regulation 38 there shall be inserted the following regulation:—
\begin{quotation}
\subsection*{“Appeals raising issues for decision by officers of Inland Revenue}

38A.---(1)  Where, on consideration of any appeal, it appears to an appeal tribunal that an issue arises which, by virtue of section 8 of the Transfer Act, falls to be decided by an officer of the Board, that tribunal shall—
\begin{enumerate}\item[]
($a$) refer the appeal to the Secretary of State pending the decision of that issue by an officer of the Board; and

($b$) require the Secretary of State to refer that issue to the Board;
\end{enumerate}
and the Secretary of State shall refer that issue accordingly.

(2) Pending the final decision of any issue which has been referred to the Board in accordance with paragraph (1) above, the Secretary of State may revise the decision under appeal, or make a further decision superseding that decision, in accordance with his determination of any issue other than one which has been so referred.

(3) On receipt by the Secretary of State of the final decision of an issue which has been referred in accordance with paragraph (1) above, he shall consider whether the decision under appeal ought to be revised under section 9 or superseded under section 10, and—
\begin{enumerate}\item[]
($a$) if so, revise it or, as the case may be, make a further decision which supersedes it; or

($b$) if not, forward the appeal to the appeal tribunal which shall determine the appeal in accordance with the final decision of the issue so referred.
\end{enumerate}

(4) In paragraphs (2) and (3) above, “final decision” has the same meaning as in regulation 11A(3) and (4).”.
\end{quotation}

(5) In regulation 41 (medical examination required by appeal tribunal)—
\begin{enumerate}\item[]
($a$) in paragraph ($a$), head (v) shall be omitted;

($b$) after paragraph ($d$) there shall be inserted the following paragraph:—
\begin{quotation}
“($dd$) is whether a person is incapable of work for the purposes of the Contributions and Benefits Act;”; and
\end{quotation}

($c$) paragraph ($e$) shall be omitted.
\end{enumerate}

\bigskip

Signed 
by authority of the Secretary of State for Social Security.

{\raggedleft
\emph{Stephen C.\ Timms
}\\*Minister of State,\\*Department of Social Security

}

14th June 1999

\small

\part{Explanatory Note}

\renewcommand\parthead{--- Explanatory Note}

\subsection*{(This note is not part of the Regulations)}

These Regulations, which are made by virtue of, or in consequence of, provisions in the Social Security Act 1998 and the Social Security Contributions (Transfer of Functions, etc.)\ Act 1999 (“the Transfer Act”), further amend the Social Security and Child Support (Decisions and Appeals) Regulations 1999 (“the principal Regulations”) which make provision in relation to new arrangements for decision making and appeals introduced by the Social Security Act 1998. In particular, these Regulations make provision in the light of the transfer, from the Secretary of State to officers of the Commissioners of Inland Revenue (“the Board”), of the function of making decisions in relation to issues arising in connection with social security contributions, statutory sick pay and statutory maternity pay under the Social Security Contributions and Benefits Act 1992.

New regulations 11A and 38A are inserted into the principal Regulations to provide for the referral to the Board of issues falling to be decided by the Board which arise on consideration of a claim or other matter, an application for revision or supersession of a decision, or an appeal. The new regulations also make provision for the seeking of a preliminary opinion from the Board on any issue referred to them, and for dealing with any matter pending a final decision on the referred question (regulation 2(3) and (4)).

Amendments are also made in regulation 41 of the principal Regulations to specify incapacity for work as one of the issues in relation to which an appeal tribunal may require a person to undergo a medical examination, and to remove provision in relation to appeals about statutory sick pay and statutory maternity pay (which by virtue of the Transfer Act, no longer fall to be determined by appeal tribunals) (regulation 2(5)).  

\end{document}