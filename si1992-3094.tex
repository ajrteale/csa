\documentclass[12pt,a4paper]{article}

\newcommand\regstitle{The Child Support Fees Regulations 1992}

\newcommand\regsnumber{1992/3094}

%\opt{newrules}{
\title{\regstitle}
%}

%\opt{2012rules}{
%\title{Child Maintenance and Other Payments Act 2008\\(2012 scheme version)}
%}

\author{S.I. 1992 No. 3094}

\date{Made 9th December 1992\\Coming into force 5th April 1993}

%\opt{oldrules}{\newcommand\versionyear{1993}}
%\opt{newrules}{\newcommand\versionyear{2003}}
%\opt{2012rules}{\newcommand\versionyear{2012}}

\usepackage{csa-regs}

\setlength\headheight{27.57402pt}

\begin{document}

\maketitle

\noindent
Whereas a draft of this instrument was laid before Parliament in accordance with section 52(2) of the Child Support Act 1991\footnote{\frenchspacing 1991 c. 48.} and approved by a resolution of each House of Parliament:

Now, therefore, the Secretary of State for Social Security, in exercise of the powers conferred by sections 47, 52(4) and 54 of the Child Support Act 1991\footnote{\frenchspacing Section 54 is cited because of the meaning ascribed to the word “prescribed”.} and of all other powers enabling him in that behalf hereby makes the following Regulations:

{\sloppy

\tableofcontents

}

\setcounter{secnumdepth}{-2}

\subsection[1. Citation, commencement and interpretation]{Citation, commencement and interpretation}

1.—(1) These Regulations may be cited as the Child Support Fees Regulations 1992 and shall come into force on 5th April 1993.

(2) In these Regulations, unless the context otherwise requires—
\begin{enumerate}\item[]
“the Act” means the Child Support Act 1991;

“assessable income” means income calculated in accordance with paragraph 5 of Schedule 1 to the Act;

“assessment fee” means a fee in respect of the assessment of child support maintenance;

%“collection fee” means a fee in respect of the Secretary of State arranging for the collection of child support maintenance which becomes due, in accordance with a maintenance assessment, after that fee becomes payable, and (if necessary) arranging for the enforcement of the obligation to pay that child support maintenance in accordance with that assessment;

%Defn of ``collection fee'' in reg 1(2) substituted (7.2.94) by SI 1994/227 reg 5(2)
“collection fee” means a fee in respect of services provided by the Secretary of State for the collection of child support maintenance or for enforcing payment of such maintenance or both such collection and such enforcement;

%Defns of ``earnings top-up'', ``the Earnings Top-up Scheme'' inserted (7.10.96) by SI 1996/1945 reg 5
“earnings top-up” means the allowance paid by the Secretary of State under the rules specified in the Earnings Top-up Scheme;

“the Earnings Top-up Scheme” means the Earnings Top-up Scheme 1996\footnote{\frenchspacing This Scheme, which applies only in certain parts of Great Britain, is an extra-statutory Scheme, introduced by the Secretary of State for Social Security, having effect on 8th October 1996. Copies of the rules of this Scheme may be obtained from the Customer Services Manager, Earnings Top-up, Norcross, Blackpool \textsc{fy5 3ta}.};

“Maintenance Assessment Procedure Regulations” means the Child Support (Maintenance Assessment Procedure) Regulations 1992\footnote{\frenchspacing S.I.1992/1813.};

“parent with care” means a person who, in respect of the same child or children, is both a parent and a person with care.
\end{enumerate}

(3) In these Regulations, unless the context otherwise requires, a reference—
\begin{enumerate}\item[]
($a$) to a numbered regulation is to the regulation in these Regulations bearing that number;

($b$) in a regulation to a numbered paragraph is to the paragraph in that regulation bearing that number;

($c$) in a paragraph to a lettered or numbered sub-paragraph is to the sub-paragraph in that paragraph bearing that letter or number.
\end{enumerate}

\amendment{
Defn. of ``collection fee'' in reg. 1(2) substituted (7.2.94) by the Child Support (Miscellaneous Amendments and Transitional Provisions) Regulations 1994 reg. 5(2) (subject to transitional provisions in reg. 12).

Definitions of ``earnings top-up'' and ``the Earnings Top-up Scheme'' inserted in reg. 1(2) (7.10.96) by the Child Support (Miscellaneous Amendments) Regulations 1996 reg. 5.
}

\subsection[2. Circumstances when fees are payable]{Circumstances when fees are payable}

2.  Where a maintenance assessment is made following an application under section 4, 6 or 7 of the Act fees shall be payable to the Secretary of State in accordance with regulations 3 and 4.

\subsection[3. Liability to pay fees]{Liability to pay fees}

3.—(1) Subject to the provisions of 
%paragraphs (4) and (5), 
paragraphs (3A) to (5),  % Words substituted (18.4.95) by SI 1995/1045 reg 20(2)
where a maintenance assessment is in force the following persons shall be liable to pay fees, in accordance with the provisions of regulation 4—
\begin{enumerate}\item[]
($a$) where an application has been made under section 4 or 7 of the Act—
\begin{enumerate}\item[]
(i) the person with care if he is a parent with care; and

(ii) the absent parent,
\end{enumerate}
with respect to whom the assessment was made;

($b$) where an application has been made under section 6 of the Act and the parent with care remains within section 6(1) of the Act, the absent parent with respect to whom the assessment was made.
\end{enumerate}

(2) In a case falling within paragraph (1)($a$), the fees payable shall be the assessment fee and, where the Secretary of State exercises his powers under section 4(2) or 7(3) of the Act, the collection fee.

%(3) In a case falling within paragraph (1)($b$), the fees payable shall be the assessment fee and the collection fee.

%Reg 3(3) substituted (7.2.94) by SI 1994/227 reg 5(3)
(3) In a case falling within paragraph (1)($b$) the fee payable shall be the assessment fee and if, but only if, collection or enforcement services (or both) are provided by the Secretary of State, the collection fee.

%Reg 3(3A) inserted (18.4.95) by SI 1995/1045 reg 20(3)
(3A) No person shall be liable to pay an assessment fee or a collection fee which would otherwise become payable on or after 18th April 1995 and before 6th April 
%1997, 
%1999,  % Word substituted (13.1.97) by SI 1996/3196 reg 4
2001,  % Word substituted (6.4.99) by SI 1999/977 reg 3
and for the purposes of this paragraph a fee becomes payable—
\begin{enumerate}\item[]
($a$) in the case of a collection fee, upon the date upon which the Secretary of State arranges for the collection of, and enforcement of the obligation to pay, child support maintenance in accordance with the assessment or the anniversary of the date upon which he so arranges;

($b$) in the case of an assessment fee upon the date upon which the maintenance assessment in the case in question is made, or the anniversary thereof.
\end{enumerate}

(4) Where—
\begin{enumerate}\item[]
($a$) an application has been made under section 6 of the Act; and

($b$) the parent with care no longer falls within section 6(1) of the Act but has not requested the Secretary of State to cease taking action under section 6 of the Act,
\end{enumerate}
the case shall for the purposes of paragraph (1) be treated as if the application had been made under section 4 of the Act.

(5) No fees shall be payable by the following categories of person—
\begin{enumerate}\item[]
($a$) any person to or in respect of whom income support, family credit or disability working allowance under Part VII of the Social Security Contributions and Benefits Act 1992\footnote{\frenchspacing 1992 c. 4.},~or income-based jobseeker’s allowance under the Jobseekers Act 1995\footnote{\frenchspacing 1995 c. 18.} is paid;

($b$) any person under the age of 16 or under the age of 19 and receiving full-time education which is not advanced education;

($c$) any person whose assessable income is nil;

($d$) an absent parent to whom the provisions of paragraph 6 of Schedule 1 to the Act (protected income) apply;

%Reg 3(5)(e) inserted (7.10.96) by SI 1996/1945 reg 6
($e$) any person to or in respect of whom earnings top-up is paid.
\end{enumerate}

(6) The provisions of paragraph (5) shall—
\begin{enumerate}\item[]
($a$) be applied in relation to any occasion when a liability to pay fees under the provisions of regulation 4 would otherwise arise; and

($b$) have no effect on the fees payable by any other person.
\end{enumerate}

(7) For the purposes of paragraph (5)($b$), “advanced education” has the same meaning as in paragraph 2 of Schedule 1 to the Maintenance Assessment Procedure Regulations (meaning of “child” for the purposes of the Act), and education is to be treated as full-time education if it satisfies the conditions set out in paragraph 3 of that Schedule.

\amendment{
Reg. 3(3) substituted (7.2.94) by the Child Support (Miscellaneous Amendments and Transitional Provision) Regulations 1994 reg. 5(3) (subject to transitional provisions in reg. 12).

Words substituted in reg. 3(1) and reg. 3(3A) inserted (18.4.95) by the Child Support and Income Support (Amendment) Regulations 1995 reg. 20.

Words inserted in reg. 3(5)($a$) (7.10.96) by the Social Security and Child Support (Jobseeker's Allowance) (Consequential Amendments) Regulations 1996 reg. 4.

Reg. 3(5)($e$) inserted (7.10.96) by the Child Support (Miscellaneous Amendments) Regulations 1996 reg. 6.

Word substituted in reg. 3(3A) (13.1.97) by the Child Support (Miscellaneous Amendments) (No. 2) Regulations 1996 reg. 4.

Word substituted in reg. 3(3A) (6.4.99) by the Child Support (Miscellaneous Amendments) Regulations 1999 reg. 3.
}

\subsection[4. Fees]{Fees}

4.—(1) The first assessment fee shall become payable on the date a maintenance assessment is made following an application under section 4, 6 or 7 of the Act and an assessment fee shall thereafter become payable on each anniversary of that date.

%(2) The first collection fee shall become payable on the date the Secretary of State arranges for the collection of child support maintenance and a collection fee shall thereafter become payable on the date the assessment fee becomes payable.

%Reg 4(2) substituted (7.2.94) by SI 1994/227 reg 5(4)
(2) Where a collection fee is payable under regulation 3(2) or 3(3) the first such fee shall become payable on the date the Secretary of State first takes action to collect or enforce payment of child support maintenance, and any subsequent fee which becomes so payable shall be payable on the date the assessment fee becomes payable.

(3) Subject to paragraphs (4) and (6)—
\begin{enumerate}\item[]
($a$) the assessment fee shall be £44.00;

($b$) the collection fee shall be £34.00.
\end{enumerate}

(4) Where the first collection fee becomes payable on a date (“the first collection date”) later than the date the first assessment fee becomes payable or an anniversary of that date, 
%the amount of that fee shall be an amount equal to the collection fee specified in paragraph (3) above, multiplied by the number of complete weeks between the first collection date and the date the assessment fee next becomes payable, and divided by 52.
the amount of that fee shall be—
\begin{enumerate}\item[]
($a$) in a case where the Secretary of State arranges for the enforcement of the obligation to pay child support maintenance in accordance with the assessment whichever is the less of the following—
\begin{enumerate}\item[]
(i) the amount specified in paragraph (3); or

(ii) that amount multiplied by the number of complete weeks between the first date in respect of which arrears are due and the date the assessment fee next becomes payable divided by 52;
\end{enumerate}

($b$) in any other case an amount equal to the collection fee specified in paragraph (3), multiplied by the number of complete weeks between the first collection date and the date the assessment fee next becomes payable, and divided by 52.
\end{enumerate}  % Words substituted (18.4.95) by SI 1995/1045 reg 21(2)

(5) The provisions of this regulation in relation to collection fees shall apply where there has been an earlier period, which has terminated
except by virtue of regulation 3(3A) above% Words inserted (18.4.95) by SI 1995/1045 reg 21(3)
, during which collection fees were payable and the Secretary of State again arranges for the collection of child support maintenance, and references to “the first collection fee” shall be construed accordingly.

(6) No additional assessment fees or collection fees shall be payable by a person with respect to whom more than one maintenance assessment is in force.

(7) Where a liability to pay assessment fees or collection fees under these Regulations arises, the fees shall become due on the fourteenth day after the date the fee invoice is given or sent by the Secretary of State.

(8) If a fee invoice is sent by post to a person’s last known or notified address, it shall, for the purposes of paragraph (7), be treated as having been given or sent on the second day after the day of posting, excluding any Sunday or any day which is a bank holiday in England, Wales, Scotland or Northern Ireland under the Banking and Financial Dealings Act 1971\footnote{\frenchspacing 1971 c. 80.}.

\amendment{
Reg. 4(2) substituted (7.2.94) by the Child Support (Miscellaneous Amendments and Transitional Provisions) Regulations 1994 reg. 5(4) (subject to transitional provisions in reg. 12).

Words substituted in reg. 4(4) and words inserted in reg. 4(5) (18.4.95) by the Child Support and Income Support (Amendment) Regulations 1995 reg. 21.
}

\bigskip

Signed by authority of the Secretary of State for Social Security.

{\raggedleft
\emph{Alistair Burt}\\*Parliamentary Under-Secretary of State,\\*Department of Social Security

}

9th December 1992

\part{Explanatory Note}

\renewcommand\parthead{--- Explanatory Note}

\subsection*{(This note is not part of the Regulations)}

 These Regulations provide for the payment of fees under the Child Support Act 1991. Regulation 1 contains interpretation provisions, including definitions of “assessment fee” and “collection fee”.

  Regulation 2 provides that where a maintenance assessment is made following an application for an assessment, fees shall be payable in accordance with regulations 3 and 4.

  Regulation 3 prescribes who is liable to pay fees and which fees are payable, and lists the categories of person who are exempted from paying fees.

  Regulation 4 prescribes when the assessment and the collection fees become payable, and the amounts of those fees.

\end{document}