\documentclass[12pt,a4paper]{article}

\newcommand\regstitle{The Child Support (Miscellaneous Amendments) Regulations 2003}

\newcommand\regsnumber{2003/328}

%\opt{newrules}{
\title{\regstitle}
%}

%\opt{2012rules}{
%\title{Child Maintenance and Other Payments Act 2008\\(2012 scheme version)}
%}

\author{S.I.\ 2003 No.\ 328}

\date{Made
20th February 2003\\
%Laid before Parliament
%7th February 2003\\
Coming into force
in accordance with regulation 1(3)
}

%\opt{oldrules}{\newcommand\versionyear{1993}}
%\opt{newrules}{\newcommand\versionyear{2003}}
%\opt{2012rules}{\newcommand\versionyear{2012}}

\usepackage{csa-regs}

\setlength\headheight{27.57402pt}

\begin{document}

\maketitle

\noindent
Whereas a draft of this instrument was laid before Parliament in accordance with section 52(2) of the Child Support Act 1991\footnote{1991 c.\ 48. Section 52(2) was amended by paragraph 15 of Schedule 3 to the Child Support Act 1995 (c.\ 34) and is substituted by section 25 of the Child Support, Pensions and Social Security Act 2000 (c.\ 19).} and approved by resolution of each House of Parliament:

Now, therefore, the Secretary of State for Work and Pensions, in exercise of the powers conferred upon him by sections 17(5), 28B(2)($c$), 28E(5), 31(8), 42, 46(5), 51, 52 and 54 of, and paragraphs 5, 6, 7, 10 and 11 of Schedule 1 to and paragraphs 2(2) and 5(1) of Schedule 4B to, the Child Support Act 1991\footnote{Section 17 was substituted by section 41 of the Social Security Act 1998 (c.\ 14). Sections 28B, 46(5) and 51 and paragraph 5 of Schedule 1 were amended by, respectively, paragraph 35 of Schedule 7 to the Social Security Act 1998, paragraph 12 of Schedule 3 to the Child Support Act 1995 and paragraph 43(2) of Schedule 7 to the Social Security Act 1998, paragraph 46 of Schedule 7 to the Social Security Act 1998 and paragraph 20(7) of Schedule 2 to the Jobseekers Act 1995 (c.\ 18). Sections 28B and 28E and paragraphs 2(2) and 5(1) of Schedule 4B were inserted by, respectively, sections 2, 5 and 6(2) of and Schedule 2 to, the Child Support Act 1995. Sections 17, 42 and 51, and paragraph 11 of Schedule 1 are amended by, and sections 28B, 46(5) and paragraph 10 of Schedule 1 are substituted by, respectively, section 9 of, paragraph 11(2) and (19) of Schedule 3 to, sections 5(2), 19 and 1(3) of, and Schedule 1 to, the Child Support, Pensions and Social Security Act 2000. Section 28B is modified by regulations made under section 28G(2)($b$), as substituted by section 7 of the Child Support, Pensions and Social Security Act 2000. Section 54 is cited because of the meaning ascribed to “prescribed”. See also S.I. 2003/192.} and section 29 of the Child Support, Pensions and Social Security Act 2000\footnote{2000 c.\ 19.}, and of all other powers enabling him in that behalf, hereby makes the following Regulations: 

\enlargethispage{\baselineskip}

{\sloppy

\tableofcontents

}

\bigskip

\setcounter{secnumdepth}{-2}

\subsection[1. Citation, interpretation and commencement]{Citation, interpretation and commencement}

1.---(1)  These Regulations may be cited as the Child Support (Miscellaneous Amendments) Regulations 2003.

(2) In these Regulations—
\begin{enumerate}\item[]
“the Collection and Enforcement Regulations” means the Child Support (Collection and Enforcement) Regulations 1992\footnote{S.I.\ 1992/1989.};

“the Decisions and Appeals Regulations” means the Social Security and Child Support (Decisions and Appeals) Regulations 1999\footnote{S.I.\ 1999/991.};

“the Departure Regulations” means the Child Support Departure Direction and Consequential Amendments Regulations 1996\footnote{S.I. 1996/2907, which is revoked, with savings, by S.I. 2001/156.};

“the Maintenance Assessment Procedure Regulations” means the Child Support (Maintenance Assessment Procedure) Regulations 1992\footnote{S.I.\ 1992/1813, which is revoked, with savings, by S.I.\ 2001/157.};

“the Maintenance Assessments and Special Cases Regulations” means the Child Support (Maintenance Assessments and Special Cases) Regulations 1992\footnote{S.I.\ 1992/1815, which is revoked, with savings, by S.I.\ 2001/155.};

“the Maintenance Calculation Procedure Regulations” means the Child Support (Maintenance Calculation Procedure) Regulations 2000\footnote{S.I.\ 2001/157.};

“the Maintenance Calculations and Special Cases Regulations” means the Child Support (Maintenance Calculations and Special Cases) Regulations 2000\footnote{S.I.\ 2001/155.};

“the Transitional Regulations” means the Child Support (Transitional Provisions) Regulations 2000\footnote{S.I.\ 2000/3186.}; and

“the Variations Regulations” means the Child Support (Variations) Regulations 2000\footnote{S.I.\ 2001/156.}.
\end{enumerate}

(3) These Regulations shall come into force as follows—
\begin{enumerate}\item[]
($a$) subject to sub-paragraphs ($b$)  to ($d$), these Regulations shall come into force on the day after the day that they are made;

($b$) regulation 3 shall come into force in relation to a particular case on the day on which section 17 of the Child Support Act 1991 as amended by the Child Support, Pensions and Social Security Act 2000 comes into force in relation to that type of case;

($c$) regulation 6(3), (7)($c$)  and (8)($a$)  shall come into force on 1st April 2003; and

($d$) regulations 2, 4, 5, 6(2), (4), (5), (6), (7)($a$)  and ($b$)  and (8)($b$), 7(3), 8(2) and (4) and 10 shall come into force on 6th April 2003.
\end{enumerate}

\subsection[2. Amendment of the Collection and Enforcement Regulations]{Amendment of the Collection and Enforcement Regulations}

2.  In regulation 8(4) of the Collection and Enforcement Regulations (interpretation of Part III—deduction from earnings orders)\footnote{Regulation 8(4) was amended by S.I.\ 1999/977.}, after sub-paragraph ($e$), there shall be added—
\begin{quotation}
“($f$) working tax credit payable under section 10 of the Tax Credits Act 2002\footnote{2002 c.\ 21.}.”.
\end{quotation}

\subsection[3. Amendment of the Decisions and Appeals Regulations]{Amendment of the Decisions and Appeals Regulations}

3.---(1)  The Decisions and Appeals Regulations shall be amended in accordance with the following paragraph.

(2) In regulation 7B (date from which a decision superseded under section 17 of the Child Support Act 1991 takes effect)\footnote{Regulation 7B was inserted by S.I. 2000/3185 and amended by S.I.\ 2002/1204.} after paragraph (17) there shall be inserted—
\begin{quotation}
“(17A) Where a superseding decision is made in a case to which regulation 6A(2)($a$)  or (3) applies, and the relevant circumstance is that a person has ceased to be a person with care in relation to a qualifying child in respect of whom the maintenance calculation was made, the decision shall take effect from the first day of the maintenance period in which that person ceased to be a person with care in relation to that qualifying child.

(17B) Where a superseding decision is made in a case to which regulation 6A(3) applies, and the relevant circumstance is that there is a further qualifying child in respect of the non-resident parent and the person with care to whom the maintenance calculation being superseded relates, the superseding decision shall take effect from—
\begin{enumerate}\item[]
($a$) subject to sub-paragraph ($b$), the first day of the maintenance period in respect of the maintenance calculation in force, following—
\begin{enumerate}\item[]
(i) where an effective application is made under section 17(1) of the Child Support Act by the non-resident parent, the date on which that application is made; or

(ii) where the application made under section 17(1) of that Act is made by the person with care, or, where a maintenance calculation has been made in response to an application by a child under section 7 of that Act, by the child, the date of notification to the non-resident parent of that application;
\end{enumerate}

($b$) the first day of the maintenance period in respect of the maintenance calculation in force where the date set out in head (i)  or (ii)  falls on the first day of that maintenance period.
\end{enumerate}

(17C) For the purposes of paragraph (17B)—
\begin{enumerate}\item[]
($a$) in head (i)  of sub-paragraph ($a$), an application is effective if, were it an application for a maintenance calculation, it would comply with regulation 3(1) of the Maintenance Calculation Procedure Regulations;

($b$) in head (ii)  of sub-paragraph ($a$), notification to the non-resident parent shall take the same form in respect of an application for a supersession as it would in regulation 5 of the Maintenance Calculation Procedure Regulations, in respect of an application for a maintenance calculation.”.
\end{enumerate}
\end{quotation}

\subsection[4. Amendment of the Departure Regulations]{Amendment of the Departure Regulations}

4.---(1)  The Departure Regulations shall be amended in accordance with the following paragraphs.

(2) In regulation 9 (departure directions and persons in receipt of income support, income-based jobseeker’s allowance, working families' tax credit or disabled person’s tax credit)\footnote{Regulation 9 was substituted by S.I.\ 1998/58 and amended by S.I.\ 1999/2566 and is revoked, with savings, by S.I.\ 2001/156.} in the heading and in paragraphs (1)($b$)  and (3)($b$), for “, working families' tax credit or disabled person’s tax credit” there shall be substituted “or working tax credit”.

(3) In regulation 12 (meaning of “benefit” for the purposes of section 28E of the Act)\footnote{Regulation 12 was amended by S.I.\ 1999/2566 and is revoked, with savings, by S.I.\ 2001/156.}, for “working families' tax credit, disabled person’s tax credit”, there shall be substituted “working tax credit”.

\subsection[5. Amendment of the Maintenance Assessment Procedure Regulations]{Amendment of the Maintenance Assessment Procedure Regulations}

5.  In regulation 35A of the Maintenance Assessment Procedure Regulations (circumstances in which a reduced benefit direction shall not be given)\footnote{Regulation 35A was inserted by S.I.\ 1995/3261 and amended by S.I.\ 1996/1345 and 1999/1047 and is revoked, with savings, by S.I.\ 2001/157.}, after paragraph ($b$)  there shall be added—
\begin{quotation}
    “or

    ($c$) 
    an amount prescribed under section 9(5)($c$)  of the Tax Credits Act 2002 (increased elements of child tax credit for children or young persons with a disability) is included in an award of child tax credit payable to the parent in question or a member of that parent’s family living with him.”. 
\end{quotation}

\subsection[6. Amendment of the Maintenance Assessments and Special Cases Regulations]{\sloppy Amendment of the Maintenance Assessments and Special Cases Regulations}

6.---(1)  The Maintenance Assessments and Special Cases Regulations shall be amended in accordance with the following paragraphs.

(2) In regulation 1(2) (citation, commencement and interpretation)\footnote{Regulation 1(2) was amended by S.I.\ 1993/913, 1995/1045, 1995/3261, 1996/1345, 1996/1803, 1996/1945, 1996/3196, 1998/58, 1999/977, 1999/1510 (C.\ 43) and 1999/2566 and is revoked, with savings, by S.I.\ 2001/155.}—
\begin{enumerate}\item[]
($a$) after the definition of “Child Benefit Rates Regulations”, there shall be inserted the following definition—
\begin{quotation}
““child tax credit” means a child tax credit under section 8 of the Tax Credits Act 2002;”;
\end{quotation}

($b$) the definitions of “disabled person’s tax credit” and “family credit” shall be omitted; and

($c$) for the definition of “working families' tax credit”, there shall be substituted the following definition—
\begin{quotation}
““working tax credit” means a working tax credit under section 10 of the Tax Credits Act 2002;”.
\end{quotation}
\end{enumerate}

(3) In regulation 9(1)($e$)(i)  (exempt income: calculation or estimation of E)\footnote{Regulation 9 was amended by S.I.\ 1993/913, 1995/1045, 1995/3261, 1996/1803, 1996/1945, 1996/2907, 1998/58 and 2002/1204 and is revoked, with savings, by S.I.\ 2001/155.}, for “an invalid care allowance”, there shall be substituted “a carer’s allowance”.

(4) In regulation 10A (assessable income: working families' tax credit or disabled person’s tax credit paid to or in respect of a parent with care or an absent parent)\footnote{Regulation 10A was inserted by S.I.\ 1996/3196 and amended by S.I.\ 1999/1510 (C.\ 43) and 1999/2566 and is revoked, with savings, by S.I.\ 2001/155.}, in the heading and in paragraph (1), for “working families' tax credit or disabled person’s tax credit”, there shall be substituted “working tax credit”.

(5) In regulation 11(2)($a$)  (protected income)\footnote{Regulation 11 was amended by S.I.\ 1994/227, 1995/1045, 1995/3261, 1996/1803, 1996/1945 and 1998/58 and is revoked, with savings, by S.I.\ 2001/155.}, after head (v), there shall be added—
\begin{quotation}
“(vi) there shall be taken into account any child tax credit which is payable to the absent parent or his partner; and”.
\end{quotation}

(6) In Schedule 1 (calculation of N and M)\footnote{Schedule 1 was amended by S.I.\ 1993/913, 1995/1045, 1996/1345, 1996/1803, 1996/1945, 1996/3196, 1998/58, 1999/977, 1999/1510 (C.\ 43) and 1999/2566 and is revoked, with savings, by S.I.\ 2001/155.}—
\begin{enumerate}\item[]
($a$) in paragraphs 2(1A) and 5(5), for “working families' tax credit or disabled person’s tax credit”, there shall be substituted “working tax credit or child tax credit”;

($b$) in paragraph 7, sub-paragraphs (2) to (5) shall be omitted;

($c$) in paragraph 12($e$), for the words after “under”, there shall be substituted “regulation 62(2A)($b$)  of the Income Support (General) Regulations 1987\footnote{S.I.\ 1987/1967. Regulation 62(2A) was inserted by S.I.\ 1992/468, substituted by S.I.\ 1999/1935 and amended by S.I.\ 2001/2319 and 2002/1589.} towards such costs;”;

($d$) after paragraph 14A, there shall be inserted—
\begin{quotation}
“14B.---(1)  Subject to sub-paragraph (2), payments to a person of working tax credit shall be treated as the income of the parent who has qualified for them by his normal engagement in remunerative work at the rate payable at the effective date.

(2) Where working tax credit is payable and the amount which is payable has been calculated by reference to the earnings of the absent parent and another person—
\begin{enumerate}\item[]
($a$) if during the period which is used to calculate his earnings under paragraph 2 or, as the case may be, paragraph 5, the normal weekly earnings of that parent exceed those of the other person, the amount payable by way of working tax credit shall be treated as the income of that parent;

($b$) if during that period the normal weekly earnings of that parent equal those of the other person, half of the amount payable by way of working tax credit shall be treated as the income of that parent; and

($c$) if during that period the normal weekly earnings of that parent are less than those of that other person, the amount payable by way of working tax credit shall not be treated as the income of that parent.”; and
\end{enumerate}
\end{quotation}

($e$) paragraph 16 shall be amended as follows—
\begin{enumerate}\item[]
(i) in sub-paragraph (1), for “(6)”, there shall be substituted “(7)”; and

(ii) after sub-paragraph (6), there shall be added—
\begin{quotation}
“(7) This paragraph shall not apply to payments of working tax credit referred to in paragraph 14B.”.
\end{quotation}
\end{enumerate}
\end{enumerate}

(7) In Schedule 2 (amounts to be disregarded when calculating or estimating N and M)\footnote{Schedule 2 was amended by S.I.\ 1993/913, 1995/1045, 1995/3261, 1996/481, 1996/1345, 1996/3196, 1998/58 and 1999/977 and is revoked, with savings, by S.I.\ 2001/155.}—
\begin{enumerate}\item[]
($a$) in paragraph 22, for the words from “paragraph 19” to the end of the paragraph, there shall be substituted “paragraph 19 of Schedule 9 to the Income Support (General) Regulations 1987\footnote{S.I.\ 1987/1967. Paragraph 19 of Schedule 9 was substituted by S.I.\ 1994/527 and amended by S.I.\ 1995/516 and 2002/668.} which would have applied if he had been in receipt of income support.”;

($b$) after paragraph 48C, there shall be inserted—
\begin{quotation}
“48D.  Any payment of child tax credit.”; and
\end{quotation}

($c$) after paragraph 48D, there shall be inserted—
\begin{quotation}
“48E.  Any payment made by a local authority relating to—
\begin{enumerate}\item[]
($a$) welfare services in respect of which the Secretary of State has paid a grant to the local authority under section 93(1) of the Local Government Act 2000\footnote{2000 c.\ 22.};

($b$) welfare services in respect of which the National Assembly for Wales has paid a grant to the local authority under section 93(2) of the Local Government Act 2000; or

($c$) housing support services in respect of which the Scottish Ministers have paid a grant to the local authority under section 91(1) of the Housing (Scotland) Act 2001\footnote{2001 asp 10.},
\end{enumerate}
where the person qualified for those services.”.
\end{quotation}
\end{enumerate}

(8) Paragraph ($a$)  of Schedule 4 (cases where child support maintenance is not to be payable)\footnote{Schedule 4 was amended by S.I. 1993/913, 1995/1045 and 1999/2566 and is revoked, with savings, by S.I.\ 2001/155.} shall be amended as follows—
\begin{enumerate}\item[]
($a$) in sub-paragraph (viii), for “invalid care allowance”, there shall be substituted “carer’s allowance”; and

($b$) sub-paragraph (xi) shall be omitted.
\end{enumerate}

\subsection[7. Amendment of the Maintenance Calculation Procedure Regulations]{Amendment of the Maintenance Calculation Procedure Regulations}

7.---(1)  The Maintenance Calculation Procedure Regulations shall be amended in accordance with the following paragraphs.

(2) In regulation 5(1) (notice of an application for a maintenance calculation), after “Act, or”, there shall be inserted “an application”.

(3) In regulation 10 (circumstances in which a reduced benefit decision shall not be given), after paragraph ($b$)  there shall be added—
\begin{quotation}
    “or

    ($c$) 
    an amount prescribed under section 9(5)($c$)  of the Tax Credits Act 2002 (increased elements of child tax credit for children or young persons with a disability) is included in an award of child tax credit payable to the parent in question or a member of that parent’s family living with him.”. 
\end{quotation}

(4) In regulation 25(1) (effective dates of maintenance calculations), after “to 29”, there shall be inserted “and 31”.

(5) Regulation 29 (effective dates of maintenance calculations in specified cases)\footnote{Regulation 29 was amended by S.I.\ 2002/1204.} shall be renumbered as paragraph (1) of that regulation and—
\begin{enumerate}\item[]
($a$) in sub-paragraph ($a$), for “the day following the day”, there shall be substituted “the date”; and

($b$) at the end there shall be added—
\begin{quotation}
“(2) Where an application is treated as made under section 6(3) of the Act, references in sub-paragraphs ($a$)  and ($c$)  of paragraph (1) to “the date the application is made” shall mean whichever is the later of—
\begin{enumerate}\item[]
($a$) the date of the claim for a prescribed benefit made by or in respect of the parent with care, as determined by regulation 6 of the Social Security (Claims and Payments) Regulations 1987\footnote{S.I.\ 1987/1968. Regulation 6 was amended by S.I.\ 1988/522, 1989/1686, 1990/725 and 2208, 1991/2284 and 2741, 1993/2113 and 2319, 1996/1460 and 2431, 1997/793, 1999/2572 and 3108, 2000/636, 897 and 1982, and 2001/567 and 892.}; and

($b$) the date on which the parent with care or her partner in the claim reports to the Secretary of State (in respect of a claim for a prescribed benefit) or to the Commissioners of Inland Revenue (in respect of a claim for a tax credit) a change of circumstances, which change—
\begin{enumerate}\item[]
(i) relates to an existing claim, in respect of the parent with care, for a prescribed benefit; and

(ii) has the effect that the parent with care is treated as applying for a maintenance calculation under section 6(1) of the Act (whether or not that section already applied to that parent with care).
\end{enumerate}
\end{enumerate}

(3) For the purposes of—
\begin{enumerate}\item[]
($a$) paragraph (1), “ceased to have effect” means ceased to have effect under paragraph 16 of Schedule 1 to the Act\footnote{\emph{See} the Child Support Act 1991 (c.\ 48); paragraph 16 of Schedule 1 was amended by Schedule 9 to the Child Support, Pensions and Social Security Act 2000.}; and

($b$) paragraph (2), “prescribed benefit” means a benefit referred to in section 6(1) of the Act or prescribed in regulations made under that section.”.
\end{enumerate}
\end{quotation}
\end{enumerate}

(6) In regulation 30(2) (revocation and savings), for “regulation 31(2)”, there shall be substituted “regulation 31(1C)($b$)  and (2)”.

(7) In regulation 31 (transitional provision—effective dates and reduced benefit decisions)\footnote{Regulation 31 was amended by S.I.\ 2002/1204.}—
\begin{enumerate}\item[]
($a$) for paragraphs (1) and (2), there shall be substituted—
\begin{quotation}
“(1) Where a maintenance assessment is, or has been, in force and an application to which regulation 29 applies is made, or is treated as made under section 6(3) of the Act, that regulation shall apply as if in paragraph (1) references to—
\begin{enumerate}\item[]
($a$) a maintenance calculation in force were to a maintenance assessment in force;

($b$) a maintenance calculation having been in force were to a maintenance assessment having been in force; and

($c$) a non-resident parent in sub-paragraph ($a$), the first time it occurs in sub-paragraph ($b$)  and in sub-paragraph ($c$)(iii), were to an absent parent.
\end{enumerate}

(1A) Where regulation 28(7) of the Child Support (Transitional Provisions) Regulations 2000 (linking provisions) applies, the effective date of the maintenance calculation shall be the date which would have been the beginning of the first maintenance period in respect of the conversion decision on or after what, but for this paragraph, would have been the relevant effective date provided for in regulation 25(2) to (4).

(1B) The provisions of Schedule 3 shall apply where—
\begin{enumerate}\item[]
($a$) an effective application for a maintenance assessment has been made under the former Act (“an assessment application”); and

($b$) an effective application for a maintenance calculation is made or an application for a maintenance calculation is treated as made under the Act (“a calculation application”).
\end{enumerate}

(1C) Where the provisions of Schedule 3 apply and, by virtue of regulation 4(3) of the Assessment Procedure Regulations, the relevant date would be—
\begin{enumerate}\item[]
($a$) before the prescribed date, the application to be proceeded with shall be treated as an application for a maintenance assessment;

($b$) on or after the prescribed date, that application shall be treated as an application for a maintenance calculation and the effective date of that maintenance calculation shall be the date which would be the assessment effective date if a maintenance assessment were to be made.
\end{enumerate}

(2) Where—
\begin{enumerate}\item[]
($a$) an application for a maintenance assessment was made before the prescribed date; and

($b$) the assessment effective date of that application would be on or after the prescribed date,
\end{enumerate}
the application shall be treated as an application for a maintenance calculation and the effective date of that maintenance calculation shall be the date which would be the assessment effective date if a maintenance assessment were to be made.”;
\end{quotation}

($b$) in paragraph (4), for the words before “is before”, there shall be substituted “Where the assessment effective date”; and

($c$) in paragraph (8)($a$)—
\begin{enumerate}\item[]
(i) after the definition of “2000 Act”, there shall be inserted—
\begin{quotation}
““absent parent” has the meaning given in section 3(2) of the former Act;

“assessment effective date” means the effective date of the maintenance assessment under regulation 30 or 33(7) of the Assessment Procedure Regulations\footnote{Regulation 30 was amended by S.I.\ 1995/123, 1045 and 3261, 1996/1945 and 1999/1047 and is revoked, with savings, by S.I.\ 2001/157. Regulation 33(7) was inserted by S.I.\ 1995/3261 and is revoked, with savings, by S.I.\ 2001/157.} or regulation 3(5), (7) or (8) of the Maintenance Arrangements and Jurisdiction Regulations\footnote{Regulation 3 was amended by S.I.\ 1995/123, 1045 and 3261 and 1999/1510 (C.\ 43) and is amended by S.I.\ 2001/161. Paragraphs (5) to (8) are omitted, with savings, by S.I.\ 2001/161.}, whichever applied to the maintenance assessment in question or would have applied had the effective date not been determined under regulation 8C or 30A of the Assessment Procedure Regulations;” and
\end{quotation}

(ii) after the definition of “prescribed date”, there shall be added—
\begin{quotation}
    “and

    “relevant date” means the date which would be the assessment effective date of the application which is to be proceeded with in accordance with Schedule 3, if a maintenance assessment were to be made.”. 
\end{quotation}
\end{enumerate}
\end{enumerate}

(8) After Schedule 2 (multiple applications) there shall be added, as Schedule 3, the Schedule set out in the Schedule to these Regulations.

\subsection[8. Amendment of the Maintenance Calculations and Special Cases Regulations]{\sloppy Amendment of the Maintenance Calculations and Special Cases Regulations}

8.---(1)  The Maintenance Calculations and Special Cases Regulations shall be amended in accordance with the following paragraphs.

(2) In regulation 1(2) (citation, commencement and interpretation)—
\begin{enumerate}\item[]
($a$) after the definition of “the Act”, there shall be inserted the following definition—
\begin{quotation}
““child tax credit” means a child tax credit under section 8 of the Tax Credits Act 2002;”;
\end{quotation}

($b$) the definition of “disabled person’s tax credit” shall be omitted; and

($c$) for the definition of “working families' tax credit”, there shall be substituted the following definition—
\begin{quotation}
““working tax credit” means a working tax credit under section 10 of the Tax Credits Act 2002;”.
\end{quotation}
\end{enumerate}

(3) In regulation 8(1)($a$)  (persons treated as non-resident parents), for “qualifying child”, there shall be substituted “child, being a child in respect of whom an application for a maintenance calculation has been made or treated as made”.

(4) In the Schedule (net weekly income)\footnote{The Schedule was amended by S.I.\ 2002/1204.}—
\begin{enumerate}\item[]
($a$) in paragraph 6(1), for “sub-paragraphs (2) to (4)”, there shall be substituted “sub-paragraphs (3) and (4)”;

($b$) paragraphs 6(2) and 9(4) shall be omitted;

($c$) in paragraph 11—
\begin{enumerate}\item[]
(i) for “working families' tax credit”, wherever it appears, including in the heading, there shall be substituted “working tax credit”;

(ii) in sub-paragraph (1)—
\begin{enumerate}\item[]
($aa$) for “sub-paragraphs (2) and (3)”, there shall be substituted “sub-paragraph (2)”; and

($bb$) “under section 128 of the Contributions and Benefits Act” shall be omitted;
\end{enumerate}

(iii) in sub-paragraph (2)—
\begin{enumerate}\item[]
($aa$) for “the weekly earnings”, there shall be substituted “the earnings”; and

($bb$) for “the normal weekly earnings”, wherever it appears, there shall be substituted “the earnings”; and

($cc$) in head ($a$), the words “(as determined in accordance with Chapter II of Part IV of the Family Credit (General) Regulations 1987)” shall be omitted;
\end{enumerate}

(iv) after sub-paragraph (2), there shall be inserted—
\begin{quotation}
“(2A) For the purposes of this paragraph, “earnings” means the employment income and the income from self-employment of the non-resident parent and the other person referred to in sub-paragraph (2), as determined for the purposes of their entitlement to working tax credit.”; and
\end{quotation}

(v) sub-paragraph (3) shall be omitted;
\end{enumerate}

($d$) paragraph 13 shall be omitted; and

($e$) at the end of Part IV (Tax Credits), there shall be added—
\begin{quotation}
\subsection*{\itshape “Child tax credit}

13A.  Payments made by way of child tax credit to a non-resident parent or his partner at the rate payable at the effective date.”.
\end{quotation}
\end{enumerate}

\subsection[9. Amendment of the Transitional Regulations]{Amendment of the Transitional Regulations}

9.---(1)  The Transitional Regulations\footnote{The Child Support (Transitional Provisions) Regulations 2000 are amended by S.I.\ 2002/1204.} shall be amended in accordance with the following paragraphs.

(2) In regulation 2(1) (interpretation)—
\begin{enumerate}\item[]
($a$) in the definition of “commencement date”, for “the Assessment Procedure Regulations or the Maintenance Arrangements and Jurisdiction Regulations”, there shall be substituted “regulation 30 or 33(7) (but not regulation 8C or 30A) of the Assessment Procedure Regulations or regulation 3(5), (7) or (8) of the Maintenance Arrangements and Jurisdiction Regulations”; and

($b$) in the definition of “Maintenance Arrangements and Jurisdiction Regulations”, the words after “1992” shall be omitted.
\end{enumerate}

(3) In regulation 3(1) (decision and notice of decision)—
\begin{enumerate}\item[]
($a$) in sub-paragraph ($a$), “has an effective date before the commencement date and” shall be omitted; and

($b$) in sub-paragraph ($c$), after “interim maintenance assessment”, there shall be inserted “(whenever made)”.
\end{enumerate}

(4) In regulation 10 (circumstances in which a transitional amount is payable), after “22”, there shall be inserted “, an amount calculated under regulation 26 of the Variations Regulations”.

(5) In regulation 11 (transitional amount—basic, reduced and most flat rate cases)—
\begin{enumerate}\item[]
($a$) in paragraph (1), for “paragraph (2)”, there shall be substituted “paragraphs (2) and (3)”;

($b$) for paragraph (2), there shall be substituted—
\begin{quotation}
“(2) Subject to paragraph (3), where regulation 10 applies and there is at the calculation date more than one maintenance assessment in relation to the same absent parent, which has the meaning given in the former Act, the amount of child support maintenance payable from the case conversion date in respect of each person with care shall be determined by applying regulation 10 and paragraph (1) as if—
\begin{enumerate}\item[]
($a$) the references to the new amount were to the apportioned amount payable in respect of the person with care; and

($b$) the references to the former assessment amount were to that amount in respect of that person with care.
\end{enumerate}

(3) Where regulation 10 applies and a conversion decision is made in a circumstance to which regulation 15(3C) applies, the amount of child support maintenance payable from the case conversion date—
\begin{enumerate}\item[]
($a$) to a person with care in respect of whom an application for a maintenance calculation has been made or treated as made which is of a type referred to in regulation 15(3C)($b$), shall be the apportioned amount payable in respect of that person with care; and

($b$) in respect of any other person with care, shall be determined by applying regulation 10 and paragraph (1) as if the references to the new amount were to the apportioned amount payable in respect of that person with care and the references to the former assessment amount were to that amount in respect of that person with care.
\end{enumerate}

(4) In this regulation, “apportioned amount” means the amount payable in respect of a person with care calculated as provided in Part I of Schedule 1 to the Act and Regulations made under that Part and, where applicable, regulations 17 to 23 and Part IV of these Regulations.”.
\end{quotation}
\end{enumerate}

(6) In regulation 15 (case conversion date)—
\begin{enumerate}\item[]
($a$) in paragraph (1), for “paragraph (2)” there shall be substituted “paragraphs (2) to (3G)”;

($b$) in paragraph (2)—
\begin{enumerate}\item[]
(i) after “paragraph (3)” there shall be inserted “or (3A)”; and

(ii) for “shall be” there shall be substituted “is”;
\end{enumerate}

($c$) for paragraph (3), there shall be substituted—
\begin{quotation}
“(3) This paragraph applies where the maintenance calculation is made with respect to a relevant person who is a relevant person in relation to the maintenance assessment whether or not with respect to a different qualifying child.

(3A) This paragraph applies where the maintenance calculation is made in relation to a partner (“A”) of a person (“B”) who is a relevant person in relation to the maintenance assessment and—
\begin{enumerate}\item[]
($a$) A or B is in receipt of a prescribed benefit; and

($b$) either—
\begin{enumerate}\item[]
(i) A is the non-resident parent in relation to the maintenance calculation and B is the absent parent in relation to the maintenance assessment; or

(ii) A is the person with care in relation to the maintenance calculation and B is the person with care in relation to the maintenance assessment.
\end{enumerate}
\end{enumerate}

(3B) The case conversion date of a conversion decision made where paragraph (3C) applies is the beginning of the first maintenance period on or after the date of notification of the conversion decision.

(3C) This paragraph applies where on or after the commencement date—
\begin{enumerate}\item[]
($a$) there is a maintenance assessment in force;

($b$) an application is made or treated as made which, but for the maintenance assessment, would result in a maintenance calculation being made with an effective date before the conversion date;

($c$) the non-resident parent in relation to the application referred to in sub-paragraph ($b$)  is the absent parent in relation to the maintenance assessment referred to in sub-paragraph ($a$); and

($d$) the person with care in relation to the application referred to in sub-paragraph ($b$)  is a different person to the person with care in relation to the maintenance assessment referred to in sub-paragraph ($a$).
\end{enumerate}

(3D) The case conversion date of a conversion decision made where paragraph (3E) applies is the beginning of the first maintenance period on or after the date on which the superseding decision referred to in paragraph (3E)($d$)  takes effect.

(3E) This paragraph applies where on or after the commencement date—
\begin{enumerate}\item[]
($a$) a maintenance assessment is in force in relation to a person (“C”) and a maintenance calculation is in force in relation to another person (“D”);

($b$) C or D is in receipt of a prescribed benefit;

($c$) either—
\begin{enumerate}\item[]
(i) C is the absent parent in relation to the maintenance assessment and D is the non-resident parent in relation to the maintenance calculation; or

(ii) C is the person with care in relation to the maintenance assessment and D is the person with care in relation to the maintenance calculation; and
\end{enumerate}

($d$) the decision relating to the prescribed benefit referred to in sub-paragraph ($b$)  is superseded on the ground that C is the partner of D.
\end{enumerate}

(3F) The case conversion date of a conversion decision made where paragraph (3G) applies is the beginning of the first maintenance period on or after the date from which entitlement to the prescribed benefit referred to in paragraph (3G)($c$)  begins.

(3G) This paragraph applies where on or after the commencement date—
\begin{enumerate}\item[]
($a$) a person (“E”) in respect of whom a maintenance assessment is in force is the partner of another person (“F”) in respect of whom a maintenance calculation is in force;

($b$) either—
\begin{enumerate}\item[]
(i) E is the absent parent in relation to the maintenance assessment and F is the non-resident parent in relation to the maintenance calculation; or

(ii) E is the person with care in relation to the maintenance assessment and F is the person with care in relation to the maintenance calculation; and
\end{enumerate}

($c$) E and F become entitled to a prescribed benefit as partners.”; and
\end{enumerate}
\end{quotation}

($d$) in paragraph (4)—
\begin{enumerate}\item[]
(i) before the definition of “maintenance assessment”, there shall be inserted—
\begin{quotation}
““absent parent” has the meaning given in the former Act;”; and
\end{quotation}

(ii) in the definition of “relevant person”, the words “, which has the meaning given in the former Act,” shall be omitted.
\end{enumerate}
\end{enumerate}

(7) In regulation 16 (conversion calculation and conversion decision)—
\begin{enumerate}\item[]
($a$) in paragraph (1)($c$), for “23”, there shall be substituted “23A”;

($b$) after paragraph (2), there shall be inserted—
\begin{quotation}
“(2A) For the purposes of sections 29 to 41B of the Act and regulations made under or by virtue of those sections, a conversion decision shall be treated on or after the case conversion date as if it were a maintenance calculation.”; and
\end{quotation}

($c$) in paragraph (3)—
\begin{enumerate}\item[]
(i) for “conversion calculation”, there shall be substituted “conversion decision”; and

(ii) for “the calculation”, there shall be substituted “the conversion calculation”.
\end{enumerate}
\end{enumerate}

(8) In regulation 22 (effect on conversion calculation—maximum amount payable where relevant departure direction is on additional cases ground)—
\begin{enumerate}\item[]
($a$) in paragraph (1), for sub-paragraphs ($a$)  and ($b$)  there shall be substituted—
\begin{quotation}
“($a$) a weekly amount calculated by aggregating the first prescribed amount with the result of applying Part I of Schedule 1 to the Act to the additional income arising under the relevant departure direction; or

($b$) a weekly amount calculated by applying Part I of Schedule 1 to the Act to the aggregate of the additional income arising under the relevant departure direction and the weekly amount of any benefit, pension or allowance received by the non-resident parent which is prescribed for the purposes of paragraph 4(1)($b$)  of that Schedule.”; and
\end{quotation}

($b$) in paragraph (3)—
\begin{enumerate}\item[]
(i) in sub-paragraph ($a$), after “(special expenses)”, there shall be inserted “or a relevant property transfer”; and

(ii) after sub-paragraph ($b$)  there shall be added—
\begin{quotation}
“and

($c$) 
any benefit, pension or allowance referred to in sub-paragraph ($b$)  shall not include—
\begin{enumerate}\item[]
(i) 
in the case of industrial injuries benefit under section 94 of the Social Security Contributions and Benefits Act 1992\footnote{1992 c.\ 4.}, any increase in that benefit under section 104 (constant attendance) or 105 (exceptionally severe disablement) of that Act;

(ii) 
in the case of a war disablement pension within the meaning in section 150(2) of that Act, any award under the following articles of the Naval, Military and Air Forces etc.\ (Disablement and Death) Service Pensions Order 1983 (“the Service Pensions Order”): article 14 (constant attendance allowance), 15 (exceptionally severe disablement allowance), 16 (severe disablement occupational allowance) or 26A (mobility supplement)\footnote{S.I.\ 1983/883. Article 26A was inserted by article 4 of S.I.\ 1983/1116 and amended by S.I.\ 1983/1521, 1986/592, 1990/1308, 1991/766, 1992/710, 1995/766, 1997/286 and 2001/409.} or any analogous allowance payable in conjunction with any other war disablement pension; and

(iii) 
any award under article 18 of the Service Pensions Order (unemployability allowances) which is an additional allowance in respect of a child of the non-resident parent where that child is not living with the non-resident parent.”.
\end{enumerate}
\end{quotation}
\end{enumerate}
\end{enumerate}

(9) In regulation 24 (phasing amount)—
\begin{enumerate}\item[]
($a$) in paragraph (3), for “paragraph (4)”, there shall be substituted “paragraphs (4) and (5)”; and

($b$) after paragraph (4), there shall be added—
\begin{quotation}
“(5) Where the new amount is calculated under regulation 26(1) of the Variations Regulations, the “relevant income” for the purposes of paragraph (2) is the additional income arising under the variation.”.
\end{quotation}
\end{enumerate}

(10) In regulation 25 (maximum transitional amount)—
\begin{enumerate}\item[]
($a$) in paragraph (1)—
\begin{enumerate}\item[]
(i) for “to which regulation 15(2)” there shall be substituted “to which regulation 15(3C)”; and

(ii) for sub-paragraph ($a$)  there shall be substituted—
\begin{quotation}
“($a$) the transitional amount payable under this Part added to, where applicable, the transitional amount payable under Part IV; and”; and
\end{quotation}
\end{enumerate}

($b$) in paragraph (3)—
\begin{enumerate}\item[]
(i) after sub-paragraph ($a$), there shall be inserted—
\begin{quotation}
“($aa$) the amount of child support maintenance payable to a person with care in respect of whom there was a maintenance assessment in force immediately before the case conversion date and in respect of whom the amount payable is not calculated by reference to a phasing amount, shall be an amount calculated as provided in sub-paragraph ($a$)  and, where applicable, regulations 17 to 23;”; and
\end{quotation}

(ii) in sub-paragraph ($b$), for “the amount calculated as provided in sub-paragraph ($a$)”, there shall be substituted “the amounts calculated as provided in sub-paragraphs ($a$)  and ($aa$)”.
\end{enumerate}
\end{enumerate}

(11) In regulation 27 (subsequent decision with effect in transitional period—amount payable), after paragraph (6), there shall be added—
\begin{quotation}
“(7) Where paragraph (1) applies and at the date of the subsequent decision there is more than one person with care in relation to the same non-resident parent—
\begin{enumerate}\item[]
($a$) the amount payable to a person with care in respect of whom the amount payable is calculated by reference to a phasing amount shall be determined by applying paragraphs (1) to (5) as if references to the new amount, the subsequent decision amount and the transitional amount were to the apportioned part of the amount in question; and

($b$) the amount payable in respect of any other person with care shall be the apportioned part of the subsequent decision amount.
\end{enumerate}

(8) In paragraph (7), “apportioned part” means the amount payable in respect of a person with care calculated as provided in Part I of Schedule 1 to the Act and Regulations made under that Part and, where applicable, Parts III and IV of these Regulations.

(9) Where a subsequent decision is made in respect of a decision which is itself a subsequent decision, paragraphs (2) to (5) shall apply as if, except in paragraphs (2)($a$)  and (4)($a$), references to the new amount were to the subsequent decision amount which applied immediately before the most recent subsequent decision.”.
\end{quotation}

(12) In paragraphs (6), (7), (7A) and (8)($a$)  of regulation 28 (linking provisions), for “conversion calculation”, wherever it appears, there shall be substituted “conversion decision”.

(13) In regulation 33(1) (savings in relation to revision of or appeal against a conversion or subsequent decision)—
\begin{enumerate}\item[]
($a$) for “15(2)” where it first occurs, there shall be substituted “15(2), (3B), (3D) or (3F)”; and

($b$) for “15(2)” where it later occurs, there shall be substituted “15(2), (3B), (3D) or (3F) as the case may be”.
\end{enumerate}

\subsection[10. Amendment of the Variations Regulations]{Amendment of the Variations Regulations}

10.  In regulation 7(5)($b$)  of the Variations Regulations (prescribed circumstances)\footnote{Regulation 7 was amended by S.I.\ 2002/1204.}, for the words after “receipt of” to “of that Act)”, there shall be substituted “working tax credit under section 10 of the Tax Credits Act 2002”. 

\bigskip

Signed 
by authority of the Secretary of State for Work and Pensions.

{\raggedleft
\emph{P.~Hollis}\\*Parliamentary Under-Secretary of State,\\*Department of Work and Pensions

}

%St Andrew's House, Edinburgh

%Dated
20th February 2003

\small

\part[Schedule --- Schedule 3 to be added to the Maintenance Calculation Procedure Regulations]{Schedule\\*Schedule~3 to be added to the Maintenance Calculation Procedure Regulations}

\renewcommand\parthead{--- Schedule}

\begin{quotation}
\part*{“Schedule 3\\*Multiple applications---transitional provisions}

\section*{\itshape No maintenance assessment or calculation in force: more than one application for maintenance by the same person under section 4 or 6, or under sections 4 and 6, of the former Act and of the Act.}

1.---(1)  Where an assessment application is made and, before a maintenance assessment under the former Act is made, the applicant makes or is treated as making, as the case may be, a calculation application under section 4 or 6 of the Act, with respect to the same person with care or with respect to a non-resident parent who is the absent parent with respect to the assessment application, as the case may be, those applications shall be treated as a single application.

(2) Where an assessment application is made by a person with care—
\begin{enumerate}\item[]
($a$) under section 4 of the former Act; or

($b$) under section 6(1) of the former Act,
\end{enumerate}
and, before a maintenance assessment under the former Act is made, the person with care—
\begin{enumerate}\item[]
(i) in a case falling within head ($a$), is treated as making a calculation application under section 6(1) of the Act; or

(ii) in a case falling within head ($b$), makes a calculation application under section 4 of the Act,
\end{enumerate}
with respect to a non-resident parent who is the absent parent with respect to the assessment application, those applications shall, if the person with care does not cease to fall within section 6(1) of the Act, be treated as a single application under section 6(1) of the former Act or of the Act, as the case may be, and shall otherwise be treated as a single application under section 4 of the former Act or of the Act, as the case may be.

\section*{\itshape No maintenance assessment or calculation in force: more than one application for maintenance by a child under section 7 of the former Act and of the Act}

2.  Where a child makes an assessment application under section 7 of the former Act and, before a maintenance assessment under the former Act is made, makes a calculation application under section 7 of the Act with respect to the same person with care and a non-resident parent who is the absent parent with respect to the assessment application, both applications shall be treated as a single application.

\section*{\itshape No maintenance assessment or calculation in force: applications by different persons for maintenance}

3.---(1)  Where the Secretary of State receives more than one application for maintenance with respect to the same person with care and absent parent or non-resident parent, as the case may be, he shall, if no maintenance assessment under the former Act or maintenance calculation under the Act, as the case may be, has been made in relation to any of the applications, determine which application he shall proceed with in accordance with sub-paragraphs (2) to (11).

(2) Where an application by a person with care is made under section 4 of the former Act or of the Act, or is made under section 6 of the former Act, or is treated as made under section 6 of the Act, and an application is made by an absent parent or non-resident parent under section 4 of the former Act or of the Act, as the case may be, the Secretary of State shall proceed with the application of the person with care.

(3) Where there is an assessment application by a qualifying child under section 7 of the former Act and a calculation application is made with respect to that child by a person who is, with respect to that child, a person with care or a non-resident parent, the Secretary of State shall proceed with the application of that person with care or non-resident parent, as the case may be.

(4) Where, in a case falling within sub-paragraph (3), there is made more than one subsequent application, the Secretary of State shall apply the provisions of sub-paragraphs (2), (7), (8) or (10), as appropriate in the circumstances of the case, to determine which application he shall proceed with.

(5) Where there is an assessment application and a calculation application by more than one qualifying child under section 7 of the former Act or of the Act, in relation to the same person with care and absent parent or non-resident parent, as the case may be, the Secretary of State shall proceed with the application of the elder or, as the case may be, eldest of the qualifying children.

(6) Where there is one absent parent and one non-resident parent in respect of the same qualifying child and an assessment application and a calculation application is received from each such person respectively, the Secretary of State shall proceed with both applications, treating them as a single application.

(7) Where a parent with care is required to authorise the Secretary of State to recover child support maintenance under section 6 of the former Act and there is a calculation application under section 4 of the Act by another person with care who has parental responsibility for (or, in Scotland, parental rights over) the qualifying child or qualifying children with respect to whom the application was made under section 6 of the former Act, the Secretary of State shall proceed with the assessment application under section 6 of the former Act by the parent with care.

(8) Where—
\begin{enumerate}\item[]
($a$) a person with care makes an assessment application under section 4 of the former Act and a different person with care makes a calculation application under section 4 of the Act and those applications are in respect of the same qualifying child or qualifying children (whether or not any of those applications is also in respect of other qualifying children);

($b$) each such person has parental responsibility for (or, in Scotland, parental rights over) that child or children; and

($c$) under regulation 20 of the Child Support (Maintenance Assessments and Special Cases) Regulations 1992 (“the Maintenance Assessments and Special Cases Regulations”) one of those persons is to be treated as an absent parent or under the provisions of regulation 8 of the Maintenance Calculations and Special Cases Regulations one of those persons is to be treated as a non-resident parent, as the case may be,
\end{enumerate}
the Secretary of State shall proceed with the application of the person who does not fall to be treated as an absent parent under regulation 20 of the Maintenance Assessments and Special Cases Regulations, or as a non-resident parent under regulation 8 of the Maintenance Calculations and Special Cases Regulations, as the case may be.

(9) Where, in a case falling within sub-paragraph (8), there is more than one person who does not fall to be treated as an absent parent under regulation 20 of the Maintenance Assessments and Special Cases Regulations or as a non-resident parent under regulation 8 of the Maintenance Calculations and Special Cases Regulations, as the case may be, the Secretary of State shall apply the provisions of paragraph (10) to determine which application he shall proceed with.

(10) Where—
\begin{enumerate}\item[]
($a$) a person with care makes an assessment application under section 4 of the former Act and a different person with care makes a calculation application under section 4 of the Act and those applications are in respect of the same qualifying child or qualifying children (whether or not any of those applications is also in respect of other qualifying children); and

($b$) either—
\begin{enumerate}\item[]
(i) none of those persons has parental responsibility for (or, in Scotland, parental rights over) that child or children; or

(ii) the case falls within sub-paragraph (8)($b$)  but the Secretary of State has not been able to determine which application he is to proceed with under the provisions of sub-paragraph (8),
\end{enumerate}
\end{enumerate}
the Secretary of State shall proceed with the application of the principal provider of day to day care, as determined in accordance with sub-paragraph (11).

(11) For the purposes of sub-paragraph (10), the application of the principal provider is, where—
\begin{enumerate}\item[]
($a$) the applications are in respect of one qualifying child, the application of that person with care to whom child benefit is paid in respect of that child;

($b$) the applications are in respect of more than one qualifying child, the application of that person with care to whom child benefit is paid in respect of those children;

($c$) the Secretary of State cannot determine which application he is to proceed with under head ($a$)  or ($b$), the application of that applicant who in the opinion of the Secretary of State is the principal provider of day to day care for the child or children in question.
\end{enumerate}

(12) Subject to sub-paragraph (13), where, in any case falling within sub-paragraphs (2) to (10), the applications are not in respect of identical qualifying children, the application that the Secretary of State is to proceed with as determined by those sub-paragraphs shall be treated as an application with respect to all of the qualifying children with respect to whom the applications were made.

(13) Where the Secretary of State is satisfied that the same person with care does not provide the principal day to day care for all of the qualifying children with respect to whom an application would but for the provisions of this paragraph be made under sub-paragraph (12), he shall make separate maintenance assessments under the former Act or maintenance calculations under the Act, as the case may be, in relation to each person with care providing such principal day to day care.

(14) For the purposes of this paragraph “day to day care” has the same meaning as in the Maintenance Assessments and Special Cases Regulations or the Maintenance Calculations and Special Cases Regulations, as the case may be.

\section*{\itshape Maintenance assessment in force: subsequent application with respect to the same persons}

4.  Where—
\begin{enumerate}\item[]
($a$) a maintenance assessment is in force under the former Act;

($b$) a calculation application is made or treated as made under the section of the Act which is the same section as the section of the former Act under which the assessment application was made; and

($c$) the calculation application relates to—
\begin{enumerate}\item[]
(i) the same person with care and qualifying child or qualifying children as the maintenance assessment; and

(ii) a non-resident parent who is the absent parent with respect to the maintenance assessment,
\end{enumerate}
\end{enumerate}
the calculation application shall not be proceeded with.

\section*{\itshape Interpretation}

5.  In this Schedule, “absent parent”, “former Act” and “maintenance assessment” have the meanings given in regulation 31(8)($a$).”
\end{quotation}

\part{Explanatory Note}

\renewcommand\parthead{— Explanatory Note}

\subsection*{(This note is not part of the Regulations)}

These Regulations provide for the amendment of regulations relating to child support.

The powers exercised to make these Regulations are those in the Child Support Act 1991 (“the 1991 Act”) and the Child Support, Pensions and Social Security Act 2000 (“the 2000 Act”). Of those in the 1991 Act, some of the powers are those prior to the amendments made to that Act by the 2000 Act, in so far as those amendments are not yet fully in force, and relate to the child support scheme presently in place (“the current scheme”). Other powers are those following amendments made to the 1991 Act by the 2000 Act, which relate to the new child support scheme provided for by those amendments (“the new scheme”).

Regulations 2, 4, 5, 6(2), (4), (5), (6), (7)($a$)  and ($b$)  and (8)($b$), 7(3), 8(2) and (4) and 10 amend various sets of Regulations, some of which relate to the current scheme, some to the new scheme and, in the case of the amendments made by regulation 2, to both of the schemes. These amendments substitute references, and make related provisions, as a result of the introduction of working tax credit and child tax credit from 6th April 2003 and the linked revocation of working families' tax credit and disabled person’s tax credit.

These Regulations make a number of amendments in addition to those relating to working tax credit and child tax credit.

Regulation 3 amends the Social Security and Child Support (Decisions and Appeals) Regulations 1999 (“the Decisions and Appeals Regulations”) and will come into force at different times for different cases as determined by commencement order made under section 86(2) of the 2000 Act. Regulation 3 makes amendments to regulation 7B of the Decisions and Appeals Regulations to provide new dates from which a child support decision which is superseded under section 17 of the 1991 Act, takes effect. Inserted paragraph (17A) of regulation 7B of the Decisions and Appeals Regulations provides an effective date where a person ceases to be a person with care in relation to a qualifying child in respect of whom the maintenance calculation was made but continues to be a person with care in relation to other qualifying children in respect of whom the maintenance calculation was made. Inserted paragraph (17B) of regulation 7B of the Decisions and Appeals Regulations provides an effective date where there is a maintenance calculation in force and there is a further qualifying child in relation to the non-resident parent and the person with care to whom that maintenance calculation applies. Inserted paragraph (17C) of regulation 7B of the Decisions and Appeals Regulations provides explanations of terms used in paragraph (17B) of regulation 7B of the Decisions and Appeals Regulations.

Regulation 6(3) and (8)($a$)  amend the Child Support (Maintenance Assessments and Special Cases) Regulations 1992 to reflect the change of name from 1st April 2003 of “invalid care allowance” to “carer’s allowance”. Regulation 6(7)($c$)  amends the same Regulations to provide that specified payments from local authorities shall not be counted as income for child support purposes and this amendment also comes into force from 1st April 2003.

The remaining provisions of these Regulations come into effect the day after the date that these Regulations are made. The amendments made by these remaining provisions are to various sets of Regulations which relate to the new scheme. Those sets will come into force at different times for different cases, again as determined by commencement order made under section 86(2) of the 2000 Act.

Regulation 7 amends the Child Support (Maintenance Calculation Procedure) Regulations 2000 (“the Maintenance Calculation Procedure Regulations”). Regulation 7(2), (4) and (6) makes minor technical amendments. Regulation 7(5) amends regulation 29 of the Maintenance Calculation Procedure Regulations to clarify the meaning of certain terms in that regulation. Regulation 7(7) makes amendments which clarify and make additions to the transitional provisions in the Maintenance Calculation Procedure Regulations and inserts a provision to set the effective date of a maintenance calculation in a case where regulation 28(7) of the Child Support (Transitional Provisions) Regulations 2000 (“the Transitional Regulations”) applies (cases where a conversion calculation ceases during the transitional period). Regulation 7(8) and the Schedule add Schedule 3 to the Maintenance Calculation Procedure Regulations.

Regulation 8(3) makes a minor clarifying amendment to regulation 8(1)($a$)  of the Child Support (Maintenance Calculations and Special Cases) Regulations 2000, which provides for persons to be treated as non-resident parents where there is shared care of a child.

Regulation 9 amends the Child Support (Transitional Provisions) Regulations 2000 (“the Transitional Regulations”). Regulation 9(2) amends two of the definitions in the Transitional Regulations, to reflect the position in regulation 31 of the Maintenance Calculation Procedure Regulations. Regulation 9(3) makes minor technical amendments to regulation 3 of the Transitional Regulations. Regulation 9(4) inserts a reference to regulation 26 of the Child Support (Variations) Regulations 2000 (“the Variations Regulations”) into regulation 10 of the Transitional Regulations, to bring cases to which that regulation 26 applies within the scope of that regulation 10. Regulation 9(5), (10) and (11) makes amendments to the Transitional Regulations for cases where there are two or more persons with care in respect of one non-resident parent and one or more person with care, but not all of them, had a maintenance assessment under the current scheme or is not affected by the phasing provisions in the Transitional Regulations. Regulation 9(6) amends regulation 15 of the Transitional Regulations to make further provision for cases where the conversion of a case from the current scheme to the new scheme will be triggered and regulation 9(13) makes an amendment consequent upon the amendments made by regulation 9(6). Regulation 9(7) and (12) makes minor technical amendments to regulations 16 and 28 of the Transitional Regulations, respectively. Regulation 9(8) makes amendments to regulation 22 of the Transitional Regulations to provide for the amounts of child support maintenance payable in cases within the scope of that provision. Regulation 9(9) amends regulation 24 of the Transitional Regulations to provide for the phasing amount which is to apply in a case to which regulation 26 of the Variations Regulations applies.

These Regulations do not impose costs on business. 

\end{document}