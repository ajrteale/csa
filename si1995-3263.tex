\documentclass[12pt,a4paper]{article}

\newcommand\regstitle{The Child Support (Compensation for Recipients of Family Credit and Disability Working Allowance) Regulations 1995}

\newcommand\regsnumber{1995/3263}

%\opt{newrules}{
\title{\regstitle}
%}

%\opt{2012rules}{
%\title{Child Maintenance and Other Payments Act 2008\\(2012 scheme version)}
%}

\author{S.I. 1995 No. 3263}

\date{Made 15th December 1995\\Coming into force 23rd January 1996
}

%\opt{oldrules}{\newcommand\versionyear{1993}}
%\opt{newrules}{\newcommand\versionyear{2003}}
%\opt{2012rules}{\newcommand\versionyear{2012}}

\usepackage{csa-regs}

\setlength\headheight{27.57402pt}

\begin{document}

\maketitle

\amendment{
Regs. revoked with a saving (5.10.99) by the Social Security and Child Support (Tax Credits) Consequential Amendments Regulations 1999 reg. 6.
}

\medskip

\noindent
Whereas a draft of this instrument was laid before Parliament in accordance with section 26(4) and (5) of the Child Support Act 1995\footnote{\frenchspacing 1995 c. 34.} and approved by a resolution of each House of Parliament:

 Now, therefore, the Secretary of State for Social Security, in exercise of the powers conferred by sections 24 and 27(2) of the Child Support Act 1995 and section 54 of the Child Support Act 1991\footnote{\frenchspacing 1991 c. 48. Section 54 which is applicable to the Child Support Act 1995 by virtue of section 27(2) of that Act is cited because of the meaning ascribed to the word “prescribed”.} and of all other powers enabling him in that behalf, hereby makes the following Regulations:

{\sloppy

\tableofcontents

}

\setcounter{secnumdepth}{-2}

\subsection[1. Citation, commencement and interpretation]{Citation, commencement and interpretation}

1.—(1) These Regulations may be cited as the Child Support (Compensation for Recipients of Family Credit and Disability Working Allowance) Regulations 1995 and shall come into force on 23rd January 1996.

(2) In these Regulations—
\begin{enumerate}\item[]
“the 1995 Act” means the Child Support Act 1995 and “the 1991 Act” means the Child Support Act 1991;

“instrument for benefit payment” has the meaning prescribed in regulation 2(1) of the Social Security (Claims and Payments) Regulations 1987\footnote{\frenchspacing S.I. 1987/1968. Relevant amendment was made to regulation 2(1) by S.I. 1994/3196 which also inserted regulation 20A.}.
\end{enumerate}

\subsection[2. Changes in child support legislation relevant for the purposes of compensation]{Changes in child support legislation relevant for the purposes of compensation}

2.  A compensation payment under section 24 of the 1995 Act shall be made where a maintenance assessment in force is reduced as a result of the coming into force of any of the child support legislation set out in the Schedule to these Regulations.

\subsection[3. Definition of relevant time]{Definition of relevant time}

3.  For the purposes of section 24 of the 1995 Act, the relevant time shall be the day immediately prior to the effective date of the revised assessment.

\subsection[4. Calculation of compensation]{Calculation of compensation}

4.  Subject to regulation 5, the compensation payment referred to in section 24(2) of the 1995 Act shall be calculated by using the following formula—
\[C = \frac{A}{2} \times B\]
where—
\begin{enumerate}\item[]
C is the compensation payment payable;

A is the amount of the reduction referred to in section 24(2) of the 1995 Act;

B is the number of complete periods of 7 days within the period commencing on the effective date of the revised assessment and ending on the date of the expiry of the award of family credit or disability working allowance in force at that effective date.
\end{enumerate}

\subsection[5. Calculation of compensation in particular cases]{Calculation of compensation in particular cases}

5.—%
%(1) The amount payable under the earlier maintenance assessment shall be the amount due under a fresh assesssment made following a review of that earlier maintenance assessment under section 16, 17, 18 or 19 of the 1991 Act where—
%\begin{enumerate}\item[]
%($a$) the review and fresh assessment were made after the prescribed change in child support legislation referred to in section 24(1) of the 1995 Act has come into force but before the revised assessment is made;
%
%($b$) that review is on grounds other than that prescribed change in legislation; and
%
%($c$) the effective date of that fresh assessment is prior to the effective date of the revised assessment.
%\end{enumerate}
%
%(2) Subject to paragraph (4) where, after a revised assessment has been made, a child support officer makes a fresh assessment following a review of the earlier maintenance assessment under section 16, 17, 18 or 19\footnote{Section 19 was substituted by section 15 of the Child Support Act 1995.} of the 1991 Act, for the purposes of the calculation of a compensation payment under section 24 of the 1995 Act the amount payable under the earlier maintenance assessment shall be that payable under that assessment prior to that review.
%
%(3) Subject to paragraph (4), where a revised assessment is reviewed on the grounds set out in section 18(6) or 19(2) of the 1991 Act, and a fresh assessment of a different amount is made, the effective date of which is the same as that of that revised assessment, the compensation payment calculated under section 24 of the 1995 Act, shall be recalculated using the amount due under that fresh assessment in place of the amount due under the revised assessment.
%
%(4) In cases where the circumstances set out in paragraphs (2) and (3) both apply, the compensation payable under section 24 of the 1995 Act shall be recalculated using the amount due under the fresh assessments referred to in those paragraphs.
%
% Reg 5(1)--(3) substituted for reg. 5(1)--(4) (1.6.99) by SI 1999/1510 reg 41
(1) Where a revised assessment is replaced by a fresh maintenance assessment of a different amount by virtue of a revision under section 16 of the 1991 Act, the compensation payment calculated under section 24 of the 1995 Act shall be recalculated using the amount due under the fresh maintenance assessment in place of the amount due under the revised assessment.

(2) Subject to paragraph (3), where the earlier assessment is replaced by a fresh assessment—
\begin{enumerate}\item[]
($a$) which was made after the revised assessment; and

($b$) the effective date of that fresh assessment is before the date on which the revised assessment was made,
\end{enumerate}
the amount payable under the fresh assessment shall be ignored for the purposes of the calculation of a compensation payment under section 24 of the 1995 Act.

(3) In a case where the circumstances in paragraphs (1) and (2) apply the compensation payable under section 24 of the 1995 Act shall be recalculated using the amount due under the fresh assessments referred to in paragraphs (1) and (2).

(5) Where the circumstances set out in paragraph (6) apply, the compensation payable under section 24 of the 1995 Act for the period of the further award shall be calculated in accordance with the following formula—
\[C = \frac{A}{2} \times 26\]
where C and A have the meaning given to them in regulation 4.

(6) The circumstances referred to in paragraph (5) are—
\begin{enumerate}\item[]
($a$) the effective date of a revised assessment is not more than 4 weeks in the case of family credit or 6 weeks in the case of disability working allowance prior to the date upon which the existing award of that benefit ends;

($b$) the person to whom that award was made has applied for a further award of that benefit to be paid from the date of expiry of the existing award;

($c$) the amount payable under the further award has not taken account of the changes in child support legislation referred to in regulation 2 which are relevant to the further award.
\end{enumerate}

(7) For the purposes of paragraph (6), the existing award means the award of family credit or disability working allowance in force at the effective date of the revised assessment and the further award means the award of family credit or disability working allowance made after the existing award has expired.

\amendment{
Reg. 5(1)--(3) substituted for reg. 5(1)--(4) (1.6.99) by the Social Security Act 1998 (Commencement No. 7 and Consequential and Transitional Provisions) Order 1999 reg. 41.
}

\subsection[6. Minimum payment]{Minimum payment}

6.—(1) Subject to paragraph (2), compensation calculated under section 24 of the 1995 Act and these Regulations by reference to a revised assessment shall not be paid if the amount of that compensation is less than £5.00.

(2) Where paragraph (3) or (4) of regulation 5 applies, the prescribed minimum set out in paragraph (1) shall be applied to the recalculated compensation payment only.

(3) Where more than one revised assessment is made and a payment has been made in respect of compensation calculated by reference to one of those revised assessments, the amount of compensation to be paid in respect of any subsequent revised assessment shall be the difference between the payment that has already been made and the compensation calculated by reference to that subsequent revised assessment.

\subsection[7. Provisions for payment]{Provisions for payment}

7.—(1) The Secretary of State shall make any compensation payment under section 24 of the 1995 Act in instalments or as a lump sum as he considers appropriate in the circumstances.

(2) Any such compensation payment shall be paid on presentation of an instrument for benefit payment or by means of an instrument of payment or by such other means as appears to the Secretary of State to be appropriate in the circumstances of any particular case.

\bigskip

Signed by authority of the Secretary of State for Social Security.

{\raggedleft
\emph{A. J. B. Mitchell}\\*Parliamentary Under-Secretary of State,\\*Department of Social Security

}

15th December 1995

\clearpage

\part[Schedule --- Changes in child support legislation relevant for the purposes of compensation]{S C H E D U L E\\*Changes in child support legislation relevant for the purposes of compensation}

\renewcommand\parthead{--- Schedule}

\noindent
\begin{tabulary}{\linewidth}{JJ}
\hline
\itshape Relevant legislation & \itshape Effect\\
\hline
Regulation 43 of the Child Support Amendment Regulations&Amends regulation 6 of the Maintenance Assessments and Special Cases Regulations (additional element)\\
\textls[25]{Regulation 44(2)($a$) and} ($b$) of the Child Support Amendment Regulations&Amend regulation 9 of the Maintenance Assessments and Special Cases Regulations (exempt income)\\
\textls[25]{Regulation 46(2)($d$) and} (6) of the Child Support Amendment Regulations&Amend regulation 11 of the Maintenance Assessments and Special Cases Regulations (protected income)\\
Regulation 48 of the Child Support Amendment Regulations&Amends regulation 15 of the Maintenance Assessments and Special Cases Regulations (housing costs)\\
Regulation 50 of the Child Support Amendment Regulations&Revokes regulation 17 of the Maintenance Assessments and Special Cases Regulations (apportionment of housing costs: exempt income)\\
Regulation 54(11) of the \textls[50]{Child Support Amend\-}ment Regulations&Amends paragraph 23 of Schedule 1 to the Maintenance Assessments and Special Cases Regulations (relevant income of child)\\
\hline
\end{tabulary}

In this Schedule—
\begin{enumerate}\item[]
“The Child Support Amendment Regulations” means the Child Support and Income Support (Amendment) Regulations 1995\footnote{\frenchspacing S.I. 1995/1045.}.

“The Maintenance Assessments and Special Cases Regulations” means the Child Support (Maintenance Assessments and Special Cases) Regulations 1992\footnote{\frenchspacing S.I. 1992/1815. Regulations 6 and 11 were amended by S.I. 1994/227 and regulations 9 and 22 by S.I. 1993/913.}.
\end{enumerate}


\part{Explanatory Note}

\renewcommand\parthead{--- Explanatory Note}

\subsection*{(This note is not part of the Regulaitons)}

These Regulations made under section 24 of the Child Support Act 1995 make provision for various matters relating to compensation for a reduction in child support maintenance attributable to changes in child support legislation.

  Regulation 2 and the Schedule set out the relevant changes in child support legislation.

  Regulation 3 defines the time at which a maintenance assessment must have been in force if the amount due under it is to form part of the calculation as to whether there has been a reduction relevant for the purposes of section 24.

  Regulation 4 sets out how the compensation payment is to be calculated.

  Regulation 5 sets out how the reduction in child support maintenance is to be calculated in particular cases.

  Regulation 6 makes provision for payment of compensation not to be made where the amount of such payment is less than a specified amount.

  Regulation 7 makes provision in relation to payment of the compensation.
  These Regulations do not impose any costs on business.

\end{document}