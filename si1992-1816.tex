\documentclass[a4paper]{article}

\usepackage[welsh,english]{babel}

\usepackage[utf8]{inputenc}
\usepackage[T1]{fontenc}

%\usepackage[2012rules]{optional}

\usepackage[osf]{mathpazo}

\usepackage{perpage} %the perpage package
\MakePerPage{footnote} %the perpage package command
\renewcommand{\thefootnote}{\fnsymbol{footnote}}

\usepackage[perpage,para,symbol]{footmisc}

%\opt{newrules}{
\title{The Child Support (Arrears, Interest and Adjustment of Maintenance Assessments) Regulations 1992}
%}

%\opt{2012rules}{
%\title{Child Maintenance and Other Payments Act 2008\\(2012 scheme version)}
%}

\author{S.I. 1992 No. 1816}

\date{Made 20th July 1992\\Coming into force 5th April 1993}

%\opt{oldrules}{\newcommand\versionyear{1993}}
%\opt{newrules}{\newcommand\versionyear{2003}}
%\opt{2012rules}{\newcommand\versionyear{2012}}

\usepackage{fancyhdr}
\pagestyle{fancy}
\fancyhead[L]{}
\fancyhead[C]{\itshape The Child Support (Arrears, Interest and Adjustment of Maintenance Assessments) Regulations 1992 (S.I.~1992/1816) \parthead\phantom{...}% (\versionyear{} scheme version)
}
\fancyhead[R]{}
\fancyfoot[C]{\thepage}
\newcommand{\parthead}{}

\usepackage{array}
\usepackage{multirow}
\usepackage[debugshow]{tabulary}
\usepackage{longtable}
\usepackage{multicol}
\usepackage{lettrine}

\usepackage[colorlinks=true]{hyperref}
\usepackage{microtype}

\hyphenation{Aw-dur-dod}
\hyphenation{bank-rupt-cy}
\hyphenation{Ec-cles-ton}
\hyphenation{Eux-ton}
\hyphenation{Hogh-ton}
\hyphenation{Pres-ton}
\hyphenation{Pru-den-tial}
\hyphenation{Riv-ing-ton}

\newcolumntype{x}[1]
	{>{\raggedright}p{#1}}
\newcommand{\tn}{\tabularnewline}
\setlength\tymin{50pt}

\newcommand\amendment[1]{\subsubsection*{Notes}{\itshape\frenchspacing\footnotesize #1 \par}}

\setlength\headheight{22.87003pt}

\begin{document}

\maketitle

\noindent
 Whereas a draft of this instrument was laid before Parliament in accordance with section 52(2) of the Child Support Act 1991\footnote{\frenchspacing 1991 c. 48.} and approved by a resolution of each House of Parliament:

Now, therefore, the Secretary of State for Social Security, in exercise of the powers conferred by sections 41, 51, 52(4) and 54 of the Child Support Act 1991\footnote{\frenchspacing Section 54 is cited because of the meaning ascribed to the word “prescribed”.} and of all other powers enabling him in that behalf hereby makes the following Regulations: 

{\sloppy

\tableofcontents

}

\setcounter{secnumdepth}{-2}

\section[Part I --- General]{Part I\\*General}

\renewcommand\parthead{--- Part I}

\subsection[1. Citation, commencement and interpretation]{Citation, commencement and interpretation}

1.—(1) These Regulations may be cited as the Child Support (Arrears, Interest and Adjustment of Maintenance Assessments) Regulations 1992 and shall come into force on 5th April 1993.

(2) In these Regulations, unless the context otherwise requires—
\begin{enumerate}\item[]
“absent parent” includes a person treated as an absent parent by virtue of regulation 20 of the Maintenance Assessments and Special Cases Regulations;

“the Act” means the Child Support Act 1991;

“arrears” means arrears of child support maintenance;

“arrears of child support maintenance” is to be construed in accordance with section 41(1) and (2) of the Act;

“arrears notice” has the meaning prescribed in regulation 2;

“due date” has the meaning prescribed in regulation 3;

“Maintenance Assessments and Special Cases Regulations” means the Child Support (Maintenance Assessments and Special Cases) Regulations 1992\footnote{\frenchspacing S.I. 1992/1815.};

“Maintenance Assessment Procedure Regulations” means the Child Support (Maintenance Assessment Procedure) Regulations 1992\footnote{\frenchspacing S.I. 1992/1813.};

“parent with care” means a person who, in respect of the same child or children, is both a parent and a person with care;

“relevant person” has the same meaning as in the Maintenance Assessment Procedure Regulations.
\end{enumerate}

(3) In these Regulations, unless the context otherwise requires, a reference—
\begin{enumerate}\item[]
($a$) to a numbered regulation is to the regulation in these Regulations bearing that number;

($b$) in a regulation to a numbered paragraph is to the paragraph in that regulation bearing that number;

($c$) in a paragraph to a lettered or numbered sub-paragraph is to the sub-paragraph in that paragraph bearing that letter or number.
\end{enumerate}

\section[Part II --- Arrears of child support maintenance and interest on arrears]{Part II\\*Arrears of child support maintenance and interest on arrears}

\renewcommand\parthead{--- Part II}

\subsection[2. Applicability of provisions as to arrears and interest and arrears notices]{Applicability of provisions as to arrears and interest and arrears notices}

2.—(1) The provisions of paragraphs (2) to (4) and regulations 3 to 9 shall apply where—
\begin{enumerate}\item[]
($a$) a case falls within section 41(1) of the Act; and

($b$) the Secretary of State is arranging for the collection of child support maintenance under section 29 of the Act.
\end{enumerate}

(2) Where the Secretary of State is considering taking action with regard to a case falling within paragraph (1), he shall serve a notice (an “arrears notice”) on the absent parent.

(3) An arrears notice shall—
\begin{enumerate}\item[]
($a$) itemize the payments of child support maintenance due and not paid;

($b$) set out in general terms the provisions as to arrears and interest contained in this regulation and regulations 3 to 9; and

($c$) request the absent parent to make payment of all outstanding arrears.
\end{enumerate}

(4) Where an arrears notice has been served under paragraph (2), no duty to serve a further notice under that paragraph shall arise in relation to further arrears unless those further arrears have arisen after an intervening continuous period of not less than 12 weeks during the course of which all payments of child support maintenance due from the absent parent have been paid on time in accordance with regulations made under section 29 of the Act.

\subsection[3. Liability to make payments of interest with respect to arrears]{Liability to make payments of interest with respect to arrears}

3.—(1) Subject to paragraph (2) and regulations 4 and 5, interest shall be payable with respect to any amount of child support maintenance due in accordance with a maintenance assessment and not paid by the date specified by the Secretary of State in accordance with regulations made under section 29 of the Act (the “due date”), and shall be payable in respect of the period commencing on that day and terminating on the date that amount is paid.

(2) Subject to paragraph (3), interest with respect to arrears shall only be payable if the Secretary of State has served an arrears notice in relation to those arrears, and shall not be payable in respect of any period terminating on a date earlier than 14 days prior to the date the arrears notice is served on the absent parent.

(3) Where the Secretary of State has served an arrears notice, the provisions of paragraph (2) shall not apply in relation to further arrears unless the conditions mentioned in regulation 2(4) are satisfied.

(4) Subject to paragraph (6), where, following a review under section 
16, 17, % words inserted (5.4.93) by SI 1993/913 reg 35
18 or 19 of the Act or an appeal under section 20 of the Act, a fresh maintenance assessment is made with retrospective effect, interest in respect of the relevant retrospective period shall be payable with respect to the arrears calculated by reference to that fresh assessment.

(5) The provisions of paragraph (4) shall apply to a fresh assessment following a review under section 
16, 17, % words inserted (5.4.93) by SI 1993/913 reg 35
18 or 19 of the Act or an appeal under section 20 of the Act prior to any adjustment of that assessment under the provisions of regulation 10.

(6) For the purposes of paragraph (4), where the review under section 
16, 17, % words inserted (5.4.93) by SI 1993/913 reg 35
18 or 19 of the Act or an appeal under section 20 of the Act results in an increased assessment, and arrears in relation to that assessment arise, no interest shall be payable with respect to the arrears relating to the additional maintenance payable under that assessment in respect of any period prior to the date the absent parent is notified of the increased assessment.

\amendment{
Words inserted in reg. 3(4), (5), (6) (5.4.93) by the Child Support (Miscellaneous Amendments) Regulations 1993 reg 35.
}

\subsection[4. Circumstances in which no liability to pay interest arises]{Circumstances in which no liability to pay interest arises}

4.—(1) An absent parent shall not be liable to make payments of interest 
%with respect to arrears in respect of any period if the conditions set out in paragraph (2) are satisfied in relation to that period.
with respect to arrears—
\begin{enumerate}\item[]
($a$) in respect of any day which falls after 17th April 1995; or

($b$) in respect of any period if either of the conditions set out in paragraph (2) is satisfied in relation to that period.
\end{enumerate}  % Words substituted (18.4.95) by SI 1995/1045 reg 7(2)

(2) The conditions referred to in paragraph (1) are—
\begin{enumerate}\item[]
($a$) the absent parent did not know, and could not reasonably have been expected to know, of the existence of the arrears; or

($b$) the arrears have arisen solely in consequence of an operational or administrative error on the part of the Secretary of State or a child support officer.
\end{enumerate}

%Reg 4(3) inserted (5.4.93) by SI 1993/913 reg 36
(3) An absent parent who pays all outstanding arrears 
%of interest % Words omitted (18.4.95) by SI 1995/1045 reg 7(3)
within 28 days of the due date shall not be liable to make payments of interest with respect to those arrears.

\amendment{
Reg. 4(3) inserted (5.4.93) by the Child Support (Miscellaneous Amendments) Regulations 1993 reg. 36.

Words substituted in reg. 4(1) and omitted in reg. 4(3) (18.4.95) by the Child Support and Income Support (Amendment) Regulations 1995 reg. 7.
}

\subsection[5. Payment of arrears by agreement]{Payment of arrears by agreement}

5.—%(1) The Secretary of State may at any time enter into an agreement in writing with an absent parent (an “arrears agreement”) for the absent parent to pay all outstanding arrears.
%
%(2) An arrears agreement shall specify the dates on which the payments of arrears shall be made and the amount to be paid on each such date.
% Reg 5(1), (2) substituted (5.4.93) by SI 1993/913 reg 37
(1) The Secretary of State may at any time enter into an agreement with an absent parent (an “arrears agreement”) for the absent parent to pay all outstanding arrears by making payments on agreed dates of agreed amounts.

(2) Where an arrears agreement has been entered into, the Secretary of State shall prepare a schedule of the dates on which payments of arrrears shall be made and the amount to be paid on each such date, and shall send a copy of the schedule to such persons as he thinks fit.

(3) If an arrears agreement is entered into within 28 days of the due date, and the terms of that agreement are adhered to by the absent parent, there shall be no liability to make payments of interest under the provisions of regulation 3 with respect to the arrears in relation to which the arrears agreement was entered into.

(4) If an arrears agreement is entered into later than 28 days after the due date and the terms of that agreement are adhered to by the absent parent, there shall, with respect to the arrears in relation to which that agreement was entered into, be no liability to make payments of interest in respect of any period commencing on the date that agreement was entered into.

(5) The Secretary of State may at any time enter into a further arrears agreement with the absent parent in relation to all arrears then outstanding.

(6) Where the terms of any arrears agreement are not adhered to by an absent parent, interest shall be payable with respect to arrears in accordance with the provisions of regulation 3.

(7) It shall be an implied term of any arrears agreement that any payment of child support maintenance that becomes due whilst that agreement is in force shall be made by the due date.

\amendment{
Reg. 5(1), (2) substituted (5.4.93) by the Child Support (Miscellaneous Amendments) Regulations 1993 reg. 37.
}

\subsection[6. Rate of interest and calculation of interest]{Rate of interest and calculation of interest}

6.—(1) The rate of interest payable where liability to pay interest under regulation 3 arises shall be one per centum per annum above the median base rate prevailing from time to time calculated on a daily basis.

(2) Interest shall be payable only with respect to arrears of child support maintenance and shall not be payable with respect to any interest that has already become due.

(3) For the purposes of paragraph (1)—
\begin{enumerate}\item[]
($a$) the median base rate, in relation to a year or part of a year, is the base rate quoted by the reference banks; or, if different base rates are quoted, the rate which, when the base rate quoted by each bank is ranked in a descending sequence of seven, is fourth in the sequence;

($b$) the reference banks are the seven largest institutions—
\begin{enumerate}\item[]
(i) authorised by the Bank of England under the Banking Act 1987\footnote{\frenchspacing 1987 c. 22.}, and

(ii) incorporated in and carrying on a deposit-taking business within the United Kingdom,
\end{enumerate}
which quote a base rate in sterling; and

($c$) the size of an institution is to be determined by reference to its total consolidated gross assets in sterling, as shown in its audited end-year accounts last published.
\end{enumerate}

(4) In paragraph (3)($c$), the reference to the consolidated gross assets of an institution is a reference to the consolidated gross assets of that institution together with any subsidiary (within the meaning of section 736 of the Companies Act 1985)\footnote{\frenchspacing Section 736 was substituted by section 144(1) of the Companies Act 1989 (c. 40).}.

%Reg 6(5) inserted (5.4.93) by SI 1993/913 reg 38.
(5) Where any calculation of interest payable under this Part of these Regulations results in a fraction of a penny, that fraction shall be disregarded.

\amendment{
Reg. 6(5) inserted (5.4.93) by the Child Support (Miscellaneous Amendments) Regulations 1993 reg. 38.

}

\subsection[7. Receipt and retention of interest paid]{Receipt and retention of interest paid}

7.—(1) Payments of interest with respect to arrears shall be made in accordance with regulations under section 29 of the Act as though they were payments of child support maintenance payable in accordance with a maintenance assessment, and shall be made within 14 days of being demanded by the Secretary of State.

(2) Subject to paragraph (3), where the Secretary of State has been authorised to recover child support maintenance under section 6 of the Act and income support
or income-based jobseeker’s allowance  % Words inserted (7.10.96) by SI 1996/1345 reg 3
is paid to or in respect of the parent with care, interest with respect to arrears relating to the period during which income support 
or income-based jobseeker’s allowance  % Words inserted (7.10.96) by SI 1996/1345 reg 3
is paid shall be payable to the Secretary of State and may be retained by him.

(3) Where a case falls within paragraph (2), but the Secretary of State considers that, if the absent parent had made payments of child support maintenance due from him in accordance with that assessment, the parent with care would not have been entitled to income support
or income-based jobseeker’s allowance% Words inserted (7.10.96) by SI 1996/1345 reg 3
, any interest shall be payable to the parent with care.

(4) Where the child support maintenance payable under a maintenance assessment is payable to more than one person, any interest in respect of arrears under that assessment shall be apportioned in the same ratio as the child support maintenance that is payable, and the provisions of paragraphs (1) to (3) shall apply to each amount of interest so apportioned.

\amendment{
Words inserted in reg. 7 (7.10.96) by the Social Security and Child Support (Jobseeker's Allowance) (Consequential Amendments) Regulations 1996 reg. 3.
}

%\subsection[8. Retention of recovered arrears of child support maintenance by the Secretary of State]{Retention of recovered arrears of child support maintenance by the Secretary of State}
%
%8.  Where the Secretary of State recovers arrears from an absent parent and income support is paid to or in respect of the person with care, the Secretary of State may retain such amount of those arrears as is equal to the difference between the amount of income support that was paid to or in respect of the person with care and the amount of income support that he is satisfied would have been paid had the absent parent paid the child support maintenance due in accordance with the maintenance assessment in force by the due dates.

% Reg 8 substituted (22.1.96) by SI 1995/3261 reg 2
\subsection[8. Retention of recovered arrears of child support maintenance by the Secretary of State]{Retention of recovered arrears of child support maintenance by the Secretary of State}

8.—(1) This regulation applies where—
\begin{enumerate}\item[]
(i) the Secretary of State recovers arrears from an absent parent under section 41 of the Act; and

(ii) income support
or income-based jobseeker’s allowance  % Words inserted (7.10.96) by SI 1996/1345 reg 3
is paid to or in respect of the person with care or was paid to or in respect of that person at the date or dates upon which the payment or payments of child support maintenance referred to in paragraph (2) should have been made.
\end{enumerate}

(2) Where paragraph (1) applies, the Secretary of State may retain such amount of those arrears as is equal to the difference between the amount of income support 
or income-based jobseeker’s allowance  % Words inserted (7.10.96) by SI 1996/1345 reg 3
that was paid to or in respect of the person with care and the amount of income support 
or income-based jobseeker’s allowance  % Words inserted (7.10.96) by SI 1996/1345 reg 3
that he is satisfied would have been paid had the absent parent paid, by the due dates, the amounts due under the child support maintenance assessment in force or to be taken to have been in force by virtue of the provisions of section 41(2A) of the Act.

\amendment{
Reg. 8 substituted (22.1.96) by the Child Support (Miscellaneous Amendments) (No.\ 2) Regulations 1995 reg. 2.

Words inserted in reg. 8 (7.10.96) by the Social Security and Child Support (Jobseeker's Allowance) (Consequential Amendments) Regulations 1996 reg. 3.

}

\section[Part III --- Attribution of payments and adjustment of the amount payable under a maintenance assessment]{\sloppy Part III\\*\textls[25]{Attribution of payments and adjustment of the} amount payable under a maintenance assessment}

\renewcommand\parthead{--- Part III}

\subsection[9. Attribution of payments]{Attribution of payments}

9.  Where a maintenance assessment is or has been in force and there are arrears of child support maintenance, the Secretary of State may attribute any payment of child support maintenance made by an absent parent to child support maintenance due as he thinks fit.

\subsection[10. Adjustment of the amount payable under a maintenance assessment]{Adjustment of the amount payable under a maintenance assessment}

%10.—(1) Where a new or a fresh maintenance assessment has retrospective effect, the amount payable under that assessment may be adjusted by a child support officer for the purpose of taking account of the retrospective effect of the assessment by such amount as, subject to the provisions of paragraph (4), he considers appropriate in the circumstances of the case.
%
%(2) Subject to paragraph (3), where the payments of child support maintenance have been over-payments or under-payments, the amount payable under a maintenance assessment may be adjusted by a child support officer for the purpose of taking account of such over-payments or under-payments by such amount as, subject to the provisions of paragraph (5), he considers appropriate in the circumstances of the case.
%
%(3) The provisions of paragraph (2) shall not apply to any case falling within section 41 of the Act.
%
%(4) Where a case falls within paragraph (1), the child support officer shall—
%\begin{enumerate}\item[]
%($a$) in the case of a new assessment, not increase the amount payable under that assessment by an amount greater than 1.5 multiplied by that assessment;
%
%($b$) in the case of a fresh assessment, not adjust the amount payable under that assessment by an amount greater than 1.5 multiplied by the difference between the amount payable under the earlier assessment and the amount payable under the fresh assessment.
%\end{enumerate}
%
%(5) Where a case falls within paragraph (2), the child support officer shall not adjust the amount payable under a maintenance assessment by an amount greater than 1.5 multiplied by the mean over-payment or the mean under-payment, as the case may be.
%
%(6) For the purposes of paragraph (5), the mean over-payment or the mean underpayment shall be the total net over-payment or the total net under-payment divided by the number of occasions on which, in respect of the period being taken into acount for the purposes of paragraph (2), there have been over-payments or, as the case may be, under-payments of child support maintenance.

%Reg 10 substituted (18.4.95) by SI 1995/1045 reg 8
10.—(1) Where for any reason, including the retrospective effect of a new or fresh maintenance assessment, there has been an overpayment of child support maintenance, a child support officer may, for the purpose of taking account of that overpayment—
\begin{enumerate}\item[]
($a$) apply the amount overpaid to reduce any arrears of child support maintenance due under any previous maintenance assessment made in respect of the same relevant persons; or

($b$) where there is no previous relevant maintenance assessment or an overpayment remains after the application for sub-paragraph ($a$), and subject to paragraph (4), adjust the amount payable under a current maintenance assessment by such amount as he considers appropriate in all the circumstances of the case having regard in particular to—
\begin{enumerate}\item[]
(i) the circumstances of the absent parent and the person with care;

\begin{sloppypar}
(ii) the amount of the overpayment in relation to the amount due under the current maintenance assessment; and
\end{sloppypar}

(iii) the period over which it would be reasonable for the overpayment to be rectified.
\end{enumerate}
\end{enumerate}

\begin{sloppypar}
(2) Where a child support officer has adjusted the amount payable under a maintenance assessment under the provisions of paragraph (1) and that maintenance assessment is subsequently reviewed under section 16, 17, 18 or 19 of the Act and a fresh maintenance assessment made, that adjustment shall, subject to paragraph (3), continue to apply to the amount payable under that fresh maintenance assessment unless a child support officer is satisfied that such adjustment would not be appropriate in all the circumstances of the case.
\end{sloppypar}

(3) Where a child support officer is satisfied that the adjustment referred to in paragraph (2) would not be appropriate, he may cancel that adjustment or he may adjust the amount payable under that fresh maintenance assessment as he sees fit, having regard to the matters specified in heads (i) to (iii) of sub-paragraph ($b$) of paragraph (1).

(4) Any adjustment under the provisions of paragraph (1), (2) or (3) shall not reduce the amount payable under a maintenance assessment to less than the minimum amount prescribed under paragraph 7 of Schedule 1 to the Act.

\amendment{
Reg. 10 substituted (18.4.95) by the Child Support and Income Support (Amendment) Regulations 1995 reg. 8.
}

% Reg 10A inserted (22.1.96) by SI 1995/3261 reg 3

\subsection[10A. Reimbursement of a repayment of overpaid child maintenance]{Reimbursement of a repayment of overpaid child maintenance}

10A.—(1) The Secretary of State may require a relevant person to repay the whole or a part of any payment by way of reimbursement made to an absent parent under section 41B(2) of the Act where the overpayment referred to in section 41B(1) of the Act arose—
\begin{enumerate}\item[]
($a$) in respect of the amount payable under a maintenance assessment calculated in accordance with Part I of Schedule 1 to the Act and where income support
or income-based jobseeker’s allowance% Words inserted (7.10.96) by SI 1996/1345 reg 3
, family credit or disability working allowance was not in payment to that person at any time during the period in which that overpayment occurred or at the date or dates on which the payment by way of reimbursement was made; or

($b$) in respect of the amount payable under an interim maintenance assessment and that amount has not been varied under regulation 8D(1) of the Maintenance Assessment Procedure Regulations following the making of a maintenance assessment calculated in accordance with Part I of Schedule 1 to the Act.
\end{enumerate}

(2) In a case falling within section 4 or 7 of the Act, where the circumstances set out in section 41B(6) apply, the Secretary of State may retain out of the child support maintenance collected by him in accordance with section 29 of the Act such sums as cover the amount of any payment by way of reimbursement required by him from the relevant person under section 41B(3) of the Act.

\amendment{
Reg. 10A inserted (22.1.96) by the Child Support (Miscellaneous Amendments) (No.\ 2) Regulations 1995 reg. 3.

Words inserted in reg. 10A (7.10.96) by the Social Security and Child Support (Jobseeker's Allowance) (Consequential Amendments) Regulations 1996 reg. 3.
}

\section[Part IV --- Miscellaneous]{Part IV\\*Miscellaneous}

\renewcommand\parthead{--- Part IV}

%\subsection[11. Notifications following an adjustment under the provisions of regulation 10]{Notifications following an adjustment under the provisions of regulation 10}
%Heading substituted (18.4.95) by SI 1995/1045 reg 9(2)
\subsection[11. Notifications following a cancellation or adjustment under the provisions of regulation 10]{Notifications following a cancellation or adjustment under the provisions of regulation 10}

11.—%(1) Where a child support officer has, under the provisions of regulation 10, adjusted the amount payable under a maintenance assessment, he shall immediately notify the relevant persons, so far as that is reasonably practicable, of the amount and period of the adjustment, and the amount payable during the period of the adjustment.
%Reg 11(1) substituted (18.4.95) by SI 1995/1045 reg 9(3)
(1) Where a child support officer has, under the provisions of regulation 10, cancelled an adjustment in accordance with the provisions of paragraph (3) of that regulation or adjusted the amount payable under a maintenance assessment, he shall immediately notify the relevant persons, so far as is reasonably practicable, of the cancellation or, of the amount and period of the adjustment, and the amount payable during the period of the adjustment.

(2) A notification under paragraph (1) shall include information as to the provisions of regulation 12(1) and regulation 13(1) in so far as it relates to time limits for an application for a review under regulation 12(1).

\amendment{
Reg. 11(1) and heading substituted (18.4.95) by the Child Support and Income Support (Amendment) Regulations 1995 reg. 9.
}

%\subsection[12. Review of adjustments under regulation 10 or of the calculation of arrears or interest]{Review of adjustments under regulation 10 or of the calculation of arrears or interest}
%
%12.—(1) Where the amount payable under a maintenance assessment has been adjusted under the provisions of regulation 10, a relevant person may apply to the Secretary of State for a review of that adjustment as if it were a case falling within section 18 of the Act and, subject to the modifications set out in paragraph (2), section 18(5) to (9) and (11) of the Act shall apply to such a review.
%
%(2) The modifications referred to in paragraph (1) are—
%\begin{enumerate}\item[]
%($a$) section 18(6) of the Act shall have effect as if for “the refusal, assessment or cancellation in question” there is substituted “the adjustment of the amount payable under regulation 10 of the Child Support (Arrears, Interest and Adjustment of Maintenance Assessments) Regulations 1992”;
%
%($b$) section 18(9) of the Act shall have effect as if for “a maintenance assessment or (as the case may be) a fresh maintenance assessment” there is substituted “a revised adjustment of the amount payable under regulation 10 of the Child Support (Arrears, Interest and Adjustment of Maintenance Assessments) Regulations 1992”.
%\end{enumerate}
%
%%(3) Where there has been a calculation of arrears due under a maintenance assessment or a calculation of the interest payable with respect to arrears, a relevant person may apply to the Secretary of State for a review of that calculation as if it were a case falling within section 18 of the Act and, subject to the modifications set out in paragraph (4), section 18(5) to (9) and (11) of the Act shall apply to such a review.
%%
%%(4) The modifications referred to in paragraph (3) are—
%%\begin{enumerate}\item[]
%%($a$) section 18(6) of the Act shall have effect as if—
%%\begin{enumerate}\item[]
%%(i) for “the refusal, assessment or cancellation in question” there is substituted “the calculation of arrears due under a maintenance assessment or the calculation of the interest payable with respect to arrears”;
%%
%%(ii) after “law” in paragraph ($c$)
%%there is inserted—
%%\begin{quotation}
%%“or
%%
%%(d) involved an arithmetical error”;
%%\end{quotation}
%%\end{enumerate}
%%
%%($b$) section 18(9) of the Act shall have effect as if for “a maintenance assessment or (as the case may be) a fresh maintenance assessment” there is substituted “a fresh calculation of the arrears due under a maintenance assessment or a fresh calculation of the interest payable with respect to arrears”.
%%\end{enumerate}
%
%% Reg 12(3), (4) omitted (5.4.93) by SI 1993/913 reg 39.
%
%(5) Where the amount payable under a maintenance assessment has been adjusted under the provisions of regulation 10 a child support officer may revise that adjustment if he is satisfied that one or more of the circumstances set out in paragraphs ($a$) to ($c$) of section 19(1) of the Act apply to that adjustment.
%
%%(6) Where there has been a calculation of the arrears due under a maintenance assessment or a calculation of interest payable with a respect to arrears, a child support officer may re-calculate the arrears or the interest if he is satisfied that one or more of the circumstances set out in paragraphs ($a$) to ($c$) of section 19(1) of the Act apply or that there has been an arithmetical error in the calculation.
%
%% Reg 12(6) omitted (5.4.93) by SI 1993/913 reg 39.

% Reg 12 substituted (18.4.95) by SI 1995/1045 reg 10

\subsection[12. Review of cancellations or adjustments under regulation 10]{Review of cancellations or adjustments under regulation 10}

\begin{sloppypar}
12.—(1) Where an adjustment made under regulation 10 has been cancelled under paragraph (3) of that regulation or where the amount payable under a maintenance assessment has been adjusted under the provisions of that regulation, a relevant person may apply to the Secretary of State for a review of that cancellation or adjustment as if it were a case falling within section 18 of the Act and—
\end{sloppypar}
\begin{enumerate}\item[]
($a$) section 18(5), (7), (8) and regulations made under section 18(11); and

($b$) subject to the modifications set out in paragraph (2), section 18(6) and (9),
\end{enumerate}
shall apply to such a review.

(2) The modifications referred to in paragraph (1) are—
\begin{enumerate}\item[]
($a$) section 18(6) of the Act shall have effect as if for the words “the refusal, assessment or cancellation in question” there are substituted the words “the adjustment of the amount payable, or the cancellation of the adjustment of the amount payable, under regulation 10 of the Child Support (Arrears, Interest and Adjustment of Maintenance Assessments) Regulations 1992”;

($b$) section 18(9) of the Act shall have effect as if for the words “a maintenance assessment or (as the case may be) a fresh maintenance assessment should be made” there are substituted the words “a cancelled adjustment should be reinstated or a revised adjustment of the amount payable under regulation 10 of the Child Support (Arrears, Interest and Adjustment of Maintenance Assessments) Regulations 1992 should be made.”.
\end{enumerate}

(3) Where an adjustment has been cancelled or the amount payable under a maintenance assessment has been adjusted under the provisions of regulation 10, a child support officer may reinstate that cancelled adjustment or revise that adjustment if he is satisfied that one or more of the circumstances set out in paragraphs ($a$) to ($c$) of section 
%19(1) 
19(2)  % Word substituted (22.1.96) by SI 1995/3261 reg 4
of the Act apply to that cancellation or that adjustment.

\amendment{
%Reg. 12(3), (4), (6) omitted (5.4.93) by the Child Support (Miscellaneous Amendments) Regulations 1993 reg. 39.

Reg. 12 substituted (18.4.95) by the Child Support and Income Support (Amendment) Regulations 1995 reg. 10.

Word substituted in reg. 12(3) (22.1.96) by the Child Support (Miscellaneous Amendments) (No.\ 2) Regulations 1995 reg. 4.
}

\subsection[13. Procedure and notifications on applications and reviews under regulation 12]{Procedure and notifications on applications and reviews under regulation 12}

13.—(1) The provisions of regulations 24 to 26 of the Maintenance Assessment Procedure Regulations shall apply to an application for a review under regulation 12(1)%
%or (3) % Words omitted (5.4.93) by SI 1993/913 reg 40($a$)
.

(2) Where a child support officer refuses an application for a review under regulation 12(1) 
%or (3) % Words omitted (5.4.93) by SI 1993/913 reg 40($a$)
on the grounds set out in section 18(6) of the Act (as applied by regulation 12), he shall immediately notify the applicant, so far as that is reasonably practicable, and shall give the reasons for his refusal in writing.

(3) Where a child support officer adjusts the amount payable under a maintenance assessment following a review under regulation 12(1) 
%or (5), 
or (3),  % Words substituted (18.4.95) by SI 1995/1045 reg 11(2)
he shall immediately notify the relevant persons, so far as that is reasonably practicable, of the amount and period of the adjustment, and the amount payable during the period of adjustment.

(4) Where a child support officer refuses to adjust the amount payable under a maintenance assessment following a review under regulation 12(1) he shall immediately notify the relevant persons, so far as that is reasonably practicable, of the refusal, and shall give the reasons for his refusal in writing.

%(5) Where a child support officer has conducted a review under regulation 12(3), or has revised the calculation of the arrears due or the interest payable with respect to arrears following a review under regulation 12(6), he shall immediately notify the relevant persons, so far as that is reasonably practicable, of his decision.
%Reg 13(5) omitted (5.4.93) by SI 1993/913 reg 40($b$)

%Reg 13(5) inserted (18.4.95) by SI 1995/1045 reg 11(3)
(5) Where a child support officer refuses to reinstate, or reinstates, a cancelled adjustment following a review under regulation 12(1), he shall immediately notify the relevant persons, so far as that is reasonably practicable, of the refusal or reinstatement, as the case may be, and shall give reasons for his refusal in writing.

%(6) A notification under 
%%paragraphs (2) to (5) 
%paragraphs (2) to (4) %Words substituted (5.4.93) by SI 1993/913 reg 40($c$)
%shall include information as to the provisions of section 20 of the Act.

% Reg 13(6), (7) substituted for reg. 13(6) (22.1.96) by SI 1995/3261 reg 5
(6) A notification under paragraphs (2), (4) and (5), and under paragraph (3) following a review under regulation 12(1), shall include information as to the provisions of section 20 of the Act.

(7) A notification under paragraph (3) following a review under regulation 12(3) shall include information as to the provisions of section 18 of the Act.

\amendment{
Words 
%substituted in reg. 13(6), words 
omitted in reg. 13(1), (2) and reg. 13(5) omitted (5.4.93) by the Child Support (Miscellaneous Amendments) Regulations 1993 reg. 40.

Words substituted in reg. 13(3) and reg. 13(5) inserted (18.4.95) by the Child Support and Income Support (Amendment) Regulations 1995 reg. 11.

Reg. 13(6), (7) substituted for reg. 13(6) (22.1.96) by the Child Support (Miscellaneous Amendments) (No.\ 2) Regulations 1995 reg. 5.
}

\subsection[14. Non-disclosure of information to third parties]{Non-disclosure of information to third parties}

14.  The provisions of regulation 10(3) of the Maintenance Assessment Procedure Regulations shall apply to any document given or sent under the provisions of regulation 11 or 13.

\subsection[15. Applicability of regulations 1(6) and 53 to 56 of the Maintenance Assessment Procedure Regulations]{Applicability of regulations 1(6) and 53 to 56 of the Maintenance Assessment Procedure Regulations}

15.  Regulations 1(6) and 53 to 56 of the Maintenance Assessment Procedure Regulations shall apply to the provisions of these Regulations.

\bigskip

Signed by authority of the Secretary of State for Social Security.

{\raggedleft
\emph{Alistair Burt}\\*Parliamentary Under-Secretary of State,\\*Department of Social Security

}

20th July 1992

\part{Explanatory Note}

\renewcommand\parthead{--- Explanatory Note}

\subsection*{(This note is not part of the Regulations)}

 These Regulations make provision in relation to arrears of child support maintenance payable under the Child Support Act 1991 (“the Act”), interest on such arrears, and the adjustment of maintenance assessments.

  Regulation 1 contains interpretation provisions.

  Regulations 2, 3 and 4 provide for the service of an arrears notice where arrears of child support maintenance have arisen, and prescribe the circumstances where liability to make payments of interest with respect to arrears arises.

  Regulation 5 provides for the payment of arrears by agreement and contains provisions as to interest where such an agreement has been entered into.

  Regulation 6 prescribes the rate of interest payable on arrears, and how interest is to be calculated.

  Regulation 7 makes provision as to the payment of interest and for the retention of interest by the Secretary of State.

  Regulation 8 prescribes circumstances where the Secretary of State may retain recovered arrears of child support maintenance.

  Regulation 9 provides for the attribution of payments where there are arrears of child support maintenance.

  Regulation 10 provides for the adjustment of amounts payable under a maintenance assessment, and regulation 11 makes provision in respect of notifications following such an adjustment.

  Regulation 12 provides for reviews of adjustments under regulation 10 and of the calculation of arrears and of interest payable with respect to arrears. Regulation 13 makes provision as to procedure and notifications on applications and reviews under regulation 12.

  Regulations 14 and 15 apply certain provisions of the Child Support (Maintenance Assessment Procedure) Regulations 1992 to the provisions of these Regulations.


\end{document}