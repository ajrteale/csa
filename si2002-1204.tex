\documentclass[12pt,a4paper]{article}

\newcommand\regstitle{The Child Support (Miscellaneous Amendments) Regulations 2002}

\newcommand\regsnumber{2002/1204}

%\opt{newrules}{
\title{\regstitle}
%}

%\opt{2012rules}{
%\title{Child Maintenance and Other Payments Act 2008\\(2012 scheme version)}
%}

\author{S.I.\ 2002 No.\ 1204}

\date{Made
29th April 2002\\
%Laid before Parliament
%26th March 2002\\
Coming into force
in accordance with regulation 1(3)
}

%\opt{oldrules}{\newcommand\versionyear{1993}}
%\opt{newrules}{\newcommand\versionyear{2003}}
%\opt{2012rules}{\newcommand\versionyear{2012}}

\usepackage{csa-regs}

\setlength\headheight{27.57402pt}

\begin{document}

\maketitle

\noindent
Whereas a draft of this instrument was laid before Parliament in accordance with section 52(2) of the Child Support Act 1991\footnote{1991 c.\ 48.} and approved by resolution of each House of Parliament:

Now, therefore, the Secretary of State for Work and Pensions, in exercise of the powers conferred upon him by sections 14(1), 16(1) and (4), 17(5), 20(4), 28B(2)($c$), 28E(1), 28G, 46(10), 51, 52(4), 54 and 57 of, and paragraphs 5, 10 and 11 of Schedule 1 and paragraphs 3, 4 and 5 of Schedule 4B to, the Child Support Act 1991\footnote{Section 51 and paragraph 5 of Schedule 1 were amended by, respectively, paragraph 46 of Schedule 7 to the Social Security Act 1998 (c.\ 14) and paragraph 20(7) of Schedule 2 to the Jobseekers Act 1995 (c.\ 18). Sections 16(1) and (4) and 17(5) were substituted by sections 40 and 41 respectively of the Social Security Act 1998. Sections 28E(1) and 28G and Schedule 4B were inserted by, respectively, sections 5 and 7 of, and Schedule 2 to, the Child Support Act 1995 (c.\ 34). Sections 14(1), 16(1) and 28E(1) and paragraph 11 of Schedule 1 were amended by, and sections 20, 28B, 28G, 46(10), Part I of Schedule 1 and Schedule 4B were substituted by, respectively, sections 12, 8, 5(4), 1(2), 10, 5(2), 7 and 19 of, and Schedule 1 and Part II of Schedule 2 to, the Child Support, Pensions and Social Security Act 2000 (c.\ 19). Section 54 is cited because of the meaning ascribed to “prescribed”.} and section 29 of the Child Support, Pensions and Social Security Act 2000\footnote{2000 c.\ 19.}, and of all other powers enabling him in that behalf, after consultation with the Council on Tribunals in accordance with section 8 of the Tribunals and Inquiries Act 1992\footnote{1992 c.\ 53.}, hereby makes the following Regulations: 

\enlargethispage{-\baselineskip}

{\sloppy

\tableofcontents

}

\bigskip

\setcounter{secnumdepth}{-2}

\subsection[1. Citation, commencement and interpretation]{Citation, commencement and interpretation}

1.---(1)  These Regulations may be cited as the Child Support (Miscellaneous Amendments) Regulations 2002.

(2) In these Regulations—
\begin{enumerate}\item[]
“the Decisions and Appeals Regulations” means the Social Security and Child Support (Decisions and Appeals) Regulations 1999\footnote{S.I.\ 1999/991.};

“the Departure Regulations” means the Child Support Departure Direction and Consequential Amendments Regulations 1996\footnote{S.I.\ 1996/2907.};

“the Information Regulations” means the Child Support (Information, Evidence and Disclosure) Regulations 1992\footnote{S.I.\ 1992/1812.};

“the Maintenance Assessments and Special Cases Regulations” means the Child Support (Maintenance Assessments and Special Cases) Regulations 1992\footnote{S.I.\ 1992/1815.};

“the Maintenance Calculation Procedure Regulations” means the Child Support (Maintenance Calculation Procedure) Regulations 2000\footnote{S.I.\ 2001/157.};

“the Maintenance Calculations and Special Cases Regulations” means the Child Support (Maintenance Calculations and Special Cases) Regulations 2000\footnote{S.I.\ 2001/155.};

“the Transitional Regulations” means the Child Support (Transitional Provisions) Regulations 2000\footnote{S.I.\ 2000/3186.};

“the Variations Regulations” means the Child Support (Variations) Regulations 2000\footnote{S.I.\ 2001/156.}; and

“the Variations Modification Regulations” means the Child Support (Variations) (Modification of Statutory Provisions) Regulations 2000\footnote{S.I.\ 2000/3173.}.
\end{enumerate}

(3) These Regulations shall come into force as follows—
\begin{enumerate}\item[]
($a$) subject to sub-paragraph ($b$), these Regulations shall come into force on the day after the day that they are made;

($b$) regulations 2 and 4($a$)  shall come into force in relation to a particular case on the day on which, respectively, sections 16, 17 and 20 and sections 51 and 54 of the Child Support Act 1991 as amended by the Child Support, Pensions and Social Security Act 2000 come into force for the purposes of that type of case.
\end{enumerate}

\subsection[2. Amendment of the Decisions and Appeals Regulations]{Amendment of the Decisions and Appeals Regulations}

2.---(1)  The Decisions and Appeals Regulations shall be amended in accordance with the following paragraphs.

(2) In regulation 3A\footnote{Regulation 3A was inserted into S.I.\ 1999/991 by regulation 5 of S.I.\ 2000/3185.} (revision of child support decisions)—
\begin{enumerate}\item[]
($a$) in paragraph (1)—
\begin{enumerate}\item[]
(i) after sub-paragraph ($c$)  there shall be inserted—
\begin{quotation}
“($cc$) if an appeal is made under section 20 of the Child Support Act against a decision within the time prescribed in regulation 31, or in a case to which regulation 32 applies within the time prescribed in that regulation, but the appeal has not been determined;”; and
\end{quotation}

(ii) at the end of sub-paragraph ($e$)  there shall be added—
\begin{quotation}
    “; or

    ($f$) 
    if the grounds for revision are that a person with respect to whom a maintenance calculation was made was not, at the time the calculation was made, a parent of a child to whom the calculation relates.”; 
\end{quotation}
\end{enumerate}

($b$) for paragraph (3) there shall be substituted—
\begin{quotation}
“(3) In paragraphs (1), (2) and (5A) and in regulation 4(3) “decision” means a decision of the Secretary of State under section 11, 12 or 46 of the Child Support Act, or a determination of an appeal tribunal on a referral under section 28D(1)($b$)  of that Act, or any supersession of a decision under section 17 of that Act, whether as originally made or as revised under section 16 of that Act.”; and
\end{quotation}

($c$) after paragraph (5) there shall be inserted—
\begin{quotation}
“(5A) Where—
\begin{enumerate}\item[]
($a$) the Secretary of State makes a decision (“decision A”) and there is an appeal;

($b$) there is a further decision in relation to the appellant (“decision B”) after the appeal but before the appeal results in a decision by an appeal tribunal (“decision C”); and

($c$) the Secretary of State would have made decision B differently if he had been aware of decision C at the time he made decision B,
\end{enumerate}
decision B may be revised at any time.”.
\end{quotation}
\end{enumerate}

(3) In regulation 6B(4)($e$)  (circumstances in which a child support decision may not be superseded) the reference to “, (19)” shall be omitted.

(4) In regulation 7B (date from which a decision superseded under section 17 of the Child Support Act takes effect)\footnote{Regulation 7B was inserted into S.I.\ 1999/991 by regulation 9 of S.I.\ 2000/3185.}—
\begin{enumerate}\item[]
($a$) after paragraph (1) there shall be inserted—
\begin{quotation}
“(1A) Where a decision is superseded by a decision made by the Secretary of State in a case to which regulation 6A(2)($a$)  or (3) applies and the relevant circumstance is that—
\begin{enumerate}\item[]
($a$) paragraph 4(2) of Schedule 1 to the Child Support Act applies, the decision shall take effect from the first day of the maintenance period on or after—
\begin{enumerate}\item[]
(i) the date on which the non-resident parent becomes the partner of a non-resident parent; or

(ii) where a maintenance calculation is first made in respect of the non-resident parent’s partner, the date on which that calculation takes effect for the purposes of the Child Support Act; or
\end{enumerate}

($b$) paragraph 4(2) of Schedule 1 to the Child Support Act ceases to apply, the decision shall take effect from the first day of the maintenance period on or after the date on which—
\begin{enumerate}\item[]
(i) the non-resident parent or his partner ceases to be a non-resident parent; or

(ii) the non-resident parent ceases to be the partner of a non-resident parent.”; and
\end{enumerate}
\end{enumerate}
\end{quotation}

($b$) paragraph (19) shall be omitted.
\end{enumerate}

(5) In regulation 31(2)\footnote{Regulation 31(2) was amended by regulation 22 of S.I.\ 1999/2570.} (time within which an appeal is to be brought) after “regulation 3(1) or (3)” in both places where it occurs, there shall be inserted “or 3A(1)”.

\subsection[3. Amendment of the Departure Regulations]{Amendment of the Departure Regulations}

3.  After sub-paragraph ($b$)  of paragraph (2) of regulation 23 of the Departure Regulations (assets capable of producing income or higher income)\footnote{Regulation 23 was amended by regulation 16 of S.I.\ 1998/58 and is revoked, with savings, by regulation 33 of S.I.\ 2001/156.}, there shall be added—
\begin{quotation}
    “; or

    ($c$) 
    if the non-applicant were a claimant, paragraph 64 of Schedule 10 to the Income Support (General) Regulations 1987\footnote{S.I.\ 1987/1967. Paragraph 64 was added by regulation 2 of S.I.\ 2001/1118.} (treatment of relevant trust payments) would apply to the asset referred to in that paragraph.”. 
\end{quotation}

\subsection[4. Amendment of the Information Regulations]{Amendment of the Information Regulations}

4.  In regulation 2(2) of the Information Regulations (persons under a duty to furnish information)\footnote{Regulation 2 was amended by regulation 2 of S.I.\ 1995/123, regulation 22 of S.I.\ 1995/1045, regulation 7 of S.I.\ 1995/3261, regulation 7 of S.I.\ 1996/1945, article 6 of S.I.\ 1999/1510 (C.\ 43) and regulation 5 of S.I.\ 2001/161.}—
\begin{enumerate}\item[]
($a$) after sub-paragraph ($a$), there shall be inserted—
\begin{quotation}
“($aa$) where regulation 8(1) of the Maintenance Calculations and Special Cases Regulations applies (persons treated as non-resident parents), a parent of or a person who provides day to day care for the child in respect of whom a maintenance calculation has been applied for or has been treated as applied for or is or has been in force, with respect to the matter listed in sub-paragraph ($l$) of regulation 3(1);”; and
\end{quotation}

($b$) in sub-paragraph ($h$)—
\begin{enumerate}\item[]
(i) in head (i), after “under” there shall be inserted “the Road Traffic (Northern Ireland) Order 1981\footnote{S.I.\ 1981/154 (N.I.\ 1).},”; and

(ii) in head (ii), after “1952” there shall be inserted “, the Prison Act (Northern Ireland) 1953\footnote{1953 c.\ 18 (N.I.).} or the Prisons (Scotland) Act 1989\footnote{1989 c.\ 45.}”.
\end{enumerate}
\end{enumerate}

\subsection[5. Amendment of the Maintenance Assessments and Special Cases Regulations]{\sloppy Amendment of the Maintenance Assessments and Special Cases Regulations}

5.  Paragraph (1) of regulation 9 of the Maintenance Assessments and Special Cases Regulations (exempt income: calculation or estimation of E)\footnote{Regulation 9 was amended by regulation 20 of S.I.\ 1993/913, regulation 44 of S.I.\ 1995/1045, regulation 42 of S.I.\ 1995/3261, regulation 19 of S.I.\ 1996/1945, regulation 68 of S.I.\ 1996/2907, regulation 11 of S.I.\ 1996/1803 and regulation 47 of S.I.\ 1998/58 and is revoked, with savings, by regulation 15 of S.I.\ 2001/155.}, shall be amended as follows—
\begin{enumerate}\item[]
($a$) in sub-paragraph ($e$), after head (ii), there shall be added—
\begin{quotation}
“(iii) if the parent were a claimant, the conditions in paragraph 13A of the relevant Schedule (income support enhanced disability premium) would be satisfied in respect of him, an amount equal to the amount specified in paragraph 15(8)($b$)  of that Schedule;”; and
\end{quotation}

($b$) in sub-paragraph ($g$), after head (ii), there shall be added—
\begin{quotation}
“(iii) if the conditions set out in paragraph 13A of the relevant Schedule (income support enhanced disability premium) are satisfied in respect of that child, an amount equal to the amount specified in paragraph 15(8)($a$)  of that Schedule or, where paragraph (2) applies, half that amount;”.
\end{quotation}
\end{enumerate}

\subsection[6. Amendment of the Maintenance Calculation Procedure Regulations]{Amendment of the Maintenance Calculation Procedure Regulations}

6.---(1)  The Maintenance Calculation Procedure Regulations shall be amended in accordance with the following paragraphs.

(2) In regulation 1(5) (commencement), for “, 54 and 55” there shall be substituted “and 54”.

(3) After regulation 9 (period within which reasons are to be given), there shall be inserted—
\begin{quotation}
\subsection*{“Period for parent to state if request still stands}

9A.  The period to be specified for the purposes of section 46(6) of the Act (period for the parent to state if her request still stands) is 4 weeks from the date on which the Secretary of State serves notice under that subsection.”.
\end{quotation}

(4) In regulation 26(1) (effective dates of maintenance calculations—\hspace{0pt}maintenance order and application under section 4 or 7), for sub-paragraph ($c$)  there shall be substituted—
\begin{quotation}
“($c$) there is a maintenance order which—
\begin{enumerate}\item[]
(i) is in force and was made on or after the date prescribed for the purposes of section 4(10)($a$)  of the Act;

(ii) relates to the person with care, the non-resident parent and all the children to whom the application referred to in sub-paragraph ($b$)  relates; and

(iii) has been in force for at least one year prior to the date of the application referred to in sub-paragraph ($b$).”.
\end{enumerate}
\end{quotation}

(5) In regulation 27 (effective dates of maintenance calculations—\hspace{0pt}maintenance order and application under section 6), in paragraph (1)($c$), for “and the non-resident parent”, there shall be substituted “, the non-resident parent and all the children to whom the application referred to in sub-paragraph ($b$)  relates”.

(6) In regulation 28($b$)  (effective dates of maintenance calculations—\hspace{0pt}maintenance order ceases), for “25 or 26” there shall be substituted “26 or 27”.

(7) In regulation 29 (effective dates of maintenance calculations in specified cases), after paragraph ($b$)  there shall be added—
\begin{quotation}
“($c$) except where the parent with care has made a request under section 6(5) of the Act, where—
\begin{enumerate}\item[]
(i) in the period of 8 weeks immediately preceding the date the application is made, or treated as made under regulation 3, a maintenance calculation (“the previous maintenance calculation”) has been in force and has ceased to have effect;

(ii) the parent with care in respect of the previous maintenance calculation is the non-resident parent in respect of the application;

(iii) the non-resident parent in respect of the previous maintenance calculation is the parent with care in respect of the application; and

(iv) the application relates to the same qualifying child, or all of the same qualifying children, and no others, as the previous maintenance calculation,
\end{enumerate}
the effective date of the maintenance calculation to which the application relates shall be the date on which the previous maintenance calculation ceased to have effect.”.
\end{quotation}

(8) In regulation 31(3) (transitional provision—effective dates and reduced benefit decisions), for “on or before” there shall be substituted “immediately before”.

\subsection[7. Amendment of the Maintenance Calculations and Special Cases Regulations]{\sloppy Amendment of the Maintenance Calculations and Special Cases Regulations}

7.---(1)  The Schedule to the Maintenance Calculations and Special Cases Regulations (net weekly income) shall be amended in accordance with the following paragraphs.

(2) In paragraph 8(1), for “of the employment” there shall be substituted “in respect of employment which are of a type which would be taken into account under paragraph 7(1)”.

(3) For paragraph 13, there shall be substituted—
\begin{quotation}
“13.---(1)  Subject to sub-paragraphs (2) and (3), payments made by way of disabled person’s tax credit under section 129 of the Contributions and Benefits Act\footnote{See section 1 of, and paragraphs 1 and 2($h$) of Schedule 1 to, the Tax Credits Act 1999 (c.\ 10).} to a non-resident parent shall be treated as the income of the non-resident parent, at the rate payable at the effective date.

(2) Where disabled person’s tax credit is payable where a non-resident parent and another person both meet the entitlement criteria for the payment and the amount which is payable has been calculated by reference to the weekly earnings of the non-resident parent and the other person—
\begin{enumerate}\item[]
($a$) where during the period which is used by the Inland Revenue to calculate the non-resident parent’s income the normal weekly earnings (as determined in accordance with Chapter II of Part V of the Disability Working Allowance (General) Regulations 1991\footnote{S.I.\ 1991/2887. Chapter II was amended by regulation 17 of S.I.\ 1993/315, regulation 39 of S.I.\ 1993/2119, regulation 3 of S.I.\ 1994/1924, regulation 3 of S.I.\ 1994/2139, regulation 3 of S.I.\ 1996/1994, regulation 2 of S.I.\ 1996/3137 and regulations 16, 17 and 26 of, and Schedule 2 to, S.I.\ 1999/2487.}) of that parent exceed those of the other person, the amount payable by way of disabled person’s tax credit shall be treated as the income of that parent;

($b$) where during that period the normal weekly earnings of that parent equal those of the other person, half of the amount payable by way of disabled person’s tax credit shall be treated as the income of that parent; and

($c$) where during that period the normal weekly earnings of that parent are less than those of that other person, the amount payable by way of disabled person’s tax credit shall not be treated as the income of that parent.
\end{enumerate}

(3) Where—
\begin{enumerate}\item[]
($a$) disabled person’s tax credit is in payment; and

($b$) not later than the effective date the person, or, if more than one, each of the persons by reference to whose entitlement that payment has been calculated is no longer the partner of the person to whom that payment is made,
\end{enumerate}
the payment shall only be treated as the income of the non-resident parent in question where he is in receipt of it.”.
\end{quotation}

\subsection[8. Amendment of the Transitional Regulations]{Amendment of the Transitional Regulations}

8.---(1)  The Transitional Regulations shall be amended in accordance with the following paragraphs.

(2) In regulation 4 (revision, supersession and appeal of conversion decisions), for paragraph (4) there shall be substituted—
\begin{quotation}
“(4) In their application to a decision referred to in these Regulations, the Decisions and Appeals Regulations shall be modified so as to provide—
\begin{enumerate}\item[]
($a$) on any revision or supersession of a conversion decision under section 16 or 17 respectively of the Act, that—
\begin{enumerate}\item[]
(i) the conversion decision may include a relevant departure direction or relevant property transfer; and

(ii) the effective date of the revision or supersession shall be as determined under the Decisions and Appeals Regulations or the case conversion date, whichever is the later;
\end{enumerate}

($b$) on any appeal in respect of a conversion decision under section 16 or 17 respectively of the Act, that the time within which the appeal must be brought shall be—
\begin{enumerate}\item[]
(i) within the time from the date of notification of the conversion decision against which the appeal is brought, to one month after the case conversion date of that decision; or

(ii) as determined under the Decisions and Appeals Regulations,
\end{enumerate}
whichever is the later.”.
\end{enumerate}
\end{quotation}

(3) Regulation 9 (amount of child support maintenance payable) shall be amended as follows—
\begin{enumerate}\item[]
($a$) in paragraph (1), for the words after “the new amount,”, there shall be substituted—
\begin{quotation}
    “unless—
\begin{enumerate}\item[]
    ($a$) 
    regulation 10 applies, in which case it shall be a transitional amount as provided for in regulations 11 and 17 to 28; or

    ($b$) 
    regulation 12 or 13 applies, in which case it shall be a transitional amount as provided for in those regulations.”; and 
\end{enumerate}
\end{quotation}

($b$) in paragraph (2), for “regulations 10 to 28” there shall be substituted “regulations 10 to 14 and 16 to 28”.
\end{enumerate}

(4) In regulation 10 (circumstances in which a transitional amount is payable), after “reduced rate”, there shall be inserted “, an amount calculated under regulation 22”.

(5) Regulation 12 (transitional amount in flat rate cases) shall be amended as follows—
\begin{enumerate}\item[]
($a$) in paragraphs (1) and (2), “, nil” shall be omitted at the end of the paragraph;

($b$) in paragraph (3), for the words after “apportioned” there shall be substituted “among the persons with care, other than any in respect of whom paragraph 8 of Part I of Schedule 1 to the Act applies, in accordance with paragraph 6(2) of that Schedule, unless paragraph (4) or (5) applies.”;

($c$) in paragraphs (4) and (5), after “paragraph 4(1)($b$)” there shall be inserted “or ($c$)”;

($d$) for paragraph (6) there shall be substituted—
\begin{quotation}
“(6) Where paragraph (4) or (5) applies, the transitional amount shall be apportioned among the persons with care, other than any in respect of whom the former assessment amount is nil and paragraph 8 of Part I of Schedule 1 to the Act applies, in accordance with paragraph 6(2) of that Schedule.”; and
\end{quotation}

($e$) in paragraph (7), “in paragraph (5)” shall be omitted.
\end{enumerate}

(6) Regulation 13 (transitional amount–certain flat rate cases) shall be renumbered as paragraph (1) of that regulation and at the end there shall be added—
\begin{quotation}
“(2) Where paragraph 4(1)($b$)  or ($c$)  of Part I of Schedule 1 to the Act applies and the former assessment amount is nil, the amount of child support maintenance payable for the year beginning on the case conversion date shall be a transitional amount equivalent to half the first prescribed amount and thereafter shall not be a transitional amount but shall be the new amount.”.
\end{quotation}

(7) In regulation 14($c$)  (certain cases where the new amount is payable), “or” shall be omitted.

(8) In regulation 15(4) (case conversion date)—
\begin{enumerate}\item[]
($a$) for “paragraph (3)” there shall be substituted “this regulation”; and

($b$) before the definition of “relevant person”, there shall be inserted—
\begin{quotation}
    ““maintenance assessment” has the meaning given in section 54 of the former Act;”. 
\end{quotation}
\end{enumerate}

(9) In regulation 17 (relevant departure decision and relevant property transfer)—
\begin{enumerate}\item[]
($a$) in the heading, for “decision” there shall be substituted “direction”; and

($b$) for paragraph (6), there shall be substituted—
\begin{quotation}
“(6) Where, but for the application of a relevant departure direction referred to in paragraph (5), the new amount would be—
\begin{enumerate}\item[]
($a$) the first prescribed amount owing to the application of paragraph 4(1)($b$)  of Part I of Schedule 1 to the Act;

($b$) the amount referred to in sub-paragraph ($a$), but is less than that amount or is nil, owing to the application of paragraph 8 of that Part; or

($c$) the nil rate under paragraph 5($a$)  of that Part,
\end{enumerate}
paragraph (5) applies where the amount of the additional income exceeds £100.”.
\end{quotation}
\end{enumerate}

(10) In regulation 21(1) (effect on conversion calculation—relevant property transfer), for “regulation 23” there shall be substituted “regulations 23 and 23A”.

(11) In regulation 22(1) (effect on conversion calculation—maximum amount payable where relevant departure direction is on additional cases ground) for “the amount of child support maintenance which the non-resident parent shall be liable to pay” there shall be substituted “the new amount”.

(12) In paragraph (2) of regulation 23 (effect of a relevant departure direction on conversion calculation—general), the words from “, other” to “costs),” shall be omitted.

(13) After regulation 23, there shall be inserted—
\begin{quotation}
\subsection*{“Effect of a relevant property transfer and a relevant departure direction—general}

23A.  Where—
\begin{enumerate}\item[]
($a$) more than one relevant property transfer applies; or

($b$) one or more relevant property transfers and one or more relevant departure directions apply,
\end{enumerate}
regulation 23 shall apply as if references to a relevant departure direction were to a relevant property transfer or to the relevant property transfers and relevant departure directions, as the case may be.”.
\end{quotation}

(14) In regulation 24 (phasing amount)—
\begin{enumerate}\item[]
($a$) in paragraph (3), for “For”, there shall be substituted “Subject to paragraph (4), for”; and

($b$) after paragraph (3), there shall be added—
\begin{quotation}
“(4) Where the new amount is calculated under regulation 22(1), “relevant income” for the purposes of paragraph (2) is the aggregate of the income calculated under regulation 22(1)($b$).”.
\end{quotation}
\end{enumerate}

(15) Regulation 27 (subsequent decision with effect in transitional period—amount payable) shall be amended as follows—
\begin{enumerate}\item[]
($a$) in paragraph (3)($b$), after “new amount,” there shall be added “and greater than the previous transitional amount,”;

($b$) in paragraph (5)($b$), after “new amount,” there shall be added “and less than the previous transitional amount,”;

($c$) in paragraph (6) for the words after “prescribed amount” there shall be substituted—
\begin{quotation}
    “, would be the first or the second prescribed amount but is less than that amount, or is nil, owing to the application of paragraph 8 of Part I of Schedule 1 to the Act, or is the nil rate.”. 
\end{quotation}
\end{enumerate}

(16) Regulation 28 (linking provisions) shall be amended as follows—
\begin{enumerate}\item[]
($a$) in paragraphs (1) and (2), for “Where” there shall be substituted “Subject to paragraph (2A), where”;

($b$) after paragraph (2), there shall be inserted—
\begin{quotation}
“(2A) Paragraph (1) or (2) shall not apply where, before any application for a maintenance calculation of a type referred to in paragraph (1) or (2) is made or treated as made, an application for a maintenance calculation is made or treated as made in relation to either the person with care or the non-resident parent (but not both of them) to whom the maintenance assessment referred to in paragraph (1) or (2) related.”;
\end{quotation}

($c$) in paragraph (4)($a$), for the words from “at the first” to the end of the sub-paragraph there shall be substituted—
\begin{quotation}
    “at—
\begin{enumerate}\item[]
    (i) 
    the first or second prescribed amount;

    (ii) 
    what would be an amount referred to in head (i)  but is less than that amount, or is nil, owing to the application of paragraph 8 of Part I of Schedule 1 to the Act; or

    (iii) 
    the nil rate; and”; 
\end{enumerate}
\end{quotation}

($d$) in paragraph (4)($b$), for the words after “other than” there shall be substituted “a rate referred to in sub-paragraph ($a$)”;

($e$) in paragraph (5), for “Where” there shall be substituted “Subject to paragraph (5A), where”;

($f$) after paragraph (5), there shall be inserted—
\begin{quotation}
“(5A) Paragraph (5) shall not apply where, before any second subsequent decision is made, an application for a maintenance calculation is made or treated as made in relation to either the person with care or the non-resident parent (but not both of them) to whom the first subsequent decision referred to in paragraph (4) related.”;
\end{quotation}

($g$) in paragraph (7)—
\begin{enumerate}\item[]
(i) for “Where” there shall be substituted “Subject to paragraph (7A), where”; and

(ii) for “an application for child support maintenance” there shall be substituted “an application for a maintenance calculation”;
\end{enumerate}

($h$) after paragraph (7), there shall be inserted—
\begin{quotation}
“(7A) Paragraph (7) shall not apply where, before an application for a maintenance calculation of a type referred to in that paragraph is made or treated as made, an application for a maintenance calculation is made or treated as made in relation to either the person with care or the non-resident parent (but not both of them) to whom the conversion calculation referred to in that paragraph related.”;
\end{quotation}

($i$) paragraph (8) shall be amended as follows—
\begin{enumerate}\item[]
(i) for “Where” there shall be substituted “Subject to paragraph (9), where”; and

(ii) for sub-paragraph ($a$), there shall be substituted—
\begin{quotation}
“($a$) a conversion calculation is in force, or pursuant to regulation 16(3) a maintenance calculation is in force, (“the calculation”) and the new amount—
\begin{enumerate}\item[]
(i) is the first or second prescribed amount;

(ii) would be an amount referred to in head (i), but is less than that amount, or is nil, owing to the application of paragraph 8 of Part I of Schedule 1 to the Act; or

(iii) is the nil rate;”; and
\end{enumerate}
\end{quotation}
\end{enumerate}

($j$) after paragraph (8), there shall be added—
\begin{quotation}
“(9) Paragraph (8) shall not apply where, before a subsequent decision of a type referred to in paragraph (8)($b$)  is made, an application for a maintenance calculation is made or treated as made in relation to the person with care or the non-resident parent (but not both of them) to whom the calculation relates.”.
\end{quotation}
\end{enumerate}

\subsection[9. Amendment of the Variations Regulations]{Amendment of the Variations Regulations}

9.---(1)  The Variations Regulations shall be amended in accordance with the following paragraphs.

(2) In regulation 7 (prescribed circumstances)—
\begin{enumerate}\item[]
($a$) in paragraph (1), for “section 28G” there shall be substituted “section 28A or 28G”; and

($b$) in paragraph (7)—
\begin{enumerate}\item[]
(i) after “agreed” there shall be inserted “and the application had been made under section 28G of the Act”; and

(ii) after “effect” there shall be inserted “and if the variation were agreed, and the application had been made under section 28A of the Act, the decision under section 11 of the Act would take effect”.
\end{enumerate}
\end{enumerate}

(3) In regulation 9(6) (procedure in relation to the determination of an application), for “4($a$)” there shall be substituted “(4)($a$)”.

(4) In regulation 16(4) (prescription of terms), for “£5000”, there shall be substituted “£4999$.$99”.

(5) In regulation 18 (assets)—
\begin{enumerate}\item[]
($a$) in sub-paragraph ($a$)  of paragraph (1), for “has the beneficial interest”, there shall be substituted “has a beneficial interest”; and

($b$) paragraph (3) shall be amended as follows—
\begin{enumerate}\item[]
(i) in sub-paragraph ($d$), before “to any” there shall be inserted “except where the asset is of a type specified in paragraph (2)($b$)  and produces income which does not form part of the net weekly income of the non-resident parent as calculated or estimated under Part III of the Schedule to the Maintenance Calculations and Special Cases Regulations,”; and

(ii) after sub-paragraph ($e$), there shall be added—
\begin{quotation}
    “; or

    ($f$) 
    where, were the non-resident parent a claimant, paragraph 22 (treatment of payments from certain trusts) or 64 (treatment of relevant trust payments) of Schedule 10 to the Income Support (General) Regulations 1987\footnote{Paragraph 22 was substituted by regulation 5(8) of S.I.\ 1991/1175 and amended by regulation 6(8) of S.I.\ 1992/1101, regulation 2(3) of S.I.\ 1993/963 and regulation 4(5) of S.I.\ 1993/1249.} would apply to the asset referred to in that paragraph.”. 
\end{quotation}
\end{enumerate}
\end{enumerate}

(6) In regulations 18(5), 19(5)($a$)  and 20(5), after “allowance” there shall be inserted “prescribed for the purposes of paragraph 4(1)($b$)  of Schedule 1 to the Act”.

\subsection[10. Amendment of the Variations Modification Regulations]{Amendment of the Variations Modification Regulations}

10.  In regulation 8(1) of the Variations Modification Regulations (modification of Schedule 4A), for ““the application for a””, there shall be substituted ““application for a””. 

\bigskip

Signed 
by authority of the Secretary of State for Work and Pensions.

{\raggedleft
\emph{P.~Hollis}\\*Parliamentary Under-Secretary of State,\\*Department of Work and Pensions

}

%Dated
21st March 2002

\small

\part{Explanatory Note}

\renewcommand\parthead{— Explanatory Note}

\subsection*{(This note is not part of the Regulations)}

These Regulations provide for the amendment of regulations relating to child support. Regulation 1 makes provision for citation, commencement and interpretation.

The powers exercised to make these Regulations are those in the Child Support Act 1991 (“the 1991 Act”) and the Child Support, Pensions and Social Security Act 2000 (“the 2000 Act”). Of those in the 1991 Act, some of the powers are those prior to the amendments made to that Act by the 2000 Act, in so far as those amendments are not yet fully in force, and others are those following amendments made to that Act by the 2000 Act.

Regulations 6, 7, 8 and 9 amend the following Regulations and their provisions will take effect when those Regulations come into force, which is at different times for different cases as determined by commencement order made under section 86(2) of the 2000 Act:
\begin{itemize}\item
    the Child Support (Maintenance Calculation Procedure) Regulations 2000 (“the Maintenance Calculation Procedure Regulations”);
\item
    the Child Support (Maintenance Calculations and Special Cases) Regulations 2000 (“the Maintenance Calculations and Special Cases Regulations”);
\item
    the Child Support (Transitional Provisions) Regulations 2000 (“the Transitional Regulations”);
\item
    the Child Support (Variations) Regulations 2000 (“the Variations Regulations”). 
\end{itemize}

Regulations 2 and 4($a$)  amend the Social Security and Child Support (Decisions and Appeals) Regulations 1999 (“the Decisions and Appeals Regulations”) and the Child Support (Information, Evidence and Disclosure) Regulations 1992 (“the Information Regulations”), respectively and will come into force at different times for different cases as determined by commencement order made under section 86(2) of the 2000 Act.

Regulations 3, 4($b$), 5 and 10 amend Regulations which are in force and come into effect the day after these Regulations are made.

Regulation 2 amends the Decisions and Appeals Regulations. Paragraph (2)($a$)(i)  inserts a new ground for revision under regulation 3A of the Decisions and Appeals Regulations. Paragraph (2)($a$)(ii)  provides for revision of certain (maintenance calculation) decisions where a person was not the parent of a relevant child; paragraph (2)($c$)  inserts a new paragraph (5A) into regulation 3A of the Decisions and Appeals Regulations to provide for certain decisions to be revised at any time. Paragraph (4) provides dates when a supersession takes effect in a case where a flat rate liability is being paid, will become payable or will cease to be payable, at a different rate in accordance with paragraph 4(2) of Schedule 1 to the 1991 Act, when non-resident parents become or cease to become partners and certain other consequential and incidental provisions as to supersession and time limits for appeals.

Regulation 3 amends the Child Support Departure Direction and Consequential Amendments Regulations 1996 to provide that certain payments in respect of variant Creutzfeldt-Jakob disease may not be taken into account for the purposes of a departure direction.

Regulation 4 amends the Information Regulations to add to the categories of persons under a duty to furnish information and regulation 5 amends the Child Support (Maintenance Assessments and Special Cases) Regulations 1992 so that an Income Support enhanced disability premium can be included in the calculation of exempt income.

Regulation 6 makes minor amendments to the Maintenance Calculation Procedure Regulations. It also specifies the period of notice for the purposes of section 46(6) of the 1991 Act, makes provisions for the effective date of maintenance calculations in specific cases and clarifies that the transitional provisions in regulation 31(4)–(7) are to apply where section 6 of the 1991 Act prior to its amendment by the 2000 Act applied immediately before the commencement date.

Regulation 7 amends the Maintenance Calculations and Special Cases Regulations as to how disabled person’s tax credit is to be taken into account in calculating the income of a non-resident parent and clarifies the type of income from self-employment which will be relevant for a maintenance calculation.

Regulation 8 amends the Transitional Regulations to provide for the time within which an appeal must be brought against a conversion decision to be either within the time from the date of notification of the conversion decision to one month after the date on which that decision takes effect, or as determined under the Decisions and Appeals Regulations, whichever is the later. Regulation 8 also amends the Transitional Regulations to clarify which cases are within the conversion provisions in regulation 15(2) of those Regulations, to provide for the transitional amount to be payable in certain flat rate cases and its apportionment between persons with care, to make provision for the effect on a conversion calculation where there is more than one relevant property transfer, or a combination of relevant property transfers and relevant departure directions, to provide for case conversion date provisions to apply to all maintenance assessments, to provide additional cases where the subsequent decision amount is payable and to provide that the linking provisions in Regulation 28 do not apply in certain circumstances.

Regulation 9 makes a similar amendment to the Variations Regulations to that made by regulation 3 and makes other minor changes to those Regulations. It also amends regulation 7 of those Regulations so the preliminary consideration provisions apply to an application made under section 28A of the 1991 Act and amends regulation 18 of those Regulations to provide that in certain circumstances land or property held as a business or trade asset is not excluded from the definition of “asset”.

Regulation 10 makes a minor amendment to the Child Support (Variations) (Modification of Statutory Provisions) Regulations 2000.

These Regulations do not impose costs on business. 

\end{document}