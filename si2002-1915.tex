\documentclass[12pt,a4paper]{article}

\newcommand\regstitle{The Child Support Appeals (Jurisdiction of Courts) Order 2002}

\newcommand\regsnumber{2002/1915}

%\opt{newrules}{
\title{\regstitle}
%}

%\opt{2012rules}{
%\title{Child Maintenance and Other Payments Act 2008\\(2012 scheme version)}
%}

\author{S.I.\ 2002 No.\ 1915 (L.\ 9)}

\date{Made
20th July 2002\\
%Laid before Parliament
%26th March 2002\\
Coming into force
in accordance with article 1(2)
}

%\opt{oldrules}{\newcommand\versionyear{1993}}
%\opt{newrules}{\newcommand\versionyear{2003}}
%\opt{2012rules}{\newcommand\versionyear{2012}}

\usepackage{csa-regs}

\setlength\headheight{27.57402pt}

\begin{document}

\maketitle

\noindent
Whereas a draft of this Order has been laid before and approved by a resolution of each House of Parliament:

Now, therefore, the Lord Chancellor in exercise of the power conferred upon him by section 45(1) and (7) of the Child Support Act 1991\footnote{1991 c.\ 48. Section 45(1) was amended by the Social Security Act 1998 (c.\ 14) Schedule 7 paragraph 42.} hereby makes the following Order: 

{\sloppy

\tableofcontents

}

\bigskip

\setcounter{secnumdepth}{-2}

\subsection[1. Citation, commencement, interpretation and extent]{Citation, commencement, interpretation and extent}

1.---(1)  This Order may be cited as the Child Support Appeals (Jurisdiction of Courts) Order 2002.

(2) Subject to paragraph (3) this Order shall come into force on the day after the date on which it is made.

(3) This Order shall not have effect in relation to a particular type of case until the day on which section 10 of the Child Support, Pensions and Social Security Act 2000\footnote{2000 c.\ 19.} comes into force for the purposes of that type of case.

(4) In this Order—
\begin{enumerate}\item[]
($a$) “the Act” means the Child Support Act 1991; and

($b$) “the Regulations” means the Social Security and Child Support (Decisions and Appeals) Regulations 1999\footnote{S.I.\ 1999/991, as amended by regulation 22 of S.I.\ 1999/2570.}.
\end{enumerate}

(5) This Order extends to England and Wales only.

\subsection[2. Revocation]{Revocation}

2.  The Child Support Appeals (Jurisdiction of Courts) Order 1993\footnote{S.I.\ 1993/961 (L.\ 12), there are no amendments.}, to the extent to which it applies in England and Wales, is revoked.

\subsection[3--5. Parentage appeals to be made to courts]{Parentage appeals to be made to courts}

3.  An appeal under section 20(5) of the Act shall be made to a court instead of to an appeal tribunal in the circumstances mentioned in article 4.

\medskip

4.  The circumstances are that—
\begin{enumerate}\item[]
($a$) the appeal will be an appeal under section 20(1)($a$)  or ($b$)  of the Act;

($b$) the determination made by the Secretary of State in making the decision to be appealed against included a determination that a particular person (whether the applicant or some other person) either was, or was not, a parent of the qualifying child in question (“a parentage determination”); and

($c$) the ground of the appeal will be that the decision to be appealed against should not have included that parentage determination.
\end{enumerate}

\medskip

5.  Regulations 31 and 32 of the Regulations shall apply to appeals brought under this Order with the following modifications—
\begin{enumerate}\item[]
($a$) for the words “an appeal tribunal” shall be substituted “a court”;

($b$) for the words “legally qualified panel member” and “panel member” shall be substituted “justices' clerk or the court”; and

($c$) in regulation 32(10) for the words “such written form as has been approved by the President” shall be substituted “written form”.
\end{enumerate}

\bigskip

%Signed 
%by authority of the Secretary of State for Work and Pensions.

{\raggedleft
\emph{Irvine of Lairg, C.}%\\*Parliamentary Under-Secretary of State,\\*Department of Work and Pensions

}

%Dated
20th July 2002

\small

\part{Explanatory Note}

\renewcommand\parthead{— Explanatory Note}

\subsection*{(This note is not part of the Order)}

This Order revokes and replaces (as regards England and Wales) the Child Support Appeals (Jurisdiction of Courts) Order 1993 (S.I.\ 1993/961 (L.\ 12)), which provides for child support appeals to be made to a court instead of to an appeal tribunal where the issue in the appeal is parentage of the qualifying child in relation to whom an application for child support maintenance has been made under the Child Support Act 1991.

The amendments are consequential on the replacement, by virtue of section 10 of the Child Support, Pensions and Social Security Act 2000, of section 20 of the Child Support Act 1991, which deals with child support appeals. This Order also makes provision for the application (with modifications) of regulations 31 and 32 of the Social Security and Child Support (Decision and Appeals) Regulations 1999 in relation to the appeals to which this Order applies. 

\end{document}