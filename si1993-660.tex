\documentclass[a4paper]{article}

\usepackage[welsh,english]{babel}

\usepackage[utf8]{inputenc}
\usepackage[T1]{fontenc}

\usepackage{textcomp}

%\usepackage[2012rules]{optional}

\usepackage[osf]{mathpazo}
\usepackage{cfr-lm}

\usepackage{perpage} %the perpage package
\MakePerPage{footnote} %the perpage package command
\renewcommand{\thefootnote}{\fnsymbol{footnote}}

\usepackage[perpage,para,symbol]{footmisc}

%\opt{newrules}{
\title{The Child Support (Amendments to Primary Legislation) (Scotland) Order 1993}
%}

%\opt{2012rules}{
%\title{Child Maintenance and Other Payments Act 2008\\(2012 scheme version)}
%}

\author{S.I. 1993 No. 660 (S.98)}

\date{Made 11th March 1993\\Laid before Parliament 15th March 1993\\Coming into force 5th April 1993}

%\opt{oldrules}{\newcommand\versionyear{1993}}
%\opt{newrules}{\newcommand\versionyear{2003}}
%\opt{2012rules}{\newcommand\versionyear{2012}}

\usepackage{fancyhdr}
\pagestyle{fancy}
\fancyhead[L]{}
\fancyhead[C]{\itshape The Child Support (Amendments to Primary Legislation) (Scotland) Order 1993 (S.I.~1993/660) \parthead%\phantom{...}% (\versionyear{} scheme version)
}
\fancyhead[R]{}
\fancyfoot[C]{\thepage}
\newcommand{\parthead}{}

\usepackage{array}
\usepackage{multirow}
\usepackage[debugshow]{tabulary}
\usepackage{longtable}
\usepackage{multicol}
\usepackage{lettrine}

\usepackage[colorlinks=true]{hyperref}
\usepackage{microtype}

\hyphenation{Aw-dur-dod}
\hyphenation{bank-rupt-cy}
\hyphenation{Ec-cles-ton}
\hyphenation{Eux-ton}
\hyphenation{Hogh-ton}
\hyphenation{Pres-ton}
\hyphenation{Pru-den-tial}
\hyphenation{Riv-ing-ton}

\newcolumntype{x}[1]
	{>{\raggedright}p{#1}}
\newcommand{\tn}{\tabularnewline}
\setlength\tymin{50pt}

\newcommand\amendment[1]{\subsubsection*{Notes}{\itshape\frenchspacing\footnotesize #1 \par}}

\begin{document}

\maketitle

\noindent
The Secretary of State, in exercise of the powers conferred on him by section 58(7) of the Child Support Act 1991\footnote{\frenchspacing 1991 c. 48.} and of all other powers enabling him in that behalf, hereby makes the following Order:

{\sloppy

\tableofcontents

}

\setcounter{secnumdepth}{-2}

\subsection[1. Citation and commencement]{Citation and commencement}

1.  This Order may be cited as the Child Support (Amendments to Primary Legislation) (Scotland) Order 1993 and shall come into force on 5th April 1993.

\subsection[2. Amendment of Family Law (Scotland) Act 1985]{Amendment of Family Law (Scotland) Act 1985}

2.—(1) The Family Law (Scotland) Act 1985\footnote{\frenchspacing 1985 c. 37; section 27 was amended by the Law Reform (Miscellaneous Provisions) (Scotland) Act 1985 (c. 73) Schedule 2, paragraph 31 and by the Law Reform (Parent and Child) (Scotland) Act 1986 (c. 9) Schedule 1, paragraph 21.} shall be amended as follows.

(2) In section 5 (variation or recall of decree of aliment), after subsection (1) there shall be inserted—
\begin{quotation}
“(1A) Without prejudice to the generality of subsection (1) above, the making of a maintenance assessment with respect to a child for whom the decree of aliment was granted is a material change of circumstances for the purposes of that subsection.”.
\end{quotation}

(3) In section 7 (agreements on aliment), after subsection (2) there shall be inserted—
\begin{quotation}
“(2A) Without prejudice to the generality of subsection (2) above, the making of a maintenance assessment with respect to a child to whom or for whose benefit aliment is payable under such an agreement is a material change of circumstances for the purposes of that subsection.”.
\end{quotation}

(4) In section 13 (orders for periodical allowance), after subsection (4) there shall be inserted—
\begin{quotation}
“(4A) Without prejudice to the generality of subsection (4) above, the making of a maintenance assessment with respect to a child who has his home with a person to whom the periodical allowance is made (being a child to whom the person making the allowance has an obligation of aliment) is a material change of circumstances for the purposes of that subsection.”.
\end{quotation}

(5) In section 16(3) (agreements on financial provision)—
\begin{enumerate}\item[]
($a$) the word “or” at the end of paragraph ($b$) shall be deleted; and

($b$) after paragraph ($c$) there shall be inserted—
\begin{quotation}
“; or

($d$) by virtue of the making of a maintenance assessment, child support maintenance has become payable by either party to the agreement with respect to a child to whom or for whose benefit periodical allowance is paid under that agreement,”.
\end{quotation}
\end{enumerate}

(6) In section 27(1) (interpretation)—
\begin{enumerate}\item[]
($a$) after the definition of “child” there shall be inserted—
\begin{quotation}
““child support maintenance” has the meaning assigned to it by section 3(6) of the Child Support Act 1991;”; and
\end{quotation}

($b$) after the definition of “incidental order” there shall be inserted—
\begin{quotation}
““maintenance assessment” has the meaning assigned to it by section 54 of the Child Support Act 1991;”.
\end{quotation}
\end{enumerate}

\bigskip

%Signed by authority of the Secretary of State for Social Security.

{\raggedleft
\emph{Fraser of Carmyllie}\\*Minister of State, Scottish Office

}

\noindent
St.\ Andrews House,\\
Edinburgh\\
11th March 1993

\part{Explanatory Note}

\renewcommand\parthead{--- Explanatory Note}

\subsection*{(This note is not part of the Order)}

Under the Child Support Act 1991 maintenance assessments can be made requiring the payment of maintenance (aliment) for children. This order amends certain provisions of the Family Law (Scotland) Act 1985 to take account of the effect of the making of a maintenance assessment on existing court orders and agreements for aliment and for financial provision on divorce.

  Article 2(2) amends section 5 of the 1985 Act so that the making of a maintenance assessment will be regarded as a change of circumstances enabling a party to apply to the court to vary or set aside an existing order affected by that assessment. A similar amendment is made for section 7 (article 2(3)) and section 13 (article 2(4)).

  Article 2(5) makes it clear that the making of a maintenance assessment will entitle parties to an agreement for financial provision on divorce (for periodical allowance) to go to court to vary or set aside any term relating to periodical allowance.

  All of the amendments relate to the court’s powers to deal with non child maintenance elements of an order or agreement.

  Article 2(6) inserts necessary definitions consequential on the amendments made.

\end{document}