\documentclass[12pt,a4paper]{article}

\newcommand\regstitle{The Child Support (Miscellaneous Amendments) Regulations 1999}

\newcommand\regsnumber{1999/977}

%\opt{newrules}{
\title{\regstitle}
%}

%\opt{2012rules}{
%\title{Child Maintenance and Other Payments Act 2008\\(2012 scheme version)}
%}

\author{S.I. 1999 No. 977}

\date{Made 25th March 1999\\Coming into force\\
Regulations 1--3, 4(1), (3) and (4), 5, 6(1), (2)($b$), (3), (4), (5)($e$), ($f$), ($g$) and ($h$), (6), (7) and 7: 6th April 1999\\
Regulations 4(2) and 6(2)($a$), (5)($a$), ($b$), ($c$) and ($d$): 4th October 1999 }

%\opt{oldrules}{\newcommand\versionyear{1993}}
%\opt{newrules}{\newcommand\versionyear{2003}}
%\opt{2012rules}{\newcommand\versionyear{2012}}

\usepackage{csa-regs}

\setlength\headheight{27.57402pt}

\begin{document}

\maketitle

\noindent
Whereas a draft of this instrument was laid before Parliament in accordance with section 52(2) of the Child Support Act 1991\footnote{\frenchspacing 1991 c. 48. Schedules 28A to 28I of and Schedules 4A and 4B to the Child Support Act 1991 were inserted by sections 1 to 9 of the Child Support Act 1995 c. 34.}, and approved by a resolution of each House of Parliament:

 Now, therefore, the Secretary of State for Social Security, in exercise of the powers conferred by section 14(3), 32(1), (8) and (9), 42(1), 47(1) and (2), 51, 52(4), 54 and 55(1) of, and paragraphs 5(1) and (2) of Schedule 1 to, the Child Support Act 1991\footnote{\frenchspacing Section 54 is cited because of the meaning ascribed to the word “prescribed”.}, and of all other powers enabling him in that behalf, hereby makes the following Regulations:

{\sloppy

\tableofcontents

}

\bigskip

\setcounter{secnumdepth}{-2}

\subsection[1. Citation, commencement and interpretation]{Citation, commencement and interpretation}

1.—(1) These Regulations may be cited as the Child Support (Miscellaneous Amendments) Regulations 1999.

(2) These Regulations shall come into force on 6th April 1999 with the exception of regulations 4(2), 6(2)($a$) and (5)($a$), ($b$), ($c$) and ($d$), which shall come into force on 4th October 1999.

(3) In these Regulations—
\begin{enumerate}\item[]
“the Act” means the Child Support Act 1991;

“the Collection and Enforcement Regulations” means the Child Support (Collection and Enforcement) Regulations 1992\footnote{\frenchspacing S.I. 1992/1989.};

“the Fees Regulations” means the Child Support Fees Regulations 1992\footnote{\frenchspacing S.I. 1992/3094.};

“the first commencement day” means 6th April 1999;

“the Information, Evidence and Disclosure Regulations” means the Child Support (Information, Evidence and Disclosure) Regulations 1992\footnote{\frenchspacing S.I. 1992/1812.};

“the Inland Revenue” means the Board of Commissioners of Inland Revenue;

“the Maintenance Assessment Procedure Regulations” means the Child Support (Maintenance Assessment Procedure) Regulations 1992\footnote{\frenchspacing S.I. 1992/1813.};

“the Maintenance Assessments and Special Cases Regulations” means the Child Support (Maintenance Assessments and Special Cases) Regulations 1992\footnote{\frenchspacing S.I. 1992/1815.}; and

“the second commencement day” means 4th October 1999.
\end{enumerate}

\subsection[2. Amendment of the Collection and Enforcement Regulations]{\sloppy Amendment of the Collection and Enforcement Regulations}

2.—(1) The Collection and Enforcement Regulations shall be amended in accordance with the following provisions of this regulation.

(2) In head ($b$) of sub-paragraph (4) of regulation 8 (interpretation of Part III)\footnote{\frenchspacing Regulation 8 has been amended by S.I. 1993/913, S.I. 1995/1045 and S.I. 1996/1945.}, following the word “forces” there shall be inserted the words “other than pay or allowances payable by his employer to him as a special member of a reserve force (within the meaning of the Reserve Forces Act 1996\footnote{\frenchspacing 1996 c. 14.})”.

(3) Regulation 25 (offences) shall be amended as follows—
\begin{enumerate}\item[]
($a$) after “orders)—” there shall be inserted “($aa$) regulation 14(1);”;

($b$) for “($a$)” there shall be substituted “($ab$)”.
\end{enumerate}

\subsection[3. Amendment of the Fees Regulations]{Amendment of the Fees Regulations}

3.  In paragraph (3A) of regulation 3 of the Fees Regulations (liability to pay fees)\footnote{\frenchspacing Paragraph (3A) was inserted into regulation 3 by S.I. 1995/1045, and amended by regulation 4 of S.I. 1996/3196.}, for “1999” there shall be substituted “2001”.

\subsection[4. Amendment of the Information, Evidence and Disclosure Regulations]{Amendment of the Information, Evidence and Disclosure Regulations}

4.—(1) The Information, Evidence and Disclosure Regulations shall be amended in accordance with the following provisions of this regulation.

(2) In head ($h$) of paragraph (2) of regulation 3 (purposes for which information or evidence may be required)\footnote{\frenchspacing Regulation 3 has been amended by S.I. 1995/1045 and 3261, S.I. 1996/1945, and 1998/58.} after the word “name,” there shall be inserted the words “the total taxable profits derived from his employment as a self-employed earner, as submitted to, or as issued to him by, the Inland Revenue,”.

(3) Regulation 9 (disclosure of information to an appropriate authority for use in the exercise of housing benefit or council tax benefit functions) is hereby revoked.

(4) In paragraph (1) of regulation 9A (disclosure of information to other persons)\footnote{\frenchspacing Regulation 9A was inserted by regulation 24 of S.I. 1995/1045 and has been amended by S.I. 1995/3261, S.I. 1996/2907 and S.I. 1998/58.}, for the words “given to him by” there shall be substituted the words “held by him for the purposes of the Act relating to”.

\subsection[5. Amendment of the Maintenance Assessment Procedure Regulations]{\sloppy Amendment of the Maintenance Assessment Procedure Regulations}

5.  In Schedule 1 to the Maintenance Assessment Procedure Regulations (meaning of “child” for the purposes of the Act)\footnote{\frenchspacing Schedule 1 has been amended by S.I. 1993/913 and S.I. 1996/1945.}, in paragraph 1, sub-paragraph (1)($a$) and (3)($a$), paragraph 4, sub-paragraph (2)($a$), and paragraph 6, for the words “youth training” there shall be substituted the words “work-based training for young people or, in Scotland, Skillseekers training”.

\subsection[6. Amendment of the Maintenance Assessments and Special Cases Regulations]{Amendment of the Maintenance Assessments and Special Cases Regulations}

6.—(1) The Maintenance Assessments and Special Cases Regulations shall be amended in accordance with the following provisions of this regulation.

(2) In paragraph (2) of regulation 1 (citation, commencement and interpretation)\footnote{\frenchspacing Regulation 1 has been amended by S.I. 1993/913, S.I. 1995/1045 and 3261, S.I. 1996/1803, 1345, 1945, 2907, and 3196, and S.I. 1998/58.}—
\begin{enumerate}\item[]
($a$) in the definition of “earnings” for the words “1 or 3” there shall be substituted the words “1, 2A or 3”;

($b$) for the words “youth training” there shall be substituted the words “work-based training for young people or, in Scotland, Skillseekers training”.
\end{enumerate}

(3) In sub-paragraph ($a$) of paragraph (3) of regulation 7 (net income: calculation or estimation of N)\footnote{\frenchspacing Regulation 7 has been amended by S.I. 1996/1345.}, for the words “youth training” there shall be substituted the words “work-based training for young people or, in Scotland, Skillseekers training”.

(4) In paragraph (4) of regulation 19 (both parents are absent)\footnote{\frenchspacing Regulation 19 has been amended by S.I. 1996/1945 and S.I. 1998/58.} for the word “parent.”\ there shall be substituted the words “parent, or, from the effective date as determined by paragraph (2) of regulation 30 of the Maintenance Assessment Procedure Regulations, whichever is the later.”.

(5) Schedule 1 (calculation of N and M)\footnote{\frenchspacing Schedule 1 has been amended by S.I. 1993/913, S.I. 1995/1045, S.I. 1996/1803, 1345, 1945, 2907 and 3196; and S.I. 1998/58.}, shall be amended as follows—
\begin{enumerate}\item[]
($a$) at the beginning of Chapter II (earnings of a self-employed earner) there shall be inserted the following paragraphs—
\begin{quotation}
“2A.—(1) Subject to paragraphs 2B, 2C, 4 and 5A, “earnings” in the case of employment as a self-employed earner shall have the meaning given by the following provisions of this paragraph.

(2) “Earnings” means the total taxable profits from self-employment of that earner as submitted to the Inland Revenue, less the following amounts—
\begin{enumerate}\item[]
($a$) any income tax relating to the taxable profits from the self-employment determined in accordance with sub-paragraph (3);

($b$) any National Insurance Contributions relating to the taxable profits from the self-employment determined in accordance with sub-paragraph (4);

($c$) one half of any premium paid in respect of a retirement annuity contract or a personal pension scheme or, where that scheme is intended partly to provide a capital sum to discharge a mortgage or charge secured upon the self-employed earner’s home, 37.5 per centum of the contributions payable.
\end{enumerate}

(3) For the purposes of sub-paragraph (2)($a$) the income tax to be deducted from the total taxable profits shall be determined in accordance with the following provisions—
\begin{enumerate}\item[]
($a$) subject to head ($d$), an amount of earnings equivalent to any personal allowance applicable to the earner by virtue of the provisions of Chapter I of Part VII of the Income and Corporation Taxes Act 1988\footnote{\frenchspacing 1988 c. 1.} (personal reliefs) shall be disregarded;

($b$) subject to head ($c$), an amount equivalent to income tax shall be calculated in relation to the earnings remaining following the application of head ($a$) (the “remaining earnings”);

($c$) the tax rate applicable at the effective date shall be applied to all the remaining earnings, where necessary increasing or reducing the amount payable to take account of the fact that the earnings relate to a period greater or less than one year;

($d$) the amount to be disregarded by virtue of head ($a$) shall be calculated by reference to the yearly rate applicable at the effective date, that amount being reduced or increased in the same proportion to that which the period represented by the taxable profits bears to the period of one year.
\end{enumerate}

(4) For the purposes of sub-paragraph (2)($b$) above, the amount to be deducted in respect of National Insurance Contributions shall be the total of—
\begin{enumerate}\item[]
($a$) the amount of Class 2 contributions (if any) payable under section 11(1) or, as the case may be, (3), of the Contributions and Benefits Act; and

($b$) the amount of Class 4 contributions (if any) payable under section 15(2) of that Act,
at the rates applicable at the effective date.
\end{enumerate}

\medskip

2B.—(1) Where—
\begin{enumerate}\item[]
($a$) a self-employed earner cannot provide the 
%child support officer 
Secretary of State  % Words substituted (1.6.99) by SI 1999/1510 reg 46
with the total taxable profit figure from self-employment for the period concerned as submitted to the Inland Revenue, but can provide a copy of his tax calculation notice; or

($b$) the 
%child support officer 
Secretary of State  % Words substituted (1.6.99) by SI 1999/1510 reg 46
becomes aware that the total taxable profit figure from the self-employment submitted by the self-employed earner has been revised by the Inland Revenue,
\end{enumerate}
the earnings of that earner shall be calculated by reference to the income from employment as a self-employed earner as set out in the tax calculation notice issued in relation to his case, and if a revision of the figures included in that notice has occurred, by reference to the revised notice.

(2) In this paragraph and elsewhere in this Schedule—
\begin{enumerate}\item[]
“submitted to” means submitted to the Inland Revenue in accordance with their requirements by or on behalf of the self-employed earner; and

a “tax calculation notice” means a document issued by the Inland Revenue containing information as to the income of a self-employed earner;

a “revision of the figures” means the revision of the figures relating to the total taxable profit of a self-employed earner following an enquiry under section 9A of the Taxes Management Act 1970\footnote{\frenchspacing 1970 c. 9. Section 9A was inserted into the Act by sections 180 and 199(1) and (2)(a) of the Finance Act 1994 (c. 9) and amended by section 133 and Schedule 19, paragraph 2 of the Finance Act 1996 (c. 8).} or otherwise by the Inland Revenue.
\end{enumerate}

\medskip

2C.  Where the 
%child support officer 
Secretary of State  % Words substituted (1.6.99) by SI 1999/1510 reg 46
accepts that it is not reasonably practicable for the self-employed earner to provide information relating to his total taxable profits from self-employment in the form submitted to, or (where paragraph 2B applies) as issued or revised by, the Inland Revenue, “earnings” in relation to that earner shall have the meaning given by paragraph 3 of this Schedule.”;
\end{quotation}

($b$) in sub-paragraph (1) of paragraph 3, for the word “Subject” there shall be substituted the words “Where paragraph 2C applies, and subject”;

($c$) in paragraph 5, after sub-paragraph (5) there shall be added the following sub-paragraph—
\begin{quotation}
“(6) This paragraph applies only where the earnings of a self-employed earner have the meaning given by paragraph 3 of this Schedule.”;
\end{quotation}

($d$) after paragraph 5 there shall be inserted the following paragraph—
\begin{quotation}
“5A.—(1) Subject to sub-paragraph (2) of this paragraph, the earnings of a self-employed earner may be determined in accordance with the provisions of paragraph 2A only where the total taxable profits concerned relate to a period of not less than 6, and not more than 15 months, which terminated not more than 24 months prior to the relevant week.

(2) Where there is more than one total taxable profit figure which would satisfy the conditions set out in sub-paragraph (1), the earnings calculation shall be based upon the figure pertaining to the latest such period.

(3) Where, in the opinion of the 
%child support officer, 
Secretary of State,  % Words substituted (1.6.99) by SI 1999/1510 reg 46
information as to the total taxable profits of the self-employed earner which would satisfy the criteria set out in sub-paragraphs (1) and (2) of this paragraph does not accurately reflect the normal weekly earnings of the self-employed earner, the earnings of that earner can be calculated by reference to the provisions of paragraphs 3 and 5 of this Schedule.”;
\end{quotation}

($e$) in sub-paragraph (1A) of paragraph 7—
\begin{enumerate}\item[]
(i) for the words “regulation 9(4)” there shall be substituted the words “regulation 10(4)”; and

(ii) for the word “1995” there shall be substituted the word “1996\footnote{\frenchspacing S.I. 1996/2567.}”;
\end{enumerate}

($f$) after paragraph 9 there shall be inserted the following paragraph—
\begin{quotation}
“9A.—(1) Where a war disablement pension includes an adult or child dependency increase—
\begin{enumerate}\item[]
($a$) if that pension, including the dependency increase, is payable to a parent, the income of that parent shall be calculated or estimated as if it did not include that amount;

\begin{sloppypar}
($b$) if that pension, including the dependency increase, is payable to some other person but includes an amount in respect of the parent, the income of the parent shall be calculated or estimated as if it included that amount.
\end{sloppypar}
\end{enumerate}

(2) For the purposes of this paragraph, a “war disablement pension” includes a war widow’s pension, a payment made to compensate for non-payment of such a pension, and a pension or payment analogous to such a pension or payment paid by the government of a country outside Great Britain.”.
\end{quotation}

($g$) in sub-paragraph (1A) of paragraph 22—
\begin{enumerate}\item[]
(i) for the words “regulation 9(4)” there shall be substituted the words “regulation 10(4)”; and

(ii) for the word “1995” there shall be substituted the word “1996\footnote{\frenchspacing S.I. 1996/2567.}”;
\end{enumerate}

($h$) after paragraph 22(1A) there shall be inserted the following paragraph—

\begin{quotation}
“22(1B).—(1) Where a war disablement pension includes a dependency allowance paid in respect of a relevant child, the relevant income of that child shall be calculated or estimated as if it included that amount.

(2) For the purposes of this paragraph, a “war disablement pension” includes a war widow’s pension, a payment made to compensate for non-payment of such a pension, and a pension or payment analogous to such a pension or payment paid by the government of a country outside Great Britain.”.
\end{quotation}
\end{enumerate}

(6) In paragraph 25 of Schedule 2 (amounts to be disregarded when calculating or estimating N and M)\footnote{\frenchspacing Schedule 2 has been amended by S.I. 1993/913, S.I. 1995/1045 and 3261, S.I. 1996/481, 1345 and 3196, and S.I. 1998/58.}, for the words “section 51” there shall be substituted the words “section 51A\footnote{\frenchspacing Section 51A was inserted into the Adoption (Scotland) Act 1978 (c. 28) by Schedule 2 of the Children (Scotland) Act 1995 (c. 36).}”.

(7) Schedule 3A to the Maintenance Assessments and Special Cases Regulations (amount to be allowed in respect of transfer of property)\footnote{\frenchspacing Schedule 3A was inserted by S.I. 1995/1045 and has been amended by S.I. 1995/3261.}, shall be amended in accordance with the following provisions of this paragraph—
\begin{enumerate}\item[]
($a$) in paragraph 1(1) in the definition of a “qualifying transfer”—
\begin{enumerate}\item[]
(i) in head ($b$), after the word “child” there shall be inserted the words “, or both whether jointly or otherwise including, in Scotland, in common property”;

(ii) for head ($d$) there shall be substituted the following head—
\begin{quotation}
“($d$) the effect of which is that (subject to any mortgage or charge) the parent with care or a relevant child is solely beneficially entitled to the property of which the property transferred forms the whole or part, or the business asset, or the parent with care is beneficially entitled to that property or that asset together with the relevant child or absent parent or both, jointly or otherwise or, in Scotland, in common property, or the relevant child is so entitled together with the absent parent;”;
\end{quotation}

(iii) for head ($e$) there shall be substituted the following head—
\begin{quotation}
“($e$) which was not made for the purpose only of compensating the parent with care either for the loss of a right to apply for, or receive, periodical payments or a capital sum in respect of herself, or for any reduction in the amount of such payments or sum;”;
\end{quotation}
\end{enumerate}

($b$) in paragraph 1(1) in the definition of a “compensating transfer”—
\begin{enumerate}\item[]
(i) for the words “or a” there shall be substituted the word “,~or”;

(ii) after the word “child” there shall be added the words “or both jointly or otherwise, or, in Scotland, in common property”;
\end{enumerate}

($c$) in sub-paragraph (1) of paragraph 4 of the Schedule—
\begin{enumerate}\item[]
(i) for the words “which is the subject of the transfer” there shall be substituted the word “transferred”;

(ii) for the words after “formula” to the end there shall be substituted the following—
\begin{quotation}
\[``QV = \frac{VP-MCP}{2} - (VAP-MCR) - VCR\]
where—
\begin{enumerate}\item[]
QV is the qualifying value,

VP is the value at the relevant date of the business asset or the property of which the estate or interest forms the whole or part,
and
for the purposes of this calculation it is assumed that the estate, interest or asset held on the relevant date by the absent parent or by the absent parent and the parent with care is held by them jointly in equal shares or, in Scotland, in common property;

MCP is the amount of any mortgage or charge outstanding immediately prior to the relevant date on the business asset or on the property of which the estate or interest forms the whole or part;

VAP is the value calculated at the relevant date of the business asset or of the property of which the estate or interest forms the whole or part beneficially owned by the absent parent immediately following the transfer (if any);

MCR is, where immediately after the transfer the absent parent is responsible for discharging a mortgage or charge on the business asset or on the property of which the estate or interest forms the whole or part, the amount calculated at the relevant date which is a proportion of any such mortgage or charge outstanding immediately following the transfer, being the same percentage as VAP bears to that property as a whole; and

VCR is the value of any charge in favour of the absent parent on the business asset or on the property of which the estate or interest forms the whole or part, being the amount specified in the court order or written maintenance agreement in relation to the charge, or the amount of a proportion of the value of the business asset or the property on the relevant date specified in the court order or written maintenance agreement.”;
\end{enumerate}
\end{quotation}
\end{enumerate}

($d$) head ($b$) of paragraph 5 shall be amended by the insertion after the words “on the relevant date” of the words “and for the purposes of this calculation it is assumed that the cash, balance or policy held on the relevant date by the absent parent and the parent with care is held by them jointly in equal shares or, in Scotland, in common property.”;

($e$) paragraph 6 shall be amended in accordance with the following paragraphs—
\begin{enumerate}\item[]
(i) in sub-paragraph ($a$), for the words “for the formula” to “MC” there shall be substituted the words “the qualifying value shall be treated as being twice the qualifying value calculated in accordance with that paragraph”;

(ii) in sub-paragraph ($b$), for the words “the value of the transfer.” there shall be substituted “treated as being twice the qualifying value calculated in accordance with that paragraph.”;
\end{enumerate}

($f$) paragraph 11 shall be amended as follows—
\begin{enumerate}\item[]
(i) in paragraph ($a$) before the words “there were substituted” there shall be inserted the words “and in head ($e$) of paragraph 8A(1), for the words “legal estate in the land””; and

(ii) after sub-paragraph ($b$) there shall be inserted the following—
\begin{quotation}
“($c$) in paragraphs 1, 2, 4 and 8A for the word “mortgage” there were substituted the words “heritable security”.”.
\end{quotation}
\end{enumerate}
\end{enumerate}

\amendment{
Words substituted in reg. 6(5) (1.6.99) by the Social Security Act 1998 (Commencement No. 7 and Consequential and Transitional Provisions) Order 1999 reg. 46.
}

\subsection[7. Transitional provisions]{Transitional provisions}

%7.  A maintenance assessment in force on the first or second commencement day shall not be reviewed solely to give effect to these Regulations but on a revision of that assessment under section 16, or a review under sections 17, 18, or 19 of the Act these Regulations shall apply to any fresh maintenance assessment made following that revision or review from the effective date of that assessment, or from the first day of the first maintenance period which begins on or after the first or second commencement day, as the case may be, whichever is the later.

% Reg 7 substituted (1.6.99) by SI 1999/1510 reg 47
7.—(1) A decision with respect to a maintenance assessment in force on the first or second commencement day shall not be superseded by a decision under section 17 of the Act solely to give effect to these Regulations.

(2) These Regulations shall apply to a fresh maintenance assessment made by virtue of—
\begin{enumerate}\item[]
($a$) a revision under section 16 of the Act of a decision with respect to a maintenance assessment; or

($b$) a decision under section 17 of the Act which supersedes a decision with respect to a maintenance assessment,
\end{enumerate}
as from whichever is the later of—
\begin{enumerate}\item[]
(i) the date as from which that revision or, as the case may be, supersession takes effect; or

(ii) the first day of the first maintenance period which begins on or after the first or second commencement day, as the case may be.
\end{enumerate}

\amendment{
Reg. 7 substituted (1.6.99) by the Social Security Act 1998 (Commencement No. 7 and Consequential and Transitional Provisions) Order 1999 reg. 47.
}

\bigskip

Signed 
by authority of the Secretary of State for Social Security.

{\raggedleft
\emph{Patricia Hollis}\\*Parliamentary Under-Secretary of State,\\*Department of Social Security

}

25th March 1999

\small

\part{Explanatory Note}

\renewcommand\parthead{--- Explanatory Note}

\subsection*{(This note is not part of the Regulations)}

These Regulations amend various regulations under the Child Support Act 1991, as amended by the Child Support Act 1995.

\begin{sloppypar}
  The Child Support (Collection and Enforcement) Regulations 1992 are amended to make it an offence, on the part of an employer, to fail to forward amounts deducted under a deduction from earnings order to the Secretary of State by the date set in regulation 14 of those Regulations. In addition provision is made to include in the definition of “earnings” subject to a deduction from earnings order payment or allowances received by an absent parent from his civilian employer while serving as a special member of a reserve force (within the meaning of the Reserve Forces Act 1996) (regulation 2).
\end{sloppypar}

  The Child Support Fees Regulations 1992 are amended to provide that no assessment fee or collection fee shall be payable where it would otherwise have become payable on or after 18th April 1995 and before 6th April 2001 (regulation 3).

  Regulation 9 of the Child Support (Information, Evidence and Disclosure) Regulations 1992, which makes provision for the disclosure of certain information to local authorities, is repealed, to avoid overlap with the provision made by section 3 of the Social Security Act 1998. In addition the information available to be disclosed between the parties to an assessment is broadened by means of an amendment to regulation 9A (regulation 4).

  The Child Support (Maintenance Assessment Procedure) Regulations 1992 are amended in order to amend certain obsolete references to youth training schemes (regulation 5).

  The Child Support (Maintenance Assessments and Special Cases) Regulations 1992 are amended by regulation 6 in the following respects. Firstly, regulation 19 is amended to remove an anomaly relating to the effective dates of assessments made where both parents are absent (regulation 6(4)). Secondly, provision is made for the earnings of a self-employed earner to be calculated on the basis of taxable profit figures submitted to the Inland Revenue by that earner, rather than by reference to the provisions at present contained in Chapter II of Schedule 1 to the Regulations (regulation 6(5)). Thirdly, in relation to transfers of property made in accordance with the terms of a court order or written maintenance agreement made or executed prior to 5th April 1993, provision is made to amend the definition of a “qualifying transfer” and for the calculation of the value of such transfers for the purposes of the allowance in exempt income (regulation 6(7)). Finally, several minor amendments are made, firstly, to bring references up to date, removing further references to youth training schemes, and to replace provisions in the Jobseeker’s Allowance legislation (regulation 6(5)), and secondly to make provision as to the treatment in income terms of certain child and dependency allowances which may be paid in conjunction with a war pension (regulation 6(5)).

  Regulation 7 makes certain transitional provisions.

  These Regulations impose no costs on business.

\end{document}