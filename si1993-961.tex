\documentclass[a4paper]{article}

\usepackage[welsh,english]{babel}

\usepackage[utf8]{inputenc}
\usepackage[T1]{fontenc}

\usepackage{textcomp}

%\usepackage[2012rules]{optional}

\usepackage[osf]{mathpazo}

%\opt{newrules}{
\title{The Child Support Appeals (Jurisdiction of Courts) Order 1993}
%}

%\opt{2012rules}{
%\title{Child Maintenance and Other Payments Act 2008\\(2012 scheme version)}
%}

\author{S.I. 1993 No. 961 (L. 12)}

\date{Made 31st March 1993\\Coming into force 5th April 1993}

%\opt{oldrules}{\newcommand\versionyear{1993}}
%\opt{newrules}{\newcommand\versionyear{2003}}
%\opt{2012rules}{\newcommand\versionyear{2012}}

\usepackage{fancyhdr}
\pagestyle{fancy}
\fancyhead[L]{}
\fancyhead[C]{\itshape The Child Support Appeals (Jurisdiction of Courts) Order 1993 (S.I.~1993/961) \parthead%\phantom{...}% (\versionyear{} scheme version)
}
\fancyhead[R]{}
\fancyfoot[C]{\thepage}
\newcommand{\parthead}{}

\usepackage{array}
\usepackage{multirow}
\usepackage[debugshow]{tabulary}
\usepackage{longtable}
\usepackage{multicol}
\usepackage{lettrine}

\usepackage[colorlinks=true]{hyperref}
\usepackage{microtype}

\hyphenation{Aw-dur-dod}
\hyphenation{bank-rupt-cy}
\hyphenation{Ec-cles-ton}
\hyphenation{Eux-ton}
\hyphenation{Hogh-ton}
\hyphenation{Pres-ton}
\hyphenation{Pru-den-tial}
\hyphenation{Riv-ing-ton}

\newcolumntype{x}[1]
	{>{\raggedright}p{#1}}
\newcommand{\tn}{\tabularnewline}
\setlength\tymin{50pt}

\newcommand\amendment[1]{\subsubsection*{Notes}{\itshape\frenchspacing\footnotesize #1 \par}}

\usepackage{perpage} %the perpage package
\MakePerPage{footnote} %the perpage package command
\renewcommand{\thefootnote}{\fnsymbol{footnote}}

\usepackage[perpage,para,symbol]{footmisc}

\begin{document}

\maketitle

\noindent
Whereas a draft of the above Order has been laid before and approved by resolution of each House of Parliament:

 Now, therefore, the Lord Chancellor in relation to England and Wales and the Lord Advocate in relation to Scotland, in exercise of the power conferred on them by sections 45(1) and (7) of the Child Support Act 1991\footnote{\frenchspacing 1991 c. 48.} and the Lord Advocate in exercise of the power conferred on him by section 58(7) of that Act, hereby make the following order—

{\sloppy

\tableofcontents

}

\setcounter{secnumdepth}{-2}

\subsection[1, 2. Title, commencement and interpretation]{Title, commencement and interpretation}

1.  This order may be cited as the Child Support Appeals (Jurisdiction of Courts) Order 1993 and shall come into force on 5th April 1993.

\medskip

2.  In this Order, “the Act” means the Child Support Act 1991.

\subsection[3--5. Parentage appeals to be made to courts]{Parentage appeals to be made to courts}

3.  An appeal under section 20 of the Act shall be made to a court instead of to a child support appeal tribunal in the circumstances mentioned in article 4.

\medskip

4.  The circumstances are that—
\begin{enumerate}\item[]
($a$) the decision against which the appeal is brought was made on the basis that a particular person (whether the applicant or some other person) either was, or was not, a parent of a child in question, and

($b$) the ground of the appeal will be that the decision should not have been made on that basis.
\end{enumerate}

\medskip

5.—(1) For the purposes of article 3 above, an appeal may be made to a court in Scotland if—
\begin{enumerate}\item[]
($a$) the child in question was born in Scotland; or

($b$) the child, the absent parent or the person with care of the child is domiciled in Scotland on the date when the appeal is made or is habitually resident in Scotland on that date.
\end{enumerate}

(2) Where an appeal to a court in Scotland is to be made to the sheriff, it shall be to the sheriff of the sheriffdom where—
\begin{enumerate}\item[]
($a$) the child in question was born; or

($b$) the child, the absent parent or the person with care of the child is habitually resident on the date when the appeal is made.
\end{enumerate}

\subsection[6, 7. Modification of section 20(2) to (4) of the Act in relation to appeals to courts]{Modification of section 20(2) to (4) of the Act in relation to appeals to courts}

6.  In relation to an appeal which is to be made to a court in accordance with this Order, the reference to the chairman of a child support appeal tribunal in section 20(2) of the Act shall be construed as a reference to the court. 

\medskip

7.  In relation to an appeal which has been made to a court in accordance with this Order, the references to the tribunal in section 20(3) and (4) of the Act shall be construed as a reference to the court.

\subsection[8. Amendment of the Law Reform (Parent and Child) (Scotland) Act 1986]{Amendment of the Law Reform (Parent and Child) (Scotland) Act 1986}

8.  In section 8 (Interpretation) of the Law Reform (Parent and Child) Scotland) Act 1986\footnote{\frenchspacing 1986 c. 9.} at the end of the definition of “action for declarator” there shall be inserted the words “but does not include an appeal under section 20 (Appeals) of the Child Support Act 1991 made to the court by virtue of an order made under section 45 (jurisdiction of courts in certain proceedings) of that Act;”.

%\bigskip
%
%Signed by authority of the Secretary of State for Social Security.
%
%{\raggedleft
%\emph{Alistair Burt}\\*Parliamentary Under-Secretary of State,\\*Department of Social Security
%
%}
%
%30th March 1993

\bigskip

{\raggedleft
\emph{Mackay of Clashfern, C}

}

Dated 31st March 1993

\bigskip

{\raggedleft
\emph{Rodger of Earlsferry}\\*Lord Advocate

}

Lord Advocate's Chambers

31st March 1993

\part{Explanatory Note}

\renewcommand\parthead{--- Explanatory Note}

\subsection*{(This note is not part of the Order)}

This Order affects appeals against decisions of child support officers on reviews under section 18 of the Child Support Act 1991 and against decisions to refuse applications for such reviews. Section 20 of the Act provides for such appeals to be made to a child support appeal tribunal, but this Order will require such appeals to be made to a court where they raise an issue of disputed parentage. The Order also makes a consequential amendment to the Law Reform (Parent and Child) (Scotland) Act 1986.

\end{document}