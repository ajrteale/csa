\documentclass[12pt,a4paper]{article}

\newcommand\regstitle{The Social Security (Child Maintenance Bonus) Regulations 1996}

\newcommand\regsnumber{1996/3195}

\usepackage[oldrules]{optional}

\opt{oldrules}{
\title{\regstitle\\(1993 scheme version)}
}

\opt{newrules}{
\title{\regstitle\\(2003 and 2012 scheme versions)}
}

%\opt{2012rules}{
%\title{Child Maintenance and Other Payments Act 2008\\(2012 scheme version)}
%}

\author{S.I. 1996 No. 3195}

\date{Made 18th December 1996\\Coming into force 7th April 1997
}

%\opt{oldrules}{\newcommand\versionyear{1993}}
%\opt{newrules}{\newcommand\versionyear{2003}}
%\opt{2012rules}{\newcommand\versionyear{2012}}

\usepackage{csa-regs}

\setlength\headheight{27.61603pt}

\begin{document}

\maketitle

\noindent
Whereas a draft of this instrument was laid before Parliament in accordance with section 26(5) of the Child Support Act 1995 and approved by resolution of each House of Parliament;

 Now, therefore, the Secretary of State for Social Security, in exercise of the powers conferred upon him by sections 10 and 26(1) to (3) of the Child Support Act 1995\footnote{\frenchspacing 1995 c. 34; the meaning assigned to the word “prescribed” is given in section 54 of the Child Support Act 1991 (c. 48) which is applied to section 10 of the Child Support Act 1995 by section 27(2) of that Act.}, sections 5(1)($p$), 6(1)($q$), 71(8), 78(2), 189(1), (3) and (4) and 191 of the Social Security Administration Act 1992\footnote{\frenchspacing 1992 c. 5; section 6(1) was substituted by the Local Government Finance Act 1992 (c. 14), Schedule 9 paragraph 12(1)($a$); section 191 is an interpretation provision and is cited because of the meaning assigned to the word “prescribe”.}, sections 136(5)($b$), 137(1) and 175(1) and (3) of the Social Security Contributions and Benefits Act 1992\footnote{\frenchspacing 1992 c. 4; section 137(1) is an interpretation provision and is cited because of the meaning assigned to the word “prescribed”.} and of all other powers enabling him in that behalf, by this instrument, which contains only regulations made by virtue of, or consequential upon, section 10 of the Child Support Act 1995 and which is made before the end of a period of 6 months beginning with the coming into force of that provision\footnote{\frenchspacing Section 170 of the Social Security Administration Act 1992 (c. 5) which relates to the Social Security Advisory Committee, is applied to Section 10 of the Child Support Act 1995 by virtue of the amendment made for subsection (5) of section 170 by the Child Support Act 1995, section 30(5), Schedule 3, paragraph 20.}, hereby makes the following Regulations:

\enlargethispage{-\baselineskip}


{\sloppy

\tableofcontents

}

\setcounter{secnumdepth}{-2}

\subsection[1. Citation, commencement and interpretation]{Citation, commencement and interpretation}

1.—(1) These Regulations may be cited as the Social Security (Child Maintenance Bonus) Regulations 1996 and shall come into force on 7th April 1997.

(2) In these Regulations—
\begin{enumerate}\item[]
“the Act” means the Child Support Act 1995;

“applicant”, except where regulation 8 (retirement) applies, means the person claiming the bonus;

“appropriate office” means an office of the Department of Social Security or the Department for Education and Employment;

“benefit week”—
\begin{enumerate}\item[]
($a$) where the relevant benefit is income support, has the meaning it has in the Income Support (General) Regulations 1987\footnote{\frenchspacing S.I. 1987/1967.} by virtue of regulation 2(1) of those Regulations; or

($b$) where the relevant benefit is a jobseeker’s allowance, has the meaning it has in the Jobseeker’s Allowance Regulations 1996\footnote{\frenchspacing S.I. 1996/207.} by virtue of regulation 1(3) of those Regulations;
\end{enumerate}

“bonus” means a child maintenance bonus;

“bonus period” comprises the days specified in regulation 4;

%“child maintenance” means any of the following payments made on or after 7th April 1997—
%\begin{enumerate}\item[]
%($a$) child support maintenance;
%
%($b$) maintenance paid by an absent parent 
%%to a parent with care 
%to a person with care  % Words substituted (6.4.97) by SI 1997/454 reg 8(2)(a)
%of a qualifying child under an agreement (whether enforceable or not) between them or by virtue of an order of a court;
%
%($c$) maintenance deducted from any benefit payable to an absent parent who is liable to maintain a qualifying child,
%\end{enumerate}
%but does not include any maintenance paid in respect of a former partner;

% Definition of ``child maintenance'' substituted (1.4.98) by SI 1998/563 reg 2(2)
“child maintenance” means maintenance in any of the following forms—
\begin{enumerate}\item[]
    ($a$) 
    child support maintenance paid or payable;

    ($b$) 
    maintenance paid or payable by an absent parent to a person with care of a qualifying child, under an agreement (whether enforceable or not) between them, or by virtue of an order of a court; or

    ($c$) 
    maintenance deducted from any benefit payable to an absent parent who is liable to maintain a qualifying child,
\end{enumerate}
    which, as the case may be, is paid, payable or deducted on or after 1st April 1998, but does not include any maintenance paid or payable in respect of a former partner;

“couple” means a married or an unmarried couple;

“income-based jobseeker’s allowance” has the same meaning as in the Jobseekers Act by virtue of section 1(4) of that Act;

“the Jobseekers Act” means the Jobseekers Act 1995\footnote{\frenchspacing 1995 c. 18.};

\begin{sloppypar}
“jobseeker’s allowance” means an income-based Jobseeker’s allowance;
\end{sloppypar}

“partner” means where a person, whether an applicant or otherwise,—
\begin{enumerate}\item[]
($a$) is a member of a married or unmarried couple, the other member of that couple;

($b$) is married polygamously to two or more members of his household, any such member; or

($c$) is a member of a marriage to which section 133(1)($b$) of the Social Security Contributions and Benefits Act 1992 (polygamous marriages) refers and the other party to the marriage has one or more additional spouses, the other party;
\end{enumerate}

“work condition” means the condition specified at regulation 3(1)($c$).
\end{enumerate}

(3) Expressions used in these Regulations and in the Child Support Act 1991 have the same meaning in these Regulations as they have in that Act

(4) For the purposes of these Regulations, the qualifying benefits are a jobseeker’s allowance and income support.

(5) In these Regulations, where—
\begin{enumerate}\item[]
($a$) a payment is made in any benefit week by an absent parent to a person with care;

($b$) the absent parent pays both child maintenance and maintenance for the person with care; and

($c$) there is no evidence as to which form of maintenance that payment is intended to represent,
\end{enumerate}
the first £5 of any such payment or, where the amount of payment is less than £5, that amount shall be treated as if it was a payment of child maintenance.

(6) For the purposes of these Regulations, child maintenance is treated as payable where it is paid under an agreement which is not enforceable

(7) Where a person is entitled to a qualifying benefit on any day but no qualifying benefit is payable to her in respect of that day, that person shall be treated for the purposes of these Regulations 
other than regulation 4 (bonus period)  % Words inserted (6.4.97) by SI 1997/454 reg 8(2)(b)
as not entitled to a qualifying benefit for that day.

(8) In these Regulations, unless the context otherwise requires, a reference—
\begin{enumerate}\item[]
($a$) to a numbered section is to the section of the Act bearing that number;

($b$) to a numbered regulation is to the regulation in these Regulations bearing that number;

($c$) in a regulation to a numbered paragraph is to the paragraph in that regulation bearing that number;

($d$) in a paragraph to a lettered or numbered sub-paragraph is to the sub-paragraph in that paragraph bearing that letter or number.
\end{enumerate}

\amendment{
Words inserted in reg. 1(7) and words substituted in sub-para (b) in definition of ``child maintenance'' in reg. 1(2) (6.4.97) by the Social Security (Miscellaneous Amendments) Regulations 1997 reg. 8(2).

Definition of ``child maintenance'' in reg. 1(2) substituted (1.4.98) by the Social Security (Miscellaneous Amendments) Regulations 1998 reg. 2(2).

\medskip

Regs. 2--13 revoked (3.3.03) by the Social Security (Child Maintenance Premium and Miscellaneous Amendments) Regulations 2000 reg. 4(1), subject to transitional provisions in reg. 4(2)--(4).
}

%\subsection[2. Application of the Regulations]{Application of the Regulations}
%
%2.—(1) Subject to paragraph (2), these Regulations apply only in a case where on or after 7th April 1997 an absent parent has paid child maintenance in respect of a qualifying child and that maintenance has been—
%\begin{enumerate}\item[]
%($a$) taken into account in determining the amount of a qualifying benefit payable to the person with care or the partner of that person; or
%
%($b$) retained by the Secretary of State in accordance with section 74A(3) of the Social Security Administration Act 1992 (payment of benefit where maintenance payments are collected by the Secretary of State)\footnote{\frenchspacing 1992 c. 5; section 74A was inserted by the Child Support Act 1995 (c. 34).}.
%\end{enumerate}
%
%(2) Regulation 6 (Secretary of State to issue estimates) applies also where a child maintenance assessment has been made but no maintenance has been paid.
%
%(3) No day falling before 7th April 1997 shall be taken into account in determining whether any condition specified in these Regulations is satisfied or whether any period specified in these Regulations commenced.
%
%\subsection[3. Entitlement to a Bonus]{Entitlement to a Bonus}
%
%3.—(1) An applicant is entitled to a bonus where—
%\begin{enumerate}\item[]
%($a$) she has claimed a bonus in accordance with regulation 10 (claiming a bonus);
%
%($b$) the claim relates to days falling within a bonus period;
%
%($c$) except where paragraph (2) applies, she satisfies the work condition, that is to say, she or her partner takes up or returns to work or increases the number of hours in which in any week she or her partner is engaged in employment or the earnings from an employment in which she or her partner are engaged is increased;
%
%($d$) as a result of satisfying the work condition any entitlement to a qualifying benefit in respect of herself and, where she has a partner, her family ceases;
%
%($e$) in a case where the qualifying benefit which ceased—
%\begin{enumerate}\item[]
%(i) was income support, the person with care has not reached the day before her 60th birthday;
%
%(ii) was a jobseeker’s allowance, the person with care has not reached the day before she attains pensionable age,
%\end{enumerate}
%at the time the work condition is satisfied; and %the work condition is satisfied within a period of 14 days following the last day in respect of which a qualifying benefit is payable.
%
%% Reg 3(1)(f) substituted (1.4.98) by SI 1998/563 reg 2(3)
%($f$) the work condition is satisfied within the period of—
%\begin{enumerate}\item[]
%(i) in a case where an applicant with care cares for one child only and that child dies, 12 months immediately following the date of death;
%
%(ii) in a case where the absent parent has—
%\begin{enumerate}\item[]
%($aa$) died;
%
%($bb$) ceased to be habitually resident in the United Kingdom; or
%
%($cc$) has been found not to be the parent of the qualifying child or children,
%\end{enumerate}
%12 weeks immediately following the first date on which any of those events occurs;
%
%(iii) in any other case, 14 days immediately following the day on which the bonus period applying to the applicant comes to an end.
%\end{enumerate}
%\end{enumerate}
%
%% Reg 3(1A) inserted (6.4.97) by SI 1997/454 reg 8(3)
%(1A) In the case of an applicant who satisfies the requirements of paragraph (1)($f$)  but whose entitlement, or whose partner’s entitlement, to a qualifying benefit ceased otherwise than as a result of satisfying the work condition, for sub-paragraph ($d$)  of paragraph (1) there shall be substituted the following sub-paragraph—
%\begin{quotation}
%“($d$) had the work condition been satisfied on the day she, or her partner, was last entitled to a qualifying benefit, that entitlement would as a consequence have ceased.”
%\end{quotation}
%
%(2) A person who is absent from work because of a trade dispute at her place of work and returns to work with the employer she worked for before the dispute began, does not thereby satisfy the requirements of paragraph (1)($c$).
%
%(3) In paragraph (2), “place of work”, in relation to any person, means the premises at which she was employed.
%
%(4) An applicant is also entitled to a bonus where she satisfies the requirements specified in regulation 8 (retirement).
%
%\amendment{
%Reg. 3(1A) inserted (6.4.97) by the Social Security (Miscellaneous Amendments) Regulations 1997 reg. 8(3).
%
%Reg. 3(1)(f) substituted (1.4.98) by the Social Security (Miscellaneous Amendments) Regulations 1998 reg. 2(3).
%}
%
%\subsection[4. Bonus Period]{Bonus Period}
%
%4.—(1) A bonus period comprises only days falling on or after 7th April 1997% 
%, other than days to which paragraph (9) applies,  % Words inserted (6.4.97) by SI 1997/454 reg 8(4)(a)
%on which—
%\begin{enumerate}\item[]
%($a$) the applicant or, where the applicant has a partner, her partner is entitled to, or is treated as entitled to a qualifying benefit whether it is payable or not;
%
%($b$) the applicant has residing with her a qualifying child; and
%
%($c$) child maintenance is either—
%\begin{enumerate}\item[]
%%(i) taken into account in determining the amount of the applicant’s income for the purposes of the qualifying benefit; or
%
%% Reg 4(1)(c)(i) substituted (6.4.97) by SI 1997/454 reg 8(4)(a)
%%(i) taken into account in determining, for the purposes of the qualifying benefit, the amount of the claimant’s income;
%
%% Reg 4(1)(c)(i) substituted (6.4.97) by SI 1997/454 reg 8(4)(a)
%(i) paid or payable to the applicant; or
%
%(ii) retained by the Secretary of State in accordance with section 74A(3) of the Social Security Administration Act 1992\footnote{\frenchspacing Section 74A was inserted by the Child Support Act 1995 (c. 34) section 25.}.
%\end{enumerate}
%\end{enumerate}
%
%(2) Any two or more bonus periods separated by any one connected period shall be treated as one bonus period.
%
%(3) For the purposes of these Regulations, “a connected period” is—
%\begin{enumerate}\item[]
%($a$) any period of not more than 12 weeks falling between two bonus periods to which paragraph (1) refers;
%
%($b$) any period of not more than 12 weeks throughout which—
%\begin{enumerate}\item[]
%% Reg 4(3)(b)(i) omitted (6.4.97) by SI 1997/454 reg 8(4)(b)
%%(i) no qualifying child resides with the applicant; or
%
%(ii) the applicant ceases to be entitled to a qualifying benefit on becoming one of a couple and the couple fail to satisfy the conditions of entitlement to a qualifying benefit; or
%\end{enumerate}
%
%($c$) any period throughout which maternity allowance is payable to the applicant; or
%
%($d$) any period of not more than 2 years throughout which incapacity benefit, severe disablement allowance or 
%%invalid care allowance 
%carer's allowance  % Words substituted (1.4.03) by SI 2002/2497 Sch 2 para 1, 2
%is payable to the applicant.
%\end{enumerate}
%
%(4) In calculating any period for the purposes of paragraph (3) no regard shall be had to any day which falls before 7th April 1997.
%
%(5) Bonus periods separated by two or more connected periods shall not link to form a single bonus period but shall instead remain separate bonus periods.
%
%(6) Where a qualifying child is temporarily absent for a period not exceeding 12 weeks from the home he shares with the applicant, the applicant shall be treated as satisfying the requirements of paragraph (1)($b$) throughout that absence.
%
%(7) A bonus period which would, but for this paragraph, have continued shall end—
%\begin{enumerate}\item[]
%($a$) where the applicant or, where the applicant has a partner, her partner, satisfies the work condition and claims a bonus, on the last day of entitlement to a qualifying benefit to which any award made on that claim applies; or
%
%%($b$) where the person with the care of the child or her partner dies, on the date of death.
%
%% Reg 4(7)(b) substituted (6.4.97) by SI 1997/454 reg 8(4)(c)
%($b$) on the date of death of a person with care of a qualifying child to whom child maintenance is payable.
%\end{enumerate}
%
%% Reg 4(8), (9) inserted (6.4.97) by SI 1997/454 reg 8(4)(d)
%(8) In paragraphs (1)($c$)(i)  and (9) “claimant”—
%\begin{enumerate}\item[]
%($a$) where the qualifying benefit is income support, means a person who claims income support; and
%
%($b$) where the qualifying benefit is a jobseeker’s allowance, means a person who claims a jobseeker’s allowance.
%\end{enumerate}
%
%\begin{sloppypar}
%(9) This paragraph applies to days on which the claimant is a person to whom—
%\end{sloppypar}
%\begin{enumerate}\item[]
%($a$) regulation 70 of the Income Support (General) Regulations 1987\footnote{\frenchspacing S.I. 1987/1967.} (urgent cases) applies other than by virtue of paragraph (2)($a$)  of that regulation (certain persons from abroad), or
%
%($b$) regulation 147 of the Jobseeker’s Allowance Regulations 1996\footnote{\frenchspacing S.I. 1996/207.} applies other than by virtue of paragraph (2)($a$)  of that regulation.
%\end{enumerate}
%
%\amendment{
%Words inserted in reg. 4(1), reg. 4(8), (9) inserted, reg. 4(1)(c)(i), (7)(b) substituted and reg. 4(3)(b)(i) omitted (6.4.97) by the Social Security (Miscellaneous Amendments) Regulations 1997 reg. 8(4).
%
%Reg. 4(1)(c)(i) substituted (1.4.98) by the Social Security (Miscellaneous Amendments) Regulations 1998 reg. 2(4).
%
%Words substituted in reg. 4 (1.4.03) by the Social Security Amendment (Carer's Allowance) Regulations 2002 Sch. 2 para. 1, 2.
%}
%
%\subsection[5. Amount payable]{Amount payable}
%
%5.—(1) The amount of the bonus shall be—
%\begin{enumerate}\item[]
%($a$) subject to the following provisions of this regulation, a sum representing the aggregate of—
%\begin{enumerate}\item[]
%(i) £5 for each benefit week in the bonus period in which the amount of child maintenance payable was not less than £5; and
%
%(ii) where in any benefit week in the bonus period the amount of child maintenance payable was less than £5, the amount that was payable;
%\end{enumerate}
%
%($b$) the amount of the child maintenance paid in the bonus period; or
%
%($c$) £1,000,
%\end{enumerate}
%whichever amount is the least.
%
%% Reg 5(2) omitted (6.4.97) by SI 1997/454 reg 8(5)
%%(2) Where the amount of child maintenance a person was liable to pay in any benefit week in the bonus period was less than £5 but the amount actually paid was higher than the amount he was required to pay, then for the purposes of paragraph (1)($a$) the weekly amount to be taken into account shall be the amount the person was liable to pay.
%
%(3) So much of any child maintenance paid in excess of the amount either—
%\begin{enumerate}\item[]
%($a$) declared for the purposes of determining the amount of qualifying benefit payable to the applicant or her partner; or
%
%($b$) retained by the Secretary of State in accordance with section 74A(3) of the Social Security Administration Act 1992,
%\end{enumerate}
%shall be disregarded in determining the amount payable under paragraph (1).
%
%% Reg 5(4) omitted (6.4.97) by SI 1997/454 reg 8(5)
%%(4) Where the child maintenance paid is child support maintenance, the amount of child maintenance to be taken into account in accordance with paragraph (1) shall be such amount as may in the particular case be certified by the child support officer.
%
%(5) Where but for this paragraph the amount of bonus payable in accordance with paragraph (1) would be less than £5, the amount of the bonus shall be Nil.
%
%\amendment{
%Reg. 5(2), (4) omitted (6.4.97) by the Social Security (Miscellaneous Amendments) Regulations 1997 reg. 8(5).
%}
%
%\subsection[6. Secretary of State to issue estimates]{Secretary of State to issue estimates}
%
%6.—(1) Where it appears to the Secretary of State that a person 
%%who is in receipt of a qualifying benefit
%with care% % Words substituted (6.4.97) by SI 1997/454 reg 8(6)
%, or the partner of such a person, may satisfy the requirements of regulation 3 (entitlement to a bonus) he may issue to 
%%the person in receipt of the qualifying benefit 
%that person  % Words substituted (6.4.97) by SI 1997/454 reg 8(6)
%a written statement of the amount he estimates may be payable by way of a bonus in his particular case, and may provide such further statements as appear appropriate in the circumstances, stating the amount he estimates may be payable.
%
%(2) The issue by the Secretary of State of a statement under paragraph (1) shall not be binding on the adjudication officer when he makes his determination on a claim for a bonus as to—
%\begin{enumerate}\item[]
%($a$) whether the applicant satisfies the conditions of entitlement to the bonus; and
%
%($b$) the amount, if any, payable where the bonus is awarded.
%\end{enumerate}
%
%\amendment{
%Words substituted in reg. 6(1) (6.4.97) by the Social Security (Miscellaneous Amendments) Regulations 1997 reg. 8(6).
%}
%
%\subsection[7. Death of a person with care of a child]{Death of a person with care of a child}
%
%7.—(1) In a case where—
%\begin{enumerate}\item[]
%($a$) the person (A) with care of a 
%%child in respect of whom child maintenance is payable dies;
%qualifying child to whom child maintenance is payable dies;  % Words substituted (6.4.97) by SI 1997/454 reg 8(6)
%
%($b$) on the date of her death, the person (A) was entitled or, where she has a partner, her partner was entitled to a qualifying benefit or had been so entitled within the 12 weeks ending on the date of her death;
%
%($c$) after the death, another person (B), who is a close relative of the person (A) and who was not before the death a person with the care of the child, becomes the person with care; and
%
%($d$) that other person was entitled or, where the other person has a partner, the other person or her partner was entitled to a qualifying benefit on the day the person (A) died or becomes entitled to a qualifying benefit within 12 weeks of the day on which the person (A) was last entitled to a qualifying benefit,
%\end{enumerate}
%then any weeks forming part of the bonus period of the person (A) which was current at the date of her death or within 12 weeks of the date on which she died shall be treated as part of the bonus period of the person (B) to the extent that those weeks are not otherwise a part of her bonus period.
%
%(2) In this Regulation, “close relative” means a parent, parent-in-law, son, son-in-law, daughter, daughter-in-law, step-parent, step-son, step-daughter, brother, sister, or the spouse of any of the preceding persons or, if that person is one of an unmarried couple, the other member of that couple.
%
%\amendment{
%Words substituted in reg. 7(1)(a) (6.4.97) by the Social Security (Miscellaneous Amendments) Regulations 1997 reg. 8(7).
%}
%
%\subsection[8. Retirement]{Retirement}
%
%8.—(1) In a case where the person with care of the child in respect of whom child maintenance is payable (the applicant) or the applicant’s partner, either—
%\begin{enumerate}\item[]
%($a$) is entitled to income support on the day before the applicant attains the age of 60; or
%
%($b$) is entitled to a jobseeker’s allowance on the day before the applicant attains pensionable age,
%\end{enumerate}
%the bonus period shall end on the day before the applicant attains 60 or, as the case may be, pensionable age and a bonus shall become payable to the applicant whether or not a claim is made for it.
%
%(2) Where an applicant who ceases to be entitled to a jobseeker’s allowance after attaining the age of 60 without satisfying the condition in paragraph (1)($b$) above, becomes entitled to income support within—
%\begin{enumerate}\item[]
%($a$) a period of 12 weeks of him ceasing to be entitled to a jobseeker’s allowance; or
%
%($b$) the duration of any connected period to which regulation 4(3) applies which immediately follows such an entitlement and which applies in his case,
%\end{enumerate}
%he shall be entitled to the bonus as though paragraph (1) were satisfied in his case and his bonus period shall be treated as though it ended on the day he becomes entitled to income support.
%
%(3) No day which falls after the day the bonus period ends in accordance with paragraph (1) or (4) or is treated as ending in accordance with paragraph (2), shall form part of that or any other bonus period.
%
%(4) Paragraph (5) shall apply where—
%\begin{enumerate}\item[]
%($a$) the applicant or the applicant’s partner—
%\begin{enumerate}\item[]
%(i) ceased to be entitled to income support in the 12 weeks preceding the date of the applicant attaining the age of 60;
%
%(ii) ceased to be entitled to a jobseeker’s allowance in the 12 weeks preceding the date of the applicant attaining pensionable age; and
%\end{enumerate}
%
%($b$) the person who ceased to be so entitled failed to satisfy the requirements of regulation 3(1)($c$) to ($f$).
%\end{enumerate}
%
%(5) Where this paragraph applies—
%\begin{enumerate}\item[]
%($a$) the bonus period shall end on the day entitlement to the qualifying benefit ceased; and
%
%($b$) a bonus shall become payable to the applicant, but only where a claim is made for it in accordance with regulation 10 (claiming a bonus).
%\end{enumerate}
%
%(6) In this regulation, “applicant” includes, where no claim is made, a person who would have been an applicant had a claim for a bonus been required.
%
%\subsection[9. Couples]{Couples}
%
%9.—(1) In the case of a couple, the person entitled to the bonus is the person who has the care of the child and to whom child maintenance is payable in respect of that child.
%
%(2) Where each member of a couple has both the care of a child and child maintenance is payable in respect of the child for whom they have care, each of them may qualify for a bonus in accordance with these Regulations where a qualifying benefit ceases to be payable to either of them because one of them, whether or not the person to whom the benefit was payable, satisfies the work condition.
%
%(3) A member of a couple to whom paragraph (2) applies shall not qualify for a bonus unless she claims it in accordance with regulation 10 (claiming a bonus).
%
%(4) Subject to paragraph (5), these Regulations shall apply to both members of a couple who separate as if they had never been one of a couple.
%
%(5) In the case of a couple who separate any entitlement to a qualifying benefit of one member of the couple during the time they were a couple shall be treated as the entitlement of both members of the couple for the purpose only of determining whether any day falls within a bonus period.
%
%\subsection[10. Claiming a bonus]{Claiming a bonus}
%
%10.—(1) A claim for a bonus shall be made in writing on a form approved for the purpose by the Secretary of State and shall be made—
%\begin{enumerate}\item[]
%($a$) not earlier than the beginning of the benefit week which precedes the benefit week in which an award of a qualifying benefit comes to an end, and
%
%($b$) except in a case to which sub-paragraph 
%%($c$) or  % Words omitted (1.4.98) by SI 1998/563 reg 2(5)(a)
%($d$) applies, not later than 28 days after the day the qualifying benefit 
%%or entitlement to the qualifying benefit  % Words omitted (6.4.97) by SI 1997/454 reg 8(8)(a)
%ceases; or
%
%%($c$) where—
%%\begin{enumerate}\item[]
%%(i) a parent with care cares for one child only; and
%%
%%(ii) that child dies,
%%\end{enumerate}
%%in the period not exceeding 12 months which begins on the date the child died and throughout which she was entitled to a relevant benefit; or
%
%% Reg 10(1)(c) substituted (6.4.97) by SI 1997/454 reg 8(8)(b), omitted (1.4.98) by SI 1998/563 reg 2(5)(b)
%%($c$) where—
%%\begin{enumerate}\item[]
%%(i) a person with care cares for one child only; and
%%
%%(ii) that child dies,
%%\end{enumerate}
%%in the period not exceeding 12 months which begins on the date the child died and throughout which she, or where she has a partner, her partner is entitled to a qualifying benefit; or
%
%($d$) in the case of a person to whom regulation 8(4) refers, not later than 28 days after the day the applicant attains the age of 60 or, as the case may be, pensionable age.
%\end{enumerate}
%
%(2) A claim for a bonus shall be delivered or sent to an appropriate office.
%
%(3) If a claim is defective at the time it is received, the Secretary of State may refer the claim to the person making it and if the form is received properly completed within one month, or such longer period as the Secretary of State may consider reasonable, from the date on which it is so referred, the Secretary of State may treat the claim as if it had been duly made in the first instance.
%
%(4) A claim which is made on the form approved for the time being is, for the purposes of paragraph (3), properly completed if it is completed in accordance with instructions on the form and defective if it is not.
%
%(5) A person who claims a bonus shall furnish such certificates, documents, information and evidence in connection with the claim, or any questions arising out of it, as may be required by the Secretary of State and shall do so within one month of being required to do so or such longer period as the Secretary of State may consider reasonable.
%
%(6) Where a person who has attained the age of 60 but has not attained pensionable age for the purposes of a jobseeker’s allowance ceases to be entitled to a jobseeker’s allowance and becomes instead entitled to income support, regulation 8 (retirement) and this regulation shall apply in his case as if he attained the age of 60 on the day he first became entitled to income support.
%
%\amendment{
%Words omitted in reg. 10(1)(b) and reg. 10(1)(c) substituted (6.4.97) by the Social Security (Miscellaneous Amendments) Regulations 1997 reg. 8(8).
%
%Words omitted in reg. 10(1)(b) and reg. 10(1)(c) omitted (1.4.98) by the Social Security (Miscellaneous Amendments) Regulations 1998 reg. 2(5).
%}
%
%\subsection[11. Claims: further provisions]{Claims: further provisions}
%
%11.—(1) A person who has made a claim may amend it at any time by notice in writing received at an appropriate office before a determination has been made on the claim, and any claim so amended may be treated as if it had been so amended in the first instance.
%
%(2) A person who has made a claim may withdraw it at any time before a determination has been made on it, by notice to an appropriate office and any such notice of withdrawal shall have effect when it is received.
%
%(3) The date on which the claim is made shall be—
%\begin{enumerate}\item[]
%($a$) in the case of a claim which meets the requirements of regulation 10(1), the date on which it is received at an appropriate office; or
%
%($b$) in the case of a claim treated under regulation 10(3) as having been duly made, the date on which the claim was received in an appropriate office in the first place.
%\end{enumerate}
%
%(4) Where the applicant proves there was good cause throughout the period from the expiry of the 28 days specified in regulation 10(1), for failure to claim the bonus within the specified time, the time for claiming the bonus shall be extended to the date on which the claim is made or to a period of 6 months, whichever is the shorter period.
%
%\subsection[12. Payment of bonus]{Payment of bonus}
%
%12.  A bonus calculated by reference to child maintenance paid during periods of entitlement to a jobseeker’s allowance and to income support shall be treated as payable—
%\begin{enumerate}\item[]
%($a$) wholly by way of a jobseeker’s allowance, where the qualifying benefit last in payment in the bonus period was a jobseeker’s allowance; or
%
%($b$) wholly by way of income support, where the qualifying benefit last in payment in the bonus period was income support.
%\end{enumerate}
%
%\subsection[13. Payments on death]{Payments on death}
%
%13.—(1) Where a person satisfies the requirements for entitlement to a bonus other than the need to make a claim, but dies within 28 days of the last day of entitlement to a qualifying benefit, the Secretary of State may appoint such person as he may think fit to claim a bonus in place of the deceased person.
%
%(2) Where the conditions specified in paragraph (3) are satisfied, a claim may be made by the person appointed for the purpose of claiming a bonus to which the deceased person would have been entitled if he had claimed it in accordance with regulation 10 (claiming a bonus).
%
%(3) Subject to the following provisions of this regulation, the following conditions are specified for the purposes of paragraph (2)—
%\begin{enumerate}\item[]
%($a$) the application to the Secretary of State to be appointed a fit person to make a claim shall be made within 6 months of the date of death; and
%
%($b$) the claim shall be made in writing within 6 months of the date the appointment was made.
%\end{enumerate}
%
%(4) Subject to paragraphs (5) and (6), the Secretary of State may, in exceptional circumstances, extend the period for making an application or a claim to such longer period as he considers appropriate in the particular case.
%
%(5) Where the period is extended in accordance with paragraph (4), the period specified in paragraph (3)($a$) or ($b$) shall be shortened by a corresponding period.
%
%(6) The Secretary of State shall not extend the period for making an application or a claim in accordance with paragraph (4) for more than 12 months from the date of death, but in calculating that period any period between the date when an application for a person to be appointed to make a claim is made and the date when the Secretary of State makes the appointment shall be disregarded.
%
%(7) A claim made in accordance with paragraph (2) shall be treated, for the purposes of these Regulations, as if made on the date of the deceased’s death.

\subsection[14. Bonus to be treated as capital for certain purposes]{Bonus to be treated as capital for certain purposes}

14.  Any bonus paid to an applicant shall be treated as capital of hers for the purposes of—
\begin{enumerate}\item[]
($a$) housing benefit;

($b$) council tax benefit;

($c$) 
%family credit 
working families' tax credit%  % Words substituted (5.10.99) by SI 1999/2566 reg 2(1) and Sch 2 Pt I
;

($d$) 
%disability working allowance 
disabled person's tax credit%  % Words substituted (5.10.99) by SI 1999/2566 reg 2(2) and Sch 2 Pt II
;

($e$) income support;

($f$) a jobseeker’s allowance.
\end{enumerate}

\amendment{
Words substituted in reg. 14(c), (d) (5.10.99) by the Social Security and Child Support (Tax Credits) Consequential Amendments Regulations 1999 reg. 2 and Sch. 2 Pts. I, II.
}

\subsection[15. Capital to be disregarded]{Capital to be disregarded}

15.  There shall be added as—
\begin{enumerate}\item[]
($a$) 
%paragraph 48 
paragraph 50  % Words substituted (6.4.97) by SI 1997/454 reg 8(9)(a)
of Schedule 4 to the Disability Working Allowance (General) Regulations 1991\footnote{\frenchspacing S.I. 1991/2887; the relevant amending instrument is S.I. 1996/1345.} (capital to be disregarded);

($b$) 
%paragraph 49 
paragraph 51  % Words substituted (6.4.97) by SI 1997/454 reg 8(9)(b)
of Schedule 3 to the Family Credit (General) Regulations 1987\footnote{\frenchspacing S.I. 1987/1973; the relevant amending instrument is S.I. 1996/1345.};

($c$) 
%paragraph 50 
paragraph 52  % Words substituted (6.4.97) by SI 1997/454 reg 8(9)(c)
of Schedule 5 to the Council Tax Benefit (General) Regulations 1992\footnote{\frenchspacing S.I. 1987/1814; the relevant amending instrument is S.I. 1996/1510.} (capital to be disregarded); and

($d$) 
%paragraph 50 
paragraph 52  % Words substituted (6.4.97) by SI 1997/454 reg 8(9)(d)
of Schedule 5 to the Housing Benefit (General) Regulations 1987\footnote{\frenchspacing S.I. 1987/1971; the relevant amending instrument is S.I. 1996/1510.} (capital to be disregarded),
\end{enumerate}
the following paragraph—
\begin{quotation}
“The amount of any child maintenance bonus payable by way of a jobseeker’s allowance or income support in accordance with section 10 of the Child Support Act 1995\footnote{\frenchspacing 1995 c. 34.}, or a corresponding payment under Article 4 of the Child Support (Northern Ireland) Order 1995\footnote{\frenchspacing S.I. 1995/2702 (N.I.13).}, but only for a period of 52 weeks from the date of receipt.”.
\end{quotation}

\amendment{
Words substituted in reg. 15(a)--(d) (6.4.97) by the Social Security (Miscellaneous Amendments) Regulations 1997 reg. 8(9).
}

\subsection[16. No deduction from bonus]{No deduction from bonus}

16.—(1) The amendments specified in this regulation shall have effect so as to prevent deductions being made from any sum payable by way of bonus.

(2) In the Social Security (Claims and Payments) Regulations 1987\footnote{\frenchspacing S.I. 1987/968; the relevant amending instruments are S.I. 1996/672 and 1460.}—
\begin{enumerate}\item[]
($a$) in Schedule 9 (deductions from benefit for direct payments), in paragraph 1, in the definition of “specified benefit”, at the end there shall be added the words “but does not include any sum payable by way of child maintenance bonus in accordance with section 10 of the Child Support Act 1995\footnote{\frenchspacing 1995 c. 34.} and the 
%Child Maintenance Bonus 
Social Security (Child Maintenance Bonus)  % Words substituted (6.4.97) by SI 1997/454 reg 8(10)
Regulations 1996;”;

($b$) in Schedule 9A (deduction of mortgage interest from benefit and payment to qualifying lenders)\footnote{\frenchspacing Schedule 9A was inserted by S.I. 1992/1026; the relevant amending instruments are S.I. 1996/672 and 1460.} in paragraph 1, in the definition of “relevant benefits” at the end there shall be added the words “but does not include any sum payable by way of child maintenance bonus in accordance with section 10 of the Child Support Act 1995 and the 
%Child Maintenance Bonus 
Social Security (Child Maintenance Bonus)  % Words substituted (6.4.97) by SI 1997/454 reg 8(10)
Regulations 1996;”.
\end{enumerate}

(3) In the Social Security (Payments on account, Overpayments and Recovery) Regulations 1988\footnote{\frenchspacing S.I. 1988/664; the relevant amending instruments are S.I. 1995/829, 1996/672 and 1345.} in regulation 16(8), in the definition of “specified benefit”, at the end there shall be added the words “but does not include any sum payable by way of child maintenance bonus in accordance with section 10 of the Child Support Act 1995 and the 
%Child Maintenance Bonus 
Social Security (Child Maintenance Bonus)  % Words substituted (6.4.97) by SI 1997/454 reg 8(10)
Regulations 1996;”.

(4) In the Social Fund (Recovery by Deductions from Benefits) Regulations 1988\footnote{\frenchspacing S.I. 1988/35.}, after regulation 3 there shall be inserted the following regulation—
\begin{quotation}
\subsection*{“Child Maintenance Bonus}

4.  In regulation 3 above, income support and a jobseeker’s allowance do not include any sum payable by way of child maintenance bonus in accordance with section 10 of the Child Support Act 1995 and the 
%Child Maintenance Bonus 
Social Security (Child Maintenance Bonus)  % Words substituted (6.4.97) by SI 1997/454 reg 8(10)
Regulations 1996.”.
\end{quotation}

\amendment{
Words substituted in reg. 16 (6.4.97) by the Social Security (Miscellaneous Amendments) Regulations 1997 reg. 8(10).
}

\bigskip

Signed by authority of the Secretary of State for Social Security.

{\raggedleft
\emph{A.\ J.\ B.\ Mitchell}\\*Parliamentary Under-Secretary of
State,\\*Department of Social Security

}

18th December 1996

\small

\part{Explanatory Note}

\renewcommand\parthead{--- Explanatory Note}

\subsection*{(This note is not part of the Regulations)}

The regulations in this instrument are made either by virtue of section 10 of the Child Support Act 1995 (c.\ 34) (“the 1995 Act”) or are consequential upon that section. The instrument is made before the end of the period of 6 months beginning with the coming into force of that provision; the regulations in it are therefore exempt from the requirement in section 172(1) of the Social Security Administration Act 1992 (c.\ 5) to refer proposals to make these Regulations to the Social Security Advisory Committee and are made without reference to that Committee.

  Section 10 of the 1995 Act introduces a child maintenance bonus for persons who are or have been entitled to income support or an income-based jobseeker’s allowance and who have also been in receipt of child maintenance payments.

  Regulation 1 contains provisions relating to citation, commencement and interpretation.

  Regulations 2 to 4 specify the circumstances in which the Regulations apply and the conditions for entitlement to the bonus.

  Regulation 5 details the method of calculating the amount of the bonus and regulation 6 enables the Secretary of State to issue estimates to claimants of their projected level of bonus.

  Regulations 7 to 9 contain detailed provisions as to the application of the Regulations in specific situations, namely, where the person with the care of the child dies, is at or approaching retirement or is one of a couple.

  Regulations 10 and 11 provide details relating to claims for the bonus and regulations 12 and 13 relate to the payment of the bonus. Regulation 14 treats the bonus as capital for certain social security benefits.

  Regulation 15 provides for payments made by way of bonus to be disregarded for up to 12 months in calculating capital for the purposes of disability working allowance, family credit, council tax benefit and housing benefit.

  Regulation 16 provides that deductions which are made from the payment of certain social security benefits are not to be made from payments by way of child maintenance bonus.

  These Regulations do not impose a charge on businesses.

\end{document}