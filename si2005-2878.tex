\documentclass[12pt,a4paper]{article}

\newcommand\regstitle{The Social Security (Civil Partnership) (Consequential Amendments) Regulations 2005}

\newcommand\regsnumber{2005/2878}

%\opt{newrules}{
\title{\regstitle}
%}

%\opt{2012rules}{
%\title{Child Maintenance and~Other Payments Act 2008\\(2012 scheme version)}
%}

\author{S.I.\ 2005 No.\ 2878}

\date{Made
13th October 2005\\
Laid before Parliament
20th October 2005\\
Coming into~force
5th December 2005
}

%\opt{oldrules}{\newcommand\versionyear{1993}}
%\opt{newrules}{\newcommand\versionyear{2003}}
%\opt{2012rules}{\newcommand\versionyear{2012}}

\usepackage{csa-regs}

\setlength\headheight{42.11603pt}

%\hbadness=10000

\begin{document}

\maketitle

\begin{sloppypar}
\noindent
The Secretary of State for Work and~Pensions, in exercise of the powers conferred upon him by sections~48(1), 113(1)($b$), 122(1), 123(1)($a$), ($d$)  and~($e$),~134,~135, 136, 137(1) and~175(1), (3), (4) and~(5) of, and~paragraphs~2(2), 3B(5)($b$)(iii) and~7B(5)($b$)(iii) of Schedule 5 to, the Social Security Contributions and~Benefits Act 1992\footnote{1992 c.~4. Section 48(1) was amended by paragraph~24 of Schedule 24 to the Civil Partnership Act 2004 (c.~33). Sections 122(1) and 137(1) are cited because of the meaning there given to the word “prescribed”. Section 123(1)($e$) was substituted by paragraph~1(1) of Schedule 9 to the Local Government Finance Act 1992 (c.~14). Section 175(1) and (4) was amended by paragraph~29(1) and (4) of Schedule 2 to the Social Security Contributions (Transfer of Functions, etc.)\ Act 1999 (c.~2). Paragraphs 3B and 7B of Schedule 5 were inserted respectively by paragraphs 9 and 11 of Schedule 11 to the Pensions Act 2004 (c.~35), and were amended by paragraph~5(5) and (12) of the Schedule to S.I.~2005/2053.}, sections~5(1)($a$), 189(1),~(4) and~(5) and~191 of the Social Security Administration Act 1992\footnote{1992 c.~5; section~191 is cited for the definition of “prescribe”.}, sections~9(4),~10(6),~79(1),~(4) and~(6) and~84 of the Social Security Act 1998\footnote{1998 c.~14; section~84 is cited for the definition of “prescribe”.} and~paragraph~20(1) and~(3) of Schedule 7 to the Child Support, Pensions and~Social Security Act 2000\footnote{2000 c.~19.}, and~of all other powers enabling him in that behalf, after consultation in respect of provisions in these Regulations relating to housing benefit and~council tax benefit with organisations appearing to him to be representative of the authorities concerned\footnote{\emph{See} section~176(1) of the Social Security Administration Act 1992 as amended by section~103 of, and paragraph~23 of Schedule 9 to, the Local Government Finance Act 1992.}, and~after agreement by the Social Security Advisory Committee that proposals in respect of these Regulations should not be referred to it\footnote{\emph{See} sections 172(1) and 173(1)($b$) of the Social Security Administration Act 1992.}, hereby makes the following Regulations: \looseness=-1
\end{sloppypar}

{\sloppy

\tableofcontents

}

\bigskip

\setcounter{secnumdepth}{-2}

\subsection[1. Citation and~commencement]{Citation and~commencement}

1.  These Regulations may be cited as the Social Security (Civil Partnership) (Consequential Amendments) Regulations 2005, and~shall come into force on 5th December 2005 (immediately after the Civil Partnership (Pensions, Social Security and~Child Support) (Consequential, etc.\ Provisions) Order 2005\footnote{S.I.~2005/2877.} comes into force).

\subsection[2. Amendment of the Social Security (Widow’s Benefit and~Retirement Pensions) Regulations 1979]{Amendment of the Social Security (Widow’s Benefit and~Retirement Pensions) Regulations 1979}

2.---(1)  The Social Security (Widow’s Benefit and~Retirement Pensions) Regulations 1979\footnote{S.I.~1979/642.} are amended as follows.

(2) In regulation 1(2) (citation, commencement and~interpretation), after the definition “bereavement allowance” insert the following definition—
\begin{quotation}
““civil partner” in relation to any person who has been in a civil partnership more than once means the last civil partner;”.
\end{quotation}

(3) In regulation 4(1)($d$)(ii) (days to be treated as days of increment)\footnote{Paragraph~(1)($d$) was inserted by S.I.~2005/453.}, after “neither married to,” insert “or in a civil partnership with,”.

\subsection[3. Amendment of the Social Security (General Benefit) Regulations 1982]{Amendment of the Social Security (General Benefit) Regulations 1982}

3.---(1)  The Social Security (General Benefit) Regulations 1982\footnote{S.I.~1982/1408.} are amended as follows.

(2) In regulation 2(1), (2) and~(3) (exceptions from disqualification for imprisonment etc.)\footnote{Regulation~2 was amended by S.I.~1983/186, S.I.~1984/1303, S.I.~1986/1561, S.I.~1991/2742, S.I.\ 1995/829, S.I.~1996/1551, S.I.~2000/1483 and S.I.~2005/1551.}, for “wife or husband” substitute “spouse or civil partner”.

\subsection[4. Amendment of the Income Support (General) Regulations 1987]{Amendment of the Income Support (General) Regulations 1987}

4.---(1)  The Income Support (General) Regulations 1987\footnote{S.I.~1987/1967.} are amended as follows.

(2) In regulation 42(4)($a$)(i) (notional income)\footnote{Paragraph~(4) was substituted by S.I.~1988/1445, and amended by S.I.~1990/1776, S.I.~1991/1559, S.I.~1993/315, S.I.~1994/527, S.I.~1995/2792, S.I.~1995/3282, S.I.~1998/2117, S.I.~1999/2640, S.I.~2002/841, S.I.~2003/455, S.I.~2005/574 and S.I.~2005/2465.}, for “widow or widower” substitute “widow, widower or surviving civil partner”.

(3) In regulation 51(3)($a$)(i) (notional capital)\footnote{Paragraph~(3)($a$)(i) was amended by S.I.~1997/65, S.I.~2002/841 and S.I.~2003/455.}, for “widow or widower” substitute “widow, widower or surviving civil partner”.

\subsection[5. Amendment of the Social Security (Claims and~Payments) Regulations 1987]{Amendment of the Social Security (Claims and~Payments) Regulations 1987}

5.---(1)  The Social Security (Claims and~Payments) Regulations 1987\footnote{S.I.~1987/1968.} are amended as follows.

(2) In regulation 6(22) (date of claim)\footnote{S.I.~2002/428 and 2005/337 made relevant amendments to paragraph~(22).}, in the definition of “family” omit “married or unmarried”.

(3) In regulation 19(3B) (time for claiming benefit)\footnote{Paragraph~(3B) was inserted by S.I.~2005/777.} after “spouse” insert “or civil partner”.

\subsection[6. Amendment of the Housing Benefit (General) Regulations 1987]{Amendment of the Housing Benefit (General) Regulations 1987}

6.---(1)  The Housing Benefit (General) Regulations 1987\footnote{S.I.~1987/1971.} are amended as follows.

(2) In regulation 18(1)($c$)  (patients)\footnote{Paragraph~(1)($c$) was amended by S.I.~2003/1995 and S.I.~2005/522.}, for “a married or unmarried couple” substitute “a couple”.

\subsection[7. Amendment of the Council Tax Benefit (General) Regulations 1992]{Amendment of the Council Tax Benefit (General) Regulations 1992}

7.---(1)  The Council Tax Benefit (General) Regulations 1992\footnote{S.I.~1992/1814.} are amended as follows.

(2) In regulation 10(1)($c$)  (patients)\footnote{Paragraph~(1)($c$) was amended by S.I.~2003/1995 and S.I.~2005/522.}, for “a married or unmarried couple” substitute “a couple”.

\subsection[8. Amendment of the Social Security and~Child Support (Decisions and~Appeals) Regulations 1999]{Amendment of the Social Security and~Child Support (Decisions and~Appeals) Regulations 1999}

8.---(1)  The Social Security and~Child Support (Decisions and~Appeals) Regulations 1999\footnote{S.I.~1999/991.} are amended as follows.

(2) In regulation 1(3) (citation, commencement and~interpretation)—
\begin{enumerate}\item[]
($a$) after the definition of “clerk to the appeal tribunal” insert the following definition—
\begin{quotation}
““couple” means—
\begin{enumerate}\item[]
($a$) 
a man and~woman who are married to each other and~are members of the same household;

($b$) 
a man and~woman who are not married to each other but are living together as husband~and~wife;

($c$) 
two people of the same sex who are civil partners of each other and~are members of the same household; or

($d$) 
two people of the same sex who are not civil partners of each other but are living together as if they were civil partners,
\end{enumerate}
and~for the purposes of paragraph~($d$), two people of the same sex are to be regarded as living together as if they were civil partners if, but only if, they would be regarded as living together as husband~and~wife were they instead two people of the opposite sex;; and”; and
\end{quotation}

($b$) in the definition of “partner”\footnote{The definition of “partner” was inserted by S.I.~2002/1379.}, for “a married couple or an unmarried couple” substitute “a couple”.
\end{enumerate}

(3) In regulation 5(2) (date from which a decision revised under section~9 takes effect)\footnote{Paragraph~(2) was inserted by S.I.~2004/2283.}, in sub-paragraphs~($a$)(ii), ($b$)  and~sub-paragraph~(ii) at the end of the paragraph, after “spouse” insert “or civil partner”.

(4) In Schedule 3A (date from which superseding decision takes effect where a claimant is in receipt of income support or jobseeker’s allowance)\footnote{Schedule 3A was inserted by S.I.~2000/1596.}, in paragraph~8($e$)\footnote{Sub-paragraph~($e$) was inserted by S.I.~2001/518.} for “a married or an unmarried couple” substitute “a couple”.

\subsection[9. Amendment of the Housing Benefit and~Council Tax Benefit (Decisions and~Appeals) Regulations 2001]{Amendment of the Housing Benefit and~Council Tax Benefit (Decisions and~Appeals) Regulations 2001}

9.---(1)  The Housing Benefit and~Council Tax Benefit (Decisions and~Appeals) Regulations 2001\footnote{S.I.~2001/1002.} are amended as follows.

(2) In regulation 1(2) (citation, commencement and~interpretation)—
\begin{enumerate}\item[]
($a$) after the definition of “Council Tax Benefit Regulations” insert the following definition—
\begin{quotation}
““couple” means—
\begin{enumerate}\item[]
($a$) 
a man and~woman who are married to each other and~are members of the same household;

($b$) 
a man and~woman who are not married to each other but are living together as husband~and~wife;

($c$) 
two people of the same sex who are civil partners of each other and~are members of the same household; or

($d$) 
two people of the same sex who are not civil partners of each other but are living together as if they were civil partners,
\end{enumerate}
and~for the purposes of paragraph~($d$), two people of the same sex are to be regarded as living together as if they were civil partners if, but only if, they would be regarded as living together as husband~and~wife were they instead two people of the opposite sex;; and”; and
\end{quotation}

($b$) in the definition of “partner”, in sub-paragraph~($a$)  for “a married couple or an unmarried couple” substitute “a couple”.
\end{enumerate}

(3) In regulation 23(1) (procedure in connection with appeals)\footnote{Paragraph~(1) was amended by S.I.~2002/1379 and S.I.~2005/337.}, for “the Social Security, Child Support and~Tax Credits (Miscellaneous Amendments) Regulations 2005” substitute “the Social Security (Civil Partnership) (Consequential Amendments) Regulations 2005\footnote{S.I.~2005/2878.}”.

\subsection[10. Amendment of the Social Security (Deferral of Retirement Pensions) Regulations 2005]{Amendment of the Social Security (Deferral of Retirement Pensions) Regulations 2005}

10.---(1)  The Social Security (Deferral of Retirement Pensions) Regulations 2005\footnote{S.I.~2005/453.} are amended as follows.

(2) In regulation 3(1)($c$)  (amount of retirement pension not included in the calculation of the lump sum), after “married to” insert “or in a civil partnership with,”. 

\bigskip

Signed 
by authority of the 
Secretary of State for~Work and~Pensions.
%I concur
%By authority of the Lord Chancellor

{\raggedleft
\emph{Philip A Hunt}\\*
%Secretary
%Minister
Parliamentary Under-Secretary 
of State,\\*Department 
for~Work and~Pensions

}

13th October 2005

\small

\part{Explanatory Note}

\renewcommand\parthead{— Explanatory Note}

\subsection*{(This note is not part of the Regulations)}

These Regulations make minor amendments to a number of statutory instruments dealing with social security to reflect the new status of civil partnership. The amendments are consequential upon the Civil Partnership Act 2004 (“the Act”). The Act, the substantive provisions of which come into force on the same date as these Regulations, enables same-sex couples to form a civil partnership by registering as civil partners of each other (and~certain overseas relationships registered abroad may be treated as civil partnerships). The Act makes provision for civil partners to be treated in the same or similar way as spouses in relation to certain benefits and~obligations.

Regulation~2 amends the Social Security (Widow’s Benefit and~Retirement Pensions) Regulations 1979 (S.I.~1979/642) so as to insert a provision relating to the circumstances in which a civil partner or a former civil partner can use the National Insurance contributions of the other party to the civil partnership to increase their category A retirement pension. It also further amends the provisions relating to the days of deferral to be treated as days on which increments can be earned so that these provisions apply to civil partners.

Regulation~3 amends the Social Security (General Benefit) Regulations 1982 (S.I.~1982/1408) to ensure that, where a beneficiary’s civil partner is undergoing imprisonment or detention in legal custody, an adult dependency increase is not payable.

Regulation~4 amends the Income Support (General) Regulations 1987 (S.I.~1987/\hspace{0pt}1967) so as to provide that a war pension paid to third party in respect of a surviving civil partner is taken into account in the same way as a similar payment made in respect of a surviving spouse.

Regulation~5 amends the Social Security (Claims and~Payments) Regulations 1987 (S.I.~1987/1968) in relation to the timing of benefit claims by civil partners.

Regulation~6 amends the Housing Benefit (General) Regulations 1987 (S.I.~1987/\hspace{0pt}1971) to provide that same-sex couples and~opposite-sex couples are treated in the same way where the calculation of the applicable amount for an in-patient is concerned. Regulation~7 makes a similar amendment to the Council Tax Benefit (General) Regulations 1992 (S.I.~1992/1814).

Regulation~8 amends the Social Security and~Child Support (Decisions and~Appeals) Regulations 1999 (S.I.~1999/991) to ensure that same-sex couples are treated in the same way as opposite-sex couples in matters of decision-making and~appeals.

Regulation~9 amends the Housing Benefit and~Council Tax Benefit (Decisions and~Appeals) Regulations 2001 (S.I.~2001/1002) to ensure that same-sex couples are treated in the same way as opposite-sex couples in matters of decision-making and~appeals in respect of Housing Benefit and~Council Tax Benefit.

Regulation~10 amends the Social Security (Deferral of Retirement Pensions) Regulations 2005 (S.I.~2005/453) in relation to the amount of retirement pension the deferrer would have earned in an accrual period in the calculation of a lump sum so that these provisions apply to civil partners.

A full regulatory impact assessment has not been produced for this instrument as it has no impact on the costs of business. However a full Regulatory Impact Assessment was produced for the Civil Partnership Bill which reflects all the costs to Government, business and~the voluntary sector. The RIA can be accessed at \url{http://www.dti.gov.uk/access/ria/pdf/ria-civilpartnerships2004.pdf}. 

\end{document}