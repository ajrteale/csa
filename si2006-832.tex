\documentclass[12pt,a4paper]{article}

\newcommand\regstitle{The Social Security (Miscellaneous Amendments) (No.~2) Regulations 2006}

\newcommand\regsnumber{2006/832}

%\opt{newrules}{
\title{\regstitle}
%}

%\opt{2012rules}{
%\title{Child Maintenance and~Other Payments Act 2008\\(2012 scheme version)}
%}

\author{S.I.\ 2006 No.\ 832}

\date{Made
20th March 2006\\
Laid before Parliament
20th March 2006\\
Coming into~force\\
except for regulation~3
10th April 2006\\
regulation~3
24th July 2006
}

%\opt{oldrules}{\newcommand\versionyear{1993}}
%\opt{newrules}{\newcommand\versionyear{2003}}
%\opt{2012rules}{\newcommand\versionyear{2012}}

\usepackage{csa-regs}

\setlength\headheight{42.11603pt}

%\hbadness=10000

\begin{document}

\maketitle

\enlargethispage{1.65405pt}

\begin{sloppypar}
\noindent
The Secretary of State for Work and~Pensions makes the following Regulations in exercise of the powers conferred upon him by sections 5(1)($a$)  to~($c$),~($i$),~($j$),~($m$)  and~($p$), 7A(1) and~(6)($d$), 111A(1A), (1B), (1D) and~(1E),~112(1A) to~(1D), 189(1) and~(4) to~(6) and~191 of the Social Security Administration Act 1992\footnote{1992 c.~5. Section 7A was inserted by section~71 of the Welfare Reform and~Pensions Act 1999 (c.~30). Section 111A was inserted by section~13 of the Social Security Administration (Fraud) Act 1997 (c.~47). Subsections (1A) to (1C) of section~111A were inserted, and~subsections (1A) to (1D) of section~112 were substituted by section~16 of the Social Security Fraud Act 2001 (c.~11). Section 189(1) was amended by paragraph~57 of Schedule 3 to the Social Security Contributions (Transfer of Functions, etc.)\ Act 1999 (c.~2) and~Schedule 6 to the Tax Credits Act 2002 (c.~21). Section 191 is cited for the meaning ascribed to the word “prescribe”.} and~sections 9(4), 10(3) and~(6), 79(1) and~(4) and~84 of the Social Security Act 1998\footnote{1998 c.~14. Section 84 is cited for the meaning ascribed to the word “prescribed”.}.
\end{sloppypar}

In accordance with section~173(1)($b$)  of the Social Security Administration Act 1992 he has obtained the agreement of the Social Security Advisory Committee that proposals to~make these Regulations should not be referred to~it. \looseness=-1

{\sloppy

\tableofcontents

}

\bigskip

\setcounter{secnumdepth}{-2}

\subsection[1. Citation and~commencement]{Citation and~commencement}

1.---(1)  These Regulations may be cited as the Social Security (Miscellaneous Amendments) (No.~2) Regulations 2006.

(2) These Regulations shall come into~force on 10th April 2006 except for regulation~3 which shall come into~force on 24th July 2006.

\subsection[2. Amendment of the Social Security (Claims and~Payments) Regulations 1987]{Amendment of the Social Security (Claims and~Payments) Regulations 1987}

2.---(1)  The Social Security (Claims and~Payments) Regulations 1987\footnote{S.I.~1987/1968.} are amended as follows.

(2) In regulation~2(1) (interpretation) omit the definition of “instrument for benefit payment”\footnote{The definition was inserted by S.I.~1994/3196 and~amended by S.I.~1999/2572.}.

(3) In regulation~4(6A)\footnote{Paragraph~(6A) was inserted by S.I.~2003/1632 and~amended by S.I.~2005/1551.} (making a claim for benefit)—
\begin{enumerate}\item[]
($a$) after sub-paragraph~($b$)  omit “or”;

($b$) at the end of sub-paragraph~($c$)  add “; or”; and

($c$) after sub-paragraph~($c$)  add—
\begin{quotation}
“($d$) who has not attained the qualifying age and~who makes a claim for a disability living allowance or a carer’s allowance.”.
\end{quotation}
\end{enumerate}

(4) In regulation~6 (date of claim)—
\begin{enumerate}\item[]
($a$) after paragraph~(1C)\footnote{Paragraph~(1C) was inserted by S.I.~2001/567 and~amended by S.I.~2001/892.} insert—
\begin{quotation}
“(1D) Subject to~paragraph~(1E) and~without prejudice to~the generality of paragraph~(1), where a properly completed claim for incapacity benefit is received in an appropriate office within one month of the claimant first notifying such an office, by whatever means, of his intention to~make that claim, the date of claim shall be the date on which that notification is made or the first day in respect of which the claim is made if later.

(1E) For the purposes of paragraph~(1D), a person who has attained the qualifying age may notify his intention and~may send or deliver his claim to~an office specified in regulation~4(6B)\footnote{Paragraph~(6B) was inserted by S.I.~2003/1632.}.”;
\end{quotation}

($b$) in paragraph~(17)\footnote{Paragraph~(17) was substituted by S.I.~2000/1596 and~amended by S.I.~2002/2497.} omit sub-paragraph~($b$);

($c$) in paragraph~(22)\footnote{Paragraph~(22) was substituted by S.I.~2000/1596 and~amended by S.I.~2002/428 and~2497 and~2005/337.} for “and~(30)” substitute “,~(30) and~(33)”;

($d$) in paragraph~(26)\footnote{Paragraph~(26) was substituted by S.I.~2000/1596.} for “18($a$)  and~($c$), 21($a$), 24 and~in paragraph~18($b$)” substitute “(18)($a$)  and~($c$), (21)($a$), (24) and~(30) and~in paragraph~(18)($b$)”; and

($e$) after paragraph~(32)\footnote{Paragraph (32) was inserted by S.I.~2004/2283.} add—
\begin{quotation}
“(33) Where a person makes a claim for a carer’s allowance within 3 months of a decision made—
\begin{enumerate}\item[]
($a$) on a claim;

($b$) on revision or supersession; or

($c$) on appeal whether by an appeal tribunal, a Commissioner or the court,
\end{enumerate}
awarding a qualifying benefit to~the disabled person, the date of claim is the first day in respect of which that qualifying benefit is payable.”.
\end{quotation}
\end{enumerate}

(5) In regulation~6A(2)\footnote{Regulation 6A was inserted by S.I.~2000/897 and~amended by S.I.~2001/3210, 2002/1703 and~2004/959.} (claims by persons subject to~work-focused interviews) after sub-paragraph~($c$)  add—
\begin{quotation}
“($d$) without prejudice to~sub-paragraphs~($a$)  and~($b$), where a properly completed claim for incapacity benefit is received in an appropriate office within one month of the claimant first notifying such an office, by whatever means, of his intention to~make that claim, the date of claim shall be the date on which that notification is made or the first day in respect of which the claim is made if later.”.
\end{quotation}

(6) For regulation~20\footnote{Regulation 20 was amended by S.I.~1994/3196 and~1999/2572.} and~its heading (time and~manner of payment: general provision) substitute—
\begin{quotation}
\subsection*{“Time of payment: general provision}

20.  Subject to~regulations 21 to~26B, benefit shall be paid in accordance with an award as soon as is reasonably practicable after the award has been made.”.
\end{quotation}

(7) Regulation 20A\footnote{Regulation 20A was inserted by S.I.~1994/3196 and~amended by S.I.~1996/672 and~1999/2572.} (payment on presentation of an instrument for benefit payment) shall be omitted.

(8) In regulation~21(1)\footnote{Paragraph (1) was substituted by S.I.~2002/2441.} (direct credit transfer) after “person claiming or entitled to~it” insert “or person appointed under regulation~33 or specified in regulation~33(1)($c$)  or ($d$)”.

(9) In regulation~26 (income support)—
\begin{enumerate}\item[]
($a$) in paragraph~(1)\footnote{Paragraph (1) was amended by S.I.~1993/1113 and~2000/1596.} (payment) omit “manner in and”; and

($b$) omit paragraphs~(2) and~(3)\footnote{Paragraphs~(2) and~(3) were amended by S.I.~1989/136 and~1999/3178.} (amounts paid by a book of serial orders).
\end{enumerate}

(10) In regulation~26B\footnote{Regulation 26B was inserted by S.I.~2002/3019.} (state pension credit)—
\begin{enumerate}\item[]
($a$) in paragraph~(1) for “,~where state pension credit is payable in accordance with paragraph~(3)($a$), to~the provisions of regulation~21 (direct credit transfer)” substitute “to~regulation~21 where payment is by direct credit transfer”;

($b$) omit paragraph~(3) (manner of payment);

($c$) in paragraph~(4) for “in accordance with paragraph~(3)($b$)” substitute “otherwise than in accordance with regulation~21”; and

($d$) omit paragraphs~(6) and~(7) (amounts paid by a book of serial orders).
\end{enumerate}

(11) In regulation~38(1) (extinguishment of right to~payment of sums by way of benefit where payment is not obtained within the prescribed period)—
\begin{enumerate}\item[]
($a$) omit sub-paragraph~($aa$)\footnote{Sub-paragraph ($aa$)  was inserted by S.I.~1996/672.} (payment by an instrument for benefit payment); and

($b$) in sub-paragraph~($c$)\footnote{Sub-paragraph ($c$) was amended by S.I.~1996/672, 1999/2572 and~2005/337.} omit “($aa$),”.
\end{enumerate}

(12) For regulation~47\footnote{Regulation 47 was substituted by S.I.~1994/3196 and~amended by S.I.~1999/2572.} and~its heading (instruments of payment, etc and~instruments for benefit payment) substitute—
\begin{quotation}
\subsection*{“Instruments of payment}

47.---(1)  Instruments of payment issued by the Secretary of State shall remain his property.

(2) Any person having an instrument of payment shall, on ceasing to~be entitled to~the benefit to~which the instrument relates, or when so required by the Secretary of State, deliver it to~the Secretary of State or such other person as he may direct.”.
\end{quotation}

(13) For the heading to~Schedule 7\footnote{The heading to Schedule 7 was amended by S.I.~2000/1596 and~paragraph 1 was amended by S.I.~1996/672.} (manner and~time of payment, and~commencement of entitlement in income support cases) and~paragraph~1 of that Schedule substitute—
\begin{quotation}
\part*{\noindent “Time of payment and~commencement of entitlement in income support cases}

1.  Except as otherwise provided in these Regulations income support shall be paid in arrears in accordance with the award.”.
\end{quotation}

(14) In Schedule 9 (deductions from benefit and~direct payment to~third parties) in paragraph~4(2A)\footnote{Sub-paragraph (2A) was inserted by S.I.~2003/2325 and~amended by S.I.~2005/777.} (miscellaneous accommodation costs) for “£18$.$80” each time it appears substitute “£19$.$60”.

\subsection[3. Further amendment of the Social Security (Claims and~Payments) Regulations 1987]{Further amendment of the Social Security (Claims and Payments) Regulations 1987}

3.---(1)  The Social Security (Claims and~Payments) Regulations 1987 are further amended as follows.

(2) In regulation~4D\footnote{Regulations 4D to 4F were inserted by S.I.~2002/3019. Regulation 4D was amended by S.I.~2003/1632 and~2005/337.} (making a claim for state pension credit)—
\begin{enumerate}\item[]
($a$) in paragraph~(3) omit “by telephone to, or”;

($b$) in paragraph~(6) omit “or by telephone”;

($c$) after paragraph~(6) insert—
\begin{quotation}
“(6A) A claim for state pension credit may be made by telephone call to~the telephone number specified by the Secretary of State.

(6B) Where the Secretary of State, in any particular case, directs that the person making the claim approves a written statement of his circumstances, provided for the purpose by the Secretary of State, a claim made by telephone is not a valid claim unless the person complies with the direction.\looseness=1

(6C) A claim made by telephone in accordance with paragraph~(6A) is defective unless the Secretary of State is provided, during that telephone call, with all the information he requires to~determine the claim.

(6D) Where a claim made by telephone in accordance with paragraph~(6A) is defective, the Secretary of State is to provide the person making it with an opportunity to correct the defect.\looseness=-1

(6E) If the person corrects the defect within one month, or such longer period as the Secretary of State considers reasonable, of the date the Secretary of State last drew attention to~the defect, the Secretary of State shall treat the claim as if it had been duly made in the first instance.”; and
\end{quotation}

($d$) in paragraph~(12) for “Paragraph~(11) does” substitute “Paragraphs~(6E) and~(11) do”.
\end{enumerate}

(3) In regulation~4F(2)\footnote{Regulation 4F was amended by S.I.~2003/1632 and~2004/2327.} (making a claim after attaining the qualifying age: date of claim)—
\begin{enumerate}\item[]
($a$) in sub-paragraph~($b$)  after “4D(3)” insert “or (6A)”; and

($b$) in sub-paragraph~($c$)  for “4D(11)” substitute “4D(6E) or (11)”.
\end{enumerate}

(4) In regulation~5(1)\footnote{Regulation 5 was amended by S.I.~2005/34.} (amendment and~withdrawal of claim) after “regulation~4(11)” insert “or 4D(6A)”.

\subsection[4. Amendment of the Jobseeker’s Allowance Regulations 1996]{Amendment of the Jobseeker’s Allowance Regulations 1996}

4.  In regulation~24(7) of the Jobseeker’s Allowance Regulations 1996\footnote{S.I.~1996/207. Paragraph (7) was amended by S.I.~2000/1978.} (provision of information and~evidence) for “in writing (unless the Secretary of State determines in any particular case to~accept notice given otherwise than in writing) to~the appropriate office” substitute—
\begin{quotation}
    “of the change to~an office of the Department for Work and~Pensions specified by the Secretary of State—
\begin{enumerate}\item[]
    (i) 
    in writing or by telephone (unless the Secretary of State determines in any particular case that notice must be in writing or may be given otherwise than in writing or by telephone); or

    (ii) 
    in writing if in any class of case he requires written notice (unless he determines in any particular case to~accept notice given otherwise than in writing)”. 
\end{enumerate}
\end{quotation}

\subsection[5. Amendment of the Social Security and~Child Support (Decisions and~Appeals) Regulations 1999]{Amendment of the Social Security and~Child Support (Decisions and~Appeals) Regulations 1999}

5.---(1)  The Social Security and~Child Support (Decisions and~Appeals) Regulations 1999\footnote{S.I.~1999/991.} are amended as follows.

(2) In regulation~3 (revision of decisions)—
\begin{enumerate}\item[]
($a$) in paragraph~(7B)\footnote{Paragraphs (7B) and (7C) were inserted by S.I.~2005/337.} after “successful” add “or lapses”;

($b$) in paragraph~(7C)—
\begin{enumerate}\item[]
(i) after “incapable of work and” insert “the decision which embodies that determination is revised or”; and

(ii) for “that embodies” substitute “which embodies”; and
\end{enumerate}

($c$) after paragraph~(7E)\footnote{Paragraph (7E) was inserted by S.I.~2005/2677.} insert—
\begin{quotation}
“(7F) A decision under regulation~17(1)($d$)  of the Income Support Regulations that a person is no longer entitled to~a disability premium because of a determination that he is not incapable of work may be revised where the decision which embodies that determination is revised or his appeal against the decision is successful.”.
\end{quotation}
\end{enumerate}

(3) In regulation~7 (date from which a decision superseded under section~10 takes effect)—
\begin{enumerate}\item[]
($a$) in paragraph~(2)($c$)\footnote{Sub-paragraph ($c$) was amended by S.I.~1999/1623 and 3178.}—
\begin{enumerate}\item[]
(i) omit head (iii); and

(ii) after head (ii)  add—
\begin{quotation}
“(iv) in the case of a disability benefit decision, where the change of circumstances is not in relation to~the disability determination embodied in or necessary to~the disability benefit decision, from the date of the change; or

(v) in any other case, except in the case of a decision which supersedes a disability benefit decision, from the date of the change.”; and
\end{quotation}
\end{enumerate}

\enlargethispage{\baselineskip}

($b$) for paragraph~(7)\footnote{Paragraph (7) was substituted by S.I.~2002/428 and amended by S.I.~2005/337.} substitute—
\begin{quotation}
“(7) A decision which is superseded in accordance with regulation~6(2)($e$)  or ($ee$)  shall be superseded—
\begin{enumerate}\item[]
($a$) subject to~sub-paragraph~($b$), from the date on which entitlement arises to~the other relevant benefit referred to~in regulation~6(2)($e$) (ii)  or ($ee$)  or to~an increase in the rate of that other relevant benefit; or

($b$) where the claimant or his partner—
\begin{enumerate}\item[]
(i) is not a severely disabled person for the purposes of section~135(5) of the Contributions and~Benefits Act (the applicable amount) or section~2(7) of the State Pension Credit Act (guarantee credit)

(ii) by virtue of his having—
\begin{enumerate}\item[]
($aa$) a non-dependant as defined by regulation~3 of the Income Support Regulations; or

($bb$) a person residing with him for the purposes of paragraph~1 of Schedule 1 to~the State Pension Credit Regulations whose presence may not be ignored in accordance with paragraph~2 of that Schedule,
\end{enumerate}
at the date the superseded decision would, but for this sub-paragraph, have had effect,
\end{enumerate}
from the date on which the claimant or his partner ceased to~have a non-dependant or person residing with him or from the date on which the presence of that person was first ignored.”.
\end{enumerate}
\end{quotation}
\end{enumerate}

(4) In Schedule 3B\footnote{Schedule 3B was inserted by S.I.~2002/3019 and amended by S.I.~2003/2274.} (date on which change of circumstances takes effect where claimant entitled to~state pension credit) after paragraph~6 add—
\begin{quotation}
“7.  Where an amount of state pension credit payable under an award is changed by a superseding decision specified in paragraph~8 the superseding decision shall take effect from the day specified in paragraph~1($b$).

\medskip

8.  The following are superseding decisions for the purposes of paragraph~7—
\begin{enumerate}\item[]
($a$) a decision which supersedes a decision specified in regulation~6(2)($b$)  to~($ee$)  and~($m$); and

\enlargethispage{\baselineskip}

($b$) a superseding decision which would, but for paragraphs~2 and~7, take effect from a date specified in regulation~7(5) to~(7), (12) to~(16) and~(29C).”.
\end{enumerate}
\end{quotation}

\subsection[6. Amendment of the Social Security (Notification of Change of Circumstances) Regulations 2001]{\sloppy Amendment of the Social Security (Notification of Change of Circumstances) Regulations 2001}

6.  In regulation~3(1) of the Social Security (Notification of Change of Circumstances) Regulations 2001\footnote{S.I.~2001/3252, to which there are amendments not relevant to these Regulations.} (changes affecting jobseeker’s allowance)—
\begin{enumerate}\item[]
($a$) omit “or sent”;

($b$) omit “in writing (except where he determines in any particular case that he will accept notice other than in writing)”; and

($c$) after “1996” add—
\begin{quotation}
“($a$) in writing or by telephone (unless the Secretary of State determines in any particular case that notice must be in writing or may be given otherwise than in writing or by telephone); or

($b$) in writing if in any class of case he requires written notice (unless he determines in any particular case to~accept notice given otherwise than in writing).”.
\end{quotation}
\end{enumerate}

\subsection[7. Revocations]{Revocations}

7.  The following regulations, which provide for instruments for benefit payment, shall be revoked—
\begin{enumerate}\item[]
($a$) the Social Security (Claims and~Payments) Amendment (No.~4) Regulations 1994\footnote{S.I.~1994/3196.}; and

($b$) regulations 2(2), (5) to~(8), 4 and~5 of the Social Security (Claims and~Payments Etc.)\ Amendment Regulations 1996\footnote{S.I.~1996/672.}.
\end{enumerate}

\bigskip

Signed 
by authority of the 
Secretary of State for~Work and~Pensions.
%I concur
%By authority of the Lord Chancellor

{\raggedleft
\emph{Anne McGuire}\\*
%Secretary
%Minister
Parliamentary Under-Secretary 
of State,\\*Department 
for~Work and~Pensions

}

20th March 2006

\small

\part{Explanatory Note}

\renewcommand\parthead{— Explanatory Note}

\subsection*{(This note is not part of the Regulations)}

These Regulations amend the Social Security (Claims and~Payments) Regulations 1987 (“the 1987 Regulations”), the Jobseeker’s Allowance Regulations 1996 (“the 1996 Regulations”), the Social Security and~Child Support (Decisions \& Appeals) Regulations 1999 (“the 1999 Regulations”) and~the Social Security (Notification of Change of Circumstances) Regulations 2001 (“the 2001 Regulations”).

Regulation 2 amends the 1987 Regulations. Paragraphs~(2), (7), (11), (12) and~(13) omit provisions which provide for instruments for benefit payment. Regulation 7 makes consequential revocations.

Regulation 2(3) allows claims for a disability living allowance and~a carer’s allowance to~be made at offices other than offices of the Department for Work and~Pensions which have been approved by the Secretary of State.

Regulation 2(4)($a$)  provides that where a person first notifies an appropriate office that he intends to~claim incapacity benefit and~his properly completed claim is received in such an office within one month, the date of claim is the date of the first notification, and~provide that a person who has attained the qualifying age for state pension credit may notify his intention and~may send or deliver his claim to~an office approved by the Secretary of State.

Regulation 2(4)($b$)  to~($e$)  allows a claim for carer’s allowance to~be treated as made on the first day for which the qualifying benefit is payable, where the claim is made within three months of the date of the award of that qualifying benefit.

Regulation 2(5) provides that where a person subject to~work-focused interviews notifies an appropriate office of his intention of making a claim for incapacity benefit and~a properly completed claim is received within one month, the date of claim is the date of first notification.

Regulation 2(6), (10)($b$)  and~(13) omit provisions which provide that benefit shall be paid by instrument of payment.

Regulation 2(8) provides for an appointee, or other specified persons acting on behalf of a person claiming or entitled to~benefit, to~arrange for it to~be paid by direct credit transfer.

Regulation 2(9)($b$), (10)($d$)  and~(12) revoke provisions concerning books of serial orders.

Regulation 2(14) increases the amount allowed for personal expenses for a person in accommodation for which benefit is paid to~his accommodation provider.

Regulation 3(2) and~(3) enables a person to~claim state pension credit by telephone unless the Secretary of State directs that the claimant must approve a written statement of his circumstances provided for the purpose by the Secretary of State. Paragraph~(4) enables a person who has made such a claim to~amend it by telephone.

Regulation 4 amends the 1996 Regulations to~provide that a change of circumstances affecting the continuance of entitlement to, or the payment of, jobseeker’s allowance shall be notified to~the Secretary of State in writing or by telephone (unless he requires the person to~give written notice or accepts another means of notification) or, if he so requires in a class of case, the change shall be notified in writing unless he accepts another means of notification in any particular case.

Regulation 5 amends the 1999 Regulations. Paragraph~(2) allows the Secretary of State to~revise a decision where a determination necessary to~it has been revised or overturned on appeal. Paragraph~(3) amends regulation~7 in two ways. First, it removes the exception from the general rule for determining the effective date of a change of circumstances which is not advantageous. Secondly, it provides that where the claimant, who would otherwise be a severely disabled person, ceases to~have a non-dependant, the effective date is the date the claimant ceased to~have a non-dependant. Paragraph~(4) amends Schedule 3B so that, subject to~exceptions, a change of circumstances must take effect on the first day of the benefit week following the change.

Regulation 6 amends the 2001 Regulations to~make provision similar to~regulation~4 for the purpose of offences relating to~failure to~notify changes of circumstances.

A full regulatory impact assessment has not been prepared for this instrument as it has no significant impact on the costs of business, charities and~voluntary bodies. 

\end{document}