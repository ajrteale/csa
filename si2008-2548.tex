\documentclass[12pt,a4paper]{article}

\newcommand\regstitle{The Child Maintenance and Other Payments Act 2008 (Commencement No.~3 and Transitional and Savings Provisions) Order 2008}

\newcommand\regsnumber{2008/2548}

%\opt{newrules}{
\title{\regstitle}
%}

%\opt{2012rules}{
%\title{Child Maintenance and~Other Payments Act 2008\\(2012 scheme version)}
%}

\author{S.I.\ 2008 No.\ 2548 (C.~110)}

\date{Made
%26th September 2008\\
%Laid before Parliament
%1st October 2008\\
%Coming into force
25th September 2008
}

%\opt{oldrules}{\newcommand\versionyear{1993}}
%\opt{newrules}{\newcommand\versionyear{2003}}
%\opt{2012rules}{\newcommand\versionyear{2012}}

\usepackage{csa-regs}

\setlength\headheight{42.11603pt}

%\hbadness=10000

\begin{document}

\maketitle

\noindent
The Secretary of State, in exercise of the powers conferred by sections 62(3) and (4) of the Child Maintenance and Other Payments Act 2008\footnote{2008 c.~6.}, makes the following Order: 

{\sloppy

\tableofcontents

}

\bigskip

\setcounter{secnumdepth}{-2}

\subsection[1. Citation and interpretation]{Citation and interpretation}

1.---(1)  This Order may be cited as the Child Maintenance and Other Payments Act 2008 (Commencement No.~3 and Transitional and Savings Provisions) Order 2008.

(2) In this Order—
\begin{enumerate}\item[]
“the 1991 Act” means the Child Support Act 1991\footnote{1991 c.~48.};

“the relevant date” means the 27th October 2008.
\end{enumerate}

(3) In this Order, except where otherwise stated, any reference to a numbered section~or Schedule is a reference to a section~of, or Schedule to, the Child Maintenance and Other Payments Act 2008.

\subsection[2. Appointed days for sections 20 and 36]{Appointed days for sections 20 and 36}

2.  The following provisions come into force on 26th September 2008 for the purpose of making regulations and on the relevant date for all other purposes—
\begin{enumerate}\item[]
($a$) section~20 (use of deduction from earnings orders as basic method of payment); and

($b$) section~36 (offence of failing to notify change of address).
\end{enumerate}

\subsection[3. Appointed day for other provisions]{Appointed day for other provisions}

3.  So far as not already in force, the following provisions come into force on the relevant date—
\begin{enumerate}\item[]
($a$) section~15($a$)  (repeal of section~6 of the 1991 Act);

($b$) sections 57(1) and 58, so far as relating to the paragraphs and entries referred to in sub-paragraphs ($c$)  and ($d$)  below;

($c$) paragraph 2(1) and (2) of Schedule 7 (minor and consequential amendments); and

($d$) in Schedule 8 (repeals), the entries relating to repeals in—
\begin{enumerate}\item[]
(i) sections 4, 8, 9, 11, 12, 14, 26, 27, 27A, 28, 28A, 28F, 28J,~29,~41 and 52 of, and Schedules 4A and 4B to, the 1991 Act;

(ii) sections 107 and 122(3) of the Social Security Administration Act 1992\footnote{1992 c.~5.}; and

(iii) section~3 of, and paragraphs 11(3) to (6), (8) to (10), (13)($a$)  and (22) of Schedule 3 to the Child Support, Pensions and Social Security Act 2000\footnote{2000 c.~19.}.
\end{enumerate}
\end{enumerate}

\subsection[4. Transitional and savings provisions]{Transitional and savings provisions}

4.---(1)  Any existing case, as defined in paragraph (2), is to be treated from the relevant date as though the person with care had made an application under section~4 of the 1991 Act.

(2) In paragraph (1) an existing case is one in which, immediately before 14th July 2008, section~6 of the 1991 Act applied in relation to a parent with care and—
\begin{enumerate}\item[]
($a$) a maintenance calculation or assessment was in force as a result of the Secretary of State acting under that section; or

($b$) no maintenance calculation or assessment had been made but the Secretary of State had given notice in accordance with regulation 5 of the Child Support (Maintenance Calculation Procedure) Regulations 2000\footnote{S.I.~2001/157.} or regulation 5 of the Child Support (Maintenance Assessment Procedure) Regulations 1992\footnote{S.I.~1992/1813, amended by S.I.~1993/913; there are other amending instruments but none is relevant.}.
\end{enumerate}

(3) Paragraph (1) is not to apply in any case where the maintenance assessment or calculation ceases to have effect, or would have ceased to have effect had it been made, before the relevant date.

(4) The repeal of section~6 of the 1991 Act is not to prevent the Secretary of State from exercising any powers that he would have otherwise had under section~41 of the 1991 Act to recover and retain arrears of child support maintenance accrued in relation to any period before the relevant date. 

\bigskip

Signed 
by authority of the 
Secretary of State for~Work and~Pensions.
%I concur

{\raggedleft
\emph{Stephen C.~Timms}\\*
Minister
%Parliamentary Under-Secretary 
of State,\\*Department for~Work and~Pensions

}

25th September 2008

\small

\part{Explanatory Note}

\renewcommand\parthead{— Explanatory Note}

\subsection*{(This note is not part of the Order)}

This Order brings into force provisions of the Child Maintenance and Other Payments Act 2008 (c.~6) as follows:

Article 2 brings into force sections 20 and 36 on 26th September 2008 for the purpose of making regulations and on 27th October 2008 for all other purposes.

Article 3 brings various other provisions fully into force on 27th October 2008. This includes section~15($a$), which repeals section~6 of the Child Support Act 1991 (“the 1991 Act”) (c.~48), and Schedules 7 and 8, which contain consequential provisions and repeals.

Article 4 makes transitional and savings provisions in relation to the repeal of section~6 of the 1991 Act. Existing cases (those where a maintenance calculation is in force or an effective date has been set) are treated from 27th October 2008 as though the person with care had applied to the Secretary of State under section~4 of the 1991 Act. Cases in which a maintenance calculation or assessment has, or would have if one had been made, ceased before 27th October 2008 shall not be treated as applied for under section~4 of the 1991 Act. Article 4(5) preserves the power of the Secretary of State to recover and retain arrears of child support maintenance for periods before 27th October 2008 when the parent with care was being paid a benefit.

A full impact assessment has not been produced for this instrument as it has no impact on the private or voluntary sectors. 

\end{document}