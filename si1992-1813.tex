\documentclass[a4paper,12pt]{article}

\newcommand\regstitle{The Child Support (Maintenance Assessment Procedure) Regulations 1992}

\newcommand\regsnumber{1992/1813}

%\opt{newrules}{
\title{\regstitle}
%}

%\opt{2012rules}{
%\title{Child Maintenance and Other Payments Act 2008\\(2012 scheme version)}
%}

\author{S.I. 1992 No. 1813}

\date{Made 20th July 1992\\Coming into force 5th April 1993}

%\opt{oldrules}{\newcommand\versionyear{1993}}
%\opt{newrules}{\newcommand\versionyear{2003}}
%\opt{2012rules}{\newcommand\versionyear{2012}}

\usepackage{csa-regs}

\setlength\headheight{27.57402pt}

\begin{document}

\maketitle

\noindent
 Whereas a draft of this instrument was laid before Parliament in accordance with section 52(2) of the Child Support Act 1991\footnote{\frenchspacing 1991 c. 48.} and approved by a resolution of each House of Parliament:

 Now, therefore, the Secretary of State for Social Security, in exercise of the powers conferred by sections 3(3), 5(3), 6(1), 12, 16, 17, 18, 42(3), 46(11), 51, 52(4), 54 and 55 of, and paragraphs 11, 14 and 16 of Schedule 1 to, the Child Support Act 1991 and of all other powers enabling him in that behalf hereby makes the following Regulations:

{\sloppy

\tableofcontents

}

\setcounter{secnumdepth}{-2}

\section[Part I --- General]{Part I\\*General}

\renewcommand\parthead{--- Part I}

\subsection[1. Citation, commencement and interpretation]{Citation, commencement and interpretation}

1.—(1) These Regulations may be cited as the Child Support (Maintenance Assessment Procedure) Regulations 1992 and shall come into force on 5th April 1993.

(2) In these Regulations, unless the context otherwise requires—
\begin{enumerate}\item[]
“the Act” means the Child Support Act 1991;

“applicable amount”%
,~except in regulation 40ZA,  % Words inserted (7.10.96) by SI 1996/1345 reg 5(2)(a)(i)
 is to be construed in accordance with Part IV of the Income Support Regulations;

“applicable amounts Schedule” means Schedule 2 to the Income Support Regulations\footnote{\frenchspacing Part I of Schedule 2 was substituted by Schedule 3 to S.I. 1991/2910 from 6.4.92.};

“award period” means a period in respect of which an award of family credit or disability working allowance is made;

“balance of the reduction period” means, in relation to a direction that is or has been in force, the portion of the period specified in a direction in respect of which no reduction of relevant benefit has been made;

“benefit week”, in relation to income support, has the same meaning as in the Income Support Regulations, 
in relation to jobseeker’s allowance has the same meaning as in the Jobseeker’s Allowance Regulations,  % Words inserted (7.10.96) by SI 1996/1345 reg 5(2)(a)(ii)
and, in relation to family credit and disability working allowance, is to be construed in accordance with the Social Security (Claims and Payments) Regulations 1987\footnote{\frenchspacing S.I. 1987/1968.};

“direction” means reduced benefit direction;

“disability working allowance” has the same meaning as in the Social Security Contributions and Benefits Act 1992\footnote{\frenchspacing 1992 c. 4.};

“day to day care” has the same meaning as in the Maintenance Assessments and Special Cases Regulations;

“effective application” means any application that complies with the provisions of regulation 2;

“effective date” means the date on which a maintenance assessment takes effect for the purposes of the Act;

“Income Support Regulations” means the Income Support (General) Regulations 1987\footnote{\frenchspacing S.I. 1987/1967.};

“Information, Evidence and Disclosure Regulations” means the Child Support (Information, Evidence and Disclosure) Regulations 1992\footnote{\frenchspacing S.I. 1992/1812.};

%Definition of ``the Jobseeker's Allowance Regulations'' inserted (7.10.96) by SI 1996/1345 reg 5(2)(a)(iii)
\begin{sloppypar}
“the Jobseeker’s Allowance Regulations” means the Jobseeker’s Allowance Regulations 1996\footnote{\frenchspacing S.I. 1996/207.};
\end{sloppypar}

%Definition of ``Maintenance Arrangements and Jurisdiction Regulations'' inserted (16.2.95) by SI 1995/123 reg. 4.
“Maintenance Arrangements and Jurisdiction Regulations” means the Child Support (Maintenance Arrangements and Jurisdiction) Regulations 1992\footnote{\frenchspacing S.I. 1992/2645.};

“Maintenance Assessments and Special Cases Regulations” means the Child Support (Maintenance Assessments and Special Cases) Regulations 1992\footnote{\frenchspacing S.I. 1992/1815.};

“maintenance period” has the meaning prescribed in regulation 33;

“obligation imposed by section 6 of the Act” is to be construed in accordance with section 46(1) of the Act;

“parent with care” means a person who, in respect of the same child or children, is both a parent and a person with care;

“the parent concerned” means the parent with respect to whom a direction is given;

“protected income level” has the same meaning as in paragraph 6(6) of Schedule 1 to the Act;

“relevant benefit” means income support, 
income-based jobseeker’s allowance,  % Words inserted (7.10.96) by SI 1996/1345 reg 5(2)(a)(iv)
family credit or disability working allowance;

“relevant person” means—
\begin{enumerate}\item[]
($a$) a person with care;

($b$) an absent parent;

($c$) a parent who is treated as an absent parent under regulation 20 of the Maintenance Assessments and Special Cases Regulations;

($d$) where the application for an assessment is made by a child under section 7 of the Act, that child,
\end{enumerate}
in respect of whom a maintenance assessment has been applied for or is or has been in force.
\end{enumerate}

(3) In these Regulations, references to a direction as being “in operation”, “suspended”, or “in force” shall be construed as follows—
\begin{enumerate}\item[]
a direction is “in operation” if, by virtue of that direction, relevant benefit is currently being reduced;

a direction is “suspended” if%
% either % Word omitted (22.1.96) by SI 1995/3261 reg 15(i)
—
\begin{enumerate}\item[]
($a$) after that direction has been given, relevant benefit ceases to be payable, or becomes payable at one of the rates indicated in regulation 40(3)
or, as the case may be, regulation 40ZA(4);  % Words inserted (7.10.96) by SI 1996/1345 reg 5(2)(b)(i)
%or % Word omitted (22.1.96) by SI 1995/3261 reg 15(ii)

($b$) at the time that the direction is given, relevant benefit is payable at one of the rates indicated in regulation 40(3)
or, as the case may be, regulation 40ZA(4);  % Words inserted (7.10.96) by SI 1996/1345 reg 5(2)(b)(i)
or  % Word inserted (22.1.96) by SI 1995/3261 reg 15(iii)

% Paragraph ($c$) inserted (22.1.96) by SI 1995/3261 reg 15(iv)
($c$) at the time that the direction is given one or more of the deductions set out in regulation 40A is being made from the income support 
or income-based jobseeker’s allowance  % Words inserted (7.10.96) by SI 1996/1345 reg 5(2)(b)(ii)
payable to or in respect of the parent concerned,
\end{enumerate}
and these Regulations provide for relevant benefit payable from a later date to be reduced by virtue of the same direction;

a direction is “in force” if it is either in operation or is suspended,
\end{enumerate}
and cognate terms shall be construed accordingly.

(4) The provisions of Schedule 1 shall have effect to supplement the meaning of “child” in section 55 of the Act.

(5) The provisions of these Regulations shall have general application to cases prescribed in regulations 19 to 26 of the Maintenance Assessments and Special Cases Regulations as cases to be treated as special cases for the purposes of the Act, and the terms “absent parent” and “person with care” shall be construed accordingly.

(6) Except where express provision is made to the contrary, where, by any provision of the Act or of these Regulations—
\begin{enumerate}\item[]
($a$) any document is given or sent to the Secretary of State, that document shall, subject to paragraph (7), be treated as having been so given or sent on the day it is received by the Secretary of State; and

($b$) any document is given or sent to any 
other  % Word inserted (13.1.97) by SI 1996/3196 reg 5
person, that document shall, if sent by post to that person’s last known or notified address, and subject to paragraph (8), be treated as having been given or sent on the second day after the day of posting, excluding any Sunday or any day which is a bank holiday in England, Wales, Scotland or Northern Ireland under the Banking and Financial Dealings Act 1971\footnote{\frenchspacing 1971 c. 80.}.
\end{enumerate}

(7) Except where the provisions of regulation 8(6), 24(2), 29(3) or 31(6)($a$) apply, the Secretary of State may treat a document given or sent to him as given or sent on such day, earlier than the day it was received by him, as he may determine, if he is satisfied that there was unavoidable delay in his receiving the document in question.

(8) Where, by any provision of the Act or of these Regulations, and in relation to a particular application, notice or notification—
\begin{enumerate}\item[]
($a$) more than one document is required to be given or sent to a person, and more than one such document is sent by post to that person but not all the documents are posted on the same day; or

($b$) documents are required to be given or sent to more than one person, and not all such documents are posted on the same day,
\end{enumerate}
all those documents shall be treated as having been posted on the later or, as the case may be, the latest day of posting.

(9) In these Regulations, unless the context otherwise requires, a reference—
\begin{enumerate}\item[]
($a$) to a numbered Part is to the Part of these Regulations bearing that number;

($b$) to a numbered Schedule is to the Schedule to these Regulations bearing that number;

($c$) to a numbered regulation is to the regulation in these Regulations bearing that number;

($d$) in a regulation or Schedule to a numbered paragraph is to the paragraph in that regulation or Schedule bearing that number;

($e$) in a paragraph to a lettered or numbered sub-paragraph is to the sub-paragraph in that paragraph bearing that letter or number.
\end{enumerate}

\amendment{
Definition of ``Maintenance Arrangements and Jurisdiction Regulations'' inserted in reg. 1(2) (16.2.95) by the Child Support (Miscellaneous Amendments) Regulations 1995 reg. 4.

Words inserted and omitted and para. ($c$) inserted in the definition of ``suspended'' in reg. 1(3) (22.1.96) by the Child Support (Miscellaneous Amendments) (No. 2) Regulations 1995 reg. 15.

Words inserted in the definitions of ``applicable amount'', ``benefit week'' and ``relevant benefit'' in reg. 1(2), words inserted in reg. 1(3) and definition of ``the Jobseeker's Allowance Regulations'' inserted in reg. 1(2) (7.10.96) by the Social Security and Child Support (Jobseeker's Allowance) (Consequential Amendments) Regulations 1996 reg. 5(2).

Word inserted in reg. 1(6)(b) (13.1.97) by the Child Support (Miscellaneous Amendments) (No. 2) Regulations 1996 reg. 5.
}

\section[Part II --- Applications for a maintenance assessment]{Part II\\*Applications for a maintenance assessment}

\renewcommand\parthead{--- Part II}

\subsection[2. Applications under section 4, 6 or 7 of the Act]{Applications under section 4, 6 or 7 of the Act}

2.—(1) Any person who applies for a maintenance assessment under section 4 or 7 of the Act shall do so on a form (a “maintenance application form”) provided by the Secretary of State.

(2) Maintenance application forms provided by the Secretary of State under section 6 of the Act or under paragraph (1) shall be supplied without charge by such persons as the Secretary of State appoints or authorises for that purpose.

(3) A completed maintenance application form shall be given or sent to the Secretary of State.

(4) Subject to paragraph (5), an application for a maintenance assessment under the Act shall be an effective application if it is made on a maintenance application form and that form has been completed in accordance with the Secretary of State’s instructions.

(5) Where an application is not effective under the provisions of paragraph (4), the Secretary of State may—
\begin{enumerate}\item[]
($a$) give or send the maintenance application form to the person who made the application, together, if he thinks appropriate, with a fresh maintenance application form, and request that the application be re-submitted so as to comply with the provisions of that paragraph; or

($b$) request the person who made the application to provide such additional information or evidence as the Secretary of State specifies,
\end{enumerate}
and if a completed application form or, as the case may be, the additional information or evidence requested is received by the Secretary of State within 14 days of the date of his request, he shall treat the application as made on the date on which the earlier or earliest application would have been treated as made had it been effective under the provisions of paragraph (4).

(6) Subject to paragraph (7), a person who has made an effective application may amend his application by notice in writing to the Secretary of State at any time before a maintenance assessment is made.

(7) No amendment under paragraph (6) shall relate to any change of circumstances arising after the effective date of a maintenance assessment resulting from an effective application.

\subsection[3. Applications on the termination of a maintenance assessment]{Applications on the termination of a maintenance assessment}

3.—(1) Where a maintenance assessment has been in force with respect to a person with care and a qualifying child and that person is replaced by another person with care, an application for a maintenance assessment with respect to that person with care and that qualifying child may for the purposes of regulation 30(2)($b$)(ii) and subject to paragraph (3) be treated as having been received on a date earlier than that on which it was received.

(2) Where a maintenance assessment has been made in response to an application by a child under section 7 of the Act and either—
\begin{enumerate}\item[]
($a$) a child support officer cancels that assessment following a request from that child; or

($b$) that child ceases to be a child for the purposes of the Act,
\end{enumerate}
any application for a maintenance assessment with respect to any other children who were qualifying children with respect to the earlier maintenance assessment may for the purposes of regulation 30(2)($b$)(ii) and subject to paragraph (3) be treated as having been received on a date earlier than that on which it was received.

(3) No application for a maintenance assessment shall be treated as having been received under paragraph (1) or (2) on a date—
\begin{enumerate}\item[]
($a$) more than 8 weeks earlier than the date on which the application was received; or

($b$) on or before the first day of the maintenance period in which the earlier maintenance assessment ceased to have effect.
\end{enumerate}

\subsection[4. Multiple applications]{Multiple applications}

4.—(1) The provisions of Schedule 2 shall apply in cases where there is more than one application for a maintenance assessment.

(2) The provisions of paragraphs 1, 2 and 3 of Schedule 2 relating to the treatment of two or more applications as a single application shall apply where no request is received for the Secretary of State to cease acting in relation to all but one of the applications.

(3) Where, under the provisions of paragraph 1, 2 or 3 of Schedule 2, two or more applications are to be treated as a single application, that application shall be treated as an application for a maintenance assessment to be made with respect to all of the qualifying children mentioned in the applications, and the effective date of that assessment shall be determined by reference to the earlier or earliest application.

\subsection[5. Notice to other persons of an application for a maintenance assessment]{Notice to other persons of an application for a maintenance assessment}

5.—(1) 
%Where %orig
Subject to paragraph (2A), where %amendment SI 1993/913 reg 2(2) eff 5.4.93
an effective application for a maintenance assessment has been made the Secretary of State shall as soon as is reasonably practicable give notice in writing of that application to the relevant persons other than the applicant.

(2) The Secretary of State shall%
, subject to paragraph (2A), %amendment SI 1993/913 reg 2(3) eff 5.4.93
 give or send to any person to whom notice has been given under paragraph (1) a form (a “maintenance enquiry form”) and a written request that the form be completed and returned to him for the purpose of enabling the application for the maintenance assessment to be proceeded with.

%amendment SI 1993/913 reg 2(4) eff 5.4.93
(2A) The provisions of paragraphs (1) and (2) shall not apply where the Secretary of State is satisfied that an application for a maintenance assessment can be dealt with in the absence of a completed and returned maintenance enquiry form. %end amendment SI 1993/913 reg 2(4) eff 5.4.93

(3) Where the person to whom notice is being given under paragraph (1) is an absent parent, that notice shall specify the effective date of the maintenance assessment if one is to be made, and set out in general terms the provisions relating to interim maintenance assessments.

\amendment{
Words substituted in reg. 5(1), inserted in reg. 5(2) and reg. 5(2A) inserted (5.4.93) by the Child Support (Miscellaneous Amendments) Regulations 1993 reg. 2. 
}

\subsection[6. Response to notification of an application for a maintenance assessment]{\sloppy Response to notification of an application for a maintenance assessment}

6.—(1) Any person who has received a maintenance enquiry form given or sent under regulation 5(2) shall complete that form in accordance with the Secretary of State’s instructions and return it to the Secretary of State within 14 days of its having been given or sent.

(2) Subject to paragraph (3), a person who has returned a completed maintenance enquiry form may amend the information he has provided on that form at any time before a maintenance assessment is made by notifying the Secretary of State in writing of the amendments.

(3) No amendment under paragraph (2) shall relate to any change of circumstances arising after the effective date of any maintenance assessment made in response to the application in relation to which the maintenance enquiry form was given or sent.

\subsection[7. Death of a qualifying child]{Death of a qualifying child}

7.—(1) Where the child support officer concerned is informed of the death of a qualifying child with respect to whom an application for a maintenance assessment has been made, he shall—
\begin{enumerate}\item[]
($a$) proceed with the application as if it had not been made with respect to that child if he has not yet made an assessment;

($b$) treat any assessment already made by him as not having been made if the relevant persons have not been notified of it and proceed with the application as if it had not been made with respect to that child.
\end{enumerate}

(2) Where all of the qualifying children with respect to whom an application for a maintenance assessment has been made have died, and either the assessment has not been made or the relevant persons have not been notified of it, the child support officer shall treat the application as not having being made.

\section[Part III --- Interim maintenance assessments]{Part III\\*Interim maintenance assessments}

\renewcommand\parthead{--- Part III}

%\subsection[8. Amount and duration of an interim maintenance assessment]{Amount and duration of an interim maintenance assessment}
%
%8.—(1) Where a child support officer serves notice under section 12(4) of the Act of his intention to make an interim maintenance assessment, he shall not make the interim assessment before the end of a period of 14 days commencing with the date that notice was given or sent.
%
%% Reg 8(1A), (1B) inserted by SI 1993/913 reg 3(2) eff 5.4.93
%(1A) There shall be 
%%two 
%four  % Word substituted (18.4.95) by SI 1995/1045 reg 28(2)
%categories of interim maintenance assessment, Category A interim maintenance assessments and 
%%Category B interim maintenance assessments. 
%Category B interim maintenance assessments, Category C interim maintenance assessments and Category D interim maintenance assessments.  %  Words substituted (18.4.95) by SI 1995/1045 reg 28(2)
%
%(1B) An interim maintenance assessment made by a child support officer shall be---
%\begin{enumerate}\item[]
%($a$) a Category A interim maintenance assessment, where 
%%the information that is required by him as to the income of the absent parent 
%any information, other than information referred to in sub-paragraph ($b$), that is required by him  % Words substituted (18.4.95) by SI 1995/1045 reg 28(3)
%to enable him to make an assessment in accordance with the provisions of Part I of Schedule 1 to the Act has not been provided by that absent parent, and that parent has that information in his possession or can reasonably be expected to acquire it;
%
%%($b$) a Category B interim maintenance assessment, where the information that is required by him as to the income of the partner or other member of the family of the absent parent or parent with care to enable him to make an assessment in accordance with the provisions of Part I of Schedule 1 to the Act has not been provided by that partner or other member of the family, and that partner or other member of the family has that information in his possession or can reasonably be expected to acquire it.
%
%% Reg 8(1B)($b$) substituted (18.4.95) by SI 1995/1045 reg 28(4)
%($b$) a Category B interim maintenance assessment, where the information that is required by him as to the income of the partner or other member of the family of the absent parent or parent with care for the purposes of the calculation of the income of that partner or other member of the family under regulation 9(2), 10, 11(2) or 12(1) of the Maintenance Assessments and Special Cases Regulations—
%\begin{enumerate}\item[]
%(i) has not been provided by that partner or other member of the family, and that partner or other member of the family has that information in his possession or can reasonably be expected to acquire it; or
%
%(ii) has been provided by that partner or other member of the family to the absent parent or parent with care, but the absent parent or parent with care has not provided it to the Secretary of State or the child support officer;
%\end{enumerate}
%
%% Reg 8(1B)($c$), ($d$) inserted (18.4.95) by SI 1995/1045 reg 28(5)
%($c$) a Category C interim maintenance assessment where—
%\begin{enumerate}\item[]
%(i) the absent parent is a self-employed earner as defined in regulation 1(2) of the Maintenance Assessments and Special Cases Regulations; and
%
%(ii) the absent parent is currently unable to provide, but has indicated that he expects within a reasonable time to be able to provide, information to enable a child support officer to determine the earnings of that absent parent in accordance with paragraphs 3 to 5 of Schedule 1 to the Maintenance Assessments and Special Cases Regulations; and
%
%(iii) no maintenance order as defined in section 8(11) of the Act or written maintenance agreement as defined in section 9(1) of the Act is in force with respect to the children in respect of whom the Category C interim maintenance assessment would be made; or
%\end{enumerate}
%
%($d$) a Category D interim maintenance assessment where it appears to a child support officer, on the basis of information available to him as to the income of the absent parent, that the amount of any maintenance assessment made in accordance with Part I of Schedule 1 of the Act applicable to that absent parent may be higher than the amount of a Category A interim maintenance assessment in force in respect of him.
%\end{enumerate}
%
%(2) The amount of child support maintenance fixed by 
%%an
%a Category A % Words substituted by SI 1993/913 reg 3(3)($a$) eff 5.4.93
%interim maintenance assessment shall be 1.5 multiplied by the amount of the maintenance requirement in respect of the qualifying child or qualifying children concerned calculated in accordance with the provisions of paragraph 1 of Schedule 1 to the Act, and paragraphs 2 to 9 of that Schedule shall not apply to 
%Category A %Words inserted by SI 1993/913 reg 3(3)($b$) eff 5.4.93
%interim maintenance assessments.
%
%%Reg 8(2A)--(2C) inserted by SI 1993/913 reg 3(4) eff 5.4.93
%(2A) 
%%The amount 
%Subject to paragraph (2D), the amount  % Words substituted (18.4.95) by SI 1995/1045 reg 28(6)
%of child support maintenance fixed by a Category B interim maintenance assessment shall be determined in accordance with paragraphs (2B) and (2C).
%
%\begin{sloppypar}
%(2B) Where a child support officer is unable to determine the exempt income---
%\end{sloppypar}
%\begin{enumerate}\item[]
%($a$) of an absent parent under regulation 9 of the Maintenance Assessments and Special Cases Regulations because he is unable to determine whether regulation 9(2) of those Regulations applies;
%
%($b$) of a parent with care under regulation 10 of those Regulations because he is unable to determine whether regulation 9(2) of those Regulations, as modified by and applied by regulation 10 of those Regulations applies,
%\end{enumerate}
%the amount of the Category B interim maintenance assessment shall be the maintenance assessment calculated in accordance with Part I of Schedule 1 to the Act on the assumption that---
%\begin{enumerate}\item[]
%(i) in a case falling within sub-paragraph ($a$), regulation 9(2) of those Regulations does apply;
%
%(ii) in a case falling within sub-paragraph ($b$), regulation 9(2) of those Regulations as modified by and applied by regulation 10 of those Regulations does apply.
%\end{enumerate}
%
%(2C) Where the disposable income of an absent parent 
%calculated in accordance with regulation 12(1)($a$) of the Maintenance Assessments and Special Cases Regulations  % Words inserted (18.4.95) by SI 1995/1045 reg 28(7)
%would, without taking account of the income of any member of his family, bring him within the provisions of paragraph 6 of Schedule 1 to the Act (protected income), and a child support officer is unable to ascertain the disposable income of the other members of his family, the amount of the Category B interim maintenance assessment shall be the maintenance assessment calculated in accordance with Part I of Schedule 1 to the Act on the assumption that the provisions of paragraph 6 of Schedule 1 to the Act do not apply to the absent parent.
%
%% Reg 8(2D)--(2I) inserted (18.4.95) by SI 1995/1045 reg 28(8)
%(2D) Where the application of the provisions of paragraph (2B) or (2C) would result in the amount of a Category B interim maintenance assessment being more than 30 per centum of the net income of the absent parent as calculated in accordance with regulation 7 of the Maintenance Assessments and Special Cases Regulations, those provisions shall not apply to that absent parent and instead, the amount of that Category B interim maintenance assessment shall be 30 per centum of his net income as so calculated and where that calculation results in a fraction of a penny, that fraction shall be disregarded.
%
%(2E) The amount of child support maintenance fixed by a category C interim maintenance assessment shall be £30.00 but a child support officer may set a lower amount, including a nil amount, if he thinks it reasonable to do so in all the circumstances of the case.
%
%(2F) Paragraph 6 of Schedule 1 to the Act shall not apply to Category C interim maintenance assessments.
%
%(2G) A child support officer shall notify the person with care where he is considering setting a lower amount for a Category C interim maintenance assessment in accordance with paragraph (2E), and shall take into account any relevant representations made by that person with care in deciding the amount of that Category C interim maintenance assessment.
%
%(2H) The amount of child support maintenance fixed by a Category D interim maintenance assessment shall be calculated or estimated by applying to the absent parent’s income, in so far as the child support officer is able to determine it at the time of the making of that Category D interim maintenance assessment, the provisions of Part I of Schedule 1 to the Act and regulations made under it, subject to the modification that—
%\begin{enumerate}\item[]
%($a$) paragraphs 6 and 8 of that Schedule shall not apply; and
%
%($b$) only paragraphs (1)($a$) and (5) of regulation 9 of the Maintenance Assessments and Special Cases Regulations shall apply; and
%
%($c$) heads ($b$) and ($c$) of sub-paragraph (3) of paragraph 1 of Schedule 1 to the Maintenance Assessments and Special Cases Regulations shall not apply.
%\end{enumerate}
%
%(2I) Where the absent parent referred to in paragraph (2H) is an employed earner as defined in regulation 1 of the Maintenance Assessments and Special Cases Regulations and the child support officer is unable to calculate the net income of that absent parent, his net income shall be estimated under the provisions of paragraph (2A)($a$) and ($b$) of regulation 1 of the Maintenance Assessments and Special Cases Regulations.
%
%%(3) 
%%%Where the provisions of regulation 30(2)($a$) or (4) apply, 
%%Except where regulation 3(5) of the Maintenance Arrangements and Jurisdiction Regulations (effective date of maintenance assessment where court order in force) or paragraph (3A), (3B), (3C) or (3D) applies, %Words substituted (16.2.95) by SI 1995/123 reg 5(2)
%%the effective date of an interim maintenance assessment shall be such date, being not earlier than the first and not later than the seventh day following the expiry of the period of 14 days specified in paragraph (1), as falls on the same day of the week as the date specified in regulation 30(2)($a$).
%
%% Reg 8(3) substituted (18.4.95) by SI 1995/1045 reg 28(9)
%(3) Except where regulation 3(5) of the maintenance Arrangements and Jurisdiction Regulations (effective date of maintenance assessment where court order in force) or paragraph (3A), (3B), (3C), (3D), (7) (7A) or (7E) applies, the effective date of an interim maintenance assessment shall be—
%\begin{enumerate}\item[]
%($a$) in respect of a Category A, Category C or Category D interim maintenance assessment, subject to sub-paragraph ($c$), such date, being not earlier than the first and not later than the seventh day following the date upon which that interim maintenance assessment was made, as falls on the same day of the week as the date determined in accordance with regulation 30(2)($a$)(ii);
%
%($b$) in respect of a Category B interim maintenance assessment—
%\begin{enumerate}\item[]
%(i) subject to head (ii) and sub-paragraph ($c$), such date, being not earlier than the first and not later than the seventh day following the expiry of the period of 14 days specified in paragraph (1), as falls on the same day of the week as the date determined in accordance with regulation 30(2)($a$)(ii);
%
%(ii) where that Category B interim maintenance assessment is made after a Category A, Category C or Category D interim maintenance assessment has been in force, the date upon which that Category A, Category C or Category D interim maintenance assessment ceased to have effect in accordance with paragraph (9A);
%\end{enumerate}
%
%($c$) in respect of a Category A, Category B, Category C or Category D interim maintenance assessment, where the application of the provisions of sub-paragraph ($a$) or ($b$)(i) would otherwise set an effective date for an interim maintenance assessment earlier than the end of a period of eight weeks from the date upon which—
%\begin{enumerate}\item[]
%(i) the maintenance enquiry form referred to in regulation 30(2)($a$)(i) was given or sent to an absent parent; or
%
%(ii) the application made by in absent parent referred to in regulation 30(2)($b$)(i) was received by the Secretary of State,
%\end{enumerate}
%in circumstances where that absent parent has complied with the provisions of regulation 30(2)($a$)(i) or ($b$)(i) or paragraph (2A) of that regulation applies, the date determined in accordance with regulation 30(2)($a$)(i) or ($b$)(i).”.
%\end{enumerate}
%
%%Reg 8(3A)--(3D) inserted (16.2.95) by SI 1995/123 reg 5(3)
%(3A) Subject to paragraph (3D), where a child support officer makes a 
%%Category A 
%Category A or Category D  % Words substituted (18.4.95) by SI 1995/1045 reg 28(10)
%interim maintenance assessment following a review of a 
%%Category A 
%Category A or Category D  % Words substituted (18.4.95) by SI 1995/1045 reg 28(10)
%interim maintenance assessment under section 16 of the Act, the effective date of that assessment shall be 
%%52 weeks 
%104 weeks  % Words substituted (18.4.95) by SI 1995/1045 reg 28(10)
%after the effective date of the previous interim maintenance assessment, disregarding any previous interim maintenance assessment made following a review under section 19 of the Act.
%
%(3B) Subject to paragraph (3D), where a child support officer reviews a 
%%Category A 
%Category A or Category D  % Words substituted (18.4.95) by SI 1995/1045 reg 28(10)
%interim maintenance assessment under section 19(1) of the Act on the grounds that it is defective because of a mistake as to its effective date or for reasons which include a mistake as to its effective date, the effective date of a 
%%Category A 
%Category A or Category D  % Words substituted (18.4.95) by SI 1995/1045 reg 28(10)
%interim maintenance assessment made following such a review shall be the correct effective date applicable to the interim maintenance assessment being reviewed, as determined in accordance with paragraph (3), (3A) or regulation 3(5) of the Maintenance Arrangements and Jurisdiction Regulations, as the case may be.
%
%(3C) Subject to paragraph (3D), where a child support officer reviews a 
%%Category A 
%Category A or Category D  % Words substituted (18.4.95) by SI 1995/1045 reg 28(10)
%interim maintenance assessment under section 19(1) of the Act on the grounds that it is defective for reasons which do not include a mistake as to its effective date, the effective date of a 
%%Category A 
%Category A or Category D  % Words substituted (18.4.95) by SI 1995/1045 reg 28(10)
%interim maintenance assessment made following such a review shall be the same as the effective date of the interim maintenance assessment that has been reviewed.
%
%(3D) Where the effective date of a Category A interim maintenance assessment made following a review under section 16 or 19(1) of the Act would by virtue of the provisions of paragraphs (3A) to (3C) be earlier than 16th February 1995, the effective date of that assessment shall be 16th February 1995.
%
%(4) 
%%Where a maintenance assessment is made 
%In cases where the effective date of an interim maintenance assessment is determined under paragraph (3), (3A), (3B), (3C)% 
%  %or (3D), 
%  , (3D), (7), (7A) or (7E)  % Words substituted (18.4.95) by SI 1995/1045 reg 28(11)
%where a maintenance assessment is made % Words substituted (16.2.95) by SI 1995/123 reg 5(4)
%after an interim maintenance assessment has been in force, child support maintenance calculated in accordance with Part I of Schedule 1 to the Act shall be payable in respect of the period preceding that during which the interim maintenance assessment was in force.
%
%(5) The child support maintenance payable under the provisions of paragraph (4) shall be payable in respect of the period between the effective date of the assessment (or, where separate assessments are made for different periods under paragraph 15 of Schedule 1 to the Act, the effective date of the assessment in respect of the earliest such period) and the effective date of the interim maintenance assessment.
%
%(6) Where a child support officer is satisfied that there was unavoidable delay by the absent parent in completing and returning a maintenance enquiry form under the provisions of regulation 6(1), or in providing information or evidence that is required by the Secretary of State for the determination of an application for a maintenance assessment, he may cancel 
%%an interim maintenance assessment 
%a Category A, Category B or Category D interim maintenance assessment  % Words substituted (18.4.95) by SI 1995/1045 reg 28(12)
%which is in force.
%
%%(7) An interim maintenance assessment shall not be cancelled under paragraph (6) with effect from a date earlier than that on which the provisions of regulation 6(1) could have been complied with.
%
%%Reg 8(7)--(7F) substituted for reg 8(7) (18.4.95) by SI 1995/1045 reg 28(13)
%(7) Where a child support officer cancels a Category A, Category B or Category D interim maintenance assessment in accordance with the provisions of paragraph (6), and he is satisfied that there was unavoidable delay for only part of the period during which that assessment was in force, and that another Category A, Category B or Category D interim maintenance assessment should be made, the effective date of that other Category A, Category B or Category D interim maintenance assessment shall, subject to paragraph (7A), be the first day of the maintenance period following the date upon which, in the opinion of the child support officer, the delay became avoidable.
%
%(7A) Where the Category A or Category B interim maintenance assessment cancelled in accordance with the provisions of paragraph (6) was made prior to 18 April 1995 and the effective date of any new Category A or Category B interim maintenance assessment would, by virtue of paragraph (7), be prior to 18 April 1995, the effective date of that new Category A or Category B interim maintenance assessment shall be the first day of the maintenance period which begins on or after 18 April 1995.
%
%(7B) Where in respect of any Category A or Category B interim maintenance assessment in force before 18 April 1995 the delay referred to in paragraph (6) became avoidable before 18 April 1995, that Category A or Category B interim maintenance assessment may not be cancelled with effect from a date earlier than the date the delay became avoidable.
%
%(7C) Subject to paragraph (6), where a child support officer is satisfied that it would be appropriate to make an interim maintenance assessment the Category of which is different from that of the interim maintenance assessment in force, he may cancel the interim maintenance assessment which is in force with effect from whichever is the later of the first day of the maintenance period in which he becomes so satisfied or the first day of the maintenance period which begins on or after 18 April 1995.
%
%(7D) In paragraph (7C), “Category” in relation to an interim maintenance assessment means Category A, Category B, Category C or Category D, as the case may be.
%
%(7E) Where a child support officer makes an interim maintenance assessment following the cancellation of an interim maintenance assessment in accordance with paragraph (7C), the effective date for the fresh interim maintenance assessment shall be the date upon which that cancellation took effect.
%
%(7F) A child support officer may cancel an interim maintenance assessment which is in force with effect from such date as he considers appropriate in all the circumstances on the grounds that—
%\begin{enumerate}\item[]
%($a$) there was a material procedural error in connection with the making of the assessment; or
%
%($b$) he is satisfied that he did not, or has subsequently ceased to have jurisdiction to make that interim maintenance assessment.
%\end{enumerate}
%
%%(8) Subject to paragraphs (6), (7) and (10), the child support maintenance payable in respect of any period in respect of which an interim maintenance assessment is in force shall not be adjusted following the making of a maintenance assessment.
%
%%Reg 8(8) substituted (18.4.95) by SI 1995/1045 reg 28(14)
%(8) Where a maintenance assessment calculated in accordance with Part I of Schedule 1 to the Act is made following an interim maintenance assessment, the amount of child support maintenance payable in respect of the period after 18 April 1995 during which that interim maintenance assessment was in force shall be that fixed by the maintenance assessment.
%
%(9) 
%%An interim maintenance assessment 
%Subject to paragraph (9A), an interim maintenance assessment  % Words substituted (18.4.95) by SI 1995/1045 reg 28(15)
%shall cease to have effect on the first day of the maintenance period during which the Secretary of State receives the information which enables a child support officer to make the maintenance assessment or assessments in relation to the same absent parent, person with care, and qualifying child or qualifying children, calculated in accordance with Part I of Schedule 1 to the Act.
%
%% Reg 8(9A) inserted (18.4.95) by SI 1995/1045 reg 28(16)
%(9A) A Category A, Category C or Category D interim maintenance assessment shall cease to have effect on the first day of the maintenance period in which the Secretary of State has received all the information that is required by him to enable a child support officer to make an assessment in accordance with the provisions of Part I of Schedule 1 to the Act with the exception of the information referred to in paragraph (1B)($b$) or the first day of the maintenance period after 18 April 1995 whichever is the later.
%
%%(10) Where a maintenance assessment calculated in accordance with Part I of Schedule 1 to the Act is made following an interim maintenance assessment and the amount of child support maintenance payable under that assessment in respect of the period during which the interim maintenance assessment was in force is higher than the amount fixed by the interim maintenance assessment determined in accordance with paragraph (2), the amount of child support maintenance payable in respect of that period shall be that fixed by the maintenance assessment calculated in accordance with Part I of Schedule 1 to the Act.
%
%% Reg 8(10) omitted (18.4.95) by SI 1995/1045 reg 28(17)
%
%(11) Subject to regulation 9(6), 
%%for the purposes of sections 17 and 18 of the Act 
%for the purposes of sections 17, 18 and 19(2) of the Act  % Words substituted (18.4.95) by SI 1995/1045 reg 28(18)
%a maintenance assessment shall not include 
%%an 
%a Category A %Words substituted by SI 1993/913 reg 3(5) eff 5.4.93
%or Category D % Words inserted (18.4.95) by SI 1995/1045 reg 28(18)
%interim maintenance assessment.
%
%(12) The provisions of regulations 29, 31, 32, 33(5) and 55 shall not apply to 
%Category A %Words inserted by SI 1993/913 reg 3(6) eff 5.4.93
%or Category D  % Words inserted (18.4.95) by SI 1995/1045 reg 28(19)
%interim maintenance assessments.
%
%%Reg 8(13) inserted by SI 1993/913 reg 3(7) eff 5.4.93
%(13) In this regulation “family” and “partner” have the same meanings as in the Maintenance Assessments and Special Cases Regulations.
%
%\amendment{
%Words inserted in reg. 8(2), (12), words substituted in reg. 8(2), (11) and reg. 8(1A), (1B), (2A)--(2C), (13) inserted (5.4.93) by the Child Support (Miscellaneous Amendments) Regulations 1993 reg. 3.
%
%Words substituted in reg. 8(3), (4) and reg. 8(3A)--(3D) inserted (16.2.95) by the Child Support (Miscellaneous Amendments) Regulations 1995 reg. 5.
%
%Words substituted in reg. 8(1A), (1B)($a$), (2A), (3A)--(3C), (4), (6), (9), (11), words inserted in reg. 8(2C), (11), (12), reg. 8(1B)($b$), (3), (8) substituted, reg. 8(7)--(7F) substituted for reg. 8(7), reg. 8(1B)($c$), ($d$), (2D)--(2I), (9A) inserted and reg. 8(10) omitted (18.4.95) by the Child Support and Income Support (Amendment) Regulations 1995 reg. 28.
%}

% Regs 8--8D substituted for reg 8 (22.1.96) by SI 1995/3261 reg 16
\subsection[8. Categories of interim maintenance assessment]{Categories of interim maintenance assessment}

8.—(1) Where a child support officer serves notice under section 12(4) of the Act of his intention to make an interim maintenance assessment, he shall not make that interim assessment before the end of a period of 14 days, commencing with the date that notice was given or sent.

(2) There shall be four categories of interim maintenance assessment, Category A, Category B, Category C, and Category D interim maintenance assessments.

(3) An interim maintenance assessment made by a child support officer shall be—
\begin{enumerate}\item[]
($a$) a Category A interim maintenance assessment, where any information, other than information referred to in sub-paragraph ($b$), that is required by him to enable him to make an assessment in accordance with the provisions of Part I of Schedule 1 to the Act has not been provided by that absent parent, and that parent has that information in his possession or can reasonably be expected to acquire it;

($b$) a Category B interim maintenance assessment, where the information that is required by him as to the income of the partner or other member of the family of the absent parent or parent with care for the purposes of the calculation of the income of that partner or other member of the family under regulation 9(2), 10, 11(2) or 12(1) of the Maintenance Assessments and Special Cases Regulations—
\begin{enumerate}\item[]
(i) has not been provided by that partner or other member of the family, and that partner or other member of the family has that information in his possession or can reasonably be expected to acquire it; or

(ii) has been provided by that partner or other member of the family to the absent parent or parent with care, but the absent parent or parent with care has not provided it to the Secretary of State or the child support officer;
\end{enumerate}

($c$) a Category C interim maintenance assessment where—
\begin{enumerate}\item[]
(i) the absent parent is a self-employed earner as defined in regulation 1(2) of the Maintenance Assessments and Special Cases Regulations; and

(ii) the absent parent is currently unable to provide, but has indicated that he expects within a reasonable time to be able to provide, information to enable a child support officer to determine the earnings of that absent parent in accordance with paragraphs 3 to 5 of Schedule 1 to the Maintenance Assessments and Special Cases Regulations; and

(iii) no maintenance order as defined in section 8(11) of the Act or written maintenance agreement as defined in section 9(1) of the Act is in force with respect to children in respect of whom the Category C interim maintenance assessment would be made; or
\end{enumerate}

($d$) a Category D interim maintenance assessment where it appears to a child support officer, on the basis of information available to him as to the income of the absent parent, that the amount of any maintenance assessment made in accordance with Part I of Schedule 1 to the Act applicable to that absent parent may be higher than the amount of a Category A interim maintenance assessment in force in respect of him.
\end{enumerate}

(4) In this regulation and in regulation 8A, “family” and “partner” have the same meanings as in the Maintenance Assessments and Special Cases Regulations.

\amendment{
Reg. 8 substituted (22.1.96) by the Child Support (Miscellaneous Amendments) (No. 2) Regulations 1995 reg. 16.
}

\subsection[8A. Amount of an interim maintenance assessment]{Amount of an interim maintenance assessment}

8A.—(1) The amount of child support maintenance fixed by a Category A interim maintenance assessment shall be 1.5 multiplied by the amount of the maintenance requirement in respect of the qualifying child or qualifying children concerned calculated in accordance with the provisions of paragraph 1 of Schedule 1 to the Act, and paragraphs 2 to 9 of that Schedule shall not apply to Category A interim maintenance assessments.

(2) Subject to paragraph (5), the amount of child support maintenance fixed by a Category B interim maintenance assessment shall be determined in accordance with paragraphs (3) and (4).

(3) Where a child support officer is unable to determine the exempt income—
\begin{enumerate}\item[]
($a$) of an absent parent under regulation 9 of the Maintenance Assessments and Special Cases Regulations because he is unable to determine whether regulation 9(2) of those Regulations applies;

($b$) of a parent with care under regulation 10 of those Regulations because he is unable to determine whether regulation 9(2) of those Regulations, as modified by and applied by regulation 10 of those Regulations applies,
\end{enumerate}
the amount of the Category B interim maintenance assessment shall be the maintenance assessment calculated in accordance with Part I of Schedule 1 to the Act on the assumption that—
\begin{enumerate}\item[]
(i) in a case falling within sub-paragraph ($a$), regulation 9(2) of those Regulations does apply;

(ii) in a case falling within sub-paragraph ($b$), regulation 9(2) of those Regulations as modified by and applied by regulation 10 of those Regulations does apply.
\end{enumerate}

%(4) Where the disposable income of an absent parent, calculated in accordance with regulation 12(1)($a$) of the Maintenance Assessments and Special Cases Regulations, would, without taking account of the income of any member of his family, bring him within the provisions of paragraph 6 of Schedule 1 to the Act (protected income), and a child support officer is unable to ascertain the disposable income of the other members of his family, the amount of the Category B interim maintenance assessment shall be the maintenance assessment calculated in accordance with Part I of Schedule 1 to the Act on the assumption that the provisions of paragraph 6 of Schedule 1 to the Act do not apply to the absent parent.

% Reg 8A(4) substituted (19.1.98) by SI 1998/58 reg 35
(4) Where a child support officer is unable to ascertain the income of other members of the family of an absent parent so that the disposable income of that absent parent can be calculated in accordance with regulation 12(1)($a$) of the Maintenance Assessments and Special Cases Regulations, the amount of the Category B interim maintenance assessment shall be the maintenance assessment calculated in accordance with Part I of Schedule 1 to the Act on the assumption that the provisions of paragraph 6 of that Schedule do not apply to the absent parent.

(5) Where the application of the provisions of paragraph (3) or (4) would result in the amount of a Category B interim maintenance assessment being more than 30 per centum of the net income of the absent parent as calculated in accordance with regulation 7 of the Maintenance Assessments and Special Cases Regulations, those provisions shall not apply to that absent parent and instead, the amount of that Category B interim maintenance assessment shall be 30 per centum of his net income as so calculated and where that calculation results in a fraction of a penny, that fraction shall be disregarded.

(6) The amount of child support maintenance fixed by a Category C interim maintenance assessment shall be £30.00 but a child support officer may set a lower amount, including a nil amount, if he thinks it reasonable to do so in all the circumstances of the case.

(7) Paragraph 6 of Schedule 1 to the Act shall not apply to Category C interim maintenance assessments.

(8) A child support officer shall notify the person with care where he is considering setting a lower amount for a Category C interim maintenance assessment in accordance with paragraph (6) and shall take into account any relevant representations made by that person with care in deciding the amount of that Category C interim maintenance assessment.

(9) The amount of child support maintenance fixed by a Category D interim maintenance assessment shall be calculated or estimated by applying to the absent parent’s income, in so far as the child support officer is able to determine it at the time of the making of that Category D interim maintenance assessment, the provisions of Part I of Schedule 1 to the Act and regulations made under it, subject to the modification that—
\begin{enumerate}\item[]
($a$) paragraphs 6 and 8 of that Schedule shall not apply;

($b$) only paragraphs (1)($a$) and (5) of regulation 9 of the Maintenance Assessments and Special Cases Regulations shall apply; and

($c$) heads ($b$) and ($c$) of sub-paragraph (3) of paragraph 1 of Schedule 1 to the Maintenance Assessments and Special Cases Regulations shall not apply.
\end{enumerate}

(10) Where the absent parent referred to in paragraph (9) is an employed earner as defined in regulation 1 of the Maintenance Assessments and Special Cases Regulations and the child support officer is unable to calculate the net income of that absent parent, his net income shall be estimated under the provisions of paragraph (2A)($a$) and ($b$) of that regulation.

\amendment{
Reg. 8A substituted for reg. 8 (22.1.96) by the Child Support (Miscellaneous Amendments) (No. 2) Regulations 1995 reg. 16.

Reg. 8A(4) substituted (19.1.98) by the Child Support (Miscellaneous Amendments) Regulations 1998 reg. 35.
}

\subsection[8B. Review of an interim maintenance assessment]{Review of an interim maintenance assessment}

8B.—(1) Subject to paragraph (4), where a child support officer—
\begin{enumerate}\item[]
($a$) makes a Category A interim maintenance assessment following a review of a Category A interim maintenance assessment under section 16 of the Act; or

($b$) makes a Category D interim maintenance assessment following a review of a Category D maintenance assessment under section 16 of the Act,
\end{enumerate}
the effective date of that assessment shall be 104 weeks after the effective date of the previous interim maintenance assessment, disregarding any previous interim maintenance assessment made following a review under section 19 of the Act.

(2) Subject to paragraph (4), where a child support officer reviews a Category A or Category D interim maintenance assessment under section 19(1)($c$)\footnote{\frenchspacing Section 19 was substituted by section 15 of the Child Support Act 1995.} of the Act on the grounds that it is defective because of a mistake as to its effective date or for reasons which include a mistake as to its effective date, the effective date of a Category A or Category D interim maintenance assessment made following such a review shall be the correct effective date applicable to the interim maintenance assessment being reviewed, as determined in accordance with paragraph (1), regulation 8C(1) or regulation 3(5) of the Maintenance Arrangements and Jurisdiction Regulations, as the case may be.

(3) Subject to paragraph (4), where a child support officer reviews a Category A or Category D interim maintenance assessment under section 19(1)($c$) of the Act on the grounds that it is defective for reasons which do not include a mistake as to its effective date, the effective date of a Category A or Category D interim maintenance assessment made following such a review shall be the same as the effective date of the interim maintenance assessment that has been reviewed.

(4) Where the effective date of a Category A interim maintenance assessment made following a review under section 16 or 19(1)($c$) of the Act would by virtue of the provisions of paragraphs (1) to (3) be earlier than 16th February 1995, the effective date of that assessment shall be 16th February 1995.

\amendment{
Reg. 8B substituted for reg. 8 (22.1.96) by the Child Support (Miscellaneous Amendments) (No. 2) Regulations 1995 reg. 16.
}

\subsection[8C. Effective date of an interim maintenance assessment]{Effective date of an interim maintenance assessment}

8C.—(1) Except where regulation 3(5) of the Maintenance Arrangements and Jurisdiction Regulations (effective date of maintenance assessment where court order in force), regulation 9(9) or 33(7) or paragraph (2) applies, the effective date of an interim maintenance assessment shall be—
\begin{enumerate}\item[]
($a$) in respect of a Category A interim maintenance assessment, subject to regulations 8B, 9(2) and (3) and sub-paragraph ($d$), such date, being not earlier than the first and not later than the seventh day following the date upon which that interim maintenance assessment was made, as falls on the same day of the week as the date determined in accordance with regulation 30(2)($a$)(ii) or ($b$)(ii) as the case may be;

($b$) in respect of a Category B interim maintenance assessment made after 22nd January 1996, subject to sub-paragraph ($d$) and to regulations 31 to 31C, the date specified in regulation 30(2)($a$)(ii) or ($b$)(ii) as the case may be;

($c$) in respect of a Category C interim maintenance assessment, subject to sub-paragraph ($d$) and regulations 31 to 31C, the date set out in sub-paragraph ($a$);

($d$) in respect of a Category A, Category B or Category C interim maintenance assessment, where the application of the provisions of sub-\hspace{0pt}paragraph ($a$), ($b$) or ($c$) would otherwise set an effective date for an interim maintenance assessment earlier than the end of a period of eight weeks from the date upon which—
\begin{enumerate}\item[]
\begin{sloppypar}
(i) the maintenance enquiry form referred to in regulation 30(2)($a$)(i) was given or sent to an absent parent; or
\end{sloppypar}

(ii) the application made by an absent parent referred to in regulation 30(2)($b$)(i) was received by the Secretary of State,
\end{enumerate}
in circumstances where that absent parent has complied with the provisions of regulation 30(2)($a$)(i) or ($b$)(i) or paragraph (2A) of that regulation applies, the date determined in accordance with regulation 30(2)($a$)(i) or ($b$)(i).
\end{enumerate}

(2) The effective date of an interim maintenance assessment made under section 12(1)($b$) or ($c$) of the Act\footnote{\frenchspacing Section 12(1)($b$) and ($c$) were inserted by section 11 of the Child Support Act 1995.} shall, subject to regulations 8B, 9(2), (3) and (9), or 33(7), and, as regards Category B and Category C interim maintenance assessments to regulations 31 to 31C, be such date, not earlier than the first and not later than the seventh day following the date upon which that interim maintenance assessment was made, as falls on the same day of the week as the effective date of the maintenance assessment calculated in accordance with Part I of Schedule 1 to the Act which is being reviewed.

(3) In cases where the effective date of an interim maintenance assessment is determined under paragraph (1), regulation 8B or 9(2), (3) or (9), where a maintenance assessment, except a maintenance assessment falling within regulation 8D(7), is made after an interim maintenance assessment has been in force, child support maintenance calculated in accordance with Part I of Schedule 1 to the Act shall be payable in respect of the period preceding that during which the interim maintenance assessment was in force.

(4) The child support maintenance payable under the provisions of paragraph (3) shall be payable in respect of the period between the effective date of the assessment (or, where separate assessments are made for different periods under paragraph 15 of Schedule 1 to the Act, the effective date of the assessment in respect of the earliest such period) and the effective date of the interim maintenance assessment.

\amendment{
Reg. 8C substituted for reg. 8 (22.1.96) by the Child Support (Miscellaneous Amendments) (No. 2) Regulations 1995 reg. 16.
}

\subsection[8D. Miscellaneous provisions in relation to interim maintenance assessments]{\sloppy Miscellaneous provisions in relation to interim maintenance assessments}

8D.—(1) Subject to paragraph (2), where a maintenance assessment calculated in accordance with Part I of Schedule 1 to the Act is made following an interim maintenance assessment, the amount of child support maintenance assessment, the amount of child support maintenance payable in respect of the period after 18th April 1995, during which that interim maintenance assessment was in force shall be that fixed by the maintenance assessment.

% Reg 8D(1A) inserted (19.1.98) by SI 1998/58 reg 36
(1A) The reference in paragraph (1) to a maintenance assessment calculated in accordance with Part I of Schedule 1 to the Act shall include a maintenance assessment falling within regulation 30A(2).

(2) Paragraph (1) shall not apply where a maintenance assessment calculated in accordance with Part I of Schedule 1 to the Act falls within paragraph (7).

(3) Subject to regulations 9(13) and 9A(6), for the purposes of sections 17, 18 and 19(1)($a$), ($b$) and ($e$) and (6), of the Act, a maintenance assessment shall not include a Category A or Category D interim maintenance assessment.

(4) The provisions of regulations 29, 31 to 31C, 32, 33(5) and 55 shall not apply to a Category A or Category D interim maintenance assessment.

(5) Subject to paragraph (6) and regulation 9(15), an interim maintenance assessment shall cease to have effect on the first day of the maintenance period during which the Secretary of State receives the information which enables a child support officer to make the maintenance assessment or assessments in relation to the same absent parent, person with care, and qualifying child or qualifying children, calculated in accordance with Part I of Schedule 1 to the Act.

(6) Subject to regulation 9(15), where a child support officer has insufficient information or evidence to enable him to make a maintenance assessment calculated in accordance with Part I of Schedule 1 to the Act for the whole of the period beginning with the effective date applicable to a particular case, an interim maintenance assessment made in that case shall cease to have effect—
\begin{enumerate}\item[]
($a$) on 18th April 1995 where by that date the Secretary of State has received the information or evidence set out in paragraph (7); or

($b$) on the first day of the maintenance period after 18th April 1995 in which the Secretary of State has received that information or evidence.
\end{enumerate}

(7) The information or evidence referred to in paragraph (6) is information or evidence enabling a child support officer to make a maintenance assessment calculated in accordance with Part I of Schedule 1 to the Act, for a period beginning after the effective date applicable to that case, in respect of the absent parent, parent with care and qualifying child or qualifying children in respect of whom the interim maintenance assessment referred to in paragraph (6) was made.

%(8) For the purposes of paragraph (6), the Secretary of State shall be treated as having received the information or evidence which has caused the interim maintenance assessment to cease to have effect on the first day upon which the absent parent in question became entitled to income support
%or income-based jobseeker’s allowance.  % Words added (7.10.96) by SI 1996/1345 reg 5(3)

% Reg 8D(8) substituted (13.1.97) by SI 1996/3196 reg 6
(8) Where the information or evidence referred to in paragraph (6)($a$) or ($b$) is that there has been an award of income support or an income-based jobseeker’s allowance, the Secretary of State shall be treated as having received that information or evidence on the first day in respect of which income support or an income-based jobseeker’s allowance was payable under that award.

\amendment{
Reg. 8D substituted for reg. 8 (22.1.96) by the Child Support (Miscellaneous Amendments) (No. 2) Regulations 1995 reg. 16.

%Words added in reg. 8D(8) (7.10.96) by the Social Security and Child Support (Jobseeker's Allowance) (Consequential Amendments) Regulations 1996 reg. 5(3).

Reg. 8D(8) substituted (13.1.97) by the Child Support (Miscellaneous Amendments) (No. 2) Regulations 1996 reg. 6.

Reg. 8D(1A) inserted (19.1.98) by the Child Support (Miscellaneous Amendments) Regulations 1998 reg. 36.
}

%\subsection[9. Cancellation of an interim maintenance assessment]{Cancellation of an interim maintenance assessment}
%
%9.—(1) An absent parent with respect to whom 
%%an interim maintenance assessment 
%a Category A 
%  or Category D  % Words inserted (18.4.95) by SI 1995/1045 reg 29
%interim maintenance assessment % Words substituted by SI 1993/913 reg 4(2) eff 5.4.93
%is in force may apply to a child support officer for that interim assessment to be cancelled.
%
%(2) Any application made under paragraph (1) shall be in writing, and shall include a statement of the grounds for the application.
%
%(3) A child support officer who receives an application under the provisions of paragraph (1), shall—
%\begin{enumerate}\item[]
%($a$) decide whether the interim maintenance assessment is to be cancelled and, if so, the date with effect from which it is to be cancelled;
%
%($b$) in any case where he does cancel an interim maintenance assessment, decide whether it is appropriate for a maintenance assessment to be made in accordance with the provisions of Part I of Schedule 1 to the Act;
%
%($c$) in any case where he has decided that it is appropriate for a maintenance assessment to be made in accordance with the provisions of Part I of Schedule 1 to the Act, make that assessment.
%\end{enumerate}
%
%(4) Where a child support officer has made a decision under paragraph (3), he shall immediately notify the applicant, so far as that is reasonably practicable, and shall give the reasons for his decision in writing.
%
%(5) A notification under paragraph (4) shall include information as to the provisions of sections 18 and 20 of the Act and regulation 24(1) and, where an assessment is made in accordance with the provisions of Part I of Schedule 1 to the Act, the provisions of sections 16 and 17 of the Act.
%
%(6) Where a child support officer has made a decision following an application under paragraph (1), the absent parent may apply to the Secretary of State for a review of that decision and, subject to the modification set out in paragraph (7), the provisions of section 18(5) to (8) of the Act shall apply to such a review.
%
%(7) The modification referred to in paragraph (6) is that section 18(6) of the Act shall have effect as if for “the refusal, assessment or cancellation in question” there is substituted “the decision following an application under regulation 9(1) of the Child Support (Maintenance Assessment Procedure) Regulations 1992”.
%
%(8) Regulations 10, 11%
%, 24 % Words inserted by SI 1993/913 reg 4(3) eff 5.4.93
% and 25 shall apply to reviews under paragraph (6).
%
%\amendment{
%Words substituted in reg. 9(1) and inserted in reg. 9(8) (5.4.93) by the Child Support (Miscellaneous Amendments) Regulations 1993 reg. 4.
%
%Words inserted in reg. 9(1) (18.4.95) by the Child Support and Income Support (Amendment) Regulations 1995 reg 29.
%}

% Regs 9, 9A subsituted for reg 9 (22.1.96) by SI 1995/3261 reg 17
\subsection[9. Cancellation of an interim maintenance assessment]{Cancellation of an interim maintenance assessment}

9.—(1) Where a child support officer is satisfied that there was unavoidable delay by the absent parent in—
\begin{enumerate}\item[]
(i) completing and returning a maintenance enquiry form under the provisions of regulation 6(1);

(ii) providing information or evidence that is required by the Secretary of State for the determination of an application for a maintenance assessment; or

(iii) providing information or evidence that is required by a child support officer to enable him to conduct or complete a review under section 16, 17, 18 or 19 of the Act,
\end{enumerate}
he may cancel an interim maintenance assessment which is in force.

(2) Where a child support officer cancels a Category A, Category B or Category D interim maintenance assessment in accordance with the provisions of paragraph (1), and he is satisfied that there was unavoidable delay for only part of the period during which that assessment was in force, and that another Category A, Category B or Category D interim maintenance assessment should be made, the effective date of that other Category A, or Category D interim maintenance assessment shall, subject to paragraph (3), be the first day of the maintenance period following the date upon which, in the opinion of the child support officer, the delay became avoidable and the effective date of that other Category B interim maintenance assessment made after 22nd January 1996 shall be the date set out in regulation 8C(1)($b$).

(3) Where the Category A or Category B interim maintenance assessment cancelled in accordance with the provisions of paragraph (1) was made prior to 18th April 1995 and the effective date of any new Category A or Category B interim maintenance assessment would, by virtue of paragraph (2), be prior to 18th April 1995, the effective date of that new Category A or Category B interim maintenance assessment shall be the first day of the maintenance period which begins on or after 18th April 1995.

(4) Where in respect of any Category A or Category B interim maintenance assessment in force before 18th April 1995 the delay referred to in paragraph (1) became avoidable before 18th April 1995, that Category A or Category B interim maintenance assessment may not be cancelled with effect from a date earlier than the date the delay became avoidable.

(5) Subject to paragraph (1), where a child support officer is satisfied that it would be appropriate to make an interim maintenance assessment the Category of which is different from that of the interim maintenance assessment in force, he may cancel the interim maintenance assessment which is in force with effect from—
\begin{enumerate}\item[]
(i) subject to sub-paragraph (ii), whichever is the later of the first day of the maintenance period in which he becomes so satisfied or the first day of the maintenance period which begins on or after 18th April 1995; or

(ii) where he is satisfied that the interim maintenance assessment in force should be replaced by a Category B interim maintenance assessment, whichever is the later of the effective date of the interim maintenance assessment in force or 22nd January 1996.
\end{enumerate}

(6) Where an interim maintenance assessment is cancelled under the provisions of paragraph (5)(ii) and that interim maintenance assessment was made immediately following a previous interim maintenance assessment, a child support officer shall also cancel that previous interim maintenance assessment with effect from the effective date of that previous interim maintenance assessment or 22nd January 1996 whichever is the later.

(7) Where an interim maintenance assessment has been cancelled in the circumstances set out in paragraph (5)(ii) or (6), payments made under that interim maintenance assessment shall be treated as payments made under the Category B interim maintenance assessment which replaces it.

(8) In paragraph (5), “Category” in relation to an interim maintenance assessment means Category A, Category B, Category C or Category D, as the case may be.

(9) Where a child support officer makes an interim maintenance assessment following the cancellation of an interim maintenance assessment in accordance with paragraph (5), the effective date of the fresh interim maintenance assessment shall be—
\begin{enumerate}\item[]
(i) subject to sub-paragraph (ii), the date upon which that cancellation took effect;

(ii) where the fresh interim maintenance assessment is a Category B interim maintenance assessment, subject to paragraphs (10) and (11), the date determined in accordance with regulation 8C(1)($b$) or 22nd January 1996, whichever is later.
\end{enumerate}

(10) Where paragraph (9)(ii) applies and the interim maintenance assessment cancelled in accordance with paragraph (5) caused a court order to cease to have effect in accordance with regulation 3(6) of the Maintenance Arrangements and Jurisdiction Regulations, the effective date of the Category B interim maintenance assessment referred to in paragraph (9)(ii) shall be the date upon which that cancellation took effect.

(11) Where paragraphs (6) and (9)(ii) apply and the interim maintenance assessment cancelled in accordance with paragraph (6) caused a court order to cease to have effect in accordance with regulation 3(6) of the Maintenance Arrangements and Jurisdiction Regulations, the effective date of the Category B interim maintenance assessment referred to in paragraph 9(ii) shall be the date upon which that cancellation in accordance with paragraph (6) took effect.

(12) A child support officer may cancel an interim maintenance assessment which is in force with effect from such date as he considers appropriate in all the circumstances on the grounds that—
\begin{enumerate}\item[]
($a$) there was a material procedural error in connection with the making of the assessment; or

($b$) he is satisfied that he did not, or has subsequently ceased to have jurisdiction to make that interim maintenance assessment.
\end{enumerate}

(13) Where a child support officer has cancelled an interim maintenance assessment under paragraph (12), a relevant person may apply to the Secretary of State for a review of that cancellation under section 18(3) of the Act and the provisions of section 18(5) to (8) shall apply to that review.

(14) Where, following a review under section 18(3) of the Act, a child support officer sets aside the cancellation of the interim maintenance assessment which has been cancelled under paragraph (12), the effective date of the reinstated interim maintenance assessment shall be the date 
%on which the cancelled interim maintenance assessment ceased to have effect 
set by the child support officer under paragraph (12) on which the cancellation referred to in that paragraph took effect  % Words substituted (19.1.98) by SI 1998/58 reg 37
or 22nd January 1996 whichever is the later.

(15) An interim maintenance assessment in force which is made under section 12(1)($b$) or ($c$) of the Act shall be cancelled by a child support officer with effect from the effective date of that interim maintenance assessment as soon as is reasonably practicable after he has received the information or evidence which enables him to carry out or to complete a review under section 16, 17, 18 or 19 of the Act.

(16) Where an interim maintenance assessment has been cancelled under paragraph (15), payments made under it shall be treated as payments made under the maintenance assessment being reviewed under section 16, 17, 18 or 19 of the Act or under any maintenance assessment made following the review which replaces for the relevant period the maintenance assessment being reviewed.

\amendment{
Reg. 9 substituted (22.1.96) by the Child Support (Miscellaneous Amendments) (No. 2) Regulations 1995 reg. 17.

Words substituted in reg. 9(14) (19.1.98) by the Child Support (Miscellaneous Amendments) Regulations 1998 reg. 37.
}

\subsection[9A. Application for cancellation of an interim maintenance assessment]{Application for cancellation of an interim maintenance assessment}

9A.—(1) An absent parent with respect to whom a Category A or Category D interim maintenance assessment is in force may apply to a child support officer for that interim maintenance assessment to be cancelled.

(2) Any application made under paragraph (1) shall be in writing, and shall include the a statement of the grounds for the application.

(3) A child support officer who receives an application under provisions of paragraph (1), shall—
\begin{enumerate}\item[]
($a$) decide whether the interim maintenance assessment is to be cancelled and, if so, the date with effect from which it is to be cancelled;

($b$) in any case where he does cancel an interim maintenance assessment, decide whether it is appropriate for a maintenance assessment to be made in accordance with the provisions of Part I of Schedule 1 to the Act;

($c$) in any case where he has decided that it is appropriate for a maintenance assessment to be made in accordance with the provisions of Part I of Schedule 1 to the Act, make that assessment.
\end{enumerate}

(4) Where a child support officer has made a decision under paragraph (3), he shall immediately notify the applicant, so far as that is reasonably practicable, and shall give the reasons for his decision in writing.

(5) A notification under paragraph (4) shall include information as to the provisions of sections 18 and 20 of the Act and regulation 24(1) and, where an assessment is made in accordance with the provisions of Part I of Schedule 1 to the Act, the provisions of sections 16 and 17 of the Act.

(6) Where a child support officer has made a decision following an application under paragraph (1), the absent parent may apply to the Secretary of State for a review of that decision and, subject to the modification set out in paragraph (7), the provisions of section 18(5) to (8) of the Act shall apply to such a review.

(7) The modification referred to in paragraph (6) is that section 18(6) of the Act shall have effect as if for “the refusal, assessment or cancellation in question” there is substituted “the decision following an application under regulation 9A(1) of the Child Support (Maintenance Assessment Procedure) Regulations 1992”.

(8) Regulations 10, 11, 24 and 25 shall apply to reviews under paragraph (6).

\amendment{
Reg. 9A substituted for reg. 9 (22.1.96) by the Child Support (Miscellaneous Amendments) (No. 2) Regulations 1995 reg. 17.
}

\section[Part IV --- Notifications following certain decisions by child support officers]{\sloppy Part IV\\*Notifications following certain decisions by child support officers}

\renewcommand\parthead{--- Part IV}

\subsection[10. Notification of a new or a fresh maintenance assessment]{Notification of a new or a fresh maintenance assessment}

10.—%(1) Where a child support officer makes a new or a fresh maintenance assessment following—
%\begin{enumerate}\item[]
%($a$) an application under section 4, 6 or 7 of the Act; or
%
%($b$) a review under section 16, 17, 18 or 19 of the Act,
%\end{enumerate}
%he shall immediately notify the relevant persons, so far as that is reasonably practicable, of the amount of the child support maintenance under that assessment.
%
% Reg 10(1) substituted (18.4.95) by SI 1995/1045 reg 30(2)
(1) Where a child support officer—
\begin{enumerate}\item[]
%($a$) makes a new or fresh maintenance assessment following an application under section 4, 6 or 7 of the Act or a review under section 16, 17, 18 or 19 of the Act;
%
%($b$) 
%makes a new interim maintenance assessment under section 12 of the Act or  % Words inserted (22.1.96) by SI 1995/3261 reg 18(2)
%substitutes an interim maintenance assessment for one which is in force in accordance with regulation 8
%or 9%Words inserted (22.1.96) by SI 1995/3261 reg 18(2)
%; or

% Reg 10(1)(a), (b) substituted (2.12.96) by SI 1996/2907 reg 67(2)
($a$) makes a new or fresh maintenance assessment following an application under section 4, 6 or 7 of the Act, a review under section 16, 17, 18 or 19 of the Act, or the giving or cancellation of a departure direction;

($b$) makes a new interim maintenance assessment under section 12 of the Act, substitutes an interim maintenance assessment for one which is in force in accordance with regulation 8 or 9, or gives or cancels a departure direction; or

($c$) makes a maintenance assessment calculated in accordance with Part I of Schedule 1 to the Act where an interim maintenance assessment is or has been in force,
\end{enumerate}
he shall immediately notify the relevant persons, so far as that is reasonably practicable, of the amount of the child support maintenance under that assessment.

(2) 
%A notification under paragraph (1) 
Subject to 
  %paragraph (2A), 
  paragraphs (2A) and (2B),  % Words substituted (18.4.95) by SI 1995/1045 reg 30(3)
a notification under paragraph (1)  % Words substituted (16.2.95) by SI 1995/123 reg 6(2)
shall set out, in relation to the maintenance assessment in question—
\begin{enumerate}\item[]
($a$) the maintenance requirement;

($b$) the effective date of the assessment;

%($c$) the absent parent’s assessable income and, where relevant, his protected income level;

% Reg 10(2)($c$) substituted (22.1.96) by SI 1995/3261 reg 18(3)
($c$) the net and assessable income of the absent parent and, where relevant, the amount determined under regulation 9(1)($b$) of the Maintenance Assessments and Special Cases Regulations (housing costs);

% Reg 10(2)($cc$) inserted (22.1.96) by SI 1995/3261 reg 18(4)
($cc$) where relevant, the absent parent’s protected income level and the amount of the maintenance assessment before the adjustment in respect of protected income specified in paragraph 6(2) of Schedule 1 to the Act was carried out;

%($d$) the assessable income of a parent with care;

% Reg 10(2)($d$) substituted (22.1.96) by SI 1995/3261 reg 19(5)
($d$) the net and assessable income of the parent with care, and, where relevant, an amount in relation to housing costs determined in the manner specified in regulation 10 of the Maintenance Assessments and Special Cases Regulations (calculation of exempt income of parent with care);

($e$) details as to the minimum amount of child support maintenance payable by virtue of regulations made under paragraph 7 of Schedule 1 to the Act; and

($f$) details as to apportionment where a case is to be treated as a special case for the purposes of the Act under section 42 of the Act;

% Reg 10(2)($h$) inserted (22.1.96) by SI 1995/3261 reg 19(6)
($h$)\footnote{\frenchspacing Where the provisions of Part II of the Schedule to S.I. 1992/2644 (C. 83) are applied a further item (sub-paragraph ($g$)) is to be included in paragraph (2) by virtue of paragraph 10 of that Schedule.} any amount determined in accordance with Schedule 3A or 3B to the Maintenance Assessments and Special Cases Regulations (qualifying transfer of property and travel costs);

% Reg 10(2)(i) inserted (2.12.96) by SI 1996/2907 reg 67(4)
($i$) where the notification under paragraph (1)($a$) or ($b$) follows the giving, or cancellation of a departure direction, the amounts calculated in accordance with Part I of Schedule 1 to the Act, or in accordance with regulation 8A, which have been changed as a result of the giving or cancellation of that departure direction.
\end{enumerate}

%Reg 10(2A) inserted (16.2.95) by SI 1995/123 reg 6(3)
(2A) Where a new Category A% 
, Category C or D  % Words inserted (18.4.95) by SI 1995/1045 reg 30(4)
interim maintenance assessment is made, or a fresh Category A% 
, Category C or D  % Words inserted (18.4.95) by SI 1995/1045 reg 30(4)
interim maintenance assessment is made following a review under section 16 or 19(1) of the Act, a notification under paragraph (1) shall set out, in relation to that interim maintenance assessment, the maintenance requirement and the effective date.

% Reg 10(2AA) inserted (2.12.96) by SI 1996/2907 reg 67(5)
(2AA) Where a fresh Category D interim maintenance assessment is made following the giving or cancellation of a departure direction, a notification under paragraph (1) shall set out in relation to that interim maintenance assessment the amounts calculated in accordance with regulation 8A which have changed as a result of the giving or cancellation of that departure direction.

%Reg 10(2B) inserted (18.4.95) by SI 1995/1045 reg 30(5)
(2B) A notification under paragraph (1) in relation to a Category B interim maintenance assessment shall set out in relation to it—
\begin{enumerate}\item[]
%($a$) the matters listed in sub-paragraphs ($a$), ($b$) and ($d$) to ($f$) of paragraph (2); and
%
%($b$) where known, the absent parent’s assessable income.

% Reg 10(2B)(a)--(c) substituted for reg. 10(2B)(a), (b) (2.12.96) by SI 1996/2907 reg 67(5)
($a$) the matters listed in sub-paragraphs ($a$), ($b$) and ($d$) to ($f$) of paragraph (2);

($b$) where known, the absent parent’s assessable income; and

($c$) where the Category B interim maintenance assessment is made following the giving or cancellation of a departure direction, the amounts calculated in accordance with regulation 8A which have changed as a result of the giving or cancellation of that departure direction.
\end{enumerate}

(3) Except where a person gives written permission to the Secretary of State that the information, in relation to him, mentioned in sub-paragraphs ($a$) and ($b$) below may be conveyed to other persons, any document given or sent under the provisions of paragraph (1) or (2) shall not contain—
\begin{enumerate}\item[]
($a$) the address of any person other than the recipient of the document in question (other than the address of the office of the child support officer concerned) or any other information the use of which could reasonably be expected to lead to any such person being located;

($b$) any other information the use of which could reasonably be expected to lead to any person, other than a qualifying child or a relevant person, being identified.
\end{enumerate}

(4) 
%A notification under paragraph (1) 
Subject to paragraph (5), a notification under paragraph (1)  % Words substituted (16.2.95) by SI 1995/123 reg 6(4)
shall include information as to the following provisions—
\begin{enumerate}\item[]
($a$) where a new maintenance assessment is made following an application under the Act or a fresh maintenance assessment is made following a review under section 16 of the Act, sections 16, 17 and 18 of the Act;

($b$) where a fresh maintenance assessment is made following a review under section 17 of the Act, or following a review under section 19 of the Act where the child support officer conducting such a review is satisfied that if an application were to be made under section 17 of the Act it would be appropriate to make a fresh maintenance assessment, sections 16 and 18 of the Act;

%($c$) where a fresh maintenance assessment is made following a review under section 18 of the Act, or following a review under section 19 of the Act where the child support officer conducting such a review is satisfied that if an application were to be made under section 18 of the Act, it would be appropriate to make a fresh maintenance assessment, sections 16, 17 and 20 of the Act.

%Reg 10(4)($c$) substituted by reg 10(4)($c$), ($d$) (7.2.94) by SI 1994/227 reg 2(2)
($c$) where a fresh maintenance assessment is made following a review under section 18 of the Act, sections 16, 17 and 20 of the Act;

%($d$) where a fresh maintenance assessment is made following a review under section 19 of the Act, sections 16, 17 and 18 of the Act.

% Reg 10(4)(d), (e) substituted for reg 10(4)(d) (2.12.96) by SI 1996/2907 reg 67(6)
($d$) where a fresh maintenance assessment is made following a review under section 19 of the Act, sections 16, 17 and 18 of the Act;

($e$) where a fresh maintenance assessment is made following the giving of a departure direction, sections 16, 17 and 18 of the Act.
\end{enumerate}

% Reg 10(5) inserted (16.2.95) by SI 1995/123 reg 6(5)
(5) Where a new Category A 
or Category D  % Words inserted (18.4.95) by SI 1995/1045 reg 30(6)
interim maintenance assessment is made or a fresh Category A 
or Category D  % Words inserted (18.4.95) by SI 1995/1045 reg 30(6)
interim maintenance assessment is made following a review under section 16 or 19(1) of the Act, a notification under paragraph (1) shall include information as to sections 16 and 19(1) of the Act.

% Reg 10(6) inserted (2.12.96) by SI 1996/2907 reg 67(7)
(6) Where a fresh Category D interim maintenance assessment is made following the giving or cancellation of a departure direction, a notification under paragraph (1) shall include information as to sections 16 and 19(1) of the Act.

\amendment{
Reg. 10(4)(c), (d) substituted for reg. 10(4)(d) (7.2.94) by the Child Support (Miscellaneous Amendments and Transitional Provisions) Regulations 1994 reg. 2(2) (subject to transitional provisions in reg. 12).

Words substituted in reg. 10(2), (4) and reg. 10(2A), (5) inserted (16.2.95) by the Child Support (Miscellaneous Amendments) Regulations 1995 reg. 6.

Words inserted in reg. 10(2A), (5), words substituted in reg. 10(2), reg. 10(2B) inserted and reg. 10(1) substituted (18.4.95) by the Child Support and Income Support (Amendment) Regulations 1995 reg. 30.

%Words inserted in reg. 10(1)($b$), 
Reg. 10(2)(cc), (h) inserted and reg. 10(2)(c), (d) substituted (22.1.96) by the Child Support (Miscellaneous Amendments) (No. 2) Regulations 1995 reg. 18.

Reg. 10(2)(i), (2AA), (6) inserted, reg. 10(2B)(a)--(c) substituted for reg. 10(2B)(a), (b), reg. 10(4)(d), (e) substituted for reg. 10(4)(d) and reg. 10(1)(a), (b) substituted (2.12.96) by the Child Support Departure Direction and Consequential Amendments Regulations 1996 reg. 67.
}

% Reg 10A inserted (19.1.98) by SI 1998/58 reg 38
\subsection[10A. Notification of increase or reduction in the amount of a maintenance assessment]{Notification of increase or reduction in the amount of a maintenance assessment}

10A.—(1) Where, in a case falling within paragraph (2B) of regulation 22 of the Maintenance Assessments and Special Cases Regulations (multiple applications relating to an absent parent)\footnote{\frenchspacing Paragraph (2B) is inserted by regulation 53.}, a child support officer has increased or reduced one or more of the other maintenance assessments referred to in that paragraph following the making of the fresh assessment referred to in sub-paragraph ($c$) of that paragraph, he shall, so far as that is reasonably practicable, immediately notify the relevant persons in respect of whom each maintenance assessment so increased or reduced was made of—
\begin{enumerate}\item[]
($a$) the making of that fresh assessment;

($b$) the amount of the increase or reduction in that maintenance assessment; and

($c$) the date on which that increase or reduction shall take effect,
\end{enumerate}
and the notification shall include information as to the provisions of section 18 of the Act.

(2) Except where a person gives written permission to the Secretary of State that the information in relation to him mentioned in sub-paragraphs ($a$) and ($b$) below may be conveyed to other persons, any document given or sent under the provisions of paragraph (1) shall not contain—
\begin{enumerate}\item[]
($a$) the address of any person other than the recipient of the document in question (other than the address of the office of the child support officer concerned) or any other information the use of which could reasonably be expected to lead to any such person being located;

($b$) any other information the use of which could reasonably be expected to lead to any person, other than a qualifying child or a relevant person, being identified.
\end{enumerate}

\amendment{
Reg. 10A inserted (19.1.98) by the Child Support (Miscellaneous Amendments) Regulations 1998 reg. 38.
}

\subsection[11. Notification of a refusal to conduct a review]{Notification of a refusal to conduct a review}

11.—(1) Where a child support officer refuses an application for a review under section 17 of the Act on the grounds set out in section 17(3) of the Act, or an application for a review under section 18 of the Act on the grounds set out in section 18(6) of the Act, he shall immediately notify the applicant, so far as that is reasonably practicable, and shall give the reasons for his refusal in writing.

%(2) A notification under paragraph (1) shall include information as to the following provisions—
%\begin{enumerate}\item[]
%($a$) where the refusal is on the grounds set out in section 17(3) of the Act, sections 16 and 18 of the Act and regulations 24(1) and 31(7);
%
%($b$) where the refusal is on the grounds set out in section 18(6) of the Act, sections 16, 17 and 20 of the Act.
%\end{enumerate}

% Reg 11(2) substituted (18.4.95) by SI 1995/1045 reg 31
(2) A notification under paragraph (1) shall include information as to the following provisions—
\begin{enumerate}\item[]
($a$) where the refusal is on the grounds set out in section 17(3) of the Act, sections 16 and 18 of the Act and regulations 24(1) and 31(7);

($b$) except where sub-paragraph ($c$) applies, where the refusal is on the grounds set out in section 18(6) of the Act, sections 16, 17 and 20 of the Act;

($c$) where the refusal is on the grounds set out in section 18(6) of the Act and relates to a decision made under regulation 9(6), sections 16 and 20 of the Act.
\end{enumerate}

\amendment{
Reg. 11(2) substituted (18.4.95) by the Child Support and Income Support (Amendment) Regulations 1995 reg. 31.
}

\subsection[12. Notification of a refusal to make a new or a fresh maintenance assessment]{Notification of a refusal to make a new or a fresh maintenance assessment}

12.—%(1) Where a child support officer refuses an application for a maintenance assessment under the Act, or refuses to make a fresh assessment following a review under 
%%section 17 or 
%section 17 of the Act or to make an assessment or a fresh assessment following a review under section %Words inserted by SI 1993/913 reg 5(2)($a$) eff 5.4.93
%18 of the Act, he shall immediately notify the following persons, so far as that is reasonably practicable—
%\begin{enumerate}\item[]
%($a$) where an application for a maintenance assessment under section 4 or 6 of the Act is refused, the applicant;
%
%($b$) where an application for a maintenance assessment under section 7 of the Act is refused, the applicant child and the other relevant persons who have been notified of the application;
%
%($c$) where there is a refusal to make a fresh assessment following a review under section 17 or 18 of the Act, the relevant persons;
%
%%Reg 12(1)($d$) inserted by SI 1993/913 reg 5(2)($b$) eff 5.4.93
%($d$) where there is a refusal to make an assessment following a review under section 18 of the Act, the applicant,
%\end{enumerate}
%and shall give the reasons for his refusal in writing.
%
%Reg 12(1) substituted (22.1.96) by SI 1995/3261 reg 19
(1) Where a child support officer—
\begin{enumerate}\item[]
($a$) refuses an application for a maintenance assessment under the Act;

($b$) refuses to make a fresh assessment following a review under section 17 of the Act;

($c$) refuses to make an assessment or a fresh assessment following a review under section 18 of the Act; or

($d$) decides not to make a maintenance assessment or a fresh assessment under section 19 of the Act,
\end{enumerate}
he shall immediately notify the following persons, so far as that is reasonably practicable—
\begin{enumerate}\item[]
(i) where an application for a maintenance assessment under section 4 or 6 of the Act is refused, the applicant;

(ii) where an application under section 7 of the Act is refused, the applicant child and the other relevant persons who have been notified of the application;

(iii) where there is a refusal to make a fresh assessment following a review under section 17 or 18(2) of the Act, or a child support officer has decided not to make a fresh assessment following a review under section 19(1)($c$) of the Act, the relevant persons; or

(iv) where there is a refusal to make an assessment following a review under section 18(1)($a$) of the Act, or a child support officer has decided not to make an assessment following a review under section 19(1)($a$) of the Act, the applicant for that assessment,
\end{enumerate}
and shall give in writing the reasons for his refusal.

(2) A notification under paragraph (1) shall include information as to the following provisions—
\begin{enumerate}\item[]
($a$) where an application for a maintenance assessment under the Act is refused, section 18 of the Act and regulation 24(1);

($b$) where there is a refusal to make a fresh assessment following a review under section 17 of the Act, sections 16 and 18 of the Act and regulation 24(1);

($c$) where there is a refusal to make a fresh assessment following a review under section 18 of the Act, sections 16, 17 and 20 of the Act;

%Reg 12(2)($d$) inserted by SI 1993/913 reg 5(3) eff 5.4.93
($d$) where there is a refusal to make an assessment following a review under section 18 of the Act, section 20 of the Act.
\end{enumerate}

\amendment{
%Words inserted in reg. 5(1) and reg. 5(1)($d$), 
Reg. 12(2)($d$) inserted (5.4.93) by the Child Support (Miscellaneous Amendments) Regulations 1993 reg. 5.

Reg. 12(1) substituted (22.1.96) by the Child Support (Miscellaneous Amendments) (No. 2) Regulations 1995 reg. 19.
}

\subsection[13. Notification of a refusal to cancel a maintenance assessment]{Notification of a refusal to cancel a maintenance assessment}

13.—(1) Where a child support officer refuses a request under paragraph 16 of Schedule 1 to the Act for a maintenance assessment to be cancelled, or refuses to cancel a maintenance assessment following a review under section 18 of the Act, he shall immediately notify the following persons, so far as that is reasonably practicable—
\begin{enumerate}\item[]
($a$) where a request for a cancellation under paragraph 16 of Schedule 1 to the Act is refused, the applicant, or, as the case may be, the applicants;

($b$) where the cancellation of a maintenance assessment following a review under section 18 of the Act is refused, the relevant persons,
\end{enumerate}
and shall give the reasons for his refusal in writing.

(2) A notification under paragraph (1) shall include information as to the following provisions—
\begin{enumerate}\item[]
($a$) where a request for a cancellation under paragraph 16 of Schedule 1 to the Act is refused, sections 16 and 18 of the Act and regulation 24(1);

($b$) where the cancellation of a maintenance assessment following a review under section 18 of the Act is refused, sections 16, 17 and 20 of the Act;

% Reg 13(2)($c$) inserted (18.4.95) by SI 1995/1045 reg 32
($c$) where the refusal is of an application for the cancellation of a Category A or a Category D interim maintenance assessment under regulation 9, sections 16 and 20 of the Act.
\end{enumerate}

\amendment{
Reg. 13(2)($c$) inserted (18.4.95) by the Child Support and Income Support (Amendment) Regulations 1995 reg. 32.
}

\subsection[14. Notification of a cancellation of a maintenance assessment]{Notification of a cancellation of a maintenance assessment}

14.—(1) Where a child support officer cancels a maintenance assessment, 
except a Category A or Category D interim maintenance assessment falling within 
  %regulation 9,  
  regulation 9A, % Words substituted (22.1.96) by SI 1995/3261 reg 20(2)
% Words inserted (18.4.95) by SI 1995/1045 reg 33
he shall immediately notify the relevant persons, so far as that is reasonably practicable, and shall give the reasons for the cancellation in writing.

(2) A notification under paragraph (1) shall include information as to the provisions of section 18 of the Act and regulations 24(1) and 
%31(8).
31A(8).  % Words substituted (22.1.96) by SI 1995/3261 reg 20(3)

\amendment{
Words inserted in reg. 14(1) (18.4.95) by the Child Support and Income Support (Amendment) Regulations 1995 reg. 33.

Words substituted in reg. 14(1), (2) (22.1.96) by the Child Support (Miscellaneous Amendments) Regulations 1995 reg. 20.
}

\subsection[15. Notification of a refusal to reinstate a cancelled maintenance assessment]{Notification of a refusal to reinstate a cancelled maintenance assessment}

15.—(1) Where a child support officer, following a review under section 18(3) of the Act, refuses to reinstate a maintenance assessment that has been cancelled
or following a review under section 19(1)($d$) of the Act decides not to reinstate a cancelled maintenance assessment%Words inserted (22.1.96) by SI 1995/3261 reg 21(2)
, he shall immediately notify the relevant persons, so far as that is reasonably practicable, and shall give the reasons for his refusal in writing.

%(2) A notification under paragraph (1) shall include information as to the provisions of section 20 of the Act.

% Reg 15(2) substituted (22.1.96) by SI 1995/3261 reg 21(3)
(2) A notification under paragraph (1) shall, where the review is carried out under section 18(3) of the Act, include information as to the provisions of section 20 of the Act and, where the review is carried out under section 19(1)($d$) of the Act, except where that review is of the cancellation of a Category A or Category D interim maintenance assessment, the provisions of section 18 of the Act and regulations 24(1) and 31A(8).

\amendment{
Words inserted in reg. 15(1) and reg. 15(2) substituted (22.1.96) by the Child Support (Miscellaneous Amendments) (No. 2) Regulations 1995 reg. 21.
}

% Reg 15A inserted (22.1.96) by SI 1995/3261 reg 22
\subsection[15A. Notification of reinstatement of a maintenance assessment]{Notification of reinstatement of a maintenance assessment}

15A.—(1) Where a child support officer, following a review under section 18(3) or 19(1)($d$) of the Act, has decided that the cancellation of a maintenance assessment should be set aside, he shall immediately notify the relevant persons, so far as that is reasonably practicable, and shall give in writing reasons for the setting aside of the cancellation and, if applicable, the date with effect from which the maintenance assessment is reinstated.

(2) A notification under paragraph (1) shall, where the review is carried out under section 18(3) of the Act, include information as to the provisions of section 20 of the Act
and where the review is carried out under section 19(1)($d$) of the Act, except where that review is of the cancellation of a Category A or Category D interim maintenance assessment, as to the provisions of section 18 of the Act and regulations 24(1) and 31A(8).  % Words added (5.8.96) by SI 1996/1945 reg 9

\amendment{
Reg. 15A inserted (22.1.96) by the Child Support (Miscellaneous Amendments) (No. 2) Regulations 1995 reg. 22.

Words inserted in reg. 15A(2) (5.8.96) by the Child Support (Miscellanous Amendments) Regulations 1996 reg. 9.
}

\subsection[16. Notification when an applicant under section 7 of the Act ceases to be a child]{\sloppy Notification when an applicant under section 7 of the Act ceases to be a child}

16.  Where a maintenance assessment has been made in response to an application by a child under section 7 of the Act and that child ceases to be a child for the purposes of the Act, a child support officer shall immediately notify, so far as that is reasonably practicable—
\begin{enumerate}\item[]
($a$) the other qualifying children 
%over the age of 12 
who have attained the age of 12 years % Words substituted by SI 1993/913 reg 6 eff 5.4.93
and the absent parent with respect to whom that maintenance assessment was made; and

($b$) the person with care.
\end{enumerate}

\amendment{
Words substituted in reg. 16($a$) (5.4.93) by the Child Support (Miscellaneous Amendments) Regulations 1993 reg. 6.
}

% Reg 16A inserted (18.12.95) by SI 1995/3261 reg 23
\subsection[16A. Notification that an appeal has lapsed]{Notification that an appeal has lapsed}

16A.  Where a case falls within section 20A(1) of the Act and the appeal that has been brought under section 20 of the Act lapses under the provisions of section 20A(2) of the Act a child support officer shall, so far as that is reasonably practicable, notify the relevant persons that that appeal has lapsed.

\amendment{
Reg. 16A inserted (18.12.95) by the Child Support (Miscellaneous Amendments) (No. 2) Regulations 1995 reg. 23.
}

\section[Part V --- Periodical reviews]{Part V\\*Periodical reviews}

\renewcommand\parthead{--- Part V}

\subsection[17. Intervals between periodical reviews and notice of a periodical review]{Intervals between periodical reviews and notice of a periodical review}

17.—%(1) Subject to regulation 18(1), a maintenance assessment that has been in force for a period of 52 weeks shall be reviewed by a child support officer under section 16 of the Act.
%
%(2) Where a review under section 17 of the Act results in a fresh maintenance assessment, the next review under the provisions of paragraph (1) shall be conducted when that fresh assessment has been in force for a period of 52 weeks.
%
% Reg 17(1), (2) substituted (5.4.93) by SI 1993/913 reg 7(2)
(1) Subject to 
%regulation 18(1),
regulation 18,  % Words substituted (22.1.96) by SI 1995/3261 reg 24(2)
where a maintenance assessment in force is---
\begin{enumerate}\item[]
($a$) an assessment that has not been previously reviewed;

($b$) a fresh assessment following an earlier review under section 16 of the Act; or

($c$) a fresh assessment following a review under section 17 of the Act
where before 22nd January 1996 a child support officer decided, in accordance with section 17(3) of the Act, to proceed with a review,  % Words inserted (22.1.96) by SI 1995/3261 reg 24(2)
\end{enumerate}
that assessment shall be reviewed by a child support officer under section 16 of the Act 
%after it has been in force for a period of 52 weeks.
after it has been in force for a period of—
\begin{enumerate}\item[]
(i) in the case of an assessment the effective date of which is on or before 18th April 1994, 52 weeks;

(ii) in the case of an assessment the effective date of which is after 18th April 1994, 104 weeks.
\end{enumerate}  % Words substituted (18.4.95) by SI 1995/1045 reg 34(2)

%(2) Where a maintenance assessment in force is a fresh assessment following a review under section 18 or 19 of the Act, that assessment shall be reviewed by a child support officer under section 16 of the Act after it has been in force for a period of 52 weeks less the period between the effective date of the previous assessment falling within paragraph (1) above and the effective date of the fresh assessment following the review under section 18 or 19 of the Act.

% Reg 17(2) substituted (18.4.95) by SI 1995/1045 reg 34(3)
%(2) Where a maintenance assessment in force is a fresh assessment following a review under section 18 or 19 of the Act, that assessment shall be reviewed by a child support officer under section 16 of the Act after it has been in force for a period of—
%\begin{enumerate}\item[]
%($a$) in the case of an assessment the effective date of which is on or before 18th April 1994, 52 weeks;
%
%($b$) in the case of an assessment the effective date of which is after 18th April 1994, 104 weeks,
%\end{enumerate}
%less, in either case, the period between the effective date of the previous assessment falling within paragraph (1) above and the effective date of the fresh assessment following the review under section 18 or 19 of the Act.

% Reg 17(2) substituted (22.1.96) by SI 1995/3261 reg 24(3)
(2) Where a maintenance assessment in force is a fresh assessment, following—
\begin{enumerate}\item[]
($a$) a review under section 17 of the Act where, after 22nd January 1996, a child support officer decided, in accordance with section 17(3) of the Act, to proceed with that review; or

($b$) a review under section 18 or 19 of the Act,
\end{enumerate}
that assessment shall be reviewed by a child support officer under section 16 of the Act after it has been in force for a period of—
\begin{enumerate}\item[]
(i) in a case where the effective date of the assessment that has been reviewed was on or before 18 April 1994, 52 weeks;

(ii) in a case where the effective date of the assessment that has been reviewed was after 18th April 1994, 104 weeks,
\end{enumerate}
less, in either case, 
%the period between the effective date of the assessment that has been reviewed and the effective date of the fresh assessment following that review.
the period between the effective date of the fresh assessment following the latest review carried out under sub-paragraph ($a$) or ($b$) and whichever is the later of—
\begin{enumerate}\item[]
($aa$) the effective date of the assessment falling within sub-paragraph ($a$) of paragraph (1); or

($bb$) the effective date of the assessment made following the review referred to in sub-paragraph ($b$) or ($c$) of paragraph (1).
\end{enumerate}  % Words substituted (19.1.98) by SI 1998/58 reg 39(2)

(3) A child support officer may decide not to conduct a review under paragraph (1) if a fresh maintenance assessment following such a review would cease to have effect within 28 days of the effective date of that fresh assessment.

(4) Before a child support officer conducts a review under section 16 of the Act, he shall give 14 days' notice of the proposed review to the relevant persons.

(5) Subject to paragraphs (6) and (7), a child support officer shall request every person to whom he is giving notice under paragraph (4) to provide, within 14 days, and in accordance with the provisions of regulations 2 and 3 of the Information, Evidence and Disclosure Regulations such information or evidence as to his current circumstances as may be specified
and shall set out the possible consequences of failure to provide that information or evidence.  % Words inserted (22.1.96) by SI 1995/3261 reg 24(4)

%(6) The provisions of paragraph (5) shall not apply in relation to any person to whom or in respect of whom income support 
%or income-based jobseeker’s allowance  % Words inserted (7.10.96) by SI 1996/1345 reg 5(4)
%is payable or to a person with care where income support 
%or income-based jobseeker’s allowance  % Words inserted (7.10.96) by SI 1996/1345 reg 5(4)
%is payable to or in respect of the absent parent.

% Reg 17(6) substituted (6.4.98) by SI 1998/58 reg 39(3)
(6) The provisions of paragraph (5) shall not apply in relation to—
\begin{enumerate}\item[]
($a$) any person to or in respect of whom income support or income-based jobseeker’s allowance is payable;

($b$) a person with care where income support or income-based jobseeker’s allowance is payable to or in respect of the absent parent;

($c$) an absent parent or parent with care to whom regulation 10A of the Maintenance Assessments and Special Cases Regulations applies; or

($d$) a parent with care where that regulation applies to the absent parent.
\end{enumerate}

(7) The provisions of paragraph (5) shall not apply in relation to a relevant person where—
\begin{enumerate}\item[]
($a$) 
%a case is 
the case is one % Words substituted (5.4.93) by SI 1993/913 reg 7(3)
prescribed in regulation 22 or 23 of the Maintenance Assessments and Special Cases Regulations as a case to be treated as a special case for the purposes of the Act;

($b$) there has been a review under section 16 
%or 17  % Words omitted (22.1.96) by SI 1995/3261 reg 24(5)
of the Act in relation to another maintenance assessment in force relating to that person;

($c$) the child support officer concerned has notified that person of the assessments following that review not earlier than 13 weeks prior to the date a review under section 16 of the Act is due under paragraph (1); and

($d$) the child support officer has no reason to believe that there has been a change in that person’s circumstances.
\end{enumerate}

\amendment{
Reg. 17(1)%, (2) 
{} substituted and words substituted in reg. 17(7)($a$) (5.4.93) by the Child Support (Miscellaneous Amendments) Regulations 1993 reg. 7.

Words substituted in reg. 17(1) %and reg. 17(2) substituted 
(18.4.95) by the Child Support and Income Support (Amendment) Regulations 1995 reg. 34.

Words inserted in reg. 17(1)($c$), (5), words substituted in reg. 17(1), words omitted in reg. 17(7)($b$) and reg. 17(2) substituted (22.1.96) by the Child Support (Miscellaneous Amendments) (No. 2) Regulations 1995 reg. 24.

%Words inserted in reg. 17(6) (7.10.96) by the Social Security and Child Support (Jobseeker's Allowance) (Consequential Amendments) Regulations 1996 reg. 5(4).

Words substituted in reg. 17(2) (19.1.98) by the Child Support (Miscellaneous Amendments) Regulations 1998 reg. 39(2).

Reg. 17(6) substituted (6.4.98) by the Child Support (Miscellaneous Amendments) Regulations 1998 reg. 39(3).
}

%\subsection[18. Review under section 17 of the Act treated as a review under section 16 of the Act]{Review under section 17 of the Act treated as a review under section 16 of the Act}
%
%18.—(1) Where, under the provisions of regulation 19(1), a child support officer gives notice of a review under section 17 of the Act, that notice is given or sent not earlier than 8 weeks prior to the next review, under the provisions of regulation 17(1), of the maintenance assessment in force, and the review under section 17 of the Act does not result in a fresh maintenance assessment by virtue of the provisions of regulation 20, 21 or 22, that review shall be treated as a review under section 16 of the Act, and the fresh assessment that would have been made but for the provisions of regulation 20, 21 or 22, as the case may be, shall be the assessment following that review.
%
%(2) Where there is a fresh assessment under the provisions of paragraph (1), the next review under the provisions of regulation 17(1) shall be of that fresh assessment.

% Reg 18 substituted (22.1.96) by SI 1995/3261 reg 25
\subsection[18. Review under section 16 of the Act to be substituted for review under section 17 of the Act]{Review under section 16 of the Act to be substituted for review under section 17 of the Act}

18.  Where after 22nd January 1996 a child support officer considers that he is likely to be required under section 17(3) of the Act to make one or more fresh maintenance assessments if he conducts a review under that section and the application for that review was received by the Secretary of State not earlier than 8 weeks prior to the date upon which the next review of the maintenance assessment in force is due under the provisions of section 16 of the Act, the child support officer shall carry out a review under section 16 of the Act instead of the review under section 17 of the Act for which application has been made.

\amendment{
Reg. 18 substituted (22.1.96) by the Child Support (Miscellaneous Amendments) (No. 2) Regulations 1995 reg. 25.
}

\section[Part VI --- Reviews on a change of circumstances]{Part VI\\*Reviews on a change of circumstances}

\renewcommand\parthead{--- Part VI}

\subsection[19. Conduct of a review on a change of circumstances]{Conduct of a review on a change of circumstances}

19.—(1) Where a child support officer proposes to conduct a review under section 17 of the Act, he shall give 14 days' notice of the proposed review to the relevant persons.

%(2) Subject to 
%%paragraphs (3) and (4), and except where the circumstances set out in regulation 17(7) apply 
%paragraphs (3), (4) and (4A)% Words substituted (5.4.93) by SI 1993/913 reg 8(2)
%, a child support officer proposing to conduct a review under section 17 of the Act shall request every person to whom he is giving notice under paragraph (1) to provide within 14 days, and in accordance with the provisions of regulations 2 and 3 of the Information, Evidence and Disclosure Regulations, such information or evidence as to his current circumstances as may be specified.
%
%%(3) The provisions of paragraph (2) shall not apply in relation to any person to whom or in respect of whom income support is payable.
%
%%Reg 19(3) substituted (18.4.95) by SI 1995/1045 reg 35
%(3) The provisions of paragraph (2) shall not apply in relation to any person to whom or in respect of whom income support is payable or to a person with care where income support is payable to or in respect of the absent parent.

% Reg 19(2), (3) substituted (22.1.96) by SI 1995/3261 reg 26(2)
(2) Any application made under section 17 of the Act after 22nd January 1996 shall be in writing and shall give details of the change of circumstances in respect of which a review is sought.

(3) Where a child support officer conducts the review in respect of which notification has been given in accordance with paragraph (1), he shall take into account any information in relation to a change of circumstances notified to him in writing by a relevant person.

%(4) Where an application for a review under section 17 of the Act is made at the time that a review under section 16 of the Act is being conducted, the child support officer concerned may proceed with the review under section 17 of the Act notwithstanding that he has not complied with the provisions of paragraph (2) if in his opinion such compliance is not required in the particular circumstances of the case.
%
%%Reg 19(4A) inserted (5.4.93) by SI 1993/913 reg 8(3)
%(4A) The provisions of paragraph (2) shall not apply in relation to a relevant person where---
%\begin{enumerate}\item[]
%($a$) the case is one prescribed in regulation 22 or 23 of the Maintenance Assessments and Special Cases Regulations as a case to be treated as a special case for the purposes of the Act;
%
%($b$) there has been a review under section 16 or 17 of the Act in relation to another maintenance assessment in force relating to that person;
%
%($c$) the child support officer concerned has notified that person of the assessments following that review not earlier than 13 weeks prior to the date the child support officer gives notice under paragraph (1); and
%
%($d$) the child support officer has no reason to believe that there has been a change in that person’s circumstances.
%\end{enumerate}

% Reg 19(4), (4A) omitted (22.1.96) by SI 1995/3261 reg 26(3)

(5) Where a maintenance assessment is in force with respect to a parent with care and an absent parent in response to an application by the parent with care under section 6 of the Act, and the parent with care authorises the Secretary of State to take action under the Act to recover child support maintenance from that absent parent in relation to an additional child of whom she is a parent with care and he is an absent parent, that authorisation shall be treated by the Secretary of State as an application for a review under section 17 of the Act.

\amendment{
%Words substituted in reg. 19(2) and reg. 19(4A) inserted (5.4.93) by the Child Support (Miscellaneous Amendments) Regulations 1993 reg. 8.

%Reg. 19(3) substituted (18.4.95) by the Child Support and Income Support (Amendment) Regulations 1995 reg. 35.

Reg. 19(2), (3) substituted and reg. 19(4), (4A) omitted (22.1.96) by the Child Support (Miscellaneous Amendments) (No. 2) Regulations 1995 reg. 26 (subject to transitional provisions in reg. 57(4)). 
}

\subsection[20. Fresh assessments following a review on a change of circumstances]{Fresh assessments following a review on a change of circumstances}

20.—(1) Subject to 
%paragraphs (2) and (3) 
paragraphs (2) to 
  %(4) 
  (5)  % Word substituted (13.1.97) by SI 1996/3196 reg 7(2)
% Words substituted (5.4.93) by SI 1993/913 reg 9(2)
and regulations 21 and 22, a child support officer who has completed a review 
of an original assessment  % Words inserted (22.1.96) by SI 1995/3261 reg 27(2)($a$)
under section 17 of the Act shall not make a fresh assessment if the difference between the amount of child support maintenance 
%fixed by the assessment 
fixed by that assessment  % Words substituted (22.1.96) by SI 1995/3261 reg 27(2)($b$)
%currently in force  % Words omitted (22.1.96) by SI 1995/3261 reg 27(2)($c$)
and the amount that would be fixed if a fresh assessment were to be made as a result of the review 
of that assessment  % Words inserted (22.1.96) by SI 1995/3261 reg 27(2)($d$)
is less than £10.00 per week.

(2) Where a child support officer who has completed a review 
of an original assessment  % Words inserted (22.1.96) by SI 1995/3261 reg 27(3)($a$)
under section 17 of the Act determines that, were a fresh assessment to be made as a result of the review
of that assessment, % Words inserted (22.1.96) by SI 1995/3261 reg 27(3)($b$)
the circumstances of the absent parent are such that the provisions of paragraph 6 of Schedule 1 to the Act would apply to that 
fresh  % Word inserted (22.1.96) by SI 1995/3261 reg 27(3)($c$)
assessment, 
%he shall not make a fresh assessment if the difference between the amount of child support maintenance fixed by the original assessment and the amount that would be fixed if a fresh assessment were to be made as a result of the review is less than £1.00 per week.
he shall not make a fresh assessment if—
\begin{enumerate}\item[]
($a$) where the amount fixed by the original assessment is less than the amount that would be fixed by the fresh assessment, the difference between the two amounts is less than £5.00 a week; and

($b$) where the amount fixed by the original assessment is more than the amount that would be fixed by the fresh assessment, the difference between the two amounts is less than £1.00 a week.
\end{enumerate} % Words substituted in reg 20(2) (7.2.94) by SI 1994/227 reg 2(3)

(3) Where a child support officer who has completed a review 
of an original assessment  % Words inserted (22.1.96) by SI 1995/3261 reg 27(4)($a$)
under section 17 of the Act determines that, were a fresh assessment to be made as a result of the review
of that assessment, % Words inserted (22.1.96) by SI 1995/3261 reg 27(4)($b$)
the children in respect of whom that 
fresh  % Word inserted (22.1.96) by SI 1995/3261 reg 27(4)($c$)
assessment would be made are not identical with the children in respect of whom the original assessment was made, he shall not make a fresh assessment if the difference between the amount of child support maintenance fixed by the original assessment and the amount that would be fixed if a fresh assessment were to be made as a result of the review 
of that original assessment  % Words inserted (22.1.96) by SI 1995/3261 reg 27(4)($d$)
is less than £1.00 per week.

%Reg 20(4) inserted (5.4.93) by SI 1993/913 reg 9(3)
(4) Where a child support officer on completing a review under section 17 of the Act determines that---
\begin{enumerate}\item[]
($a$) the absent parent is, by virtue of paragraph 5(4) of Schedule 1 to the Act, to be taken for the purposes of that Schedule to have no assessable income; or

($b$) the case falls within paragraph 7(2) of Schedule 1 to the Act,
\end{enumerate}
he shall make a fresh maintenance assessment.

% Reg 20(5) inserted (13.1.97) by SI 1996/3196 reg 7(3)
(5) Where a child support officer, on completing a review under section 17 of the Act of a case falling within sub-paragraph (4) of paragraph 5 of Schedule 1 to the Act, determines that the case no longer falls within that sub-paragraph, he shall make a fresh assessment.

\amendment{
Words substituted in reg. 20(1) and reg. 20(4) inserted (5.4.93) by the Child Support (Miscellaneous Amendments) Regulations 1993 reg. 9.

Words substituted in reg. 20(2) (7.2.94) by the Child Support (Miscellaneous Amendments and Transitional Provisions) Regulations 1994 reg. 2(3) (subject to transitional provisions in reg. 12).

Words inserted in reg. 20(1), (2), (3) and words substituted and omitted in reg. 20(1) (22.1.96) by the Child Support (Miscellaneous Amendments) (No. 2) Regulations 1995 reg. 27. 

Word substituted in reg. 20(1) and reg. 20(5) inserted (13.1.97) by the Child Support (Miscellaneous Amendments) (No. 2) Regulations 1996 reg. 7.

Under the Child Support (Miscellaneous Amendments) (No. 2) Regulations 1996 reg. 16, a maintenance assessment in force on 13.1.97 continues in force notwithstanding reg. 20(5) until its next review under s. 16, 17, 18 or 19 of the Act, and the effective date of any fresh assessment affected by reg. 20(5) shall not be earlier than the first day of the first maintenance period which commences on or after 13.1.97.
}

\subsection[21. Fresh assessments following a review on a change of circumstances: special case prescribed by regulation 22 of the Maintenance Assessments and Special Cases Regulations]{Fresh assessments following a review on a change of circumstances: special case prescribed by regulation 22 of the Maintenance Assessments and Special Cases Regulations}

21.—(1) The provisions of paragraphs (2) and (3) shall apply on a review under section 17 of the Act where a case is to be treated as a special case for the purposes of the Act by virtue of regulation 22 of the Maintenance Assessments and Special Cases Regulations.

(2) Where there is a change in the circumstances of the absent parent (whether or not there is also a change in the circumstances of one or more of the persons with care), a child support officer shall not make fresh assessments if the difference between the aggregate amount of child support maintenance fixed by the 
original  % Word inserted (22.1.96) by SI 1995/3261 reg 28(2)($a$)
assessments 
%currently in force  % Words omitted (22.1.96) by SI 1995/3261 reg 28(2)($b$)
and the aggregate amount that would be fixed if fresh assessments were to be made as a result of the review 
of those original assessments  % Words inserted (22.1.96) by SI 1995/3261 reg 28(2)($c$)
is less than £10.00 per week or, where the circumstances of the absent parent are such that the provisions of paragraph 6 of Schedule 1 to the Act would apply to those fresh assessments, 
%that difference is less than £1.00 per week.
that difference is less than—
\begin{enumerate}\item[]
($a$) where the aggregate amount fixed by the original assessments is less than the aggregate amount that would be fixed by the fresh assessments, £5.00 a week; and

($b$) where the aggregate amount fixed by the original assessments is more than the aggregate amount that would be fixed by the fresh assessments, £1.00 a week.
\end{enumerate} % Words substituted in reg 21(2) (7.2.94) by SI 1994/227 reg 2(4)

(3) Where there is a change in the circumstances of one or more of the persons with care but not in that of the absent parent, the provisions of regulation 20 shall apply in relation to 
%each fresh assessment.
a review of each original assessment.  % Words substituted (22.1.96) by SI 1995/3261 reg 28(3)

\amendment{
Words substituted in reg. 21(2) (7.2.94) by the Child Support (Miscellaneous Amendments and Transitional Provisions) Regulations 1994 reg. 2(4) (subject to transitional provisions in reg. 12).

Words inserted in reg. 21(1), words substituted in reg. 21(2) and words omitted in reg. 21(1) (22.1.96) by the Child Support (Miscellaneous Amendments) (No. 2) Regulations 1995 reg. 28.
}

\subsection[22. Fresh assessments following a review on a change of circumstances: special case prescribed by regulation 23 of the Maintenance Assessments and Special Cases Regulations]{Fresh assessments following a review on a change of circumstances: special case prescribed by regulation 23 of the Maintenance Assessments and Special Cases Regulations}

22.—(1) The provisions of paragraph (2) shall apply on a review under section 17 of the Act where a case is to be treated as a special case for the purposes of the Act by virtue of regulation 23 of the Maintenance Assessments and Special Cases Regulations.

(2) Where there is a change in the circumstances of the person with care or in the circumstances of one or more of the absent parents, the provisions of regulation 20 shall apply to 
%each fresh assessment.
a review of each original assessment.  % Words substituted (22.1.96) by SI 1995/3261 reg 29

\amendment{
Words substituted in reg. 22(2) (22.1.96) by the Child Support (Miscellaneous Amendments) (No. 2) Regulations 1995 reg. 29.
}

\subsection[23. Reviews conducted under section 19 of the Act as if a review under section 17 of the Act had been applied for]{Reviews conducted under section 19 of the Act as if a review under section 17 of the Act had been applied for}

23.  The provisions of regulations 20, 21 and 22 shall apply to a review under section 19 of the Act which has been conducted as if an application for a review under section 17 of the Act had been made.

\section[Part VII --- Reviews of a decision by a child support officer]{Part VII\\*Reviews of a decision by a child support officer}

\renewcommand\parthead{--- Part VII}

\subsection[24. Time limits for an application for a review of a decision by a child support officer]{Time limits for an application for a review of a decision by a child support officer}

24.—(1) Subject to paragraph (2), the Secretary of State shall not refer any application for a review under section 18(1), (3) or (4) of the Act or under section 18 of the Act as extended by regulation 9(6) to a child support officer unless that application is received by the Secretary of State within 28 days of the date of notification to the applicant of the decision whose review he seeks.

(2) Where the Secretary of State receives an application for a review under section 18(1), (3) or (4) of the Act or under section 18 of the Act as extended by regulation 9(6) more than 28 days after the date of notification to the applicant of the decision whose review he seeks, the Secretary of State may refer that application to a child support officer if he is satisfied that there was unavoidable delay in making the application.

% Reg 24(3) inserted (5.4.93) by SI 1993/913 reg 10(2)
(3) Where---
\begin{enumerate}\item[]
($a$) a child support officer refuses an application for a maintenance assessment on the grounds of lack of jurisdiction;

($b$) the applicant makes no application at that stage for that refusal to be reviewed under section 18(1)($a$) of the Act but applies to a court for a maintenance order in relation to the children concerned;

($c$) the court refuses to make a maintenance order on the grounds of lack of jurisdiction; and

($d$) the applicant then makes an application for the refusal mentioned in sub-paragraph ($a$) to be reviewed under section 18(1)($a$) of the Act,
\end{enumerate}
the date the applicant is notified of the court’s decision shall, for the purposes of paragraphs (1) and (2), be treated as the date of notification to the applicant of the decision whose review he seeks.

\amendment{
Reg. 24(3) inserted (5.4.93) by the Child Maintenance (Miscellaneous Amendments) Regulations 1993 reg. 10.
}

\subsection[25. Notice of a review of a decision by a child support officer]{Notice of a review of a decision by a child support officer}

25.—(1) Where on an application for a review under section 18 of the Act a child support officer proposes to conduct such a review, he shall give 14 days' notice of the proposed review to the relevant persons.

(2) A child support officer proposing to conduct a review under section 18 of the Act shall—
\begin{enumerate}\item[]
($a$) send to the relevant persons the applicant’s reasons for making the application for the review;

($b$) where a maintenance assessment is in force, send to the relevant persons the information that was included, under the provisons of regulation 10(2), in the notification of that assessment made under the provisions of regulation 10(1);

($c$) invite representations, either in person or in writing, from the relevant persons on any matter relating to the review and set out the provisions of paragraphs (3) to (6) in relation to such representations.
\end{enumerate}

(3) Subject to paragraph (4), where the child support officer conducting the review does not within 14 days of the date on which notice of the review was given receive a request from a relevant person to make representations in person, or receives such a request and arranges for an appointment for such representations to be made but that appointment is not kept, he may complete the review in the absence of such representations from that person.

(4) Where the child support officer conducting the review is satisfied that there was good reason for failure to keep an appointment, he shall provide for a further opportunity for the making of representations by the relevant person concerned before he completes the review.

(5) Where the child support officer conducting the review does not receive written representations from a relevant person within 14 days of the date on which notice of the review was given, he may complete the review in the absence of written representations from that person.

(6) Except where a person gives written permission to the Secretary of State that the information, in relation to him, mentioned in sub-paragraphs ($a$) and ($b$) below may be conveyed to other persons, any document given or sent under the provisions of paragraph (1) or (2) shall not contain—
\begin{enumerate}\item[]
($a$) the address of any person other than the recipient of the document in question (other than the address of the office of the child support officer concerned) or any other information the use of which could reasonably be expected to lead to any such person being located;

($b$) any other information the use of which could reasonably be expected to lead to any person other than a qualifying child or relevant person being identified.
\end{enumerate}

\subsection[26. Procedure on a review of a decision by a child support officer]{Procedure on a review of a decision by a child support officer}

26.—(1) Where the Secretary of State has referred more than one application for a review to a child support officer under section 18 of the Act in relation to the same decision and that child support officer proposes to conduct a review but has not given notice under regulation 25(1), he shall give notice to the relevant persons under regulation 25(1) and shall conduct one review taking account of all the representations made and all the evidence before him.

(2) Where the child support officer conducting a review under section 18 of the Act has given notice under regulation 25(1) and has a further application referred to him by the Secretary of State in relation to the same decision before he has completed his review, he shall notify the person who has made that further application that he is already conducting a review of that decision and that he will take into account the information contained in that application.

%Reg 26A inserted (5.4.93) by SI 1993/913 reg 11
\subsection[26A. Review under section 18 of the Act where parentage is an issue]{Review under section 18 of the Act where parentage is an issue}

26A.  Where an applicant for a review under section 18 of the Act gives as one, but not the only, reason for making the application that---
\begin{enumerate}\item[]
($a$) the decision of which he seeks the review has been made on the basis that a particular person (whether the applicant or some other person) either is, or is not, a parent of a child in question; and

($b$) the decision should not have been made on that basis,
\end{enumerate}
the Secretary of State shall treat the application as two applications, one relating solely to the issue of parentage and the other relating to all other matters giving rise to the application, and shall proceed accordingly.

\amendment{
Reg. 26A inserted (5.4.93) by the Child Support (Miscellaneous Amendments) Regulations 1993 reg. 11.
}

\subsection[27. Review following an application under section 18(1)($b$) of the Act]{Review following an application under section 18(1)($b$) of the Act}

27.  Where a child support officer has completed a review 
of an original assessment as defined in section 17(1) of the Act  % Words inserted (22.1.96) by SI 1995/3261 reg 30
following an application for a review under section 18(1)($b$) of the Act, regulations 20 to 22 shall apply in relation to any fresh assessment following that review.

\amendment{
Words inserted in reg. 27 (22.1.96) by the Child Support (Miscellaneous Amendments) (No. 2) Regulations 1995 reg. 30.
}

%\subsection[28. Reviews conducted under section 19 of the Act as if a review under section 18(1)($b$) of the Act had been applied for]{Reviews conducted under section 19 of the Act as if a review under section 18(1)($b$) of the Act had been applied for}
%
%28.  The provisions of regulation 27 shall apply to a review under section 19 of the Act which has been conducted as if an application for a review under section 18(1)($b$) of the Act had been made.

% Reg 28 substituted (22.1.96) by SI 1995/3261 reg 31
\subsection[28. Reviews conducted under section 19(1)($b$) of the Act]{Reviews conducted under section 19(1)($b$) of the Act}

28.  Where a child support officer has completed a review under section 19(1)($b$) of the Act of an original assessment as defined in section 17(1) of the Act regulations 20 to 22 shall apply in relation to any fresh assessment following that review.

\amendment{
Reg. 28 substituted (22.1.96) by the Child Support (Miscellaneous Amendments) (No. 2) Regulations 1995 reg. 31.
}

\subsection[29. Extension of provisions of section 18(2) of the Act]{Extension of provisions of section 18(2) of the Act}

29.—(1) The provisions of section 18(2) of the Act shall apply where a maintenance assessment has been in force but is no longer in force if the condition specified in paragraph (2) is satisfied.

(2) The condition mentioned in paragraph (1) is that, subject to paragraph (3), the application for a review under section 18(2) of the Act as extended by this regulation is received by the Secretary of State within 28 days of the date of notification to the applicant of the maintenance assessment whose review he seeks.

(3) Where the Secretary of State receives such an application more than 28 days after the date of notification to the applicant of the maintenance assessment whose review he seeks, the Secretary of State may refer that application to a child support officer if he is satisfied that there was unavoidable delay.

\section[Part VIII --- Commencement and termination of maintenance assessments and maintenance periods]{\sloppy Part VIII\\*Commencement and termination of maintenance assessments and maintenance periods}

\renewcommand\parthead{--- Part VIII}

\subsection[30. Effective dates of new maintenance assessments]{Effective dates of new maintenance assessments}

30.—(1) Subject to 
%regulation 8(3) (interim maintenance assessments)
regulations 8C (effective dates of interim maintenance assessments), 30A (effective dates in particular cases), 33(7) (maintenance periods)  % Words substituted (22.1.96) by SI 1995/3261 reg 32(2)
and to regulation 3(5)% 
%and (7)
, (7) and (8)  % Words substituted (18.4.95) by SI 1995/1045 reg 36(2)
of the Maintenance Arrangements and Jurisdiction Regulations (maintenance assessments where court order in force),  %Words inserted (16.2.95) by SI 1995/123 reg 7
 the effective date of a new maintenance assessment following an application under section 4, 6 or 7 of the Act shall be the date determined in accordance with paragraphs (2) to (4).

%(2) Where no maintenance assessment is in force with respect to the person with care and absent parent, the effective date of a new assessment shall be—
%\begin{enumerate}\item[]
%($a$) the date a maintenance enquiry form is given or sent to an absent parent in a case where the application for a maintenance assessment is made by a person with care or by a child under section 7 of the Act; or
%
%($b$) the date an effective maintenance application form is received by the Secretary of State in a case where the application for a maintenance assessment—
%\begin{enumerate}\item[]
%(i) is made by an absent parent; or
%
%(ii) is an application in relation to which the provisions of regulation 3 have been applied.
%\end{enumerate}
%\end{enumerate}

% Reg 30(2) substituted (18.4.95) by SI 1995/1045 reg 36(3)
(2) Where no maintenance assessment made in accordance with Part I of Schedule 1 to the Act is in force with respect to the person with care and absent parent, the effective date of a new assessment shall be—
\begin{enumerate}\item[]
($a$) in a case where the application for a maintenance assessment is made by a person with care or by a child under section 7 of the Act—
\begin{enumerate}\item[]
(i) eight weeks from the date on which a maintenance enquiry form has been given or sent to an absent parent, where such date is on or after 18th April 1995 and where within four weeks of the date that form was given or sent, it has been returned by the absent parent to the Secretary of State and it contains his name, address and written confirmation that he is the parent of the child or children in respect of whom the application for a maintenance assessment was made;

(ii) in all other circumstances, the date a maintenance enquiry form is given or sent to an absent parent;
\end{enumerate}

($b$) in a case where the application for a maintenance assessment is made by an absent parent—
\begin{enumerate}\item[]
(i) eight weeks from the date on which an application made by an absent parent was received by the Secretary of State, where such date is on or after 18 April 1995 and where, on, or within four weeks of, the date of receipt of that maintenance application, the absent parent has provided his name, address and written confirmation that he is the parent of the child or children in respect of whom the application was made;

(ii) in all other circumstances, the date an effective maintenance application form is received by the Secretary of State;
\end{enumerate}

% Reg 30(2)($c$) inserted (22.1.96) by SI 1995/3261 reg 32(3)
($c$) in a case where the application for a maintenance assessment is an application in relation to which the provisions of regulation 3 have been applied, the date an effective maintenance application form is received by the Secretary of State.
\end{enumerate}

% Reg 30(2A) inserted (18.4.95) by SI 1995/1045 reg 36(4)
(2A) Where a child support officer is satisfied that there was unavoidable delay by the absent parent in providing the information listed in sub-paragraphs ($a$)(i) or ($b$)(i) of paragraph (2) within the time specified in those sub-paragraphs, he may apply the provisions of those sub-paragraphs for the purpose of setting the effective date of a maintenance assessment even though that information was not provided within the time specified in those sub-paragraphs.

(3) The provisions of regulation 1(6)($b$) shall not apply to paragraph (2)($a$).

(4) Where a child support officer is satisfied that an absent parent has deliberately avoided receipt of a maintenance enquiry form, he may determine the date on which the form would have been given or sent but for such avoidance, and that date shall be the relevant date for the purposes of paragraph (2)($a$).

\amendment{
Words inserted in reg. 30(1) (16.2.95) by the Child Support (Miscellaneous Amendments) Regulations 1995 reg. 7.

Words substituted in reg. 30(1), reg. 30(2A) inserted and reg. 30(2) substituted (18.4.95) by the Child Support and Income Support (Amendment) Regulations 1995 reg. 36.

Words substituted in reg. 30(1) and reg. 30(2)($c$) inserted (22.1.96) by the Child Support (Miscellaneous Amendments) (No. 2) Regulations 1995 reg. 32.
}

% Reg 30A inserted (22.1.96) by SI 1995/3261 reg 33
\subsection[30A. Effective dates of new maintenance assessments in particular cases]{Effective dates of new maintenance assessments in particular cases}

30A.—(1) Subject to regulation 33(7), where a new maintenance assessment is made in accordance with Part I of Schedule 1 to the Act following an interim maintenance assessment which has ceased to have effect in the circumstances set out in regulation 8D(6), the effective date of that maintenance assessment shall be the date upon which that interim maintenance assessment ceased to have effect in accordance with that regulation.

%(2) Where a child support officer receives the information or evidence to enable him to make a maintenance assessment calculated in accordance with Part I of Schedule 1 to the Act for a period prior to the date upon which an interim maintenance assessment has ceased to have effect in the circumstances set out in regulation 8D(6), that maintenance assessment shall, subject to regulation 33(7), have effect for the period from the date set by regulation 3(7) of the Maintenance Arrangements and Jurisdiction Regulations or regulation 30(2)($a$) or ($b$), as the case may be, to the effective date of the maintenance assessment referred to in paragraph (1).

% Reg 30A(2) substituted (13.1.97) by SI 1996/3196 reg 8(2)
(2) Where a child support officer receives the information or evidence to enable him to make a maintenance assessment, calculated in accordance with the provisions of Part I of Schedule 1 to the Act, for the period from the date set by regulation 3(7) of the Maintenance Arrangements and Jurisdiction Regulations or regulation 30(2)($a$) or ($b$), as the case may be, to the effective date of the maintenance assessment referred to in paragraph (1), the maintenance assessment first referred to in this paragraph shall, subject to regulation 33(7), have effect for that period.

% Reg 30A(3)--(7) inserted (13.1.97) by SI 1996/3195 reg 8(3)
(3) The effective date of a new maintenance assessment made in respect of a person with care and an absent parent shall, where the circumstances set out in paragraph (4) apply, be the first day of the first maintenance period after the 
%child support officer 
Secretary of State  % Words substituted (19.1.98) by SI 1998/58 reg 40
has received the information or evidence referred to in paragraph (4)($c$) or 13th January 1997, whichever is the later.

(4) The circumstances referred to in paragraph (3) are where—
\begin{enumerate}\item[]
($a$) paragraphs (1) and (2) do not apply to that person with care and that absent parent;

($b$) no maintenance assessment made in accordance with the provisions of Part I of Schedule 1 to the Act is in force in relation to that person with care and that absent parent; and

($c$) on or after 13th January 1997, a child support officer has sufficient information or evidence to enable him to make a new maintenance assessment, calculated in accordance with the provisions of Part I of Schedule 1 to the Act, in relation to that person with care and that absent parent but in respect only of a period beginning after the effective date applicable in their case by virtue of regulation 30(2).
\end{enumerate}

(5) Where the information or evidence referred to in paragraph (3) is that there has been an award of income support or an income-based jobseeker’s allowance, the Secretary of State shall be treated as having received the information or evidence which enables a child support officer to make the assessment referred to in that paragraph on the first day in respect of which income support or an income-based jobseeker’s allowance was payable under that award.

(6) Where, in a case falling within paragraph (3), a child support officer receives the information or evidence to enable him to make a maintenance assessment calculated in accordance with the provisions of Part I of Schedule 1 to the Act, for the period from the effective date applicable to that case under regulation 30(2)($a$) or ($b$), as the case may be, to the effective date of the assessment referred to in paragraph (3), the maintenance assessment first referred to in this paragraph shall have effect for that period.

(7) Paragraphs (3) to (6) shall not apply where a case falls within regulation 33(7), or regulation 3 of the Maintenance Arrangements and Jurisdiction Regulations (relationship between maintenance assessments and certain court orders).

\amendment{
Reg. 30A inserted (22.1.96) by the Child Support (Miscellaneous Amendments) (No. 2) Regulations 1995 reg. 33.

Reg. 30A(2) substituted and reg. 30A(3)--(7) inserted (13.1.97) by the Child Support (Miscellaneous Amendments) (No. 2) Regulations 1996 reg. 8.

Under the Child Support (Miscellaneous Amendments) (No. 2) Regulations 1996 reg. 16, maintenance assessments in force on 13.1.97 shall not be reviewed solely to give effect to reg. 30A(2) as amended or to reg. 30A(3)--(7), and the effective date of any fresh assessment affected by the amendment to reg. 30A(2) or by reg. 30A(3)--(7) shall not be earlier than the first day of the first maintenance period which commences on or after 13.1.97.

Words substituted in reg. 30A(3) (19.1.98) by the Child Support (Miscellaneous Amendments) Regulations 1998 reg. 40.
}

%\subsection[31. Effective dates of maintenance assessments following a review under sections 16 to 19 of the Act]{Effective dates of maintenance assessments following a review under sections 16 to 19 of the Act}
%
%31.—(1) Where a fresh maintenance assessment is made following a review under section 16 of the Act, the effective date of that assessment shall be 
%%52 weeks 
%104 weeks  % Words substituted (18.4.95) by SI 1995/1045 reg 37(2)
%after the effective date of the previous assessment
%disregarding any previous assessment made following a review made under section 18 or 19 of the Act. % Words inserted (7.2.94) by SI 1994/227 reg 2(5)
%
%(2) Subject to paragraph (4), where an application is made under section 17 of the Act for a review of a maintenance assessment in force, and a fresh maintenance assessment is made in accordance with the provisions of regulation 20, 21 or 22, the effective date of that assessment shall be the first day of the maintenance period in which the application is received.
%
%(3) Where a case falls within regulation 18(1), the effective date of the fresh assessment shall be the first day of the maintenance period in which the assessment is made.
%
%(4) Where an application is made under section 17 of the Act for a review of a maintenance assessment in force following the death of a qualifying child and a fresh maintenance assessment is made in accordance with the provisions of regulation 20, 21 or 22, the effective date of that assessment shall be the first day of the maintenance period during the course of which that child died.
%
%(5) Where, following a review under section 18(1)($a$) of the Act, a maintenance assessment is made following a refusal to make a maintenance assessment, the effective date of that assessment shall be the effective date of the assessment that would have been made if the application for a maintenance assessment had not been refused.
%
%(6) 
%%Subject to paragraphs (7), (10) and (11), 
%Subject to paragraphs (6A), (6B), (6C), (9) and (10),  % Words substituted (18.4.95) by SI 1995/1045 reg 37(3)
%where an application is made under section 18(2) of the Act for a review of a maintenance assessment in force, the effective date of a fresh assessment (if one is made) following such a review shall be—
%\begin{enumerate}\item[]
%($a$) where the application is received by the Secretary of State within 28 days of the date of notification of that assessment, or on a later date but the Secretary of State is satisfied that there was unavoidable delay, the effective date as determined on the review;
%
%($b$) subject to sub-paragraph ($a$), where the application is received by the Secretary of State later than 28 days after the date of notification of that assessment, the first day of the maintenance period in which the application is received.
%\end{enumerate}
%
%% Reg 31(6A)--(6C) inserted (18.4.95) by SI 1995/1045 reg 37(4)
%(6A) Subject to paragraph (6C), where an application is made under section 18(2) of the Act for a review of a maintenance assessment in force following notification being given to the relevant person that the child support officer does not propose to review the assessment in consequence of the coming into force of the provisions mentioned in paragraph (6B), the effective date of a fresh assessment (if one is made) following such a review shall be—
%\begin{enumerate}\item[]
%($a$) where the application is received within 28 days of the Secretary of State notifying the relevant person of the child support officer’s decision, or on a later date where the Secretary of State is satisfied that there was unavoidable delay, the effective date as determined on the review;
%
%($b$) subject to sub-paragraph ($a$), where the application is received by the Secretary of State later than 28 days after the date of the notification of the child support officer’s decision, the first day of the maintenance period in which the application is received.
%\end{enumerate}
%
%(6B) Paragraph (6A) applies to the following provisions of the Income Support and Child Support (Amendment) Regulations 1995—
%\begin{enumerate}\item[]
%($a$) regulation 44(2);
%
%($b$) regulation 45;
%
%($c$) regulation 46(2)($d$) and ($e$);
%
%($d$) regulation 51.
%\end{enumerate}
%
%(6C) Where the application made under section 18(2) is made following notification being given to the relevant person that the child support officer has determined that the amount to be allowed in the computation of the relevant person’s exempt income in accordance with Schedule 3A to the Child Support (Maintenance Assessments and Special Cases) Regulations is nil by reason of the failure of the relevant person to furnish within a reasonable time the evidence required by paragraph 2 of that Schedule—
%\begin{enumerate}\item[]
%($a$) where the Secretary of State is satisfied that there was good cause for the delay in furnishing the evidence the effective date of any assessment made in consequence of the review shall be the effective date which would have been applicable to the assessment had the evidence been furnished timeously;
%
%($b$) where the Secretary of State is not satisfied that there was good cause for the delay, the effective date of any revised assessment shall be the first day of the maintenance period in which the relevant person provides that evidence.
%\end{enumerate}
%
%(7) Where an application is made under section 18(1)($b$) of the Act for a review of a refusal of an application under section 17 of the Act for the review of a maintenance assessment which is in force, the effective date of a fresh maintenance assessment (if one is made) shall be the date determined under paragraph (2).
%
%(8) Where, following a review under section 18(3) of the Act, a cancelled maintenance assessment is reinstated, the effective date of the reinstated assessment shall be the date on which the cancelled assessment ceased to have effect.
%
%(9) Where there has been a misrepresentation or failure to disclose a material fact on the part of the person with care or absent parent in connection with an application for a maintenance assessment under the Act, or a review under section 16 or 17 of the Act, and that misrepresentation or failure has resulted in an incorrect assessment or a series of incorrect assessments, the effective date of a fresh assessment (or of a fresh assessment in relation to the earliest relevant period) following discovery of the misrepresentation or failure shall be the effective date of the incorrect assessment or the first incorrect assessment, as the case may be.
%
%(10) Where a fresh maintenance assessment is made on a review under section 18 or 19 of the Act by reason of an assessment having been made in ignorance of a material fact or having been based on a mistake as to a material fact and that ignorance or mistake, as the case may be, is attributable to an operational or administrative error on the part of the Secretary of State or of a child support officer, the effective date of that fresh assessment shall be the effective date of the assessment that has been reviewed.
%
%(11) Subject to paragraphs (9), (10), (12), (13) and (14), where a fresh maintenance assessment is made under section 19 of the Act, the effective date of the assessment shall be the first day of the maintenance period in which the assessment is made.
%
%(12) Where a fresh maintenance assessment is made under section 19 of the Act following the death of a qualifying child, the effective date of that assessment shall be the first day of the maintenance period during which that child died.
%
%(13) Where a child support officer on a review under section 18 or 19 of the Act is satisfied that a maintenance assessment which is or has been in force is defective by reason of a mistake as to the effective date of that assessment, the effective date of a fresh assessment shall be that determined in accordance with 
%%regulation 30 or in accordance with paragraphs (1) to (12), as the case may be.
%paragraphs (1) to (12), regulation 8(3), regulation 30, or in accordance with regulation 3(5) or (7) of the Maintenance Arrangements and Jurisdiction Regulations, as the case may be.  % Words substituted (16.2.95) by SI 1995/123 reg 8
%
%%(14) Where a child support officer on a review under section 19 of the Act is satisfied that if an application were to be made under section 18 of the Act it would be appropriate to make a fresh maintenance assessment, and does so, the effective date of that fresh assessment shall be determined in accordance with paragraphs (5) to (8).
%
%% Reg 31(14) substituted (18.4.95) by SI 1995/1045 reg 37(5)
%(14) Where a child support officer following a review under section 19(1) of the Act makes a fresh maintenance assessment or on a review under section 19(2) of the Act is satisfied that if an application were to be made under section 18 of the Act it would be appropriate to make a fresh maintenance assessment, and does so, the effective date of that fresh assessment shall—
%\begin{enumerate}\item[]
%($a$) be determined in accordance with paragraph (5) or (8); or
%
%($b$) be determined in accordance with paragraph (7), subject to the modification that that paragraph shall have effect as if for “the date determined under paragraph (2)” there is substituted “the first day of the maintenance period in which the child support officer is first satisfied that a review under section 19(1) of the Act should be undertaken or the first day of the maintenance period following 18th April 1995, whichever is the later”; or
%
%($c$) (subject to paragraphs (9) or (10)), be the first day of the maintenance period in which the child support officer is satisfied that a review under section 19 of the Act should be undertaken or the first day of the maintenance period following 18th April 1995 whichever is the later.
%\end{enumerate}
%
%\amendment{
%Words inserted in reg. 31(1) (7.2.94) by the Child Support (Miscellaneous Amendments and Transitional Provisions) Regulations 1994 reg. 2(5) (subject to transitional provisions in reg. 12).
%
%Words substituted in reg. 31(13) (16.2.95) by the Child Support (Miscellaneous Amendments) Regulations 1995 reg. 8.
%
%Words substituted in reg. 31(1), (6), reg. 31(6A)--(6C) inserted and reg. 31(14) substituted (18.4.95) by the Child Support and Income Support (Amendment) Regulations 1995 reg. 37.
%}

% Regs 31--31C substituted for reg 31 (22.1.96) by SI 1995/3261 reg 34
\subsection[31. Effective dates of maintenance assessments following a review under section 16 or 17 of the Act]{Effective dates of maintenance assessments following a review under section 16 or 17 of the Act}

31.—%(1) Subject to paragraph (2), where a fresh maintenance assessment is made following a review under section 16 of the Act, the effective date of that assessment shall be 104 weeks after the effective date of the previous assessment disregarding any previous assessment made following a review made under section 17 of the Act, where after 22nd January 1996 a child support officer decided, in accordance with section 17(3) of the Act, to proceed with a review, or under section, 18 or 19 of the Act or any interim maintenance assessment made under section 12(1)($b$) or ($c$) of the Act.
%
% Reg 31(1) substituted (7.10.96) by SI 1996/1945 reg 10
(1) Subject to paragraph (2), where a fresh maintenance assessment is made following a review under section 16 of the Act, the effective date of that assessment shall be 104 weeks after the effective date of the previous assessment disregarding—
\begin{enumerate}\item[]
($a$) any previous assessment made following a review under section 17 of the Act, where, after 22nd January 1996, a child support officer decided, in accordance with section 17(3) of the Act, to proceed with a review;

($b$) any previous assessment made following a review under section 18 or 19 of the Act;

($c$) any interim maintenance assessment made under section 12(1A)($b$) or ($c$) of the Act, except a Category B interim maintenance assessment made under paragraph ($b$) or ($c$) of that section where that interim maintenance assessment is the assessment being reviewed under section 16 of the Act.
\end{enumerate}

(2) Where a fresh maintenance assessment is made following a review under section 16 of the Act in the circumstances set out in regulation 18, the effective date of that fresh maintenance assessment shall be the date determined under paragraph (3).

(3) Subject to paragraphs (4), (5) and (6), where an application is made under section 17 of the Act for a review of a maintenance assessment in force, and a fresh maintenance assessment is made in accordance with the provisions of regulation 20, 21 or 22, the effective date of that assessment shall be the first day of the maintenance period in which the application is received.

(4) Where an application is made under section 17 of the Act for a review of a maintenance assessment in force following the death of a qualifying child and a fresh maintenance assessment is made in accordance with the provisions of regulation 20, 21 or 22, the effective date of that assessment shall be the first day of the maintenance period during the course of which that child died.

(5) Where a child support officer has carried out a review of an original assessment under section 17(4A) of the Act, the effective date of any fresh assessment made under section 17(6) of the Act shall be the date determined under paragraph (3).

(6) Where a fresh maintenance assessment is made under section 17(7) of the Act following a review of a subsequent assessment, the effective date of that fresh assessment shall be the effective date of that subsequent assessment.

\amendment{
Reg. 31 substituted (22.1.96) by the Child Support (Miscellaneous Amendments) (No. 2) Regulations 1995 reg. 34.

Reg. 31(1) substituted (7.10.96) by the Child Support (Miscellaneous Amendments) Regulations 1996 reg. 10.
}

\subsection[31A. Effective dates of maintenance assessments following a review under section 18 of the Act]{Effective dates of maintenance assessments following a review under section 18 of the Act}

31A.—(1) Where, following a review under section 18(1)($a$) of the Act, a maintenance assessment is made following a refusal to make a maintenance assessment, the effective date of that assessment shall be the effective date of the assessment that would have been made if the application for a maintenance assessment had not been refused.

(2) Subject to paragraphs (3) to (6) and to regulation 31C, where an application is made under section 18(2) of the Act for a review of a maintenance assessment in force at the time of that application, the effective date of a fresh assessment (if one is made) following such a review shall be—
\begin{enumerate}\item[]
($a$) where the application is received by the Secretary of State within 28 days of the date of notification of that assessment, or on a later date but the Secretary of State is satisfied that there was unavoidable delay, the effective date as determined by the child support officer dealing with the review;

($b$) subject to sub-paragraph ($a$), where the application is received by the Secretary of State later than 28 days after the date of notification of that assessment, the first day of the maintenance period in which the application is received.
\end{enumerate}

(3) Subject to paragraph (5), where an application is made under section 18(2) of the Act for a review of a maintenance assessment in force following notification being given to the relevant person that the child support officer does not propose to review the assessment in consequence of the coming into force of the provisions mentioned in paragraph (4), the effective date of a fresh assessment (if one is made) following such a review shall be—
\begin{enumerate}\item[]
($a$) where the application is received within 28 days of the Secretary of State notifying the relevant person of the child support officer’s decision, or on a later date where the Secretary of State is satisfied that there was unavoidable delay, the effective date as determined by the child support officer dealing with the review;

($b$) subject to sub-paragraph ($a$), where the application is received by the Secretary of State later than 28 days after the date of the notification of the child support officer’s decision, the first day of the maintenance period in which the application is received.
\end{enumerate}

(4) Paragraph (3) applies to the following provisions of the Child Support and Income Support (Amendment) Regulations 1995—
\begin{enumerate}\item[]
($a$) regulation 44(2);

($b$) regulation 45;

($c$) regulation 46(2)($d$) and ($e$);

($d$) regulation 51.
\end{enumerate}

(5) Where the application made under section 18(2) of the Act is made following notification being given to the relevant person that the child support officer has determined that the amount to be allowed in the computation of the relevant person’s exempt income in accordance with Schedule 3A to the Maintenance Assessments and Special Cases Regulations is nil by reason of the failure of the relevant person to furnish within a reasonable time the evidence required by paragraph 2 of that Schedule—
\begin{enumerate}\item[]
($a$) where the Secretary of State is satisfied that there was good cause for the delay in furnishing the evidence the effective date of any assessment made in consequence of the review shall be the effective date which would have been applicable to the assessment had the evidence been furnished timeously;

($b$) where the Secretary of State is not satisfied that there was good cause for the delay, the effective date of any revised assessment shall be the first day of the maintenance period in which the relevant person provides that evidence.
\end{enumerate}

(6) The effective date of any fresh maintenance assessment, made following a review under section 18(6A) of the Act of a maintenance assessment made after the original assessment, shall be the effective date of the maintenance assessment which has been reviewed.

(7) Where, an application is made under section 18(1)($b$) of the Act, for a review of a refusal of an application under section 17 of the Act for the review of a maintenance assessment, the effective date of a fresh maintenance assessment (if one is made) shall be the date determined under regulation 31(3).

(8) Where, following a review under section 18(3) of the Act, a cancelled maintenance assessment is reinstated, the effective date of the reinstated assessment shall be the date on which the cancelled assessment ceased to have effect.

\amendment{
Reg. 31A substituted for reg. 31 (22.1.96) by the Child Support (Miscellaneous Amendments) (No. 2) Regulations 1995 reg. 34.
}

\subsection[31B. Effective dates of maintenance assessments following a review under section 19 of the Act]{Effective dates of maintenance assessments following a review under section 19 of the Act}

31B.—(1) Where a maintenance assessment is made following a review under section 19(1)($a$) of the Act of a refusal to make a maintenance assessment, the effective date of that maintenance assessment shall be the date determined under regulation 31A(1).

(2) Where a fresh maintenance assessment is made, following a review under section 19(1)($b$) of the Act of a refusal of an application under section 17 of the Act for review of a maintenance assessment, the effective date of that fresh maintenance assessment shall be the date determined under regulation 31(3) to (6).

(3) Subject to paragraph (5) and regulation 31C, where a child support officer has carried out a review of a maintenance assessment on the grounds set out in section 19(2) of the Act, the effective date of any fresh maintenance assessment made following that review shall be the effective date as determined by the child support officer dealing with the review.

(4) Subject to paragraph (5) and regulation 31C, where a child support officer has carried out a review of a maintenance assessment on the grounds set out in section 19(6) of the Act, the effective date of any fresh assessment made following such review shall be the first day of the maintenance period in which the child support officer suspected that he might be required to make one or more fresh maintenance assessments if an application under section 17 of the Act were made.

(5) Where a fresh maintenance assessment is made under section 19 of the Act following the death of a qualifying child, the effective date of that assessment shall be the first day of the maintenance period during which that child died.

\amendment{
Reg. 31B substituted for reg. 31 (22.1.96) by the Child Support (Miscellaneous Amendments) (No. 2) Regulations 1995 reg. 34.
}

\subsection[31C. Provisions as to effective dates of maintenance assessments in specific cases]{Provisions as to effective dates of maintenance assessments in specific cases}

31C.—(1) Where there has been a misrepresentation or failure to disclose a material fact on the part of the person with care or absent parent in connection with an application for a maintenance assessment under the Act, a review under section 16 of the Act, or with information or evidence requested by a child support officer on a review under section 17, 18 or 19 of the Act and that misrepresentation or failure has resulted in an incorrect assessment or a series of incorrect assessments, the effective date of a fresh assessment (or of a fresh assessment in relation to the earliest relevant period) following discovery of the misrepresentation or failure shall be the effective date of the incorrect assessment or the first incorrect assessment, as the case may be.

(2) Where a fresh maintenance assessment is made on a review under section 18 or 19 of the Act by reason of an assessment having been made in ignorance of a material fact or having been based on a mistake as to a material fact and that ignorance or mistake, as the case may be, is attributable to an operational or administrative error on the part of the Secretary of State or of a child support officer, the effective date of that assessment shall be the effective date of the assessment that has been reviewed.

(3) Where a child support officer on a review under section 18 or 19 of the Act is satisfied that a maintenance assessment which is or has been in force is defective by reason of a mistake as to the effective date of that assessment, the effective date of a fresh assessment shall be 
%that determined in accordance with paragraph (1) or (2), regulations 8C(1), 30 to 31B, 33(7), or in accordance with regulation 3(5), (7) or (8) of the Maintenance Arrangements and Jurisdiction Regulations, as the case may be.
the correct effective date applicable to the maintenance assessment which is being reviewed.  % Words substituted (7.10.96) by SI 1996/1945 reg 11

\amendment{
Reg. 31C substituted for reg. 31 (22.1.96) by the Child Support (Miscellaneous Amendments) (No. 2) Regulations 1995 reg. 34.

Words substituted in reg. 31C(3) (7.10.96) by the Child Support (Miscellaneous Amendments) Regulations 1996 reg. 11.
}

\subsection[32. Cancellation of a maintenance assessment]{Cancellation of a maintenance assessment}

32.  Where a child support officer cancels a maintenance assessment under paragraph 16(2) or (3) of Schedule 1 to the Act, the assessment shall cease to have effect from the date of receipt of the request for the cancellation of the assessment or from such later date as the child support officer may determine.

%Reg 32A inserted (5.4.93) by SI 1993/913 reg 12
\subsection[32A. Cancellation of maintenance assessments made under section 7 of the Act where the child is no longer habitually resident in Scotland]{Cancellation of maintenance assessments made under section 7 of the Act where the child is no longer habitually resident in Scotland}

32A.—(1) Where a maintenance assessment made in response to an application by a child under section 7 of the Act is in force and that child ceases to be habitually resident in Scotland, a child support officer shall cancel that assessment.

(2) In any case where paragraph (1) applies, the assessment shall cease to have effect from the date that the child support officer determines is the date on which the child concerned ceased to be habitually resident in Scotland.

\amendment{
Reg. 32A inserted (5.4.93) by the Child Support (Miscellaneous Amendments) Regulations 1993 reg. 12.
}

% Reg 32B inserted (22.1.96) by SI 1995/3261 reg 35, moved after reg 32A (22.1.96) by SI 1995/3265 reg 2
\subsection[32B. Notification of intention to cancel a maintenance assessment under paragraph 16(4A) of Schedule 1 to the Act]{Notification of intention to cancel a maintenance assessment under paragraph 16(4A) of Schedule 1 to the Act}

32B.—(1) A child support officer shall, if it is reasonably practicable to do so, give written notice to the relevant persons of his intention to cancel a maintenance assessment under paragraph 16(4A) of Schedule 1 to the Act.

(2) Where a notice under paragraph (1) has been given, a child support officer shall not cancel that maintenance assessment before the end of a period of 14 days commencing with the date that notice was given or sent.

\amendment{
Reg. 32B inserted (22.1.96) by the Child Support (Miscellaneous Amendments) (No. 2) Regulations 1995 reg. 35.

Reg. 32B moved to appear after reg. 32A (22.1.96) by the Child Support (Miscellaneous Amendments) (No. 3) Regulations 1995 reg. 2.
}

\subsection[33. Maintenance periods]{Maintenance periods}

33.—(1) The child support maintenance payable under a maintenance assessment shall be calculated at a weekly rate and be in respect of successive maintenance periods, each such period being a period of 7 days.

(2) Subject to paragraph (6), the first maintenance period shall commence on the effective date of the first maintenance assessment, and each succeeding maintenance period shall commence on the day immediately following the last day of the preceding maintenance period.

(3) The maintenance periods in relation to a fresh maintenance assessment following a review under section 16, 17, 18 or 19 of the Act shall coincide with the maintenance periods in relation to the earlier assessment, had it continued in force, and the first maintenance period in relation to a fresh assessment shall commence on the day following the last day of the last maintenance period in relation to the earlier assessment.

(4) The amount of child support maintenance payable in respect of a maintenance period which includes the effective date of a fresh maintenance assessment shall be the amount of maintenance payable under that fresh assessment.

(5) The amount of child support maintenance payable in respect of a maintenance period during the course of which a cancelled maintenance assessment ceases to have effect shall be the amount of maintenance payable under that assessment.

%(6) Where a case is to be treated as a special case for the purposes of the Act by virtue of regulation 22 of the Maintenance Assessments and Special Cases Regulations (multiple applications relating to an absent parent) and an application is made by a person with care in relation to an absent parent where there is already a maintenance assessment in force in relation to that absent parent and a different person with care, the maintenance periods in relation to an assessment made in response to that application shall coincide with the maintenance periods in relation to the earlier maintenance assessment, 
%except where regulation 3(7) of the Maintenance Arrangements and Jurisdiction Regulations or paragraph (8) applies,  % Words inserted (22.1.96) by SI 1995/3261 reg 36(2)
%and the first such period shall commence not later than 7 days after the date of notification to the relevant persons of the later maintenance assessment.

% Reg 33(6) substituted (5.8.96) by SI 1996/1945 reg 11(2)
(6) Where a case is to be treated as a special case for the purposes of the Act by virtue of regulation 22 of the Maintenance Assessments and Special Cases Regulations (multiple applications relating to an absent parent) and an application is made by a person with care in relation to an absent parent where—
\begin{enumerate}\item[]
($a$) there is already a maintenance assessment in force in relation to that absent parent and a different person with care; or

($b$) sub-paragraph ($a$) does not apply, but before a maintenance assessment is made in relation to that application, a maintenance assessment is made in relation to that absent parent and a different person with care,
\end{enumerate}
the maintenance periods in relation to an assessment made in response to that application shall coincide with the maintenance periods in relation to the earlier maintenance assessment, except where regulation 3(7) of the Maintenance Arrangements and Jurisdiction Regulations or paragraph (8) applies, and the first such period shall, subject to paragraph (9), commence not later than 7 days after the date of notification to the relevant persons of the later maintenance assessment.

% Reg 33(7), (8) inserted (22.1.96) by SI 1995/3261 reg 36(3)
(7) Subject to regulation 3(7) of the Maintenance Arrangements and Jurisdiction Regulations and to paragraph (8), the effective date of a maintenance assessment made in response to an application falling within paragraph (6) shall be the date upon which the first maintenance period in relation to that application commences in accordance with that paragraph.

(8) The first maintenance period in relation to a maintenance assessment which is made in response to an application falling within paragraph (6) and which immediately follows an interim maintenance assessment shall commence on the effective date of that interim maintenance assessment or 22nd January 1996 whichever is the later, and the effective date of that maintenance assessment shall be the date upon which that first maintenance period commences.

% Reg 33(9) added (5.8.96) by SI 1996/1945 reg 11(3)
(9) Where the case is one to which, if paragraphs (6) and (7) did not apply, regulation 30(2)($a$)(i) or ($b$)(i) would apply, and the first maintenance period would, under the provisions of paragraph (6), commence during the 8 week period referred to in sub-paragraph ($a$) or ($b$) of that regulation, the first maintenance period shall commence not later than 7 days after the expiry of that period of 8 weeks.

\amendment{
Words inserted in reg. 33(6) and reg. 33(7), (8) inserted (22.1.96) by the Child Support (Miscellaneous Amendments) (No. 2) Regulations 1995 reg. 36.

Reg. 33(9) added and reg. 33(6) substituted (5.8.96) by the Child Support (Miscellaneous Amendments) Regulations 1996 reg. 12.

Under the Child Support (Miscellaneous Amendments) Regulations 1996 reg. 25(1), reg. 33(9) does not apply to any applications made prior to 5.8.96 and the previous version of reg. 33(6) applies:
\begin{quotation}
(6) Where a case is to be treated as a special case for the purposes of the Act by virtue of regulation 22 of the Maintenance Assessments and Special Cases Regulations (multiple applications relating to an absent parent) and an application is made by a person with care in relation to an absent parent where there is already a maintenance assessment in force in relation to that absent parent and a different person with care, the maintenance periods in relation to an assessment made in response to that application shall coincide with the maintenance periods in relation to the earlier maintenance assessment, 
except where regulation 3(7) of the Maintenance Arrangements and Jurisdiction Regulations or paragraph (8) applies,  % Words inserted (22.1.96) by SI 1995/3261 reg 36(2)
and the first such period shall commence not later than 7 days after the date of notification to the relevant persons of the later maintenance assessment.
\end{quotation}
}

\section[Part IX --- Reduced benefit directions]{Part IX\\*Reduced benefit directions}

\renewcommand\parthead{--- Part IX}

\subsection[34. Prescription of disability working allowance for the purposes of section 6 of the Act]{Prescription of disability working allowance for the purposes of section 6 of the Act}

34.  Disability working allowance shall be a benefit of a prescribed kind for the purposes of section 6 of the Act.

\subsection[35. Periods for compliance with obligations imposed by section 6 of the Act]{Periods for compliance with obligations imposed by section 6 of the Act}

35.—(1) Where the Secretary of State considers that a parent has failed to comply with an obligation imposed by section 6 of the Act he shall serve written notice on that parent that, unless she complies with that obligation, he intends to refer the case to a child support officer for the child support officer to take action under section 46 of the Act if the child support officer considers such action to be appropriate.

%(2) The Secretary of State shall not refer a case to a child support officer prior to the expiry of a period of 6 weeks from the date he serves notice under paragraph (1) on the parent in question, and the notice shall contain a statement to that effect.

% Reg 35(2) substituted (7.10.96) by SI 1996/1945 reg 13(2)
(2) The Secretary of State shall not refer a case to a child support officer prior to the expiry of a period of—
\begin{enumerate}\item[]
($a$) 2 weeks from the date he serves notice under paragraph (1) on the parent in question; or

($b$) 6 weeks from that date, where, before the expiry of 2 weeks from service of that notice, he has received from the parent in question in writing her reasons why she believes that if she were to be required to comply with an obligation imposed by section 6 of the Act, there would be a risk, as a result of that compliance, of her or any child or children living with her suffering harm or undue distress,
\end{enumerate}
and the notice shall contain a statement setting out the provisions of sub-paragraphs ($a$) and ($b$).

(3) Where 
%the Secretary of State refers a case to a child support officer and the 
a  % Words substituted (7.10.96) by SI 1996/1945 reg 13(3)
child support officer serves written notice on a parent under section 46(2) of the Act, the period to be specified in that notice shall be 14 days.

\amendment{
Words substituted in reg. 35(3) and reg. 35(2) substituted (7.10.96) by the Child Support (Miscellaneous Amendments) Regulations 1996 reg. 13.

Under the Child Support (Miscellaneous Amendments) Regulations 1996 reg. 25(2) the previous versions of reg. 35(2), (3) apply to any case where the failure to comply referred to in reg. 35(1) arose prior to 7.10.96:
\begin{quotation}
(2) The Secretary of State shall not refer a case to a child support officer prior to the expiry of a period of 6 weeks from the date he serves notice under paragraph (1) on the parent in question, and the notice shall contain a statement to that effect.

(3) Where 
the Secretary of State refers a case to a child support officer and the 
child support officer serves written notice on a parent under section 46(2) of the Act, the period to be specified in that notice shall be 14 days.
\end{quotation}

}

% Reg 35A inserted (22.1.96) by SI 1995/3261 reg 37
\subsection[35A. Circumstances in which a reduced benefit direction shall not be given]{Circumstances in which a reduced benefit direction shall not be given}

35A.  A child support officer shall not after 22nd January 1996 give a reduced benefit direction where—
\begin{enumerate}\item[]
($a$) income support is paid to or in respect of the parent in question and the applicable amount of the claimant for income support includes one or more of the amounts set out in paragraph 15(3), (4) or (6) of Part IV of Schedule 2 to the Income Support (General) Regulations 1987\footnote{\frenchspacing S.I. 1987/1967. Part IV of Schedule 2 was substituted by S.I. 1995/559.}; or

($aa$) income-based jobseeker’s allowance is paid to or in respect of the parent in question and the applicable amount of the claimant for income-based jobseeker’s allowance includes one or more of the amounts set out in paragraph 20(4), (5) or (7) of Schedule 1 to the Jobseeker’s Allowance Regulations; or

($b$) an amount equal to one or more of the amounts specified in sub-paragraph ($a$) is included, by virtue of regulation 9 of the Maintenance Assessments and Special Cases Regulations, in the exempt income of the parent in question and family credit or disability working allowance is paid to or in respect of that parent.
\end{enumerate}

\amendment{
Reg. 35A inserted (22.1.96) by the Child Support (Miscellaneous Amendments) (No. 2) Regulations 1995 reg. 37.

Reg. 35A(aa) inserted (7.10.96) by the Social Security and Child Support (Jobseeker's Allowance) (Consequential Amendments) Regulations 1996 reg. 5(5).
}

\subsection[36. Amount of and period of reduction of relevant benefit under a reduced benefit direction]{Amount of and period of reduction of relevant benefit under a reduced benefit direction}

36.—(1) The reduction in the amount payable by way of a relevant benefit to, or in respect of, the parent concerned and the period of such reduction by virtue of a direction shall be determined in accordance with paragraphs (2) to (9).

(2) Subject to paragraph (6) and regulations 37, 38(7)%
% and 40, 
, 40 and 40ZA,  % Words substituted (7.10.96) by SI 1996/1345 reg 5(6)(a)
there shall be a reduction for a period of 
%26 weeks 
156 weeks  % Words substituted (7.10.96) by SI 1996/1945 reg 14(2)
from the day specified in the direction under the provisions of section 46(9) of the Act in respect of each such week equal to
%\[0.2 \times \mathrm{B}\]
\[0.4 \times B.\]  % Formula substituted (7.10.96) by SI 1996/1945 reg 14(2)
where B is an amount equal to the weekly amount, in relation to the week in question, specified in column (2) of paragraph 1(1)($e$) of the applicable amounts Schedule.

% Reg 36(3) omitted (7.10.96) by SI 1996/1945 reg 14(3)

%(3) Subject to paragraph (6) and regulations 37, 38(7)%
%% and 40, 
%, 40 and 40ZA,  % Words substituted (7.10.96) by SI 1996/1345 reg 5(6)(a)
%at the end of the period specified in paragraph (2) there shall be a reduction from the day immediately succeeding the last day of that period for a period of 52 weeks of an amount in respect of each such week equal to
%\[0.1 \times \mathrm{B}\]
%where B has the same meaning as in paragraph (2).

(4) 
%Subject to paragraph (5), 
Subject to paragraphs 
(4A),  % Words inserted (7.10.96) by SI 1996/1945 reg 14(4)
(5), (5A) and (5B),  % Words inserted (18.4.95) by SI 1995/1045 reg 38(2)
a direction shall come into operation on the first day of the second benefit week following the review, carried out by the adjudication officer in consequence of the direction, of the relevant benefit that is payable.

% Reg 36(4A) inserted (7.10.96) by SI 1996/1945 reg 14(5)
(4A) Subject to paragraphs (5), (5A) and (5B), where a reduced benefit direction (“the subsequent direction”) is made on a day when a reduced benefit direction (“the earlier direction”) is in force in respect of the same parent, the subsequent direction shall come into operation on the day immediately following the day on which the earlier direction ceased to be in force.

(5) Where the relevant benefit is income support and the provisions of regulation 26(2) of the Social Security (Claims and Payments) Regulations 1987\footnote{\frenchspacing S.I. 1987/1968; relevant amending instruments are S.I. 1988/522 and 1989/136.} (deferment of payment of different amount of income support) apply, a direction shall come into operation on such later date as may be determined by the Secretary of State in accordance with those provisions.

% Reg 38(5A)--(5E) inserted (18.4.95) by SI 1995/1045 reg 38(3)
(5A) Where the relevant benefit is family credit or disability working allowance and, at the time a direction is given, a lump sum payment has already been made under the provisions of regulation 27(1A) of the Social Security (Claims and Payments) Regulations 1987\footnote{\frenchspacing S.I. 1987/1968. Regulation 27(1A) was inserted by S.I. 1993/2113.} (payment of family credit or disability working allowance by lump sum) the direction shall, subject to paragraph (5B), come into operation on the first day of any benefit week which immediately follows the period in respect of which the lump sum payment was made, or the first day of any benefit week which immediately follows 18th April 1995 if later.

(5B) Where the period in respect of which the lump sum payment was made is not immediately followed by a benefit week, but family credit or disability working allowance again becomes payable, or income support 
or income-based jobseeker’s allowance  % Words inserted (7.10.96) by SI 1996/1345 reg 5(6)(b)
becomes payable, during a period of 52 weeks from the date the direction was given, the direction shall come into operation on the first day of the second benefit week which immediately follows the expiry of a period of 14 days from service of the notice specified in paragraph (5C).

(5C) Where paragraph (5B) applies, the parent to or in respect of whom family credit or disability working allowance has again become payable, or income support 
or income-based jobseeker’s allowance  % Words inserted (7.10.96) by SI 1996/1345 reg 5(6)(c)(i)
has become payable, shall be notified in writing by a child support officer that the amount of family credit, disability working allowance% 
%or income support 
, income support or income-based jobseeker’s allowance  % Words substituted (7.10.96) by SI 1996/1345 reg 5(6)(c)(ii)
paid to or in respect of her will be reduced in accordance with the provisions of paragraph (5B) if she continues to fail to comply with the obligations imposed by section 6 of the Act.

(5D) Where—
\begin{enumerate}\item[]
($a$) family credit or disability working allowance has been paid by lump sum under the provisions of regulation 27(1A) of the Social Security (Claims and Payments) Regulations 1987 (whether or not a benefit week immediately follows the period in respect of which the lump sum payment was made); and

($b$) where income support 
or income-based jobseeker’s allowance  % Words inserted (7.10.96) by SI 1996/1345 reg 5(6)(d)(i)
becomes payable to or in respect of a parent to or in respect of whom family credit or disability working allowance was payable at the time the direction referred to in paragraph (5A) was made, 
\end{enumerate}
income support 
or, as the case may be, income-based jobseeker’s allowance  % Words inserted (7.10.96) by SI 1996/1345 reg 5(6)(d)(ii)
shall become a relevant benefit for the purposes of that direction and the amount payable by way of income support 
or, as the case may be, income-based jobseeker’s allowance  % Words inserted (7.10.96) by SI 1996/1345 reg 5(6)(d)(ii)
shall be reduced in accordance with that direction.

(5E) In circumstances to which paragraph (5A) or (5B) applies, where no relevant benefit has become payable during a period of 52 weeks from that date on which a direction was given, it shall lapse.

%(6) Where the benefit payable is income support 
%or income-based jobseeker’s allowance  % Words inserted (7.10.96) by SI 1996/1345 reg 5(6)(b)
%and there is a change in the benefit week whilst a direction is in operation, the periods of the reductions specified in paragraphs (2) and (3) shall be—
%\begin{enumerate}\item[]
%($a$) where the reduction is that specified in paragraph (2), a period greater than 25 weeks but less than 26 weeks;
%
%($b$) where the reduction is that specified in paragraph (3), a period greater than 51 weeks but less than 52 weeks,
%\end{enumerate}
%and ending on the last day of the last benefit week falling entirely within the period of 26 weeks specified in paragraph (2), or the period of 52 weeks specified in paragraph (3), as the case may be.

% Reg 36(6) substituted (7.10.96) by SI 1996/1945 reg 14(6)
(6) Where the benefit payable is income support or income-based jobseekers allowance and there is a change in the benefit week whilst a direction is in operation, the period of the reduction specified in paragraph (2) shall be a period greater than 155 weeks but less than 156 weeks and ending on the last day of the last benefit week falling entirely within the period of 156 weeks specified in that paragraph.

(7) Where the weekly amount specified in column (2) of paragraph 1(1)($e$) of the applicable amounts Schedule changes on a day when a direction is in operation, the amount of the reduction of the relevant benefit shall be changed—
\begin{enumerate}\item[]
($a$) where the benefit is income support
or income-based jobseeker’s allowance% Words inserted (7.10.96) by SI 1996/1345 reg 5(6)(b)
, from the first day of the first benefit week to commence for the parent concerned on or after the day that weekly amount changes;

($b$) where the benefit is family credit or disability working allowance, from the first day of the next award period of that benefit for the parent concerned commencing on or after the day that weekly amount changes.
\end{enumerate}

(8) Only one direction in relation to a parent shall be in force at any one time.

% Reg 36(9) omitted (7.10.96) by SI 1996/1945
%(9) Where a direction has been in operation for the aggregate of the periods specified in paragraphs (2) and (3) (“the full period”), no further direction shall be given with respect to the same parent on account of that parent’s failure to comply with the obligations imposed by section 6 of the Act in relation to any child in relation to whom the direction that has been in operation for the full period was given.

\amendment{
Words substituted in reg. 36(4) and reg. 36(5A)--(5E) inserted (18.4.95) by the Child Support and Income Support (Amendment) Regulations 1995 reg. 38.

Words inserted in reg. 36(5B), (5C), (5D)($b$), 
%(6), 
(7) and words substituted in reg. 36(2), %(3), 
(5C) (7.10.96) by the Social Security and Child Support (Jobseeker's Allowance) (Consequential Amendments) Regulations 1996 reg. 5(6).

Word inserted in reg. 36(4), words and formula substituted in reg. 36(2), reg. 36(6) inserted, reg. 36(4A) inserted and reg. 36(3), (9) omitted (7.10.96) by the Child Support (Miscellaneous Amendments) Regulations 1996 reg. 14.

Under the Child Support (Miscellaneous Amendments) Regulations 1996 reg. 25(3) (as amended by the Social Security and Child Support (Jobseeker's Allowance) (Transitional Provisions) (Amendment) Regulations 1996 reg. 3(2)) the following version of this regulation applies to a parent in respect of whom a reduced benefit direction was given prior to 7.10.96:

\begin{quotation}
36.—(1) The reduction in the amount payable by way of a relevant benefit to, or in respect of, the parent concerned and the period of such reduction by virtue of a direction shall be determined in accordance with paragraphs (2) to (9).

(2) Subject to paragraph (6) and regulations 37, 38(7)%
% and 40, 
, 40 and 40ZA,  % Words substituted (7.10.96) by SI 1996/1345 reg 5(6)(a)
there shall be a reduction for a period of 
26 weeks 
from the day specified in the direction under the provisions of section 46(9) of the Act in respect of each such week equal to
\[0.2 \times B\]
where B is an amount equal to the weekly amount, in relation to the week in question, specified in column (2) of paragraph 1(1)($e$) of the applicable amounts Schedule.

(3) Subject to paragraph (6) and regulations 37, 38(7)%
% and 40, 
, 40 and 40ZA,  % Words substituted (7.10.96) by SI 1996/1345 reg 5(6)(a)
at the end of the period specified in paragraph (2) there shall be a reduction from the day immediately succeeding the last day of that period for a period of 52 weeks of an amount in respect of each such week equal to
\[0.1 \times B\]
where B has the same meaning as in paragraph (2).

(4) 
%Subject to paragraph (5), 
Subject to paragraphs 
(5), (5A) and (5B),  % Words inserted (18.4.95) by SI 1995/1045 reg 38(2)
a direction shall come into operation on the first day of the second benefit week following the review, carried out by the adjudication officer in consequence of the direction, of the relevant benefit that is payable.

(5) Where the relevant benefit is income support and the provisions of regulation 26(2) of the Social Security (Claims and Payments) Regulations 1987\footnote{\frenchspacing S.I. 1987/1968; relevant amending instruments are S.I. 1988/522 and 1989/136.} (deferment of payment of different amount of income support) apply, a direction shall come into operation on such later date as may be determined by the Secretary of State in accordance with those provisions.

% Reg 38(5A)--(5E) inserted (18.4.95) by SI 1995/1045 reg 38(3)
(5A) Where the relevant benefit is family credit or disability working allowance and, at the time a direction is given, a lump sum payment has already been made under the provisions of regulation 27(1A) of the Social Security (Claims and Payments) Regulations 1987\footnote{\frenchspacing S.I. 1987/1968. Regulation 27(1A) was inserted by S.I. 1993/2113.} (payment of family credit or disability working allowance by lump sum) the direction shall, subject to paragraph (5B), come into operation on the first day of any benefit week which immediately follows the period in respect of which the lump sum payment was made, or the first day of any benefit week which immediately follows 18th April 1995 if later.

(5B) Where the period in respect of which the lump sum payment was made is not immediately followed by a benefit week, but family credit or disability working allowance again becomes payable, or income support 
or income-based jobseeker’s allowance  % Words inserted (7.10.96) by SI 1996/1345 reg 5(6)(b)
becomes payable, during a period of 52 weeks from the date the direction was given, the direction shall come into operation on the first day of the second benefit week which immediately follows the expiry of a period of 14 days from service of the notice specified in paragraph (5C).

(5C) Where paragraph (5B) applies, the parent to or in respect of whom family credit or disability working allowance has again become payable, or income support 
or income-based jobseeker’s allowance  % Words inserted (7.10.96) by SI 1996/1345 reg 5(6)(c)(i)
has become payable, shall be notified in writing by a child support officer that the amount of family credit, disability working allowance% 
%or income support 
, income support or income-based jobseeker’s allowance  % Words substituted (7.10.96) by SI 1996/1345 reg 5(6)(c)(ii)
paid to or in respect of her will be reduced in accordance with the provisions of paragraph (5B) if she continues to fail to comply with the obligations imposed by section 6 of the Act.

(5D) Where—
\begin{enumerate}\item[]
($a$) family credit or disability working allowance has been paid by lump sum under the provisions of regulation 27(1A) of the Social Security (Claims and Payments) Regulations 1987 (whether or not a benefit week immediately follows the period in respect of which the lump sum payment was made); and

($b$) where income support 
or income-based jobseeker’s allowance  % Words inserted (7.10.96) by SI 1996/1345 reg 5(6)(d)(i)
becomes payable to or in respect of a parent to or in respect of whom family credit or disability working allowance was payable at the time the direction referred to in paragraph (5A) was made, 
\end{enumerate}
income support 
or, as the case may be, income-based jobseeker’s allowance  % Words inserted (7.10.96) by SI 1996/1345 reg 5(6)(d)(ii)
shall become a relevant benefit for the purposes of that direction and the amount payable by way of income support 
or, as the case may be, income-based jobseeker’s allowance  % Words inserted (7.10.96) by SI 1996/1345 reg 5(6)(d)(ii)
shall be reduced in accordance with that direction.

(5E) In circumstances to which paragraph (5A) or (5B) applies, where no relevant benefit has become payable during a period of 52 weeks from that date on which a direction was given, it shall lapse.

(6) Where the benefit payable is income support 
or income-based jobseeker’s allowance  % Words inserted (7.10.96) by SI 1996/1345 reg 5(6)(b)
and there is a change in the benefit week whilst a direction is in operation, the periods of the reductions specified in paragraphs (2) and (3) shall be—
\begin{enumerate}\item[]
($a$) where the reduction is that specified in paragraph (2), a period greater than 25 weeks but less than 26 weeks;

($b$) where the reduction is that specified in paragraph (3), a period greater than 51 weeks but less than 52 weeks,
\end{enumerate}
and ending on the last day of the last benefit week falling entirely within the period of 26 weeks specified in paragraph (2), or the period of 52 weeks specified in paragraph (3), as the case may be.

(7) Where the weekly amount specified in column (2) of paragraph 1(1)($e$) of the applicable amounts Schedule changes on a day when a direction is in operation, the amount of the reduction of the relevant benefit shall be changed—
\begin{enumerate}\item[]
($a$) where the benefit is income support
or income-based jobseeker’s allowance% Words inserted (7.10.96) by SI 1996/1345 reg 5(6)(b)
, from the first day of the first benefit week to commence for the parent concerned on or after the day that weekly amount changes;

($b$) where the benefit is family credit or disability working allowance, from the first day of the next award period of that benefit for the parent concerned commencing on or after the day that weekly amount changes.
\end{enumerate}

(8) Only one direction in relation to a parent shall be in force at any one time.

(9) Where a direction has been in operation for the aggregate of the periods specified in paragraphs (2) and (3) (“the full period”), no further direction shall be given with respect to the same parent on account of that parent’s failure to comply with the obligations imposed by section 6 of the Act in relation to any child in relation to whom the direction that has been in operation for the full period was given.
\end{quotation}
}

\subsection[37. Modification of reduction under a reduced benefit direction to preserve minimum entitlement to relevant benefit]{Modification of reduction under a reduced benefit direction to preserve minimum entitlement to relevant benefit}

37.  Where in respect of any benefit week the amount of the relevant benefit that would be payable after it has been reduced following a direction would, but for this regulation, be nil or less than the minimum amount of that benefit that is payable as determined—
\begin{enumerate}\item[]
($a$) in the case of income support, by regulation 26(4) of the Social Security (Claims and Payments) Regulations 1987;

($aa$) in the case of income-based jobseeker’s allowance, by regulation 
%26A(10)\footnote{\frenchspacing Regulation 26A was inserted by S.I. 1996/207.} of those Regulations;
87A of the Jobseeker’s Allowance Regulations 1996\footnote{\frenchspacing S.I. 1996/207; regulation 87A was inserted by S.I. 1996/1517.};  % Words substituted (28.10.96) by SI 1996/2538 reg 6(2)(a)

($b$) in the case of family credit and disability working allowance, by regulation 27(2) 
%of those Regulations,
of the Social Security (Claims and Payments) Regulations 1987,  % Words substituted (28.10.96) by SI 1996/2538 reg 6(2)(b)
\end{enumerate}
the amount of that reduction shall be decreased to such extent as to raise the amount of that benefit to the minimum amount that is payable.

\amendment{
Reg. 37($aa$) inserted (7.10.96) by the Social Security and Child Support (Jobseeker's Allowance) (Consequential Amendments) Regulations 1996 reg. 5(7).

Words substituted in reg. 37($aa$), ($b$) (28.10.96) by the Social Security and Child Support (Jobseeker's Allowance) (Miscellaneous Amendments) Regulations 1996 reg. 6(2).
}

\subsection[38. Suspension of a reduced benefit direction when relevant benefit ceases to be payable]{Suspension of a reduced benefit direction when relevant benefit ceases to be payable}

38.—(1) Where relevant benefit ceases to be payable to, or in respect of, the parent concerned at a time when a direction is in operation, that direction shall, subject to paragraph (2), be suspended for a period of 52 weeks from the date the relevant benefit has ceased to be payable.

(2) Where a direction has been suspended for a period of 52 weeks and no relevant benefit is payable at the end of that period, it shall cease to be in force.

(3) Where a direction is suspended and relevant benefit again becomes payable to or in respect of the parent concerned, the amount payable by way of that benefit shall, subject to regulations 40, 
40ZA,  % Word inserted (7.10.96) by SI 1996/1345 reg 5(8)
41 and 42, be reduced in accordance with that direction for the balance of the reduction period.

(4) The amount or, as the case may be, amounts of the reduction to be made during the balance of the reduction period shall be determined in accordance with regulation 36(2).
% and (3).  % Words omitted (7.10.96) by SI 1996/1945 reg 15

(5) No reduction in the amount of benefit under paragraph (3) shall be made before the expiry of a period of 14 days from service of the notice specified in paragraph (6), and the provisions of regulation 36(4) shall apply as to the date when the direction again comes into operation.

(6) Where relevant benefit again becomes payable to or in respect of a parent with respect to whom a direction is suspended she shall be notified in writing by a child support officer that the amount of relevant benefit paid to or in respect of her will again be reduced, in accordance with the provisions of paragraph (3), if she continues to fail to comply with the obligations imposed by section 6 of the Act.

(7) Where a direction has ceased to be in force by virtue of the provisions of paragraph (2), a further direction in respect of the same parent given on account of that parent’s failure to comply with the obligations imposed by section 6 of the Act in relation to one or more of the same qualifying children shall, unless it also ceases to be in force by virtue of the provisions of paragraph (2), be in operation for the balance of the reduction period relating to the direction that has ceased to be in force, and the provisions of paragraph (4) shall apply to it.

\amendment{
Word inserted in reg. 38(3) (7.10.96) by the Social Security and Child Support (Jobseeker's Allowance) (Consequential Amendments) Regulations 1996 reg. 5(8).

Words omitted in reg. 38(4) (7.10.96) by the Child Support (Miscellaneous Amendments) Regulations 1996 reg. 15.
}

\subsection[39. Reduced benefit direction where family credit or disability working allowance is payable and income support becomes payable]{\sloppy Reduced benefit direction where family credit or disability working allowance is payable and income support becomes payable}

39.—(1) Where a direction is in operation in respect of a parent to whom or in respect of whom family credit or disability working allowance is payable, and income support 
or income-based jobseeker’s allowance  % Words inserted (7.10.96) by SI 1996/1345 reg 5(9)(a)
becomes payable to or in respect of that parent, income support 
or, as the case may be, income-based jobseeker’s allowance  % Words inserted (7.10.96) by SI 1996/1345 reg 5(9)(b)
shall become a relevant benefit for the purposes of that direction, and the amount payable by way of income support 
or, as the case may be, income-based jobseeker’s allowance  % Words inserted (7.10.96) by SI 1996/1345 reg 5(9)(b)
shall be reduced in accordance with that direction for the balance of the reduction period.

(2) The amount or, as the case may be, the amounts of the reduction to be made during the balance of the reduction period shall be determined in accordance with regulation 36(2).
%and (3).  % Words omitted (7.10.96) by SI 1996/1945 reg 16

\amendment{
Words inserted in reg. 39(1) (7.10.96) by the Social Security and Child Support (Jobseeker's Allowance) (Consequential Amendments) Regulations 1996 reg. 5(9).

Words omitted in reg. 39(2) (7.10.96) by the Child Support (Miscellaneous Amendments) Regulations 1996 reg. 16.
}

\subsection[40. Suspension of a reduced benefit direction when a modified applicable amount is payable]{Suspension of a reduced benefit direction when a modified applicable amount is payable}

40.—(1) Where a direction is given or is in operation at a time when income support is payable to or in respect of the parent concerned but her applicable amount falls to be calculated under the provisions mentioned in paragraph (3), that direction shall be suspended for so long as the applicable amount falls to be calculated under the provisions mentioned in that paragraph, or 52 weeks, whichever period is the shorter.

% Reg 40(1A) inserted (18.4.95) by SI 1995/1045 reg 39(2)
(1A) Where a direction is given or is in operation at a time when income support is payable to or in respect of the parent concerned, but her applicable amount includes a residential allowance under regulation 17 of, and paragraph 2A of Schedule 2 to, the Income Support Regulations\footnote{\frenchspacing Regulation 17 was amended and paragraph 2A added by S.I. 1992/3147. Paragraph 2A(1) was substituted by S.I. 1994/542.} (applicable amounts for those in residential care or nursing homes), that direction shall be suspended for as long as her applicable amount includes a residential allowance under regulation 17 and paragraph 2A of Schedule 2, or 52 weeks, whichever period is the shorter.

(2) Where a case falls within paragraph (1) 
or (1A)  % Words inserted (18.4.95) by SI 1995/1045 reg 39(3)
and a direction has been suspended for a period of 52 weeks, it shall cease to be in force.

(3) The provisions of paragraph (1) shall apply where the applicable amount in relation to the parent concerned falls to be calculated under—
\begin{enumerate}\item[]
($a$) regulation 19 of and Schedule 4 to the Income Support Regulations (applicable amounts for persons in residential care and nursing homes);

($b$) regulation 21 of and paragraphs 1 to 3 of Schedule 7 to the Income Support Regulations (patients);

($c$) regulation 21 of and paragraphs 10B, 10C
%, 10D % Words omitted (5.4.93) by SI 1993/913 reg 13
and 13 of Schedule 7 to the Income Support Regulations (persons in residential accommodation).
\end{enumerate}

\amendment{
Words omitted in reg. 40(3)($c$) (5.4.93) by the Child Support (Miscellaneous Amendments) Regulations 1993 reg. 13.

Words inserted in reg. 40(2) and reg. 40(1A) inserted (18.4.95) by the Child Support and Income Support (Amendment) Regulations 1995 reg. 39.
}

% Reg 40ZA inserted (7.10.96) by SI 1996/1345 reg 5(10)
\subsection[40ZA. Suspension of a reduced benefit direction in the case of modified applicable amounts in jobseeker’s allowance]{Suspension of a reduced benefit direction in the case of modified applicable amounts in jobseeker’s allowance}

40ZA.—(1) Where a direction is given or is in operation at a time when income-based jobseeker’s allowance is payable to or in respect of the parent concerned but her applicable amount falls to be calculated under the provisions mentioned in paragraph (4), that direction shall be suspended for so long as the applicable amount falls to be calculated under those provisions, or 52 weeks, whichever period is the shorter.

(2) Where a direction is given or is in operation at a time when income-based jobseeker’s allowance is payable to or in respect of the parent concerned, but her applicable amount includes a residential allowance under regulation 83($c$) of and paragraph 3 of Schedule 1 to the Jobseeker’s Allowance Regulations (persons in residential care or nursing homes), that direction shall be suspended for as long as her applicable amount includes such a residential allowance, or 52 weeks, whichever period is the shorter.

(3) Where a case falls within paragraph (1) or (2) and a direction has been suspended for a period of 52 weeks, it shall cease to be in force.

(4) The provisions of paragraph (1) shall apply where the applicable amount in relation to the parent concerned falls to be calculated under—
\begin{enumerate}\item[]
($a$) regulation 85 of and paragraph 1 or 2 of Schedule 5 to the Jobseeker’s Allowance Regulations (patients);

($b$) regulation 85 of and paragraph 8, 9 or 15 of Schedule 5 to the Jobseeker’s Allowance Regulations (persons in residential accommodation); or

($c$) regulation 86 of and Schedule 4 to the Jobseeker’s Allowance Regulations (applicable amounts for persons in residential care and nursing homes).
\end{enumerate}

\amendment{
Reg. 40ZA inserted (7.10.96) by the Social Security and Child Support (Jobseeker's Allowance) (Consequential Amendments) Regulations 1996 reg. 5(10).

\medskip

Reg. 40A revoked (13.1.97) by the Child Support (Miscellaneous Amendments) (No. 2) Regulations 1996 reg. 9.

Under the Child Support (Miscellaneous Amendments) (No. 2) Regulations 1996 reg. 16(3), reg. 40A (as shown below) continues to apply to a reduced benefit direction which as of 13.1.97 is suspended under its provisions:
\begin{quotation}
% Reg 40A inserted (22.1.96) by SI 1995/3261 reg 38
\subsection*{\sloppy \itshape Suspension of a reduced benefit direction where certain deductions are being made from income support}

40A.—(1) A reduced benefit direction made after 22nd January 1996 shall be suspended where, on the date it is given, one or more of the deductions specified in paragraph (2) are being made from income support 
or an income-based jobseeker’s allowance  % Words inserted (28.10.96) by SI 1996/2538 reg 6(3)
paid to or in respect of the parent concerned.

(2) The deductions relevant for the purposes of paragraph (1) are—
\begin{enumerate}\item[]
(i) deductions in respect of arrears of housing costs, fuel or water charges under paragraph 3, 5, 6 or 7 of Schedule 9 to the Social Security (Claims and Payments) Regulations 1987\footnote{\frenchspacing S.I. 1987/1968. Paragraph 3 of Schedule 9 was amended by S.I. 1988/522, S.I. 1992/1026 and S.I. 1992/2595 and paragraph 5 by S.I. 1988/522, S.I. 1991/2284 and S.I. 1992/2595. Paragraph 6 was amended by S.I. 1988/522, S.I. 1991/2284, S.I. 1992/2595 and S.I. 1994/2319. Paragraph 7 was amended by S.I. 1992/2595 and S.I. 1994/2319.};

(ii) deductions in respect of overpaid benefit under regulation 15, 16 or 17 of the Social Security (Payments on Account, Overpayments and Recovery) Regulations 1988\footnote{\frenchspacing S.I. 1988/664. Regulations 15, 16 and 17 were amended by S.I. 1988/688 and S.I. 1991/2742.};

(iii) deductions in respect of arrears of Community Charge liability under regulation 2 or 4 of the Community Charges (Deductions from Income Support) (No.\ 2) Regulations 1990\footnote{\frenchspacing S.I. 1990/545. Regulation 2 was amended by S.I. 1992/1026 and S.I. 1993/2113.};

(iv) deductions in respect of arrears of Council Tax liability under regulation 5 or 7 of the Council Tax (Deductions from Income Support) Regulations 1993\footnote{\frenchspacing S.I. 1993/494.};

(v) deductions in respect of fines under regulation 4 of the Fines (Deductions from Income Support) Regulations 1992\footnote{\frenchspacing S.I. 1992/2182. Regulation 4 was substituted by S.I. 1993/495.};

(vi) deductions in respect of social fund awards under section 78(1) to (3) of the Social Security Administration Act 1992\footnote{\frenchspacing 1992 c. 5.}.
\end{enumerate}

(3) When income support 
or an income-based jobseeker’s allowance  % Words inserted (28.10.96) by SI 1996/2538 reg 6(3)
payable to or in respect of the parent concerned is no longer subject to the deductions relevant for the purposes of paragraph (1), the reduced benefit direction shall cease to be suspended at the end of a period of 14 days after notification has been served under regulation 49A.
\end{quotation}
}

% Reg 40A revoked (13.1.97) by SI 1996/3196 reg 9
%% Reg 40A inserted (22.1.96) by SI 1995/3261 reg 38
%\subsection[40A. Suspension of a reduced benefit direction where certain deductions are being made from income support]{Suspension of a reduced benefit direction where certain deductions are being made from income support}
%
%40A.—(1) A reduced benefit direction made after 22nd January 1996 shall be suspended where, on the date it is given, one or more of the deductions specified in paragraph (2) are being made from income support 
%or an income-based jobseeker’s allowance  % Words inserted (28.10.96) by SI 1996/2538 reg 6(3)
%paid to or in respect of the parent concerned.
%
%(2) The deductions relevant for the purposes of paragraph (1) are—
%\begin{enumerate}\item[]
%(i) deductions in respect of arrears of housing costs, fuel or water charges under paragraph 3, 5, 6 or 7 of Schedule 9 to the Social Security (Claims and Payments) Regulations 1987\footnote{\frenchspacing S.I. 1987/1968. Paragraph 3 of Schedule 9 was amended by S.I. 1988/522, S.I. 1992/1026 and S.I. 1992/2595 and paragraph 5 by S.I. 1988/522, S.I. 1991/2284 and S.I. 1992/2595. Paragraph 6 was amended by S.I. 1988/522, S.I. 1991/2284, S.I. 1992/2595 and S.I. 1994/2319. Paragraph 7 was amended by S.I. 1992/2595 and S.I. 1994/2319.};
%
%(ii) deductions in respect of overpaid benefit under regulation 15, 16 or 17 of the Social Security (Payments on Account, Overpayments and Recovery) Regulations 1988\footnote{\frenchspacing S.I. 1988/664. Regulations 15, 16 and 17 were amended by S.I. 1988/688 and S.I. 1991/2742.};
%
%(iii) deductions in respect of arrears of Community Charge liability under regulation 2 or 4 of the Community Charges (Deductions from Income Support) (No.\ 2) Regulations 1990\footnote{\frenchspacing S.I. 1990/545. Regulation 2 was amended by S.I. 1992/1026 and S.I. 1993/2113.};
%
%(iv) deductions in respect of arrears of Council Tax liability under regulation 5 or 7 of the Council Tax (Deductions from Income Support) Regulations 1993\footnote{\frenchspacing S.I. 1993/494.};
%
%(v) deductions in respect of fines under regulation 4 of the Fines (Deductions from Income Support) Regulations 1992\footnote{\frenchspacing S.I. 1992/2182. Regulation 4 was substituted by S.I. 1993/495.};
%
%(vi) deductions in respect of social fund awards under section 78(1) to (3) of the Social Security Administration Act 1992\footnote{\frenchspacing 1992 c. 5.}.
%\end{enumerate}
%
%(3) When income support 
%or an income-based jobseeker’s allowance  % Words inserted (28.10.96) by SI 1996/2538 reg 6(3)
%payable to or in respect of the parent concerned is no longer subject to the deductions relevant for the purposes of paragraph (1), the reduced benefit direction shall cease to be suspended at the end of a period of 14 days after notification has been served under regulation 49A.
%
%\amendment{
%Reg. 40A inserted (22.1.96) by the Child Support (Miscellaneous Amendments) (No. 2) Regulations 1995 reg. 38.
%
%Words inserted in reg. 40A(1), (3) (28.10.96) by the Social Security and Child Support (Jobseeker's Allowance) (Miscellaneous Amendments) Regulations 1996 reg. 6(3).
%}

\subsection[41. Termination of a reduced benefit direction following compliance with obligations imposed by section 6 of the Act]{Termination of a reduced benefit direction following compliance with obligations imposed by section 6 of the Act}

41.—(1) Where a parent with care with respect to whom a direction is in force complies with the obligations imposed by section 6 of the Act, that direction shall cease to be in force on the date determined in accordance with paragraph (2) or (3), as the case may be.

(2) Where the direction is in operation, it shall cease to be in force on the last day of the benefit week during the course of which the parent concerned complied with the obligations imposed by section 6 of the Act.

(3) Where the direction is suspended, it shall cease to be in force on the date on which the parent concerned complied with the obligations imposed by section 6 of the Act.

\subsection[42. Review of a reduced benefit direction]{Review of a reduced benefit direction}

42.—(1) Where a parent with care with respect to whom a direction is in force 
or some other person % Words inserted (5.4.93) by SI 1993/913 reg 14(2)($a$)
gives the Secretary of State reasons—
\begin{enumerate}\item[]
($a$) additional to any reasons given by 
%her 
the parent with care % Word substituted (5.4.93) by SI 1993/913 reg 14(2)($b$)
in response to the notice served on her under section 46(2) of the Act for having failed to comply with the obligations imposed by section 6 of the Act; or

($b$) as to why 
%she 
the parent with care % Word substituted (5.4.93) by SI 1993/913 reg 14(2)($c$)
should no longer be required to comply with the obligations imposed by section 6 of the Act,
\end{enumerate}
the Secretary of State shall refer the matter to a child support officer who shall conduct a review of the direction (“a review”) to determine whether the direction is to continue or is to cease to be in force.

(2) Where a parent with care with respect to whom a direction is in force 
or some other person % Words inserted (5.4.93) by SI 1993/913 reg 14(3)
gives a child support officer reasons of the kind mentioned in paragraph (1), a child support officer shall conduct a review to determine whether the direction is to continue or is to cease to be in force.

%Reg 42(2A), (2B) inserted (5.4.93) by SI 1993/913 reg 14(4)
(2A) Where a direction is in force and the Secretary of State becomes aware that a question arises as to whether the welfare of a child is likely to be affected by the direction continuing to be in force, he shall refer the matter to a child support officer who shall conduct a review to determine whether the direction is to continue or is to cease to be in force.

(2B) Where a direction is in force and a child support officer becomes aware that a question arises as to whether the welfare of a child is likely to be affected by the direction continuing to be in force, a child support officer shall conduct a review to determine whether the direction is to continue or is to cease to be in force.

(3) A review shall not be carried out by the child support officer who gave the direction with respect to the parent concerned.

(4) Where the child support officer who is conducting a review considers that the parent concerned is no longer to be required to comply with the obligations imposed by section 6 of the Act, the direction shall cease to be in force on the date determined in accordance with paragraph (5) or (6), as the case may be.

(5) Where the direction is in operation, it shall cease to be in force on the last day of the benefit week during the course of which 
%the parent concerned gave the reasons specified in paragraph (1).
the reasons specified in paragraph (1) were given % Words substituted (5.4.93) by SI 1993/913 reg 14(5)
or the Secretary of State or a child support officer becomes aware of a question of a kind mentioned in paragraph (2A) or (2B).  % Words inserted (18.4.95) by SI 1995/1045 reg 40

(6) Where the direction is suspended, it shall cease to be in force on the date on which %the parent concerned gave the reasons specified in paragraph (1)
the reasons specified in paragraph (1) were given. % Words substituted (5.4.93) by SI 1993/913 reg 14(5)

%Reg 42(7) omitted (5.4.93) by SI 1993/913 reg 14(6)
%(7) The provisions of section 20 of the Act shall apply in relation to a decision of a child support officer following a review.

(8) A child support officer shall on completing a review immediately notify the parent concerned of his decision, so far as that is reasonably practicable, and shall give the reasons for his decision in writing.

%(9) A notification under paragraph (8) shall include information as to the provisions of section 20 of the Act.
%Reg 42(9)--(11) substituted for reg 42(9) (5.4.93) by SI 1993/913 reg 14(7)
(9) A parent with care who is aggrieved by a decision of a child support officer following a review may appeal to a child support appeal tribunal against that decision.

(10) Sections 20(2) to (4) and 21 of the Act shall apply in relation to appeals under paragraph (9) as they apply in relation to appeals under section 20 of the Act.

(11) A notification under paragraph (8) shall include information as to the provisions of paragraphs (9) and (10).

\amendment{
Words inserted in reg. 42(1), (2), words substituted in reg. 42(1), (2), (5), (6), reg. 42(2A), (2B) inserted, reg. 42(9)--(11) substituted for reg. 42(9) and reg. 42(7) omitted (5.4.93) by the Child Support (Miscellaneous Amendments) Regulations 1993 reg. 14.

Words inserted in reg. 42(5) (18.4.95) by the Child Support and Income Support (Amendment) Regulations 1995 reg. 40.
}

\subsection[43. Termination of a reduced benefit direction where a maintenance assessment is made following an application by a child under section 7 of the Act]{\sloppy Termination of a reduced benefit direction where a maintenance assessment is made following an application by a child under section 7 of the Act}

43.  Where a qualifying child of a parent with respect to whom a direction is in force applies for a maintenance assessment to be made with respect to him under section 7 of the Act, and an assessment is made in response to that application in respect of all of the qualifying children in relation to whom the parent concerned failed to comply with the obligations imposed by section 6 of the Act, that direction shall cease to be in force from the date determined in accordance with regulation 45.

\subsection[44. Termination of a reduced benefit direction where a maintenance assessment is made following an application by an absent parent under section 4 of the Act]{\sloppy Termination of a reduced benefit direction where a maintenance assessment is made following an application by an absent parent under section 4 of the Act}

44.  Where—
\begin{enumerate}\item[]
($a$) an absent parent applies for a maintenance assessment to be made under section 4 of the Act with respect to all of his qualifying children in relation to whom the other parent of those children is a person with care;

($b$) a direction is in force with respect to that other parent following her failure to comply with the obligations imposed by section 6 of the Act in relation to those qualifying children; and

($c$) an assessment is made in response to that application by the absent parent for a maintenance assessment,
\end{enumerate}
that direction shall cease to be in force on the date determined in accordance with regulation 45.

\subsection[45. Date from which a reduced benefit direction ceases to be in force following a termination under regulation 43 or 44]{Date from which a reduced benefit direction ceases to be in force following a termination under regulation 43 or 44}

45.—(1) The date a direction ceases to be in force under the provisions of regulation 43 or 44 shall be determined in accordance with paragraphs (2) and (3).

(2) Where the direction is in operation, it shall cease to be in force on the last day of the benefit week during the course of which the Secretary of State is supplied with the information that enables a child support officer to make the assessment.

(3) Where the direction is suspended, it shall cease to be in force on the date on which the Secretary of State is supplied with the information that enables a child support officer to make the assessment.

\subsection[46. Cancellation of a reduced benefit direction in cases of error]{Cancellation of a reduced benefit direction in cases of error}

46.  Where a child support officer is satisfied that a direction was given as a result of an error on the part of the Secretary of State or a child support officer, or though not given as a result of such an error has not subsequently ceased to be in force as a result of such an error, the child support officer shall cancel the direction and it shall be treated as not having been given, or as having ceased to be in force on the date it would have ceased to be in force if that error had not been made, as the case may be.

\subsection[47. Reduced benefit directions where there is an additional qualifying child]{\sloppy Reduced benefit directions where there is an additional qualifying child}

47.—(1) Where a direction is in operation or would be in operation but for the provisions of regulation 40 
or 40ZA  % Words inserted (7.10.96) by SI 1996/1345 reg 5(11)
and a child support officer gives a further direction with respect to the same parent on account of that parent failing to comply with the obligations imposed by section 6 of the Act in relation to an additional qualifying child of whom she is a person with care, the earlier direction shall cease to be in force on the last day of the benefit week preceding the benefit week on the first day of which, in accordance with the provisions of regulation 36(4), the further direction comes into operation, or would come into operation but for the provisions of regulation 40
or 40ZA.  % Words inserted (7.10.96) by SI 1996/1345 reg 5(11)

(2) Where a further direction comes into operation in a case falling within paragraph (1), the provisions of regulation 36 shall apply to it.

%(3) Where a direction has ceased to be in force by virtue of regulation 38(2) and a child support officer gives a direction with respect to the same parent on account of that parent’s failure to comply with the obligations imposed by section 6 of the Act in relation to an additional qualifying child, no further direction shall be given with respect to that parent on account of her failure to comply with the obligations imposed by section 6 of the Act in relation to one or more children in relation to whom the direction that has ceased to be in force by virtue of regulation 38(2) was given.

% Reg 47(3) substituted (7.10.96) by SI 1996/1945 reg 17(2)
(3) Where—
\begin{enumerate}\item[]
($a$) a direction (“the earlier direction”) has ceased to be in force by virtue of regulation 38(2); and

($b$) a child support officer gives a direction (“the further direction”) with respect to the same parent on account of that parent’s failure to comply with the obligations imposed by section 6 of the Act in relation to an additional qualifying child,
\end{enumerate}
as long as that further direction remains in force, no additional direction shall be brought into force with respect to that parent on account of her failure to comply with the obligations imposed by section 6 of the Act in relation to one or more children in relation to whom the earlier direction was given.

(4) Where a case falls within paragraph (1) or (3) and the further direction, but for the provisions of this paragraph would cease to be in force by virtue of the provisions of regulation 41 or 42, but the earlier direction would not have ceased to be in force by virtue of the provisions of those regulations, the later direction shall continue in force for a period (“the extended period”) calculated in accordance with the provisions of paragraph (5) and the reduction of relevant benefit 
%shall be determined in accordance with paragraphs (6) and (7).
for the extended period shall be determined in accordance with regulation 36(2).  % Words substituted (7.10.96) by SI 1996/1945 reg 17(3)

(5) The extended period for the purposes of paragraph (4) shall be 
%\[(78 - F - S) \ \mathrm{weeks}\]
\[(156-F-S) \mathrm{\ weeks.}\]  % Formula substituted (7.10.96) by SI 1996/1945 reg 17(4)
 where—
\begin{enumerate}\item[]
F is the number of weeks for which the earlier direction was in operation; and

S is the number of weeks for which the later direction has been in operation.
\end{enumerate}

% Reg 47(6), (7) omitted (7.10.96) by SI 1996/1945 reg 17(5)
%(6) Where the extended period calculated in accordance with paragraph (5) is greater than 52 weeks, there shall be a reduction of relevant benefit in respect of the number of weeks in excess of 52 determined in accordance with regulation 36(2), and a reduction of relevant benefit in respect of the remaining 52 weeks determined in accordance with regulation 36(3).
%
%(7) Where the extended period calculated in accordance with paragraph (5) is equal to or less than 52 weeks, there shall be a reduction of relevant benefit in respect of that period determined in accordance with regulation 36(3).

(8) In this regulation “an additional qualifying child” means a qualifying child of whom the parent concerned is a person with care and who was either not such a qualifying child at the time the earlier direction was given or had not been born at the time the earlier direction was given.

\amendment{
Words inserted in reg. 47(1) (7.10.96) by the Social Security and Child Support (Jobseeker's Allowance) (Consequential Amendments) Regulations 1996 reg. 5(11).

Words in reg. 47(4) substituted, formula in reg. 47(5) substituted, reg. 47(3) substituted and reg. 47(6), (6) omitted (7.10.96) by the Child Support (Miscellaneous Amendments) Regulations 1996 reg. 17.

Under the Child Support (Miscellaneous Amendments) Regulations 1996 reg. 25(4) (as amended by Social Security and Child Support (Jobseeker's Allowance) (Transitional Provisions) (Amendment) Regulations 1996 reg. 3(3)) the following version of reg. 47 continues to apply to any reduced benefit direction made prior to that 7.10.96, and in relation to an earlier direction referred to in reg. 47(4) which was in force prior to 7.10.96 whether or not the further direction referred to in reg. 47(4) was made after 7.10.96:

\begin{quotation}
47.—(1) Where a direction is in operation or would be in operation but for the provisions of regulation 40 
or 40ZA  % Words inserted (7.10.96) by SI 1996/1345 reg 5(11)
and a child support officer gives a further direction with respect to the same parent on account of that parent failing to comply with the obligations imposed by section 6 of the Act in relation to an additional qualifying child of whom she is a person with care, the earlier direction shall cease to be in force on the last day of the benefit week preceding the benefit week on the first day of which, in accordance with the provisions of regulation 36(4), the further direction comes into operation, or would come into operation but for the provisions of regulation 40
or 40ZA.  % Words inserted (7.10.96) by SI 1996/1345 reg 5(11)

(2) Where a further direction comes into operation in a case falling within paragraph (1), the provisions of regulation 36 shall apply to it.

(3) Where a direction has ceased to be in force by virtue of regulation 38(2) and a child support officer gives a direction with respect to the same parent on account of that parent’s failure to comply with the obligations imposed by section 6 of the Act in relation to an additional qualifying child, no further direction shall be given with respect to that parent on account of her failure to comply with the obligations imposed by section 6 of the Act in relation to one or more children in relation to whom the direction that has ceased to be in force by virtue of regulation 38(2) was given.

(4) Where a case falls within paragraph (1) or (3) and the further direction, but for the provisions of this paragraph would cease to be in force by virtue of the provisions of regulation 41 or 42, but the earlier direction would not have ceased to be in force by virtue of the provisions of those regulations, the later direction shall continue in force for a period (“the extended period”) calculated in accordance with the provisions of paragraph (5) and the reduction of relevant benefit 
shall be determined in accordance with paragraphs (6) and (7).

(5) The extended period for the purposes of paragraph (4) shall be 
\[(78 - F - S) \ \mathrm{weeks}\]
 where—
\begin{enumerate}\item[]
F is the number of weeks for which the earlier direction was in operation; and

S is the number of weeks for which the later direction has been in operation.
\end{enumerate}

(6) Where the extended period calculated in accordance with paragraph (5) is greater than 52 weeks, there shall be a reduction of relevant benefit in respect of the number of weeks in excess of 52 determined in accordance with regulation 36(2), and a reduction of relevant benefit in respect of the remaining 52 weeks determined in accordance with regulation 36(3).

(7) Where the extended period calculated in accordance with paragraph (5) is equal to or less than 52 weeks, there shall be a reduction of relevant benefit in respect of that period determined in accordance with regulation 36(3).

(8) In this regulation “an additional qualifying child” means a qualifying child of whom the parent concerned is a person with care and who was either not such a qualifying child at the time the earlier direction was given or had not been born at the time the earlier direction was given.
\end{quotation}
}

\subsection[48. Suspension and termination of a reduced benefit direction where the sole qualifying child ceases to be a child or where the parent concerned ceases to be a person with care]{\sloppy Suspension and termination of a reduced benefit direction where the sole qualifying child ceases to be a child or where the parent concerned ceases to be a person with care}

48.—(1) Where, whilst a direction is in operation—
\begin{enumerate}\item[]
($a$) there is, in relation to that direction, only one qualifying child, and that child ceases to be a child within the meaning of the Act; or

($b$) the parent concerned ceases to be a person with care,
\end{enumerate}
the direction shall be suspended from the last day of the benefit week during the course of which the child ceases to be a child within the meaning of the Act, or the parent concerned ceases to be a person with care, as the case may be.

(2) Where, under the provisions of paragraph (1), a direction has been suspended for a period of 52 weeks and no relevant benefit is payable at that time, it shall cease to be in force.

(3) If during the period specified in paragraph (1) the former child again becomes a child within the meaning of the Act or the parent concerned again becomes a person with care and relevant benefit is payable to or in respect of that parent, a reduction in the amount of that benefit shall be made in accordance with the provisions of paragraphs (3) to (7) of regulation 38.

\subsection[49. Notice of termination of a reduced benefit direction]{Notice of termination of a reduced benefit direction}

49.—(1) Where a direction ceases to be in force under the provisions of regulations 41 to 44 or 46 to 48, or is suspended under the provisions of regulation 48, a child support officer shall serve notice of such termination or suspension, as the case may be, on the adjudication officer and shall specify the date on which the direction ceases to be in force or is suspended, as the case may be.

(2) Any notice served under paragraph (1) shall set out the reasons why the direction has ceased to be in force or has been suspended.

(3) The parent concerned shall be served with a copy of any notice served under paragraph (1).

\amendment{
Reg. 49A revoked (13.1.97) by the Child Support (Miscellaneous Amendments) (No. 2) Regulations 1996 reg. 9.

Under the Child Support (Miscellaneous Amendments) (No. 2) Regulations 1996 reg. 16(3), reg. 49A (as shown below) continues to apply to a reduced benefit direction which as of 13.1.97 is suspended under reg. 40A:
\begin{quotation}
\subsection*{\itshape Notice of termination of suspension of a reduced benefit direction}

49A.—(1) Where the deductions relevant for the purposes of regulation 40A cease to be made, a child support officer shall, so far as is reasonably practicable, serve on the parent concerned notice of the date from which the suspension of the reduced benefit direction shall cease.

(2) The adjudication officer shall be served with a copy of any notice served under paragraph (1).
\end{quotation}
}

% Reg 49A revoked (13.1.97) by SI 1996/3196 reg 9
%% Reg 49A inserted (22.1.96) by SI 1995/3261 reg 39
%\subsection[49A. Notice of termination of suspension of a reduced benefit direction]{Notice of termination of suspension of a reduced benefit direction}
%
%49A.—(1) Where the deductions relevant for the purposes of regulation 40A cease to be made, a child support officer shall, so far as is reasonably practicable, serve on the parent concerned notice of the date from which the suspension of the reduced benefit direction shall cease.
%
%(2) The adjudication officer shall be served with a copy of any notice served under paragraph (1).
%
%\amendment{
%Reg. 49A inserted (22.1.96) by the Child Support (Miscellaneous Amendments) (No. 2) Regulations 1995 reg. 39.
%}

\subsection[50. Rounding provisions]{Rounding provisions}

50.  Where any calculation made under this Part of these Regulations results in a fraction of a penny, that fraction shall be treated as a penny if it exceeds one half, and shall otherwise be disregarded.

\section[Part X --- Miscellaneous provisions]{Part X\\*Miscellaneous provisions}

\renewcommand\parthead{--- Part X}

\subsection[51. Persons who are not persons with care]{Persons who are not persons with care}

51.—(1) For the purposes of the Act the following categories of person shall not be persons with care—
\begin{enumerate}\item[]
($a$) a local authority;

($b$) a person with whom a child who is looked after by a local authority is placed by that authority under the provisions of the Children Act 1989\footnote{\frenchspacing 1989 c. 41.}
except where that person is a parent of such a child and the local authority allow the child to live with that parent under section 23(5) of that Act; %Words inserted (5.4.93) by SI 1993/913 reg 15

($c$) in Scotland, a person with whom a child is boarded out by a local authority under the provisions of section 21 of the Social Work (Scotland) Act 1968\footnote{\frenchspacing 1968 c. 49.}.
\end{enumerate}

(2) In paragraph (1) above—
\begin{enumerate}\item[]
“local authority” means, in relation to England and Wales, the council of a county, a metropolitan district, a London Borough or the Common Council of the City of London and, in relation to Scotland, a regional council or an islands council;

“a child who is looked after by a local authority” has the same meaning as in section 22 of the Children Act 1989.
\end{enumerate}

\amendment{
Words inserted in reg. 51(1)($b$) (5.4.93) by the Child Support (Miscellaneous Amendments) Regulations 1993 reg. 15.
}

\subsection[52. Terminations of maintenance assessments]{Terminations of maintenance assessments}

52.—(1) Where the Secretary of State is satisfied that a question arises as to whether a maintenance assessment has ceased to have effect under the provisions of paragraph 16(1)($a$) to ($d$) of Schedule 1 to the Act, he shall refer that question (a “termination question”) to a child support officer.

(2) Where a child support officer has made a decision on a termination question (a “termination decision”) he shall immediately notify the following persons of his decision, so far as that is reasonably practicable—
\begin{enumerate}\item[]
($a$) in a case falling within paragraph 16(1)($a$) of Schedule 1 to the Act, the surviving relevant persons;

($b$) in a case falling within paragraph 16(1)($b$), ($c$) or ($d$) of Schedule 1 to the Act, the relevant persons.
\end{enumerate}

(3) Any notification under paragraph (2) shall give the reasons for the termination decision made, include information as to the provisions of section 18 of the Act, and explain the provisions of paragraph (4).

(4) The persons specified in paragraph (2) may apply to the Secretary of State for a review of a termination decision as if it were a case falling within section 18 of the Act and, subject to the modifications set out in paragraph (5), section 18(5) to (9) and (11) of the Act shall apply to such a review.

(5) The modifications referred to in paragraph (4) are—
\begin{enumerate}\item[]
($a$) section 18(6) of the Act shall have effect as if for “the refusal, assessment or cancellation” there is substituted “the termination decision”;

($b$) section 18(9) of the Act shall have effect as if for “a maintenance assessment or (as the case may be) a fresh maintenance assessment” there is substituted “a different termination decision”.
\end{enumerate}

(6) The provisions of regulation 24 as to time limits for an application for a review of a decision by a child support officer shall apply to reviews under paragraph (4).

(7) Where a child support officer has completed a review of a termination decision he shall immediately notify the persons specified in paragraph (2), so far as that is reasonably practicable, of the review decision, give the reasons for that decision in writing, and notify them of the provisions of section 20 of the Act.

(8) Where a case falls within regulation 19 of the Maintenance Assessments and Special Cases Regulations and both absent parents have made an application for a maintenance assessment under section 4 of the Act, the Secretary of State shall be under the duty imposed by section 4(6) of the Act only if both absent parents have, under section 4(5) of the Act, requested the Secretary of State to cease acting under section 4 of the Act.

\subsection[53. Authorisation of representative]{Authorisation of representative}

53.—(1) A person may authorise a representative, whether or not legally qualified, to receive notices and other documents on his behalf and to act on his behalf in relation to the making of applications and the supply of information under any provision of the Act or these Regulations.

(2) Where a person has authorised a representative for the purposes of paragraph (1) who is not legally qualified, he shall confirm that authorisation in writing to the Secretary of State.

\subsection[54. Correction of accidental errors in decisions]{Correction of accidental errors in decisions}

54.—(1) Subject to regulation 56, accidental errors in any decision or record of a decision may at any time be corrected by a child support officer and a correction made to, or to the record of, a decision shall be deemed to be part of the decision or of that record.

(2) A child support officer who has made a correction under the provisions of paragraph (1) shall immediately notify the persons who were notified of the decision that has been corrected, so far as that is reasonably practicable.

\subsection[55. Setting aside of decisions on certain grounds]{Setting aside of decisions on certain grounds}

55.—(1) Subject to paragraph (7) and regulation 56, on an application made by a relevant person, a decision may be set aside by a child support officer on the grounds that the interests of justice so require, and in particular that a relevant document in relation to that decision was not sent to, or was not received at an appropriate time by the person making the application or his representative or was sent but not received at an appropriate time by the child support officer who gave the decision.

(2) Any application made under paragraph (1) shall be in writing, shall include a statement of the grounds for the application, and shall be made by giving or sending it to the Secretary of State within 28 days of the date of notification of the decision in question.

(3) Where an application to set aside a decision is being considered by a child support officer under paragraph (1), he shall notify the relevant persons other than the applicant of the application and they shall be given 14 days to make representations as to that application.

(4) The provisions of regulation 25(6) shall apply to notifications under paragraph (5).

(5) A child support officer who has made a determination on an application to set aside a decision shall immediately notify the relevant persons, so far as that is reasonably practicable, and shall give the reasons for his determination in writing.

(6) For the purposes of determining an application to set aside a decision under this regulation, there shall be disregarded regulation 1(6)($b$) and any provision in any enactment or instrument to the effect that any notice or other document required or authorised to be given or sent to any person shall be deemed to have been given or sent if it was sent by post to that person’s last known or notified address.

(7) The provisions of paragraphs (1) to (6) shall not apply to any document given or sent under any provision of Part IX.

\subsection[56. Provisions common to regulations 54 and 55]{Provisions common to regulations 54 and 55}

56.—(1) In determining whether the time limits specified in regulation 17, 19, 24 or 25 have been complied with, there shall be disregarded any day falling before the day on which notification is given of a correction made to, or to the record of, a decision made under regulation 54 or on which notification is given that a decision shall not be set aside following an application made under regulation 55, as the case may be.

(2) The power to correct errors under regulation 54 or set aside decisions under regulation 55 shall not be taken to limit any other powers to correct errors or set aside decisions that are exercisable apart from these Regulations.

%Reg 57 inserted (5.4.93) by SI 1993/913 reg 16
\subsection[57. Action by the Secretary of State on receipt of an application under section 17 or 18 of the Act where a question as to the entitlement to benefit arises]{Action by the Secretary of State on receipt of an application under section 17 or 18 of the Act where a question as to the entitlement to benefit arises}

57.—(1) Where an application for a review under section 17 or 18 of the Act has been made to the Secretary of State and he is of the opinion that the application gives rise to a question as to the entitlement to benefit of any person, he may disclose the information contained in that application to an adjudication officer or, in the case of housing benefit or council tax benefit, to an appropriate authority.

(2) Where the Secretary of State discloses information under paragraph (1), he need not refer the application to a child support officer earlier than the expiration of a period of 28 days beginning with the date prescribed in paragraph (3).

(3) The date prescribed for the purposes of paragraph (2) is the second day after the date the Secretary of State receives the application for a review under section 17 or 18 of the Act, excluding any Saturday, Sunday, or any day which is a bank holiday in England, Wales, Scotland or Northern Ireland under the Banking and Financial Dealings Act 1971\footnote{\frenchspacing 1971 c. 80.}.

(4) In this regulation---
\begin{enumerate}\item[]
($a$) “benefit” is to be construed in accordance with the benefit Acts;

($b$) “appropriate authority” means–
\begin{enumerate}\item[]
(i) in relation to housing benefit, the housing or local authority concerned; and

(ii) in relation to council tax benefit the billing authority or, in Scotland, the levying authority.
\end{enumerate}
\end{enumerate}

\amendment{
Reg. 57 inserted (5.4.93) by the Child Support (Miscellaneous Amendments) Regulations 1993 reg. 16.
}

\bigskip

Signed by authority of the Secretary of State for Social Security.

{\raggedleft
\emph{Alistair Burt}\\*Parliamentary Under-Secretary of State,\\*Department of Social Security

}

20th July 1992

\small

\part*{S C H E D U L E S}

\part[Schedule 1 --- Meaning of ``child'' for the purposes of the Act]{Schedule 1\\*Meaning of ``child'' for the purposes of the Act}

\renewcommand\parthead{--- Schedule 1}

\subsection*{Persons of 16 or 17 years of age who are not in full-time non-advanced education}

1.—(1) Subject to sub-paragraph (3), the conditions which must be satisfied for a person to be a child within section 55(1)($c$) of the Act are—
\begin{enumerate}\item[]
($a$) the person is registered for work or for training under youth training with—
\begin{enumerate}\item[]
(i) the Department of Employment;

(ii) the Ministry of Defence;

(iii) in England and Wales, a local education authority within the meaning of the Education Acts 1944 to 1992;

(iv) in Scotland, an education authority within the meaning of section 135(1) of the Education (Scotland) Act 1980\footnote{\frenchspacing 1980 c. 44.} (interpretation); or

(v) for the purposes of applying Council Regulation (EEC)
No.\ 1408/\hspace{0pt}71\footnote{\frenchspacing O.J. No.L149, 5.7.1971; Regulations (EEC) No. 1408/71 and No. 574/72 were restated in amended form in Council Regulation (EEC) No. 2001/83 (O.J. No.L230, 22.8.1983) and further amended by Council Regulations (EEC) Nos. 1660/85 (O.J. No.L160, 20.6.1985); 1661/85 (O.J. No.L160, 20.6.1985); Commission Regulation (EEC) No. 513/86 (O.J. No.L51, 28.2.1986) and Articles 60 and 220 of, and Point 1, Part VIII of Annex I to the Act of Accession to the European Communities of Spain and Portugal; 3811/86 (O.J. No.L355, 16.12.1986).}, any corresponding body in another member State;
\end{enumerate}

($b$) the person is not engaged in remunerative work, other than work of a temporary nature that is due to cease before the end of the extension period which applies in the case of that person;

($c$) the extension period which applies in the case of that person has not expired; and

($d$) immediately before the extension period begins, the person is a child for the purposes of the Act without regard to this paragraph.
\end{enumerate}

(2) For the purposes of paragraphs ($b$), ($c$) and ($d$) of sub-paragraph (1), the extension period—
\begin{enumerate}\item[]
($a$) begins on the first day of the week in which the person would no longer be a child for the purposes of the Act but for this paragraph; and

($b$) where a person ceases to fall within section 55(1)($a$) of the Act or within paragraph 5—
\begin{enumerate}\item[]
(i) on or after the first Monday in September, but before the first Monday in January of the following year, ends on the last day of the week which falls immediately before the week which includes the first Monday in January in that year;

(ii) on or after the first Monday in January but before the Monday following Easter Monday in that year, ends on the last day of the week which falls 12 weeks after the week which includes the first Monday in January in that year;

(iii) at any other time of the year, ends on the last day of the week which falls 12 weeks after the week which includes the Monday following Easter Monday in that year.
\end{enumerate}
\end{enumerate}

(3) A person shall not be a child for the purposes of the Act under this paragraph if—
\begin{enumerate}\item[]
($a$) he is engaged in training under youth training; or

($b$) he is entitled to income support
or income-based jobseeker’s allowance.  % Words inserted (7.10.96) by SI 1996/1345 reg 5(12)
\end{enumerate}

\amendment{
Words inserted in para. 1(3)(b) (7.10.96) by the Social Security and Child Support (Jobseeker's Allowance) (Consequential Amendments) Regulations 1996 reg. 5(12).
}

\subsection*{Meaning of “advanced education” for the purposes of section 55 of the Act}

2.  For the purposes of section 55 of the Act “advanced education” means education of the following description—
\begin{enumerate}\item[]
($a$) a course in preparation for a degree, a Diploma of Higher Education, a higher national diploma, a higher national diploma or higher national certificate of the Business and 
%Technician 
Technology % word substituted (5.4.93) by SI 1993/913 reg 17
Education Council or the Scottish Vocational Education Council or a teaching qualification; or

($b$) any other course which is of a standard above that of an ordinary national diploma, a national diploma or national certificate of the Business and 
%Technician 
Technology % word substituted (5.4.93) by SI 1993/913 reg 17
 Education Council or the Scottish Vocational Education Council, the advanced level of the General Certificate of Education, a Scottish certificate of education (higher level) or a Scottish certificate of sixth year studies.
\end{enumerate}

\amendment{
Words substituted in para. 2 (5.4.93) by the Child Support (Miscellaneous Amendments) Regulations 1993 reg. 27.
}

\subsection*{Circumstances in which education is to be treated as full-time education}

3.  For the purposes of section 55 of the Act education shall be treated as being full-time if it is received by a person attending a course of education at a recognised educational establishment and the time spent receiving instruction or tuition, undertaking supervised study, examination or practical work or taking part in any exercise, experiment or project for which provision is made in the curriculum of the course, exceeds 12 hours per week, so however that in calculating the time spent in pursuit of the course, no account shall be taken of time occupied by meal breaks or spent on unsupervised study, whether undertaken on or off the premises of the educational establishment.

\subsection*{Interruption of full-time education}

4.—(1) Subject to sub-paragraph (2), in determining whether a person falls within section 55(1)($b$) of the Act no account shall be taken of a period (whether beginning before or after the person concerned attains age 16) of up to 6 months of any interruption to the extent to which it is accepted that the interruption is attributable to a cause which is reasonable in the particular circumstances of the case; and where the interruption or its continuance is attributable to the illness or disability of mind or body of the person concerned, the period of 6 months may be extended for such further period as a child support officer considers reasonable in the particular circumstances of the case.

(2) The provisions of sub-paragraph (1) shall not apply to any period of interruption of a person’s full-time education which is likely to be followed immediately or which is followed immediately by a period during which—
\begin{enumerate}\item[]
($a$) provision is made for the training of that person, and for an allowance to be payable to that person, under youth training; or

($b$) he is receiving education by virtue of his employment or of any office held by him.
\end{enumerate}

\subsection*{Circumstances in which a person who has ceased to receive full-time education is to be treated as continuing to fall within section 55(1) of the Act}

5.—(1) Subject to sub-paragraphs (2) and (5), a person who has ceased to receive full-time education (which is not advanced education) shall, if—
\begin{enumerate}\item[]
($a$) he is under the age of 16 when he so ceases, from the date on which he attains that age; or

($b$) he is 16 or over when he so ceases, from the date on which he so ceases,
\end{enumerate}
be treated as continuing to fall within section 55(1) of the Act up to and including the week including the terminal date or if he attains the age of 19 on or before that date up to and including the week including the last Monday before he attains that age.

(2) In the case of a person specified in sub-paragraph (1)($a$) or ($b$) who had not attained the upper limit of compulsory school age when he ceased to receive full-time education, the terminal date in his case shall be that specified in paragraph ($a$), ($b$) or ($c$) of sub-paragraph (3), whichever next follows the date on which he would have attained that age.

(3) In this paragraph the “terminal date” means—
\begin{enumerate}\item[]
($a$) the first Monday in January; or

($b$) the Monday following Easter Monday; or

($c$) the first Monday in September, education ceased.
\end{enumerate}
whichever first occurs after the date on which the person’s said education ceased.

(4) In this paragraph “compulsory school age” means—
\begin{enumerate}\item[]
($a$) in England and Wales, compulsory school age as determined in accordance with section 9 of the Education Act 1962\footnote{\frenchspacing 10 \& 11 Eliz. 2 c. 12 as amended by the Education (School-leaving Dates) Act 1976 (c. 5).};

($b$) in Scotland, school age as determined in accordance with sections 31 and 33 of the Education (Scotland) Act 1980\footnote{\frenchspacing 1980 c. 44.}.
\end{enumerate}

(5) A person shall not be treated as continuing to fall within section 55(1) of the Act under this paragraph if he is engaged in remunerative work, other than work of a temporary nature that is due to cease before the terminal date.

(6) Subject to sub-paragraphs (5) and (8), a person whose name was entered as a candidate for any external examination in connection with full-time education (which is not advanced education), which he was receiving at that time, shall so long as his name continued to be so entered before ceasing to receive such education be treated as continuing to fall within section 55(1) of the Act for any week in the period specified in sub-paragraph (7).

(7) Subject to sub-paragraph (8), the period specified for the purposes of sub-paragraph (6) is the period beginning with the date when that person ceased to receive such education ending with—
\begin{enumerate}\item[]
($a$) whichever of the dates in sub-paragraph (3) first occurs after the conclusion of the examination (or the last of them, if there are more than one); or

($b$) the expiry of the week which includes the last Monday before his 19th birthday.
\end{enumerate}

(8) The period specified in sub-paragraph (7) shall, in the case of a person who has not attained the age of 16 when he so ceased, begin with the date on which he attained that age.

\subsection*{Interpretation}

6.  In this Schedule—
\begin{enumerate}\item[]
“Education Acts 1944 to 1992” has the meaning prescribed in section 94(2) of the Further and Higher Education Act 1992\footnote{\frenchspacing 1992 c. 13.};

“remunerative work” means work of not less than 24 hours a week—
\begin{enumerate}\item[]
($a$) in respect of which payment is made; or

($b$) which is done in expectation of payment;
\end{enumerate}

“week” means a period of 7 days beginning with a Monday;

“youth training” means—
\begin{enumerate}\item[]
($a$) arrangements made under section 2 of the Employment and Training Act 1973\footnote{\frenchspacing 1973 c. 50; section 2 is substituted by the Employment Act 1988 (c. 19), section 25(1).} (functions of the Secretary of State) or section 2 of the Enterprise and New Towns (Scotland) Act 1990\footnote{\frenchspacing 1990 c. 35.};

($b$) arrangements made by the Secretary of State for persons enlisted in Her Majesty’s forces for any special term of service specified in regulations made under section 2 of the Armed Forces Act 1966\footnote{\frenchspacing 1966 c. 45.} (power of Defence Council to make regulations as to engagement of persons in regular forces); or

($c$) for the purposes of the application of Council Regulation (EEC) No.\ 1408/71, any corresponding provisions operated in another member State,
\end{enumerate}
for purposes which include the training of persons who, at the begining of their training, are under the age of 18.
\end{enumerate}

\part[Schedule 2 --- Multiple applications]{Schedule 2\\*Multiple applications}

\renewcommand\parthead{--- Schedule 2}

\subsection*{No maintenance assessment in force: more than one application for a maintenance assessment by the same person under section 4 or 6 or under sections 4 and 6 of the Act}

1.—(1) Where a person makes an effective application for a maintenance assessment under section 4 or 6 of the Act and, before that assessment is made, makes a subsequent effective application under that section with respect to the same absent parent or person with care, as the case may be, those applications shall be treated as a single application.

(2) Where a parent with care makes an effective application for a maintenance assessment—
\begin{enumerate}\item[]
($a$) under section 4 of the Act; or

($b$) under section 6 of the Act,
\end{enumerate}
and, before that assessment is made, makes a subsequent effective application—
\begin{enumerate}\item[]
($c$) in a case falling within paragraph ($a$), under section 6 of the Act; or

($d$) in a case falling within paragraph ($b$), under section 4 of the Act,
\end{enumerate}
with respect to the same absent parent, those applications shall, if the parent with care does not cease to fall within section 6(1) of the Act, be treated as a single application under section 6 of the Act, and shall otherwise be treated as a single application under section 4 of the Act.

\subsection*{No maintenance assessment in force: more than one application by a child under section 7 of the Act}

2.  Where a child makes an effective application for a maintenance assessment under section 7 of the Act and, before that assessment is made, makes a subsequent effective application under that section with respect to the same person with care and absent parent, both applications shall be treated as a single application for a maintenance assessment.

\subsection*{No maintenance assessment in force: applications by different persons for a maintenance assessment}

3.—(1) Where the Secretary of State receives more than one effective application for a maintenance assessment with respect to the same person with care and absent parent, he shall refer each such application to a child support officer and, if no maintenance assessment has been made in relation to any of the applications, the child support officer shall determine which application he shall proceed with in accordance with sub-paragraphs (2) to (11).

(2) Where there is an application by a person with care under section 4 or 6 of the Act and an application by an absent parent under section 4 of the Act, the child support officer shall proceed with the application of the person with care.

(3) Where there is an application for a maintenance assessment by a qualifying child under section 7 of the Act and a subsequent application is made with respect to that child by a person who is, with respect to that child, a person with care or an absent parent, the child support officer shall proceed with the application of that person with care or absent parent, as the case may be.

(4) Where, in a case falling within sub-paragraph (3), there is more than one subsequent application, the child support officer shall apply the provisions of sub-paragraph (2), (8), (9) or (11), as is appropriate in the circumstances of the case, to determine which application he shall proceed with.

(5) Where there is an application for a maintenance assessment by more than one qualifying child under section 7 of the Act in relation to the same person with care and absent parent, the child support officer shall proceed with the application of the elder or, as the case may be, eldest of the qualifying children.

(6) Where a case is to be treated as a special case for the purposes of the Act under regulation 19 of the Maintenance Assessments and Special Cases Regulations (both parents are absent) and an effective application is received from each absent parent, the child support officer shall proceed with both applications, treating them as a single application for a maintenance assessment.

(7) Where, under the provisions of regulation 20 of the Maintenance Assessments and Special Cases Regulations (persons treated as absent parents), two persons are to be treated as absent parents and an effective application is received from each such person, the child support officer shall proceed with both applications, treating them as a single application for a maintenance assessment.

(8) Where there is an application under section 6 of the Act by a parent with care and an application under section 4 of the Act by another person with care who has parental responsibility for (or, in Scotland, parental rights over) the qualifying child or qualifying children with respect to whom the application under section 6 of the Act was made, the child support officer shall proceed with the application under section 6 of the Act by the parent with care.

(9) Where—
\begin{enumerate}\item[]
($a$) more than one person with care makes an application for a maintenance assessment under section 4 of the Act in respect of the same qualifying child or qualifying children (whether or not any of those applications is also in respect of other qualifying children);

\begin{sloppypar}
($b$) each such person has parental responsibility for (or, in Scotland, parental rights over) that child or children; and
\end{sloppypar}

($c$) under the provisions of regulation 20 of the Maintenance Assessments and Special Cases Regulations one of those persons is to be treated as an absent parent,
\end{enumerate}
the child support officer shall proceed with the application of the person who does not fall to be treated as an absent parent under the provisions of regulation 20 of those Regulations.

\begin{sloppypar}
(10) Where, in a case falling within sub-paragraph (9), there is more than one person who does not fall to be treated as an absent parent under the provisions of regulation 20 of those Regulations, the child support officer shall apply the provisions of paragraph (11) to determine which application he shall proceed with.
\end{sloppypar}

(11) Where—
\begin{enumerate}\item[]
($a$) more than one person with care makes an application for a maintenance assessment under section 4 of the Act in respect of the same qualifying child or qualifying children (whether or not any of those applications is also in respect of other qualifying children); and

($b$) either—
\begin{enumerate}\item[]
(i) none of those persons has parental responsibility for (or, in Scotland, parental rights over) that child or children; or

(ii) the case falls within sub-paragraph (9)($b$),
\end{enumerate}
but the child support officer has not been able to determine which application he is to proceed with under the provisions of sub-paragraph (9),
\end{enumerate}
the child support officer shall proceed with the application of the principal provider of day to day care, as determined in accordance with sub-paragraph (12).

(12) Where—
\begin{enumerate}\item[]
($a$) the applications are in respect of one qualifying child, the application of that person with care with whom the child spends the greater or, as the case may be, the greatest proportion of his time;

($b$) the applications are in respect of more than one qualifying child, the application of that person with care with whom the children spend the greater or, as the case may be, the greatest proportion of their time, taking account of the time each qualifying child spends with each of the persons with care in question;

($c$) the child support officer cannot determine which application he is to proceed with under paragraph ($a$) or ($b$), and child benefit is paid in respect of the qualifying child or qualifying children to one but not any other of the applicants, the application of the applicant to whom child benefit is paid;

($d$) the child support officer cannot determine which application he is to proceed with under paragraph ($a$), ($b$) or ($c$), the application of that applicant who in the opinion of the child support officer is the principal provider of day to day care for the child or children in question.
\end{enumerate}

(13) Subject to sub-paragraph (14), where, in any case falling within sub-paragraphs (2) to (11), the applications are not in respect of identical qualifying children, the application that the child support officer is to proceed with as determined by those paragraphs shall be treated as an application with respect to all of the qualifying children with respect to whom the applications were made.

(14) Where the child support officer is satisfied that the same person with care does not provide the principal day to day care for all of the qualifying children with respect to whom an assessment would but for the provisions of this paragraph be made under sub-paragraph (13), he shall make separate assessments in relation to each person with care providing such principal day to day care.

\subsection*{Maintenance assessment in force: subsequent application for a maintenance assessment with respect to the same persons}

4.  Where a maintenance assessment is in force and a subsequent application is made under the same section of the Act for an assessment with respect to the same person with care, absent parent, and qualifying child or qualifying children as those with respect to whom the assessment in force has been made, that application shall not be proceeded with unless the Secretary of State treats that application as an application for a review under section 17 of the Act.

\subsection*{Maintenance assessment in force: subsequent application for a maintenance assessment under section 6 of the Act}

5.  Where a maintenance assessment is in force following an application under section 4 or 7 of the Act and the person with care makes an application under section 6 of the Act, any maintenance assessment made in response to that application shall replace the assessment currently in force.

\subsection*{Maintenance assessment in force: subsequent application for a maintenance assessment in respect of additional children}

6.—%(1) Where a maintenance assessment made in response to an application by an absent parent under section 4 of the Act is in force and that assessment is not in respect of all of his children who are in the care of the person with care with respect to whom that assessment has been made, an assessment made in response to an application by that person with care under section 4 of the Act with respect to—
%\begin{enumerate}\item[]
%($a$) the children in respect of whom the assessment currently in force was made; and
%
%($b$) the additional child or, as the case may be, one or more of the additional children in that person’s care who are children of that absent parent,
%\end{enumerate}
%shall replace the assessment currently in force.
%
%(2) Where—
%\begin{enumerate}\item[]
%($a$) a maintenance assessment made in response to an application by an absent parent or a person with care under section 4 of the Act is in force;
%
%($b$) that assessment is not in respect of all of the children of the absent parent who are in the care of the person with respect to whom that assessment has been made; and
%
%($c$) the absent parent makes a subsequent application in respect of an additional qualifying child or additional qualifying children of his in the care of the same person,
%\end{enumerate}
%that application shall be treated as an application for a maintenance assessment in respect of all of the qualifying children concerned, and the assessment made shall replace the assessment currently in force.
%
% Para 6(1) substituted for para 6(1)--(2) (19.1.98) by SI 1998/58 reg 41(2)
(1) Where there is in force a maintenance assessment made in response to an application under section 4 of the Act by an absent parent or person with care and that assessment is not in respect of all of the absent parent’s children who are in the care of the person with care with respect to whom that assessment was made—
\begin{enumerate}\item[]
($a$) if that absent parent or that person with care makes an application under section 4 of the Act with respect to the children in respect of whom the assessment currently in force was made and the additional child or one or more of the additional children in the care of that person with care who are children of that absent parent, an assessment made in response to that application shall replace the assessment currently in force;

($b$) if that absent parent or that person with care makes an application under section 4 of the Act in respect of an additional qualifying child or additional qualifying children of that absent parent in the care of that person with care, that application shall be treated as an application for a maintenance assessment in respect of all the qualifying children concerned and the assessment made shall replace the assessment currently in force.
\end{enumerate}

(3) Where a maintenance assessment made in response to an application by a child under section 7 of the Act is in force and the person with care 
or the absent parent  % Words inserted (19.1.98) by SI 1998/58 reg 41(3)(a)
of that child makes an application for a maintenance assessment under section 4 of the Act in respect of 
%that child and all other children of the absent parent who are in her care
one or more 
  %children of the absent parent who are in her care, 
  children of that absent parent who are in the care of that person with care  % Words substituted (19.1.98) by SI 1998/58 reg 41(3)(b)
that application shall be treated as an application for a maintenance assessment with respect to all the children of the absent parent who are in her care, and % Words substituted (5.4.93) by SI 1993/913 reg 18
that assessment shall replace the assessment currently in force.

\amendment{
Words substituted in para. 6(3) (5.4.93) by the Child Support (Miscellaneous Amendments) Regulations 1993 reg. 18.

Words inserted and substituted in para. 6(3) and para. 6(1) substituted for para. 6(1), (2) (19.1.98) by the Child Support (Miscellaneous Amendments) Regulations 1998 reg. 41.
}

\part{Explanatory Note}

\renewcommand\parthead{--- Explanatory Note}

\subsection*{(This note is not part of the Regulations)}

 These Regulations provide for various procedural matters relating to applications for a maintenance assessment under the Child Support Act 1991 (“the Act”), reviews of an assessment, and effective dates, and make provision in respect of reduced benefit directions.

  Regulation 1 and Schedule 1 contains interpretation provisions and provisions relating to the service and receipt of documents.

  Regulations 2 to 4 and Schedule 2 set out the procedures in relation to applications for a maintenance assessment under the Act.

  Regulations 5 to 7 set out the procedures when an effective application for a maintenance assessment has been made.

  Regulations 8 and 9 prescribe the amount of child support maintenance payable under an interim maintenance assessment and make other provision in respect of interim maintenance assessments.

  Regulations 10 to 16 make provision in respect of notifications following decisions by child support officers.

  Regulations 17 and 18 make provision as to periodical reviews under section 16 of the Act, and regulations 19 to 23 make provision as to reviews on a change of circumstances under section 17 or 19 of the Act.

  Regulations 24 to 29 make provision as to reviews, under section 18 of the Act, of a decision by a child support officer.

  Regulations 30 to 33 prescribe the effective dates of maintenance assessments, and the dates on which maintenance assessments cease to have effect, and make provision in respect of maintenance periods.

  Regulations 34 to 50 make provision as to the amount and duration of reduced benefit directions, following a failure to comply with obligations imposed by section 6 of the Act.

  Regulations 51 to 56 make provision in respect of miscellaneous matters: regulation 51 prescribes persons who are not persons with care for the purposes of the Act; regulation 52 makes provision in respect of terminations of maintenance assessments; regulation 53 provides for the authorisation of representatives; and regulations 54 to 56 provide for the correction of accidental errors in decisions and the setting aside of decisions on certain grounds.





\end{document}