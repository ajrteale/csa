\documentclass[12pt,a4paper]{article}

\newcommand\regstitle{The State Pension Credit (Decisions and Appeals-Amendments) Regulations 2003}

\newcommand\regsnumber{2003/1581}

%\opt{newrules}{
\title{\regstitle}
%}

%\opt{2012rules}{
%\title{Child Maintenance and Other Payments Act 2008\\(2012 scheme version)}
%}

\author{S.I.\ 2003 No.\ 1581}

\date{Made
17th June 2003\\
%Laid before Parliament
%15th December 2003\\
Coming into force
in accordance with regulation 1
}

%\opt{oldrules}{\newcommand\versionyear{1993}}
%\opt{newrules}{\newcommand\versionyear{2003}}
%\opt{2012rules}{\newcommand\versionyear{2012}}

\usepackage{csa-regs}

\setlength\headheight{27.57402pt}

\begin{document}

\maketitle

\noindent
Whereas a draft of this Instrument was laid before Parliament in accordance with section 80(1)($b$)  of the Social Security Act 1998\footnote{1998 c.\ 14 (“the 1998 Act”).} and paragraph 20(4) of Schedule 7 to the Child Support, Pensions and Social Security Act 2000\footnote{2000 c.\ 19 (“the 2000 Act”).} and approved by resolution of each House of Parliament.

Now, therefore, the Secretary of State for Work and Pensions, in exercise of the powers conferred upon him by sections 79(1) and 84 of, and paragraph 9 of Schedule 2 to, the Social Security Act 1998\footnote{Section 84 is cited because of the meaning ascribed to the word “prescribe”.} and paragraphs 6(4)($b$), 20(1)($b$)  and 23(1) of Schedule 7 to the Child Support, Pensions and Social Security Act 2000\footnote{Paragraph 23(1) is cited because of the meaning ascribed to the word “prescribed”.} and of all other powers enabling him in that behalf, after consultation, in respect of regulation 3, with both the Council on Tribunals in accordance with section 8(1) of the Tribunals and Inquiries Act 1992\footnote{1992 c.\ 53.} and with organisations appearing to him to be representative of the authorities concerned\footnote{\emph{See} section 176(1) of the Social Security Administration Act 1992 (c.\ 5) (“the Administration Act”).} and after agreement by the Social Security Advisory Committee that proposals in respect of these Regulations should not be referred to it\footnote{\emph{See} sections 170 and 173(1)($b$) of the Administration Act; paragraph 104 of Schedule 7 to the 1998 Act added Chapter II of Part I of that Act, and section 73 of the 2000 Act added Schedule 7 to that Act, to the list of “relevant enactments” in respect of which regulations must normally be referred to the Committee.}, hereby makes the following Regulations: 

{\sloppy

\tableofcontents

}

\bigskip

\setcounter{secnumdepth}{-2}

\subsection[1. Citation and commencement]{Citation and commencement}

1.  These Regulations may be cited as the State Pension Credit (Decisions and Appeals-Amendments) Regulations 2003 and shall come into force on the day after the day on which they are made.

\subsection[2. Amendment of the Social Security and Child Support (Decisions and Appeals) Regulations 1999]{Amendment of the Social Security and Child Support (Decisions and Appeals) Regulations 1999}

2.  In paragraph 5 of Schedule 2 to the Social Security and Child Support (Decisions and Appeals) Regulations 1999\footnote{S.I.\ 1999/991; paragraph 5 was substituted by S.I.\ 2002/1379.} (decisions against which no appeal lies: claims and payments)—
\begin{enumerate}\item[]
($a$) after sub-paragraph ($b$), there shall be inserted the following sub-paragraph—
\begin{quotation}
“($bb$) regulation 4D (making a claim for state pension credit) or 4E\footnote{Regulations 4D and 4E were inserted by S.I.\ 2002/3019.} (making a claim before attaining the qualifying age);”;
\end{quotation}

($b$) after sub-paragraph ($m$), there shall be inserted the following sub-paragraph—
\begin{quotation}
“($mm$) regulation 26B\footnote{Regulation 26B was inserted by S.I.\ 2002/3019.} (payment of state pension credit);”.
\end{quotation}
\end{enumerate}

\subsection[3. Amendment of the Housing Benefit and Council Tax Benefit (Decisions and Appeals) Regulations 2001]{Amendment of the Housing Benefit and Council Tax Benefit (Decisions and Appeals) Regulations 2001}

3.  At the end of the Schedule to the Housing Benefit and Council Tax Benefit (Decisions and Appeals) Regulations 2001\footnote{S.I.\ 2001/1002 to which there are amendments which are not relevant to these Regulations.} (decisions against which no appeal lies), there shall be added the following paragraph—
\begin{quotation}
“6.  No appeal shall lie against the calculation or estimate of the claimant's, or the claimant’s partner's, income or capital used by a relevant authority in accordance with regulation 23(1) of the Housing Benefit Regulations or regulation 15(1) of the Council Tax Benefit Regulations (calculation of claimant’s income in savings credit only cases), as modified, in both cases, by the Housing Benefit and Council Tax Benefit (State Pension Credit) Regulations 2003\footnote{S.I.\ 2003/325.}.”.
\end{quotation}

\bigskip

Signed 
by authority of the Secretary of State for Work and Pensions.

{\raggedleft
\emph{Malcolm Wicks}\\*Minister of State,\\*Department of Work and Pensions

}

%St Andrew's House, Edinburgh

%Dated
17th June 2003

\small

\part{Explanatory Note}

\renewcommand\parthead{— Explanatory Note}

\subsection*{(This note is not part of the Regulations)}

These Regulations amend—
\begin{itemize}
\item    the Social Security and Child Support (Decisions and Appeals) Regulations 1999 (S.I.\ 1999/991) to provide that certain administrative decisions relating to claims for, and payment of, state pension credit, do not attract a right of appeal (regulation 2);
\item
    the Housing Benefit and Council Tax Benefit (Decisions and Appeals) Regulations 2001 (S.I.\ 2001/1002) to provide that no appeal lies against calculations and estimates of the income and capital of a claimant for housing benefit or council tax benefit or his partner, where that claimant or his partner has been awarded state pension credit comprising only the savings credit (regulation 3). 
\end{itemize}

These Regulations do not impose a charge on business. 

\end{document}