\documentclass[12pt,a4paper]{article}

\newcommand\regstitle{The Social Security Amendment (Joint Claims) Regulations 2001}

\newcommand\regsnumber{2001/518}

%\opt{newrules}{
\title{\regstitle}
%}

%\opt{2012rules}{
%\title{Child Maintenance and~Other Payments Act 2008\\(2012 scheme version)}
%}

\author{S.I.\ 2001 No.\ 518}

\date{Made
22nd February 2001\\
Laid before Parliament
26th February 2001\\
Coming into~force
19th March 2001
}

%\opt{oldrules}{\newcommand\versionyear{1993}}
%\opt{newrules}{\newcommand\versionyear{2003}}
%\opt{2012rules}{\newcommand\versionyear{2012}}

\usepackage{csa-regs}

\setlength\headheight{27.61603pt}

%\hbadness=10000

\begin{document}

\maketitle

\noindent
The Secretary of State for Education and~Employment, in relation to regulation~2(2) to (4) and~(6) and~the Secretary of State for Social Security in relation to the remainder of these Regulations, in exercise of the powers conferred by sections~1(2C) and~(4), 4(5), 5(3), 21, 35(1) and~36(1), (2) and~(4) of, and~paragraph~8A(1) of Schedule~1 to, the Jobseekers Act 1995\footnote{1995 c.~18; section~1(4) was amended, and~section~1(2C) and~paragraph~8A of Schedule~1 inserted, by section~59 of, and~Schedule~7 to, the Welfare Reform and~Pensions Act 1999 (c.~30), paragraphs 2(3) and~(4) and~16(2). Section~35(1) is an interpretation provision and~is cited because of the meaning ascribed to the words “prescribed” and~“regulations”.}, sections~22(5), 122(1), 136(5)($b$), 137(1) and~175(1) and~(3) of the Social Security Contributions and~Benefits Act 1992\footnote{1992 c.~4; section~22(5) was amended by paragraph~22 of Schedule 2 to the Jobseekers Act 1995; sections~122(1) and 137(1) are cited because of the meaning ascribed to the words “prescribe” and “prescribed” respectively.}, sections~8(1)($c$), 10,~12(1)($b$),~39(2) and~79(1) and~(4) of, and~paragraph~9 of Schedule~3 to, the Social Security Act 1998\footnote{1998 c.~14; section~39(2) applies section~191 of the Social Security Administration Act 1992 (c.~5) to the powers in Chapter II and is cited because of the meaning ascribed in section~191 to the word “prescribe”.} and~sections~5(1)($i$), 189(1) and~(4) and~191 of the Social Security Administration Act 1992\footnote{Section~191 is an interpretation provision and is cited because of the meaning ascribed to the word “prescribe”.} and~of all other powers enabling each of them in that behalf, after consultation, in relation to regulation~6, with organisations appearing to him to be representative of the authorities concerned\footnote{\emph{See} section~176(1)($b$) of the Social Security Administration Act 1992.}, by this Instrument which contains only regulations made by virtue of, or consequential upon, section~59 of, and~Schedule~7 to, the Welfare Reform and~Pensions Act 1999 and~which is made before the end of the period of six months beginning with the coming into force of those provisions\footnote{\emph{See} section~173(5)(b) of the Social Security Administration Act 1992 and section~91(3) of the Welfare Reform and Pensions Act 1999.}, hereby make the following Regulations: 

{\sloppy

\tableofcontents

}

\bigskip

\setcounter{secnumdepth}{-2}

\subsection[1. Citation, commencement and~interpretation]{Citation, commencement and~interpretation}

1.---(1)  These Regulations shall be cited as the Social Security Amendment (Joint Claims) Regulations 2001 and~shall, subject to paragraph~(2) below, come into force on 19th March 2001.

(2) Regulation 2(1) to (4), (6) and~(7) of these Regulations shall come into force immediately after the Jobseeker’s Allowance (Joint Claims) Regulations 2000\footnote{S.I.~2000/1978.}.

\subsection[2, Amendment of the Jobseeker’s Allowance Regulations 1996]{Amendment of the Jobseeker’s Allowance Regulations 1996}

2.---(1)  The Jobseeker’s Allowance Regulations 1996\footnote{S.I.~1996/207; the relevant amending instrument is S.I.~2000/1978.} shall be amended in accordance with the following paragraphs of this regulation.

(2) In regulation~3A(1) (prescribed description of a joint-claim couple for the purposes of section~1(4)), after the words “where at least one member” there shall be inserted the words “is aged 18 or over and”.

(3) For regulation~3D(1)($c$)  (further circumstances in which a joint-claim couple may be entitled to a joint-claim jobseeker’s allowance), there shall be substituted the following sub-paragraph—
\begin{quotation}
“($c$) the other member satisfies the condition in section~1(2)($e$)  and~($h$)  but is not required to satisfy the other conditions in section~1(2B)($b$)\footnote{Section 1(2B) was inserted by section~59 of, and Schedule 7 to, the Welfare Reform and Pensions Act 1999 (c.~30), paragraph~2(3).} because, subject to paragraph~(3), he is a person to whom any paragraph~in Schedule~A1 applies; and”.
\end{quotation}

(4) In regulation~3E(2) (entitlement of a member of a joint-claim couple to a jobseeker’s allowance without a claim being made jointly by the couple), in both sub-paragraphs ($g$)  and~($k$), for the words “more than 16 hours” there shall be substituted the words “16 hours or more”.

(5) At the end of regulation~47(4)($b$)(ii)  (jobseeking period), there shall be added the words “or on which the claimant is a member of a joint-claim couple and~a joint-claim jobseeker’s allowance is not payable or is reduced because he is subject to sanctions by virtue of section~20A”.

(6) In Schedule~A1 (categories of members of a joint-claim couple who are not required to satisfy the conditions in section~1(2B)($b$))—
\begin{enumerate}\item[]
($a$) in the heading, for “Regulation 3D(1)($c$)(iii)” there shall be substituted “Regulation 3D(1)($c$)”;

($b$) for paragraph~2(1) and~(2) there shall be substituted the following sub-paragraphs—
\begin{quotation}
“2.---(1)  A member—
\begin{enumerate}\item[]
($a$) who, at the date of claim, is aged 16 or over but under 19 and~is receiving full-time education for the purposes of section~142 of the Benefits Act;

($b$) who, at the date of claim, is a full-time student; or

($c$) to whom ($a$)  or ($b$)  does not apply but to whom sub-paragraph~(1A) or (2) does apply.
\end{enumerate}

(1A) This sub-paragraph~applies to a member who—
\begin{enumerate}\item[]
($a$) as at the date of claim—
\begin{enumerate}\item[]
(i) had applied to an educational establishment to commence a full-time course of study commencing from the beginning of the next academic term or, as the case may be, the next academic year after the date of claim and~that application has not been rejected; or\looseness=-1

(ii) had been allocated a place on a full-time course of study commencing from the beginning of the next academic term or, as the case may be, the next academic year; and
\end{enumerate}

($b$) is either—
\begin{enumerate}\item[]
(i) aged 16 or over but under 19 and~is receiving full-time education for the purposes of section~142 of the Benefits Act; or

(ii) a full-time student.
\end{enumerate}
\end{enumerate}

(2) This sub-paragraph~applies to a member who has applied to an educational establishment to commence a full-time course of study (other than a course of study beyond a first degree course or a comparable course)—
\begin{enumerate}\item[]
($a$) within one month of—
\begin{enumerate}\item[]
(i) the last day of a previous course of study; or

(ii) the day on which the member received examination results relating to a previous course of study; and
\end{enumerate}

($b$) who is either—
\begin{enumerate}\item[]
(i) aged 16 or over but under 19 and~is receiving full-time education for the purposes of section~142 of the Benefits Act; or

(ii) a full-time student.”.
\end{enumerate}
\end{enumerate}
\end{quotation}
\end{enumerate}

(7) In the first column of paragraph~20M(4)($i$)  of Schedule~1 (applicable amounts), for “20I(3)” there shall be substituted “20I(4)”.

\subsection[3. Amendment of the Social Security (Credits) Regulations 1975]{Amendment of the Social Security (Credits) Regulations 1975}

3.  In regulation~8A(5) of the Social Security (Credits) Regulations 1975\footnote{S.I.~1975/556; regulation~8A was inserted by S.I.~1996/2367.} (credits for unemployment)—
\begin{enumerate}\item[]
($a$) after sub-paragraph~($c$)  there shall be inserted the following sub-paragraph—
\begin{quotation}
“($cc$) a week in respect of which a joint-claim jobseeker’s allowance was not payable or was reduced pursuant to section~20A of that Act because the person was subject to sanctions for the purposes of that section, even though the couple of which he was a member satisfied the conditions for entitlement to that allowance;”;
\end{quotation}

($b$) after sub-paragraph~($d$)  there shall be inserted the following sub-paragraph—
\begin{quotation}
“($dd$) a week in respect of which a joint-claim jobseeker’s allowance was payable in respect of a joint-claim couple of which the person is a member only by virtue of regulation~146C of the Jobseeker’s Allowance Regulations 1996\footnote{Regulation 146C was inserted by S.I.~2000/1978.} (circumstances in which a joint-claim jobseeker’s allowance is payable where a joint-claim couple is a couple in hardship);”.
\end{quotation}
\end{enumerate}

\subsection[4. Amendment of the Social Security and~Child Support (Decisions and~Appeals) Regulations 1999]{Amendment of the Social Security and~Child Support (Decisions and~Appeals) Regulations 1999}

4.  In the Social Security and~Child Support (Decisions and~Appeals) Regulations 1999\footnote{S.I.~1999/991; the relevant amending instrument is S.I.~2000/1596.}—
\begin{enumerate}\item[]
($a$) in regulation~1(3) (citation, commencement and~interpretation) after the definition of “the Jobseeker’s Allowance Regulations” there shall be inserted the following definitions—
\begin{quotation}
    ““a joint-claim couple” has the same meaning as in section~1(4) of the Jobseekers Act 1995;

    “a joint-claim jobseeker’s allowance” has the same meaning as in section~1(4) of the Jobseekers Act 1995;”; 
\end{quotation}

($b$) in regulation~26 (decisions against which an appeal lies), after paragraph~($c$)  there shall be inserted the following—
\begin{quotation}
    “; or

    ($d$) 
    under section~59 of, and~Schedule~7 to, the Welfare Reform and~Pensions Act 1999\footnote{1999 c.~30.} (couples to make joint-claim for jobseeker’s allowance) where one member of the couple is working and~the Secretary of State has decided that both members of the couple are not engaged in remunerative work,”; 
\end{quotation}

($c$) in paragraph~8 of Schedule~3A—
\begin{enumerate}\item[]
(i) in sub-paragraph~($a$), after the words “section~3(1)($a$)” there shall be inserted the words “or 3A(1)($a$)”;

(ii) after paragraph~($d$), there shall be inserted the following—
\begin{quotation}
    “; or

    ($e$) 
    a joint-claim couple ceases to be a married or an unmarried couple,”. 
\end{quotation}
\end{enumerate}
\end{enumerate}

\subsection[5. Amendment of the Social Security (Claims and~Payments) Regulations 1987]{Amendment of the Social Security (Claims and~Payments) Regulations 1987}

5.  After regulation~30 of the Social Security (Claims and~Payments) Regulations 1987\footnote{S.I.~1987/1968.}, there shall be inserted the following regulation—
\begin{quotation}
\subsection*{“Payments of arrears of joint-claim jobseeker’s allowance where the nominated person can no longer be traced}

30A.  Where—
\begin{enumerate}\item[]
($a$) an award of joint-claim jobseeker’s allowance has been awarded to a joint-claim couple;

($b$) that couple ceases to be a joint-claim couple; and

($c$) the member of the joint-claim couple nominated for the purposes of section~3B of the Jobseekers Act cannot be traced,\looseness=-1
\end{enumerate}
arrears on the award of joint-claim jobseeker’s allowance shall be paid to the other member of the former joint-claim couple.”.
\end{quotation}

\subsection[6. Amendment of the Housing Benefit (General) Regulations 1987 and~of the Council Tax Benefit (General) Regulations 1992]{Amendment of the Housing Benefit (General) Regulations 1987 and~of the Council Tax Benefit (General) Regulations 1992}

6.  In both the Housing Benefit (General) Regulations 1987\footnote{S.I.~1987/1971; the relevant amending instrument is S.I.~1996/1510.} and~the Council Tax Benefit (General) Regulations 1992\footnote{S.I.~1992/1814; the relevant amending instrument is S.I.~1996/1510.}—
\begin{enumerate}\item[]
($a$) after paragraph~4 of Schedule~4 (sums to be disregarded in the calculation of income other than earnings), there shall be inserted the following paragraph—
\begin{quotation}
“4A.  Where the claimant is a member of a joint-claim couple for the purposes of the Jobseekers Act 1995 and~his partner is on an income-based jobseeker’s allowance, the whole of the claimant’s income.”;
\end{quotation}

($b$) after paragraph~5 of Schedule~5 (capital to be disregarded), there shall be inserted the following paragraph—
\begin{quotation}
“5A.  Where the claimant is a member of a joint-claim couple for the purposes of the Jobseekers Act 1995 and~his partner is on an income-based jobseeker’s allowance, the whole of the claimant’s capital.”.
\end{quotation}
\end{enumerate}

\bigskip

Signed in relation to regulation~2(2) to (4) and~(6) by authority of the Secretary of State for Education and~Employment. 
%I concur
%By authority of the Lord Chancellor

{\raggedleft
\emph{Michael Willis}\\*
%Secretary
%Minister
Parliamentary Under-Secretary 
of State,\\*Department 
for~Education and~Employment

}

20th February 2001

\bigskip

Signed 
in relation to the remainder of these Regulations 
by authority of the 
Secretary of State for~Work and~Pensions.
%I concur
%By authority of the Lord Chancellor

{\raggedleft
\emph{P.~Hollis}\\*
%Secretary
%Minister
Parliamentary Under-Secretary 
of State,\\*Department 
for~Work and~Pensions

}

22nd February 2001

\small

\part{Explanatory Note}

\renewcommand\parthead{— Explanatory Note}

\subsection*{(This note is not part of the Regulations)}

These Regulations are made by virtue of, or in consequence of, provisions in section~59 of, and~Schedule~7 to, the Welfare Reform and~Pensions Act 1999 (c.~30). The Instrument is made before the end of the period of six months beginning with the coming into force of those provisions; the regulations in it are therefore exempted from the requirement in section~172(1) of the Social Security Administration Act 1992 (c.~5) to refer proposals to make these Regulations to the Social Security Advisory Committee and~are made without reference to that Committee.

Regulation 2 amends the Jobseeker’s Allowance Regulations 1996 (S.I.~1996/\hspace{0pt}207) by—
\begin{enumerate}\item[]
($a$) clarifying the definition of a joint-claim couple so that it includes a couple where at least one member must be aged 18 or over (regulation~2(2));

($b$) clarifying the position as to when a joint-claim couple may be entitled to a joint-claim jobseeker’s allowance whilst one member is not required to satisfy the conditions in section~1(2B)($b$)  of the Jobseekers Act 1995 (c.~18) and~making a consequential amendment  (regulation~2(3) and~(6)($a$));

($c$) ensuring that joint claims do not have to be made in certain circumstances where one member of the couple is working 16 hours per week (regulation~2(4));

($d$) providing that days where a member of a joint-claim couple satisfies the conditions for entitlement to a contribution-based jobseeker’s allowance and~a joint-claim jobseeker’s allowance is not payable or is reduced because he is subject to sanctions for the purposes of section~20A of the Jobseekers Act 1995, shall be treated as a day of entitlement to a contribution-based jobseeker’s allowance (regulation~2(5));\looseness=-1

($e$) making an amendment which clarifies the rule as to when those receiving full-time education or those who are full-time students may be exempt from having to comply with the jobseeking conditions (regulation~2(6)($b$));

($f$) correcting a reference in paragraph~20M of Schedule~1 (regulation~2(7)).
\end{enumerate}

Regulation 3 amends regulation~8A of the Social Security (Credits) Regulations 1975 (S.I.~1975/556) by preventing credits from being awarded where a joint-claim jobseeker’s allowance is not payable or is reduced because a person is subject to sanctions pursuant to section~20A of the Jobseekers Act 1995 or where such a couple is only receiving a joint-claim jobseeker’s allowance because they are a couple in hardship.\looseness=-1

Regulation 4 amends the Social Security and~Child Support (Decisions and~Appeals) Regulations 1999 (S.I.~1999/991) by inserting definitions into regulation~1(3), providing a new right of appeal in regulation~26 against a decision that a couple are required to make a joint claim and~the reason for that decision is that one member of the couple who is working is not engaged in remunerative work and~providing in paragraph~8 of Schedule~3A of the Regulations an effective date where a joint-claim couple separate.\looseness=-1

Regulation 5 amends the Social Security (Claims and~Payments) Regulations 1987 (S.I.~1987/1968) by providing that where a member of a joint-claim couple to whom a joint-claim jobseeker’s allowance is payable disappears, that allowance shall be payable to the other member of that couple.

\enlargethispage{\baselineskip}

\begin{sloppypar}
Regulation 6 amends the Housing Benefit (General) Regulations 1987 (S.I.~1987/1971) and~the Council Tax Benefit (General) Regulations 1992 (S.I.~1992/1814) by ensuring that where a claimant for those benefits is a member of a joint-claim couple and~his partner is getting income-based jobseeker’s allowance, the whole of his income and~capital will nevertheless be disregarded.
\end{sloppypar}

These Regulations do not impose any charge on business. 

\end{document}