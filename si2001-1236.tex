\documentclass[12pt,a4paper]{article}

\newcommand\regstitle{The Child Support (Civil Imprisonment) (Scotland) Regulations 2001}

\newcommand\regsnumber{2001/1236}

%\opt{newrules}{
\title{\regstitle}
%}

%\opt{2012rules}{
%\title{Child Maintenance and Other Payments Act 2008\\(2012 scheme version)}
%}

\author{S.I.\ 2001 No.\ 1236 (S. 3)}

\date{Made
27th March 2001\\
Laid before Parliament
3rd April 2001\\
Coming into force
24th April 2001
}

%\opt{oldrules}{\newcommand\versionyear{1993}}
%\opt{newrules}{\newcommand\versionyear{2003}}
%\opt{2012rules}{\newcommand\versionyear{2012}}

\usepackage{csa-regs}

\setlength\headheight{27.57402pt}

\begin{document}

\maketitle

\noindent
The Secretary of State for Social Security, in exercise of the powers conferred upon him by section 40A of the Child Support Act 1991\footnote{1991 c.\ 48. Section 40A was inserted by the Child Support, Pensions and Social Security Act 2000 (c.\ 19), section 17(2).} and of all other powers enabling him in that behalf, hereby makes the following Regulations: 

{\sloppy

\tableofcontents

}

\bigskip

\setcounter{secnumdepth}{-2}

\subsection[1. Citation, commencement and interpretation]{Citation, commencement and interpretation}

1.---(1)  These Regulations may be cited as the Child Support (Civil Imprisonment) (Scotland) Regulations 2001 and shall come into force on 24th April 2001.

(2) In these Regulations the “1991 Act” means the Child Support Act 1991.

\subsection[2. Expenses of commitment to prison]{Expenses of commitment to prison}

2.  The amount to be included in the warrant under section 40A(2)(ii) of the 1991 Act (sheriff’s warrant for committal to prison of liable person) in respect of the expenses of commitment shall be such amount as, in the view of the sheriff, is equal to the expenses reasonably incurred by the Secretary of State in respect of the expenses of commitment.

\subsection[3. Reduction of period of imprisonment]{Reduction of period of imprisonment}

3.---(1)  For the purposes of subsection (6) of section 40A of the 1991 Act (reduction of period of imprisonment for part payment) the following paragraphs shall apply.

(2) Where, after the sheriff has issued a warrant for committal to prison under section 40A of the 1991 Act, part payment of the amount stated in the warrant is made, the period of imprisonment specified in the warrant shall be reduced proportionately so that for the period of imprisonment specified in the warrant, there shall be substituted a period of imprisonment of such number of days as bears the same proportion to the number of days specified in the warrant as the amount remaining unpaid under the warrant bears to the amount specified in the warrant.

(3) Where the part payment is of such an amount as would, under paragraph (2) above, reduce the period of imprisonment to such number of days as have already been served (or would be so served in the course of the day of payment) the period of imprisonment shall be reduced to the period already served plus one day. 

\bigskip

Signed 
by authority of the Secretary of State for Social Security.

{\raggedleft
\emph{P.~Hollis}\\*Parliamentary Under-Secretary of State,\\*Department of Social Security

}

%Dated
27th March 2001

\small

\part{Explanatory Note}

\renewcommand\parthead{— Explanatory Note}

\subsection*{(This note is not part of the Regulations)}

Section 40A of the Child Support Act 1991 (inserted by section 17 of the Child Support, Pensions and Social Security Act 2000) provides for a sheriff committing to prison a liable person who fails to pay under a liability order. The sheriff may do so only if he is satisfied that the liable person has wilfilly refused to pay or has culpably neglected to pay.

If the sheriff orders imprisonment it will be in respect of an amount specified in the sheriff’s warrant. The amount will be made up of the arrears of child maintenance and an amount to be determined in respect of the expenses of commitment. Regulation 2 sets out the manner of determination of the amount of expenses of commitment.

Section 40A specifies that the maximum period of imprisonment which the sheriff may impose is six weeks. However, that period may be reduced where there is part payment of the amount in respect of which the warrant for committal to prison was issued. Regulation 3 provides for the reduction of that period of imprisonment.

Where part payment of the amount is made, the period of imprisonment will be proportionately reduced. That period will be the number of days which bears the same proportion to the number of days specified in the warrant as the amount remaining unpaid in respect of the warrant bears to the amount in respect of which the warrant was originally granted. Where part payment would result in the period of imprisonment being reduced to the number of days already served, the period of imprisonment will be the period already served plus one further day.

These Regulations for Scotland make provision comparable to that already made for England and Wales in Regulation 34 of the Child Support (Collection and Enforcement) Regulations 1992 (SI 1992/1989). 

\end{document}