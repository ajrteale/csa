\documentclass[12pt,a4paper]{article}

\newcommand\regstitle{The Child Maintenance and Other Payments Act 2008 (Commencement No.~12 and Savings Provisions) and the Welfare Reform Act 2012 (Commencement No.~15) Order 2013}

\newcommand\regsnumber{2013/2947}

\title{\regstitle}

\author{S.I.\ 2013 No.\ 2947 (C.~121)}

\date{Made
%19th November 2013\\
%%%Laid before Parliament
%%%27th June 2013\\
%Coming into force
19th November 2013
}

%\opt{oldrules}{\newcommand\versionyear{1993}}
%\opt{newrules}{\newcommand\versionyear{2003}}
%\opt{2012rules}{\newcommand\versionyear{2012}}

\usepackage{csa-regs}

\setlength\headheight{42.11603pt}

%\hbadness=10000

\begin{document}

\maketitle

\enlargethispage{\baselineskip}

\noindent
The Secretary of State for Work and Pensions, in exercise of the powers conferred by section 62(3) and (4) of the Child Maintenance and Other Payments Act 2008\footnote{2008 c.~6.} and section 150(3) and (4) of the Welfare Reform Act 2012\footnote{2012 c.~5.}, makes the following Order: 

{\sloppy

\tableofcontents

}

\bigskip

\setcounter{secnumdepth}{-2}

\subsection[1. Citation and Interpretation]{Citation and Interpretation}

1.---(1)  This Order may be cited as the Child Maintenance and Other Payments Act 2008 (Commencement No.~12 and Savings Provisions) and the Welfare Reform Act 2012 (Commencement No.~15) Order 2013.

(2) In this Order—
\begin{enumerate}\item[]
“1991 Act” means the Child Support Act 1991\footnote{1991 c.~48.};

“2000 Act” means the Child Support, Pensions and Social Security Act 2000\footnote{2000 c.~19. The relevant amendments to the Child Support Act 1991 (“the 1991 Act”) were made by sections 1 and 26 of, and paragraph 11(1), (2) and (20) of Schedule 3 to, the Child Support, Pensions and Social Security Act 2000 (“the 2000 Act”).};

“2008 Act” means the Child Maintenance and Other Payments Act 2008;

“new calculation rules” means Part~I of Schedule 1 to the 1991 Act as amended by the provisions of the 2008 Act specified in article 2.
\end{enumerate}

(3) In this Order, a reference to an existing case is to a case in which there is—
\begin{enumerate}\item[]
($a$) a maintenance assessment in force;

($b$) a maintenance calculation, made otherwise than in accordance with the new calculation rules, in force;

($c$) an application for a maintenance assessment which has been made but not determined; or

($d$) an application for a maintenance calculation, which falls to be made otherwise than in accordance with the new calculation rules, which has been made but not determined.
\end{enumerate}

(4) In this Order, subject to paragraph (5)—
\begin{enumerate}\item[]
“maintenance calculation”, “non-resident parent”, “person with care” and “qualifying child” have the meanings given in the 1991 Act\footnote{The definition of “maintenance calculation” was substituted for the definition of “maintenance assessment” in section 54 of the 1991 Act by section 26 of, and paragraph 11(1) and (20)($d$)  of Schedule 3 to, the 2000 Act. The term “non-resident parent” was substituted for the term “absent parent” by section 26 of, and paragraph 11(1) and (2) of Schedule 3 to, the 2000 Act. The definition of “qualifying child” in section 3(1) of the 1991 Act was amended by section 26 of, and paragraph 11(1) and (2) of Schedule 3 to, the 2000 Act.};

“absent parent” and “maintenance assessment” have the meanings given in the 1991 Act before its amendment by the 2000 Act.
\end{enumerate}

(5) In this Order—
\begin{enumerate}\item[]
($a$) a reference to a non-resident parent includes reference to a person who is—
\begin{enumerate}\item[]
(i) alleged to be the non-resident parent for the purposes of an application for child support maintenance under the 1991 Act, or

(ii) treated as the non resident parent for the purposes of the 1991 Act\footnote{A person may be treated as a non-resident parent for the purposes of the 1991 Act under regulation 50(2) of the Child Support Maintenance Calculation Regulations 2012 (S.I.~2012/2677) or regulation 8(2) of the Child Support (Maintenance Calculations and Special Cases) Regulations 2000 (S.I.~2001/155).}; and
\end{enumerate}

($b$) a reference to an absent parent includes reference to a person who is—
\begin{enumerate}\item[]
(i) alleged to be the absent parent for the purposes of an application for child support maintenance under the 1991 Act, or

(ii) treated as the absent parent for the purposes of the 1991 Act\footnote{A person may be treated as an absent parent for the purposes of the 1991 Act under regulation 20(2) of the Child Support (Maintenance Assessments and Special Cases) Regulations 1992 (S.I.~1992/1815).}.
\end{enumerate}
\end{enumerate}

\subsection[2. Appointed day for the coming into effect of the new calculation rules]{Appointed day for the coming into effect of the new calculation rules}

2.  The following provisions of the 2008 Act come into force for all purposes, in so far as those provisions are not already in force, on 25th November 2013, except where the saving in article 3 applies—
\begin{enumerate}\item[]
($a$) sections 16 (changes to the calculation of maintenance), 17 (power to regulate supersession) and 18 (determination of applications for a variation) and Schedule 4 (changes to the calculation of maintenance);

($b$) section 57(1) and paragraph 1(1) of Schedule 7 (minor and consequential amendments), so far as relating to the sub-paragraphs of paragraph~1 referred to in paragraph ($c$);

($c$) paragraph 1(2), (28) and (29) of Schedule 7;

($d$) section 58 (repeals), so far as relating to the entries referred to in paragraph ($e$); and

($e$) in Schedule 8 (repeals), the entries relating to—
\begin{enumerate}\item[]
(i) Schedule 1 (maintenance calculations) to the 1991 Act;

(ii) Schedule 24 (social security, child support and tax credits) to the Civil Partnership Act 2004\footnote{2004 c.~33.}.
\end{enumerate}
\end{enumerate}

\subsection[3. New calculation rules not to apply to existing cases]{New calculation rules not to apply to existing cases}

3.---(1)  The provisions of the 2008 Act referred to in article 2 do not apply to an existing case.

(2) Subject to articles 4 and 5, the saving in paragraph (1) applies until liability in relation to the maintenance assessment or maintenance calculation ceases to accrue (whether because the applicant has requested the Secretary of State cease acting, because it has otherwise ceased or because the power in paragraph 1(1) of Schedule 5 to the 2008 Act has been exercised in relation to that case) or, where an application has been made but not determined, until the date notified to the person with care as the date on which the Secretary of State has ceased acting.

\subsection[4. Thirteen week linking rule where case closed voluntarily]{Thirteen week linking rule where case closed voluntarily}

4.---(1)  This article applies where—
\begin{enumerate}\item[]
($a$) on or after the date on which this Order is made, the applicant in relation to an existing case makes a request to the Secretary of State under section 4(5) or 7(6) of the 1991 Act\footnote{Section 4 was amended by section 18(1) of the Child Support Act 1995 (c.~34), paragraph 19 of Schedule 7, and Schedule 8, to the Social Security Act 1998 (c.~14) (“the 1998 Act”), sections 1(2) and 2(1) to (3) of, and paragraph 11(1) to (3) of Schedule 3 to, the 2000 Act, section 35(1) of, and Schedule 8 to, the Child Maintenance and Other Payments Act 2008 (c.~6) (“the 2008 Act”) and S.I.~2012/2007. Section 7 was amended by paragraph 21 of Schedule 7, and Schedule 8, to the 1998 Act, section 1(2) of, and paragraph 11(1),(2) and (4) of Schedule 3 to, the 2000 Act, section 35(2) of the 2008 Act and S.I.~2012/2007.} to cease acting; and

($b$) a further application is made under section 4 or 7 of the 1991 Act in relation to the same qualifying child, person with care and non-resident parent before the expiry of 13 weeks from the date of cessation of action by the Secretary of State.
\end{enumerate}

(2) Where this article applies, for the purposes of calculating the amount of child maintenance in response to the further application referred to in paragraph (1)($b$)  the saving in article 3 continues to apply (and so that new application is an existing case).

(3) For the purposes of paragraph (1)($b$)—
\begin{enumerate}\item[]
($a$) the date an application is made is—
\begin{enumerate}\item[]
(i) where made by telephone, the date it (the telephone call) is made; and

(ii) where made by post, the date of receipt by the Secretary of State; and
\end{enumerate}

($b$) the date of cessation of action by the Secretary of State is—
\begin{enumerate}\item[]
(i) where there is a maintenance assessment or maintenance calculation in force, the date on which the liability under that assessment or calculation ends as a result of the request to cease acting;

(ii) where there is an application still to be determined, the date notified to the person with care as the date on which the Secretary of State has ceased acting.
\end{enumerate}
\end{enumerate}

\subsection[5. New calculation rules to apply to existing cases related to a new application]{New calculation rules to apply to existing cases related to a new application}

5.---(1)  Subject to article 4, where an application is made under section 4 or~7 of the 1991 Act on or after 25th November 2013, but before the Secretary of State begins to exercise the power in paragraph 1(1) of Schedule 5 to the 2008 Act, and that application satisfies paragraph (2) or (3), paragraph (5) applies.

(2) An application satisfies this paragraph where—
\begin{enumerate}\item[]
($a$) the non-resident parent in relation to the application is also the non-resident parent or absent parent in relation to an existing case; and

($b$) the person with care in relation to the application is not the person with care in relation to the existing case in sub-paragraph ($a$).
\end{enumerate}

(3) An application satisfies this paragraph where—
\begin{enumerate}\item[]
($a$) the non-resident parent in relation to the application (“$\mathcal{A}$”) is a partner of a non-resident parent or an absent parent in relation to an existing case (“$\mathcal{B}$”); and

($b$) $\mathcal{A}$ or $\mathcal{B}$ is in receipt of a prescribed benefit.
\end{enumerate}

(4) For the purposes of paragraph (3)—
\begin{enumerate}\item[]
“partner” has the meaning given in paragraph 10C(4) (references to various terms) of Schedule 1 to the 1991 Act as amended by the 2000 Act\footnote{Part~I of Schedule 1 to the 1991 Act was substituted by section 1(3) of, and Schedule 1 to, the 2000 Act.};
%($a$) 

“prescribed benefit” means a benefit prescribed, or treated as prescribed, for the purposes of paragraph 4(1)($c$)  (flat rate) of Schedule 1 to the 1991 Act as amended by the 2000 Act.
\end{enumerate}

(5) Where this paragraph applies, the saving in article 3 ceases to apply in relation to the existing case referred to in paragraph (2)($a$)  or (3)($a$)  (and accordingly the new calculation rules apply) on the date from which the maintenance calculation made in response to the application referred to in paragraph (1) takes effect.

\subsection[6. Appointed day for provisions of the Welfare Reform Act 2012]{Appointed day for provisions of the Welfare Reform Act 2012}

6.  Sections 136 (supporting maintenance agreements), 140 (fees) and 141 (review of fees regulations) of the Welfare Reform Act 2012\footnote{2012 c.~5. Section 136 was amended by S.I.~2012/2007.} come into force on 25th November 2013. 

\bigskip

\pagebreak[3]

Signed 
by authority of the 
Secretary of State for~Work and~Pensions.
%I concur
%By authority of the Lord Chancellor

{\raggedleft
\emph{Steve Webb}\\*
%Secretary
Minister
%Parliamentary Under Secretary 
of State\\*Department 
for~Work and~Pensions

}

19th November 2013

\small

\part{Explanatory Note}

\renewcommand\parthead{— Explanatory Note}

\subsection*{(This note is not part of the Regulations)}

This Order brings into force provisions of the Child Maintenance and Other Payments Act 2008 (c.~6) (“the 2008 Act”) for the purpose of applying new rules for calculating child support maintenance to all cases other than existing cases. It also brings into force provisions in the Welfare Reform Act 2012 (c.~5).

The 2008 Act amends the statutory scheme for calculation, collection and enforcement of child support maintenance, as originally set out in the Child Support Act 1991 (c.~48) (“the 1991 Act”) and amended by the Child Support, Pensions and Social Security Act 2000 (c.~19) (“the 2000 Act”). The amendments made by the 2000 Act were brought into force by the Child Support, Pensions and Social Security Act 2000 (Commencement No.~12) Order 2003 (S.I.~2003/192) for new applications after 3rd March 2003 and for existing cases related to such applications. However, the original provisions of the 1991 Act remained in force for a substantial number of cases, effectively resulting in two separate schemes. The further amendments made by the 2008 Act, together with the 2000 Act amendments, constitute a third scheme (“the new calculation rules”).\looseness=1

The Child Maintenance and Other Payments Act 2008 (Commencement No.~10 and Transitional Provisions) Order 2012 (S.I.~2012/3042) brought the amendments made by the 2008 Act into force for new applications made on or after 10th December 2012 where there were four or more children with the same person with care and non-resident parent and no existing case with the same person with care and non-resident parent and for existing cases related to such applications. The Child Maintenance and Other Payments Act 2008 (Commencement No.~11 and Transitional Provisions) Order 2013 (S.I.~2013/1860) brought those amendments into force for new applications made on or after 29th July 2013 where there were two or three children with the same person with care and non-resident parent and no existing case with the same person with care and non-resident parent and for existing cases related to such applications.

Article 2 brings into force on 25th November 2013 the amendments made by the 2008 Act, in so far as they are not yet in force, for all purposes except where the saving in article 3 applies.

Article 3 is a savings provision which provides that the commencement of the 2008 Act provisions does not apply to existing cases. The calculation rules applicable to the case before the commencement of the 2008 Act provisions continue to apply until liability ceases to accrue in relation to that case (subject to article 4), until the Secretary of State notifies the parent with care that he has ceased acting where there is an application that has been made but not determined, or until article 5 applies.

Article 4 provides for a thirteen week linking rule so that the new calculation rules will not apply to an existing case if a person asks the Secretary of State to cease acting and reapplies to the statutory scheme within 13 weeks.

Article 5 provides for the application of the new calculation rules to an existing case where: a new application is made in relation to the non-resident parent in the existing case and there is a different parent with care; or a non-resident parent is the partner of a non-resident parent named in a new application and either of those non-resident parents claims a prescribed benefit.

Article 6 brings section 136 of the Welfare Reform Act 2012 into force. Section 136 inserts provision into the 1991 Act which enables the Secretary of State to take appropriate steps to encourage the making and keeping of family-based maintenance agreements, including inviting an applicant to consider whether it is possible to make such an agreement, before an application for child support maintenance is accepted. Article 6 also brings into force sections 140 and 141 of the Welfare Reform Act 2012, which amend section 6 of the 2008 Act (section 6 gives the Secretary of State power to make regulations about charging fees). Section 140 clarifies that section 6 can be used to make provision for the apportionment of fees and waiver of fees and matters to be taken into account in determining such apportionment and waiver. Section 141 requires the Secretary of State to review the effect of the first regulations made under section 6 and provides when that review must be carried out and the action that must be taken following that review. 

\end{document}