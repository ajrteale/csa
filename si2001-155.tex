\documentclass[12pt,a4paper]{article}

\newcommand\regstitle{The Child Support (Maintenance Calculations and Special Cases) Regulations 2000}

\newcommand\regsnumber{2001/155}

%\opt{newrules}{
\title{\regstitle}
%}

%\opt{2012rules}{
%\title{Child Maintenance and Other Payments Act 2008\\(2012 scheme version)}
%}

\author{S.I. 2001 No. 155}

\date{Made 18th January 2001\\
%Laid before Parliament 6th December 2000\\
Coming into force in accordance with regulation~1(4) and (5)
}

%\opt{oldrules}{\newcommand\versionyear{1993}}
%\opt{newrules}{\newcommand\versionyear{2003}}
%\opt{2012rules}{\newcommand\versionyear{2012}}

\usepackage{csa-regs}

\setlength\headheight{27.61603pt}

\begin{document}

\maketitle

\amendment{
Regs. revoked (10.12.12 for 2012 scheme cases only) by the Child Support (Meaning of Child and New Calculation Rules) (Consequential and Miscellaneous Amendment) Regulations 2012 reg.~10(e).
}

\medskip

\noindent
Whereas a draft of this Instrument was laid before Parliament in accordance with section~52(2) and (2A) of the Child Support Act 1991\footnote{1991 c.\ 48. Section~52 is amended by section~25 of the Child Support, Pensions and Social Security Act 2000 (c.\ 19).} and approved by a resolution of each House of Parliament:

Now, therefore, the Secretary of State for Social Security, in exercise of the powers conferred upon him by sections~14(1) and (1A), 42, 51, 52(4) and 54 of, and paragraphs 3(2), 4(1)($b$)  and ($c$), 4(3), 5($a$), 7(3), 9, 10 and 10C(2)($b$)  of Schedule~1 to, the Child Support Act 1991\footnote{Sections~14(1) and (1A), 51 and 54 are amended and Part~I of Schedule~1 is substituted respectively by sections~12 of and paragraphs 11(7), 11(9) and 11(20) of Schedule~3 to and section~1 of the Child Support, Pensions and Social Security Act 2000. Section~54 is cited because of the meaning ascribed to the word “prescribed”.}, and of all other powers enabling him in that behalf, hereby makes the following Regulations: 

\enlargethispage{\baselineskip}

{\sloppy

\tableofcontents

}

\bigskip

\setcounter{secnumdepth}{-2}

\section[Part~I --- General]{Part~I\\*General}

\renewcommand\parthead{--- Part~I}

\subsection[1. Citation, commencement and interpretation]{Citation, commencement and interpretation}

1.---(1)  These Regulations may be cited as the Child Support (Maintenance Calculations and Special Cases) Regulations 2000.

(2) In these Regulations, unless the context otherwise requires—
\begin{enumerate}\item[]
“the Act” means the Child Support Act 1991;

% Definitions of ``care home'', ``care home service'' inserted (5.11.03) by SI 2003/2779 reg 6(2)(c)
“care home” has the meaning assigned to it by section~3 of the Care Standards Act 2000;

“care home service” has the meaning assigned to it by section~2(3) of the Regulation of Care (Scotland) Act 2001;

% Definition of ``child tax credit'' inserted in reg.~1(2) (6.4.03) by SI 2003/328 reg 8(2)(a)
“child tax credit” means a child tax credit under section~8 of the Tax Credits Act 2002;

“Contributions and Benefits Act” means the Social Security Contributions and Benefits Act 1992\footnote{\frenchspacing 1992 c. 4.};

“Contributions and Benefits (Northern Ireland) Act” means the Social Security Contributions and Benefits (Northern Ireland) Act 1992\footnote{\frenchspacing 1992 c. 7.};

%“couple” means a man and a woman who are—
%\begin{enumerate}\item[]
%($a$) 
%married to each other and are members of the same household; or
%
%($b$) 
%not married to each other but are living together as husband and wife;
%\end{enumerate}

% Definition of ``couple'' substituted (5.12.05) by SI 2005/2877 Sch 4 para 7(2)(a)
“couple” means—
\begin{enumerate}\item[]
($a$) 
a man and woman who are married to each other and are members of the same household;

($b$) 
a man and woman who are not married to each other but are living together as husband and wife;

($c$) 
two people of the same sex who are civil partners of each other and are members of the same household; or

($d$) 
two people of the same sex who are not civil partners of each other but are living together as if they were civil partners,
\end{enumerate}
and for the purposes of sub-paragraph~($d$), two people of the same sex are to be regarded as living together as if they were civil partners if, but only if, they would be regarded as living together as husband and wife were they instead two people of the opposite sex;

“course of advanced education” means—
\begin{enumerate}\item[]
($a$) 
a full-time course leading to a postgraduate degree or comparable qualification, a first degree or comparable qualification, a Diploma of Higher Education, a higher national diploma, a higher national diploma or higher national certificate of the Business and Technology Education Council or the Scottish Qualifications Authority or a teaching qualification; or

($b$) 
any other full-time course which is a course of a standard above that of an ordinary national diploma, a national diploma or national certificate of the Business and Technology Education Council or the Scottish Qualifications Authority, the advanced level of the General Certificate of Education, a Scottish certificate of education (higher level), a Scottish certificate of sixth year studies or a Scottish National Qualification at Higher Level;
\end{enumerate}

“day” includes any part of a day;

“day to day care” means—
\begin{enumerate}\item[]
($a$) 
care of not less than 104 nights in total during the 12 month period ending with the relevant week; or

($b$) 
where, in the opinion of the Secretary of State, a period other than 12 months is more representative of the current arrangements for the care of the child in question, care during that period of not less in total than the number of nights which bears the same ratio to 104 nights as that period bears to 12 months, and for the purpose of this definition—
\begin{enumerate}\item[]
(i)
where a child is a boarder at a boarding school or is a patient in a hospital or other circumstances apply, such as where the child stays with a person who is not a parent of the child, and which the Secretary of State regards as temporary, the person who, but for those circumstances, would otherwise provide day to day care of the child shall be treated as providing day to day care during the periods in question; and

(ii)
“relevant week” shall have the meaning ascribed to it in the definition in this paragraph, except that in a case where notification is given under regulation~7C of the Decisions and Appeals Regulations\footnote{\frenchspacing Regulation 7C was inserted by S.I. 2000/119.} to the relevant persons on different dates, “relevant week” means the period of 7 days immediately preceding the date of the latest notification;
\end{enumerate}
\end{enumerate}

“Decisions and Appeals Regulations” means the Social Security and Child Support (Decisions and Appeals) Regulations 1999\footnote{S.I.~1999/991. The Regulations were amended by S.I. 1999/1446, 1623, 1662, 1670, 2570, 2677, 3178, 2000/119, 127, 897, 1596 and 1982.};

% Definition of ``disabled person's tax credit'' omitted (6.4.03) by SI 2003/328 reg 8(2)(b)
%“disabled person’s tax credit” means a disabled person’s tax credit under section~129 of the Contributions and Benefits Act\footnote{\frenchspacing \emph{See} section~1 of, and paragraphs 1 and 2($h$) of Schedule~1 to, the Tax Credits Act 1999 (c. 10).};

“effective date” means the date on which a maintenance calculation takes effect for the purposes of the Act;

“employed earner” has the same meaning as in section~2(1)($a$)  of the Contributions and Benefits Act except that it shall include—
\begin{enumerate}\item[]
($a$) 
a person gainfully employed in Northern Ireland; 
%and  % Word omitted (30.4.12) by SI 2012/712 reg 6(2)(a)

\pagebreak
($b$) 
a person to whom section~44(2A)\footnote{Section~44(2A) is inserted by section~22 of the Child Support, Pensions and Social Security Act 2000.} of the Act applies;
and % Words added (30.4.12) by SI 2012/712 reg 6(2)(b)

($c$) a person gainfully employed outside the United Kingdom if the person’s income from that employment is chargeable to tax under the Income Tax (Earnings and Pensions) Act 2003 or would be were it not for any double taxation arrangements under Part~II of the Taxation (International and Other Provisions) Act 2010;
\end{enumerate}

“family” means—
\begin{enumerate}\item[]
($a$) 
a couple (including the members of a polygamous marriage) and any member of the same household for whom one or more of them is responsible and who is a child; or

($b$) 
a person who is not a member of a couple and a member of the same household for whom that person is responsible and who is a child;
\end{enumerate}

“home” means—
\begin{enumerate}\item[]
($a$) 
the dwelling in which a person and any family of his normally live; or

($b$) 
if he or they normally live in more than one home, the principal home of that person and any family of his, and for the purpose of determining the principal home in which a person normally lives no regard shall be had to residence in 
%a residential care home or a nursing home 
a care home or an independent hospital or the provision of a care home service or an independent health care service  % Words substituted (5.11.03) by SI 2003/2779 reg 6(2)(a)
during a period which does not exceed 52 weeks or, where it appears to the Secretary of State that the person will return to his principal home after that period has expired, such longer period as the Secretary of State considers reasonable to allow for the return of that person to that home;
\end{enumerate}

“Income Support Regulations” means the Income Support (General) Regulations 1987\footnote{\frenchspacing S.I. 1987/1967; the relevant amending instruments are S.I. 1988/663, 1228, 1445, 2022, 1989/534, 1034, 1678, 1990/547, 1168, 1776; 1991/236, 387, 503, 1559.};

% Definitions of ``independent health care service'', ``independent hospital'' inserted (5.11.03) by SI 2003/2779 reg 6(2)(d)
“independent health care service” has the meaning assigned to it by section~2(5)($a$)  and ($b$)  of the Regulation of Care (Scotland) Act 2001;

“independent hospital” has the meaning assigned to it by section~2 of the Care Standards Act 2000;

“the Jobseekers Act” means the Jobseekers Act 1995\footnote{\frenchspacing 1995 c. 18.};

“Maintenance Calculation Procedure Regulations” means the Child Support (Maintenance Calculation Procedure) Regulations 2000\footnote{\frenchspacing S.I. 2001/157.};

“net weekly income” has the meaning given in the Schedule~to these Regulations;

% Definition omitted (5.11.03) by SI 2003/2779 reg 6(2)(b)
%“nursing home” has the same meaning as in regulation~19(3) of the Income Support Regulations;

“occupational pension scheme” means such a scheme within the meaning in section~1 of the Pension Schemes Act 1993\footnote{\frenchspacing 1993 c. 48.} and which is approved for the purposes of Part~XIV of the Income and Corporation Taxes Act 1988\footnote{\frenchspacing 1988 c. 1.}
or is a statutory scheme to which section~594 of that Act applies%  % Words inserted (5.11.03) by SI 2003/2779 reg 6(2)(e)
;

“partner” means—
\begin{enumerate}\item[]
($a$) 
in relation to a member of a couple, the other member of that couple;

($b$) 
in relation to a member of a polygamous marriage, any other member of that marriage with whom he lives;
\end{enumerate}

“patient” means a person (other than a person who is serving a sentence of imprisonment or detention in a young offender institution within the meaning of the Criminal Justice Act 1982\footnote{1982 c.\ 48. The Act is amended by the Criminal Justice Act 1988 (c.~33).} or the Prisons (Scotland) Act 1989\footnote{\frenchspacing 1989 c. 45.} who is regarded as receiving free in-patient treatment within the meaning of the Social Security (Hospital In-Patients) Regulations 1975\footnote{\frenchspacing S.I. 1975/555; relevant amending instruments are S.I. 1977/1693, 1987/1683 and 1999/1326.};

“person” does not include a local authority;

“personal pension scheme” means such a scheme within the meaning in section~1 of the Pension Schemes Act 1993 and which is approved for the purposes of Part~XIV of the Income and Corporation Taxes Act 1988;

“polygamous marriage” means any marriage during the subsistence of which a party to it is married to more than one person and in respect of which any ceremony of marriage took place under the law of a country which at the time of that ceremony permitted polygamy;

“prisoner” means a person who is detained in custody pending trial or sentence upon conviction or under a sentence imposed by a court other than a person whose detention is under the Mental Health Act 1983\footnote{\frenchspacing 1983 c. 20.} or the Mental Health (Scotland) Act 1984\footnote{\frenchspacing 1984 c. 36.};

“relevant week” means—
\begin{enumerate}\item[]
($a$) 
in relation to an application for child support maintenance—
\begin{enumerate}\item[]
(i)
where the application is made by a non-resident parent, the period of 7 days immediately before the application is made; and

(ii)
in any other case, the period of 7 days immediately before the date of notification to the non-resident parent and for this purpose “the date of notification to the non-resident parent” means the date on which the non-resident parent is first given notice by the Secretary of State under the Maintenance Calculation Procedure Regulations that an application for a maintenance calculation has been made
%, or treated as made, as the case may be,  % Words omitted (27.10.08) by SI 2008/2543 reg 7(2)
in relation to which the non-resident parent is named as the parent of the child to whom the application relates;
\end{enumerate}

($b$) 
where a decision (“the original decision”) is to be—
\begin{enumerate}\item[]
(i)
revised under section~16 of the Act; or

(ii)
superseded by a decision under section~17 of the Act on the grounds that the original decision was made in ignorance of, or was based upon a mistake as to, some material fact or was erroneous in point of law,
\end{enumerate}
the period of 7 days which was the relevant week for the purposes of the original decision;

($c$) 
where a decision (“the original decision”) is to be superseded under section~17 of the Act—
\begin{enumerate}\item[]
(i)
on an application made for the purpose on the basis that a material change of circumstances has occurred since the original decision was made, the period of 7 days immediately preceding the date on which that application was made;\looseness=-1

(ii)
subject to sub-paragraph~($b$), in a case where a relevant person is given notice under regulation~7C of the Decisions and Appeals Regulations, the period of 7 days immediately preceding the date of that notification,
\end{enumerate}
except that where, under paragraph~15 of Schedule~1 to the Act, the Secretary of State makes separate maintenance calculations in respect of different periods in a particular case, because he is aware of one or more changes of circumstances which occurred after the date which is applicable to that case, the relevant week for the purposes of each separate maintenance calculation made to take account of each such change of circumstances shall be the period of 7 days immediately before the date on which notification was given to the Secretary of State of the change of circumstances relevant to that separate maintenance calculation;
\end{enumerate}

% Definition omitted (5.11.03) by SI 2003/2779 reg 6(2)(b)
%“residential care home” has the same meaning as in regulation~19(3) of the Income Support Regulations;

“retirement annuity contract” means an annuity contract for the time being approved by the Board of Inland Revenue as having for its main object the provision of a life annuity in old age or the provision of an annuity for a partner or dependant and in respect of which relief from income tax may be given on any premium;

%“self-employed earner” has the same meaning as in section~2(1)($b$)  of the Contributions and Benefits Act except that it shall include a person gainfully employed in Northern Ireland otherwise than in employed earner’s employment (whether or not he is also employed in such employment);

% Definition substituted (30.4.12) by SI 2012/712 reg 6(3)
“self-employed earner” has the same meaning as in section~2(1)($b$)  of the Contributions and Benefits Act except that it includes a person gainfully employed otherwise than in employed earner’s employment (whether or not he is also employed in such employment)---
\begin{enumerate}\item[]
($a$) 
in Northern Ireland; or

($b$) 
outside the United Kingdom if the person’s income from that gainful employment is chargeable to tax under the Income Tax (Trading and Other Income) Act 2005 or would be were it not for any double taxation arrangements made under Part~II of the Taxation (International and Other Provisions) Act 2010;
\end{enumerate}

% Definition inserted (6.10.03) by SI 2002/3019 reg 27(2)
“state pension credit” means the social security benefit of that name payable under the State Pension Credit Act 2002;

“student” means a person, other than a person in receipt of a training allowance, who is aged less than 19 and attending a full-time course of advanced education or who is aged 19 or over and attending a full-time course of study at an educational establishment; and for the purposes of this definition—
\begin{enumerate}\item[]
($a$) 
a person who has started on such a course shall be treated as attending it throughout any period of term or vacation within it, until the last day of the course or such earlier date as he abandons it or is dismissed from it;

($b$) 
a person on a sandwich course (within the meaning of paragraph~1(1) of Schedule~5 to the Education (Mandatory Awards) (No.\ 2) Regulations 1993\footnote{\frenchspacing S.I. 1993/2914.}) shall be treated as attending a full-time course of advanced education or, as the case may be, of study;
\end{enumerate}

%“training allowance” means an allowance payable under section~2 of the Employment and Training Act 1973\footnote{1973 c.\ 50. Section~2 was substituted by the Employment Act 1988 (c.\ 19), section~25(1).}, or section~2 of the Enterprise and New Towns (Scotland) Act 1990\footnote{\frenchspacing 1990 c. 35.};

% Definition of ``training allowance'' substituted (5.11.03) by SI 2003/2779 reg 6(2)(f)
“training allowance” means a payment under section~2 of the Employment and Training Act 1973 (“the 1973 Act”)\footnote{1973 c.\ 50. Section~2 was substituted by section~25(1) of the Employment Act 1988 (c.\ 19).}, or section~2 of the Enterprise and New Towns (Scotland) Act 1990 (“the 1990 Act”)\footnote{1990 c.\ 35.}, which is paid—
\begin{enumerate}\item[]
($a$) 
to a person for his maintenance; and

($b$) 
in respect of a period during which that person—
\begin{enumerate}\item[]
(i) 
is undergoing training pursuant to arrangements made under section~2 of the 1973 Act or section~2 of the 1990 Act; and

(ii) 
has no net weekly income of a type referred to in Part~II or Part~III of the Schedule;
\end{enumerate}
\end{enumerate}

% Definition of ``war widow's pension'' inserted (5.11.03) by SI 2003/2779 reg 6(2)(g)
“war widow’s pension” means any pension or allowance payable for a widow which is—
\begin{enumerate}\item[]
($a$) 
granted in respect of a death due to service or war injury and payable by virtue of the Air Force (Constitution) Act 1917\footnote{7 \& 8 Geo.~5 c.\ 51. Section~3 was amended by and section~13 was repealed by S.I.~1964/488. Section~4 was amended by the Armed Forces Act 1981 (c.\ 55). Sections~5 and 11 were repealed by the Statute Law Revision Act 1927 (17 \& 18 Geo.\ 5 c.\ 42). Section~6 was repealed by the Statute Law (Repeals) Act 1976 (c.\ 16). Section~7 and Schedule~1 were repealed by the Naval Discipline Act 1957 (5 \& 6 Eliz.~2 c.\ 53). Sections~8 to 10 were repealed by the Defence (Transfer of Functions) Act 1964 (c.\ 15). Section~12 and Schedule~2 were repealed by the Revision of the Army and Air Forces Acts (Transitional Provisions) Act 1955 (3 \& 4 Eliz.~2 c.\ 20).}, the Personal Injuries (Emergency Provisions) Act 1939\footnote{2 \& 3 Geo.\ 6 c.\ 82. Section~2 was amended, and sections~3, 4 and 5 were repealed, by the Statute Law Revision Act 1953 (1 \& 2 Eliz.~2 c.~5). Section~6 was repealed by the Theft Act 1968 (c.\ 60). Section~8 was modified by the Northern Ireland Act 1998 (c.\ 47). Section~9 was amended by the Statute Law Revision Act 1950 (14 Geo.\ 6 c.\ 6).}, the Pensions (Navy, Army, Air Force and Mercantile Marine) Act 1939\footnote{2 \& 3 Geo.\ 6 c.\ 83. Section~1 was repealed by S.I.~1964/488. Section~2 was repealed by the War Orphans Act 1942 (5 \& 6 Geo.\ 6 c.\ 8). Sections~3,~4,~5,~6,~7 and 10 were amended by the Pensions (Mercantile Marine) Act 1942 (5 \& 6 Geo.\ 6 c.\ 26). Section~4 was amended by the Pilotage Act 1983 (c.\ 21). Section~5 was amended by the Armed Forces Act 1981 (c.\ 55). Section~6 was amended by the Merchant Shipping Act 1970 (c.\ 36). Section~8 was repealed by the Theft Act 1968. Section~9 was repealed by S.I.~1965/145.}, the Polish Resettlement Act 1947\footnote{10 \& 11 Geo.\ 6 c.\ 19. Section~2 and the Schedule~were amended by the National Assistance Act 1948 (11 \& 12 Geo.\ 6 c.\ 29). Section~3 was amended by S.I.~1951/174 and 1968/1699, the Supplementary Benefits Act 1976 (c.\ 71) and the Social Security Act 1980 (c.\ 30). Section~4 was amended by S.I.~1968/1699, the National Health Service Act 1977 (c.\ 49), the Social Security Act 1980, the Mental Health Act 1983 (c.\ 20) and the Health Authorities Act 1995 (c.\ 17). Sections~6, 7 and 12 and the Schedule~were amended by the Social Security Act 1980. Sections~8 and 9 were repealed, and sections~10 and 12 were amended, by the Statute Law Revision Act 1953. Section~11 was amended by the Mental Health Act 1983 and the Mental Health (Scotland) Act 1984 (c.\ 36).} or Part~VII or section~151 of the Reserve Forces Act 1980\footnote{1980 c.\ 9. Part~VII was amended by the Armed Forces Act 1981, the Army Act 1992 (c.\ 39), the Statute Law (Repeals) Act 1993 (c.\ 50) and the Reserve Forces Act 1996 (c.\ 14).};

($b$) 
payable under so much of any Order in Council, Royal Warrant, order or scheme as relates to death due to service in the armed forces of the Crown, wartime service in the merchant navy or war injuries;

($c$) 
payable in respect of death due to peacetime service in the armed forces of the Crown before 3rd September 1939, and payable at rates, and subject to conditions, similar to those of a pension within sub-paragraph~($b$); or

($d$) 
payable under the law of a country other than the United Kingdom and of a character substantially similar to a pension within sub-paragraph~($a$), ($b$)  or ($c$),
\end{enumerate}
and “war widower’s pension”
and “surviving civil partner’s war pension”  % Words inserted (5.12.05) by SI 2005/2877 Sch 4 para 7(2)(b)
shall be construed accordingly;

% Definition of ``the Welfare Reform Act'' inserted (27.10.08) by SI 2008/1544 reg 61(2)
“the Welfare Reform Act” means the Welfare Reform Act 2007\footnote{2007 c.~5.};

“work-based training for young people or, in Scotland, Skillseekers training” means—
\begin{enumerate}\item[]
($a$) 
arrangements made under section~2 of the Employment and Training Act 1973 or section~2 of the Enterprise and New Towns (Scotland) Act 1990; or

($b$) 
arrangements made by the Secretary of State for persons enlisted in Her Majesty’s forces for any special term of service specified in regulations made under section~2 of the Armed Forces Act 1966\footnote{\frenchspacing 1966 c. 45.} (power of Defence Council to make regulations as to engagement of persons in regular forces),
\end{enumerate}
for purposes which include the training of persons who, at the beginning of their training, are under the age of 18;

%“working families' tax credit” means a working families' tax credit under section~128 of the Contributions and Benefits Act\footnote{\frenchspacing See Section~1 of, and paragraphs 1 and 2($g$) of Schedule~1 to, the Tax Credits Act 1999 (c. 10).}; and

% Definition of ``working families' tax credit'' substituted (6.4.03) by SI 2003/328 reg 8(2)(c)
“working tax credit” means a working tax credit under section~10 of the Tax Credits Act 2002;

“year” means a period of 52 weeks.
\end{enumerate}

%(3) The following other description of children is prescribed for the purposes of paragraph~10C(2)($b$)  of Schedule~1 to the Act (relevant other children)—
%\begin{enumerate}\item[]
%    children other than qualifying children in respect of whom the non-resident parent or his partner would receive child benefit under Part~IX of the Contributions and Benefits Act but who do not solely because the conditions set out in section~146 of that Act (persons outside Great Britain) are not met. 
%\end{enumerate}

% Reg 1(3) substituted (30.9.13) by SI 2013/1517 reg 6(2)
(3) For the purposes of paragraph 10C(2)($b$)  of Schedule 1 to the Act (which provides for other descriptions of relevant other children to be prescribed) “relevant other child” includes a child, other than a qualifying child, in respect of whom the non-resident parent or the non-resident parent’s partner—
\begin{enumerate}\item[]
($a$) would receive child benefit under Part IX of the Contributions and Benefits Act, but in respect of whom they do not do so, solely because the conditions set out in section~146 of that Act (persons outside Great Britain) are not met; or

($b$) has made an election under section 13A(1) of the Social Security Administration Act 1992 (election not to receive child benefit) for payments of child benefit not to be made.
\end{enumerate}

(4) Subject to paragraph~(5), these Regulations shall come into force in relation to a particular case on the day on which Part~I of Schedule~1 to the 1991 Act as amended by the Child Support, Pensions and Social Security Act 2000 comes into force in relation to that type of case.

(5) Paragraphs (1) and (2) of regulation~4 and, for the purposes of those provisions, this regulation~shall come into force on 31st January 2001.

\amendment{
Definition of ``child tax credit'' inserted in reg.~1(2), definition of ``working families' tax credit'' substituted in reg.~1(2) and definition of ``disabled person's tax credit'' omitted from reg.~1(2) (6.4.03) by the Child Support (Miscellaneous Amendments) Regulations 2003 reg.~8(2).

Definition of ``state pension credit'' inserted in reg.~1(2) (6.10.03) by the State Pension Credit (Consequential, Transitional and Miscellaneous Provisions) Regulations 2002 reg.~27(2).

Words inserted in definition of ``occupational pension scheme'' in reg.~1(2), words substituted in definition of ``home'' in reg.~1(2), definitions of ``care home'', ``care home service'', ``independent health care service'', ``independent hospital'', ``war widow's pension'' inserted in reg.~1(2), definition of ``training allowance'' in reg.~1(2) substituted and definitions of ``nursing home'' and ``residential care home'' in reg.~1(2) omitted (5.11.03) by the Child Support (Miscellaneous Amendments) (No. 2) Regulations 2003 reg.~6(2).

Words inserted in definition of ``war widow's pension'' in reg.~1(2) and definition of ``couple'' in reg.~1(2) substituted (5.12.05) by the Civil Partnership (Pensions, Social Security and Child Support) (Consequential, etc. Provisions) Order 2005 Sch. 4 para. 7(2).

Definition of ``the Welfare Reform Act'' inserted in reg.~1(2)
(27.10.08) by the Employment and Support Allowance (Consequential Provisions) (No.~2) Regulations 2008 reg.~61(2).

Words omitted in sub-para. (a)(ii) of definition of ``relevant week'' in reg.~1(2) (27.10.08) by the Child Support (Consequential Provisions) Regulations 2008 reg.~7(2).

Words added to definition of ``employed earner'' in reg.~1(2) and definition of ``self-employed earner'' in reg.~1(2) substituted (30.4.12) by the Child Support (Miscellaneous Amendments) Regulations 2012 reg.~6(2), (3).

Reg.~1(3) substituted (30.9.13) by the Child Support (Miscellaneous Amendments) Regulations 2013 reg.~6(2).
}

%\enlargethispage{\baselineskip}

\section[Part~II --- Calculation of child support maintenance]{Part~II\\*Calculation of child support maintenance}

\renewcommand\parthead{--- Part~II}

\subsection[2. Calculation of amounts]{Calculation of amounts}

2.---(1)  Where any amount is to be considered in connection with any calculation made under these Regulations or under Schedule~1 to the Act, it shall be calculated as a weekly amount and, except where the context otherwise requires, any reference to such an amount shall be construed accordingly.

(2) Subject to paragraph~(3), where any calculation made under these Regulations or under Schedule~1 to the Act results in a fraction of a penny that fraction shall be treated as a penny if it is either one half or exceeds one half, otherwise it shall be disregarded.

(3) Where the calculation of the basic rate of child support maintenance or the reduced rate of child support maintenance results in a fraction of a pound that fraction shall be treated as a pound if it is either one half or exceeds one half, otherwise it shall be disregarded.

(4) In taking account of any amounts or information required for the purposes of making a maintenance calculation, the Secretary of State shall apply the dates or periods specified in these Regulations as applicable to those amounts or information, provided that if he becomes aware of a material change of circumstances occurring after such date or period, but before the effective date, he shall take that change of circumstances into account.

(5) Information required for the purposes of making a maintenance calculation in relation to the following shall be the information applicable at the effective date—
\begin{enumerate}\item[]
($a$) the number of qualifying children;

($b$) the number of relevant other children;

($c$) whether the non-resident parent receives a benefit, pension or allowance prescribed for the purposes of paragraph~4(1)($b$)  of Schedule~1 to the Act;

($d$) whether the non-resident parent or his partner receives a benefit prescribed for the purposes of paragraph~4(1)($c$)  of Schedule~1 to the Act; and

($e$) whether paragraph~5($a$)  of Schedule~1 to the Act applies to the non-resident parent.
\end{enumerate}

\subsection[3. Reduced Rate]{Reduced Rate}

3.  The reduced rate is an amount calculated as follows—
\[F + (A \times T)\]
where—
\begin{enumerate}\item[]
    $F$ is the flat rate liability applicable to the non-resident parent under paragraph~4 of Schedule~1 to the Act;

    $A$ is the amount of the non-resident parent’s net weekly income between £100 and £200; and

    $T$ is the percentage determined in accordance with the following Table— 
\end{enumerate}

{\scriptsize\noindent\hbadness=10000
%\begin{tabulary}{\linewidth}{JJJJJJJJJJJJJ}
\begin{tabular}{p{79pt}p{10pt}p{17pt}p{8pt}p{18pt}p{10pt}p{10pt}p{9pt}p{18pt}p{10pt}p{17pt}p{10pt}p{18pt}}
\hline
&	\multicolumn{4}{p{84pt}}{\itshape 1 qualifying child of the non-resident parent}	& \multicolumn{4}{p{78pt}}{\itshape 2 qualifying children of the non-resident parent}	& \multicolumn{4}{p{84pt}}{\itshape 3 or more qualifying children of the non-resident parent}\\
\hline
Number of relevant other children of the non-resident parent	&0	&1	&2	&3 or more	&0	&1	&2	&3 or more	&0	&1	&2	&3 or more\\
\hline
$T$ (\%)	&25	&20$.$5	&19	&17$.$5	&35	&29	&27	&25	&45	&37$.$5	&35	&32$.$5\\
\hline
%\end{tabulary}
\end{tabular}

}

\subsection[4. Flat rate]{Flat rate}

4.---(1)  The following benefits, pensions and allowances are prescribed for the purposes of paragraph~4(1)($b$)  of Schedule~1 to the Act—
\begin{enumerate}\item[]
($a$) under the Contributions and Benefits Act—
\begin{enumerate}\item[]
(i) bereavement allowance under section~39B\footnote{\frenchspacing Sections~39A and 39B were inserted by section~55 of the Welfare Reform and Pensions Act 1999 (c. 30).};

(ii) category A retirement pension under section~44\footnote{Section~44 was amended by paragraph~32 of Schedule~8 to the Pension Schemes Act 1993 (c.\ 48), sections~1–3, 5, 6, 8–10 of and Schedule~1 to, the Social Security (Incapacity for Work) Act 1994 (c.\ 18), sections~127–134 of, and Schedules 1 and 2 to, the Pensions Act 1995 (c.\ 26) and by sections~30(2) and 35(5)–(7) of the Child Support, Pensions and Social Security Act 2000. Section~44(4) was substituted by section~68 of the Social Security Act 1998 (c.\ 47).};

(iii) category B retirement pension under section~48C\footnote{Section~48C was inserted by paragraph~7 of Schedule~8 to the Welfare Reform and Pensions Act 1999.};

(iv) category C and category D retirement pensions under section~78\footnote{Section~78 was amended by sections~127–134 of, and Schedules 4 and 5 to, the Pensions Act 1995.};

(v) incapacity benefit under section~30A\footnote{Section~30A was inserted by section~1(1) of the Social Security (Incapacity for Work) Act 1994.};

(vi) 
%invalid care allowance 
carer's allowance  % Words substituted (1.4.03) by SI 2002/2497
under section~70;

(vii) maternity allowance under section~35\footnote{Section~35 was amended by section~67 of the Social Security Act 1988, section~2 of the Still-Birth Definition Act 1992 (c.\ 29) and by section~53 of the Welfare Reform and Pensions Act 1999. Sections~35(1) and (1A) were substituted by section~53 of the Welfare Reform and Pensions Act 1999.};

(viii) severe disablement allowance under section~68\footnote{Section~68 is prospectively repealed with savings by section~65 of, and Part~IV of Schedule~13 to, the Welfare Reform and Pensions Act 1999.};

(ix) industrial injuries benefit under section~94;

(x) widowed mother’s allowance under section~37;

(xi) widowed parent’s allowance under section~39A; and

(xii) widow’s pension under section~38;
\end{enumerate}

($b$) contribution-based jobseeker’s allowance under section~1 of the Jobseekers Act;

($c$) a social security benefit paid by a country other than the United Kingdom;

($d$) a training allowance (other than work-based training for young people or, in Scotland, Skillseekers training); 
%and % Word omitted (16.3.05) by SI 2005/785 reg 6(2)(a)

($e$) a war disablement pension 
%or war widow’s pension  % Words omitted (5.11.03) by SI 2003/2779 reg 6(3)(a)
within the meaning of section~150(2) of the Contributions and Benefits Act\footnote{Section~150(2) was amended by section~132 of, and paragraph~13 of Schedule~4 to, the Pensions Act 1995.} or a pension which is analogous to such a pension paid by the government of a country outside Great Britain;

%and % Word omitted (16.3.05) by SI 2005/785 reg 6(2)(a)
% Reg 4(1)(f) added (5.11.03) by SI 2003/2779 reg 6(3)(b) 
%    ($f$) 
%    a war widow’s pension or a war widower’s pension%

% Reg 4(1)(f) substituted (5.12.05) by SI 2005/2877 Sch 4 para 7(3)
($f$) a war widow’s pension, war widower’s pension or surviving civil partner’s war pension;
%
% Reg 4(1)(g) added (16.3.05) by SI 2005/785 reg 6(2)(b)
%and

($g$) a payment under a scheme mentioned in section~1(2) of the Armed Forces (Pensions and Compensation) Act 2004\footnote{2004 c.\ 32.} (compensation schemes for armed and reserve forces);
%
% Reg 4(1)(h) added (27.10.08) by SI 2008/1554 reg 61(3)(a)
and

($h$) contributory employment and~support allowance under section~2 of the Welfare Reform Act.
\end{enumerate}

(2) The benefits prescribed for the purposes of paragraph~4(1)($c$)  of Schedule~1 to the Act are—
\begin{enumerate}\item[]
($a$) income support under section~124 of the Contributions and Benefits Act; and

($b$) income-based jobseeker’s allowance under section~1 of the Jobseekers Act;
%
% Reg 4(2)(c) added (6.10.03) by SI 2002/3019 reg 27(3)
and

    ($c$) 
    state pension credit;
%
% Reg 4(2)(d) added (27.10.08) by SI 2008/1554 reg 61(3)(b)
and

($d$) income-related employment and~support allowance under section~4 of the Welfare Reform Act.
\end{enumerate}

(3) Where the non-resident parent is liable to a pay a flat rate by virtue of paragraph~4(2) of Schedule~1 to the Act—
\begin{enumerate}\item[]
($a$) if he has one partner, then the amount payable by the non-resident parent shall be half the flat rate; and

($b$) if he has more than one partner, then the amount payable by the non-resident parent shall be the result of apportioning the flat rate equally among him and his partners.
\end{enumerate}

\amendment{
Words substituted in reg.~4(1)(a)(vi) (1.4.03) by the Social Security Amendment (Carer's Allowance) Regulations 2002 Sch. 2 para. 1, 2.

Reg. 4(2)(c) added (6.10.03) by the State Pension Credit (Consequential, Transitional and Miscellaneous Provisions) Regulations 2002 reg.~27(3).

Words omitted in reg.~4(1)(e) and reg.~4(1)(f) added (5.11.03) by the Child Support (Miscellaneous Amendments) (No. 2) Regulations 2003 reg.~6(3).

Reg. 4(1)(g) added (16.3.05) by the Child Support (Miscellaneous Amendments) Regulations 2005 reg.~6(2).

Reg. 4(1)(f) substituted (5.12.05) by the Civil Partnership (Pensions, Social Security and Child Support) (Consequential, etc. Provisions) Order 2005 Sch. 4 para. 7(3).

Reg.~4(1)(h), (2)(d) added
(27.10.08) by the Employment and Support Allowance (Consequential Provisions) (No.~2) Regulations 2008 reg.~61(3).
}

\subsection[5. Nil rate]{Nil rate}

5.  The rate payable is nil where the non-resident parent is—
\begin{enumerate}\item[]
($a$) a student;

($b$) a child within the meaning given in section~55(1) of the Act;

($c$) a prisoner;

($d$) a person who is 16 or 17 years old and—
\begin{enumerate}\item[]
(i) in receipt of income support% 
%or income-based jobseeker’s allowance
,~income-based jobseeker’s allowance or~income-related employment and~support allowance%  % Words substituted (27.10.08) by SI 2008/1554 reg 61(4)
; or

(ii) a member of a couple whose partner is in receipt of income support% 
%or income-based jobseeker’s allowance
,~income-based jobseeker’s allowance or~income-related employment and~support allowance%  % Words substituted (27.10.08) by SI 2008/1554 reg 61(4)
;
\end{enumerate}

($e$) a person receiving an allowance in respect of work-based training for young people, or in Scotland, Skillseekers training;

($f$) a person 
%in a residential care home or nursing home 
who is resident in a care home or an independent hospital or is being provided with a care home service or an independent health care service  % Words substituted (5.11.03) by SI 2003/2779 reg 6(4)
who—
\begin{enumerate}\item[]
(i) is in receipt of a pension, benefit or allowance specified in regulation~4(1) or (2); or

(ii) has the whole or part of the cost of his accommodation met by a local authority.
\end{enumerate}

% Reg 5(g), (gg), (h) omitted (6.4.09) by SI 2009/396 reg 5
%($g$) a patient in hospital who is in receipt of income support whose applicable amount includes an amount under paragraph~
%%1($a$)  or ($b$)  
%1($b$)  or 2  % Words substituted (21.5.03) by SI 2003/1195 reg 7(a)(i)
%of Schedule~7 to the Income Support Regulations (patient for more than 
%%6 
%52  % Words substituted (21.5.03) by SI 2003/1195 reg 7(a)(ii)
%weeks);
%
%% Reg 5(gg) inserted (6.10.03) by SI 2002/3019 reg 27(4)
%($gg$) a patient in hospital who is in receipt of state pension credit and in respect of whom paragraph~2(1) of Schedule~III to the State Pension Credit Regulations
%2002%  % Word inserted (21.5.03) by SI 2003/1195 reg 7(b)(i)
%\footnote{S.I.~2002/1792.} (patient for 
%%at least 13 but not exceeding 
%more than  % Words substituted (21.5.03) by SI 2003/1195 reg 7(b)(ii)
%52 weeks) applies;
%or  % Word inserted (16.3.05) by SI 2005/785 reg 6(3)(a)
%
%($h$) a person in receipt of a benefit specified in regulation~4(1) the amount of which has been reduced in accordance with the provisions of regulations 
%%4($d$)  
%4  % Word substituted (21.5.03) by SI 2003/1195 reg 7(c)
%and 6 of the Social Security Hospital In-Patients Regulations 1975 (circumstances in which personal benefit is to be adjusted and adjustment of personal benefit after 52 weeks in hospital)\footnote{S.I.~1975/555. Regulation 4($d$) has been amended by regulation~2(3)($b$) of S.I.~1987/1683 and regulation~6 has been amended by section~18(1) of the Social Security Act 1986 (c.\ 50), regulation~3 of S.I.~1977/1963, regulation~2 of S.I.~1987/31 and regulation~2 of S.I.~1987/1683.}%
%; or  % Word omitted (16.3.05) by SI 2005/785 reg 6(3)(b)
%
% Reg 5(i) omitted (16.9.04) by SI 2004/2415 reg 7(2)
%($i$) a person who would be liable to pay the flat rate because he satisfies the description in paragraph~4(1)($c$)  of Schedule~1 to the Act but his net weekly income, inclusive of—
%\begin{enumerate}\item[]
%($aa$) any benefit, pension or allowance that he receives which is prescribed for the purposes of paragraph~4(1)($b$)  of Schedule~1 to the Act; and
%
%($bb$) any benefit that he or his partner receives which is prescribed for the purposes of paragraph~4(1)($c$)  of Schedule~1 to the Act,
%\end{enumerate}
%is less than £5 a week.
\end{enumerate}

\amendment{
Word inserted in reg.~5(gg) and words substituted in reg.~5(g), (gg), (h) (21.5.03) by the Social Security (Hospital In-Patients and Miscellaneous Amendments) Regulations 2003 reg.~7.

Reg. 5(gg) inserted (6.10.03) by the State Pension Credit (Consequential, Transitional and Miscellaneous Provisions) Regulations 2002 reg.~27(4).

Words substituted in reg.~5(f) (5.11.03) by the Child Support (Miscellaneous Amendments) (No. 2) Regulations 2003 reg.~6(4).

Reg. 5(i) omitted (16.9.04) by the Child Support (Miscellaneous Amendments) Regulations 2004 reg.~7(2).

Word inserted after reg.~5(gg) and word omitted after reg 5(h) (16.3.05) by the Child Support (Miscellaneous Amendments) Regulations 2005 reg.~6(3).

Words substituted in reg.~5(d)
(27.10.08) by the Employment and Support Allowance (Consequential Provisions) (No.~2) Regulations 2008 reg.~61(4).

Reg.~5(g), (gg), (h) omitted (6.4.09) by the Child Support (Miscellaneous Amendments) Regulations 2009 reg.~5.}

\subsection[6. Apportionment]{Apportionment}

6.  If, in making the apportionment required by regulation~4(3) or paragraph~6 of Part~I of Schedule~1 to the Act, the effect of the application of regulation~2(2) (rounding) would be such that the aggregate amount of child support maintenance payable by a non-resident parent would be different from the aggregate amount payable before any apportionment, the Secretary of State shall adjust that apportionment so as to eliminate that difference; and that adjustment shall be varied from time to time so as to secure that, taking one week with another and so far as is practicable, each person with care receives the amount which she would have received if no adjustment had been made under this paragraph.

\subsection[7. Shared care]{Shared care}

7.---(1)  For the purposes of paragraphs 7 and 8 of Part~I of Schedule~1 to the Act a night will count for the purposes of shared care where the non-resident parent—
\begin{enumerate}\item[]
($a$) has the care of a qualifying child overnight; and

($b$) the qualifying child stays at the same address as the non-resident parent.
\end{enumerate}

(2) For the purposes of paragraphs 7 and 8 of Part~I of Schedule~1 to the Act, a non-resident parent has the care of a qualifying child when he is looking after the child.

(3) Subject to paragraph~(4), in determining the number of nights for the purposes of shared care, the Secretary of State shall consider the 12 month period ending with the relevant week and for this purpose “relevant week” has the same meaning as in the definition of day to day care in regulation~1 of these Regulations.

(4) The circumstances in which the Secretary of State may have regard to a number of nights over less than a 12 month period are where there has been no pattern for the frequency with which the non-resident parent looks after the qualifying child for the 12 months preceding the relevant week, or the Secretary of State is aware that a change in that frequency is intended, and in that case he shall have regard to such lesser period as may seem to him to be appropriate, and the Table in paragraph~7(4) and the period in paragraph~8(2) of Schedule~1 to the Act shall have effect subject to the adjustment described in paragraph~(5).

(5) Where paragraph~(4) applies, the Secretary of State shall adjust the number of nights in that lesser period by applying to that number the ratio which the period of 12 months bears to that lesser period.

(6) Where a child is a boarder at a boarding school, or is a patient in a hospital, the person who, but for those circumstances, would otherwise have care of the child overnight shall be treated as providing that care during the periods in question.

\section[Part~III --- Special cases]{Part~III\\*Special cases}

\renewcommand\parthead{--- Part~III}

\subsection[8. Persons treated as non-resident parents]{Persons treated as non-resident parents}

8.---(1)  Where the circumstances of a case are that—
\begin{enumerate}\item[]
($a$) two or more persons who do not live in the same household each provide day to day care for the same 
%qualifying child
child, being a child in respect of whom an application for a maintenance calculation has been made% 
%or treated as made%  % Words omitted (27.10.08) by SI 2008/2543 reg 7(3)
  % Words substituted (21.2.03) by SI 2003/328 reg 8(3)
; and

($b$) at least one of those persons is a parent of the child,
\end{enumerate}
that case shall be treated as a special case for the purposes of the Act.

(2) For the purposes of this special case a parent who provides day to day care for a child of his is to be treated as a non-resident parent for the purposes of the Act in the following circumstances—
\begin{enumerate}\item[]
($a$) a parent who provides such care to a lesser extent than the other parent, person or persons who provide such care for the child in question; or

($b$) where the persons mentioned in paragraph~(1)($a$)  include both parents and the circumstances are such that care is provided to the same extent by both but each provides care to an extent greater than or equal to any other person who provides such care for that child—
\begin{enumerate}\item[]
(i) the parent who is not in receipt of child benefit for the child in question; or

(ii) if neither parent is in receipt of child benefit for that child, the parent who, in the opinion of the Secretary of State, will not be the principal provider of day to day care for that child.
\end{enumerate}
\end{enumerate}

(3) For the purposes of this regulation and regulation~10---
\begin{enumerate}\item[]
($a$)  % Words became sub-paragraph (a) (30.9.13) by SI 2013/1517 reg 6(3)
“child benefit” means child benefit payable under Part~IX of the Contributions and Benefits Act;

% Reg 8(3)(b) inserted (30.9.13) by SI 2013/1517 reg 6(3)
($b$) where a person has made an election under section 13A(1) of the Social Security Administration Act 1992 (election not to receive child benefit) for payments of child benefit not to be made, that person is to be treated as being in receipt of child benefit.
\end{enumerate}

\amendment{
Words subsituted in reg.~8(1)(a) (21.2.03) by the Child Support (Miscellaneous Amendments) Regulations 2003 reg.~8(3).

Words omitted in reg.~8(1)(a) (27.10.08) by the Child Support (Consequential Provisions) Regulations 2008 reg.~7(3).

Reg.~8(3)(b) inserted (30.9.13) by the Child Support (Miscellaneous Amendments) Regulations 2013 reg.~6(3).
}

\subsection[9. Care provided in part by a local authority]{Care provided in part by a local authority}

9.---(1)  This regulation~applies where paragraph~(2) applies and the rate of child support maintenance payable is the basic rate, or the reduced rate, or has been calculated following agreement to a variation where the non-resident parent’s liability would otherwise have been a flat rate or the nil rate.

(2) Where the circumstances of a case are that the care of the qualifying child is shared between the person with care and a local authority and—
\begin{enumerate}\item[]
($a$) the qualifying child is in the care of the local authority for 52 nights or more in the 12 month period ending with the relevant week; or

($b$) where, in the opinion of the Secretary of State, a period other than the 12 month period mentioned in sub-paragraph~($a$)  is more representative of the current arrangements for the care of the qualifying child, the qualifying child is in the care of the local authority during that period for no fewer than the number of nights which bears the same ratio to 52 nights as that period bears to 12 months; or%\looseness=-1

($c$) it is intended that the qualifying child shall be in the care of the local authority for a number of nights in a period from the effective date,
\end{enumerate}
that case shall be treated as a special case for the purposes of the Act.

(3) In a case where this regulation~applies, the amount of child support maintenance which the non-resident parent is liable to pay the person with care of that qualifying child is the amount calculated in accordance with the provisions of Part~I of Schedule~1 to the Act and decreased in accordance with this regulation.

(4) First, there is to be a decrease according to the number of nights spent or to be spent by the qualifying child in the care of the local authority during the period under consideration.

(5) Where paragraph~(2)($b$)  or ($c$)  applies, the number of nights in the period under consideration shall be adjusted by the ratio which the period of 12 months bears to the period under consideration.

(6) After any adjustment under paragraph~(5), the amount of the decrease for one child is set out in the following Table—

\begin{center}
\footnotesize
\begin{tabular}{ll}
\hline
\itshape Number of nights in care of local authority	& \itshape Fraction to subtract\\
\hline
52–103	&One-seventh\\
104–155	&Two-sevenths\\
156–207	&Three-sevenths\\
208–259	&Four-sevenths\\
260–262	&Five-sevenths\\
\hline
\end{tabular}
\end{center}

(7) If the non-resident parent and the person with care have more than one qualifying child, the applicable decrease is the sum of the appropriate fractions in the Table divided by the number of such qualifying children.

(8) In a case where the amount of child support maintenance which the non-resident parent is liable to pay in relation to the same person with care is to be decreased in accordance with the provisions of both this regulation~and of paragraph~7 of Part~I of Schedule~1 to the Act, read with regulation~7 of these Regulations, the applicable decrease is the sum of the appropriate fractions derived under those provisions.

\begin{sloppypar}
(9) If the application of this regulation~would decrease the weekly amount of child support maintenance (or the aggregate of all such amounts) payable by the non-resident parent to less than the rate stated in or prescribed for the purposes of paragraph~4(1) of Part~I of Schedule~1 to the Act, he is instead liable to pay child support maintenance at a rate equivalent to that rate, apportioned (if appropriate) in accordance with paragraph~6 of Part~I of Schedule~1 to the Act and regulation~6.
\end{sloppypar}

(10) Where a qualifying child is a boarder at a boarding school or is an in-patient at a hospital, the qualifying child shall be treated as being in the care of the local authority for any night that the local authority would otherwise have been providing such care.

(11) A child is in the care of a local authority for any night in which he is being looked after by the local authority within the meaning of section~22 of the Children Act 1989\footnote{1989 c.\ 41.} or section~17(6) of the Children (Scotland) Act 1995\footnote{1995 c.\ 36.}.

\subsection[10. Care provided for relevant other child by a local authority]{Care provided for relevant other child by a local authority}

10.  Where a child other than a qualifying child is cared for in part or in full by a local authority and the non-resident parent or his partner receives child benefit for that child, the child is a relevant other child for the purposes of Schedule~1 to the Act.

\subsection[11. Non-resident parent liable to pay maintenance under a maintenance order]{\sloppy\hbadness=1522 Non-resident parent liable to pay maintenance under a maintenance order}

11.---(1)  Subject to paragraph~(2), where the circumstances of a case are that—
\begin{enumerate}\item[]
($a$) an application for child support maintenance is made 
%or treated as made, as the case may be,  % Words omitted (27.10.08) by SI 2008/2543 reg 7(4)
with respect to a qualifying child and a non-resident parent; and

($b$) an application for child support maintenance for a different child cannot be made under the Act but that non-resident parent is liable to pay maintenance 
%under a maintenance order for that child,
%\end{enumerate}
%that case shall be treated as a special case for the purposes of the Act.
for that child—
\begin{enumerate}\item[]
    (i) 
    under a maintenance order;

    (ii) 
    in accordance with the terms of an order made by a court outside Great Britain; or

    (iii)
    under the legislation of a jurisdiction outside the United Kingdom,
\end{enumerate}
\end{enumerate}
    that case shall be treated as a special case for the purpose of the Act.
% Words substituted (16.3.05) by SI 2005/785 reg 6(4)

(2) This regulation~applies where the rate of child support maintenance payable is the basic rate, or the reduced rate, or has been calculated following agreement to a variation where the non-resident parent’s liability would otherwise have been a flat rate or the nil rate.

(3) Where this regulation~applies, 
subject to paragraph~(5),  % Words inserted (5.11.03) by SI 2003/2779 reg 6(5)(a)
the amount of child support maintenance payable by the non-resident parent shall be ascertained by—
\begin{enumerate}\item[]
($a$) calculating the amount of maintenance payable as if the number of qualifying children of that parent included any children with respect to whom he is liable to make payments under the order referred to in paragraph~(1)($b$); and

($b$) apportioning the amount so calculated between the qualifying children and the children with respect to whom he is liable to make payments under the order referred to in paragraph~(1)($b$),
\end{enumerate}
and the amount payable shall be the amount apportioned to the qualifying children, and the amount payable to each person with care shall be that amount subject to the application of apportionment under paragraph~6 of Schedule~1 to the Act and the shared care provisions in paragraph~7 of Part~I of that Schedule.

(4) In a case where this regulation~applies paragraph~7 of Part~I of Schedule~1 to the Act (shared care) and regulation~10 (care provided in part by local authority) shall not apply in relation to a child in respect of whom the non-resident parent is liable to make payments under a maintenance order as provided in paragraph~(1)($b$).

% Reg 11(5) added (5.11.03) by SI 2003/2779 reg 6(5)(b)
\begin{sloppypar}
(5) If the application of paragraph~(3) would decrease the weekly amount of child support maintenance (or the aggregate of all such amounts) payable by the non-resident parent to the person with care (or all of them) to an amount which is less than a figure equivalent to the flat rate of child support maintenance payable under paragraph~4(1) of Schedule~1 to the Act, the non-resident parent shall instead be liable to pay child support maintenance at a rate equivalent to that flat rate apportioned (if appropriate) as provided in paragraph~6 of Schedule~1 to the Act.
\end{sloppypar}

\amendment{
Words inserted in reg.~11(3) and reg.~11(5) added (5.11.03) by the Child Support (Miscellaneous Amendments) (No. 2) Regulations 2003 reg.~6(5).

Words substituted in reg.~11(1) (16.3.05) by the Child Support (Miscellaneous Amendments) Regulations 2005 reg.~6(4).

Words omitted in reg.~11(1)(a) (27.10.08) by the Child Support (Consequential Provisions) Regulations 2008 reg.~7(4).
}

\subsection[12. Child who is a boarder or an in-patient in hospital]{Child who is a boarder or an in-patient in hospital}

12.---(1)  Where the circumstances of the case are that—
\begin{enumerate}\item[]
($a$) a qualifying child is a boarder at a boarding school or is an in-patient in a hospital; and

($b$) by reason of those circumstances, the person who would otherwise provide day to day care is not doing so,
\end{enumerate}
that case shall be treated as a special case for the purposes of the Act.

(2) For the purposes of this case, section~3(3)($b$)  of the Act shall be modified so that for the reference to the person who usually provides day to day care for the child there shall be substituted a reference to the person who would usually be providing such care for that child but for the circumstances specified in paragraph~(1).

\subsection[13. Child who is allowed to live with his parent under section~23(5) of the Children Act 1989]{Child who is allowed to live with his parent under section~23(5) of the Children Act 1989}

13.---(1)  Where the circumstances of a case are that a qualifying child who is in the care of a local authority in England and Wales is allowed by the authority to live with a parent of his under section~23(5) of the Children Act 1989, that case shall be treated as a special case for the purposes of the Act.

(2) For the purposes of this case, section~3(3)($b$)  of the Act shall be modified so that for the reference to the person who usually provides day to day care for the child there shall be substituted a reference to the parent of the child with whom the local authority allow the child to live with under section~23(5) of the Children Act 1989.

\subsection[14. Person with part-time care who is not a non-resident parent]{Person with part-time care who is not a non-resident parent}

14.---(1)  Where the circumstances of a case are that—
\begin{enumerate}\item[]
($a$) two or more persons who do not live in the same household each provide day to day care for the same qualifying child; and

($b$) those persons do not include any parent who is treated as a non-resident parent of that child by regulation~8(2),
\end{enumerate}
that case shall be treated as a special case for the purposes of the Act.

(2) For the purposes of this case—
\begin{enumerate}\item[]
($a$) the person whose application for a maintenance calculation is being proceeded with shall, subject to sub-paragraph~($b$), be entitled to receive all of the child support maintenance payable under the Act in respect of the child in question;

($b$) on request being made to the Secretary of State by—
\begin{enumerate}\item[]
(i) that person; or

(ii) any other person who is providing day to day care for that child and who intends to continue to provide that care,
\end{enumerate}
the Secretary of State may make arrangements for the payment of any child support maintenance payable under the Act to the persons who provide such care in the same ratio as that in which it appears to the Secretary of State that each is to provide such care for the child in question;\looseness=-1

($c$) before making an arrangement under sub-paragraph~($b$), the Secretary of State shall consider all of the circumstances of the case and in particular the interests of the child, the present arrangements for the day to day care of the child in question and any representations or proposals made by the persons who provide such care for that child.
\end{enumerate}

\section[Part~IV --- Revocation and savings]{Part~IV\\*Revocation and savings}

\renewcommand\parthead{--- Part~IV}

\subsection[15. Revocation and savings]{Revocation and savings}

15.---(1)  Subject to 
the Child Support (Transitional Provisions) Regulations 2000 and % Words inserted (3.3.03) by SI 2003/347 reg 2(1), (2)(a)
paragraphs (2), (3) and (4), the Child Support (Maintenance Assessments and Special Cases) Regulations 1992\footnote{S.I.~1992/1815.} (“the 1992 Regulations”) shall be revoked with respect to a particular case with effect from the date that these Regulations come into force with respect to that type of case (“the commencement date”).

(2) Where before the commencement date in respect of a particular case—
\begin{enumerate}\item[]
($a$) an application was made and not determined for—
\begin{enumerate}\item[]
(i) a maintenance assessment;

(ii) a departure direction; or

(iii) a revision or supersession of a decision;
\end{enumerate}

($b$) the Secretary of State had begun but not completed a revision or supersession of a decision on his own initiative;

($c$) any time limit provided for in Regulations for making an application for a revision or a departure direction had not expired; or

($d$) any appeal was made but not decided or any time limit for making an appeal had not expired,
\end{enumerate}
the provisions of the 1992 Regulations shall continue to apply for the purposes of—
\begin{enumerate}\item[]
($aa$) the decision on the application referred to in sub-paragraph~($a$);

($bb$) the revision or supersession referred to in sub-paragraph~($b$);

($cc$) the ability to apply for the revision or the departure direction referred to in sub-paragraph~($c$)  and the decision whether to revise or to give a departure direction following any such application;

($dd$) any appeal outstanding or made during the time limit referred to in sub-paragraph~($d$); or

($ee$) any revision, supersession, appeal or application for a departure direction in relation to a decision, ability to apply or appeal referred to in sub-paragraphs ($aa$)  to ($dd$) above.
\end{enumerate}

(3) Where immediately before the commencement date in respect of a particular case an interim maintenance assessment was in force, the provisions of the 1992 Regulations shall continue to apply for the purposes of the decision under section~17 of the Act to make a maintenance assessment calculated in accordance with Part~I of Schedule~1 to the Act before its amendment by the 2000 Act and any revision, supersession or appeal in relation to that decision.

(4) Where under regulation~28(1) of the Child Support (Transitional Provisions) Regulations 2000\footnote{S.I.~2000/3186.} an application for a maintenance calculation is treated as an application for a maintenance assessment, the provisions of the 1992 Regulations shall continue to apply for the purposes of the determination of the application and any revision, supersession or appeal in relation to any such assessment made.

(5) Where after the commencement date a maintenance assessment is revised from a date which is prior to the commencement date the 1992 Regulations shall apply for the purposes of that revision.

(6) For the purposes of this regulation—
\begin{enumerate}\item[]
($a$) “departure direction”, “maintenance assessment” and “interim maintenance assessment” have the same meaning as in section~54 of the Act before its amendment by the 2000 Act;

($b$) “revision or supersession” means a revision or supersession of a decision under section~16 or 17 of the Act before their amendment by the 2000 Act; and

($c$) “2000 Act” means the Child Support, Pensions and Social Security Act 2000.
\end{enumerate}

\amendment{
Words inserted in reg.~15(1) (3.3.03) by the Child Support (Transitional Provision) (Miscellaneous Amendments) Regulations 2003 reg.~2(1), (2)(c).
}

\bigskip

Signed 
by authority of the Secretary of State for Social Security.

{\raggedleft
\emph{P.~Hollis}\\*Parliamentary Under-Secretary of State,\\*Department of Social Security

}

18th January 2001

\small

\part[Schedule~--- Net weekly income]{Schedule\\*Net weekly income}

\section[Part~I --- General]{Part~I\\*General}

\renewcommand\parthead{--- Schedule~Part~I}

\subsection*{\itshape Net weekly income}

1.  Net weekly income means the aggregate of the net weekly income of the non-resident parent provided for in this Schedule.

\subsection*{\itshape Amounts to be disregarded when calculating income}

2.  The following amounts shall be disregarded when calculating the net weekly income of the non-resident parent—
\begin{enumerate}\item[]
($a$) where a payment is made in a currency other than sterling, an amount equal to any banking charge or commission payable in converting that payment to sterling;

($b$) any amount payable in a country outside the United Kingdom where there is a prohibition against the transfer to the United Kingdom of that amount.
\end{enumerate}

\vfill

\section[Part~II --- Employed earner]{Part~II\\*Employed earner}

\renewcommand\parthead{--- Schedule~Part~II}

\subsection*{\itshape Net weekly income of employed earner}

3.---(1)  The net weekly income of the non-resident parent as an employed earner shall be—
\begin{enumerate}\item[]
($a$) his earnings provided for in paragraph~4 less the deductions provided for in paragraph~5 and calculated or estimated by reference to the relevant week as provided for in paragraph~6; or

($b$) where the Secretary of State is satisfied that the person is unable to provide evidence or information relating to the deductions provided for in paragraph~5, the non-resident parent’s net earnings estimated by the Secretary of State on the basis of information available to him as to the non-resident parent’s net income.
\end{enumerate}

(2) Where any provision of these Regulations requires the income of a person to be estimated, and that or any other provision of these Regulations requires that the amount of such estimated income is to be taken into account for any purpose, after deducting from it a sum in respect of income tax, or of primary Class~1 contributions under the Contributions and Benefits Act or, as the case may be, the Contributions and Benefits (Northern Ireland) Act, or contributions paid by that person towards an occupational pension scheme or personal pension scheme, then,
\begin{enumerate}\item[]
($a$) subject to sub-paragraph~($c$), the amount to be deducted in respect of income tax shall be calculated by applying to that income the rates of income tax applicable at the effective date less only the personal relief to which that person is entitled under Chapter I of Part~VII of the Income and Corporation Taxes Act 1988 (personal relief); but if the period in respect of which that income is to be estimated is less than a year, the amount of the personal relief deductible under this paragraph~shall be calculated on a pro-rata basis and the amount of income to which each tax rate applies shall be determined on the basis that the ratio of that amount to the full amount of the income to which each tax rate applies is the same as the ratio of the proportionate part of that personal relief to the full personal relief;

($b$) subject to sub-paragraph~($c$), the amount to be deducted in respect of Class~1 contributions under the Contributions and Benefits Act or, as the case may be, the Contributions and Benefits (Northern Ireland) Act, shall be calculated by applying to that income the appropriate primary percentage applicable on the effective date;

($c$) in relation to any bonus or commission which may be included in that person’s income—
\begin{enumerate}\item[]
(i) the amount to be deducted in respect of income tax shall be calculated by applying to the gross amount of that bonus or commission the rate or rates of income tax applicable at the effective date;

(ii) the amount to be deducted in respect of primary Class~1 contributions under the Contributions and Benefits Act or, as the case may be, the Contributions and Benefits (Northern Ireland) Act, shall be calculated by applying to the gross amount of that bonus or commission the appropriate main primary percentage applicable on the effective date but no deduction shall be made in respect of the portion (if any) of the bonus or commission which, if added to the estimated income, would cause such income to exceed the upper earnings limit for Class~1 contributions as provided for in section~5(1)($b$)  of the Contributions and Benefits Act or, as the case may be, the Contributions and Benefits (Northern Ireland) Act;
\end{enumerate}

($d$) the amount to be deducted in respect of any sums or contributions towards an occupational pension scheme or personal pension scheme shall be the full amount of any such payments made or, where that scheme is intended partly to provide a capital sum to discharge a mortgage secured upon that parent’s home, 75 per centum of any such payments made.
\end{enumerate}

\subsection*{\itshape Earnings}

4.---(1)  Subject to sub-paragraph~(2), “earnings” means, in the case of employment as an employed earner, any remuneration or profit derived from that employment and includes—
\begin{enumerate}\item[]
($a$) any bonus, commission, payment in respect of overtime, royalty or fees;

($b$) any holiday pay except any payable more than 4 weeks after termination of the employment;

($c$) any payment by way of a retainer;

($d$) any statutory sick pay under Part~XI of the Contributions and Benefits Act or statutory maternity pay under Part~XII of the Contributions and Benefits Act; 

% Para 4(1)(dd) inserted (16.9.04) by SI 2004/2415 reg 7(3)
($dd$) any statutory paternity pay under Part~XIIZA of the Contributions and Benefits Act or any statutory adoption pay under Part~XIIZB of that Act\footnote{Part~XIIZA was inserted by section~2 of the Employment Act 2002 (c.\ 22) and Part~XIIZB was inserted by section~4 of that Act.};
and

($e$) any payment in lieu of notice, and any compensation in respect of the absence or inadequacy of any such notice, but only in so far as such payment or compensation represents loss of income.
\end{enumerate}

(2) Earnings for the purposes of this Part~of Schedule~1 do not include—
\begin{enumerate}\item[]
($a$) any payment in respect of expenses wholly, exclusively and necessarily incurred in the performance of the duties of the employment;

($b$) any tax-exempt allowance made by an employer to an employee;

($c$) any gratuities paid by customers of the employer;

($d$) any payment in kind;

($e$) any advance of earnings or any loan made by an employer to an employee;

($f$) any amount received from an employer during a period when the employee has withdrawn his services by reason of a trade dispute;

($g$) any payment made in respect of the performance of duties as—
\begin{enumerate}\item[]
(i) an auxiliary coastguard in respect of coast rescue activities;

(ii) a part-time fireman in a fire brigade maintained in pursuance of the Fire Services Acts 1947 to 1959\footnote{10 \& 11 Geo.\ 6 c.\ 41, 14 \& 15 Geo.\ 6 c.\ 27, 7 \& 8 Eliz.~2 c.\ 44.};

(iii) a person engaged part-time in the manning or launching of a lifeboat;

(iv) a member of any territorial or reserve force prescribed in Part~I of Schedule~3 to the Social Security (Contributions) Regulations 1979\footnote{S.I.~1979/591. Part~I of Schedule~3 was substituted by regulation~6 of S.I.~1980/1975 and paragraph~9 was substituted by regulation~4 of S.I.~1994/1553.};
\end{enumerate}

($h$) any payment made by a local authority to a member of that authority in respect of the performance of his duties as a member;

($i$) any payment where—
\begin{enumerate}\item[]
(i) the employment in respect of which it was made has ceased; and

(ii) a period of the same length as the period by reference to which it was calculated has expired since that cessation but prior to the effective date; or
\end{enumerate}

($j$) where, in any week or other period which falls within the period by reference to which earnings are calculated, earnings are received both in respect of a previous employment and in respect of a subsequent employment, the earnings in respect of the previous employment.
\end{enumerate}

\amendment{
Para. 4(1)(dd) inserted (16.9.04) by the Child Support (Miscellaneous Amendments) Regulations 2004 reg.~7(3).
}

\subsection*{\itshape Deductions}

5.---(1)  The deductions to be taken from gross earnings to calculate net income for the purposes of this Part~of the Schedule~are any amount deducted from those earnings by way of—
\begin{enumerate}\item[]
($a$) income tax;

($b$) primary Class~1 contributions under the Contributions and Benefits Act or under the Contributions and Benefits (Northern Ireland) Act; or

($c$) any sums paid by the non-resident parent towards an occupational pension scheme or personal pension scheme or, where that scheme is intended partly to provide a capital sum to discharge a mortgage secured upon that parent’s home, 75 per centum of any such sums.
\end{enumerate}

(2) For the purposes of sub-paragraph~(1)($a$), 
except for cases falling within sub-paragraph~(3),  % Words inserted (30.4.12) by SI 2012/712 reg 6(4)
amounts deducted by way of income tax shall be the amounts actually deducted, including in respect of payments which are not included as earnings in paragraph~4.

% Para 5(3), (4) inserted (30.4.12) by SI 2012/712 reg 6(5)
(3) For the purposes of sub-paragraph~(1)($a$), where an employed earner is gainfully employed outside the United Kingdom, amounts deducted by way of income tax shall be---
\begin{enumerate}\item[]
($a$) the amounts actually deducted in respect of income tax applicable to the income in question, whether that is paid in full in Great Britain or outside Great Britain, or partly paid both in Great Britain and outside Great Britain; or

($b$) where insufficient or unreliable evidence or information is provided by the non-resident parent as to the actual amounts deducted, the amounts that would have been deducted had that employed earner been gainfully employed in Great Britain.
\end{enumerate}

(4) For the purposes of sub-paragraph~(1)($b$), where an employed earner is gainfully employed outside the United Kingdom, amounts deducted by way of primary Class 1 contributions\footnote{Primary Class 1 contributions are defined for Great Britain in Part~I of the Social Security Contributions and Benefits Act 1992 (c.~4) and for Northern Ireland in the Social Security Contributions and Benefits (Northern Ireland) Act 1992 (c.~7). These are payments for National Insurance paid by those that are employed.} shall be the amounts actually deducted under the Contributions and Benefits Act or under the Contributions and Benefits (Northern Ireland) Act and amounts actually deducted outside the United Kingdom for payments of a similar nature.

\amendment{
Words inserted in para.~5(2) and para.~5(3), (4) inserted (30.4.12) by the Child Support (Miscellaneous Amendments) Regulations 2012 reg.~6(4), (5).
}

\subsection*{\itshape Calculation or estimate}

6.---(1)  Subject to 
%sub-paragraphs (2) to (4)
sub-paragraphs (3) and (4)%  % Words substituted (6.4.03) by SI 2003/328 reg 8(4)(a)
, the amount of earnings to be taken into account for the purpose of calculating net income shall be calculated or estimated by reference to the average earnings at the relevant week having regard to such evidence as is available in relation to that person’s earnings during such period as appears appropriate to the Secretary of State, beginning not earlier than 8 weeks before the relevant week and ending not later than the date of the calculation, and for the purposes of the calculation or estimate he may consider evidence of that person’s cumulative earnings during the period beginning with the start of the year of assessment (within the meaning of section~832 of the Income and Corporation Taxes Act 1988) in which the relevant week falls and ending with a date no later than the date when the calculation is made.

% Para 6(2) omitted (6.4.03) by SI 2003/328 reg 8(4)(b)
%(2) Subject to sub-paragraph~(4), where a person has claimed, or has been paid, working families' tax credit or disabled person’s tax credit on any day during the period beginning not earlier than 8 weeks before the relevant week and ending not later than the date on which the calculation is made, the Secretary of State may have regard to the amount of earnings taken into account in determining entitlement to those tax credits in order to calculate or estimate the amount of earnings to be taken into account for the purposes of calculating net earnings, notwithstanding the fact that entitlement to those tax credits may have been determined by reference to earnings attributable to a period other than that specified in sub-paragraph~(1).

(3) Where a person’s earnings during the period of 52 weeks ending with the relevant week include a bonus or commission 
%made in anticipation of the calculation of profits  % Words omitted (27.10.08) by SI 2008/2544 reg 6
which is paid separately from, or in relation to a longer period than, the other earnings with which it is paid, the amount of that bonus or commission shall be determined for the purposes of the calculation of earnings by aggregating any such payments received in that period and dividing by 52.

(4) Where a calculation would, but for this sub-paragraph, produce an amount which, in the opinion of the Secretary of State, does not accurately reflect the normal amount of the earnings of the person in question, such earnings, or any part of them, shall be calculated by reference to such other period as may, in the particular case, enable the normal weekly earnings of that person to be determined more accurately, and for this purpose the Secretary of State shall have regard to—
\begin{enumerate}\item[]
($a$) the earnings received, or due to be received from any employment in which the person in question is engaged, has been engaged or is due to be engaged; and

($b$) the duration and pattern, or the expected duration and pattern, of any employment of that person.
\end{enumerate}

\amendment{
Words substituted in para. 6(1) and para. 6(2) omitted (6.4.03) by the Child Support (Miscellaneous Amendments) Regulations 2003 reg.~8(4)(a), (b).

Words omitted in para. 6(3) (27.10.08) by the Child Support (Miscellaneous Amendments) (No.~2) Regulations 2008 reg.~6.
}

% Para 6A inserted (30.4.12) by SI 2012/712 reg 6(6)
\subsection*{\sloppy\itshape Estimate of net weekly income of employed earner where insufficient information available}

6A.---(1)  Where the 
%Commission 
Secretary of State  % Words substituted (1.8.12) by SI 2012/2007 Sch para 114(2)
is calculating net weekly income of an employed earner under Part II of the Schedule and the information available in relation to that income is insufficient or unreliable, the 
%Commission 
Secretary of State  % Words substituted (1.8.12) by SI 2012/2007 Sch para 114(2)
may estimate that income and, in doing so, may make any assumptions as to any fact.

(2) Where the 
%Commission 
Secretary of State  % Words substituted (1.8.12) by SI 2012/2007 Sch para 114(2)
is satisfied that the non-resident parent is engaged in a particular occupation as an employee, the assumptions referred to in sub-paragraph~(1) may include an assumption that the non-resident parent has the average net weekly income of a person engaged in that occupation in the United Kingdom or any part of the United Kingdom.

\amendment{
Para. 6A inserted (30.4.12) by the Child Support (Miscellaneous Amendments) Regulations 2012 reg.~6(6).

Words substituted in para.~6A(1), (2) (1.8.12) by the Public Bodies (Child Maintenance and Enforcement Commission: Abolition and Transfer of Functions) Order 2012 Sch. para.~114(2).
}

\section[Part~III --- Self-employed earner]{Part~III\\*Self-employed earner}

\renewcommand\parthead{--- Schedule~Part~III}

\subsection*{\itshape 
%Figures submitted to the Inland Revenue
Net weekly income of non-resident parent as a self-employed earner  % Title substituted (1.8.07) by SI 2007/1979 reg 5(2)(a)
}

7.---%
%(1)  Subject to sub-paragraph~(6) the net weekly income of the non-resident parent as a self-employed earner shall be his gross earnings calculated by reference to one of the following, as the Secretary of State may decide, less the deductions to which sub-paragraph~(3) applies—
%\begin{enumerate}\item[]
%($a$) the total taxable profits from self-employment of that earner as submitted to the Inland Revenue in accordance with their requirements by or on behalf of that earner; or
%
%($b$) the income from self-employment as a self-employed earner as set out on the tax calculation notice or, as the case may be, the revised notice.
%\end{enumerate}
%
% Para 7(1) substituted (1.8.07) by SI 2007/1979 reg 5(2)(b)
(1) Subject to sub-paragraph~(6) and to paragraph~8, the net weekly income of the non-resident parent as a self-employed earner shall be his gross earnings less the deductions to which sub-paragraph~(3) applies.

% Para 7(1A) inserted (1.8.07) by SI 2007/1979 reg 5(2)(c)
(1A) In this paragraph~and paragraph~8 a person’s “gross earnings” are his taxable profits calculated in accordance with Part~II of the Income Tax (Trading and Other Income) Act 2005\footnote{2005 c.\ 5.}.

%(2) Where the information referred to in head~($a$)  or ($b$)  of sub-paragraph~(1) is made available to the Secretary of State he may nevertheless require the information referred to in the other head~from the non-resident parent and where the Secretary of State becomes aware that a revised notice has been issued he may require and use this in preference to the other information referred to in sub-paragraph~(1)($a$)  and ($b$).

% Para 7(2) substituted (1.8.07) by SI 2007/1979 reg 5(2)(d)
(2) The non-resident parent shall provide to the Secretary of State on demand a copy of—
\begin{enumerate}\item[]
($a$) any tax calculation notice issued to him by Her Majesty’s Revenue and Customs; and

($b$) any revised tax calculation notice issued to him by Her Majesty’s Revenue and Customs.
\end{enumerate}

(3) This paragraph~applies to the following deductions—
\begin{enumerate}\item[]
($a$) any income tax relating to the gross earnings from the self-employment determined in accordance with sub-paragraph~(4);

($b$) any National Insurance contributions relating to the gross earnings from the self-employment determined in accordance with sub-paragraph~(5); and

($c$) any premiums paid by the non-resident parent in respect of a retirement annuity contract or a personal pension scheme or, where that scheme is intended partly to provide a capital sum to discharge a mortgage or a charge secured upon the parent’s home, 75 per centum of the contributions payable.
\end{enumerate}

(4) For the purpose of sub-paragraph~(3)($a$), the income tax to be deducted from the gross earnings shall be determined in accordance with the following provisions—
\begin{enumerate}\item[]
($a$) subject to head~($d$), an amount of gross earnings 
%equivalent to any personal allowance 
calculated as if it were equivalent to any personal allowance which would be  % Words substituted (16.3.05) by SI 2005/785 reg 6(5)
applicable to the earner by virtue of the provisions of Chapter I of Part~VII of the Income and Corporation Taxes Act 1988 (personal relief) shall be disregarded;

($b$) subject to head~($c$), an amount equivalent to income tax shall be calculated in relation to the gross earnings remaining following the application of head~($a$)  (the “remaining earnings”);

($c$) the tax rate applicable at the effective date shall be applied to all the remaining earnings, where necessary increasing or reducing the amount payable to take account of the fact that the earnings related to a period greater or less than one year; and

($d$) the amount to be disregarded by virtue of head~($a$)  shall be calculated by reference to the yearly rate applicable at the effective date, that amount being reduced or increased in the same proportion to that which the period represented by the gross earnings bears to the period of one year.
\end{enumerate}

(5) For the purposes of sub-paragraph~(3)($b$), the amount to be deducted in respect of National Insurance contributions shall be the total of—
\begin{enumerate}\item[]
($a$) the amount of Class~2 contributions (if any) payable under section~11(1) or, as the case may be, (3) of the Contributions and Benefits Act or under section~11(1) or (3) of the Contributions and Benefits (Northern Ireland) Act; and

($b$) the amount of Class~4 contributions (if any) payable under section~15(2) of that Act, or under section~15(2) of the Contributions and Benefits (Northern Ireland) Act,
\end{enumerate}
at the rates applicable at the effective date.

(6) The net weekly income of a self-employed earner may only be determined in accordance with this paragraph~where the earnings concerned relate to a period which terminated not more than 24 months prior to the relevant week.

% Para 7(7) omitted (1.8.07) by SI 2007/1979 reg 5(2)(e)
%(7) In this paragraph—
%\begin{enumerate}\item[]
%“tax calculation notice” means a document issued by the Inland Revenue containing information as to the income of the self-employed earner; and
%
%“revised notice” means a notice issued by the Inland Revenue where there has been a tax calculation notice and there is a revision of the figures relating to the income of a self-employed earner following an enquiry under section~9A of the Taxes Management Act 1970\footnote{1970 c.\ 9. Section~9A was inserted by sections~180 and 199(1) and (2) of the Finance Act 1994 (c.\ 9) and amended by section~133 and Schedule~19, paragraph~2 of the Finance Act 1996 (c.\ 8).} or otherwise by the Inland Revenue.
%\end{enumerate}

(8) Any request by the Secretary of State in accordance with sub-paragraph~(2) for the provision of information shall set out the possible consequences of failure to provide such information, including details of the offences provided for in section~14A of the Act\footnote{Section~14A is inserted by section~13 of the Child Support, Pensions and Social Security Act 2000 (c.\ 19).} for failing to provide, or providing false, information.

\amendment{
Words substituted in para. 7(4)(a) (16.3.05) by the Child Support (Miscellaneous Amendments) Regulations 2005 reg.~6(5).

Para. 7(1A) inserted, para. 7(1), (2) and title substituted and para. 7(7) omitted (1.8.07) by the Child Support (Miscellaneous Amendments) Regulations reg.~5(2).
}

\subsection*{\itshape Figures calculated using gross receipts less deductions}

8.---(1)  Where—
\begin{enumerate}\item[]
($a$) the conditions of paragraph~7(6) are not satisfied; or

($b$) the Secretary of State accepts that it is not reasonably practicable for the self-employed earner to provide information relating to his gross earnings from self-employment in the forms submitted to, or as issued or revised by, the Inland Revenue%; or
%
% Para 8(1)(c) and ``or'' before it omitted (1.8.07) by SI 2007/1979 reg 5(3)(a)
%($c$) in the opinion of the Secretary of State, information as to the gross earnings of the self-employed earner which has satisfied the criteria set out in paragraph~7 does not accurately reflect the normal weekly earnings of the self-employed earner
,
\end{enumerate}
net income means in the case of employment as a self-employed earner his earnings calculated by reference to the gross receipts 
%of the employment 
in respect of employment which are of a type which would be taken into account under paragraph~7(1)  % Words substituted (30.4.02) by SI 2002/1204 reg 7(2)
less the deductions provided for in sub-paragraph~(2).

(2) The deductions to be taken from the gross receipts to calculate net earnings for the purposes of this paragraph~are—
\begin{enumerate}\item[]
($a$) any expenses which are reasonably incurred and are wholly and exclusively defrayed for the purposes of the earner’s business in the period by reference to which his earnings are determined under paragraph~9(2) or~(3);

($b$) any value added tax paid in the period by reference to which his earnings are determined in excess of value added tax received in that period;

($c$) any amount in respect of income tax determined in accordance with sub-paragraph~(4);

($d$) any amount of National Insurance contributions determined in accordance with sub-paragraph~(4); and

($e$) any premium paid by the non-resident parent in respect of a retirement annuity contract or a personal pension scheme or, where that scheme is intended partly to provide a capital sum to discharge a mortgage or a charge secured upon the parent’s home, 75 per centum of contributions payable.
\end{enumerate}

(3) For the purposes of sub-paragraph~(2)($a$)—
\begin{enumerate}\item[]
($a$) such expenses include—
\begin{enumerate}\item[]
(i) repayment of capital on any loan used for the replacement, in the course of business, of equipment or machinery, or the repair of an existing business asset except to the extent that any sum is payable under an insurance policy for its repair;

(ii) any income expended in the repair of an existing business asset except to the extent that any sum is payable under an insurance policy for its repair; and

(iii) any payment of interest on a loan taken out for the purposes of the business;
\end{enumerate}

($b$) such expenses do not include—
\begin{enumerate}\item[]
% Para 8(3)(b)(i), (iii), (iv), (v), (vii) omitted (1.8.07) by SI 2007/1979 reg 5(3)(b)
%(i) repayment of capital on any other loan taken out for the purposes of the business;

(ii) any capital expenditure;

%(iii) the depreciation of any capital assets;

%(iv) any sum employed, or intended to be employed, in the setting up or expansion of the business;

%(v) any loss incurred before the beginning of the period by reference to which earnings are determined;

(vi) any expenses incurred in providing business entertainment%; or
%
%(vii) any loss incurred in any other employment in which he is engaged as a self-employed earner
.
\end{enumerate}
\end{enumerate}

(4) For the purposes of sub-paragraph~(2)($c$)  and ($d$), the amounts in respect of income tax and National Insurance contributions to be deducted from the gross receipts shall be determined in accordance with paragraph~7(4) and (5) of this Schedule~as if in paragraph~7(4) references to gross earnings were references to taxable earnings and in this sub-paragraph~“taxable earnings” means the gross receipts of the earner less the deductions mentioned in sub-paragraph~(2)($a$)  and~($b$).

\amendment{
Words substituted in para. 8(1) (30.4.02) by the Child Support (Miscellaneous Amendments) Regulations 2002 reg.~7(2).

Para. 8(1)(c), (3)(b)(i), (iii), (iv), (v), (vii) omitted (1.8.07) by the Child Support (Miscellaneous Amendments) Regulations reg.~5(2).
}

\subsection*{\itshape Rules for calculation under paragraph~8}

9.---(1)  This paragraph~applies only where the net income of a self-employed earner is calculated or estimated under paragraph~8 of this Schedule.

(2) Where—
\begin{enumerate}\item[]
($a$) a non-resident parent has been a self-employed earner for 52 weeks or more, including the relevant week, the amount of his net weekly income shall be determined by reference to the average of the earnings which he has received in the 52 weeks ending with the relevant week; or

($b$) a non-resident parent has been a self-employed earner for a period of less than 52 weeks including the relevant week, the amount of his net weekly income shall be determined by reference to the average of the earnings which he has received during that period.
\end{enumerate}

(3) Where a calculation would, but for this sub-paragraph, produce an amount which, in the opinion of the Secretary of State, does not accurately reflect the normal weekly income of the non-resident parent in question, such earnings, or any part of them, shall be calculated by reference to such other period as may, in the particular case, enable the normal weekly earnings of the non-resident parent to be determined more accurately and for this purpose the Secretary of State shall have regard to—
\begin{enumerate}\item[]
($a$) the earnings from self-employment received, or due to be received, by him; and

($b$) the duration and pattern, or the expected duration and pattern, of any self-employment of that non-resident parent.
\end{enumerate}

% Para 9(4) omitted (6.4.03) by SI 2003/328 reg 8(4)(b)
%(4) Where a person has claimed, or has been paid, working families' tax credit or disabled person’s tax credit on any day during the period beginning not earlier than 8 weeks before the relevant week and ending not later than the date on which the calculation is made, the Secretary of State may have regard to the amount of earnings taken into account in determining entitlement to those tax credits in order to calculate or estimate the amount of earnings to be taken into account for the purposes of calculating net income, notwithstanding the fact that entitlement to those tax credits may have been determined by reference to earnings attributable to a period other than that specified in sub-paragraph~(2).

\amendment{
Para. 9(4) omitted (6.4.03) by the Child Support (Miscellaneous Amendments) Regulations 2003 reg.~8(4)(b).
}

% Para 9A inserted (30.4.12) by SI 2012/712 reg 6(7)
\subsection*{\sloppy\itshape Estimate of net weekly income of self-\hspace{0pt}employed earner where insufficient information available}

9A.---(1)  Where the 
%Commission 
Secretary of State  % Words substituted (1.8.12) by SI 2012/2007 Sch para 114(2)
is calculating net weekly income of a self-employed earner under Part III of the Schedule and the information available in relation to that income is insufficient or unreliable, the 
%Commission 
Secretary of State  % Words substituted (1.8.12) by SI 2012/2007 Sch para 114(2)
may estimate that income and, in doing so, may make any assumptions as to any fact.

(2) Where the 
%Commission 
Secretary of State  % Words substituted (1.8.12) by SI 2012/2007 Sch para 114(2)
is satisfied that the non-resident parent is engaged in a particular occupation as a self-employed earner, the assumptions referred to in sub-paragraph~(1) may include an assumption that the non-resident parent has the average net weekly income of a person engaged in that occupation in the United Kingdom or any part of the United Kingdom.

\amendment{
Para. 9A inserted (30.4.12) by the Child Support (Miscellaneous Amendments) Regulations 2012 reg.~6(6).

Words substituted in para.~9A(1), (2) (1.8.12) by the Public Bodies (Child Maintenance and Enforcement Commission: Abolition and Transfer of Functions) Order 2012 Sch. para.~114(3).
}

\subsection*{\itshape Income from board or lodging}

10.  In a case where a non-resident parent is a self-employed earner who provides board and lodging, his earnings shall include payments received for that provision where those payments are the only or main source of income of that earner.

\section[Part~IV --- Tax credits]{Part~IV\\*Tax credits}

\renewcommand\parthead{--- Schedule~Part~IV}

\subsection*{\itshape 
%Working families' tax credit
Working tax credit  % Words substituted (6.4.03) by SI 2003/328 reg 8(4)(c)(i)
}

11.---(1)  Subject to 
%sub-paragraphs (2) and (3)
sub-paragraph~(2)%  % Words substituted (6.4.03) by SI 2003/328 reg 8(4)(c)(ii)(aa)
, payments by way of 
%working families' tax credit
working tax credit  % Words substituted (6.4.03) by SI 2003/328 reg 8(4)(c)(i)
%under section~128 of the Contributions and Benefits Act\footnote{\emph{See} section~1 of, and paragraphs 1 and 2($g$) of Schedule~1 to, the Tax Credits Act 1999 (c. 19).},  % Words omitted (6.4.03) by SI 2003/328 reg 8(4)(c)(ii)(bb)
shall be treated as the income of the non-resident parent where he has qualified for them by his engagement in, and normal engagement in, remunerative work, at the rate payable at the effective date.

(2) Where 
%working families' tax credit
working tax credit  % Words substituted (6.4.03) by SI 2003/328 reg 8(4)(c)(i)
is payable and the amount which is payable has been calculated by reference to 
%the weekly earnings 
the earnings  % Words substituted (6.4.03) by SI 2003/328 reg 8(4)(c)(iii)(aa)
of the non-resident parent and another person—
\begin{enumerate}\item[]
($a$) where during the period which is used by the Inland Revenue to calculate his income 
%the normal weekly earnings 
the earnings  % Words substituted (6.4.03) by SI 2003/328 reg 8(4)(c)(iii)(bb)
%(as determined in accordance with Chapter II of Part~IV of the Family Credit (General) Regulations 1987\footnote{S.I.~1987/1973. Relevant amending instruments are S.I.~1988/1438 and 1970, 1990/574, 1991/1520, 1992/573 and 2155, 1993/315 and 2119, 1994/527, 1924 and 2139, 1995/516, 1996/462, 2545, 3137, and 1997/2793.})  % Words omitted (6.4.03) by SI 2003/328 reg 8(4)(c)(iii)(cc)
of that parent exceed those of the other person, the amount payable by way of 
%working families' tax credit
working tax credit  % Words substituted (6.4.03) by SI 2003/328 reg 8(4)(c)(i)
shall be treated as the income of that parent;

($b$) where during that period 
%the normal weekly earnings 
the earnings  % Words substituted (6.4.03) by SI 2003/328 reg 8(4)(c)(iii)(bb)
of that parent equal those of the other person, half of the amount payable by way of 
%working families' tax credit
working tax credit  % Words substituted (6.4.03) by SI 2003/328 reg 8(4)(c)(i)
shall be treated as the income of that parent; and

($c$) where during that period 
%the normal weekly earnings 
the earnings  % Words substituted (6.4.03) by SI 2003/328 reg 8(4)(c)(iii)(bb)
of that parent are less than those of that other person, the amount payable by way of 
%working families' tax credit
working tax credit  % Words substituted (6.4.03) by SI 2003/328 reg 8(4)(c)(i)
shall not be treated as the income of that parent.
\end{enumerate}

% Para 11(2A) inserted (6.4.03) by SI 2003/328 reg 8(4)(c)(iv)
(2A) For the purposes of this paragraph, “earnings” means the employment income and the income from self-employment of the non-resident parent and the other person referred to in sub-paragraph~(2), as determined for the purposes of their entitlement to working tax credit.

% Para 11(3) omitted (6.4.03) by SI 2003/328 reg 8(4)(c)(v)
%(3) Where—
%\begin{enumerate}\item[]
%($a$) 
%%working families' tax credit
%working tax credit  % Words substituted (6.4.03) by SI 2003/328 reg 8(4)(c)(i)
%is in payment; and
%
%($b$) not later than the effective date the person, or, if more than one, each of the persons by reference to whose engagement, and normal engagement, in remunerative work that payment has been calculated is no longer the partner of the person to whom that payment is made,
%\end{enumerate}
%the payment in question shall only be treated as the income of the non-resident parent in question where he is in receipt of it.

\amendment{
Words substituted in para. 11 and heading, words omitted in para. 11(1), (2)(a), para. 11(2A) inserted and para. 11(3) omitted (6.4.03) by the Child Support (Miscellaneous Amendments) Regulations 2003 reg.~8(4)(c).
}

\subsection*{\itshape Employment Credits}

12.  Payments made by way of employment credits under section~2(1) of the Employment and Training Act 1973 to a non-resident parent who is participating in a scheme arranged under section~2(2) of the Employment and Training Act 1973 and known as the New Deal 50 plus shall be treated as the income of the non-resident parent, at the rate payable at the effective date.

\amendment{
Para. 13 omitted (6.4.03) by the Child Support (Miscellaneous Amendments) Regulations 2003 reg.~8(4)(d).
}

%% Para 13 omitted (6.4.03) by SI 2003/328 reg 8(4)(d)
%\subsection*{\itshape Disabled Person’s Tax Credits}
%
%%13.  Payments made by way of disabled person’s tax credit under section~129 of the Contributions and Benefits Act\footnote{\emph{See} section~1 of, and paragraphs 1 and 2($h$) of Schedule~1 to, the Tax Credits Act 1999.} to a non-resident parent shall be treated as the income of the non-resident parent at the rate payable at the effective date.
%
%% Para 13 substituted (30.4.02) by SI 2002/1204 reg 7(3)
%13.---(1)  Subject to sub-paragraphs (2) and (3), payments made by way of disabled person’s tax credit under section~129 of the Contributions and Benefits Act\footnote{See section~1 of, and paragraphs 1 and 2($h$) of Schedule~1 to, the Tax Credits Act 1999 (c.\ 10).} to a non-resident parent shall be treated as the income of the non-resident parent, at the rate payable at the effective date.
%
%(2) Where disabled person’s tax credit is payable where a non-resident parent and another person both meet the entitlement criteria for the payment and the amount which is payable has been calculated by reference to the weekly earnings of the non-resident parent and the other person—
%\begin{enumerate}\item[]
%($a$) where during the period which is used by the Inland Revenue to calculate the non-resident parent’s income the normal weekly earnings (as determined in accordance with Chapter II of Part~V of the Disability Working Allowance (General) Regulations 1991\footnote{S.I.~1991/2887. Chapter II was amended by regulation~17 of S.I.~1993/315, regulation~39 of S.I.~1993/2119, regulation~3 of S.I.~1994/1924, regulation~3 of S.I.~1994/2139, regulation~3 of S.I.~1996/1994, regulation~2 of S.I.~1996/3137 and regulations 16, 17 and 26 of, and Schedule~2 to, S.I.~1999/2487.}) of that parent exceed those of the other person, the amount payable by way of disabled person’s tax credit shall be treated as the income of that parent;
%
%($b$) where during that period the normal weekly earnings of that parent equal those of the other person, half of the amount payable by way of disabled person’s tax credit shall be treated as the income of that parent; and
%
%($c$) where during that period the normal weekly earnings of that parent are less than those of that other person, the amount payable by way of disabled person’s tax credit shall not be treated as the income of that parent.
%\end{enumerate}
%
%(3) Where—
%\begin{enumerate}\item[]
%($a$) disabled person’s tax credit is in payment; and
%
%($b$) not later than the effective date the person, or, if more than one, each of the persons by reference to whose entitlement that payment has been calculated is no longer the partner of the person to whom that payment is made,
%\end{enumerate}
%the payment shall only be treated as the income of the non-resident parent in question where he is in receipt of it.
%
%\amendment{
%Para. 13 substituted (30.4.02) by the Child Support (Miscellaneous Amendments) Regulations 2002 reg.~7(3).
%}

% Para 13A inserted (6.4.03) by SI 2003/328 reg 8(4)(e)
\subsection*{\itshape Child tax credit}

13A.  Payments made by way of child tax credit to a non-resident parent or his partner at the rate payable at the effective date.

\amendment{
Para. 13A added (6.4.03) by the Child Support (Miscellaneous Amendments) Regulations 2003 reg.~8(4)(e).
}

\section[Part~V --- Other income]{Part~V\\*Other income}

\renewcommand\parthead{--- Schedule~Part~V}

\subsection*{\itshape Amount}

14.  The amount of other income to be taken into account in calculating or estimating net weekly income shall be the aggregate of the payments to which paragraph~15 applies, net of any income tax deducted and otherwise determined in accordance with this Part.

\subsection*{\itshape Types}

15.  This paragraph~applies to any periodic payment of pension or other benefit under an occupational or personal pension scheme or a retirement annuity contract or other such scheme for the provision of income in retirement whether or not approved by the Inland Revenue.

\subsection*{\itshape Calculation or estimate and period}

16.---(1)  The amount of any income to which this Part~applies shall be calculated or estimated—
\begin{enumerate}\item[]
($a$) where it has been received in respect of the whole of the period of 26 weeks which ends at the end of the relevant week, by dividing such income received in that period by 26;

($b$) where it has been received in respect of part of the period of 26 weeks which ends at the end of the relevant week, by dividing such income received in that period by the number of complete weeks in respect of which such income is received and for this purpose income shall be treated as received in respect of a week if it is received in respect of any day in the week in question.
\end{enumerate}

(2) Where a calculation or estimate to which this Part~applies would, but for this sub-paragraph, produce an amount which, in the opinion of the Secretary of State, does not accurately reflect the normal amount of the other income of the non-resident parent in question, such income, or any part of it, shall be calculated by reference to such other period as may, in the particular case, enable the other income of that parent to be determined more accurately and for this purpose the Secretary of State shall have regard to the nature and pattern of receipt of such income. 

% Para 17 added (16.9.04) by SI 2004/2415 reg 7(4)
\section[Part~VI --- Benefits, pensions and allowances]{Part~VI\\*Benefits, pensions and allowances}

\renewcommand\parthead{--- Schedule~Part~VI}

17.---(1)  Subject to paragraph~(2), the net weekly income of a non-resident parent shall include payments made by way of benefits, pensions and allowances prescribed in regulation~4 for the purposes of paragraph~4(1)($b$)  and ($c$)  of Schedule~1 to the Act, to a non-resident parent or his partner at the rate payable at the effective date.

(2) Paragraph (1) applies only for the purpose of establishing whether the non-resident parent is a person to whom paragraph~5($b$)  of Schedule~1 to the Act applies.

\amendment{
Para. 17 added (16.9.04) by the Child Support (Miscellaneous Amendments) Regulations 2004 reg.~7(4).
}

\part{Explanatory Note}

\renewcommand\parthead{--- Explanatory Note}

\subsection*{(This note is not part of the Regulations)}

These Regulations provide for various matters relating to the calculation of child support maintenance under the Child Support Act 1991 (“the Act”) and also make provision for special cases under the Act, consequent upon the introduction of changes to the child support system made by the Child Support, Pensions and Social Security Act 2000 (c.\ 48). Subject to savings for transitional purposes, these Regulations revoke the Child Support (Maintenance Assessments and Special Cases) Regulations 1992 (S.I.~1992/1815). Apart from paragraphs (1) and (2) of regulation~4, which come into force on 31st January 2001, these Regulations come into force at different times for different cases according to the dates on which provisions of the Child Support, Pensions and Social Security Act 2000 which are relevant to these Regulations are commenced for different types of cases.

Regulation 1 contains provisions relating to citation, commencement and interpretation.

Regulation 2 contains general provisions regarding the calculation of child support maintenance under the Act. The Schedule~to these Regulations prescribes the amounts to be taken into account to calculate net weekly income for the purposes of Schedule~1 to the Act.

Regulation 3 prescribes the method of calculating the reduced rate of child support maintenance and regulation~4 prescribes the benefits, pensions and allowances for the purposes of paragraph~4(1) of Schedule~1 to the Act (flat rate cases).

Regulation 5 prescribes the circumstances for which the rate payable is nil.

Regulation 6 provides a general rule for adjusting the child support maintenance payable following apportionment and regulation~7 prescribes the circumstances in which a night will count for the purposes of paragraphs 7 and 8 of Part~I of Schedule~1 to the Act (shared care).

Regulations 8 to 14 prescribe the circumstances in which cases are to be treated as special cases for the purposes of the Act. These include cases where persons are treated as non-resident parents; where care of a qualifying child or a relevant other child is provided in part by a local authority; where the non-resident parent is liable to pay maintenance under a maintenance order; where a child is a boarder or an in-patient in hospital; where a child is allowed to live with his parent under section~23(5) of the Children Act 1989 and where a person with part-time care of the child is not a non-resident parent.

Regulation 15 revokes the Child Support (Maintenance Assessments and Special Cases) Regulations 1992, with savings for transitional purposes.

The impact on business of these Regulations was covered in the Regulatory Impact Assessment (RIA) for the Child Support, Pensions, and Social Security Act 2000, in accordance with which, and in consequence of which, these Regulations are made. A copy of that RIA has been placed in the libraries of both Houses of Parliament and can be obtained from the Department of Social Security, Regulatory Impact Unit, Adelphi, 1--11 John Adam Street, London \textsc{\lowercase{WC2N 6HT}}. 

\end{document}