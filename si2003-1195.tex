\documentclass[12pt,a4paper]{article}

\newcommand\regstitle{The Social Security (Hospital In-Patients and Miscellaneous Amendments) Regulations 2003}

\newcommand\regsnumber{2003/1195}

%\opt{newrules}{
\title{\regstitle}
%}

%\opt{2012rules}{
%\title{Child Maintenance and Other Payments Act 2008\\(2012 scheme version)}
%}

\author{S.I.\ 2003 No.\ 1195}

\date{Made
30th April 2003\\
Laid before Parliament
6th May 2003\\
Coming into force
in accordance with regulation 1
}

%\opt{oldrules}{\newcommand\versionyear{1993}}
%\opt{newrules}{\newcommand\versionyear{2003}}
%\opt{2012rules}{\newcommand\versionyear{2012}}

\usepackage{csa-regs}

\setlength\headheight{27.57402pt}

\begin{document}

\maketitle

\noindent
The Secretary of State for Work and Pensions, in exercise of the powers conferred upon him by sections 52(4) and 54 of, and paragraph 5($a$)  of Schedule 1 to, the Child Support Act 1991\footnote{1991 c.\ 48; section 54 is cited for the meaning ascribed to the word “prescribed”; Schedule 1 was substituted by section 1 of the Child Support, Pensions and Social Security Act 2000 (c.\ 19).}, sections 123(1)($a$), ($d$)  and ($e$), 130(4), 131(10), 135(1), 137(1) and 175(1), (3) and (4) of the Social Security Contributions and Benefits Act 1992\footnote{1992 c.\ 4; section 123(1)($e$) and section 131 were substituted by the Local Government Finance Act 1992 (c.\ 14), Schedule 9, paragraphs 1(1) and 4 respectively; section 137(1) is an interpretation provision and is cited because of the meaning ascribed to the word “prescribed”.}, sections 73(1)($b$)  and 189(1), (4) and (5) of the Social Security Administration Act 1992\footnote{1992 c.\ 5; section 73 was amended by the Jobseekers Act 1995 (c.\ 18), Schedule 2, paragraph 49.}, sections 4(5) and 36(1), (2) and (4)($a$)  of the Jobseekers Act 1995\footnote{1995 c.\ 18.} and sections 2(3) and (6), 13(1), 17(1) and 19(1) of the State Pension Credit Act 2002\footnote{2002 c.\ 16; section 17(1) is an interpretation provision and is cited because of the meaning ascribed to the word “prescribed”.}, and all other powers enabling him in that behalf, after consultation in respect of provisions in these Regulations relating to housing benefit and council tax benefit with organisations appearing to him to be representative of the authorities concerned\footnote{\emph{See} the Social Security Administration Act 1992, section 176(1)($a$) which was amended by the Local Government Finance Act 1992, Schedule 9, paragraph 23.}, and after agreement by the Social Security Advisory Committee that the proposals to make these Regulations should not be referred to it\footnote{\emph{See} the Social Security Administration Act 1992, section 173(1)($b$).}, hereby makes the following Regulations: 

{\sloppy

\tableofcontents

}

\bigskip

\setcounter{secnumdepth}{-2}

\subsection[1. Citation and commencement]{Citation and commencement}

1.  These Regulations may be cited as the Social Security (Hospital In-Patients and Miscellaneous Amendments) Regulations 2003 and shall come into force for the purposes of—
\begin{enumerate}\item[]
($a$) this regulation and regulations 7 to 10, on 21st May 2003;

($b$) regulation 2—
\begin{enumerate}\item[]
(i) in so far as it relates to a particular beneficiary other than a beneficiary in receipt of incapacity benefit or severe disablement allowance, on 21st May 2003 if it is his appropriate pay day or, if not, on his appropriate pay day next following 21st May 2003, and for the purposes of this sub-paragraph “appropriate pay day” shall be construed in accordance with regulation 22(3) of, and Schedule 6 to, the Social Security (Claims and Payments) Regulations 1987\footnote{S.I.\ 1987/1968.}; and

(ii) in so far as it relates to a particular beneficiary in receipt of incapacity benefit or severe disablement allowance, on 21st May 2003;
\end{enumerate}

($c$) regulation 3, in so far as it relates to a beneficiary paid in advance, on the first day of the first benefit week to commence for that beneficiary on or after 21st May 2003, but otherwise on the first day of the benefit week which includes that date, and for the purpose of this paragraph “benefit week” has the same meaning as in the Income Support (General) Regulations 1987\footnote{S.I.\ 1987/1967.};

($d$) regulations 4(1) and 4(3) to (5) and regulations 5(1) and 5(3) to (5) on 26th May 2003;

($e$) regulations 4(2) and 5(2) on 6th October 2003; and

($f$) regulation 6, in so far as it relates to a beneficiary paid in advance, on the first day of the first benefit week to commence for that beneficiary on or after 21st May 2003, but otherwise on the first day of the benefit week which includes that date, and for the purpose of this paragraph “benefit week” has the same meaning as in the Jobseeker’s Allowance Regulations 1996\footnote{S.I.\ 1996/207.}.
\end{enumerate}

\subsection[2. Amendment of the Social Security (Hospital In-Patients) Regulations 1975]{Amendment of the Social Security (Hospital In-Patients) Regulations 1975}

2.---(1)  The Social Security (Hospital In-Patients) Regulations 1975\footnote{S.I.\ 1975/555.} shall be amended in accordance with the following paragraphs of this regulation.

(2) In regulation 2(1)\footnote{The definition of “38\% of the basic pension” was substituted by S.I.\ 2002/685.} (interpretation) omit the definition of “38\% of the basic pension”.

(3) For regulation 4\footnote{Regulation 4 was amended by S.I.\ 1987/1683.} substitute—
\begin{quotation}
\subsection*{“Circumstances in which personal benefit is to be adjusted}

4. Where a person—
\begin{enumerate}\item[]
($a$) receives, or has received, free in-patient treatment continuously for a period of more than 52 weeks; and

($b$) satisfies the conditions for the receipt of a personal benefit which is specified in Schedule 2 to these regulations,
\end{enumerate}
for any part of the period after the 52nd week the weekly rate of that benefit shall be adjusted in accordance with regulation 6.”.
\end{quotation}

(4) In regulation 4A(1)\footnote{Regulation 4A was inserted by S.I.\ 1986/903 and amended by S.I.\ 1987/1683.} (provision for adjusting benefit for part of a week) omit “5 or”.

(5) Regulation 5\footnote{Regulation 5 was substituted by S.I.\ 1987/1683.} (adjustment of personal benefit after 6 weeks in hospital) shall be omitted.

(6) In regulation 6\footnote{Paragraph (2) of regulation 6 was substituted by S.I.\ 1987/1683.} (adjustment of personal benefit after 52 weeks in hospital) for “4($d$)”, in both places where it occurs, substitute “4”.

(7) In regulation 9\footnote{Regulation 9 was substituted by S.I.\ 1977/1693 and amended by S.I.\ 1987/1683.} (circumstances in which dependency benefit is to be adjusted), in paragraph ($b$)  for “6” substitute “52”.

(8) In regulation 11 (adjustment of dependency benefit where dependant is husband or wife of beneficiary and is in hospital)—
\begin{enumerate}\item[]
($a$) for paragraph (1) substitute—
\begin{quotation}
“(1) Subject to paragraph (3), in a case to which regulation 9($d$)  applies, for any part of a period of continuous free in-patient treatment which the dependant receives after the 52nd week of such continuous treatment, the weekly rate of dependency benefit payable to the beneficiary shall be adjusted so that 20\% of the basic pension is payable.”; and
\end{quotation}

($b$) in paragraph (3) for “(1)($b$)” substitute “(1)”.
\end{enumerate}

(9) In regulation 17 (calculation of periods)\footnote{Regulation 17(1) and (5) was amended by S.I.\ 1987/1683.} omit—
\begin{enumerate}\item[]
($a$) paragraphs (2), (3) and (6) (persons in prescribed accommodation who become hospital in-patients); and

($b$) in paragraph (4), the words “(or is regarded under this regulation as having received)”.
\end{enumerate}

\subsection[3. Amendment of the Income Support (General) Regulations 1987]{\sloppy Amendment of the Income Support (General) Regulations 1987}

3.---(1)  The Income Support (General) Regulations 1987 shall be amended in accordance with the following paragraphs of this regulation.

(2) In regulation 2(1A)\footnote{Paragraph (1A) was inserted by S.I.\ 2001/3767.} (interpretation), in sub-paragraph ($a$)  for “six” substitute “52”.

(3) In Schedule 2 (applicable amounts), in paragraph 13A(2)\footnote{Paragraph 13A was inserted by S.I.\ 2000/2629.} (enhanced disability premium) for “six”, in both places where it occurs, substitute “52”.

(4) In Schedule 3\footnote{Schedule 3 was substituted by S.I.\ 1995/1613.} (housing costs), in paragraph 18(7)($g$)  (no deductions in respect of non-dependants) for “six”, in both places where it occurs, substitute “52”.

(5) In Schedule 7 (applicable amounts in special cases)—
\begin{enumerate}\item[]
($a$) in paragraph 1\footnote{Paragraph 1 was amended by S.I.\ 2003/526.} (patients)—
\begin{enumerate}\item[]
(i) for “3 and 18” substitute “and 3”;

(ii) for “six” substitute “52”; and

(iii) omit sub-paragraph ($a$)  in columns (1) and (2);
\end{enumerate}

($b$) in paragraph 13 (persons in residential accommodation)—
\begin{enumerate}\item[]
(i) in sub-paragraph (1), in column (1) omit “Subject to sub-paragraph (2),”; and

(ii) omit sub-paragraph (2) in columns (1) and (2); and
\end{enumerate}

($c$) in paragraph 13B\footnote{Paragraph 13B was inserted by S.I.\ 1994/2139.} (Polish resettlement: persons temporarily absent from accommodation), in sub-paragraph ($a$)  in column (2)—
\begin{enumerate}\item[]
(i) in head (i)  for “6” substitute “52”; and

(ii) omit head (ii).
\end{enumerate}
\end{enumerate}

\subsection[4. Amendment of the Housing Benefit (General) Regulations 1987]{\sloppy Amendment of the Housing Benefit (General) Regulations 1987}

4.---(1)  The Housing Benefit (General) Regulations 1987\footnote{S.I.\ 1987/1971.} shall be amended in accordance with the following paragraphs of this regulation.

(2) In regulation 16\footnote{Regulation 16 was modified by S.I.\ 2003/325.} (applicable amounts), as modified in respect of any person who has attained the qualifying age for state pension credit—
\begin{enumerate}\item[]
($a$) in paragraph (2) omit “13 weeks but not exceeding 52 weeks, or, in exceptional circumstances, is unlikely to substantially exceed”;

($b$) in paragraph (3)($a$)—
\begin{enumerate}\item[]
(i) for “determined in accordance with paragraph (1) shall be reduced by an amount equal to 38” substitute “shall be equal to 20”; and

(ii) after “the Contributions and Benefits Act” add “(“the standard reduction”)”;
\end{enumerate}

($c$) in paragraph (3)($b$)  for “20 per cent.\ of the weekly rate of the basic pension for the time being specified in section 44(4) of the Contributions and Benefits Act (“the standard reduction”)” substitute “the standard reduction”; and

($d$) in paragraph (7) for “13” substitute “52”.
\end{enumerate}

(3) In regulation 18\footnote{Regulation 18 was amended by S.I.\ 1996/1803 and 2432 and 2003/526 and was modified by S.I.\ 2003/325.} (patients) for “6 weeks” or “six weeks”, wherever those words occur, substitute “52 weeks”.

(4) In regulation 63(7)($e$)\footnote{Regulation 63 was amended by S.I.\ 1990/546 and 1775, 1991/235 and 387, 1992/50 and 2148, 1993/317, 963 and 1249, 1996/1510 and 2518, 1998/563 and 2002/668; sub-paragraph ($e$) was added by S.I.\ 1992/50.} (no deductions in respect of non-dependants) for “six”, in both places where it occurs, substitute “52”.

(5) In Schedule 2 (applicable amounts), in paragraph 13A(2)\footnote{Paragraph 13A was inserted by S.I.\ 2000/2629.} (enhanced disability premium) for “six”, in both places where it occurs, substitute “52”.

\subsection[5. Amendment of Council Tax Benefit (General) Regulations 1992]{\sloppy Amendment of Council Tax Benefit (General) Regulations 1992}

5.---(1)  The Council Tax Benefit (General) Regulations 1992\footnote{S.I.\ 1992/1814.} shall be amended in accordance with the following paragraphs of this regulation.

(2) In regulation 8\footnote{Regulation 8 was modified by S.I.\ 2003/325.} (applicable amounts), as modified in respect of any person who has attained the qualifying age for state pension credit—
\begin{enumerate}\item[]
($a$) in paragraph (2) omit “13 weeks but not exceeding 52 weeks, or, in exceptional circumstances, is unlikely to substantially exceed”;

($b$) in paragraph (3)($a$)—
\begin{enumerate}\item[]
(i) for “determined in accordance with paragraph (1) shall be reduced by an amount equal to 38” substitute “shall be equal to 20”; and

(ii) after “the Contributions and Benefits Act” add “(“the standard reduction”)”;
\end{enumerate}

($c$) in paragraph (3)($b$)  for “20 per cent.\ of the weekly rate of the basic pension for the time being specified in section 44(4) of the Contributions and Benefits Act (“the standard reduction”)” substitute “the standard reduction”; and

($d$) in paragraph (7) for “13” substitute “52”.
\end{enumerate}

(3) In regulation 10\footnote{Regulation 10 was amended by S.I.\ 1993/688, 1996/1803 and 2432 and 2003/526 and was modified by S.I.\ 2003/325.} (patients) for “6 weeks” or “six weeks”, wherever those words occur, substitute “52 weeks”.

(4) In regulation 52(7)($d$)\footnote{Regulation 52 was amended by S.I.\ 1993/688, 963 and 1249, 1996/1510 and 2518, 2001/1605 and 2003/526.} (no deductions in respect of non-dependants) for “six”, in both places where it occurs, substitute “52”.

(5) In Schedule 1 (applicable amounts), in paragraph 14A(2)\footnote{Paragraph 14A was inserted by S.I.\ 2000/2629.} (enhanced disability premium) for “six”, in both places where it occurs, substitute “52”.

\subsection[6. Amendment of the Jobseeker’s Allowance Regulations 1996]{Amendment of the Jobseeker’s Allowance Regulations 1996}

6.---(1)  The Jobseeker’s Allowance Regulations 1996 shall be amended in accordance with the following paragraphs of this regulation.

(2) In regulation 1(3F)\footnote{Paragraph (3F) was inserted by S.I.\ 2001/3767.} (interpretation), in sub-paragraph ($a$)  for “six” substitute “52”.

(3) In Schedule 1 (applicable amounts)—
\begin{enumerate}\item[]
($a$) in paragraph 15A(2)\footnote{Paragraph 15A was inserted by S.I.\ 2000/2629.} (enhanced disability premium), for “six”, in both places where it occurs, substitute “52”; and

($b$) in paragraph 20IA(2)\footnote{Paragraph 20IA was inserted by S.I.\ 2000/2629.} (enhanced disability premium) for “six” substitute “52”.
\end{enumerate}

(4) In Schedule 2 (housing costs), in paragraph 17(7)($g$)\footnote{Regulation 17 was amended by S.I.\ 1996/2518 and 2538, 1997/827, 1999/2860, 2000/440 and 2003/526.} (no deductions in respect of non-dependants) for “six”, in both places where it occurs, substitute “52”.

(5) In Schedule 5 (applicable amounts in special cases)—
\begin{enumerate}\item[]
($a$) in paragraph 1 (person other than claimant who is a patient), in column (1) for “6” substitute “52”; and

($b$) in paragraph 15 (persons in residential accommodation)—
\begin{enumerate}\item[]
(i) in sub-paragraph (1), in column (1) omit “Subject to sub-paragraph (2),”; and

(ii) omit sub-paragraph (2) in columns (1) and (2).
\end{enumerate}
\end{enumerate}

(6) In Schedule 5A\footnote{Schedule 5A was inserted by S.I.\ 2000/1978.} (applicable amounts of joint-claim couples in special cases), in paragraph 1 for “six”, in both places where it occurs in column (1), substitute “52”.

\subsection[7. Amendment of the Child Support (Maintenance Calculations and Special Cases) Regulations 2000]{Amendment of the Child Support (Maintenance Calculations and Special Cases) Regulations 2000}

7.  In the Child Support (Maintenance Calculations and Special Cases) Regulations 2000\footnote{S.I.\ 2001/155; these regulations come into force in relation to a particular case on the day on which Part I of Schedule 1 to the Child Support Act 1991 as amended by the Child Support, Pensions and Social Security Act 2000 comes into force in relation to that type of case.}, in regulation 5 (nil rate)—
\begin{enumerate}\item[]
($a$) in paragraph ($g$)—
\begin{enumerate}\item[]
(i) for “1($a$)  or ($b$)” substitute “1($b$)  or 2”; and

(ii) for “6” substitute “52”;
\end{enumerate}

($b$) in paragraph ($gg$)\footnote{Paragraph ($gg$)  was inserted by S.I.\ 2002/3019.}—
\begin{enumerate}\item[]
(i) after “Regulations” insert “2002”; and

(ii) for “at least 13 but not exceeding” substitute “more than”; and
\end{enumerate}

($c$) in paragraph ($h$)  for “4($d$)” substitute “4”.
\end{enumerate}

\subsection[8. Amendment of the State Pension Credit Regulations 2002]{Amendment of the State Pension Credit Regulations 2002}

8.---(1)  The State Pension Credit Regulations 2002\footnote{S.I.\ 2002/1792.} shall be amended in accordance with the following paragraphs of this regulation.

(2) In Schedule II (housing costs), in paragraph 14(7)($e$)  (no deductions in respect of non-dependants) for “13”, in both places where it occurs, substitute “52”.

(3) In Schedule III (special groups), in paragraph 2 (further provisions in the case of patients)—
\begin{enumerate}\item[]
($a$) in sub-paragraph (1) omit “13 weeks but not exceeding 52 weeks, or, in exceptional circumstances, is unlikely to substantially exceed”;

($b$) in sub-paragraph (2)($a$)—
\begin{enumerate}\item[]
(i) for “determined by taking the amount for the time being specified in regulation 6(1)($b$)  and reducing it by an amount equal to 38” substitute “equal to 20”; and

(ii) after the “1992 Act” add “(“the standard reduction”)”;
\end{enumerate}

($c$) in sub-paragraph (2)($b$), for “20 per cent.\ of the weekly rate of the basic pension for the time being specified in section 44(4) of the 1992 Act (“the standard reduction”)” substitute “the standard reduction”;

($d$) omit sub-paragraph (3);

($e$) in sub-paragraph (4) for “sub-paragraphs (2) and (3), the basic pension” substitute “sub-paragraph (2), the standard reduction”;

($f$) in sub-paragraph (5) omit “or (3)”; and

($g$) in sub-paragraph (6) omit “13 weeks and”.
\end{enumerate}

\subsection[9. Amendment of the State Pension Credit (Consequential, Transitional and Miscellaneous Provisions) Regulations 2002]{Amendment of the State Pension Credit (Consequential, Transitional and Miscellaneous Provisions) Regulations 2002}

9.  In the State Pension Credit (Consequential, Transitional and Miscellaneous Provisions) Regulations 2002\footnote{S.I.\ 2002/3019.}, in regulation 36\footnote{Regulation 36 was amended by S.I.\ 2002/3197.} (persons entitled to income support immediately before the appointed day) omit paragraphs (15) to (19).

\subsection[10. Amendment of the Housing Benefit and Council Tax Benefit (State Pension Credit) Regulations 2003]{Amendment of the Housing Benefit and Council Tax Benefit (State Pension Credit) Regulations 2003}

10.  In the Housing Benefit and Council Tax Benefit (State Pension Credit) Regulations 2003\footnote{S.I.\ 2003/325.} omit regulation 31. 

\bigskip

Signed 
by authority of the Secretary of State for Work and Pensions.

{\raggedleft
\emph{Malcolm Wicks}\\*Parliamentary Under-Secretary of State,\\*Department of Work and Pensions

}

%St Andrew's House, Edinburgh

%Dated
30th April 2003

\small

\part{Explanatory Note}

\renewcommand\parthead{— Explanatory Note}

\subsection*{(This note is not part of the Regulations)}

These Regulations amend the Social Security (Hospital In-Patients) Regulations 1975 in respect of specified benefits and other regulations in respect of income support, housing benefit, council tax benefit, jobseeker’s allowance, state pension credit and child support maintenance calculations.

Regulations 2(3) and (5) provide that a beneficiary’s weekly rate of specified benefits will not be adjusted after he has received 6 weeks of free in-patient treatment but only after he has received 52 weeks of such in-patient treatment. Regulations 3(5)($a$), 4(2) and (3), 5(2) and (3) and 8(3) make similar provision in respect of income support, housing benefit, council tax benefit and state pension credit respectively. Provision is made in regulation 2(8) and in other regulations, in particular regulation 6(5)($a$)  in respect of income-based jobseeker’s allowances, for the amount of dependency benefit to be adjusted after a dependant of a beneficiary has received 52 weeks of free in-patient treatment.

Regulation 2(9) stops the downrating of specified benefits immediately a person enters hospital to receive free in-patient treatment after living in prescribed residential accommodation provided and managed by a local authority. Regulations 3(5)($b$)  and 6(5)($b$)  make similar provision in respect of income support and income-based jobseeker’s allowance.

Regulations 3(2) and (5)($c$)  and 6(2) specify the period as a hospital in-patient which is to be regarded as a temporary absence from a specified care home.

Regulations 3(4), 6(4), 4(4), 5(4), and 8(2) provide that, to calculate the applicable amount for the purposes of a beneficiary’s income support or income-based jobseeker’s allowance, the maximum amount of his housing benefit or council tax benefit or the housing costs element of his state pension credit respectively, no deduction will be made for a non-dependant receiving free in-patient treatment when the treatment has continued for over 52 weeks.

Regulation 7 provides for child support maintenance to be payable at a nil rate after a non-resident parent in receipt of income support or state pension credit has received 52 weeks of free in-patient treatment.

Regulations 9 and 10 remove provisions which are rendered superfluous by the previous amendments.

These Regulations do not impose a charge on business. 

\end{document}