\documentclass[12pt,a4paper]{article}

\newcommand\regstitle{The Personal Independence Payment (Supplementary Provisions and Consequential Amendments) Regulations 2013}

\newcommand\regsnumber{2013/388}

\title{\regstitle}

\author{S.I.\ 2013 No.\ 388}

\date{Made
25th February 2013\\
Laid before Parliament
4th March 2013\\
Coming into force
in accordance with regulations 2 and 3
}

%\opt{oldrules}{\newcommand\versionyear{1993}}
%\opt{newrules}{\newcommand\versionyear{2003}}
%\opt{2012rules}{\newcommand\versionyear{2012}}

\usepackage{csa-regs}

\setlength\headheight{42.11603pt}

%\hbadness=10000

\begin{document}

\maketitle

%\enlargethispage{\baselineskip}

\noindent
The Secretary of State for Work and Pensions makes the following Regulations in exercise of the powers conferred by sections 92(1) and (2) and 94(1) of the Welfare Reform Act 2012\footnote{2012 c.~5.}.

These Regulations contain only regulations made by virtue of, or consequential upon, Part~IV of the Welfare Reform Act 2012 and are made before the end of the period of 6 months beginning with the coming into force of those provisions\footnote{See section 173(5)($b$)  of the Social Security Administration Act 1992 (“the 1992 Act”). Sections 92 and 94 of the Welfare Reform Act 2012 (“the 2012 Act”) are relevant enactments for the purpose of section 170(5) of the 1992 Act by virtue of an amendment made by paragraph 26($a$)  of Schedule 9 to the 2012 Act.}. 

{\sloppy

\tableofcontents

}

\bigskip

\setcounter{secnumdepth}{-2}

\subsection[1--3. Citation and commencement]{Citation and commencement}

1.  These Regulations may be cited as the Personal Independence Payment (Supplementary Provisions and Consequential Amendments) Regulations 2013.

\medskip

2.  Except as specified in regulation 3, they come into force on 8th April 2013.

\medskip

3.  Paragraphs 25, 26, 34, 39 and 49 of the Schedule, and regulation 8 in so far as it relates to those paragraphs, come into force on 6th May 2013.

\subsection[4--7. Extent and application]{Extent and application}

4.  Subject to regulations 5 to 7, these Regulations extend and apply to England and Wales, Scotland and Northern Ireland.

\medskip

5.  The amendments made by paragraphs 4 to 11, 13, 15 to 17, 19 and 21 to 51 of the Schedule have the same extent and application as the provisions amended by those paragraphs.

\medskip

6.  Paragraph 20 of the Schedule extends and applies to England and Wales only.

\medskip

7.  Paragraphs 3, 12, 14 and 18 of the Schedule extend to England and Wales but apply in relation to England only.

\subsection[8. Supplementary provisions and consequential amendments]{\sloppy Supplementary provisions and consequential amendments}

8.  The Schedule has effect. 

\bigskip

\pagebreak[3]

Signed 
by authority of the 
Secretary of State for~Work and~Pensions.
%I concur
%By authority of the Lord Chancellor

{\raggedleft
\emph{Esther McVey}\\*
%Secretary
%Minister
Parliamentary Under Secretary 
of State\\*Department 
for~Work and~Pensions

}

25th February 2013

\small

\part[Schedule --- Personal independence payment: supplementary provisions and consequential amendments]{Schedule\\*Personal independence payment: supplementary provisions and consequential amendments}

\section[Part I --- Supplementary provisions]{Part I\\*Supplementary provisions}

\renewcommand\parthead{--- Schedule Part I}

\subsection*{\itshape Adjustment of personal independence payment where medical expenses are paid from public funds under war pensions instruments}

1.—(1) Sub-paragraph (2) applies where a person (“$\mathcal{P}$”) is provided with relevant accommodation.

(2) Subject to paragraph 2, where there are payable in respect of $\mathcal{P}$ both a payment under article 25B or article 21 and personal independence payment which is attributable to the daily living component in accordance with section 78 of the 2012 Act, the personal independence payment, in so far as it is so attributable, is to be adjusted by deducting from it the amount of the payment under article 25B or article 21, as the case may be, and only the balance is payable.

(3) In sub-paragraph (2)—
\begin{enumerate}\item[]
“article 25B” means article 25B of the Personal Injuries (Civilians) Scheme 1983 (medical expenses)\footnote{S.I.~1983/686. Article 25B was substituted by the Personal Injuries (Civilians) Scheme 1999 (S.I.~1999/262).} and includes that article as applied by article 48B of that Scheme;

“article 21” means article 21 of the Service Pensions Order 2006 (medical expenses)\footnote{S.I.~2006/606. Paragraph (1) was amended by S.I.~2006/1455.};
\end{enumerate}
and in both this paragraph and paragraph 2 “relevant accommodation” means accommodation provided as necessary ancillary to nursing care where the medical expenses involved are wholly borne by the Secretary of State pursuant to article 25B or article 21 and “the 2012 Act” means the Welfare Reform Act 2012.

\subsection*{\itshape Exemption from paragraph 1}

2.—(1) Paragraph 1 does not apply to $\mathcal{P}$ in respect of the first 28 days of any period during which the amount of any personal independence payment attributable to the daily living component in accordance with section 78 of the 2012 Act would be liable to be adjusted by virtue of paragraph 1.

(2) For the purposes of sub-paragraph (1), two or more distinct periods separated by an interval not exceeding 28 days, or by two such intervals, are to be treated as a continuous period equal in duration to the aggregate of such distinct periods and ending on the last day of the later or last such period.

(3) For the purposes of this paragraph, a “relevant day” in relation to $\mathcal{P}$ means a day which fell not earlier than 28 days before the first day on which $\mathcal{P}$ was provided with relevant accommodation and either—
\begin{enumerate}\item[]
($a$) was a day when $\mathcal{P}$ was undergoing medical treatment in a hospital or similar institution in any of the circumstances mentioned in regulation 29 of the Social Security (Personal Independence Payment) Regulations 2013\footnote{S.I.~2013/377.}; or

($b$) was a day when $\mathcal{P}$ was, or would but for regulation 30 of those Regulations have been, prevented from receiving personal independence payment attributable to the daily living component by virtue of regulation 29(1) of those Regulations,
\end{enumerate}
and where there is in relation to $\mathcal{P}$ a relevant day, sub-paragraph (1) has effect as if for “28 days” there was substituted such lesser number of days as is produced by subtracting from 28 the number of relevant days in that case.

\subsection*{\itshape Amendment of the Council Tax (Additional Provisions for Discount Disregards) Regulations 1992}

3.  In paragraph 3($a$)  of the Schedule\footnote{Paragraph 3($a$)  was amended by S.I.~1996/637. There are other amendments no relevant to these Regulations.} to the Council Tax (Additional Provisions for Discount Disregards) Regulations 1992\footnote{S.I.~1992/552.}—
\begin{enumerate}\item[]
($a$) omit “or” at the end of paragraph (iii); and

($b$) after paragraph (iv)  insert—
\begin{quotation}
“or

(v) the standard or enhanced rate of the daily living component of personal independence payment under section 78(3) of the Welfare Reform Act 2012;”.
\end{quotation}
\end{enumerate}

\subsection*{\itshape Amendment of the Disabled Persons (Badges for Motor Vehicles) (England) Regulations 2000}

4.—(1) The Disabled Persons (Badges for Motor Vehicles) (England) Regulations 2000\footnote{S.I.~2000/682.} are amended as follows.

(2) In regulation 4 (descriptions of disabled persons), after paragraph (2)($f$)  insert—
\begin{quotation}
“($g$) receives the mobility component of personal independence payment at either the standard rate or the enhanced rate under section~79(3) of the Welfare Reform Act 2012 by virtue of obtaining a score of at least 8 points in relation to the “moving around” activity in an assessment carried out under the Social Security (Personal Independence Payment) Regulations 2013.”.
\end{quotation}

(3) In regulation 6 (fee for issue and period of issue of a badge), in paragraph~(2)($b$)\footnote{Regulation 6(2) was substituted by S.I.~2007/2531. There is an amendment which is not relevant to these Regulations.}—
\begin{enumerate}\item[]
($a$) for “or 4(2)($d$)” substitute “,~4(2)($d$)  or 4(2)($g$)”; and

($b$) in sub-paragraph (ii)  for “or the mobility supplement” substitute “,~the mobility supplement or the mobility component of personal independence payment.”.
\end{enumerate}

\section[Part II --- Consequential amendments]{Part II\\*Consequential amendments}

\renewcommand\parthead{--- Schedule Part II}

\subsection*{\itshape Amendment of the Social Security Contributions and Benefits Act 1992}

5.  In section 70 (carer’s allowance), in subsection (2), of the Social Security Contributions and Benefits Act 1992\footnote{1992 c.~4.} after “middle rate” insert “or personal independence payment by virtue of entitlement to the daily living component at the standard or enhanced rate”.

\subsection*{\itshape Amendment of the Social Security Administration Act 1992}

6.  After section 159D\footnote{Section 159D was inserted by the Welfare Reform Act 2012, Schedule 2, paragraph 23.} of the Social Security Administration Act 1992\footnote{1992 c.~5.} insert—
\begin{quotation}
\subsubsection*{“159E.  Effect of alteration of rates of personal independence payment}

(1) Subject to such exceptions and conditions as may be prescribed, subsection (2) or (3) shall have effect where—
\begin{enumerate}\item[]
($a$) an award of personal independence payment is in force in favour of any person (“the recipient”); and

($b$) an alteration in the rate of any component of personal independence payment affects the amount of personal independence payment to which he is entitled.
\end{enumerate}

(2) Where, as a result of the alteration, the amount of personal independence payment to which the recipient is entitled is increased or reduced, then, as from the commencing date, the amount of personal independence payment in the case of the recipient under the award shall be the increased or reduced amount, without any further decision of the Secretary of State; and the award shall have effect accordingly.

(3) Where, notwithstanding the alteration, the recipient continues on and after the commencing date to be entitled to the same amount by way of personal independence payment as before, the award shall continue in force accordingly.

(4) Subsection (5) applies where a statement is made in the House of Commons by or on behalf of the Secretary of State which specifies—
\begin{enumerate}\item[]
($a$) the amount of the alteration in the rate of any component of personal independence payment which he proposes to make by an order under section 150 or 152 or by or under any other enactment, and

($b$) the date on which he proposes to bring the alteration in force (“the proposed commencing date”).
\end{enumerate}

(5) If, in a case where this subsection applies, an award of personal independence payment is made in favour of a person before the proposed commencing date and after the date on which the statement is made, the award—
\begin{enumerate}\item[]
($a$) may provide for personal independence payment to be paid as from the proposed commencing date by reference to the rates of the component of personal independence payment which will be in force on that date, or

($b$) may be expressed in terms of the rates of those components in force at the date of the award.
\end{enumerate}

(6) In this section—
\begin{enumerate}\item[]
    “alteration” means alteration by or under any enactment;

    “the commencing date”, in relation to an alteration, means the date on which the alteration comes into force in relation to the recipient;

    “component”, in relation to personal independence payment, means the daily living component or mobility component (see sections 78 and 79 of the Welfare Reform Act 2012).”. 
\end{enumerate}
\end{quotation}

\subsection*{\itshape\sloppy Amendment of the Local Government Finance Act 1992}

7.—(1) The Local Government Finance Act 1992\footnote{1992 c.~14. Regulation 2(1A) was inserted by S.I.~1994/268. There are amendments which are not relevant to these Regulations.} is amended as follows.

(2) In section 13 (reduced amounts), in subsection (10)—
\begin{enumerate}\item[]
($a$) the text from “an order” to the end becomes paragraph ($a$); and

($b$) after that paragraph insert—
\begin{quotation}
“; or

($b$) regulations made, or falling to be made, under Part~IV of the Welfare Reform Act 2012.”.
\end{quotation}
\end{enumerate}

(3) In section 80 (reduced amounts), in subsection (10)—
\begin{enumerate}\item[]
($a$) the text from “an order” to the end becomes paragraph ($a$); and

($b$) after that paragraph insert—
\begin{quotation}
“; or

($b$) regulations made, or falling to be made, under Part~IV of the Welfare Reform Act 2012.”.
\end{quotation}
\end{enumerate}

(4) In paragraph 15 of Schedule 2 (administration: supply of information to authorities), in sub-paragraph (2)($a$), after “the Social Security Acts” insert “or Part~IV of the Welfare Reform Act 2012”.

\subsection*{\itshape Amendment of the Social Security Benefit (Persons Abroad) Regulations 1975}

8.  In regulation 2 (modification of the Act in relation to incapacity benefit, severe disablement allowance, unemployability supplement and maternity allowance), in paragraph (1A), of the Social Security Benefit (Persons Abroad) Regulations 1975\footnote{S.I.~1975/563.} for “or disability living allowance” substitute “,~disability living allowance or personal independence payment under Part~IV of the Welfare Reform Act 2012”.

\subsection*{\itshape Amendment of the Social Security (Invalid Care Allowance) Regulations 1976}

9.  In regulation 9 (conditions relating to residence and presence in Great Britain), for paragraph (2)($b$), of the Social Security (Invalid Care Allowance) Regulations 1976\footnote{S.I.~1976/409. Regulation 9(2)($b$)  was amended by S.I.~1991/2742 and 1996/2744.} substitute—
\begin{quotation}
“($b$) if his absence is temporary and for the specific purpose of caring for the severely disabled person who is also absent from Great Britain and where any of the following is payable in respect of that disabled person for that day—
\begin{enumerate}\item[]
(i) attendance allowance;

(ii) the care component of disability living allowance at the highest or middle rate prescribed in accordance with section 72(3) of the Contributions and Benefits Act;

(iii) the daily living component of personal independence payment at the standard or enhanced rate prescribed in accordance with section 78(3) of the Welfare Reform Act 2012; or

(iv) a payment specified in regulation 3(1) of these Regulations.”.
\end{enumerate}
\end{quotation}

\subsection*{\itshape Amendment of the Social Security (Overlapping Benefits) Regulations 1979}

10.—(1) The Social Security (Overlapping Benefits) Regulations 1979\footnote{S.I.~1979/597.} are amended as follows.

(2) In regulation 2 (interpretation), in paragraph (1)\footnote{There are amendments to regulation 2(1) which are not relevant to these Regulations.}, in the appropriate places insert—
\begin{quotation}
““the 2012 Act” means the Welfare Reform Act 2012;”;

““the daily living component of personal independence payment” means a payment in accordance with section 78 of the 2012 Act;”; and

““personal independence payment” means personal independence payment under Part~IV of the 2012 Act;”.
\end{quotation}

(3) In regulation 6 (adjustments of personal benefit under Chapters I and II of Part II of the Act by reference to industrial injuries benefits and benefits not under the Act, and adjustment of industrial injuries benefits), for paragraph (3)\footnote{Regulation 6(3) was amended by S.I.~1991/2742.} substitute—
\begin{quotation}
“(3) Paragraph (1) and Schedule 1 have effect in relation to—
\begin{enumerate}\item[]
($a$) the following allowances and payments—
\begin{enumerate}\item[]
(i) an attendance allowance;

(ii) the care component of disability living allowance; and

(iii) the daily living component of personal independence payment; and
\end{enumerate}

($b$) any benefit by reference to which an allowance or payment under paragraph ($a$)  above is to be adjusted;
\end{enumerate}
as requiring adjustment where both that allowance or payment and the benefit are payable in respect of the same person (whether or not one or both of them are payable to that person).”.
\end{quotation}

(4) In regulation 16 (persons to be treated as entitled to benefit for certain purposes)\footnote{Regulation 16 was amended by S.I.~1996/1345 and 2008/1554.}—
\begin{enumerate}\item[]
($a$) after “Part~I of the Welfare Reform Act” where it appears for the first time insert “,~under Part~IV of the 2012 Act”; and

($b$) after “Part~I of the Welfare Reform Act and regulations made under it” insert “,~Part~IV of the 2012 Act and regulations made under it”.
\end{enumerate}

(5) In regulation 17 (prevention of double adjustments) for “or Part~I of the Welfare Reform Act” substitute “,~Part~I of the Welfare Reform Act or Part~IV of the 2012 Act”.

(6) In Schedule 1 (personal benefits which are required to be adjusted by reference to benefits not under Chapters I and II of Part II of the Act)—
\begin{enumerate}\item[]
($a$) in column (1) for “Personal benefit under the Act” substitute “Personal benefit”;

($b$) in column (1), in paragraph 5\footnote{Paragraph 5 of column (1) of Schedule 1 was amended by S.I.~1991/2742.}, for “or the care component of disability living allowance” substitute “,~the care component of disability living allowance or the daily living component of personal independence payment”.
\end{enumerate}

\subsection*{\itshape Amendment of the Income Support (General) Regulations 1987}

11.—(1) The Income Support (General) Regulations 1987\footnote{S.I.~1987/1967.} are amended as follows.

(2) In regulation 2 (interpretation), in paragraph (1)\footnote{The definition of “the benefits Acts” was inserted by S.I.~1996/206 and amended by S.I.~2008/1554. There are other amendments to regulation 2(1) which are not relevant to these Regulations.}—
\begin{enumerate}\item[]
($a$) in the appropriate places insert—
\begin{quotation}
““the 2012 Act” means the Welfare Reform Act 2012;”;

““personal independence payment” means personal independence payment under Part~IV of the 2012 Act;”; and
\end{quotation}

($b$) in the definition of “the benefits Acts” for “and Part~I of the Welfare Reform Act” substitute “,~Part~I of the Welfare Reform Act and Part~IV of the 2012 Act”.
\end{enumerate}

\begin{sloppypar}
(3) In regulation 4 (temporary absence from Great Britain), in paragraph~(2)($c$)(v)($aa$)\footnote{Regulation 4(2)($c$)(v)  was inserted by S.I.~1988/663. There are amendments which are not relevant to these Regulations.}, after “disability living allowance” insert “or the enhanced rate of the daily living component of personal independence payment”.
\end{sloppypar}

(4) In paragraph 4 of Schedule 1B (prescribed categories of person: persons caring for another person), in sub-paragraph ($a$)\footnote{Schedule 1B was inserted by S.I.~1996/206.}—
\begin{enumerate}\item[]
($a$) in paragraph (i)\footnote{There are amendments to paragraph 4($a$)(i)  which are not relevant to these Regulations.} for “or the care component of disability living allowance at the highest or middle rate prescribed in accordance with section 72(3) of the Contributions and Benefits Act” substitute “,~the care component of disability living allowance at the highest or middle rate prescribed in accordance with section 72(3) of the Contributions and Benefits Act or the daily living component of personal independence payment at the standard or enhanced rate in accordance with section 78(3) of the 2012 Act”;

($b$) in paragraph (iii)  after “disability living allowance” insert “or personal independence payment”; and

($c$) after paragraph (iiia) \footnote{Paragraph 4($a$)(iiia)  was amended by S.I.~1996/1517.} insert—
\begin{quotation}
“; or

(iv) the person being cared for has claimed entitlement to the daily living component of personal independence payment in accordance with regulation 33 of the Universal Credit, Personal Independence Payment, Jobseeker’s Allowance and Employment and Support Allowance (Claims and Payments) Regulations 2013 (advance claim for and award of personal independence payment)\footnote{S.I.~2013/380.}, an award at the standard or enhanced rate has been made in respect of that claim and, where the period for which the award is payable has begun, that person is in receipt of the payment;”.
\end{quotation}
\end{enumerate}

(5) In Schedule 2 (applicable amounts)—
\begin{enumerate}\item[]
($a$) in paragraph 7 (premiums), in sub-paragraph (2)\footnote{Paragraph 7(2) was inserted by S.I.~1990/1776 and amended by S.I.~1991/2742 and 2002/2497.}—
\begin{enumerate}\item[]
(i) omit “or” after “attendance allowance,”; and

(ii) after “Social Security Act” insert “or the daily living component of personal independence payment at the standard or enhanced rate in accordance with section 78(3) of the 2012 Act”;
\end{enumerate}

($b$) in paragraph 12 (additional condition for the higher pensioner and disability premiums)—
\begin{enumerate}\item[]
(i) in sub-paragraph (1)($a$)(i)\footnote{Paragraph 12(1)($a$)(i)  was amended by S.I.~1991/2742. There are other amendments which are not relevant to these Regulations.} after “disability living allowance,” insert “personal independence payment,”; and

(ii) in sub-paragraph (1)($d$)\footnote{Paragraph 12(1)($d$)  was substituted by S.I.~2002/3019 and amended by S.I.~2004/1141. There are other amendments which are not relevant to these Regulations.}—
\begin{enumerate}\item[]
($aa$) for “or disability living allowance” substitute “,~disability living allowance or personal independence payment”; and

($bb$) in paragraph (i)  for “or the Social Security (Disability Living Allowance) Regulations 1991” substitute “,~the Social Security (Disability Living Allowance) Regulations 1991 or regulations made under section 86(1) (hospital in-patients) of the 2012 Act”;
\end{enumerate}
\end{enumerate}

($c$) in paragraph 13 (severe disability premium)—
\begin{enumerate}\item[]
(i) in sub-paragraph (2)($a$)(i)\footnote{Paragraph 13(2)($a$)(i)  was amended by S.I.~1991/2742.}—
\begin{enumerate}\item[]
($aa$) omit “or” after “attendance allowance,”; and

($bb$) after “Social Security Act” insert “or the daily living component of personal independence payment at the standard or enhanced rate in accordance with section 78(3) of the 2012 Act”;
\end{enumerate}

(ii) in sub-paragraph (2)($b$)(i)\footnote{Paragraph 13(2)($b$)(i)  was amended by S.I.~1991/2742.}—
\begin{enumerate}\item[]
($aa$) omit “or” after “attendance allowance,”; and

($bb$) after “Social Security Act” insert “or the daily living component of personal independence payment at the standard or enhanced rate in accordance with section 78(3) of the 2012 Act”;
\end{enumerate}

(iii) in sub-paragraph (3)($a$)\footnote{Paragraph 13(3)($a$)  was amended by S.I.~1991/2742.}—
\begin{enumerate}\item[]
($aa$) omit “or” after “attendance allowance,”; and

($bb$) after “Social Security Act” insert “or the daily living component of personal independence payment at the standard or enhanced rate in accordance with section 78(3) of the 2012 Act”;
\end{enumerate}

(iv) after sub-paragraph (3A)($b$)  insert—
\begin{quotation}
“($c$) as being in receipt of the daily living component of personal independence payment at the standard or enhanced rate in accordance with section 78(3) of the 2012 Act if he would, but for a suspension of benefit in accordance with regulations under section 86(1) (hospital in-patients) of the 2012 Act, be so in receipt.”; and
\end{quotation}

\begin{sloppypar}
(v) in paragraph 13A (enhanced disability premium)\footnote{Paragraph 13A was inserted by S.I.~2000/2629. There are amendments which are not relevant to these Regulations.}, for sub-paragraph~(1) substitute—
\end{sloppypar}
\begin{quotation}
“(1) Subject to sub-paragraph (2), the condition is that—
\begin{enumerate}\item[]
($a$) the claimant; or

($b$) the claimant’s partner (if any) who is aged less than~60,
\end{enumerate}
is a person to whom sub-paragraph (1ZA) applies.

(1ZA) This sub-paragraph applies to the person mentioned in sub-paragraph (1) where—
\begin{enumerate}\item[]
($a$) the care component of disability living allowance is, or would, but for a suspension of benefit in accordance with regulations under section 113(2) of the Contributions and Benefits Act or but for an abatement as a consequence of hospitalisation, be payable to that person at the highest rate prescribed under section 72(3) of that Act; or

($b$) the daily living component of personal independence payment is, or would, but for regulations made under section 86(1) (hospital in-patients) of the 2012 Act, be payable to that person at the enhanced rate in accordance with section 78(2) of that Act.”; and
\end{enumerate}
\end{quotation}
\end{enumerate}

($d$) in paragraph 14 (disabled child premium)\footnote{Paragraph 14 was substituted by S.I.~2007/719 and amended by S.I 2011/674.}, after sub-paragraph (1)($c$)  insert—
\begin{quotation}
“; or

($d$) a young person who is in receipt of personal independence payment or who would, but for regulations made under section~86(1) (hospital in-patients) of the 2012 Act, be so in receipt provided that the young person continues to be a member of the family”.
\end{quotation}
\end{enumerate}

(6) In paragraph 18 of Schedule 3 (housing costs: non-dependant deductions)\footnote{Schedule 3 was substituted by S.I.~1995/1613. There are amendments to paragraph 18 which are not relevant to these Regulations.}—
\begin{enumerate}\item[]
(i) omit “or” at the end of sub-paragraph (6)($b$)(i);

(ii) after sub-paragraph (6)($b$)(ii)  insert—
\begin{quotation}
“; or

(iii) the daily living component of personal independence payment”; and
\end{quotation}

(iii) in sub-paragraph (8)($a$), for “or disability living allowance” substitute “,~disability living allowance or personal independence payment”.
\end{enumerate}

(7) In Schedule 9 (sums to be disregarded in the calculation of income other than earnings)—
\begin{enumerate}\item[]
($a$) in paragraph 6\footnote{Paragraph 6 was amended by S.I.~1991/2742 and 2008/3157.} after “allowance” insert “or the mobility component of personal independence payment”; and

($b$) in paragraph 9\footnote{Paragraph 9 was substituted by S.I.~1993/518 and amended by S.I.~2003/1121.} for “or the care component of disability living allowance” substitute “,~the care component of disability living allowance or the daily living component of personal independence payment”.
\end{enumerate}

\subsection*{\itshape Amendment of the Council Tax (Discount Disregards) Order 1992}

12.  In article 3 (the severely mentally impaired) of the Council Tax (Discount Disregards) Order 1992\footnote{S.I.~1992/548.}, after paragraph (2)($k$)\footnote{Paragraph (2)($k$)  was inserted by S.I.~1996/636.} insert—
\begin{quotation}
“($l$) the standard or enhanced rate of the daily living component of personal independence payment under section 78(3) of the Welfare Reform Act 2012.”.
\end{quotation}

\subsection*{\itshape Amendment of the Child Support (Maintenance Assessments and Special Cases) Regulations 1992}

13.—(1) The Child Support (Maintenance Assessments and Special Cases) Regulations 1992\footnote{S.I.~1992/1815.} are amended as follows.

(2) In Schedule 2 (amounts to be disregarded when calculating or estimating $N$ and $M$)—
\begin{enumerate}\item[]
($a$) for paragraph 8 substitute—
\begin{quotation}
“8.  Any disability living allowance, personal independence payment, mobility supplement or any payment intended to compensate for the non-payment of any such allowance, payment or supplement.”; and
\end{quotation}

($b$) in paragraph 15\footnote{There are amendments to paragraph 15 which are not relevant to these Regulations.} after “disability living allowance,” insert “personal independence payment,”.
\end{enumerate}

(3) In Schedule 4 (cases where child support maintenance is not to be payable)—
\begin{enumerate}\item[]
($a$) omit “and” at the end of paragraph ($b$)(x)\footnote{Paragraph ($b$)(x) was inserted by S.I.~2005/785.}; and

($b$) after paragraph ($c$)  insert—
\begin{quotation}
“; and

($d$) personal independence payment in accordance with Part~IV of the Welfare Reform Act 2012”.
\end{quotation}
\end{enumerate}

\subsection*{\itshape Amendment of the National Assistance (Assessment of Resources) Regulations 1992}

14.—(1) The National Assistance (Assessment of Resources) Regulations 1992\footnote{S.I.~1992/2977.} are amended as follows.

(2) In regulation 2 (interpretation), in paragraph (1)\footnote{The definition of “permanent resident” in relation to England was inserted by S.I.~2001/1066. There are other amendments to regulation 2(1) which are not relevant to these Regulations.}, after the definition of “permanent resident” insert—
\begin{quotation}
““personal independence payment” means personal independence payment under Part~IV of the Welfare Reform Act 2012;”.
\end{quotation}

(3) In Schedule 3 (sums to be disregarded in the calculation of income other than earnings)—
\begin{enumerate}\item[]
($a$) in paragraph 4 after “disability living allowance” insert “or the mobility component of personal independence payment”; and

($b$) in paragraph 6—
\begin{enumerate}\item[]
(i) omit “or” at the end of paragraph ($a$); and

(ii) after paragraph ($b$)  insert—
\begin{quotation}
“; or

($c$) the daily living component of any personal independence payment”.
\end{quotation}
\end{enumerate}
\end{enumerate}

\subsection*{\itshape Amendment of the Social Security (Incapacity Benefit) Regulations 1994}

15.  In regulation 26 (person whose benefit is not to be reduced under section~30DD(1))\footnote{Regulation 26 was inserted by S.I.~2000/3120.} of the Social Security (Incapacity Benefit) Regulations 1994\footnote{S.I.~1994/2946.} after “Contributions and Benefits Act” where it appears the second time insert “or the enhanced rate of the daily living component of personal independence payment under section 78(2) of the Welfare Reform Act 2012”.

\subsection*{\itshape Amendment of the Jobseeker’s Allowance Regulations 1996}

16.—(1) The Jobseeker’s Allowance Regulations 1996\footnote{S.I.~1996/207.} are amended as follows.

(2) In regulation 1 (citation, commencement and interpretation), in paragraph~(3)\footnote{The definition of the “the benefits Acts” was inserted by S.I.~2008/3157. There are other amendments to regulation 1(3) which are not relevant to these Regulations.}—
\begin{enumerate}\item[]
($a$) in the appropriate place insert—
\begin{quotation}
““the 2012 Act” means the Welfare Reform Act 2012;”;

““personal independence payment” means personal independence payment under Part~IV of the 2012 Act;”;

““the Universal Credit etc.\ Claims and Payments Regulations” means the Universal Credit, Personal Independence Payment, Jobseeker’s Allowance and Employment and Support Allowance (Claims and Payments) Regulations 2013;”; and
\end{quotation}

($b$) in the definition of “the benefits Acts” for “and Part~I of the Welfare Reform Act 2007” substitute “,~Part~I of the Welfare Reform Act 2007 and Part~IV of the 2012 Act”.
\end{enumerate}

(3) In regulation 51 (remunerative work), in paragraph (3)($c$)\footnote{Regulation 51(3)($c$)  was amended by S.I.~1996/1516 and 2003/511.}—
\begin{enumerate}\item[]
($a$) in paragraph (i)  for “or the care component of disability living allowance at the highest or middle rate” substitute “,~the care component of disability living allowance at the highest or middle rate or the daily living component of personal independence payment at the standard or enhanced rate”;

($b$) in paragraph (ii)  for “or a disability living allowance” substitute “,~disability living allowance or personal independence payment”; and

($c$) after paragraph (iv)  insert—
\begin{quotation}
“; or

(v) a person who has claimed personal independence payment and has an award of the daily living component at the standard or enhanced rate under section 78 of the 2012 Act for a period commencing after the date on which that claim was made”.
\end{quotation}
\end{enumerate}

(4) In regulation 140 (meaning of “person in hardship”), in paragraph (1)($h$)\footnote{Regulation 140(1)($h$)  was amended by S.I.~1996/1516.}—
\begin{enumerate}\item[]
($a$) in paragraph (i)  for “or the care component of disability living allowance at one of the two higher rates prescribed under section 72(4) of the Benefits Act” substitute “,~the care component of disability living allowance at one of the two higher rates prescribed under section 72(4) of the Benefits Act or the daily living component of personal independence payment at the standard or enhanced rate in accordance with section 78 of the 2012 Act”;

($b$) in paragraph (ii)  for “or disability living allowance” substitute “,~disability living allowance or personal independence payment”; and

($c$) after paragraph (iii)  insert—
\begin{quotation}
“or

(iv) has claimed personal independence payment and has an award of the daily living component of personal independence payment at the standard or enhanced rate in accordance with section 78 of the 2012 Act for a period commencing after the date on which that claim was made,”.
\end{quotation}
\end{enumerate}

(5) In regulation 146A (meaning of “couple in hardship”)\footnote{Regulation 146A was inserted by S.I.~2000/1978. There are amendments to regulation 146A which are not relevant to these Regulations.}, in paragraph (1)($e$)—
\begin{enumerate}\item[]
($a$) in paragraph (i)  for “or the care component of disability living allowance at one of the two higher rates prescribed under section 72(4) of the Benefits Act” substitute “,~the care component of disability living allowance at one of the two higher rates prescribed under section 72(4) of the Benefits Act or the daily living component of personal independence payment at the standard or enhanced rate in accordance with section 78 of the 2012 Act”;

($b$) in paragraph (ii)  for “or disability living allowance” substitute “,~disability living allowance or personal independence payment”; and

($c$) after paragraph (iii)  insert—
\begin{quotation}
“; or

(iv) has claimed personal independence payment and has an award of the daily living component of personal independence payment at the standard or enhanced rate in accordance with section 78 of the 2012 Act for a period commencing after the date on which that claim was made”.
\end{quotation}
\end{enumerate}

(6) In paragraph 3 of Schedule A1 (categories of members of a joint-claim couple who are not required to satisfy the conditions in section 1(2B)($b$): member caring for another person)\footnote{Schedule A1 was inserted by S.I.~2000/1978. There is an amendment to paragraph 3 which is not relevant to these Regulations.}, in sub-paragraph ($a$)—
\begin{enumerate}\item[]
($a$) in paragraph (i)  for “or the care component of disability living allowance at the highest or middle rate prescribed in accordance with section 72(3) of the Benefits Act” substitute “,~the care component of disability living allowance at the highest or middle rate prescribed in accordance with section 72(3) of the Benefits Act or the daily living component of personal independence payment at the standard or enhanced rate in accordance with section 78(3) of the 2012 Act”;

($b$) in paragraph (iv)  after “disability living allowance” insert “or personal independence payment”; and

($c$) after paragraph (v)  insert—
\begin{quotation}
“or

(vi) the person being cared for has claimed entitlement to the daily living component of personal independence payment in accordance with regulation 33 of the Universal Credit etc. Claims and Payments Regulations (advance claim for and award of personal independence payment), an award of the standard or enhanced rate of the daily living component has been made in respect of that claim and, where the period for which the award is payable has begun, that person is in receipt of that payment;”.
\end{quotation}
\end{enumerate}

(7) In Schedule 1 (applicable amounts)—
\begin{enumerate}\item[]
($a$) in paragraph 8(2)\footnote{There are amendments to paragraph 8(2) which are not relevant to these Regulations.} for “or the care component of disability living allowance at the highest or middle rate prescribed in accordance with section 72(3) of the Benefits Act” substitute “the care component of disability living allowance at the highest or middle rate prescribed in accordance with section 72(3) of the Benefits Act or the daily living component of personal independence payment at the standard or enhanced rate prescribed in accordance with section 78(3) of the 2012 Act”;

($b$) in paragraph 14 (additional conditions for higher pensioner and disability premium), in sub-paragraph (1)—
\begin{enumerate}\item[]
(i) after paragraph ($c$)  insert—
\begin{quotation}
“($ca$) the claimant or, as the case may be, his partner, is in receipt of personal independence payment or is a person whose personal independence payment is payable, in whole or in part, to another in accordance with regulation 58(2) of the Universal Credit etc.\ Claims and Payments Regulations (payment to another person on the claimant’s behalf);”;
\end{quotation}

(ii) after paragraph ($f$)  insert—
\begin{quotation}
“($fa$) the claimant or, as the case may be, his partner, is a person who is entitled to the mobility component of personal independence payment but to whom the component is not payable in accordance with regulation 61 of the Universal Credit etc.\ Claims and Payments Regulations (cases where mobility component of personal independence payment not payable);”;
\end{quotation}

(iii) omit “or” at the end of paragraph ($g$)(i); and

(iv) after paragraph ($g$)(ii)\footnote{There are amendments to paragraph 14(1)($g$)(ii)  which are not relevant to these Regulations} insert—
\begin{quotation}
“; or

(iii) entitled to personal independence payment but no amount is payable in accordance with regulations made under section 86(1) (hospital in-patients) of the 2012 Act”;
\end{quotation}
\end{enumerate}

($c$) in paragraph 15 (severe disability premium)\footnote{There are amendments to paragraph 15 which are not relevant to these Regulations.}—
\begin{enumerate}\item[]
(i) in sub-paragraph (1)($a$)  for “or the care component of disability living allowance at the highest or middle rate prescribed in accordance with section 72(3) of the Benefits Act” substitute “,~the care component of disability living allowance at the highest or middle rate prescribed in accordance with section 72(3) of the Benefits Act or the daily living component of personal independence payment at the standard or enhanced rate in accordance with section 78(3) of the 2012 Act”;

(ii) in sub-paragraph (2)($a$)  for “or the care component of disability living allowance at the highest or middle rate prescribed in accordance with section 72(3) of the Benefits Act” substitute “,~the care component of disability living allowance at the highest or middle rate prescribed in accordance with section 72(3) of the Benefits Act or the daily living component of personal independence payment at the standard or enhanced rate in accordance with section 78(3) of the 2012 Act”;

(iii) in sub-paragraph (4)($a$)  for “or the care component of disability living allowance at the highest or middle rate prescribed in accordance with section 72(3) of the Benefits Act” substitute “,~the care component of disability living allowance at the highest or middle rate prescribed in accordance with section 72(3) of the Benefits Act or the daily living component of personal independence payment at the standard or enhanced rate in accordance with section 78(3) of the 2012 Act”; and

(iv) after sub-paragraph (5)($a$)  insert—
\begin{quotation}
“($aa$) as being in receipt of the daily living component of personal independence payment at the standard or enhanced rate in accordance with section 78 of the 2012 Act if he would, but for regulations made under section 86(1) (hospital in-patients) of the 2012 Act, be so in receipt;”;
\end{quotation}
\end{enumerate}

($d$) in paragraph 15A (enhanced disability premium)\footnote{Paragraph 15A was inserted by S.I.~2000/2629. There are amendments to paragraph 15A which are not relevant to these Regulations.} for sub-paragraph (1) substitute—
\begin{quotation}
“(1) Subject to sub-paragraph (2), the condition is that—
\begin{enumerate}\item[]
($a$) the claimant; or

($b$) the claimant’s partner (if any),
\end{enumerate}
is a person who has not attained the qualifying age for state pension credit and is a person to whom sub-paragraph (1ZA) applies.

(1ZA) This sub-paragraph applies to the person mentioned in sub-paragraph (1) where—
\begin{enumerate}\item[]
($a$) the care component of disability living allowance is, or would, but for a suspension of benefit in accordance with regulations under section 113(2) of the Benefits Act or but for an abatement as a consequence of hospitalisation, be payable to that person at the highest rate prescribed under section~72(3) of the Benefits Act; or

($b$) the daily living component of personal independence payment is, or would, but for a suspension of benefits in accordance with regulations under section 86(1) (hospital in-patients) of the 2012 Act, be payable to that person at the enhanced rate in accordance with section 78(2) of the 2012 Act.”;\looseness=-1
\end{enumerate}
\end{quotation}

($e$) in paragraph 16 (disabled child premium)\footnote{Paragraph 16 was substituted by S.I.~2007/719 and paragraph (2) was amended by 2011/674. There are other amendments which are not relevant to these Regulations.}—
\begin{enumerate}\item[]
(i) after sub-paragraph (1)($a$)  insert—
\begin{quotation}
“($aa$) a young person who is in receipt of personal independence payment or who would, but for regulations made under section 86(1) (hospital in-patients) of the 2012 Act, be so in receipt, provided that the young person continues to be a member of the family;”; and
\end{quotation}

(ii) in sub-paragraph (2) after “sub-paragraph (1)($a$)” insert “,~($aa$)”;
\end{enumerate}

($f$) in paragraph 20D(2)\footnote{Paragraph 20D was inserted by S.I.~2000/1978 and paragraph (2) was amended by S.I.~2003/511. There are other amendments to paragraph 20D which are not relevant to these Regulations.} for “or the care component of disability living allowance at the highest or middle rate prescribed in accordance with section~72(3) of the Benefits Act” substitute “the care component of disability living allowance at the highest or middle rate prescribed in accordance with section 72(3) of the Benefits Act or the daily living component of personal independence payment at the standard or enhanced rate in accordance with section 78(3) of the 2012 Act”;

($g$) in paragraph 20H (additional conditions for higher pensioner and disability premium)\footnote{Paragraph 20H was inserted by S.I.~2000/1978. There are amendments to paragraph 20H which are not relevant to these Regulations.}, in sub-paragraph (1)—
\begin{enumerate}\item[]
(i) after sub-paragraph ($c$)  insert—
\begin{quotation}
“($ca$) is in receipt of personal independence payment or is a person whose personal independence payment is payable, in whole or in part, to another in accordance with regulation~58(2) of the Universal Credit etc.\ Claims and Payments Regulations (payment to another person on the claimant’s behalf);”;
\end{quotation}

(ii) after paragraph ($g$)  insert—
\begin{quotation}
“($ga$) is a person who is entitled to the mobility component of personal independence payment but to whom the component is not payable in accordance with regulation 61 of the Universal Credit etc.\ Claims and Payments Regulations (cases where mobility component of personal independence payment not payable);”; and
\end{quotation}

(iii) after paragraph ($h$)(ii)  insert—
\begin{quotation}
“or

(iii) entitled to personal independence payment but no amount is payable in accordance with regulations under section~86(1) (hospital in-patients) of the 2012 Act,”;
\end{quotation}
\end{enumerate}

($h$) in paragraph 20I (severe disability premium)\footnote{Paragraph 20I was inserted by S.I.~2000/1978. There are amendments to paragraph 20I which are not relevant to these Regulations.}—
\begin{enumerate}\item[]
(i) in sub-paragraph (1)($a$)  for “or the care component of disability living allowance at the highest or middle rate prescribed in accordance with section 72(3) of the Benefits Act” substitute “,~the care component of disability living allowance at the highest or middle rate prescribed in accordance with section 72(3) of the Benefits Act or the daily living component of personal independence payment at the standard or enhanced rate in accordance with section 78(3) of the 2012 Act”;

(ii) in sub-paragraph (3)($a$)  for “or the care component of disability living allowance at the highest or middle rate prescribed in accordance with section 72(3) of the Benefits Act” substitute “,~the care component of disability living allowance at the highest or middle rate prescribed in accordance with section 72(3) of the Benefits Act or the daily living component of personal independence payment at the standard or enhanced rate in accordance with section 78(3) of the 2012 Act”; and

(iii) after sub-paragraph (4)($b$)  insert—
\begin{quotation}
“($c$) as being in receipt of the daily living component of personal independence payment at the standard or enhanced rate in accordance with section 78 of the 2012 Act if he would, but for regulations made under section 86(1) (hospital in-patients) of the 2012 Act, be so in receipt.”; and
\end{quotation}
\end{enumerate}

($i$) in paragraph 20IA (enhanced disability premium)\footnote{Paragraph 20IA was inserted by S.I.~2000/2629. There are amendments to paragraph 20IA which are not relevant to these Regulations.}, for sub-paragraph (1) substitute—
\begin{quotation}
“(1) Subject to sub-paragraph (2), the condition is that in respect of a member of a joint-claim couple who has not attained the qualifying age for state pension credit—
\begin{enumerate}\item[]
($a$) the care component of disability living allowance is, or would, but for a suspension of benefit in accordance with regulations under section 113(2) of the Benefits Act or but for an abatement as a consequence of hospitalisation, be payable at the highest rate prescribed under section 72(3) of the Benefits Act; or

($b$) the daily living component of personal independence payment is, or would, but for regulations made under section~86(1) (hospital in-patients) of the 2012 Act, be payable at the enhanced rate in accordance with section 78(2) of the 2012 Act.”.
\end{enumerate}
\end{quotation}
\end{enumerate}

(8) In paragraph 17\footnote{There are amendments to paragraph 17 which are not relevant to these Regulations.} of Schedule 2 (housing costs: non-dependant deductions)—
\begin{enumerate}\item[]
($a$) after sub-paragraph (6)($b$)(ii)  insert—
\begin{quotation}
“,~or

(iii) the daily living component of personal independence payment”; and
\end{quotation}

($b$) in sub-paragraph (8)($a$)  for “or disability living allowance” substitute “,~disability living allowance or personal independence payment”.
\end{enumerate}

(9) In Schedule 7 (sums to be disregarded in the calculation of income other than earnings)—
\begin{enumerate}\item[]
($a$) in paragraph 7\footnote{Paragraph 7 was substituted by S.I.~2008/3157.} after “disability living allowance” insert “or the mobility component of personal independence payment”; and

($b$) in paragraph 10\footnote{Paragraph 10 was amended by S.I.~2001/3767.} for “or the care component of disability living allowance” substitute “,~the care component of disability living allowance or the daily living component of personal independence payment”.
\end{enumerate}

\subsection*{\itshape Amendment of the Social Security Benefit (Computation of Earnings) Regulations 1996}

17.  In Schedule 2 (child care charges to be deducted in the calculation of earnings) to the Social Security Benefit (Computation of Earnings) Regulations 1996\footnote{S.I.~1996/2745}—
\begin{enumerate}\item[]
($a$) in paragraph 8($b$)  after “following pensions” insert “,~payments”;

($b$) after paragraph 8($b$)(vi)  insert—
\begin{quotation}
“(vii) personal independence payment under Part~IV of the Welfare Reform Act 2012;”; and
\end{quotation}

($c$) in paragraph 8($c$)—
\begin{enumerate}\item[]
(i) after ``pension'' insert ``, payment'';

(ii) for “or (vi)” substitute “,~(vi)  or (vii)”;

(iii) the text from “in consequence of his becoming a patient” to the end becomes paragraph (i); and

(iv) after that paragraph insert—
\begin{quotation}
“; or

(ii) in accordance with regulations made section 86(1) (hospital in-patients) of the Welfare Reform Act 2012”.
\end{quotation}
\end{enumerate}
\end{enumerate}

\subsection*{\itshape Amendment of the Housing Renewal Grants Regulations 1996}

18.—(1) The Housing Renewal Grants Regulations 1996\footnote{S.I.~1996/2890.} are amended as follows.

(2) In regulation 2 (interpretation), in paragraph (1)\footnote{Regulation 2(1) was amended in relation to England by S.I.~2003\slash 2504. There are other amendments which are not relevant to these Regulations.} in the appropriate places insert—
\begin{quotation}
““the 2012 Act” means the Welfare Reform Act 2012;”; and

““personal independence payment” means personal independence payment under Part~IV of the 2012 Act;”.
\end{quotation}

(3) In regulation 19 (treatment of child care charges)—
\begin{enumerate}\item[]
($a$) after paragraph (3)($d$)(vii)\footnote{Regulation 19(3)($d$)  was amended in relation to England by S.I.~2009/1807.} insert—
\begin{quotation}
“(viii) personal independence payment;”;
\end{quotation}

($b$) omit “or” at the end of paragraph (3)($f$);

($c$) after paragraph (3)($g$)  insert—
\begin{quotation}
“; or

($h$) personal independence payment would be payable but for regulations under section 85 (care home residents) or section 86(1) (hospital in-patients) of the 2012 Act”;
\end{quotation}

($d$) omit “or” at the end of paragraph (8)($b$)(ii); and

($e$) after paragraph (8)($b$)(iii)  insert—
\begin{quotation}
“; or

(iv) in respect of whom personal independence payment is payable, or would, but for regulations made under section 85 (care home residents) or section 86(1) (hospital in-patients) of the 2012 Act, be payable”.
\end{quotation}
\end{enumerate}

(4) In Schedule 1 (applicable amounts)—
\begin{enumerate}\item[]
($a$) in paragraph 12 (additional condition for the higher pensioner and disability premiums) after sub-paragraph (1)($a$)(v)  insert—
\begin{quotation}
“(vi) is in receipt of personal independence payment or would but for regulations made under section 85 (care home residents) or section 86(1) (hospital in-patients) of the 2012 Act, be so in receipt; or”;
\end{quotation}

($b$) in paragraph 13 (severe disability premium)—
\begin{enumerate}\item[]
(i) in sub-paragraph (2)($a$)(i)  for “or the care component of disability living allowance at the highest or middle rate prescribed in accordance with section 72(3) of the 1992 Act” substitute “,~the care component of disability living allowance at the highest or middle rate prescribed in accordance with section 72(3) of the 1992 Act or the daily living component of personal independence payment at the standard or enhanced rate prescribed in accordance with section 78(3) of the 2012 Act”;

(ii) in sub-paragraph (2)($b$)(i)  for “or the care component of disability living allowance at the highest or middle rate prescribed in accordance with section 72(3) of the 1992 Act” substitute “,~the care component of disability living allowance at the highest or middle rate prescribed in accordance with section 72(3) of the 1992 Act or the daily living component of personal independence payment at the standard or enhanced rate prescribed in accordance with section 78(3) of the 2012 Act”;

(iii) in sub-paragraph (2)($b$)(ii)  after “allowance” in each place where it appears insert “or payment”;

(iv) after sub-paragraph (4)($b$)\footnote{There is an amendment to paragraph 13(4)($b$)  which is not relevant to these Regulations.} insert—
\begin{quotation}
“; or

($c$) the daily living component of personal independence payment if he would, but for regulations made under section~85 (care home residents) or section 86(1) (hospital in-patients) of the 2012 Act, be so in receipt”; and
\end{quotation}

(v) after sub-paragraph (5)($b$)  insert—
\begin{quotation}
“; or

($c$) a person receiving the daily living component of personal independence payment”;
\end{quotation}
\end{enumerate}

($c$) in paragraph 13A (enhanced disability premium)\footnote{Paragraph 13A was subsituted in relation to England by S.I.~2009/1807.} after paragraph ($b$)  insert—
\begin{quotation}
“; or

($c$) the daily living component of personal independence payment is payable, or but for regulations made under section 85 (care home residents) or section 86(1) (hospital in-patients) of the 2012 Act, would be payable, at the enhanced rate prescribed under section 78(2) of the 2012 Act”;
\end{quotation}

($d$) in paragraph 14 (disabled child premium) after sub-paragraph (1)($c$)  insert—
\begin{quotation}
“; or

($d$) is a young person who is in receipt of personal independence payment or who would, but for regulations made under section~85 (care home residents) or section 86(1) (hospital in-patients) of the 2012 Act, be so in receipt, provided that the young person continues to be a member of the family”; and
\end{quotation}

($e$) in paragraph 15 (carer premium), in sub-paragraph (2)($b$)\footnote{Paragraph 15(2) was amended in relation to England by S.I.~2003/2504.}, for “or the care component of disability living allowance at the highest or middle rate prescribed in accordance with section 72(3) of the 1992 Act” substitute “,~the care component of disability living allowance at the highest or middle rate prescribed in accordance with section 72(3) of the 1992 Act or the daily living component of personal independence payment at the standard or enhanced rate prescribed in accordance with section 78(3) of the 2012 Act”.
\end{enumerate}

(5) In Schedule 1A (applicable amounts for persons who have attained or whose partner has attained the qualifying age for state pension credit)\footnote{Schedule 1A was inserted in relation to England by S.I.~2005/3323.}—
\begin{enumerate}\item[]
($a$) in paragraph 7 (severe disability premium)—
\begin{enumerate}\item[]
(i) in sub-paragraph (2)($a$)(i)  for “or the care component of disability living allowance at the highest or middle rate prescribed in accordance with section 72(3) of the 1992 Act” substitute “,~the care component of disability living allowance at the highest or middle rate prescribed in accordance with section 72(3) of the 1992 Act or the daily living component of personal independence payment at the standard or enhanced rate prescribed in accordance with section 78(3) of the 2012 Act”;

(ii) in sub-paragraph (2)($b$)(i)  for “or the care component of disability living allowance at the highest or middle rate prescribed in accordance with section 72(3) of the 1992 Act” substitute “,~the care component of disability living allowance at the highest or middle rate prescribed in accordance with section 72(3) of the 1992 Act or the daily living component of personal independence payment prescribed in accordance with section 78(3) of the 2012 Act at the standard or enhanced rate”;

(iii) in sub-paragraph (2)($b$)(ii)  after “allowance” in each place where it appears insert “or payment”; and

(iv) after sub-paragraph (5)($b$)  insert—
\begin{quotation}
“; or

($c$) a person receiving the daily living component of personal independence payment prescribed in accordance with section 78 of the 2012 Act at the standard or enhanced rate, if he would, but for regulations made under section 85 (care home residents) or section 86(1) (hospital in-patients) of the 2012 Act, be so in receipt;”;
\end{quotation}
\end{enumerate}

($b$) for paragraph 8 (enhanced disability premium) substitute—
\begin{quotation}
“8.  The condition is that—
\begin{enumerate}\item[]
($a$) the care component of disability living allowance is, or would, but for a suspension of benefit in accordance with regulations under section 113(2) of the 1992 Act or but for an abatement as a consequence of hospitalisation, be payable at the highest rate prescribed under section 72(3) of the 1992 Act, or

($b$) the daily living component of personal independence payment is, or would, but for regulations under section 85 (care home residents) or section 86(1) (hospital in-patients) of the 2012 Act, be payable at the enhanced rate under section~78(2) of that Act,
\end{enumerate}
in respect of a child or young person who is a member of the relevant person’s family.”; and
\end{quotation}

($c$) in paragraph 9 (disabled child premium), after sub-paragraph (1)($c$)  insert—
\begin{quotation}
“; or

($d$) is a young person who is in receipt of personal independence payment or who would, but for regulations made under section~85 (care home residents) or section 86(1) (hospital in-patients) of the 2012 Act, be so in receipt, provided that the young person continues to be a member of the family”.
\end{quotation}
\end{enumerate}

(6) In paragraph 5 of Schedule 3 (sums to be disregarded in the determination of income other than earnings) after “disability living allowance” insert “or personal independence payment”.

\subsection*{\itshape Amendment of the Child Support Departure Direction and Consequential Amendments Regulations 1996}

19.  In regulation 15 (illness or disability) of the Child Support Departure Direction and Consequential Amendments Regulations 1996\footnote{These Regulations were revoked with savings by S.I.~2001/156, as amended by S.I.~2003/347.}—
\begin{enumerate}\item[]
($a$) in paragraph (3)—
\begin{enumerate}\item[]
(i) in sub-paragraph ($b$)  for “either of the allowances” substitute “a disability benefit”;

(ii) for “the allowance” substitute “the disability benefit”;
\end{enumerate}

($b$) for paragraph (4) substitute—
\begin{quotation}
“(4) Subject to paragraph (4A), where the Secretary of State considers that a person who has made an application in respect of special expenses falling within paragraph (1) or his dependant may be entitled to a disability benefit—
\begin{enumerate}\item[]
($a$) if that applicant or his dependant has at the date of that application, or within a period of six weeks beginning with the giving or sending to that person of notification of the possibility of entitlement to a disability benefit, applied for a disability benefit, the application made by that applicant shall not be determined until a decision has been made by the Secretary of State on the eligibility for that disability benefit of that applicant or that dependant;

($b$) if that applicant or his dependant has failed to apply for a disability benefit within the six week period specified in sub-paragraph ($a$), the Secretary of State shall determine the application for a departure direction made by that applicant on the basis that that applicant has income equivalent to the highest rate prescribed in respect of that disability benefit by or under those sections.”;
\end{enumerate}
\end{quotation}

($c$) for paragraph (4A)\footnote{Paragraph (4A) was inserted by S.I.~1998/58.} substitute—
\begin{quotation}
“(4A) Paragraphs (3) and (4) do not apply where the dependant of an applicant is adjudged eligible for a disability benefit and in all the circumstances of the case the Secretary of State considers that the costs being met by the applicant in respect of the items listed in paragraph (1) shall constitute special expenses for the purposes of paragraph 2(2) of Schedule 4B to the Act without the deductions in paragraph (3) being made.”; and
\end{quotation}

($d$) after paragraph (6)($c$)  insert—
\begin{quotation}
“($d$) “disability benefit” means disability living allowance under section 71 of the Contributions and Benefits Act, personal independence payment under Part~IV of the Welfare Reform Act 2012 or attendance allowance under section 64 of the Contributions and Benefits Act.”.
\end{quotation}
\end{enumerate}

\subsection*{\itshape Amendment of the Education (Student Loans) Regulations 1998}

20.  In Schedule 2 (terms of loans) of the Education (Student Loans) Regulations 1998\footnote{S.I.~1998/211.}, in the definition of “disability related benefits” in paragraph 1\footnote{There are amendments to paragraph 1 which are not relevant to these Regulations.}, after “Social Security Contributions and Benefits Act 1992,” insert “personal independence payment under Part~IV of the Welfare Reform Act 2012,”.

\subsection*{\itshape Amendment of the Social Security and Child Support (Decisions and Appeals) Regulations 1999}

21.  In Schedule 3A (date) to the Social Security and Child Support (Decisions and Appeals) Regulations 1999\footnote{S.I.~1999/991.}, after paragraph 3($h$)\footnote{Schedule 3A was inserted by S.I.~2000/1596 and paragraph 3($h$)  was inserted by S.I.~2006/2377.} insert—
\begin{quotation}
“; or

($i$) regulations under section 86(1) (hospital in-patients) of the Welfare Reform Act 2012 apply, or cease to apply, to the claimant for a period of less than one week”.
\end{quotation}

\subsection*{\itshape Amendment of the Maternity and Parental Leave etc.\ Regulations 1999}

22.—(1) The Maternity and Parental Leave etc.\ Regulations 1999\footnote{S.I.~1999/3312.} are amended as follows.

(2) In regulation 2 (interpretation), in paragraph (1)\footnote{There are amendments to Regulation 2(1) which are not relevant to these Regulations.}, in the appropriate places insert—
\begin{quotation}
““personal independence payment” means personal independence payment under Part~IV of the Welfare Reform Act 2012;”.
\end{quotation}

(3) In regulation 14 (extent of entitlement), in paragraph (1A)\footnote{Regulation 14(1A) was inserted by S.I.~2001/4010.}, after “disability living allowance” insert “or personal independence payment”.

(4) In regulation 15 (when parental leave may be taken)\footnote{Regulation 15 was substituted by S.I.~2001/4010.}, in paragraph (3), after “disability living allowance” insert “or personal independence payment”.

(5) In Schedule 2 (default provisions in respect of parental leave)—
\begin{enumerate}\item[]
($a$) in paragraph 2($c$)—
\begin{enumerate}\item[]
(i) after “disability living allowance” insert “or personal independence payment”; and

(ii) after “that allowance” insert “or payment”; and
\end{enumerate}

($b$) in paragraph 7 after “disability living allowance” insert “or personal independence payment”.
\end{enumerate}

\subsection*{\itshape\sloppyword{Amendment of the Social Security (Immigration and Asylum) Consequential Amendments Regulations 2000}}

23.—(1) The Social Security (Immigration and Asylum) Consequential Amendments Regulations 2000\footnote{S.I.~2000/636.} are amended as follows.

(2) In regulation 1 (citation, commencement and interpretation), in paragraph~(3)\footnote{There are amendments to regulation 1(3) which are not relevant to these Regulations.}, in the appropriate place insert—
\begin{quotation}
““personal independence payment” means personal independence payment under Part~IV of the Welfare Reform Act 2012;”.
\end{quotation}

(3) In regulation 2 (persons not excluded from specified benefits under section~115 of the Immigration and Asylum Act 1999)—
\begin{enumerate}\item[]
($a$) in paragraph (2)\footnote{There are amendments to regulation 2(2) which are not relevant to these Regulations.} after “Contributions and Benefits Act” insert “or personal independence payment”; and

($b$) in paragraph (3) after “Contributions and Benefits Act” insert “or personal independence payment”.
\end{enumerate}

(4) In the heading to Part II of the Schedule (persons not excluded from certain benefits under section 115 of the Immigration and Asylum Act 1999)\footnote{There are amendments to the heading to Part II of the Schedule which are not relevant to these Regulations.} after “disability living allowance,” insert “personal independence payment,”.

\subsection*{\itshape Amendment of the Child Support (Variations) Regulations 2000}

24.  In regulation 11 (special expenses---illness or disability of relevant other child) of the Child Support (Variations) Regulations 2000\footnote{S.I.~2001/156.}—
\begin{enumerate}\item[]
($a$) in paragraph (2)($a$)(i)  after “disability living allowance” insert “,~personal independence payment”;

($b$) omit “or” at the end of paragraph (2)($a$)(ii);

($c$) after paragraph (2)($a$)(iii)  insert—
\begin{quotation}
“or

(iv) he would receive personal independence payment but for regulations under section 86(1) (hospital in-patients) of the Welfare Reform Act 2012 and he remains part of the applicant’s family;”;
\end{quotation}

($d$) after the definition of “disability living allowance” in paragraph (2)($a$)  insert—
\begin{quotation}
“(iia) “personal independence payment” means an allowance payable under section 78 of the Welfare Reform Act 2012 (daily living component);”; and
\end{quotation}

($e$) in paragraph (3)($b$)\footnote{Regulation 11(3) was substituted by S.I.~2005/785.} after “disability living allowance” insert “or personal independence payment”.
\end{enumerate}

\subsection*{\itshape Amendment of the Representation of the People (England and Wales) Regulations 2001}

25.  In regulation 53 (additional requirements for applications for a proxy vote for a definite or indefinite period on grounds of blindness or any other disability)\footnote{Regulation 53(5)($b$)  was amended by S.I.~2006/2910.} of the Representation of the People (England and Wales) Regulations 2001\footnote{S.I.~2001/341.}, in paragraph~(5)($b$), after “Benefits Act 1992)” insert “or the enhanced rate of the mobility component of personal independence payment (payable under section~79(2) of the Welfare Reform Act 2012)”.

\subsection*{\itshape Amendment of the Representation of the People (Scotland) Regulations 2001}

26.  In regulation 53 (additional requirements for applications for a proxy vote for a definite or indefinite period on grounds of blindness or any other disability)\footnote{Regulation 53(5)($b$)  was amended by S.I.~2007/925.} of the Representation of the People (Scotland) Regulations 2001\footnote{S.I.~2001/497.}, in paragraph (5)($b$), after “Benefits Act 1992)” insert “or the enhanced rate of the mobility component of personal independence payment (payable under section 79(2) of the Welfare Reform Act 2012)”.

\subsection*{\itshape Amendment of the State Pension Credit Regulations 2002}

27.—(1) The State Pension Credit Regulations 2002\footnote{S.I.~2002/1792.} are amended as follows.

(2) In regulation 1 (citation, commencement and interpretation), in paragraph~(2)\footnote{There are amendments to regulation 1(2) which are not relevant to these Regulations.}, in the appropriate places insert—
\begin{quotation}
““the 2012 Act” means the Welfare Reform Act 2012;”; and

“personal independence payment” means personal independence payment under Part~IV of the 2012 Act;”.
\end{quotation}

(3) In regulation 15 (income for the purposes of the Act), after paragraph (1)($a$), insert—
\begin{quotation}
“($aa$) personal independence payment;”.
\end{quotation}

(4) In Schedule 1—
\begin{enumerate}\item[]
($a$) in paragraph 1 (severe disablement)—
\begin{enumerate}\item[]
(i) in sub-paragraph (1)($a$)(i)  for “or the care component of disability living allowance at the highest or middle rate prescribed in accordance with section 72(3) of the 1992 Act” substitute “,~the care component of disability living allowance at the highest or middle rate prescribed in accordance with section 72(3) of the 1992 Act or the daily living component of personal independence payment at the standard or enhanced rate in accordance with section 78(3) of the 2012 Act”;

(ii) in sub-paragraph (1)($b$)(i)  for “or the care component of disability living allowance at the highest or middle rate prescribed in accordance with section 72(3) of the 1992 Act” substitute “,~the care component of disability living allowance at the highest or middle rate prescribed in accordance with section 72(3) of the 1992 Act or the daily living component of personal independence payment at the standard or enhanced rate in accordance with section 78(3) of the 2012 Act”;

(iii) in sub-paragraph (1)($c$)(i)  for “or the care component of disability living allowance at the highest or middle rate prescribed in accordance with section 72(3) of the 1992 Act” substitute “,~the care component of disability living allowance at the highest or middle rate prescribed in accordance with section 72(3) of the 1992 Act or the daily living component of personal independence payment at the standard or enhanced rate in accordance with section 78(3) of the 2012 Act”;

(iv) in sub-paragraph (2)($a$)  for “or the care component of disability living allowance at the highest or middle rate prescribed in accordance with section 72(3) of the 1992 Act” substitute “,~the care component of disability living allowance at the highest or middle rate prescribed in accordance with section 72(3) of the 1992 Act or the daily living component of personal independence payment at the standard or enhanced rate in accordance with section 78(3) of the 2012 Act”;

(v) in sub-paragraph (2)($a$)(i)  after “allowance” insert “or payment”; and

(vi) after sub-paragraph (2)($b$)\footnote{There is an amendment to paragraph 1(2)($b$)  which is not relevant to these Regulations.} insert—
\begin{quotation}
“($ba$) for the purposes of sub-paragraph (1)($b$)  as being in receipt of the daily living component of personal independence payment at the standard or enhanced rate in accordance with section 78 of the 2012 Act if he would, but for regulations made under section 86(1) (hospital in-patients) of that Act, be so in receipt;” and
\end{quotation}
\end{enumerate}

($b$) in paragraph 2 (persons residing with the claimant whose presence is ignored), in sub-paragraph (2)($a$), for “or the care component of disability living allowance at the highest or middle rate prescribed in accordance with section 72(3) of the 1992 Act” substitute “,~the care component of disability living allowance at the highest or middle rate prescribed in accordance with section 72(3) of the 1992 Act or the daily living component of personal independence payment at the standard or enhanced rate in accordance with section 78(3) of the 2012 Act”.
\end{enumerate}

(5) In Schedule 2 (housing costs)—
\begin{enumerate}\item[]
($a$) in paragraph 1 (housing costs), after sub-paragraph (2)($a$)(iii)($dd$)\footnote{Paragraph 1(2)($a$)(iii)($dd$)  was inserted by S.I.~2008/1554.} insert—
\begin{quotation}
“or

($ee$) is a person in respect of whom personal independence payment is payable or would be payable but for regulations under section 86(1) (hospital in-patients) of the 2012 Act”; and
\end{quotation}

($b$) in paragraph 14 (persons residing with the claimant)—
\begin{enumerate}\item[]
(i) omit “or” at the end of sub-paragraph (6)($b$)(i);

(ii) after sub-paragraph (6)($b$)(ii)  insert—
\begin{quotation}
“; or

(iii) the daily living component of personal independence payment”; and
\end{quotation}

(iii) in sub-paragraph (8)($a$)  for “or disability living allowance” substitute “,~disability living allowance or personal independence payment”.
\end{enumerate}
\end{enumerate}

(6) In paragraph 1 of Schedule 3 (special groups: polygamous marriages), in sub-paragraph (9), for “or the care component of disability living allowance at the highest or middle rate prescribed in accordance with section 72(3) of the 1992 Act” substitute “,~the care component of disability living allowance at the highest or middle rate prescribed in accordance with section 72(3) of the 1992 Act or the daily living component of personal independence payment at the standard or enhanced rate in accordance with section 78(3) of the 2012 Act”.

(7) After paragraph 20(2)($b$)  of Schedule 5 (income from capital) insert—
\begin{quotation}
“($ba$) personal independence payment;”.
\end{quotation}

(8) In paragraph 4(1)($a$)  of Schedule 6 (sums disregarded from claimant’s earnings)
\begin{enumerate}\item[]
($a$) omit “or” at the end of paragraph (vii)\footnote{Paragraph 4(1)($a$)(vii)  was inserted by S.I.~2008/1554.}; and

($b$) after paragraph (vii)  insert—
\begin{quotation}
“(viii) personal independence payment; or”.
\end{quotation}
\end{enumerate}

\subsection*{\itshape Amendment of the Working Tax Credit (Entitlement and Maximum Rate) Regulations 2002}

28.—(1) The Working Tax Credit (Entitlement and Maximum Rate) Regulations 2002\footnote{S.I.~2002/2005.} are amended as follows.

(2) In regulation 2 (interpretation), in paragraph (1)\footnote{There are amendments to regulation 2(1) which are not relevant to these Regulations.}, in the appropriate place insert—
\begin{quotation}
““personal independence payment” means personal independence payment under Part~IV of the Welfare Reform Act 2012;”.
\end{quotation}

(3) In regulation 9 (disability element and workers who are to be treated as at a disadvantage in getting a job)\footnote{Regulation 9 was substituted by S.I.~2003/701. There are amendments to regulation 9 which are not relevant to these Regulations.}, after paragraph (4)($c$)  insert—
\begin{quotation}
“($d$) personal independence payment.”.
\end{quotation}

(4) In regulation 13 (entitlement to child care element of working tax credit), after paragraph (6)($h$)  insert—
\begin{quotation}
“($i$) personal independence payment.”.
\end{quotation}

(5) In regulation 14(4)—
\begin{enumerate}\item[]
($a$) omit “or” at the end of sub-paragraph ($b$)\footnote{There are amendments to regulation 14(4)($b$)  which are not relevant to these Regulations.}; and

($b$) after sub-paragraph ($c$)  insert—
\begin{quotation}
“; or

($d$) personal independence payment is payable in respect of that child, or would be payable but for regulations under section~86(1) (hospital in-patients) of the Welfare Reform Act 2012”.
\end{quotation}
\end{enumerate}

(6) In regulation 17 (severe disability element)—
\begin{enumerate}\item[]
($a$) in paragraph (1) after “paragraph (2)” insert “or (3)”; and

($b$) after paragraph (2) insert—
\begin{quotation}
“(3) A person satisfies this paragraph if the enhanced rate of the daily living component of personal independence payment under section 78(2) of the Welfare Reform Act 2012—
\begin{enumerate}\item[]
($a$) is payable in respect of that person; or

($b$) would be so payable but for regulations made under section~86(1) (hospital in-patients) of that Act.”.
\end{enumerate}
\end{quotation}
\end{enumerate}

\subsection*{\itshape Amendment of the Tax Credits (Definition and Calculation of Income) Regulations 2002}

29.  In regulation 7 (social security income)\footnote{There are amendments to regulation 7 which are not relevant to these Regulations.} of the Tax Credits (Definition and Calculation of Income) Regulations 2002\footnote{S.I.~2002/2006.}, in Table 3 after the final entry insert—
\begin{quotation}
“28. Personal independence payment under Part~IV of the Welfare Reform Act 2012.”.
\end{quotation}

\subsection*{\itshape Amendment of the Child Tax Credit Regulations 2002}

30.—(1) The Child Tax Credit Regulations 2002\footnote{S.I.~2002/2007.} are amended as follows.

(2) In regulation 2 (interpretation), in paragraph (1)\footnote{There are amendments to regulation 2(1) which are not relevant to these Regulations.}, in the appropriate place insert—
\begin{quotation}
““personal independence payment” means personal independence payment under Part~IV of the Welfare Reform Act 2012;”.
\end{quotation}

(3) In regulation 8 (prescribed conditions for a disabled or severely disabled child or qualifying young person)—
\begin{enumerate}\item[]
($a$) in paragraph (1)($b$)  after “paragraph (3)” insert “or (4)”;

($b$) after paragraph (2)($c$)  insert—
\begin{quotation}
“; or

($d$) personal independence payment is payable in respect of that person, or would be so payable but for regulations made under section 86(1) (hospital in-patients) of the Welfare Reform Act 2012”; and
\end{quotation}

($c$) after paragraph (3) insert—
\begin{quotation}
“(4) A person satisfies the requirements of this paragraph if the daily living component of personal independence payment—
\begin{enumerate}\item[]
($a$) is payable in respect of that person, or

($b$) would be so payable but for regulations made under section~86(1) (hospital in-patients) of the Welfare Reform Act 2012,
\end{enumerate}
at the enhanced rate under section 78(2) of that Act.”.
\end{quotation}
\end{enumerate}

\subsection*{\itshape Amendment of the Tax Credits (Claims and Notifications) Regulations 2002}

31.—(1) The Tax Credits (Claims and Notifications) Regulations 2002\footnote{S.I.~2002/2014.} are amended as follows.

(2) In regulation 2 (interpretation)\footnote{There are amendments to regulation 2 which are not relevant to these Regulations.} in the appropriate place insert—
\begin{quotation}
““personal independence payment” means personal independence payment under Part~IV of the Welfare Reform Act 2012;”.
\end{quotation}

(3) In regulation 26A (date of notification---disability element and severe disability element of child tax credit)\footnote{Regulation 26A was substituted by S.I.~2009/697. There are amendments which are not relevant to these Regulations.} after each place where “disability living allowance” occurs insert “or personal independence payment”.

\subsection*{\itshape Amendment of the Flexible Working (Eligibility, Complaints and Remedies) Regulations 2002}

32.  In regulation 2 (interpretation)\footnote{The definition of “disabled” was inserted by S.I.~2006/3314. There are other amendments to regulation 2 which are not relevant to these Regulations.} of the Flexible Working (Eligibility, Complaints and Remedies) Regulations 2002\footnote{S.I.~2002/3236.}, in the definition of “disabled” in paragraph (1), after “Benefits Act 1992” insert “or personal independence payment under Part~IV of the Welfare Reform Act 2012”.

\subsection*{\itshape\sloppy Amendment of the Government Resources and Accounts Act 2000 (Rights of Access of Comptroller and Auditor General) Order 2003}

33.  In article 2 (grant payments) of the Government Resources and Accounts Act 2000 (Rights of Access of Comptroller and Auditor General) Order 2003\footnote{S.I.~2003/1325.}, in paragraph (2)—
\begin{enumerate}\item[]
($a$) for the word “Acts” substitute “legislation”;

($b$) omit “and” at the end of sub-paragraph ($i$); and

($c$) after sub-paragraph ($j$)  insert—
\begin{quotation}
“,~and

($k$) Part~IV of the Welfare Reform Act 2012”.
\end{quotation}
\end{enumerate}

\subsection*{\itshape Amendment of the European Parliamentary Elections Regulations 2004}

34.  In paragraph 23 of Schedule 2 (absent voting: additional requirements for applications for a proxy vote for a definite or indefinite period on grounds of blindness or other disability)\footnote{Schedule 2 was substituted by S.I.~2009/186. There are amendments to paragraph 23 which are not relevant to these Regulations.} to the European Parliamentary Elections Regulations 2004\footnote{S.I.~2004/293.}—
\begin{enumerate}\item[]
($a$) omit “or” at the end of paragraph (6)($c$); and

($b$) after paragraph (6)($d$)—
\begin{quotation}
“; or

($e$) the application states that the applicant is in receipt of the enhanced rate of the mobility component of personal independence payment (payable under section 79(2) of the Welfare Reform Act 2012) because of the disability specified in the application”.
\end{quotation}
\end{enumerate}

\subsection*{\itshape Amendment of the Non-Contentious Probate Fees Order 2004}

35.  In paragraph 1 of Schedule 1A (remissions and part remissions: interpretation)\footnote{Schedule 1A was substituted by S.I.~2009/1497. There are amendments to paragraph 1 which are not relevant to these Regulations.} to the Non-Contentious Probate Fees Order 2004\footnote{S.I.~2004/3120.}, in the definition of “excluded benefits” in sub-paragraph (1)—
\begin{enumerate}\item[]
($a$) omit “and” at the end of paragraph ($f$); and

($b$) after paragraph ($g$)  insert—
\begin{quotation}
“and

($h$) personal independence payment under Part~IV of the Welfare Reform Act 2012;”.
\end{quotation}
\end{enumerate}

\subsection*{\itshape Amendment of the Housing Benefit Regulations 2006}

36.—(1) The Housing Benefit Regulations 2006\footnote{S.I.~2006/213.} are amended as follows.

(2) In regulation 2 (interpretation)\footnote{There are amendments to regulation 2 which are not relevant to these Regulations.}, in paragraph (1)—
\begin{enumerate}\item[]
($a$) in the appropriate places insert—
\begin{quotation}
““the 2012 Act” means the Welfare Reform Act 2012;”;

““personal independence payment” means personal independence payment under Part~IV of the 2012 Act;”;
\end{quotation}

($b$) in the definition of “the benefit Acts” after “the Jobseekers Act” insert “,~Part~IV of the 2012 Act”; and

($c$) in the definition of “person who requires overnight care”—
\begin{enumerate}\item[]
(i) omit “or” at the end of sub-paragraph ($a$)(ii);

(ii) after sub-paragraph ($a$)(ii)  insert—
\begin{quotation}
“(iia) is in receipt of the daily living component of personal independence payment in accordance with section 78 of the 2012 Act; or”; and
\end{quotation}

(iii) in sub-paragraph ($a$)(iii)  for “or (ii)” substitute “,~(ii)  or (iia)”.
\end{enumerate}
\end{enumerate}

(3) In regulation 28 (treatment of child care charges)—
\begin{enumerate}\item[]
($a$) after paragraph (11)($d$)(vii)\footnote{Regulation 28(11)($d$)(vii)  was inserted by S.I.~2008/1082.} insert—
\begin{quotation}
“(viii) personal independence payment;”;
\end{quotation}

($b$) omit “or” at the end of paragraph (13)($b$); and

($c$) after paragraph (13)($c$)  insert—
\begin{quotation}
“; or

($d$) in respect of whom personal independence payment is payable, or would be payable but for regulations made under section 86(1) (hospital in-patients) of the 2012 Act”.
\end{quotation}
\end{enumerate}

(4) In regulation 74 (non-dependant deductions)\footnote{Regulation 74 was substituted by S.I.~2007/2868. There are amendments which are not relevant to these Regulations.}—
\begin{enumerate}\item[]
($a$) omit “or” at the end of sub-paragraph (6)($b$)(i);

($b$) after sub-paragraph (6)($b$)(ii)  insert—
\begin{quotation}
“or

(iii) the daily living component of personal independence payment;”; and
\end{quotation}

($c$) in paragraph (9)($a$)  for “or disability living allowance” insert “,~disability living allowance or personal independence payment”.
\end{enumerate}

(5) In regulation 79 (date on which change of circumstances is to take effect), in paragraph (6), after “the Act” insert “or Part~IV of the 2012 Act”.

(6) In Schedule 3 (applicable amounts)—
\begin{enumerate}\item[]
($a$) in paragraph 7(2) for “or the care component of disability living allowance at the highest or middle rate prescribed in accordance with section 72(3) of the Act” substitute “the care component of disability living allowance at the highest or middle rate prescribed in accordance with section 72(3) of the Act or the daily living component of personal independence payment at the standard or enhanced rate in accordance with section 78(3) of the 2012 Act”;

($b$) in paragraph 13 (additional condition for the disability premium), in sub-paragraph (1)($a$)—
\begin{enumerate}\item[]
(i) in paragraph (i)  after “disability living allowance,” insert “personal independence payment,”; and

(ii) after paragraph (iii)\footnote{Paragraph 13(1)($a$)(iii)  was amended by S.I.~2005/2502.} insert—
\begin{quotation}
“(iiia) would be in receipt of personal independence payment but for regulations made under section 86(1) (hospital in-patients) of the 2012 Act; or”;
\end{quotation}
\end{enumerate}

($c$) in paragraph 14 (severe disability premium)—
\begin{enumerate}\item[]
(i) in sub-paragraph (2)($a$)(i)  for “or the care component of disability living allowance at the highest or middle rate prescribed in accordance with section 72(3) of the Act” substitute “the care component of disability living allowance at the highest or middle rate prescribed in accordance with section 72(3) of the Act or the daily living component of personal independence payment at the standard or enhanced rate in accordance with section 78(3) of the 2012 Act”;

(ii) in sub-paragraph (2)($b$)(i)  for “or the care component of disability living allowance at the highest or middle rate prescribed in accordance with section 72(3) of the Act” substitute “the care component of disability living allowance at the highest or middle rate prescribed in accordance with section 72(3) of the Act or the daily living component of personal independence payment at the standard or enhanced rate in accordance with section 78(3) of the 2012 Act”;

(iii) in sub-paragraph (2)($b$)(ii)  after “allowance” in each place where it appears insert “or payment”;

(iv) in sub-paragraph (4)($a$)  for “or the care component of disability living allowance at the highest or middle rate prescribed in accordance with section 72(3) of the Act” substitute “the care component of disability living allowance at the highest or middle rate prescribed in accordance with section 72(3) of the Act or the daily living component of personal independence payment at the standard or enhanced rate in accordance with section 78(3) of the 2012 Act”;

(v) after sub-paragraph (5)($b$)  insert—
\begin{quotation}
“($c$) as being in receipt of the daily living component of personal independence payment at the standard or enhanced rate in accordance with section 78 of the 2012 Act, if he would, but for regulations made under section 86(1) (hospital in-patients) of the 2012 Act, be so in receipt.”;
\end{quotation}
\end{enumerate}

\sloppyword{
($d$) in paragraph 15 (enhanced disability premium), after sub-paragraph~(1)($b$)\footnote{Paragraph 15(1) was substituted by S.I.~2008/1082. There is an amendment to paragraph 15(1)($b$)  which is not relevant to these Regulations.} insert---
}
\begin{quotation}
“; or

($c$) the enhanced rate of the daily living component of personal independence payment is, or would, but for regulations made under section 86(1) (hospital in-patients) of the 2012 Act, be payable in respect of—
\begin{enumerate}\item[]
(i) the claimant; or

(ii) a member of the claimant’s family,
\end{enumerate}
who has not attained the qualifying age for state pension credit”; and
\end{quotation}

($e$) in paragraph 16 (disabled child premium), after sub-paragraph ($c$)\footnote{Paragraph 16($c$)  was substituted by S.I.~2011/674.} insert—
\begin{quotation}
“; or

($d$) is a young person who is in receipt of personal independence payment or who would, but for regulations made under section~86(1) (hospital in-patients) of the 2012 Act be so in receipt, provided that the young person continues to be a member of the family”.
\end{quotation}
\end{enumerate}

(7) In paragraph 6 of Schedule 5 (sums to be disregarded in the calculation of income other than earnings) after “disability living allowance” insert “or personal independence payment”.

\subsection*{\itshape Amendment of the Housing Benefit (Persons who have attained the qualifying age for state pension credit) Regulations 2006}

37.—(1) The Housing Benefit (Persons who have attained the qualifying age for state pension credit) Regulations 2006\footnote{S.I.~2006/214.} are amended as follows.

(2) In regulation 2 (interpretation), in paragraph (1)\footnote{There are amendments to regulation 2(1) which are not relevant to these Regulations.}—
\begin{enumerate}\item[]
($a$) in the appropriate places insert—
\begin{quotation}
““the 2012 Act” means the Welfare Reform Act 2012;”;

““personal independence payment” means personal independence payment under Part~IV of the 2012 Act;”;
\end{quotation}

($b$) in the definition of “the benefit Acts” after “Welfare Reform Act” insert “,~Part~IV of the 2012 Act”;

($c$) in the definition of “person who requires overnight care”—
\begin{enumerate}\item[]
(i) omit “or” at the end of sub-paragraph ($a$)(ii);

(ii) after sub-paragraph ($a$)(ii)  insert—
\begin{quotation}
“(iia) is in receipt of the daily living component of personal independence payment in accordance with section 78 of the 2012 Act; or”; and
\end{quotation}

(iii) in sub-paragraph ($a$)(iii)  for “or (ii)” substitute “,~(ii)  or (iia)”.
\end{enumerate}
\end{enumerate}

(3) In regulation 29 (meaning of “income”) after paragraph (1)($j$)(i)\footnote{There are amendments to regulation 13(1)($j$)(i)  which are not relevant to these Regulations.} insert—
\begin{quotation}
“(ia) personal independence payment;”.
\end{quotation}

(4) In regulation 31 (treatment of child care charges)—
\begin{enumerate}\item[]
($a$) after paragraph (11)($d$)(vii)\footnote{Regulation 31(11)($d$)(vii)  was inserted by S.I.~2008/1082.} insert—
\begin{quotation}
“(viii) personal independence payment;”;
\end{quotation}

($b$) omit “or” at the end of paragraph (13)($b$); and

($c$) after paragraph (13)($c$)  insert—
\begin{quotation}
“; or

($d$) in respect of whom personal independence payment is payable, or would be payable but for regulations made under section 86(1) (hospital in-patients) of the 2012 Act”.
\end{quotation}
\end{enumerate}

(5) In regulation 55 (non-dependant deductions)\footnote{Regulation 55 was substituted by S.I.~2007/2869. There are amendments to regulation 55 which are not relevant to these Regulations.}—
\begin{enumerate}\item[]
($a$) omit “or” at the end of paragraph (6)($b$)(i);

($b$) after paragraph (6)($b$)(ii)  insert—
\begin{quotation}
“; or

(iii) the daily living component of personal independence payment”; and
\end{quotation}

($c$) in paragraph (10)($a$)  for “or disability living allowance” substitute “,~disability living allowance or personal independence payment”.
\end{enumerate}

(6) In regulation 59 (date on which change of circumstances is to take effect), in paragraph (6), after “the Act” insert “or Part~IV of the 2012 Act”.

(7) In Schedule 3 (applicable amounts)—
\begin{enumerate}\item[]
($a$) for paragraph 5(2) substitute—
\begin{quotation}
“(2) For the purposes of the carer premium under paragraph 9, a person shall be treated as being in receipt of a carer’s allowance under section 70 of the Act by virtue of sub-paragraph (1)($a$)  only if and for so long as the person in respect of whose care the allowance has been claimed remains in receipt of—
\begin{enumerate}\item[]
($a$) attendance allowance;

($b$) the care component of disability living allowance at the highest or middle rate prescribed in accordance with section~72(3) of the Act; or

($c$) the daily living component of personal independence payment at the standard or enhanced rate in accordance with section 78(3) of the 2012 Act.”;
\end{enumerate}
\end{quotation}

($b$) in paragraph 6 (severe disability premium)—
\begin{enumerate}\item[]
(i) in sub-paragraph (2)($a$)(i)  for “or the care component of disability living allowance at the highest or middle rate prescribed in accordance with section 72(3) of the Act” substitute “,~the care component of disability living allowance at the highest or middle rate prescribed in accordance with section 72(3) of the Act or the daily living component of personal independence payment at the standard or enhanced rate in accordance with section 78(3) of the 2012 Act”;

(ii) in sub-paragraph (2)($b$)(i)  for “or the care component of disability living allowance at the highest or middle rate prescribed in accordance with section 72(3) of the Act” substitute “,~the care component of disability living allowance at the highest or middle rate prescribed in accordance with section 72(3) of the Act or the daily living component of personal independence payment at the standard or enhanced rate in accordance with section 78(3) of the 2012 Act”;

(iii) in sub-paragraph (2)($b$)(ii)  after “allowance” in each place where it appears insert “or payment”;

(iv) in sub-paragraph (6)($a$)  for “or the care component of disability living allowance at the highest or middle rate prescribed in accordance with section 72(3) of the Act” substitute “,~the care component of disability living allowance at the highest or middle rate prescribed in accordance with section 72(3) of the Act or the daily living component of personal independence payment at the standard or enhanced rate in accordance with section 78(3) of the 2012 Act”;

(v) after sub-paragraph (7)($b$)  insert—
\begin{quotation}
“($c$) as being in receipt of the daily living component of personal independence payment at the standard or enhanced rate in accordance with section 78 of the 2012 Act, if he would, but for regulations made under section 86(1) (hospital in-patients) of the 2012 Act, be so in receipt.”;
\end{quotation}
\end{enumerate}

($c$) in paragraph 7 (enhanced disability premium) for sub-paragraph (1)\footnote{Paragraph 7(1) was substituted by S.I.~2011/674.} substitute—
\begin{quotation}
“(1) Subject to sub-paragraph (2), the condition is that—
\begin{enumerate}\item[]
($a$) the care component of disability living allowance is, or would, but for a suspension of benefit in accordance with regulations under section 113(2) of the Act or but for an abatement as a consequence of hospitalisation, be payable at the highest rate prescribed under section 72(3) of the Act; or

($b$) the daily living component of personal independence payment is, or would, but for regulations made under section~86(1) (hospital in-patients) of the 2012 Act, be payable at the enhanced rate under section 78(2) of the 2012 Act;
\end{enumerate}
in respect of a child or young person who is a member of the claimant’s family.”; and
\end{quotation}

($d$) in paragraph 8 (disabled child premium), after sub-paragraph ($c$)\footnote{Paragraph 8($c$)  was substituted by S.I.~2011/674.} insert—
\begin{quotation}
“; or

($d$) is a young person who is in receipt of personal independence payment or who would, but for regulations made under section~86(1) (hospital in-patients) of the 2012 Act be so in receipt, provided that the young person continues to be a member of the family”.
\end{quotation}
\end{enumerate}

(8) In paragraph 5(1)($a$)  of Schedule 4 (sums disregarded from claimant’s earnings)—
\begin{enumerate}\item[]
($a$) omit “or” at the end of paragraph (vii)\footnote{Paragraph 5(1)($a$)(vii)  was inserted by S.I.~2008/1082.}; and

($b$) after paragraph (vii)  insert—
\begin{quotation}
“(viii) personal independence payment; or”.
\end{quotation}
\end{enumerate}

(9) After paragraph 21(2)($b$)  of Schedule 6 (capital to be disregarded) insert—
\begin{quotation}
“($ba$) personal independence payment;”.
\end{quotation}

\subsection*{\itshape Amendment of the Naval, Military and Air Forces Etc.\ (Disablement and Death) Service Pensions Order 2006}

38.—(1) The Naval, Military and Air Forces Etc.\ (Disablement and Death) Service Pensions Order 2006\footnote{S.I.~2006/606.} is amended as follows.

(2) In article 20 (mobility supplement), after paragraph (1)($c$)(iii)  insert—
\begin{quotation}
“(iv) has been in receipt of the mobility component of personal independence payment at the enhanced rate under section 79(2) of the Welfare Reform Act 2012; or”.
\end{quotation}

(3) In article 56 (abatement of awards in respect of social security benefits), after paragraph (3)($g$)\footnote{There are amendments to article 56(3) which are not relevant to these Regulations.} insert—
\begin{quotation}
“($h$) Part~IV of the Welfare Reform Act 2012.”.
\end{quotation}

\subsection*{\itshape Amendment of the National Assembly for Wales (Representation of the People) Order 2007}

39.  In paragraph 4 of Schedule 1 (additional requirements for applications for a proxy vote on grounds of blindness or other disability: absent voting at assembly elections) to the National Assembly for Wales (Representation of the People) Order 2007\footnote{S.I.~2007/236}, after sub-paragraph (5)($b$)  insert—
\begin{quotation}
“; or

($c$) the application states that the applicant is in receipt of the enhanced rate of the mobility component of personal independence payment (payable under section 79(2) of the Welfare Reform Act 2012) because of the disability specified in the application”.
\end{quotation}

\subsection*{\itshape\sloppyword{Amendment of the Employment and Support Allowance Regulations 2008}}

40.—(1) The Employment and Support Allowance Regulations 2008\footnote{S.I.~2008/794.} are amended as follows.

(2) In regulation 2 (interpretation), in paragraph (1)\footnote{There are amendments to regulation 2(1) which are not relevant to these Regulations.} in the appropriate places insert—
\begin{quotation}
““the 2012 Act” means the Welfare Reform Act 2012;”;

““daily living component” means the daily living component of personal independence payment under section 78 of the 2012 Act;”;

““personal independence payment” means personal independence payment under Part~IV of the 2012 Act;”.
\end{quotation}

(3) In regulation 18 (circumstances in which the condition that the claimant is not receiving education does not apply) after “disability living allowance” insert “or personal independence payment”.

(4) In regulation 158 (meaning of “person in hardship”), in paragraph (3)—
\begin{enumerate}\item[]
($a$) in sub-paragraph ($b$)  for “or the care component” substitute “,~the care component or the daily living component”;

($b$) in sub-paragraph ($c$)  for “or disability living allowance” substitute “,~disability living allowance or personal independence payment”;

($c$) in sub-paragraph ($d$)(i)  for “or the care component” substitute “,~the care component or the daily living component”; and

($d$) in sub-paragraph ($d$)(ii)  for “or disability living allowance” substitute “,~disability living allowance or personal independence payment”.
\end{enumerate}

(5) In Schedule 4 (amounts)—
\begin{enumerate}\item[]
($a$) for paragraph 4(2) substitute—
\begin{quotation}
“(2) For the purposes of the carer premium under paragraph 8, a claimant is to be treated as being in receipt of a carer’s allowance by virtue of sub-paragraph (1)($a$)  only if and for so long as the person in respect of whose care the allowance has been claimed remains in receipt of—
\begin{enumerate}\item[]
($a$) attendance allowance;

($b$) the care component of disability living allowance at the highest or middle rate prescribed in accordance with section~72(3) of the Contributions and Benefits Act; or

($c$) the daily living component of personal independence payment at the standard or enhanced rate in accordance with section 78(3) of the 2012 Act.”;
\end{enumerate}
\end{quotation}

($b$) in paragraph 6 (severe disability premium)—
\begin{enumerate}\item[]
(i) in sub-paragraph (2)($a$)(i)\footnote{Paragraph 6(2)($a$)(i)  was amended by S.I.~2011/2425.} after “the care component” insert “,~the daily living component”;

(ii) in sub-paragraph (2)($b$)(i)\footnote{Paragraph 6(2)($b$)(i)  was amended by S.I.~2011/2425.} after “the care component” insert “,~the daily living component”;

(iii) in sub-paragraph (2)($b$)(ii)  after “care component” in each place where it appears insert “,~the daily living component”;

(iv) in sub-paragraph (4)($a$)  after “attendance allowance,” insert “the daily living component”; and

(v) after sub-paragraph (5)($b$)  insert—
\begin{quotation}
“($c$) as being in entitled to, and in receipt of, the daily living component if the person would, but for regulations under section 86(1) (hospital in-patients) of the 2012 Act, be so entitled and in receipt.”; and
\end{quotation}
\end{enumerate}

($c$) in paragraph 7 (enhanced disability premium)—
\begin{enumerate}\item[]
(i) omit “or” at the end of sub-paragraph (1)($a$); and

(ii) after sub-paragraph (1)($b$)(ii)  insert—
\begin{quotation}
“; or

($c$) the daily living component is, or would, but for regulations made under section 86(1) (hospital in-patients) of the 2012 Act, be payable at the enhanced rate under section 78(2) of that Act in respect of—
\begin{enumerate}\item[]
(i) the claimant; or

(ii) the claimant’s partner (if any) who is aged less then the qualifying age for state pension credit”.
\end{enumerate}
\end{quotation}
\end{enumerate}
\end{enumerate}

(6) In Schedule 6 (housing costs)—
\begin{enumerate}\item[]
($a$) in paragraph 15 (linking rule), in sub-paragraph (11)($b$)\footnote{Paragraph 15(11)($b$)  was substituted by S.I.~2011/2428.}, after “disability living allowance” insert “or personal independence payment”; and

($b$) in paragraph 19 (non-dependant deductions)—
\begin{enumerate}\item[]
(i) omit “or” at the end of sub-paragraph (6)($b$)(i);

(ii) after sub-paragraph (6)($b$)(ii)  insert—
\begin{quotation}
“; or

(iii) the daily living component”; and
\end{quotation}

(iii) in sub-paragraph (8)($a$)  for “or disability living allowance” substitute “,~disability living allowance or personal independence payment”.
\end{enumerate}
\end{enumerate}

(7) In Schedule 8 (sums to be disregarded in the calculation of income other than earnings)—
\begin{enumerate}\item[]
($a$) in paragraph 8 after “disability living allowance” insert “or the mobility component of personal independence payment”; and

($b$) in paragraph 11 for “or the care component of disability living allowance” substitute “,~the care component of disability living allowance or the daily living component”.
\end{enumerate}

\subsection*{\itshape\sloppy Amendment of the Magistrates’ Courts Fees Order 2008}

41.  In paragraph 1 of Schedule 2 (remissions and part-remissions: interpretation) to the Magistrates’ Courts Fees Order 2008\footnote{S.I.~2008/1052.}, in the definition of “excluded benefits” in sub-paragraph (1)\footnote{There are amendments to paragraph 1(1) which are not relevant to these Regulations.}—
\begin{enumerate}\item[]
($a$) omit “and” at the end of paragraph ($f$); and

($b$) after paragraph ($g$)  insert—
\begin{quotation}
“and

($h$) personal independence payment under Part~IV of the Welfare Reform Act 2012;”.
\end{quotation}
\end{enumerate}

\subsection*{\itshape Amendment of the Civil Proceedings Fees Order 2008}

42.  In paragraph 1 of Schedule 2 (remissions and part remissions: interpretation) to the Civil Proceedings Fees Order 2008\footnote{S.I.~2008/1053.}, in the definition of “excluded benefits” in sub-paragraph (1)\footnote{There are amendments to paragraph 1(1) which are not relevant to these Regulations.}—
\begin{enumerate}\item[]
($a$) omit “and” at the end of paragraph ($f$); and

($b$) after paragraph ($g$)  insert—
\begin{quotation}
“and

($h$) personal independence payment under Part~IV of the Welfare Reform Act 2012;”.
\end{quotation}
\end{enumerate}

\subsection*{\itshape\sloppy Amendment of the Family Proceedings Fees Order 2008}

43.  In paragraph 1 of Schedule 2 (remissions and part remissions: interpretation) to the Family Proceedings Fees Order 2008\footnote{S.I.~2008/1054.}, in the definition of “excluded benefits” in sub-paragraph (1)\footnote{There are amendments to paragraph 1(1) which are not relevant to these Regulations.}—
\begin{enumerate}\item[]
($a$) omit “and” at the end of paragraph ($f$); and

($b$) after paragraph ($g$)  insert—
\begin{quotation}
“and

($h$) personal independence payment under Part~IV of the Welfare Reform Act 2012;”.
\end{quotation}
\end{enumerate}

\subsection*{\itshape Amendment of the Education (Student Loans) (Repayment) Regulations 2009}

44.  In regulation 9 (interpretation) of the Education (Student Loans) (Repayment) Regulations 2009\footnote{S.I.~2009/470.}, in the definition of “disability-related benefit” in sub-paragraph~(1)\footnote{There are amendments to regulation 9(1) which are not relevant to these Regulations.}, after “Social Security Contributions and Benefits Act 1992,” insert “personal independence payment under Part~IV of the Welfare Reform Act 2012,”.

\subsection*{\itshape Amendment of the Supreme Court Fees Order 2009}

45.  In paragraph 1 of Schedule 2 (remissions and part remissions: interpretation) to the Supreme Court Order 2009\footnote{S.I.~2009/2131.}, in the definition of “excluded benefits” in sub-paragraph (1)—
\begin{enumerate}\item[]
($a$) omit “and” at the end of paragraph ($f$); and

($b$) after paragraph ($g$)  insert—
\begin{quotation}
“and

($h$) personal independence payment under Part~IV of the Welfare Reform Act 2012;”.
\end{quotation}
\end{enumerate}

\subsection*{\itshape\sloppyword{Amendment of the Social Security (Contributions Credits for Parents and Carers) Regulations 2010}}

46.  In regulation 2 (interpretation) of the Social Security (Contributions Credits for Parents and Carers) Regulations 2010\footnote{S.I.~2010/19.}, in the definition of “relevant benefit” after sub-paragraph (1)($e$)(ii)  insert—
\begin{quotation}
“($f$) the daily living component of personal independence payment in accordance with section 78 of the Welfare Reform Act 2012.”.
\end{quotation}

\subsection*{\itshape\sloppy Amendment to the Upper Tribunal (Immigration and Asylum Chamber) (Judicial Review) (England and Wales) Fees Order 2011}

47.  In paragraph 1 of Schedule 2 (remissions and part remissions: interpretation) to the Upper Tribunal (Immigration and Asylum Chamber) (Judicial Review) (England and Wales) Fees Order 2011\footnote{S.I.~2011/2344.}, in the definition of “excluded benefits” in sub-paragraph (1)—
\begin{enumerate}\item[]
($a$) omit “and” at the end of paragraph ($f$); and

($b$) after paragraph ($g$)  insert—
\begin{quotation}
“and

($h$) personal independence payment under Part~IV of the Welfare Reform Act 2012;”.
\end{quotation}
\end{enumerate}

\subsection*{\itshape Amendment to the Social Security (Information-\hspace{0pt}sharing in relation to Welfare Services etc.)\ Regulations 2012}

48.  In regulation 4 (prescribed benefits) of the Social Security (Information-sharing in relation to Welfare Services etc.)\ Regulations 2012\footnote{S.I.~2012/1483.}—
\begin{enumerate}\item[]
($a$) omit “and” at the end of sub-paragraph ($e$) ; and

($b$) after sub-paragraph ($f$)  insert—
\begin{quotation}
“; and

($g$) personal independence payment under Part~IV of the 2012 Act”.
\end{quotation}
\end{enumerate}

\subsection*{\itshape Amendment to the Police and Crime Commissioner Elections Order 2012}

49.  In paragraph 15 of Schedule 2 (absent voting in PCC elections: additional requirements referred to in paragraph 14(4)) to the Police and Crime Commissioner Elections Order 2012\footnote{S.I.~2012/1917.}—
\begin{enumerate}\item[]
($a$) omit “or” at the end of sub-paragraph (6)($a$); and

($b$) after sub-paragraph (6)($b$)  insert—
\begin{quotation}
“,~or

($c$) the application states that the applicant is in receipt of the enhanced rate of the mobility component of personal independence payment (payable under section 79(2) of the Welfare Reform Act 2012) because of the disability specified in the application”.
\end{quotation}
\end{enumerate}

\subsection*{\itshape Amendment to the Child Support Maintenance Calculation Regulations 2012}

50.  In regulation 64 (illness or disability of relevant other child) of the Child Support Maintenance Calculation Regulations 2012\footnote{S.I.~2012/2677.}—
\begin{enumerate}\item[]
($a$) in paragraph (2)($a$)(i)  after “disability living allowance” insert “or personal independence payment”;

($b$) omit “or” at the end of paragraph (2)($a$)(ii);

($c$) after paragraph (2)($a$)(iii)  insert—
\begin{quotation}
“or

(iv) that person would receive personal independence payment but for regulations under section 86(1) (hospital in-patients) of the Welfare Reform Act 2012, and remains part of the applicant’s family,”;
\end{quotation}

($d$) after paragraph (2)($f$)  insert—
\begin{quotation}
“($g$) “personal independence payment” means the daily living component of personal independence payment under section 78 of the Welfare Reform Act 2012.”;
\end{quotation}

($e$) in paragraph (3) after “the amount of the allowance” insert “or payment”; and

($f$) in paragraph (3)($b$)  after “disability living allowance” insert “or personal independence payment”.
\end{enumerate}

\subsection*{\itshape Amendment to the Benefit Cap (Housing Benefit) Regulations 2012}

51.  In regulation 2 (amendment of the Housing Benefit Regulations 2006) of the Benefit Cap (Housing Benefit) Regulations 2012\footnote{S.I.~2012/2994.} in paragraph (5), in regulation~75F inserted by those Regulations—
\begin{enumerate}\item[]
($a$) after sub-paragraph ($e$)  insert—
\begin{quotation}
“($ea$) the claimant, the claimant’s partner or a young person for whom the claimant or the claimant’s partner is responsible, is receiving a personal independence payment;”;
\end{quotation}

($b$) in sub-paragraph ($f$)  for “or ($e$)” substitute “,~($e$)  or ($ea$)”; and

($c$) after sub-paragraph ($f$)(iii)  insert—
\begin{quotation}
“(iv) that payment is not payable in accordance with regulations made under section 85 (care home residents) or section~86(1) (hospital in-patients) of the Welfare Reform Act 2012;”.
\end{quotation}
\end{enumerate}

\part{Explanatory Note}

\renewcommand\parthead{— Explanatory Note}

\subsection*{(This note is not part of the Regulations)}

The provisions contained in these Regulations are consequential upon, or supplementary to, provisions in Part~IV of the Welfare Reform Act 2012 (c.~5).

Paragraph 1 of the Schedule makes provision for adjusting personal independence payment where medical expenses are paid from public funds under war pensions instruments.

Paragraph 2 of the Schedule provides exemptions to paragraph 1.

Paragraph 3 of the Schedule amends the Council Tax (Additional Provision for Discount Disregards) Regulations 1992 which makes provision for care workers in relation to council tax disregards. Both the standard and enhanced rates of the daily living component of personal independence payment have been added to the list of qualifying benefits in respect of the person being cared for.

Paragraph 4 of the Schedule amends the Disabled Persons (Badges for Motor Vehicles) (England) Regulations 2000 to provide that persons who score at least 8 points on the “moving around” activity of the assessment for the mobility component of personal independence payment may be issued with a badge in accordance with those Regulations. An amendment to regulation 6 also provides that the period of issue of a badge shall end on the date when the badge holder will cease to receive personal independence payment, where this is less than three years from the date of issue.

Paragraphs 5 to 51 of the Schedule ensure that appropriate references to personal independence payment are inserted to both primary and secondary legislation where there are references to disability living allowance. 

\end{document}