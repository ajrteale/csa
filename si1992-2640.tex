\documentclass[12pt,a4paper]{article}

\newcommand\regstitle{The Child Support Commissioners (Procedure) Regulations 1992}

\newcommand\regsnumber{1992/2640}

%\opt{newrules}{
\title{\regstitle}
%}

%\opt{2012rules}{
%\title{Child Maintenance and Other Payments Act 2008\\(2012 scheme version)}
%}

\author{S.I. 1992 No. 2640}

\date{Made 26th October 1992\\Laid before Parliament 29th October 1992\\Coming into force 5th April 1993}

%\opt{oldrules}{\newcommand\versionyear{1993}}
%\opt{newrules}{\newcommand\versionyear{2003}}
%\opt{2012rules}{\newcommand\versionyear{2012}}

\usepackage{csa-regs}

\begin{document}

\maketitle

\amendment{
Regs. revoked (1.6.99) by the Child Support Commissioners (Procedure) Regulations 1999 reg. 2(a).
}

% Regs revoked (1.6.99) by SI 1999/1305 reg 2(a)
%\noindent
%The Lord Chancellor, in exercise of the powers conferred by sections 22(3), 24(6) and (7), and 25(2), (3) and (5) of the Child Support Act 1991\footnote{\frenchspacing 1991 c. 48.} and of all other powers enabling him in that behalf, after consultation with the Lord Advocate and, in accordance with section 8 of the Tribunals and Inquiries Act 1992\footnote{\frenchspacing 1992 c. 53.}, with the Council on Tribunals, hereby makes the following Regulations:
%
%{\sloppy
%
%\tableofcontents
%
%}
%
%\setcounter{secnumdepth}{-2}
%
%\section[Part I --- Introduction]{Part I\\*Introduction}
%
%\subsection[1. Citation, commencement and interpretation]{Citation, commencement and interpretation}
%
%\renewcommand\parthead{--- Part I}
%
%1.—(1) These Regulations may be cited as the Child Support Commissioners (Procedure) Regulations 1992 and shall come into force on 5th April 1993.
%
%(2) In these Regulations, unless the context otherwise requires—
%\begin{enumerate}\item[]
%“the Act” means the Child Support Act 1991;
%
%“appeal tribunal” means a child support appeal tribunal;
%
%“the chairman”, for the purposes of regulations 2 and 3, means—
%\begin{enumerate}\item[]
%(i) the person who was the chairman of the appeal tribunal which gave the decision against which leave to appeal is being sought; or
%
%(ii) where the application for leave to appeal to a Commissioner was dealt with under regulation 2(2), the chairman who dealt with the application;
%\end{enumerate}
%
%“Chief Commissioner” means the Chief Child Support Commissioner appointed under section 22(1) of the Act;
%
%“Commissioner” means the Chief or any other Child Support Commissioner appointed under section 22(1) of the Act and includes a Tribunal of Commissioners constituted under paragraph 5 of Schedule 4 to the Act;
%
%% Definition of ``full statement of the tribunal's decision'' inserted (28.4.97) by SI 1997/955 reg 10
%“full statement of the tribunal’s decision” means the statement referred to in regulation 13(3A) of the Child Support Appeal Tribunals (Procedure) Regulations 1992\footnote{\frenchspacing S.I. 1992/2641. Paragraph (3A) of regulation 13 was inserted by S.I. 1996/182, substituted by S.I. 1996/2450 and amended by S.I. 1996/2907.};
%
%\begin{sloppypar}
%“proceedings” means any proceedings before a Commissioner, whether by way of an application for leave to appeal to, or from, a Commissioner, or by way of an appeal or otherwise;
%\end{sloppypar}
%
%“respondent” means any person, other than the applicant or appellant, who participated as a party to the proceedings before the appeal tribunal, and any other person who, pursuant to a direction given under regulation 7(1)($a$), is served with notice of the appeal; and
%
%“summons” in relation to Scotland, means “citation” and regulation 14 shall be construed accordingly.
%\end{enumerate}
%
%(3) In these Regulations, unless the context otherwise requires, a reference—
%\begin{enumerate}\item[]
%($a$) to a numbered regulation is to the regulation in these Regulations bearing that number;
%
%($b$) in a regulation to a numbered paragraph is to the paragraph in that regulation bearing that number;
%
%($c$) in a paragraph to a lettered sub-paragraph is to the sub-paragraph in that paragraph bearing that letter.
%\end{enumerate}
%
%% Reg 1(4) inserted (14.4.97) by SI 1997/802 reg 2
%(4) In these Regulations, for the purposes of any proceedings relating to an application for a departure direction which has been decided by an appeal tribunal under section 28D(1)($b$)\footnote{\frenchspacing Section 28D of the Child Support Act 1991 was inserted by the Child Support Act 1995, section 4.} of the Act, the term ‘party to the proceedings’ shall include the Secretary of State.
%
%\amendment{
%Reg. 1(4) inserted (14.4.97) by the Child Support Commissioners (Procedure) (Amendment) Regulations 1997 reg. 2.
%
%Definition of ``full statement of the tribunal's decision'' inserted in reg. 1(2) (28.4.97) by the Social Security (Adjudication) and Commissioners Procedure and Child Support Commissioners (Procedure) Amendment Regulations 1997 reg. 10.
%}
%
%\section[Part II --- Applications for leave to appeal and appeals to a Commissioner]{\sloppy Part II\\*Applications for leave to appeal and appeals to a Commissioner}
%
%\renewcommand\parthead{--- Part II}
%
%\subsection[2. Application to the chairman of an appeal tribunal or to a Commissioner for leave to appeal to a Commissioner]{Application to the chairman of an appeal tribunal or to a Commissioner for leave to appeal to a Commissioner}
%
%2.—(1) An application for leave to appeal to a Commissioner from the decision of an appeal tribunal shall be made—
%\begin{enumerate}\item[]
%($a$) in the case of an application to the chairman of an appeal tribunal, within the period of 3 months beginning with the date on which 
%%notice of the decision of the tribunal 
%the full statement of the tribunal’s decision  % Words substituted (28.4.97) by SI 1997/955 reg 11
%was given or sent to the applicant; or
%
%($b$) in the case of an application to a Commissioner, within the period of 42 days beginning with the date on which notice of the refusal of leave to appeal by the chairman of the appeal tribunal was given or sent to the applicant.
%\end{enumerate}
%
%(2) Where in any case it is impracticable, or it would be likely to cause undue delay for an application for leave to appeal against a decision of an appeal tribunal to be determined by the person who was the chairman of that tribunal, that application shall be determined by any other person qualified under paragraph 3 of Schedule 3 to the Act to act as a chairman of appeal tribunals.
%
%(3) Subject to paragraph (4), an application may be made to a Commissioner for leave to appeal against a decision of an appeal tribunal only where the applicant has been refused leave to appeal by the chairman of the appeal tribunal.
%
%(4) Where there has been a failure to apply to the chairman of the tribunal, either within the time specified in paragraph (1)($a$) or at all, an application for leave to appeal may be made to a Commissioner who may, if for special reasons he thinks fit, accept and proceed to consider and determine the application.
%
%(5) A Commissioner may accept and proceed to consider and determine an application for leave to appeal under paragraph (3) notwithstanding that the period specified for making the application has expired if for special reasons he thinks fit.
%
%\amendment{
%Words substituted in reg. 2(1)(a) (28.4.97) by the Social Security (Adjudication) and Commissioners Procedure and Child Support Commissioners (Procedure) Amendment Regulations 1997 reg. 11.
%}
%
%\subsection[3. Notice of application for leave to appeal to a Commissioner]{Notice of application for leave to appeal to a Commissioner}
%
%3.—(1) An application for leave to appeal shall be brought by a notice in writing to the clerk to the 
%%tribunal at the Central Office of Child Support Appeal Tribunals 
%appeal tribunal  % Words substituted (14.4.97) by SI 1997/802 reg 3
%at Anchorage Two, Anchorage Quay, Salford Quays, Manchester, \textsc{\lowercase{M5 2YN}} or, as the case may be, to a Commissioner, and shall contain—
%\begin{enumerate}\item[]
%($a$) the name and address of the applicant;
%
%($b$) the grounds on which the applicant intends to rely;
%
%($c$) an address for service of notices and other documents on the applicant; and
%
%($d$) where the applicant is to be represented by a person who is not a barrister, advocate or solicitor, the written authority of the applicant for that person to represent him,
%\end{enumerate}
%and the notice shall have annexed to it 
%%a copy of the decision against which leave to appeal is being sought.
%a copy of the full statement of the tribunal’s decision against which leave to appeal is being sought.  % Words substituted (28.4.97) by SI 1997/955 reg 12
%
%(2) In the case of an application for leave to appeal to a Commissioner made to a Commissioner where the applicant has been refused leave to appeal by the chairman of an appeal tribunal the notice shall also have annexed to it a copy of the decision refusing leave to appeal, and shall state the date on which the applicant was given notice of the refusal of leave.
%
%(3) Where the applicant has failed to apply within the time specified in regulation 2(1)($a$) or, as the case may be, 2(1)($b$) for leave to appeal, the notice of application for leave to appeal shall, in addition to complying with paragraph (1), state the grounds relied upon for seeking acceptance of the application notwithstanding that the relevant period has expired.
%
%(4) In a case where the application for leave to appeal is made by 
%%a child support officer he 
%a child support officer or by the Secretary of State under section 24(1A)\footnote{\frenchspacing Subsection (1A) of section 24 of the Child Support Act 1991 was inserted by the Child Support Act 1995, section 30(5), Schedule 3, paragraph 7.} of the Act, the child support officer or the Secretary of State, as the case may be,  % Words substituted (14.4.97) by SI 1997/802 reg 4
%shall send a copy of the application to each person who was a party to the proceedings before the appeal tribunal, and in any other case the clerk to the tribunal or, as the case may be, the Office of the Commissioner shall send a copy of the application to each person, other than the applicant, who was such a party.
%
%(5) An applicant for leave to appeal to a Commissioner may at any time before the application is determined withdraw it by giving written notice of withdrawal to the clerk to the tribunal or, as the case may be, to the Commissioner.
%
%\amendment{
%Words substituted in reg. 3(1), (4) (14.4.97) by the Child Support Commissioners (Procedure) (Amendment) Regulations 1997 regs. 3, 4.
%
%Words substituted in reg. 3(1) (28.4.97) by the Social Security (Adjudication) and Commissioners Procedure and Child Support Commissioners (Procedure) Amendment Regulations 1997 reg. 12.
%}
%
%\subsection[4. Determination of applications for leave to appeal]{Determination of applications for leave to appeal}
%
%4.—(1) The determination of an application for leave to appeal to a Commissioner made to the chairman of an appeal tribunal shall be recorded in writing by the chairman and a copy of the determination shall be sent by the clerk to the tribunal to the applicant and every other person to whom notice of the application was given under regulation 3(4).
%
%(2) Unless a Commissioner directs to the contrary, where a Commissioner grants leave to appeal on an application made in accordance with regulation 3, notice of appeal shall be deemed to have been duly given on the date when notice of the determination was given to the applicant and the notice of application shall be deemed to be a notice of appeal duly served under regulation 5.
%
%(3) If on consideration of an application for leave to appeal to him from the decision of an appeal tribunal the Commissioner grants leave he may, with the consent of the applicant and each respondent, treat the application as an appeal and determine any question arising on the application as though it were a question arising on an appeal.
%
%\subsection[5. Notice of appeal]{Notice of appeal}
%
%5.—(1) Subject to regulation 4(2), an appeal shall be brought by a notice to a Commissioner containing—
%\begin{enumerate}\item[]
%($a$) the name and address of the appellant;
%
%($b$) the date on which leave to appeal was granted;
%
%($c$) the grounds on which the appellant intends to rely;
%
%($d$) an address for service of notices and other documents on the appellant,
%\end{enumerate}
%and the notice shall have annexed to it a copy of the determination granting leave to appeal and a copy of the 
%%decision 
%full statement of the tribunal’s decision  % Words substituted (28.4.97) by SI 1997/955 reg 13
%against which leave to appeal has been granted.
%
%\amendment{
%Words substituted in reg. 5(1) (28.4.97) by the Social Security (Adjudication) and Commissioners Procedure and Child Support Commissioners (Procedure) Amendment Regulations 1997 reg. 13.
%}
%
%\subsection[6. Time limit for appealing]{Time limit for appealing}
%
%6.—(1) Subject to paragraph (2), a notice of appeal shall not be valid unless it is served on a Commissioner within 42 days of the date on which the applicant was given notice in writing that leave to appeal had been granted.
%
%(2) A Commissioner may accept a notice of appeal served after the expiry of the period prescribed by paragraph (1) if for special reasons he thinks fit.
%
%\subsection[7. Directions on notice of appeal]{Directions on notice of appeal}
%
%7.—(1) As soon as practicable after the receipt of a notice of appeal a Commissioner shall give such directions as appear to him to be necessary, specifying—
%\begin{enumerate}\item[]
%($a$) the parties who are to be respondents to the appeal; and
%
%($b$) the order in which and the time within which any party is to be allowed to make written observations on the appeal or on the observations made by any other party.
%\end{enumerate}
%
%(2) If in any case two or more persons who were parties to the proceedings before the appeal tribunal give notice of appeal to a Commissioner, a Commissioner shall direct which one of them is to be treated as the appellant, and thereafter, but without prejudice to any rights or powers conferred on appellants by the Act or these Regulations, any other person who has given notice of appeal shall be treated as a respondent.
%
%(3) Subject to regulation 23(2)($b$), the time specified in directions given under paragraph (1)($b$) as being the time within which written observations are to be made shall be not less than 30 days beginning with the day on which the notice of the appeal or, as the case may be, the observations were sent to the party concerned.
%
%\subsection[8. Acknowledgement of a notice of appeal and notification to each respondent]{Acknowledgement of a notice of appeal and notification to each respondent}
%
%8.  There shall be sent by the office of the Child Support Commissioners—
%\begin{enumerate}\item[]
%($a$) to the appellant, an acknowledgement of the receipt of the notice of appeal; and
%
%($b$) to each respondent, a copy of the notice of appeal.
%\end{enumerate}
%
%\subsection[9. Secretary of State as respondent to an appeal]{Secretary of State as respondent to an appeal}
%
%9.  
%Except where he is already a party to the proceedings by virtue of regulation 1(4) or of regulation 1 of the Child Support Appeal Tribunals (Procedure) Regulations 1992\footnote{\frenchspacing S.I. 1992/2641.},  % Words inserted (14.4.97) by SI 1997/802 reg 5
%the Secretary of State may at any time apply to a Commissioner for leave to intervene in an appeal pending before a Commissioner, and if such leave is granted the Secretary of State shall thereafter be treated as a respondent to that appeal.
%
%\amendment{
%Words inserted in reg. 9 (14.4.97) by the Child Support Commissioners (Procedure) (Amendment) Regulations 1997 reg. 5.
%}
%
%
%\section[Part III --- General procedure]{Part III\\*General procedure}
%
%\renewcommand\parthead{--- Part III}
%
%\subsection[10. Other directions]{Other directions}
%
%10.—(1) Where it appears to a Commissioner that an application or appeal which is made to him gives insufficient particulars to enable the question at issue to be determined, he may direct the party making the application or appeal or any respondent to furnish such further particulars as may reasonably be required.
%
%(2) At any stage of the proceedings a Commissioner may, either of his own motion or on application, give such directions or further directions as he may consider necessary or desirable for the efficient and effective despatch of the proceedings.
%
%(3) Without prejudice to the provisions of paragraph (2), a Commissioner may direct any party to any proceedings before him to make such written observations as may seem to him necessary to enable the question at issue to be determined.
%
%(4) An application under paragraph (2) shall be made in writing to a Commissioner and shall set out the direction which the applicant is seeking to have made and the grounds for the application.
%
%(5) Unless a Commissioner otherwise determines, an application made pursuant to paragraph (2) shall be copied by the office of the Child Support Commissioners to the other parties.
%
%(6) The powers to give directions conferred by paragraphs (2) and (3) include power to revoke or vary any such direction.
%
%\subsection[11. Requests for oral hearings]{Requests for oral hearings}
%
%11.—(1) Subject to paragraphs (2) and (3), a Commissioner may determine an application for leave to appeal or an appeal without an oral hearing.
%
%(2) Where in any proceedings before a Commissioner a request is made by any party for an oral hearing the Commissioner shall grant the request unless, after considering all the circumstances of the case and the reasons put forward in the request for the hearing, he is satisfied that the application or appeal can properly be determined without a hearing, in which event he may proceed to determine the case without a hearing and he shall in writing either before giving his determination or decision, or in it, inform the person making the request that it has been refused.
%
%(3) A Commissioner may of his own motion at any stage, if he is satisfied that an oral hearing is desirable, direct such a hearing.
%
%\subsection[12. Representation at an oral hearing]{Representation at an oral hearing}
%
%12.  At any oral hearing a party may conduct his case himself (with assistance from any person if he wishes) or be represented by any person whom he may appoint for the purpose.
%
%\subsection[13. Oral hearings]{Oral hearings}
%
%13.—(1) This regulation applies to any oral hearing to which these Regulations apply.
%
%(2) Reasonable notice (being not less than 10 days beginning with the day on which notice is given and ending on the day before the hearing of the case is to take place) of the time and place of any oral hearing before a Commissioner shall be given to the parties by the office of the Child Support Commissioners.
%
%(3) If any party to whom notice of an oral hearing has been given in accordance with these Regulations should fail to appear at the hearing, the Commissioner may, having regard to all the circumstances including any explanation offered for the absence, proceed with the case notwithstanding that party’s absence, or may give such directions with a view to the determination of the case as he thinks fit.
%
%(4) Any oral hearing before a Commissioner shall be in public except where the Commissioner for special reasons directs otherwise, in which case the hearing or any part thereof shall be in private.
%
%(5) Where a Commissioner holds an oral hearing the applicant or appellant and every respondent shall be entitled to be present and be heard.
%
%(6) Any person entitled to be heard at an oral hearing may—
%\begin{enumerate}\item[]
%($a$) address the Commissioner;
%
%($b$) with the leave of the Commissioner but not otherwise, give evidence, call witnesses and put questions directly to any other person called as a witness.
%\end{enumerate}
%
%(7) Nothing in these Regulations shall prevent a member of the Council on Tribunals or of the Scottish Committee of the Council in his capacity as such from being present at an oral hearing before a Commissioner notwithstanding that the hearing is not in public.
%
%\subsection[14. Summoning of witnesses]{Summoning of witnesses}
%
%14.—(1) Subject to paragraph (2), a Commissioner may summon any person to attend as a witness at an oral hearing, at such time and place as may be specified in the summons, to answer any questions or produce any documents in his custody or under his control which relate to any matter in question in the proceedings.
%
%(2) No person shall be required to attend in obedience to a summons under paragraph (1) unless he has been given at least 7 days' notice of the hearing or, if less than 7 days, has informed the Commissioner that he accepts such notice as he has been given.
%
%(3) A Commissioner may upon the application of a person summoned under this regulation set the summons aside.
%
%(4) A Commissioner may require any witness to give evidence on oath and for that purpose there may be administered an oath in due form.
%
%\subsection[15. Postponement and adjournment]{Postponement and adjournment}
%
%15.—(1) A Commissioner may, either of his own motion or on an application by any party to the proceedings, postpone an oral hearing.
%
%(2) An oral hearing, once commenced, may be adjourned by the Commissioner at any time either on the application of any party to the proceedings or of his own motion.
%
%\subsection[16. Withdrawal of applications for leave to appeal and appeals]{Withdrawal of applications for leave to appeal and appeals}
%
%16.—(1) At any time before it is determined, an application to a Commissioner for leave to appeal against a decision of an appeal tribunal may be withdrawn by the applicant by giving written notice to a Commissioner of his intention to do so.
%
%(2) At any time before the decision is made, an appeal to a Commissioner may, with the leave of a Commissioner, be withdrawn by the appellant.
%
%(3) A Commissioner may, on application by the party concerned, give leave to reinstate any application or appeal which has been withdrawn in accordance with paragraphs (1) and (2) and, on giving leave, he may make such directions as to the future conduct of the proceedings as he thinks fit.
%
%\subsection[17. Irregularities]{Irregularities}
%
%17.  Any irregularity resulting from failure to comply with the requirements of these Regulations before a Commissioner has determined the application or appeal shall not by itself invalidate any proceedings, and the Commissioner, before reaching his decision, may waive the irregularity or take such steps as he thinks fit to remedy the irregularity whether by amendment of any document, or the giving of any notice or directions or otherwise.
%
%\section[Part IV --- Decisions]{Part IV\\*Decisions}
%
%\renewcommand\parthead{--- Part IV}
%
%\subsection[18. Determinations and decisions of a Commissioner]{Determinations and decisions of a Commissioner}
%
%18.—(1) The determination of a Commissioner on an application for leave to appeal shall be in writing and signed by him.
%
%(2) The decision of a Commissioner on an appeal shall be in writing and signed by him and, except in respect of a decision made with the consent of the parties, he shall record the reasons.
%
%(3) A copy of the determination or decision and any reasons shall be sent to the parties by the office of the Child Support Commissioners.
%
%(4) Without prejudice to paragraphs (2) and (3), a Commissioner may announce his determination or decision at the conclusion of an oral hearing.
%
%(5) When giving his decision on an application or appeal, whether in writing or orally, a Commissioner shall omit any reference to the surname of any child to whom the appeal relates and any other information which would be likely, whether directly or indirectly, to identify that child.
%
%\subsection[19. Correction of accidental errors in decisions]{Correction of accidental errors in decisions}
%
%19.—(1) Subject to regulation 21, accidental errors in any decision or record of a decision may at any time be corrected by the Commissioner who gave the decision.
%
%(2) A correction made to, or to the record of, a decision shall become part of the decision or record thereof and written notice thereof shall be given by the office of the Child Support Commissioners to any party to whom notice of the decision had previously been given.
%
%\subsection[20. Setting aside of decisions on certain grounds]{Setting aside of decisions on certain grounds}
%
%20.—(1) Subject to the following provisions of this regulation and regulation 21, on an application made by any party a decision may be set aside by the Commissioner who gave the decision in a case where it appears just to do so on the ground that—
%\begin{enumerate}\item[]
%($a$) a document relating to the proceedings was not sent to, or was not received at an appropriate time by, a party or his representative, or was not received at an appropriate time by the Commissioner; or
%
%($b$) a party or his representative had not been present at an oral hearing which had been held in the course of the proceedings; or
%
%($c$) there has been some other procedural irregularity or mishap.
%\end{enumerate}
%
%(2) An application under this regulation shall be made in writing to a Commissioner within 30 days from the date on which notice in writing of the decision was given by the office of the Child Support Commissioners to the party making the application.
%
%(3) Where an application to set aside a decision is made under paragraph (1), each party shall be sent by the office of the Child Support Commissioners a copy of the application and shall be afforded a reasonable opportunity of making representations on it before the application is determined.
%
%(4) Notice in writing of a determination of an application to set aside a decision shall be given by the office of the Child Support Commissioners to each party and shall contain a statement giving the reasons for the determination.
%
%\subsection[21. Provisions common to regulations 19 and 20]{Provisions common to regulations 19 and 20}
%
%21.—(1) In regulations 19 and 20 the word “decision” shall include determinations of applications for leave to appeal as well as decisions on appeals.
%
%(2) Subject to a direction by a Commissioner to the contrary, in calculating any time for applying for leave to appeal against a Commissioner’s decision there shall be disregarded any day falling before the day on which notice was given of a correction of a decision or the record thereof pursuant to regulation 19 or on which notice was given of a determination that a decision shall not be set aside under regulation 20, as the case may be.
%
%(3) There shall be no appeal against a correction or a refusal to correct under regulation 19 or a determination given under regulation 20.
%
%(4) If it is impracticable or likely to cause undue delay for a decision or record of a decision to be dealt with pursuant to regulation 19 or 20 by the Commissioner who gave the decision, the Chief Commissioner or another Commissioner may deal with the matter.
%
%\section[Part V --- Miscellaneous and supplemental]{Part V\\*Miscellaneous and supplemental}
%
%\renewcommand\parthead{--- Part V}
%
%\subsection[22. Confidentiality]{Confidentiality}
%
%%22.—(1) No information such as is mentioned in paragraph (2), and which has been furnished for the purposes of any proceedings to which these Regulations apply, shall be disclosed except with the written consent of the person to whom the information relates.
%%
%%(2) The information mentioned in paragraph (1) is—
%%\begin{enumerate}\item[]
%%($a$) the address, other than the address of the office of the Commissioner concerned and the place where the oral hearing (if any) is to be held; and
%%
%%($b$) any other information the use of which could reasonably be expected to lead to a person being located.
%%\end{enumerate}
%
%% Reg 22 substituted (14.4.97) by SI 1997/802 reg 6
%22.—(1) No information such as is mentioned in paragraph (2), and which has been provided for the purposes of any proceedings to which these Regulations apply, shall be disclosed if, before the expiry of the period of 21 days specified in paragraph (3), written notification has been received from the person to whom the information relates that he does not consent to such disclosure.
%
%(2) The information referred to in paragraph (1) is—
%\begin{enumerate}\item[]
%($a$) the address of the person referred to in that paragraph; and
%
%($b$) any other information the use of which could reasonably be expected to lead to that person being located.
%\end{enumerate}
%
%(3) Except where the proceedings relate to an application for leave to appeal to a Commissioner or to an appeal in either case made under section 46(7) of the Act (Failure to comply with obligations imposed by section 6) or regulation 42(9) of the Child Support (Maintenance Assessment Procedure) Regulations 1992\footnote{\frenchspacing S.I. 1992/1813.} (Review of a reduced benefit direction), the Office of the Commissioner shall notify the person to whom the information referred to in paragraphs (1) and (2) relates of the provisions of those paragraphs and that disclosure of that information may be made unless the written notification specified in paragraph (1) is received before the expiry of the period of 21 days beginning with the date the notification by the Office of the Commissioner was given or sent to that person.
%
%\amendment{
%Reg. 22 substituted (14.4.97) by the Child Support Commissioners (Procedure) (Amendment) Regulations 1997 reg. 6.
%}
%
%\subsection[23. General powers of a Commissioner]{General powers of a Commissioner}
%
%23.—(1) Subject to the provisions of these Regulations, and without prejudice to regulations 7 and 10, a Commissioner may adopt such procedure in relation to any proceedings before him as he sees fit.
%
%(2) A Commissioner may, if he thinks fit—
%\begin{enumerate}\item[]
%($a$) subject to regulations 2(5) and 6(2), extend the time specified by or under these Regulations for doing any act, notwithstanding that the time specified may have expired;
%
%($b$) abridge the time so specified; or
%
%($c$) expedite the proceedings in such manner as he thinks fit.
%\end{enumerate}
%
%(3) Subject to paragraph (4), a Commissioner may, if he thinks fit, either on the application of a party or of his own motion, strike out for want of prosecution any application for leave to appeal or any appeal.
%
%(4) Before making an order under paragraph (3), the Commissioner shall send notice to the party against whom it is proposed that it shall be made giving him an opportunity to show cause why it should not be made.
%
%(5) A Commissioner may, on application by the party concerned, give leave to reinstate any application or appeal which has been struck out in accordance with paragraph (3) and, on giving leave, he may make such directions as to the future conduct of the proceedings as he thinks fit.
%
%(6) Nothing in these Regulations shall be construed as derogating from any other power which is exercisable apart from these Regulations.
%
%%Reg 23A inserted (4.12.95) by SI 1995/2907 reg 2
%%\subsection[23A. Delegation of functions to nominated officers]{Delegation of functions to nominated officers}
%%
%%23A.—(1) All or any of the following functions of a Commissioner may be exercised by a nominated officer, that is to say:
%%\begin{enumerate}\item[]
%%($a$) giving directions under regulation 7(1) and (2) (directions on notice of appeal);
%%
%%($b$) granting leave under regulation 9 to the Secretary of State to intervene in an appeal;
%%
%%($c$) making any direction under regulation 10(1), (2) and (3) (other directions);
%%
%%($d$) making orders for oral hearings under regulation 11(2) and (3);
%%
%%($e$) summoning witnesses under regulation 14(1) and (2) and setting aside under regulation 14(3) a witness summons made by a nominated officer;
%%
%%($f$) ordering the postponement of oral hearings under regulation 15(1);
%%
%%($g$) giving leave for the withdrawal of any appeal under regulation 16(2);
%%
%%($h$) making any order for the extension or abridgement of time, or for expediting the proceedings, under regulation 23(2)($a$), ($b$) and ($c$).
%%\end{enumerate}
%%
%%(2) Any party may, within 10 days of being given the decision of the nominated officer, in writing request a Commissioner to consider, and confirm or replace with his own, that decision, but such a request shall not stop the proceedings unless so ordered by the Commissioner.
%%
%%(3) In this regulation, “nominated officer” means an officer authorised by the Lord Chancellor (or, in Scotland, by the Secretary of State) in accordance with paragraph 4A of Schedule 4 to the Act.
%
%%Reg 23A revoked and re-inserted (1.3.96) by SI 1996/243 reg 2
%\subsection[23A. Delegation of functions to nominated officers]{Delegation of functions to nominated officers}
%
%23A.—(1) All or any of the following functions of a Commissioner may be exercised by a nominated officer, that is to say:
%\begin{enumerate}\item[]
%($a$) giving directions under regulation 7(1) and (2) (directions on notice of appeal);
%
%($b$) granting leave under regulation 9 to the Secretary of State to intervene in an appeal;
%
%($c$) making any direction under regulation 10(1), (2) and (3) (other directions);
%
%($d$) making orders for oral hearings under regulation 11(2) and (3);
%
%($e$) summoning witnesses under regulation 14(1) and (2) and setting aside under regulation 14(3) a witness summons made by a nominated officer;
%
%($f$) ordering the postponement of oral hearings under regulation 15(1);
%
%($g$) giving leave for the withdrawal of any appeal under regulation 16(2);
%
%($h$) making any order for the extension or abridgement of time, or for expediting the proceedings, under regulation 23(2)($a$), ($b$) and ($c$).
%\end{enumerate}
%
%(2) Any party may, within 10 days of being given the decision of the nominated officer, in writing request a Commissioner to consider, and confirm or replace with his own, that decision, but such a request shall not stop the proceedings unless so ordered by the Commissioner.
%
%(3) In this regulation, “nominated officer” means an officer authorised by the Lord Chancellor (or, in Scotland, by the Secretary of State) in accordance with paragraph 4A of Schedule 4 to the Act.
%
%\amendment{
%Reg. 23A inserted (4.12.95) by the Child Support Commissioners (Procedure) (Amemdment) Regulations 1995 reg. 2.
%
%Reg. 23A revoked and re-inserted (1.3.96) by the Child Support Commissioners (Procedure) (Amemdment) Regulations 1996 reg. 2.
%}
%
%\subsection[24. Manner of and time for service of notices, etc.]{Manner of and time for service of notices, etc.}
%
%24.—(1) Any notice or other document required or authorised to be given or sent to any party under the provisions of these Regulations shall be deemed to have been given or sent if it was sent by post properly addressed and pre-paid to that party at his ordinary or last notified address.
%
%(2) Any notice or other document given, sent or served by post shall be deemed to have been given on the day on which it was posted.
%
%(3) Any notice or other document required to be given, sent or submitted to or served on a Commissioner—
%\begin{enumerate}\item[]
%($a$) shall be given, sent or submitted to an office of the Child Support Commissioners;
%
%($b$) shall be deemed to have been given, sent or submitted if it was sent by post properly addressed and pre-paid to an office of the Child Support Commissioners.
%\end{enumerate}
%
%\subsection[25. Application to a Commissioner for leave to appeal to the Courts]{Application to a Commissioner for leave to appeal to the Courts}
%
%25.—(1) A person who was a party to the proceedings in which the original decision or appeal decision was given (both of those expressions having the meaning assigned to them by section 25 of the Act) may appoint any person for the purpose of making an application for leave to appeal under section 25 of the Act.
%
%(2) An application to a Commissioner under section 25 of the Act for leave to appeal against a decision of a Commissioner shall be made in writing and shall be made within 3 months from the date on which the applicant was given written notice of the decision.
%
%(3) In a case where the Chief Commissioner considers that it is impracticable, or would be likely to cause undue delay, for such an application to be determined by the Commissioner who decided the case, that application shall be determined—
%\begin{enumerate}\item[]
%($a$) where the decision was a decision of an individual Commissioner, by the Chief Commissioner or a Commissioner selected by the Chief Commissioner; and
%
%($b$) where the decision was a decision of a Tribunal of Commissioners, by a differently constituted Tribunal of Commissioners selected by the Chief Commissioner.
%\end{enumerate}
%
%(4) If the office of Chief Commissioner is vacant, or if the Chief Commissioner is unable to act, paragraph (3) shall have effect as if the expression “the Chief Commissioner” referred to such other of the Commissioners as may have been nominated to act for the purpose either by the Chief Commissioner or, if he has not made such a nomination, by the Lord Chancellor.
%
%(5) Regulations 16(1) and 16(3) shall apply to applications to a Commissioner for leave to appeal from a Commissioner as they do to the proceedings therein set out.
%
%\bigskip
%
%{\raggedleft
%\emph{Mackay of Clashfern, C.}
%
%
%}
%
%26th October 1992
%
%\small
%
%\part{Explanatory Note}
%
%\renewcommand\parthead{--- Explanatory Note}
%
%\subsection*{(This note is not part of the Regulations)}
%
% These Regulations make provision for the procedure to be followed in proceedings before a Child Support Commissioner under the Child Support Act 1991.
%
%  Part I is introductory. Part II contains provisions about making applications to the chairman of a child support appeal tribunal or a Child Support Commissioner for leave to appeal, and appeals to a Child Support Commissioner. In particular provision is made for the time within which an application or appeal must be made (regulations 2 and 6), the contents of the notice of application or appeal (regulations 3 and 5), for enabling a Child Support Commissioner to give directions as to who are to be respondents to the application or appeal and enabling the Secretary of State to apply to intervene in proceedings.
%
%  Part III contains general procedural provisions about oral hearings, postponements and adjournments, withdrawal of applications and appeals and irregularities in proceedings. Part IV makes provision about determinations and decisions of Child Support Commissioners and confers power to correct or set aside decisions in certain circumstances. Part V contains miscellaneous and supplementary provisions.
%



\end{document}