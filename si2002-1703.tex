\documentclass[12pt,a4paper]{article}

\newcommand\regstitle{The Social Security (Jobcentre Plus Interviews) Regulations 2002}

\newcommand\regsnumber{2002/1703}

%\opt{newrules}{
\title{\regstitle}
%}

%\opt{2012rules}{
%\title{Child Maintenance and~Other Payments Act 2008\\(2012 scheme version)}
%}

\author{S.I.\ 2002 No.\ 1703}

\date{Made
3rd July 2002\\
Laid before Parliament
8th July 2002\\
Coming into~force
30th September 2002
}

%\opt{oldrules}{\newcommand\versionyear{1993}}
%\opt{newrules}{\newcommand\versionyear{2003}}
%\opt{2012rules}{\newcommand\versionyear{2012}}

\usepackage{csa-regs}

\setlength\headheight{42.11603pt}

%\hbadness=10000

\begin{document}

\maketitle

\begin{sloppypar}
\noindent
The Secretary of State for Work and~Pensions, in exercise of the powers conferred upon him by sections 2A(1), (3) to~(6) and~(8), 2B(6) and~(7),~5(1)($a$)  and~($b$), 6(1)($a$)  and~($b$), 7A, 189(1), (4) and~(5) and~191 of the Social Security Administration Act 1992\footnote{1992 c.~5. Sections 2A, 2B and~7A were inserted by the Welfare Reform and~Pensions Act 1999 (c.~30), sections 57 and~71 respectively; section~191 is an interpretation provision and~is cited because of the meaning ascribed to~the word “prescribe”. Section 2A(8) is cited because of the meaning ascribed to~the word “specified”.} and~section~68 of, and~paragraphs~3(1),~4(4),~6(8),~20(3) and~23(1) of Schedule 7 to, the Child Support, Pensions and~Social Security Act 2000\footnote{2000 c.~19. Paragraph 23(1) is cited because of the meaning ascribed to~the word “prescribed”.}, and~of all other powers enabling him in that behalf, after consultation with the Council on Tribunals in accordance with section~8(1) of the Tribunals and~Inquiries Act 1992\footnote{1992 c.~53.} and~in respect of provisions in these Regulations relating to~housing benefit and~council tax benefit with organisations appearing to~him to~be representative of the authorities concerned\footnote{\emph{See} section~176(1) of the Social Security Administration Act 1992.}, and~after agreement by the Social Security Advisory Committee that proposals in respect of these Regulations should not be referred to~it\footnote{\emph{See} section~173(1)($b$)  of the Social Security Administration Act 1992. Section 68 of, and~Schedule 7 to, the Child Support, Pensions and~Social Security Act 2000 were included in the list of “relevant enactments” in section~170(5) by section~73 of the 2000 Act}, hereby makes the following Regulations: 
\end{sloppypar}

{\sloppy

\tableofcontents

}

\bigskip

\setcounter{secnumdepth}{-2}

\subsection[1. Citation and~commencement]{Citation and~commencement}

1.  These Regulations may be cited as the Social Security (Jobcentre Plus Interviews) Regulations 2002 and~shall come into~force on 30th September 2002.\looseness=-1

\subsection[2. Interpretation and~application]{Interpretation and~application}

2.---(1)  In these Regulations, unless the context otherwise requires—
\begin{enumerate}\item[]
“the 1998 Act” means the Social Security Act 1998\footnote{1998 c.~14.};

“benefit week” means any period of seven days corresponding to~the week in respect of which the relevant specified benefit is due to~be paid;

“bereavement benefit” means any benefit, other than a bereavement payment, falling within section~20(1)($ea$)  of the Social Security Contributions and~Benefits Act 1992\footnote{1992 c.~4. Section 20(1)($ea$)  was inserted by the Welfare Reform and Pensions Act 1999, section~70.};

“the Careers Service” means—
\begin{enumerate}\item[]
($a$) 
in England~and~Wales, a person with whom the Secretary of State or, as the case may be, the National Assembly for Wales, have made arrangements under section~10(1) of the Employment and~Training Act 1973 or a local education authority to~whom a direction has been given by the Secretary of State or the National Assembly for Wales under section~10(2) of that Act;

($b$) 
in Scotland, a person with whom the Scottish Ministers have made arrangements under section~10(1) of the Employment and Training Act 1973 or any education authority to~whom a direction has been given by the Scottish Ministers under section~10(2) of that Act;\looseness=-1
\end{enumerate}

“the Connexions Service” means a person of any description with whom the Secretary of State has made an arrangement under section~114(2)($a$)  of the Learning and~Skills Act 2000\footnote{2000 c.~21.} and~section~10(1) of the Employment and~Training Act 1973 and~any person to~whom he has given a direction under section~114(2)($b$)  of the Learning and~Skills Act 2000 and~section~10(2) of the Employment and~Training Act 1973;

“interview” means a work-focused interview with a person who has claimed a specified benefit and~which is conducted for any or all of the following purposes—
\begin{enumerate}\item[]
($a$) 
assessing that person’s prospects for existing or future employment (whether paid or voluntary);

($b$) 
assisting or encouraging that person to~enhance his prospects of such employment;

($c$) 
identifying activities which that person may undertake to strengthen his existing or future prospects of employment;

($d$) 
identifying current or future employment or training opportunities suitable to~that person’s needs; and

($e$) 
identifying educational opportunities connected with the existing or future employment prospects or needs of that person;
\end{enumerate}

“officer” means a person who is an officer of, or who is providing services to~or exercising functions of, the Secretary of State;

“specified benefit” means income support, incapacity benefit, invalid care allowance, severe disablement allowance and~any bereavement benefit.
\end{enumerate}

(2) For the purposes of these Regulations—
\begin{enumerate}\item[]
($a$) a person shall be deemed to~be in remunerative work where he is in remunerative work within the meaning prescribed in regulation~4 of the Housing Benefit (General) Regulations 1987\footnote{S.I.~1987/1971; the relevant amending instruments are S.I.~1991/1599, 1993/2118, 1995/560, 1996/1550 and 1999/2165.}; but

($b$) a person shall be deemed not to~be in remunerative work where—
\begin{enumerate}\item[]
(i) he is not in remunerative work in accordance with sub-\hspace{0pt}paragraph~($a$)  above; or

(ii) he is in remunerative work in accordance with sub-paragraph~($a$)  above and~is not entitled to~income support but would not be prevented from being entitled to~income support solely by being in such work; and
\end{enumerate}

($c$) a person shall be deemed to~be engaged in part-time work where he is engaged in work for which payment is made but he is not engaged or deemed to~be engaged in remunerative work.
\end{enumerate}

(3) Except in a case where regulation~16(2) applies, regulations 3 to~15 apply in respect of a person who makes a claim for a specified benefit on or after 30th September 2002 at an office of the Department for Work and~Pensions which is designated by the Secretary of State as a Jobcentre Plus Office\footnote{Offices designated as Jobcentre Plus Offices are identified in two lists entitled “Jobcentre Plus Pathfinder Offices” and “Jobcentre Plus Work-focused Interview Extension Sites” which are available from the Department for Work and Pensions, \textsc{\lowercase{W2W1, 4S25}}, Quarry House, Quarry Hill, Leeds, \textsc{\lowercase{LS2 7UA}}.} or at an office of a relevant authority (being a person within section~72(2) of the Welfare Reform and~Pensions Act 1999\footnote{1999 c.~30.}) which displays the \textsc{\lowercase{ONE}} logo\footnote{Offices displaying the \textsc{\lowercase{ONE}} logo are identified in a list entitled “\textsc{\lowercase{ONE}} sites—a complete list” which is available from the Department for Work and Pensions, \textsc{\lowercase{W2W1, 4S25}}, Quarry House, Quarry Hill, Leeds, \textsc{\lowercase{LS2~7UA}}.}.

\enlargethispage{-\baselineskip}

(4) Where a claim for benefit is made by a person (“the appointee”) on behalf of another, references in these Regulations to~a person claiming benefit shall be treated as a reference to~the person on whose behalf the claim is made and~not to~the appointee.

(5) In these Regulations, unless the context otherwise requires, a reference—
\begin{enumerate}\item[]
($a$) to~a numbered regulation~is to~a regulation~in these Regulations bearing that number;

($b$) in a regulation~to~a numbered paragraph~or sub-paragraph~is to~the paragraph~or sub-paragraph~in that regulation~bearing that number;

($c$) to~a numbered Schedule is to~the Schedule to~these Regulations bearing that number.
\end{enumerate}

\subsection[3. Requirement for person claiming a specified benefit to~take part in an interview]{Requirement for person claiming a specified benefit to take part in an interview}

3.---(1)  Subject to~regulations 6 to~9, a person who—
\begin{enumerate}\item[]
($a$) makes a claim for a specified benefit;

($b$) on the day on which he makes that claim, has attained the age of 16 but has not attained the age of 60; and

($c$) is not in remunerative work,
\end{enumerate}
is required to~take part in an interview.

(2) An officer shall, except where paragraph~(3) applies, conduct the interview.

(3) An officer may, if he considers it appropriate in all the circumstances, arrange for a person who has not attained the age of 18 to~attend an interview with the Careers Service or with the Connexions Service.

\subsection[4. Continuing entitlement to~specified benefit dependent on an interview]{Continuing entitlement to~specified benefit dependent on an interview}

\enlargethispage{-\baselineskip}

4.---(1)  Subject to~regulations 6 to~9, a person who has not attained the age of~60 and~who is entitled to~a specified benefit, shall be required to~take part in an interview as a condition of his continuing to~be entitled to~the full amount of benefit which is payable apart from these Regulations where paragraph~(2) applies and—
\begin{enumerate}\item[]
($a$) in the case of a lone parent who has attained the age of 18 and~who is neither claiming incapacity benefit nor severe disablement allowance, paragraph~(3) applies; or

($b$) in any other case, any of the circumstances specified in paragraph~(4) apply or where paragraph~(5) applies.
\end{enumerate}

(2) This paragraph~applies in the case of a person who has taken part in an interview under regulation~3 or who would have taken part in such an interview but for the requirement being waived in accordance with regulation~6 or deferred in accordance with regulation~7.

(3) A lone parent to~whom paragraph~(1)($a$)  applies shall be required to~take part in an interview—
\begin{enumerate}\item[]
($a$) after the expiry of six months from the date on which—
\begin{enumerate}\item[]
(i) he took part in an interview under regulation~3; or

(ii) a determination was made under regulation~6(1);
\end{enumerate}

($b$) where the lone parent took part, failed to~take part or was treated, by virtue of regulation~6(2), as having taken part, in an interview pursuant to~the requirement arising in sub-paragraph~($a$), after the expiry of six months from the date on which—
\begin{enumerate}\item[]
(i) he took part in that interview;

(ii) he failed to~take part in that interview; or

(iii) the determination was made under regulation~6(1); and
\end{enumerate}

($c$) where the lone parent took part, failed to~take part or was treated, by virtue of regulation~6(2), as having taken part, in an interview pursuant to~the requirement arising in sub-paragraph~($b$), after the expiry of twelve months from the date on which—
\begin{enumerate}\item[]
(i) he last took part in an interview;

(ii) he last failed to~take part in an interview; or

(iii) a determination was last made under regulation~6(1).
\end{enumerate}
\end{enumerate}

(4) The circumstances specified in this paragraph~are those where—
\begin{enumerate}\item[]
($a$) it is determined in accordance with a personal capability assessment that a person is incapable of work and~therefore, continues to~be entitled to~a specified benefit;

\enlargethispage{-\baselineskip}

($b$) a person’s entitlement to~an invalid care allowance ceases whilst entitlement to~another specified benefit continues;

($c$) a person becomes engaged or ceases to~be engaged in part-time work;

($d$) a person has been undergoing education or training arranged by the officer and~that education or training comes to~an end; and

($e$) a person who has not attained the age of 18 and~who has previously taken part in an interview, attains the age of 18.
\end{enumerate}

(5) A requirement to~take part in an interview arises under this paragraph~where a person has not been required to~take part in an interview by virtue of paragraph~(4) for at least 36 months.

(6) In this regulation, “lone parent” has the meaning it bears in regulation~2(1) of the Income Support (General) Regulations 1987\footnote{S.I.~1987/1967.}.

\subsection[5. Time when interview is to~take place]{Time when interview is to~take place}

5.  An officer shall arrange for an interview to~take place as soon as reasonably practicable after—
\begin{enumerate}\item[]
($a$) the claim is made;

($b$) the requirement under regulation~4(1) arises; or,

($c$) in a case where regulation~7(1) applies, the time when that requirement is to~apply by virtue of regulation~7(2).
\end{enumerate}

\subsection[6. Waiver of requirement to~take part in an interview]{Waiver of requirement to~take part in an interview}

6.---(1)  A requirement imposed by these Regulations to~take part in an interview shall not apply where an officer determines that an interview would not—
\begin{enumerate}\item[]
($a$) be of assistance to~the person concerned; or

($b$) be appropriate in the circumstances.
\end{enumerate}

(2) A person in relation to~whom a requirement to~take part in an interview has been waived under paragraph~(1) shall be treated for the purposes of—
\begin{enumerate}\item[]
($a$) regulation~3 or 4; and

\enlargethispage{-\baselineskip}

($b$) any claim for, or entitlement to, a specified benefit,
\end{enumerate}
as having complied with that requirement.

\subsection[7. Deferment of requirement to~take part in an interview]{Deferment of requirement to~take part in an interview}

7.---(1)  An officer may determine, in the case of any particular person, that the requirement to~take part in an interview shall be deferred at the time the claim is made or the requirement to~take part in an interview arises or applies because an interview would not at that time—
\begin{enumerate}\item[]
($a$) be of assistance to~that person; or

($b$) be appropriate in the circumstances.
\end{enumerate}

(2) Where the officer determines in accordance with paragraph~(1) that the requirement to~take part in an interview shall be deferred, he shall also determine when that determination is made, the time when the requirement to~take part in an interview is to~apply in the person’s case.

(3) Where a requirement to~take part in an interview has been deferred in accordance with paragraph~(1), then until—
\begin{enumerate}\item[]
($a$) a determination is made under regulation~6(1);

($b$) the person takes part in an interview; or

($c$) a relevant decision has been made in relation to~that person in accordance with regulation~11(4),
\end{enumerate}
that person shall be treated for the purposes of any claim for, or entitlement to, a specified benefit as having complied with that requirement.

\subsection[8. Exemptions]{Exemptions}

8.---(1)  Subject to~paragraph~(2), persons who, on the day on which the claim for a specified benefit is made or the requirement to~take part in an interview under regulation~4 or 7(2) arises or applies—
\begin{enumerate}\item[]
($a$) are engaged in remunerative work; or

\enlargethispage{-\baselineskip}

($b$) are claiming, or are entitled to, a jobseeker’s allowance,
\end{enumerate}
shall be exempt from the requirement to~take part in an interview.

(2) Paragraph (1)($b$)  shall not apply where—
\begin{enumerate}\item[]
($a$) a joint-claim couple (as defined for the purposes of section~1(4) of the Jobseekers Act 1995\footnote{1995 c.~18; the definition of “joint-claim couple” was inserted by the Welfare Reform and Pensions Act 1999, section~59 and Schedule 7, paragraph 2(4)($b$).}) have claimed a jobseeker’s allowance; and

($b$) a member of that couple is a person to~whom regulation~3D(1)($c$)  of the Jobseeker’s Allowance Regulations 1996\footnote{S.I.~1996/207; regulation 3D was inserted by S.I.~2000/1978 and amended by S.I.~2001/518.} (further circumstances in which a joint-claim couple may be entitled to~a joint-claim jobseeker’s allowance) applies.
\end{enumerate}

\subsection[9. Claims for two or more specified benefits]{Claims for two or more specified benefits}

9.  A person who would otherwise be required under these Regulations to~take part in interviews relating to~more than one specified benefit—
\begin{enumerate}\item[]
($a$) is only required to~take part in one interview; and

($b$) that interview counts for the purposes of all those benefits.
\end{enumerate}

\subsection[10. The interview]{The interview}

10.---(1)  The officer shall inform a person who is required to~take part in an interview of the place and~time of the interview.

(2) The officer may determine that an interview is to~take place in the person’s home where it would, in his opinion, be unreasonable to~expect that person to~attend elsewhere because that person’s personal circumstances are such that attending elsewhere would cause him undue inconvenience or endanger his health.

\subsection[11. Taking part in an interview]{Taking part in an interview}

11.---(1)  The officer shall determine whether a person has taken part in an interview.

(2) A person shall be regarded as having taken part in an interview if and~only if—
\begin{enumerate}\item[]
($a$) he attends for the interview at the place and~time notified to~him by the officer; and

($b$) he provides answers (where asked) to~questions and~appropriate information about—
\begin{enumerate}\item[]
(i) the level to~which he has pursued any educational qualifications;

\enlargethispage{-\baselineskip}

(ii) his employment history;

(iii) any vocational training he has undertaken;

(iv) any skills he has acquired which fit him for employment;

(v) any paid or unpaid employment he is engaged in;

(vi) any medical condition which, in his opinion, puts him at a disadvantage in obtaining employment; and

(vii) any caring or childcare responsibilities he has.
\end{enumerate}
\end{enumerate}

(3) A person who has not attained the age of 18 shall also be regarded as having taken part in an interview if he attends an interview with the Careers Service or with the Connexions Service at the place and~time notified to~him by an officer.

(4) Where an officer determines that a person has failed to~take part in an interview and~good cause has not been shown for that failure within five working days of the day on which the interview was to~take place, a relevant decision shall be made for the purposes of section~2B of the Social Security Administration Act 1992\footnote{1992 c.~5; section~2B was inserted by the Welfare Reform and Pensions Act 1999, section~57.}.

\subsection[12. Failure to~take part in an interview]{Failure to~take part in an interview}

12.---(1)  A person in respect of whom a relevant decision has been made in accordance with regulation~11(4) shall, subject to~paragraph~(12), suffer the consequences set out below.

(2) Those consequences are—
\begin{enumerate}\item[]
($a$) where the interview arose in connection with a claim for a specified benefit, that the person to~whom the claim relates is to~be regarded as not having made a claim for a specified benefit;

\enlargethispage{-\baselineskip}

($b$) where an interview which arose in connection with a claim for a specified benefit was deferred and~benefit became payable by virtue of regulation~7(3), that the person’s entitlement to~that benefit shall terminate from the first day of the next benefit week following the date on which the relevant decision was made;

($c$) where the claimant has an award of benefit and~the requirement for the interview arose under regulation~4, the claimant’s benefit shall be reduced as from the first day of the next benefit week following the day the relevant decision was made, by a sum equal (but subject to~paragraphs~(3) and~(4)) to~20 per cent.\ of the amount applicable on the date the deduction commences in respect of a single claimant for income support aged not less than 25.
\end{enumerate}

(3) Benefit reduced in accordance with paragraph~(2)($c$)  shall not be reduced below ten pence per week.

(4) Where two or more specified benefits are in payment to~a claimant, a deduction made in accordance with this regulation~shall be applied, except in a case to~which paragraph~(5) applies, to~the specified benefits in the following order of priority—
\begin{enumerate}\item[]
($a$) income support;

($b$) incapacity benefit;

($c$) any bereavement benefit;

($d$) invalid care allowance;

($e$) severe disablement allowance.
\end{enumerate}

(5) Where the amount of the reduction is greater than some (but not all) of the specified benefits listed in paragraph~(4), the reduction shall be made against the first benefit in that list which is the same as, or greater than, the amount of the reduction.

(6) For the purpose of determining whether a specified benefit is the same as, or greater than, the amount of the reduction for the purposes of paragraph~(5), ten pence shall be added to~the amount of the reduction.

(7) In a case where the whole of the reduction cannot be applied against any one specified benefit because no one benefit is the same as, or greater than, the amount of the reduction, the reduction shall be applied against the first benefit in payment in the list of priorities at paragraph~(4) and~so on against each benefit in turn until the whole of the reduction is exhausted or, if this is not possible, the whole of the specified benefits are exhausted, subject in each case to~ten pence remaining in payment.

(8) Where the rate of any specified benefit payable to~a claimant changes, the rules set out above for a reduction in the benefit payable shall be applied to~the new rates and~any adjustments to~the benefits against which the reductions are made shall take effect from the beginning of the first benefit week to~commence for that claimant following the change.

\enlargethispage{-\baselineskip}

(9) Where a claimant whose benefit has been reduced in accordance with this regulation~subsequently takes part in an interview, the reduction shall cease to~have effect on the first day of the benefit week in which the requirement to~take part in an interview was met.

(10) For the avoidance of doubt, a person who is regarded as not having made a claim for any benefit because he failed to~take part in an interview shall be required to~make a new claim in order to~establish entitlement to~any specified benefit.

(11) For the purposes of determining the amount of any benefit payable, a claimant shall be treated as receiving the amount of any specified benefit which would have been payable but for a reduction made in accordance with this regulation.

(12) The consequences set out in this regulation~shall not apply in the case of a person who brings new facts to~the notice of the Secretary of State within one month of the date on which the decision was notified and—
\begin{enumerate}\item[]
($a$) those facts could not reasonably have been brought to~the Secretary of State’s notice within five working days of the day on which the interview was to~take place; and

($b$) those facts show that he had good cause for his failure to~take part in the interview.
\end{enumerate}

(13) In paragraphs~(2) and~(12), the “decision” means the decision that the person failed without good cause to~take part in an interview.

\subsection[13. Circumstances where regulation~12 does not apply]{Circumstances where regulation~12 does not apply}

13.  The consequences of a failure to~take part in an interview set out in regulation~12 shall not apply where—
\begin{enumerate}\item[]
($a$) he is no longer required to~take part in an interview as a condition for continuing to~be entitled to~the full amount of benefit which is payable apart from these Regulations; or

($b$) the person attains the age of 60.
\end{enumerate}

\subsection[14. Good cause]{Good cause}

\enlargethispage{-\baselineskip}

14.  Matters to~be taken into~account in determining whether a person has shown good cause for his failure to~take part in an interview include—
\begin{enumerate}\item[]
($a$) that the person misunderstood the requirement to~take part in the interview due to~any learning, language or literacy difficulties of the person or any misleading information given to~the person by the officer;

($b$) that the person was attending a medical or dental appointment, or accompanying a person for whom the claimant has caring responsibilities to~such an appointment, and~that it would have been unreasonable, in the circumstances, to~rearrange the appointment;

($c$) that the person had difficulties with his normal mode of transport and~that no reasonable alternative was available;

($d$) that the established customs and~practices of the religion to~which the person belongs prevented him attending on that day or at that time;

($e$) that the person was attending an interview with an employer with a view to~obtaining employment;

($f$) that the person was actually pursuing employment opportunities as a self-employed earner;

($g$) that the person or a dependant of his or a person for whom he provides care suffered an accident, sudden illness or relapse of a chronic condition;%\looseness=-1

($h$) that he was attending the funeral of a close friend or relative on the day fixed for the interview;

($i$) that a disability from which the person suffers made it impracticable for him to~attend at the time fixed for the interview.
\end{enumerate}

\subsection[15. Appeals]{Appeals}

15.---(1)  This regulation~applies to~any relevant decision made under regulation~11(4) or any decision under section~10 of the 1998 Act superseding such a decision.%\looseness=-1

(2) This regulation~applies whether the decision is as originally made or as revised under section~9 of the 1998 Act.

(3) In the case of a decision to~which this regulation~applies, the person in respect of whom the decision was made shall have a right of appeal under section~12 of the 1998 Act to~an appeal tribunal.

\subsection[16. Revocations and~transitional provision]{Revocations and~transitional provision}

\enlargethispage{-\baselineskip}

16.---(1)  Subject to~paragraph~(2), the Social Security (Work-focused Interviews) Regulations 2000\footnote{S.I.~2000/897.} (“the 2000 Regulations”) and~the Social Security (Jobcentre Plus Interviews) Regulations 2001\footnote{S.I.~2001/3210.} (“the 2001 Regulations”) are hereby revoked to~the extent specified in Schedule 1.

(2) Notwithstanding paragraph~(1), both the 2000 Regulations (except for regulations 4, 5 and~12(2)($a$)  and~($b$)) and~the 2001 Regulations (except for regulations 3 and~11(2)($a$)  and~($b$)) shall continue to~apply as if these Regulations had not come into~force for the period specified in paragraph~(3) in the case of a person who, on the day before the day on which these Regulations come into~force, is both a relevant person and~entitled to~a specified benefit for the purposes of those Regulations.

(3) The period specified for the purposes of paragraph~(2) shall be the period beginning on the day on which these Regulations come to~force and~ending on the day on which the person—
\begin{enumerate}\item[]
($a$) ceases to~be a relevant person for the purposes of the 2000 Regulations or, as the case may be, the 2001 Regulations;

($b$) is not entitled to~any specified benefit for the purposes of those Regulations; or

($c$) attains the age of 60,
\end{enumerate}
whichever shall first occur.

\subsection[17. Amendments to~regulations]{Amendments to~regulations}

17.  The amendments to~regulations prescribed in Schedule 2 shall have effect. 

\bigskip

Signed 
by authority of the 
Secretary of State for~Work and~Pensions.
%I concur
%By authority of the Lord Chancellor

{\raggedleft
\emph{N.~Brown}\\*
%Secretary
Minister
%Parliamentary Under-Secretary 
of State,\\*Department 
for~Work and~Pensions
%Ministry of Justice
%Two of the Commissioners of Inland~Revenue

}

3rd July 2002

\small

\enlargethispage{-\baselineskip}

\part[Schedule 1 --- Revocations]{Schedule 1\\*Revocations}

\renewcommand\parthead{--- Schedule 1}

%\begin{tabulary}{\linewidth}{JJ}
\begin{longtable}{p{106.61865pt}p{259.3711pt}}
\hline
\itshape Regulations	& \itshape Extent of revocation\\
\hline
\endhead
\hline
\endlastfoot
\hbadness=10000 The Social Security (Work-focused Interviews) Regulations 2000	&

Regulation~2 save for the words “In these Regulations—” and~the definitions of “the Council Tax Benefit Regulations”, “the Claims and~Payments Regulations” and~“the Housing Benefit Regulations” in paragraph~(1);\\

&regulations 3 to~15;\\

&regulation~16(1) and~(2);\\

&Schedules 1 to~3;\\

&In Schedule 6, paragraphs~2($c$)  and~3($b$).\\

\hbadness=6348 The Social Security (Jobcentre Plus Interviews) Regulations 2001	
&
Regulation~2 save for paragraph~(6)($c$);\newline regulations 3 to~14;\newline In regulation~15, the words “, the Social Security (Work-focused Interviews) Regulations 2000”;\\

&Schedule 1;\\

&paragraph~2 of Schedule 2.\\
%\end{tabulary}
\end{longtable}

\part[Schedule 2 --- Amendments to regulations]{Schedule 2\\*Amendments to regulations}

\renewcommand\parthead{--- Schedule 2}

1.  For regulation~6A(5) of the Social Security (Claims and~Payments) Regulations 1987\footnote{S.I.~1987/1968; the relevant amending instrument is S.I.~2000/897.} (claims by persons subject to~work-focused interviews) there shall be substituted the following paragraph—
\begin{quotation}
“(5) In regulation~4 and~this regulation, “work-focused interview” means an interview which a person is required to~take part in under the Social Security (Jobcentre Plus Interviews) Regulations 2002.”.
\end{quotation}

\medskip

\enlargethispage{-\baselineskip}

2.  In the Housing Benefit (General) Regulations 1987\footnote{S.I.~1987/1971; the relevant amending instrument is S.I.~2000/897.}—
\begin{enumerate}\item[]
($a$) in regulation~71(7) (who may claim), for the words from the beginning to~the word “refers” there shall be substituted the words “Where the claim is made at an office displaying the \textsc{\lowercase{ONE}} logo”;

($b$) in regulation~72(4) (time and~manner in which claims are to~be made)—
\begin{enumerate}\item[]
(i) in sub-paragraph~($d$), for the words “and~is neither engaged in remunerative work nor residing in an area identified in Schedule 1 to~the Work-focused Interviews Regulations”, there shall be substituted the words “and~is not engaged in remunerative work”;

(ii) for sub-paragraph~($e$), there shall be substituted the following sub-paragraph—
\begin{quotation}
“($e$) may be sent or delivered, where the claimant has attained the age of 16 but not the age of 60, to~the office of a designated authority displaying the \textsc{\lowercase{ONE}} logo.”.
\end{quotation}
\end{enumerate}
\end{enumerate}

\medskip

3.  For the definition of “designated authority” in regulation~1(2) of the Child Support (Maintenance Assessment Procedure) Regulations 1992\footnote{S.I.~1992/1813; the relevant amending instruments are S.I.~1999/1047 and 2000/897.}, there shall be substituted the following definition—
\begin{quotation}
    ““designated authority” means—
\begin{enumerate}\item[]
    ($a$) 
    the Secretary of State;

    ($b$) 
    a person providing services to~the Secretary of State;

    ($c$) 
    a local authority;

    ($d$) 
    a person providing services to, or authorised to~exercise any functions of, any such authority;”. 
\end{enumerate}
\end{quotation}

\medskip

4.  In regulation~62(4) of the Council Tax Benefit (General) Regulations 1992\footnote{S.I.~1992/1814; the relevant amending instrument is S.I.~2000/897.} (time and~manner in which claims are to~be made)—
\begin{enumerate}\item[]
($a$) in sub-paragraph~($d$), for the words “and~is neither engaged in remunerative work nor residing in an area identified in Schedule 1 to~the Work-focused Interviews Regulations”, there shall be substituted the words “and~is not engaged in remunerative work”;

($b$) for sub-paragraph~($e$), there shall be substituted the following sub-paragraph—
\begin{quotation}
“($e$) may be sent or delivered, where the claimant has attained the age of 16 but not the age of 60, to~the office of a designated authority displaying the \textsc{\lowercase{ONE}} logo.”.
\end{quotation}
\end{enumerate}

\medskip

5.  For the definition of “designated authority” in regulation~1(2) of the Child Support Departure Direction and~Consequential Amendments Regulations 1996\footnote{S.I.~1996/2907; the relevant amending instruments are S.I.~1999/1047 and 2000/897.}, there shall be substituted the following definition—
\begin{quotation}
    ““designated authority” means—
\begin{enumerate}\item[]
    ($a$) 
    the Secretary of State;

    ($b$) 
    a person providing services to~the Secretary of State;

    ($c$) 
    a local authority; or

    ($d$) 
    a person providing services to, or authorised to~exercise any functions of, any such authority;”.
\end{enumerate}
\end{quotation}

\medskip 

6.  In the Social Security and~Child Support (Decisions and~Appeals) Regulations 1999\footnote{S.I.~1999/991; the relevant amending instrument is S.I.~2000/897.}—
\begin{enumerate}\item[]
($a$) in regulation~1(3) (interpretation)—
\begin{enumerate}\item[]
(i) for the definition of “designated authority” there shall be substituted the following definition—
\begin{quotation}
    ““designated authority” means—
\begin{enumerate}\item[]
    ($a$)     the Secretary of State;

    ($b$) 
    a person providing services to~the Secretary of State;

    ($c$) 
    a local authority;

    ($d$) 
    a person providing services to, or authorised to~exercise any functions of, any such authority;”; 
\end{enumerate}
\end{quotation}

(ii) for the definition of “work-focused interview”, there shall be substituted the following definition—
\begin{quotation}
““work-focused interview” means an interview which a person is required to~take part in under the Social Security (Jobcentre Plus Interviews) Regulations 2002;”;
\end{quotation}

(iii) the definition of “the Work-focused Interviews Regulations” shall be omitted;
\end{enumerate}

($b$) in regulation~3(11) (revision of decisions), for sub-paragraph~($f$)  there shall be substituted the following sub-paragraph—
\begin{quotation}
“($f$) in the case of a person who is, or would be, required to~take part in a work-focused interview, an office of the Department for Work and~Pensions which is designated by the Secretary of State as a Jobcentre Plus Office or an office of a designated authority which displays the \textsc{\lowercase{ONE}} logo.”.
\end{quotation}
\end{enumerate}

\medskip

7.  At the end of regulation~4($d$)  of the Social Security (Work-focused Interviews for Lone Parents) and~Miscellaneous Amendments Regulations 2000\footnote{S.I.~2000/1926; the relevant amending instrument is S.I.~2001/3210.} (circumstances where requirement to~take part in an interview does not apply) there shall be added the words “or the Social Security (Jobcentre Plus Interviews) Regulations 2002”.

\medskip

8.  In the Housing Benefit and~Council Tax Benefit (Decisions and~Appeals) Regulations 2001\footnote{S.I.~2001/1002.}—
\begin{enumerate}\item[]
($a$) in regulation~1(2) (interpretation)—
\begin{enumerate}\item[]
(i) the definitions of “designated authority”, “work-focused interview” and~of “the Work-focused Interviews Regulations” shall be omitted;

(ii) in the definition of “official error”, paragraph~($c$)  shall be omitted;
\end{enumerate}

($b$) in regulation~4(8) (revision of decisions), the words from “or, in a case” to~the end of the paragraph~shall be omitted;

($c$) in regulation~5(3)($c$)  (late application for a revision), the words from “or, in a case” to~the end of the sub-paragraph~shall be omitted;

($d$) in regulation~7 (decisions superseding earlier decisions)—
\begin{enumerate}\item[]
(i) paragraph~(2)($f$)  shall be omitted;

(ii) in paragraph~(7), the words from “or, in a case” to~the end of the paragraph~shall be omitted;
\end{enumerate}

($e$) in regulation~20(1)($c$)  (making of appeals and~applications), the words from “, or in a case” to~the end of the sub-paragraph~shall be omitted.
\end{enumerate}

\part{Explanatory Note}

\renewcommand\parthead{— Explanatory Note}

\subsection*{(This note is not part of the Regulations)}

These Regulations impose a requirement on persons who claim, or are entitled to, certain benefits (specified in regulation~2(1)) to~take part in a work-focused interview (“an interview”).

Regulation~3 specifies those persons who are required to~take part in an interview when claiming a specified benefit and~regulation~4 prescribes when persons are required to~take part in an interview as a condition of their continuing entitlement to~those benefits.

Regulation~5 prescribes the time when the interview is to~take place. Regulation~6 provides that the requirement to~take part in an interview can be waived where an interview would not be of assistance to~the person or it would not be appropriate in the circumstances of the case and~regulation~7 specifies that an interview can be deferred. Regulation~8 prescribes circumstances when a person is exempted from the requirement to~take part in an interview. Regulation~9 specifies when a requirement to~take part in two or more interviews is satisfied by the person taking part in a single interview.

Regulation~10 provides for the person to~be advised of the time and~place of the interview and~provides that an interview can take place in the person’s home if the interviewer considers that it would be unreasonable to~require that person to~attend elsewhere.

Regulation~11 prescribes circumstances as to~when a person is to~be regarded as having taken part in an interview and~regulation~12 details the consequences of a failure to~take part in an interview. Regulation~13 specifies the circumstances where those consequences to~not apply and~regulation~14 specifies the matters to~be taken into~account in determining whether a person had good cause for his failure to~take part in an interview.

Regulation~15 provides that a decision that a person has failed to~take part in an interview without good cause can be appealed to~an appeal tribunal under section~12 of the Social Security Act 1998 (c.~14).

Regulation~16 and~Schedule 1 revoke previous regulations, with savings and~a transitional provision, which imposed requirements on persons in certain areas to~take part in interviews.

Regulation~17 and~Schedule 2 make amendments to~other regulations which are consequential both on these Regulations and~the revocation and~transitional effect of the regulations referred to~in Schedule 1.

These Regulations do not impose a charge on business. 

\end{document}