\documentclass[12pt,a4paper]{article}

\newcommand\regstitle{The Social Security (Miscellaneous Amendments) (No.\ 2) Regulations 1998}

\newcommand\regsnumber{1998/865}

%\opt{newrules}{
\title{\regstitle}
%}

%\opt{2012rules}{
%\title{Child Maintenance and Other Payments Act 2008\\(2012 scheme version)}
%}

\author{S.I. 1998 No. 865}

\date{Made 19th March 1998\\Laid before Parliament 19th March 1998\\Coming into force 20th March 1998
}

%\opt{oldrules}{\newcommand\versionyear{1993}}
%\opt{newrules}{\newcommand\versionyear{2003}}
%\opt{2012rules}{\newcommand\versionyear{2012}}

\usepackage{csa-regs}

\setlength\headheight{27.57402pt}

\begin{document}

\maketitle

\begin{center}
\itshape This Statutory Instrument has been made in consequence of a defect in S.I.\ 1998/563 and is being issued free of charge to all known recipients of that Statutory Instrument. 
\end{center}

\noindent
The Secretary of State for Social Security, in exercise of powers conferred upon her by sections 64(1), 68(4)($c$), 70(4), 71(6), 146A, 147(1), (4) and (5) of the Social Security Contributions and Benefits Act 1992\footnote{\frenchspacing 1992 c. 4; section 146A was inserted by the Asylum and Immigration Act 1996 (c. 49); section 147(1) is an interpretation provision and is cited because of the meaning ascribed to the word “prescribed”.}, sections 10(1), (5)($a$)  and ($b$)  and (7) and 26(1) to (3) of the Child Support Act 1995\footnote{\frenchspacing 1995 c. 34.}, paragraph 7A of Schedule 2 to the Abolition of Domestic Rates (Scotland) Act 1987\footnote{\frenchspacing 1987 c. 47 (“the 1987 Act”); paragraph 7A was inserted by paragraph 36(1) of Schedule 12 to the Local Government Finance Act 1988 (c. 41). The 1987 Act was repealed by Schedule 14 to the Local Government Finance Act 1992 (c. 14) but paragraph 7A of Schedule 2 continues to have effect for the purposes of amending the Community Charges (Deductions from Income Support) (Scotland) Regulations 1989 (S.I. 1989/507) (S.59) by virtue of Article 2 of the Local Government Finance Act 1992 (Recovery of Community Charge) Saving Order 1993 (S.I. 1993/1780). Paragraph 7A of Schedule 2 to the 1987 Act as continued in effect by virtue of that Order, was amended by paragraph 10 of Schedule 2 to the Jobseekers Act 1995 (c. 18).}, section 146(6) of, and paragraph 6 of Schedule 4 to, the Local Government Finance Act 1988\footnote{\frenchspacing 1988 c. 41 (“the 1988 Act”); section 146(6) is cited because of the meaning given to the word “prescribed”. Paragraph 6 of Schedule 4 was repealed by Schedule 14 to the Local Government Finance Act 1992 (c. 14), but continues to have effect for the purposes of amending the Community Charges (Deductions from Income Support) (No. 2) Regulations 1990 (S.I. 1990/545) by virtue of Article 2 of the Local Government Finance Act 1992 (Recovery of Community Charge) Saving Order 1993 (S.I. 1993/1780). Paragraph 6 of Schedule 4 to the 1988 Act as continued in effect by virtue of that Order, was amended by paragraph 18 of Schedule 2 to the Jobseekers Act 1995.}, sections 24 and 30 of the Criminal Justice Act 1991\footnote{\frenchspacing 1991 c. 53.} and section 116(1) of, and paragraphs 1 and 6 of Schedule 4 and paragraph 6 of Schedule 8 to, the Local Government Finance Act 1992\footnote{\frenchspacing 1992 c. 14; paragraph 6 of Schedule 4 and paragraph 6 of Schedule 8 were amended by the Jobseekers Act 1995, Schedule 2, paragraphs 75 and 76 respectively. Section 116(1) is an interpretation provision and is cited because of the meaning given to the word “prescribed”.} and of all other powers enabling her in that behalf, and after agreement by the Social Security Advisory Committee that these Regulations should not be referred to it\footnote{\frenchspacing See sections 170 and 173(1)($b$) of the Social Security Administration Act 1992 (c. 5); paragraph 20 of Schedule 3 to the Child Support Act 1995 (c. 34) added that Act to the list of “relevant enactments” in respect of which regulations must normally be referred to the Committee.}, hereby makes the following Regulations:  


{\sloppy

\tableofcontents

}

\setcounter{secnumdepth}{-2}

\subsection[1. Citation and commencement]{Citation and commencement}

1.  These Regulations may be cited as the Social Security (Miscellaneous Amendments) (No.\ 2) Regulations 1998 and shall come into force on 20th March 1998.

\subsection[2. Amendment of the Social Security (Miscellaneous Amendments) Regulations 1998]{Amendment of the Social Security (Miscellaneous Amendments) Regulations 1998}

2.  In regulation 1(1) of the Social Security (Miscellaneous Amendments) Regulations 1998\footnote{\frenchspacing S.I. 1998/563.} for the words “and regulation 16” there shall be substituted the words “and regulations 2, 3, 16, 18(2)($a$)  and ($d$)  to ($g$)  and 18(1) in so far as it relates those sub-paragraphs”. 

\bigskip

Signed by authority of the Secretary of State for Social Security.

{\raggedleft
\emph{John Y.~Denham}\\*Parliamentary Under-Secretary of
State,\\*Department of Social Security

}

19th March 1998

\small

\part{Explanatory Note}

\renewcommand\parthead{--- Explanatory Note}

\subsection*{(This note is not part of the Regulations)}

These Regulations amend the Social Security (Miscellaneous Amendments) Regulations 1998 (S.I.\ 1998/563) by inserting a commencement date for regulations 2, 3 and 18 which was omitted from those Regulations.

These Regulations do not impose a charge on business. 

\end{document}