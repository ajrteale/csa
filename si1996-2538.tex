\documentclass[a4paper]{article}

\usepackage[welsh,english]{babel}

\usepackage[utf8]{inputenc}
\usepackage[T1]{fontenc}

\usepackage{textcomp}

%\usepackage[2012rules]{optional}

\usepackage[osf]{mathpazo}

\usepackage{perpage} %the perpage package
\MakePerPage{footnote} %the perpage package command
\renewcommand{\thefootnote}{\fnsymbol{footnote}}

\usepackage[perpage,para,symbol]{footmisc}

%\opt{newrules}{
\title{The Social Security and Child Support (Jobseeker’s Allowance) (Miscellaneous Amendments) Regulations 1996}
%}

%\opt{2012rules}{
%\title{Child Maintenance and Other Payments Act 2008\\(2012 scheme version)}
%}

\author{S.I. 1996 No. 2538}

\date{Made 7th October 1996\\Laid before Parliament 8th October 1996\\Coming into force 28th October 1996}

%\opt{oldrules}{\newcommand\versionyear{1993}}
%\opt{newrules}{\newcommand\versionyear{2003}}
%\opt{2012rules}{\newcommand\versionyear{2012}}

\usepackage{fancyhdr}
\pagestyle{fancy}
\fancyhead[L]{}
\fancyhead[C]{\itshape The Social Security and Child Support (Jobseeker’s Allowance) (Miscellaneous Amendments) Regulations 1996 (S.I.~1996/2538) \parthead%\phantom{...}% (\versionyear{} scheme version)
}
\fancyhead[R]{}
\fancyfoot[C]{\thepage}
\newcommand{\parthead}{}

\usepackage{array}
\usepackage{multirow}
\usepackage[debugshow]{tabulary}
\usepackage{longtable}
\usepackage{multicol}
\usepackage{lettrine}

\usepackage[colorlinks=true]{hyperref}
\usepackage{microtype}

\hyphenation{Aw-dur-dod}
\hyphenation{bank-rupt-cy}
\hyphenation{Ec-cles-ton}
\hyphenation{Eux-ton}
\hyphenation{Hogh-ton}
\hyphenation{Pres-ton}
\hyphenation{Pru-den-tial}
\hyphenation{Riv-ing-ton}

\newcolumntype{x}[1]
	{>{\raggedright}p{#1}}
\newcommand{\tn}{\tabularnewline}
\setlength\tymin{50pt}

\newcommand\amendment[1]{\subsubsection*{Notes}{\itshape\frenchspacing\footnotesize #1 \par}}

\setlength\headheight{22.87003pt}

\newcommand\fnote[1]{\footnote{\frenchspacing #1}}

\begin{document}

\maketitle

\noindent
The Secretary of State for Education and Employment, in relation to regulation 2(2)($a$) and ($b$) and (3) and the Secretary of State for Social Security, in relation to the remainder of the Regulations, in exercise of the powers conferred by sections 3(1), 4(5), 5(3), 8, 12(4), 20(4), 21, 26, 35(1), 36(2) to (5) and 40 of, and paragraph 3 of Schedule 1 to, the Jobseekers Act 1995\footnote{\frenchspacing 1995 c. 18; section 35(1) is an interpretation provision and is cited because of the meanings ascribed to the words “prescribed” and “regulations”.}, and sections 61 and 71 of the Social Security Administration Act 1992\footnote{\frenchspacing 1992 c. 5.} and of all other powers enabling each of them in that behalf, and after agreement by the Social Security Advisory Committee that proposals in respect of these Regulations should not be referred to it\footnote{\frenchspacing \emph{See} the Social Security Administration Act 1992, section 173(1)($b$).} hereby make the following Regulations:

{\sloppy

\tableofcontents

}

\setcounter{secnumdepth}{-2}

\subsection[1. Citation and commencement]{Citation and commencement}

1.  These Regulations may be cited as the Social Security and Child Support (Jobseeker’s Allowance) (Miscellaneous Amendments) Regulations 1996 and shall come into force on 28th October 1996.

\subsection[2. Amendment of the Jobseeker’s Allowance Regulations 1996]{Amendment of the Jobseeker’s Allowance Regulations 1996}

2.—(1) The Jobseeker’s Allowance Regulations 1996\footnote{\frenchspacing S.I. 1996/207.} shall be amended in accordance with the following provisions of this regulation.

(2) In regulation 1(3) (definitions)—
\begin{enumerate}\item[]
($a$) in the definition of “benefit week”\footnote{\frenchspacing The definition of “benefit week” was amended by S.I. 1996/1517.}—
\begin{enumerate}\item[]
(i) for head ($a$) there shall be substituted the following head—
\begin{quotation}
“($a$) where—
\begin{enumerate}\item[]
(i) the Secretary of State requires attendance otherwise than at regular two weekly intervals, or in the case of a claimant who is paid benefit in accordance with Part III, other than regulation 20A, of the Claims and Payments Regulations at the time he provides a signed declaration as referred to in regulation 24(6), the “benefit week” ends on such day as the Secretary of State may specify in a notice in writing given or sent to the claimant;

(ii) in accordance with an award of income support that includes the relevant day, the “benefit week” ends on a Saturday, the “benefit week” shall end on a Saturday, or on such other day as the Secretary of State may specify in a notice in writing given or sent to the claimant; or

(iii) in accordance with an award of unemployment benefit that includes the relevant day, the claimant is paid benefit in respect of a period of seven days ending on the week-day specified in a written notice given to him by the Secretary of State for the purpose of his claiming unemployment benefit, and that day is a Saturday, the “benefit week” shall end on a Saturday or on such other day as the Secretary of State may specify in a notice in writing given or sent to the claimant;”;
\end{enumerate}
\end{quotation}

(ii) in head ($b$) for the words “less than a week;” there shall be substituted the words—
\begin{quotation}
\begin{enumerate}\item[]
“less than a week,
\end{enumerate}\noindent
and in this definition “relevant day” has the meaning it has in the Jobseeker’s Allowance (Transitional Provisions) Regulations 1995\footnote{\frenchspacing S.I. 1995/3276.}.”;
\end{quotation}
\end{enumerate}

($b$) in the definition of “close relative” the word “, IV” shall be omitted;

($c$) after the definition of “earnings” the following definitions shall be inserted—
\begin{quotation}
““earnings top-up” means the allowance paid by the Secretary of State under the Earnings Top-up Scheme;

“the Earnings Top-up Scheme” means the Earnings Top-up Scheme 1996\footnote{\frenchspacing This Scheme, which applies only in certain areas of Great Britain, is an extra-statutory Scheme introduced by the Secretary of State for Social Security having effect on 8th October 1996. Copies of the Rules of the Scheme may be obtained from the Customer Services Manager, Earnings Top-up, Norcross, Blackpool FY5 3TA and will be available for inspection at the Department of Social Security, 9th Floor, Adelphi, 1--11 John Adam Street, London WC2N 6HT.} as amended from time to time;”.
\end{quotation}
\end{enumerate}

(3) In regulation 4 (interpretation of Parts II, IV and V), in the definition of “close relative” after the word “means” there shall be inserted the words “,~except in Part IV,”.

(4) In regulation 47 (jobseeking period) after paragraph (2) the following paragraph shall be inserted—
\begin{quotation}
“(2A) Any period in which a claimant is entitled to a jobseeker’s allowance in accordance with regulation 11(3) of the Jobseeker’s Allowance (Transitional Provisions) Regulations 1995\footnote{\frenchspacing S.I. 1995/3276; relevant amending instrument is S.I. 1996/1515.} shall, for the purposes of paragraph (1), be treated as a period in which he satisfies the conditions specified in paragraphs ($a$) to ($c$) and ($e$) to ($i$) of subsection (2) of section 1.”.
\end{quotation}

(5) After regulation 47 (jobseeking period) the following regulation shall be inserted—
\begin{quotation}
\subsection*{“Jobseeking periods: periods of interruption of employment}

47A.  For the purposes of section 2(4)($b$)(i) and for determining any waiting days—
\begin{enumerate}\item[]
($a$) where a jobseeking period or a linked period commences on 7th October 1996, any period of interruption of employment ending within the 8 weeks preceding the day the jobseeking period or linked period commenced,

($b$) where a jobseeking period or a linked period commences after 7th October 1996, any period of interruption of employment ending within the 12 weeks preceding that date,
shall be treated as a jobseeking period.”.
\end{enumerate}
\end{quotation}

(6) In regulation 48 (linking periods)—
\begin{enumerate}\item[]
($a$) in paragraph (2) after sub-paragraph ($d$) the following sub-paragraph shall be added—
\begin{quotation}
“($e$) a period which includes 6th October during which the claimant attends court in response to a summons for jury service and which was immediately preceded by a period of entitlement to unemployment benefit.”;
\end{quotation}

($b$) after paragraph (2) the following paragraph shall be inserted—
\begin{quotation}
“(2A) A period is a linked period for the purposes of section 2(4)($b$)(ii) of the Act only where it ends within 12 weeks or less of the commencement of a jobseeking period or of some other linked period.”.
\end{quotation}
\end{enumerate}

(7) In regulation 85 (special cases)—
\begin{enumerate}\item[]
($a$) in paragraph (4) at the beginning there shall be inserted the words “Subject to paragraph (4A),”;

($b$) after paragraph (4) there shall be inserted the following paragraph—
\begin{quotation}
“(4A) In paragraph (4) “person from abroad” does not include any person in Great Britain who left the territory of Montserrat after 1st November 1995 because of the effect on that territory of a volcanic eruption.”;
\end{quotation}

($c$) for paragraph (5) the following paragraph shall be substituted—
\begin{quotation}
“(5) A person shall continue to be treated as being in residential accommodation within the meaning of paragraph (4) if—
\begin{enumerate}\item[]
($a$) he is in, or only temporarily absent from, such residential accommodation, and the same accommodation subsequently becomes a residential care home for so long as he remains in that accommodation; or

($b$) on 31st March 1993 he was in, or only temporarily absent from, accommodation of a kind mentioned in regulation 21(3B) to (3E) of the Income Support Regulations\footnote{\frenchspacing Relevant amending instruments are S.I. 1993/518, 2119, 1994/2139 and 1995/516.}.”.
\end{enumerate}
\end{quotation}
\end{enumerate}

(8) In regulation 129 (date on which child support maintenance is to be treated as paid)—
\begin{enumerate}\item[]
($a$) in paragraph (1)($a$) at the beginning there shall be inserted the words “subject to sub-paragraph ($aa$),”;

($b$) after paragraph (1)($a$) there shall be inserted the following sub-paragraph—
\begin{quotation}
“($aa$) in the case of any amount of a payment which represents arrears of maintenance for a week prior to the benefit week in which the claimant first became entitled to an income-based jobseeker’s allowance, on the day of the week in which it became due which corresponds to the first day of the benefit week;”;
\end{quotation}

($c$) for paragraph (2) the following paragraph shall be substituted—
\begin{quotation}
“(2) Where a payment to which paragraph (1)($b$) refers is made to the Secretary of State and then transmitted to the person entitled to receive it, the payment shall be treated as paid on the first day of the benefit week in which it is transmitted or, where it is not practicable to take it into account in that week, the first day of the first succeeding benefit week in which it is practicable to take the payment into account.”.
\end{quotation}
\end{enumerate}

(9) In regulation 141 (circumstances in which an income-based jobseeker’s allowance is payable to a person in hardship) in paragraph (2) for the words “a reason for the delay” there shall be substituted the words “the sole reason for the delay” and at the end the words “provided he satisfies the conditions of entitlement specified in paragraph ($d$)(ii) of subsection (2) of section 1.” shall be added.

(10) In regulation 142 (further circumstances in which an income-based jobseeker’s allowance is payable to a person in hardship) in paragraph (2)—
\begin{enumerate}\item[]
($a$) for sub-paragraph ($a$) there shall be substituted the following sub-paragraph—
\begin{quotation}
“($a$) the 15th day following the date of claim disregarding any waiting days; or”;
\end{quotation}

($b$) sub-paragraph ($b$) shall be omitted;

($c$) for the words “a reason for the delay” there shall be substituted the words “the sole reason for the delay” and at the end the words “provided he satisfies the conditions of entitlement specified in paragraph ($d$)(ii) of subsection (2) of section 1.” shall be added.
\end{enumerate}

(11) In Schedule 1 (applicable amounts)—
\begin{enumerate}\item[]
($a$) for sub-paragraph (1)($b$) of paragraph 8 there shall be substituted the following sub-paragraph—
\begin{quotation}
“($b$) for any period spent by a claimant in undertaking a course of training or instruction provided or approved by the Secretary of State for Education and Employment under section 2 of the Employment and Training Act 1973\footnote{\frenchspacing 1973 c. 50; as amended by sections 9 and 11 and Schedule 2, Part II paragraph 9 and Schedule 3 to the Employment and Training Act 1981 (c. 57).}, or by Scottish Enterprise or Highlands and Islands Enterprise under section 2 of the Enterprise and New Towns (Scotland) Act 1990\footnote{\frenchspacing 1990 c. 35.} or for any period during which he is in receipt of a training allowance.”;
\end{quotation}

($b$) in paragraph 12, for sub-head (ii) of paragraph (1)($c$) there shall be substituted the following sub-head—
\begin{quotation}
“(ii) satisfies the requirements of either sub-head (i) or (ii) of paragraph 12(1)($a$); and”.
\end{quotation}
\end{enumerate}

(12) In Schedule 2 (housing costs)—
\begin{enumerate}\item[]
($a$) in paragraph 4, in sub-paragraph (2)($b$) for the word “proceeding” there shall be substituted the word “preceding”;

($b$) in paragraph 13(5), after head ($b$) there shall be inserted the following head—
\begin{quotation}
“($bb$) a personal rate of contribution-based jobseeker’s allowance that is equal to, or exceeds, the applicable amount in his case; or”;
\end{quotation}

($c$) in paragraph 17—
\begin{enumerate}\item[]
(i) in sub-paragraph (2)($b$) for the words “£10.00” there shall be substituted the words “£12.00”;

(ii) in sub-paragraph (7)($b$) for the words “an allowance payable”\footnote{\frenchspacing Relevant amending instrument is S.I. 1996/1517.} to the end there shall be substituted the words “a training allowance paid in connection with a Youth Training Scheme established under section 2 of the Employment and Training Act 1973 or section 2 of the Enterprise and New Towns (Scotland) Act 1990; or”.
\end{enumerate}
\end{enumerate}

(13) In Schedule 5 (applicable amounts in special cases)—
\begin{enumerate}\item[]
($a$) in paragraph 7 of Column (2), for the word “£13.35” there shall be substituted the word “£13.75”;

($b$) in paragraph 13(2) of Column (1)—
\begin{enumerate}\item[]
(i) the word “($b$)” shall be omitted;

(ii) for the word “($c$)” there shall be substituted the word “($b$)”; and

(iii) at the end the words “had claimed a jobseeker’s allowance.” shall be added.
\end{enumerate}
\end{enumerate}

(14) In Schedule 8 (capital to be disregarded), after sub-paragraph ($b$) of paragraph 12 the following sub-paragraph shall be inserted—
\begin{quotation}
“($c$) any allowance paid by the Secretary of State under the Earnings Top-up Scheme,”.
\end{quotation}

\subsection[3. Amendment of the Jobseeker’s Allowance (Transitional Provisions) Regulations 1995]{Amendment of the Jobseeker’s Allowance (Transitional Provisions) Regulations 1995}

3.—(1) The Jobseeker’s Allowance (Transitional Provisions) Regulations 1995\footnote{\frenchspacing S.I. 1995/3276; relevant amending instruments are S.I. 1996/1515 and 1996/2519.} shall be amended in accordance with the following provisions of this regulation.

(2) In regulation 1(2) in the definition of “training allowance” after the words “Employment and Training Act 1973” there shall be inserted the words “or section 2 of the Enterprise and New Towns (Scotland) Act 1990\footnote{\frenchspacing 1990 c. 35.},”.

(3) In regulation (5A) (Transition from Unemployment Benefit to a Jobseeker’s Allowance: further provisions) for paragraph (2) there shall be substituted the following paragraph—
\begin{quotation}
“(2) For the purposes of paragraph (1), a person who is disqualified for receiving unemployment benefit in accordance with section 28 of the Benefits Act as in force on 6th October 1996 for the benefit week that includes 7th October 1996 shall be treated as having an award of unemployment benefit for that week.”.
\end{quotation}

(4) In regulation 14 (claimants subject to disqualification or reduction in benefit) after paragraph (1) the following paragraph shall be inserted—
\begin{quotation}
“(1A) A period of disqualification for receiving unemployment benefit as referred to in paragraph (1) shall be treated as a period during which a contribution-based jobseeker’s allowance was not payable to the claimant under section 19 and days during that period shall be treated as days of entitlement to a contribution-based jobseeker’s allowance for the purposes of section 5(1).”.
\end{quotation}

\subsection[4. Amendment of the Social Security (Back to Work Bonus) Regulations 1996]{Amendment of the Social Security (Back to Work Bonus) Regulations 1996}

4.—(1) The Social Security (Back to Work Bonus) Regulations 1996\footnote{\frenchspacing S.I. 1996/193; relevant amending instrument is S.I. 1996/1511.} shall be amended in accordance with the following provisions of this regulation.

(2) In regulation 1(2) (definitions) in the definition of “training allowance” after the words “Employment and Training Act 1973” there shall be inserted the words “or section 2 of the Enterprise and New Towns (Scotland) Act 1990\footnote{\frenchspacing 1990 c. 35.},”.

(3) In regulation 5 (periods of entitlement which do not qualify)—
\begin{enumerate}\item[]
($a$) in paragraph (2) for the words “sums earned” there shall be substituted the words “earnings in any benefit week”;

($b$) after paragraph (2) the following paragraph shall be inserted—
\begin{quotation}
“(2A) Paragraph (2)—
\begin{enumerate}\item[]
($a$) shall not apply where during a benefit week there are days of entitlement to a qualifying benefit; and

($b$) the formula set out in regulation 8(1)($c$) shall apply to any earnings in the benefit week in which those days fall, except that “N” shall represent the number of days of entitlement to the qualifying benefit.”.
\end{enumerate}
\end{quotation}
\end{enumerate}

(4) In regulation 7 (requirements for a bonus) in both sub-paragraphs ($b$)(i) and ($c$)(i) of paragraph (7), for the words “claims, or whose partner claims” there shall be substituted the words “becomes entitled to, or whose partner becomes entitled to,”.

(5) In regulation 15 (single claimants who are couples), in paragraph (4) after the words “paragraph (1)($a$)” there shall be inserted the words “or (1A)($a$)”.

(6) In regulation 17 (persons attaining pensionable age)—
\begin{enumerate}\item[]
($a$) after paragraph (3) the following paragraph shall be inserted—
\begin{quotation}
“(3A) Where a person who ceases to be entitled to a jobseeker’s allowance after attaining the age of 60 becomes entitled to income support within 12 weeks of ceasing to be entitled to a jobseeker’s allowance, or after an intervening period as provided for in regulation 2, he shall be entitled to a bonus in accordance with paragraph (1)($a$) on the day he claims income support, and notwithstanding paragraph (3) his bonus period shall be treated as ending on the day he claims income support.”;
\end{quotation}

($b$) in paragraph (4) for the words “12 weeks of” there shall be substituted the words “12 weeks before”;

($c$) after paragraph (4) the following paragraph shall be inserted—
\begin{quotation}
“(4A) Where a person to whom paragraph (4) applies becomes entitled to income support within the period of 12 weeks after he attains the age of 60, or, as the case may be, pensionable age, or after an intervening period as provided for in regulation 2, he shall be entitled to a bonus notwithstanding his failure to satisfy any one of the conditions specified in regulation 7.”.
\end{quotation}
\end{enumerate}

(7) In regulation 21 (share fishermen) after the words “share fishermen” there shall be inserted the words “who are entitled to a contribution-based jobseeker’s allowance,”.

\subsection[5. Amendment of the Social Security (General Benefit) Regulations 1982]{Amendment of the Social Security (General Benefit) Regulations 1982}

\begin{sloppypar}
5.—(1) The Social Security (General Benefit) Regulations 1982\footnote{\frenchspacing S.I. 1982/1408; relevant amending instruments are S.I. 1984/1259 and 1987/1968.} shall be amended in accordance with the following provisions of this regulation.
\end{sloppypar}

(2) In regulation 9, (payment of benefit and suspension of payments pending a decision on appeals or references, arrears and repayments) after paragraph (5) the following paragraphs shall be inserted—
\begin{quotation}
“(5A) Where a person—
\begin{enumerate}\item[]
($a$) has received a contribution-based jobseeker’s allowance in respect of one or more days in one or more periods of entitlement to a jobseeker’s allowance; and

($b$) is subsequently awarded a contribution-based jobseeker’s allowance in respect of one or more days which fell before the days mentioned in sub-paragraph ($a$) (“the earlier period”); and

($c$) in consequence of the award mentioned in sub-paragraph ($b$) the number of days on which a person was entitled to a contribution-based jobseeker’s allowance exceeds the number of days specified for the purposes of section 5(1) of the Jobseekers Act 1995 (duration of a contributions-based jobseeker’s allowance),
\end{enumerate}
then any benefit which would, but for this provision, have become overpaid if the amount due under the subsequent award was paid shall be treated as having been paid in respect of the earlier period and the amount due to be paid under the subsequent award shall be reduced accordingly.

(5B) Where a person—
\begin{enumerate}\item[]
($a$) has received a contribution-based jobseeker’s allowance in respect of one or more days in one or more periods of entitlement to a jobseeker’s allowance;

($b$) is subsequently awarded unemployment benefit in respect of one or more days that fell before 7th October 1996 or in the benefit week that includes 7th October 1996 (“the earlier period”); and

($c$) in consequence of the award mentioned in sub-paragraph ($b$) the number of days on which a person was entitled to a contribution-based jobseeker’s allowance exceeds the number of days specified for the purposes of section 5(1) of the Jobseekers Act 1995 (duration of a contribution-based jobseeker’s allowance) or regulation 7(3) (claims for entitlement to a jobseeker’s allowance) of the Jobseeker’s Allowance (Transitional Provisions) Regulations 1995\footnote{\frenchspacing S.I. 1995/3276; relevant amending instrument is S.I. 1996/1515.},
\end{enumerate}
then any benefit which would, but for this provision, have become overpaid if the amount due under the subsequent award was paid shall be treated as having been paid in respect of the earlier period and the amount due under the subsequent award shall be reduced accordingly.

(5C) Where on appeal or review a decision is reversed or varied or revised and by reason thereof any sum on account of a contribution-based jobseeker’s allowance is shown to have been paid to any person in respect of days for which he was not entitled to it, then, in determining for the purposes of section 5(1) of the Jobseekers Act 1995 whether that person has exhausted his right to that benefit and what is the last day for which he was entitled to it—
\begin{enumerate}\item[]
($a$) any period for which such sum has been paid in pursuance of the original decision shall be treated as if it was a period for which that person was entitled to that benefit notwithstanding that that period is not a period of entitlement to a contribution-based jobseeker’s allowance;

($b$) where any sum has been so paid to such a person and that sum or any part thereof is recovered, then there shall be excluded for the purposes of the said determination under section 5(1) of the Jobseekers Act 1995 a number of days (to the nearest whole number) equal to the number to be obtained by dividing the amount recovered by one seventh (rounded to the nearest penny) of the weekly rate at which benefit was paid.
\end{enumerate}

(5D) Paragraph (5C) shall not apply to a period for which there would have been entitlement to a contribution-based jobseeker’s allowance but for a payment by the Secretary of State in accordance with section 182 of the Employment Rights Act 1996\footnote{\frenchspacing 1996 c. 18.} (employee’s rights on insolvency of employer), in respect of a sum owed by that person’s former employer, where the Secretary of State, in calculating the payment, has made a deduction from that sum on account of any contribution-based jobseeker’s allowance received.”.
\end{quotation}

\subsection[6. Amendment of the Child Support (Maintenance Assessment Procedure) Regulations 1992]{Amendment of the Child Support (Maintenance Assessment Procedure) Regulations 1992}

6.—(1) The Child Support (Maintenance Assessment Procedure) Regulations 1992\footnote{\frenchspacing S.I. 1992/1813.} shall be amended in accordance with the following provisions of this regulation.

(2) In regulation 37 (modification of reduction under a reduced benefit direction to preserve minimum entitlement to relevant benefit)—
\begin{enumerate}\item[]
($a$) in paragraph ($aa$)\footnote{\frenchspacing Paragraph ($aa$) was inserted by S.I. 1996/1345.} for the words “26A(10) of those Regulations;” there shall be substituted the words “87A of the Jobseeker’s Allowance Regulations 1996\footnote{\frenchspacing S.I. 1996/207; regulation 87A was inserted by S.I. 1996/1517.};”;

($b$) in paragraph ($b$) at the end for the words “of those regulations” there shall be substituted the words “of the Social Security (Claims and Payments) Regulations 1987”.
\end{enumerate}

(3) In regulation 40A\footnote{\frenchspacing Regulation 40A was inserted by S.I. 1995/3261 from 22.1.96.} in both paragraphs (1) and (3) after the words “income support” there shall be inserted the words “or an income-based jobseeker’s allowance”.

\subsection[7. Revocations]{Revocations}

7.  Regulation 27 of the Jobseeker’s Allowance and Income Support (General) (Amendment) Regulations 1996\footnote{\frenchspacing S.I. 1996/1517.} is revoked.

\bigskip

Signed in relation to regulation 2(2)($a$) and ($b$) and (3) of the Regulations by authority of the Secretary of State for Education and Employment.

{\raggedleft
\emph{Eric Forth}\\*Minister of State,\\*Department for Education and Employment

}

7th October 1996

\bigskip

Signed 
in relation to the remainder of the regulations
by authority of the Secretary of State for Social Security.

{\raggedleft
\emph{Roger Evans}\\*Parliamentary Under-Secretary of State,\\*Department of Social Security

}

7th October 1996

\bigskip

\part{Explanatory Note}

\renewcommand\parthead{--- Explanatory Note}

\subsection*{(This note is not part of the Regulations)}

These Regulations amend the Jobseeker’s Allowance Regulations 1996 (S.I.\ 1996/207), the Jobseeker’s Allowance (Transitional Provisions) Regulations 1995 (S.I.\ 1995/3276), the Social Security (Back to Work Bonus) Regulations 1996 (S.I.\ 1996/193), the Social Security (General Benefit) Regulations 1982 (S.I.\ 1982/1408) and the Child Support (Maintenance Assessment Procedure) Regulations 1992 (S.I.\ 1992/1813).

  Regulation 2 makes a number of amendments to the Jobseeker’s Allowance Regulations;
\begin{itemize}
\item the definition of “benefit week” is amended to provide for jobseekers who have a Saturday benefit week ending,
\item the definition of “close relative” is amended in relation to young persons,
\item a definition of the Earnings Top-up Scheme is inserted, and arrears of earnings top-up are to be treated as capital and disregarded when assessing jobseeker’s allowance,
a period of interruption of employment for unemployment benefit purposes is included as a “linked period”,
\item people coming to Great Britain from Montserrat because of the volcanic activity there are excluded from the definition of “persons from abroad”,
\item a person in residential accommodation provided by a local authority who is still in that accommodation when it becomes a residential care home for the purposes of the Regulations will be treated as being in local authority residential accommodation, notwithstanding that the local authority may no longer be under a duty to provide or make arrangements for providing accommodation for him,
\item the date child support maintenance is treated as paid is altered, as are the dates on which hardship payments are made,
\item days of entitlement to a contribution-based jobseeker’s allowance can count towards the qualifying period for help with mortgage interest available in respect of income-based jobseeker’s allowance,
\item a definition of training and training allowance is added, and the amount deducted in respect of the earnings of a non-dependant is increased to £12 to align jobseeker’s allowance with income support.
\end{itemize}

  Regulation 3 makes a number of amendments to the Jobseeker’s Allowance (Transitional Provisions) Regulations;
\begin{itemize}
\item the definition of “training allowance” is amended,
\item periods of disqualification for receiving unemployment benefit are treated as counting towards the maximum number of days a jobseeker’s allowance is payable.
\end{itemize}

  Regulation 4 makes a number of amendments to the Social Security (Back to Work Bonus) Regulations 1996;
\begin{itemize}
\item the definition of “training allowance” is amended,
\item the provisions on whether earnings should be taken into account are amended,
\item technical amendments are made to regulations 7 and 15,
\item two further circumstances in which a person aged 60 or over who claims income support may be entitled to a bonus are added to regulation 17,
\item the earnings to be taken into account in respect of share fishermen are limited to those earned whilst entitled to contribution-based jobseeker’s allowance.
\end{itemize}

  The Social Security (General Benefit) Regulations are amended so that rules on the payment of benefit and suspension of payments pending a decision on appeals or references, and arrears and repayment similar to those that apply to unemployment benefit will apply to a jobseeker’s allowance.

  The Child Support (Maintenance Assessment Procedure) Regulations are amended so that where benefit is reduced under a reduced benefit direction, the minimum amount of a jobseeker’s allowance which is payable shall be untouched, and the reduced benefit direction may be suspended when specified deductions are made from a jobseeker’s allowance.

  These Regulations revoke regulation 27 of the Jobseeker’s Allowance and Income Support (General) (Amendment) Regulations 1996.

  A copy of the Rules of the Earnings Top-up Scheme may be obtained from:
 The Customer Services Manager,
Earnings Top-up,
Norcross,
Blackpool {\scshape FY}5 3TA.

  These Regulations do not impose a charge on businesses.

\end{document}