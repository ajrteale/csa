\documentclass[12pt,a4paper]{article}

\newcommand\regstitle{The Child Benefit, Child Support and Social Security (Miscellaneous Amendments) Regulations 1996}

\newcommand\regsnumber{1996/1803}

%\opt{newrules}{
\title{\regstitle}
%}

%\opt{2012rules}{
%\title{Child Maintenance and Other Payments Act 2008\\(2012 scheme version)}
%}

\author{S.I. 1996 No. 1803}

\date{Made 8th July 1996\\Coming into force in accordance with regulation 1
}

%\opt{oldrules}{\newcommand\versionyear{1993}}
%\opt{newrules}{\newcommand\versionyear{2003}}
%\opt{2012rules}{\newcommand\versionyear{2012}}

\usepackage{csa-regs}

\setlength\headheight{27.57402pt}

\begin{document}

\maketitle

\noindent
Whereas a draft of this instrument was laid before Parliament in accordance with section 52(2) of the Child Support Act 1991\footnote{\frenchspacing 1991 c. 48.}, section 176(1) of the Social Security Contributions and Benefits Act 1992\footnote{\frenchspacing 1992 c. 4.} and under section 37(2) of the Jobseekers Act 1995\footnote{\frenchspacing 1995 c. 18.} and approved by resolution of each House of Parliament;

 Now, therefore, the Secretary of State for Social Security, in conjunction with the Treasury\footnote{\frenchspacing \emph{See} section 145(5) of the Social Security Contributions and Benefits Act 1992 (c. 4).}, in exercise of the powers conferred by sections 11(2), 43(1)($b$), 51, 52(4) and 54 of, and paragraphs 1(3) and (5), 2(1), 4(3), 5(1) and (2), 6(2) and 9 of Schedule 1 to, the Child Support Act 1991\footnote{\frenchspacing Section 54 is an interpretation provision and is cited because of the meaning assigned to the word “prescribed”.}, sections 123(1)($a$), ($d$) and ($e$), 135(1), 136(4) and (5)($a$) and ($b$), 137(1) and (2)($i$), 144(1) and (2), 145(1), 147(1) and 175(1), (3) and (4) of, and paragraph 4 of Schedule 9 to, the Social Security Contributions and Benefits Act 1992\footnote{\frenchspacing Sections 123, 135 and 137 are amended to have effect in relation to council tax benefit by Schedule 9 to the Local Government Finance Act 1992 (c. 14). Sections 137(1) and 147(1) are interpretation provisions and are cited because of the meaning assigned to the word “prescribed”.}, sections 1(1), 5(1)($i$), 7(1), 27(1)($b$), 73(1)($a$), 189(1), (3), (4) and (5) and 191 of the Social Security Administration Act 1992\footnote{\frenchspacing 1992 c. 5; section 191 is an interpretation provision and is cited because of the meaning assigned to the word “prescribed”.}, and sections 4(5), 12(2) and (4)($a$) and ($b$), 35(1) and 36(1), (2) and (4)($a$) of the Jobseekers Act 1995\footnote{\frenchspacing Section 35(1) is an interpretation provision and is cited because of the meaning assigned to the word “prescribed”.} and of all other powers enabling him in that behalf, after consultation, in respect of provisions in these Regulations relating to housing benefit and council tax benefit, with organisations appearing to him to be representative of the authorities concerned\footnote{\frenchspacing \emph{See} section 176(1) of the Social Security Administration Act 1992 (c. 5).} and after reference to the Social Security Advisory Committee of proposals in respect of regulations 2 to 6 and 18 to 48\footnote{\frenchspacing \emph{See} section 172(1) of the Social Security Administration Act 1992 (c. 5).}, hereby makes the following Regulations:—

{\sloppy

\tableofcontents

}

\setcounter{secnumdepth}{-2}

\subsection[1. Citation, commencement and interpretation]{Citation, commencement and interpretation}

1.—(1) These Regulations may be cited as the Child Benefit, Child Support and Social Security (Miscellaneous Amendments) Regulations 1996 and shall come into force—
\begin{enumerate}\item[]
($a$) for the purposes of regulations 1, 22 to 28, 30 and 35($b$), on 1st April 1997;

($b$) for the purposes of regulations 2 to 21 and 37 to 49, on 7th April 1997;

($c$) for the purposes of regulations 29, 31 to 35($a$) and 36—
\begin{enumerate}\item[]
(i) in any case where rent is payable at intervals of one month or any other interval which is not a week or a multiple thereof, on 1st April 1997, and

(ii) in any other case, on 7th April 1997.
\end{enumerate}
\end{enumerate}

(2) Regulations 37 to 41 of these Regulations shall have effect in relation to any particular claimant at the beginning of the first benefit week to commence for that claimant on or after 7th April 1997 which applies in his case and for the purpose of this paragraph, “benefit week” and “claimant” have the same meaning as in regulation 2(1) of the Income Support Regulations.

(3) Regulations 42 to 46 of these Regulations shall have effect in relation to any particular claimant at the beginning of the first benefit week to commence for that claimant on or after 7th April 1997 which applies in his case and for the purpose of this paragraph, “benefit week” has the same meaning as in regulation 1(3) of the Jobseeker’s Allowance Regulations\footnote{\frenchspacing The definition of “benefit week” in regulation 2(1) of the Income Support Regulations is amended by S.I. 1988/1445.}.

(4) In these Regulations—
\begin{enumerate}\item[]
“the Adjudication Regulations” means the Social Security (Adjudication) Regulations 1995\footnote{\frenchspacing S.I. 1995/1801.};

“the Child Benefit Regulations” means the Child Benefit (General) Regulations 1976\footnote{\frenchspacing S.I. 1976/965.};

“the Child Benefit Rates Regulations” means the Child Benefit and Social Security (Fixing and Adjustment of Rates) Regulations 1976\footnote{\frenchspacing S.I. 1976/1267.};

“the Child Support Maintenance Assessments Regulations” means the Child Support (Maintenance Assessments and Special Cases) Regulations 1992\footnote{\frenchspacing S.I. 1992/1815.};

“the Claims and Payments Regulations” means the Social Security (Claims and Payments) Regulations 1987\footnote{\frenchspacing S.I. 1987/1968.};

“the Council Tax Benefit Regulations” means the Council Tax Benefit (General) Regulations 1992\footnote{\frenchspacing S.I. 1992/1814.};

“the Housing Benefit Regulations” means the Housing Benefit (General) Regulations 1987\footnote{\frenchspacing S.I. 1987/1971.};

“the Income Support Regulations” means the Income Support (General) Regulations 1987\footnote{\frenchspacing S.I. 1987/1967.};

\begin{sloppypar}
“the Jobseeker’s Allowance Regulations” means the Jobseeker’s Allowance Regulations 1996\footnote{\frenchspacing S.I. 1996/207.};
\end{sloppypar}

“the Overlapping Benefits Regulations” means the Social Security (Overlapping Benefits) Regulations 1979\footnote{\frenchspacing S.I. 1979/597.}.
\end{enumerate}

\subsection[2. Amendment of regulation 59 of the Adjudication Regulations]{Amendment of regulation 59 of the Adjudication Regulations}

2.  In regulation 59(1)($d$) of the Adjudication Regulations (review of decisions involving payment or increase of child benefit), for the word “12” there shall be substituted the word “6”.

\subsection[3. Amendment of regulation 9 of the Child Benefit Regulations]{Amendment of regulation 9 of the Child Benefit Regulations}

3.—(1) In regulation 9 of the Child Benefit Regulations\footnote{\frenchspacing Regulation 9 is amended by S.I. 1984/337.} (persons exempt from tax)—
\begin{enumerate}\item[]
($a$) in paragraph (1), after the words “his spouse” wherever those words occur, there shall be inserted the words “or partner”;

($b$) after paragraph (1), there shall be inserted the following paragraph—
\begin{quotation}
\begin{sloppypar}
“(1A) For the purpose of paragraph (1), “partner” means any person who is living with another person as his spouse.”.
\end{sloppypar}
\end{quotation}
\end{enumerate}

(2) Paragraph (1) shall not apply in the case of any person who was entitled to child benefit on 6th April 1997 and to whom regulation 9(1) of the Child Benefit Regulations applies on the date these Regulations come into force, for so long as his entitlement to child benefit continues.

\subsection[4. Insertion of regulation 9A into the Child Benefit Regulations]{Insertion of regulation 9A into the Child Benefit Regulations}

4.—(1) After regulation 9 of the Child Benefit Regulations, there shall be inserted the following regulation—
\begin{quotation}
\subsection*{“Child living with another person as his spouse}

9A.—(1) Except in the circumstances specified in paragraph (2), benefit shall not be payable to any person in respect of a child for any week in which that child is living with another person as his spouse (referred to in this regulation as “the partner”) and that child—
\begin{enumerate}\item[]
($a$) is under the age of 18 and not receiving full-time education; or

($b$) is under the age of 19 and receiving full-time education.
\end{enumerate}

(2) The specified circumstances are that—
\begin{enumerate}\item[]
($a$) the person to whom benefit is payable is not the partner of that child; and

($b$) the partner of that child is receiving full-time education.”.
\end{enumerate}
\end{quotation}

(2) Paragraph (1) shall not apply in the case of any person who was entitled to child benefit on 6th April 1997 for so long as that entitlement continues.

\subsection[5. Amendment of regulation 2 of the Child Benefit Rates Regulations]{Amendment of regulation 2 of the Child Benefit Rates Regulations}

5.—(1) Regulation 2 of the Child Benefit Rates Regulations\footnote{\frenchspacing Regulation 2 is amended by S.I. 1977/1328, 1980/110, 1991/502, 1993/965 and 1995/559.} (weekly rates of child benefit) shall be amended in accordance with the following paragraphs.

(2) In paragraph (1)—
\begin{enumerate}\item[]
($a$) the words “paragraph (2) of this regulation and” shall be omitted;

($b$) for sub-paragraph ($a$), there shall be substituted the following sub-paragraph—
\begin{quotation}
“($a$) subject to paragraphs (2ZA) to (4) of this regulation—
\begin{enumerate}\item[]
(i) in a case where in any week a child (not being a child to whom head (ii) of this sub-paragraph applies) is the only child or, if not the only child, the elder or eldest child in respect of whom child benefit is payable to a person, £10.80,

(ii) in a case where in any week that child is the only child or, if not the only child, the elder or eldest child of a lone parent in respect of whom child benefit is payable to a person, £17.10; and”.
\end{enumerate}
\end{quotation}
\end{enumerate}

(3) For paragraph (2) there shall be substituted—
\begin{quotation}
“(2) For the purpose of paragraph (1)($a$)(ii) of this regulation, “lone parent” means a person who is living with a child and, in any week, that person—
\begin{enumerate}\item[]
($a$) either has—
\begin{enumerate}\item[]
(i) no spouse, or

(ii) is not residing with his spouse, and
\end{enumerate}

($b$) is not living with any other person as his spouse.”.
\end{enumerate}
\end{quotation}

(4) In paragraph (2ZA), after the words “paragraph (1)($a$)” in both places where those words occur, there shall be inserted the word “(i)”.

(5) In paragraph (2ZB), after the words “paragraph (1)($a$)” there shall be inserted the words “(i) or (ii)”.

(6) For paragraph (2A) there shall be substituted the following paragraph—
\begin{quotation}
“(2A) Subject to paragraph (2AB), a person who is residing with a parent of the child shall be entitled to the weekly rate of child benefit payable under paragraph (1)($a$)(i) of this regulation in respect of that child (“A”).”.
\end{quotation}

(7) After that paragraph (2A), there shall be inserted the following paragraph—
\begin{quotation}
“(2AB) Paragraph (2A) shall not apply where the person is also entitled to child benefit in respect of another child (“B”) for whom he is treated as responsible under section 143(1)($a$) of the Social Security Contributions and Benefits Act 1992\footnote{\frenchspacing 1992 c. 4.} and he is not residing with a parent of B, in which case he shall be entitled, in respect of A, to the weekly rate of child benefit payable under paragraph (1)($a$)(ii) of this regulation.”.
\end{quotation}

(8) In paragraph (4)—
\begin{enumerate}\item[]
($a$) in sub-paragraph ($a$), for the words “paragraph (2) of this regulation shall not apply”, there shall be substituted the words “the weekly rate of child benefit specified in paragraph (1)($a$)(i) of this regulation shall be payable in respect of that child”;

($b$) in sub-paragraph ($b$), for the words “paragraph (2) of this regulation shall not apply to that child”, there shall be substituted the words “the weekly rate of child benefit specified in paragraph (1)($a$)(i) of this regulation shall be payable in respect of that child”;

($c$) for sub-paragraph ($c$), there shall be substituted the following sub-paragraph—
\begin{quotation}
“($c$) in any week a person has been paid the weekly rate of child benefit for the time being specified in paragraph (1)($a$)(ii) of this regulation and that week is one throughout which that person subsequently becomes entitled to such an allowance or increase as is referred to in sub-paragraph ($a$) above, the amount corresponding to the difference between the weekly rates of child benefit for the time being specified in paragraph (1)($a$)(i) and (ii) of this regulation shall be treated as if it had been paid on account of any such allowance or increase payable in respect of that week.”.
\end{quotation}
\end{enumerate}

\subsection[6. Revocation of regulation 4 of the Child Benefit Rates Regulations]{Revocation of regulation 4 of the Child Benefit Rates Regulations}

6.  Regulation 4 of the Child Benefit Rates Regulations (transitional provisions for the purposes of regulation 2(2)) is hereby revoked.

\subsection[7. Amendment of regulation 1 of the Child Support Maintenance Assessments Regulations]{Amendment of regulation 1 of the Child Support Maintenance Assessments Regulations}

7.  In regulation 1(2) of the Child Support Maintenance Assessments Regulations (interpretation)\footnote{\frenchspacing Regulation 1(2) is amended by S.I. 1993/913, 1995/1045 and 1995/3261.} after the definition of “the Act” there shall be inserted the following definition—
\begin{quotation}
““Child Benefit Rates Regulations” means the Child Benefit and Social Security (Fixing and Adjustment of Rates) Regulations 1976\footnote{\frenchspacing S.I. 1976/1267 is amended by S.I. 1977/1328, 1980/110, 1991/502, 1993/965, 1995/559 and 1996/1803.};”.
\end{quotation}

\subsection[8. Amendment of regulation 3 of the Child Support Maintenance Assessments Regulations]{Amendment of regulation 3 of the Child Support Maintenance Assessments Regulations}

8.  In regulation 3(1) of the Child Support Maintenance Assessments Regulations\footnote{\frenchspacing Regulation 3(1) is amended by S.I. 1994/227.} (calculation of AG)—
\begin{enumerate}\item[]
($a$) for sub-paragraph ($c$) there shall be substituted the following sub-paragraph—
\begin{quotation}
“($c$) an amount equal to—
\begin{enumerate}\item[]
(i) the amount specified in paragraph 3($b$) of the relevant Schedule; or

(ii) where the person with care is a lone parent as defined in regulation 2(1) of the Income Support Regulations, the amount specified in paragraph 3($a$) of the relevant Schedule.”;
\end{enumerate}
\end{quotation}

($b$) sub-paragraph ($d$) shall be omitted.
\end{enumerate}

\subsection[9. Amendment of regulation 4 of the Child Support Maintenance Assessments Regulations]{Amendment of regulation 4 of the Child Support Maintenance Assessments Regulations}

9.  In regulation 4 of the Child Support Maintenance Assessments Regulations (basic rate of child benefit) for the words “regulation 2(1) of the Child Benefit and Social Security (Fixing and Adjustment of Rates) Regulations 1976\footnote{\frenchspacing S.I. 1976/1267; the relevant amending instruments are S.I. 1977/1328, 1991/502, 543, 1595.} (rates of child benefit)” there shall be substituted the words “regulation 2(1)($a$)(i) or 2(1)($b$) of the Child Benefit Rates Regulations (weekly rate for only, elder or eldest child and for other children)”.

\subsection[10. Amendment of regulation 6 of the Child Support Maintenance Assessments Regulations]{Amendment of regulation 6 of the Child Support Maintenance Assessments Regulations}

10.  In regulation 6(2)($b$) of the Child Support Maintenance Assessments Regulations\footnote{\frenchspacing Regulation 6(2) is amended by S.I. 1995/1045.} (the additional element) for the words “regulation 3(1)($c$)” there shall be substituted the words “regulation 3(1)($c$)(i)”.

\subsection[11. Amendment of regulation 9 of the Child Support Maintenance Assessments Regulations]{Amendment of regulation 9 of the Child Support Maintenance Assessments Regulations}

11.—(1) Regulation 9 of the Child Support Maintenance Assessments Regulations\footnote{\frenchspacing Regulation 9 is amended by S.I. 1993/913, 1995/1045 and 1995/3261.} (exempt income: calculation or estimation of E) shall be amended in accordance with the following paragraphs.

(2) In paragraph (1)—
\begin{enumerate}\item[]
($a$) in sub-paragraph ($c$)—
\begin{enumerate}\item[]
(i) for head (ii) there shall be substituted the following head—
\begin{quotation}
“(ii) if he were a claimant, the conditions in paragraph 3($a$) of the relevant Schedule would be satisfied;”;
\end{quotation}

(ii) for the words “column 2 of paragraph 15(1) of that Schedule (income support lone parent premium)” there shall be substituted the words “that sub-paragraph”;
\end{enumerate}

($b$) in sub-paragraph ($f$)—
\begin{enumerate}\item[]
(i) after the words “that parent”, there shall be inserted the words “but he is not a lone parent as defined in regulation 2(1) of the Income Support Regulations,”;

(ii) after the words “specified in”, there shall be inserted the words “sub-paragraph ($b$) of”.
\end{enumerate}
\end{enumerate}

(3) In paragraph (2)($c$)(iv), for the words “3 of the Schedule” there shall be substituted the words “3($b$) of the relevant Schedule”.

(4) In paragraph (3)—
\begin{enumerate}\item[]
($a$) for the words “any amounts” there shall be substituted the words “any amount”;

($b$) for the words “and ($f$)” there shall be substituted the words “or ($f$)”.
\end{enumerate}

\subsection[12. Amendment of regulation 11 of the Child Support Maintenance Assessments Regulations]{\sloppy Amendment of regulation 11 of the Child Support Maintenance Assessments Regulations}

12.—(1) Regulation 11 of the Child Support Maintenance Assessments Regulations\footnote{\frenchspacing Regulation 11(1) is amended by S.I. 1994/227, 1995/1045 and 1995/3261.} (protected income) shall be amended in accordance with the following paragraphs.

(2) In paragraph (1)—
\begin{enumerate}\item[]
($a$) for sub-paragraph ($c$) there shall be substituted the following sub-paragraph—
\begin{quotation}
“($c$) where, if the absent parent were a claimant, the conditions in paragraph 3($a$) of the relevant Schedule (income support family premium) would be satisfied, an amount equal to the amount specified in that sub-paragraph;”;
\end{quotation}

($b$) in sub-paragraph ($f$)—
\begin{enumerate}\item[]
(i) after the word “satisfied” there shall be inserted the words “but he is not a lone parent as defined in regulation 2(1) of the Income Support Regulations”;

(ii) after the words “specified in” there shall be inserted the words “sub-paragraph ($b$) of”.
\end{enumerate}
\end{enumerate}

(3) In paragraph (3)—
\begin{enumerate}\item[]
($a$) for the words “any amounts” there shall be substituted the words “any amount”;

($b$) for the words “and ($f$)” there shall be substituted the words “or ($f$)”; and

($c$) the words “income support lone parent premium and” shall be omitted.
\end{enumerate}

\subsection[13. Amendment of regulation 19 of the Child Support Maintenance Assessments Regulations]{\sloppy Amendment of regulation 19 of the Child Support Maintenance Assessments Regulations}

13.  In regulation 19(2)($c$) of the Child Support Maintenance Assessments Regulations (both parents are absent) for the words “not include any amount mentioned in regulation 3(1)($d$) (income support lone parent premium)” there shall be substituted the words “include the amount specified in regulation 3(1)($c$)(i) but not the amount specified in regulation 3(1)($c$)(ii) (income support family premium)”.

\subsection[14. Amendment of regulation 23 of the Child Support Maintenance Assessments Regulations]{\sloppy Amendment of regulation 23 of the Child Support Maintenance Assessments Regulations}

14.  In regulation 23(2) of the Child Support Maintenance Assessments Regulations\footnote{\frenchspacing Regulation 23(2) is amended by S.I. 1994/227.} (person caring for children of more than one absent parent) for the words “, ($c$) or ($d$)” there shall be substituted the words “or ($c$)”.

\subsection[15. Amendment of regulation 26 of the Child Support Maintenance Assessments Regulations]{\sloppy Amendment of regulation 26 of the Child Support Maintenance Assessments Regulations}

15.  In regulation 26(1)($b$) of the Child Support Maintenance Assessments Regulations\footnote{\frenchspacing Regulation 26(1) is amended by S.I. 1995/1045.} (cases where child support maintenance is not to be payable), in head (ii), for the word “11(1)($f$)” there shall be substituted the words “11(1)($c$) or ($f$)”.

\subsection[16. Amendment of regulation 28 of the Child Support Maintenance Assessments Regulations]{\sloppy Amendment of regulation 28 of the Child Support Maintenance Assessments Regulations}

16.  In regulation 28(1)($b$) of the Child Support Maintenance Assessments Regulations\footnote{\frenchspacing Regulation 28(1) is amended by S.I. 1993/913.} (amount payable where absent parent is in receipt of income support or other prescribed benefit), for the word “3” there shall be substituted the words “3($a$) or ($b$)”.

\subsection[17. Amendment of Schedule 1 to the Child Support Maintenance Assessments Regulations]{Amendment of Schedule 1 to the Child Support Maintenance Assessments Regulations}

17.—(1) Schedule 1 to the Child Support Maintenance Assessments Regulations\footnote{\frenchspacing Schedule 1 is amended by S.I. 1993/613 and 1995/1045.} (calculation of N and M) shall be amended in accordance with the following paragraphs.

(2) After sub-paragraph (5) of paragraph 7 there shall be added the following sub-paragraph—
\begin{quotation}
“(6) Where child benefit in respect of a relevant child is in payment at the rate specified in regulation 2(1)($a$)(ii) of the Child Benefit Rates Regulations, the difference between that rate and the basic rate applicable to that child, as defined in regulation 4.”.
\end{quotation}

(3) In paragraph 20—
\begin{enumerate}\item[]
($a$) in head (ii) of sub-paragraph ($a$), for the words “to ($d$)” there shall be substituted the words “and ($c$)”;

($b$) in sub-paragraph ($b$), for the words “3(1)($c$)” there shall be substituted the words “3(1)($c$)(i)”.
\end{enumerate}

(4) For sub-paragraph ($a$) of paragraph 28 there shall be substituted the following sub-paragraph—
\begin{quotation}
“($a$) if the parent satisfies the conditions for payment of the rate of child benefit specified in regulation 2(1)($a$)(ii) of the Child Benefit Rates Regulations, an amount representing the difference between that rate and the basic rate, as defined in regulation 4;”.
\end{quotation}

\subsection[18. Amendment of regulation 2 of the Claims and Payments Regulations]{Amendment of regulation 2 of the Claims and Payments Regulations}

18.  In regulation 2(3) of the Claims and Payments Regulations\footnote{\frenchspacing Regulation 2(3) is amended by S.I. 1992/247.} (treatment as separate benefits), the words from “and so shall” to the end of that paragraph shall be omitted.

\subsection[19. Amendment of regulation 9 of the Claims and Payments Regulations]{Amendment of regulation 9 of the Claims and Payments Regulations}

19.  In regulation 9(3) of the Claims and Payments Regulations\footnote{\frenchspacing Regulation 9(3) is amended by S.I. 1989/136.} (interchange of claims for child benefit with claims for other benefits), the words “(except an increase in child benefit)” shall be omitted.

\subsection[20. Amendment of Schedule 1 to the Claims and Payments Regulations]{Amendment of Schedule 1 to the Claims and Payments Regulations}

20.  In Part II of Schedule 1 to the Claims and Payments Regulations (interchange of claims for child benefit with claims for other benefits), the words “Increase in child benefit under regulation 2(2) of the Child Benefit and Social Security (Fixing and Adjustment of Rates) Regulations 1976” shall be omitted.

\subsection[21. Amendment of Schedule 8 to the Claims and Payments Regulations]{Amendment of Schedule 8 to the Claims and Payments Regulations}

21.  For paragraph 2($a$) of Schedule 8 to the Claims and Payments Regulations\footnote{\frenchspacing Paragraph 2 of Schedule 8 is amended by S.I. 1991/2741.} (election to have child benefit paid weekly), there shall be substituted the following sub-paragraph—
\begin{quotation}
“($a$) he is a lone parent within the meaning set out in regulation 2(2) of the Child Benefit and Social Security (Fixing and Adjustment of Rates) Regulations 1976\footnote{\frenchspacing S.I. 1976/1267; the relevant amending instrument is S.I. 1996/1803.}, or”.
\end{quotation}

\subsection[22. Amendment of regulation 10 of the Council Tax Benefit Regulations]{Amendment of regulation 10 of the Council Tax Benefit Regulations}

22.  In regulation 10(1)($b$) of the Council Tax Benefit Regulations\footnote{\frenchspacing Regulation 10(1)(b) is amended by S.I. 1993/688.} (applicable amounts relating to patients), the words “8 or” shall be omitted.

\subsection[23. Amendment of regulation 26 of the Council Tax Benefit Regulations]{Amendment of regulation 26 of the Council Tax Benefit Regulations}

23.—(1) Regulation 26 of the Council Tax Benefit Regulations\footnote{\frenchspacing Regulation 26 is amended by S.I. 1993/688, 963, 1249, 1994/578, 2137 and 1995/560, 626, 2303.} (notional income) shall be amended in accordance with the following paragraphs.

(2) After paragraph (4A) there shall be inserted the following paragraphs—
\begin{quotation}
“(4B) Paragraph (4C) shall apply where a claimant is in receipt of both child benefit and an increase in child benefit under regulation 2(2) of the Child Benefit and Social Security (Fixing and Adjustment of Rates) Regulations 1976\footnote{\frenchspacing S.I. 1976/1267; relevant amending instruments are S.I. 1980/110 and 1993/965.} (increase in respect of the only, elder or eldest child of a single parent) in respect of the week up to and including 6th April 1997.

(4C) In a case where this paragraph applies, the appropriate authority shall, where it has selected 1st April 1997 to apply in its area as the date at which the claimant is treated as possessing benefit at the altered rate in accordance with paragraph (4), treat the claimant, for the period from 1st April 1997 to 6th April 1997, as possessing any child benefit which is payable to him from 7th April 1997 at the rates specified in regulation 2(1)($a$)(ii)\footnote{\frenchspacing The relevant amending instrument is S.I. 1996/1803.} (weekly rate for only, elder or eldest child of a lone parent) and, as the case may be, ($b$) (weekly rate for other children) of the Child Benefit and Social Security (Fixing and Adjustment of Rates) Regulations 1976, to the extent that such child benefit is payable to him at those rates from that date.”.
\end{quotation}

(3) In paragraph (6), after the words “(4A)” there shall be inserted the words “and (4C)”.

\subsection[24. Amendment of regulation 40 of the Council Tax Benefit Regulations]{Amendment of regulation 40 of the Council Tax Benefit Regulations}

24.  In regulation 40(3)($b$) of the Council Tax Benefit Regulations (students who are not excluded from entitlement to council tax benefit), for the words “the lone parent premium” there shall be substituted the words “a family premium under paragraph 3($a$) of Schedule 1”.

\subsection[25. Amendment of Schedule 1 to the Council Tax Benefit Regulations]{Amendment of Schedule 1 to the Council Tax Benefit Regulations}

25.—(1) Schedule 1 to the Council Tax Benefit Regulations (applicable amounts) shall be amended in accordance with the following paragraphs.

(2) In Part II (family premium), in paragraph 3, after the words “shall be”, there shall be inserted the words—
\begin{quotation}
“($a$) where the claimant is a lone parent and no premium is applicable under paragraph 9, 10, 11 or 12, £22.05;

($b$) in any other case,”.
\end{quotation}

(3) In Part III (premiums)—
\begin{enumerate}\item[]
($a$) in paragraph 4, for the words “paragraphs 8” there shall be substituted the words “paragraphs 9”;

($b$) paragraph 8 (lone parent premium) shall be omitted.
\end{enumerate}

(4) In Part IV (amounts of premiums), in paragraph 19, the entries in both columns of sub-paragraph (1) (lone parent premium) shall be omitted.

\subsection[26. Amendment of Schedule 3 to the Council Tax Benefit Regulations]{Amendment of Schedule 3 to the Council Tax Benefit Regulations}

26.  In paragraph 4 of Schedule 3 to the Council Tax Benefit Regulations (disregard of certain sums in the calculation of a lone parent’s earnings), for the words “lone parent premium under” there shall be substituted the words “family premium under paragraph 3($a$) of”.

\subsection[27. Amendment of Schedule 4 to the Council Tax Benefit Regulations]{Amendment of Schedule 4 to the Council Tax Benefit Regulations}

27.  In Schedule 4 to the Council Tax Benefit Regulations (sums to be disregarded in the calculation of income other than earnings)—
\begin{enumerate}\item[]
($a$) in paragraph 46(1), (disregard of maintenance in the calculation of income other than earnings), for the words “the family premium” there shall be substituted the words “a family premium”;

($b$) after paragraph 57\footnote{\frenchspacing Paragraph 57 is substituted by S.I. 1996/462.} there shall be inserted the following paragraph—
\begin{quotation}
“58.  Where regulation 26(4C) applies to a claimant and—
\begin{enumerate}\item[]
($a$) he is in receipt of both child benefit and an increase in child benefit under regulation 2(2) of the Child Benefit and Social Security (Fixing and Adjustment of Rates) Regulations 1976\footnote{\frenchspacing S.I. 1976/1267; relevant amending instruments are S.I. 1980/110 and 1993/965.} (increase in respect of the only, elder or eldest child of a single parent) in respect of the week up to and including 6th April 1997; and

($b$) child benefit is payable to him from 7th April 1997 at the rates specified in regulation 2(1)($a$)(ii)\footnote{\frenchspacing The relevant amending instrument is S.I. 1996/1803.} (weekly rate for only, elder or eldest child of a lone parent) and, as the case may be, ($b$) of those Regulations (weekly rate for other children) as payable from that date,
\end{enumerate}
any child benefit and increase in child benefit specified in sub-paragraph ($a$) to the extent that it is payable in respect of the period between 1st April 1997 and 6th April 1997.”.
\end{quotation}
\end{enumerate}

\subsection[28. Amendment of Schedule 5A to the Council Tax Benefit Regulations]{Amendment of Schedule 5A to the Council Tax Benefit Regulations}

28.  In paragraph 2($c$)(ii) of Schedule 5A to the Council Tax Benefit Regulations\footnote{\frenchspacing Schedule 5A is inserted by S.I. 1996/194.} (conditions for an extended payment of council tax benefit), for the words “paragraph 8 (lone parent premium)” there shall be substituted the words “paragraph 3($a$) (rate of family premium applicable to a lone parent)”.

\subsection[29. Amendment of regulation 18 of the Housing Benefit Regulations]{Amendment of regulation 18 of the Housing Benefit Regulations}

29.  In regulation 18(1)($b$) of the Housing Benefit Regulations (applicable amounts relating to patients), for the words “paragraphs 8 or” there shall be substituted the word “paragraph”.

\subsection[30. Amendment of regulation 35 of the Housing Benefit Regulations]{Amendment of regulation 35 of the Housing Benefit Regulations}

30.—(1) Regulation 35 of the Housing Benefit Regulations\footnote{\frenchspacing Regulation 35 is amended by S.I. 1988/1971, 1990/127, 546, 1991/1175, 1599, 1992/1101, 2148, 1993/317, 963, 1249, 1994/578, 2137 and 1995/560, 626, 2303.} (notional income) shall be amended in accordance with the following paragraphs.

(2) After paragraph (4A) there shall be inserted the following paragraphs—
\begin{quotation}
“(4B) Paragraph (4C) shall apply where a claimant is in receipt of both child benefit and an increase in child benefit under regulation 2(2) of the Child Benefit and Social Security (Fixing and Adjustment of Rates) Regulations 1976\footnote{\frenchspacing S.I. 1976/1267; relevant amending instruments are S.I. 1980/110 and 1993/965.} (increase in respect of the only, elder or eldest child of a single parent) in respect of the week up to and including 6th April 1997.

(4C) In a case where this paragraph applies, the appropriate authority shall, in a case in which the claimant’s weekly amount of eligible rent falls to be calculated in accordance with regulation 69(2)($b$) (calculation of weekly amounts), treat the claimant, for the period from 1st April 1997 to 6th April 1997, as possessing any child benefit which is payable to him from 7th April 1997 at the rates specified in regulation 2(1)($a$)(ii)\footnote{\frenchspacing The relevant amending instrument is S.I. 1996/1803.} (weekly rate for only, elder or eldest child of a lone parent) and, as the case may be, ($b$) (weekly rate for other children) of the Child Benefit and Social Security (Fixing and Adjustment of Rates) Regulations 1976, to the extent that such child benefit is payable to him at those rates from that date.”.
\end{quotation}

(3) In paragraph (6), after the words “(4A)” there shall be inserted the words “and (4C)”.

\subsection[31. Amendment of regulation 48A of the Housing Benefit Regulations]{Amendment of regulation 48A of the Housing Benefit Regulations}

31.  In regulation 48A(2)($b$) of the Housing Benefit Regulations\footnote{\frenchspacing Regulation 48A is inserted by S.I. 1990/1549 and amended by S.I. 1990/1657, 1991/235, 1992/432 and 1995/626.} (students who are treated as liable to make payments in respect of a dwelling), for the words “the lone parent premium” there shall be substituted the words “a family premium under paragraph 3($a$) of Schedule 2”.

\subsection[32. Amendment of regulation 51 of the Housing Benefit Regulations]{Amendment of regulation 51 of the Housing Benefit Regulations}

32.  In regulation 51(2)($c$) of the Housing Benefit Regulations\footnote{\frenchspacing Regulation 51(2)(c) is amended by S.I. 1995/626.} (exclusions from reductions in the amounts of eligible rent relating to students), in head (i), for the words “the lone parent premium” there shall be substituted the words “a family premium under paragraph 3($a$) of Schedule 2”.

\subsection[33. Amendment of Schedule 2 to the Housing Benefit Regulations]{Amendment of Schedule 2 to the Housing Benefit Regulations}

\begin{sloppypar}
33.—(1) Schedule 2 to the Housing Benefit Regulations (applicable amounts) shall be amended in accordance with the following paragraphs.
\end{sloppypar}

(2) In Part II (family premium), in paragraph 3, after the words “shall be”, there shall be inserted the words—
\begin{quotation}
“($a$) where the claimant is a lone parent and no premium is applicable under paragraph 9, 9A, 10 or 11, £22.05;

($b$) in any other case,”.
\end{quotation}

(3) In Part III (premiums)—
\begin{enumerate}\item[]
($a$) in paragraph 4\footnote{\frenchspacing Paragraph 4 is amended by S.I. 1990/1775.}, for the words “paragraphs 8” there shall be substituted the words “paragraphs 9”;

($b$) paragraph 8\footnote{\frenchspacing Paragraph 8 is amended by S.I. 1990/546.} (lone parent premium) shall be omitted.
\end{enumerate}

(4) In Part IV (amounts of premiums), in paragraph 15, the entries in both columns of sub-paragraph (1) (lone parent premium) shall be omitted.

\subsection[34. Amendment of Schedule 3 to the Housing Benefit Regulations]{Amendment of Schedule 3 to the Housing Benefit Regulations}

34.  In paragraph 4 of Schedule 3 to the Housing Benefit Regulations\footnote{\frenchspacing Paragraph 4 is substituted by S.I. 1990/1775.} (disregard of certain sums in the calculation of a lone parent’s earnings), for the words “lone parent premium under” there shall be substituted the words “family premium under paragraph 3($a$) of”.

\subsection[35. Amendment of Schedule 4 to the Housing Benefit Regulations]{Amendment of Schedule 4 to the Housing Benefit Regulations}

35.  In Schedule 4 to the Housing Benefit Regulations (sums to be disregarded in the calculation of income other than earnings)—
\begin{enumerate}\item[]
($a$) in paragraph 47(1)\footnote{\frenchspacing Paragraph 47 is added by S.I. 1991/2695.} (disregard of maintenance in the calculation of income other than earnings), for the words “the family premium” there shall be substituted the words “a family premium”;

($b$) after paragraph 58\footnote{\frenchspacing Paragraph 58 is substituted by S.I. 1996/462.} there shall be inserted the following paragraph—
\begin{quotation}
“59.  Where regulation 35(4C) applies to a claimant and—
\begin{enumerate}\item[]
($a$) he is in receipt of both child benefit and an increase in child benefit under regulation 2(2) of the Child Benefit and Social Security (Fixing and Adjustment of Rates) Regulations 1976\footnote{\frenchspacing S.I. 1976/1267; relevant amending instruments are S.I. 1980/110 and 1993/965.} (increase in respect of the only, elder or eldest child of a single parent) in respect of the week up to and including 6th April 1997; and

($b$) child benefit is payable to him from 7th April 1997 at the rates specified in regulation 2(1)($a$)(ii)\footnote{\frenchspacing The relevant amending instrument is S.I. 1996/1803.} (weekly rate for only, elder or eldest child of a lone parent) and, as the case may be, ($b$) of those Regulations (weekly rate for other children) as payable from that date,
\end{enumerate}
any child benefit and increase in child benefit specified in sub-paragraph ($a$) to the extent that it is payable in respect of the period between 1st April 1997 and 6th April 1997.”.
\end{quotation}
\end{enumerate}

\subsection[36. Amendment of Schedule 5A to the Housing Benefit Regulations]{Amendment of Schedule 5A to the Housing Benefit Regulations}

36.  In paragraph 2($c$)(ii) of Schedule 5A to the Housing Benefit Regulations\footnote{\frenchspacing Schedule 5A is inserted by S.I. 1996/194.} (conditions for an extended payment of housing benefit), for the words “paragraph 8 (lone parent premium)” there shall be substituted the words “paragraph 3($a$) (rate of family premium applicable to a lone parent)”.

\subsection[37. Amendment of regulation 42 of the Income Support Regulations]{Amendment of regulation 42 of the Income Support Regulations}

37.  In regulation 42 of the Income Support Regulations\footnote{\frenchspacing Regulation 42 is amended by S.I. 1988/663, 1992/468 and 1995/2303.} (notional income)—
\begin{enumerate}\item[]
($a$) for sub-paragraph ($d$) of paragraph (2) there shall be substituted the following sub-paragraph—
\begin{quotation}
“($d$) child benefit to which paragraph (2D) refers;”;
\end{quotation}

($b$) after paragraph (2C), there shall be inserted the following paragraph—
\begin{quotation}
“(2D) This paragraph refers to child benefit payable in accordance with regulation 2(1)($a$)(ii) of the Child Benefit and Social Security (Fixing and Adjustment of Rates) Regulations 1976\footnote{\frenchspacing S.I. 1976/1267; the relevant amending instrument is S.I. 1996/1803.} (weekly rate for only, elder or eldest child of a lone parent) but only to the extent that it exceeds the amount specified in regulation 2(1)($a$)(i) of those Regulations.”.
\end{quotation}
\end{enumerate}

\subsection[38. Amendment of regulation 57 of the Income Support Regulations]{Amendment of regulation 57 of the Income Support Regulations}

38.  In regulation 57 of the Income Support Regulations (period over which payments other than periodical payments are to be taken into account), in both head (ii) of paragraph (1)($b$)\footnote{\frenchspacing Regulation 57(1)(b) is amended by S.I. 1990/1776.} and paragraph (2)($b$), the words “and lone parent” shall be omitted.

\subsection[39. Amendment of Schedule 2 to the Income Support Regulations]{Amendment of Schedule 2 to the Income Support Regulations}

\begin{sloppypar}
39.—(1) Schedule 2 to the Income Support Regulations (applicable amounts) shall be amended in accordance with the following paragraphs.
\end{sloppypar}

(2) In Part II (family premium), in paragraph 3\footnote{\frenchspacing Paragraph 3 is amended by S.I. 1988/1445.}, after the words “shall be”, there shall be inserted the words—
\begin{quotation}
“($a$) where the claimant is a lone parent and no premium is applicable under paragraph 9, 9A, 10 or 11, £15.75;

($b$) in any other case,”.
\end{quotation}

(3) In Part III (premiums)—
\begin{enumerate}\item[]
($a$) in paragraph 4\footnote{\frenchspacing Paragraph 4 is amended by S.I. 1988/1445 and 1990/1776.}, for the words “paragraphs 8” there shall be substituted the words “paragraphs 9”;

($b$) paragraph 8 (lone parent premium) shall be omitted.
\end{enumerate}

(4) In Part IV (amounts of premiums), in paragraph 15, the entries in both columns of sub-paragraph (1) (lone parent premium) shall be omitted.

\subsection[40. Amendment of Schedule 7 to the Income Support Regulations]{Amendment of Schedule 7 to the Income Support Regulations}

40.  In Schedule 7 to the Income Support Regulations (applicable amounts in special cases)—
\begin{enumerate}\item[]
($a$) in Column (2) of paragraph 1($b$)\footnote{\frenchspacing Paragraph 1(b) is amended by S.I. 1988/1445.} (applicable amounts in relation to patients), the words “8 or” shall be omitted; and

($b$) in Column (2) of paragraph 10C\footnote{\frenchspacing Paragraph 10C is inserted by S.I. 1988/2022 and amended by S.I. 1990/547, 1992/3147 and 1995/559.} (applicable amounts for lone parents in residential accommodation temporarily), in sub-paragraph ($b$), the words “, or ($d$) in so far as that amount relates to the lone parent premium under paragraph 8 of Schedule 2” shall be omitted.
\end{enumerate}

\subsection[41. Amendment of Schedule 8 to the Income Support Regulations]{Amendment of Schedule 8 to the Income Support Regulations}

41.  In paragraph 5 of Schedule 8 to the Income Support Regulations\footnote{\frenchspacing Paragraph 5 is amended by S.I. 1988/1445 and 1989/534.} (disregard of certain sums in the calculation of a lone parent’s earnings), for the words “lone parent premium under” there shall be substituted the words “family premium under paragraph 3($a$) of”.

\subsection[42. Amendment of regulation 105 of the Jobseeker’s Allowance Regulations]{Amendment of regulation 105 of the Jobseeker’s Allowance Regulations}

42.  In regulation 105 of the Jobseeker’s Allowance Regulations (notional income)—
\begin{enumerate}\item[]
($a$) for sub-paragraph ($c$) of paragraph (2), there shall be substituted the following sub-paragraph—
\begin{quotation}
“($c$) child benefit to which paragraph (2A) refers;”;
\end{quotation}

($b$) after paragraph (2), there shall be inserted the following paragraph—
\begin{quotation}
“(2A) This paragraph refers to child benefit payable in accordance with regulation 2(1)($a$)(ii) of the Child Benefit and Social Security (Fixing and Adjustment of Rates) Regulations 1976\footnote{\frenchspacing S.I. 1976/1267; the relevant amending instrument is S.I. 1996/1803.} (weekly rate for only, elder or eldest child of a lone parent) but only to the extent that it exceeds the amount specified in regulation 2(1)($a$)(i) of those Regulations.”.
\end{quotation}
\end{enumerate}

\subsection[43. Amendment of regulation 121 of the Jobseeker’s Allowance Regulations]{Amendment of regulation 121 of the Jobseeker’s Allowance Regulations}

43.  In regulation 121 of the Jobseeker’s Allowance Regulations (period over which payments other than periodical payments are to be taken into account), in both head (ii) of paragraph (1)($b$) and paragraph (2)($b$), the words “and lone parent” shall be omitted.

\subsection[44. Amendment of Schedule 1 to the Jobseeker’s Allowance Regulations]{Amendment of Schedule 1 to the Jobseeker’s Allowance Regulations}

44.—(1) Schedule 1 to the Jobseeker’s Allowance Regulations (applicable amounts) shall be amended in accordance with the following paragraphs.

(2) In Part II (family premium), in paragraph 4, after the words “shall be”, there shall be inserted the words—
\begin{quotation}
“($a$) where the claimant is a lone parent and no premium is applicable under paragraph 10, 11, 12 or 13, £15.75;

($b$) in any other case,”.
\end{quotation}

(3) In Part III (premiums)—
\begin{enumerate}\item[]
($a$) in paragraph 5, for the words “paragraphs 9” there shall be substituted the words “paragraphs 10”;

($b$) paragraph 9 (lone parent premium) shall be omitted.
\end{enumerate}

(4) In Part IV (amounts of premiums), in paragraph 20, the entries in both columns of sub-paragraph (1) (lone parent premium) shall be omitted.

\subsection[45. Amendment of Schedule 5 to the Jobseeker’s Allowance Regulations]{Amendment of Schedule 5 to the Jobseeker’s Allowance Regulations}

45.  In the entry in paragraph ($b$) of column (2) of paragraph 9 of Schedule 5 to the Jobseeker’s Allowance Regulations (applicable amount for lone parents who are in residential accommodation temporarily), for the words “regulation 83($d$), ($e$) or ($f$) in so far as that amount relates to the lone parent premium under paragraph 9 of Schedule 1” there shall be substituted the words “regulation 83($d$) or ($f$)”.

\subsection[46. Amendment of Schedule 6 to the Jobseeker’s Allowance Regulations]{Amendment of Schedule 6 to the Jobseeker’s Allowance Regulations}

46.  In paragraph 6 of Schedule 6 to the Jobseeker’s Allowance Regulations (disregard of certain sums in the calculation of a lone parent’s earnings), for the words “lone parent premium under” there shall be substituted the words “family premium under paragraph 4($a$) of”.

\subsection[47. Amendment of regulation 8 of the Overlapping Benefits Regulations]{Amendment of regulation 8 of the Overlapping Benefits Regulations}

47.  In regulation 8 of the Overlapping Benefits Regulations\footnote{\frenchspacing Regulation 8 is amended by S.I. 1991/547, 1992/589 and 1993/965.} (adjustment of benefit or increase in benefit by reference to child benefit)—
\begin{enumerate}\item[]
($a$) for paragraphs (2) and (3) there shall be substituted the following paragraphs—
\begin{quotation}
“(2) Where child benefit is payable to a beneficiary at the rate for the time being specified in regulation 2(1)($a$)(ii) of the Child Benefit and Social Security (Fixing and Adjustment of Rates) Regulations 1976\footnote{\frenchspacing S.I. 1976/1267; the relevant amending instrument is S.I. 1996/1803.} (in this regulation referred to as the “Child Benefit Rates Regulations”) (weekly rate for only, elder or eldest child of a lone parent) and for the same period, in respect of the same child, any benefit or increase in benefit under the Contributions and Benefits Act is or would be payable to a beneficiary, the weekly rate of that benefit or increase thereof shall be reduced by—
\begin{enumerate}\item[]
($a$) in a case where that benefit is guardian’s allowance payable to any person under section 77 of that Act, an amount equal to the amount, less £0.75, by which the rate is specified in regulation 2(1)($a$)(i) of the Child Benefit Rates Regulations (weekly rate for only, elder or eldest child) exceeds the rate specified in regulation 2(1)($b$) of those Regulations (weekly rate for other children); and

($b$) in any other case, an amount equal to the amount, less £0.75, by which the rate specified in regulation 2(1)($a$)(ii) of the Child Benefit Rates Regulations exceeds the rate specified in regulation 2(1)($b$) of those Regulations.
\end{enumerate}

(3) Subject to paragraph (6) of this regulation, where child benefit is payable to a beneficiary at the rate for the time being specified in regulation 2(1)($a$)(i) of the Child Benefit Rates Regulations (weekly rate for only, elder or eldest child) and for the same period, in respect of the same child, any benefit or increase in benefit under the Contributions and Benefits Act is or would be payable to a beneficiary, the weekly rate of that benefit or increase thereof shall be reduced by an amount equal to the amount, less £0.75, by which the rate specified in regulation 2(1)($a$)(i) of the Child Benefit Rates Regulations exceeds the rate specified in regulation 2(1)($b$) of those Regulations.”;
\end{quotation}

($b$) paragraphs (4), (5) and (7) shall be omitted.
\end{enumerate}

\subsection[48. Transitional provision relating to applications for review]{Transitional provision relating to applications for review}

48.  Where an applicant for a review of a decision relating to child benefit—
\begin{enumerate}\item[]
($a$) makes his application on or before 7th October 1997; and

($b$) in respect of any week or weeks prior to 7th April 1997 but no more than 26 weeks before the date of the application referred to in paragraph ($a$) of this regulation (“the relevant period”), would have satisfied the conditions, as were then in force, in regulation 2(2) of the Child Benefit Rates Regulations relating to an increase in the weekly rate of child benefit; and

($c$) was not in receipt of an increase in the weekly rate of child benefit under regulation 2(2) of those Regulations in respect of the relevant period,
\end{enumerate}
that application for review shall be treated, in addition, as if it were a claim for an increase in the weekly rate of child benefit under regulation 2(2) of those Regulations in respect of the relevant period.

\subsection[49. Transitional provision relating to maintenance assessments]{Transitional provision relating to maintenance assessments}

%49.  A maintenance assessment in force on 7th April 1997 shall not be reviewed solely to give effect to regulations 7 to 17 of these Regulations but on a review of that assessment under section 16, 17 or 18 of the Child Support Act 1991\footnote{\frenchspacing 1991 c. 48.}, those regulations shall have effect from the effective date of any fresh maintenance assessment made following that review, or from the first day of the first maintenance period which commences on or after 7th April 1997, whichever is the later.

% Reg 49 substituted (1.6.99) by SI 1999/1510 reg 42
49.—(1) A decision with respect to a maintenance assessment in force on 7th April 1997 shall not be superseded by a decision under section 17 of the Child Support Act 1991 (“the Act”) solely to give effect to these Regulations.

(2) These Regulations shall apply to a fresh maintenance assessment made by virtue of—
\begin{enumerate}\item[]
($a$) a revision under section 16 of the Act of a decision with respect to a maintenance assessment; or

($b$) a decision under section 17 of the Act which supersedes a decision with respect to a maintenance assessment,
\end{enumerate}
as from the effective date of that revision under section 16 of the Act or, as the case may be, decision under section 17 of the Act.

\amendment{
Reg. 49 substituted (1.6.99) by the Social Security Act 1998 (Commencement No. 7 and Consequential and Transitional Provisions) Order 1999 reg. 42.
}

\bigskip

Signed by authority of the Secretary of State for Social Security.

{\raggedleft
\emph{A J B Mitchell}\\*Parliamentary Under-Secretary of State,\\*Department of Social Security

}

4th July 1996

\bigskip

We consent,

{\raggedleft
\emph{Simon Burns}\\*\emph{Liam Fox}\\*Two of the Lords Commissioners of Her Majesty’s Treasury

}

8th July 1996

\part{Explanatory Note}

\renewcommand\parthead{--- Explanatory Note}

\subsection*{(This note is not part of the Regulations)}

These Regulations, in particular regulation 5, amend the Child Benefit and Social Security (Fixing and Adjustment of Rates) Regulations 1976 (S.I.\ 1976/1267) (“the Child Benefit Rates Regulations”) so as to specify a composite rate of child benefit to be payable in respect of the only, elder or eldest child of a lone parent rather than two distinct rates.

  These Regulations also make certain other amendments which are consequential on, or relate to, the above—
\begin{itemize}
\item they amend the Social Security (Adjudication) Regulations 1995 (S.I.\ 1995/1801) by reducing the time limit for submitting applications for review of decisions relating to child benefit (regulation 2);

\item they omit a transitional provision in the Child Benefit Rates Regulations which is no longer necessary (regulation 6);

\item they make consequential amendments to the Child Support (Maintenance Assessment and Special Cases) Regulations 1992 (S.I.\ 1992/\hspace{0pt}1815) (regulations 7, 9 and 17) and to the Social Security (Claims and Payments) Regulations 1987 (S.I.\ 1987/1968) (regulations 18 to 21);

\item they amend the Council Tax Benefit (General) Regulations 1992 (S.I.\ 1992/1814) and the Housing Benefit (General) Regulations 1987 (S.I.\ 1987/1971) so as to provide that a claimant is treated as possessing certain child benefit and that certain child benefit is disregarded in the calculation of income other than earnings in certain circumstances (regulations 23, 27($b$), 30 and 35($b$));

\item they amend provisions in the Income Support (General) Regulations 1987 (S.I.\ 1987/1967) and the Jobseeker’s Allowance Regulations 1996 (S.I.\ 1996/207) relating to the calculation of notional income in so far as it relates to the new composite rate of child benefit (regulations 37 and 42);

\item they amend the Social Security (Overlapping Benefits) Regulations 1979 (S.I.\ 1979/597) so as to provide for the reduction of awards of certain benefits and increases in those benefits where child benefit is being paid in respect of the only, elder or eldest child of a lone parent (regulation 47).
\end{itemize}

  Regulations 48 and 49 are transitional provisions relating respectively to the treatment of applications for review of decisions relating to child benefit made on or before 7th October 1997 and of maintenance assessments for child support which are in force on 7th April 1997.

  These Regulations also replace the premium which is applicable in relation to lone parents in receipt of council tax benefit (regulations 22, 24 to 27($a$) and 28), housing benefit (regulations 29, 31 to 35($a$) and 36), income support (regulations 38 to 41) and jobseeker’s allowance (regulations 43 to 46), with an additional element to the family premium. They also provide for consequential amendments both elsewere in the relevant regulations relating to those benefits and in the Child Support (Maintenance Assessments and Special Cases) Regulations 1992 (regulations 8 and 10 to 16) in connection with this change.

  These Regulations also make amendments to the Child Benefit (General) Regulations 1976 (S.I.\ 1976/965) which are unrelated to the above. In particular, they provide that unmarried partners shall not be entitled to child benefit in any week where they are exempt from United Kingdom income tax (regulation 3) and that no person shall be entitled to child benefit in respect of a child who, in any week, is living with another person as his spouse (regulation 4). Transitional protection is also provided for in both cases.

  The Report of the Social Security Advisory Committee dated 2nd May 1996 on the proposals referred to them in respect of regulations 2 to 6 and 18 to 48 of these Regulations, together with a statement showing the extent to which those Regulations give effect to the Recommendations of the Committee, and in so far as they do not give effect to them, the reasons why not, are contained in Command Paper Cm.\ 3296, published by Her Majesty’s Stationery Office.

  These Regulations do not impose a charge on business.


\end{document}