\documentclass[a4paper]{article}

\usepackage[welsh,english]{babel}

\usepackage[utf8]{inputenc}
\usepackage[T1]{fontenc}

\usepackage{textcomp}

%\usepackage[2012rules]{optional}

\usepackage{lmodern}
\usepackage[osf]{mathpazo}

%\usepackage{perpage} %the perpage package
%\MakePerPage{footnote} %the perpage package command
%\renewcommand{\thefootnote}{\fnsymbol{footnote}}
%
%\usepackage[perpage,para,symbol]{footmisc}

%\opt{newrules}{
\title{The Child Support Departure Direction and Consequential Amendments
Regulations 1996}
%}

%\opt{2012rules}{
%\title{Child Maintenance and Other Payments Act 2008\\(2012 scheme version)}
%}

\author{S.I. 1996 No. 2907}

\date{Made 20th November 1996\\Coming into force 2nd December 1996
}

%\opt{oldrules}{\newcommand\versionyear{1993}}
%\opt{newrules}{\newcommand\versionyear{2003}}
%\opt{2012rules}{\newcommand\versionyear{2012}}

\usepackage{fancyhdr}
\pagestyle{fancy}
\fancyhead[L]{}
\fancyhead[C]{\itshape The Child Support Departure Direction and Consequential
Amendments Regulations 1996 (S.I.~1996/2907) \parthead%\phantom{...}%(\versionyear{} scheme version)
}
\fancyhead[R]{}
\fancyfoot[C]{\thepage}
\newcommand{\parthead}{}

\usepackage{array}
\usepackage{multirow}
\usepackage[debugshow]{tabulary}
\usepackage{longtable}
\usepackage{multicol}
\usepackage{lettrine}

\usepackage[colorlinks=true]{hyperref}
\usepackage{microtype}

\hyphenation{Aw-dur-dod}
\hyphenation{bank-rupt-cy}
\hyphenation{Ec-cles-ton}
\hyphenation{Eux-ton}
\hyphenation{Hogh-ton}
\hyphenation{Pres-ton}
\hyphenation{Pru-den-tial}
\hyphenation{Riv-ing-ton}

\newcolumntype{x}[1]
	{>{\raggedright}p{#1}}
\newcommand{\tn}{\tabularnewline}
\setlength\tymin{50pt}

\newcommand\amendment[1]{\subsubsection*{Notes}{\itshape\frenchspacing\footnotesize
#1 \par}}

\setlength\headheight{22.87003pt}

\newcommand\fnote[1]{\footnote{\frenchspacing #1}}

\begin{document}

\maketitle

\noindent
Whereas a draft of this instrument was laid before Parliament in accordance with
section 52(2) of the Child Support Act 1991\footnote{\frenchspacing 1991.c.48.
Section 28A to 28I of and Schedules 4A and 4B to the Child Support Act 1991 were
inserted by the Child Support Act 1995 (1995 c. 34).} and approved by a
resolution of each House of Parliament:

Now, therefore, the Secretary of State for Social Security, in exercise of the
powers conferred by sections 14(3), 21, 28A(3), 28B(2)($b$), 28C, 28E(5), 28F,
28G, 28I(4)($c$), 42, 51, 52(4) and 54\footnote{\frenchspacing Section 54 is cited
because of the meaning ascribed to the words “maintenance assessment” and
“prescribed”.} of, and paragraph 5 of Schedule 1, paragraphs 2, 4, 6, 7 and 9 of
Schedule 4A and Schedule 4B to the Child Support Act 1991 and of all other
powers enabling him in that behalf, after consultation with the Council on
Tribunals in accordance with section 8 of the Tribunals and Inquiries Act
1992\footnote{\frenchspacing 1992 c. 53.}, hereby makes the following
Regulations:

{\sloppy

\tableofcontents

}

\setcounter{secnumdepth}{-2}

\section[Part I --- General]{Part I\\*General}

\subsection[1. Citation, commencement and interpretation]{Citation, commencement and interpretation}

\renewcommand\parthead{--- Part I}

1.—(1) These Regulations
may be cited as the Child Support Departure Direction and Consequential
Amendments Regulations 1996 and shall come into force on 2nd December 1996.

(2) In these Regulations, unless the context otherwise requires—
\begin{enumerate}\item[]
“the Act” means the Child Support Act 1991;

“the Appeal Regulations” means the Child Support Appeal Tribunals (Procedure)
Regulations 1992\footnote{\frenchspacing S.I. 1992/2641. Regulations 3 and 11 were amended by S.I. 1995/1045 and S.I. 1996/2450. Regulation 5, 6, and 7 were amended by S.I. 1996/2450 and regulation 13 by S.I. 1996/182 and 1996/2450.};

“applicant” has the same meaning as in Schedule 4B to the Act;

“application” means an application for a departure direction;

“Arrears Regulations” means the Child Support (Arrears, Interest and Adjustment
of Maintenance Assessments) Regulations 1992\footnote{\frenchspacing S.I. 1992/1816.};

“Contributions and Benefits Act” means the Social Security Contributions and
Benefits Act 1992\footnote{\frenchspacing 1992 c. 4. Regulation 10 was substituted by S.I. 1995/1045.};

“Departure Direction Anticipatory Application Regulations” means the Child
Support Departure Direction (Anticipatory Application) Regulations 1996\footnote{\frenchspacing S.I. 1996/635.};

“departure direction application form” means the form provided by the Secretary
of State in accordance with regulation 4(1);

“effective date” in relation to a departure direction means the date on which
that direction takes effect;

“Information, Evidence and Disclosure Regulations” means the Child Support
(Information, Evidence and Disclosure) Regulations 1992\footnote{\frenchspacing S.I. 1992/1812. Regulation 9A was inserted by S.I. 1995/1045 and amended by S.I. 1995/3261, which also substituted regulation 10 and inserted regulation 10A.};

“Maintenance Arrangements and Jurisdiction Regulations” means the Child Support
(Maintenance Arrangements and Jurisdiction) Regulations 1992\footnote{\frenchspacing S.I. 1992/2645. Regulation 8 was amended by S.I. 1995/913.};

“Maintenance Assessment Procedure Regulations” means the Child Support
(Maintenance Assessment Procedure) Regulations 1992\footnote{\frenchspacing S.I. 1992/1813. Regulation 10 was amended by S.I. 1994/227, 1995/123, 1995/1045 and 1995/3261.};

“Maintenance Assessments and Special Cases Regulations” means the Child Support
(Maintenance Assessments and Special Cases) Regulations 1992\footnote{\frenchspacing S.I. 1992/1815. Regulation 9 was amended by S.I. 1995/1045 and 1996/1945 and regulation 22 by S.I. 1993/913 and 1995/1045.};

“maintenance period” has the same meaning as in regulation 33 of the Maintenance
Assessment Procedure Regulations;

“non-applicant” means—
\begin{enumerate}\item[]
($a$) where the application has been made by a person with care, the absent parent;

($b$) where the application has been made by an absent parent, the person with
care;
\end{enumerate}

“partner” has the same meaning as in paragraph (2) of regulation 1 of the
Maintenance Assessments and Special Cases Regulations\footnote{\frenchspacing Paragraph (2) of regulation 1 has been amended by S.I. 1993/913, 1995/1045 and 3261.};

“relevant person” means—
\begin{enumerate}\item[]
($a$) an absent parent, or a person who is treated as an absent parent under
regulation 20 of the Maintenance Assessments and Special Cases Regulations
(persons treated as absent parents), whose liability under a maintenance
assessment may be affected by any departure direction given following an
application;

($b$) a person with care, or a child to whom section 7 of the Act applies, where
the amount of child support maintenance payable under a maintenance assessment
relevant to that person with care or that child may be affected by any departure
direction given following an application.
\end{enumerate}
\end{enumerate}

(3) In these Regulations, unless the context otherwise requires, a reference—
\begin{enumerate}\item[]
($a$) to the Schedule, is to the Schedule to these Regulations;

($b$) to a numbered regulation is to the regulation in these Regulations bearing
that number;

($c$) in a regulation or the Schedule to a numbered paragraph is to the paragraph
in that regulation or the Schedule bearing that number;

($d$) in a paragraph to a lettered or numbered sub-paragraph is to the
sub-paragraph in that paragraph bearing that letter or number.
\end{enumerate}

\subsection[2. Documents]{Documents}

2.—(1) Except where express provision is made to the contrary, where,
under any provision of these Regulations—
\begin{enumerate}\item[]
($a$) any document is given or sent to the Secretary of State, that document shall,
subject to paragraph (2), be treated as having been so given or sent on the date
it is received by the Secretary of State; and

($b$) any document is given or sent to any other person, that document shall, if
sent by post to that person’s last known or notified address, and subject to
paragraph (3), be treated as having been given or sent on the second day after
the day of posting, excluding any Sunday or any day which is a Bank Holiday in
England, Wales, Scotland or Northern Ireland under the Banking and Financial
Dealings Act 1971\footnote{\frenchspacing 1971 c. 80.}.
\end{enumerate}

(2) The Secretary of State may treat any document given or sent to him as given
or sent on such day, earlier than the day it was received by him, as he may
determine, if he is satisfied that there was unavoidable delay in his receiving
the document in question.

(3) Where, by any provision of these Regulations, and in relation to a
particular application, notice or notification—
\begin{enumerate}\item[]
($a$) more than one document is required to be given or sent to a person, and more
than one such document is sent by post to that person but not all the documents
are posted on the same day; or

($b$) documents are required to be given or sent to more than one person, and not
all such documents are posted on the same day,
\end{enumerate}
all those documents shall be treated as having been posted on the later or, as
the case may be, the latest day of posting.

\subsection[3. Determination of amounts]{Determination of amounts}

3.—(1) Where any amount is required to be determined for
the purposes of these Regulations, it shall be determined as a weekly amount
and, except where the context otherwise requires, any reference to such an
amount shall be construed accordingly.

(2) Where any calculation made under these Regulations results in a fraction of
a penny that fraction shall be treated as a penny if it is either one half or
exceeds one half and shall be otherwise disregarded.

\section[Part II --- Procedure on an application for a departure direction and preliminary consideration]{Part II\\*Procedure on an application for a departure direction and preliminary consideration}

\renewcommand\parthead{--- Part II}

\subsection[4, Application for a departure direction]{Application for a departure direction}

4.—(1) Every application shall
be made in writing on a form (a “departure direction application form”) provided
by the Secretary of State, or in such other manner, being in writing, as the
Secretary of State may accept as sufficient in the circumstances of any
particular case.

(2) Departure direction application forms shall be supplied without charge by
such persons as the Secretary of State authorises for that purpose.

(3) Every application shall be given or sent to the Secretary of State or to
such persons as he may authorise for that purpose.

(4) Where an application is defective at the date when it is received, or has
been made in writing but not on the departure direction application form
provided by the Secretary of State, the Secretary of State may refer that
application to the person who made it or, as the case may be, supply him with a
departure direction application form.

(5) In a case to which paragraph (4) applies, if the departure direction
application form is received by the Secretary of State properly completed—
\begin{enumerate}\item[]
($a$) within the specified period, he shall treat the application as if it had been
duly made in the first instance;

($b$) outside the specified period, unless he is satisfied that the delay has been
unavoidable, he shall treat the application as a fresh application made on the
date upon which the properly completed departure direction application form was
received.
\end{enumerate}

(6) An application which is made on a departure direction application form is,
for the purposes of paragraph (5), properly completed if completed in accordance
with the instructions on the form and defective if not so completed.

(7) In a case to which paragraph (4) applies, the specified period for the
purposes of paragraph (5) shall be the period of 14 days commencing with the
date upon which, in accordance with paragraph (4), the application is referred
to the person who made the defective application or a departure direction
application form is given or sent to the person who made a written application
but not on a departure direction application form.

(8) For the purposes of paragraph (7), the provisions of regulation 2 shall
apply to an application referred to in paragraph (4).

(9) A person applying for a departure direction may authorise a representative,
whether or not legally qualified, to receive notices and other documents on his
behalf, and to act on his behalf in relation to an application.

(10) Where a person has, under paragraph (9), authorised a representative who is
not legally qualified, he shall confirm that authorisation in writing, or as
otherwise required, to the Secretary of State, unless such authorisation has
already been approved by the Secretary of State under regulation 53 of the
Maintenance Assessment Procedure Regulations (authorisation of representative).

(11) This paragraph applies where a departure direction has effect, and a later
application by the applicant in response to whose application that direction was
given is made on grounds which are—
\begin{enumerate}\item[]
($a$) additional to the grounds in respect of which the earlier direction was
given;

($b$) not additional to the grounds in respect of which the earlier direction was
given but there has been a change of circumstances in respect of one or more but
not all of those grounds.
\end{enumerate}

(12) Where—
\begin{enumerate}\item[]
($a$) paragraph (11)($a$) applies, the later application may be treated as an
application in respect of which the earlier direction was given;

($b$) paragraph (11)($b$) applies, the later application may be treated as an
application in respect of which the earlier direction was given in relation to
which there have been no changes of circumstances.
\end{enumerate}

(13) Where a departure direction has effect and the Secretary of State is
satisfied that a ground in respect of which the application for that direction
was made no longer applies, he shall treat the applicant for that direction as
having applied for a later direction in respect of the grounds which remain
applicable.

(14) Regulation 8(1) shall apply to cases to which paragraph (11) applies but
only in relation to the additional grounds or, as the case may be, those in
relation to which there has been stated to be a change of circumstances and
shall not apply to cases to which paragraph (13) applies.

\subsection[5. Amendment or withdrawal of application]{Amendment or withdrawal of application}

5. A person who has made an application
may amend or withdraw his application by notice in writing to the Secretary of
State at any time prior to a determination being made in relation to that
application.

\subsection[6. Provision of information]{Provision of information}

6.—(1) Where an application has been made, the Secretary
of State may request further information or evidence from the applicant to
enable that application to be determined.

(2) Any information or evidence requested by the Secretary of State in
accordance with paragraph (1) shall be given within 14 days of the request for
such information or evidence having been given or sent.

(3) Where the time limit specified in paragraph (2) is not complied with, the
Secretary of State may determine that application, in the absence of that
information or evidence.

\subsection[7. Rejection of application on completion of a preliminary consideration]{Rejection of application on completion of a preliminary consideration}

7.—(1) The
Secretary of State may, on completing a preliminary consideration of an
application, reject that application on the ground set out in section 28B(2)($b$)
of the Act if it appears to him that the difference between the current amount
and the revised amount is less than £1.00.

(2) Where an application has been rejected in accordance with paragraph (1), the
Secretary of State shall, as soon as reasonably practicable, give notice of that
rejection to the relevant persons.

\subsection[8. Procedure in relation to the determination of an application]{Procedure in relation to the determination of an application}

8.—(1) Subject to
paragraph (4), where an application has not failed within the meaning of section
28D of the Act, the Secretary of State shall—
\begin{enumerate}\item[]
($a$) give notice of that application to the relevant persons other than the
applicant;

($b$) send to them details of the grounds on which the application has been made
and any relevant information or evidence the applicant has given, except
details, information or evidence falling within paragraph (2);

($c$) invite representations in writing from the relevant persons other than the
applicant on any matter relating to that application; and

($d$) set out the provisions of paragraphs (2), (5) and (6) in relation to such
representations.
\end{enumerate}

(2) The details, information or evidence referred to in paragraphs (1)($b$), (6)
and (7) are—
\begin{enumerate}\item[]
($a$) medical evidence or medical advice that has not been disclosed to the
applicant or a relevant person and which the Secretary of State considers would
be harmful to the health of the applicant or that relevant person if disclosed
to him;

($b$) the address of a relevant person, or of any child in relation to whom the
assessment was made in respect of which the application has been made, or any
other information which could reasonably be expected to lead to that person or
that child being located, where that person has not agreed to disclosure of that
address or that information, it is not known to the other party to that
assessment and—
\begin{enumerate}\item[]
(i) the Secretary of State is satisfied that that address or that information is
not necessary for the determination of that application; or

(ii) the Secretary of State is satisfied that that address or that information is
necessary for the determination of that application and that there would be a
risk of harm or undue distress to that person or that child if disclosure were
made.
\end{enumerate}
\end{enumerate}

(3) Subject to paragraph (4), the notice referred to in paragraph (1)($a$) shall
be given as soon as reasonably practicable after—
\begin{enumerate}\item[]
($a$) completion of the preliminary consideration of that application under section
28B of the Act; or

($b$) where the Secretary of State has requested information or evidence under
regulation 6, receipt of that information or evidence or the expiry of the
period of 14 days referred to in regulation 6(2).
\end{enumerate}

(4) The provisions of paragraphs (1) and (3) shall not apply where information
or evidence requested in accordance with regulation 6 has not been received by
the Secretary of State within the period specified in paragraph (2) of that
regulation and the Secretary of State is satisfied on the information or
evidence available to him that a departure direction should not be given.

(5) Where the Secretary of State does not receive written representations from a
relevant person within 14 days of the date on which representations were invited
under paragraph (1), (6) or (7) he may, in the absence of written
representations from that person, proceed to determine the application.

(6) The Secretary of State may, if he considers it reasonable to do so, send to
the applicant a copy of any written representations made following an invitation
under paragraph (1)($c$), whether or not they were received within the time
specified in paragraph (5), except to the extent that the representations
contain information or evidence which falls within paragraph (2), and invite him
to submit representations in writing on any matters contained in those
representations.

(7) Where any information or evidence requested by the Secretary of State under
regulation 6 is received after notification has been given under paragraph (1),
the Secretary of State may, if he considers it reasonable to do so and except
where that information or evidence falls within paragraph (2), send a copy of
such information or evidence to the relevant persons and invite them to submit
representations in writing on that information or evidence.

(8) In deciding whether to give a departure direction, the Secretary of State
shall take into account—
\begin{enumerate}\item[]
($a$) any information or evidence given by the applicant for that direction; and

($b$) any written representations made by the applicant or by a relevant person and
received by him at the date upon which he determines the application, and may in
addition take into account—
\begin{enumerate}\item[]
(i) any relevant information or evidence received by him or by a child support
officer, in relation to any application for a maintenance assessment or for a
review of a maintenance assessment made in respect of the absent parent, person
with care and any child in respect of whom the current assessment was made;

(ii) any relevant information or evidence acquired by him in connection with any
of his functions under any of the benefit Acts or the Jobseekers Act 1995\footnote{\frenchspacing 1995 c. 18.}.
\end{enumerate}
\end{enumerate}

(9) Where the Secretary of State has determined an application he shall, as soon
as is reasonably practicable—
\begin{enumerate}\item[]
($a$) notify the relevant persons of that determination;

($b$) where a departure direction has been given, refer the case to a child support
officer.
\end{enumerate}

(10) A notification under paragraph (9)($a$) shall set out—
\begin{enumerate}\item[]
($a$) the reasons for that determination;

($b$) where a departure direction has been given, the basis on which the amount of
child support maintenance is to be fixed by any assessment made in consequence
of that direction.
\end{enumerate}

(11) The Secretary of State may reconsider any application which has been
determined by him or by a child support appeal tribunal under section 28D(1)($b$)
of the Act where, after the determination of that application, he receives
further information or evidence which he is satisfied is relevant to that
application.

\subsection[9. Departure directions and persons in receipt of income support or income-based
jobseeker’s allowance]{Departure directions and persons in receipt of income support or income-based
jobseeker’s allowance}

9.—(1) The costs referred to in regulations 13 to 18 shall
not constitute special expenses where they are incurred by a person to or in
respect of whom income support or income-based jobseeker’s allowance is paid.

(2) A transfer shall not constitute a transfer of property for the purposes of
paragraph 3(1)($b$) or 4(1)($b$) of Schedule 4B to the Act, or of regulations 21 and
22, where the application is made by an absent parent to or in respect of whom
income support or income-based jobseeker’s allowance is paid at the time that
application is made.

(3) A case shall not constitute a case under regulations 23 to 29 where the
application is made by an absent parent to or in respect of whom income support
or income-based jobseeker’s allowance is paid.

\subsection[10. Departure directions and interim maintenance assessments]{Departure directions and interim maintenance assessments}

10.—(1) For the purposes
of section 28A(1) of the Act, the term “maintenance assessment” does not
include—
\begin{enumerate}\item[]
($a$) a Category A or Category C interim maintenance assessment;

($b$) a Category B interim maintenance assessment where the application is made
under paragraph 2 of Schedule 4B to the Act in respect of expenses prescribed by
regulation 18 and that Category B interim maintenance assessment was made
because the applicant fell within paragraph (3)($b$) of regulation 8 of the
Maintenance Assessment Procedure Regulations\footnote{\frenchspacing S.I. 1992/1813. Regulation 8 was substituted by S.I. 1995/3261.};

($c$) a Category D interim maintenance assessment, where the application is made
under paragraph 3 or 4 of Schedule 4B to the Act or by an absent parent under
paragraph 2 or 5 of that Schedule.
\end{enumerate}

(2) For the purposes of this regulation, Category A, Category B, Category C and
Category D interim maintenance assessments are defined in regulation 8(3) of the
Maintenance Assessment Procedure Regulations (categories of interim maintenance
assessment).

\subsection[11. Departure application and review under section 17 of the Act]{Departure application and review under section 17 of the Act}

11. Where the
effective date of any fresh assessment made on completion of a review under
section 17 of the Act\footnote{\frenchspacing Section 17 of the Child Support Act 1991 was amended by section 12 of the Child Support Act 1995.} is later than the effective date of any departure
direction given in response to an application for a direction, the provisions of
regulations 20, 21 and 22 of the Maintenance Assessment Procedure Regulations
shall apply to that review as if for references in those regulations to the
original assessment there were substituted references to the fresh assessment
made in consequence of the departure direction.

\subsection[12. Meaning of “benefit” for the purposes of section 28E of the Act]{Meaning of “benefit” for the purposes of section 28E of the Act}

12. For the
purposes of section 28E of the Act, “benefit” means income support, income-based
jobseeker’s allowance, family credit, disability working allowance, housing
benefit, and council tax benefit.

\section[Part III --- Special expenses]{Part III\\*Special expenses}

\renewcommand\parthead{--- Part III}

\subsection[13. Costs incurred in travelling to work]{Costs incurred in travelling to work}

13.—(1) Subject to
paragraphs (2) and (3), the following costs shall constitute expenses for the
purposes of paragraph 2(2) of Schedule 4B to the Act where they are incurred by
the applicant for the purposes of travel between his home and his normal place
of work—
\begin{enumerate}\item[]
($a$) the cost of purchasing a ticket for such travel;

($b$) the cost of purchasing fuel, where such travel is by a vehicle which is not
carrying fare-paying passengers; or

($c$) in exceptional circumstances, the taxi fare for a journey which must
unavoidably be undertaken during hours when no other reasonable mode of travel
is available,
\end{enumerate}
and any minor incidental costs, such as tolls or fees for the use of a
particular road or bridge, incurred in connection with such travel.

(2) Where the Secretary of State considers any costs referred to in paragraph
(1) to be unreasonably high or to have been unreasonably incurred he may
substitute such lower amount as he considers reasonable, including a nil amount.

(3) Costs which can be set off against the income of the applicant under the
Income and Corporation Taxes Act 1988\footnote{\frenchspacing 1988 c. 1.} shall not constitute expenses for the
purposes of paragraph (1).

\subsection[14. Contact costs]{Contact costs}

14.—(1) Where at the time a departure direction is applied for a
set pattern has been established as to frequency of contact between the absent
parent and a child in respect of whom the current assessment was made, the
following costs, based upon that pattern and incurred by that absent parent for
the purpose of maintaining contact with that child, shall, subject to paragraphs
(2) to (6), constitute expenses for the purposes of paragraph 2(2) of Schedule
4B to the Act—
\begin{enumerate}\item[]
($a$) the cost of purchasing a ticket for travel for the purpose of maintaining
that contact;

($b$) the cost of purchasing fuel, where travel is for the purpose of maintaining
that contact and is by a vehicle which is not carrying fare-paying passengers;
or

($c$) the taxi fare for a journey or part of a journey to maintain that contact
where the Secretary of State is satisfied that the disability of the absent
parent makes it impracticable to use any other form of transport which might
otherwise have been available to him,
\end{enumerate}
and any minor incidental costs, such as tolls or fees for the use of a
particular road or bridge, incurred in connection with such travel.

(2) Subject to paragraph (3), where the Secretary of State considers any costs
referred to in paragraph (1) to be unreasonably high or to have been
unreasonably incurred he may substitute such lower amount as he considers
reasonable, including a nil amount.

(3) Any lower amount substituted by the Secretary of State under paragraph (2)
shall not be so low as to make it impossible, in the Secretary of State’s
opinion, for contact to be maintained at the frequency specified in any court
order made in respect of the absent parent and the child mentioned in paragraph
(1) where the absent parent is maintaining contact at that frequency.

(4) Paragraph (1) shall not apply where regulation 20 of the Maintenance
Assessments and Special Cases Regulations (persons treated as absent parents)
applies to the applicant.

(5) Where sub-paragraph ($c$) of paragraph (1) applies and the applicant has, at
the date an application is made, received, or at that date is in receipt of,
financial assistance from any source to meet, wholly or in part, costs of
maintaining contact with the child who is referred to in paragraph (1), which
arise wholly from his disability and which are in excess of the costs which
would be incurred if that disability did not exist, only the net amount of the
costs referred to in that sub-paragraph, after the deduction of that financial
assistance, shall constitute special expenses for the purposes of paragraph 2(2)
of Schedule 4B to the Act.

(6) For the purposes of this regulation, a person is disabled if he is blind,
deaf or dumb or is substantially or permanently handicapped by illness, injury,
mental disorder or congenital deformity.

(7) Where, at the time a departure direction is applied for, no set pattern has
been established as to frequency of contact between the absent parent and a
child in respect of whom the current assessment was made, but the Secretary of
State is satisfied that that absent parent and the person with care of that
child have agreed upon a pattern of contact for the future, the costs mentioned
in paragraph (1) and which are based upon that intended pattern of contact shall
constitute expenses for the purposes of paragraph 2(2) of Schedule 4B to the
Act, and paragraphs (2) to (6) shall apply to that application.

\subsection[15. Illness or disability]{Illness or disability}

15.—(1) Subject to paragraphs (2) to (4), the costs being
met by the applicant in respect of the items listed in sub-paragraphs ($a$) to
($m$), which arise from long-term illness or disability of that applicant or a
dependant of that applicant and which are in excess of the costs which would be
incurred if that illness or disability did not exist, shall constitute special
expenses for the purposes of paragraph 2(2) of Schedule 4B to the Act—
\begin{enumerate}\item[]
($a$) personal care and attendance;

($b$) personal communication needs;

($c$) mobility;

($d$) domestic help;

($e$) medical aids where these cannot be provided under the health service;

($f$) heating;

($g$) clothing;

($h$) laundry requirements;

($i$) payments for food essential to comply with a diet recommended by a medical
practitioner;

($j$) adaptations required to the applicant’s home;

($k$) day care;

($l$) rehabilitation; or

($m$) respite care.
\end{enumerate}

(2) Where the Secretary of State considers any costs referred to in paragraph
(1) to be unreasonably high or to have been unreasonably incurred he may
substitute such lower amount as he considers reasonable, including a nil amount.

(3) Where—
\begin{enumerate}\item[]
($a$) an applicant or his dependant has, at the date an application is made,
received, or at that date is in receipt of, financial assistance from any source
in respect of his long-term illness or disability or that of his dependant; or

($b$) that applicant or his dependant is adjudged eligible for either of the
allowances referred to in paragraph (4),
\end{enumerate}
only the net amount of the costs incurred in respect of the items listed in
paragraph (1), after the deduction of the financial assistance referred to in
sub-paragraph ($a$) and, where applicable, the allowance referred to in
sub-paragraph ($b$) shall constitute special expenses for the purposes of
paragraph 2(2) of Schedule 4B to the Act.

(4) Where the Secretary of State considers that a person who has made an
application in respect of special expenses falling within paragraph (1) or his
dependant may be entitled to disability living allowance under section 71 of the
Contributions and Benefits Act or attendance allowance under section 64 of that
Act—
\begin{enumerate}\item[]
($a$) if that applicant or his dependant has at the date of that application, or
within a period of six weeks beginning with the giving or sending to him of
notification of the possibility of entitlement to either of those allowances,
applied for either of those allowances, the application made by that applicant
shall not be determined until a decision has been made by the adjudicating
authority on the eligibility for that allowance of that applicant or that
dependant;

($b$) if that applicant or his dependant has failed to apply for either of those
allowances within the six week period specified in sub-paragraph ($a$), the
Secretary of State shall determine the application for a departure direction
made by that applicant on the basis that that applicant has income equivalent to
the highest rate prescribed in respect of that allowance by or under those
sections.
\end{enumerate}

(5) For the purposes of this regulation, a dependant of an applicant shall be—
\begin{enumerate}\item[]
($a$) where the applicant is an absent parent—
\begin{enumerate}\item[]
(i) the partner of that absent parent;

(ii) any child of whom that absent parent or his partner is a parent and who
lives with them; or
\end{enumerate}

($b$) where the applicant is a parent with care—
\begin{enumerate}\item[]
(i) the partner of that parent with care;

(ii) any child of whom that parent with care or her partner is a parent and who
lives with them, except any child in respect of whom the absent parent against
whom the current assessment is made is the parent.
\end{enumerate}
\end{enumerate}

(6) For the purposes of this regulation—
\begin{enumerate}\item[]
($a$) a person is disabled if he is blind, deaf or dumb or is substantially or
permanently handicapped by illness, injury, mental disorder or congenital
deformity;

($b$) “long-term illness” means an illness from which the applicant or his
dependant is suffering at the date of the application and which is likely to
last for at least 52 weeks from that date or if likely to be shorter than 52
weeks, for the rest of the life of that applicant or his dependant;

($c$) “the health service” has the same meaning as in section 128 of the National
Health Service Act 1977\footnote{\frenchspacing 1977 c. 49.} or in section 108(1) of the National Health Service
(Scotland) Act 1978\footnote{\frenchspacing 1978 c. 29.}.
\end{enumerate}

\subsection[16. Debts incurred before the absent parent became an absent parent]{Debts incurred before the absent parent became an absent parent}

16.—(1) Subject
to paragraphs (2) to (4), repayment of debts incurred—
\begin{enumerate}\item[]
($a$) for the joint benefit of the applicant and the non-applicant parent;

($b$) for the benefit of the non-applicant parent where the applicant remains
legally liable to repay the whole or part of that debt;

($c$) for the benefit of any person who at the time the debt was incurred—
\begin{enumerate}\item[]
(i) was a child;

(ii) lived with the applicant and non-applicant parent; and

(iii) of whom the applicant or the non-applicant parent is the parent, or both
are the parents; or
\end{enumerate}

($d$) for the benefit of any child with respect to whom the current assessment was
made,
\end{enumerate}
shall constitute expenses for the purposes of paragraph 2(2) of Schedule 4B to
the Act where those debts were incurred before the absent parent became an
absent parent in relation to a child with respect to whom the current assessment
was made and at a time when the applicant and the non-applicant parent were a
married or unmarried couple who were living together.

(2) Paragraph (1) shall not apply to repayment of—
\begin{enumerate}\item[]
($a$) a debt which would otherwise fall within paragraph (1) where the applicant
has retained for his own use and benefit the asset the purchase of which
incurred the debt;

($b$) a debt incurred for the purposes of any trade or business;

($c$) a gambling debt;

($d$) a fine imposed on the applicant;

($e$) unpaid legal costs in respect of separation or divorce from the non-applicant
parent;

($f$) amounts due after use of a credit card;

($g$) a debt incurred by the applicant to pay any of the items listed in
sub-paragraphs ($c$) to ($f$) and ($j$);

($h$) amounts payable by the applicant under a mortgage or loan taken out on the
security of any property except where that mortgage or loan was taken out to
facilitate the purchase of, or to pay for repairs or improvements to, any
property which is the home of the parent with care and any child in respect of
whom the current assessment was made;

($i$) amounts payable by the applicant in respect of a policy of insurance of a
kind referred to in paragraph 3(4) or (5) of Schedule 3 to the Maintenance
Assessments and Special Cases Regulations\footnote{\frenchspacing Paragraph 3(4) was amended by S.I. 1995/1045 and paragraph 3(5) by S.I. 1994/227.} (eligible housing costs) except
where that policy of insurance was obtained or retained to discharge a mortgage
or charge taken out to facilitate the purchase of, or to pay for repairs or
improvements to, any property which is the home of the parent with care and any
child in respect of whom the current assessment was made;

($j$) a bank overdraft except where the overdraft was, at the time it was taken
out, agreed to be for a specified amount repayable over a specified period;

($k$) a loan obtained by the applicant, other than a loan obtained from a
qualifying lender or the applicant’s current or former employer;

($l$) a debt in respect of which a departure direction has already been given and
which has not been repaid during the period for which that direction was in
force except where the maintenance assessment in respect of which that direction
was given was cancelled or ceased to have effect and, during the period for
which that direction was in force, a further maintenance assessment was made in
respect of the same applicant, non-applicant and qualifying child with respect
to whom the earlier assessment was made; or

($m$) any other debt which the Secretary of State is satisfied it is reasonable to
exclude.
\end{enumerate}

(3) Except where the repayment is of an amount which is payable under a mortgage
or loan, or in respect of a policy of insurance, which falls within the
exception set out in sub-paragraph ($h$) or ($i$) of paragraph (2), repayment of a
debt shall not constitute expenses for the purposes of paragraph (1) where the
Secretary of State is satisfied that the applicant has taken responsibility for
repayment of that debt, as, or as part of, a financial settlement with the
non-applicant parent or by virtue of a court order.

(4) Where an applicant has incurred a debt partly to repay a debt or debts
repayment of which would have fallen within paragraph (1), the repayment of that
part of the debt incurred which is referable to the debts repayment of which
would have fallen within that paragraph shall constitute expenses for the
purposes of paragraph 2(2) of Schedule 4B to the Act.

(5) For the purposes of this regulation—
\begin{enumerate}\item[]
($a$) “married or unmarried couple” has the meaning set out in regulation 1 of the
Maintenance Assessments and Special Cases Regulations;

($b$) “non-applicant parent” means—
\begin{enumerate}\item[]
(i) where the applicant is the person with care, the absent parent;

(ii) where the applicant is the absent parent, the partner of that absent parent
at the time the debt in respect of which the application is made was entered
into;
\end{enumerate}

($c$) “qualifying lender” has the meaning given to it in section 376(4) of the
Income and Corporation Taxes Act 1988\footnote{\frenchspacing 1988 c. 1.};

($d$) “repairs and improvements” means major repairs necessary to maintain the
fabric of the home and any of the measures set out in sub-paragraphs ($a$) to ($j$)
of paragraph 2 of Schedule 3 to the Maintenance Assessments and Special Cases
Regulations (eligible housing costs) and other improvements which the Secretary
of State considers reasonable in the circumstances where those measures or other
improvements are undertaken with a view to improving fitness for occupation of
the home.
\end{enumerate}

\subsection[17. Pre-1993 financial commitments]{Pre-1993 financial commitments}

17.—(1) A financial commitment entered into by an
absent parent before 5th April 1993, except any commitment of a kind listed in
paragraph (2)($b$) to ($g$) and ($j$) of regulation 16 or which has been wholly or
partly taken into account in the calculation of a maintenance assessment shall
constitute expenses for the purposes of paragraph 2(2) of Schedule 4B to the Act
where—
\begin{enumerate}\item[]
($a$) there was in force on 5th April 1993 and at the date that commitment was
entered into, a court order or maintenance agreement made before 5th April 1993
in respect of that absent parent and every child in respect of whom, before that
date, he was, or was found, or adjudged to be, the parent; and

($b$) the Secretary of State is satisfied that it is impossible for the absent
parent to withdraw from that commitment or unreasonable to expect him to do so.
\end{enumerate}

(2) For the purposes of paragraph (1)—
\begin{enumerate}\item[]
($a$) “court order” means an order made under the enactments listed in or
prescribed under section 8(11) of the Act, for the making or securing the making
of financial provision for the benefit of a child in respect of whom the current
assessment was made;

($b$) “maintenance agreement” means an agreement in writing for the making or
securing the making of financial provision for the benefit of a child in respect
of whom the current assessment was made.
\end{enumerate}

\subsection[18. Costs incurred in supporting certain children]{Costs incurred in supporting certain children}

18.—(1) The costs incurred by a
parent in supporting a child who is not his child but who is part of his family
(a “relevant child”) shall constitute special expenses for the purposes of
paragraph 2(2) of Schedule 4B to the Act if the conditions set out in paragraph
(2) are satisfied and shall, if those conditions are satisfied, equal the amount
specified in paragraph (3).

(2) The conditions referred to in paragraph (1) are—
\begin{enumerate}\item[]
($a$) the child became a relevant child prior to 5th April 1993;

($b$) subject to paragraph (7), the liability of the absent parent of a relevant
child to pay maintenance to or for the benefit of that child under a court
order, a written maintenance agreement or a maintenance assessment is less than
the amount specified in paragraph (4), or there is no such liability; and

($c$) the net income of the parent’s current partner where the relevant child is
the child of that partner, calculated in accordance with paragraph (5), is less
than the amount calculated in accordance with paragraph (6) (“the partner’s
outgoings”).
\end{enumerate}

(3) The amount referred to in paragraph (1) constituting special expenses for a
case falling within this regulation is the difference between the amount
specified in paragraph (4) and, subject to paragraph (7), the liability of the
absent parent of a relevant child to pay maintenance of a kind mentioned in
paragraph (2)($b$), and if there is no such liability is the amount specified in
paragraph (4).

(4) The amount referred to in paragraphs (2)($b$) and (3) is the aggregate of—
\begin{enumerate}\item[]
($a$) an amount in respect of each relevant child equal to the personal allowance
for that child specified in column (2) of paragraph 2 of the relevant Schedule
(income support personal allowance);

($b$) if the conditions set out in paragraph 14($b$) and ($c$) that Schedule (income
support disabled child premium) are satisfied in respect of a relevant child, an
amount equal to the amount specified in column (2) of paragraph 15(6) of that
Schedule in respect of each such child;

($c$) an amount equal to the income support family premium specified in paragraph 3
of that Schedule that would be payable if the parent were a claimant, except
where the family includes other children of the parent; and

($d$) an amount equal to the income support lone parent premium specified in column
(2) of paragraph 15(1) of that Schedule that would be payable, if the parent
were a claimant, except where the family includes children of the parent.
\end{enumerate}

(5) For the purposes of paragraph (2)($c$), the net income of the parent’s partner
shall be the aggregate of—
\begin{enumerate}\item[]
($a$) the income of that partner, calculated in accordance with regulation 7(1) of
the Maintenance Assessments and Special Cases Regulations (but excluding the
amount mentioned in sub-paragraph ($d$) of that regulation) as if that partner
were an absent parent to whom that regulation applied;

($b$) the child benefit payable in respect of each relevant child; and

($c$) any income, other than earnings, in excess of £10.00 per week in respect of
each relevant child.
\end{enumerate}

(6) For the purposes of paragraph (2)($c$), a current partner’s outgoings shall be
the aggregate of—
\begin{enumerate}\item[]
($a$) an amount equal to the amount specified in column (2) of paragraph 1(1)($e$) of
the relevant Schedule (income support personal allowance for a single claimant
aged not less than 25);

($b$) where a departure direction has already been given in a case falling within
regulation 27 in respect of the housing costs attributable to the partner, the
amount determined in accordance with regulation 40(7) as the housing costs the
partner is able to contribute;

($c$) the amount of any reduction in the parent’s exempt income, calculated under
paragraph (1) of regulation 9 of the Maintenance Assessments and Special Cases
Regulations\footnote{\frenchspacing Paragraph (1) of regulation 9 was amended by regulation 44(2) of S.I. 1995/1045. Paragraph (2) was amended by regulation 9(2)($c$) of S.I. 1993/913 and regulation 44(3) of S.I. 1995/1045.}, in consequence of the application of paragraph (2) of that
regulation; and

($d$) the amount specified in paragraph (3).
\end{enumerate}

(7) The Secretary of State may, if he is satisfied that it is appropriate in the
particular circumstances of the case, treat a liability of a kind mentioned in
paragraph (2)($b$) as not constituting a liability for the purposes of that
paragraph and of paragraph (3).

(8) For the purposes of this regulation—
\begin{enumerate}\item[]
($a$) a child who is not the child of a particular person is a part of that
person’s family where that child is the child of a current or former partner of
that person;

($b$) “relevant Schedule” means Schedule 2 to the Income Support (General)
Regulations 1987\footnote{\frenchspacing S.I. 1987/1967. Paragraphs 1 and 2 of Schedule 2 were substituted by Schedule 4 to S.I. 1995/559; paragraph 15 was substituted by Schedule 5 to that instrument.}.
\end{enumerate}

\subsection[19. Special expenses for a case falling within regulation 13, 14, 16 or 17]{Special expenses for a case falling within regulation 13, 14, 16 or 17}

19.—(1)
This regulation applies where the expenses of an applicant fall within one or
more of the descriptions of expenses falling within regulation 13 (travel to
work costs), 14 (contact costs), 16 (debts incurred before the absent parent
became an absent parent) or 17 (pre-1993 financial commitments).

(2) Special expenses for the purposes of paragraph 2(2) of Schedule 4B to the
Act in respect of the expenses mentioned in paragraph (1) shall be—
\begin{enumerate}\item[]
($a$) where the expenses fall within only one description of expenses, those
expenses in excess of £15.00;

($b$) where the expenses fall within more than one description of expenses, the
aggregate of those expenses in excess of £15.00.
\end{enumerate}

\subsection[20. Application for a departure direction in respect of special expenses other than
those with respect to which a direction has already been given]{Application for a departure direction in respect of special expenses other than
those with respect to which a direction has already been given}

20. Where a
departure direction with respect to special expenses falling within one or more
of the descriptions of expenses falling within regulation 13, 14, 16 or 17 has
already been given and an application with respect to special expenses falling
within one or more of those descriptions of expenses is made where none of those
expenses are ones with respect to which the earlier direction has been given,
the special expenses with respect to which any later direction is given shall be
the expenses, determined in accordance with regulation 13, 14, 16 or 17, as the
case may be, with respect to which the later application is made, and the
provisions of regulation 19 shall not apply.

\section[Part IV --- Property or capital transfers]{Part IV\\*Property or capital transfers}

\renewcommand\parthead{--- Part IV}

\subsection[21. Prescription of certain terms for the
purposes of paragraphs 3 and 4 of Schedule 4B to the Act]{Prescription of certain terms for the
purposes of paragraphs 3 and 4 of Schedule 4B to the Act}

21.—(1) For the purposes
of paragraphs 3(1)($a$) and 4(1)($a$) of Schedule 4B to the Act—
\begin{enumerate}\item[]
($a$) a court order means an order made—
\begin{enumerate}\item[]
(i) under one or more of the enactments listed in or prescribed under section
8(11) of the Act; and

(ii) in connection with the transfer of property of a kind defined in paragraph
(2);
\end{enumerate}

($b$) an agreement means a written agreement made in connection with the transfer
of property of a kind defined in paragraph (2).
\end{enumerate}

(2) Subject to paragraphs (3) to (5), for the purposes of paragraph 3(1)($b$) and
4(1)($b$) of Schedule 4B to the Act, a transfer of property is a transfer by the
absent parent of his beneficial interest in any asset to the person with care,
to a child in respect of whom the current assessment was made, or to trustees
where the object or one of the objects of the trust is the provision of
maintenance.

(3) Where a transfer of property would not originally have fallen within
paragraph (2) but the Secretary of State is satisfied that some or all of the
amount of that property transferred was subsequently transferred to the person
currently with care of a child in respect of whom the current assessment was
made, the transfer of that property to the person currently with care shall
count as a transfer of property for the purposes of paragraph 3(1)($b$) and
4(1)($b$) of Schedule 4B to the Act.

(4) Where, if the Act had been in force at the time a transfer of property
falling within paragraph (2) was made, the person who, at the time the
application is made is the person with care would have been the absent parent
and the person who, at the time the application is made is the absent parent
would have been the person with care, that transfer shall not count as a
transfer of property for the purposes of this regulation.

(5) For the purposes of paragraph 3(3) of Schedule 4B to the Act, the effect of
a transfer of property is properly reflected in the current assessment if—
\begin{enumerate}\item[]
($a$) the amount of child support maintenance payable under any fresh maintenance
assessment which would be made in consequence of a departure direction differs
from the amount of child support maintenance payable under that current
assessment by less than £1.00; or

($b$) the transfer referred to in paragraph (2) was for a specified period only and
that period ended before the effective date of any departure direction which
would otherwise have been given.
\end{enumerate}

\subsection[22. Value of a transfer of property and its equivalent weekly value for a case
falling within paragraph 3 of Schedule 4B to the Act]{Value of a transfer of property and its equivalent weekly value for a case
falling within paragraph 3 of Schedule 4B to the Act}

22.—(1) Where the conditions
specified in paragraph 3(1) of Schedule 4B to the Act are satisfied, the value
of a transfer of property for the purposes of that paragraph shall be that part
of the transfer made by the absent parent (making allowance for any transfer by
the person with care to the absent parent) which the Secretary of State is
satisfied is in lieu of maintenance.

(2) The Secretary of State shall, in determining the value of a transfer of
property in accordance with paragraph (1), assume that, unless evidence to the
contrary is provided to him—
\begin{enumerate}\item[]
($a$) the person with care and the absent parent had equal beneficial interests in
the assets in relation to which the court order or agreement was made;

($b$) where the person with care was married to the absent parent, one half of the
value of the transfer was a transfer for the benefit of the person with care;
and

($c$) where the person with care has never been married to the absent parent, none
of the value of the transfer was a transfer for the benefit of the person with
care.
\end{enumerate}

(3) The equivalent weekly value of a transfer of property shall be determined in
accordance with the provisions of the Schedule.

(4) For the purposes of regulation 21 and this regulation, the term
“maintenance” means the normal day-to-day living expenses of the child with
respect to whom the current assessment was made.

\section[Part V --- Additional cases]{Part V\\*Additional cases}

\renewcommand\parthead{--- Part V}

\subsection[23. Assets capable of producing income or higher income]{Assets capable of producing income or higher income}

23.—(1)
Subject to paragraphs (2) and (3), a case shall constitute a case for the
purposes of paragraph 5(1) of Schedule 4B to the Act where—
\begin{enumerate}\item[]
($a$) the Secretary of State is satisfied that any asset in which the non-applicant
has a beneficial interest, or which he has the ability to control—
\begin{enumerate}\item[]
(i) is capable of being utilised to produce income but has not been so utilised;

(ii) has been invested in such a way that the income obtained from it is less
than might reasonably be expected;

(iii) is a chose in action which has not been enforced where the Secretary of
State is satisfied that such enforcement would be reasonable;

(iv) in Scotland, is monies due or an obligation owed, whether immediately
payable or otherwise and whether the payment or obligation is secured or not and
the Secretary of State is satisfied that requiring payment of the monies or the
implementation of the obligation would be reasonable; or

(v) has not been sold where the Secretary of State is satisfied that the sale of
the asset would be reasonable;
\end{enumerate}

($b$) any asset has been transferred by the non-applicant to trustees and the
non-applicant is a beneficiary of the trust so created; or

($c$) any asset has become subject to a trust created by legal implication of which
the non-applicant is a beneficiary.
\end{enumerate}

(2) Paragraph (1) shall not apply where—
\begin{enumerate}\item[]
($a$) the total value of the asset or assets referred to in that paragraph does not
exceed £10,000.00 after deduction of the amount owing under any mortgage or
charge on that asset; or

($b$) the Secretary of State is satisfied that any asset referred to in that
paragraph is being retained by the non-applicant to be used for a purpose which
the Secretary of State considers reasonable in all the circumstances of the
case.
\end{enumerate}

(3) No application may be made under this regulation where income support or
income-based jobseeker’s allowance is paid to or in respect of the
non-applicant.

(4) For the purposes of this regulation the term “asset” means—
\begin{enumerate}\item[]
($a$) money, whether in cash or on deposit;

($b$) a beneficial interest in land and rights in or over land;

($c$) shares as defined in section 744 of the Companies Act 1985\footnote{\frenchspacing 1985 c. 6.}, stock and
unit trusts as defined in section 6 of the Charging Orders Act 1979\footnote{\frenchspacing 1979 c. 53.}, gilt
edged securities as defined in paragraph 1 of Schedule 2 to the Capital Gains
Tax Act 1979\footnote{\frenchspacing 1979 c. 14.}, and other similar financial instruments.
\end{enumerate}

(5) For the purposes of paragraph (4) the term “asset” includes any asset
falling within that paragraph which is located outside Great Britain.

\subsection[24. Diversion of income]{Diversion of income}

24. A case shall constitute a case for the purposes of
paragraph 5(1) of Schedule 4B to the Act where—
\begin{enumerate}\item[]
($a$) the non-applicant has the ability to control the amount of income he
receives, including earnings from employment or self-employment and dividends
from shares, whether or not the whole of that income is derived from the company
or business from which his earnings are derived; and

($b$) the Secretary of State is satisfied that the non-applicant has unreasonably
reduced the amount of his income which would otherwise fall to be taken into
account under regulation 7 or 8 of the Maintenance Assessments and Special Cases
Regulations by diverting it to other persons or for purposes other than the
provision of such income for himself.
\end{enumerate}

\subsection[25. Life-style inconsistent with declared income]{Life-style inconsistent with declared income}

25.—(1) Subject to paragraph (2), a
case shall constitute a case for the purposes of paragraph 5(1) of Schedule 4B
to the Act where the Secretary of State is satisfied that the current
maintenance assessment is based upon a level of income of the non-applicant
which is substantially lower than the level of income required to support the
overall life-style of that non-applicant.

(2) Paragraph (1) shall not apply where—
\begin{enumerate}\item[]
($a$) income support or income-based jobseeker’s allowance is paid to or in respect
of the non-applicant;

($b$) the Secretary of State is satisfied that the life-style of the non-applicant
is paid for—
\begin{enumerate}\item[]
(i) out of capital belonging to him; or

(ii) by his partner unless the non-applicant is able to influence or control the
amount of income received by that partner.
\end{enumerate}
\end{enumerate}

(3) Where the Secretary of State is satisfied in a particular case that the
provisions of paragraph (1) would apply but for the provisions of paragraph
(2)($b$)(ii), he may, whether or not any application on that ground has been made,
consider whether the case falls within regulation 27.

\subsection[26. Unreasonably high housing costs]{Unreasonably high housing costs}

26. A case shall constitute a case for the
purposes of paragraph 5(1) of Schedule 4B to the Act where—
\begin{enumerate}\item[]
($a$) the housing costs of the non-applicant exceed the limits set out in paragraph
(1) of regulation 18 of the Maintenance Assessments and Special Cases
Regulations (excessive housing costs);

($b$) the non-applicant falls within paragraph (2) of that regulation or would fall
within that paragraph if it applied to parents with care; and

($c$) the Secretary of State is satisfied that the housing costs of the
non-applicant are substantially higher than is necessary taking into account any
special circumstances applicable to that non-applicant.
\end{enumerate}

\subsection[27. Partner’s contribution to housing costs]{Partner’s contribution to housing costs}

27. A case shall constitute a case for
the purposes of paragraph 5(1) of Schedule 4B to the Act where a partner of the
non-applicant occupies the home with him and the Secretary of State considers
that it is reasonable for that partner to contribute to the payment of the
housing costs of the non-applicant.

\subsection[28. Unreasonably high travel costs]{Unreasonably high travel costs}

28. A case shall constitute a case for the
purposes of paragraph 5(1) of Schedule 4B to the Act where an amount in respect
of travel to work costs has been included in the calculation of exempt income of
the non-applicant under regulation 9(1)($i$) of the Maintenance Assessments and
Special Cases Regulations\footnote{\frenchspacing Sub-paragraph ($i$) was added to regulation 9(1) by regulation 44(2)(b) of S.I. 1995/1045.} (exempt income: calculation or estimation of E)
or, as the case may be, under regulation 10 of those Regulations (exempt income:
calculation or estimation of F)\footnote{\frenchspacing Regulation 10 was amended by regulation 45 of S.I. 1995/1045.} applying regulation 9(1)($i$), and the
Secretary of State is satisfied that, in all the circumstances of the case, that
amount is unreasonably high.

\subsection[29. Travel costs to be disregarded]{Travel costs to be disregarded}

29. A case shall constitute a case for the
purposes of paragraph 5(1) of Schedule 4B to the Act where—
\begin{enumerate}\item[]
($a$) an amount in respect of travel to work costs has, in the calculation of a
maintenance assessment, been included in the calculation of the exempt income of
the non-applicant under regulation 9(1)($i$) of the Maintenance Assessments and
Special Cases Regulations or, as the case may be, under regulation 10 of those
Regulations applying regulation 9(1)($i$); and

($b$) the Secretary of State is satisfied that the non-applicant has sufficient
income remaining after the deduction of the amount that would be payable under
that assessment, had the amount referred to in sub-paragraph ($a$) not been
included in its calculation, for it to be inappropriate for all or part of that
amount to be included in the exempt income of the non-applicant.
\end{enumerate}

\section[Part VI --- Factors to be taken into account for the purposes of section 28F of the Act]{Part VI\\*Factors to be taken into account for the purposes of section 28F of the Act}

\renewcommand\parthead{--- Part VI}

\subsection[30. Factors to be taken into account and not to be taken into account in
determining whether it would be just and equitable to give a departure
direction]{Factors to be taken into account and not to be taken into account in
determining whether it would be just and equitable to give a departure
direction}

30.—(1) The factors to be taken into account in determining whether it
would be just and equitable to give a departure direction in any case shall
include—
\begin{enumerate}\item[]
($a$) where the application is made on any ground—
\begin{enumerate}\item[]
(i) whether, in the opinion of the Secretary of State, the giving of a departure
direction would be likely to result in a relevant person ceasing paid
employment;

(ii) if the applicant is the absent parent, the extent, if any, of his liability
to pay child maintenance under a court order or other agreement in the period
prior to the effective date of the maintenance assessment;
\end{enumerate}

($b$) where an application is made on the ground that the case falls within
regulations 13 to 20 (special expenses), whether, in the opinion of the
Secretary of State—
\begin{enumerate}\item[]
(i) the financial arrangements made by the applicant could have been such as to
enable the whole or part of the expenses cited to be paid without a departure
direction being given;

(ii) the applicant has at his disposal financial resources which are currently
utilised for the payment of expenses other than those arising from essential
everyday requirements and which could be used to pay the whole or part of the
expenses cited.
\end{enumerate}
\end{enumerate}

(2) The following factors are not to be taken into account in determining
whether it would be just and equitable to give a departure direction in any
case—
\begin{enumerate}\item[]
($a$) the fact that the conception of a child in respect of whom the current
assessment was made was not planned by one or both of the parents;

($b$) whether the parent with care or the absent parent was responsible for the
breakdown of the relationship between them;

($c$) the fact that the parent with care or the absent parent has formed a new
relationship with a person who is not a parent of the child in respect of whom
the current assessment was made;

($d$) the existence of particular arrangements for contact with the child in
respect of whom the current assessment was made, including whether any
arrangements made are being adhered to by the parents;

($e$) the failure by an absent parent to make payments under a maintenance order, a
written maintenance agreement, or a maintenance assessment;

($f$) representations made by persons other than the relevant persons.
\end{enumerate}

\section[Part VII --- Effective date and duration of a departure direction]{Part VII\\*Effective date and duration of a departure direction}

\renewcommand\parthead{--- Part VII}

\subsection[31. Refusal to give a
departure direction under section 28F(4)of the Act]{Refusal to give a
departure direction under section 28F(4)of the Act}

31. The Secretary of State
shall not give a departure direction in accordance with section 28F of the Act
if he is satisfied that the difference between the current amount and the
revised amount is less than £1.00.

\subsection[32. Effective date of a departure direction]{Effective date of a departure direction}

32.—(1) Where an application is made on
the grounds set out in section 28A(2)($a$) of the Act (the effect of the current
assessment) and that application is given or sent within 28 days of the date of
notification of the current assessment (whether or not that assessment has been
made following an interim maintenance assessment), a departure direction given
in response to that application shall take effect—
\begin{enumerate}\item[]
($a$) where it is given on grounds that relate to the whole of the period between
the effective date of the current assessment and the date on which that
assessment is made, on the effective date of that assessment;

($b$) in a case not falling within sub-paragraph ($a$), on the first day of the
maintenance period following the date upon which the circumstances giving rise
to that application first arose.
\end{enumerate}

(2) Where an application is made on the grounds set out in section 28A(2)($a$) of
the Act (the effect of the current assessment) and that application is given or
sent later than 28 days after the date of notification of the current assessment
(whether or not that assessment has been made following an interim maintenance
assessment)—
\begin{enumerate}\item[]
($a$) subject to sub-paragraph ($b$), a departure direction given in response to that
application shall take effect on the first day of the maintenance period during
which that application is received;

($b$) where the Secretary of State is satisfied that there was unavoidable delay,
he may, for the purposes of determining the date on which a departure direction
takes effect, treat the application as if it were given or sent within 28 days
of the date of notification of the current assessment.
\end{enumerate}

(3) The provisions of paragraphs (1) and (2) are subject to the provisions of
paragraph (6) and of regulations 47 to 50.

(4) Subject to paragraph (6), where an application for a departure direction is
made on the grounds set out in section 28A(2)($b$) of the Act (a material change
in the circumstances of the case since the current assessment was made), any
departure direction given shall take effect on the first day of the maintenance
period during which the application was received.

(5) An application may be made on the grounds set out in section 28A(2)($b$) of
the Act only if the material change in the circumstances on which it is based
has already occurred.

(6) Where—
\begin{enumerate}\item[]
($a$) an application has been determined in accordance with regulation 15(4)($b$);

($b$) a subsequent application is made with respect to special expenses falling
within regulation 15(1) each of which is an expense in respect of which the
earlier application was made; and

($c$) the Secretary of State is satisfied that there was good cause for the
applicant or his dependant not applying for disability living allowance or, as
the case may be, attendance allowance within the six week period specified in
regulation 15(4)($a$),
\end{enumerate}
any departure direction given in response to the later application shall take
effect from the date that the earlier direction had effect, or would have had
effect if an earlier direction had been given.

(7) Where, under the provisions of regulation 4(12), a later application is
treated as an application in respect of grounds for which the earlier direction
was given, or in respect of grounds for which the earlier direction was given in
relation to which there have been no changes of circumstances, and a direction
is given, that direction shall take effect in accordance with the provisions of
paragraphs (1), (2) and (4) as applied to the additional grounds or, as the case
may be, the grounds in respect of which there has been a change of
circumstances, and the earlier direction shall cease to have effect immediately
before the coming into force of that direction.

(8) Where a direction is given following an application that is treated as
having been made by virtue of the provisions of regulation 4(13), that direction
shall take effect on the first day of the maintenance period during which the
Secretary of State is satisfied that a ground in respect of which the
application for the earlier direction was made no longer applies, and the
earlier direction shall cease to have effect immediately before the direction
that is given takes effect.

\subsection[33. Cancellation of a departure direction following a review under section 16, 17,
18 or 19 of the Act or on a change of circumstances]{Cancellation of a departure direction following a review under section 16, 17,
18 or 19 of the Act or on a change of circumstances}

33.—(1) Where the Secretary
of State is satisfied that, following a review under section 16, 17, 18 or 19 of
the Act or a change in the circumstances of the case, it is no longer
appropriate for a departure direction to continue to have effect, he shall
cancel that direction.

(2) A departure direction that is cancelled under the provisions of paragraph
(1) shall cease to have effect on the first day of the maintenance period during
which the Secretary of State is given or sent, or becomes aware of, the
information which leads him to become satisfied that it is no longer appropriate
for the departure direction to continue to have effect.

(3) Where a departure direction has effect and the applicant in respect of whom
the direction was given applies for a further departure direction in respect of
the same grounds, any departure direction given in response to that application
shall take effect in accordance with the provisions of paragraphs (1), (2) and
(4) of regulation 32, and the earlier direction shall cease to have effect
immediately prior to the coming into effect of the later direction.

\subsection[34. Cancellation of a departure direction on recognition of an error]{Cancellation of a departure direction on recognition of an error}

34.—(1) Where
the Secretary of State is satisfied that a departure direction was given in
error, he shall cancel that direction.

(2) The cancellation of a departure direction under paragraph (1) shall take
effect from the date on which that direction took effect.

\subsection[35. Termination and suspension of departure directions]{Termination and suspension of departure directions}

35.—(1) Subject to paragraphs
(2), (3) and (4), where a departure direction has effect in relation to the
amount of child support maintenance fixed by a maintenance assessment which is
cancelled or ceases to have effect, that departure direction shall cease to have
effect and shall not subsequently take effect.

(2) Where a child support officer ceases to have jurisdiction to make a
maintenance assessment and subsequently acquires jurisdiction to make a
maintenance assessment in respect of the same absent parent, person with care
and any child with respect to whom the earlier assessment was made, a departure
direction for a case falling within paragraph 3 or 4 of Schedule 4B to the Act
shall again take effect from the effective date of the fresh maintenance
assessment.

(3) Where a departure direction had effect in relation to the amount of child
support maintenance fixed by a maintenance assessment which is, under regulation
8(2) of the Maintenance Arrangements and Jurisdiction Regulations (maintenance
assessments and maintenance orders made in error), treated as not having been
cancelled or not having ceased to have effect, that departure direction shall
again take effect from the date that maintenance assessment was cancelled or
ceased to have effect, except where there has, since that maintenance assessment
was cancelled or ceased to have effect, been a material change of circumstances
relevant to that departure direction.

(4) Where—
\begin{enumerate}\item[]
($a$) a departure direction is in force in respect of an interim maintenance
assessment or a maintenance assessment made in accordance with the provisions of
Part I of Schedule 1 to the Act;

($b$) that interim maintenance assessment is replaced by another (“the later
interim maintenance assessment”) or, as the case may be, that maintenance
assessment is replaced by an interim maintenance assessment; and

($c$) by virtue of regulation 10 a departure direction would not be given if that
interim maintenance assessment or that later interim maintenance assessment had
been in force at the time that departure direction was given, 
\end{enumerate}
that departure
direction shall be suspended until that interim maintenance assessment or that
later interim maintenance assessment has been cancelled or has ceased to have
effect and shall again take effect from the effective date of an interim
maintenance assessment to which regulation 10 does not apply, or of a
maintenance assessment made in accordance with the provisions of Part I of
Schedule 1 to the Act, which follows the interim maintenance assessment referred
to in sub-paragraph ($c$).

(5) For the purposes of paragraph (4), a departure direction which is in force
shall include a departure direction which is suspended.

\section[Part VIII --- Maintenance assessment following a departure direction]{Part VIII\\*Maintenance assessment following a departure direction}

\renewcommand\parthead{--- Part VIII}

\subsection[36. Effect of a
departure direction—general]{Effect of a
departure direction—general}

36.—(1) Except where a case falls within regulation
22, 41, 42 or 43, a departure direction shall specify, as the basis on which the
amount of child support maintenance is to be fixed by any fresh assessment made
in consequence of the direction, that the amount of net income or exempt income
of the parent with care or absent parent or the amount of protected income of
the absent parent be increased or, as the case may be, decreased in accordance
with those provisions of regulations 37, 38 and 40 which are applicable to the
particular case.

(2) Where the provisions of paragraph (1) apply to a departure direction, the
amount of child support maintenance fixed by a fresh maintenance assessment
shall be determined in accordance with the provisions of Part I of Schedule 1 to
the Act, but with the substitution of the amounts changed in consequence of the
direction for the amounts determined in accordance with those provisions.

\subsection[37. Effect of a departure direction in respect of special expenses—exempt
income]{Effect of a departure direction in respect of special expenses—exempt
income}

37.—(1) Subject to paragraph (2), where a departure direction is given in
respect of special expenses, the exempt income of the absent parent or, as the
case may be, the parent with care shall be increased by the amount constituting
the special expenses or the aggregate of the special expenses determined in
accordance with regulations 13 to 20.

(2) Where a departure direction is given with respect to costs incurred in
travelling to work or expenses which include such costs, and a component of
exempt income has been determined in accordance with regulation 9(1)($i$) of the
Maintenance Assessments and Special Cases Regulations or regulation 10 of those
Regulations applying regulation 9(1)($i$), the increase in exempt income
determined in accordance with paragraph (1) shall be reduced by that component
of exempt income.

(3) A departure direction with respect to special expenses for a case falling
within regulation 16 shall be given only for the repayment period remaining
applicable to that debt at the date on which that direction takes effect except—
\begin{enumerate}\item[]
($a$) where in consequence of the applicant’s unemployment or incapacity for work,
the repayment period of that debt has been extended by agreement with the
creditor, a departure direction may be given to cover the additional weeks
allowed for repayment; or

($b$) where the Secretary of State is satisfied that, as a consequence of the
income of the applicant having been substantially reduced the repayment period
of that debt has been extended by agreement with the creditor, a departure
direction may be given for such repayment period as the Secretary of State
considers is reasonable.
\end{enumerate}

(4) Where paragraph (4) of regulation 16 applies, a departure direction may be
given in respect only of repayment of that part of the debt incurred which is
referable to the debt, repayment of which would have fallen within paragraph (1)
of that regulation, based upon the amount, rate of repayment and repayment
period agreed in respect of that part at the time it was taken out.

\subsection[38. Effect of a departure direction in respect of special expenses—protected
income]{Effect of a departure direction in respect of special expenses—protected
income}

38.—(1) Subject to paragraphs (2) and (3), where a departure direction is
given with respect to special expenses in response to an absent parent’s
application, his protected income shall be determined in accordance with
paragraph (1) of regulation 11 of the Maintenance Assessments and Special Cases
Regulations\footnote{\frenchspacing Sub-paragraphs ($a$) to ($k$) of paragraph (1) have been amended by regulation 4(4) of S.I. 1994/227, by regulation 46(2)($a$), ($b$) and ($c$) of S.I. 1995/1045, and by regulation 43(1), (2) and (3) of S.I. 1995/3261. Sub-paragraph ($kk$) was added to paragraph (1) of regulation 11 by regulation 46(2)($d$) of S.I. 1995/1045.} with the modification that the increase of exempt income as
determined in accordance with regulation 37 shall be added to the aggregate of
the amounts mentioned in sub-paragraphs ($a$) to ($kk$) of paragraph (1) of
regulation 11 of the Maintenance Assessments and Special Cases Regulations.

(2) Protected income shall not be increased in accordance with paragraph (1) on
account of special expenses constituted by costs falling within regulation 18
(costs incurred in supporting certain children).

(3) Where a departure direction is given with respect to costs which include
costs incurred in travelling to work, the absent parent’s protected income shall
be determined in accordance with paragraph (1), but without inclusion of the
amount determined in accordance with sub-paragraph ($kk$) of regulation 11(1) of
the Maintenance Assessments and Special Cases Regulations within the aggregate
of the amounts mentioned in that regulation.

\subsection[39. Effect of a departure direction in respect of a transfer of property]{Effect of a departure direction in respect of a transfer of property}

39.—(1)
Where a departure direction is given in respect of a transfer of property for a
case falling within paragraph 3 of Schedule 4B to the Act—
\begin{enumerate}\item[]
($a$) where the exempt income of an absent parent includes a component of exempt
income determined in accordance with regulation 9(1)($bb$) of the Maintenance
Assessments and Special Cases Regulations\footnote{\frenchspacing Sub-paragraph ($bb$) was added to paragraph (1) of regulation 9 by regulation 44(2)($a$) of 1995/1045.}, the exempt income of the absent
parent shall be reduced by that component of exempt income;

($b$) subject to sub-paragraph ($c$) and paragraphs (2) and (3), the fresh
maintenance assessment made in consequence of the direction shall be the
maintenance assessment calculated in accordance with the provisions of
paragraphs 1 to 5 and 7 to 10 of Part I of Schedule 1 to the Act, as modified by
sub-paragraph ($a$) where that sub-paragraph is applicable to the case in
question, reduced by the equivalent weekly value of the property transferred as
determined in accordance with regulation 22;

($c$) where the equivalent weekly value is nil, the fresh maintenance assessment
made in consequence of the direction shall be the maintenance assessment
calculated in accordance with the provisions of Part I of Schedule 1 to the Act,
as modified by sub-paragraph ($a$), where that sub-paragraph is applicable to the
case in question.
\end{enumerate}

(2) The amount of child support maintenance fixed by an assessment made in
consequence of a direction falling within paragraph (1) shall not be less than
the amount prescribed by regulation 13 of the Maintenance Assessments and
Special Cases Regulations.

(3) Where there has been a transfer by the applicant of property to trustees as
set out in regulation 21(2) and the equivalent weekly value is greater than nil,
any monies paid to the parent with care out of that trust fund for maintenance
of a child with respect to whom the current assessment was made shall be
disregarded in calculating the assessable income of that parent with care in
accordance with the provisions of Part I of Schedule 1 to the Act.

(4) A departure direction falling within paragraph (1) shall cease to have
effect at the end of the number of years of liability, as defined in paragraph 1
of the Schedule, for the case in question.

(5) Where a departure direction has ceased to have effect under the provisions
of paragraph (4), the exempt income of an absent parent shall be determined as
if regulation 9(1)($bb$) of the Maintenance Assessments and Special Cases
Regulations were omitted.

(6) Where a departure direction is given in respect of a transfer of property
for a case falling within paragraph 4 of Schedule 4B to the Act, the exempt
income of the absent parent shall be reduced by the component of exempt income
determined in accordance with regulation 9(1)($bb$) of the Maintenance Assessments
and Special Cases Regulations.

(7) This regulation is subject to regulation 42.

\subsection[40. Effect of a departure direction in respect of additional cases]{Effect of a departure direction in respect of additional cases}

40.—(1) This
regulation applies where a departure direction is given for an additional case
falling within paragraph 5 of Schedule 4B to the Act.

(2) In a case falling within paragraph (1)($a$) of regulation 23 (assets capable
of producing income or higher income), subject to paragraph (4), the net income
of the non-applicant shall be increased by an amount calculated by applying
interest at the statutory rate prescribed for a judgment debt\footnote{\frenchspacing See Order 42, rule 1 of the Rules of the Supreme Court, S.I. 1965/1776.} or, in
Scotland, at the statutory rate in respect of interest included in or payable
under a decree in the Court of Session\footnote{\frenchspacing See Act of Sederunt (Rules of the Court of Session 1994) 1994.} at the date on which the departure
direction is given to—
\begin{enumerate}\item[]
($a$)any monies falling within that paragraph;

($b$)the net value of any asset, other than monies, falling within that paragraph,
after deduction of the amount owing on any mortgage or charge on that asset,
\end{enumerate}
less any income received in respect of that asset which has been taken into
account in the calculation of the current assessment.

(3) In a case falling within paragraph (1)($b$) or ($c$) of regulation 23, subject
to paragraph (4), the net income of the non-applicant shall be increased by an
amount calculated by applying interest at the statutory rate prescribed for a
judgment debt, or, in Scotland, at the statutory rate in respect of interest
included in or payable under a decree in the Court of Session at the date of the
application to the value of the asset subject to the trust less any income
received from the trust which has been taken into account in the calculation of
the current assessment.

(4) In a case to which regulation 24 (diversion of income) applies, the net
income of the non-applicant who is a parent of a child in respect of whom the
current assessment is made shall be increased by the amount by which the
Secretary of State is satisfied that that parent has reduced his income.

(5) In a case to which regulation 25 (life-style inconsistent with declared
income) applies, the net income of the non-applicant who is a parent of a child
in respect of whom the current assessment is made shall be increased by the
difference between the two levels of income referred to in paragraph (1) of that
regulation.

(6) In a case to which regulation 26 applies (unreasonably high housing costs)
the amount of housing costs included in exempt income and the amount referred to
in regulation 11(1)($b$) of the Maintenance Assessments and Special Cases
Regulations shall not exceed the amounts set out in regulation 18(1)($a$) or ($b$),
as the case may be, of the Maintenance Assessments and Special Cases Regulations
(excessive housing costs) and the provisions of regulation 18(2) of those
Regulations shall not apply.

(7) In a case to which regulation 27 applies (partner’s contribution to housing
costs) that part of the exempt income constituted by the eligible housing costs
determined in accordance with regulation 14 of the Maintenance Assessments and
Special Cases Regulations (eligible housing costs) shall, subject to paragraphs
(8) and (9), be reduced by the percentage of the housing costs which the
Secretary of State considers appropriate, taking into account the income of that
parent and the income or estimated income of that partner.

(8) Where paragraph (7) applies, the housing costs determined in accordance with
regulation 11(1)($b$) of the Maintenance Assessments and Special Cases Regulations
(protected income) shall remain unchanged.

(9) Where a Category B interim maintenance assessment is in force in respect of
a non-applicant, the whole of the eligible housing costs may be deducted from
the exempt income of that non-applicant.

(10) In a case to which regulation 28 (unreasonably high travel costs) or
regulation 29 (travel costs to be disregarded) applies, for the component of
exempt income determined in accordance with regulation 9(1)($i$) of the
Maintenance Assessments and Special Cases Regulations or in accordance with that
regulation as applied by regulation 10 of those Regulations and, in the case of
an absent parent, for the amount determined in accordance with regulation
11(1)($kk$) of those Regulations, there shall be substituted such amount,
including a nil amount, as the Secretary of State considers to be appropriate in
all the circumstances of the case.

\section[Part IX --- Maintenance assessment following a departure direction: particular cases]{Part IX\\*Maintenance assessment following a departure direction: particular cases}

\renewcommand\parthead{--- Part IX}

\subsection[41. Child support maintenance payable where effect of a departure direction
would be to decrease an absent parent’s assessable income but case still fell
within paragraph 2(3) of Schedule 1 to the Act]{Child support maintenance payable where effect of a departure direction
would be to decrease an absent parent’s assessable income but case still fell
within paragraph 2(3) of Schedule 1 to the Act}

41.—(1) Subject to regulation 42
and paragraph (8), where the effect of a departure direction would, but for the
following provisions of this regulation, be to reduce an absent parent’s
assessable income and his assessable income following that direction would be
such that the case fell within paragraph 2(3) of Schedule 1 to the Act
(additional element of maintenance payable), the amount of child support
maintenance payable shall be determined in accordance with paragraphs (2) to
(5).
(2) There shall be calculated the amount equal to A x P, where A is equal to the
amount that would be the absent parent’s assessable income if the departure
direction referred to in paragraph (1) had been given and P has the value
prescribed in regulation 5 of the Maintenance Assessments and Special Cases
Regulations.
(3) There shall be determined the amount that would be payable under a
maintenance assessment calculated by reference to the circumstances at the time
the application is made, in accordance with the provisions of Part I of Schedule
1 to the Act.
(4) The lower of the amounts calculated in accordance with paragraph (2) and
determined in accordance with paragraph (3) shall constitute the revised amount
for the purposes of regulation 7 (rejection of application on completion of a
preliminary consideration) and regulation 31 (refusal to give a departure
direction under section 28F(4) of the Act), and the Secretary of State may apply
regulation 7 and shall apply regulation 31 in relation to the current amount and
the revised amount as so construed.
(5) Subject to paragraph (7), where the application of the provisions of
paragraph (4) results in a departure direction being given, the amount of child
support maintenance payable following that direction shall be determined by the
child support officer as being the revised amount as defined in paragraph (4).
(6) Where the assessable income of an absent parent changes following a review
under section 16, 17, 18 or 19 of the Act, the provisions of paragraphs (2) to
(5) shall be applied to—
($a$)the amount calculated under paragraph (2) which takes account of the change
in assessable income; and
($b$)the amount that would be payable under the maintenance assessment calculated
in accordance with the provisions of Part I of Schedule 1 to the Act which takes
account of that change in assessable income.
(7) Where the provisions of paragraph 6 of Schedule 1 to the Act (protected
income) as modified by the provisions of regulation 38 apply following a
departure direction, and the amount of child support maintenance payable under
those provisions is lower than that payable under paragraph (5), the amount of
child support maintenance payable shall be that payable under those provisions.
(8) Where a departure direction given in accordance with the provisions of
paragraphs (1) to (7) has effect, those provisions shall apply, subject to the
modifications set out in paragraph (9), where—
($a$)the effect of a later direction would, but for the provisions of paragraphs
(2) to (5), be to change the absent parent’s assessable income and his
assessable income following the direction would be such that the case fell
within paragraph 2(3) of Schedule 1 to the Act (additional element of
maintenance payable); and
($b$)that assessable income following the later direction would be less than the
assessable income would be if it were calculated in accordance with the
provisions of Part I of Schedule 1 to the Act by reference to the circumstances
at the time the application for the later direction is made.
(9) The modifications referred to in paragraph (8) are—
($a$)in paragraph (2), A would be the absent parent’s assessable income following
the later direction but for the provisions of paragraphs (3) to (5);
($b$)the references to regulation 7 in paragraph (4) are omitted.
Application of regulation 41 where there is a transfer of property falling
within paragraph 3 of Schedule 4B to the Act42.—(1) Where the application of
regulation 41 to a case would result in a change in the amount of child support
maintenance payable and a direction is given in respect of a transfer of
property falling within paragraph 3 of Schedule 4B to the Act, regulation 41
shall be applied subject to the modifications set out in paragraphs (2) and (3).
(2) Where the exempt income of an absent parent includes a component of exempt
income determined in accordance with regulation 9(1)($bb$) of the Maintenance
Assessments and Special Cases Regulations, that amount shall be excluded—
($a$)in calculating the amount A defined in paragraph (2) of regulation 41;
($b$)in calculating the maintenance assessment specified in paragraph (3) of
regulation 41.
(3) For the purposes of this regulation, the revised amount for the purposes of
regulations 7 and 31 shall be the amount as defined in paragraph (4) of
regulation 41, subject to paragraph (2) of this regulation, less the amount
determined in accordance with regulation 22 (the value of a transfer of property
and its equivalent weekly value for a case falling within paragraph 3 of
Schedule 4B to the Act).
(4) Where the application of the provisions of paragraph (3) results in a
departure direction being given, the amount of child support maintenance payable
following that direction shall be the revised amount as defined in paragraph
(3).
Maintenance assessment following a departure direction for certain cases falling
within regulation 22 of the Maintenance Assessments and Special Cases
Regulations43.—(1) Where the provisions of regulation 41 or 42 are applicable to
a case falling within regulation 22 of the Maintenance Assessments and Special
Cases Regulations(33) (multiple applications relating to an absent parent),
those provisions shall apply for the purposes of determining the total
maintenance payable in consequence of a departure direction.
(2) In a case falling within paragraph (1), the amount of child support
maintenance payable in respect of each application for child support maintenance
following the direction shall be the lower of—
($a$)the amount as determined in accordance with paragraph (3) of regulation 41,
subject to the modification that regulation 22 of the Maintenance Assessments
and Special Cases Regulations is applied in determining the amount that would be
payable (“Y”);
($b$)the amount calculated by the formula—
where A and P have the same meanings as in regulation 41(2) and Q is the sum of
the amounts calculated in accordance with sub-paragraph ($a$) for each assessment.
(3) Where, in a case falling within regulation 22 of the Maintenance Assessments
and Special Cases Regulations, a departure direction has been given in respect
of an absent parent in a case falling within paragraph 3 of Schedule 4B to the
Act (property or capital transfers), the equivalent weekly value of the transfer
of property as calculated in accordance with regulation 22 of these Regulations
shall be deducted from the amount of the maintenance assessment in respect of
the person with care or child to or in respect of whom the property transfer was
made.
Maintenance assessment following a departure direction where there is a phased
maintenance assessment44.—(1) Where a departure direction is given in a case
falling within a relevant enactment, the assessment made in consequence of that
direction shall be the assessment that fixes the amount of child support
maintenance that would be payable but for the provisions of that enactment (“the
unadjusted departure amount”).
(2) Where a departure direction takes effect on the effective date of a
maintenance assessment to which the provisions of a relevant enactment become
applicable, those provisions shall remain applicable to that case following the
departure direction.
(3) Where a departure direction takes effect on a date later than the date on
which the provisions of a relevant enactment become applicable to a maintenance
assessment, the amount of child support maintenance payable in consequence of
that direction shall be—
($a$)where the unadjusted departure amount is more than the formula amount, the
phased amount plus the difference between the unadjusted departure amount and
the formula amount;
($b$)where the unadjusted departure amount is more than the phased amount but less
than the formula amount, the phased amount;
($c$)where the unadjusted departure amount is less than the phased amount, the
unadjusted departure amount.
(4) Regulation 31 shall have effect for cases falling within paragraphs (1) to
(3) as if “current amount” referred to the amount payable under the maintenance
assessment that would be in force when the departure direction is given but for
the provisions of the relevant enactment and “revised amount” referred to the
unadjusted departure amount.
(5) Where a child support officer determines that, were a fresh maintenance
assessment to be made as a result of a review under section 17, 18 or 19 of the
Act in relation to a case to which the provisions of paragraphs (1) to (3) have
been applied, and the amount payable under it (“the reviewed unadjusted
departure amount”) would be—
($a$)more than the unadjusted departure amount, the amount of child support
maintenance payable shall be the amount determined in accordance with paragraph
(3), plus the difference between the unadjusted departure amount and the
reviewed unadjusted departure amount;
($b$)less than the unadjusted departure amount but more than the phased amount,
the amount of child support maintenance payable shall be the phased amount;
($c$)less than the phased amount, the amount of child support maintenance payable
shall be the reviewed unadjusted departure amount.
(6) In this regulation—
“the 1992 enactment” means Part II of the Schedule to the Child Support Act 1991
(Commencement No.3 and Transitional Provisions) Order 1992(34) (modification of
maintenance assessment in certain cases);
“the 1994 enactment” means Part III of the Child Support (Miscellaneous
Amendments and Transitional Provisions) Regulations 1994(35) (transitional
provisions);
“formula amount” has the same meaning as in the relevant enactment;
“phased amount” means—
($a$)where the 1992 enactment is applicable to the particular case, the modified
amount as defined in paragraph 6 of that enactment;
($b$)where the 1994 enactment is applicable to the particular case, the
transitional amount as defined in regulation 6(1) of that enactment;
“relevant enactment” means—
($a$)the 1992 enactment where that enactment is applicable to the particular case;($b$)the 1994 enactment where that enactment is applicable to the particular case.PART XMISCELLANEOUSRegular payments condition45.—(1) For the purposes of section
28C(2)($b$) of the Act (regular payments condition—reduced payments), reduced
payments shall, subject to paragraph (3), be such payments as would be equal to
the payments of child support maintenance fixed by the fresh maintenance
assessment that would be made if the circumstances of the case were those set
out in paragraph (2).
(2) The circumstances referred to in paragraph (1) are—
($a$)the Secretary of State is satisfied that the case is one which falls within
paragraph 2 of Schedule 4B to the Act (special expenses);
($b$)the Secretary of State is satisfied that the expenses claimed by the absent
parent are both being incurred and, for a case falling within regulation 13
(costs incurred in travelling to work), 14 (contact costs) or 15 (illness or
disability), are neither unreasonably high nor being unreasonably incurred, and
that it is just and equitable to give a departure direction in respect of the
whole of those expenses; and
($c$)a departure direction is given in response to the application.
(3) Where the Secretary of State considers it likely that the expenses incurred
by the absent parent are lower than those claimed by him or are not reasonably
incurred, he may fix such amount as he considers to be reasonable in all the
circumstances of the case.
(4) Where the absent parent, following written notice under section 28C(8) of
the Act, fails within 28 days of that notice to comply with the regular payments
condition that was imposed on him, the application shall lapse.
Special case—departure direction having effect from date earlier than effective
date of current assessment46.—(1) A case shall be treated as a special case for
the purposes of the Act if the conditions specified in paragraph (2) are
satisfied.
(2) The conditions are—
($a$)liability to pay child support maintenance commenced earlier than the
effective date of the maintenance assessment in force (“the current
assessment”);
($b$)an application is made or treated as made in relation to the current
assessment which results in a departure direction being given in respect of that
assessment;
($c$)the applicant was unable to make an application on a date falling within a
period in respect of which an earlier assessment had effect because he had not
been notified of that earlier assessment during that period; and
($d$)if the applicant had been able to make such an application and had done so,
the Secretary of State is satisfied that a departure direction would have been
given in response to that application.
(3) Where a case falls within paragraph (2), references to “the current
assessment” and “the current amount” in these Regulations shall be construed as
including references to an earlier assessment falling within paragraph (2)($c$)
and to the amount of child support maintenance fixed by it, and these
Regulations shall be applied to such an earlier assessment accordingly.
PART XITRANSITIONAL PROVISIONSTransitional provisions—application before 2nd
December 199647.—(1) This paragraph applies in any case where an application for
a departure direction has been made before 2nd December 1996(36).
(2) Where paragraph (1) applies, the Secretary of State shall request the
applicant to inform him in writing before 2nd December 1997—
($a$)whether he wishes the application to be treated as an application under these
Regulations in respect of the maintenance assessment in force on 2nd December
1996; and
($b$)whether there have been any changes in the circumstances which are relevant
for the determination or, as the case may be, redetermination of the application
which have occurred since his application and, if so, what those changes are.
(3) Where the applicant fully complies with the request set out in paragraph
(2), and states that he wishes the application to be treated as described in
paragraph (2)($a$), the Secretary of State shall treat the application as an
application under these Regulations which contains the statement mentioned in
section 28A(2)($a$) of the Act, and paragraphs (4) to (10) and regulation 48 shall
apply.
(4) Where the applicant informs the Secretary of State that there have not been
any changes of the kind mentioned in paragraph (2)($b$), the Secretary of State
shall nevertheless invite representations in writing from the relevant persons
other than the applicant.
(5) Where the applicant informs the Secretary of State that there have been
changes in the circumstances of the kind mentioned in paragraph (2)($b$), the
Secretary of State shall—
($a$)give notice that he has been informed of such changes to the relevant persons
other than the applicant;
($b$)send to them the information as to such changes which the applicant has given
except where the Secretary of State considers that information to be information
of the kind falling within paragraph (2) of regulation 8;
($c$)invite representations in writing from the relevant persons other than the
applicant as to such changes; and
($d$)set out the provisions of paragraph (6) in relation to such representations.
(6) The following provisions shall apply to information provided under paragraph
(2)($b$) or representations made following an invitation made in accordance with
paragraph (4) or (5)($c$)—
($a$)paragraphs (2) to (11) of regulation 8, subject to the modification set out
in paragraph (7);
($b$)in relation to an applicant, regulations 6 and 7.
(7) The modification of regulation 8 mentioned in paragraph (6)($a$) is that for
the references to paragraph (1) or, as the case may be, paragraph (1)($c$) of that
regulation, there were substituted references to paragraph (5) or, as the case
may be, paragraph (5)($c$) of this regulation.
(8) Where the Secretary of State has not determined the application in
accordance with the Departure Direction Anticipatory Application Regulations, a
determination shall be made in accordance with these Regulations.
(9) Where the Secretary of State has determined the application in accordance
with the Departure Direction Anticipatory Application Regulations, he shall
determine whether there have been any changes in—
($a$)the circumstances referred to in paragraph (2)($b$);
($b$)the relevant provisions of these Regulations compared with the corresponding
provisions of the Departure Direction Anticipatory Application Regulations.
(10) Where the Secretary of State determines that there have been no changes of
the kind referred to in paragraph (9)($a$) or ($b$), and the relevant persons other
than the applicant have not made any representations in accordance with
paragraph (4), his determination of the application in accordance with the
Departure Direction Anticipatory Application Regulations shall take effect.
(11) Where the Secretary of State determines that there have been changes of the
kind referred to in paragraph (9)($a$) or ($b$), or where the relevant persons other
than the applicant have made representations, he shall make a determination of
the application, taking those changes and representations into account, in
accordance with these Regulations.
Effective date of departure direction for a case falling within regulation
4748.—(1) Where the determination made by the Secretary of State by application
of the provisions of paragraphs (1) to (10) of regulation 47 is to give a
departure direction, that direction shall take effect on the first day of the
first maintenance period commencing on or after 2nd December 1996.
(2) Where a case falls within paragraph (1) of regulation 47, and the applicant
complies with the request for information mentioned in paragraph (2) of that
regulation but not by the date mentioned in that paragraph, his response shall
be treated as an application for a departure direction.
Transitional provisions—no application before 2nd December 199649.—(1) Where—
($a$)a maintenance assessment was in force on 2nd December 1996;
($b$)no application has been made before that date by any of the persons with
respect to whom that assessment was made; and
($c$)an application is made by one of those persons on the grounds set out in
section 28A(2)($a$) of the Act (the effect of the current assessment) on or after
that date and before 2nd December 1997,
any departure direction given in response to that application shall take effect
on the first day of the first maintenance period commencing on or after 2nd
December 1996.
Transitional provisions—new maintenance assessment made before 2nd December 1996
whose effective date is on or after 2nd December 199650. Where a new maintenance
assessment is made before 2nd December 1996 but the effective date of that
assessment is a date on or after 2nd December 1996—
($a$)the provisions of paragraph (1) of regulation 32 shall apply as if for the
reference to an application being given or sent within 28 days of the date of
notification of the current assessment there were substituted a reference to an
application being given or sent before 30th December 1996;
($b$)the provisions of paragraph (2) of regulation 32 shall apply as if for the
reference to an application being given or sent later than 28 days after the
date of notification of the current assessment there were substituted a
reference to an application being given or sent after 29th December 1996.
PART XIIREVOCATIONRevocation of the Departure Direction Anticipatory Application
Regulations51. The Departure Direction Anticipatory Application Regulations are
hereby revoked.
PART XIIICONSEQUENTIAL AMENDMENTSAmendment of regulation 1 of the Appeals
Regulations52. In paragraph (2) of regulation 1 of the Appeals Regulations
(citation, commencement and interpretation)—
($a$)in the definition of “party to the proceedings”—
(i)in sub-paragraph ($d$), after the word “officer” there shall be inserted the
words “except where the proceedings relate only to an appeal under section 28H
of the Act or to a referral;”;
(ii)after sub-paragraph ($d$) there shall be added the following sub-paragraph—
“(dd)the Secretary of State where the proceedings relate to an appeal under
section 28H of the Act;”;
($b$)in the definition of “proceedings”, for the words “or application” there
shall be substituted the words “, application or referral”;
($c$)after the definition of “proceedings”, there shall be inserted the following
definition—
““referral” means a reference by the Secretary of State to a tribunal under
section 28D(1)($b$) of the Act;”;
($d$)in the definition of “tribunal”, after the words “section 21 of” there shall
be inserted the words “or regulations made under paragraph 9 of Schedule 4A
to,”.
Amendment of regulation 3 of the Appeals Regulations53.—(1) Regulation 3 of the
Appeals Regulations (making an appeal or application and time limits), shall be
amended in accordance with the following provisions of this regulation.
(2) In sub-paragraph ($a$) of paragraph (1), after the words “section 20(1)” there
shall be inserted the words “, 28H(1)”.
(3) In paragraph (5), for the words “as the case may be, paragraph (4)” there
shall be substituted the words “(4) or in section 20(2) or 28H(3) of the Act, as
the case may be”.
Amendment of regulation 5 of the Appeals Regulations54.—(1) Regulation 5 of the
Appeals Regulations (directions) shall be amended in accordance with the
following provisions of this regulation.
(2) In paragraph (1), after the words “a chairman may” there shall be inserted
the words “, subject to paragraph (3),”.
(3) In paragraph (2), after the word “may” there shall be inserted the words “,
subject to paragraph (3),”.
(4) After paragraph (2), there shall be added the following paragraphs—
“(3) In the case of an appeal under section 28H of the Act or of a referral, as
the case may be, a direction requiring the Secretary of State to provide
information shall have effect only if that information is information of which
he is aware or which he has in his possession in connection with his functions
under the Act.
(4) Where a chairman is considering whether to give a direction under paragraph
(1) or the terms of any direction, he may direct that an oral hearing be held by
a tribunal to determine whether a direction shall be given under that paragraph
and the terms of any direction which may be given.
(5) The provisions of these Regulations shall apply to a hearing held under the
provisions of paragraph (4).”.
Amendment of regulation 6 of the Appeals Regulations55. After paragraph (1B) of
regulation 6 of the Appeals Regulations (striking out of proceedings), there
shall be inserted the following paragraph—
“(1C) In the case of an appeal under section 28H of the Act, no direction shall
be given under paragraph (1B) requiring the Secretary of State to provide
information other than information of which he is aware or which he has in his
possession in connection with his functions under the Act.”.
Amendment of regulation 7 of the Appeals Regulations56.—(1) Regulation 7 of the
Appeals Regulations (withdrawal of appeals and applications), shall be amended
in accordance with the following provisions of this regulation.
(2) In heads (i) and (ii) of sub-paragraph ($b$) of paragraph (1), after the words
“child support officer” there shall be added the words “or, in the case of an
appeal under section 28H of the Act, the Secretary of State”.
(3) In paragraph (1A), after the words “child support officer” there shall be
inserted the words “or, in the case of an appeal under section 28H of the Act,
the Secretary of State”.
Amendment of regulation 10 of the Appeals Regulations57. In paragraph (1) of
regulation 10 of the Appeals Regulations (summoning of witnesses), for the words
“appeal or application” wherever they appear there shall be substituted the
words “appeal, application or referral”.
Amendment of regulation 11 of the Appeals Regulations58.—(1) Regulation 11 of
the Appeals Regulations (hearings) shall be amended in accordance with the
following provisions of this regulation.
(2) In paragraphs (1), (2A) and (2B), for the words “appeal or application”
wherever they appear there shall be substituted the words “appeal, application
or referral”.
(3) In paragraph (8), after sub-paragraph ($e$), there shall be inserted the
following sub-paragraph—
“(ee)any person undergoing training to enable him to act in the name of the
Secretary of State in relation to applications for a departure direction under
section 28A of the Act and any person acting on behalf of the Secretary of State
in the training or supervision of persons undergoing that training or in the
monitoring of standards of decisions made by persons on behalf of the Secretary
of State in relation to those applications;”.
Insertion of regulations 11A and 11B into the Appeals Regulations59. After
regulation 11 of the Appeals Regulations there shall be inserted the following
regulations—
“Hearing by chairman sitting alone11A.—(1) The prescribed circumstances for the
purpose of paragraph 9 of Schedule 4A to the Act (child support appeal
tribunals) are—
($a$)in relation to a referral, where an application has been made on the grounds
set out in paragraph 3 or 4 of Schedule 4B to the Act;
($b$)in relation to an appeal under section 28H of the Act, where that appeal is
against the rejection of an application by the Secretary of State under section
28B(2) of the Act or a decision of the Secretary of State on an application made
on the grounds set out in paragraph 3 or 4 of Schedule 4B to the Act; or
($c$)in relation to an appeal under section 28H of the Act or to any referral,
where a chairman has directed that an oral hearing be held by a tribunal under
regulation 5(4).
(2) Where the circumstances set out in sub-paragraph ($a$), ($b$) or ($c$) of
paragraph (1) apply, a chairman may decide that the appeal or referral shall be
dealt with by a tribunal constituted by the chairman of the tribunal sitting
alone.
Consideration of more than one appeal under section 28H of the Act11B. A
tribunal which is considering an appeal under section 28H of the Act in respect
of a departure direction which relates to a maintenance assessment may, if it
considers it appropriate to do so, consider at the same time any appeal under
that section in respect of another departure direction which relates to the same
maintenance assessment.”.
Amendment of regulation 13 of the Appeals Regulations60.—(1) Regulation 13 of
the Appeals Regulations (decisions) shall be amended in accordance with the
following provisions of this regulation.
(2) In paragraph (3A), for the words “and of the terms of any direction under
section 20(4) of the Act” there shall be substituted the words “, of the terms
of any direction under section 20(4) of the Act and of the terms of any decision
made by the tribunal under section 28H(4)($c$) of the Act or on a referral”.
(3) After paragraph (3E), there shall be inserted the following paragraph—
“(3F) Paragraphs (1) and (3D) shall not apply where the tribunal is constituted
in accordance with the provisions of regulation 11A.”.
(4) For paragraph (4), there shall be substituted the following paragraph—
“(4) A child support officer may apply to the tribunal or another tribunal for
directions or further directions and the tribunal may give such directions or
further directions as it thinks fit where the child support officer—
($a$)to whom a case is referred by the Secretary of State under section 20(3) of
the Act (procedure following a successful appeal) is uncertain, having regard to
the terms of the decision and of any directions contained in it, how he should
deal with the case; or
($b$)who has been notified of a decision of a tribunal on an appeal under section
28H of the Act or on a referral is uncertain, having regard to the terms of that
decision or of any departure direction given by that tribunal, how he should
deal with the case.”.
Amendment of regulation 14 of the Appeals Regulations61.—(1) Regulation 14 of
the Appeals Regulations (corrections) shall be amended in accordance with the
following provisions of this regulation.
(2) In paragraph (1), after the words “Subject to” there shall be inserted the
words “paragraph (3) and”.
(3) After paragraph (2), there shall be added the following paragraph—
“(3) Paragraphs (1) and (2) shall not apply to referrals.”.
Amendment of regulation 10 of the Arrears Regulations62. In paragraph (2) of
regulation 10 of the Arrears Regulations (adjustment of the amount payable under
a maintenance assessment), after the word “subsequently” there shall be inserted
the words “revised as a result of a departure direction having been given or”.
Amendment of regulation 8 of the Information, Evidence and Disclosure
Regulations63. Regulation 8 of the Information, Evidence and Disclosure
Regulations (disclosure of information to a court or tribunal) shall be numbered
paragraph (1) of that regulation and after paragraph (1) there shall be added
the following paragraph—
“(2) For the purposes of this regulation “proceedings” includes the
determination of an application referred to a child support appeal tribunal
under section 28D(1)($b$) of the Act.”.
Amendment of regulation 9A of the Information, Evidence and Disclosure
Regulations64. For sub-paragraph ($c$) of paragraph (2) of regulation 9A of the
Information, Evidence and Disclosure Regulations (disclosure of information to
other persons), there shall be substituted the following sub-paragraph—
“($c$)the personal representative of a relevant person where—
(i)a review or appeal was pending at the date of death of that person and the
personal representative is dealing with that review or appeal on behalf of that
person; or
(ii)an application for a departure direction had been made but not determined at
the date of death of that person and the personal representative is dealing with
that application on behalf of that person.”.
Amendment of regulation 10 of the Information, Evidence and Disclosure
Regulations65. In paragraph (1) of regulation 10 of the Information, Evidence
and Disclosure Regulations (disclosure of information by the Secretary of
State), for the words “or in connection with an assessment which is or has been
in force” there shall be substituted the words “,an assessment which is or has
been in force or in connection with a departure direction.”.
Amendment of regulation 10A of the Information, Evidence and Disclosure
Regulations66. In paragraph (2) of regulation 10A of the Information, Evidence
and Disclosure Regulations (disclosure of information by a child support
officer), the words “or in connection with” shall be omitted and at the end,
there shall be added the words “or in connection with a departure direction”.
Amendment of regulation 10 of the Maintenance Assessment Procedure
Regulations67.—(1) Regulation 10 of the Maintenance Assessment Procedure
Regulations (notification of a new or a fresh maintenance assessment), shall be
amended in accordance with the following provisions of this regulation.
(2) For sub-paragraphs ($a$) and ($b$) of paragraph (1), there shall be substituted
the following sub-paragraphs—
“($a$)makes a new or fresh maintenance assessment following an application under
section 4, 6 or 7 of the Act, a review under section 16, 17, 18 or 19 of the
Act, or the giving or cancellation of a departure direction;
($b$)makes a new interim maintenance assessment under section 12 of the Act,
substitutes an interim maintenance assessment for one which is in force in
accordance with regulation 8 or 9, or gives or cancels a departure direction;
or”.
(3) In paragraph (2), after sub-paragraph ($h$), there shall be added the
following sub-paragraph—
“(i)where the notification under paragraph (1)($a$) or ($b$) follows the giving, or
cancellation of a departure direction, the amounts calculated in accordance with
Part I of Schedule 1 to the Act, or in accordance with regulation 8A, which have
been changed as a result of the giving or cancellation of that departure
direction.”.
(4) After paragraph (2A), there shall be added the following paragraph—
“(2AA) where a fresh Category D interim maintenance assessment is made following
the giving or cancellation of a departure direction, a notification under
paragraph (1) shall set out in relation to that interim maintenance assessment
the amounts calculated in accordance with regulation 8A which have changed as a
result of the giving or cancellation of that departure direction.”.
(5) For sub-paragraphs ($a$) and ($b$) of paragraph (2B) there shall be substituted
the following sub-paragraphs—
“($a$)the matters listed in sub-paragraphs ($a$), ($b$) and ($d$) to ($f$) of paragraph
(2);
($b$)where known, the absent parent’s assessable income; and
($c$)where the Category B interim maintenance assessment is made following the
giving or cancellation of a departure direction, the amounts calculated in
accordance with regulation 8A which have changed as a result of the giving or
cancellation of that departure direction.”.
(6) In paragraph (4) for sub-paragraph ($d$) there shall be substituted the
following sub-paragraphs—
“($d$)where a fresh maintenance assessment is made following a review under
section 19 of the Act, sections 16, 17 and 18 of the Act;
($e$)where a fresh maintenance assessment is made following the giving of a
departure direction, sections 16, 17 and 18 of the Act.”.
(7) After paragraph (5) there shall be added the following paragraph—
“(6) Where a fresh Category D interim maintenance assessment is made following
the giving or cancellation of a departure direction, a notification under
paragraph (1) shall include information as to sections 16 and 19(1) of the
Act.”.
Amendment of the Maintenance Assessments and Special Cases Regulations68.—(1)
The Maintenance Assessments and Special Cases Regulations shall be amended in
accordance with the following provisions of this regulation.
(2) In paragraph (1) of regulation 1, after the definition of “day to day care”
there shall be inserted the following definition—
““Departure Direction and Consequential Amendments Regulations” means the Child
Support Departure Direction and Consequential Amendments Regulations 1996(37);”.
(3) In paragraph (4) of regulation 1, there shall be inserted at the beginning
the words “These Regulations are subject to the provisions of Parts VIII and IX
of the Departure Direction and Consequential Amendments Regulations and”.
(4) In paragraph (2)($c$) of regulation 9, after head (iv) there shall be added
the following head—
“(v)where a departure direction has been given on the grounds that a case falls
within regulation 27 of the Departure Direction and Consequential Amendments
Regulations (partner’s contribution to housing costs), the amount of the housing
costs which corresponds to the percentage of the housing costs mentioned in
regulation 40(7) of those Regulations.”.
(5) In regulation 22—
($a$)in paragraph (2), after the words “and in these Regulations” there shall be
inserted the words “, and subject to paragraph (2ZA),”; and
($b$)after paragraph (2), there shall be inserted the following paragraph—
“(2ZA) Where a case falls within regulation 39(1)($a$) of the Departure Direction
and Consequential Amendment Regulations, for the purposes of assessing the
amount of child support maintenance payable in respect of an application for
child support maintenance before a departure direction in respect of the
maintenance assessment in question is given, for references to the assessable
income of an absent parent in the Act and in these Regulations there shall be
substituted references to the amount calculated by the formula—
where A, T, B and D have the same meanings as in paragraph (2).”.

\bigskip

Signed by authority of the Secretary of State for Social Security.

{\raggedleft
\emph{A.\ J.\ B.\ Mitchell}\\*Parliamentary Under-Secretary of
State,\\*Department of Social Security

}

20th November 1996

\clearpage

\part[Schedule]{S C H E D U L E}

\renewcommand\parthead{--- Schedule}

SCHEDULEEQUIVALENT WEEKLY VALUE OF A TRANSFER OF A PROPERTY1.—(1) Subject to
paragraphs 3 and 4, the equivalent weekly value of a transfer of property shall
be calculated by multiplying the value of a transfer of property determined in
accordance with regulation 22(1) and (2) by the relevant factor specified in the
Table set out in paragraph 2 (“the Table”).
(2) For the purposes of sub-paragraph (1), the relevant factor is the number in
the Table at the intersection of the column for the statutory rate and of the
row for the number of years of liability.
(3) In sub-paragraph (2)—
($a$)“the statutory rate” means interest at the statutory rate prescribed for a
judgment debt(38) or, in Scotland, the statutory rate in respect of interest
included in or payable under a decree in the Court of Session(39), which in
either case applies at the date of the court order or written agreement relating
to the transfer of the property;
($b$)“the number of years of liability” means the number of years, beginning on
the date of the court order or written agreement relating to the transfer of
property and ending on—
(i)the date specified in that order or agreement as the date on which
maintenance for the youngest child in respect of whom that order or agreement
was made shall cease; or
(ii)if no such date is specified, the date on which the youngest child specified
in the order or agreement reaches the age of 18,
and where that period includes a fraction of a year, that fraction shall be
treated as a full year if it is either one half or exceeds one half of a year,
and shall otherwise be disregarded.
2.  The Table referred to in paragraph 1(1) is set out below—
THE TABLENumber of years of liabilityStatutory
rate8.0%10.0%12.0%12.5%14.0%15.0%1.02077.02115.02154.02163.02192.022122.01078.01108.01138.01145.01168.011833.00746.00773.00801.00808.00828.008424.00581.00607.00633.00640.00660.006745.00482.00507.00533.00540.00560.005746.00416.00442.00468.00474.00495.005087.00369.00395.00421.00428.00448.004628.00335.00360.00387.00394.00415.004299.00308.00334.00361.00368.00389.0040310.00287.00313.00340.00347.00369.0038311.00269.00296.00324.00331.00353.0036712.00255.00282.00310.00318.00340.0035513.00243.00271.00299.00307.00329.0034414.00233.00261.00290.00298.00320.0033615.00225.00253.00282.00290.00313.0032916.00217.00246.00276.00283.00307.0032317.00211.00240.00270.00278.00302.0031818.00205.00234.00265.00273.00297.003143.
The equivalent weekly value of the property transferred shall be nil if the
value of the transfer of the property is less than £5,000.
4. The Secretary of State may determine a lower equivalent weekly value than
that determined in accordance with paragraphs 1 and 2 where the amount of child
support maintenance that would be payable in consequence of a departure
direction specifying that value is lower than the amount of maintenance that was
payable under the court order or written agreement referred to in regulation 21.
5. In this Schedule, “maintenance” has the same meaning as in paragraph 3(2) of
Schedule 4B to the Act.


\part{Explanatory Note}

\renewcommand\parthead{--- Explanatory Note}

\subsection*{(This note is not part of the Regulations)}

These Regulations provide for an application for a departure direction to be
made, the effect of which, if given, would be to vary a child support
maintenance assessment determined in accordance with the formula provisions of
Part I of Schedule 1 to the Child Support Act 1991, and the regulations made
under it.

Regulations 1 to 3 contain interpretation provisions, provisions relating to
documents and rounding provisions.

Regulations 4 to 12 contain provisions relating to the manner in which an
application is to be made, to the Secretary of State’s preliminary consideration
of an application, where income support or income-based jobseeker’s allowance is
payable, interim maintenance assessments, and reviews under section 17 of the
Child Support Act 1991.

Regulations 13 to 29 and the Schedule make provision in relation to cases in
which a departure direction may be given: regulations 13 to 20 relate to special
expenses, regulations 21 and 22 and the Schedule to property or capital
transfers and regulations 23 to 29 to additional cases where a departure
direction may be given.

The Schedule contains a table for calculating the equivalent weekly value of a
transfer of property. The factors in the table are derived from the standard
formula used in annuity calculations.

Regulation 30 prescribes factors to be taken into account and not to be taken
into account in determining whether it would be just and equitable to give a
departure direction.

Regulations 31 to 35 contain provisions as to the effective date and the
duration of a departure direction.

Regulations 36 to 44 contain provisions as to the maintenance assessment which
is to be made in consequence of a departure direction.

Regulations 45 and 46 contain provisions as to the imposition of a regular
payments condition, and a departure direction having effect from a date earlier
than the effective date of the current assessment.

Regulations 47 to 50 contain transitional provisions and regulation 51 revokes
the Child Support Departure Direction (Anticipatory Application) Regulations
1996.

Regulations 52 to 68 provide for amendments to be made to five sets of Child
Support Regulations which are consequential on the introduction of the
departures system.

  These Regulations impose no costs on business.

\end{document}