\documentclass[12pt,a4paper]{article}

\newcommand\regstitle{The Social Security (Deferral of Retirement Pensions, Shared Additional Pension and~Graduated Retirement Benefit) (Miscellaneous Provisions) Regulations 2005}

\newcommand\regsnumber{2005/2677}

%\opt{newrules}{
\title{\regstitle}
%}

%\opt{2012rules}{
%\title{Child Maintenance and~Other Payments Act 2008\\(2012 scheme version)}
%}

\author{S.I.\ 2005 No.\ 2677}

\date{Made
26th September 2005\\
Laid before Parliament
3rd October 2005\\
Coming into~force
6th April 2006
}

%\opt{oldrules}{\newcommand\versionyear{1993}}
%\opt{newrules}{\newcommand\versionyear{2003}}
%\opt{2012rules}{\newcommand\versionyear{2012}}

\usepackage{csa-regs}

\setlength\headheight{42.11603pt}

%\hbadness=10000

\begin{document}

\maketitle

\enlargethispage{1.65405pt}

\noindent
The Secretary of State for~Work and~Pensions, in exercise of the powers conferred upon him by sections 62(1)($a$)  and~($c$), 122(1), 136(5)($a$)  and~($b$),~137(1) and~175(1) and~(3) to (5) of, and~paragraphs~A1(1) and~(3) and~3C(2) and~(4) of Schedule~5 and~paragraph~1(1) and~(3) of Schedule~5A to, the Social Security Contributions and~Benefits Act 1992\footnote{1992 c.~4. Section 62(1) is amended by paragraph~7 of Schedule~4 to the Pensions Act 1995 (c.~26), paragraph~17 of Schedule~11 to the Pensions Act 2004 (c.~35) and by S.I.~2005/2053. Section 175(1) is amended by paragraph~29 of Schedule~3 to the Social Security Contributions (Transfer of Functions, etc.)\ Act 1999 (c.~2) (“the Transfer of Functions Act”). Section 175(1), (3) and (4) is applied to powers conferred by the State Pension Credit Act 2002 (c.~16) by section~19(1) of that Act. Paragraphs~A1 and 3C of Schedule~5 are inserted and Schedule~5A added respectively by paragraphs~4, 9 and 15 of Schedule~11 to the Pensions Act 2004. Paragraph~3C of Schedule~5 is amended to apply to civil partners by S.I.~2005/2053. Sections 122(1) and 137(1) are cited for~the definitions of “prescribe” and “prescribed” respectively.}, sections 5(1)($i$) and~189(1),~(4) and~(6) of the Social Security Administration Act 1992\footnote{1992 c.~5.}, sections 9(1), 10(3) and~(6), 11(1), 18(1)($a$), 79(1) and~(4) and~84 of the Social Security Act 1998\footnote{1998 c.~14. Section 18(1) is amended by paragraph~29 of Schedule~7 to the Transfer of Functions Act. Section 84 is cited for~the definition of “prescribe”.}, paragraphs~3(1), 4(4) and~(6), 20(1) and~(3) and~23(1) of Schedule~7 to the Child Support, Pensions and~Social Security Act 2000\footnote{2000 c.~19. Paragraph~23(1) of Schedule~7 is cited for~the definition of “prescribed”.}, sections 15(6)($a$)  and~($b$)  and~17(1) of the State Pension Credit Act 2002\footnote{Section 17(1) is cited for~the definitions of “prescribed” and “regulations”.} and~paragraph~27 of Schedule~11 to the Pensions Act 2004\footnote{2004 c.~35.}, and~of all other powers enabling him in that behalf, after agreement by the Social Security Advisory Committee that proposals to make regulations 2 to 5 and~7 to 13 should not be referred to it\footnote{\emph{See} sections 170 and 173(1)($b$) of the Social Security Administration Act 1992. Paragraph~104 of Schedule~7 to the Social Security Act 1998, section~73 of the Child Support, Pensions and Social Security Act 2000 and paragraph~20 of Schedule~2 to the State Pension Credit Act 2002 respectively added the relevant provisions of those Acts to the list of “relevant enactments” in respect of which regulations must normally be referred to the Committee. Section 173(7) defines “regulations”.} and~in so far as these Regulations concern housing benefit and~council tax benefit, after consultation with organisations appearing to the Secretary of State to be representative of the authorities concerned\footnote{\emph{See} section~176(1)($a$) of the Social Security Administration Act 1992.}, hereby makes the following Regulations: 

{\sloppy

\tableofcontents

}

\bigskip

\setcounter{secnumdepth}{-2}

\section[Part I --- General]{Part I\\*General}

\renewcommand\parthead{--- Part I}

\subsection[1. Citation, commencement and~interpretation]{Citation, commencement and~interpretation}

1.---(1)  These Regulations may be cited as the Social Security (Deferral of Retirement Pensions, Shared Additional Pension and~Graduated Retirement Benefit) (Miscellaneous Provisions) Regulations 2005 and~shall come into force on 6th April 2006.

(2) In these Regulations—
\begin{enumerate}\item[]
“the Claims and~Payments Regulations” means the Social Security (Claims and~Payments) Regulations 1987\footnote{S.I.~1987/1968.};

“the Housing Benefit Regulations” means the Housing Benefit (General) Regulations 1987\footnote{S.I.~1987/1971.}.
\end{enumerate}

\section[Part II --- Deferral of retirement pensions and shared additional pension]{Part II\\*Deferral of retirement pensions and shared additional pension}

\renewcommand\parthead{--- Part II}

\subsection[2. Interpretation]{Interpretation}

2.---(1)  In this Part—
\begin{enumerate}\item[]
“elector” means the person who may make an election under paragraph~A1(1) or~3C(2) of Schedule~5 or~paragraph~1(1) of Schedule~5A;

“retirement pension” means a Category A or~a Category B retirement pension.
\end{enumerate}

(2) In this Part, references to Schedules 5 and~5A are to those Schedules to the Social Security Contributions and~Benefits Act 1992.

\subsection[3. Timing of election]{Timing of election}

3.---(1)  The period for~making an election under—
\begin{enumerate}\item[]
($a$) paragraph~A1(1) of Schedule~5 (choice between increase of pension and~lump sum where pensioner’s entitlement is deferred); and

($b$) paragraph~1(1) of Schedule~5A (choice between pension increase and~lump sum where entitlement to shared additional pension is deferred),
\end{enumerate}
is, subject to paragraph~(4), three months starting on the date shown on the notice issued by the Secretary of State following the claim for~retirement pension or~shared additional pension, confirming that the elector~is required to make that election.

(2) The period for~making an election under paragraph~3C(2) of Schedule~5 (choice between increase of pension and~lump sum where pensioner’s deceased spouse or~civil partner has deferred entitlement) is, subject to paragraph~(4), three months starting on the date shown on the notice issued by the Secretary of State following $W$’s claim for~retirement pension or, if later, the date of $S$’s death, confirming that the elector~is required to make that election\footnote{“$W$” and “$S$” have the same meaning for~the purposes of this sub-paragraph~as for~the purposes of paragraph~3C of Schedule~5.}.

(3) Where more than one notice has been issued by the Secretary of State in accordance with paragraph~(1) or~(2), the periods prescribed in those paragraphs~shall only commence from the date shown on the latest such notice.

(4) The periods specified in paragraphs~(1) and~(2) may be extended by the Secretary of State if he considers it reasonable to do so in any particular case.

(5) Nothing in this regulation~shall prevent the making of an election on or~after claiming retirement pension or, as the case may be, shared additional pension but before the issue of the notice referred to in paragraph~(1) or~(2).

\subsection[4. Manner of making election]{Manner of making election}

4.  An election under paragraph~A1(1) or~3C(2) of Schedule~5 or~under paragraph~1(1) of Schedule~5A may be made—
\begin{enumerate}\item[]
($a$) in writing to an office specified by the Secretary of State for~accepting such elections; or

($b$) except where the Secretary of State directs in any particular case that the election must be made in accordance with paragraph~($a$), by telephone call to the telephone number specified by the Secretary of State.
\end{enumerate}

\subsection[5. Change of election]{Change of election}

5.---(1)  Subject to paragraphs~(2) and~(6), this regulation~applies in the case of an election which—
\begin{enumerate}\item[]
($a$) has been made under paragraph~A1(1) or~3C(2) of Schedule~5 or~under paragraph~1(1) of Schedule~5A; or

($b$) has been treated as made under paragraph~A1(2) or~3C(3) of Schedule~5 or~under paragraph~1(2) of Schedule~5A.
\end{enumerate}

(2) This regulation~does not apply in the case of an election which is—
\begin{enumerate}\item[]
($a$) made, or~treated as made, by an elector~who has subsequently died; or

($b$) treated as having been made by virtue of regulation~30(5D) or~(5F) of the Claims and~Payments Regulations\footnote{Regulation~30(5D) to (5F) is inserted by S.I.~2005/455 and amended by S.I.~2005/1551.}.
\end{enumerate}

(3) An election specified in paragraph~(1) may be changed by way of application made no later than the last day of the period specified in paragraph~(4).

(4) The period specified for~the purposes of paragraph~(3) is, subject to paragraph~(5), three months starting on the date shown on the written notification issued by the Secretary of State to the elector, confirming the election which the elector~has made or~is treated as having made.

(5) The period specified in paragraph~(4) may be extended by the Secretary of State if he considers it reasonable to do so in any particular case.

(6) An election specified in paragraph~(1) may not be changed where—
\begin{enumerate}\item[]
($a$) there has been a previous change of election under this regulation~in respect of the same period of deferment;

\begin{sloppypar}
($b$) the application is to change the election to one under paragraph~A1(1)($a$)  or~3C(2)($a$)  of Schedule~5 or~paragraph~1(1)($a$)  of Schedule~5A and~any amount paid to him by way of, or~on account of, a lump sum pursuant to Schedule~5 or~5A, has not been repaid in full to the Secretary of State within the period specified in paragraph~(4) or, as the case may be, (5); or
\end{sloppypar}

\begin{sloppypar}
($c$) the application is to change the election to one under paragraph~A1(1)($b$)  or~3C(2)($b$)  of Schedule~5 or~paragraph~1(1)($b$)  of Schedule~5A and~the amount actually paid by way of an increase of retirement pension or~shared additional pension, or~actually paid on account of such an increase, would exceed the amount to which the elector~would be entitled by way of a lump sum.
\end{sloppypar}
\end{enumerate}

(7) For~the purposes of paragraph~(6)($b$), repayment in full of the amount paid by way of, or~on account of, a lump sum shall only be treated as having occurred if repaid to the Secretary of State in the currency in which that amount was originally paid.

(8) Where the application is to change the election to one under paragraph~A1(1)($b$)  or~3C(2)($b$)  of Schedule~5 or~paragraph~1(1)($b$)  of Schedule~5A and~paragraph~(6)($c$)  does not apply, any amount paid by way of an increase of retirement pension or~shared additional pension, or~on account of such an increase, in respect of the period of deferment for~which the election was originally made, shall be treated as having been paid on account of the lump sum to which the elector~is entitled under paragraph~3A or~7A of Schedule~5 or, as the case may be, paragraph~4 of Schedule~5A.

(9) An application under paragraph~(3) to change an election may be made—
\begin{enumerate}\item[]
($a$) in writing to an office specified by the Secretary of State for~accepting such applications; or

($b$) except where the Secretary of State directs in any particular case that the application must be made in accordance with sub-paragraph~($a$), by telephone call to the telephone number specified by the Secretary of State.
\end{enumerate}

\subsection[6. Amendment of the Social Security (Retirement Pensions etc.)\ (Transitional Provisions) Regulations 2005]{Amendment of the Social Security (Retirement Pensions etc.)\ (Transitional Provisions) Regulations 2005}

6.  Regulation~2(6)($a$)  of the Social Security (Retirement Pensions etc.)\ (Transitional Provisions) Regulations 2005\footnote{S.I.~2005/469.} (modification of Schedule~5) is omitted.

\section[Part III --- Deferral of graduated retirement benefit]{Part III\\*Deferral of graduated retirement benefit}

\renewcommand\parthead{--- Part III}

\subsection[7. Amendment of the Social Security (Graduated Retirement Benefit) Regulations 2005]{Amendment of the Social Security (Graduated Retirement Benefit) Regulations 2005}

7.---(1)  The Social Security (Graduated Retirement Benefit) Regulations 2005\footnote{S.I.~2005/454. Schedule 1 has effect by virtue of section~36(4) of the National Insurance Act 1965 (c.~51) as amended by those Regulations. Section 36 of that Act was repealed by the Social Security Act 1973 (c.~38) with effect from 6th April 1975 but continues in force by virtue of regulations made under Schedule 3 to the Social Security (Consequential Provisions) Act 1975 (c.~18) or~under Schedule 3 to the Social Security (Consequential Provisions) Act 1992 (c.~6).} shall be amended in accordance with the following paragraphs.

(2) In Schedule~1 (increases of graduated retirement benefit and~lump sums)—
\begin{enumerate}\item[]
($a$) in paragraph~2(1), omit “,~on claiming his pension either”;

($b$) in paragraph~12—
\begin{enumerate}\item[]
(i) for~sub-paragraph~(2) substitute—
\begin{quotation}
“(2) The election referred to in sub-paragraph~(1) shall be made—
\begin{enumerate}\item[]
($a$) on the date on which he claims graduated retirement benefit; or

($b$) within the period after claiming graduated retirement benefit prescribed in paragraph~20B,
\end{enumerate}
and~in the manner prescribed in paragraph~20C.”;
\end{quotation}

(ii) in sub-paragraph~(4), for~“and~within the time specified in regulations made under paragraph~A1(4) of Schedule~5” substitute “,~manner and~within the period prescribed, in paragraph~20D”;
\end{enumerate}

($c$) in paragraph~17—
\begin{enumerate}\item[]
(i) for~sub-paragraph~(3), substitute—
\begin{quotation}
“(3) The election referred to in sub-paragraph~(2) shall be made within the period prescribed in paragraph~20B and~in the manner prescribed in paragraph~20C.”;
\end{quotation}

(ii) in sub-paragraph~(4), for~“(3)($b$)” substitute “(3)”;

(iii) for~sub-paragraph~(5) substitute—
\begin{quotation}
“(5) A person who has made an election under sub-paragraph~(2) (including one that the person is treated by sub-paragraph~(4) as having made) may change the election in the circumstances, manner and~within the period prescribed in paragraph~20D.”.
\end{quotation}
\end{enumerate}

($d$) after paragraph~20, insert—
\begin{quotation}
\section*{“Part IIA\\*Elections under Part II}

\subsection*{\itshape Scope and~interpretation}

20A.---(1)  This Part applies in respect of elections which a person makes or~is treated as having made under Part II.

(2) In this Part, “elector” means the person who may make an election under paragraph~12(1) or~17(2).

\subsection*{\itshape Timing of election}

20B.---(1)  The period for~making an election under paragraph~12(1) is, subject to sub-paragraph~(4), three months starting on the date shown on the notice issued by the Secretary of State following the claim for~graduated retirement benefit, confirming that the elector~is required to make that election.

\begin{sloppypar}
(2) The period for~making an election under paragraph~17(2) is, subject to sub-paragraph~(4), three months starting on the date shown on the notice issued by the Secretary of State following $W$’s claim for~a Category A or~Category B retirement pension or, if later, the date of $S$’s death, confirming that the elector~is required to make that election\footnote{“$W$” and “$S$” have the same meaning for~the purposes of this sub-paragraph~as for~the purposes of paragraph~17 of Schedule 1 to those Regulations.}.
\end{sloppypar}

(3) Where more than one notice has been issued by the Secretary of State in accordance with sub-paragraph~(1) or~(2), the periods prescribed in those sub-paragraphs~shall only commence from the date shown on the latest such notice.

(4) The periods specified in sub-paragraphs~(1) and~(2) may be extended by the Secretary of State if he considers it reasonable to do so in any particular case.

(5) Nothing in this paragraph~shall prevent the making of an election on or~after claiming graduated retirement benefit or, as the case may be, Category A or~Category B retirement pension, but before the issue of the notice referred to in sub-paragraph~(1) or~(2).

\subsection*{\itshape Manner of making election}

20C.  An election under paragraph~12(1) or~17(2) may be made—
\begin{enumerate}\item[]
($a$) in writing to an office specified by the Secretary of State for~accepting such elections; or

($b$) except where the Secretary of State directs in any particular case that the election must be made in accordance with sub-paragraph~($a$), by telephone call to the telephone number specified by the Secretary of State.
\end{enumerate}

\subsection*{\itshape Change of election}

20D.---(1)  Subject to sub-paragraphs~(2) and~(6), this paragraph~applies in the case of an election which—
\begin{enumerate}\item[]
($a$) has been made under paragraph~12(1) or~17(2); or

($b$) has been treated as made under paragraph~12(3) or~17(4).
\end{enumerate}

(2) This paragraph~does not apply in the case of an election which is—
\begin{enumerate}\item[]
($a$) made, or~treated as made, by an elector~who has subsequently died; or

($b$) treated as having been made by virtue of regulation~30(5D) or~(5F) of the Social Security (Claims and Payments) Regulations 1987.
\end{enumerate}

(3) An election specified in sub-paragraph~(1) may be changed by way of application made no later than the last day of the period specified in sub-paragraph~(4).

(4) The period specified for~the purposes of sub-\hspace{0pt}paragraph~(3) is, subject to sub-paragraph~(5), three months after the date shown on the written notification issued by the Secretary of State to the elector, confirming the election which the elector~has made or~is treated as having made.

(5) The period specified in sub-paragraph~(4) may be extended by the Secretary of State if he considers it reasonable to do so in any particular case.

(6) An election specified in sub-paragraph~(1) may not be changed where—
\begin{enumerate}\item[]
($a$) there has been a previous change of election under this paragraph~in respect of the same period of deferment;

($b$) the application is to change the election to one under paragraph~12(1)($a$)  or~17(2)($a$)  and~any amount paid to him by way of, or~on account of, a lump sum pursuant to paragraph~15 or~19, has not been repaid in full to the Secretary of State within the period specified in sub-paragraph~(4) or, as the case may be, (5); or

($c$) the application is to change the election to one under paragraph~12(1)($b$)  or~17(2)($b$)  and~the amount actually paid by way of an increase of graduated retirement benefit, or~actually paid on account of such an increase, would exceed the amount to which the elector~would be entitled by way of a lump sum.
\end{enumerate}

(7) For~the purposes of sub-paragraph~(6)($b$), repayment in full of the amount paid by way of, or~on account of, a lump sum shall only be treated as having occurred if repaid to the Secretary of State in the currency in which that amount was originally paid.

(8) Where the application is to change the election to one under paragraph~12(1)($b$)  or~17(2)($b$)  and~sub-paragraph~(6)($c$)  does not apply, any amount paid by way of an increase of graduated retirement benefit, or~on account of such an increase, in respect of the period of deferment for~which the election was originally made, shall be treated as having been paid on account of the lump sum to which the elector~is entitled under paragraph~15 or~19.

(9) An application under sub-paragraph~(3) to change an election may be made—
\begin{enumerate}\item[]
($a$) in writing to an office specified by the Secretary of State for~accepting such applications; or

($b$) except where the Secretary of State directs in any particular case that the application must be made in accordance with paragraph~($a$), by telephone call to the telephone number specified by the Secretary of State.”.
\end{enumerate}
\end{quotation}
\end{enumerate}

(3) In Schedule~2 (modification of Schedule~1), omit paragraphs~5 and~10.

\section[Part IV --- Payments]{Part IV\\*Payments}

\renewcommand\parthead{--- Part IV}

\subsection[8. Amendment of the Claims and~Payments Regulations]{Amendment of the Claims and~Payments Regulations}

8.  In the Claims and~Payments Regulations, after regulation~21 insert—
\begin{quotation}
\subsection*{\itshape “Delayed payment of lump sum}

21A.---(1)   This regulation~applies where—
\begin{enumerate}\item[]
($a$) a person (“$P$”) is entitled to a lump sum under, as the case may be—
\begin{enumerate}\item[]
(i) Schedule~5 to the Contributions and~Benefits Act\footnote{Schedule 5 is amended, so far as is relevant, by Schedule 11 to the Pensions Act 2004 (c.~35).} (pension increase or~lump sum where entitlement to retirement pension is deferred);

(ii) Schedule~5A to that Act\footnote{Schedule 5A is inserted by paragraph~15 of Schedule 11 to the Pensions Act 2004.} (pension increase or~lump sum where entitlement to shared additional pension is deferred); or

(iii) Schedule~1 to the Social Security (Graduated Retirement Benefit) Regulations 2005\footnote{S.I.~2005/454.} (further provisions replacing section~36(4) of the National Insurance Act 1965: increases of graduated retirement benefit and~lump sums);
\end{enumerate}
or

($b$) the Secretary of State decides to make a payment on account of such a lump sum.
\end{enumerate}

(2) Subject to paragraph~(3), for~the purposes of section~7 of the Finance (No.~2) Act 2005\footnote{2005 c.~22.} (charge to income tax of lump sum), $P$ may elect to be paid the lump sum in the tax year (“the later year of assessment”) next following the tax year which would otherwise be the applicable year of assessment by virtue of section~8 of that Act (meaning of “applicable year of assessment” in section~7)\footnote{Section 8(5) provides that subsections (6) and (7) apply where social security regulations make related provision.}.

(3) $P$ may not elect in accordance with paragraph~(2) (“a tax election”) unless he elects on the same day as he chooses a lump sum in accordance with, as the case may be—
\begin{enumerate}\item[]
($a$) paragraph~A1 or~3C of Schedule~5 to the Contributions and~Benefits Act\footnote{Paragraphs~A1 and 3C are inserted respectively by paragraphs~4 and 9 of Schedule 11 to the Pensions Act 2004.};

($b$) paragraph~1 of Schedule~5A to that Act;

($c$) paragraph~12 or~17 of Schedule~1 to the Social Security (Graduated Retirement Benefit) Regulations 2005,
\end{enumerate}
or~within a month of that day.

(4) A tax election may be made in writing to an office specified by the Secretary of State for~accepting such elections or, except where in any particular case the Secretary of State directs that the election must be made in writing, it may be made by telephone call to the number specified by the Secretary of State.

(5) If $P$ makes a tax election, payment of the lump sum, or~any payment on account of the lump sum, shall be made in the first month of the later year of assessment or~as soon as reasonably practicable after that month, unless $P$ revokes the tax election before the payment is made.

(6) If $P$ makes no tax election in accordance with paragraphs~(2) and~(3), or~revokes a tax election, payment of the lump sum or~any payment on account of the lump sum shall be made as soon as reasonably practicable after $P$—
\begin{enumerate}\item[]
($a$) elected for~a lump sum, or~was treated as having so elected; or

($b$) revoked a tax election.
\end{enumerate}

(7) If $P$ dies before the beginning of the later year of assessment—
\begin{enumerate}\item[]
($a$) any tax election in respect of $P$’s lump sum shall cease to have effect; and

($b$) no person appointed under regulation~30 to act on $P$’s behalf may make a tax election.
\end{enumerate}

(8) In this regulation~“the later year of assessment” has the meaning given by section~8(5) of the Finance (No.~2) Act 2005.”.
\end{quotation}

\section[Part V --- Decisions]{Part V\\*Decisions}

\renewcommand\parthead{--- Part V}

\subsection[9. Amendment of the Social Security and~Child Support (Decisions and~Appeals) Regulations 1999]{Amendment of the Social Security and~Child Support (Decisions and~Appeals) Regulations 1999}

9.---(1)  The Social Security and~Child Support (Decisions and~Appeals) Regulations 1999\footnote{S.I.~1999/991.} shall be amended in accordance with the following paragraphs.

(2) In regulation~1(3) (interpretation)—
\begin{enumerate}\item[]
($a$) after the definition of “financially qualified panel member” insert—
\begin{quotation}
““the Graduated Retirement Benefit Regulations” means the Social Security (Graduated Retirement Benefit) Regulations 2005\footnote{S.I.~2005/454.};”; and
\end{quotation}

($b$) after the definition of “date of notification” insert—
\begin{quotation}
““the Deferral of Retirement Pensions etc.\ Regulations” means the Social Security (Deferral of Retirement Pensions, Shared Additional Pension and~Graduated Retirement Benefit) (Miscellaneous Provisions) Regulations 2005\footnote{S.I.~2005/2677.};”.
\end{quotation}
\end{enumerate}

(3) In regulation~3 (revision of decisions), after paragraph~(7C)\footnote{Paragraph (7C) is inserted by S.I.~2005/337.} insert—
\begin{quotation}
“(7D) Where—
\begin{enumerate}\item[]
($a$) a person elects for~an increase of—
\begin{enumerate}\item[]
(i) a Category A or~Category B retirement pension in accordance with paragraph~A1 or~3C of Schedule~5 to the Contributions and~Benefits Act\footnote{Paragraphs~A1 and 3C are inserted respectively by paragraphs 4 and 9 of Schedule 11 to the Pensions Act 2004 (c.~35).} (pension increase or~lump sum where entitlement to retirement pension is deferred);

(ii) a shared additional pension in accordance with paragraph~1 of Schedule~5A to that Act\footnote{Schedule 5A is inserted by paragraph~15 of Schedule 11 to the Pensions Act 2004.} (pension increase or~lump sum where entitlement to shared additional pension is deferred); or, as the case may be,

(iii) graduated retirement benefit in accordance with paragraph~12 or~17 of Schedule~1 to the Graduated Retirement Benefit Regulations (further provisions replacing section~36(4) of the National Insurance Act 1965: increases of graduated retirement benefit and~lump sums);\end{enumerate}

($b$) the Secretary of State decides that the person or~his partner is entitled to state pension credit and~takes into account the increase of pension or~benefit in making or~superseding that decision; and

($c$) the person’s election for~an increase is subsequently changed in favour of a lump sum in accordance with regulation~5 of the Deferral of Retirement Pensions etc.\ Regulations or, as the case may be, paragraph~20D of Schedule~1 to the Graduated Retirement Benefit Regulations\footnote{Paragraph 20D is inserted by S.I.~2005/2677.},
\end{enumerate}
the Secretary of State may revise the state pension credit decision.

(7E) Where—
\begin{enumerate}\item[]
($a$) a person is awarded a Category A or~Category B retirement pension, shared additional pension or, as the case may be, graduated retirement benefit;

($b$) an election is made, or~treated as made, in respect of the award in accordance with paragraph~A1 or~3C of Schedule~5 or~paragraph~1 of Schedule~5A to the Contributions and~Benefits Act or, as the case may be, in accordance with paragraph~12 or~17 of Schedule~1 to the Graduated Retirement Benefit Regulations; and

($c$) the election is subsequently changed in accordance with regulation~5 of the Deferral of Retirement Pensions etc.\ Regulations or, as the case may be, paragraph~20D of Schedule~1 to the Graduated Retirement Benefit Regulations,
\end{enumerate}
the Secretary of State may revise the award.”.
\end{quotation}

(4) In regulation~6 (supersession of decisions), after paragraph~(2)($n$)\footnote{Sub-paragraph ($n$)  is inserted by S.I.~2005/337.} insert—
\begin{quotation}
“($o$) is a decision that a person is entitled to state pension credit and—
\begin{enumerate}\item[]
(i) the person or~his partner makes, or~is treated as having made, an election for~a lump sum in accordance with—
\begin{enumerate}\item[]
($aa$) paragraph~A1 or~3C of Schedule~5 to the Contributions and~Benefits Act\footnote{Paragraphs~A1 and 3C are inserted respectively by paragraphs 4 and 9 of Schedule 11 to the Pensions Act 2004 (c.~35).};

($bb$) paragraph~1 of Schedule~5A to that Act\footnote{Schedule 5A is inserted by paragraph 15 of Schedule 11 to the Pensions Act 2004.}; or, as the case may be,

($cc$) paragraph~12 or~17 of Schedule~1 to the Graduated Retirement Benefit Regulations;
\end{enumerate}
or

(ii) such a lump sum is repaid in consequence of an application to change an election for~a lump sum in accordance with regulation~5 of the Deferral of Retirement Pensions etc.\ Regulations or, as the case may be, paragraph~20D of Schedule~1 to the Graduated Retirement Benefit Regulations.”.
\end{enumerate}
\end{quotation}

(5) In regulation~7 (date from which a decision superseded under section~10 takes effect), after paragraph~(7) insert—
\begin{quotation}
“(7A) Where a decision is superseded in accordance with regulation~6(2)($o$), the superseding decision shall take effect from the day on which a lump sum, or~a payment on account of a lump sum, is paid or~repaid if that day is the first day of the benefit week but, if it is not, from the next following such day.”.
\end{quotation}

(6) After regulation~13 insert—
\begin{quotation}
\subsection*{“Retirement pension after period of deferment}

13A.---(1)  This regulation~applies where—
\begin{enumerate}\item[]
($a$) a person claims a Category A or~Category B retirement pension, shared additional pension or, as the case may be, graduated retirement benefit;

($b$) an election is required by, as the case may be—
\begin{enumerate}\item[]
(i) paragraph~A1 or~3C of Schedule~5 to the Contributions and~Benefits Act (pension increase or~lump sum where entitlement to retirement pension is deferred);

(ii) paragraph~1 of Schedule~5A to that Act (pension increase or~lump sum where entitlement to shared additional pension is deferred); or, as the case may be,\looseness=1

(iii) paragraph~12 or~17 of Schedule~1 to the Graduated Retirement Benefit Regulations (further provisions replacing section~36(4) of the National Insurance Act 1965: increases of graduated retirement benefit and~lump sums); and\looseness=-1 
\end{enumerate}

($c$) no election is made when the claim is made.
\end{enumerate}

(2) In the circumstances specified in paragraph~(1) the Secretary of State may decide the claim before any election is made, or~is treated as made, for~an increase or~lump sum.

(3) When an election is made, or~is treated as made, the Secretary of State shall revise the decision which he made in pursuance of paragraph~(2).”.
\end{quotation}

\subsection[10. Amendment of the Housing Benefit and~Council Tax Benefit (Decisions and~Appeals) Regulations 2001]{Amendment of the Housing Benefit and~Council Tax Benefit (Decisions and~Appeals) Regulations 2001}

10.---(1)  The Housing Benefit and~Council Tax Benefit (Decisions and Appeals) Regulations 2001\footnote{S.I.~2001/1002.} shall be amended in accordance with the following paragraphs.\looseness=-1

(2) In regulation~4 (revision of decisions), after paragraph~(7C)\footnote{Paragraph (7C) is inserted by S.I.~2003/2275.} insert—
\begin{quotation}
“(7D) Where—
\begin{enumerate}\item[]
($a$) a person elects for~an increase of—
\begin{enumerate}\item[]
(i) a Category A or~Category B retirement pension in accordance with paragraph~A1 or~3C of Schedule~5 to the Contributions and~Benefits Act\footnote{Paragraphs~A1 and 3C are inserted respectively by paragraphs 4 and 9 of Schedule 11 to the Pensions Act 2004 (c.~35).} (pension increase or~lump sum where entitlement to retirement pension is deferred);\looseness=-1

(ii) a shared additional pension in accordance with paragraph~1 of Schedule~5A to that Act\footnote{Schedule 5A is inserted by paragraph 15 of Schedule 11 to the Pensions Act 2004.\looseness=-1} (pension increase or~lump sum where entitlement to shared additional pension is deferred); or, as the case may be,

\pagebreak[3]

(iii) graduated retirement benefit in accordance with paragraph~12 or~17 of Schedule~1 to the Social Security (Graduated Retirement Benefit) Regulations 2005\footnote{S.I.~2005/454.} (further provisions replacing section~36(4) of the National Insurance Act 1965: increases of graduated retirement benefit and~lump sums);
\end{enumerate}

($b$) the relevant authority decides that the person or~his partner is entitled to housing benefit or~council tax benefit and~takes into account the increase of pension or~benefit in making or~superseding that decision; and

($c$) the person’s election for~an increase is changed so that he is entitled to a lump sum,
\end{enumerate}
the relevant authority may revise the housing benefit or~council tax benefit decision.”.
\end{quotation}

(3) In regulation~7 (decisions superseding earlier decisions), after paragraph~(2)($i$)\footnote{Sub-paragraph ($i$) is inserted by S.I.~2003/2275.} insert—
\begin{quotation}
“($j$) where—
\begin{enumerate}\item[]
(i) the claimant or~his partner makes, or~is treated as having made, an election for~a lump sum in accordance with—
\begin{enumerate}\item[]
($aa$) paragraph~A1 or~3C of Schedule~5 to the Contributions and~Benefits Act;

($bb$) paragraph~1 of Schedule~5A to that Act; or, as the case may be,

($cc$)  paragraph~12 or~17 of Schedule~1 to the Social Security (Graduated Retirement Benefit) Regulations 2005;
\end{enumerate}
or

(ii) such a lump sum is repaid in consequence of an application to change an election for~a lump sum in accordance with regulation~5 of the Social Security (Deferral of Retirement Pensions, Shared Additional Pension and~Graduated Retirement Benefit) (Miscellaneous Provisions) Regulations 2005\footnote{S.I.~2005/2677.} or, as the case may be, paragraph~20D of Schedule~1 to the Social Security (Graduated Retirement Benefit) Regulations 2005\footnote{Paragraph 20D is inserted by S.I.~2005/2677.}.”.
\end{enumerate}
\end{quotation}

(4) In regulation~8 (date from which decision superseding an earlier decision takes effect), after paragraph~(14)\footnote{Paragraph (14) is inserted by S.I.~2003/2275.} insert—
\begin{quotation}
“(14A) Where a decision is superseded in accordance with regulation~7(2)($j$), the superseding decision shall take effect from the day on which a lump sum, or~a payment on account of a lump sum, is paid or~repaid if that day is the first day of the benefit week but, if it is not, from the next following such day.”.
\end{quotation}

\section[Part VI --- Amendment of benefit regulations]{Part VI\\*Amendment of benefit regulations}

\renewcommand\parthead{--- Part VI}

\subsection[11. Amendment of the Housing Benefit Regulations]{Amendment of the Housing Benefit Regulations}

11.---(1)  The Housing Benefit Regulations\footnote{S.I.~1987/1971. The relevant amending instruments are S.I.~1999/1539, 2002/1397, 2003/325 and 2004/2327.} as modified in their application to persons to whom regulation~2(1) of the Housing Benefit and~Council Tax Benefit (State Pension Credit) Regulations 2003\footnote{S.I 2003/325. The relevant amending instrument is S.I.~2003/2275.} applies shall be amended in accordance with the following paragraphs.

(2) In regulation~2(1) (interpretation), after the definition of “gateway office” insert—
\begin{quotation}
\begin{sloppypar}
““the Graduated Retirement Benefit Regulations” means the Social Security (Graduated Retirement Benefit) Regulations 2005\footnote{S.I.~2005/454.};”.
\end{sloppypar}
\end{quotation}

(3) In regulation~36 (notional income)—
\begin{enumerate}\item[]
\begin{sloppypar}
($a$) at the beginning of paragraph~(6) insert “Subject to paragraph~(6A),”;
\end{sloppypar}

($b$) after paragraph~(6) insert—
\begin{quotation}
“(6A) Paragraph~(6) shall not apply in respect of the amount of an increase of pension or~benefit where a person, having made an election in favour of that increase of pension or~benefit under Schedule~5 or~5A to the Contributions and~Benefits Act\footnote{Schedule 5A is inserted by paragraph 15 of Schedule 11 to the Pensions Act 2004 (c.~35).} or~under Schedule~1 to the Graduated Retirement Benefit Regulations, changes that election in accordance with regulations made under Schedule~5 or~5A to that Act in favour of a lump sum.

(6B) In paragraph~(6A), “lump sum” means a lump sum under Schedule~5 or~5A to the Contributions and~Benefits Act or~under Schedule~1 to the Graduated Retirement Benefit Regulations.”.
\end{quotation}
\end{enumerate}

(4) In Schedule~5ZA, in Part I (capital to be disregarded), after paragraph~25A\footnote{Paragraph 25A is inserted by S.I.~2003/2275.} insert—
\begin{quotation}
“25B.  Where a person elects to be entitled to a lump sum under Schedule~5 or~5A to the Contributions and~Benefits Act or~under Schedule~1 to the Graduated Retirement Benefit Regulations, or~is treated as having made such an election, and~a payment has been made pursuant to that election, an amount equal to—
\begin{enumerate}\item[]
($a$) except where sub-paragraph~($b$)  applies, the amount of any payment or~payments made on account of that lump sum;

($b$) the amount of that lump sum,
\end{enumerate}
but only for~so long as that person does not change that election in favour of an increase of pension or~benefit.”.
\end{quotation}

\subsection[12. Amendment of the Council Tax Benefit (General) Regulations 1992]{Amendment of the Council Tax Benefit (General) Regulations 1992}

12.---(1)  The Council Tax Benefit (General) Regulations 1992\footnote{S.I.~1992/1814. The relevant amending instruments are S.I.~1999/1539, 2003/325 and 2004/2327.} as modified in their application to persons to whom regulation~12(1) of the Housing Benefit and~Council Tax Benefit (State Pension Credit) Regulations 2003 applies shall be amended in accordance with the following paragraphs.

(2) In regulation~2(1) (interpretation), after the definition of “gateway office”, insert—
\begin{quotation}\sloppy
““the Graduated Retirement Benefit Regulations” means the Social Security (Graduated Retirement Benefit) Regulations 2005\footnote{S.I.~2005/454.};”.
\end{quotation}

(3) In regulation~28 (notional income)—
\begin{enumerate}\item[]
\begin{sloppypar}
($a$) at the beginning of paragraph~(6) insert “Subject to paragraph~(6A),”;
\end{sloppypar}

($b$) after paragraph~(6) insert—
\begin{quotation}
“(6A) Paragraph~(6) shall not apply in respect of the amount of an increase of pension or~benefit where a person, having made an election in favour of that increase of pension or~benefit under Schedule~5 or~5A to the Contributions and~Benefits Act 1992\footnote{Schedule 5A is inserted by paragraph 15 of Schedule 11 to the Pensions Act 2004 (c.~35).} or~under Schedule~1 to the Graduated Retirement Benefit Regulations, changes that election in accordance with regulations made under Schedule~5 or~5A to that Act in favour of a lump sum.

(6B) In paragraph~(6A), “lump sum” means a lump sum under Schedule~5 or~5A to the Contributions and~Benefits Act 1992 or~under Schedule~1 to the Graduated Retirement Benefit Regulations.”.
\end{quotation}
\end{enumerate}

(4) In Schedule~5ZA, in Part I (capital to be disregarded), after paragraph~25A insert—
\begin{quotation}
“25B.  Where a person elects to be entitled to a lump sum under Schedule~5 or~5A to the Contributions and~Benefits Act 1992 or~under Schedule~1 to the Graduated Retirement Benefit Regulations, or~is treated as having made such an election, and~a payment has been made pursuant to that election, an amount equal to—
\begin{enumerate}\item[]
($a$) except where sub-paragraph~($b$)  applies, the amount of any payment or~payments made on account of that lump sum;

($b$) the amount of that lump sum,
\end{enumerate}
but only for~so long as that person does not change that election in favour of an increase of pension or~benefit.”.
\end{quotation}

\subsection[13. Amendment of the State Pension Credit Regulations 2002]{Amendment of the State Pension Credit Regulations 2002}

13.---(1)  The State Pension Credit Regulations 2002\footnote{S.I.~2002/1792.} shall be amended in accordance with the following paragraphs.

(2) In regulation~1(2) (interpretation), after the definition of “full-time student” insert—
\begin{quotation}\sloppy
““the Graduated Retirement Benefit Regulations” means the Social Security (Graduated Retirement Benefit) Regulations 2005\footnote{S.I.~2005/454.};”.
\end{quotation}

(3) In regulation~18 (notional income)—
\begin{enumerate}\item[]
($a$) at the beginning of paragraph~(6) insert “Subject to paragraph~(7),”;

($b$) after paragraph~(6) add—
\begin{quotation}
“(7) Paragraph~(6) shall not apply in respect of the amount of an increase of pension or~benefit where a person, having made an election in favour of that increase of pension or~benefit under Schedule~5 or~5A to the 1992 Act\footnote{Schedule 5A is inserted by paragraph 15 of Schedule 11 to the Pensions Act 2004 (c.~35).} or~under Schedule~1 to the Graduated Retirement Benefit Regulations, changes that election in accordance with regulations made under Schedule~5 or~5A to that Act in favour of a lump sum.

(8) In paragraph~(7), “lump sum” means a lump sum under Schedule~5 or~5A to the 1992 Act or~under Schedule~1 to the Graduated Retirement Benefit Regulations.”.
\end{quotation}
\end{enumerate}

(4) In Schedule~V, in Part I (capital disregarded for~the purpose of calculating income), after paragraph~23 insert—
\begin{quotation}
“23A.  Where a person elects to be entitled to a lump sum under Schedule~5 or~5A to the 1992 Act or~under Schedule~1 to the Graduated Retirement Benefit Regulations, or~is treated as having made such an election, and~a payment has been made pursuant to that election, an amount equal to—
\begin{enumerate}\item[]
($a$) except where sub-paragraph~($b$)  applies, the amount of any payment or~payments made on account of that lump sum;

($b$) the amount of that lump sum,
\end{enumerate}
but only for~so long as that person does not change that election in favour of an increase of pension or~benefit.”.
\end{quotation}

\bigskip

Signed 
by authority of the 
Secretary of State for~Work and~Pensions.
%I concur
%By authority of the Lord Chancellor

{\raggedleft
\emph{Stephen C.~Timms}\\*
%Secretary
Minister
%Parliamentary Under-Secretary 
of State,\\*Department 
for~Work and~Pensions

}

26th September 2005

\small

\part{Explanatory Note}

\renewcommand\parthead{— Explanatory Note}

\subsection*{(This note is not part of the Regulations)}

These Regulations make provision relating to changes to the regime for~deferring entitlement to state pension made by the Pensions Act 2004 (c.~35) which provide for~a choice between increments and~a lump sum for~those who have deferred their entitlement to retirement pension, shared additional pension or~graduated retirement benefit, for~12 months or~more.

Part II makes provision in relation to deferral of retirement pension and~shared additional pension. Regulation~3 prescribes the period within which an election between increments and~lump sums of retirement pension and~shared additional pension must be made and~regulation~4 prescribes the manner in which such elections must be made. Regulation~5 prescribes the circumstances and~manner in which, and~time within which, changes to such elections can be made. Regulation~6 omits a transitional provision relating to deferral of retirement pension.

Part III makes equivalent provision to Part II in relation to deferral of graduated retirement benefit.

Part IV relates to payments. Regulation~8 amends the Social Security (Claims and~Payments) Regulations 1987 (S.I.~1987/1968). It provides that when a person chooses a lump sum he may elect to be paid it in the tax year following the tax year which would otherwise be the year for~assessing tax on the lump sum.

Part V relates to decisions. Regulation~9 amends the Child Support and~Social Security (Decisions and~Appeals) Regulations 1999 (S.I.~1999/991). Paragraphs~(3),~(4) and~(5) provide for~the revision or~supersession of a state pension credit decision when a person becomes entitled to a lump sum. Paragraph~(3) also provides for~revision of a retirement pension, shared additional pension or~graduated retirement benefit decision when an election is changed pursuant to provision made in Parts II and~III of these Regulations. Paragraph~(6) provides that a claim for~such a pension or~benefit following deferment may be decided pending an election for~increments or~a lump sum.

Regulation~10 amends the Housing Benefit and~Council Tax Benefit (Decisions and~Appeals) Regulations 2001 (S.I.~2001/1002) to provide for~the revision or~supersession of a housing benefit or~council tax benefit decision when a person becomes entitled to a lump sum.

Part 6 amends various benefit regulations in so far as they relate to deferral of retirement pension, shared additional pension and~graduated retirement benefit. Regulations 11 and~12 amend respectively the Housing Benefit (General) Regulations 1987 (S.I.~1987/1971) and~the Council Tax Benefit (General) Regulations 1992 (S.I.~1992/1814) as modified by the Housing Benefit and~Council Tax Benefit (State Pension Credit) Regulations 2003 (S.I.~2003/325) for~persons who have attained the qualifying age for~state pension credit and~regulation~13 amends the State Pension Credit Regulations 2002 (S.I.~2002/1792).

In regulations 11 to 13, paragraph~(2) prescribes a definition of the Graduated Retirement Benefit Regulations for~the purposes of those benefits, paragraph~(3) provides an exception to the notional income rule in those benefits where a person having deferred their pension or~benefit in favour of an increase of pension or~benefit, changes that election in favour of a lump sum and~paragraph~(4) provides that an amount of capital equal to the amount of a payment on account of a lump sum or~the amount of the lump sum itself, is to be disregarded in the calculation of income in the case of state pension credit and~capital in the case of housing benefit and~council tax benefit.

A full regulatory impact assessment has not been produced for~this instrument as it has no impact on the costs of business, charities and~voluntary bodies. 

\end{document}