\documentclass[12pt,a4paper]{article}

\newcommand\regstitle{The Social Security Act 1998 (Commencement No.\ 12 and Consequential and Transitional Provisions) Order 1999}

\newcommand\regsnumber{1999/3178}

%\opt{newrules}{
\title{\regstitle}
%}

%\opt{2012rules}{
%\title{Child Maintenance and Other Payments Act 2008\\(2012 scheme version)}
%}

\author{S.I. 1999 No. 3178 (C. 81)}

\date{Made
25th November 1999\\
%Laid before Parliament
%27th January 2000\\
%Coming into force
%17th February 2000
}

%\opt{oldrules}{\newcommand\versionyear{1993}}
%\opt{newrules}{\newcommand\versionyear{2003}}
%\opt{2012rules}{\newcommand\versionyear{2012}}

\usepackage{csa-regs}

\setlength\headheight{27.57402pt}

\begin{document}

\maketitle

\noindent
The Secretary of State for Social Security, in exercise of the powers conferred upon him by sections 79(3) and (4) and 87(2) and (3) of the Social Security Act 1998\footnote{\frenchspacing 1998 c. 14.} and of all other powers enabling him in that behalf, after consultation with the Council on Tribunals in accordance with section 8 of the Tribunals and Inquiries Act 1992\footnote{\frenchspacing 1992 c. 53.}, hereby makes the following Order:  

{\sloppy

\tableofcontents

}

\bigskip

\setcounter{secnumdepth}{-2}

\subsection[1. Citation and interpretation]{Citation and interpretation}

1.---(1)  This Order may be cited as the Social Security Act 1998 (Commencement No.\ 12 and Consequential and Transitional Provisions) Order 1999.

(2) In this Order, unless the context otherwise requires---
\begin{enumerate}\item[]
($a$) “the Act” means the Social Security Act 1998;

($b$) “the 1997 Regulations” means the Social Security (Recovery of Benefits) (Appeals) Regulations 1997\footnote{\frenchspacing S.I. 1997/2237.};

($c$) “the Adjudication Regulations” means the Social Security (Adjudication) Regulations 1995\footnote{\frenchspacing S.I. 1995/1801; relevant amending instruments are S.I. 1996/182 and 2450.};

($d$) “legally qualified panel member” has the meaning it bears in regulation 1(3) of the Regulations;

($e$) “the No.\ 8 Order” means the Social Security Act 1998 (Commencement No.\ 8, and Savings and Consequential and Transitional Provisions) Order 1999\footnote{\frenchspacing S.I. 1999/1958 (C. 51).};

($f$) “the No.\ 9 Order” means the Social Security Act 1998 (Commencement No.\ 9, and Savings and Consequential and Transitional Provisions) Order 1999\footnote{\frenchspacing S.I. 1999/2422 (C. 61).};

($g$) “the No.\ 11 Order” means the Social Security Act 1998 (Commencement No.\ 11, and Savings and Consequential and Transitional Provisions) Order 1999\footnote{\frenchspacing S.I. 1999/2860 (C. 75).};

($h$) “the Regulations” means the Social Security and Child Support (Decisions and Appeals) Regulations 1999\footnote{\frenchspacing S.I. 1999/991; to which there are amendments not relevant to this Order.}; and

($i$) “relevant benefit” means income support, child’s special allowance under section 56 of the Contributions and Benefits Act or, as the case may be, a social fund payment mentioned in section 138(1)($a$)  or (2) of that Act,
\end{enumerate}
and references to sections and Schedules are references to sections of, and Schedules to, the Act.

\subsection[2. Appointed day]{Appointed day}

2.---(1)  Subject to paragraph (2) below, 29th November 1999 is the day appointed for the coming into force of the---
\begin{enumerate}\item[]
($a$) the provisions specified in Schedule 1 to this Order; and

($b$) sections 1 and 4 to 7 in relation to---
\begin{enumerate}\item[]
(i) statutory sick pay under Part XI of the Contributions and Benefits Act; and

(ii) statutory maternity pay under Part XII of that Act,
\end{enumerate}
\end{enumerate}
in so far as those provisions are not already in force.

(2) Paragraph (1)($a$)  above shall not apply in relation to---
\begin{enumerate}\item[]
($a$) housing benefit;

($b$) council tax benefit; nor

($c$) decisions to which article 4(6) of the Social Security Contributions (Transfer of Functions, etc.)\ Act 1999 (Commencement No.\ 1 and Transitional Provisions) Order 1999\footnote{\frenchspacing S.I. 1999/527 (C. 11).} applies.
\end{enumerate}

(3) 31st March 2000 is the day appointed for the coming into force of paragraph 10 of Schedule 1 (appeal tribunals: supplementary provisions) and sections 5(3) and 7(7) in so far as they relate to it.

\subsection[3. Consequential amendments]{Consequential amendments}

3.---(1)  The amendments and revocations effected by this Order---
\begin{enumerate}\item[]
($a$) shall take effect as from 29th November 1999; and

($b$) shall not have effect in relation to---
\begin{enumerate}\item[]
(i) working families' tax credit;

(ii) disabled person’s tax credit;

(iii) decisions to which article 4(6) of the Social Security Contributions (Transfer of Functions, etc.)\ Act 1999 (Commencement No.\ 1 and Transitional Provisions) Order 1999 applies.
\end{enumerate}
\end{enumerate}

(2) The Social Security Benefit (Dependency) Regulations 1977\footnote{\frenchspacing S.I. 1977/343; amendment was made by the Health and Social Services and Social Security Adjudications Act 1983 (c. 41), Schedule 8, paragraph 1(3)($a$).} shall be amended in accordance with Schedule 2 to this Order.

(3) The Social Security (General Benefit) Regulations 1982\footnote{\frenchspacing S.I. 1982/1408; relevant amending instruments are S.I. 1983/186 and 1993/1985.} shall be amended in accordance with Schedule 3 to this Order.

(4) The Social Fund Maternity and Funeral Expenses (General) Regulations 1987\footnote{\frenchspacing S.I. 1987/481; relevant amending instrument is S.I. 1997/2538.} shall be amended in accordance with Schedule 4 to this Order.

(5) The Income Support (General) Regulations 1987\footnote{\frenchspacing S.I. 1987/1967; relevant amending instruments are S.I. 1989/534, 663, 1678, 1990/671, 1993/2119, 1995/516, 2303, 1996/206, 462 and 1997/65 and 2863.} shall be amended in accordance with Schedule 5 to this Order.

(6) The Social Security (Claims and Payments) Regulations 1987\footnote{\frenchspacing S.I. 1987/1968; relevant amending instruments are S.I. 1988/522, 1989/136 and 1686, 1990/2208, 1991/387, 2741 and 2284, 1992/247, 1026 and 2595, 1993/1113 and 2113, 1994/2319 and 2943, 1995/1613, 1996/1460 and 2306 and 1998/1174 and 1381.} shall be amended in accordance with Schedule 6 to this Order.

(7) The Housing Benefit (General) Regulations 1987\footnote{\frenchspacing S.I. 1987/1971; relevant amending instruments are S.I. 1995/560 and 626 and 1996/1510.} and the Council Tax Benefit (General) Regulations 1992\footnote{\frenchspacing S.I. 1992/1814; relevant amending instruments are S.I. 1995/560 and 626 and 1996/1510.} shall be amended in accordance with Schedule 7 to this Order.

(8) The Social Fund (Application for Review) Regulations 1988\footnote{\frenchspacing S.I. 1988/34; relevant amending instruments are S.I. 1988/1843 and 1990/580.} shall be amended in accordance with Schedule 8 to this Order.

(9) The Social Security (Payments on Account, Overpayments and Recovery) Regulations 1988\footnote{\frenchspacing S.I. 1988/664; relevant amending instruments are S.I. 1991/2742, 1996/30 and 1345.} shall be amended in accordance with Schedule 9 to this Order.

(10) The Community Charges (Deductions from Income Support) (Scotland) Regulations 1989\footnote{\frenchspacing S.I. 1989/507 (S. 59); relevant amending instruments are S.I. 1990/113, 1992/1026, 1993/2113 and 1996/2344.} shall be amended in accordance with Schedule 10 to this Order.

(11) The Community Charges (Deductions from Income Support) (No.\ 2) Regulations 1990\footnote{\frenchspacing S.I. 1990/545; relevant amending instruments are S.I. 1992/1026, 1993/2113 and 1996/2344.} shall be amended in accordance with Schedule 11 to this Order.

(12) The Fines (Deductions from Income Support) Regulations 1992\footnote{\frenchspacing S.I. 1992/2182; relevant amending instruments are S.I. 1993/495, 1996/2344 and 1997/827.} shall be amended in accordance with Schedule 12 to this Order.

(13) The Council Tax (Deductions from Income Support) Regulations 1993\footnote{\frenchspacing S.I. 1993/494; relevant amending instruments are S.I. 1996/2344 and 1997/827.} shall be amended in accordance with Schedule 13 to this Order.

(14) The Employment Protection (Recoupment of Jobseeker’s Allowance and Income Support) Regulations 1996\footnote{\frenchspacing S.I. 1996/2349.} shall be amended in accordance with Schedule 14 to this Order.

(15) The Social Security (Back to Work Bonus) (No.\ 2) Regulations 1996\footnote{\frenchspacing S.I. 1996/2570; the relevant amending instrument is S.I. 1997/454.} shall be amended in accordance with Schedule 15 to this Order.

(16) The Social Security Benefit (Computation of Earnings) Regulations 1996\footnote{\frenchspacing S.I. 1996/2745.} shall be amended in accordance with Schedule 16 to this Order.

(17) The Social Security (Recovery of Benefits) Regulations 1997\footnote{\frenchspacing S.I. 1997/2205.} shall be amended in accordance with Schedule 17 to this Order.

(18) The Social Fund Winter Fuel Payment Regulations 1998\footnote{\frenchspacing S.I. 1998/19, to which there are amendments not relevant to this Order.} shall be amended in accordance with Schedule 18 to this Order.

(19) The Social Security and Child Support (Decisions and Appeals) Regulations 1999 shall be amended in accordance with Schedule 19 to this Order.

(20) The No.\ 8 Order, the No.\ 9 Order and the No.\ 11 Order shall be amended in accordance with Schedule 20 to this Order.

\subsection[4. Transitional provisions]{Transitional provisions}

4.  Schedules 21 to 23 to this Order shall take effect as from 29th November 1999. 

\bigskip

Signed 
by authority of the Secretary of State for Social Security.

{\raggedleft
\emph{Angela Eagle
}\\*Parliamentary Under-Secretary of State,\\*Department of Social Security

}

25th November 1999

\small

\bigskip

\part[Schedule 1 --- Provisions brought into force on 29th November 1999]{Schedule 1\\* Provisions brought into force on 29th November 1999}

\renewcommand\parthead{--- Schedule 1}

{\footnotesize\noindent
%\begin{tabulary}{\textwidth}{JJ}
\begin{longtable}{p{183pt}p{183pt}}
\hline
\itshape Provision of the Act	& \itshape Subject matter\\
\hline
\endhead
\hline
\endlastfoot
Section 1	&Transfer of functions to the Secretary of State\\
Section 2	&Use of computers\\
Section 4	&Unified appeal tribunals\\
Section 8\footnote{\frenchspacing Section 8 was amended by paragraph 22 of Schedule 7 to the Social Security Contributions (Transfer of Functions, etc.) Act 1999 (c. 2) (“the Transfer Act”).}	&Decisions by the Secretary of State\\
Sections 9 and 10\footnote{\frenchspacing Section 10 was amended by paragraph 23 of Schedule 7 to the Transfer Act.}	&Revision of decisions and decisions superseding earlier decisions\\
Section 11	&Regulations with respect to decisions\\
Section 12 and Schedules 2 and 3\footnote{\frenchspacing Section 12 and Schedule 3 were amended by paragraphs 25 and 36 respectively of Schedule 7 to the Transfer Act.}	&Appeal to appeal tribunal\\
Section 13\footnote{\frenchspacing Sections 13 and 14 were amended by paragraphs 26 and 27 respectively of Schedule 7 to the Transfer Act.\label{fn:31}}	&Redetermination etc.\ of appeals by tribunal\\
Section 14\footref{fn:31} and Schedule 4	&Appeal from tribunal to Commissioner\\
Section 15	&Appeal from Commissioner on point of law\\
Section 16 and Schedule 5	&Procedure\\
Section 17	&Finality of decisions\\
Sections 18 to 20\footnote{\frenchspacing Sections 18, 19 and 20 were amended by paragraphs 29, 30 and 31 respectively of Schedule 7 to the Transfer Act.}	&Matters arising as respects decisions, and medical examination required by the Secretary of State and appeal tribunal\\
Sections 21\footnote{\frenchspacing Section 21 was amended by paragraph 32 of Schedule 7 to the Transfer Act.} to 27	&Suspension and termination of benefit, decisions and appeals involving issues that arise on appeal in other cases, and restrictions on entitlement to benefit in certain cases of error\\
Section 28\footnote{\frenchspacing Section 28 was amended by paragraph 34 of Schedule 7 to the Transfer Act.}	&Correction of errors and setting aside of decisions\\
Section 33	&Christmas bonus\\
Sections 36 to 38	&Appropriate officers, the social fund Commissioner and inspectors, and reviews of determinations\\
Section 39\footnote{\frenchspacing Section 39 was amended by paragraph 35 of Schedule 7 to the Transfer Act.}	&Interpretation etc.\ of Chapter II of Part I of the Act\\
Section 74	&Provision of information\\
Schedule 7\footnote{\frenchspacing Certain provisions of Schedule 7 were repealed by Schedule 10 to the Transfer Act.} in the respects specified below, and section 86(1) in so far as it relates to them---	&Minor and consequential amendments\\
\hspace{1em}    Paragraph 4(1) and (2) &
	House of Commons disqualification\\
\hspace{1em}
    Paragraph 11 &
	Provision as to forfeiture\\
\hspace{1em}
    Paragraphs 15 and 17 &
	Recovery of outstanding community charge by deductions from income support or jobseeker’s allowance\\
\hspace{1em}
    Paragraph 47 &
	Interpretation of the Child Support Act 1991\footnote{\frenchspacing 1991 c. 48.}\\
\hspace{1em}
    Paragraph 55 &
	Recovery of fines etc.\ by deductions from income support or jobseeker’s allowance\\
\hspace{1em}
    Paragraphs 66 to 73 &
	Miscellaneous provisions relating to benefits, and social fund\\
\hspace{1em}
    Paragraph 79(1) &
	Regulations about claims for and payments of benefit\\
\hspace{1em}
    Paragraph 81 &
	Overpayments\\
\hspace{1em}
    Paragraph 83 &
	Legal proceedings\\
\hspace{1em}
    Paragraph 84 &
	Issues arising in proceedings\\
\hspace{1em}
    Paragraph 88 &

	Unauthorised disclosure of information relating to particular persons\\
\hspace{1em}
    Paragraph 89 &

	Regulations as to notification of deaths\\
\hspace{1em}
    Paragraph 95 &

	Effect of alteration in the component rates of income support\\
\hspace{1em}
    Paragraph 97 &

	Implementation of increases in income support due to attainment of particular ages\\
\hspace{1em}
    Paragraph 101 &

	Destination of payments\\
\hspace{1em}
    Paragraph 102 &

	Financial review and report\\
\hspace{1em}
    Paragraph 103 &

	Allocations from social fund\\
\hspace{1em}
    Paragraphs 106 to 108 &

	Reciprocal agreements, and travelling expenses\\
\hspace{1em}
    Paragraph 109 &

	Regulations and orders under the Administration Act\\
\hspace{1em}
    Paragraph 110 &

	Instruments containing provisions under certain provisions to be subject to the affirmative Parliamentary procedure\\
\hspace{1em}
    Paragraph 111 &

	Interpretation of the Administration Act\\
\hspace{1em}
    Paragraph 112 &

	Extent of the Administration Act\\
\hspace{1em}
    Paragraph 113 &

	Unauthorised disclosure of information by persons employed in social security administration or adjudication\\
\hspace{1em}
    Paragraph 116 &

	Supplementary benefit\\
\hspace{1em}
    Paragraph 117 &

	Recovery of council tax etc.\ by deductions from income support or jobseeker’s allowance\\
\hspace{1em}
    Paragraphs 118, 120($b$)  and 121(2) &

	Appointment of chairmen of certain tribunals, and tribunals under supervision of Council on Tribunals\\
\hspace{1em}
    Paragraphs 123 to 125 &

	Judicial pensions—the offices which may be qualifying judicial offices, retirement provisions: the relevant offices, and retirement dates: transitional provisions\\
\hspace{1em}
    Paragraph 129 &

	Pension Schemes—disclosure of information between government departments etc.\\
\hspace{1em}
    Paragraph 147 &

	Industrial tribunals—power to provide for recoupment of benefits\\
\hspace{1em}
    Paragraphs 149 to 153 &

	Review of, and appeals against, certificates of recoverable benefits, and general interpretation of the Social Security (Recovery of Benefits) Act 1997\footnote{\frenchspacing 1997 c. 27.}\\

Schedule 8 in respect of the legislation specified below, and section 86(2) in so far as it relates to it---&
	Repeals\\
\hspace{1em}%
The House of Commons Disqualifica\-\hspace*{1em}tion Act 1975\footnote{\frenchspacing 1975 c. 24.}\\
%{\setlength{\leftmargin}{-1em}%
\hspace{1em}%
The Health and Social Services and \hspace*{1em}%
Social Security Adjudications Act 1983\footnote{\frenchspacing 1983 c. 41.}
%}
\\
\hspace{1em}%
In the Administration Act---\\
\hspace*{1em}\hspace{1em}%
    Part II\\
\hspace{1em}\hspace{1em}%
    Section 116(6)\\
\hspace{1em}\hspace{1em}%
    In section 189, in subsection (1), the \hspace*{2em}words “subsection (2) below and to”, \hspace*{2em}subsection (2), in subsection (4), the \hspace*{2em}words “24 or”, in subsection (5), the \hspace*{2em}words “(other than the power con\-\hspace*{2em}ferred by section 24 above)”, in sub\-\hspace*{2em}section (6), the word “24,” and sub\-\hspace*{2em}section (10)\\
\hspace{1em}\hspace{1em}%
    Section 190(4)\\
\hspace{1em}\hspace{1em}%
    In section 191, the definitions of \hspace*{2em}“Commissioner”, “the disablement \hspace*{2em}questions”, “5 year general qualifica\-\hspace*{2em}tion”, “President” and “10 year gen\-\hspace*{2em}eral qualification”\\
\hspace{1em}\hspace{1em}%
    In section 192(5), the words “section \hspace*{2em}24;”\\
\hspace{1em}\hspace{1em}%
    In Schedule 4, in Part I, the entry \hspace*{2em}headed “\emph{Adjudication officers}”, in the \hspace*{2em}entry headed “\emph{Adjudicating bodies}”, \hspace*{2em}paragraphs ($b$)  to ($d$)  and in the entry \hspace*{2em}headed “\emph{The Social Fund}”, the words \hspace*{2em}“A social fund officer” \\
\hspace*{1em}The Social Security (Consequential Pro\-\hspace*{1em}visions) Act 1992\footnote{\frenchspacing 1992 c. 6.}\\
\hspace{1em}The Local Government Finance Act \hspace*{1em}1992\footnote{\frenchspacing 1992 c. 14.}\\
\hspace{1em}The Tribunals and Inquiries Act 1992\footnote{\frenchspacing 1992 c. 53.}\\
\hspace{1em}The Judicial Pensions and Retirement \hspace*{1em}Act 1993\footnote{\frenchspacing 1993 c. 8.}\\
\hspace{1em}%
The Pension Schemes Act 1993\footnote{\frenchspacing 1993 c. 48.}\\
\hspace{1em}%
Social Security (Incapacity for Work) \hspace*{1em}Act 1994\footnote{\frenchspacing 1994 c. 18.}\\
%\hspace{1em}
%In Schedule 1, paragraphs 11 and 46 to 48\\
\hspace{1em}%
The Deregulation and Contracting Out \hspace*{1em}Act 1994\footnote{\frenchspacing 1994 c. 40.}\\
\hspace{1em}%
In the Pensions Act 1995\footnote{\frenchspacing 1995 c. 26.}\\
\hspace{1em}%
The Industrial Tribunals Act 1996\footnote{\frenchspacing 1996 c. 17.}\\
\hspace{1em}%
The Arbitration Act 1996\footnote{\frenchspacing 1996 c. 23.}\\
\hspace{1em}%
The Social Security (Recovery of Bene\-\hspace*{1em}fits) Act 1997\footnote{\frenchspacing 1997 c. 27.}\\
\hspace{1em}%
The Social Security Administration \hspace*{1em}(Fraud) Act 1997\footnote{\frenchspacing 1997 c. 47.}\\
\hspace{1em}%
The Social Security Act 1998\footnote{\frenchspacing 1998 c. 14.}\\
%\end{tabulary}
\end{longtable}

}

\part[Schedule 2 --- Amendment of the Social Security Benefit (Dependency) Regulations 1977]{Schedule 2\\*Amendment of the Social Security Benefit (Dependency) Regulations 1977}

\renewcommand\parthead{--- Schedule 2}

1.  In regulation 1(2)\footnote{\frenchspacing Amendment was made by the Health and Social Services and Social Security Adjudications Act 1983, (c. 41), Schedule 8, paragraph 1(3)($a$).} (interpretation), for the definition of “the determining authority” there shall be substituted the following definition---
\begin{quotation}
    ““determining authority” means, as the case may require, the Secretary of State, an appeal tribunal constituted under Chapter I of Part I of the Social Security Act 1998 (“an appeal tribunal”) or a Commissioner within the meaning of section 39(1) of that Act;”. 
\end{quotation}

\medskip

2.  In regulation 3\footnote{\frenchspacing Regulation 3 was amended by S.I. 1983/1001, 1984/1699, 1985/1305, 1994/2945 and 1996/1345.} (allocation of contributions for spouse or children) in paragraph (4)($b$)---
\begin{enumerate}\item[]
($a$) for the words “varied on review” there shall be substituted the word “superseded”;

($b$) for the words “determining authority” there shall be substituted the words “the Secretary of State”; and

($c$) subject to sub-paragraph ($a$)  above, for the word “review” in each place in which it occurs there shall be substituted the word “supersession”.
\end{enumerate}

\part[Schedule 3 --- Amendment of the Social Security (General Benefit) Regulations 1982]{Schedule 3\\*Amendment of the Social Security (General Benefit) Regulations 1982}

\renewcommand\parthead{--- Schedule 3}

1.  In regulation 1(2)\footnote{\frenchspacing Amendment was made by the Health and Social Services and Social Security Adjudications Act 1983, Schedule 8, paragraph 1(3)($a$) and 1991/2742.} (interpretation), for the definition of “determining authority” there shall be substituted the following definition---
\begin{quotation}
    ““determining authority” means, as the case may require, the Secretary of State, an appeal tribunal constituted under Chapter I of Part I of the Social Security Act 1998 (“an appeal tribunal”) or a Commissioner within the meaning of section 39(1) of that Act;”. 
\end{quotation}

\medskip

2.  In paragraphs (7) and (8) of regulation 11 (further definition of the principles of assessment of disablement and prescribed degrees of disablement), for the words from “the medical appeal tribunal” to “(as the case may be)” there shall be substituted the words “the Secretary of State or, as the case may be, an appeal tribunal”.

\medskip

3.  In paragraphs (2) to (4) of regulation 40\footnote{\frenchspacing Regulation 40 was amended by the Health and Social Services and Social Security Adjudications Act 1983 Schedule 8, paragraph 1(3)($a$) and S.I. 1983/186.} (disqualification for receipt of benefit, suspension of proceedings on claims and suspension of payment of benefit), for the words “adjudication officer, a social security appeal tribunal or the Commissioner” there shall be substituted the words “determining authority”.

\part[Schedule 4 --- Amendment of the Social Fund Maternity and Funeral Expenses (General) Regulations 1987]{Schedule 4\\*Amendment of the Social Fund Maternity and Funeral Expenses (General) Regulations 1987}

\renewcommand\parthead{--- Schedule 4}

In regulation 4(4)\footnote{\frenchspacing Paragraph (4) was added by S.I. 1997/2538.} (provisions against double payment)---
\begin{enumerate}\item[]
($a$) in sub-paragraph ($a$), for the word “reviewed” there shall be substituted the word “revised”; and

($b$) in sub-paragraph ($b$), the words “on that review” shall be omitted.
\end{enumerate}

\part[Schedule 5 --- Amendment of the Income Support (General) Regulations 1987]{Schedule 5\\*Amendment of the Income Support (General) Regulations 1987}

\renewcommand\parthead{--- Schedule 5}

1.  In regulation 5(2)($b$)(ii)  (persons treated as engaged in remunerative work), for the words “of review” there shall be substituted the words “on which a superseding decision is made under section 10 (decisions superseding earlier decisions) of the Social Security Act 1998”.

\medskip

2.  In regulation 38 (calculation of net profit of self-employed earners)---
\begin{enumerate}\item[]
($a$) in paragraph (7), for the words “An adjudication officer” there shall be substituted the words “The Secretary of State”; and

($b$) in paragraph (10), for the words “an adjudication officer” there shall be substituted the words “the Secretary of State”.
\end{enumerate}

\medskip

3.  For paragraph (1) of regulation 41\footnote{\frenchspacing Regulation 41(1) was amended by S.I. 1990/671 and 1997/65.} (capital treated as income) there shall be substituted the following paragraph---
\begin{quotation}
“(1) Capital which is payable by instalments which are outstanding on---
\begin{enumerate}\item[]
($a$) the first day in respect of which income support is payable or the date of the determination of the claim, whichever is earlier; or

($b$) in the case of a supersession, the date of that supersession,
\end{enumerate}
shall be treated as income if the aggregate of the instalments outstanding and the amount of the claimant’s capital otherwise calculated in accordance with Chapter VI of this Part exceeds £8,000 or, in a case where regulation 45($b$)\footnote{\frenchspacing Regulation 45 was substituted by S.I. 1996/462.} applies, £16,000.”.
\end{quotation}

\medskip

4.  In regulation 42 (notional income)---
\begin{enumerate}\item[]
($a$) in paragraphs (2B)\footnote{\frenchspacing Regulation 42(2B) was inserted by S.I. 1995/2303.} and (6)\footnote{\frenchspacing Regulation 42(6) was amended by S.I. 1999/2554.}, for the words “adjudication officer” in each place in which they occur there shall be substituted the words “Secretary of State”; and

($b$) in paragraph (5), for the words “subsequent review the adjudication officer” there shall be substituted the words “revision or supersession the Secretary of State”.
\end{enumerate}

\medskip

5.  For paragraph (1) of regulation 44\footnote{\frenchspacing Regulation 44(1) was amended by S.I. 1993/2119.} (modifications in respect of children and young persons) there shall be substituted the following paragraph---
\begin{quotation}
“(1) Any capital of a child or young person payable by instalments which are outstanding on---
\begin{enumerate}\item[]
($a$) the first day in respect of which income support is payable or at the date of the determination of the claim, whichever is the earlier; or

($b$) in the case of a supersession, the date of that supersession,
\end{enumerate}
shall be treated as income if the aggregate of the instalments outstanding and the amount of that child’s or young person’s other capital calculated in accordance with Chapter VI of this Part in like manner as for the claimant would exceed £3,000.”.
\end{quotation}

\medskip

6.  In regulation 49($b$)(i)  (calculation of capital in the United Kingdom), for the words “review, the date of any subsequent review” there shall be substituted the words “supersession, the date of that supersession”.

\medskip

7.  In regulation 69 (disregard of changes occurring during summer vacation) for the words “an adjudication officer shall disregard” there shall be substituted the words “there shall be disregarded”.

\medskip

8.  In regulation 70(4)($b$)\footnote{\frenchspacing Regulation 70(4) was amended by S.I. 1989/1323.} (urgent cases) for the words “adjudication officer” there shall be substituted the words “Secretary of State”.

\medskip

9.  In Schedule 3\footnote{\frenchspacing Schedule 3 was substituted by S.I. 1995/1613; relevant amending instruments are S.I. 1995/2927, 1996/1944 and 2518, 1997/827 and 1999/264.} (housing costs)---
\begin{enumerate}\item[]
($a$) in paragraph 13---
\begin{enumerate}\item[]
(i) for sub-paragraph (6) there shall be substituted the following sub-paragraph---
\begin{quotation}
“(6) Where sub-paragraph (4) does not apply and the claimant (or other member of the family) was able to meet the financial commitments for the dwelling occupied as the home when these were entered into, no restriction shall be made under this paragraph during the 26 weeks immediately following the date on which---
\begin{enumerate}\item[]
($a$) the claimant became entitled to income support where the claimant’s housing costs fell within one of the cases in sub-paragraph (1) on that date; or

($b$) a decision took effect which was made under section 10 (decisions superseding earlier decisions) of the Social Security Act 1998 on the ground that the claimant’s housing costs fell within one of the cases in sub-paragraph (1),
\end{enumerate}
nor during the next 26 weeks if and so long as the claimant uses his best endeavours to obtain cheaper accommodation.”; and
\end{quotation}

(ii) in sub-paragraph (8)($b$) , the words “on appeal or review” shall be omitted;
\end{enumerate}

($b$) in paragraph 14---
\begin{enumerate}\item[]
(i) in sub-paragraph (1)($a$)(i) , for the words “held, on appeal or review,” there shall be substituted the word “determined”;

(ii) in sub-paragraph (1)($b$) , for the words “held on appeal or review” there shall be substituted the word “determined”; and

(iii) in sub-paragraph (2)($a$) , the words “on review” shall be omitted; and
\end{enumerate}

($c$) in paragraph 18(2) and (7)($a$) , for the words “adjudication officer” there shall be substituted the words “Secretary of State”.
\end{enumerate}

\medskip

10.  In paragraph 3(4) of Schedule 3B\footnote{\frenchspacing Schedule 3B was inserted by S.I. 1989/534 and paragraph 3(4) was added by S.I. 1989/1678.} (protected sum)---
\begin{enumerate}\item[]
($a$) the words “on review” in each place in which they occur shall be omitted; and

($b$) in paragraph ($a$), for the words “that review” there shall be substituted the words “that determination”.
\end{enumerate}

\part[Schedule 6 --- Amendment of the Social Security (Claims and Payments) Regulations 1987]{Schedule 6\\*Amendment of the Social Security (Claims and Payments) Regulations 1987}

\renewcommand\parthead{--- Schedule 6}

1.  Subject to paragraphs 2 and 4 to 6, 10, 12 and 17 for the words “adjudicating authority” in each place in which they occur there shall be substituted the words “Secretary of State”.

\medskip

2.  In regulation 2(1) (interpretation)---
\begin{enumerate}\item[]
($a$) the definition of “adjudicating authority” shall be omitted; and

($b$) in paragraph ($c$)  of the definition of “claim for benefit”\footnote{\frenchspacing The definition of “claim for benefit” was amended by S.I. 1989/1686 and 1992/247.} for the words “the review of an award or”---
\begin{enumerate}\item[]
(i) in the first place in which they occur there shall be substituted the words “a revision under section 9 of the Social Security Act 1998 or a supersession under section 10 of that Act of”; and

(ii) in the second place in which they occur there shall be substituted the words “a revision or a supersession of”.
\end{enumerate}
\end{enumerate}

\medskip

3.  In regulation 3\footnote{\frenchspacing Regulation 3 was amended by S.I. 1989/136, 1994/2943, 1996/1460 and 1999/2556.} (claims not required for entitlement to benefit in certain cases), for paragraph ($g$)  there shall be substituted the following paragraph---
\begin{quotation}
“($g$) in the case of a jobseeker’s allowance where–
\begin{enumerate}\item[]
(i) payment of benefit has been suspended in the circumstance prescribed in regulation 16(2) of the Social Security and Child Support (Decisions and Appeals) Regulations 1999; and

(ii) the claimant whose benefit has been suspended satisfies the conditions of entitlement (apart from the requirement to claim) to that benefit immediately before the suspension ends;”.
\end{enumerate}
\end{quotation}

\medskip

4.  In regulation 13\footnote{\frenchspacing Regulation 13 was amended by S.I. 1991/2741 and 2284, 1992/247 and 1994/2319.} (advance claims and awards)---
\begin{enumerate}\item[]
($a$) in paragraph (1), for the words “that authority” there shall be substituted the words “the Secretary of State”; and

($b$) in paragraph (2), for the words “An award under paragraph (1)($b$)  shall be reviewed by the adjudicating authority” there shall be substituted the words “A decision pursuant to paragraph (1)($b$)  to award benefit may be revised under section 9 of the Social Security Act 1998”.
\end{enumerate}

\medskip

5.  In regulation 13A\footnote{\frenchspacing Regulation 13A was inserted by S.I. 1991/2741.} (advance award of disability living allowance)---
\begin{enumerate}\item[]
($a$) in paragraph (1), for the words “that authority” there shall be substituted the words “the Secretary of State”; and

($b$) in paragraph (3), for the words “An award under paragraph (1) or (2) shall be reviewed by the adjudicating authority” there shall be substituted the words “A decision pursuant to paragraph (1) or (2) to award benefit may be revised under section 9 of the Social Security Act 1998”.
\end{enumerate}

\medskip

6.  In regulation 13C(3)\footnote{\frenchspacing Regulation 13C was inserted by S.I. 1991/2741.} (further claim for and award of disability living allowance), for the words “An award under paragraph (2)($b$)  shall be reviewed by the adjudicating authority” there shall be substituted the words “A decision pursuant to paragraph (2)($b$)  to award benefit may be revised under section 9 of the Social Security Act 1998”.

\medskip

7.  In regulation 17(4) (duration of awards), the words from “; and where those” to the end of that paragraph shall be omitted.

\medskip

8.  In regulation 26\footnote{\frenchspacing Regulation 26 was amended by S.I. 1988/522, 1989/136 and 1993/1113.} (income support)---
\begin{enumerate}\item[]
($a$) in paragraph (1), for the words “the day when any change of circumstances affecting entitlement is to have” there shall be substituted the words “the date from which a superseding decision on the ground of a relevant change of circumstances has”; and

($b$) in paragraphs (2) and (3), the words “on review” shall be omitted.
\end{enumerate}

\medskip

9.  In regulation 26A\footnote{\frenchspacing Regulation 26A was inserted by S.I. 1996/1460 and amended by S.I. 1998/1174.} (jobseeker’s allowance)–
\begin{enumerate}\item[]
($a$) in paragraph (4)–
\begin{enumerate}\item[]
(i) for the words “an award of jobseeker’s allowance is revised” there shall be substituted the words “a decision in respect of a claim for jobseeker’s allowance is superseded”; and

(ii) for the words “revised award” there shall be substituted the word “supersession”;
\end{enumerate}

($b$) in paragraph (5), for the words “revised award” in both places in which they occur there shall be substituted the word “supersession”;

($c$) in paragraph (6)–
\begin{enumerate}\item[]
(i) for the words “revised award” there shall be substituted the word “supersession”; and

(ii) for the words “the award is again revised, the award, as again revised,” there shall be substituted the words “a further superseding decision is made, that further superseding decision”;
\end{enumerate}

($d$) in paragraph (7), for the words “revised award, that revised award” there shall be substituted the words “supersession, that supersession”; and

($e$) in paragraph (8), for the words “will be impracticable to give effect to that revised award in accordance with the other provisions of this regulation, the revised award” there shall be substituted the words “is impracticable for a supersession to have effect in accordance with the other provisions of this regulation, the supersession”.
\end{enumerate}

\medskip

10.  In regulation 31(4) (time and manner of payments of industrial injuries gratuities), the words “but any such decision may be varied by the adjudicating authority by whom the award of that gratuity is varied” shall be omitted.

\medskip

11.  In the heading to Part V (suspension and extinguishment), the words “Suspension and” shall be omitted.

\medskip

12.  Regulations 37 to 37B\footnote{\frenchspacing Regulations 37, 37A and 37B were substituted for regulation 37 by S.I. 1992/247. Regulation 37 was amended by S.I. 1993/2113, 1996/1460 and 2306. Regulation 37A was substituted by S.I. 1998/1381. Regulations 37AA and 37AB were inserted by S.I. 1994/2319. Regulation 37AA was amended by S.I. 1996/2306 and 1460.} (suspension, withholding and payment of withheld benefit), are hereby revoked.

\medskip

13.  In regulation 38(2A)\footnote{\frenchspacing Regulation 38(2A) was inserted by S.I. 1989/1686 and amended by S.I. 1993/2113.} (extinguishment of right to payment of sums by way of benefit where payment is not obtained within the prescribed period)–
\begin{enumerate}\item[]
($a$) in sub-paragraph ($a$), for the words “the Secretary of State has” there shall be substituted the word “he”; and

($b$) in sub-paragraph ($c$)–
\begin{enumerate}\item[]
(i) the words “the Secretary of State has certified” shall be omitted;

(ii) the word “that” in each place in which it occurs shall be omitted; and

(iii) in head (ii), for the word “him” there shall be substituted the words “the Secretary of State”.
\end{enumerate}
\end{enumerate}

\medskip

14.  In paragraph 1(2)($a$)\footnote{\frenchspacing Paragraph 1 was amended by S.I. 1996/1460.} of Schedule 2 (special provisions relating to claims for jobseeker’s allowance during periods connected with public holidays), for the words “an adjudication officer” there shall be substituted the words “the Secretary of State”.

\medskip

15.  In Schedule 7 (manner and time of payment, effective date of change of circumstances and commencement of entitlement in income support cases)–
\begin{enumerate}\item[]
($a$) in the heading, for the words “change of circumstances” there shall be substituted the words “superseding decision”;

($b$) for the heading to paragraph 7 (date when change of circumstances is to take effect) there shall be substituted the heading “Date from which superseding decision on ground of change of circumstances takes effect”; and

($c$) in paragraph 7\footnote{\frenchspacing Paragraph 7 was substituted by S.I. 1990/2208 and amended by S.I. 1991/387, 1992/247 and 1998/1174.}–
\begin{enumerate}\item[]
(i) in sub-paragraph (1), for the words “changed because of a change of circumstances that change of circumstances” there shall be substituted the words “changed by a superseding decision made on the ground of a change of circumstances that superseding decision”;

(ii) in sub-paragraph (2), for the words “the decision given on review” there shall be substituted the words “the superseding decision”; and

(iii) for sub-paragraphs (4) to (6) there shall be substituted the following sub-paragraphs–
\begin{quotation}
“(4) A superseding decision made in consequence of a payment of income being treated as paid on a particular day under regulation 31(1)($b$)  or (2) or 39C(3) of the Income Support Regulations (date on which income is treated as paid) shall have effect on the day on which that payment is treated as paid.

(5) Where–
\begin{enumerate}\item[]
($a$) it is decided upon supersession on the ground of a relevant change of circumstances that the amount of income support is, or is to be, reduced; and

($b$) the Secretary of State certifies that it is impracticable for a superseding decision to have effect from the day prescribed in the preceding sub-paragraphs (other than where sub-paragraph (3)($f$)  or (4) applies),
\end{enumerate}
that superseding decision shall have effect–
\begin{enumerate}\item[]
(i) where the relevant change has occurred, from the first day of the benefit week following that in which that superseding decision is made; or

(ii) where the relevant change is expected to occur, from the first day of the benefit week following that in which that change of circumstances is expected to occur.
\end{enumerate}

(6) Where–
\begin{enumerate}\item[]
($a$) a superseding decision (“the former supersession”) was made on the ground of a relevant change of circumstances in the cases set out in sub-paragraph (3)($b$)  to ($f$); and

($b$) that superseding decision is itself superseded by a subsequent decision because the circumstances which gave rise to the former supersession cease to apply (“the second change”),
\end{enumerate}
that subsequent decision shall have effect from the date of the second change.”.
\end{quotation}
\end{enumerate}
\end{enumerate}

\medskip

16.  In Schedule 9 (deductions from benefit and direct payment to third parties)–
\begin{enumerate}\item[]
($a$) in paragraph 3(1)\footnote{\frenchspacing Paragraph 3(1) was amended by S.I. 1992/1026.}, for the word “its” there shall be substituted the word “his”;

($b$) in paragraph 6(4)\footnote{\frenchspacing Paragraph 6(4) was amended by S.I. 1992/2595.}, for the words “that determination falls to be reviewed” there shall be substituted the words “a decision which embodies that determination falls to be superseded”; and

($c$) in paragraph 7(2)\footnote{\frenchspacing Paragraph 7(2) was amended by S.I. 1992/2595.}, for the words “the authority” there shall be substituted the words “the Secretary of State”.
\end{enumerate}

\medskip

17.  In paragraph 3(1)\footnote{\frenchspacing Schedule 9A was inserted by S.I. 1992/1026 and paragraph 3(1) was amended by S.I. 1995/1613 and 1996/1460.} of Schedule 9A (deductions of mortgage interest from benefit and payment to qualifying lenders), for the words “adjudicating authority in accordance with regulation 34A, shall be paid by the Secretary of State” there shall be substituted the words “Secretary of State in accordance with regulation 34A, shall be paid”.

\part[Schedule 7 --- Amendment of the Housing Benefit (General) Regulations 1987 and the Council Tax Benefit (General) Regulations 1992]{Schedule 7\\*Amendment of the Housing Benefit (General) Regulations 1987 and the Council Tax Benefit (General) Regulations 1992}

\renewcommand\parthead{--- Schedule 7}

1.  In regulation 2(1)\footnote{\frenchspacing The definition of “adjudication officer” was inserted by S.I. 1995/626.\label{fn:82}} (interpretation) of the Housing Benefit (General) Regulations 1987 and regulation 2(1)\footref{fn:82} (interpretation) of the Council Tax Benefit (General) Regulations 1992 the definition of “adjudication officer” shall be omitted.

\medskip

2.  In regulation 95(4A)\footnote{\frenchspacing Paragraph (4A) was inserted by S.I. 1995/560 and amended by S.I. 1996/1510.} (withholding of benefit) of the Housing Benefit (General) Regulations 1987 and regulation 80(2A)\footnote{\frenchspacing Paragraph (2A) was inserted by S.I. 1995/560 and amended by S.I. 1996/1510.} (withholding of benefit) of the Council Tax Benefit (General) Regulations 1992 for the words “an adjudication officer” in each place in which they occur there shall be substituted the words “the Secretary of State”.

\part[Schedule 8 --- Amendment of the Social Fund (Application for Review) Regulations 1988]{Schedule 8\\*Amendment of the Social Fund (Application for Review) Regulations 1988}

\renewcommand\parthead{--- Schedule 8}

In regulation 2\footnote{\frenchspacing Regulation 2 was amended by S.I. 1988/1843 and 1990/580.} (manner of making application for review or further review and time limits), for the words–
\begin{enumerate}\item[]
($a$) “a social fund officer” in each place in which they occur there shall be substituted the words “an appropriate officer”; and

($b$) “the social fund officer” in each place in which they occur there shall be substituted the words “the appropriate officer”.
\end{enumerate}

\part[Schedule 9 --- Amendment of the Social Security (Payments on Account, Overpamyents and Recovery) Regulations 1988]{Schedule 9\\*Amendment of the Social Security (Payments on Account, Overpamyents and Recovery) Regulations 1988}

\renewcommand\parthead{--- Schedule 9}

1.  In regulation 1(2)\footnote{\frenchspacing Regulation 1(2) was amended by S.I. 1988/1725, 1989/136, 1991/2742, 1995/829 and 1996/1345.} (interpretation), for the definition of “adjudicating authority” there shall be substituted the following definition–
\begin{quotation}
    ““adjudicating authority” means, as the case may require, the Secretary of State, an appeal tribunal constituted under Chapter I of Part I of the Social Security Act 1998 or a Commissioner within the meaning of section 39(1) of that Act;”. 
\end{quotation}

\medskip

2.  In regulation 2(1)\footnote{\frenchspacing Regulation 2(1) was amended by S.I. 1996/30.} (making of interim payments), for the words “a reference, review,” there shall be substituted the word “an”.

\medskip

3.  In regulation 5(2) (offsetting prior payment against subsequent award)–
\begin{enumerate}\item[]
($a$) for Case 1 there shall be substituted the following case–
\begin{quotation}
\subsubsection*{“Case 1: Payment pursuant to a decision which is revised or superseded, or overturned on appeal}

Where a person has been paid a sum by way of benefit pursuant to a decision which is subsequently revised under section 9 of the Social Security Act 1998, superseded by a decision under section 10 of that Act or overturned on appeal”; and
\end{quotation}

($b$) in Case 2, the words “, on review or appeal,” shall be omitted.
\end{enumerate}

\medskip

4.  In regulation 8(2)\footnote{\frenchspacing Regulation 8(2) was amended by S.I. 1996/1345.} (duplication of prescribed payments), for the words “on review” there shall be substituted the words “by way of revision or supersession”.

\medskip

5.  In the heading to Part VI, the words “REVISION OF DETERMINATION AND” shall be omitted.

\medskip

6.  In regulation 12 (circumstances in which determination need not be revised)–
\begin{enumerate}\item[]
($a$) in the heading, for the word “revised” there shall be substituted the words “reversed, varied, revised or superseded”;

($b$) for the words “or revision of determination” there shall be substituted the words “, revision or supersession”; and

($c$) for the words “for reviewing and revising the determination under which payment was made” there shall be substituted the words “for the decision pursuant to which the payment was made to be revised under section 9 of the Social Security Act 1998 or superseded under section 10 of that Act”.
\end{enumerate}

\medskip

7.  In regulation 23 (increase of amount of award on appeal or review)–
\begin{enumerate}\item[]
($a$) in the heading, for the word “review” there shall be substituted the word “otherwise”;

($b$) for the words “on review by an adjudicating authority” there shall be substituted the word “otherwise”; and

($c$) for the words “were the earnings subsequently reviewed under regulation 24” there shall be substituted the words “, where a notice of variation of protected earnings is given under regulation 24, were the earnings stated in that notice”.
\end{enumerate}

\medskip

8.  In regulation 24\footnote{\frenchspacing Regulation 24 was amended by S.I. 1988/688.} (review of determination of protected earnings)–
\begin{enumerate}\item[]
($a$) for the heading there shall be substituted the heading “Notice of variation of protected earnings”;

($b$) paragraph (1) shall be omitted; and

($c$) for paragraph (2) there shall be substituted the following paragraph–
\begin{quotation}
“(2) The Secretary of State shall give a claimant’s employer written notice varying the deduction notice where a decision as to a claimant’s protected earnings is revised or superseded.”.
\end{quotation}
\end{enumerate}

\medskip

9.  In regulation 25(2)($b$)  (power to serve further deduction notice on resumption of employment), for the words “reviewed under regulation 24” there shall be substituted the word “varied”.

\medskip

10.  In regulation 26 (right of Secretary of State to recover direct from claimant), for the words from “the Secretary of State has” to “for him” there shall be substituted the words “, at any time, it is not practicable for the Secretary of State”.

\part[Schedule 10 --- Amendment of the Community Charges (Deductions from Income Support) (Scotland) Regulations 1989]{Schedule 10\\*Amendment of the Community Charges (Deductions from Income Support) (Scotland) Regulations 1989}

\renewcommand\parthead{--- Schedule 10}

1.  In regulation 1(2)\footnote{\frenchspacing Regulation 1 was amended by S.I. 1990/113 and 1996/2344.} (interpretation)–
\begin{enumerate}\item[]
($a$) the definitions of “the 1975 Act” and “adjudication officer” shall be omitted;

($b$) after the definition of “the 1986 Act” there shall be inserted the following definition–
\begin{quotation}
““the 1998 Act” means the Social Security Act 1998;”;
\end{quotation}

($c$) for the definition of “Commissioner” there shall be substituted the following definition–
\begin{quotation}
    ““Commissioner” has the meaning it bears in section 39(1) of the 1998 Act;”; and 
\end{quotation}

($d$) for the definition of “tribunal” there shall be substituted the following definition–
\begin{quotation}
““tribunal” means an appeal tribunal constituted under Chapter I of Part I of the 1998 Act;”.
\end{quotation}
\end{enumerate}

\medskip

2.---(1)  In the heading to regulation 2\footnote{\frenchspacing Regulation 2 was amended by S.I. 1990/113, 1992/1026 and 1996/2344.} (deductions from income support or jobseeker’s allowance) for the word “Deductions” there shall be substituted the words “Application for deductions”.

(2) Paragraphs (4) and (5) of regulation 2 shall be omitted.

\medskip

3.  For regulations 2A\footnote{\frenchspacing Regulation 2A was inserted by S.I. 1996/2344.} (deductions from debtor’s jobseeker’s allowance) and 3 (notification of decision) there shall be substituted the following regulation–
\begin{quotation}
\subsection*{“Deductions from debtor’s income support or jobseeker’s allowance}

3.---(1)  Subject to paragraph (4) and regulation 4, where the Secretary of State receives an application from a levying authority in respect of a debtor who is entitled to income support or income-based jobseeker’s allowance and the amount payable by way of that benefit, after any deduction under this paragraph, is 10 pence or more, the Secretary of State may deduct a sum from that benefit which is equal to 5 per cent.\ of the personal allowance–
\begin{enumerate}\item[]
($a$) for a couple where–
\begin{enumerate}\item[]
(i) a summary warrant or decree is made; and

(ii) that benefit is payable, in respect of both members of a couple both of whom are aged not less than 18; and
\end{enumerate}

($b$) in any other case, for a single claimant aged not less than 25,
\end{enumerate}
and pay that sum to the levying authority towards satisfaction of any outstanding sum which is or forms part of the amount in respect of which the summary warrant or decree was granted.

(2) Subject to paragraph (3) and regulation 4, where–
\begin{enumerate}\item[]
($a$) the Secretary of State receives an application from a levying authority in respect of a debtor who is entitled to contribution-based jobseeker’s allowance; and

($b$) the amount of contribution-based jobseeker’s allowance payable before any deduction under this paragraph is equal to or more than one-third of the age-related amount applicable to the debtor under section 4(1)($a$)  of the Jobseekers Act,
\end{enumerate}
the Secretary of State may deduct a sum from that benefit which is equal to one-third of the age-related amount applicable to the debtor under section 4(1)($a$)  of the Jobseekers Act and pay that sum to the levying authority towards satisfaction of any outstanding sum which is or forms part of the amount in respect of which the summary warrant or decree was granted.

(3) Where the sum that would otherwise fall to be deducted under paragraph (2) includes a fraction of a penny, the sum to be deducted shall be rounded down to the next whole penny.

(4) Before making a deduction under paragraph (1) the Secretary of State shall make any deduction which falls to be made in respect of a liability mentioned in any of the following provisions of the Social Security (Claims and Payments) Regulations 1987–
\begin{enumerate}\item[]
($a$) regulation 34A\footnote{\frenchspacing Regulation 34A was inserted by S.I. 1992/1026.} (mortgage interest);

($b$) paragraph 3\footnote{\frenchspacing Paragraph 3 was amended by S.I. 1992/1026 and 2595, 1995/1613 and 2927, and 1996/1460.} (housing costs) of Schedule 9;

($c$) paragraph 5\footnote{\frenchspacing Paragraph 5 was amended by S.I. 1991/2284, 1992/2595 and 1996/1460.} (rent and certain service charges for fuel) of Schedule 9; and

($d$) paragraph 6\footnote{\frenchspacing Paragraph 6 was amended by S.I. 1991/2284, 1992/2595, 1994/2319 and 1996/1460.} (fuel costs) of Schedule 9.
\end{enumerate}

(5) Subject to regulations 5 and 6, a decision of the Secretary of State under this regulation shall be final.

(6) The Secretary of State shall notify the debtor in writing of a decision to make a deduction under this regulation as soon as is practicable and at the same time shall notify the debtor of his right of appeal.”.
\end{quotation}

\medskip

4.  For paragraph (1)\footnote{\frenchspacing Regulation 4(1) was amended by S.I. 1990/113, 1993/2113 and 1996/2344.} of regulation 4 (circumstances, time of making and termination of deductions) there shall be substituted the following paragraph–
\begin{quotation}
“(1) The Secretary of State–
\begin{enumerate}\item[]
($a$) shall make deductions under regulation 3 only where the debtor is entitled to income support or jobseeker’s allowance throughout any benefit week; and

($b$) shall not determine any application under regulation 2 which relates to a debtor in respect of whom–
\begin{enumerate}\item[]
(i) he is making deductions; or

(ii) deductions fall to be made,
\end{enumerate}
pursuant to an earlier application under regulation 3 until no deductions pursuant to that earlier application fall to be made.”.
\end{enumerate}
\end{quotation}

\medskip

5.  For regulations 5 (appeal) and 6 (review) there shall be substituted the following regulations–
\begin{quotation}
\subsection*{“Revision and supersession}

5.  Any decision of the Secretary of State under regulation 3 may be revised under section 9 of the 1998 Act or superseded under section 10 of that Act as though the decision were made under section 8(1)($c$)  of that Act.

\subsection*{Appeal}

6.  Any decision of the Secretary of State under regulation 3 (whether as originally made or as revised under regulation 5) may be appealed to a tribunal as though the decision were made on an award of a relevant benefit (within the meaning of section 8(3) of the 1998 Act) under section 8(1)($c$)  of the 1998 Act.”.
\end{quotation}

\medskip

6.  Regulations 7 to 11 and Schedules 1 and 2 are hereby revoked.

\part[Schedule 11 --- Amendment of the Community Charges (Deductions from Income Support) (No.\ 2) Regulations 1990]{Schedule 11\\*Amendment of the Community Charges (Deductions from Income Support) (No.\ 2) Regulations 1990}

\renewcommand\parthead{--- Schedule 11}

1.  In regulation 1(2)\footnote{\frenchspacing Regulation 1 was amended by S.I. 1996/2344.} (interpretation)–
\begin{enumerate}\item[]
($a$) the definitions of “the 1975 Act” and “adjudication officer” shall be omitted;

($b$) after the definition of “the 1986 Act” there shall be inserted the following definition–
\begin{quotation}
““the 1998 Act” means the Social Security Act 1998;”;
\end{quotation}

($c$) for the definition of “Commissioner” there shall be substituted the following definition–
\begin{quotation}
    ““Commissioner” has the meaning it bears in section 39(1) of the 1998 Act;”; and 
\end{quotation}

($d$) for the definition of “tribunal” there shall be substituted the following definition–
\begin{quotation}
““tribunal” means an appeal tribunal constituted under Chapter I of Part I of the 1998 Act;”.
\end{quotation}
\end{enumerate}

\medskip

2.---(1)  In the heading to regulation 2\footnote{\frenchspacing Regulation 2 was amended by S.I. 1992/1026, 1193/2113 and 1996/2344.} (deductions from income support or jobseeker’s allowance) for the word “Deductions” there shall be substituted the words “Application for deductions”.

(2) Paragraphs (4) to (6) of regulation 2 shall be omitted.

\medskip

3.  For regulations 2A\footnote{\frenchspacing Regulation 2A was inserted by S.I. 1996/2344.} (deductions from debtor’s jobseeker’s allowance) and 3 (notification of decision) there shall be substituted the following regulation–
\begin{quotation}
\subsection*{“Deductions from debtor’s income support or jobseeker’s allowance}

3.---(1)  Subject to paragraph (4) and regulation 4, where the Secretary of State receives an application from an authority in respect of a debtor who is entitled to income support or income-based jobseeker’s allowance and the amount payable by way of that benefit, after any deduction under this paragraph, is 10 pence or more, the Secretary of State may deduct a sum from that benefit which is equal to 5 per cent.\ of the personal allowance–
\begin{enumerate}\item[]
($a$) set out in paragraph 1(1)($e$)  of Schedule 2 to the Income Support (General) Regulations 1987 or, as the case may be, of Schedule 1 to the Jobseeker’s Allowance Regulations 1996 for a couple where–
\begin{enumerate}\item[]
(i) a liability order is made; and

(ii) that benefit is payable, in respect of both members of a couple both of whom are aged not less than 18; and
\end{enumerate}

($b$) in any other case, for a single claimant aged not less than 25 set out in paragraph 1(3)($c$)  of Schedule 2 to the Income Support (General) Regulations 1987 or, as the case may be, paragraph 1(3)($e$)  of Schedule 1 to the Jobseeker’s Allowance Regulations 1996,
\end{enumerate}
and pay that sum to the authority towards satisfaction of any outstanding sum which is or forms part of the amount in respect of which the liability order was made.

(2) Subject to paragraph (3) and regulation 4, where–
\begin{enumerate}\item[]
($a$) the Secretary of State receives an application from an authority in respect of a debtor who is entitled to contribution-based jobseeker’s allowance; and

($b$) the amount of contribution-based jobseeker’s allowance payable before any deduction under this paragraph is equal to or more than one-third of the age-related amount applicable to the debtor under section 4(1)($a$)  of the Jobseekers Act,
\end{enumerate}
the Secretary of State may deduct a sum from that benefit which is equal to one-third of the age-related amount applicable to the debtor under section 4(1)($a$)  of the Jobseekers Act and pay that sum to the authority towards satisfaction of any outstanding sum which is or forms part of the amount in respect of which the liability order was made.

(3) Where the sum that would otherwise fall to be deducted under paragraph (2) includes a fraction of a penny, the sum to be deducted shall be rounded down to the next whole penny.

(4) Before making a deduction under paragraph (1) the Secretary of State shall make any deduction which falls to be made in respect of a liability mentioned in any of the following provisions of the Social Security (Claims and Payments) Regulations 1987–
\begin{enumerate}\item[]
($a$) regulation 34A\footnote{\frenchspacing Regulation 34A was inserted by S.I. 1992/1026.} (mortgage interest);

($b$) paragraph 3\footnote{\frenchspacing Paragraph 3 was amended by S.I. 1992/1026 and 2595, 1995/1613 and 2927 and 1996/1460.} (housing costs) of Schedule 9;

($c$) paragraph 5\footnote{\frenchspacing Paragraph 5 was amended by S.I. 1991/2284, 1992/2595 and 1996/1460.} (rent and certain service charges for fuel) of Schedule 9;

($d$) paragraph 6\footnote{\frenchspacing Paragraph 6 was amended by S.I. 1991/2284, 1992/2595, 1994/2319 and 1996/1460.} (fuel costs) of Schedule 9; and

($e$) paragraph 7\footnote{\frenchspacing Paragraph 7 was amended by S.I. 1992/2595, 1993/478, 1994/2319 and 1996/1460.} (water charges) of Schedule 9.
\end{enumerate}

(5) Subject to regulations 5 and 6, a decision of the Secretary of State under this regulation shall be final.

(6) The Secretary of State shall notify the debtor in writing of a decision to make a deduction under this regulation as soon as is practicable and at the same time shall notify the debtor of his right of appeal.”.
\end{quotation}

\medskip

4.  For paragraph (1)\footnote{\frenchspacing Regulation 4(1) was amended by S.I. 1996/2344.} of regulation 4 (circumstances, time of making and termination of deductions) there shall be substituted the following paragraph–
\begin{quotation}
“(1) The Secretary of State–
\begin{enumerate}\item[]
($a$) shall make deductions under regulation 3 only where the debtor is entitled to income support or jobseeker’s allowance throughout any benefit week; and

($b$) shall not determine any application under regulation 2 which relates to a debtor in respect of whom–
\begin{enumerate}\item[]
(i) he is making deductions; or

(ii) deductions fall to be made,
\end{enumerate}
pursuant to an earlier application under regulation 3 until no deductions pursuant to that earlier application fall to be made.”.
\end{enumerate}
\end{quotation}

\medskip

5.  For regulations 5 (appeal) and 6 (review) there shall be substituted the following regulations–
\begin{quotation}
\subsection*{“Revision and supersession}

5.  Any decision of the Secretary of State under regulation 3 may be revised under section 9 of the 1998 Act or superseded under section 10 of that Act as though the decision were made under section 8(1)($c$)  of that Act.

\subsection*{Appeal}

6.  Any decision of the Secretary of State under regulation 3 (whether as originally made or as revised under regulation 5) may be appealed to a tribunal as though the decision were made on an award of a relevant benefit (within the meaning of section 8(3) of the 1998 Act) under section 8(1)($c$)  of the 1998 Act.”.
\end{quotation}

\medskip

6.  Regulations 7 to 11 and Schedules 1 and 2 are hereby revoked.

\part[Schedule 12 --- Amendment of the Fines (Deductions from Income Support) Regulations 1992]{Schedule 12\\*Amendment of the Fines (Deductions from Income Support) Regulations 1992}

\renewcommand\parthead{--- Schedule 12}

1.  In regulation 1(2)\footnote{\frenchspacing There are amendments to regulation 1 which are not relevant to this Order.} (interpretation)–
\begin{enumerate}\item[]
($a$) for the definition of “the 1992 Act” there shall be substituted the following definition–
\begin{quotation}
““the 1998 Act” means the Social Security Act 1998;”;
\end{quotation}

($b$) the definitions of “adjudication officer” and “appropriate appeal court” shall be omitted;

($c$) for the definition of “Commissioner” there shall be substituted the following definition–
\begin{quotation}
    ““Commissioner” has the meaning it bears in section 39(1) of the 1998 Act;”; and 
\end{quotation}

($d$) for the definition of “tribunal” there shall be substituted the following definition–
\begin{quotation}
““tribunal” means an appeal tribunal constituted under Chapter I of Part I of the 1998 Act.”.
\end{quotation}
\end{enumerate}

\medskip

2.  For regulations 4 to 6A\footnote{\frenchspacing Regulation 4 was substituted by S.I. 1993/495 and amended by S.I. 1996/2344 and 1997/827. Regulation 6 was amended, and regulation 6A was inserted, by S.I. 1996/2344.} (reference to adjudication officer, notification of decision, and deductions from offender’s income support or jobseeker’s allowance) there shall be substituted the following regulation–
\begin{quotation}
\subsection*{“Deductions from offender’s income support or jobseeker’s allowance}

4.---(1)  Subject to regulation 7, where–
\begin{enumerate}\item[]
($a$) the Secretary of State receives an application from a court in respect of an offender who is entitled to income support or income-based jobseeker’s allowance;

($b$) the amount payable by way of that benefit, after any deduction under this paragraph, is 10 pence or more; and

($c$) the aggregate amount payable under one or more of the following provisions, namely, paragraphs 3(2)($a$), 5(6), 6(2)($a$)  and 7(3)($a$)  and (5)($a$)  of Schedule 9 to the Claims and Payments Regulations, and regulation 2 of the Council Tax (Deductions from Income Support) Regulations 1993, together with the amount to be deducted under this paragraph does not exceed an amount equal to 3 times 5 per cent.\ of the personal allowance for a single claimant aged not less than 25 years,
\end{enumerate}
the Secretary of State may deduct a sum from that benefit which is equal to 5 per cent.\ of the personal allowance for a single claimant aged not less than 25 and pay that sum to the court towards satisfaction of the fine or the sum required to be paid by compensation order.

(2) Subject to paragraphs (3) and (4) and regulation 7, where–
\begin{enumerate}\item[]
($a$) the Secretary of State receives an application from a court in respect of an offender who is entitled to contribution-based jobseeker’s allowance; and

($b$) the amount of contribution-based jobseeker’s allowance payable before any deduction under this paragraph is equal to or more than one-third of the age-related amount applicable to the offender under section 4(1)($a$)  of the Jobseekers Act,
\end{enumerate}
the Secretary of State may deduct a sum from that benefit which is equal to one-third of the age-related amount applicable to the offender under section 4(1)($a$)  of the Jobseekers Act and pay that sum to the court towards satisfaction of the fine or the sum required to be paid by compensation order.

(3) No deduction shall be made under paragraph (2) where a deduction is being made from the offender’s contribution-based jobseeker’s allowance under the Community Charges (Deductions from Income Support) (No.\ 2) Regulations 1990, the Community Charges (Deductions from Income Support) (Scotland) Regulations 1989 or the Council Tax (Deductions from Income Support) Regulations 1993.

(4) Where the sum that would otherwise fall to be deducted under paragraph (2) includes a fraction of a penny, the sum to be deducted shall be rounded down to the next whole penny.

(5) The Secretary of State shall notify the offender and the court in writing of a decision to make a deduction under this regulation so far as is practicable within 14 days from the date on which he made the decision and at the same time shall notify the offender of his right of appeal.”.
\end{quotation}

\medskip

3.  In regulation 7\footnote{\frenchspacing Regulation 7 was amended by S.I. 1996/2344.} (circumstances, time of making and termination of deductions)–
\begin{enumerate}\item[]
($a$) in paragraph (1), for the words “income support or jobseeker’s allowance under regulation 6 or 6A” there shall be substituted the words “under regulation 4”; and

($b$) for paragraph (5) there shall be substituted the following paragraph–
\begin{quotation}
“(5) The Secretary of State shall not determine any application under regulation 2 which relates to an offender in respect of whom–
\begin{enumerate}\item[]
($a$) he is making deductions; or

($b$) deductions fall to be made,
\end{enumerate}
pursuant to an earlier application under that regulation until no deductions pursuant to that earlier application fall to be made.”.
\end{quotation}
\end{enumerate}

\medskip

4.  For regulations 9 (appeal) and 10 (review) there shall be substituted the following regulations–
\begin{quotation}
\subsection*{“Revision and supersession}

9.  Any decision of the Secretary of State under regulation 4 may be revised under section 9 of the 1998 Act or superseded under section 10 of that Act as though the decision were made under section 8(1)($c$)  of that Act.

\subsection*{Appeal}

10.  Any decision of the Secretary of State under regulation 4 (whether as originally made or as revised under regulation 9) may be appealed to a tribunal as though the decision were made on an award of a relevant benefit (within the meaning of section 8(3) of the 1998 Act) under section 8(1)($c$)  of the 1998 Act.”.
\end{quotation}

\medskip

5.  Regulations 11 to 15 and Schedules 1 and 2 are hereby revoked.

\part[Schedule 13 --- Amendment of the Council Tax (Deductions from Income Support) Regulations 1993]{Schedule 13\\*Amendment of the Council Tax (Deductions from Income Support) Regulations 1993}

\renewcommand\parthead{--- Schedule 13}

1.  In regulation 1(2)\footnote{\frenchspacing There are amendments to regulation 1 which are not relevant to this Order.} (interpretation)–
\begin{enumerate}\item[]
($a$) for the definition of “the Administration Act” there shall be substituted the following definition–
\begin{quotation}
““the 1998 Act” means the Social Security Act 1998;”;
\end{quotation}

($b$) the definitions of “adjudication officer” and “appropriate appeal court” shall be omitted;

($c$) for the definition of “Commissioner”there shall be substituted the following definition–
\begin{quotation}
    ““Commissioner” has the meaning it bears in section 39(1) of the 1998 Act;”; and 
\end{quotation}

($d$) for the definition of “tribunal” there shall be substituted the following definition–
\begin{quotation}
““tribunal” means an appeal tribunal constituted under Chapter I of Part I of the 1998 Act.”.
\end{quotation}
\end{enumerate}

\medskip

2.  For regulations 5 to 7A\footnote{\frenchspacing Regulation 5 was amended by S.I. 1996/2344 and 1997/827. Regulation 7 was amended, and regulation 7A was inserted, by S.I. 1996/2344.} (reference to adjudication officer, notification of decision, and deductions from debtor’s income support or jobseeker’s allowance) there shall be substituted the following regulation–
\begin{quotation}
\subsection*{“Deductions from debtor’s income support or jobseeker’s allowance}

5.---(1)  Subject to regulation 8, where–
\begin{enumerate}\item[]
($a$) the Secretary of State receives an application from an authority in respect of a debtor who is entitled to income support or income-based jobseeker’s allowance;

($b$) the amount payable by way of that benefit, after any deduction under this paragraph, is 10 pence or more; and

($c$) the aggregate amount payable under one or more of the following provisions, namely, paragraphs 3(2)($a$), 5(6), 6(2)($a$)  and 7(3)($a$)  and (5)($a$)  of Schedule 9 to the Claims and Payments Regulations together with the amount to be deducted under this paragraph does not exceed an amount equal to 3 times 5 per cent.\ of the personal allowance for a single claimant aged not less than 25 years,
\end{enumerate}
the Secretary of State may deduct a sum from that benefit which is equal to 5 per cent.\ of the personal allowance for a single claimant aged not less than 25 and pay that sum to the authority towards satisfaction of any outstanding sum which is or forms part of the amount in respect of which the liability order was made or the summary warrant or the decree was granted.

(2) Subject to paragraph (3) and regulation 8, where–
\begin{enumerate}\item[]
($a$) the Secretary of State receives an application from an authority in respect of a debtor who is entitled to contribution-based jobseeker’s allowance; and

($b$) the amount of contribution-based jobseeker’s allowance payable before any deduction under this paragraph is equal to or more than one-third of the age-related amount applicable to the debtor under section 4(1)($a$)  of the Jobseekers Act,
\end{enumerate}
the Secretary of State may deduct a sum from that benefit which is equal to one-third of the age-related amount applicable to the debtor under section 4(1)($a$)  of the Jobseekers Act and pay that sum to the authority towards satisfaction of any outstanding sum which is or forms part of the amount in respect of which the liability order was made or the summary warrant or the decree was granted.

(3) Where the sum that would otherwise fall to be deducted under paragraph (2) includes a fraction of a penny, the sum to be deducted shall be rounded down to the next whole penny.

(4) The Secretary of State shall notify the debtor and the authority concerned in writing of a decision to make a deduction under this regulation so far as is practicable within 14 days from the date on which he made the decision and at the same time shall notify the debtor of his right of appeal.”.
\end{quotation}

\medskip

3.  For paragraph (4) of regulation 8 (circumstances, time of making and termination of deductions) there shall be substituted the following paragraph–
\begin{quotation}
“(4) The Secretary of State shall not determine any application under regulation 2 or 3 which relates to a debtor in respect of whom–
\begin{enumerate}\item[]
($a$) he is making deductions; or

($b$) deductions fall to be made,
\end{enumerate}
pursuant to an earlier application under either of those regulations until no deductions pursuant to that earlier application fall to be made.”.
\end{quotation}

\medskip

4.  For regulations 10 (appeal) and 11 (review) there shall be substituted the following regulations–
\begin{quotation}
\subsection*{“Revision and supersession}

10.  Any decision of the Secretary of State under regulation 5 may be revised under section 9 of the 1998 Act or superseded under section 10 of that Act as though the decision were made under section 8(1)($c$)  of that Act.

\subsection*{Appeal}

11.  Any decision of the Secretary of State under regulation 5 (whether as originally made or as revised under regulation 10) may be appealed to a tribunal as though the decision were made on an award of a relevant benefit (within the meaning of section 8(3) of the 1998 Act) under section 8(1)($c$)  of the 1998 Act.”.
\end{quotation}

\medskip

5.  Regulations 12 to 16 and Schedules 1 and 2 are hereby revoked.

\part[Schedule 14 --- Amendment of the Employment Protection (Recoupment of Jobseeker's Allowance and Income Support) Regulations 1996]{Schedule 14\\*Amendment of the Employment Protection (Recoupment of Jobseeker's Allowance and Income Support) Regulations 1996}

\renewcommand\parthead{--- Schedule 14}

1.  In the heading to Part IV (determination and review of benefit recouped) the words “and review” shall be omitted.

\medskip

2.  For paragraphs (2) and (3) of regulation 10 (provisions relating to the determination of amount paid by way of or paid as on account of benefit) there shall be substituted the following paragraphs–
\begin{quotation}
“(2) Where an employee has given notice in writing to the Secretary of State under paragraph (1) above that he does not accept that an amount specified in the recoupment notice is correct, the Secretary of State shall make a decision as to the amount of jobseeker’s allowance or, as the case may be, income support paid in respect of the period to which the prescribed element is attributable or, as appropriate, in respect of so much of the protected period as falls before the date on which the employer complies with Regulation 6 above.

(2A) The Secretary of State may revise either upon application made for the purpose or on his own initiative a decision under paragraph (2) above.

(2B) The employee shall have a right of appeal to an appeal tribunal constituted under Chapter I of Part I of the 1998 Act against a decision of the Secretary of State whether as originally made under paragraph (2) or as revised under paragraph (2A) above.

(2C) The Social Security and Child Support (Decisions and Appeals) Regulations 1999 shall apply for the purposes of paragraphs (2A) and (2B) above as though a decision of the Secretary of State under paragraph (2A) above were made under section 9 of the 1998 Act and any appeal from such a decision were made under section 12 of that Act.

(2D) In this Regulation “the 1998 Act” means the Social Security Act 1998.

(3) Where the Secretary of State recovers too much money from an employer under these Regulations the Secretary of State shall pay to the employee an amount equal to the excess.”.
\end{quotation}

\part[Schedule 15 --- Amendment of the Social Security (Back to Work Bonus) (No.\ 2) Regulations 1996]{Schedule 15\\*Amendment of the Social Security (Back to Work Bonus) (No.\ 2) Regulations 1996}

\renewcommand\parthead{--- Schedule 15}

1.  In regulations 5(4)($c$)(ii)  and (6)\footnote{\frenchspacing Regulation 5 was amended by S.I. 1997/454.}, 8(5)($b$)  and 25(1), for the words “adjudication officer” there shall be substituted the words “Secretary of State”.

\medskip

2.  In regulation 8(4)($c$)  and ($d$), for the word “review” there shall be substituted the words “revision or supersession”.

\medskip

3.  In regulation 9(2), the words “upon the adjudication officer” shall be omitted.

\part[Schedule 16 --- Amendment of the Social Security Benefit (Computation of Earnings) Regulations 1996]{Schedule 16\\*Amendment of the Social Security Benefit (Computation of Earnings) Regulations 1996}

\renewcommand\parthead{--- Schedule 16}

1.  Subject to the following provisions of this Schedule, for the words “the adjudicating authority” and “an adjudicating authority” in each place in which they occur there shall be substituted the words “the Secretary of State”.

\medskip

2.  In regulation 2(1) (interpretation), the definition of “adjudicating authority” shall be omitted.

\medskip

3.  In regulation 4 (notional earnings)–
\begin{enumerate}\item[]
($a$) in paragraph (1), for the words “of the determination of the claim or of any subsequent review the adjudicating authority shall treat the claimant” there shall be substituted the words “on which a decision falls to be made by the Secretary of State under Chapter II of Part I of the Social Security Act 1998 or regulations made thereunder the claimant shall be treated”; and

($b$) in paragraph (2), for the words “the adjudicating authority shall treat the claimant” there shall be substituted the words “the claimant shall be treated”.
\end{enumerate}

\medskip

4.  In regulation 6(8) (calculation of earnings of employed earners), in paragraph ($b$)(ii)($bb$)  of the definition of “part-time employment”, for the words “of review” there shall be substituted the words “on which a revision or supersession of a decision falls to be made”.

\medskip

5.  In regulation 13(8) (calculation of net profit of self-employed earners), for the words “The adjudicating authority shall refuse to make a deduction” there shall be substituted the words “A deduction shall not be made”.

\medskip

6.  In regulation 14(2) (deduction of tax and contributions for self-employed earners) for the words “of the determination of the claim or of any subsequent review” in both places in which they occur there shall be substituted the words “on which a decision is made by the Secretary of State under Chapter II of Part I of the Social Security Act 1998 or regulations made thereunder”.

\medskip

7.  Regulation 16 (transitional provision to suspend benefit and make interim payments) is hereby revoked.

\part[Schedule 17 --- Amendment of the Social Security (Recovery of Benefits) Regulations 1997]{Schedule 17\\*Amendment of the Social Security (Recovery of Benefits) Regulations 1997}

\renewcommand\parthead{--- Schedule 17}

For paragraph (3) of regulation 12 (transitional provisions) there shall be substituted the following paragraphs–
\begin{quotation}
“(3) Any appeal under section 98 of the 1992 Act made on or after 6th October 1997 which has not been determined before 29th November 1999 shall be referred to an appeal tribunal constituted in accordance with paragraph (3I) below.

(3A) Any appeal duly made before 6th October 1997 which has not been referred to a medical appeal tribunal or a social security appeal tribunal shall be referred to and determined by an appeal tribunal constituted in accordance with paragraph (3I) below.

(3B) Any appeal duly made before 6th October 1997 and referred to a medical appeal tribunal shall be determined by an appeal tribunal constituted in accordance with paragraph (3I) below which shall determine all issues.

(3C) Any appeal duly made before 6th October 1997 and referred to a social security appeal tribunal shall be determined by an appeal tribunal which shall consist of a legally qualified panel member and in making its determination, the appeal tribunal shall be bound by any decision of a medical appeal tribunal to which a question under section 98(5) of the 1992 Act was referred.

(3D) An appeal tribunal constituted in accordance with paragraph (3I) below shall completely rehear any appeal made under section 98 of the 1992 Act which stands adjourned immediately before 29th November 1999.

(3E) Where a Commissioner holds that the decision of a medical appeal tribunal or a social security appeal tribunal on an appeal made before 6th October 1997 was erroneous in law and refers the case to an appeal tribunal, that appeal tribunal shall be constituted in accordance with paragraph (3I) below and shall determine all issues in accordance with the Commissioner’s direction.

(3F) Regulation 11 of the Social Security (Recoupment) Regulations 1990 (“the 1990 Regulations”) and regulation 12 of those Regulations shall have effect in relation to any appeal under section 98 of the 1992 Act made on or after 6th October 1997 with the modification that for the word “chairman” in each place in which it occurs there were substituted the words “legally qualified panel member”

(3G) Regulation 13 of the 1990 Regulations shall have effect in relation to any appeal under section 98 of the 1992 Act made on or after 6th October 1997.

(3H) Any other transitional question arising from an appeal made under section 98 of the 1992 Act in consequence of the coming into force of the Social Security and Child Support (Decisions and Appeals) Regulations 1999 (“the 1999 Regulations”) shall be determined by a legally qualified panel member who may for this purpose give such directions consistent with these regulations as are necessary.

(3I) For the purposes of paragraphs (3) to (3B) and (3E) above an appeal tribunal shall be constituted under Chapter I of Part I of the Social Security Act 1998 as though the appeal were made under section 11(1)($b$)  of the 1997 Act.

(3J) In this regulation, “legally qualified panel member” has the meaning it bears in regulation 1(3) of the 1999 Regulations.”.
\end{quotation}

\part[Schedule 18 --- Amendment of the Social Fund Winter Fuel Payment Regulations 1998]{Schedule 18\\*Amendment of the Social Fund Winter Fuel Payment Regulations 1998}

\renewcommand\parthead{--- Schedule 18}

1.  In regulation 1(2)\footnote{\frenchspacing There are amendments to regulation 1(2) which are not relevant to this Order.} (interpretation), the definition of “the Administration Act” shall be omitted.

\medskip

2.  For paragraph (2) of regulation 4 (official records) there shall be substituted the following paragraph–
\begin{quotation}
“(2) Paragraph (1) shall not apply so as to exclude the revision of a decision under section 9 of the Social Security Act 1998 or the supersession of a decision under section 10\footnote{\frenchspacing Section 10 was amended by paragraph 23 of Schedule 7 to the Social Security Contributions (Transfer of Functions, etc.) Act 1999 (c. 2).} of that Act or the consideration of fresh evidence in connection with the revision or supersession of a decision.”.
\end{quotation}

\part[Schedule 19 --- Amendment of the Social Security and Child Support (Decisions and Appeals) Regulations 1999]{Schedule 19\\*Amendment of the Social Security and Child Support (Decisions and Appeals) Regulations 1999}

\renewcommand\parthead{--- Schedule 19}

1.  In regulation 7\footnote{\frenchspacing Regulation 7 was amended by S.I. 1999/1623.} (date from which a decision superseded under section 10 takes effect)–
\begin{enumerate}\item[]
($a$) for paragraph (1) there shall be substituted the following paragraph–
\begin{quotation}
“(1) This regulation–
\begin{enumerate}\item[]
($a$) is, except for paragraph (2)($b$), subject to regulations 26\footnote{\frenchspacing Regulation 26 was amended by S.I. 1988/522, 1989/136 and 1993/1113.} (income support) and 26A\footnote{\frenchspacing Regulation 26A was inserted by S.I. 1996/1460 and amended by S.I. 1998/1174.} (jobseeker’s allowance) of, and paragraph 7\footnote{\frenchspacing Paragraph 7 was substituted by S.I. 1990/2208 and amended by S.I. 1991/387, 1992/247 and 1998/1174.} (date from which superseding decision on ground of change of circumstances takes effect) of Schedule 7 to, the Claims and Payments Regulations; and

($b$) contains exceptions to the provisions of section 10(5) as to the date from which a decision under section 10 which supersedes an earlier decision is to take effect.”; and
\end{enumerate}
\end{quotation}

($b$) in paragraph (2)—
\begin{enumerate}\item[]
(i) for sub-paragraph ($a$)  there shall be substituted the following sub-paragraph–
\begin{quotation}
“($a$) from the date the change occurred or, where the change does not have effect until a later date, from the first date on which such effect occurs where–
\begin{enumerate}\item[]
(i) the decision is advantageous to the claimant; and

(ii) the change was notified to an appropriate office within one month of the change occurring or within such longer period as may be allowed under regulation 8 for the claimant’s failure to notify the change on an earlier date;” and
\end{enumerate}
\end{quotation}

(ii) head (i)  of sub-paragraph ($c$)  shall be omitted.
\end{enumerate}
\end{enumerate}

\medskip

2.  In paragraph 8 of Schedule 2 (decisions against which no appeal lies), for the words “one falling within regulation 4 of those Regulations” there shall be substituted the words “a decision whether benefit is sufficient for a deduction to be made”.

\part[Schedule 20 --- Amendment of the No.\ 8 Order, the No.\ 9 Order and the No.\ 11 Order]{Schedule 20\\*Amendment of the No.\ 8 Order, the No.\ 9 Order and the No.\ 11 Order}

\renewcommand\parthead{--- Schedule 20}

1.  The following provisions, namely–
\begin{enumerate}\item[]
($a$) paragraphs (5) and (9) to (12) of article 4 (consequential modifications) of, and Schedules 5 and 9 to 11 to, the No.\ 8 Order;

($b$) paragraphs (2), (8), (9) and (14) of article 3 (consequential modifications) of, and Schedules 2, 7, 8 and 13 to, the No.\ 9 Order; and

($c$) paragraphs (2) to (6), (9), (10) and (13) to (15) of article 3 (consequential modifications) of, and Schedules 2 to 6, 9, 10 and 13 to 15 to, the No.\ 11 Order, 
\end{enumerate}
are hereby revoked.

\medskip

2.  In–
\begin{enumerate}\item[]
($a$) article 3(1) (savings) of the No.\ 8 Order;

($b$) article 5 (savings) of the No.\ 9 Order; and

($c$) article 5 (savings) of the No.\ 11 Order,
\end{enumerate}
after the words “cease to have effect)” there shall be inserted the words “, section 86(2) and Schedule 8 (repeals)”.

\part[Schedule 21 --- Transitional provisions in relation to the recovery of benefits]{Schedule 21\\*Transitional provisions in relation to the recovery of benefits}

\renewcommand\parthead{--- Schedule 21}

1.---(1)  Regulation 2 of the 1997 Regulations shall, notwithstanding regulation 59 of the Regulations continue to have effect until 29th December 2000 in relation to any certificate of recoverable benefits in respect of which a right of appeal arose before 29th November 1999 subject to the modifications specified in sub-paragraph (2) below.

(2) The modifications referred to in sub-paragraph (1) above are as if–
\begin{enumerate}\item[]
($a$) for the word “chairman” in each place in which it occurs there were substituted the words “legally qualified panel member”;

($b$) in paragraph (2) the words “of a medical appeal tribunal” were omitted;

($c$) for paragraph (7) there were substituted the following paragraph–
\begin{quotation}
“(7) Notwithstanding paragraph (2), no appeal may be brought after 29th December 2000.”;
\end{quotation}

($d$) in paragraph (18) the words from “, notwithstanding that a condition” to the end were omitted; and

($e$) after paragraph (18) there were added the following paragraph–
\begin{quotation}
“(19) In this regulation “legally qualified panel member” has the meaning it bears in regulation 1(3) of the Social Security and Child Support (Decisions and Appeals) Regulations 1999.”.
\end{quotation}
\end{enumerate}

\medskip

2.  An appeal duly made (but not determined) before 29th November 1999 against a certificate of recoverable benefits shall be referred to and determined by an appeal tribunal under section 12 of the Social Security (Recovery of Benefits) Act 1997\footnote{\frenchspacing 1997 c. 27.}.

\medskip

3.---(1)  Notwithstanding regulation 39 and Chapter III of Part V of the Regulations, this paragraph applies where a direction (“the direction”) was given under regulation 4(1) of the 1997 Regulations.

(2) An appeal tribunal shall hold an oral hearing of an appeal if–
\begin{enumerate}\item[]
($a$) sub-paragraph (3) below applies; or

($b$) the chairman or, in the case of an appeal tribunal which has only one member, that member, is satisfied that such a hearing is necessary to enable the tribunal to reach a decision.
\end{enumerate}

(3) This sub-paragraph applies where a notification that a party to the proceedings wishes an oral hearing to be held is received by a clerk to a medical appeal tribunal before 29th November 1999 or by a clerk to an appeal tribunal (notwithstanding that it was sent to a clerk to a medical appeal tribunal) after that date within–
\begin{enumerate}\item[]
($a$) 10 days of receipt by that party of the direction; or

($b$) such other period as–
\begin{enumerate}\item[]
(i) the clerk to, or the chairman of, the medical appeal tribunal may have directed; or

(ii) where head (i)  above does not apply, a clerk to an appeal tribunal may direct.
\end{enumerate}
\end{enumerate}

(4) An appeal tribunal shall determine an appeal without an oral hearing where sub-paragraph (2) does not apply.

\medskip

4.  An appeal tribunal shall completely rehear any appeal to a medical appeal tribunal in relation to a certificate of recoverable benefits which stands adjourned immediately before 29th November 1999.

\medskip

5.  A copy of a statement of–
\begin{enumerate}\item[]
($a$) the reasons for a decision of a medical appeal tribunal in relation to a certificate of recoverable benefits; and

($b$) its findings on questions of fact material thereto,
\end{enumerate}
shall be supplied to each party to the proceedings before that tribunal, if requested by any such party within 21 days of the date on which notification of that decision was given or sent.

\medskip

6.---(1)  Subject to paragraph 1(2) above and sub-paragraph (3) below–
\begin{enumerate}\item[]
($a$) regulation 11; and

($b$) regulations 2(16) and (17) and 12 in so far as they relate to regulation 11,
\end{enumerate}
of the 1997 Regulations shall, notwithstanding regulation 59 of the Regulations, continue to have effect, subject to the modifications specified in sub-paragraph (2) below, in relation to any application to set aside a decision of a medical appeal tribunal in relation to a certificate of recoverable benefits.

(2) The modifications referred to in sub-paragraph (1) above are as if in–
\begin{enumerate}\item[]
($a$) regulation 2(16), for the words “the chairman” there were substituted the words “the legally qualified panel member”;

($b$) regulation 11(1), for the words “the tribunal which gave the decision or by another medical appeal tribunal” there were substituted the words “a legally qualified panel member within the meaning of regulation 1(3) of the Social Security and Child Support (Decisions and Appeals) Regulations 1999”;

($c$) regulation 11(2)–
\begin{enumerate}\item[]
(i) for the words “the tribunal shall” in both places in which they occur there were substituted the words “the legally qualified panel member shall”; and

(ii) for the words “it is satisfied” there were substituted the words “he is satisfied”;
\end{enumerate}

($d$) regulation 11(3)($b$), after the words “the office of the clerk to the tribunal which made the relevant decision” there were inserted the words “or to a clerk to an appeal tribunal”; and

($e$) regulation 11(4)–
\begin{enumerate}\item[]
(i) for the words “the chairman of the tribunal” there were substituted the words “a legally qualified panel member”; and

(ii) for the words “a chairman” there were substituted the words “a legally qualified panel member”.
\end{enumerate}
\end{enumerate}

(3) Sub-paragraph (1) above shall not apply in any case where an application to set aside a decision of a medical appeal tribunal is made after 29th December 2000.

\medskip

7.---(1)  Subject to sub-paragraph (2) below, any decision of a medical appeal tribunal under section 12 of the Social Security (Recovery of Benefits) Act 1997 shall be treated as a decision of an appeal tribunal under that section.

(2) Where sub-paragraph (1) above applies, any application for leave to appeal which is made for the purposes of section 14(10)($a$)\footnote{\frenchspacing Section 14(7) to (10) of the Social Security Act 1998 applies by virtue of section 13(3) of the Social Security (Recovery of Benefits) Act 1997 (c. 27) as amended by the Social Security Act 1998, section 86(1) and Schedule 7, paragraph 152(3).} shall be–
\begin{enumerate}\item[]
($a$) made no later than three months after the date on which a copy of the statement of the reasons for the decision of the medical appeal tribunal was given or sent to the applicant; and

($b$) determined by a legally qualified panel member.
\end{enumerate}

\part[Schedule 22 --- Transitional provisions in relation to relevant benefits]{Schedule 22\\*Transitional provisions in relation to relevant benefits}

\renewcommand\parthead{--- Schedule 22}

1.  A decision which fell to be made before 29th November 1999, but which was not made before that date–
\begin{enumerate}\item[]
($a$) on a claim for; or

($b$) under or by virtue of Part II of the Administration Act in relation to,
\end{enumerate}
a relevant benefit (other than a decision which fell to be made on review or on appeal) shall be made by the Secretary of State under paragraph ($a$)  or, as the case may be, paragraph ($c$)  of section 8(1).

\medskip

2.---(1)  Any application duly made before 29th November 1999 under Part II of the Administration Act for a review of a decision in relation to a relevant benefit which was not decided before that date shall on and after that date be treated as an application to the Secretary of State–
\begin{enumerate}\item[]
($a$) where the application is not in respect of a decision given on appeal and is made–
\begin{enumerate}\item[]
(i) within three months of the date on which the applicant was notified of the decision, or within such longer period as may be allowed under sub-paragraph (2) below; and

(ii) other than on the ground of a relevant change of circumstances,
\end{enumerate}
for a revision of that decision under section 9; or

($b$) in any other case, for a decision under section 10 to supersede that decision.
\end{enumerate}

(2) Subject to sub-paragraphs (3) and (4) below, the period of three months specified in sub-paragraph (1)($a$)  above may be extended where an application for such an extension is made before 29th December 2000 by a claimant or a person acting on his behalf containing–
\begin{enumerate}\item[]
($a$) the grounds on which an extension of time is sought; and

($b$) sufficient details of the decision to enable it to be identified.
\end{enumerate}

(3) An application for an extension of time shall not be granted under sub-paragraph (2) above unless the Secretary of State is satisfied that–
\begin{enumerate}\item[]
($a$) it is reasonable to grant that application;

($b$) the application for review has merit; and

($c$) special circumstances are relevant to the application for an extension of time as a result of which it was not practicable for the application for review to be made within three months of the date of the adjudication officer’s decision being notified to the claimant.
\end{enumerate}

(4) In deciding whether to grant an extension of time no account shall be taken of the following factors–
\begin{enumerate}\item[]
($a$) that the claimant or any person acting for him misunderstood or was unaware of the law applicable to his case (including misunderstanding or being unaware of the period specified in sub-paragraph (1)($a$)  above); or

($b$) that a Commissioner or a court has taken a different view of the law from that previously understood and applied by the adjudication officer.
\end{enumerate}

(5) Where, by virtue of sub-paragraph (1)($b$)  above–
\begin{enumerate}\item[]
($a$) a decision is made under section 10 which is advantageous to the applicant; and

($b$) the same decision could have been made on a review prior to 29th November 1999,
\end{enumerate}
that decision shall take effect from the date on which it would have taken effect had the decision been so made.

\medskip

3.---(1)  A decision (other than a decision of a social security appeal tribunal or a Commissioner) made before 29th November 1999–
\begin{enumerate}\item[]
($a$) on a claim for; or

($b$) under or by virtue of Part II of the Administration Act in relation to,
\end{enumerate}
a relevant benefit shall be treated as a decision of the Secretary of State under paragraph ($a$)  or, as the case may be, paragraph ($c$)  of section 8(1).

(2) Where, before 29th November 1999, any person was required to give notice to the claimant of a decision referred to in sub-paragraph (1) above, and such notice was not given before that date, the Secretary of State shall give notice to the claimant of that decision.

\medskip

4.---(1)  This paragraph applies where the time limit for making an appeal to a social security appeal tribunal in respect of a decision in relation to a relevant benefit made before 29th November 1999 has not expired before that date.

(2) Where sub-paragraph (1) above applies, regulation 3 of the Adjudication Regulations as it relates to the period within which an appeal may be made, or an extension of that period, shall, notwithstanding regulation 59 of the Regulations, continue to have effect, subject to the modifications in sub-paragraph (3) below, with respect to any appeal to an appeal tribunal made on or after 29th November 1999 in relation to that decision.

(3) The modifications referred to in sub-paragraph (2) above are as if–
\begin{enumerate}\item[]
($a$) references to a tribunal, a chairman or a person considering the application were references to a legally qualified panel member; and

($b$) in paragraph (3E)\footnote{\frenchspacing Paragraph (3E) was inserted by S.I. 1996/182.}, for the words from “6 years” to the end of the paragraph there were substituted the words “29th December 2000”.
\end{enumerate}

(4) Notwithstanding regulation 3 of the Regulations, the Secretary of State may revise under section 9 a decision made before 29th November 1999 on a claim for or award of a relevant benefit (other than a decision made on appeal)–
\begin{enumerate}\item[]
($a$) pursuant to an application for a review of a decision made within three months of the notification of that decision; or

($b$) where an appeal has been duly made against that decision but not determined.
\end{enumerate}

(5) Where a decision is revised pursuant to sub-paragraph (4) above the appeal shall lapse unless the decision as revised is not more advantageous to the appellant than the decision before it was revised.

\medskip

5.  An appeal to a social security appeal tribunal in relation to a relevant benefit which was duly made before 29th November 1999 and which was not determined before that date shall, without prejudice to Chapter III of Part V of the Regulations, be treated as an appeal duly made to an appeal tribunal in relation to a decision of the Secretary of State under section 8.

\medskip

6.---(1)  This paragraph applies where a clerk to a social security appeal tribunal has before 29th November 1999 given a direction under regulation 22(1) of the Adjudication Regulations in connection with an appeal in relation to a relevant benefit to that tribunal, and the notification mentioned in paragraph (1A)\footnote{\frenchspacing Paragraph (1A) of regulation 22 was inserted by S.I. 1996/2450.} of that regulation 22 has not been received by the clerk before that date.

(2) A notification in response to such a direction given under that regulation 22(1) shall be–
\begin{enumerate}\item[]
($a$) in writing; and

($b$) made within 14 days of receipt of the direction or within such other period as the clerk to an appeal tribunal may direct.
\end{enumerate}

(3) An appeal may be struck out by the clerk to an appeal tribunal where the notification referred to in sub-paragraph (2) above is not received within the period specified in that sub-paragraph.

(4) An appeal which has been struck out in accordance with sub-paragraph (3) above shall be treated for the purpose of reinstatement as if it had been struck out under regulation 46 of the Regulations.

(5) An oral hearing of the appeal shall be held where–
\begin{enumerate}\item[]
($a$) a notification is received by the clerk to the appeal tribunal under sub-paragraph (2) above; or

($b$) the chairman of the appeal tribunal or, in the case of an appeal tribunal which has only one member, that member, is satisfied that such a hearing is necessary to enable the tribunal to reach a decision.
\end{enumerate}

\medskip

7.  Where an appeal to a social security appeal tribunal in relation to a relevant benefit has been struck out under regulation 7 of the Adjudication Regulations, a legally qualified panel member may reinstate the appeal on an application by any party to the proceedings made not later than three months from the date of the order under paragraph (1) of that regulation if he is satisfied that–
\begin{enumerate}\item[]
($a$) the applicant did not receive a notice under paragraph (2) of that regulation; and

($b$) the conditions in paragraph (2A) of that regulation were not satisfied,
\end{enumerate}
and the appeal shall be treated as an appeal to an appeal tribunal in relation to a decision of the Secretary of State under section 8.

\medskip

8.  An appeal tribunal shall completely rehear any appeal to a social security appeal tribunal in relation to a relevant benefit which stands adjourned immediately before 29th November 1999.

\medskip

9.  A copy of a statement of–
\begin{enumerate}\item[]
($a$) the reasons for a decision of a social security appeal tribunal in relation to a relevant benefit; and

($b$) its findings on questions of fact material thereto,
\end{enumerate}
shall be supplied to each party to the proceedings before that tribunal, if requested by any such party within 21 days of the date on which notification of that decision was given or sent.

\medskip

10.---(1)  Subject to sub-paragraph (2) below, any decision of a social security appeal tribunal in relation to a relevant benefit shall be treated as a decision of an appeal tribunal made under section 12.

(2) Where sub-paragraph (1) above applies, any application for leave to appeal which is made for the purposes of section 14(10)($a$)  shall be–
\begin{enumerate}\item[]
($a$) made no later than three months after the date on which a copy of the statement of the reasons for the decision of the social security appeal tribunal was given or sent to the applicant; and

($b$) determined by a legally qualified panel member.
\end{enumerate}

\medskip

11.---(1)  Subject to sub-paragraph (3) below, regulation 10 of the Adjudication Regulations and regulation 3 of those Regulations in so far as it relates to that regulation 10, shall, notwithstanding regulation 59 of the Regulations, continue to have effect, subject to the modifications specified in sub-paragraph (2) below, in relation to any application to set aside a decision of a social security appeal tribunal in relation to a relevant benefit.

(2) The modifications referred to in sub-paragraph (1) above are as if in regulation 3 for the reference to a chairman and in regulation 10(1) the first reference to the adjudicating authority which gave the decision and to an authority of like status, there were substituted references to a legally qualified panel member.

(3) Paragraph (1) above shall not apply in any case where an application to set aside a decision of a social security appeal tribunal is made after 29th December 2000.

\medskip

12.  Where, immediately before 29th November 1999, a payment of relevant benefit was suspended or withheld by virtue of any provision of Part V of the Social Security (Claims and Payments) Regulations 1987\footnote{\frenchspacing S.I. 1987/1968; relevant amending instruments are S.I. 1992/247, 1993/2113, 1994/2319 and 1996/1460 and 2306.} (suspension and extinguishment), the provisions of Chapter I of Part III of the Regulations (suspension and termination) shall apply with respect to that suspension or withholding as if it were a suspension imposed by virtue of those provisions.

\medskip

13.  For the purpose of section 10(1)($b$) , a decision of a Commissioner made before 29th November 1999 in relation to a relevant benefit shall be treated as a decision of a Commissioner made under section 14.

\part[Schedule 23 --- Transitional provisions in relation to the social fund]{Schedule 23\\*Transitional provisions in relation to the social fund}

\renewcommand\parthead{--- Schedule 23}

1.  An application to–
\begin{enumerate}\item[]
($a$) the social fund shall, from 29th November 1999, be determined by an appropriate officer;

($b$) to a social fund officer for a review shall, from 29th November 1999, be treated as an application for a review by an appropriate officer.
\end{enumerate}

\medskip

2.  A determination of a social fund officer shall be treated as a determination of an appropriate officer from 29th November 1999.

\medskip

3.  In this Schedule “appropriate officer” has the meaning it bears in section 36(1). 

\part{Explanatory Note}

\renewcommand\parthead{--- Explanatory Note}

\subsection*{(This note is not part of the Order)}

This Order provides for the coming into force on 29th November 1999 of further provisions of the Social Security Act 1998 (“the Act”) introducing new arrangements for decision-making in relation to income support, the social fund, child’s special allowance and the recovery of benefits from compensation.

The provisions brought into force by article 2 and Schedule 1 relate in particular to the transfer of decision-making functions from adjudication officers to the Secretary of State, from social fund officers to appropriate officers of the Secretary of State and from social security appeal tribunals and medical appeal tribunals to appeal tribunals constituted under Chapter I of Part I of the Act.

Article 3 and Schedules 2 to 20 make consequential amendments in other statutory instruments in so far as they are concerned with, or make reference to, existing arrangements for decision-making and appeals.

Article 4 and Schedules 21 to 23 make transitional provision, in particular as to the manner in which matters are to be dealt with on or after 29th November 1999 which are awaiting determination under the existing srrangements for decision-making and appeals immediately before that date. 

\part{Note as to Earlier Commencement Orders}

\renewcommand\parthead{--- Note as to Earlier Commencement Orders}

\subsection*{(This note is not part of the Order)}

The following provisions have been brought into force by the Social Security Act 1998 (Commencement No.\ 1) Order 1998 (S.I.\ 1998/2209), the Social Security Act 1998 (Commencement No.\ 2) Order 1998 (S.I.\ 1998/2780), the Social Security Act 1998 (Commencement No.\ 3) Order 1999 (S.I.\ 1999/418), the Social Security Act 1998 (Commencement No.\ 4) Order 1999 (S.I.\ 1999/526), the Social Security Act 1998 (Commencement No.\ 5) Order 1999 (S.I.\ 1999/528), the Social Security Act 1998 (Commencement No.\ 6) Order 1999 (S.I.\ 1999/1055), the Social Security Act 1998 (Commencement No.\ 7 and Consequential and Transitional Provisions) Order 1999 (S.I.\ 1999/1510), the Social Security Act 1998 (Commencement No.\ 8, and Savings and Consequential and Transitional Provisions) Order 1999 (S.I.\ 1999/1958), the Social Security Act 1998 (Commencement No.\ 9, and Savings and Consequential and Transitional Provisions) Order 1999 (S.I.\ 1999/2422), the Social Security Act 1998 (Commencement No.\ 10 and Transitional Provisions) Order 1999 (S.I.\ 1999/2739) and the Social Security Act 1998 (Commencement No.\ 11, and Savings and Consequential and Transitional Provisions) Order 1999 (S.I.\ 1999/2860). 

{\noindent\footnotesize
%\begin{tabulary}{\textwidth}{JJJ}
\begin{longtable}{p{127.04362pt}p{139.74577pt}p{87.20068pt}}
\hline
\itshape Provision of Social Security Act 1998	& \itshape Date of Commencement	& \itshape S.I. Number\\
\hline
\endhead
\hline
\endlastfoot
\footnote{In this note an asterisk indicates that the provision or provisions in the entry to which it relates has or have been commenced in part only.\label{fn:124}}Section 1($a$) 	& 5th July, 6th September and 5th and 18th October 1999	&1999/1958, 2422, 2739 and 2860\\
Section 1($c$) 	&1st June 1999	&1999/1510\\
Section 2 (except section 2(2)($a$))	&8th September 1998	&1998/2209\\
\footref{fn:124}Section 2(2)($a$) 	&5th July, 6th September and 5th and 18th October 1999	&1999/1958, 2422, 2739 and 2860\\
Section 3	&8th September 1998	&1998/2209\\
\footref{fn:124}Section 4(1)($a$)  and (2)($a$) 	&5th July, 6th September and 5th and 18th October 1999	&1999/1958, 2422, 2739 and 2860\\
Section 4(1)($b$)  and (2)($b$) 	&1st June 1999	&1999/1510\\
\footref{fn:124}Section 4(1)($c$)  and (2)($c$) 	&18th October 1999&	1999/2860\\
Section 5	&1st June 1999	&1999/1510\\
Sections 6 and 7	&4th March and 1st June 1999	&1999/528 and 1510\\
\footref{fn:124}Section 8(1)($a$)  and ($c$), (2), (3)($a$), (4) and (5)	&5th July, 6th September and 5th and 18th October 1999	&1999/1958, 2422, 2739 and 2860\\
\footref{fn:124}Section 8(3)($b$) 	&18th October 1999	&1999/2860\\
\footref{fn:124}Section 8(3)($d$)  and ($e$) 	&5th October 1999	&1999/2739\\
\footref{fn:124}Section 8(3)($g$) 	&5th July 1999	&1999/1958\\
\footref{fn:124}Sections 9 to 12	&4th March, 5th July, 6th September and 5th and 18th October 1999	&1999/528, 1958, 2422, 2739 and 2860\\
\footref{fn:124}Section 13	&5th July, 6th September and 5th and 18th October 1999	&1999/1958, 2422, 2739 and 2860\\
\footref{fn:124}Section 14 and Schedule 4	&4th March, 5th July, 6th September and 5th and 18th October 1999	&1999/528, 1958, 2422, 2739 and 2860\\
\footref{fn:124}Section 15	&4th March, 5th July, 6th September and 5th and 18th October 1999	&1999/528, 1958, 2422, 2739 and 2860\\
\footref{fn:124}Section 16 and Schedule 5	&8th September 1998, 4th March, 6th April, 5th July, 6th September and 5th and 18th October 1999	&1998/2209 and 1999/528, 1958, 2422, 2739 and 2860\\
\footref{fn:124}Sections 17 and 18(1)	&4th March, 5th July, 6th September and 5th and 18th October 1999	&1999/528, 1958, 2422, 2739 and 2860\\
Section 18(2)	&18th October 1999	&1999/2860\\
\footref{fn:124}Section 19	&5th July, 6th September and 5th and 18th October 1999	&1999/528, 2422, 2739 and 2860\\
\footref{fn:124}Sections 20 to 26 (except section 26(8))	&4th March, 5th July, 6th September and 5th and 18th October 1999	&1999/528, 1958, 2422, 2739 and 2860\\
Section 26(8)	&1st June 1999	&1999/1510\\
\footref{fn:124}Section 27	&5th July, 6th September and 5th and 18th October 1999	&1999/1958, 2422, 2739 and 2860\\
\footref{fn:124}Section 28	&4th March 1999	&1999/528\\
\footref{fn:124}Section 28 (except subsections (3)($d$)  and ($e$))	&5th July 1999	&1999/1958\\
\footref{fn:124}Section 28 (except subsections (3)($c$)  to ($e$))	&5th July, 6th September, 5th and 18th October 1999	&1999/2422, 2739 and 2860\\
\footref{fn:124}Sections 29 and 30	&5th July 1999	&1999/1958\\
Section 31	&4th March and 6th September 1999	&1999/528 and 2422\\
Section 32 and 34	&18th October 1999	&1999/2860\\
\footref{fn:124}Section 38(1)($a$)  and (3)	&4th March 1999	&1999/528\\
\footref{fn:124}Section 39	&5th July, 6th September and 5th and 18th October 1999	&1999/1958, 2422, 2739 and 2860\\
Section 40&16th November and 7th December 1998&1998/2780\\Sections 41 to 44&4th March and 1st June 1999&1999/528 and 1510\\Sections 45 to 47&4th March and 18th October 1999&1999/528 and 2860\\Sections 48 and 49&8th September 1998&1998/2209\\Section 50(1)&8th September 1998&1998/2209\\Section 51&23rd February and 6th April 1999&1999/418\\Section 52&8th September 1998&1998/2209\\Section 53&8th September 1998 and 6th April 1999&1998/2209\\Section 54&4th March and 6th April 1999&1999/526\\Section 55&8th September 1998&1998/2209\\Sections 56 and 57&4th March and 6th April 1999&1999/526\\Section 59&8th September 1998&1998/2209\\Sections 60 and 61&4th March and 6th April 1999&1999/526\\Section 62&6th April 1999&1999/526\\Section 63&4th March and 6th April 1999&1999/526\\Section 64&6th April 1999&1999/526\\Section 65&8th September 1998 and 6th April 1999&1998/2209\\Section 68&8th September 1998&1998/2209\\Sections 70 and 71&5th April 1999&1999/1055\\Section 73&6th April 1999&1998/2209\\Section 74&4th March 1999&1999/528\\Section 75&5th October 1998&1998/2209\\Section 76&16th November 1998&1998/2780\\\footref{fn:124}Schedule 1, paragraphs 1 to 9 and 11 to 13&4th March and 1st June 1999&1999/528 and 1510\\\footref{fn:124}Schedule 2, paragraphs 4, 5, 6($a$)  and 9&4th March and 5th July 1999&1999/528 and 1958\\\footref{fn:124}Schedule 3, paragraphs 1, 2, 3($a$)  and ($c$) , 4 to 7 and 9&4th March and 5th July 1999&1999/528 and 1958\\Schedule 7 in the respects specified below and section 86(1) in so far as it relates to them---\\\\paragraphs 1 and 2&1st June 1999&1999/1510\\\footref{fn:124}paragraph 4(2)&1st June and 18th October 1999&1999/1510 and 2860\\paragraph 4(3)&1st June 1999&1999/1510\\\footref{fn:124}paragraphs 5 to 7&18th October 1999&1999/2860\\\footref{fn:124}paragraphs 8 and 9&4th March and 18th October 1999&1999/528 and 2860\\\footref{fn:124}paragraph 10&18th October 1999&1999/2860\\\footref{fn:124}paragraph 11&5th July and 6th September 1999&1999/1958 and 2422\\paragraphs 12 to 14&6th April 1999&1999/526\\\footref{fn:124}paragraph 15&18th October 1999&1999/2860\\paragraph 16&6th April 1999&1999/418\\\footref{fn:124}paragraph 17&18th October 1999&1999/2860\\paragraphs 18 to 26&1st June 1999&1999/1510\\paragraph 27&8th September 1998 and 1st June 1999&1998/2209 and 1999/1510\\paragraphs 28 to 34&1st June 1999&1999/1510\\paragraph 35&4th March and 1st June 1999&1999/528 and 1510\\paragraphs 36 to 38&1st June 1999&1999/1510\\paragraph 39 and 40&4th March and 1st June 1999&1999/528 and 1510\\paragraphs 41 and 42&1st June 1999&1999/1510\\paragraphs 43 and 44&4th March and 1st June 1999&1999/528 and 1510\\paragraph 45&1st June 1999&1999/1510\\paragraph 46&16th November 1998, 4th March and 1st June 1999&1998/2780 and 1999/528 and 1510\\\footref{fn:124}paragraph 47&1st June 1999&1999/1510\\paragraph 48&1st June 1999&1999/1510\\paragraph 49&8th September 1998&1998/2209\\paragraphs 50 and 51&1st June 1999&1999/1510\\paragraphs 52 to 54&4th March and 1st June 1999&1999/528 and 1510\\\footref{fn:124}paragraph 55&18th October 1999&1999/2860\\\\paragraph 56&8th September 1998 and 6th April 1999&1998/2209\\paragraph 57&6th April 1999&1998/2209\\paragraph 58(1)&6th April 1999&1999/418\\paragraph 58(2)&6th April 1999&1998/2209\\paragraphs 59 to 61&6th April 1999&1999/418\\paragraph 62&6th September 1999&1999/2422\\\footref{fn:124}paragraphs 63 to 65&5th July 1999&1999/1958\\\footref{fn:124}paragraphs 66 to 70&5th July, 6th September and 18th October 1999&1999/1958, 2422 and 2860\\\footref{fn:124}paragraph 71&8th September 1998 and 6th April, 5th July, 6th September and 18th October 1999&1998/2209 and 1999/418, 1958, 2422 and 2860\\paragraph 72(3) and (4)&5th April 1999&1999/1055\\paragraphs 74 and 75&6th April 1999&1999/418\\\footref{fn:124}paragraph 76&6th September 1999&1999/2422\\paragraph 77(1), (6) to (9), (11), (12) and (14) to (16)&8th September 1998 and 6th April 1999&1998/2209\\paragraph 77(2) to (5)&6th April 1999&1999/418\\\footref{fn:124}paragraph 78&6th September 1999&1999/2422\\\footref{fn:124}paragraphs 79(1) and 81&5th July, 6th September and 5th and 18th October 1999&1999/1958, 2422, 2739 and 2860\\\footref{fn:124}paragraph 82&18th October 1999&1999/2860\\\footref{fn:124}paragraph 84&5th July 1999&1999/1958\\paragraph 85&6th April 1999&1999/526\\paragraph 86&6th April 1999&1998/2209 and 1999/526\\paragraph 87&6th April 1999&1999/526\\\footref{fn:124}paragraphs 88 and 89&5th July, 6th September and 5th and 18th October 1999&1999/1958, 2422, 2739 and 2860\\paragraph 90&6th April 1999&1999/418\\paragraph 91&8th September 1998 and 6th April 1999&1998/2209\\paragraphs 92 to 94&6th April 1999&1999/418\\\footref{fn:124}paragraphs 96 and 98&18th October 1999&1999/2860\\paragraph 99&8th September 1998 and 6th April 1999&1998/2209 and 1999/418 and 526\\paragraph 100(1)&6th April 1999&1998/2209\\\\paragraph 100(2)&6th April 1999&1999/526\\\footref{fn:124}paragraph 101&5th July 1999&1999/1958\\\footref{fn:124}paragraph 102&5th July, 6th September and 18th October 1999&1999/1958, 2422 and 2860\\paragraph 104&4th March 1999&1999/528\\paragraph 105&5th July 1999&1999/528\\\footref{fn:124}paragraphs 106 to 108&5th July, 6th September and 5th and 18th October 1999&1999/1958, 2422, 2739 and 2860\\\footref{fn:124}paragraph 109&6th September and 5th and 18th October 1999&1999/2422, 2739 and 2860\\paragraph 110(1)($a$) &8th September 1998 and 6th April 1999&1998/2209\\paragraph 110(1)($b$) &6th April 1999&1999/418\\\footref{fn:124}paragraph 111($b$) &5th July 1999&1999/1958\\\footref{fn:124}paragraph 112&6th September and 5th October 1999&1999/2422 and 2739\\paragraph 114&8th September 1998 and 6th April 1999&1998/2209\\\footref{fn:124}paragraph 115&5th July 1999&1999/1958\\\footref{fn:124}paragraph 117&18th October 1999&1999/2860\\\footref{fn:124}paragraph 118(1)&1st June and 18th October 1999&1999/1510 and 2860\\\footref{fn:124}paragraph 118(2)&18th October 1999&1999/2860\\\footref{fn:124}paragraph 119&18th October 1999&1999/2860\\\footref{fn:124}paragraph 121&4th March 1999&1999/528\\paragraph 121(1)&1st June 1999&1999/1510\\\footref{fn:124}paragraph 121(2)&6th September, 5th and 18th October 1999&1999/2422, 2739 and 2860\\paragraphs 122, 123(1)($b$)  and 124(1)($b$) &1st June 1999&1999/1510\\\footref{fn:124}paragraph 123(1)($a$) &6th September and 5th October 1999&1999/2422 and 2739\\\footref{fn:124}paragraphs 123(2) and 124(2)&1st June and 18th October 1999&1999/1510 and 2860\\\footref{fn:124}paragraph 124(1)($a$) &6th September and 5th October 1999&1999/2422 and 2739\\paragraphs 126 to 128&6th April 1999&1999/418\\\footref{fn:124}paragraphs 129 and 130(2)&5th July 1999&1999/1958\\\footref{fn:124}paragraph 131&4th March and 5th July 1999&1999/528 and 1958\\paragraph 133&6th April 1999&1999/418\\paragraphs 134 to 146&18th October 1999&1999/2860\\\footref{fn:124}paragraph 147&18th October 1999&1999/2860\\\footref{fn:124}paragraph 149&4th March 1999&1999/528\\\footref{fn:124}Section 86(2) and Schedule 8&8th September 1998 and 5th and 6th April, 1st June, 5th July, 6th September and 18th October 1999&1998/2209 and 1999/418, 526, 1055, 1510, 1958, 2422 and 2860\\
%\end{tabulary}
\end{longtable}

}

\end{document}