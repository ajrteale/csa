\documentclass[12pt,a4paper]{article}

\newcommand\regstitle{The Social Security and Child Support (Miscellaneous Amendments) Regulations 2000}

\newcommand\regsnumber{2000/1596}

%\opt{newrules}{
\title{\regstitle}
%}

%\opt{2012rules}{
%\title{Child Maintenance and Other Payments Act 2008\\(2012 scheme version)}
%}

\author{S.I. 2000 No. 1596}

\date{Made
15th June 2000\\
%Laid before Parliament
%27th January 2000\\
Coming into force
19th June 2000}

%\opt{oldrules}{\newcommand\versionyear{1993}}
%\opt{newrules}{\newcommand\versionyear{2003}}
%\opt{2012rules}{\newcommand\versionyear{2012}}

\usepackage{csa-regs}

\setlength\headheight{27.57402pt}

\begin{document}

\maketitle

\noindent
Whereas a draft of this Instrument was laid before Parliament in accordance with section 80(1) of the Social Security Act 1998\footnote{\frenchspacing 1998 c. 14.} and approved by a resolution of each House of Parliament;

Now, therefore, the Secretary of State for Social Security, in exercise of the powers set out in the Schedule to this Instrument and of all other powers enabling him in that behalf, after agreement by the Social Security Advisory Committee that proposals to make the Regulations should not be referred to it\footnote{\frenchspacing \emph{See} sections 170 and 173(1)($b$) of the Social Security Administration Act 1992 (c. 5) (“the 1992 Act”); the Social Security Act 1998 is a relevant enactment for the purposes of section 170(5) of the 1992 Act by virtue of the amendment of section 170(5) of the 1992 Act by the Social Security Act 1998, Schedule 7, paragraph 104($a$).} and after consultation with the Council on Tribunals in accordance with section 8 of the Tribunals and Inquiries Act 1992\footnote{\frenchspacing 1992 c. 53.}, hereby makes the following Regulations: 

{\sloppy

\tableofcontents

}

\bigskip

\setcounter{secnumdepth}{-2}

\subsection[1. Citation, commencement and interpretation]{Citation, commencement and interpretation}

1.---(1)  These Regulations may be cited as the Social Security and Child Support (Miscellaneous Amendments) Regulations 2000 and shall come into force on 19th June 2000.

(2) In these Regulations—
\begin{enumerate}\item[]
“the Claims and Payments Regulations” means the Social Security (Claims and Payments) Regulations 1987\footnote{\frenchspacing S.I. 1987/1968; relevant amending instruments are S.I. 1993/1113, 1996/1460, 1997/793 and 2290, 1998/1174 and 1999/3178 (C. 81).};

“the Departure Regulations” means the Child Support Departure Direction and Consequential Amendments Regulations 1996\footnote{\frenchspacing S.I. 1996/2907; relevant amending instruments are S.I. 1999/1047 and 2000/897.};

“the Industrial Injuries Regulations” means the Social Security (Industrial Injuries) (Prescribed Diseases) Regulations 1985\footnote{\frenchspacing S.I. 1985/967, to which there are amendments not relevant to these Regulations.};

“the Maintenance Regulations” means the Child Support (Maintenance Assessment Procedure) Regulations 1992\footnote{\frenchspacing S.I. 1992/1813; relevant amending instruments are S.I. 1999/1047 and 2000/897.}; and

“the principal Regulations” means the Social Security and Child Support (Decisions and Appeals) Regulations 1999\footnote{\frenchspacing S.I. 1999/991; relevant amending instruments are S.I. 1999/1466, 1623 and 3178 (C. 81) and 2000/119 and 897.}.
\end{enumerate}

\subsection[2. Amendment of the Industrial Injuries Regulations]{Amendment of the Industrial Injuries Regulations}

2.  In regulation 7(4) for the words “review of the assessment” to the end of that regulation there shall be substituted the words “a supersession of the assessment relating to the relevant period.”.

\subsection[3--5. Amendment of the Claims and Payments Regulations]{Amendment of the Claims and Payments Regulations}

3.  In regulation 6—
\begin{enumerate}\item[]
($a$) for paragraphs (16) to (26) there shall be substituted the following paragraphs—
\begin{quotation}
“(16) Where a person has claimed a relevant benefit and that claim (“the original claim”) has been refused in the circumstances specified in paragraph (17), and a further claim is made in the additional circumstances specified in paragraph (18), that further claim shall be treated as made—
\begin{enumerate}\item[]
($a$) on the date of the original claim; or

($b$) on the first date in respect of which the qualifying benefit was payable,
\end{enumerate}
whichever is the later.

(17) The circumstances referred to in paragraph (16) are that the ground for refusal was—
\begin{enumerate}\item[]
($a$) in the case of severe disablement allowance, that the claimant’s disablement was less than 80 per cent.;

($b$) in the case of invalid care allowance, that the disabled person was not a severely disabled person within the meaning of section 70(2) of the Contributions and Benefits Act\footnote{\frenchspacing 1992 c. 4.};

($c$) in any case, that the claimant had not been awarded a qualifying benefit.
\end{enumerate}

(18) The additional circumstances referred to in paragraph (16) are that—
\begin{enumerate}\item[]
($a$) the claimant had made a claim for the qualifying benefit not later than ten days after the date of the original claim, and the claim for the qualifying benefit had not been decided;

($b$) after the original claim had been decided the claim for the qualifying benefit had been decided in the claimant’s or the disabled person’s favour; and

($c$) the further claim was made within three months of the date on which the claim for the qualifying benefit was decided.
\end{enumerate}

(19) Where a person has been awarded a relevant benefit and that award (“ the original award”) has been terminated in the circumstances specified in paragraph (20), and a further claim is made in the additional circumstances specified in paragraph (21), that further claim shall be treated as made—
\begin{enumerate}\item[]
($a$) on the date of termination of the original award; or

($b$) on the first date in respect of which the qualifying benefit again becomes payable,
\end{enumerate}
whichever is the later.

(20) The circumstances referred to in paragraph (19) are that the award of the qualifying benefit has itself been terminated or reduced by means of a revision, supersession or appeal in such a way as to affect the award of the relevant benefit.

(21) The additional circumstances referred to in paragraph (19) are that—
\begin{enumerate}\item[]
\begin{sloppypar}
($a$) after the original award has been terminated the claim for the qualifying benefit is decided in the claimant’s or the disabled person’s favour; and
\end{sloppypar}

($b$) the further claim is made within three months of the date on which the qualifying benefit is re-awarded, following revision, supersession or appeal.
\end{enumerate}

(22) In paragraphs (16) to (21)—
\begin{enumerate}\item[]
\begin{sloppypar}
“relevant benefit” means any of the following, namely—
\end{sloppypar}
\begin{enumerate}\item[]
($a$) 
benefit under Parts II to V of the Contributions and Benefits Act except incapacity benefit;

($b$) 
income support;

($c$) 
a jobseeker’s allowance;

\begin{sloppypar}
($d$) 
a social fund payment mentioned in section 138(1)($a$)  or (2) of the Contributions and Benefits Act;
\end{sloppypar}

($e$) 
child benefit;
\end{enumerate}

“qualifying benefit” means—
\begin{enumerate}\item[]
($a$) 
in relation to severe disablement allowance, the highest rate of the care component of disability living allowance;

($b$) 
in relation to invalid care allowance, any benefit or payment referred to in section 70(2) of the Contributions and Benefits Act;

($c$) 
in relation to a social fund payment in respect of maternity or funeral expenses, any benefit referred to in regulation 5(1)($a$)  or 7(1)($a$)  of the Social Fund Maternity and Funeral Expenses (General) Regulations 1987\footnote{\frenchspacing S.I. 1987/481; relevant amending instruments are S.I. 1988/36, 1991/2742, 1996/1443, 1997/792 and 2000/528.};

($d$) 
any other relevant benefit which has the effect of making another relevant benefit payable or payable at an increased rate;
\end{enumerate}

“the disabled person” means the person for whom the invalid care allowance claimant is caring in accordance with section 70(1)($a$)  of the Contributions and Benefits Act.
\end{enumerate}

(23) Where a person has ceased to be entitled to incapacity benefit, and a further claim for that benefit is made in the circumstances specified in paragraph (24), that further claim shall be treated as made—
\begin{enumerate}\item[]
($a$) on the date on which entitlement to incapacity benefit ceased; or

($b$) on the first date in respect of which the qualifying benefit was payable,
\end{enumerate}
whichever is the later.

(24) The circumstances referred to in paragraph (23) are that—
\begin{enumerate}\item[]
($a$) entitlement to incapacity benefit ceased on the ground that the claimant was not incapable of work;

($b$) at the date that entitlement ceased the claimant had made a claim for a qualifying benefit and that claim had not been decided;

($c$) after entitlement had ceased, the claim for the qualifying benefit was decided in the claimant’s favour; and

($d$) the further claim for incapacity benefit was made within three months of the date on which the claim for the qualifying benefit was decided.
\end{enumerate}

(25) In paragraphs (23) and (24) “qualifying benefit” means any of the payments referred to in regulation 10(2)($a$)  of the Social Security (Incapacity for Work) (General) Regulations 1995\footnote{\frenchspacing S.I. 1995/311; the relevant amending instrument is S.I. 1995/987.}.

(26) In paragraphs 18($a$)  and ($c$), 21($a$), 24 and in paragraph 18($b$)  where the word appears for the second time, “decided” includes the making of a decision following a revision, supersession or an appeal, whether by the Secretary of State, an appeal tribunal, a Commissioner or the court.”; and
\end{quotation}

($b$) in paragraph (29) for the words “paragraphs (16) and (21)” there shall be substituted the words “paragraphs (16) and (19)”.
\end{enumerate}

\medskip

4.---(1)  In regulation 26(1) the words “, the date from which a superseding decision on the ground of a relevant change of circumstances has effect” shall be omitted;

(2) In regulation 26A paragraphs (4) to (8) shall be omitted.

\medskip

5.  In Schedule 7 the words “, \lowercase{EFFECTIVE DATE OF SUPERSEDING DECISION}” shall be omitted from the heading and paragraph 7 shall be omitted.

\subsection[6--9. Amendment of the Maintenance Regulations]{Amendment of the Maintenance Regulations}

6.  In regulation 1(2), in the definition of “official error” for sub-paragraph ($b$)  there shall be substituted—
\begin{quotation}
\begin{enumerate}\item[]
“($b$) a person employed by a designated authority acting on behalf of the authority, which no person outside that authority caused or to which no person outside that authority materially contributed,
\end{enumerate}\noindent
but excludes any error of law which is only shown to have been an error by virtue of a subsequent decision of a Child Support Commissioner or the court;”.
\end{quotation}

\medskip

7.  In regulation 17, in paragraph (6)($a$)(i)  for the words “the date as from which the decision had effect” there shall be substituted the words “the date on which the decision was made”.

\medskip

8.  In regulation 20, in paragraph (3)($a$)(i)  for the words “since the decision was made” there shall be substituted the words “since the date from which the decision had effect”.

\medskip

9.  In regulation 23—
\begin{enumerate}\item[]
($a$) in paragraph (2) for the words “Where a decision is superseded” there shall be substituted the words “Subject to paragraph (19), where a decision is superseded”;

($b$) in paragraph (4) for the words “Where a superseding decision is made” there shall be substituted the words “Subject to paragraph (19), where a superseding decision is made”; and

($c$) after paragraph (18) there shall be added the following paragraph—
\begin{quotation}
“(19) Where a superseding decision is made in a case to which regulation 20(2)($a$)  or (3) applies and the material circumstance is the death of a qualifying child or a qualifying child ceasing to be a qualifying child, the decision shall take effect as from the first day of the maintenance period in which the change occurred.”.
\end{quotation}
\end{enumerate}

\subsection[10--13. Amendment of the Departure Regulations]{Amendment of the Departure Regulations}

10.  In regulation 1(2), in the definition of “official error” for sub-paragraph ($b$)  there shall be substituted—
\begin{quotation}
\begin{enumerate}\item[]
“($b$) a person employed by a designated authority acting on behalf of the authority, which no person outside that authority caused or to which no person outside that authority materially contributed,
\end{enumerate}\noindent
but excludes any error of law which is only shown to have been an error by virtue of a subsequent decision of a Child Support Commissioner or the court;”.
\end{quotation}

\medskip

11.  In regulation 8 paragraph (8) shall be omitted.

\medskip

12.  In regulation 32A, in paragraph (2)($a$)(i)  for the words “the date from which the decision had effect” there shall be substituted the words “the date on which the decision was made”.

\medskip

13.  In regulation 32E—
\begin{enumerate}\item[]
($a$) in paragraph (2) for the words “(3) and (5)” there shall be substituted the words “(3), (5) and (12)”;

($b$) in paragraph (3) for the words “Where a decision” there shall be substituted the words “Subject to paragraph (12), where a decision”; and

($c$) after paragraph (11) there shall be added the following paragraph—
\begin{quotation}
“(12) Where a superseding decision is made in a case to which regulation 32D(2)($a$)  or (3) applies and the material circumstance is the death of a qualifying child or a qualifying child ceasing to be a qualifying child, the decision shall take effect as from the first day of the maintenance period in which the change occurred.”.
\end{quotation}
\end{enumerate}

\subsection[14--35. Amendment of the principal Regulations]{Amendment of the principal Regulations}

14.  In regulation 1(3)—
\begin{enumerate}\item[]
($a$) after the definition of “designated authority” there shall be inserted the following definition—
\begin{quotation}
    ““family” has the same meaning as in section 137 of the Contributions and Benefits Act;”; 
\end{quotation}

($b$) in the definition of “official error” for sub-paragraph ($b$)  there shall be substituted—
\begin{quotation}
\begin{enumerate}\item[]
“($b$) a person employed by a designated authority acting on behalf of the authority, which no person outside that authority caused or to which no person outside that authority materially contributed,
\end{enumerate}\noindent
but excludes any error of law which is only shown to have been an error by virtue of a subsequent decision of a Commissioner or the court;” ; and
\end{quotation}

($c$) after the definition of “referral” there shall be inserted the following definition—
\begin{quotation}
““relevant credit” means a credit of contributions or earnings resulting from a decision in accordance with regulations made under section 22(5) of the Contributions and Benefits Act;”.
\end{quotation}
\end{enumerate}

\medskip

15.  In regulation 3 for paragraph (7) there shall be substituted the following paragraph—
\begin{quotation}
“(7) A decision under section 8 or 10 may be revised where—
\begin{enumerate}\item[]
($a$) the Secretary of State, appeal tribunal or Commissioner has awarded entitlement to a relevant benefit; and

($b$) on the date that entitlement arises, the claimant or a member of his family becomes entitled to, and is paid, another relevant benefit or an increase in the rate of another relevant benefit.”.
\end{enumerate}
\end{quotation}

\medskip

16.  In regulation 6(2) for sub-paragraph ($e$)  there shall be substituted the following sub-paragraph—
\begin{quotation}
“($e$) is a decision where—
\begin{enumerate}\item[]
(i) the claimant has been awarded entitlement to a relevant benefit; and

(ii) on a date after that entitlement arises, the claimant or a member of his family becomes entitled to, and is paid, another relevant benefit or an increase in the rate of another relevant benefit;”.
\end{enumerate}
\end{quotation}

\medskip

17.  In regulation 7—
\begin{enumerate}\item[]
($a$) for sub-paragraph ($a$)  of paragraph (1) there shall be substituted the following sub-paragraph—
\begin{quotation}
“($a$) is, except for paragraph (2)($b$), subject to Schedule 3A; and”;
\end{quotation}

($b$) after sub-paragraph ($b$)  of paragraph (2), there shall be inserted the following sub-paragraph—
\begin{quotation}
“($bb$) where the decision is advantageous to the claimant and is made on the Secretary of State’s own initiative, from the date on which the Secretary of State commenced action with a view to supersession;”;
\end{quotation}

($c$) for paragraph (5) there shall be substituted the following paragraph—
\begin{quotation}
“(5) Where the Secretary of State supersedes a decision made by an appeal tribunal or a Commissioner on the grounds specified in regulation 6(2)($c$)  (ignorance of, or mistake as to, a material fact), the decision under section 10 shall take effect, in a case where, as a result of that ignorance of or mistake as to material fact, the decision to be superseded was more advantageous to the claimant than it would otherwise have been and which either—
\begin{enumerate}\item[]
($a$) does not relate to a disability benefit decision or an incapacity benefit decision where there has been an incapacity determination; or

($b$) relates to a disability benefit decision or an incapacity benefit decision where there has been an incapacity determination, and the Secretary of State is satisfied that at the time the decision was made the claimant or payee knew or could reasonably have been expected to know of the fact in question and that it was relevant to the decision,
\end{enumerate}
from the date on which the decision of the appeal tribunal or the Commissioner took, or was to take, effect.”;
\end{quotation}

($d$) for paragraph (7) there shall be substituted the following paragraph—
\begin{quotation}
“(7) A decision which falls to be superseded under regulation 6(2)($e$)  shall be superseded as from the date on which the claimant or member of his family becomes entitled to and receives the relevant benefit or increase in benefit referred to in regulation 6(2)($e$)(ii).”; and
\end{quotation}

($e$) in paragraph (9)($a$)  for the words “the date of that decision” there shall be substituted the words “the date on which the Secretary of State commenced action with a view to supersession”.
\end{enumerate}

\medskip

18.  In regulation 7A—
\begin{enumerate}\item[]
($a$) in the heading and in paragraph (1) for the words “and 7(2)($c$)” there shall be substituted the words “, 7(2)($c$)  and (5)”;

($b$) in the definition of “incapacity benefit decision” after the words “relevant benefit” there shall be inserted the words “or relevant credit”; and

($c$) in the definition of “incapacity determination” for the words “all work test” there shall be substituted the words “personal capability assessment”\footnote{\frenchspacing The Social Security (Incapacity for Work) Miscellaneous Amendments Regulations 1999, (S.I. 1999/3109) re-named the “all work test” the “ personal capability assessment”.}.
\end{enumerate}

\medskip

19.  After regulation 12 there shall be inserted the following regulation—
\begin{quotation}
\subsection*{“Recrudescence of a prescribed disease}

12A.---(1)  This regulation applies to a decision made under sections 108 to 110 of the Contributions and Benefits Act where a disease is subsequently treated as a recrudescence under regulation 7 of the Social Security (Industrial Injuries) (Prescribed Diseases) Regulations 1985\footnote{\frenchspacing S.I. 1985/967.}.

(2) Where this regulation applies Chapter II of Part I of the Act shall apply as if section 8(2) did not apply.”.
\end{quotation}

\medskip

20.  In regulation 16 for paragraph (4) there shall be substituted the following paragraph—
\begin{quotation}
“(4) For the purposes of section 21(3)($c$)  an appeal is pending where a decision of an appeal tribunal, a Commissioner or a court has been made and the Secretary of State—
\begin{enumerate}\item[]
($a$) is awaiting receipt of that decision or (in the case of an appeal tribunal decision) is considering whether to apply for a statement of the reasons for it, or has applied for such a statement and is awaiting receipt thereof; or

($b$) has received that decision or (in the case of an appeal tribunal decision) the statement of the reasons for it, and is considering whether to apply for leave to appeal, or, where leave to appeal has been granted, is considering whether to appeal;
\end{enumerate}
and the Secretary of State shall give written notice of his proposal to make a request for a statement of the reasons for a tribunal decision, to apply for leave to appeal, or to appeal, as soon as reasonably practicable.”.
\end{quotation}

\medskip

21.  In regulation 20—
\begin{enumerate}\item[]
($a$) paragraph (1)($c$)  shall be omitted; and

($b$) for paragraphs (2) and (3) there shall be substituted the following paragraphs—
\begin{quotation}
“(2) Where regulation 16(3)($b$)(i)  applies, payment of a benefit suspended shall be made if the Secretary of State—
\begin{enumerate}\item[]
($a$) does not, in the case of a decision of an appeal tribunal, apply for a statement of the reasons for that decision within the period of one month specified in regulation 53(4);

($b$) does not, in the case of a decision of an appeal tribunal, a Commissioner or a court, make an application for leave to appeal and (where leave to appeal is granted) make the appeal within the time prescribed for the making of such applications and appeals;

($c$) withdraws an application for leave to appeal or the appeal; or

($d$) is refused leave to appeal, in circumstances where it is not open to him to renew the application for leave or to make a further application for leave to appeal.
\end{enumerate}

(3) Where regulation 16(3)($b$)(ii)  applies, payment of a benefit suspended shall be made if the Secretary of State, in relation to the decision of a Commissioner or the court in a different case—
\begin{enumerate}\item[]
($a$) does not make an application for leave to appeal and (where leave to appeal is granted) make the appeal within the time prescribed for the making of such applications and appeals;

($b$) withdraws an application for leave to appeal or the appeal; or

($c$) is refused leave to appeal, in circumstances where it is not open to him to renew the application for leave or to make a further application for leave to appeal.”.
\end{enumerate}
\end{quotation}
\end{enumerate}

\medskip

22.  In regulation 26, after paragraph ($b$)  there shall be inserted the following—
\begin{quotation}
“or

($c$) under Schedule 6 to the Contributions and Benefits Act\footnote{\frenchspacing 1992 c. 4.} (assessment of extent of disablement) in relation to sections 103 (disablement benefit) and 108 (prescribed diseases) of that Act for the purposes of industrial injuries benefit under Part V of that Act.”.
\end{quotation}

\medskip

23.  In regulation 33, after paragraph (9), there shall be added the following paragraph—
\begin{quotation}
“(10) The Secretary of State may discontinue action on an appeal where the appeal has not been forwarded to the clerk to an appeal tribunal or to a legally qualified panel member and the appellant has given written notice that he does not wish the appeal to continue.”.
\end{quotation}

\amendment{
Regs. 24, 25 revoked (3.11.08) by the Tribunals, Courts and Enforcement Act 2007 (Transitional and Consequential Provisions) Order 2008 Sch.~2.
}

% Regs 24, 25 revoked (3.11.08) by SI 2008/2683 Sch 2
%\medskip
%
%24.  In regulation 36—
%\begin{enumerate}\item[]
%($a$) in paragraph (1) the words “, including an appeal tribunal determining a misconceived appeal as a preliminary issue in accordance with regulation 48,” shall be omitted;
%
%($b$) in paragraph (2) for the words “paragraphs (3) to (5)” there shall be substituted the words “paragraphs (3) to (5), (8) and (9)”;
%
%($c$) in paragraph (2)($a$)(i)  for the words “all work test” there shall be substituted the words “personal capability assessment”;
%
%($d$) in paragraph (2)($b$)(i)  after the words “the appeal” there shall be inserted the words “(not being an appeal where the only issue is whether there should be a declaration of an industrial accident under section 29(2))”;
%
%($e$) in paragraph (5) for the words “or (3)” there shall be substituted the words “, (3) or (9)”;
%
%($f$) in paragraph (6) for the words “a disability working allowance” there shall be substituted the words “a disabled person’s tax credit”\footnote{\frenchspacing The Tax Credits Act 1999 (c. 10); section 1 and Schedule 1, paragraphs 1($b$) and 2($h$), substituted the words “disabled person’s tax credit ” for the words “disability working allowance” in section 129 of the Social Security Contributions and Benefits Act 1992 (c. 4).};
%
%($g$) in paragraph (7) for the words “all work test” there shall be substituted the words “personal capability assessment”; and
%
%($h$) after paragraph (7) there shall be added the following paragraphs—
%\begin{quotation}
%“(8) A person shall not act as a medically qualified panel member of an appeal tribunal in any appeal if he has at any time advised or prepared a report upon any person whose medical condition is relevant to the issue in the appeal, or has at any time regularly attended such a person.
%
%(9) Subject to paragraph (5), an appeal tribunal determining a misconceived appeal as a preliminary issue in accordance with regulation 48 shall consist of a legally qualified panel member.”.
%\end{quotation}
%\end{enumerate}
%
%\medskip
%
%25.  In regulation 42—
%\begin{enumerate}\item[]
%($a$) in paragraph (1)—
%\begin{enumerate}\item[]
%(i) the words “the consideration and determination of” and “before an appeal tribunal” shall be omitted; and
%
%(ii) for the words “the chairman, or in the case of an appeal tribunal which has only one member, in the opinion of that member,” there shall be substituted the words “a legally qualified panel member”; and
%\end{enumerate}
%
%($b$) in paragraph (2) for the words “the chairman, or in the case of an appeal tribunal which has only one member, that member,” there shall be substituted the words “a legally qualified panel member”.
%\end{enumerate}

\medskip

26.  In regulation 47 for the words “or regulation 48(2)” there shall be substituted the words “or 48”.

\medskip

27.  In regulation 49(10), after the words “any witness” there shall be added the words “or of any person whom the chairman, or in the case of an appeal tribunal which has only one member, that member, permits to be present in order to assist the clerk”.

\medskip

28.  In regulation 53(4)—
\begin{enumerate}\item[]
($a$) the words “a copy of” shall be omitted; and

($b$) after the words “regulation 54” there shall be added the words “and following that application the chairman or, as the case may be, that member shall record a statement of the reasons and a copy of that statement shall be sent or given to every party to the proceedings as soon as may be practicable”.
\end{enumerate}

\amendment{
Regs. 29--32 revoked (3.11.08) by the Tribunals, Courts and Enforcement Act 2007 (Transitional and Consequential Provisions) Order 2008 Sch.~2.
}

% Regs 29-32 revoked (3.11.08) by SI 2008/2683 Sch 2
%\medskip
%
%29.  In regulation 54—
%\begin{enumerate}\item[]
%($a$) in paragraph (1) the words “a copy of” shall be omitted, and after the words “are satisfied, but” there shall be inserted the words “, subject to paragraph (13),”;
%
%($b$) after paragraph (12) there shall be added the following paragraph—
%\begin{quotation}
%“(13) In calculating the time specified for applying in writing for a statement of the reasons for the tribunal’s decision there shall be disregarded any day which falls before the day on which notice was given of—
%\begin{enumerate}\item[]
%($a$) a correction of a decision or the record thereof pursuant to regulation 56; or
%
%($b$) a determination that a decision shall not be set aside following an application made under regulation 57.”.
%\end{enumerate}
%\end{quotation}
%\end{enumerate}
%
%\medskip
%
%30.  In regulation 56(1) for the words “, or where the clerk refers the matter to a legally qualified panel member, that member,” there shall be substituted the words “or a legally qualified panel member”.
%
%\medskip
%
%31.  In regulation 57—
%\begin{enumerate}\item[]
%($a$) for paragraph (3) there shall be substituted the following paragraph—
%\begin{quotation}
%“(3) An application under this regulation shall—
%\begin{enumerate}\item[]
%($a$) be made within one month of the date on which—
%\begin{enumerate}\item[]
%(i) a copy of the decision notice is sent or given to the parties to the proceedings in accordance with regulation 53(3); or
%
%(ii) the statement of the reasons for the decision is given or sent in accordance with regulation 53(4),
%\end{enumerate}
%whichever is the later;
%
%($b$) be in writing and signed by a party to the proceedings or, where the party has provided written authority to a representative to act on his behalf, that representative;
%
%($c$) contain particulars of the grounds on which it is made; and
%
%($d$) be sent to the clerk to the appeal tribunal.”; and
%\end{enumerate}
%\end{quotation}
%
%($b$) after paragraph (5) there shall be added the following paragraphs—
%\begin{quotation}
%“(6) The time within which an application under this regulation must be made may be extended by a period not exceeding one year where the conditions specified in paragraphs (7) to (11) are satisfied.
%
%(7) An application for an extension of time shall be made in accordance with paragraph (3)($b$)  to ($d$), shall include details of any relevant special circumstances for the purposes of paragraph (9) and shall be determined by a legally qualified panel member.
%
%(8) An application for an extension of time shall not be granted unless the panel member is satisfied that—
%\begin{enumerate}\item[]
%($a$) if the application is granted there are reasonable prospects that the application to set aside will be successful; and
%
%($b$) it is in the interests of justice for the application for an extension of time to be granted.
%\end{enumerate}
%
%(9) For the purposes of paragraph (8) it is not in the interests of justice to grant an application for an extension of time unless the panel member is satisfied that—
%\begin{enumerate}\item[]
%($a$) the special circumstances specified in paragraph (10) are relevant to that application; or
%
%($b$) some other special circumstances exist which are wholly exceptional and relevant to that application,
%\end{enumerate}
%and as a result of those special circumstances, it was not practicable for the application to set aside to be made within the time limit specified in paragraph (3)($a$).
%
%(10) For the purposes of paragraph (9)($a$)  the special circumstances are that—
%\begin{enumerate}\item[]
%($a$) the applicant or a spouse or dependant of the applicant has died or suffered serious illness;
%
%($b$) the applicant is not resident in the United Kingdom; or
%
%($c$) normal postal services were disrupted.
%\end{enumerate}
%
%(11) In determining whether it is in the interests of justice to grant an application for an extension of time, the panel member shall have regard to the principle that the greater the amount of time that has elapsed between the expiry of the time within which the application to set aside is to be made and the making of the application for an extension of time, the more compelling should be the special circumstances on which the application for an extension is based.
%
%(12) An application under this regulation for an extension of time which has been refused may not be renewed.”.
%\end{quotation}
%\end{enumerate}
%
%\medskip
%
%32.  After regulation 57 there shall be inserted the following regulations—
%\begin{quotation}
%\subsection*{“Provisions common to regulations 56 and 57}
%
%57A.---(1)  In calculating any time specified for appealing to a Commissioner from a decision of an appeal tribunal there shall be disregarded any day falling before the day on which notice was given of a correction of a decision or the record thereof pursuant to regulation 56 or on which notice is given of a determination that a decision shall not be set aside following an application made under regulation 57, as the case may be.
%
%(2) There shall be no appeal against a correction made under regulation 56 or a refusal to make such a correction or against a determination given under regulation 57.
%
%(3) Nothing in this Chapter shall be construed as derogating from any power to correct errors or set aside decisions which is exercisable apart from these Regulations.
%
%\subsection*{Interpretation of Chapter V}
%
%57B.  In Chapter V, except in regulation 58, “Commissioner” includes Child Support Commissioner.”.
%\end{quotation}

\medskip

33.  In regulation 58(6) after the words “of that tribunal,” shall be inserted the words “or if the President considers it necessary or expedient for the purpose of supervising panel members or in the monitoring of decision-making by panel members,”.

\medskip

34.  In Schedule 2—
\begin{enumerate}\item[]
($a$) in paragraph 5 sub-paragraph ($b$)  shall be omitted;

($b$) in paragraph 13 the words “which embodies a determination” shall be omitted; and

($c$) in paragraph 19(3) the words “which embodies a determination” shall be omitted.
\end{enumerate}

\medskip

35.  After Schedule 3 there shall be inserted the following Schedule—
\begin{quotation}
\section*{\sloppy\noindent ``Schedule 3A\\*Date on which change of circumstances takes effect in certain cases where a claimant is in receipt of income support or jobseeker’s allowance}

\subsection*{Income Support}

1.  Subject to paragraphs 2 to 6, where the amount of income support payable under an award is changed by a superseding decision made on the ground of a change of circumstances, that superseding decision shall take effect—
\begin{enumerate}\item[]
($a$) where income support is paid in arrears, from the first day of the benefit week in which the relevant change of circumstances occurs or is expected to occur; or

($b$) where income support is paid in advance, from the date of the relevant change of circumstances, or the day on which the relevant change of circumstances is expected to occur, if either of those days is the first day of the benefit week and otherwise from the next following such day,
\end{enumerate}
and for the purposes of this paragraph any period of residence in temporary accommodation under arrangements for training made under section 2 of the Employment and Training Act 1973\footnote{\frenchspacing 1973 c. 50.} or section 2 of the Enterprise and New Towns (Scotland) Act 1990\footnote{\frenchspacing 1990 c. 35.} for a period which is expected to last for seven days or less shall not be regarded as a change of circumstances.

\medskip

2.  In the cases set out in paragraph 3, the superseding decision shall take effect from the day on which the relevant change of circumstances occurs or is expected to occur.

\medskip

3.  The cases referred to in paragraph 2 are where—
\begin{enumerate}\item[]
($a$) income support is paid in arrears and entitlement ends, or is expected to end, for a reason other than that the claimant no longer satisfies the provisions of section 124(1)($b$)  of the Contributions and Benefits Act\footnote{\frenchspacing 1992 c. 4.};

($b$) a child or young person referred to in regulation 16(6) of the Income Support Regulations\footnote{\frenchspacing S.I. 1987/1967.} (child in care of local authority or detained in custody) lives, or is expected to live, with the claimant for part only of the benefit week;

($c$) a claimant or his partner (as defined in regulation 2(1) of the Income Support Regulations) enters, or is expected to enter, a nursing home or a residential care home (as defined in regulation 19(3) of those Regulations) or residential accommodation (as defined in regulation 21(3)($a$)  to ($d$)  of those Regulations) for a period of not more than 8 weeks;

($d$) a person referred to in paragraph 1, 2, 3 or 18 of Schedule 7 to the Income Support Regulations—
\begin{enumerate}\item[]
(i) ceases, or is expected to cease, to be a patient; or

(ii) a member of his family ceases, or is expected to cease, to be a patient,
\end{enumerate}
in either case for a period of less than a week;

($e$) a person referred to in paragraph 8 of Schedule 7 to the Income Support Regulations—
\begin{enumerate}\item[]
(i) ceases to be a prisoner; or

(ii) becomes a prisoner;
\end{enumerate}

($f$) a person to whom section 126 of the Contributions and Benefits Act (trade disputes) applies—
\begin{enumerate}\item[]
(i) becomes incapable of work by reason of disease or bodily or mental disablement; or

(ii) enters the maternity period (as defined in section 126(2) of that Act) or the day is known on which that person is expected to enter the maternity period;
\end{enumerate}

($g$) during the currency of the claim, a claimant makes a claim for a relevant social security benefit—
\begin{enumerate}\item[]
(i) the result of which is that his benefit week changes; or

(ii) under regulation 13 of the Claims and Payment Regulations and an award of that benefit on the relevant day for the purposes of that regulation means that his benefit week is expected to change.
\end{enumerate}
\end{enumerate}

\medskip

4.  A superseding decision made in consequence of a payment of income being treated as paid on a particular day under regulation 31(1)($b$)  or (2) or 39C(3) of the Income Support Regulations (date on which income is treated as paid) shall take effect from the day on which that payment is treated as paid.

\medskip

5.  Where—
\begin{enumerate}\item[]
($a$) it is decided upon supersession on the ground of a relevant change of circumstances that the amount of income support is, or is to be, reduced; and

($b$) the Secretary of State certifies that it is impracticable for a superseding decision to take effect from the day prescribed in the preceding paragraphs of this Schedule (other than where paragraph 3($g$)  or 4 applies),
\end{enumerate}
that superseding decision shall take effect—
\begin{enumerate}\item[]
(i) where the relevant change has occurred, from the first day of the benefit week following that in which that superseding decision is made; or

(ii) where the relevant change is expected to occur, from the first day of the benefit week following that in which that change of circumstances is expected to occur.
\end{enumerate}

\medskip

6.  Where—
\begin{enumerate}\item[]
($a$) a superseding decision (“the former supersession”) was made on the ground of a relevant change of circumstances in the cases set out in paragraphs 3($b$)  to ($g$) ; and

($b$) that superseding decision is itself superseded by a subsequent decision because the circumstances which gave rise to the former supersession cease to apply (“the second change”), that subsequent decision shall take effect from the date of the second change.
\end{enumerate}

\subsection*{Jobseeker’s Allowance}

7.  Subject to paragraphs 8 to 11, where a decision in respect of a claim for jobseeker’s allowance is superseded on the ground that there has been or there is expected to be, a relevant change of circumstances, the supersession shall take effect from the first day of the benefit week (as defined in regulation 1(3) of the Jobseeker’s Allowance Regulations) in which that relevant change of circumstances occurs or is expected to occur.

\medskip

8.  Where the relevant change of circumstances giving rise to the supersession is that—
\begin{enumerate}\item[]
($a$) entitlement to jobseeker’s allowance ends, or is expected to end, for a reason other than that the claimant no longer satisfies the provisions of section 3(1)($a$)  of the Jobseekers Act\footnote{\frenchspacing 1995 c. 18.}; or

($b$) a child or young person who is normally in the care of a local authority or who is detained in custody lives, or is expected to live, with the claimant for a part only of the benefit week; or

($c$) the claimant or his partner enters, or is expected to enter, a nursing home or residential care home for a period of not more than 8 weeks; or

($d$) the partner of the claimant or a member of his family ceases, or is expected to cease, to be a hospital in-patient for a period of less than a week,
\end{enumerate}
the supersession shall take effect from the date that the relevant change of circumstances occurs or is expected to occur.

\medskip

9.  Where the relevant change of circumstances giving rise to a supersession is any of those specified in paragraph 8, and, in consequence of those circumstances ceasing to apply, a further superseding decision is made, that further superseding decision shall take effect from the date that those circumstances ceased to apply.

\medskip

10.  Where, under the provisions of regulation 96 or 102C(3) of the Jobseeker’s Allowance Regulations\footnote{\frenchspacing S.I. 1996/207; the relevant amending instrument is S.I. 1998/1174.}, income is treated as paid on a certain date and that payment gives rise, or is expected to give rise, to a relevant change of circumstance resulting in a supersession, that supersession shall take effect from that date.

\medskip

11.  Where a relevant change of circumstances occurs which results, or is expected to result, in a reduced award of jobseeker’s allowance then, if the Secretary of State is of the opinion that it is impracticable for a supersession to take effect in accordance with the preceding paragraphs of this Schedule, the supersession shall take effect from the first day of the benefit week following that in which the relevant change of circumstances occurs.”.
\end{quotation}

\bigskip

Signed 
by authority of the Secretary of State for Social Security.

{\raggedleft
\emph{P.\ Hollis
}\\*Parliamentary Under-Secretary of State,\\*Department of Social Security

}

15th June 2000

\small

\part[Schedule --- Provisions conferring powers exercised in making these Regulations]{Schedule\\*Provisions conferring powers exercised in making these Regulations}

\renewcommand\parthead{--- Schedule}

{\footnotesize

%\begin{tabulary}{\linewidth}{JJ}
\begin{longtable}{p{181.94467pt}p{184.07935pt}}
\hline
Column (1)	& Column (2)\\
\itshape Provision	& \itshape Relevant Amendments\\
\hline
\endhead
\hline
\endlastfoot
Social Security Administration Act 1992\\
\hspace{1em}section 5(1)($b$) \\
Social Security Contributions and Benefits Act 1992\\
\hspace{1em}section 108(4)\\
\hspace{1em}section 109(2)\\
Social Security Act 1998\\
\hspace{1em}section 5(3)\\
\hspace{1em}section 7(6) and (7)\\
\hspace{1em}section 9(1) and (4)\\
\hspace{1em}section 10(3), (5) and (6)\\
\hspace{1em}section 12(1)\\
\hspace{1em}section 14(11)\\
\hspace{1em}section 16(1)\\
\hspace{1em}section 21\\
\hspace{1em}section 28(1)\\
\hspace{1em}section 32\\
\hspace{1em}section 79(1), (3), (4), (6) and (7)\\
\hspace{1em}Schedule 1, paragraphs 11 and 12\\
\hspace{1em}Schedule 2, paragraph 9\\
\hspace{1em}Schedule 3, paragraph 9\\
\hspace{1em}Schedule 5, paragraphs 1 to 4, 6 and 9\\
\pagebreak[2]
Child Support Act 1991\\
\hspace{1em}section 16(1) and (4)	&Social Security Act 1998, section 40\\
\hspace{1em}section 17(3) and (5)	&Social Security Act 1998, section 41\\
\hspace{1em}Schedule 4A, paragraphs 2, 4 and 7	&Child Support Act 1995, section 1(1) and Schedule 1\\
%\end{tabulary}
\end{longtable}

}

\part{Explanatory Note}

\renewcommand\parthead{--- Explanatory Note}

\subsection*{(This note is not part of the Regulations)}

These Regulations amend the Social Security (Industrial Injuries) (Prescribed Diseases) Regulations 1985 (“the Industrial Injuries Regulations”), the Social Security (Claims and Payments) Regulations 1987 (“the Claims and Payments Regulations”), the Child Support (Maintenance Assessment Procedure) Regulations 1992 (“the Maintenance Regulations”), the Child Support Departure Direction and Consequential Amendments Regulations 1996 (“the Departure Regulations”) and the Social Security and Child Support (Decisions and Appeals) Regulations 1999 (“the principal Regulations”).

Regulation 2 amends the Industrial Injuries Regulations in consequence of the changes to the decision-making process for social security introduced by the Social Security Act 1998 (c.\ 14) (“the Act”).

Regulation 3 amends the Claims and Payments Regulations to provide that in cases where an award of benefit has the effect of making another relevant benefit payable or payable at an increased rate, the periods of entitlement to the two benefits shall be the same in the circumstances specified.

Regulation 6 amends the Maintenance Regulations as to the definition of “official error”. Regulation 9 provides for the effective date of a superseding decision on a change of circumstances where a child ceases to be a qualifying child.

Regulation 10 amends the Departure Regulations as to the definition of “official error”. Regulation 13 provides for the effective date of a superseding decision on a change of circumstances where a child ceases to be a qualifying child.

Regulation 14 amends regulation 1(3) of the principal Regulations in particular as to the definition of “official error”. Regulation 15 amends regulation 3(7) as to the power to revise a decision. Regulation 16 amends regulation 6(2)($e$)  as to the power to supersede a decision. Regulation 17 amends regulation 7 by altering the effective date of a superseding decision in specified circumstances and transfers into these Regulations provisions previously found in the Claims and Payments Regulations concerning effective dates for income support and jobseeker’s allowance. Regulation 19 inserts a new provision to disapply section 8(2) of the Act to cases where there is a recrudescence of a prescribed disease. Regulation 20 amends regulation 16 so as to define when an appeal is pending for the purposes of section 21(3)($c$)  of the Act. Regulation 21 amends regulation 20 so as to specify when payments of benefit suspended shall be payable. Regulation 22 amends regulation 26 to provide a right of appeal against a decision made on a percentage assessment of disability in industrial injury benefit cases. Regulation 23 amends regulation 33 by providing for action on an appeal to be discontinued in certain circumstances. Regulation 24 amends regulation 36 as to the composition of an appeal tribunal. Regulation 25 amends regulation 42 to provide that the decision whether to withhold medical advice and evidence from a claimant shall be taken by a legally qualified panel member. Regulation 26 amends regulation 47 to include all appeals struck out under regulation 48. Regulation 27 amends regulation 49 by adding a category to those people entitled to be present at a hearing. Regulation 28 amends regulation 53 by specifying when statements of reasons are to be recorded. Regulation 29 amends regulation 54 to take account of the operation of regulations 56 and 57. Regulation 30 amends regulation 56 by providing that accidental errors in decisions can be corrected either by a clerk to an appeal tribunal or a legally qualified panel member. Regulation 31 amends regulation 57 by incorporating into this regulation the rules relating to the setting aside of a decision of an appeal tribunal on certain grounds. Regulation 32 inserts regulations 57A and 57B clarifying the way in which certain time limits are to be calculated and the meaning of “Commissioner” in Chapter V. Regulation 33 amends regulation 58 to specify circumstances in which an application for leave to appeal to a tribunal may be determined by a legally qualified panel member other than the person who constituted the tribunal concerned. There are also other amendments to the principal Regulations in consequence of changes in terminology effected by legislation since the principal Regulations were made.

These Regulations do not impose a charge on business. 

\end{document}