\documentclass[12pt,a4paper]{article}

\newcommand\regstitle{Child Support, Pensions and Social Security Act 2000}

\newcommand\regsnumber{c.~19}

%\opt{newrules}{
\title{\regstitle}
%}

%\opt{2012rules}{
%\title{Child Maintenance and Other Payments Act 2008\\(2012 scheme version)}
%}

\author{2000 Chapter 19}

\date{Royal Assent
28th July 2000\\
%Laid before Parliament
%27th January 2000\\
%Coming into force
%19th June 2000
}

%\opt{oldrules}{\newcommand\versionyear{1993}}
%\opt{newrules}{\newcommand\versionyear{2003}}
%\opt{2012rules}{\newcommand\versionyear{2012}}

\usepackage{csa-regs}

\setlength\headheight{27.57402pt}

\renewcommand\siprefix{\relax}

\begin{document}

\maketitle

{\noindent\large
An Act to amend the law relating to child support; to amend the law relating to occupational and personal pensions and war pensions; to amend the law relating to social security benefits and social security administration; to amend the law relating to national insurance contributions; to amend Part III of the Family Law Reform Act 1969 and Part III of the Family Law Act 1986; and for connected purposes.

}

\bigskip

\lettrine{B}{e it enacted} by the Queen’s most Excellent Majesty, by and with the advice and consent of the Lords Spiritual and Temporal, and Commons, in this present Parliament assembled, and by the authority of the same, as follows:—  

{\sloppy

\tableofcontents

}

\bigskip

\setcounter{secnumdepth}{-2}

\part[Part I --- Child Support]{Part I\\*Child Support}

\renewcommand\parthead{--- Part I}

%Maintenance calculations and interim and default maintenance decisions
%1Maintenance calculations and terminology
%
%(1) In the [1991 c. 48. ] Child Support Act 1991 (“the 1991 Act”), for section 11 (maintenance assessments) there shall be substituted—
%“11Maintenance calculations
%
%(1) An application for a maintenance calculation made to the Secretary of State shall be dealt with by him in accordance with the provision made by or under this Act.
%
%(2) The Secretary of State shall (unless he decides not to make a maintenance calculation in response to the application, or makes a decision under section 12) determine the application by making a decision under this section about whether any child support maintenance is payable and, if so, how much.
%
%(3) Where—
%
%($a$) a parent is treated under section 6(3)  as having applied for a maintenance calculation; but
%
%($b$) the Secretary of State becomes aware before determining the application that the parent has ceased to fall within section 6(1),
%
%he shall, subject to subsection (4) , cease to treat that parent as having applied for a maintenance calculation.
%
%(4) If it appears to the Secretary of State that subsection (10)  of section 4 would not have prevented the parent with care concerned from making an application for a maintenance calculation under that section he shall—
%
%($a$) notify her of the effect of this subsection; and
%
%($b$) if, before the end of the period of one month beginning with the day on which notice was sent to her, she asks him to do so, treat her as having applied not under section 6 but under section 4. 
%
%(5) Where subsection (3)  applies but subsection (4)  does not, the Secretary of State shall notify—
%
%($a$) the parent with care concerned; and
%
%($b$) the non-resident parent (or alleged non-resident parent), where it appears to him that that person is aware that the parent with care has been treated as having applied for a maintenance calculation.
%
%(6) The amount of child support maintenance to be fixed by a maintenance calculation shall be determined in accordance with Part I of Schedule 1 unless an application for a variation has been made and agreed.
%
%(7) If the Secretary of State has agreed to a variation, the amount of child support maintenance to be fixed shall be determined on the basis he determines under section 28F(4) .
%
%(8) Part II of Schedule 1 makes further provision with respect to maintenance calculations.”
%
%(2) In the 1991 Act—
%
%($a$) for “maintenance assessment”, wherever it occurs, there shall be substituted “maintenance calculation”; and
%
%($b$) for “assessment” (or any variant of that term), wherever it occurs, there shall be substituted “calculation” (or the corresponding variant) preceded, where appropriate, by “a” instead of “an”.
%
%(3) For Part I of Schedule 1 to the 1991 Act, there shall be substituted the Part I set out in Schedule 1 to this Act.
%2Applications under section 4 of the Child Support Act 1991
%
%(1) In section 4 of the 1991 Act (child support maintenance), subsection (10)  shall be amended as follows.
%
%(2) In paragraph ($a$), after “maintenance order” there shall be inserted “made before a prescribed date”.
%
%(3) After paragraph ($a$), there shall be inserted—
%
%“($aa$) a maintenance order made on or after the date prescribed for the purposes of paragraph ($a$)  is in force in respect of them, but has been so for less than the period of one year beginning with the date on which it was made; or”.
%3Applications by persons claiming or receiving benefit
%
%For section 6 of the 1991 Act (applications by those receiving benefit) there shall be substituted—
%“6Applications by those claiming or receiving benefit
%
%(1) This section applies where income support, an income-based jobseeker’s allowance or any other benefit of a prescribed kind is claimed by or in respect of, or paid to or in respect of, the parent of a qualifying child who is also a person with care of the child.
%
%(2) In this section, that person is referred to as “the parent”.
%
%(3) The Secretary of State may—
%
%($a$) treat the parent as having applied for a maintenance calculation with respect to the qualifying child and all other children of the non-resident parent in relation to whom the parent is also a person with care; and
%
%($b$) take action under this Act to recover from the non-resident parent, on the parent’s behalf, the child support maintenance so determined.
%
%(4) Before doing what is mentioned in subsection (3), the Secretary of State must notify the parent in writing of the effect of subsections (3)  and (5)  and section 46. 
%
%(5) The Secretary of State may not act under subsection (3)  if the parent asks him not to (a request which need not be in writing).
%
%(6) Subsection (1)  has effect regardless of whether any of the benefits mentioned there is payable with respect to any qualifying child.
%
%(7) Unless she has made a request under subsection (5) , the parent shall, so far as she reasonably can, comply with such regulations as may be made by the Secretary of State with a view to the Secretary of State’s being provided with the information which is required to enable—
%
%($a$) the non-resident parent to be identified or traced;
%
%($b$) the amount of child support maintenance payable by him to be calculated; and
%
%($c$) that amount to be recovered from him.
%
%(8) The obligation to provide information which is imposed by subsection (7) —
%
%($a$) does not apply in such circumstances as may be prescribed; and
%
%($b$) may, in such circumstances as may be prescribed, be waived by the Secretary of State.
%
%(9) If the parent ceases to fall within subsection (1), she may ask the Secretary of State to cease acting under this section, but until then he may continue to do so.
%
%(10) The Secretary of State must comply with any request under subsection (9)  (but subject to any regulations made under subsection (11) ).
%
%(11) The Secretary of State may by regulations make such incidental or transitional provision as he thinks appropriate with respect to cases in which he is asked under subsection (9)  to cease to act under this section.
%
%(12) The fact that a maintenance calculation is in force with respect to a person with care does not prevent the making of a new maintenance calculation with respect to her as a result of the Secretary of State’s acting under subsection (3) .”
%4Default and interim maintenance decisions
%
%For section 12 of the 1991 Act (interim maintenance assessments) there shall be substituted—
%“12Default and interim maintenance decisions
%
%(1) Where the Secretary of State—
%
%($a$) is required to make a maintenance calculation; or
%
%($b$) is proposing to make a decision under section 16 or 17,
%
%and it appears to him that he does not have sufficient information to enable him to do so, he may make a default maintenance decision.
%
%(2) Where an application for a variation has been made under section 28A(1)  in connection with an application for a maintenance calculation (or in connection with such an application which is treated as having been made), the Secretary of State may make an interim maintenance decision.
%
%(3) The amount of child support maintenance fixed by an interim maintenance decision shall be determined in accordance with Part I of Schedule 1. 
%
%(4) The Secretary of State may by regulations make provision as to default and interim maintenance decisions.
%
%(5) The regulations may, in particular, make provision as to—
%
%($a$) the procedure to be followed in making a default or an interim maintenance decision; and
%
%($b$) a default rate of child support maintenance to apply where a default maintenance decision is made.”
%Applications for a variation
%5Departure from usual rules for calculating maintenance
%
%(1) The 1991 Act shall be amended as follows.
%
%(2) For sections 28A to 28C (which deal respectively with applications for departure directions, their preliminary consideration, and the imposition of a regular payments condition) there shall be substituted—
%“Variations
%28AApplication for variation of usual rules for calculating maintenance
%
%(1) Where an application for a maintenance calculation is made under section 4 or 7, or treated as made under section 6, the person with care or the non-resident parent or (in the case of an application under section 7) either of them or the child concerned may apply to the Secretary of State for the rules by which the calculation is made to be varied in accordance with this Act.
%
%(2) Such an application is referred to in this Act as an “application for a variation”.
%
%(3) An application for a variation may be made at any time before the Secretary of State has reached a decision (under section 11 or 12(1) ) on the application for a maintenance calculation (or the application treated as having been made under section 6).
%
%(4) A person who applies for a variation—
%
%($a$) need not make the application in writing unless the Secretary of State directs in any case that he must; and
%
%($b$) must say upon what grounds the application is made.
%
%(5) In other respects an application for a variation is to be made in such manner as may be prescribed.
%
%(6) Schedule 4A has effect in relation to applications for a variation.
%28BPreliminary consideration of applications
%
%(1) Where an application for a variation has been duly made to the Secretary of State, he may give it a preliminary consideration.
%
%(2) Where he does so he may, on completing the preliminary consideration, reject the application (and proceed to make his decision on the application for a maintenance calculation without any variation) if it appears to him—
%
%($a$) that there are no grounds on which he could agree to a variation;
%
%($b$) that he has insufficient information to make a decision on the application for the maintenance calculation under section 11 (apart from any information needed in relation to the application for a variation), and therefore that his decision would be made under section 12(1) ; or
%
%($c$) that other prescribed circumstances apply.
%28CImposition of regular payments condition
%
%(1) Where—
%
%($a$) an application for a variation is made by the non-resident parent; and
%
%($b$) the Secretary of State makes an interim maintenance decision,
%
%the Secretary of State may also, if he has completed his preliminary consideration (under section 28B) of the application for a variation and has not rejected it under that section, impose on the non-resident parent one of the conditions mentioned in subsection (2)  (a “regular payments condition”).
%
%(2) The conditions are that—
%
%($a$) the non-resident parent must make the payments of child support maintenance specified in the interim maintenance decision;
%
%($b$) the non-resident parent must make such lesser payments of child support maintenance as may be determined in accordance with regulations made by the Secretary of State.
%
%(3) Where the Secretary of State imposes a regular payments condition, he shall give written notice of the imposition of the condition and of the effect of failure to comply with it to—
%
%($a$) the non-resident parent;
%
%($b$) all the persons with care concerned; and
%
%($c$) if the application for the maintenance calculation was made under section 7, the child who made the application.
%
%(4) A regular payments condition shall cease to have effect—
%
%($a$) when the Secretary of State has made a decision on the application for a maintenance calculation under section 11 (whether he agrees to a variation or not);
%
%($b$) on the withdrawal of the application for a variation.
%
%(5) Where a non-resident parent has failed to comply with a regular payments condition, the Secretary of State may in prescribed circumstances refuse to consider the application for a variation, and instead reach his decision under section 11 as if no such application had been made.
%
%(6) The question whether a non-resident parent has failed to comply with a regular payments condition is to be determined by the Secretary of State.
%
%(7) Where the Secretary of State determines that a non-resident parent has failed to comply with a regular payments condition he shall give written notice of his determination to—
%
%($a$) that parent;
%
%($b$) all the persons with care concerned; and
%
%($c$) if the application for the maintenance calculation was made under section 7, the child who made the application.”
%
%(3) In section 28D (determination of applications)—
%
%($a$) for subsection (1)  there shall be substituted—
%
%“(1) Where an application for a variation has not failed, the Secretary of State shall, in accordance with the relevant provisions of, or made under, this Act—
%
%($a$) either agree or not to a variation, and make a decision under section 11 or 12(1) ; or
%
%($b$) refer the application to an appeal tribunal for the tribunal to determine what variation, if any, is to be made.”;
%
%($b$) in each of subsections (2)  and (3), for “an application for a departure direction” there shall be substituted “an application for a variation”; and
%
%($c$) in subsection (2) , in paragraph ($a$)  “lapsed or” shall be omitted, at the end of paragraph ($b$)  “or” shall be inserted, and after that paragraph there shall be inserted—
%
%“($c$) the Secretary of State has refused to consider it under section 28C(5).”
%
%(4) In section 28E (matters to be taken into account)—
%
%($a$) in subsections (1), (3)  and (4) , for “any application for a departure direction” (wherever appearing) there shall be substituted “whether to agree to a variation”; and
%
%($b$) in subsection (4)($a$), for “a departure direction were made” there shall be substituted “the Secretary of State agreed to a variation”.
%
%(5) For section 28F (departure directions) there shall be substituted—
%“28FAgreement to a variation
%
%(1) The Secretary of State may agree to a variation if—
%
%($a$) he is satisfied that the case is one which falls within one or more of the cases set out in Part I of Schedule 4B or in regulations made under that Part; and
%
%($b$) it is his opinion that, in all the circumstances of the case, it would be just and equitable to agree to a variation.
%
%(2) In considering whether it would be just and equitable in any case to agree to a variation, the Secretary of State—
%
%($a$) must have regard, in particular, to the welfare of any child likely to be affected if he did agree to a variation; and
%
%($b$) must, or as the case may be must not, take any prescribed factors into account, or must take them into account (or not) in prescribed circumstances.
%
%(3) The Secretary of State shall not agree to a variation (and shall proceed to make his decision on the application for a maintenance calculation without any variation) if he is satisfied that—
%
%($a$) he has insufficient information to make a decision on the application for the maintenance calculation under section 11, and therefore that his decision would be made under section 12(1) ; or
%
%($b$) other prescribed circumstances apply.
%
%(4) Where the Secretary of State agrees to a variation, he shall—
%
%($a$) determine the basis on which the amount of child support maintenance is to be calculated in response to the application for a maintenance calculation (including an application treated as having been made); and
%
%($b$) make a decision under section 11 on that basis.
%
%(5) If the Secretary of State has made an interim maintenance decision, it is to be treated as having been replaced by his decision under section 11, and except in prescribed circumstances any appeal connected with it (under section 20) shall lapse.
%
%(6) In determining whether or not to agree to a variation, the Secretary of State shall comply with regulations made under Part II of Schedule 4B.”
%6Applications for a variation: further provisions
%
%(1) For Schedule 4A to the 1991 Act there shall be substituted the Schedule 4A set out in Part I of Schedule 2. 
%
%(2) For Schedule 4B to that Act there shall be substituted the Schedule 4B set out in Part II of Schedule 2. 
%7Variations: revision and supersession
%
%For section 28G of the 1991 Act (effect and duration of departure directions) there shall be substituted—
%“28GVariations: revision and supersession
%
%(1) An application for a variation may also be made when a maintenance calculation is in force.
%
%(2) The Secretary of State may by regulations provide for—
%
%($a$) sections 16, 17 and 20; and
%
%($b$) sections 28A to 28F and Schedules 4A and 4B,
%
%to apply with prescribed modifications in relation to such applications.
%
%(3) The Secretary of State may by regulations provide that, in prescribed cases (or except in prescribed cases), a decision under section 17 made otherwise than pursuant to an application for a variation may be made on the basis of a variation agreed to for the purposes of an earlier decision without a new application for a variation having to be made.”
%Revision and supersession of decisions
%8Revision of decisions
%
%(1) Section 16 of the 1991 Act (revision of decisions) shall be amended as follows.
%
%(2) In subsection (1), for “of the Secretary of State under section 11, 12 or 17” there shall be substituted “to which subsection (1A) applies”.
%
%(3) After subsection (1), there shall be inserted—
%
%“(1A)This subsection applies to—
%
%($a$) a decision of the Secretary of State under section 11, 12 or 17;
%
%($b$) a reduced benefit decision under section 46;
%
%($c$) a decision of an appeal tribunal on a referral under section 28D(1)($b$) .
%
%(1B)Where the Secretary of State revises a decision under section 12(1) —
%
%($a$) he may (if appropriate) do so as if he were revising a decision under section 11; and
%
%($b$) if he does that, his decision as revised is to be treated as one under section 11 instead of section 12(1)  (and, in particular, is to be so treated for the purposes of an appeal against it under section 20).”
%9Decisions superseding earlier decisions
%
%(1) Section 17 of the 1991 Act (decisions superseding earlier decisions) shall be amended as follows.
%
%(2) In subsection (1), for paragraph ($c$)  there shall be substituted—
%
%“($c$) any reduced benefit decision under section 46;
%
%($d$) any decision of an appeal tribunal on a referral under section 28D(1)($b$) ;
%
%($e$) any decision of a Child Support Commissioner on an appeal from such a decision as is mentioned in paragraph ($b$)  or ($d$) .”
%
%(3) For subsection (4)  there shall be substituted—
%
%“(4) Subject to subsection (5)  and section 28ZC, a decision under this section shall take effect as from the beginning of the maintenance period in which it is made or, where applicable, the beginning of the maintenance period in which the application was made.
%
%(4A) In subsection (4) , a “maintenance period” is (except where a different meaning is prescribed for prescribed cases) a period of seven days, the first one beginning on the effective date of the first decision made by the Secretary of State under section 11 or (if earlier) his first default or interim maintenance decision (under section 12) in relation to the non-resident parent in question, and each subsequent one beginning on the day after the last day of the previous one.”
%Appeals
%10Appeals to appeal tribunals
%
%For section 20 of the 1991 Act (appeals to appeal tribunals) there shall be substituted—
%“20Appeals to appeal tribunals
%
%(1) A qualifying person has a right of appeal to an appeal tribunal against—
%
%($a$) a decision of the Secretary of State under section 11, 12 or 17 (whether as originally made or as revised under section 16);
%
%($b$) a decision of the Secretary of State not to make a maintenance calculation under section 11 or not to supersede a decision under section 17;
%
%($c$) a reduced benefit decision under section 46;
%
%($d$) the imposition (by virtue of section 41A) of a requirement to make penalty payments, or their amount;
%
%($e$) the imposition (by virtue of section 47) of a requirement to pay fees.
%
%(2) In subsection (1), “qualifying person” means—
%
%($a$) in relation to paragraphs ($a$)  and ($b$) —
%
%(i) the person with care, or non-resident parent, with respect to whom the Secretary of State made the decision, or
%
%(ii) in a case relating to a maintenance calculation which was applied for under section 7, either of those persons or the child concerned;
%
%($b$) in relation to paragraph ($c$), the person in respect of whom the benefits are payable;
%
%($c$) in relation to paragraph ($d$), the parent who has been required to make penalty payments; and
%
%($d$) in relation to paragraph ($e$), the person required to pay fees.
%
%(3) A person with a right of appeal under this section shall be given such notice as may be prescribed of—
%
%($a$) that right; and
%
%($b$) the relevant decision, or the imposition of the requirement.
%
%(4) Regulations may make—
%
%($a$) provision as to the manner in which, and the time within which, appeals are to be brought; and
%
%($b$) such provision with respect to proceedings before appeal tribunals as the Secretary of State considers appropriate.
%
%(5) The regulations may in particular make any provision of a kind mentioned in Schedule 5 to the [1998 c. 14. ] Social Security Act 1998. 
%
%(6) No appeal lies by virtue of subsection (1)($c$)  unless the amount of the person’s benefit is reduced in accordance with the reduced benefit decision; and the time within which such an appeal may be brought runs from the date of notification of the reduction.
%
%(7) In deciding an appeal under this section, an appeal tribunal—
%
%($a$) need not consider any issue that is not raised by the appeal; and
%
%($b$) shall not take into account any circumstances not obtaining at the time when the Secretary of State made the decision or imposed the requirement.
%
%(8) If an appeal under this section is allowed, the appeal tribunal may—
%
%($a$) itself make such decision as it considers appropriate; or
%
%($b$) remit the case to the Secretary of State, together with such directions (if any) as it considers appropriate.”
%11Redetermination of appeals
%
%After section 23 of the 1991 Act there shall be inserted—
%“23ARedetermination of appeals
%
%(1) This section applies where an application is made to a person under section 24(6) ($a$)  for leave to appeal from a decision of an appeal tribunal.
%
%(2) If the person who constituted, or was the chairman of, the appeal tribunal considers that the decision was erroneous in law, he may set aside the decision and refer the case either for redetermination by the tribunal or for determination by a differently constituted tribunal.
%
%(3) If each of the principal parties to the case expresses the view that the decision was erroneous in point of law, the person shall set aside the decision and refer the case for determination by a differently constituted tribunal.
%
%(4) The “principal parties” are—
%
%($a$) the Secretary of State; and
%
%($b$) those who are qualifying persons for the purposes of section 20(2)  in relation to the decision in question.”
%Information
%12Information required by Secretary of State
%
%In section 14 of the 1991 Act (information required by the Secretary of State), in subsection (1), after “such an application” there shall be inserted “(or application treated as made), or needed for the making of any decision or in connection with the imposition of any condition or requirement under this Act,”.
%13Information— offences
%
%After section 14 of the 1991 Act there shall be inserted—
%“14AInformation —offences
%
%(1) This section applies to—
%
%($a$) persons who are required to comply with regulations under section 4(4)  or 7(5) ; and
%
%($b$) persons specified in regulations under section 14(1)($a$) .
%
%(2) Such a person is guilty of an offence if, pursuant to a request for information under or by virtue of those regulations—
%
%($a$) he makes a statement or representation which he knows to be false; or
%
%($b$) he provides, or knowingly causes or knowingly allows to be provided, a document or other information which he knows to be false in a material particular.
%
%(3) Such a person is guilty of an offence if, following such a request, he fails to comply with it.
%
%(4) It is a defence for a person charged with an offence under subsection (3)  to prove that he had a reasonable excuse for failing to comply.
%
%(5) A person guilty of an offence under this section is liable on summary conviction to a fine not exceeding level 3 on the standard scale.”
%14Inspectors
%
%(1) Section 15 of the 1991 Act (powers of inspectors) shall be amended as follows.
%
%(2) For subsections (1)  to (4)  there shall be substituted—
%
%“(1) The Secretary of State may appoint, on such terms as he thinks fit, persons to act as inspectors under this section.
%
%(2) The function of inspectors is to acquire information which the Secretary of State needs for any of the purposes of this Act.
%
%(3) Every inspector is to be given a certificate of his appointment.
%
%(4) An inspector has power, at any reasonable time and either alone or accompanied by such other persons as he thinks fit, to enter any premises which—
%
%($a$) are liable to inspection under this section; and
%
%($b$) are premises to which it is reasonable for him to require entry in order that he may exercise his functions under this section,
%
%and may there make such examination and inquiry as he considers appropriate.
%
%(4A) Premises liable to inspection under this section are those which are not used wholly as a dwelling house and which the inspector has reasonable grounds for suspecting are—
%
%($a$) premises at which a non-resident parent is or has been employed;
%
%($b$) premises at which a non-resident parent carries out, or has carried out, a trade, profession, vocation or business;
%
%($c$) premises at which there is information held by a person (“A”) whom the inspector has reasonable grounds for suspecting has information about a non-resident parent acquired in the course of A’s own trade, profession, vocation or business.”
%
%(3) In subsection (6) , for the words from “any person who” to the end of paragraph ($d$)  there shall be substituted “any such person”.
%
%(4) After subsection (10)  there shall be inserted—
%
%“(11) In this section, “premises” includes—
%
%($a$) moveable structures and vehicles, vessels, aircraft and hovercraft;
%
%($b$) installations that are offshore installations for the purposes of the [1971 c. 61. ] Mineral Workings (Offshore Installations) Act 1971; and
%
%($c$) places of all other descriptions whether or not occupied as land or otherwise,
%
%and references in this section to the occupier of premises are to be construed, in relation to premises that are not occupied as land, as references to any person for the time being present at the place in question.”
%Parentage
%15Presumption of parentage in child support cases
%
%(1) In section 26(2)  of the 1991 Act (cases in which the Secretary of State may assume a person to be the parent of a child for the purpose of making a maintenance calculation under that Act), before Case A there shall be inserted—
%“Case A1
%
%Where—
%
%($a$) the child is habitually resident in England and Wales;
%
%($b$) the Secretary of State is satisfied that the alleged parent was married to the child’s mother at some time in the period beginning with the conception and ending with the birth of the child; and
%
%($c$) the child has not been adopted.
%Case A2
%
%Where—
%
%($a$) the child is habitually resident in England and Wales;
%
%($b$) the alleged parent has been registered as father of the child under section 10 or 10A of the [1953 c. 20. ] Births and Deaths Registration Act 1953, or in any register kept under section 13 (register of births and still-births) or section 44 (Register of Corrections Etc) of the [1965 c. 49. ] Registration of Births, Deaths and Marriages (Scotland) Act 1965, or under Article 14 or 18(1)($b$)(ii)  of the [S.I. 1976/1041 (N.I. 14).] Births and Deaths Registration (Northern Ireland) Order 1976; and
%
%($c$) the child has not subsequently been adopted.
%Case A3
%
%Where the result of a scientific test (within the meaning of section 27A) taken by the alleged parent would be relevant to determining the child’s parentage, and the alleged parent—
%
%($a$) refuses to take such a test; or
%
%($b$) has submitted to such a test, and it shows that there is no reasonable doubt that the alleged parent is a parent of the child.”
%
%(2) In that provision, after Case B there shall be inserted—
%“Case B1
%
%Where the Secretary of State is satisfied that the alleged parent is a parent of the child in question by virtue of section 27 or 28 of that Act (meaning of “mother” and of “father” respectively).”
%Disqualification from driving
%16Disqualification from driving
%
%(1) After section 39 of the 1991 Act there shall be inserted—
%“39ACommitment to prison and disqualification from driving
%
%(1) Where the Secretary of State has sought—
%
%($a$) in England and Wales to levy an amount by distress under this Act; or
%
%($b$) to recover an amount by virtue of section 36 or 38,
%
%and that amount, or any portion of it, remains unpaid he may apply to the court under this section.
%
%(2) An application under this section is for whichever the court considers appropriate in all the circumstances of—
%
%($a$) the issue of a warrant committing the liable person to prison; or
%
%($b$) an order for him to be disqualified from holding or obtaining a driving licence.
%
%(3) On any such application the court shall (in the presence of the liable person) inquire as to—
%
%($a$) whether he needs a driving licence to earn his living;
%
%($b$) his means; and
%
%($c$) whether there has been wilful refusal or culpable neglect on his part.
%
%(4) The Secretary of State may make representations to the court as to whether he thinks it more appropriate to commit the liable person to prison or to disqualify him from holding or obtaining a driving licence; and the liable person may reply to those representations.
%
%(5) In this section and section 40B, “driving licence” means a licence to drive a motor vehicle granted under Part III of the [1988 c. 52. ] Road Traffic Act 1988. 
%
%(6) In this section “the court” means—
%
%($a$) in England and Wales, a magistrates' court;
%
%($b$) in Scotland, the sheriff.”
%
%(2) In section 40 of the 1991 Act (commitment to prison), subsections (1)  and (2)  shall be omitted.
%
%(3) Before section 41 of the 1991 Act there shall be inserted—
%“40BDisqualification from driving: further provision
%
%(1) If, but only if, the court is of the opinion that there has been wilful refusal or culpable neglect on the part of the liable person, it may—
%
%($a$) order him to be disqualified, for such period specified in the order but not exceeding two years as it thinks fit, from holding or obtaining a driving licence (a “disqualification order”); or
%
%($b$) make a disqualification order but suspend its operation until such time and on such conditions (if any) as it thinks just.
%
%(2) The court may not take action under both section 40 and this section.
%
%(3) A disqualification order must state the amount in respect of which it is made, which is to be the aggregate of—
%
%($a$) the amount mentioned in section 35(1), or so much of it as remains outstanding; and
%
%($b$) an amount (determined in accordance with regulations made by the Secretary of State) in respect of the costs of the application under section 39A.
%
%(4) A court which makes a disqualification order shall require the person to whom it relates to produce any driving licence held by him, and its counterpart (within the meaning of section 108(1)  of the [1988 c. 52. ] Road Traffic Act 1988).
%
%(5) On an application by the Secretary of State or the liable person, the court—
%
%($a$) may make an order substituting a shorter period of disqualification, or make an order revoking the disqualification order, if part of the amount referred to in subsection (3)  (the “amount due”) is paid to any person authorised to receive it; and
%
%($b$) must make an order revoking the disqualification order if all of the amount due is so paid.
%
%(6) The Secretary of State may make representations to the court as to the amount which should be paid before it would be appropriate to make an order revoking the disqualification order under subsection (5)($a$), and the person liable may reply to those representations.
%
%(7) The Secretary of State may make a further application under section 39A if the amount due has not been paid in full when the period of disqualification specified in the disqualification order expires.
%
%(8) Where a court—
%
%($a$) makes a disqualification order;
%
%($b$) makes an order under subsection (5) ; or
%
%($c$) allows an appeal against a disqualification order,
%
%it shall send notice of that fact to the Secretary of State; and the notice shall contain such particulars and be sent in such manner and to such address as the Secretary of State may determine.
%
%(9) Where a court makes a disqualification order, it shall also send the driving licence and its counterpart, on their being produced to the court, to the Secretary of State at such address as he may determine.
%
%(10) Section 80 of the [1980 c. 43. ] Magistrates' Courts Act 1980 (application of money found on defaulter) shall apply in relation to a disqualification order under this section in relation to a liable person as it applies in relation to the enforcement of a sum mentioned in subsection (1)  of that section.
%
%(11) The Secretary of State may by regulations make provision in relation to disqualification orders corresponding to the provision he may make under section 40(11) .
%
%(12) In the application to Scotland of this section—
%
%($a$) in subsection (2)  for “section 40” substitute “section 40A”;
%
%($b$) in subsection (3)  for paragraph ($a$)  substitute—
%
%“($a$) the appropriate amount under section 38;”;
%
%($c$) subsection (10)  is omitted; and
%
%($d$) for subsection (11)  substitute—
%
%“(11) The power of the Court of Session by Act of Sederunt to regulate the procedure and practice in civil proceedings in the sheriff court shall include power to make, in relation to disqualification orders, provision corresponding to that which may be made by virtue of section 40A(8) .””
%
%(4) In section 164(5)  of the [1988 c. 52. ] Road Traffic Act 1988 (power of constables to require production of driving licence etc.), after “Road Traffic Offenders Act 1988” there shall be inserted “, section 40B of the Child Support Act 1991”.
%
%(5) In section 27(3)  of the [1988 c. 53. ] Road Traffic Offenders Act 1988 (offence of failing to produce a licence), for the word “then,” there shall be substituted “, or if the holder of the licence does not produce it and its counterpart as required by section 40B of the [1991 c. 48. ] Child Support Act 1991, then,”.
%17Civil imprisonment: Scotland
%
%(1) In section 40 of the 1991 Act (commitment to prison), for subsections (12)  to (14)  there shall be substituted—
%
%“(12) This section does not apply to Scotland.”
%
%(2) After section 40 there shall be inserted—
%“40ACommitment to prison: Scotland
%
%(1) If, but only if, the sheriff is satisfied that there has been wilful refusal or culpable neglect on the part of the liable person he may—
%
%($a$) issue a warrant for his committal to prison; or
%
%($b$) fix a term of imprisonment and postpone the issue of the warrant until such time and on such conditions (if any) as he thinks just.
%
%(2) A warrant under this section—
%
%($a$) shall be made in respect of an amount equal to the aggregate of—
%
%(i) the appropriate amount under section 38; and
%
%(ii) an amount (determined in accordance with regulations made by the Secretary of State) in respect of the expenses of commitment; and
%
%($b$) shall state that amount.
%
%(3) No warrant may be issued under this section against a person who is under the age of 18. 
%
%(4) A warrant issued under this section shall order the liable person—
%
%($a$) to be imprisoned for a specified period; but
%
%($b$) to be released (unless he is in custody for some other reason) on payment of the amount stated in the warrant.
%
%(5) The maximum period of imprisonment which may be imposed by virtue of subsection (4)  is six weeks.
%
%(6) The Secretary of State may by regulations make provision for the period of imprisonment specified in any warrant issued under this section to be reduced where there is part payment of the amount in respect of which the warrant was issued.
%
%(7) A warrant issued under this section may be directed to such person as the sheriff thinks fit.
%
%(8) The power of the Court of Session by Act of Sederunt to regulate the procedure and practice in civil proceedings in the sheriff court shall include power to make provision—
%
%($a$) as to the form of any warrant issued under this section;
%
%($b$) allowing an application under this section to be renewed where no warrant is issued or term of imprisonment is fixed;
%
%($c$) that a statement in writing to the effect that wages of any amount have been paid to the liable person during any period, purporting to be signed by or on behalf of his employer, shall be sufficient evidence of the facts stated;
%
%($d$) that, for the purposes of enabling an inquiry to be made as to the liable person’s conduct and means, the sheriff may issue a citation to him to appear before the sheriff and (if he does not obey) may issue a warrant for his arrest;
%
%($e$) that for the purpose of enabling such an inquiry, the sheriff may issue a warrant for the liable person’s arrest without issuing a citation;
%
%($f$) as to the execution of a warrant of arrest.”
%Financial penalties
%18Financial penalties
%
%(1) In section 41 of the 1991 Act (arrears of child support maintenance), subsections (3)  to (5)  (which provide for the payment of interest on arrears) shall cease to have effect.
%
%(2) For section 41A of the 1991 Act (arrears: alternative to interest payments) there shall be substituted—
%“41APenalty payments
%
%(1) The Secretary of State may by regulations make provision for the payment to him by non-resident parents who are in arrears with payments of child support maintenance of penalty payments determined in accordance with the regulations.
%
%(2) The amount of a penalty payment in respect of any week may not exceed 25% of the amount of child support maintenance payable for that week, but otherwise is to be determined by the Secretary of State.
%
%(3) The liability of a non-resident parent to make a penalty payment does not affect his liability to pay the arrears of child support maintenance concerned.
%
%(4) Regulations under subsection (1)  may, in particular, make provision—
%
%($a$) as to the time at which a penalty payment is to be payable;
%
%($b$) for the Secretary of State to waive a penalty payment, or part of it.
%
%(5) The provisions of this Act with respect to—
%
%($a$) the collection of child support maintenance;
%
%($b$) the enforcement of an obligation to pay child support maintenance,
%
%apply equally (with any necessary modifications) to penalty payments payable by virtue of regulations under this section.
%
%(6) The Secretary of State shall pay penalty payments received by him into the Consolidated Fund.”
%19Reduced benefit decisions
%
%For section 46 of the 1991 Act (failure to comply with obligations imposed by section 6) there shall be substituted—
%“46Reduced benefit decisions
%
%(1) This section applies where any person (“the parent”)—
%
%($a$) has made a request under section 6(5) ;
%
%($b$) fails to comply with any regulation made under section 6(7) ; or
%
%($c$) having been treated as having applied for a maintenance calculation under section 6, refuses to take a scientific test (within the meaning of section 27A).
%
%(2) The Secretary of State may serve written notice on the parent requiring her, before the end of a specified period—
%
%($a$) in a subsection (1)($a$)  case, to give him her reasons for making the request;
%
%($b$) in a subsection (1)($b$)  case, to give him her reasons for failing to do so; or
%
%($c$) in a subsection (1)($c$)  case, to give him her reasons for her refusal.
%
%(3) When the specified period has expired, the Secretary of State shall consider whether, having regard to any reasons given by the parent, there are reasonable grounds for believing that—
%
%($a$) in a subsection (1)($a$)  case, if the Secretary of State were to do what is mentioned in section 6(3) ;
%
%($b$) in a subsection (1)($b$)  case, if she were to be required to comply; or
%
%($c$) in a subsection (1)($c$)  case, if she took the scientific test,
%
%there would be a risk of her, or of any children living with her, suffering harm or undue distress as a result of his taking such action, or her complying or taking the test.
%
%(4) If the Secretary of State considers that there are such reasonable grounds, he shall—
%
%($a$) take no further action under this section in relation to the request, the failure or the refusal in question; and
%
%($b$) notify the parent, in writing, accordingly.
%
%(5) If the Secretary of State considers that there are no such reasonable grounds, he may, except in prescribed circumstances, make a reduced benefit decision with respect to the parent.
%
%(6) In a subsection (1)($a$)  case, the Secretary of State may from time to time serve written notice on the parent requiring her, before the end of a specified period—
%
%($a$) to state whether her request under section 6(5)  still stands; and
%
%($b$) if so, to give him her reasons for maintaining her request,
%
%and subsections (3)  to (5)  have effect in relation to such a notice and any response to it as they have effect in relation to a notice under subsection (2)($a$)  and any response to it.
%
%(7) Where the Secretary of State makes a reduced benefit decision he must send a copy of it to the parent.
%
%(8) A reduced benefit decision is to take effect on such date as may be specified in the decision.
%
%(9) Reasons given in response to a notice under subsection (2)  or (6)  need not be given in writing unless the Secretary of State directs in any case that they must.
%
%(10) In this section—
%
%($a$) “comply” means to comply with the requirement or with the regulation in question; and “complied” and “complying” are to be construed accordingly;
%
%($b$) “reduced benefit decision” means a decision that the amount payable by way of any relevant benefit to, or in respect of, the parent concerned be reduced by such amount, and for such period, as may be prescribed;
%
%($c$) “relevant benefit” means income support or an income-based jobseeker’s allowance or any other benefit of a kind prescribed for the purposes of section 6; and
%
%($d$) “specified”, in relation to a notice served under this section, means specified in the notice; and the period to be specified is to be determined in accordance with regulations made by the Secretary of State.”

\amendment{
Ss. 1--19 are not yet in force.
}

\section{Miscellaneous}

%20Voluntary payments
%
%(1) After section 28I of the 1991 Act there shall be inserted—
%“Voluntary payments
%28JVoluntary payments
%
%(1) This section applies where—
%
%($a$) a person has applied for a maintenance calculation under section 4(1)  or 7(1), or is treated as having applied for one by virtue of section 6;
%
%($b$) the Secretary of State has neither made a decision under section 11 or 12 on the application, nor decided not to make a maintenance calculation; and
%
%($c$) the non-resident parent makes a voluntary payment.
%
%(2) A “voluntary payment” is a payment—
%
%($a$) on account of child support maintenance which the non-resident parent expects to become liable to pay following the determination of the application (whether or not the amount of the payment is based on any estimate of his potential liability which the Secretary of State has agreed to give); and
%
%($b$) made before the maintenance calculation has been notified to the non-resident parent or (as the case may be) before the Secretary of State has notified the non-resident parent that he has decided not to make a maintenance calculation.
%
%(3) In such circumstances and to such extent as may be prescribed—
%
%($a$) the voluntary payment may be set off against arrears of child support maintenance which accrued by virtue of the maintenance calculation taking effect on a date earlier than that on which it was notified to the non-resident parent;
%
%($b$) the amount payable under a maintenance calculation may be adjusted to take account of the voluntary payment.
%
%(4) A voluntary payment shall be made to the Secretary of State unless he agrees, on such conditions as he may specify, that it may be made to the person with care, or to or through another person.
%
%(5) The Secretary of State may by regulations make provision as to voluntary payments, and the regulations may in particular—
%
%($a$) prescribe what payments or descriptions of payment are, or are not, to count as “voluntary payments”;
%
%($b$) prescribe the extent to which and circumstances in which a payment, or a payment of a prescribed description, counts.”
%
%(2) Section 41B of the 1991 Act (repayment of overpaid child support maintenance) shall be amended as follows.
%
%(3) After subsection (1)  there shall be inserted—
%
%“(1A)This section also applies where the non-resident parent has made a voluntary payment and it appears to the Secretary of State—
%
%($a$) that he is not liable to pay child support maintenance; or
%
%($b$) that he is liable, but some or all of the payment amounts to an overpayment,
%
%and, in a case falling within paragraph ($b$), it also appears to him that subsection (1)($a$)  or ($b$)  applies.”
%
%(4) For subsection (7)  there shall be substituted—
%
%“(7) For the purposes of this section—
%
%($a$) a payment made by a person under a maintenance calculation which was not validly made; and
%
%($b$) a voluntary payment made in the circumstances set out in subsection (1A)($a$),
%
%shall be treated as an overpayment of child support maintenance made by a non-resident parent.”
%21Recovery of child support maintenance by deduction from benefit
%
%For section 43 of the 1991 Act (contribution to maintenance by deduction from benefit) there shall be substituted—
%“43Recovery of child support maintenance by deduction from benefit
%
%(1) This section applies where—
%
%($a$) a non-resident parent is liable to pay a flat rate of child support maintenance (or would be so liable but for a variation having been agreed to), and that rate applies (or would have applied) because he falls within paragraph 4(1)($b$)  or ($c$)  or 4(2)  of Schedule 1; and
%
%($b$) such conditions as may be prescribed for the purposes of this section are satisfied.
%
%(2) The power of the Secretary of State to make regulations under section 5 of the [1992 c. 5. ] Social Security Administration Act 1992 by virtue of subsection (1)(p) (deductions from benefits) may be exercised in relation to cases to which this section applies with a view to securing that payments in respect of child support maintenance are made or that arrears of child support maintenance are recovered.
%
%(3) For the purposes of this section, the benefits to which section 5 of the 1992 Act applies are to be taken as including war disablement pensions and war widows' pensions (within the meaning of section 150 of the [1992 c. 4. ] Social Security Contributions and Benefits Act 1992 (interpretation)).”
%22Jurisdiction
%
%(1) Section 44 of the 1991 Act (jurisdiction) shall be amended as follows.
%
%(2) In subsection (1), after “United Kingdom” there shall be inserted “, except in the case of a non-resident parent who falls within subsection (2A) ”.
%
%(3) After subsection (2)  there shall be inserted—
%
%“(2A) A non-resident parent falls within this subsection if he is not habitually resident in the United Kingdom, but is—
%
%($a$) employed in the civil service of the Crown, including Her Majesty’s Diplomatic Service and Her Majesty’s Overseas Civil Service;
%
%($b$) a member of the naval, military or air forces of the Crown, including any person employed by an association established for the purposes of Part XI of the [1996 c. 14. ] Reserve Forces Act 1996;
%
%($c$) employed by a company of a prescribed description registered under the [1985 c. 6. ] Companies Act 1985 in England and Wales or in Scotland, or under the [S.I. 1986/1032 (N.I. 6).] Companies (Northern Ireland) Order 1986; or
%
%($d$) employed by a body of a prescribed description.”
%
%(4) Subsection (3)  shall cease to have effect.
%23Abolition of the child maintenance bonus
%
%Section 10 of the [1995 c. 34. ] Child Support Act 1995 (which provides for the child maintenance bonus) shall cease to have effect.

\amendment{
Ss. 20--23 are not yet in force.
}

\subsection{24. Periodical reviews}

Article 3(4)  of the  Social Security Act 1998 (Commencement No.\ 2) Order 1998 (which saved section 16 of the 1991 Act for certain purposes) is revoked; and accordingly that section shall cease to have effect for all purposes.

%\subsection{25. Regulations}
%
%In section 52 of the 1991 Act (regulations and orders), for subsection (2)  there shall be substituted—
%
%“(2) No statutory instrument containing (whether alone or with other provisions) regulations made under—
%
%($a$) section 6(1), 12(4)  (so far as the regulations make provision for the default rate of child support maintenance mentioned in section 12(5)($b$) ), 28C(2)($b$), 28F(2)($b$), 30(5A) , 41(2) , 41A, 41B(6) , 43(1), 44(2A) ($d$), 46 or 47;
%
%($b$) paragraph 3(2)  or 10A(1)  of Part I of Schedule 1; or
%
%($c$) Schedule 4B,
%
%or an order made under section 45(1)  or (6) , shall be made unless a draft of the instrument has been laid before Parliament and approved by a resolution of each House of Parliament.
%
%(2A) No statutory instrument containing (whether alone or with other provisions) the first set of regulations made under paragraph 10(1)  of Part I of Schedule 1 as substituted by section 1(3)  of the Child Support, Pensions and Social Security Act 2000 shall be made unless a draft of the instrument has been laid before Parliament and approved by a resolution of each House of Parliament.”
%26Amendments
%
%Schedule 3 (amendment of enactments) shall have effect.
%27Temporary compensation payment scheme
%
%(1) This section applies where—
%
%($a$) a maintenance assessment is made before a prescribed date following an application for one under section 4, 6 or 7 of the 1991 Act; or
%
%($b$) a fresh maintenance assessment has been made following either a periodic review under section 16 of the 1991 Act or a review under section 17 of that Act (as they had effect before their substitution by section 40 or 41 respectively of the [1998 c. 14. ] Social Security Act 1998),
%
%and the effective date of the assessment is earlier than the date on which the assessment was made, with the result that arrears of child support maintenance have become due under the assessment.
%
%(2) The Secretary of State may in regulations provide that this section has effect as if it were modified so as—
%
%($a$) to apply to cases of arrears of child support maintenance having become due additional to those referred to in subsection (1) ;
%
%($b$) not to apply to any such case as is referred to in subsection (1) .
%
%(3) If this section applies, the Secretary of State may in prescribed circumstances agree with the absent parent, on terms specified in the agreement, that—
%
%($a$) the absent parent will not be required to pay the whole of the arrears, but only some lesser amount; and
%
%($b$) the Secretary of State will not, while the agreement is complied with, take action to recover any of the arrears.
%
%(4) The terms which may be specified are to be prescribed in or determined in accordance with regulations made by the Secretary of State.
%
%(5) An agreement may be entered into only if it is made before 1st April 2002 and expires before 1st April 2003. 
%
%(6) If the absent parent enters into such an agreement, the Secretary of State may, while the absent parent complies with it, refrain from taking action under the 1991 Act to recover the arrears.
%
%(7) Upon the expiry of the agreement, if the absent parent has complied with it—
%
%($a$) he ceases to be liable to pay the arrears; and
%
%($b$) the Secretary of State may make payments of such amounts and at such times as he may determine to the person with care.
%
%(8) If the absent parent fails to comply with the agreement he becomes liable to pay the full amount of any outstanding arrears (as well as any other amount payable in accordance with the assessment).
%
%(9) The Secretary of State may by regulations provide for this section to have effect as if there were substituted for the dates in subsection (5)  such later dates as are prescribed.
%
%(10) In this section, “prescribed” means prescribed in regulations made by the Secretary of State.
%
%(11) Regulations under this section shall be made by statutory instrument.
%
%(12) No statutory instrument containing regulations under subsection (9)  is to be made unless a draft of the instrument has been laid before Parliament and approved by a resolution of each House of Parliament; but otherwise a statutory instrument containing regulations under this section shall be subject to annulment in pursuance of a resolution of either House of Parliament.
%28Pilot schemes
%
%(1) Any regulations made under—
%
%($a$) provisions inserted or substituted in the 1991 Act by this Part of this Act (or Schedule 1, 2 or 3); and
%
%($b$) in so far as they are consequential on or supplementary to any such regulations, regulations made under any other provisions in the 1991 Act,
%
%may be made so as to have effect for a specified period not exceeding 12 months.
%
%(2) Any regulations which, by virtue of subsection (1), are to have effect for a limited period are referred to in this section as “a pilot scheme”.
%
%(3) A pilot scheme may provide that its provisions are to apply only in relation to—
%
%($a$) one or more specified areas or localities;
%
%($b$) one or more specified classes of person;
%
%($c$) persons selected by reference to prescribed criteria, or on a sampling basis.
%
%(4) A pilot scheme may make consequential or transitional provision with respect to the cessation of the scheme on the expiry of the specified period.
%
%(5) A pilot scheme (“the previous scheme”) may be replaced by a further pilot scheme making the same provision as that made by the previous scheme (apart from the specified period), or similar provision.
%
%(6) A statutory instrument containing (whether alone or with other provisions) a pilot scheme shall not be made unless a draft of the instrument has been laid before Parliament and approved by resolution of each House of Parliament.
%29Interpretation, transitional provisions, savings, etc
%
%(1) In this Part, “the 1991 Act” means the [1991 c. 48. ] Child Support Act 1991. 
%
%(2) The Secretary of State may in regulations make such transitional and transitory provisions, and such incidental, supplementary, savings and consequential provisions, as he considers necessary or expedient in connection with the coming into force of this Part or any provision in it.
%
%(3) The regulations may, in particular—
%
%($a$) provide for the amount of child support maintenance payable by or to any person to be at a transitional rate (or more than one such rate successively) resulting from the phasing-in by way of prescribed steps of any increase or decrease in the amount payable following the coming into force of this Part or any provision in it;
%
%($b$) provide for a departure direction or any finding in relation to a previous determination of child support maintenance to be taken into account in a decision as to the amount of child support maintenance payable by or to any person.
%
%(4) Section 175(3)  and (5)  of the [1992 c. 4. ] Social Security Contributions and Benefits Act 1992 (supplemental power in relation to regulations) applies to regulations made under this section as it applies to regulations made under that Act.
%
%(5) The power to make regulations under this section is exercisable by statutory instrument.
%
%(6) A statutory instrument containing regulations under this section shall be subject to annulment in pursuance of a resolution of either House of Parliament.

\amendment{
Ss. 25--29 are not yet in force.
}

\part[Part II --- Pensions]{Part II\\*Pensions}

\section[Chapter I --- State pensions]{Chapter I --- State pensions}

\renewcommand\parthead{--- Part II Chapter I}

%State second pension
%30Earnings from which pension derived
%
%(1) In section 22 of the [1992 c. 4. ] Social Security Contributions and Benefits Act 1992 (earnings from which earnings factors are derived), after subsection (2)  there shall be inserted—
%
%“(2A) For the purposes specified in subsection (2)($b$)  above, in the case of the first appointed year or any subsequent tax year a person’s earnings factor shall be treated as derived only from those of his earnings on which primary Class 1 contributions have been paid or treated as paid.”
%
%(2) In section 44 of that Act (Category A retirement pension), in subsection (6) —
%
%($a$) before paragraph ($a$)  there shall be inserted—
%
%“(za)where the relevant year is the first appointed year or any subsequent year, to the aggregate of his earnings factors derived from those of his earnings upon which primary Class 1 contributions have been paid or treated as paid in respect of that year;”;
%
%and
%
%($b$) in paragraph ($a$), after “subsequent tax year” there shall be inserted “before the first appointed year”.
%
%(3) After that section there shall be inserted—
%“44ADeemed earnings factors
%
%(1) For the purposes of section 44(6) (za) above, if any of the conditions in subsection (2)  below is satisfied for a relevant year, a pensioner is deemed to have an earnings factor for that year which—
%
%($a$) is derived from earnings on which primary Class 1 contributions were paid; and
%
%($b$) is equal to the amount which, when added to any other earnings factors taken into account under that provision, produces an aggregate of earnings factors equal to the low earnings threshold.
%
%(2) The conditions referred to in subsection (1)  above are that—
%
%($a$) the pensioner would, apart from this section, have an earnings factor for the year—
%
%(i) equal to or greater than the qualifying earnings factor for the year; but
%
%(ii) less than the low earnings threshold for the year;
%
%($b$) invalid care allowance—
%
%(i) was payable to the pensioner throughout the year; or
%
%(ii) would have been so payable but for the fact that under regulations the amount payable to him was reduced to nil because of his receipt of other benefits;
%
%($c$) for the purposes of paragraph 5(7)($b$)  of Schedule 3, the pensioner is taken to be precluded from regular employment by responsibilities at home throughout the year by virtue of—
%
%(i) the fact that child benefit was payable to him in respect of a child under the age of six; or
%
%(ii) his satisfying such other condition as may be prescribed;
%
%($d$) the pensioner is a person satisfying the requirement in subsection (3)  below to whom long-term incapacity benefit was payable throughout the year, or would have been so payable but for the fact that—
%
%(i) he did not satisfy the contribution conditions in paragraph 2 of Schedule 3; or
%
%(ii) under regulations the amount payable to him was reduced to nil because of his receipt of other benefits or of payments from an occupational pension scheme or personal pension scheme.
%
%(3) The requirement referred to in subsection (2)($d$)  above is that—
%
%($a$) for one or more relevant years the pensioner has paid, or (apart from this section) is treated as having paid, primary Class 1 contributions on earnings equal to or greater than the qualifying earnings factor; and
%
%($b$) the years for which he has such a factor constitute at least one tenth of his working life.
%
%(4) For the purposes of subsection (3)($b$)  above—
%
%($a$) a pensioner’s working life shall not include—
%
%(i) any tax year before 1978-79; or
%
%(ii) any year in which he is deemed under subsection (1)  above to have an earnings factor by virtue of fulfilling the condition in subsection (2)($b$)  or ($c$)  above; and
%
%($b$) the figure calculated by dividing his working life by ten shall be rounded to the nearest whole year (and any half year shall be rounded down).
%
%(5) The low earnings threshold for the first appointed year and subsequent tax years shall be £9,500 (but subject to section 148A of the Administration Act).
%
%(6) In subsection (2)($d$)(ii)  above, “occupational pension scheme” and “personal pension scheme” have the meanings given by subsection (6)  of section 30DD above for the purposes of subsection (5)  of that section.”
%
%(4) For the purposes of subsection (1)  of section 44A of the [1992 c. 4. ] Social Security Contributions and Benefits Act 1992, a pensioner is deemed to have an earnings factor in relation to any relevant year as specified in that subsection if—
%
%($a$) severe disablement allowance was payable to him throughout the year; and
%
%($b$) he satisfies the requirement in subsection (3)  of that section.
%31Calculation
%
%(1) In section 45 of the [1992 c. 4. ] Social Security Contributions and Benefits Act 1992 (calculation of additional pension in a Category A retirement pension), in subsection (2) —
%
%($a$) after “shall be” there shall be inserted “the sum of the following”;
%
%($b$) in paragraph ($b$), after “after 1987-88” there shall be inserted “but before the first appointed year”; and
%
%($c$) after that paragraph there shall be inserted “; and
%
%($c$) in relation to any tax years falling within subsection (3A)  below, the weekly equivalent of the amount calculated in accordance with Schedule 4A to this Act.”
%
%(2) In that section the following subsection shall be inserted after subsection (3) —
%
%“(3A) The following tax years fall within this subsection—
%
%($a$) the first appointed year;
%
%($b$) subsequent tax years.”
%
%(3) After Schedule 4 to that Act there shall be inserted the Schedule set out in Schedule 4 to this Act.
%32Calculation of Category B retirement pension
%
%(1) In section 46 of the [1992 c. 4. ] Social Security Contributions and Benefits Act 1992 (modifications of section 45 for calculating the additional pension in certain benefits), after subsection (2)  there shall be inserted—
%
%“(3) For the purpose of determining the additional pension falling to be calculated under section 45 above by virtue of section 48BB below in a case where the deceased spouse died under pensionable age, the following definition shall be substituted for the definition of “N” in section 45(4)($b$)  above—
%
%    ““N” =
%    ($a$) 
%
%    the number of tax years which begin after 5th April 1978 and end before the date when the deceased spouse dies, or
%    ($b$) 
%
%    the number of tax years in the period—
%    (i) 
%
%    beginning with the tax year in which the deceased spouse (“S”) attained the age of 16 or, if later, 1978-79, and
%    (ii) 
%
%    ending immediately before the tax year in which S would have attained pensionable age if S had not died earlier,
%
%    whichever is the smaller number.”” 
%
%(2) In section 48BB of that Act (Category B retirement pension: entitlement by reference to benefits under section 39A or 39B), in subsection (5)  for “section 46(2)” there shall be substituted “section 46(3)”.
%
%(3) In paragraph 5 of Schedule 8 to the [1999 c. 30. ] Welfare Reform and Pensions Act 1999 (welfare benefits: minor and consequential amendments), sub-paragraph ($b$), and the word “and” immediately preceding it, shall be omitted.
%33Revaluation
%
%(1) After section 148 of the [1992 c. 5. ] Social Security Administration Act 1992 there shall be inserted—
%“148ARevaluation of low earnings threshold
%
%(1) The Secretary of State shall in the tax year preceding the first appointed year and in each subsequent tax year review the general level of earnings obtaining in Great Britain and any changes in that level which have taken place during the review period.
%
%(2) In this section, “the review period” means—
%
%($a$) in the case of the first review under this section, the period beginning with 1st October 1998 and ending on 30th September in the tax year preceding the first appointed year; and
%
%($b$) in the case of each subsequent review under this section, the period since—
%
%(i) the end of the last period taken into account in a review under this section; or
%
%(ii) such other date (whether earlier or later) as the Secretary of State may determine.
%
%(3) If on such a review it appears to the Secretary of State that the general level of earnings has increased during the review period, he shall make an order under this section.
%
%(4) An order under this section shall be an order directing that, for the purposes of the Contributions and Benefits Act—
%
%($a$) there shall be a new low earnings threshold for the tax years after the tax year in which the review takes place; and
%
%($b$) the amount of that threshold shall be the amount specified in subsection (5)  below—
%
%(i) increased by the percentage by which the general level of earnings increased during the review period; and
%
%(ii) rounded to the nearest £100 (taking any amount of £50 as nearest to the next whole £100).
%
%(5) The amount referred to in subsection (4)($b$)  above is—
%
%($a$) in the case of the first review under this section, £9,500; and
%
%($b$) in the case of each subsequent review, the low earnings threshold for the year in which the review takes place.
%
%(6) This section does not require the Secretary of State to direct any increase where it appears to him that the increase would be inconsiderable.
%
%(7) If on any review under subsection (1)  above the Secretary of State determines that he is not required to make an order under this section, he shall instead lay before each House of Parliament a report explaining his reasons for arriving at that determination.
%
%(8) For the purposes of any review under subsection (1)  above the Secretary of State shall estimate the general level of earnings in such manner as he thinks fit.”
%
%(2) Section 148 of the [1992 c. 5. ] Social Security Administration Act 1992 (revaluation of earnings factors) shall have effect as if—
%
%($a$) the amounts for the first appointed year and any subsequent tax year that are to be reviewed under that section,
%
%($b$) the amounts for those years to which any directions by an order under subsection (4)  of that section are to be applied, and
%
%($c$) accordingly, the amounts for the purpose of maintaining the value of which that section has effect,
%
%included the parts of the surplus in an earnings factor referred to in paragraphs 2(2)($a$), 5(2)($a$)  and 7(2)($a$)  of Schedule 4A to the [1992 c. 4. ] Social Security Contributions and Benefits Act 1992. 
%
%(3) Nothing in section 148 of the [1992 c. 5. ] Social Security Administration Act 1992 shall require, or ever have required, the earnings factors used for computing a surplus in an earnings factor for any year under section 44(5A)  of the [1992 c. 4. ] Social Security Contributions and Benefits Act 1992 to be treated as increased in any case in which that surplus, or any part of it, is itself reviewed under section 148 of the [1992 c. 5. ] Social Security Administration Act 1992. 
%
%(4) In section 128(3)  of the [1995 c. 26. ] Pensions Act 1995 (revaluation of surpluses in earnings factors under section 44(5A)  of the Social Security Contributions and Benefits Act 1992), after “1992” there shall be inserted “for the purposes of section 45(1)  and (2)($a$)  and ($b$)  of that Act”.
%34Report of Government Actuary: rebates etc
%
%In each of sections 42(1)($a$)(ii) , 42B(1)($a$)  and 45A(1)($a$)  of the [1993 c. 48. ] Pension Schemes Act 1993 (reports by Government Actuary on cost of providing benefits equivalent to benefits which are foregone) for “which, under section 48A,” there shall be substituted “(or parts of benefits) which, in accordance with section 48A below and Schedule 4A to the [1992 c. 4. ] Social Security Contributions and Benefits Act 1992,”.
%35Supplementary
%
%(1) The [1992 c. 4. ] Social Security Contributions and Benefits Act 1992 shall be amended as follows.
%
%(2) In section 21(5A) ($b$)  (contribution conditions)—
%
%($a$) after “22(1)($a$)” there shall be inserted “, (2A) ”; and
%
%($b$) for “44(6) ($a$)” there shall be substituted “44(6) (za) and ($a$)”.
%
%(3) In section 39 (rate of widowed mother’s allowance and widow’s pension), in subsections (1), (2)  and (3), after “sections 44 to 45B” there shall be inserted “and Schedule 4A”.
%
%(4) In section 39C (rate of widowed parent’s allowance and bereavement allowance), in subsections (1), (3)  and (4) , after “sections 44 to 45A” there shall be inserted “and Schedule 4A”.
%
%(5) In section 44 (Category A retirement pension), in subsection (5A) , after “section 45” there shall be inserted “and Schedule 4A”.
%
%(6) In that subsection, for the words from “that year,” to “surplus” there shall be substituted “that year,
%
%($b$) the amount of the surplus is the amount of that excess, and
%
%($c$) for the purposes of section 45(1)  and (2)($a$)  and ($b$)  below, the adjusted amount of the surplus”.
%
%(7) In subsection (6)  of that section, after “section 45” there shall be inserted “or Schedule 4A”.
%
%(8) In section 45 (the additional element in a Category A retirement pension)—
%
%($a$) in subsections (1)  and (2)($a$)  and ($b$), before “amount” (in each place) there shall be inserted “adjusted”; and
%
%($b$) in subsection (6) , for “the amount of any surpluses” there shall be substituted “any amount”.
%
%(9) In section 48A(4)  (Category B retirement pension for married person), after “sections 44 to 45B above” there shall be inserted “and Schedule 4A below”.
%
%(10) In section 48B (Category B retirement pension for widows and widowers), in subsections (2)  and (3), after “sections 44 to 45B above” there shall be inserted “and Schedule 4A below”.
%
%(11) In section 48BB (Category B retirement pension: entitlement by reference to benefits under section 39A or 39B), in subsections (5)  and (6) , after “sections 44 to 45A above” there shall be inserted “and Schedule 4A below”.
%
%(12) In section 48C(4)  (Category B retirement pension: general), after “sections 44 to 45B above” there shall be inserted “and Schedule 4A below”.
%
%(13) In section 51 (Category B retirement pension for widowers), in subsections (2)  and (3), after “sections 44 to 45A above” there shall be inserted “and Schedule 4A below”.
%
%(14) In section 122(1)  (interpretation of Parts I to VI), at the appropriate place in alphabetical order, there shall be inserted—
%
%““first appointed year” means such tax year, no earlier than 2002-03, as may be appointed by order, and “second appointed year” means such subsequent tax year as may be so appointed;”.
%
%(15)In section 176 (Parliamentary control), after subsection (3)  there shall be inserted—
%
%“(4) Subsection (3)  above does not apply to a statutory instrument by reason only that it contains an order appointing the first or second appointed year (within the meanings given by section 122(1)  above).”
%Report on pensions uprating
%36Report on cost of pension uprating in line with general earnings level
%
%The Government Actuary or the Deputy Government Actuary shall report to the Secretary of State his opinion on the effect on the level of the National Insurance Fund, and the effect which might be expected on the rates of contributions, in each year up to and including 2005-06 of annual increases in the basic pension by the percentage increase in the general level of earnings; and the Secretary of State shall lay a copy of the report before Parliament.

\amendment{
Ss. 30--36 are not yet in force.
}

\subsection{\itshape Earnings factors}

%37Revaluation of earnings factors
%
%In section 148(2)  of the [1992 c. 5. ] Social Security Administration Act 1992 (revaluation of earnings factors), for the words from “place” to the end there shall be substituted “place—
%
%($a$) since the end of the period taken into account for the last review under this section, or
%
%($b$) since such other date (whether earlier or later) as he may determine;
%
%and for the purposes of any such review the Secretary of State shall estimate the general level of earnings in such manner as he thinks fit.”

\amendment{
S. 37 is not yet in force.
}

\subsubsection{38. Modification of earnings factors}

(1) In section 48A(5)  of the 1993 Act (power to modify the application of section 44(5)  of the 1992 Act where in any year a pensioner’s earnings derive only partially from contracted-out employment), after “44(5)” there shall be inserted “or (5A) ”.

(2) Subsection (1)  shall have effect—
\begin{enumerate}\item[]
($a$) in relation to the application of section 44(5A)  of the 1992 Act by virtue of sections 39C(1)  and 48BB(5)  of that Act;

($b$) in relation to the application of section 44(5A)  of the 1992 Act in the circumstances described in section 128(4)  to (6)  of the 1995 Act.
\end{enumerate}

(3) In relation to the period—
\begin{enumerate}\item[]
($a$) beginning with 6th April 2000, and

($b$) ending with the day before the first regulations under section 48A(5)  of the 1993 Act (as amended by subsection (1)  above) come into force,
\end{enumerate}
the Secretary of State shall be taken to have, and to have had, power to calculate and pay relevant pensions by reference to section 44(5)  of the 1992 Act as modified by regulations under section 48A(5)  of the 1993 Act.

(4) For the purpose of applying subsection (3)  above—
\begin{enumerate}\item[]
($a$) the substitution made by section 128(1)  of the 1995 Act shall be ignored; and

($b$) references in enactments to section 44(5A)  of the 1992 Act shall (so far as necessary) be treated as references to section 44(5).
\end{enumerate}

(5) The first regulations under section 48A(5)  of the 1993 Act (as amended by subsection (1)  above) may include provision in relation to—
\begin{enumerate}\item[]
($a$) revising the calculation of a relevant pension;

($b$) paying a relevant pension in accordance with a revised calculation.
\end{enumerate}

(6) Relevant pensions are pensions which fall to be calculated—
\begin{enumerate}\item[]
($a$) in the circumstances described in section 128(4)  to (6)  of the 1995 Act; and

($b$) in relation to persons where, by virtue of section 48A(1)  of the 1993 Act, section 44(6)  of the 1992 Act has effect in any tax year as mentioned in section 48A(1)  of the 1993 Act in relation to some but not all of a person’s earnings.
\end{enumerate}

(7) For the purposes of this section—
\begin{enumerate}\item[]
($a$) the 1992 Act is the Social Security Contributions and Benefits Act 1992;

($b$) the 1993 Act is the Pension Schemes Act 1993;

($c$) the 1995 Act is the Pensions Act 1995. 
\end{enumerate}

\subsection{\itshape Preservation of rights in respect of additional pensions}

\subsubsection{39. Preservation of rights in respect of additional pensions}

(1) In the provisions of the Social Security Contributions and Benefits Act 1992 that are set out in subsection (2)  (provisions relating to additional pensions for surviving spouses)—
\begin{enumerate}\item[]
($a$) the references to 5th April 2000 (wherever occurring) shall have effect, and be deemed always to have had effect, as references to 5th October 2002; and

($b$) the references to 6th April 2000 (wherever occurring) shall have effect, and be deemed always to have had effect, as references to 6th October 2002. 
\end{enumerate}

(2) Those provisions are—
\begin{enumerate}\item[]
($a$) sections 39(3)  and 39C(4)  (widowed mother’s allowance and widowed parent’s allowance);

($b$) sections 48BB(7), 48C(3)  and 51(3)  (Category B retirement pensions); and

($c$) paragraphs 4(3), 5A(2)  and (3)  and 6(3)  and (4)  of Schedule 5 (deferred pensions).
\end{enumerate}

(3) For section 52(3)  of the Welfare Reform and Pensions Act 1999 (power to substitute a later year for references to year 2000 in prescribed provisions of the Social Security Contributions and Benefits Act 1992) there shall be substituted—
\begin{quotation}
“(3) The regulations may amend (or further amend) any prescribed provision set out in section 39(2)  of the Child Support, Pensions and Social Security Act 2000 (which sets out provisions falling within subsection (2)  of this section) so as to substitute a reference to a later date for—
\begin{enumerate}\item[]
($a$) any reference in that provision to 5th October 2002 or 6th October 2002; or

($b$) any reference to a date inserted in that provision by a substitution made by virtue of this subsection.”
\end{enumerate}
\end{quotation}

(4) After section 52(4)  of that Act of 1999 there shall be inserted—
\begin{quotation}
“(4A) The regulations may provide, for the purposes of any provision made by virtue of subsection (4) , for a case in which a person who, as a consequence of receiving incorrect or incomplete information, did not give any consideration to—
\begin{enumerate}\item[]
($a$) the taking of a step which is a step he might have taken had he considered the matter on the basis of correct and complete information, or

($b$) refraining from taking a step which is a step he did take but might have refrained from taking had he considered the matter on that basis,
\end{enumerate}
to be treated as a case in which his failure to take the step, or his taking of the step he did take, was in reliance on the incorrect or incomplete information and as a case in which that step is one which he would have taken, or (as the case may be) would not have taken, had the information been correct and complete.”
\end{quotation}

(5) In section 52(6)  of that Act of 1999 (supplemental provisions of regulations relating to the scheme), after paragraph ($e$)  there shall be inserted—
\begin{quotation}
“($ea$) prescribing the matters that may be relied on, and the presumptions that may be made, in the determination of whether or not the prescribed conditions have been satisfied;”.
\end{quotation}

\section{\itshape Other provisions}

%40Home responsibilities protection
%
%In paragraph 5 of Schedule 3 to the [1992 c. 4. ] Social Security Contributions and Benefits Act 1992 (contribution conditions for entitlement to Category A and B retirement pension, widowed mother’s allowance and widow’s pension), after sub-paragraph (7)  (reduction of number of years for which contribution conditions must be satisfied) there shall be inserted—
%
%“(7A) Regulations may provide that a person is not to be taken for the purposes of sub-paragraph (7)($b$)  above as precluded from regular employment by responsibilities at home unless he meets the prescribed requirements as to the provision of information to the Secretary of State.”

\amendment{
S. 40 is not yet in force.
}

\subsection{41. Sharing of state scheme rights}

(1) In section 49 of the Welfare Reform and Pensions Act 1999 (creation of state scheme pension debits and credits), for subsection (4)  there shall be substituted—
\begin{quotation}
“(4) The Secretary of State may by regulations make provision about the calculation and verification of cash equivalents for the purposes of this section.

(4A) The power conferred by subsection (4)  above includes power to provide—
\begin{enumerate}\item[]
($a$) for calculation or verification in such manner as may be approved by or on behalf of the Government Actuary, and

($b$) for things done under the regulations to be required to be done in accordance with guidance from time to time prepared by a person prescribed by the regulations.”
\end{enumerate}
\end{quotation}

(2) In section 45B of the Social Security Contributions and Benefits Act 1992 (pension sharing resulting in reduction of additional Category A retirement pension), for subsection (7)  there shall be substituted—
\begin{quotation}
“(7) The Secretary of State may by regulations make provision about the calculation and verification of cash equivalents for the purposes of this section.

(7A) The power conferred by subsection (7)  above includes power to provide—
\begin{enumerate}\item[]
($a$) for calculation or verification in such manner as may be approved by or on behalf of the Government Actuary, and

($b$) for things done under the regulations to be required to be done in accordance with guidance from time to time prepared by a person prescribed by the regulations.”
\end{enumerate}
\end{quotation}

(3) In section 55A of that Act (shared additional pension), for subsection (6)  there shall be substituted—
\begin{quotation}
“(6) The Secretary of State may by regulations make provision about the calculation and verification of cash equivalents for the purposes of this section.

(6A) The power conferred by subsection (6)  above includes power to provide—
\begin{enumerate}\item[]
($a$) for calculation or verification in such manner as may be approved by or on behalf of the Government Actuary, and

($b$) for things done under the regulations to be required to be done in accordance with guidance from time to time prepared by a person prescribed by the regulations.”
\end{enumerate}
\end{quotation}

(4) In section 55B of that Act (pension sharing resulting in reduction of shared additional pension), for subsection (7)  there shall be substituted—
\begin{quotation}
“(7) The Secretary of State may by regulations make provision about the calculation and verification of cash equivalents for the purposes of this section.

(7A) The power conferred by subsection (7)  above includes power to provide—
\begin{enumerate}\item[]
($a$) for calculation or verification in such manner as may be approved by or on behalf of the Government Actuary, and

($b$) for things done under the regulations to be required to be done in accordance with guidance from time to time prepared by a person prescribed by the regulations.”
\end{enumerate}
\end{quotation}

%42Disclosure of state pension information
%
%(1) This section applies to any state pension information which is held in relation to any individual—
%
%($a$) by the Secretary of State; or
%
%($b$) in connection with the provision of any services provided to the Secretary of State for purposes connected with his functions relating to social security, by the person providing those services.
%
%(2) Regulations may confer a power on the Secretary of State to disclose, or to authorise the disclosure of, any information to which this section applies in any case in which—
%
%($a$) the person to whom the disclosure is made is a person falling within subsection (3)  who has, in the prescribed manner, applied to the Secretary of State for the disclosure of the information; and
%
%($b$) it appears to the Secretary of State that the prescribed conditions for the making of a disclosure of the information in question to that person have been satisfied.
%
%(3) A person falls within this subsection if—
%
%($a$) he is the trustee or manager of an occupational pension scheme of which the individual to whom the information relates is a member;
%
%($b$) he is the trustee or manager of a personal pension scheme of which that individual is a member;
%
%($c$) he is the employer in relation to an occupational pension scheme of which that individual is a member;
%
%($d$) he is the employer in relation to any employed earner’s employment of that individual which is not contracted-out employment; or
%
%($e$) he is proposing to provide services to that individual in circumstances in which the provision of the services, or the proposal to do so, may involve the giving of advice or forecasts to which the information to which this section applies may be relevant.
%
%(4) The Secretary of State shall secure that his powers under this section are exercised so that at least the following is prescribed for the purposes of subsection (2)($b$), namely—
%
%($a$) in the case of an application for information made by a person falling within paragraph ($e$)  of subsection (3), a condition that the individual to whom the information relates has consented to the making of the application and to the disclosure; and
%
%($b$) in any other case, either that condition or the alternative condition set out in subsection (5).
%
%(5) The alternative condition is—
%
%($a$) that such steps as may be prescribed have been taken for the purpose of ascertaining whether the individual to whom the information relates objects to the making of the application for the disclosure of information relating to him; and
%
%($b$) that the prescribed time has elapsed without any objection by that individual.
%
%(6) A person applying to the Secretary of State, in accordance with regulations under this section, for the disclosure of any information relating to an individual shall be entitled, for the purpose of making the application, to make such disclosures of information relating to that individual as may be authorised by the regulations.
%
%(7) In this section the reference, in relation to an individual, to state pension information is a reference to the following information about that individual—
%
%($a$) his date of birth, and the age at which and date on which he attains pensionable age—
%
%(i) for the purposes of the [1993 c. 48. ] Pension Schemes Act 1993, in relation to any guaranteed minimum pension to which he is entitled; and
%
%(ii) in accordance with the rules in paragraph 1 of Schedule 4 to the [1995 c. 26. ] Pensions Act 1995;
%
%($b$) the amount of any basic retirement pension a present or future entitlement to which has already accrued to that individual, and the amount of any additional retirement pension such an entitlement to which has already accrued to that individual;
%
%($c$) a projection of the amount of the basic retirement pension to which that individual is likely to become entitled, or might become entitled in particular circumstances; and
%
%($d$) a projection of the amount of the additional retirement pension to which that individual is likely to become entitled, or might become entitled in particular circumstances.
%
%(8) Regulations under this section shall be made by statutory instrument, which shall be subject to annulment in pursuance of a resolution of either House of Parliament.
%
%(9) Subsections (4)  to (6)  of section 189 of the [1992 c. 5. ] Social Security Administration Act 1992 (supplemental and incidental powers etc.) shall apply in relation to any power to make regulations under this section as they apply in relation to the powers to make regulations that are conferred by that Act.
%
%(10) For the purposes of section 121E of the [1992 c. 5. ] Social Security Administration Act 1992 (supply of information by the Inland Revenue to the Secretary of State for the purposes of the Secretary of State’s functions relating to social security), the Secretary of State’s functions relating to social security shall be taken to include any power conferred on him by regulations under this section.
%
%(11) In this section—
%
%    “basic retirement pension” and “additional retirement pension” mean any basic or, as the case may be, additional pension under the [1992 c. 4. ] Social Security Contributions and Benefits Act 1992;
%
%    “contracted-out employment” has the same meaning as in the [1993 c. 48. ] Pension Schemes Act 1993;
%
%    “employed earner” has the same meaning as it has in Parts I to V of the [1992 c. 4. ] Social Security Contributions and Benefits Act 1992 (by virtue of section 2(1)  of that Act);
%
%    “employer”—
%    ($a$) 
%
%    in relation to any occupational pension scheme, has the same meaning as in Part I of the [1995 c. 26. ] Pensions Act 1995; and
%    ($b$) 
%
%    in relation to employed earner’s employment, has the same meaning as in the [1993 c. 48. ] Pension Schemes Act 1993;
%
%    “member”, in relation to an occupational pension scheme, has the same meaning as in Part I of the [1995 c. 26. ] Pensions Act 1995;
%
%    “occupational pension scheme” and “personal pension scheme” have the same meanings as in the [1993 c. 48. ] Pension Schemes Act 1993;
%
%    “prescribed” means prescribed by or determined in accordance with regulations;
%
%    “regulations” means regulations made by the Secretary of State;
%
%    “trustee” and “manager”, in relation to an occupational pension scheme, have the same meanings as in Part I of the [1995 c. 26. ] Pensions Act 1995.  
%
%Chapter IIOccupational and Personal Pension Schemes
%Selection of trustees and of directors of corporate trustees
%43Member-nominated trustees
%
%(1) Section 16 of the [1995 c. 26. ] Pensions Act 1995 (requirements for trustees to be nominated and selected by members of the scheme) shall be amended in accordance with subsections (2)  to (8)  of this section.
%
%(2) In subsection (1)  (duty of trustees to make arrangements for selection of member-nominated trustees)—
%
%($a$) the words “(subject to section 17)” and in paragraph ($b$), the words “, and the appropriate rules,” shall be omitted; and
%
%($b$) in paragraph ($a$), for “persons selected” there shall be substituted “the selection of persons nominated”.
%
%(3) In subsection (3)($a$)  (selected persons to be trustees), for “in accordance with the appropriate rules” there shall be substituted “as a member-nominated trustee”.
%
%(4) In subsection (4)  (procedure for filling vacancies unfilled because of insufficient nominations), for “the appropriate rules” there shall be substituted “regulations”.
%
%(5) In subsection (5)  (period of service as a member-nominated trustee), after “six years” there shall be inserted “but for a member-nominated trustee to be eligible for selection again at the end of any period of service as such a trustee.”
%
%(6) After subsection (6)  there shall be inserted—
%
%“(6A) The arrangements must provide that, where the employer so requires, a person who is not a qualifying member of the scheme must have the employer’s approval to qualify for selection as a member-nominated trustee.”
%
%(7) In subsection (8)  (persons ceasing to be member-nominated trustees on ceasing to be qualifying members of the scheme)—
%
%($a$) for “The arrangements must” there shall be substituted “The arrangements—
%
%($a$) must”; and
%
%($b$) at the end there shall be inserted “; and
%
%($b$) may provide for a member-nominated trustee who—
%
%(i) is a qualifying member of one of the following descriptions, that is to say, an active, deferred or pensioner member, and
%
%(ii) ceases (without ceasing to be a qualifying member) to be a qualifying member of that description,
%
%to cease, by virtue of that fact, to be a trustee.”
%
%(8) After subsection (8)  there shall be inserted—
%
%“(9) Regulations may make provision in relation to arrangements under this section—
%
%($a$) supplementing the requirements of this section as to the matters to be contained in the arrangements; and
%
%($b$) providing for the manner in which, and the time within which, persons are, for the purposes of the arrangements, to be nominated and selected as member-nominated trustees.
%
%(10) This section does not apply in the case of a trust scheme if—
%
%($a$) every member of the scheme is a trustee of the scheme and no other person is such a trustee;
%
%($b$) every trustee of the scheme is a company; or
%
%($c$) the scheme is of a prescribed description.”
%
%(9) Section 17 of that Act (exceptions to section 16 where the employer’s alternative proposals are approved) shall cease to have effect.
%44Corporate trustees
%
%(1) Section 18 of the [1995 c. 26. ] Pensions Act 1995 (requirements for member-nominated directors of trustee company) shall be amended in accordance with subsections (2)  to (9)  of this section.
%
%(2) In subsection (1)  (duty of corporate trustee to make arrangements for selection of member-nominated directors)—
%
%($a$) for the words from “and the employer” to “satisfied” there shall be substituted “and there is no trustee of the scheme who is not a company”;
%
%($b$) the words “, subject to section 19,” and in paragraph ($b$), the words “, and the appropriate rules,” shall be omitted; and
%
%($c$) in paragraph ($a$), for “persons selected” there shall be substituted “the selection of persons nominated”.
%
%(3) In subsection (3)($a$)  (selected persons to be directors), for “in accordance with the appropriate rules” there shall be substituted “as a member-nominated director”.
%
%(4) In subsection (4)  (procedure for filling vacancies unfilled because of insufficient nominations), for “the appropriate rules” there shall be substituted “regulations”.
%
%(5) In subsection (5)  (period of service as a member-nominated director), after “six years” there shall be inserted “but for a member-nominated director to be eligible for selection again at the end of any period of service as such a director.”
%
%(6) After subsection (6)  there shall be inserted—
%
%“(6A) The arrangements must provide that, where the employer so requires, a person who is not a qualifying member of the scheme must have the employer’s approval to qualify for selection as a member-nominated director.”
%
%(7) In subsection (7)  (persons ceasing to be member-nominated directors on ceasing to be qualifying members of the scheme)—
%
%($a$) for “The arrangements must” there shall be substituted “The arrangements—
%
%($a$) must”; and
%
%($b$) at the end there shall be inserted “; and
%
%($b$) may provide for a member-nominated director who—
%
%(i) is a qualifying member of one of the following descriptions, that is to say, an active, deferred or pensioner member, and
%
%(ii) ceases (without ceasing to be a qualifying member) to be a qualifying member of that description,
%
%to cease, by virtue of that fact, to be a director.”
%
%(8) For subsection (8)  (companies that are trustees of two or more different trust schemes) there shall be substituted—
%
%“(8) Where—
%
%($a$) the same company is a trustee of two or more schemes by reference to each of which this section applies to the company, and
%
%($b$) the company does not, in the prescribed manner, elect that this subsection should not apply,
%
%the preceding provisions of this section and section 21(8)  shall have effect as if those schemes were a single scheme and the members of each of the schemes were members of that single scheme.”
%
%(9) After subsection (8)  there shall be inserted—
%
%“(9) Regulations may make provision in relation to arrangements under this section—
%
%($a$) supplementing the requirements of this section as to the matters to be contained in the arrangements; and
%
%($b$) providing for the manner in which, and the time within which, persons are, for the purposes of the arrangements, to be nominated and selected as member-nominated directors.
%
%(10) This section does not apply in the case of a trust scheme if the scheme is of a prescribed description.”
%
%(10) Sections 19 and 20 of that Act (exceptions to section 18 where the employer’s alternative proposals are approved and meaning of “appropriate rules”) shall cease to have effect.
%45Employer’s proposals for selection of trustees or directors
%
%(1) After section 18 of the [1995 c. 26. ] Pensions Act 1995 there shall be inserted—
%“Further provisions about the selection of trustees and directors
%18AEmployer’s proposals for selection of trustees or directors
%
%(1) Where, in the case of any trust scheme—
%
%($a$) the employer makes proposals for the adoption of arrangements for the nomination and selection of the trustees of the scheme,
%
%($b$) the proposed arrangements comply with all the requirements of section 16 and do not contain anything inconsistent with those requirements,
%
%($c$) the proposed arrangements comply with such other requirements as may be prescribed,
%
%($d$) the proposed arrangements are approved under such procedure for obtaining the views of members of the scheme as may be prescribed, and
%
%($e$) such other conditions are satisfied as may be prescribed,
%
%the trustees of the scheme shall secure that the proposed arrangements are made and implemented.
%
%(2) Where, in the case of any company which is trustee of a trust scheme of which there is no trustee who is not a company—
%
%($a$) the employer makes proposals for the adoption of arrangements for the nomination and selection of the directors of the company,
%
%($b$) the proposed arrangements comply with all the requirements of section 18 and do not contain anything inconsistent with those requirements,
%
%($c$) the proposed arrangements comply with such other requirements as may be prescribed,
%
%($d$) the proposed arrangements are approved under such procedure for obtaining the views of members of the scheme as may be prescribed, and
%
%($e$) such other conditions are satisfied as may be prescribed,
%
%the company shall secure that the proposed arrangements are made and implemented.
%
%(3) Arrangements made and implemented under this section may include provision that is different from that for which provision is made by regulations under section 16(9)  or 18(9) .
%
%(4) Regulations may make provision—
%
%($a$) as to the manner in which, and the time within which, arrangements proposed and approved for the purposes of this section are to be implemented by the trustees of a trust scheme or by a company which is a trustee of a trust scheme; and
%
%($b$) as to what is to happen where an approval for the purposes of this section of any arrangements ceases, in accordance with regulations, to have effect.
%
%(5) Regulations about the manner in which anything is approved for the purposes of this section may provide—
%
%($a$) for it to be treated as approved in accordance with the prescribed procedure where the Authority determine that prescribed conditions have been satisfied in relation to any departures from that procedure that have occurred; and
%
%($b$) for persons who do not object to it to be treated as having approved it.
%
%(6) Regulations may provide that, for the purposes of this section and any arrangements under this section, arrangements are to be taken as complying with the requirements of section 16 or 18, and as being consistent with those requirements, notwithstanding that nominations made for the purposes of the arrangements by a person or organisation which—
%
%($a$) represents for any particular purposes the interests of persons who are comprised in the membership of the scheme in question, and
%
%($b$) is of such a description as is specified in the regulations,
%
%are to be treated under the arrangements as nominations, or as the only nominations, made by qualifying members of the scheme.
%
%(7) Provision made by or under the preceding provisions of this section with respect to member-nominated trustees does not apply in the case of a trust scheme if—
%
%($a$) every member of the scheme is a trustee of the scheme and no other person is such a trustee; or
%
%($b$) every trustee of the scheme is a company.
%
%(8) Provision made by or under the preceding provisions of this section does not apply if the scheme is of a prescribed description.”
%
%(2) In section 68(2)($b$)  of that Act (power of trustee to modify scheme), for “17(2)” there shall be substituted “18A(1)”.
%
%(3) In section 117(2)($c$)  of that Act (overriding requirements), for “17(2)” there shall be substituted “18A(1)”.
%46Non-compliance in relation to arrangements or proposals
%
%(1) In section 21 of the [1995 c. 26. ] Pensions Act 1995 (consequences for trustees of failure to implement arrangements)—
%
%($a$) in subsections (1)  and (2) , the words “, or the appropriate rules,” shall be omitted;
%
%($b$) in subsections (1)  and (3), for “17(2)”, in each place, there shall be substituted “18A(1)”;
%
%($c$) in subsection (2) , for “19(2)”, in each place, there shall be substituted “18A(2)”;
%
%($d$) in subsection (3), the words “or rules” shall be omitted;
%
%($e$) in subsection (4) , for “17(2) , 18(1)  or 19(2)” there shall be substituted “18(1)  or 18A(1)  or (2)” and the words “(or further arrangements)” in paragraph ($a$), paragraph ($b$)  and the word “and” immediately preceding it shall be omitted;
%
%($f$) subsection (5)  shall cease to have effect;
%
%($g$) in subsection (6) , for “20” there shall be substituted “18A”;
%
%(h)in subsection (7), for “16 to 20” there shall be substituted “16 and 18” and the words “and this section”, paragraph ($b$)  and the word “and” immediately preceding paragraph ($b$)  shall be omitted;
%
%(i) in subsection (8) ($a$), for the words from “of the appropriate” to “given” there shall be substituted “for the purposes of section 18A of proposed arrangements must be given, in accordance with regulations under that section,”; and
%
%($j$) paragraph ($b$)  of subsection (8)  and the word “and” immediately preceding it shall be omitted.
%
%(2) In subsection (1)  of that section, after paragraph ($b$)  there shall be inserted “or
%
%($c$) regulations under section 16(9) ($b$)  have not been complied with,”.
%
%(3) In subsection (2)  of that section, after paragraph ($b$)  there shall be inserted “or
%
%($c$) regulations under section 18(9) ($b$)  have not been complied with,”.
%
%(4) After subsection (2)  of that section there shall be inserted—
%
%“(2A) Section 10 applies to an employer who has made a proposal for the purposes of section 18A but who contravenes any requirements of any regulations under section 18A relating to the submission of that proposal for approval.”
%
%(5) After subsection (6)  there shall be inserted—
%
%“(6A) In sections 16 to 18A “company” means a company within the meaning given by section 735(1)  of the [1985 c. 6. ] Companies Act 1985 or a company which may be wound up under Part V of the [1986 c. 45. ] Insolvency Act 1986 (unregistered companies).”
%Winding-up of schemes
%47Information to be given to the Authority
%
%(1) In section 22(1)  and (3)  of the [1995 c. 26. ] Pensions Act 1995 (circumstances in which provisions apply to a trust scheme the employer in relation to which has been subjected to an insolvency procedure), for “26”, in each case, there shall be substituted “26A”.
%
%(2) After section 26 of that Act there shall be inserted—
%“26AInformation to be given to the Authority in a s. 22 case
%
%(1) If at any time while section 22 applies in relation to a scheme—
%
%($a$) the trustees of the scheme do not include at least one person who the practitioner or official receiver has informed them is a person about whose independent status he is satisfied, and
%
%($b$) the trustees have no other reasonable grounds for believing that their number includes at least one person about whose independent status the practitioner or official receiver is satisfied,
%
%it shall be the duty of the trustees, as soon as reasonably practicable after it first appears to any one or more of them as mentioned in paragraphs ($a$)  and ($b$), to give notice to the Authority that the scheme appears not to have an independent trustee.
%
%(2) If a trust scheme is without trustees at any time while section 22 applies to it, it shall be the duty of every person involved in the administration of the scheme, as soon as reasonably practicable after it first appears to him that the scheme is without trustees, to give notice to the Authority that the scheme has no trustees.
%
%(3) No person shall be required to give a notice under subsection (1)  or (2)  at any time when it appears to him on reasonable grounds—
%
%($a$) that it is the intention of the practitioner or official receiver, for the purpose of complying with his duty under section 23(1)($b$), to make or secure the appointment of any person as a trustee of the scheme; and
%
%($b$) that the appointment will be made within the period specified by or under section 23(2)  for the performance of that duty.
%
%(4) No person shall be required to give a notice under subsection (2)  at any time when it appears to him, on reasonable grounds, that the Authority are already aware that the scheme has no trustees.
%
%(5) Where the practitioner or official receiver at any time informs the trustees of a trust scheme that he is not, or is no longer, satisfied about a person’s independent status, no account shall be taken for the purposes of subsection (1)($a$)  of any information that he was so satisfied which was given by the practitioner or official receiver to the trustees before that time.
%
%(6) References in this section to the practitioner or official receiver being satisfied about a person’s independent status are references to his being satisfied for the purposes of section 23 that that person is an independent person.
%
%(7) If subsection (1)  is not complied with, section 10 applies to any trustee who has failed to take all such steps as are reasonable to secure compliance.
%
%(8) Section 10 applies to any person who fails to comply with a duty imposed on him by subsection (2).
%26BInformation to be given in cases where s. 22 disapplied
%
%(1) Where, at any time—
%
%($a$) section 22 would apply in relation to a trust scheme but for regulations under section 118,
%
%($b$) the employer in relation to the scheme is the sole trustee of the scheme,
%
%($c$) there are persons involved in the administration of the scheme, and
%
%($d$) none of those persons has received an employer’s assurance relating to the scheme,
%
%it shall be the duty of every person who is involved in the administration of the scheme, as soon as reasonably practicable after it first appears to him as mentioned in paragraphs ($a$)  and ($b$), to give notice to the Authority that the case is one falling within paragraphs ($a$)  to ($d$) .
%
%(2) For the purposes of this section a person has received an employer’s assurance relating to a scheme if during the period while section 22 would have applied in relation to the scheme but for regulations under section 118—
%
%($a$) he has been informed by the person who is the employer in relation to the scheme that there is no reason why the employer should not continue to act as a trustee of the scheme;
%
%($b$) he has not subsequently been informed by the person who is the employer in relation to the scheme that that has ceased to be the case; and
%
%($c$) the trustees of the scheme have not changed since he was informed as mentioned in paragraph ($a$) .
%
%(3) No person shall be required to give a notice under subsection (1) —
%
%($a$) at any time when it appears to him, on reasonable grounds, that the Authority are already aware that the case is one falling within paragraphs ($a$)  to ($d$)  of that subsection;
%
%($b$) if a period is prescribed for the purposes of this paragraph, at any time in the prescribed period after the event by virtue of which the scheme became a scheme in relation to which section 22 would apply but for regulations under section 118; or
%
%($c$) at any other time that is prescribed for the purposes of this subsection.
%
%(4) Section 10 applies to any person who fails to comply with any duty imposed on him by subsection (1) .
%26CConstruction of ss. 26A and 26B
%
%(1) In sections 26A and 26B references, in relation to a scheme, to a person involved in the administration of the scheme are (subject to subsection (2) ) references to any person who is so involved otherwise than as—
%
%($a$) the employer in relation to that scheme;
%
%($b$) a trustee of the scheme;
%
%($c$) the auditor of the scheme or its actuary;
%
%($d$) a legal adviser of the trustees of the scheme;
%
%($e$) a fund manager for the scheme;
%
%($f$) a person acting on behalf of a person who is involved in the administration of the scheme;
%
%($g$) a person providing services to a person so involved;
%
%(h)a person acting in his capacity as an employee of a person so involved;
%
%(i) a person who would fall within any of paragraphs ($f$)  to (h) if persons acting in relation to the scheme in any capacity mentioned in the preceding paragraphs were treated as involved in the administration of a scheme.
%
%(2) In this section references, in relation to a scheme, to a person involved in the administration of the scheme do not include references to persons of a particular description if regulations provide for persons of that description to be excluded from those references.
%
%(3) If regulations so provide in relation to any provision of section 26A or 26B, so much of that provision as requires any notice to be given as soon as reasonably practicable after a particular time shall have effect as a requirement to give that notice within such period after that time as may be prescribed.”
%
%(3) In subsection (2)  of section 118 of that Act (powers to provide for sections 22 to 26 not to apply in the case of certain schemes), for “sections 22 to 26” there shall be substituted “some or all of the provisions of sections 22 to 26C”.
%
%(4) After that subsection there shall be inserted—
%
%“(3) Regulations may modify sections 26A and 26B for the purpose of requiring prescribed persons, in addition to or instead of the persons who (apart from the regulations) would be required to provide information to the Authority under those sections, to be subject to the duties imposed by those sections.”
%
%(5) In section 178($b$)  of the [1993 c. 48. ] Pension Schemes Act 1993 (regulations providing for who is to be treated as a trustee of a scheme), at the end there shall be inserted “or sections 22 to 26C of the Pensions Act 1995”.
%48Modification of scheme to secure winding-up
%
%After section 71 of the [1995 c. 26. ] Pensions Act 1995 (effect of modification orders under section 69) there shall be inserted—
%“71AModification by Authority to secure winding-up
%
%(1) The Authority may at any time while—
%
%($a$) an occupational pension scheme is being wound up, and
%
%($b$) the employer in relation to the scheme is subject to an insolvency procedure,
%
%make an order modifying that scheme with a view to ensuring that it is properly wound up.
%
%(2) The Authority shall not make such an order except on an application made to them, at a time such as is mentioned in subsection (1), by the trustees or managers of the scheme.
%
%(3) Except in so far as regulations otherwise provide, an application for the purposes of this section must be made in writing.
%
%(4) Regulations may make provision—
%
%($a$) for the form and manner in which an application for the purposes of this section is to be made to the Authority;
%
%($b$) for the matters which are to be contained in such an application;
%
%($c$) for the documents which must be attached to an application for the purposes of this section or which must otherwise be delivered to the Authority with or in connection with any such application;
%
%($d$) for persons to be required, before such time as may be prescribed, to give such notifications of the making of an application for the purposes of this section as may be prescribed;
%
%($e$) for the matters which are to be contained in a notification of such an application;
%
%($f$) for persons to have the opportunity, for a prescribed period, to make representations to the Authority about the matters to which such an application relates;
%
%($g$) for the manner in which the Authority are to deal with any such application.
%
%(5) The power of the Authority to make an order under this section—
%
%($a$) shall be limited to what they consider to be the minimum modification necessary to enable the scheme to be properly wound up; and
%
%($b$) shall not include power to make any modification that would have a significant adverse effect on—
%
%(i) the accrued rights of any member of the scheme; or
%
%(ii) any person’s entitlement under the scheme to receive any benefit.
%
%(6) A modification of an occupational pension scheme by an order of the Authority under this section shall be as effective in law as if—
%
%($a$) it had been made under powers conferred by or under the scheme;
%
%($b$) the modification made by the order were capable of being made in exercise of such powers notwithstanding any enactment, rule of law or rule of the scheme that would have prevented their exercise for the making of that modification; and
%
%($c$) the exercise of such powers for the making of that modification would not have been subject to any enactment, rule of law or rule of the scheme requiring the implementation of any procedure or the obtaining of any consent in connection with the making of a modification.
%
%(7) Regulations may provide that, in prescribed circumstances, this section—
%
%($a$) does not apply in the case of occupational pension schemes of a prescribed class or description; or
%
%($b$) in the case of occupational pension schemes of a prescribed class or description applies with prescribed modifications.
%
%(8) The times when an employer in relation to an occupational pension scheme shall be taken for the purposes of this section to be subject to an insolvency procedure are—
%
%($a$) in the case of a trust scheme, while section 22 applies in relation to the scheme; and
%
%($b$) in the case of a scheme that is not a trust scheme, while section 22 would apply in relation to the scheme if it were a trust scheme;
%
%and for the purposes of this subsection no account shall be taken of modifications or exclusions contained in any regulations under section 118. 
%
%(9) The Authority shall not be entitled to make an order under this section in relation to a public service pension scheme.”
%49Reports about winding-up
%
%(1) After section 72 of the [1995 c. 26. ] Pensions Act 1995 there shall be inserted—
%“Supervision of winding-up
%72AReports to Authority about winding-up
%
%(1) Where—
%
%($a$) an occupational pension scheme is being wound up, and
%
%($b$) the winding-up is one beginning at a time (whether before or after the passing of this Act) by reference to which regulations provide that it is to be a winding-up to which this section applies,
%
%it shall be the duty of the trustees or managers, in accordance with this section, to make periodic reports in writing to the Authority about the progress of the winding-up.
%
%(2) In the case of each winding-up, the first report to be made under this section shall be made—
%
%($a$) except in a case to which paragraph ($b$)  applies—
%
%(i) after the end of the prescribed period beginning with the day on which the winding-up began; and
%
%(ii) before the end of the prescribed period that begins with the end of the period that applies for the purposes of sub-paragraph (i) ;
%
%and
%
%($b$) in a case where the winding-up began before the coming into force of the regulations which (for the purposes of subsection (1)($b$) ) prescribe the time by reference to which the winding-up is one to which this section applies, before such date as may be prescribed by those regulations.
%
%(3) Subject to subsection (4) , each subsequent report made under this section in the case of a winding-up shall be made no more than twelve months after the date which (apart from any postponement under subsection (4) ) was the latest date for the making of the previous report required to be made in the case of that winding-up.
%
%(4) If, in the case of any report required to be made under subsection (3), the Authority consider (whether on an application made for the purpose or otherwise) that it would be appropriate to do so, they may, at any time before the latest time for the making of that report, postpone that latest time by such period as they think fit.
%
%(5) The latest time for making a report shall not be postponed under subsection (4)  by more than twelve months.
%
%(6) Subject to the application of the limit specified in subsection (5)  to the cumulative period of the postponements, more than one postponement may be made under subsection (4)  in the case of the same report.
%
%(7) A report under this section—
%
%($a$) must contain such information and statements as may be prescribed; and
%
%($b$) must be made in accordance with the prescribed requirements.
%
%(8) Regulations may—
%
%($a$) provide that, in prescribed circumstances, there shall be no obligation to make a report that would otherwise fall to be made under this section;
%
%($b$) make provision for the period within which, and the manner in which, applications may be made for a postponement under subsection (4); and
%
%($c$) modify subsections (3)  and (5)  by substituting periods of different lengths for the periods for the time being specified in those subsections.
%
%(9) If there is any failure by the trustees or managers of any scheme to comply with their duty to make a report in accordance with the requirements imposed by or under this section—
%
%($a$) section 3 applies, if the scheme is a trust scheme, to any trustee who has failed to take all such steps as are reasonable to secure compliance; and
%
%($b$) section 10 applies (irrespective of the description of scheme involved) to any trustee or manager who has failed to take all such steps.”
%
%(2) In section 124 of that Act (interpretation of Part I), after subsection (3)  there shall be inserted—
%
%“(3A) In a case of the winding-up of an occupational pension scheme in pursuance of an order of the Authority under section 11 or of an order of a court, the winding-up shall (subject to subsection (3E)) be taken for the purposes of this Part to begin—
%
%($a$) if the order provides for a time to be the time when the winding-up begins, at that time; and
%
%($b$) in any other case, at the time when the order comes into force.
%
%(3B)In a case of the winding-up of an occupational pension scheme in accordance with a requirement or power contained in the rules of the scheme, the winding-up shall (subject to subsections (3C) to (3E)) be taken for the purposes of this Part to begin—
%
%($a$) at the time (if any) which under those rules is the time when the winding-up begins; and
%
%($b$) if paragraph ($a$)  does not apply, at the earliest time which is a time fixed by the trustees or managers as the time from which steps for the purposes of the winding-up are to be taken.
%
%(3C)Subsection (3B) shall not require a winding-up of a scheme to be treated as having begun at any time before the end of any period during which effect is being given—
%
%($a$) to a determination under section 38 that the scheme is not for the time being to be wound up; or
%
%($b$) to a determination in accordance with the rules of the scheme to postpone the commencement of a winding-up.
%
%(3D)In subsection (3B)($b$)  the reference to the trustees or managers of the scheme shall have effect in relation to any scheme the rules of which provide for a determination that the scheme is to be wound up to be made by persons other than the trustees or managers as including a reference to those other persons.
%
%(3E)Subsections (3A)  to (3D) above do not apply for such purposes as may be prescribed.”
%
%(3) After section 49 of that Act (other responsibilities of trustees employers etc.) there shall be inserted—
%“49ARecord of winding-up decisions
%
%(1) Except so far as regulations otherwise provide, the trustees or managers of an occupational pension scheme shall keep written records of—
%
%($a$) any determination for the winding-up of the scheme in accordance with its rules;
%
%($b$) decisions as to the time from which steps for the purposes of the winding-up of the scheme are to be taken;
%
%($c$) determinations under section 38;
%
%($d$) determinations in accordance with the rules of the scheme to postpone the commencement of a winding-up of the scheme.
%
%(2) For the purpose of this section—
%
%($a$) the determinations and decisions of which written records must be kept under this section include determinations and decisions by persons who—
%
%(i) are not trustees or managers of a scheme, but
%
%(ii) are entitled, in accordance with the rules of a scheme, to make a determination for its winding-up;
%
%and
%
%($b$) regulations may, in relation to such determinations or decisions as are mentioned in paragraph ($a$), impose obligations to keep written records on the persons making the determinations or decisions (as well as, or instead of, on the trustees or managers).
%
%(3) Regulations may provide for the form and content of any records that are required to be kept under this section.
%
%(4) Section 3 applies to any trustee of a scheme who fails to take all such steps as are reasonable to secure compliance by the trustees of that scheme with the obligations imposed on them by this section.
%
%(5) Section 10 applies to any trustee or manager of a scheme who fails to take all such steps as are reasonable to secure compliance by the trustees or managers of that scheme with those obligations.”
%50Directions for facilitating winding-up
%
%After the section 72A inserted in the [1995 c. 26. ] Pensions Act 1995 by section 49 there shall be inserted—
%“72BDirections by Authority for facilitating winding-up
%
%(1) Subject to the following provisions of this section, the Authority shall have power, at any time after the winding-up of an occupational pension scheme has begun, to give directions under this section if they consider that the giving of the direction is appropriate on any of the grounds set out in subsection (2).
%
%(2) Those grounds are—
%
%($a$) that the trustees or managers of the scheme are not taking all the steps in connection with the winding-up that the Authority consider would be being taken if the trustees or managers were acting reasonably;
%
%($b$) that steps being taken by the trustees or managers for the purposes of the winding-up involve things being done with what the Authority consider to be unreasonable delay;
%
%($c$) that the winding-up is being obstructed or unreasonably delayed by the failure of any person—
%
%(i) to provide information to the trustees or managers;
%
%(ii) to provide information to a person involved in the administration of the scheme;
%
%(iii) to provide information to a person of a prescribed description; or
%
%(iv) to take any step (other than the provision of information) that he has been asked to take by the trustees or managers;
%
%($d$) that the winding-up would be likely to be facilitated or accelerated by the taking by any person other than the trustees or managers of any other steps;
%
%($e$) that in any prescribed circumstances not falling within paragraphs ($a$)  to ($d$) —
%
%(i) the provision by any person of any information to the trustees or managers or to any other person, or
%
%(ii) the taking of any other step by any person,
%
%would be likely to facilitate or accelerate the progress of the winding-up.
%
%(3) Except in prescribed circumstances, the power of the Authority to give a direction under this section in the case of a winding-up shall be exercisable only where—
%
%($a$) periodic reports about the progress of the winding-up are required to be made under section 72A; and
%
%($b$) the first report that has to be made for the purposes of that section in the case of that winding-up either has been made or should have been made.
%
%(4) Regulations may provide that, in prescribed circumstances, the Authority shall not give a direction on the ground set out in subsection (2)($e$)  except in response to an application made by the trustees or managers of the scheme for the giving of a direction on that ground.
%
%(5) A direction under this section is a direction in writing given to and imposing requirements on—
%
%($a$) any or all of the trustees or managers of the scheme;
%
%($b$) a person who is involved in its administration; or
%
%($c$) a person of a prescribed description.
%
%(6) The requirements that may be imposed by a direction under this section are any requirement for the person to whom it is given, within such period specified in the direction as the Authority may consider reasonable—
%
%($a$) to provide the trustees or managers with all such information as may be specified or described in the direction;
%
%($b$) to provide a person involved in the administration of the scheme with all such information as may be so specified or described;
%
%($c$) to provide a person who is of a prescribed description with all such information as may be so specified or described;
%
%($d$) to take such steps (other than the provision of information) as may be so specified or described.
%
%(7) If, at any time before the end of a period within which any step is required by a direction under this section to be taken by any person, the Authority consider (whether on an application made for the purpose or otherwise) that it would be appropriate to do so, they may extend (or further extend) that period until such time as they think fit.
%
%(8) Regulations may—
%
%($a$) impose limitations on the steps that a person may be required to take by a direction under this section;
%
%($b$) make provision for the period within which, and the manner in which, applications may be made for a period to be extended (or further extended) under subsection (7) .
%
%(9) In this section references, in relation to a scheme, to a person involved in the administration of the scheme are (subject to subsection (10) ) references to any person who is so involved otherwise than as—
%
%($a$) the employer in relation to that scheme;
%
%($b$) a trustee or manager of the scheme;
%
%($c$) the auditor of the scheme or its actuary;
%
%($d$) a legal adviser of the trustees or managers of the scheme;
%
%($e$) a fund manager for the scheme;
%
%($f$) a person acting on behalf of a person who is involved in the administration of the scheme;
%
%($g$) a person providing services to a person so involved;
%
%(h)a person acting in his capacity as an employee of a person so involved;
%
%(i) a person who would fall within any of paragraphs ($f$)  to (h) if persons acting in relation to the scheme in any capacity mentioned in the preceding paragraphs were treated as involved in the administration of a scheme.
%
%(10) In this section references, in relation to a scheme, to a person involved in the administration of the scheme do not include references to persons of a particular description if regulations provide for persons of that description to be excluded from those references.
%72CDuty to comply with directions under s. 72B
%
%(1) It shall be the duty of any person to whom a direction is given under section 72B to comply with it.
%
%(2) Where a direction is given under section 72B to the trustees of a trust scheme, section 3 applies to any trustee who fails, without reasonable excuse, to take all such steps as are reasonable to secure compliance with it.
%
%(3) Section 10 applies to any trustee or manager of a scheme who fails, without reasonable excuse, to take all such steps as are reasonable to secure compliance by the trustees or managers of that scheme with any direction given to them under section 72B.
%
%(4) Section 10 applies to any person who—
%
%($a$) is a person to whom a direction under section 72B is given otherwise than in the capacity of a trustee or manager; and
%
%($b$) without reasonable excuse, fails to comply with that direction.
%
%(5) For the purposes of this section it shall not be a reasonable excuse in relation to any failure to provide information in pursuance of a direction under section 72B that the provision of that information would (but for the duty imposed by subsection (1)  of this section) involve a breach by any person of a duty owed to another not to disclose that information.”
%Other provisions
%51Restriction on index-linking where annuity tied to investments
%
%(1) In section 51(2)  of the [1995 c. 26. ] Pensions Act 1995 (annual increases in rate of pension), for “Subject to section 52” there shall be substituted “Subject to sections 51A and 52”.
%
%(2) After section 51 of that Act there shall be inserted—
%“51ARestriction on increase where annuity tied to investments
%
%(1) No increase under section 51 is required to be made, at any time on or after the relevant date, of so much of any pension under a money purchase scheme as—
%
%($a$) is payable by way of an annuity the amount of which for any year after the first year of payment is determined (whether under the terms of the scheme or under the terms of the annuity contract in pursuance of which it is payable) by reference to fluctuations in the value of, or the return from, particular investments;
%
%($b$) does not represent benefits payable in respect of the protected rights of any member of the scheme; and
%
%($c$) satisfies such other conditions (if any) as may be prescribed.
%
%(2) For the purposes of this section it shall be immaterial whether the annuity in question is payable out of the funds of the scheme in question or under an annuity contract entered into for the purposes of the scheme.
%
%(3) In this section “the relevant date” means the date appointed for the coming into force of section 51 of the Child Support, Pensions and Social Security Act 2000.”
%52Information for members of schemes etc
%
%(1) In subsection (1)  of section 113 of the [1993 c. 48. ] Pension Schemes Act 1993 (regulations as to information to be provided to scheme members etc.), for the word “and” at the end of paragraph ($c$)  there shall be substituted—
%
%“($ca$) of the pensions and other benefits an entitlement to which would be likely to accrue to the member, or be capable of being secured by him, in respect of the rights that may arise under it; and”.
%
%(2) After subsection (3)  of that section there shall be inserted—
%
%“(3A) The regulations may provide for the information that must be given to be determined, in whole or part, by reference to guidance which—
%
%($a$) is prepared and from time to time revised by a prescribed body; and
%
%($b$) is for the time being approved by the Secretary of State.
%
%(3B)The regulations may, in relation to cases where a scheme is being wound up, contain—
%
%($a$) provision conferring power on the Regulatory Authority, at times before the period expires, to extend any period specified in the regulations as the period within which a requirement imposed by the regulations must be complied with; and
%
%($b$) provision as to the contents of any application for the exercise of such a power and as to the form and manner in which, and the time within which, any such application must be made.”
%53Jurisdiction of the Pensions Ombudsman
%
%(1) Section 146 of the [1993 c. 48. ] Pension Schemes Act 1993 (functions of the Pensions Ombudsman) shall be amended as follows.
%
%(2) In subsection (1), after paragraph ($b$)  there shall be inserted—
%
%“(ba)a complaint made to him by or on behalf of an independent trustee of a trust scheme who, in connection with any act or omission which is an act or omission either—
%
%(i) of trustees of the scheme who are not independent trustees, or
%
%(ii) of former trustees of the scheme who were not independent trustees,
%
%alleges maladministration of the scheme,”.
%
%(3) In that subsection, for the words after sub-paragraph (ii)  of paragraph ($d$)  there shall be substituted—
%
%“and in a case falling within sub-paragraph (ii)  references in this Part to the scheme to which the reference relates are references to each of the schemes,
%
%($e$) any dispute not falling within paragraph ($f$)  between different trustees of the same occupational pension scheme,
%
%($f$) any dispute, in relation to a time while section 22 of the [1995 c. 26. ] Pensions Act 1995 (schemes subject to insolvency procedures) applies in relation to an occupational pension scheme, between an independent trustee of the scheme and either—
%
%(i) trustees of the scheme who are not independent trustees, or
%
%(ii) former trustees of the scheme who were not independent trustees, and
%
%($g$) any question relating, in the case of an occupational pension scheme with a sole trustee, to the carrying out of the functions of that trustee.”
%
%(4) After that subsection there shall be inserted—
%
%“(1A)The Pensions Ombudsman shall not investigate or determine any dispute or question falling within subsection (1)($c$)  to ($g$)  unless it is referred to him—
%
%($a$) in the case of a dispute falling within subsection (1)($c$), by or on behalf of the actual or potential beneficiary who is a party to the dispute,
%
%($b$) in the case of a dispute falling within subsection (1)($d$), by or on behalf of any of the parties to the dispute,
%
%($c$) in the case of a dispute falling within subsection (1)($e$), by or on behalf of at least half the trustees of the scheme,
%
%($d$) in the case of a dispute falling within subsection (1)($f$), by or on behalf of the independent trustee who is a party to the dispute,
%
%($e$) in the case of a question falling within subsection (1)($g$), by or on behalf of the sole trustee.
%
%(1B)For the purposes of this Part, any reference to or determination by the Pensions Ombudsman of a question falling within subsection (1)($g$)  shall be taken to be the reference or determination of a dispute.”
%
%(5) In subsection (3)  (persons responsible for the management of the scheme to be the trustees and managers and employer), after “occupational pension scheme” there shall be inserted “or a personal pension scheme”.
%
%(6) For paragraph ($a$)  of subsection (6)  (exclusion of the Ombudsman’s jurisdiction where court proceedings have been begun) there shall be substituted—
%
%“($a$) if, before the making of the complaint or the reference of the dispute—
%
%(i) proceedings in respect of the matters which would be the subject of the investigation have been begun in any court or employment tribunal, and
%
%(ii) those proceedings are proceedings which have not been discontinued or which have been discontinued on the basis of a settlement or compromise binding all the persons by or on whose behalf the complaint or reference is made;”.
%
%(7) In subsection (7)  (persons who are actual or potential beneficiaries)—
%
%($a$) after paragraph ($b$)  there shall be inserted—
%
%“(ba)a person who is entitled to a pension credit as against the trustees or managers of the scheme;” and
%
%($b$) in sub-paragraph (i)  of paragraph ($c$), for “paragraph ($a$)  or ($b$)” there shall be substituted “paragraph ($a$), ($b$)  or (ba)”.
%
%(8) In subsection (8)  (interpretation) after the definition of “employer” there shall be inserted—
%
%““independent trustee”, in relation to a scheme, means—
%
%($a$) a trustee of the scheme appointed under section 23(1)($b$)  of the [1995 c. 26. ] Pensions Act 1995 (appointment of independent trustee by insolvency practitioner or official receiver),
%
%($b$) a person appointed under section 7(1)  of that Act to replace a trustee falling within paragraph ($a$)  or this paragraph;”.
%
%(9) In subsection (1) —
%
%($a$) for “complaints and disputes” there shall be substituted “matters”;
%
%($b$) in paragraph ($b$), for the words from “is to” to the end of the paragraph there shall be substituted “are references to the other scheme referred to in that sub-paragraph”; and
%
%($c$) in paragraphs ($c$)  and ($d$), the words “which arises”, in each place where they occur, shall be omitted.
%
%(10) Subsection (6)  does not have effect in relation to proceedings begun before the day appointed under section 86 for the coming into force of this section.
%54Investigations by the Pensions Ombudsman
%
%(1) The [1993 c. 48. ] Pension Schemes Act 1993 shall be amended as follows.
%
%(2) In section 148(5)  (meaning of parties to an investigation for the purposes of staying proceedings), after paragraph ($b$)  there shall be inserted—
%
%“(ba)any actual or potential beneficiary of the scheme whose interests are or may be affected by the matters to which the complaint or dispute relates,
%
%(bb)any actual or potential beneficiary of the scheme whose interests it is reasonable to suppose might be affected by—
%
%(i) the Pensions Ombudsman’s determination of the complaint or dispute, or
%
%(ii) directions that may be given by the Ombudsman in consequence of that determination,”.
%
%(3) For subsection (1)  of section 149 (obligation to allow persons to comment on allegations in complaint or reference) there shall be substituted—
%
%“(1) Where the Pensions Ombudsman proposes to conduct an investigation into a complaint made or dispute referred under this Part, he shall—
%
%($a$) give every person against whom allegations are made in the complaint or reference an opportunity to comment on those allegations,
%
%($b$) give every person responsible for the management of the scheme to which the complaint or reference relates an opportunity to make representations to him about the matters to which the complaint or dispute relates, and
%
%($c$) give every actual or potential beneficiary of that scheme whose interests are or may be affected by the matters to which the complaint or dispute relates, an opportunity to make representations about those matters.
%
%(1A)Subject to subsection (1B), subsection (1)  shall not require an opportunity to make comments or representations to be given to any person if the Pensions Ombudsman is satisfied that that person is—
%
%($a$) a person who, as the person or one of the persons making the complaint or reference, has had his opportunity to make comments or representations about the matters in question; or
%
%($b$) a person whose interests in relation to the matters to which the complaint or dispute relates are being represented, in accordance with rules under this section, by a person who has been given an appropriate opportunity to make comments or representations.
%
%(1B)The Pensions Ombudsman shall, under subsection (1), give an opportunity to make comments and representations to a person falling within subsection (1A)($a$)  in any case in which that person is a person who, in accordance with rules, is appointed or otherwise determined, after the making of the complaint or reference, to represent the interests of other persons in relation to the matters to which the complaint or dispute relates.”
%
%(4) In subsection (3)  of section 149 (matters as to which rules may be made), for “and” at the end of paragraph ($b$)  there shall be substituted—
%
%“(ba)for the interests of all of a number of persons who—
%
%(i) are actual or potential beneficiaries of the scheme to which the complaint or reference relates, and
%
%(ii) appear to have the same interest in relation to any of the matters to which the complaint or dispute relates,
%
%to be represented for the purposes of the investigation by such one or more of them, or such other person, as may be appointed by the Ombudsman or otherwise determined in accordance with the rules,”.
%
%(5) In that subsection, at the end of paragraph ($c$), there shall be inserted “and
%
%($d$) for the payment of legal expenses incurred by a party to an investigation (as defined in section 148(5) ) out of funds held for the purposes of the scheme to which the complaint or reference relates.”
%
%(6) After subsection (7)  of section 149 there shall be inserted—
%
%“(8) References in this section to the matters to which a complaint or dispute relates include references to any matter which it is reasonable to suppose might form the subject of—
%
%($a$) the Pensions Ombudsman’s determination of the complaint or dispute, or
%
%($b$) any directions that may be given by the Ombudsman in consequence of that determination.”
%
%(7) In subsection (1)  of section 151 (persons to be given notice of a determination by the Ombudsman), at the end of paragraph ($b$)  there shall be inserted “and
%
%($c$) to every other person who was required under section 149 to be given an opportunity—
%
%(i) to comment on an allegation in the complaint or reference, or
%
%(ii) to make representations about matters to which the complaint or reference relates,”.
%
%(8) In subsection (3)  of section 151 (persons bound by determinations), for “and” at the end of paragraph ($b$)  there shall be substituted—
%
%“(ba)any person who under section 149 was given such an opportunity to make any such comment or representation as is mentioned in subsection (1)($c$)  of this section,
%
%(bb)any person whose interests were represented by a person falling within any of the preceding paragraphs, and”;
%
%( )and, in paragraph ($c$)  of that subsection for “paragraph ($a$)  or ($b$)” there shall be substituted “any of paragraphs ($a$)  to (bb)”;
%
%(9) Nothing in any provision made by this section shall—
%
%($a$) apply in relation to any complaint or reference made to the Pensions Ombudsman before the day on which this section comes into force; or
%
%($b$) authorise the making of any provision applying in relation to any such complaint or reference.
%55Prohibition on different rules for overseas residents etc
%
%After section 66 of the [1995 c. 26. ] Pensions Act 1995 there shall be inserted—
%“Treatment of overseas residents etc.
%66AProhibition on different rules for overseas residents etc
%
%(1) This section applies where an occupational pension scheme contains provisions contravening subsection (2)  or (3) .
%
%(2) Except so far as regulations otherwise provide, provisions of an occupational pension scheme contravene this subsection to the extent that they would (apart from this section) have an effect with respect to—
%
%($a$) the entitlement of any person to benefits under the scheme, or
%
%($b$) the payment to any person of benefits under the scheme,
%
%which would be different according to whether or not a place outside the United Kingdom is specified by that person as the place to which he requires payments of benefits under the scheme to be made to him.
%
%(3) Except so far as regulations otherwise provide, provisions of an occupational pension scheme contravene this subsection to the extent that they would (apart from this section) have an effect with respect to—
%
%($a$) the entitlement of any person to remain a member of the scheme,
%
%($b$) the eligibility of any person to remain a person by or in respect of whom contributions are made towards or under the scheme, or
%
%($c$) the making by or in respect of any person who is a member of the scheme of any contributions towards or under the scheme,
%
%which would be different according to whether that person works wholly in the United Kingdom or wholly or partly outside the United Kingdom.
%
%(4) Provisions contravening subsection (2)  shall have effect, in relation to all times after the coming into force of section 55 of the Child Support, Pensions and Social Security Act 2000, as if they made the same provision in relation to a person who requires payments of benefits to be made to a place outside the United Kingdom as they make in relation to a person in whose case all payments of benefits fall to be made to a place in the United Kingdom.
%
%(5) Provisions contravening subsection (3)  shall have effect, in relation to all times after the coming into force of section 55 of the Child Support, Pensions and Social Security Act 2000, as if they made the same provision in relation to persons working wholly or partly outside the United Kingdom as they make in relation to persons working wholly in the United Kingdom.
%
%(6) This section—
%
%($a$) shall be without prejudice to any enactment under which any amount is to be or may be deducted, or treated as deducted, from amounts payable by way of benefits under the scheme or treated as so payable; and
%
%($b$) shall not apply in relation to so much of any provision of a scheme as is required for securing compliance with the conditions of any approval, exemption or relief given or available under the Tax Acts.”
%56Miscellaneous amendments and alternative to anti-franking rules
%
%Schedule 5 (which contains miscellaneous amendments of the [1993 c. 48. ] Pension Schemes Act 1993 and the [1995 c. 26. ] Pensions Act 1995 and makes provision for an alternative to the anti-franking rules in Part III of that Act of 1993) shall have effect.
%Chapter IIIWar Pensions
%57Rights of appeal
%
%(1) After section 5 of the [1943 c. 39. ] Pensions Appeal Tribunals Act 1943 there shall be inserted—
%“5AAppeals in other cases
%
%(1) Where, in the case of any such claim as is referred to in section 1, 2 or 3 of this Act, the Minister makes a specified decision—
%
%($a$) he shall notify the claimant of the decision, specifying the ground on which it is made, and
%
%($b$) thereupon an appeal against the decision shall lie to the Tribunal on the issue whether the decision was rightly made on that ground.
%
%(2) For the purposes of subsection (1), a “specified decision” is a decision (other than a decision which is capable of being the subject of an appeal under any other provision of this Act) which is of a kind specified by the Minister in regulations made by statutory instrument.
%
%(3) Regulations under this section shall not be made unless a draft of the regulations has been laid before, and approved by a resolution of, each House of Parliament.”
%
%(2) In subsection (2)  of section 6 of that Act (further appeals to High Court in cases of appeals brought under sections 1 to 4), for the words from “section one” to “four” there shall be substituted “sections 1, 2, 3, 4 or 5A”.
%
%(3) In subsection (2A)  of that section (setting aside of decisions of Tribunal on appeals under sections 1, 2, 3 or 4), for “or 4” there shall be substituted “, 4 or 5A”.
%
%(4) Section 1(2)  of the [1949 c. 12. ] Pensions Appeal Tribunals Act 1949 (no right of appeal against rejection of claims relating to service before 3rd September 1939) shall cease to have effect.
%58Time limit for appeals
%
%(1) In section 8 of the [1943 c. 39. ] Pensions Appeal Tribunals Act 1943 (time limit for appeals), in subsection (1)  (notice of appeal to be given within twelve months of notification of decision or assessment), for the words from “twelve months after” to “in any other case,” there shall be substituted “six months after”.
%
%(2) After subsection (3)  of that section there shall be inserted—
%
%“(4) The Minister may by regulations made by statutory instrument amend subsections (1)  and (3)  so as to substitute a different number of months for any number of months specified there.
%
%(5) The Minister may by regulations made by statutory instrument provide that the Tribunal may, in circumstances prescribed in the regulations, allow an appeal to be brought not later than twelve months after the end of any period limited by this section.
%
%(6) Regulations under subsection (4)  or (5)  shall not be made unless a draft of the regulations has been laid before, and approved by a resolution of, each House of Parliament.”
%
%(3) Subsection (1)  shall not have effect in relation to—
%
%($a$) decisions from which an appeal lies to the Tribunal under sections 1 to 4 of the [1943 c. 39. ] Pensions Appeal Tribunals Act 1943 and which are made before the day on which that subsection comes into force, or
%
%($b$) decisions or assessments from which an appeal lies to the Tribunal under section 5(2)  of that Act and which are made before the day on which that subsection comes into force.
%
%(4) In relation to decisions falling within subsection (3)($a$)  of this section, section 8 of the [1943 c. 39. ] Pensions Appeal Tribunals Act 1943 shall have effect as if for paragraphs ($a$)  to ($c$)  of subsection (1)  of that section there were substituted “the day on which section 58(1)  of the Child Support, Pensions and Social Security Act 2000 came into force”.
%
%(5) In section 6(1)  of the [1921 c. 49. ] War Pensions Act 1921 (notice of appeal to be given within twelve months of notification of rejection of claim), for “twelve” there shall be substituted “six”.
%
%(6) Subsection (5)  shall not have effect in relation to any appeal if the decision or assessment appealed against was made before the day on which that subsection comes into force.
%59Matters relevant on appeal to Pensions Appeal Tribunal
%
%Before section 6 of the [1943 c. 39. ] Pensions Appeal Tribunals Act 1943 (constitution, jurisdiction and procedure of Pensions Appeal Tribunal), there shall be inserted—
%“5BMatters relevant on appeal
%
%In deciding any appeal, a Pensions Appeal Tribunal—
%
%($a$) need not consider any issue that is not raised by the appellant or the Minister in relation to the appeal; and
%
%($b$) shall not take into account any circumstances not obtaining at the time when the decision appealed against was made.”
%60Constitution and procedure of Pensions Appeal Tribunals
%
%(1) In sub-paragraph (2)  of paragraph 2 of the Schedule to the [1943 c. 39. ] Pensions Appeal Tribunals Act 1943 (remuneration for members of Pensions Appeal Tribunals), after “remuneration” there shall be inserted “and allowances”.
%
%(2) After that sub-paragraph there shall be inserted—
%
%“(2A) Subject to sub-paragraphs (3)  and (4)  below, a member of such a Tribunal shall hold and vacate his office in accordance with the terms of his appointment, but shall be eligible for reappointment.”
%
%(3) After paragraph 2 of that Schedule, there shall be inserted—
%
%“2A(1) The Lord Chancellor shall ensure that the appointments made by him under paragraph 2 above have the effect, in the case of each of the Tribunals, that the persons holding office as members of that Tribunal at all times include—
%
%($a$) persons who are legally qualified;
%
%($b$) persons who are medically qualified;
%
%($c$) persons with knowledge or experience of service in Her Majesty’s naval, military or air forces; and
%
%($d$) other persons.
%
%(2) For the purposes of this Schedule a person is legally qualified if—
%
%($a$) he has a seven year general qualification within the meaning of section 71 of the [1990 c. 41. ] Courts and Legal Services Act 1990;
%
%($b$) he is an advocate or solicitor in Scotland of at least seven years' standing; or
%
%($c$) he is a member of the Bar of Northern Ireland or solicitor of the Supreme Court of Northern Ireland of at least seven years' standing.
%
%(3) For the purposes of this Schedule a person is medically qualified if he is a duly qualified medical practitioner of at least seven years' standing.
%
%(4) In making any appointment under paragraph 2 it shall be the duty of the Lord Chancellor to have regard to the desirability of having as members of the Tribunals persons with knowledge or experience of matters relating to the disability of persons.
%
%2B(1) A President of Pensions Appeal Tribunals and a Deputy President of Pensions Appeal Tribunals may be appointed for each part of the United Kingdom.
%
%(2) The person entitled to appoint a person under this paragraph to be a President or Deputy President of Pensions Appeal Tribunals shall be—
%
%($a$) in the case of an appointment for England and Wales, the Lord Chancellor;
%
%($b$) in the case of an appointment for Scotland, the Lord President of the Court of Session; and
%
%($c$) in the case of an appointment for Northern Ireland, the Lord Chief Justice of Northern Ireland.
%
%(3) Only legally qualified members of a Pensions Appeal Tribunal shall be eligible for appointment under this paragraph.
%
%(4) A person shall cease to be President or Deputy President of Pensions Appeal Tribunals if he ceases to be a member of any such Tribunal.
%
%(5) The Deputy President of Pensions Appeal Tribunals for any part of the United Kingdom shall carry out such of the functions of the President for that part of the United Kingdom as that President assigns to him.
%
%(6) If at any time the President of Pensions Appeal Tribunals for any part of the United Kingdom is temporarily unable to carry out his functions under this Schedule, those functions shall be carried out by the Deputy President for that part of the United Kingdom.”
%
%(4) For paragraph 3 of that Schedule (constitution of Tribunal for particular hearings) there shall be substituted—
%
%“3The members of the Tribunal hearing a particular appeal shall in every case include a legally qualified member; and only a legally qualified member may preside as chairman for the hearing of any appeal.
%
%3A(1) The President of Pensions Appeal Tribunals for any part of the United Kingdom may give directions as to—
%
%($a$) the number of members of the Tribunal who should hear an appeal in that part of the United Kingdom;
%
%($b$) the extent to which the members hearing such an appeal must include—
%
%(i) medically qualified persons; and
%
%(ii) persons who are neither legally qualified nor medically qualified;
%
%($c$) the extent to which in the case of such an appeal the members hearing it must include persons satisfying other requirements specified by the President;
%
%($d$) the manner of determining the members who are to serve as the chairman and members of the Tribunal for the hearing of such an appeal.
%
%3BThe President of Pensions Appeal Tribunals for any part of the United Kingdom may give directions as to the practice and procedure to be followed by such Tribunals in that part of the United Kingdom.
%
%3C(1) The power to give directions under paragraphs 3A and 3B shall be exercisable in relation to a particular appeal, to a category of appeal or to appeals generally.
%
%(2) If at any time there is, in the case of any part of the United Kingdom, neither a President of Pensions Appeal Tribunals nor a Deputy President, the power of the President to give directions under paragraphs 3A and 3B above shall be exercisable—
%
%($a$) in the case of England and Wales, by the Lord Chancellor;
%
%($b$) in the case of Scotland, by the Lord President of the Court of Session; and
%
%($c$) in the case of Northern Ireland, by the Lord Chief Justice of Northern Ireland.
%
%(3) The power to give directions under paragraphs 3A and 3B above includes power to vary or revoke directions previously given.”
%
%(5) In Schedule 11 to the [1990 c. 41. ] Courts and Legal Services Act 1990 (judges barred from legal practice), at the end there shall be inserted “Member of a Pensions Appeal Tribunal”.
%61Composition of central advisory committee
%
%(1) In section 9 of the [1970 c. 44. ] Chronically Sick and Disabled Persons Act 1970 (central advisory committee on war pensions to include chairmen of not less than twelve of the war pensions committees), in subsection (1), for “chairmen of not less than twelve” there shall be substituted “chairman of at least one”.
%
%(2) In section 3 of the [1921 c. 49. ] War Pensions Act 1921 (constitution of central advisory committee), for “representatives of any committees” there shall be substituted “at least one person from one of the committees”.

\amendment{
Ss. 42--61 are not yet in force.
}

\part[Part III --- Social Security]{Part III\\*Social Security}

\renewcommand\parthead{--- Part III}

%Loss of benefit
%62Loss of benefit for breach of community order
%
%(1) If—
%
%($a$) a court makes a determination that a person (“the offender”) has failed without reasonable excuse to comply with the requirements of a relevant community order made in respect of him,
%
%($b$) the Secretary of State is notified in accordance with regulations under section 64 of the determination, and
%
%($c$) the offender is a person with respect to whom the conditions for any entitlement to a relevant benefit are or become satisfied,
%
%then, even though those conditions are satisfied, the following restrictions shall apply in relation to the payment of that benefit in the offender’s case.
%
%(2) Subject to subsections (3)  to (5) , the relevant benefit shall not be payable in the offender’s case for the prescribed period.
%
%(3) Where the relevant benefit is income support, the benefit shall be payable in the offender’s case for the prescribed period as if the applicable amount used for the determination under section 124(4)  of the [1992 c. 4. ] Social Security Contributions and Benefits Act 1992 of the amount of the offender’s entitlement for that period were reduced in such manner as may be prescribed.
%
%(4) The Secretary of State may by regulations provide that, where the relevant benefit is jobseeker’s allowance, any income-based jobseeker’s allowance shall be payable, during the whole or a part of the prescribed period, as if one or more of the following applied—
%
%($a$) the rate of the allowance were such reduced rate as may be prescribed;
%
%($b$) the allowance were payable only if there is compliance by the offender with such obligations with respect to the provision of information as may be imposed by the regulations;
%
%($c$) the allowance were payable only if the circumstances are otherwise such as may be prescribed.
%
%(5) Where the relevant benefit is a payment under section 2 of the [1973 c. 50. ] Employment and Training Act 1973 (under which training allowances are payable), that benefit shall not be payable for the prescribed period except to such extent (if any) as may be prescribed.
%
%(6) Where the determination by a court that was made in the offender’s case is quashed or otherwise set aside by the decision of that or any other court, all such payments and other adjustments shall be made in his case as would be necessary if the restrictions imposed by or under this section in respect of that determination had not been imposed.
%
%(7) The length of any period prescribed for the purposes of any of subsections (2)  to (5)  shall not exceed twenty-six weeks.
%
%(8) In this section—
%
%    “income-based jobseeker’s allowance” and “joint-claim jobseeker’s allowance” have the same meanings as in the [1995 c. 18. ] Jobseekers Act 1995;
%
%    “relevant benefit” means—
%    ($a$) 
%
%    income support;
%    ($b$) 
%
%    any jobseeker’s allowance other than joint-claim jobseeker’s allowance;
%    ($c$) 
%
%    any benefit under the [1992 c. 4. ] Social Security Contributions and Benefits Act 1992 (other than income support) which is prescribed for the purposes of this section; or
%    ($d$) 
%
%    any prescribed payment under section 2 of the [1973 c. 50. ] Employment and Training Act 1973 (under which training allowances are payable);
%
%    “relevant community order” means—
%    ($a$) 
%
%    a community service order;
%    ($b$) 
%
%    a probation order;
%    ($c$) 
%
%    a combination order;
%    ($d$) 
%
%    such other description of community order within the meaning of the Powers of Criminal Courts (Sentencing) Act 2000 as may be prescribed for the purposes of this section; or
%    ($e$) 
%
%    any order falling in England and Wales to be treated as an order specified in paragraphs ($a$)  to ($d$) . 
%
%(9) In relation to a relevant benefit falling within paragraph ($d$)  of the definition of that expression in subsection (8) , references in this section to the conditions for entitlement to that benefit being or becoming satisfied with respect to any person are references to there having been or, as the case may be, the taking of a decision to make a payment of such benefit to that person.
%
%(10) In relation to any time before the coming into force of the Powers of Criminal Courts (Sentencing) Act 2000, the reference to that Act in subsection (8)  shall be taken to be a reference to Part I of the [1991 c. 53. ] Criminal Justice Act 1991. 
%
%(11) In the application to Scotland of this section—
%
%($a$) in subsection (1)  after the word “excuse” insert “(or, in the case of a probation order, failed)”;
%
%($b$) for paragraph ($b$)  of that subsection substitute—
%
%“($b$) the Secretary of State is notified in accordance with an Act of Adjournal made under section 64 of the determination”; and
%
%($c$) in subsection (8) —
%
%(i) in the definition of relevant benefit, paragraph ($d$)  does not apply in the case of any payment made by or on behalf of the Scottish Ministers; and
%
%(ii) in the definition of relevant community order, for paragraphs ($c$)  to ($e$)  substitute—
%
%“($c$) such other description of order made under the [1995 c. 46. ] Criminal Procedure (Scotland) Act 1995 as may be prescribed for the purposes of this section; or
%
%($d$) any order falling in Scotland to be treated as an order specified in paragraphs ($a$)  to ($c$) .”
%63Loss of joint-claim jobseeker’s allowance
%
%(1) Subsections (2)  and (3)  shall have effect, subject to the other provisions of this section, where—
%
%($a$) the conditions for the entitlement of any joint-claim couple to a joint-claim jobseeker’s allowance are or become satisfied at any time; and
%
%($b$) the restriction in subsection (2)  of section 62 would apply in the case of at least one of the members of the couple if the entitlement were an entitlement of that member to a relevant benefit.
%
%(2) The allowance shall not be payable in the couple’s case for so much of the prescribed period as is a period for which—
%
%($a$) in the case of each of the members of the couple, the restriction in subsection (2)  of section 62 would apply if the entitlement were an entitlement of that member to a relevant benefit; or
%
%($b$) that restriction would so apply in the case of one of the members of the couple and the other member of the couple is subject to sanctions for the purposes of section 20A of the [1995 c. 18. ] Jobseekers Act 1995 (denial or reduction of joint-claim jobseeker’s allowance).
%
%(3) For any part of the period for which subsection (2)  does not apply, the allowance—
%
%($a$) shall be payable in the couple’s case as if the amount of the allowance were reduced to an amount calculated using the method prescribed for the purposes of this subsection; but
%
%($b$) shall be payable only to the member of the couple who is not the person in relation to whom the court has made a determination.
%
%(4) The Secretary of State may by regulations provide in relation to cases to which subsection (2)  would otherwise apply that joint-claim jobseeker’s allowance shall be payable in a couple’s case, during the whole or a part of so much of the prescribed period as falls within paragraph ($a$)  or ($b$)  of that subsection, as if one or more of the following applied—
%
%($a$) the rate of the allowance were such reduced rate as may be prescribed;
%
%($b$) the allowance were payable only if there is compliance by each of the members of the couple with such obligations with respect to the provision of information as may be imposed by the regulations;
%
%($c$) the allowance were payable only if the circumstances are otherwise such as may be prescribed.
%
%(5) Subsection (6)  of section 20A of the [1995 c. 18. ] Jobseekers Act 1995 (calculation of reduced amount) shall apply for the purposes of subsection (3)  above as it applies for the purposes of subsection (5)  of that section.
%
%(6) Subsection (6)  of section 62 shall apply for the purposes of this section in relation to any determination relating to one or both members of the joint-claim couple as it applies for the purposes of that section in relation to the determination relating to the offender.
%
%(7) The length of any period prescribed for the purposes of subsection (2)  or (3)  shall not exceed twenty-six weeks.
%
%(8) In this section—
%
%    “joint-claim couple” and “joint-claim jobseeker’s allowance” have the same meanings as in the [1995 c. 18. ] Jobseekers Act 1995; and
%
%    “relevant benefit” has the same meaning as in section 62.  
%
%64Information provision
%
%(1) A court in Great Britain shall, before making a relevant community order in relation to any person, explain to that person in ordinary language the consequences by virtue of sections 62 and 63 of a failure to comply with the order.
%
%(2) The Secretary of State may by regulations require the Chief Probation Officer for any area in England and Wales, or such other person as may be prescribed, to notify the Secretary of State at the prescribed time and in the prescribed manner—
%
%($a$) of the laying by a person employed or appointed by a probation committee of any information that a person has failed to comply with the requirements of a relevant community order;
%
%($b$) of any such determination as is mentioned in section 62(1) ;
%
%($c$) of such information about the offender, and in the possession of the person giving the notification, as may be prescribed; and
%
%($d$) of any circumstances by virtue of which any payment or adjustment might fall to be made by virtue of section 62(6)  or 63(6) .
%
%(3) The High Court of Justiciary may, by Act of Adjournal, make provision requiring the clerk of the court in which any proceedings are commenced that could result in a determination of a failure to comply with a relevant community order to notify the Secretary of State at such time and in such manner as may be specified in the Act of Adjournal of—
%
%($a$) the commencement of the proceedings;
%
%($b$) any such determination made in the proceedings;
%
%($c$) such information about the offender as may be so specified; and
%
%($d$) any circumstances by virtue of which any payment or adjustment might fall to be made by virtue of section 62(6)  or 63(6) .
%
%(4) Where it appears to the Secretary of State that—
%
%($a$) the laying of any information that has been laid in England and Wales, or
%
%($b$) the commencement of any proceedings that have been commenced in Scotland,
%
%could result in a determination the making of which would result in the imposition by or under one or both of sections 62 and 63 of any restrictions, it shall be the duty of the Secretary of State to notify the person in whose case those restrictions would be imposed, or (as the case may be) the members of any joint-claim couple in whose case they would be imposed, of the consequences under those sections of such a determination in the case of that person, or couple.
%
%(5) A notification required to be given by the Secretary of State under subsection (4)  must be given as soon as reasonably practicable after it first appears to the Secretary of State as mentioned in that subsection.
%
%(6) The Secretary of State may by regulations make such provision as he thinks fit for the purposes of sections 62 to 65 of this Act about—
%
%($a$) the use by a person within subsection (7)  of information relating to community orders or social security;
%
%($b$) the supply of such information by a person within that subsection to any other person (whether or not within that subsection); and
%
%($c$) the purposes for which a person to whom such information is supplied under the regulations may use it.
%
%(7) The persons within this subsection are—
%
%($a$) the Secretary of State;
%
%($b$) a person providing services to the Secretary of State;
%
%($c$) a person employed or appointed by a probation committee;
%
%($d$) a person employed by a council constituted under section 2 of the [1994 c. 39. ] Local Government etc. (Scotland) Act 1994. 
%
%(8) Regulations under subsection (6)  may, in particular, authorise information supplied to a person under the regulations—
%
%($a$) to be used for the purpose of amending or supplementing other information held by that person; and
%
%($b$) where so used, to be supplied to any other person to whom, and used for any purpose for which, the information amended or supplemented could be supplied or used.
%
%(9) The explanation given to the offender by the court in pursuance of subsection (1)  shall be treated as part of the explanation required to be given to the offender for the purposes of section 228(5)  or 238(4)  of the [1995 c. 46. ] Criminal Procedure (Scotland) Act 1995. 
%
%(10) In this section “relevant community order” has the same meaning as in section 62. 
%
%(11) For the purposes of this section proceedings that could result in such a determination as is mentioned in subsection (3)  are commenced in Scotland when, and only when, a warrant to arrest the offender or to cite the offender to appear before a court is issued under section 232(1)  or 239(4)  of the [1995 c. 46. ] Criminal Procedure (Scotland) Act 1995. 
%65Loss of benefit regulations
%
%(1) In the loss of benefit provisions “prescribed” means prescribed by or determined in accordance with regulations made by the Secretary of State.
%
%(2) Regulations prescribing a period for the purposes of any of the loss of benefit provisions may contain provision for determining the time from which the period is to run.
%
%(3) Regulations under any of the loss of benefit provisions shall be made by statutory instrument which (except in the case of regulations to which subsection (4)  applies) shall be subject to annulment in pursuance of a resolution of either House of Parliament.
%
%(4) A statutory instrument containing (whether alone or with other provisions)—
%
%($a$) a provision prescribing the manner in which the applicable amount is to be reduced for the purposes of section 62(3),
%
%($b$) a provision prescribing the manner in which an amount of joint-claim jobseeker’s allowance is to be reduced for the purposes of section 63(3)($a$),
%
%($c$) a provision the making of which is authorised by section 62(4)  or 63(4) ,
%
%($d$) a provision prescribing benefits under the [1992 c. 4. ] Social Security Contributions and Benefits Act 1992 as benefits that are to be relevant benefits for the purposes of section 62, or
%
%($e$) a provision that any description of order is to be a relevant community order for the purposes of that section,
%
%shall not be made unless a draft of the instrument has been laid before, and approved by a resolution of, each House of Parliament.
%
%(5) Subsections (4)  to (6)  of section 189 of the [1992 c. 5. ] Social Security Administration Act 1992 (supplemental and incidental powers etc.) shall apply in relation to any power to make regulations that is conferred by the loss of benefit provisions as they apply in relation to the powers to make regulations that are conferred by that Act.
%
%(6) The provision that may be made in exercise of the powers to make regulations that are conferred by the loss of benefit provisions shall include different provision for different areas.
%
%(7) Where regulations made under section 62(8)  prescribe a description of order made under the [1995 c. 46. ] Criminal Procedure (Scotland) Act 1995 as a relevant community order for the purposes of that section, the regulations may make such modifications of that section as appear to the Secretary of State to be necessary in consequence of so prescribing.
%
%(8) In this section “the loss of benefit provisions” means sections 62 to 64 of this Act.
%66Appeals relating to loss of benefit
%
%In paragraph 3 of Schedule 3 to the [1998 c. 14. ] Social Security Act 1998 (decisions against which an appeal lies), after sub-paragraph ($d$)  there shall be inserted “; or
%
%($e$) section 62 or 63 of the Child Support, Pensions and Social Security Act 2000.”
%Investigation powers
%67Investigation powers
%
%Schedule 6 to this Act (which amends the enforcement provisions contained in Part VI of the [1992 c. 5. ] Social Security Administration Act 1992) shall have effect.
%Housing benefit and council tax benefit etc.
%68Housing benefit and council tax benefit: revisions and appeals
%
%Schedule 7 (which makes provision for the revision of decisions made in connection with claims for housing benefit or council tax benefit and for appeals against such decisions) shall have effect.
%69Discretionary financial assistance with housing
%
%(1) The Secretary of State may by regulations make provision conferring a power on relevant authorities to make payments by way of financial assistance (“discretionary housing payments”) to persons who—
%
%($a$) are entitled to housing benefit or council tax benefit, or to both; and
%
%($b$) appear to such an authority to require some further financial assistance (in addition to the benefit or benefits to which they are entitled) in order to meet housing costs.
%
%(2) Regulations under this section may include any of the following—
%
%($a$) provision prescribing the circumstances in which discretionary housing payments may be made under the regulations;
%
%($b$) provision conferring (subject to any provision made by virtue of paragraph ($c$)  or ($d$)  of this subsection or an order under section 70) a discretion on a relevant authority—
%
%(i) as to whether or not to make discretionary housing payments in a particular case; and
%
%(ii) as to the amount of the payments and the period for or in respect of which they are made;
%
%($c$) provision imposing a limit on the amount of the discretionary housing payment that may be made in any particular case;
%
%($d$) provision restricting the period for or in respect of which discretionary housing payments may be made;
%
%($e$) provision about the form and manner in which claims for discretionary housing payments are to be made and about the procedure to be followed by relevant authorities in dealing with and disposing of such claims;
%
%($f$) provision imposing conditions on persons claiming or receiving discretionary housing payments requiring them to provide a relevant authority with such information as may be prescribed;
%
%($g$) provision entitling a relevant authority that are making or have made a discretionary housing payment, in such circumstances as may be prescribed, to cancel the making of further such payments or to recover a payment already made;
%
%(h)provision requiring or authorising a relevant authority to review decisions made by the authority with respect to the making, cancellation or recovery of discretionary housing payments.
%
%(3) Regulations under this section shall be made by statutory instrument, which shall be subject to annulment in pursuance of a resolution of either House of Parliament.
%
%(4) Subsections (4)  to (6)  of section 189 of the [1992 c. 5. ] Social Security Administration Act 1992 (supplemental and incidental powers etc.) shall apply in relation to any power to make regulations under this section as they apply in relation to the powers to make regulations that are conferred by that Act.
%
%(5) Any power to make regulations under this section shall include power to make different provision for different areas or different relevant authorities.
%
%(6) In section 176(1)  of that Act (consultation with representative organisation on subordinate legislation relating to housing benefit or council tax benefit), after paragraph ($a$)  there shall be inserted—
%
%“($aa$) regulations under section 69 of the Child Support, Pensions and Social Security Act 2000;”.
%
%(7) In this section—
%
%    “prescribed” means prescribed by or determined in accordance with regulations made by the Secretary of State; and
%
%    “relevant authority” means an authority administering housing benefit or council tax benefit. 
%
%70Grants towards cost of discretionary housing payments
%
%(1) The Secretary of State may, out of money provided by Parliament, make to a relevant authority such payments as he thinks fit in respect of—
%
%($a$) the cost to that authority of the making of discretionary housing payments; and
%
%($b$) the expenses involved in the administration by that authority of any scheme for the making of discretionary housing payments.
%
%(2) The following provisions, namely—
%
%($a$) subsections (1), (3), (4) , (5)($b$), (7)($b$)  and (8)  of section 140B of the [1992 c. 5. ] Social Security Administration Act 1992 (calculation of amount of subsidy payable to authorities administering housing benefit or council tax benefit), and
%
%($b$) section 140C of that Act (payment of subsidy),
%
%shall apply in relation to payments under this section as they apply in relation to subsidy under section 140A of that Act.
%
%(3) The Secretary of State may by order make provision—
%
%($a$) imposing a limit on the total amount of expenditure in any year that may be incurred by a relevant authority in making discretionary housing payments;
%
%($b$) imposing subsidiary limits on the expenditure that may be incurred in any year by a relevant authority in making discretionary housing payments in the circumstances specified in the order.
%
%(4) An order imposing a limit by virtue of subsection (3)($a$)  or ($b$)  may fix that limit either by specifying the amount of the limit or by providing for the means by which it is to be determined.
%
%(5) An order under this section shall be made by statutory instrument, which shall be subject to annulment in pursuance of a resolution of either House of Parliament.
%
%(6) Subsections (4)  to (6)  of section 189 of the [1992 c. 5. ] Social Security Administration Act 1992 (supplemental and incidental powers etc.) shall apply in relation to any power to make an order under this section as they apply in relation to the powers to make an order that are conferred by that Act.
%
%(7) Any power to make an order under this section shall include power to make different provision for different areas or different relevant authorities.
%
%(8) In this section—
%
%    “discretionary housing payment” means any payment made by virtue of regulations under section 69;
%
%    “relevant authority” means an authority administering housing benefit or council tax benefit;
%
%    “subsidy” has the same meaning as in sections 140A to 140G of the [1992 c. 5. ] Social Security Administration Act 1992;
%
%    “year” means a financial year within the meaning of the [1992 c. 14. ] Local Government Finance Act 1992.  
%
%71Recovery of housing benefit
%
%For subsection (3)  of section 75 of the [1992 c. 5. ] Social Security Administration Act 1992 (overpayments of housing benefit) there shall be substituted—
%
%“(3) An amount recoverable under this section shall be recoverable—
%
%($a$) except in such circumstances as may be prescribed, from the person to whom it was paid; and
%
%($b$) where regulations so provide, from such other person (as well as, or instead of, the person to whom it was paid) as may be prescribed.”

\amendment{
Ss. 62--71 are not yet in force.
}

\section{\itshape Child benefit}

\subsection{72. Child benefit disregards}

In section 143(3)($c$)  of the Social Security Contributions and Benefits Act 1992 (disregard of days of absence in the case of children in residential accommodation in pursuance of arrangements made under the specified enactments), for sub-paragraph (iii)  and the word “or” immediately preceding it there shall be substituted—
\begin{quotation}
“(iii) the Social Work (Scotland) Act 1968;

(iv) the National Health Service (Scotland) Act 1978;

(v) the Education (Scotland) Act 1980;

(vi) the Mental Health (Scotland) Act 1984; or

(vii) the Children (Scotland) Act 1995.”
\end{quotation}

%Social Security Advisory Committee
%73Social Security Advisory Committee
%
%(1) Section 170 of the [1992 c. 5. ] Social Security Administration Act 1992 (functions of the Social Security Advisory Committee in relation to the relevant enactments and the relevant Northern Ireland enactments) shall be amended as follows.
%
%(2) In the definition in subsection (5)  of “relevant enactments”, after paragraph (ae) there shall be inserted—
%
%“(af)section 42, sections 62 to 65 and sections 68 to 70 of the Child Support Pensions and Social Security Act 2000 and Schedule 7 to that Act;”.
%
%(3) In the definition in that subsection of “relevant Northern Ireland enactments”, after paragraph (ae) there shall be inserted—
%
%“(af)any provisions in Northern Ireland which correspond to section 42, any of sections 62 to 65, 68 to 70 of the Child Support, Pensions and Social Security Act 2000 or Schedule 7 to that Act; and”.

\amendment{
S. 73 is not yet in force.
}

\part[Part IV --- National Insurance Contributions]{Part IV\\*National Insurance Contributions}

\renewcommand\parthead{--- Part IV}

\section{\itshape Great Britain}

\subsection{74. Contributions in respect of benefits in kind: Great Britain}

(1) In section 1(2)($b$)  of the Social Security Contributions and Benefits Act 1992 (Class 1A contributions), the words “in respect of cars made available for private use and car fuel” shall be omitted.

(2) For section 10 of that Act (Class 1A contributions) there shall be substituted—
\begin{quotation}
\subsection*{“10. Class 1A contributions: benefits in kind etc}

(1) Where—
\begin{enumerate}\item[]
($a$) for any tax year an earner is chargeable to income tax under Schedule E on an amount which for the purposes of the Income Tax Acts is or falls to be treated as an emolument received by him from any employment (“the relevant employment”),

($b$) the relevant employment is both employed earner’s employment and employment to which Chapter II of Part V of the 1988 Act (employment as a director or with annual emoluments of more than £8,500) applies, and

($c$) the whole or a part of the emolument falls, for the purposes of Class 1 contributions, to be left out of account in the computation of the earnings paid to or for the benefit of the earner,
\end{enumerate}
a Class 1A contribution shall be payable for that tax year, in accordance with this section, in respect of that earner and so much of the emolument as falls to be so left out of account.

(2) Subject to section 10ZA below, a Class 1A contribution for any tax year shall be payable by—
\begin{enumerate}\item[]
($a$) the person who is liable to pay the secondary Class 1 contribution relating to the last (or only) relevant payment of earnings in that tax year in relation to which there is a liability to pay such a Class 1 contribution; or

($b$) if paragraph ($a$)  above does not apply, the person who, if the emolument in respect of which the Class 1A contribution is payable were earnings in respect of which Class 1 contributions would be payable, would be liable to pay the secondary Class 1 contribution.
\end{enumerate}

(3) In subsection (2)  above “relevant payment of earnings” means a payment which for the purposes of Class 1 contributions is a payment of earnings made to or for the benefit of the earner in respect of the relevant employment.

(4) The amount of the Class 1A contribution in respect of any emolument shall be the Class 1A percentage of so much of it as falls to be left out of account as mentioned in subsection (1)($c$)  above.

(5) In subsection (4)  above “the Class 1A percentage” means a percentage rate equal to the percentage rate specified as the secondary percentage in section 9(2)  above for the tax year in question.

(6) No Class 1A contribution shall be payable for any tax year in respect of so much of any emolument as is taken for the purposes of the making of Class 1B contributions for that year to be included in a PAYE settlement agreement.

(7) For the purposes of this section—
\begin{enumerate}\item[]
($a$) the amounts which for the purposes of the Income Tax Acts are or fall to be treated as emoluments received by an earner from any employment shall be determined (subject to paragraph ($b$)  below) disregarding sections 198, 201, 201AA and 332(3)  of the 1988 Act (deductions for expenses etc.);\ but

($b$) where an amount which is deductible in respect of any matter under any of those sections is at least equal to the whole of any corresponding amount which (but for this paragraph) would fall by reference to that matter to be included in those emoluments, the whole of the corresponding amount shall be treated as not so included.
\end{enumerate}

(8) The Treasury may by regulations—
\begin{enumerate}\item[]
($a$) modify the effect of subsection (7)  above by adding any enactment contained in the Income Tax Acts to the list of sections of the 1988 Act contained in paragraph ($a$)  of that subsection; or

($b$) make such amendments of subsection (7)  above as appear to them to be necessary or expedient in consequence of any alteration of the provisions of the Income Tax Acts relating to the charge to tax under Schedule E.
\end{enumerate}

(9) The Treasury may by regulations provide—
\begin{enumerate}\item[]
($a$) for Class 1A contributions not to be payable, in prescribed circumstances, by prescribed persons or in respect of prescribed persons or emoluments;

($b$) for reducing Class 1A contributions in prescribed circumstances.
\end{enumerate}

(10) In this section “the 1988 Act” means the Income and Corporation Taxes Act 1988.”
\end{quotation}

(3) For subsection (6)  of section 4 of that Act (power to treat emoluments in respect of share acquisitions etc.\ as earnings) there shall be substituted—
\begin{quotation}
“(6) Regulations may make provision for the purposes of this Part—
\begin{enumerate}\item[]
($a$) for treating any amount on which an employed earner is chargeable to income tax under Schedule E as remuneration derived from the earner’s employment; and

($b$) for treating any amount which in accordance with regulations under paragraph ($a$)  above constitutes remuneration as an amount of remuneration paid, at such time as may be determined in accordance with the regulations, to or for the benefit of the earner in respect of his employment.”
\end{enumerate}
\end{quotation}

(4) In paragraph 5($b$)  of Schedule 1 to that Act (power to modify section 10 for cases where a car is made available by reason of more than one employment), for “a car is made available” there shall be substituted “something is provided or made available”.

(5) In paragraph 8(1)($ia$) of that Schedule (power to provide by regulations for repayment in prescribed cases of the whole or a part of a Class 1B contribution), after “part” there shall be inserted “of a Class 1A or”.

(6) In section 120(4)  of the Social Security Administration Act 1992 (proof of previous offences relating to Class 1A contributions), for “car” there shall be substituted “amount”.

(7) In section 162(5)($c$)  of that Act (appropriate national health service allocation of Class 1A contributions), for “cash equivalents of the benefits of the cars and car fuel” there shall be substituted “emoluments”.

(8) This section shall have effect in relation to the tax year beginning with 6th April 2000 and subsequent tax years.

(9) Regulations made by statutory instrument under any power conferred by virtue of this section may be made so as to have retrospective effect in relation to any time in the tax year in which they are made (including, in the case of regulations made in the tax year in which this Act is passed, any time in that tax year before the passing of this Act).

\subsection{75. Third party providers of benefits in kind: Great Britain}

(1) After section 10 of the Social Security Contributions and Benefits Act 1992 there shall be inserted—
\begin{quotation}
\subsection*{“10ZA. Liability of third party provider of benefits in kind}

(1) This section applies, where—
\begin{enumerate}\item[]
($a$) a Class 1A contribution is payable for any tax year in respect of the whole or any part of an emolument received by an earner;

($b$) the emolument, in so far as it is one in respect of which such a contribution is payable, consists in a benefit provided for the earner or a member of his family or household;

($c$) the person providing the benefit is a person other than the person (“the relevant employer”) by whom, but for this section, the Class 1A contribution would be payable in accordance with section 10(2)  above; and

($d$) the provision of the benefit by that other person has not been arranged or facilitated by the relevant employer.
\end{enumerate}

(2) For the purposes of this Act if—
\begin{enumerate}\item[]
($a$) the person providing the benefit pays an amount for the purpose of discharging any liability of the earner to income tax for any tax year, and

($b$) the income tax in question is tax chargeable in respect of the provision of the benefit or of the making of the payment itself,
\end{enumerate}
the amount of the payment shall be treated as if it were an emolument consisting in the provision of a benefit to the earner in that tax year and falling, for the purposes of Class 1 contributions, to be left out of account in the computation of the earnings paid to or for the benefit of the earner.

(3) Subject to subsection (4)  below, the liability to pay any Class 1A contribution in respect of—
\begin{enumerate}\item[]
($a$) the benefit provided to the earner, and

($b$) any further benefit treated as so provided in accordance with subsection (2)  above,
\end{enumerate}
shall fall on the person providing the benefit, instead of on the relevant employer.

(4) Subsection (3)  above applies in the case of a Class 1A contribution for the tax year beginning with 6th April 2000 only if the person providing the benefit in question gives notice in writing to the Inland Revenue on or before 6th July 2001 that he is a person who provides benefits in respect of which a liability to Class 1A contributions is capable of falling by virtue of this section on a person other than the relevant employer.

(5) The Treasury may by regulations make provision specifying the circumstances in which a person is or is not to be treated for the purposes of this Act as having arranged or facilitated the provision of any benefit.

(6) In this section references to a member of a person’s family or household shall be construed in accordance with section 168(4)  of the Income and Corporation Taxes Act 1988. 

\subsection*{10ZB. Non-cash vouchers provided by third parties}

(1) In section 10ZA above references to the provision of a benefit include references to the provision of a non-cash voucher.

(2) Where—
\begin{enumerate}\item[]
($a$) a non-cash voucher is received by any person from employment to which Chapter II of Part V of the Income and Corporation Taxes Act 1988 does not apply, and

($b$) the case would be one in which the conditions in section 10ZA(1)($a$)  to ($d$)  above would be satisfied in relation to the provision of that voucher if that Chapter did apply to that employment,
\end{enumerate}
sections 10 and 10ZA above shall have effect in relation to the provision of that voucher, and to any such payment in respect of the provision of that voucher as is mentioned in section 10ZA(2)  above, as if that employment were employment to which that Chapter applied.

(3) In this section “non-cash voucher” has the same meaning as in section 141 of the Income and Corporation Taxes Act 1988.”
\end{quotation}

(2) After subsection (3)  of section 110ZA of the Social Security Administration Act 1992 (premises liable to inspection) there shall be inserted—
\begin{quotation}
“(3A) The references in subsection (3)  above to a trade or business include references to the administration of any scheme for the provision of benefits to persons by reason of their employment.”
\end{quotation}

(3) Subsection (1)  shall have effect in relation to the tax year beginning with 6th April 2000 and subsequent tax years.

(4) Regulations made by virtue of this section under section 10ZA(5)  of the Social Security Contributions and Benefits Act 1992 may be made so as to have retrospective effect in relation to any time in the tax year in which they are made (including, in the case of regulations made in the tax year in which this Act is passed, any time in that tax year before the passing of this Act).

\subsection{76. Collection etc.\ of NICs: Great Britain}

(1) Schedule 1 to the Social Security Contributions and Benefits Act 1992 (supplementary provisions relating to contributions) shall be amended in accordance with subsections (2)  to (5).

(2) In paragraph 7(2)($b$)  (application of sections 100 to 100D and 102 to 104 of the Taxes Management Act 1970 in relation to certain penalties), for “104” there shall be substituted “105”.

(3) For sub-paragraph (2)($e$)  of paragraph 7B (power to provide for interest to be charged on late payment in the case of payment outside the PAYE system) there shall be substituted—
\begin{quotation}
“($e$) require interest to be paid on contributions that are not paid by the due date, and provide for determining the date from which such interest is to be calculated;”.
\end{quotation}

(4) After sub-paragraph (5)  of that paragraph there shall be inserted—
\begin{quotation}
“(5A) Regulations under this paragraph may, in relation to any penalty imposed by such regulations, make provision applying (with or without modifications) any enactment applying for the purposes of income tax that is contained in Part X of the Taxes Management Act 1970 (penalties).”
\end{quotation}

(5) After that paragraph there shall be inserted—
\begin{quotation}
“7BA. The Inland Revenue may by regulations provide for amounts in respect of contributions or interest that fall to be paid or repaid in accordance with any regulations under this Schedule to be set off, or to be capable of being set off, in prescribed circumstances and to the prescribed extent, against any such liabilities under regulations under this Schedule of the person entitled to the payment or repayment as may be prescribed.”
\end{quotation}

(6) In section 8(1)  of the Social Security Contributions (Transfer of Functions, etc.)\ Act 1999 (decisions to be made by an Inland Revenue officer and appealable under section 11)—
\begin{enumerate}\item[]
($a$) paragraph ($j$)  (interest under regulations made by virtue of paragraph 7B(2)($e$)  of Schedule 1 to the Social Security Contributions and Benefits Act 1992) shall cease to have effect; and

($b$) in paragraph ($l$), for “paragraphs ($j$)  and ($k$)” there shall be substituted “paragraph ($k$)”, and the words “amount of interest or” shall be omitted.
\end{enumerate}

(7) Subsection (6)  has effect in relation to interest accruing on sums becoming due in respect of the tax year beginning with 6th April 2000 or any subsequent tax year.

\subsection{77. Liability of earner for secondary contributions: Great Britain}

(1) In paragraph 3 of Schedule 1 to the Social Security Contributions and Benefits Act 1992 (prohibition on deduction or recovery of Class 1 contributions), sub-paragraph (2)  shall be omitted.

(2) After that paragraph there shall be inserted—
\begin{quotation}
\subsection*{“Prohibition on recovery of employer’s contributions}

3A.---(1) Subject to sub-paragraph (2)  below, a person who is or has been liable to pay any secondary Class 1 or any Class 1A or Class 1B contributions shall not—
\begin{enumerate}\item[]
($a$) make, from earnings paid by him, any deduction in respect of any such contributions for which he or any other person is or has been liable;

($b$) otherwise recover any such contributions (directly or indirectly) from any person who is or has been a relevant earner; or

($c$) enter into any agreement with any person for the making of any such deduction or otherwise for the purpose of so recovering any such contributions.
\end{enumerate}

(2) Sub-paragraph (1)  above does not apply to the extent that an agreement between—
\begin{enumerate}\item[]
($a$) a secondary contributor, and

($b$) any person (“the earner”) in relation to whom the secondary contributor is, was or will be such a contributor in respect of the contributions to which the agreement relates,
\end{enumerate}
allows the secondary contributor to recover (whether by deduction or otherwise) the whole or any part of any secondary Class 1 contribution payable in respect of a gain that is treated as remuneration derived from that earner’s employment by virtue of section 4(4)($a$)  above.

(3) Sub-paragraph (2)  above does not authorise any recovery (whether by deduction or otherwise)—
\begin{enumerate}\item[]
($a$) in pursuance of any agreement entered into before 19th May 2000; or

($b$) in respect of any liability to a contribution arising before the day of the passing of the Child Support, Pensions and Social Security Act 2000. 
\end{enumerate}

(4) In this paragraph—
\begin{enumerate}\item[]
    “agreement” includes any arrangement or understanding (whether or not legally enforceable); and

    “relevant earner”, in relation to a person who is or has been liable to pay any contributions, means an earner in respect of whom he is or has been so liable. 
\end{enumerate}

\subsection*{Transfer of liability to be borne by earner}

3B.---(1) This paragraph applies where—
\begin{enumerate}\item[]
($a$) an election is jointly made by—
\begin{enumerate}\item[]
(i) a secondary contributor, and

(ii) a person (“the earner”) in relation to whom the secondary contributor is or will be such a contributor in respect of contributions on share option gains by the earner,
\end{enumerate}
for the whole or a part of any liability of the secondary contributor to contributions on any such gains to be transferred to the earner; and

($b$) the election is one in respect of which the Inland Revenue have, before it was made, given by notice to the secondary contributor their approval to both—
\begin{enumerate}\item[]
(i) the form of the election; and

(ii) the arrangements made in relation to the proposed election for securing that the liability transferred by the election will be met.
\end{enumerate}
\end{enumerate}

(2) Any liability which—
\begin{enumerate}\item[]
($a$) arises while the election is in force, and

($b$) is a liability to pay the contributions on share option gains by the earner, or the part of them, to which the election relates,
\end{enumerate}
shall be treated for the purposes of this Act, the Administration Act and Part II of the Social Security Contributions (Transfer of Functions, etc.)\ Act 1999 as a liability falling on the earner, instead of on the secondary contributor.

(3) Subject to sub-paragraph (7)($b$)  below, an election made for the purposes of sub-paragraph (1)  above shall continue in force from the time when it is made until whichever of the following first occurs, namely—
\begin{enumerate}\item[]
($a$) it ceases to have effect in accordance with its terms;

($b$) it is revoked jointly by both parties to the election;

($c$) notice is given to the earner by the secondary contributor terminating the effect of the election.
\end{enumerate}

(4) An approval given to the secondary contributor for the purposes of sub-paragraph (1)($b$)  above may be given either—
\begin{enumerate}\item[]
($a$) for an election to be made by the secondary contributor and a particular person; or

($b$) for all elections to be made, or to be made in particular circumstances, by the secondary contributor and particular persons or by the secondary contributor and persons of a particular description.
\end{enumerate}

(5) The grounds on which the Inland Revenue shall be entitled to refuse an approval for the purposes of sub-paragraph (1)($b$)  above shall include each of the following—
\begin{enumerate}\item[]
($a$) that it appears to the Inland Revenue that adequate arrangements have not been made for securing that the liabilities transferred by the proposed election or elections will be met by the person or persons to whom they would be so transferred; and

($b$) that it appears to the Inland Revenue that they do not have sufficient information to determine whether or not grounds falling within paragraph ($a$)  above exist.
\end{enumerate}

(6) If, at any time after they have given an approval for the purposes of sub-paragraph (1)($b$)  above, it appears to the Inland Revenue—
\begin{enumerate}\item[]
($a$) that the arrangements that were made or are in force for securing that liabilities transferred by elections to which the approval relates are met are proving inadequate or unsatisfactory in any respect, or

($b$) that any election to which the approval relates has resulted, or is likely to result, in the avoidance or non-payment of the whole or any part of any secondary Class 1 contributions,
\end{enumerate}
the Inland Revenue may withdraw the approval by notice to the secondary contributor.

(7) The withdrawal by the Inland Revenue of any approval given for the purposes of sub-paragraph (1)($b$)  above—
\begin{enumerate}\item[]
($a$) may be either general or confined to a particular election or to particular elections; and

($b$) shall have the effect that the election to which the withdrawal relates has no effect on contributions on share option gains in respect of any right to acquire shares obtained after—
\begin{enumerate}\item[]
(i) the date on which notice of the withdrawal of the approval is given; or

(ii) such later date as the Inland Revenue may specify in that notice.
\end{enumerate}
\end{enumerate}

(8) Where the Inland Revenue have refused or withdrawn their approval for the purposes of sub-paragraph (1)($b$)  above, the person who applied for it or, as the case may be, to whom it was given may appeal to the Special Commissioners against the Inland Revenue’s decision.

(9) On an appeal under sub-paragraph (8)  above the Special Commissioners may—
\begin{enumerate}\item[]
($a$) dismiss the appeal;

($b$) remit the decision appealed against to the Inland Revenue with a direction to make such decision as the Special Commissioners think fit; or

($c$) in the case of a decision to withdraw an approval, quash that decision and direct that that decision is to be treated as never having been made.
\end{enumerate}

(10) Subject to sub-paragraph (12)  below, an election under sub-paragraph (1)  above shall not apply to any contributions in respect of gains realised before it was made.

(11) Regulations made by the Inland Revenue may make provision with respect to the making of elections for the purposes of this paragraph and the giving of approvals for the purposes of sub-paragraph (1)($b$)  above; and any such regulations may, in particular—
\begin{enumerate}\item[]
($a$) prescribe the matters that must be contained in such an election;

($b$) provide for the manner in which such an election is to be capable of being made and of being confined to particular liabilities or the part of particular liabilities; and

($c$) provide for the making of applications for such approvals and for the manner in which those applications are to be dealt with.
\end{enumerate}

(12) Where—
\begin{enumerate}\item[]
($a$) an election is made under this paragraph before the end of the period of three months beginning with the date of the passing of the Child Support, Pensions and Social Security Act 2000, and

($b$) that election is expressed to relate to liabilities for contributions arising on or after 19th May 2000 and before the making of the election,
\end{enumerate}
this paragraph shall have effect in relation to those liabilities as if sub-paragraph (2)  above provided for them to be deemed to have fallen on the earner (instead of on the secondary contributor); and the secondary contributor shall accordingly be entitled to reimbursement from the earner for any payment made by that contributor in or towards the discharge of any of those liabilities.

(13) In this paragraph references to contributions on share option gains by the earner are references to any secondary Class 1 contributions payable in respect of a gain that is treated as remuneration derived from the earner’s employment by virtue of section 4(4)($a$)  above.

(14) In this paragraph “the Special Commissioners” means the Commissioners for the special purposes of the Income Tax Acts.”
\end{quotation}

(3) In section 6(4)  of that Act (persons by whom Class 1 contributions are payable), for the words from “paragraph 3” onwards there shall be substituted “paragraphs 3 to 3B of Schedule 1 to this Act.”

(4) In paragraph 8(1)  of Schedule 1 to that Act (general regulations), after paragraph ($c$)  there shall be inserted—
\begin{quotation}
“($ca$) for requiring a secondary contributor to notify a person to whom any of his liabilities are transferred by an election under paragraph 3B above of—
\begin{enumerate}\item[]
(i) any transferred liability that arises;

(ii) the amount of any transferred liability that arises; and

(iii) the contents of any notice of withdrawal by the Inland Revenue of any approval that relates to that election;”.
\end{enumerate}
\end{quotation}

(5) In section 8(1)  of the Social Security Contributions (Transfer of Functions, etc.)\ Act 1999 (decisions to be taken by officers of the Inland Revenue), after paragraph ($i$)  there shall be inserted—
\begin{quotation}
“($ia$) to decide whether to give or withdraw an approval for the purposes of paragraph 3B(1)($b$)  of Schedule 1 to the Social Security Contributions and Benefits Act 1992;”.
\end{quotation}

(6) In section 10 of that Act of 1999 (regulations about varying or superseding decisions), at the beginning of subsection (1)  there shall be inserted “Subject to subsection (2A)  below,”, and after subsection (2)  there shall be inserted—
\begin{quotation}
“(2A) The decisions in relation to which provision may be made by regulations under this section shall not include decisions falling within section 8(1)($ia$) above.”
\end{quotation}

(7) In section 12(4)  of that Act of 1999 (appeals to be heard by General Commissioners), after “Subject to” there shall be inserted “paragraph 3B(8)  of Schedule 1 to the Social Security Contributions and Benefits Act 1992 (which provides for appeals under that paragraph to be heard by the Special Commissioners), to”.

\section{\itshape Northern Ireland}

\subsection{78. Contributions in respect of benefits in kind: Northern Ireland}

(1) In section 1(2)($b$)  of the Social Security Contributions and Benefits (Northern Ireland) Act 1992 (Class 1A contributions), the words “in respect of cars made available for private use and car fuel” shall be omitted.

(2) For section 10 of that Act (Class 1A contributions) there shall be substituted—
\begin{quotation}
\subsection*{“10. Class 1A contributions: benefits in kind etc}

(1) Where—
\begin{enumerate}\item[]
($a$) for any tax year an earner is chargeable to income tax under Schedule E on an amount which for the purposes of the Income Tax Acts is or falls to be treated as an emolument received by him from any employment (“the relevant employment”),

($b$) the relevant employment is both employed earner’s employment and employment to which Chapter II of Part V of the 1988 Act (employment as a director or with annual emoluments of more than £8,500) applies, and

($c$) the whole or a part of the emolument falls, for the purposes of Class 1 contributions, to be left out of account in the computation of the earnings paid to or for the benefit of the earner,
\end{enumerate}
a Class 1A contribution shall be payable for that tax year, in accordance with this section, in respect of that earner and so much of the emolument as falls to be so left out of account.

(2) Subject to section 10ZA below, a Class 1A contribution for any tax year shall be payable by—
\begin{enumerate}\item[]
($a$) the person who is liable to pay the secondary Class 1 contribution relating to the last (or only) relevant payment of earnings in that tax year in relation to which there is a liability to pay such a Class 1 contribution; or

($b$) if paragraph ($a$)  above does not apply, the person who, if the emolument in respect of which the Class 1A contribution is payable were earnings in respect of which Class 1 contributions would be payable, would be liable to pay the secondary Class 1 contribution.
\end{enumerate}

(3) In subsection (2)  above “relevant payment of earnings” means a payment which for the purposes of Class 1 contributions is a payment of earnings made to or for the benefit of the earner in respect of the relevant employment.

(4) The amount of the Class 1A contribution in respect of any emolument shall be the Class 1A percentage of so much of it as falls to be left out of account as mentioned in subsection (1)($c$)  above.

(5) In subsection (4)  above “the Class 1A percentage” means a percentage rate equal to the percentage rate specified as the secondary percentage in section 9(2)  above for the tax year in question.

(6) No Class 1A contribution shall be payable for any tax year in respect of so much of any emolument as is taken for the purposes of the making of Class 1B contributions for that year to be included in a PAYE settlement agreement.

(7) For the purposes of this section—
\begin{enumerate}\item[]
($a$) the amounts which for the purposes of the Income Tax Acts are or fall to be treated as emoluments received by an earner from any employment shall be determined (subject to paragraph ($b$)  below) disregarding sections 198, 201, 201AA and 332(3)  of the 1988 Act (deductions for expenses etc.);\ but

($b$) where an amount which is deductible in respect of any matter under any of those sections is at least equal to the whole of any corresponding amount which (but for this paragraph) would fall by reference to that matter to be included in those emoluments, the whole of the corresponding amount shall be treated as not so included.
\end{enumerate}

(8) The Treasury may by regulations—
\begin{enumerate}\item[]
($a$) modify the effect of subsection (7)  above by adding any enactment contained in the Income Tax Acts to the list of sections of the 1988 Act contained in paragraph ($a$)  of that subsection; or

($b$) make such amendments of subsection (7)  above as appear to them to be necessary or expedient in consequence of any alteration of the provisions of the Income Tax Acts relating to the charge to tax under Schedule E.
\end{enumerate}

(9) The Treasury may by regulations provide—
\begin{enumerate}\item[]
($a$) for Class 1A contributions not to be payable, in prescribed circumstances, by prescribed persons or in respect of prescribed persons or emoluments;

($b$) for reducing Class 1A contributions in prescribed circumstances.
\end{enumerate}

(10) In this section “the 1988 Act” means the Income and Corporation Taxes Act 1988.”
\end{quotation}

(3) For subsection (6)  of section 4 of that Act (power to treat emoluments in respect of share acquisitions etc.\ as earnings) there shall be substituted—
\begin{quotation}
“(6) Regulations may make provision for the purposes of this Part—
\begin{enumerate}\item[]
($a$) for treating any amount on which an employed earner is chargeable to income tax under Schedule E as remuneration derived from the earner’s employment; and

($b$) for treating any amount which in accordance with regulations under paragraph ($a$)  above constitutes remuneration as an amount of remuneration paid, at such time as may be determined in accordance with the regulations, to or for the benefit of the earner in respect of his employment.”
\end{enumerate}
\end{quotation}

(4) In paragraph 5($b$)  of Schedule 1 to that Act (power to modify section 10 for cases where a car is made available by reason of more than one employment), for “a car is made available” there shall be substituted “something is provided or made available”.

(5) In paragraph 8(1)($ia$) of that Schedule (power to provide by regulations for repayment in prescribed cases of the whole or a part of a Class 1B contribution), after “part” there shall be inserted “of a Class 1A or”.

(6) In section 114(4)  of the Social Security Administration (Northern Ireland) Act 1992 (proof of previous offences relating to Class 1A contributions), for “car” there shall be substituted “amount”.

(7) In section 142(5)($c$)  of that Act (appropriate health service allocation of Class 1A contributions), for “cash equivalents of the benefits of the cars and car fuel” there shall be substituted “emoluments”.

(8) This section shall have effect in relation to the tax year beginning with 6th April 2000 and subsequent tax years.

(9) Regulations made by statutory instrument under any power conferred by virtue of this section may be made so as to have retrospective effect in relation to any time in the tax year in which they are made (including, in the case of regulations made in the tax year in which this Act is passed, any time in that tax year before the passing of this Act).

\subsection{79. Third party providers of benefits in kind: Northern Ireland}

(1) After section 10 of the Social Security Contributions and Benefits (Northern Ireland) Act 1992 there shall be inserted—
\begin{quotation}
\subsection*{“10ZA. Liability of third party provider of benefits in kind}

(1) This section applies, where—
\begin{enumerate}\item[]
($a$) a Class 1A contribution is payable for any tax year in respect of the whole or any part of an emolument received by an earner;

($b$) the emolument, in so far as it is one in respect of which such a contribution is payable, consists in a benefit provided for the earner or a member of his family or household;

($c$) the person providing the benefit is a person other than the person (“the relevant employer”) by whom, but for this section, the Class 1A contribution would be payable in accordance with section 10(2)  above; and

($d$) the provision of the benefit by that other person has not been arranged or facilitated by the relevant employer.
\end{enumerate}

(2) For the purposes of this Act if—
\begin{enumerate}\item[]
($a$) the person providing the benefit pays an amount for the purpose of discharging any liability of the earner to income tax for any tax year, and

($b$) the income tax in question is tax chargeable in respect of the provision of the benefit or of the making of the payment itself,
\end{enumerate}
the amount of the payment shall be treated as if it were an emolument consisting in the provision of a benefit to the earner in that tax year and falling, for the purposes of Class 1 contributions, to be left out of account in the computation of the earnings paid to or for the benefit of the earner.

(3) Subject to subsection (4)  below, the liability to pay any Class 1A contribution in respect of—
\begin{enumerate}\item[]
($a$) the benefit provided to the earner, and

($b$) any further benefit treated as so provided in accordance with subsection (2)  above,
\end{enumerate}
shall fall on the person providing the benefit, instead of on the relevant employer.

(4) Subsection (3)  above applies in the case of a Class 1A contribution for the tax year beginning with 6th April 2000 only if the person providing the benefit in question gives notice in writing to the Inland Revenue on or before 6th July 2001 that he is a person who provides benefits in respect of which a liability to Class 1A contributions is capable of falling by virtue of this section on a person other than the relevant employer.

(5) The Treasury may by regulations make provision specifying the circumstances in which a person is or is not to be treated for the purposes of this Act as having arranged or facilitated the provision of any benefit.

(6) In this section references to a member of a person’s family or household shall be construed in accordance with section 168(4)  of the Income and Corporation Taxes Act 1988. 

\subsection*{10ZB. Non-cash vouchers provided by third parties}

(1) In section 10ZA above references to the provision of a benefit include references to the provision of a non-cash voucher.

(2) Where—
\begin{enumerate}\item[]
($a$) a non-cash voucher is received by any person from employment to which Chapter II of Part V of the Income and Corporation Taxes Act 1988 does not apply, and

($b$) the case would be one in which the conditions in section 10ZA(1)($a$)  to ($d$)  above would be satisfied in relation to the provision of that voucher if that Chapter did apply to that employment,
\end{enumerate}
sections 10 and 10ZA above shall have effect in relation to the provision of that voucher, and to any such payment in respect of the provision of that voucher as is mentioned in section 10ZA(2)  above, as if that employment were employment to which that Chapter applied.

(3) In this section “non-cash voucher” has the same meaning as in section 141 of the Income and Corporation Taxes Act 1988.”
\end{quotation}

(2) After subsection (3)  of section 104ZA of the Social Security Administration (Northern Ireland) Act 1992 (premises liable to inspection) there shall be inserted—
\begin{quotation}
“(3A) The references in subsection (3)  above to a trade or business include references to the administration of any scheme for the provision of benefits to persons by reason of their employment.”
\end{quotation}

(3) Subsection (1)  shall have effect in relation to the tax year beginning with 6th April 2000 and subsequent tax years.

(4) Regulations made by virtue of this section under section 10ZA(5)  of the  Social Security Contributions and Benefits (Northern Ireland) Act 1992 may be made so as to have retrospective effect in relation to any time in the tax year in which they are made (including, in the case of regulations made in the tax year in which this Act is passed, any time in that tax year before the passing of this Act).

\subsection{80. Collection etc.\ of NICs: Northern Ireland.}

(1) Schedule 1 to the Social Security Contributions and Benefits (Northern Ireland) Act 1992 (supplementary provisions relating to contributions) shall be amended in accordance with subsections (2)  to (5).

(2) In paragraph 7(2)($b$)  (application of sections 100 to 100D and 102 to 104 of the Taxes Management Act 1970 in relation to certain penalties), for “104” there shall be substituted “105”.

(3) For sub-paragraph (2)($e$)  of paragraph 7B (power to provide for interest to be charged on late payment in the case of payment outside the PAYE system) there shall be substituted—
\begin{quotation}
“($e$) require interest to be paid on contributions that are not paid by the due date, and provide for determining the date from which such interest is to be calculated;”.
\end{quotation}

(4) After sub-paragraph (5)  of that paragraph there shall be inserted—
\begin{quotation}
“(5A) Regulations under this paragraph may, in relation to any penalty imposed by such regulations, make provision applying (with or without modifications) any enactment applying for the purposes of income tax that is contained in Part X of the Taxes Management Act 1970 (penalties).”
\end{quotation}

(5) After that paragraph there shall be inserted—
\begin{quotation}
“7BA. The Inland Revenue may by regulations provide for amounts in respect of contributions or interest that fall to be paid or repaid in accordance with any regulations under this Schedule to be set off, or to be capable of being set off, in prescribed circumstances and to the prescribed extent, against any such liabilities under regulations under this Schedule of the person entitled to the payment or repayment as may be prescribed.”
\end{quotation}

(6) In Article 7(1)  of the Social Security Contributions (Transfer of Functions, etc.)\ (Northern Ireland) Order 1999 (decisions to be made by an Inland Revenue officer and appealable under Article 10)—
\begin{enumerate}\item[]
($a$) sub-paragraph ($j$)  (interest under regulations made by virtue of paragraph 7B(2)($e$)  of Schedule 1 to the Social Security Contributions and Benefits (Northern Ireland) Act 1992) shall cease to have effect; and

($b$) in sub-paragraph ($l$), for “sub-paragraphs ($j$)  and ($k$)” there shall be substituted “sub-paragraph ($k$)”, and the words “amount of interest or” shall be omitted.
\end{enumerate}

(7) Subsection (6)  has effect in relation to interest accruing on sums becoming due in respect of the tax year beginning with 6th April 2000 or any subsequent tax year.

\subsection{81. Liability of earner for secondary contributions: Northern Ireland}

(1) In paragraph 3 of Schedule 1 to the Social Security Contributions and Benefits (Northern Ireland) Act 1992 (prohibition on deduction or recovery of Class 1 contributions), sub-paragraph (2)  shall be omitted.

(2) After that paragraph there shall be inserted—
\begin{quotation}
\subsection*{“Prohibition on recovery of employer’s contributions}

3A.---(1) Subject to sub-paragraph (2)  below, a person who is or has been liable to pay any secondary Class 1 or any Class 1A or Class 1B contributions shall not—
\begin{enumerate}\item[]
($a$) make, from earnings paid by him, any deduction in respect of any such contributions for which he or any other person is or has been liable;

($b$) otherwise recover any such contributions (directly or indirectly) from any person who is or has been a relevant earner; or

($c$) enter into any agreement with any person for the making of any such deduction or otherwise for the purpose of so recovering any such contributions.
\end{enumerate}

(2) Sub-paragraph (1)  above does not apply to the extent that an agreement between—
\begin{enumerate}\item[]
($a$) a secondary contributor, and

($b$) any person (“the earner”) in relation to whom the secondary contributor is, was or will be such a contributor in respect of the contributions to which the agreement relates,
\end{enumerate}
allows the secondary contributor to recover (whether by deduction or otherwise) the whole or any part of any secondary Class 1 contribution payable in respect of a gain that is treated as remuneration derived from that earner’s employment by virtue of section 4(4)($a$)  above.

(3) Sub-paragraph (2)  above does not authorise any recovery (whether by deduction or otherwise)—
\begin{enumerate}\item[]
($a$) in pursuance of any agreement entered into before 19th May 2000; or

($b$) in respect of any liability to a contribution arising before the day of the passing of the Child Support, Pensions and Social Security Act 2000. 
\end{enumerate}

(4) In this paragraph—
\begin{enumerate}\item[]
    “agreement” includes any arrangement or understanding (whether or not legally enforceable); and

    “relevant earner”, in relation to a person who is or has been liable to pay any contributions, means an earner in respect of whom he is or has been so liable. 
\end{enumerate}

\subsection*{Transfer of liability to be borne by earner}

3B.---(1) This paragraph applies where—
\begin{enumerate}\item[]
($a$) an election is jointly made by—
\begin{enumerate}\item[]
(i) a secondary contributor, and

(ii) a person (“the earner”) in relation to whom the secondary contributor is or will be such a contributor in respect of contributions on share option gains by the earner,
\end{enumerate}
for the whole or a part of any liability of the secondary contributor to contributions on any such gains to be transferred to the earner; and

($b$) the election is one in respect of which the Inland Revenue have, before it was made, given by notice to the secondary contributor their approval to both—
\begin{enumerate}\item[]
(i) the form of the election; and

(ii) the arrangements made in relation to the proposed election for securing that the liability transferred by the election will be met.
\end{enumerate}
\end{enumerate}

(2) Any liability which—
\begin{enumerate}\item[]
($a$) arises while the election is in force, and

($b$) is a liability to pay the contributions on share option gains by the earner, or the part of them, to which the election relates,
\end{enumerate}
shall be treated for the purposes of this Act, the Administration Act and Part III of the Social Security Contributions (Transfer of Functions, etc.)\ (Northern Ireland) Order 1999 as a liability falling on the earner, instead of on the secondary contributor.

(3) Subject to sub-paragraph (7)($b$)  below, an election made for the purposes of sub-paragraph (1)  above shall continue in force from the time when it is made until whichever of the following first occurs, namely—
\begin{enumerate}\item[]
($a$) it ceases to have effect in accordance with its terms;

($b$) it is revoked jointly by both parties to the election;

($c$) notice is given to the earner by the secondary contributor terminating the effect of the election.
\end{enumerate}

(4) An approval given to the secondary contributor for the purposes of sub-paragraph (1)($b$)  above may be given either—
\begin{enumerate}\item[]
($a$) for an election to be made by the secondary contributor and a particular person; or

($b$) for all elections to be made, or to be made in particular circumstances, by the secondary contributor and particular persons or by the secondary contributor and persons of a particular description.
\end{enumerate}

(5) The grounds on which the Inland Revenue shall be entitled to refuse an approval for the purposes of sub-paragraph (1)($b$)  above shall include each of the following—
\begin{enumerate}\item[]
($a$) that it appears to the Inland Revenue that adequate arrangements have not been made for securing that the liabilities transferred by the proposed election or elections will be met by the person or persons to whom they would be so transferred; and

($b$) that it appears to the Inland Revenue that they do not have sufficient information to determine whether or not grounds falling within paragraph ($a$)  above exist.
\end{enumerate}

(6) If, at any time after they have given an approval for the purposes of sub-paragraph (1)($b$)  above, it appears to the Inland Revenue—
\begin{enumerate}\item[]
($a$) that the arrangements that were made or are in force for securing that liabilities transferred by elections to which the approval relates are met are proving inadequate or unsatisfactory in any respect, or

($b$) that any election to which the approval relates has resulted, or is likely to result, in the avoidance or non-payment of the whole or any part of any secondary Class 1 contributions,
\end{enumerate}
the Inland Revenue may withdraw the approval by notice to the secondary contributor.

(7) The withdrawal by the Inland Revenue of any approval given for the purposes of sub-paragraph (1)($b$)  above—
\begin{enumerate}\item[]
($a$) may be either general or confined to a particular election or to particular elections; and

($b$) shall have the effect that the election to which the withdrawal relates has no effect on contributions on share option gains in respect of any right to acquire shares obtained after—
\begin{enumerate}\item[]
(i) the date on which notice of the withdrawal of the approval is given; or

(ii) such later date as the Inland Revenue may specify in that notice.
\end{enumerate}
\end{enumerate}

(8) Where the Inland Revenue have refused or withdrawn their approval for the purposes of sub-paragraph (1)($b$)  above, the person who applied for it or, as the case may be, to whom it was given may appeal to the Special Commissioners against the Inland Revenue’s decision.

(9) On an appeal under sub-paragraph (8)  above the Special Commissioners may—
\begin{enumerate}\item[]
($a$) dismiss the appeal;

($b$) remit the decision appealed against to the Inland Revenue with a direction to make such decision as the Special Commissioners think fit; or

($c$) in the case of a decision to withdraw an approval, quash that decision and direct that that decision is to be treated as never having been made.
\end{enumerate}

(10) Subject to sub-paragraph (12)  below, an election under sub-paragraph (1)  above shall not apply to any contributions in respect of gains realised before it was made.

(11) Regulations made by the Inland Revenue may make provision with respect to the making of elections for the purposes of this paragraph and the giving of approvals for the purposes of sub-paragraph (1)($b$)  above; and any such regulations may, in particular—
\begin{enumerate}\item[]
($a$) prescribe the matters that must be contained in such an election;

($b$) provide for the manner in which such an election is to be capable of being made and of being confined to particular liabilities or the part of particular liabilities; and

($c$) provide for the making of applications for such approvals and for the manner in which those applications are to be dealt with.
\end{enumerate}

(12) Where—
\begin{enumerate}\item[]
($a$) an election is made under this paragraph before the end of the period of three months beginning with the date of the passing of the Child Support, Pensions and Social Security Act 2000, and

($b$) that election is expressed to relate to liabilities for contributions arising on or after 19th May 2000 and before the making of the election,
\end{enumerate}
this paragraph shall have effect in relation to those liabilities as if sub-paragraph (2)  above provided for them to be deemed to have fallen on the earner (instead of on the secondary contributor); and the secondary contributor shall accordingly be entitled to reimbursement from the earner for any payment made by that contributor in or towards the discharge of any of those liabilities.

(13) In this paragraph references to contributions on share option gains by the earner are references to any secondary Class 1 contributions payable in respect of a gain that is treated as remuneration derived from the earner’s employment by virtue of section 4(4)($a$)  above.

(14) In this paragraph “the Special Commissioners” means the Commissioners for the special purposes of the Income Tax Acts.”
\end{quotation}

(3) In section 6(4)  of that Act (persons by whom Class 1 contributions are payable), for the words from “paragraph 3” onwards there shall be substituted “paragraphs 3 to 3B of Schedule 1 to this Act.”

(4) In paragraph 8(1)  of Schedule 1 to that Act (general regulations), after paragraph ($c$)  there shall be inserted—
\begin{quotation}
“($ca$) for requiring a secondary contributor to notify a person to whom any of his liabilities are transferred by an election under paragraph 3B above of—
\begin{enumerate}\item[]
(i) any transferred liability that arises;

(ii) the amount of any transferred liability that arises; and

(iii) the contents of any notice of withdrawal by the Inland Revenue of any approval that relates to that election;”.
\end{enumerate}
\end{quotation}

(5) In Article 7(1)  of the Social Security Contributions (Transfer of Functions, etc.)\ (Northern Ireland) Order 1999 (decisions to be taken by officers of the Inland Revenue), after sub-paragraph ($i$)  there shall be inserted—
\begin{quotation}
“($ia$) to decide whether to give or withdraw an approval for the purposes of paragraph 3B(1)($b$)  of Schedule 1 to the Contributions and Benefits Act;”.
\end{quotation}

(6) In Article 9 of that Order (regulations about varying or superseding decisions), at the beginning of paragraph (1)  there shall be inserted “Subject to paragraph (2A)  below,”, and after paragraph (2)  there shall be inserted—
\begin{quotation}
“(2A) The decisions in relation to which provision may be made by regulations under this Article shall not include decisions falling within Article 7(1)($ia$) of this Order.”
\end{quotation}

(7) In Article 11(4)  of that Order (appeals to be heard by General Commissioners), after “Subject to” there shall be inserted “paragraph 3B(8)  of Schedule 1 to the Contributions and Benefits Act (which provides for appeals under that paragraph to be heard by the Special Commissioners), to”.

\part[Part V --- Miscellaneous and supplemental]{Part V\\*Miscellaneous and supplemental}

\renewcommand\parthead{--- Part V}

%\section{\itshape Miscellaneous}
%
%82Tests for determining parentage
%
%(1) Part III of the [1969 c. 46. ] Family Law Reform Act 1969 (tests for determining parentage) shall be amended in accordance with subsections (2)  to (4) .
%
%(2) In section 20 (power of the court to require tests)—
%
%($a$) for subsections (1A) and (1B) (nomination of the person by whom tests are to be carried out) there shall be substituted—
%
%“(1A)Tests required by a direction under this section may only be carried out by a body which has been accredited for the purposes of this section by—
%
%($a$) the Lord Chancellor, or
%
%($b$) a body appointed by him for the purpose.”;
%
%($b$) in subsection (2) —
%
%(i) for “person responsible for” there shall be substituted “individual”, and
%
%(ii) after “this section” there shall be inserted “(“the tester”)”;
%
%($c$) in subsection (4) , for “the person who made the report” there shall be substituted “the tester”; and
%
%($d$) in subsection (5) —
%
%(i) for “the person responsible for carrying out the tests taken for the purpose of giving effect to the direction, or any” there shall be substituted “the tester, or any other”,
%
%(ii) for “that person” there shall be substituted “the tester or that other person”, and
%
%(iii) after “and where” there shall be inserted “the tester or”.
%
%(3) In section 21 (consents, etc, required for the taking of blood samples), in subsection (3), for the words “if the person who has the care and control of him consents” there shall be substituted—
%
%“($a$) if the person who has the care and control of him consents; or
%
%($b$) where that person does not consent, if the court considers that it would be in his best interests for the sample to be taken.”
%
%(4) In section 22(1)  (power of Lord Chancellor to make further provision relating to tests for determining parentage)—
%
%($a$) in paragraph ($a$)  (power to provide that bodily samples are not to be taken except by such medical practitioners as may be appointed by the Lord Chancellor), for the words from “such medical practitioners” to the end there shall be substituted “registered medical practitioners or members of such professional bodies as may be prescribed by the regulations;”, and
%
%($b$) for paragraph ($e$)  (power to provide that scientific tests are not to be carried out except by persons appointed by the Lord Chancellor) there shall be substituted—
%
%“($e$) prescribe conditions which a body must meet in order to be eligible for accreditation for the purposes of section 20 of this Act;”.
%
%(5) The amendments made by this section shall not have effect in relation to any proceedings pending at the commencement of this section.
%83Declarations of status
%
%(1) Part III of the [1986 c. 55. ] Family Law Act 1986 (declarations of status) shall be amended as follows.
%
%(2) After section 55 there shall be inserted—
%“55ADeclarations of parentage
%
%(1) Subject to the following provisions of this section, any person may apply to the High Court, a county court or a magistrates' court for a declaration as to whether or not a person named in the application is or was the parent of another person so named.
%
%(2) A court shall have jurisdiction to entertain an application under subsection (1)  above if, and only if, either of the persons named in it for the purposes of that subsection—
%
%($a$) is domiciled in England and Wales on the date of the application, or
%
%($b$) has been habitually resident in England and Wales throughout the period of one year ending with that date, or
%
%($c$) died before that date and either—
%
%(i) was at death domiciled in England and Wales, or
%
%(ii) had been habitually resident in England and Wales throughout the period of one year ending with the date of death.
%
%(3) Except in a case falling within subsection (4)  below, the court shall refuse to hear an application under subsection (1)  above unless it considers that the applicant has a sufficient personal interest in the determination of the application (but this is subject to section 27 of the [1991 c. 48. ] Child Support Act 1991).
%
%(4) The excepted cases are where the declaration sought is as to whether or not—
%
%($a$) the applicant is the parent of a named person;
%
%($b$) a named person is the parent of the applicant; or
%
%($c$) a named person is the other parent of a named child of the applicant.
%
%(5) Where an application under subsection (1)  above is made and one of the persons named in it for the purposes of that subsection is a child, the court may refuse to hear the application if it considers that the determination of the application would not be in the best interests of the child.
%
%(6) Where a court refuses to hear an application under subsection (1)  above it may order that the applicant may not apply again for the same declaration without leave of the court.
%
%(7) Where a declaration is made by a court on an application under subsection (1)  above, the prescribed officer of the court shall notify the Registrar General, in such a manner and within such period as may be prescribed, of the making of that declaration.”
%
%(3) Section 58(5)($b$)  (prohibition of declarations of illegitimacy) shall be omitted.
%
%(4) After section 60(4)  there shall be inserted—
%
%“(5) An appeal shall lie to the High Court against—
%
%($a$) the making by a magistrates' court of a declaration under section 55A above,
%
%($b$) any refusal by a magistrates' court to make such a declaration, or
%
%($c$) any order under subsection (6)  of that section made on such a refusal.”
%
%(5) Schedule 8 (which makes amendments consequential on subsection (1) ) shall have effect.
%
%(6) Nothing in this Act shall affect any proceedings pursuant to an application under—
%
%($a$) section 56(1)($a$)  of the [1986 c. 55. ] Family Law Act 1986, or
%
%($b$) section 27 of the [1991 c. 48. ] Child Support Act 1991,
%
%which are pending immediately before the commencement of this section.

\amendment{
Ss. 82, 83 are not yet in force.
}

\section{\itshape Supplemental}

\subsection{84. Expenses}

There shall be paid out of money provided by Parliament—
\begin{enumerate}\item[]
($a$) any expenditure incurred by the Secretary of State for or in connection with the carrying out of his functions under this Act; and

($b$) any increase attributable to this Act in the sums which are payable out of money so provided under any other Act.
\end{enumerate}

\subsection{85. Repeals}

(1) The enactments mentioned in Schedule 9 (which include some spent provisions) are hereby repealed to the extent specified in the third column of that Schedule.

(2) The repeals specified in that Schedule have effect subject to the commencement provisions and savings contained, or referred to, in the notes set out in that Schedule.

\subsection{86. Commencement and transitional provisions}

(1) This section applies to the following provisions of this Act—
\begin{enumerate}\item[]
($a$) Part I (other than section 24);

($b$) Part II (other than sections 38 and 39 and paragraphs 4 to 6, 8(1), (3)  and (4)  and 13 of Schedule 5);

($c$) Part III;

($d$) sections 82 and 83 and Schedule 8;

($e$) Parts I to VII and IX of Schedule 9. 
\end{enumerate}

(2) The provisions of this Act to which this section applies shall come into force on such day as may be appointed by order made by statutory instrument; and different days may be appointed under this section for different purposes.

(3) The power to make an order under subsection (2)  shall be exercisable—
\begin{enumerate}\item[]
($a$) except in a case falling within paragraph ($b$), by the Secretary of State; and

($b$) in the case of an order bringing into force any of the provisions of sections 82 and 83, Schedule 8 or Part IX of Schedule 9, by the Lord Chancellor.
\end{enumerate}

(4) In the case of Part I (other than section 24) and of sections 62 to 66, the power under subsection (2)  to appoint different days for different purposes includes power to appoint different days for different areas.

(5) The Secretary of State may by regulations make such transitional provision as he considers necessary or expedient in connection with the bringing into force of any of the following provisions of this Act—
\begin{enumerate}\item[]
($a$) sections 43 to 46 and section (1)  of Part III of Schedule 9;

($b$) sections 68 to 70 and Schedule 7 and Part VII of Schedule 9. 
\end{enumerate}

(6) Regulations under subsection (5)  shall be made by statutory instrument subject to annulment in pursuance of a resolution of either House of Parliament.

(7) Section 174(2)  to (4)  of the Pensions Act 1995 (supplementary provision in relation to powers to make subordinate legislation under that Act) shall apply in relation to the power to make regulations under subsection (5)  as it applies to any power to make regulations under that Act.

(8) In this section “subordinate legislation” has the same meaning as in the  Interpretation Act 1978. 

\subsection{87. Short title and extent}

(1) This Act may be cited as the Child Support, Pensions and Social Security Act 2000. 

(2) The following provisions of this Act extend to Northern Ireland—
\begin{enumerate}\item[]
($a$) so much of section 46 as amends section 21(3)  of the Pensions Act 1995;

($b$) sections 57 to 61 (except section 60(5) );

($c$) section 73;

($d$) sections 78 to 81;

($e$) in Schedule 3, paragraphs 8 and 9, and in paragraph 11, sub-paragraph (2)  (and sub-paragraph (1)  so far as it relates to that sub-paragraph);

($f$) paragraph 6 of Schedule 5; and

($g$) this Part, except—
\begin{enumerate}\item[]
(i) sections 82 and 83 and Schedule 8; and

(ii) so much of this Part as gives effect to any repeal other than the repeals mentioned in subsection (3) .
\end{enumerate}
\end{enumerate}

(3) The repeals mentioned in subsection (2)($g$)  (which extend to Northern Ireland) are—
\begin{enumerate}\item[]
($a$) the repeals, in Part I of Schedule 9, that relate to the Tax Credits Act 1999;

($b$) the repeals, in sections (1), (6)  and (11)  of Part III of that Schedule, that relate to—
\begin{enumerate}\item[]
(i) section 21(3)  of the Pensions Act 1995;

(ii) paragraph 49($a$)(ii)  of Schedule 3 to the Pensions (Northern Ireland) Order 1995; and

(iii) section 52(5)  of the Pension Schemes (Northern Ireland) Act 1993;
\end{enumerate}

($c$) the repeals in Part IV of that Schedule (except so far as relating to the Courts and Legal Services Act 1990); and

($d$) the repeals in section (2)  of Part VIII of that Schedule.
\end{enumerate}

(4) Subject to that, this Act does not extend to Northern Ireland.

\small

\part*{\textls{SCHEDULES}}

%SCHEDULE 1Substituted Part I of Schedule 1 to the Child Support Act 1991
%“PART ICalculation of weekly amount of child support maintenance
%General rule
%
%1(1) The weekly rate of child support maintenance is the basic rate unless a reduced rate, a flat rate or the nil rate applies.
%
%(2) Unless the nil rate applies, the amount payable weekly to a person with care is—
%
%($a$) the applicable rate, if paragraph 6 does not apply; or
%
%($b$) if paragraph 6 does apply, that rate as apportioned between the persons with care in accordance with paragraph 6,
%
%as adjusted, in either case, by applying the rules about shared care in paragraph 7 or 8. 
%Basic rate
%
%2(1) The basic rate is the following percentage of the non-resident parent’s net weekly income—
%
%    15% where he has one qualifying child;
%
%    20% where he has two qualifying children;
%
%    25% where he has three or more qualifying children. 
%
%(2) If the non-resident parent also has one or more relevant other children, the appropriate percentage referred to in sub-paragraph (1)  is to be applied instead to his net weekly income less—
%
%    15% where he has one relevant other child;
%
%    20% where he has two relevant other children;
%
%    25% where he has three or more relevant other children. 
%
%Reduced rate
%
%3(1) A reduced rate is payable if—
%
%($a$) neither a flat rate nor the nil rate applies; and
%
%($b$) the non-resident parent’s net weekly income is less than £200 but more than £100. 
%
%(2) The reduced rate payable shall be prescribed in, or determined in accordance with, regulations.
%
%(3) The regulations may not prescribe, or result in, a rate of less than £5. 
%Flat rate
%
%4(1) Except in a case falling within sub-paragraph (2) , a flat rate of £5 is payable if the nil rate does not apply and—
%
%($a$) the non-resident parent’s net weekly income is £100 or less; or
%
%($b$) he receives any benefit, pension or allowance prescribed for the purposes of this paragraph of this sub-paragraph; or
%
%($c$) he or his partner (if any) receives any benefit prescribed for the purposes of this paragraph of this sub-paragraph.
%
%(2) A flat rate of a prescribed amount is payable if the nil rate does not apply and—
%
%($a$) the non-resident parent has a partner who is also a non-resident parent;
%
%($b$) the partner is a person with respect to whom a maintenance calculation is in force; and
%
%($c$) the non-resident parent or his partner receives any benefit prescribed under sub-paragraph (1)($c$) .
%
%(3) The benefits, pensions and allowances which may be prescribed for the purposes of sub-paragraph (1)($b$)  include ones paid to the non-resident parent under the law of a place outside the United Kingdom.
%Nil rate
%
%5The rate payable is nil if the non-resident parent—
%
%($a$) is of a prescribed description; or
%
%($b$) has a net weekly income of below £5. 
%Apportionment
%
%6(1) If the non-resident parent has more than one qualifying child and in relation to them there is more than one person with care, the amount of child support maintenance payable is (subject to paragraph 7 or 8) to be determined by apportioning the rate between the persons with care.
%
%(2) The rate of maintenance liability is to be divided by the number of qualifying children, and shared among the persons with care according to the number of qualifying children in relation to whom each is a person with care.
%Shared care—basic and reduced rate
%
%7(1) This paragraph applies only if the rate of child support maintenance payable is the basic rate or a reduced rate.
%
%(2) If the care of a qualifying child is shared between the non-resident parent and the person with care, so that the non-resident parent from time to time has care of the child overnight, the amount of child support maintenance which he would otherwise have been liable to pay the person with care, as calculated in accordance with the preceding paragraphs of this Part of this Schedule, is to be decreased in accordance with this paragraph.
%
%(3) First, there is to be a decrease according to the number of such nights which the Secretary of State determines there to have been, or expects there to be, or both during a prescribed twelve-month period.
%
%(4) The amount of that decrease for one child is set out in the following Table—
%Number of nights	Fraction to subtract
%52 to 103	One-seventh
%104 to 155	Two-sevenths
%156 to 174	Three-sevenths
%175 or more	One-half
%
%(5) If the person with care is caring for more than one qualifying child of the non-resident parent, the applicable decrease is the sum of the appropriate fractions in the Table divided by the number of such qualifying children.
%
%(6) If the applicable fraction is one-half in relation to any qualifying child in the care of the person with care, the total amount payable to the person with care is then to be further decreased by £7 for each such child.
%
%(7) If the application of the preceding provisions of this paragraph would decrease the weekly amount of child support maintenance (or the aggregate of all such amounts) payable by the non-resident parent to the person with care (or all of them) to less than £5, he is instead liable to pay child support maintenance at the rate of £5 a week, apportioned (if appropriate) in accordance with paragraph 6. 
%Shared care—flat rate
%
%8(1) This paragraph applies only if—
%
%($a$) the rate of child support maintenance payable is a flat rate; and
%
%($b$) that rate applies because the non-resident parent falls within paragraph 4(1)($b$)  or ($c$)  or 4(2).
%
%(2) If the care of a qualifying child is shared as mentioned in paragraph 7(2)  for at least 52 nights during a prescribed 12-month period, the amount of child support maintenance payable by the non-resident parent to the person with care of that child is nil.
%Regulations about shared care
%
%9The Secretary of State may by regulations provide—
%
%($a$) for which nights are to count for the purposes of shared care under paragraphs 7 and 8, or for how it is to be determined whether a night counts;
%
%($b$) for what counts, or does not count, as “care” for those purposes; and
%
%($c$) for paragraph 7(3)  or 8(2)  to have effect, in prescribed circumstances, as if the period mentioned there were other than 12 months, and in such circumstances for the Table in paragraph 7(4)  (or that Table as modified pursuant to regulations made under paragraph 10A(2)($a$) ), or the period mentioned in paragraph 8(2) , to have effect with prescribed adjustments.
%Net weekly income
%
%10(1) For the purposes of this Schedule, net weekly income is to be determined in such manner as is provided for in regulations.
%
%(2) The regulations may, in particular, provide for the Secretary of State to estimate any income or make an assumption as to any fact where, in his view, the information at his disposal is unreliable, insufficient, or relates to an atypical period in the life of the non-resident parent.
%
%(3) Any amount of net weekly income (calculated as above) over £2,000 is to be ignored for the purposes of this Schedule.
%Regulations about rates, figures, etc.
%
%10A(1) The Secretary of State may by regulations provide that—
%
%($a$) paragraph 2 is to have effect as if different percentages were substituted for those set out there;
%
%($b$) paragraph 3(1)  or (3), 4(1), 5, 7(7)  or 10(3)  is to have effect as if different amounts were substituted for those set out there.
%
%(2) The Secretary of State may by regulations provide that—
%
%($a$) the Table in paragraph 7(4)  is to have effect as if different numbers of nights were set out in the first column and different fractions were substituted for those set out in the second column;
%
%($b$) paragraph 7(6)  is to have effect as if a different amount were substituted for that mentioned there, or as if the amount were an aggregate amount and not an amount for each qualifying child, or both.
%Regulations about income
%
%10BThe Secretary of State may by regulations provide that, in such circumstances and to such extent as may be prescribed—
%
%($a$) where the Secretary of State is satisfied that a person has intentionally deprived himself of a source of income with a view to reducing the amount of his net weekly income, his net weekly income shall be taken to include income from that source of an amount estimated by the Secretary of State;
%
%($b$) a person is to be treated as possessing income which he does not possess;
%
%($c$) income which a person does possess is to be disregarded.
%References to various terms
%
%10C(1) References in this Part of this Schedule to “qualifying children” are to those qualifying children with respect to whom the maintenance calculation falls to be made.
%
%(2) References in this Part of this Schedule to “relevant other children” are to—
%
%($a$) children other than qualifying children in respect of whom the non-resident parent or his partner receives child benefit under Part IX of the [1992 c. 4. ] Social Security Contributions and Benefits Act 1992; and
%
%($b$) such other description of children as may be prescribed.
%
%(3) In this Part of this Schedule, a person “receives” a benefit, pension, or allowance for any week if it is paid or due to be paid to him in respect of that week.
%
%(4) In this Part of this Schedule, a person’s “partner” is—
%
%($a$) if they are a couple, the other member of that couple;
%
%($b$) if the person is a husband or wife by virtue of a marriage entered into under a law which permits polygamy, another party to the marriage who is of the opposite sex and is a member of the same household.
%
%(5) In sub-paragraph (4)($a$), “couple” means a man and a woman who are—
%
%($a$) married to each other and are members of the same household; or
%
%($b$) not married to each other but are living together as husband and wife.”.
%
%SCHEDULE 2Substituted Schedules 4A and 4B to the 1991 Act
%Part ISubstituted Schedule 4A
%“SCHEDULE 4AApplications for a variation
%Interpretation
%
%1In this Schedule, “regulations” means regulations made by the Secretary of State.
%Applications for a variation
%
%2Regulations may make provision—
%
%($a$) as to the procedure to be followed in considering an application for a variation;
%
%($b$) as to the procedure to be followed when an application for a variation is referred to an appeal tribunal under section 28D(1)($b$) .
%Completion of preliminary consideration
%
%3Regulations may provide for determining when the preliminary consideration of an application for a variation is to be taken to have been completed.
%Information
%
%4If any information which is required (by regulations under this Act) to be furnished to the Secretary of State in connection with an application for a variation has not been furnished within such period as may be prescribed, the Secretary of State may nevertheless proceed to consider the application.
%Joint consideration of applications for a variation and appeals
%
%5(1) Regulations may provide for two or more applications for a variation with respect to the same application for a maintenance calculation to be considered together.
%
%(2) In sub-paragraph (1), the reference to an application for a maintenance calculation includes an application treated as having been made under section 6. 
%
%(3) An appeal tribunal considering an application for a variation under section 28D(1)($b$)  may consider it at the same time as an appeal under section 20 in connection with an interim maintenance decision, if it considers that to be appropriate.”
%Part IISubstituted Schedule 4B
%“SCHEDULE 4BApplications for a variation: The Cases and Controls
%Part IThe Cases
%General
%
%1(1) The cases in which a variation may be agreed are those set out in this Part of this Schedule or in regulations made under this Part.
%
%(2) In this Schedule “applicant” means the person whose application for a variation is being considered.
%Special expenses
%
%2(1) A variation applied for by a non-resident parent may be agreed with respect to his special expenses.
%
%(2) In this paragraph “special expenses” means the whole, or any amount above a prescribed amount, or any prescribed part, of expenses which fall within a prescribed description of expenses.
%
%(3) In prescribing descriptions of expenses for the purposes of this paragraph, the Secretary of State may, in particular, make provision with respect to—
%
%($a$) costs incurred by a non-resident parent in maintaining contact with the child, or with any of the children, with respect to whom the application for a maintenance calculation has been made (or treated as made);
%
%($b$) costs attributable to a long-term illness or disability of a relevant other child (within the meaning of paragraph 10C(2)  of Schedule 1);
%
%($c$) debts of a prescribed description incurred, before the non-resident parent became a non-resident parent in relation to a child with respect to whom the maintenance calculation has been applied for (or treated as having been applied for)—
%
%(i) for the joint benefit of both parents;
%
%(ii) for the benefit of any such child; or
%
%(iii) for the benefit of any other child falling within a prescribed category;
%
%($d$) boarding school fees for a child in relation to whom the application for a maintenance calculation has been made (or treated as made);
%
%($e$) the cost to the non-resident parent of making payments in relation to a mortgage on the home he and the person with care shared, if he no longer has an interest in it, and she and a child in relation to whom the application for a maintenance calculation has been made (or treated as made) still live there.
%
%(4) For the purposes of sub-paragraph (3)($b$) —
%
%($a$) “disability” and “illness” have such meaning as may be prescribed; and
%
%($b$) the question whether an illness or disability is long-term shall be determined in accordance with regulations made by the Secretary of State.
%
%(5) For the purposes of sub-paragraph (3)($d$), the Secretary of State may prescribe—
%
%($a$) the meaning of “boarding school fees”; and
%
%($b$) components of such fees (whether or not itemised as such) which are, or are not, to be taken into account,
%
%and may provide for estimating any such component.
%Property or capital transfers
%
%3(1) A variation may be agreed in the circumstances set out in sub-paragraph (2)  if before 5th April 1993—
%
%($a$) a court order of a prescribed kind was in force with respect to the non-resident parent and either the person with care with respect to the application for the maintenance calculation or the child, or any of the children, with respect to whom that application was made; or
%
%($b$) an agreement of a prescribed kind between the non-resident parent and any of those persons was in force.
%
%(2) The circumstances are that in consequence of one or more transfers of property of a prescribed kind and exceeding (singly or in aggregate) a prescribed minimum value—
%
%($a$) the amount payable by the non-resident parent by way of maintenance was less than would have been the case had that transfer or those transfers not been made; or
%
%($b$) no amount was payable by the non-resident parent by way of maintenance.
%
%(3) For the purposes of sub-paragraph (2) , “maintenance” means periodical payments of maintenance made (otherwise than under this Act) with respect to the child, or any of the children, with respect to whom the application for a maintenance calculation has been made.
%Additional cases
%
%4(1) The Secretary of State may by regulations prescribe other cases in which a variation may be agreed.
%
%(2) Regulations under this paragraph may, for example, make provision with respect to cases where—
%
%($a$) the non-resident parent has assets which exceed a prescribed value;
%
%($b$) a person’s lifestyle is inconsistent with his income for the purposes of a calculation made under Part I of Schedule 1;
%
%($c$) a person has income which is not taken into account in such a calculation;
%
%($d$) a person has unreasonably reduced the income which is taken into account in such a calculation.
%Part IIRegulatory Controls
%
%5(1) The Secretary of State may by regulations make provision with respect to the variations from the usual rules for calculating maintenance which may be allowed when a variation is agreed.
%
%(2) No variations may be made other than those which are permitted by the regulations.
%
%(3) Regulations under this paragraph may, in particular, make provision for a variation to result in—
%
%($a$) a person’s being treated as having more, or less, income than would be taken into account without the variation in a calculation under Part I of Schedule 1;
%
%($b$) a person’s being treated as liable to pay a higher, or a lower, amount of child support maintenance than would result without the variation from a calculation under that Part.
%
%(4) Regulations may provide for the amount of any special expenses to be taken into account in a case falling within paragraph 2, for the purposes of a variation, not to exceed such amount as may be prescribed or as may be determined in accordance with the regulations.
%
%(5) Any regulations under this paragraph may in particular make different provision with respect to different levels of income.
%
%6The Secretary of State may by regulations provide for the application, in connection with child support maintenance payable following a variation, of paragraph 7(2)  to (7)  of Schedule 1 (subject to any prescribed modifications).”
%
%SCHEDULE 3Amendment of enactments relating to child support
%The Army Act 1955 (3 \& 4 Eliz. 2 c.18)
%
%1(1) Section 150A of the Army Act 1955 (enforcement of maintenance assessment by deductions from pay) shall be amended as follows.
%
%(2) In subsections (1), (2)($a$), (3)($a$)  (twice) and (4) , for “maintenance assessment” there shall be substituted “maintenance calculation”.
%
%(3) In subsection (3)  (twice), for “the assessment” there shall be substituted “the calculation”.
%The Air Force Act 1955 (3 \& 4 Eliz. 2 c.19)
%
%2(1) Section 150A of the Air Force Act 1955 (enforcement of maintenance assessment by deductions from pay) shall be amended as follows.
%
%(2) In subsections (1), (2)($a$), (3)($a$)  (twice) and (4) , for “maintenance assessment” there shall be substituted “maintenance calculation”.
%
%(3) In subsection (3)  (twice), for “the assessment” there shall be substituted “the calculation”.
%The Matrimonial Causes Act 1973 (c. 18)
%
%3(1) The Matrimonial Causes Act 1973 shall be amended as follows.
%
%(2) In section 29 (duration of continuing financial provision orders in favour of children, and age limit on making certain orders in their favour)—
%
%($a$) in subsections (5)($a$), (7)  (three times) and (8) ($a$), for “maintenance assessment” there shall be substituted “maintenance calculation”;
%
%($b$) in subsections (5)($a$)  and ($b$)(ii)  and (6) ($b$), for “current assessment” there shall be substituted “current calculation”;
%
%($c$) in subsection (6) ($b$), for “maintenance assessments” there shall be substituted “maintenance calculations”; and
%
%($d$) in subsection (6) ($b$), for “those assessments” there shall be substituted “those calculations”.
%
%(3) In section 31 (variation, discharge, etc, of certain orders for financial relief)—
%
%($a$) in subsections (11) ($c$)  and (12) ($a$)  and ($c$), for “maintenance assessment” there shall be substituted “maintenance calculation”; and
%
%($b$) in subsection (11) (twice), for “the assessment” there shall be substituted “the calculation”.
%
%(4) In section 52 (interpretation), in subsection (1), for “maintenance assessment” there shall be substituted “maintenance calculation”.
%The Domestic Proceedings and Magistrates Courts Act 1978 (c. 22)
%
%4(1) The Domestic Proceedings and Magistrates Courts Act 1978 shall be amended as follows.
%
%(2) In section 5 (age limit on making orders for financial provision for children and duration of such orders)—
%
%($a$) in subsections (5)($a$), (7)  (three times) and (8) ($a$), for “maintenance assessment” there shall be substituted “maintenance calculation”;
%
%($b$) in subsections (5)($a$)  and ($b$)(ii)  and (6) ($b$), for “current assessment” there shall be substituted “current calculation”; and
%
%($c$) in subsection (6) ($b$), for “those assessments” there shall be substituted “those calculations”.
%
%(3) In section 20 (variation, revival and revocation of orders for periodical payments)—
%
%($a$) in subsections (9A)($c$)  and (9B)($a$)  and ($c$), for “maintenance assessment” there shall be substituted “maintenance calculation”; and
%
%($b$) in subsection (9A) (three times), for “the assessment” there shall be substituted “the calculation”.
%
%(4) In section 88 (interpretation), in subsection (1), for “maintenance assessment” there shall be substituted “maintenance calculation”.
%The Family Law (Scotland) Act 1985 (c. 37)
%
%5(1) The Family Law (Scotland) Act 1985 shall be amended as follows.
%
%(2) In section 5 (variation and recall of decrees of aliment), in subsection (1A), for “maintenance assessment” there shall be substituted “maintenance calculation”.
%
%(3) In section 7 (agreements about aliment), in subsection (2A) , for “maintenance assessment” there shall be substituted “maintenance calculation”.
%
%(4) In section 13 (orders for periodical allowance), in subsection (4A), for “maintenance assessment” there shall be substituted “maintenance calculation”.
%
%(5) In section 16 (agreements about financial provision), in subsection (3)($d$), for “maintenance assessment” there shall be substituted “maintenance calculation”.
%
%(6) In section 27 (interpretation), in subsection (1), for “maintenance assessment” there shall be substituted “maintenance calculation”.
%The Insolvency Act 1986 (c. 45)
%
%6In section 281 of the Insolvency Act 1986 (effect of discharge on a bankrupt), in subsection (5)($b$), for “maintenance assessment” there shall be substituted “maintenance calculation”.
%The Debtors (Scotland) Act 1987 (c. 18)
%
%7(1) The Debtors (Scotland) Act 1987 shall be amended as follows.
%
%(2) In section 72 (effect of sequestration on diligence against earnings), in subsection (4A), for “maintenance assessment” there shall be substituted “maintenance calculation”.
%
%(3) In section 106 (interpretation), in the definition of “maintenance order”, in paragraph ($j$), for “maintenance assessment” there shall be substituted “maintenance calculation”.
%The Income and Corporation Taxes Act 1988 (c. 1)
%
%8(1) The Income and Corporation Taxes Act 1988 shall be amended as follows.
%
%(2) In section 347B (qualifying maintenance payments)—
%
%($a$) in subsections (8)  and (9) ($a$)  and ($c$), for “maintenance assessment” there shall be substituted “maintenance calculation”;
%
%($b$) in subsection (9) ($b$)  and ($c$), for “the assessment” there shall be substituted “the calculation”; and
%
%($c$) for subsection (11)  there shall be substituted—
%
%“(11) In this section “maintenance calculation” means a maintenance calculation made under the Child Support Act 1991 or a maintenance assessment made under the Child Support (Northern Ireland) Order 1991.”
%
%(3) In section 617 (social security benefits and contributions), in subsection (2)(ae), for “section 24 of the Child Support Act 1995 or under any corresponding enactment” there shall be substituted “any enactment corresponding to section 24 of the Child Support Act 1995”.
%The Finance Act 1988 (c. 39)
%
%9In the Finance Act 1988, in each of subsection (5A)  of section 36 (annual payments) and subsection (8A) of section 38 (maintenance payments under existing obligations: 1989-90 onwards), for “maintenance assessment made” there shall be substituted “maintenance calculation or maintenance assessment made respectively”.
%The Children Act 1989 (c. 41)
%
%10(1) Schedule 1 to the Children Act 1989 (financial provision for children) shall be amended as follows.
%
%(2) In paragraph 3—
%
%($a$) in sub-paragraph (5)($a$), (7)  (three times) and (8) ($a$), for “maintenance assessment” there shall be substituted “maintenance calculation”;
%
%($b$) in sub-paragraph (5)($a$)  and ($b$)(ii)  and (6) ($b$), for “current assessment” there shall be substituted “current calculation”;
%
%($c$) in sub-paragraph (6) ($b$), for “maintenance assessments” there shall be substituted “maintenance calculations”; and
%
%($d$) in sub-paragraph (6) ($b$), for “those assessments” there shall be substituted “those calculations”.
%
%(3) In paragraph 6—
%
%($a$) in sub-paragraph (9)  (three times), for “the assessment” there shall be substituted “the calculation”; and
%
%($b$) in sub-paragraph (9) ($c$), for “maintenance assessment” there shall be substituted “maintenance calculation”.
%
%(4) In paragraph 16(3), for “maintenance assessment” there shall be substituted “maintenance calculation”.
%The Child Support Act 1991 (c. 48)
%
%11(1) The 1991 Act shall be amended as follows.
%
%(2) For “absent parent” (or any variant of that expression), wherever it occurs, there shall be substituted “non-resident parent” (or the corresponding variant) preceded, where appropriate, by “a” instead of “an”.
%
%(3) In section 4 (child support maintenance)—
%
%($a$) in subsection (4)($a$), after “be” there shall be inserted “identified or”; and
%
%($b$) in subsection (9) , after “an application” there shall be inserted “treated as made”.
%
%(4) In section 7 (right of a child in Scotland to apply for assessment)—
%
%($a$) in subsection (1), for paragraph ($b$)  there shall be substituted—
%
%“($b$) no parent has been treated under section 6(3)  as having applied for a maintenance calculation with respect to the child.”; and
%
%($b$) in subsection (10) —
%
%(i) after “qualifying child if” there shall be inserted “($a$)”;
%
%(ii) after “maintenance order” there shall be inserted “made before a prescribed date”; and
%
%(iii) at the end there shall be inserted “or
%
%($b$) a maintenance order made on or after the date prescribed for the purposes of paragraph ($a$)  is in force in respect of them, but has been so for less than the period of one year beginning with the date on which it was made.”.
%
%(5) In section 8 (role of the courts with respect to maintenance for children)—
%
%($a$) in subsection (1), after “duly made” there shall be inserted “(or treated as made)”;
%
%($b$) in subsection (3), at the beginning insert “Except as provided in subsection (3A) ,”;
%
%($c$) for subsection (3A)  there shall be substituted—
%
%“(3A) Unless a maintenance calculation has been made with respect to the child concerned, subsection (3)  does not prevent a court from varying a maintenance order in relation to that child and the non-resident parent concerned—
%
%($a$) if the maintenance order was made on or after the date prescribed for the purposes of section 4(10) ($a$)  or 7(10) ($a$) ; or
%
%($b$) where the order was made before then, in any case in which section 4(10)  or 7(10)  prevents the making of an application for a maintenance calculation with respect to or by that child.”; and
%
%($d$) in subsection (6) , for paragraph ($b$)  there shall be substituted—
%
%“($b$) the non-resident parent’s net weekly income exceeds the figure referred to in paragraph 10(3)  of Schedule 1 (as it has effect from time to time pursuant to regulations made under paragraph 10A(1)($b$) ); and”.
%
%(6) In section 9 (agreements about maintenance), in subsection (6) , for paragraphs ($a$)  and ($b$)  there shall be substituted—
%
%“($a$) no parent has been treated under section 6(3)  as having applied for a maintenance calculation with respect to the child; or
%
%($b$) a parent has been so treated but no maintenance calculation has been made,”.
%
%(7) In section 14 (information required by Secretary of State), in subsection (1), after “any application” there shall be inserted “made or treated as made”.
%
%(8) In section 26 (disputes about parentage), in subsection (1), after “made” there shall be inserted “or treated as made”.
%
%(9) In section 27A (recovery of fees for scientific tests)—
%
%($a$) in subsection (1)($a$), after “made” there shall be inserted “or treated as made”; and
%
%($b$) in subsection (1)($b$), after “made” there shall be inserted “or, as the case may be, treated as made”.
%
%(10) In section 28 (power of the Secretary of State to bring or defend actions of declarator), in subsection (1)($a$) —
%
%($a$) after “made”, where it first occurs, there shall be inserted “or treated as made”; and
%
%($b$) for “or assessment was made” there shall be substituted “was made or treated as made or the calculation was made”.
%
%(11) In section 28ZA (decisions involving issues that arise on appeal in other cases), in subsection (1) —
%
%($a$) in paragraph ($a$), for the words “in relation to a maintenance assessment” there shall be substituted “or with respect to a reduced benefit decision under section 46”; and
%
%($b$) for paragraph ($b$)  there shall be substituted—
%
%“($b$) an appeal is pending against a decision given in relation to a different matter by a Child Support Commissioner or a court.”
%
%(12) In section 28ZB (appeals involving issues that arise on appeal in other cases)—
%
%($a$) in subsection (1), for paragraph ($a$)  there shall be substituted—
%
%“($a$) an appeal (“appeal A”) in relation to a decision or the imposition of a requirement falling within section 20(1)  is made to an appeal tribunal, or from an appeal tribunal to a Child Support Commissioner;”; and
%
%($b$) in subsection (4) , for the words “or assessment” there shall be substituted “or the imposition of the requirement”.
%
%(13) In section 28ZC (restrictions on liability in certain cases of error)—
%
%($a$) in subsection (1)($b$)(i) , at the end there shall be inserted “or one treated as having been so made, or under section 46 as to the reduction of benefit”;
%
%($b$) in subsection (1)($b$)(ii) , for the words from “a decision” to the end there shall be substituted “any decision (made after the commencement date) referred to in section 16(1A)”;
%
%($c$) in subsection (1)($b$)(iii) , for the words from “a decision” to the end there shall be substituted “any decision (made after the commencement date) referred to in section 17(1)”;
%
%($d$) in subsection (3), after “liability” there shall be inserted “or the reduction of a person’s benefit”; and
%
%($e$) in subsection (6) , in the definition of “adjudicating authority”, at the end there shall be inserted “or, in the case of a decision made on a referral under section 28D(1)($b$), an appeal tribunal”.
%
%(14) Sections 28H (departure directions: decisions and appeals) and 28I (transitional provisions relating to departure directions) shall cease to have effect.
%
%(15)In section 30 (collection and enforcement of certain forms of maintenance), for subsection (2)  there shall be substituted—
%
%“(2) The Secretary of State may, except in prescribed cases, arrange for the collection of any periodical payments, or secured periodical payments, of a prescribed kind which are payable for the benefit of a child even though he is not arranging for the collection of child support maintenance with respect to that child.”.
%
%(16)In section 32 (regulations about deduction from earnings orders), in subsection (2) , after paragraph ($b$)  there shall be inserted—
%
%“(bb)for the amount or amounts which are to be deducted from the liable person’s earnings not to exceed a prescribed proportion of his earnings (as determined by the employer);”.
%
%(17)In section 33 (liability orders), after subsection (5)  there shall be inserted—
%
%“(6) Where regulations have been made under section 29(3)($a$) —
%
%($a$) the liable person fails to make a payment (for the purposes of subsection (1)($a$)  of this section); and
%
%($b$) a payment is not paid (for the purposes of subsection (3) ),
%
%unless the payment is made to, or through, the person specified in or by virtue of those regulations for the case of the liable person in question.”
%
%(18)In section 47 (fees), after subsection (3)  there shall be inserted—
%
%“(4) The provisions of this Act with respect to—
%
%($a$) the collection of child support maintenance;
%
%($b$) the enforcement of any obligation to pay child support maintenance,
%
%shall apply equally (with any necessary modifications) to fees payable by virtue of regulations made under this section.”
%
%(19)In section 51 (supplementary power to make regulations), in subsection (2) —
%
%($a$) for paragraph ($a$)(ii)  and (iii)  there shall be substituted—
%
%“(ii) the making of decisions under section 11;
%
%(iii) the making of decisions under section 16 or 17;”; and
%
%($b$) for paragraph ($b$)  there shall be substituted—
%
%“($b$) extending the categories of case to which section 16, 17 or 20 applies;”.
%
%(20)In section 54 (interpretation)—
%
%($a$) in the definition of “application for a departure direction”, for “departure direction” there shall be substituted “variation”, and after “28A” there shall be inserted “or 28G”;
%
%($b$) after the definition of “deduction from earnings order” there shall be inserted—
%
%““default maintenance decision” has the meaning given in section 12;”;
%
%($c$) in the definition of “interim maintenance assessment”, for the word “assessment” there shall be substituted the word “decision”;
%
%($d$) for the definition of “maintenance assessment” there shall be substituted—
%
%““maintenance calculation” means a calculation of maintenance made under this Act and, except in prescribed circumstances, includes a default maintenance decision and an interim maintenance decision;”;
%
%($e$) the definitions of “assessable income”, “current assessment”, “departure direction” and “maintenance requirement” shall be omitted; and
%
%($f$) after the definition of “qualifying child” there shall be inserted—
%
%““voluntary payment” has the meaning given in section 28J.”.
%
%(21)In section 58 (short title, commencement and extent)—
%
%($a$) in subsection (9) , after “35” there shall be inserted “, 40”; and
%
%($b$) in subsection (10) , after “28” there shall be inserted “, 40A”.
%
%(22)In Schedule 1 (maintenance assessments)—
%
%($a$) paragraph 13 (which relates to assessments under which the amount payable is nil) shall cease to have effect;
%
%($b$) in paragraph 14 (which provides for consolidated applications and assessments), the existing text shall be sub-paragraph (1)  of that paragraph, and after that sub-paragraph there shall be inserted—
%
%“(2) In sub-paragraph (1), the references (however expressed) to applications for maintenance calculations include references to applications treated as made.”; and
%
%($c$) in paragraph 16 (which is about the termination of assessments)—
%
%(i) in sub-paragraph (1), paragraphs ($d$)  and ($e$)  shall cease to have effect,
%
%(ii) sub-paragraphs (2)  to (9)  shall cease to have effect; and
%
%(iii) in sub-paragraph (10) , the words “, or should be cancelled” shall cease to have effect.
%The Social Security Administration Act 1992 (c. 5)
%
%12In section 7A of the Social Security Administration Act 1992 (sharing of functions as regards certain claims and information), in subsection (6) ($a$) —
%
%($a$) after “application” there shall be inserted “(or an application treated as having been made)”; and
%
%($b$) for “maintenance assessment” there shall be substituted “maintenance calculation”.
%The Child Support Act 1995 (c. 34)
%
%13(1) The Child Support Act 1995 shall be amended as follows.
%
%(2) In section 18 (deferral of right to apply for maintenance assessment), subsection (5)  (which enables the Secretary of State by order to repeal any of the provisions of section 18) shall cease to have effect.
%
%(3) Section 24 (which provides for the making of regulations under which compensation could be paid for a reduction in child support maintenance attributable to changes in child support legislation, and which is now spent) shall cease to have effect.
%Prisoners' Earnings Act 1996 (c. 33)
%
%14In section 1 of the Prisoners' Earnings Act 1996 (power to make deductions and impose levies), in subsection (4) , in paragraph ($d$)  of the definition of “net weekly earnings”, for “maintenance assessment” there shall be substituted “maintenance calculation”.
%The Social Security Act 1998 (c. 14)
%
%15(1) The Social Security Act 1998 shall be amended as follows.
%
%(2) In Schedule 2 (decisions against which no appeal lies), for paragraph 8 and the heading preceding it there shall be substituted—
%“Reduction in accordance with reduced benefit decision
%
%8A decision to reduce the amount of a person’s benefit in accordance with a reduced benefit decision (within the meaning of section 46 of the Child Support Act).”.

%SCHEDULE 4Additional pension
%
%The Schedule to be inserted after Schedule 4 to the [1992 c. 4. ] Social Security Contributions and Benefits Act 1992 is as follows—
%“SCHEDULE 4AAdditional pension
%Part IThe amount
%
%1(1) The amount referred to in section 45(2)($c$)  above is to be calculated as follows—
%
%($a$) take for each tax year concerned the amount for the year which is found under the following provisions of this Schedule;
%
%($b$) add the amounts together;
%
%($c$) divide the sum of the amounts by the number of relevant years;
%
%($d$) the resulting amount is the amount referred to in section 45(2)($c$)  above, except that if the resulting amount is a negative one the amount so referred to is nil.
%
%(2) For the purpose of applying sub-paragraph (1)  above in the determination of the rate of any additional pension by virtue of section 39(1), 39C(1), 48A(4)  or 48B(2)  above, in a case where the deceased spouse died under pensionable age, the divisor used for the purposes of sub-paragraph (1)($c$)  above shall be whichever is the smaller of the alternative numbers referred to below (instead of the number of relevant years).
%
%(3) The first alternative number is the number of tax years which begin after 5th April 1978 and end before the date when the entitlement to the additional pension commences.
%
%(4) The second alternative number is the number of tax years in the period—
%
%($a$) beginning with the tax year in which the deceased spouse attained the age of 16 or, if later, 1978-79; and
%
%($b$) ending immediately before the tax year in which the deceased spouse would have attained pensionable age if he had not died earlier.
%
%(5) For the purpose of applying sub-paragraph (1)  above in the determination of the rate of any additional pension by virtue of section 48BB(5)  above, in a case where the deceased spouse died under pensionable age, the divisor used for the purposes of sub-paragraph (1)($c$)  above shall be whichever is the smaller of the alternative numbers referred to below (instead of the number of relevant years).
%
%(6) The first alternative number is the number of tax years which begin after 5th April 1978 and end before the date when the deceased spouse dies.
%
%(7) The second alternative number is the number of tax years in the period—
%
%($a$) beginning with the tax year in which the deceased spouse attained the age of 16 or, if later, 1978-79; and
%
%($b$) ending immediately before the tax year in which the deceased spouse would have attained pensionable age if he had not died earlier.
%
%(8) In this paragraph “relevant year” has the same meaning as in section 44 above.
%Part IISurplus earnings factor
%
%2(1) This Part of this Schedule applies if for the tax year concerned there is a surplus in the pensioner’s earnings factor.
%
%(2) The amount for the year is to be found as follows—
%
%($a$) calculate the part of the surplus for that year falling into each of the bands specified in the appropriate table below;
%
%($b$) multiply the amount of each such part in accordance with the last order under section 148 of the Administration Act to come into force before the end of the final relevant year;
%
%($c$) multiply each amount found under paragraph ($b$)  above by the percentage specified in the appropriate table in relation to the appropriate band;
%
%($d$) add together the amounts calculated under paragraph ($c$)  above.
%
%(3) The appropriate table for persons attaining pensionable age after the end of the first appointed year but before 6th April 2009 is as follows—
%Table 1
%Amount of surplus	Percentage
%Band 1. 	Not exceeding LET	40 + 2N
%Band 2. 	Exceeding LET but not exceeding $3LET - 2QEF$	10 + N/2
%Band 3. 	Exceeding 3LET - 2QEF	20 + N
%
%(4) The appropriate table for persons attaining pensionable age on or after 6th April 2009 is as follows—
%Table 2
%Amount of surplus	Percentage
%Band 1. 	Not exceeding LET	40
%Band 2. 	Exceeding LET but not exceeding 3LET - 2QEF	10
%Band 3. 	Exceeding 3LET - 2QEF	20
%
%(5) Regulations may provide, in relation to persons attaining pensionable age after such date as may be prescribed, that the amount found under this Part of this Schedule for the second appointed year or any subsequent tax year is to be calculated using only so much of the surplus in the pensioner’s earnings factor for that year as falls into Band 1 in the table in sub-paragraph (4)  above.
%
%(6) For the purposes of the tables in this paragraph—
%
%($a$) the value of N is 0. 5 for each tax year by which the tax year in which the pensioner attained pensionable age precedes 2009-10;
%
%($b$) “LET” means the low earnings threshold for that year as specified in section 44A above;
%
%($c$) “QEF” means the qualifying earnings factor for the tax year concerned.
%
%(7) In the calculation of “2QEF” the amount produced by doubling QEF shall be rounded to the nearest whole £100 (taking any amount of £50 as nearest to the previous whole £100).
%
%(8) In this paragraph “final relevant year” has the same meaning as in section 44 above.
%Part IIIContracted-out employment etc
%Introduction
%
%3(1) This Part of this Schedule applies if the following condition is satisfied in relation to each tax week in the tax year concerned.
%
%(2) The condition is that any earnings paid to or for the benefit of the pensioner in the tax week in respect of employment were in respect of employment qualifying him for a pension provided by a salary related contracted-out scheme or by a money purchase contracted-out scheme or by an appropriate personal pension scheme.
%
%(3) If the condition is satisfied in relation to one or more tax weeks in the tax year concerned, Part II of this Schedule does not apply in relation to the year.
%The amount
%
%4The amount for the year is amount C where—
%
%($a$) amount C is equal to amount A minus amount B, and
%
%($b$) amounts A and B are calculated as follows.
%Amount A
%
%5(1) Amount A is to be calculated as follows.
%
%(2) If there is an assumed surplus in the pensioner’s earnings factor for the year—
%
%($a$) calculate the part of the surplus for that year falling into each of the bands specified in the appropriate table below;
%
%($b$) multiply the amount of each such part in accordance with the last order under section 148 of the Administration Act to come into force before the end of the final relevant year;
%
%($c$) multiply each amount found under paragraph ($b$)  above by the percentage specified in the appropriate table in relation to the appropriate band;
%
%($d$) add together the amounts calculated under paragraph ($c$)  above.
%
%(3) The appropriate table for persons attaining pensionable age after the end of the first appointed year but before 6th April 2009 is as follows—
%Table 3
%Amount of surplus	Percentage
%Band 1. 	Not exceeding LET	40 + 2N
%Band 2. 	Exceeding LET but not exceeding 3LET - 2QEF	10 + N/2
%Band 3. 	Exceeding 3LET - 2QEF	20 + N
%
%(4) The appropriate table for persons attaining pensionable age on or after 6th April 2009 is as follows—
%Table 4
%Amount of surplus	Percentage
%Band 1. 	Not exceeding LET	40
%Band 2. 	Exceeding LET but not exceeding 3LET - 2QEF	10
%Band 3. 	Exceeding 3LET - 2QEF	20
%Amount B (first case)
%
%6(1) Amount B is to be calculated in accordance with this paragraph if the pensioner’s employment was entirely employment qualifying him for a pension provided by a salary related contracted-out scheme or by a money purchase contracted-out scheme.
%
%(2) If there is an assumed surplus in the pensioner’s earnings factor for the year—
%
%($a$) multiply the amount of the assumed surplus in accordance with the last order under section 148 of the Administration Act to come into force before the end of the final relevant year;
%
%($b$) multiply the amount found under paragraph ($a$)  above by the percentage specified in sub-paragraph (3)  below.
%
%(3) The percentage is—
%
%($a$) 20 + N if the person attained pensionable age after the end of the first appointed year but before 6th April 2009;
%
%($b$) 20 if the person attained pensionable age on or after 6th April 2009. 
%Amount B (second case)
%
%7(1) Amount B is to be calculated in accordance with this paragraph if the pensioner’s employment was entirely employment qualifying him for a pension provided by an appropriate personal pension scheme.
%
%(2) If there is an assumed surplus in the pensioner’s earnings factor for the year—
%
%($a$) calculate the part of the surplus for that year falling into each of the bands specified in the appropriate table below;
%
%($b$) multiply the amount of each such part in accordance with the last order under section 148 of the Administration Act to come into force before the end of the final relevant year;
%
%($c$) multiply each amount found under paragraph ($b$)  above by the percentage specified in the appropriate table in relation to the appropriate band;
%
%($d$) add together the amounts calculated under paragraph ($c$)  above.
%
%(3) The appropriate table for persons attaining pensionable age after the end of the first appointed year but before 6th April 2009 is as follows—
%Table 5
%Amount of surplus	Percentage
%Band 1. 	Not exceeding LET	40 + 2N
%Band 2. 	Exceeding LET but not exceeding 3LET - 2QEF	10 + N/2
%Band 3. 	Exceeding 3LET - 2QEF	20 + N
%
%(4) The appropriate table for persons attaining pensionable age on or after 6th April 2009 is as follows—
%Table 6
%Amount of surplus	Percentage
%Band 1. 	Not exceeding LET	40
%Band 2. 	Exceeding LET but not exceeding 3LET - 2QEF	10
%Band 3. 	Exceeding 3LET - 2QEF	20
%Interpretation
%
%8(1) In this Part of this Schedule “salary related contracted-out scheme”, “money purchase contracted-out scheme” and “appropriate personal pension scheme” have the same meanings as in the Pension Schemes Act 1993. 
%
%(2) For the purposes of this Part of this Schedule the assumed surplus in the pensioner’s earnings factor for the year is the surplus there would be in that factor for the year if section 48A(1)  of the Pension Schemes Act 1993 (no primary Class 1 contributions deemed to be paid) did not apply in relation to any tax week falling in the year.
%
%(3) Section 44A above shall be ignored in applying section 44(6)  above for the purpose of calculating amount B.
%
%(4) For the purposes of this Part of this Schedule—
%
%($a$) the value of N is 0. 5 for each tax year by which the tax year in which the pensioner attained pensionable age precedes 2009-10;
%
%($b$) “LET” means the low earnings threshold for that year as specified in section 44A above;
%
%($c$) “QEF” is the qualifying earnings factor for the tax year concerned.
%
%(5) In the calculation of “2QEF” the amount produced by doubling QEF shall be rounded to the nearest whole £100 (taking any amount of £50 as nearest to the previous whole £100).
%
%(6) In this Part of this Schedule “final relevant year” has the same meaning as in section 44 above.
%Part IVOther cases
%
%9The Secretary of State may make regulations containing provisions for finding the amount for a tax year in—
%
%($a$) cases where the circumstances relating to the pensioner change in the course of the year;
%
%($b$) such other cases as the Secretary of State thinks fit.”

\amendment{
Schs. 1--4 are not yet in force.
}

\part[Schedule 5 --- Pensions: miscellaneous amendments and alternative to anti-franking rules]{Schedule 5\\*Pensions: miscellaneous amendments and alternative to anti-franking rules}

\section[Part I --- Miscellaneous amendments]{Part I\\*Miscellaneous amendments}

\renewcommand\parthead{--- Schedule 5 Part I}

%Guaranteed minimum for widows and widowers
%
%1(1) In section 17 of the 1993 Act (guaranteed minimum for widow or widower), after subsection (4)  there shall be inserted—
%
%“(4A) The scheme must provide for the widow or widower’s pension to be payable to the widow or widower—
%
%($a$) for any period for which a Category B retirement pension is payable to the widow or widower by virtue of the earner’s contributions or would be so payable but for section 43(1)  of the [1992 c. 4. ] Social Security Contributions and Benefits Act 1992 (persons entitled to more than one retirement pension);
%
%($b$) for any period for which widowed parent’s allowance or bereavement allowance is payable to the widow or widower by virtue of the earner’s contributions; and
%
%($c$) in the case of a widow or widower whose entitlement by virtue of the earner’s contributions to a widowed parent’s allowance or bereavement allowance has come to an end at a time after the widow or widower attained the age of 45, for so much of the period beginning with the time when the entitlement came to an end as neither—
%
%(i) comprises a period during which the widow or widower and a person of the opposite sex are living together as husband and wife; nor
%
%(ii) falls after the time of any remarriage by the widow or widower.”
%
%(2) In subsection (5)  of that section—
%
%($a$) for “must provide” there shall be substituted “must also make provision”;
%
%($b$) the words “Category B retirement pension,”, in the first place where they occur, and the words from “or for which” onwards shall be omitted.
%
%(3) In subsection (6)  of that section, for “must provide” there shall be substituted “must also make provision”.
%Transfer of rights to overseas personal pension schemes
%
%2(1) In section 20(1)  of the 1993 Act (power to make provision for transfer of rights relating to guaranteed minimum pensions to an occupational or a personal pension scheme)—
%
%($a$) in paragraph ($a$), for “or to a personal pension scheme” there shall be substituted “, to a personal pension scheme or to an overseas arrangement”; and
%
%($b$) in paragraph ($b$), for “or a personal pension scheme” there shall be substituted “, a personal pension scheme or an overseas arrangement”.
%
%(2) In section 28(2)($b$)  of that Act (effect may be given to protected rights by a transfer to an occupational or personal pension scheme)—
%
%($a$) in sub-paragraph (i) , for “or to a personal pension scheme” there shall be substituted “, to a personal pension scheme or to an overseas arrangement”; and
%
%($b$) in sub-paragraph (ii) , for “or to an occupational pension scheme” there shall be substituted “, to an occupational pension scheme or to an overseas arrangement”.
%
%(3) In section 181(1)  of that Act (interpretation), there shall be inserted, at the appropriate place in the alphabetical order—
%
%““overseas arrangement” means a scheme or arrangement which—
%
%($a$) has effect, or is capable of having effect, so as to provide benefits on termination of employment or on death or retirement to or in respect of earners;
%
%($b$) is administered wholly or primarily outside Great Britain;
%
%($c$) is not an appropriate scheme; and
%
%($d$) is not an occupational pension scheme;”.
%Protected rights
%
%3(1) Section 28 of the 1993 Act (ways of giving effect to protected rights) shall be amended as follows.
%
%(2) In subsection (4)  (giving effect to protected rights at or after retirement age), for paragraph ($d$)  there shall be substituted—
%
%“($d$) the amount of the lump sum is equal to the value on that date of the protected rights to which effect is being given.”
%
%(3) After that subsection there shall be inserted—
%
%“(4A) Subject to subsection (4B), in the case of an occupational pension scheme, effect may be given to protected rights by the provision of a lump sum if—
%
%($a$) the trustees or managers of the scheme are satisfied that the member is terminally ill and likely to die within the period of twelve months beginning with the date on which the lump sum becomes payable; and
%
%($b$) the amount of the lump sum is equal to the value on that date of the protected rights to which effect is being given.
%
%(4B)The value of the protected rights to which effect may be given under subsection (4A)  in a case in which the member is a married person on the date on which the lump sum becomes payable shall not exceed one half of the value on that date of all the member’s protected rights.”
%
%(4) In subsections (3)  and (5) , for “or (4)”, in each case, there shall be substituted “, (4)  or (4A) ”.

\amendment{
Paras. 1--3 are not yet in force.
}

\subsection*{Review and alteration of rates of contribution}

4. In section 42(1)($a$)(i)  and (3)  of the 1993 Act (review of percentages mentioned in section 41), for “41(1A)($a$)  and ($b$)” there shall be substituted “41(1A) and (1B)”.

\subsection*{Contributions equivalent premiums: Great Britain}

5.---(1) For subsection (4)  of section 58 of the 1993 Act (calculation of contributions equivalent premiums) there shall be substituted—
\begin{quotation}
“(4) Subject to subsection (4A), the amount of the contributions equivalent premium shall be equal to the sum of the following amounts—
\begin{enumerate}\item[]
($a$) the amount of every reduction made under section 41 (as from time to time in force) in the amount of Class 1 contributions payable in respect of the earner’s employment in employment which was contracted-out by reference to the scheme; and

($b$) the total amount by which the reductions falling within paragraph ($a$)  would have been larger if the amount of the contributions falling to be reduced had in each case been at least equal to the amount of the reduction of those contributions provided for by section 41. 
\end{enumerate}

(4A) The amounts brought into account in accordance with subsection (4)($b$)  shall not include any amount which, by virtue of regulations made under section 41(1D) so as to avoid the payment of trivial or fractional amounts, is an amount that was not payable by the Inland Revenue to the secondary contributor.”
\end{quotation}

(2) In section 61(2)  of that Act (recovery of amount of premium attributable to primary Class 1 contributions), after “attributable to” there shall be inserted “any actual reductions of”.

(3) In section 63(1)  of that Act (amounts to be certified by the Inland Revenue), for paragraph ($b$)  there shall be substituted—
\begin{quotation}
“($b$) the sum of the amounts specified in section 58(4);”.
\end{quotation}

(4) This paragraph shall have effect, and be deemed to have had effect, in relation to any contributions equivalent premium payable on or after 6th April 1999. 

\subsection*{Contributions equivalent premiums: Northern Ireland}

6.---(1) For subsection (4)  of section 54 of the Pension Schemes (Northern Ireland) Act 1993 (calculation of contributions equivalent premiums) there shall be substituted—
\begin{quotation}
“(4) Subject to subsection (4A), the amount of the contributions equivalent premium shall be equal to the sum of the following amounts—
\begin{enumerate}\item[]
($a$) the amount of every reduction made under section 37 (as from time to time in force) in the amount of Class 1 contributions payable in respect of the earner’s employment in employment which was contracted-out by reference to the scheme; and

($b$) the total amount by which the reductions falling within paragraph ($a$)  would have been larger if the amount of the contributions falling to be reduced had in each case been at least equal to the amount of the reduction of those contributions provided for by section 37. 
\end{enumerate}

(4A) The amounts brought into account in accordance with subsection (4)($b$)  shall not include any amount which, by virtue of regulations made under section 37(1D) so as to avoid the payment of trivial or fractional amounts, is an amount that was not payable by the Inland Revenue to the secondary contributor.”
\end{quotation}

(2) In section 57(2)  of that Act (recovery of amount of premium attributable to primary Class 1 contributions), after “attributable to” there shall be inserted “any actual reductions of”.

(3) In section 59(1)  of that Act (amounts to be certified by the Inland Revenue), for paragraph ($b$)  there shall be substituted—
\begin{quotation}
“($b$) the sum of the amounts specified in section 54(4);”.
\end{quotation}

(4) This paragraph shall have effect, and be deemed to have had effect, in relation to any contributions equivalent premium payable on or after 6th April 1999. 

%Use of cash equivalent for annuity
%
%7Section 95(4)  of the 1993 Act (cash equivalent of rights under a money purchase contracted-out scheme not to be used for purchase of annuity) shall cease to have effect.

\amendment{
Para. 7 is not yet in force.
}

\subsection*{Transfer values where pension in payment}

8.---(1) In section 97(2)  of the 1993 Act (regulations about calculation of cash equivalents), for the “and” at the end of paragraph ($a$)  there shall be substituted—
\begin{quotation}
“($aa$) for a cash equivalent, including a guaranteed cash equivalent, to be reduced so as to take account of the extent (if any) to which an entitlement has arisen under the scheme to the present payment of the whole or any part of—
\begin{enumerate}\item[]
(i) any pension; or

(ii) any benefit in lieu of pension;
\end{enumerate}
and”.
\end{quotation}

%(2) In section 98(7)  of that Act (loss of right to cash equivalent)—
%
%($a$) after “right” there shall be inserted “if”; and
%
%($b$) paragraph ($a$)  (loss of right on the whole or any part of a pension becoming payable) shall cease to have effect.

(3) In section 124(1)  of the 1995 Act (interpretation), in the definition of “pensioner member”, after “other benefits” there shall be inserted “and who is not an active member of the scheme”.

(4) Sub-paragraph (2)  has effect in relation to any case in which the whole or any part of a pension or other benefit becomes payable on or after the coming into force of that sub-paragraph.

\amendment{
Para. 8(2) is not yet in force.

\medskip
%}
%
%Information about contracting-out
%
%9For section 156 of the 1993 Act (provision of information as to guaranteed minimum pensions) there shall be substituted—
%“156Information for purposes of contracting-out
%
%(1) The Secretary of State or the Inland Revenue may give to the trustees or managers of an occupational pension scheme or appropriate scheme such information as appears to the Secretary of State or Inland Revenue appropriate to give to them for the purpose of enabling them to comply with their obligations under Part III.
%
%(2) The Secretary of State or Inland Revenue may also give to such persons as may be prescribed any information that they could give under subsection (1)  to trustees or managers of a scheme.”
%Register of disqualified trustees
%
%10(1) In subsection (7)  of section 30 of the 1995 Act (disclosure of contents of register of disqualified trustees), for the words from “and” onwards there shall be substituted “but the arrangements made by the Authority for the register must secure that the contents of the register are not disclosed or otherwise made available to members of the public except in accordance with section 30A.”
%
%(2) After that subsection there shall be inserted—
%
%“(8) Nothing in subsection (7)  requires the Authority to exclude any matter from a report published under section 103.”
%
%(3) After that section there shall be inserted—
%“30AAccessibility of register of disqualified trustees
%
%(1) The Authority shall make arrangements that secure that the disqualification register is open, during the normal working hours of the Authority, for inspection in person and without notice at—
%
%($a$) the principal office used by them for the carrying out of their functions under this Act; and
%
%($b$) such other offices (if any) of theirs as they consider to be places where it would be reasonable for a copy of the register to be kept open for inspection.
%
%(2) If a request is made to the Authority—
%
%($a$) to state whether a particular person identified in the request is a person appearing in the disqualification register as disqualified in respect of a scheme specified in the request, or
%
%($b$) to state whether a particular person identified in the request is a person appearing in that register as disqualified in respect of all trust schemes,
%
%it shall be the duty of the Authority promptly to comply with the request in such manner as they consider reasonable.
%
%(3) The Authority may, in such manner as they think fit, publish a summary of the disqualification register if (subject to subsections (6)  to (8) ) the summary—
%
%($a$) contains all the information described in subsection (4);
%
%($b$) arranges that information in the manner described in subsection (5) ;
%
%($c$) does not (except by identifying a person as disqualified in respect of all trust schemes) identify any of the schemes in respect of which persons named in the summary are disqualified; and
%
%($d$) does not disclose any other information contained in the register.
%
%(4) That information is—
%
%($a$) the full names and titles, so far as the Authority have a record of them, of all the persons appearing in the register as persons who are disqualified;
%
%($b$) the dates of birth of such of those persons as are persons whose dates of birth are matters of which the Authority have a record; and
%
%($c$) in the case of each person whose name is included in the published summary, whether that person appears in the register—
%
%(i) as disqualified in respect of only one scheme;
%
%(ii) as disqualified in respect of two or more schemes but not in respect of all trust schemes; or
%
%(iii) as disqualified in respect of all trust schemes.
%
%(5) For the purposes of paragraph ($c$)  of subsection (4) , the information contained in the published summary must be arranged in three separate lists, one for each of the descriptions of disqualification specified in the three sub-paragraphs of that paragraph.
%
%(6) The Authority shall ensure, in the case of any published summary, that a person is not identified in the summary as a disqualified person if it appears to them that the determination by virtue of which that person appears in the register—
%
%($a$) is the subject of any pending review, appeal or legal proceedings which could result in that person’s removal from the register; or
%
%($b$) is a determination which might still become the subject of any such review, appeal or proceedings.
%
%(7) The Authority shall ensure, in the case of any published summary, that the particulars relating to a person do not appear in a particular list mentioned in subsection (5)  if it appears to them that a determination by virtue of which that person’s particulars would appear in that list—
%
%($a$) is the subject of any pending review, appeal or legal proceedings which could result in such a revocation or other overturning of a disqualification of that person as would require his particulars to appear in a different list; or
%
%($b$) is a determination which might still become the subject of any such review, appeal or proceedings.
%
%(8) Where subsection (7)  prevents a person’s particulars from being included in a particular list in the published summary, they shall be included, instead, in the list in which they would have been included if the disqualification to which the review, appeal or proceedings relate had already been revoked or otherwise overturned.
%
%(9) For the purposes of this section a determination is one which might still become the subject of a review, appeal or proceedings if, and only if, in the case of that determination—
%
%($a$) the time for the making of an application for a review, or for the bringing of an appeal or other proceedings, has not expired; and
%
%($b$) there is a reasonable likelihood that such an application might yet be made, or that such an appeal or such proceedings might yet be brought.
%
%(10) In this section—
%
%    “the disqualification register” means the register kept by the Authority under section 30(7) ; and
%
%    “name”, in relation to a person any of whose names is recorded by the Authority as an initial, means that initial.” 
%
%Conditions of payment of surplus to employer
%
%11(1) Section 37 of the 1995 Act (payment of surplus to employer) shall be amended as follows.
%
%(2) For paragraph ($d$)  of subsection (4)  (conditions of payment of surplus) there shall be substituted—
%
%“($d$) the annual rates of the pensions under the scheme are increased, at intervals of not more than twelve months, by at least the relevant percentage,”.
%
%(3) After subsection (5)  there shall be inserted—
%
%“(5A) For the purposes of subsection (4)($d$), the relevant percentage is the percentage which, for the purposes of the increases of the annual rates of the pensions under the scheme—
%
%($a$) falls to be computed by reference to a period which, except in the case of the first increase—
%
%(i) begins with the end of the period by reference to which the last preceding increase was made; and
%
%(ii) ends with a date which falls after the date of the last preceding increase;
%
%and
%
%($b$) is equal to whichever is the lesser of—
%
%(i) the percentage increase in the retail prices index over the period by reference to which the increase is made; and
%
%(ii) the equivalent over that period of 5 per cent. per annum.”
%
%(4) In subsection (6)  (interpretation of section), for the words from the beginning to the end of paragraph ($a$)  there shall be substituted—
%
%“(6) In this section—
%
%($a$) “annual rate” has the same meaning as in section 54, and”.
%
%(5) The preceding provisions of this paragraph have effect in relation to payments made to an employer at any time after the commencement of this paragraph.
%Duties relating to statements of contributions
%
%12(1) In section 41 of the 1995 Act (provision of documents for members), for subsection (5)  there shall be substituted—
%
%“(5) Regulations may in the case of occupational pension schemes provide for—
%
%($a$) prescribed persons,
%
%($b$) persons with prescribed qualifications or experience, or
%
%($c$) persons approved by the Secretary of State,
%
%to act for the purposes of subsection (2)  instead of scheme auditors or actuaries.
%
%(5A) Regulations may impose duties on the trustees or managers of an occupational pension scheme to disclose information to, and make documents available to, a person acting under subsection (5).
%
%(5B)If any duty imposed under subsection (5A)  is not complied with, sections 3 and 10 apply to any trustee, and section 10 applies to any manager, who has failed to take all such steps as are reasonable to secure compliance.”
%
%(2) In section 49 of that Act, in subsection (9)  (duties in event of employer’s failure to pay contributions in prescribed period), after paragraph ($b$)  there shall be inserted “; and
%
%($c$) except in prescribed circumstances, any person acting instead of an auditor for the purposes of section 41(2)($b$)  in relation to the scheme must give notice of the failure, within the prescribed period, to the Authority.”
%
%(3) In that section, there shall be inserted after subsection (10) —
%
%“(10A)Section 10 applies to a person who fails to comply with subsection (9) ($c$) .”
%
%(4) In section 88 of that Act (payment schedule to money purchase schemes: supplementary), after subsection (4)  there shall be inserted—
%
%“(5) Except in prescribed circumstances, any person acting instead of an auditor for the purposes of section 41(2)($b$)  in relation to an occupational pension scheme to which section 87 applies must, where any amounts payable in accordance with the payment schedule have not been paid on or before the due date, give notice of that fact, within the prescribed period, to the Authority.
%
%(6) Section 10 applies to a person so acting who fails to comply with subsection (5).”
%
%\amendment{
Paras. 9--12 are not yet in force.
}

\subsection*{Interpretation of Part I}

13. In this Part of this Schedule—
\begin{enumerate}\item[]
    “the 1993 Act” means the Pension Schemes Act 1993; and

    “the 1995 Act” means the Pensions Act 1995.  
\end{enumerate}

%Part IIAlternative to anti-franking rules
%Cases in which alternative applies
%
%14(1) Subject to the following provisions of this paragraph, this Part of this Schedule applies, instead of Chapter III of Part IV of the 1993 Act (anti-franking rules), in the case of a person (“the pensioner”) who is entitled to benefits under any occupational pension scheme if the benefits to which he is entitled under the scheme include a guaranteed minimum pension.
%
%(2) This Part of this Schedule does not apply in the pensioner’s case, instead of Chapter III of Part IV of the 1993 Act, unless—
%
%($a$) the pensioner is a member of the scheme who, in relation to that scheme, left pensionable service after the coming into force of this Part of this Schedule;
%
%($b$) the pensioner is the widow or widower of a member of the scheme whose pensionable service ended (by death or otherwise) after the coming into force of this Part of this Schedule; or
%
%($c$) sub-paragraph (3)  applies to the benefits to which the pensioner is entitled under the scheme.
%
%(3) This sub-paragraph applies to the benefits to which the pensioner is entitled under the scheme if—
%
%($a$) the time at which the benefits first become payable is after the coming into force of this Part of this Schedule;
%
%($b$) the benefits do not first become payable in respect of the death of a member of the scheme to whom benefits had already become payable under the scheme before the coming into force of this Part of this Schedule; and
%
%($c$) the trustees or managers of the scheme have elected, in the prescribed manner, that this Part of this Schedule should apply to benefits first becoming payable under the scheme after the coming into force of this Part of this Schedule.
%
%(4) This Part of this Schedule does not apply in the pensioner’s case (and, accordingly, Chapter III of Part IV of the 1993 Act does) if the scheme is a scheme of a prescribed description, unless the trustees or managers of the scheme have elected, in the prescribed manner, that this Part of this Schedule should apply in the case of the scheme.
%
%(5) An election for the purposes of any provision of this paragraph—
%
%($a$) shall not be exercisable differently in relation to different members of the scheme; and
%
%($b$) once exercised, shall be irrevocable.
%Alternative rules
%
%15(1) Where this Part of this Schedule applies in the pensioner’s case, the amount of the benefits to which he is entitled under the scheme shall not be less than the amount of the benefits to which he would have been entitled under the scheme if his entitlement fell to be calculated by the method set out in sub-paragraph (2).
%
%(2) That method is as follows—
%
%    Step 1: compute the amount of any benefits consisting in the guaranteed minimum pension to which the pensioner is entitled;
%
%    Step 2: compute what would have been the amount of those benefits on the assumptions set out in sub-paragraph (3) ;
%
%    Step 3: determine the extent (if any) to which attributing an amount of benefits equal to the amount computed in accordance with Step 2 to rights accruing before 6th April 1997 would leave any such rights unused;
%
%    Step 4: compute, in accordance with sub-paragraph (4) , the amount of such of the benefits to which the pensioner is entitled under the scheme as are attributable to rights accruing before 6th April 1997 (if any) which, applying the determination in Step 3, would be left unused after the attribution of the amount mentioned in that Step to rights so accruing;
%
%    Step 5: compute the amount resulting, on the required assumption, from the application of the statutory revaluations and increases in the case of the benefits computed in accordance with Step 4;
%
%    Step 6: compute, in accordance with sub-paragraph (4) , the amount of such of the benefits to which the pensioner is entitled under the scheme as are attributable to rights accruing on or after 6th April 1997;
%
%    Step 7: compute the amount resulting, on the required assumption, from the application of the statutory revaluations and increases in the case of the benefits computed in accordance with Step 6;
%
%    Step 8: aggregate the results of Steps 1, 5 and 7 to give the minimum benefits required by sub-paragraph (1) . 
%
%(3) The assumptions referred to in Step 2 in sub-paragraph (2)  are—
%
%($a$) that no increases are required to be made in accordance with section 15 or 109 of the 1993 Act (deferment increases and indexation);
%
%($b$) that increases in accordance with section 16(1)  of that Act (revaluation of earnings factors for early leavers) of any earner’s earnings factors are to be calculated as if references to the final relevant year were references to whichever is the earlier of—
%
%(i) the final relevant tax year; and
%
%(ii) the tax year immediately preceding that in which the member in question left service that qualified him for salary-related benefits under the scheme; and
%
%($c$) that no increases in accordance with any provision included in the scheme by virtue of section 16(3)  of that Act (increases of weekly equivalent for person leaving contracted-out service before final relevant year) are to be made for any year after the tax year immediately preceding that in which the member in question left service that qualified him for salary-related benefits under the scheme.
%
%(4) For the purposes of Steps 4 and 6 in sub-paragraph (2) —
%
%($a$) if (apart from this sub-paragraph) there would be a difference between the two Steps in the level of salary taken as the level by reference to which any salary-related benefits are to be computed, the level used for Step 4 must be no lower than that used for Step 6; and
%
%($b$) statutory revaluations and increases shall not be attributed to rights accruing at any time.
%
%(5) For the purposes of Steps 5 and 7 in sub-paragraph (2) , the required assumption is that the benefits in whose case the statutory revaluations and increases are applied comprise a whole pension deriving from the rights to which they are taken to be attributable for the purposes of Step 4 or, as the case may be, Step 6. 
%
%(6) Subject to sub-paragraph (7), references in this paragraph to the statutory revaluations and increases are references to—
%
%($a$) the revaluations required to be made in accordance with Chapter II of Part IV of the 1993 Act (revaluation of accrued benefits); and
%
%($b$) the increases required to be made by virtue of section 51 of the 1995 Act (indexation).
%
%(7) For the purpose of applying the statutory revaluations and increases for the purposes of Steps 5 and 7 in sub-paragraph (2) —
%
%($a$) money may be used in a way allowed by section 110(1)  of the 1993 Act (use of money to pay guaranteed minimum pension increase for subsequent year); and
%
%($b$) any deductions authorised by section 53(1)  or (2)  of the 1995 Act (permitted deductions from statutory increases) may be made.
%
%(8) In this paragraph “the pensioner” has the meaning given by paragraph 14. 
%
%(9) Any reference in this paragraph to a provision of the 1993 Act includes a reference to any enactment re-enacted in that provision.
%Relationship between alternative rules and other rules
%
%16(1) Paragraph 15 shall not apply to benefits consisting in an alternative to a short service benefit provided for under section 73(2)($b$)  of the 1993 Act, except to the extent that—
%
%($a$) that paragraph would apply for the computation of the short service benefit to which those benefits are an alternative; and
%
%($b$) the amount of any of the alternative benefits falls to be computed wholly or partly by reference to the value of what would have been the short service benefit.
%
%(2) Section 94 of the 1993 Act (right to cash equivalent) shall have effect as if the provisions of this Part of this Schedule were included for the purposes of that section in the applicable rules.
%
%(3) Subject to sub-paragraph (4) , the preceding provisions of this Part of this Schedule override any provision of an occupational pension scheme with which they are inconsistent except a provision which, under subsection (3)  of section 129 of the 1993 Act, is a protected provision for the purposes of subsection (2)  of that section.
%
%(4) The preceding provisions of this Part of this Schedule shall be without prejudice to any person’s entitlement to exercise—
%
%($a$) any right of commutation, forfeiture or surrender of the whole or any part of any benefits computed in accordance with this Part of this Schedule;
%
%($b$) any charge or lien on the whole or any part of any such benefits; or
%
%($c$) any right of set-off against the whole or any part of any such benefits;
%
%and, accordingly, the computations to be done under paragraph 15 shall be done disregarding anything falling within any of paragraphs ($a$)  to ($c$) .
%Supplemental
%
%17(1) In this Part of this Schedule references to rights accruing to a member of a scheme before 6th April 1997 include references—
%
%($a$) in relation to salary-related benefits, to rights accruing at any time in respect of service before that date; and
%
%($b$) in relation to benefits of any description, to rights that derive from any transfer of accrued rights or transfer payment and represent rights accruing under any other scheme before that date;
%
%and a reference in this Part of this Schedule to rights accruing on or after that date shall be construed accordingly.
%
%(2) For the purposes of this Part of this Schedule rights to money purchase benefits that are attributable to payments in respect of employment are rights accruing before 6th April 1997 in so far only as that employment was employment carried on before that date; and a reference in this Part of this Schedule to rights accruing on or after that date shall be construed accordingly.
%
%(3) In this Part of this Schedule—
%
%    “the 1993 Act” means the [1993 c. 48. ] Pension Schemes Act 1993;
%
%    “the 1995 Act” means the [1995 c. 26. ] Pensions Act 1995; and
%
%    “salary-related benefits” means benefits that are not money purchase benefits. 
%
%(4) Expressions defined for the purposes of the 1993 Act have the same meanings in this Part of this Schedule as they have in that Act.
%
%(5) Any power of the Secretary of State to make regulations under this Part of this Schedule shall be exercisable by statutory instrument subject to annulment in pursuance of a resolution of either House of Parliament.
%
%(6) The Secretary of State may by order make such modifications of paragraphs 14 to 16 as he considers appropriate.
%
%(7) An order under sub-paragraph (6)  shall be made by statutory instrument subject to annulment in pursuance of a resolution of either House of Parliament.
%
%(8) Subsections (2)  to (5)  of section 182 of the 1993 Act (supplemental provision in connection with powers to make subordinate legislation under that Act) shall apply—
%
%($a$) to any power of the Secretary of State to make regulations under this Part of this Schedule, and
%
%($b$) to the power of the Secretary of State to make an order under sub-paragraph (6) ,
%
%as they apply to his powers to make regulations and orders under that Act.
%
%(9) In section 178($a$)  of the 1993 Act (regulations providing for who is to be treated as a manager of a scheme), for the words from “or Part III” to “1999” there shall be substituted “, Part III or IV of the [1999 c. 30. ] Welfare Reform and Pensions Act 1999 or Part II of Schedule 5 to the [1999 c. 30. ] Child Support, Pensions and Social Security Act 2000”. 

\amendment{
Sch. 5 Pt. II and Schs. 6--8 are not yet in force.
}

%SCHEDULE 6Social security investigation powers
%Preliminary
%
%1Part VI of the [1992 c. 5. ] Social Security Administration Act 1992 (enforcement) shall be amended as follows.
%Replacement for inspector’s powers
%
%2The following sections shall be substituted for section 110 (appointment and powers of inspectors)—
%“109AAuthorisations for investigators
%
%(1) An individual who for the time being has the Secretary of State’s authorisation for the purposes of this Part shall be entitled, for any one or more of the purposes mentioned in subsection (2)  below, to exercise any of the powers which are conferred on an authorised officer by sections 109B and 109C below.
%
%(2) Those purposes are—
%
%($a$) ascertaining in relation to any case whether a benefit is or was payable in that case in accordance with any provision of the relevant social security legislation;
%
%($b$) investigating the circumstances in which any accident, injury or disease which has given rise, or may give rise, to a claim for—
%
%(i) industrial injuries benefit, or
%
%(ii) any benefit under any provision of the relevant social security legislation,
%
%occurred or may have occurred, or was or may have been received or contracted;
%
%($c$) ascertaining whether provisions of the relevant social security legislation are being, have been or are likely to be contravened (whether by particular persons or more generally);
%
%($d$) preventing, detecting and securing evidence of the commission (whether by particular persons or more generally) of benefit offences.
%
%(3) An individual has the Secretary of State’s authorisation for the purposes of this Part if, and only if, the Secretary of State has granted him an authorisation for those purposes and he is—
%
%($a$) an official of a Government department;
%
%($b$) an individual employed by an authority administering housing benefit or council tax benefit;
%
%($c$) an individual employed by an authority or joint committee that carries out functions relating to housing benefit or council tax benefit on behalf of the authority administering that benefit; or
%
%($d$) an individual employed by a person authorised by or on behalf of any such authority or joint committee as is mentioned in paragraph ($b$)  or ($c$)  above to carry out functions relating to housing benefit or council tax benefit for that authority or committee.
%
%(4) An authorisation granted for the purposes of this Part to an individual of any of the descriptions mentioned in subsection (3)  above—
%
%($a$) must be contained in a certificate provided to that individual as evidence of his entitlement to exercise powers conferred by this Part;
%
%($b$) may contain provision as to the period for which the authorisation is to have effect; and
%
%($c$) may restrict the powers exercisable by virtue of the authorisation so as to prohibit their exercise except for particular purposes, in particular circumstances or in relation to particular benefits or particular provisions of the relevant social security legislation.
%
%(5) An authorisation granted under this section may be withdrawn at any time by the Secretary of State.
%
%(6) Where the Secretary of State grants an authorisation for the purposes of this Part to an individual employed by a local authority, or to an individual employed by a person who carries out functions relating to housing benefit or council tax benefit on behalf of a local authority—
%
%($a$) the Secretary of State and the local authority shall enter into such arrangements (if any) as they consider appropriate with respect to the carrying out of functions conferred on that individual by or in connection with the authorisation granted to him; and
%
%($b$) the Secretary of State may make to the local authority such payments (if any) as he thinks fit in respect of the carrying out by that individual of any such functions.
%
%(7) The matters on which a person may be authorised to consider and report to the Secretary of State under section 139A below shall be taken to include the carrying out by any such individual as is mentioned in subsection (3)($b$)  to ($d$)  above of any functions conferred on that individual by virtue of any grant by the Secretary of State of an authorisation for the purposes of this Part.
%
%(8) The powers conferred by sections 109B and 109C below shall be exercisable in relation to persons holding office under the Crown and persons in the service of the Crown, and in relation to premises owned or occupied by the Crown, as they are exercisable in relation to other persons and premises.
%109BPower to require information
%
%(1) An authorised officer who has reasonable grounds for suspecting that a person—
%
%($a$) is a person falling within subsection (2)  below, and
%
%($b$) has or may have possession of or access to any information about any matter that is relevant for any one or more of the purposes mentioned in section 109A(2)  above,
%
%may, by written notice, require that person to provide all such information described in the notice as is information of which he has possession, or to which he has access, and which it is reasonable for the authorised officer to require for a purpose so mentioned.
%
%(2) The persons who fall within this subsection are—
%
%($a$) any person who is or has been an employer or employee within the meaning of any provision made by or under the Contributions and Benefits Act;
%
%($b$) any person who is or has been a self-employed earner within the meaning of any such provision;
%
%($c$) any person who by virtue of any provision made by or under that Act falls, or has fallen, to be treated for the purposes of any such provision as a person within paragraph ($a$)  or ($b$)  above;
%
%($d$) any person who is carrying on, or has carried on, any business involving the supply of goods for sale to the ultimate consumers by individuals not carrying on retail businesses from retail premises;
%
%($e$) any person who is carrying on, or has carried on, any business involving the supply of goods or services by the use of work done or services performed by persons other than employees of his;
%
%($f$) any person who is carrying on, or has carried on, an agency or other business for the introduction or supply, to persons requiring them, of persons available to do work or to perform services;
%
%($g$) any local authority acting in their capacity as an authority responsible for the granting of any licence;
%
%(h)any person who is or has been a trustee or manager of a personal or occupational pension scheme;
%
%(i) any person who is or has been liable to make a compensation payment or a payment to the Secretary of State under section 6 of the [1997 c. 27. ] Social Security (Recovery of Benefits) Act 1997 (payments in respect of recoverable benefits); and
%
%($j$) the servants and agents of any such person as is specified in any of paragraphs ($a$)  to (i)  above.
%
%(3) The obligation of a person to provide information in accordance with a notice under this section shall be discharged only by the provision of that information, at such reasonable time and in such form as may be specified in the notice, to the authorised officer who—
%
%($a$) is identified by or in accordance with the terms of the notice; or
%
%($b$) has been identified, since the giving of the notice, by a further written notice given by the authorised officer who imposed the original requirement or another authorised officer.
%
%(4) The power of an authorised officer under this section to require the provision of information shall include a power to require the production and delivery up and (if necessary) creation of, or of copies of or extracts from, any such documents containing the information as may be specified or described in the notice imposing the requirement.
%
%(5) No one shall be required under this section to provide any information (whether in documentary form or otherwise) that tends to incriminate either himself or, in the case of a person who is married, his spouse.
%109CPowers of entry
%
%(1) An authorised officer shall be entitled, at any reasonable time and either alone or accompanied by such other persons as he thinks fit, to enter any premises which—
%
%($a$) are liable to inspection under this section; and
%
%($b$) are premises to which it is reasonable for him to require entry in order to exercise the powers conferred by this section.
%
%(2) An authorised officer who has entered any premises liable to inspection under this section may—
%
%($a$) make such an examination of those premises, and
%
%($b$) conduct any such inquiry there,
%
%as appears to him appropriate for any one or more of the purposes mentioned in section 109A(2)  above.
%
%(3) An authorised officer who has entered any premises liable to inspection under this section may—
%
%($a$) question any person whom he finds there;
%
%($b$) require any person whom he finds there to do any one or more of the following—
%
%(i) to provide him with such information,
%
%(ii) to produce and deliver up and (if necessary) create such documents or such copies of, or extracts from, documents,
%
%as he may reasonably require for any one or more of the purposes mentioned in section 109A(2)  above; and
%
%($c$) take possession of and either remove or make his own copies of any such documents as appear to him to contain information that is relevant for any of those purposes.
%
%(4) The premises liable to inspection under this section are any premises (including premises consisting in the whole or a part of a dwelling house) which an authorised officer has reasonable grounds for suspecting are—
%
%($a$) premises which are a person’s place of employment;
%
%($b$) premises from which a trade or business is being carried on or where documents relating to a trade or business are kept by the person carrying it on or by another person on his behalf;
%
%($c$) premises from which a personal or occupational pension scheme is being administered or where documents relating to the administration of such a scheme are kept by the person administering the scheme or by another person on his behalf;
%
%($d$) premises where a person who is the compensator in relation to any such accident, injury or disease as is referred to in section 109A(2)($b$)  above is to be found;
%
%($e$) premises where a person on whose behalf any such compensator has made, may have made or may make a compensation payment is to be found.
%
%(5) An authorised officer applying for admission to any premises in accordance with this section shall, if required to do so, produce the certificate containing his authorisation for the purposes of this Part.
%
%(6) Subsection (5)  of section 109B applies for the purposes of this section as it applies for the purposes of that section.”
%Exercise of powers on behalf of local authorities
%
%3For sections 110A and 110B (inspectors appointed by local authorities etc. for the purposes of housing benefit or council tax benefit), there shall be substituted—
%“110AAuthorisations by local authorities
%
%(1) An individual who for the time being has the authorisation for the purposes of this Part of an authority administering housing benefit or council tax benefit (“a local authority authorisation”) shall be entitled, for any one or more of the purposes mentioned in subsection (2)  below, to exercise any of the powers which, subject to subsection (8)  below, are conferred on an authorised officer by sections 109B and 109C above.
%
%(2) Those purposes are—
%
%($a$) ascertaining in relation to any case whether housing benefit or council tax benefit is or was payable in that case;
%
%($b$) ascertaining whether provisions of the relevant social security legislation that relate to housing benefit or council tax benefit are being, have been or are likely to be contravened (whether by particular persons or more generally);
%
%($c$) preventing, detecting and securing evidence of the commission (whether by particular persons or more generally) of benefit offences relating to housing benefit or council tax benefit.
%
%(3) An individual has the authorisation for the purposes of this Part of an authority administering housing benefit or council tax benefit if, and only if, that authority have granted him an authorisation for those purposes and he is—
%
%($a$) an individual employed by that authority;
%
%($b$) an individual employed by another authority or joint committee that carries out functions relating to housing benefit or council tax benefit on behalf of that authority;
%
%($c$) an individual employed by a person authorised by or on behalf of—
%
%(i) the authority in question,
%
%(ii) any such authority or joint committee as is mentioned in paragraph ($b$)  above,
%
%to carry out functions relating to housing benefit or council tax benefit for that authority or committee;
%
%($d$) an official of a Government department.
%
%(4) Subsection (4)  of section 109A above shall apply in relation to a local authority authorisation as it applies in relation to an authorisation under that section.
%
%(5) A local authority authorisation may be withdrawn at any time by the authority that granted it or by the Secretary of State.
%
%(6) The certificate or other instrument containing the grant or withdrawal by any local authority of any local authority authorisation must be issued under the hand of either—
%
%($a$) the officer designated under section 4 of the [1989 c. 42. ] Local Government and Housing Act 1989 as the head of the authority’s paid service; or
%
%($b$) the officer who is the authority’s chief finance officer (within the meaning of section 5 of that Act).
%
%(7) It shall be the duty of any authority with power to grant local authority authorisations to comply with any directions of the Secretary of State as to—
%
%($a$) whether or not such authorisations are to be granted by that authority;
%
%($b$) the period for which authorisations granted by that authority are to have effect;
%
%($c$) the number of persons who may be granted authorisations by that authority at any one time; and
%
%($d$) the restrictions to be contained by virtue of subsection (4)  above in the authorisations granted by that authority for those purposes.
%
%(8) The powers conferred by sections 109B and 109C above shall have effect in the case of an individual who is an authorised officer by virtue of this section as if those sections had effect—
%
%($a$) with the substitution for every reference to the purposes mentioned in section 109A(2)  above of a reference to the purposes mentioned in subsection (2)  above; and
%
%($b$) with the substitution for every reference to the relevant social security legislation of a reference to so much of it as relates to housing benefit or council tax benefit.
%
%(9) Nothing in this section conferring any power on an authorised officer in relation to housing benefit or council tax benefit shall require that power to be exercised only in relation to cases in which the authority administering the benefit is the authority by whom that officer’s authorisation was granted.”
%Consequential amendments
%
%4In section 111 (delay and obstruction of inspector)—
%
%($a$) in subsection (3), for “section 110(5)” there shall be substituted “an authorisation granted under section 109A or 110A”; and
%
%($b$) in subsection (4) —
%
%(i) for “section 110(5)  above any power conferred by section 110 above” there shall be substituted “an authorisation granted under section 109A or 110A above, any power conferred by section 109B or 109C above”; and
%
%(ii) for the words “section 110”, where they occur at the end of the subsection, there shall be substituted “sections 109B and 109C”.
%
%5In section 111A(1)  (dishonest representations), before “social security legislation” there shall be inserted “relevant”.
%
%6In section 112(1)  (false representations), before “social security legislation” there shall be inserted “relevant”.
%
%7(1) In subsection (1)  of section 113 (breach of regulations)—
%
%($a$) for “Acts to which section 110 above applies” there shall be substituted “legislation to which this section applies”;
%
%($b$) for the words “that Act”, in the first place where they occur, there shall be substituted “that legislation”; and
%
%($c$) for the words “that Act”, where they occur in paragraph ($b$), there shall be substituted “any enactment contained in the legislation in question”.
%
%(2) After that subsection there shall be inserted—
%
%“(1A)The legislation to which this section applies is—
%
%($a$) the relevant social security legislation; and
%
%($b$) the enactments specified in section 121DA(1)  so far as relating to contributions, statutory sick pay or statutory maternity pay.”
%
%8After section 121D (but still in Part VI) there shall be inserted—
%“121DAInterpretation of Part VI
%
%(1) In this Part “the relevant social security legislation” means the provisions of any of the following, except so far as relating to contributions, working families' tax credit, disabled person’s tax credit, statutory sick pay or statutory maternity pay, that is to say—
%
%($a$) the Contributions and Benefits Act;
%
%($b$) this Act;
%
%($c$) the Pensions Act, except Part III;
%
%($d$) section 4 of the [1994 c. 18. ] Social Security (Incapacity for Work) Act 1994;
%
%($e$) the [1995 c. 18. ] Jobseekers Act 1995;
%
%($f$) the [1997 c. 27. ] Social Security (Recovery of Benefits) Act 1997;
%
%($g$) Parts I and IV of the [1998 c. 14. ] Social Security Act 1998;
%
%(h)Part V of the [1999 c. 30. ] Welfare Reform and Pensions Act 1999;
%
%(i) the [1975 c. 60. ] Social Security Pensions Act 1975;
%
%($j$) the [1973 c. 38. ] Social Security Act 1973;
%
%($k$) any subordinate legislation made, or having effect as if made, under any enactment specified in paragraphs ($a$)  to ($j$)  above.
%
%(2) In this Part “authorised officer” means a person acting in accordance with any authorisation for the purposes of this Part which is for the time being in force in relation to him.
%
%(3) For the purposes of this Part—
%
%($a$) references to a document include references to anything in which information is recorded in electronic or any other form;
%
%($b$) the requirement that a notice given by an authorised officer be in writing shall be taken to be satisfied in any case where the contents of the notice—
%
%(i) are transmitted to the recipient of the notice by electronic means; and
%
%(ii) are received by him in a form that is legible and capable of being recorded for future reference.
%
%(4) In this Part “premises” includes—
%
%($a$) moveable structures and vehicles, vessels, aircraft and hovercraft;
%
%($b$) installations that are offshore installations for the purposes of the [1971 c. 61. ] Mineral Workings (Offshore Installations) Act 1971; and
%
%($c$) places of all other descriptions whether or not occupied as land or otherwise;
%
%and references in this Part to the occupier of any premises shall be construed, in relation to premises that are not occupied as land, as references to any person for the time being present at the place in question.
%
%(5) In this Part—
%
%    “benefit” includes any allowance, payment, credit or loan;
%
%    “benefit offence” means a criminal offence committed in connection with a claim for benefit under a provision of the relevant social security legislation, or in connection with the receipt or payment of such a benefit; and
%
%    “compensation payment” has the same meaning as in the [1997 c. 27. ] Social Security (Recovery of Benefits) Act 1997.  
%
%(6) In this Part—
%
%($a$) any reference to a person authorised to carry out any function relating to housing benefit or council tax benefit shall include a reference to a person providing services relating to the benefit directly or indirectly to an authority administering it; and
%
%($b$) any reference to the carrying out of a function relating to such a benefit shall include a reference to the provision of any services relating to it.
%
%(7) In this section “subordinate legislation” has the same meaning as in the [1978 c. 30. ] Interpretation Act 1978.”
%
%9In paragraph 5 of Schedule 10 to the [1992 c. 5. ] Social Security Administration Act 1992 (transitional provisions for supplementary benefit), for the words before sub-paragraph ($a$)  there shall be substituted “Part VI of this Act shall have effect as if the following Acts were included in the Acts comprised in the relevant social security legislation”.
%
%SCHEDULE 7Housing benefit and council tax benefit: revisions and appeals
%Introductory
%
%1(1) In this Schedule “relevant authority” means an authority administering housing benefit or council tax benefit.
%
%(2) In this Schedule “relevant decision” means any of the following—
%
%($a$) a decision of a relevant authority on a claim for housing benefit or council tax benefit;
%
%($b$) any decision under paragraph 4 of this Schedule which supersedes a decision falling within paragraph ($a$), within this paragraph or within paragraph ($b$)  of sub-paragraph (1)  of that paragraph;
%
%but references in this Schedule to a relevant decision do not include references to a decision under paragraph 3 to revise a relevant decision.
%Decisions on claims for benefit
%
%2Where at any time a claim for housing benefit or council tax benefit is decided by a relevant authority—
%
%($a$) the claim shall not be regarded as subsisting after that time; and
%
%($b$) accordingly, the claimant shall not (without making a further claim) be entitled to the benefit on the basis of circumstances not obtaining at that time.
%Revision of decisions
%
%3(1) Any relevant decision may be revised or further revised by the relevant authority which made the decision—
%
%($a$) either within the prescribed period or in prescribed cases or circumstances; and
%
%($b$) either on an application made for the purpose by a person affected by the decision or on their own initiative;
%
%and regulations may prescribe the procedure by which a decision of a relevant authority may be so revised.
%
%(2) In making a decision under sub-paragraph (1), the relevant authority need not consider any issue that is not raised by the application or, as the case may be, did not cause them to act on their own initiative.
%
%(3) Subject to sub-paragraphs (4)  and (5)  and paragraph 18, a revision under this paragraph shall take effect as from the date on which the original decision took (or was to take) effect.
%
%(4) Regulations may provide that, in prescribed cases or circumstances, a revision under this paragraph shall take effect as from such other date as may be prescribed.
%
%(5) Where a decision is revised under this paragraph, for the purposes of any rule as to the time allowed for bringing an appeal, the decision shall be regarded as made on the date on which it is so revised.
%
%(6) Except in prescribed circumstances, an appeal against a decision of the relevant authority shall lapse if the decision is revised under this paragraph before the appeal is determined.
%Decisions superseding earlier decisions
%
%4(1) Subject to sub-paragraph (4) , the following, namely—
%
%($a$) any relevant decision (whether as originally made or as revised under paragraph 3), and
%
%($b$) any decision under this Schedule of an appeal tribunal or a Commissioner,
%
%may be superseded by a decision made by the appropriate relevant authority, either on an application made for the purpose by a person affected by the decision or on their own initiative.
%
%(2) In this paragraph “the appropriate relevant authority” means the authority which made the decision being superseded, the decision appealed against to the tribunal or, as the case may be, the decision to which the decision being appealed against to the Commissioner relates.
%
%(3) In making a decision under sub-paragraph (1), the relevant authority need not consider any issue that is not raised by the application or, as the case may be, did not cause them to act on their own initiative.
%
%(4) Regulations may prescribe the cases and circumstances in which, and the procedure by which, a decision may be made under this paragraph.
%
%(5) Subject to sub-paragraph (6)  and paragraph 18, a decision under this paragraph shall take effect as from the date on which it is made or, where applicable, the date on which the application was made.
%
%(6) Regulations may provide that, in prescribed cases or circumstances, a decision under this paragraph shall take effect as from such other date as may be prescribed.
%Use of experts by relevant authorities
%
%5Where it appears to a relevant authority that a matter in relation to which a relevant decision falls to be made by them involves a question of fact requiring special expertise, they may direct that, in dealing with that matter, they shall have the assistance of one or more persons appearing to them to have knowledge or experience which would be relevant in determining that question.
%Appeal to appeal tribunal
%
%6(1) Subject to sub-paragraph (2) , this paragraph applies to any relevant decision (whether as originally made or as revised under paragraph 3) of a relevant authority which—
%
%($a$) is made on a claim for, or on an award of, housing benefit or council tax benefit; or
%
%($b$) does not fall within paragraph ($a$)  but is of a prescribed description.
%
%(2) This paragraph does not apply to—
%
%($a$) any decision terminating or reducing the amount of a person’s housing benefit or council tax benefit that is made in consequence of any decision made under regulations under section 2A of the Administration Act (work-focused interviews);
%
%($b$) any decision of a relevant authority as to the application or operation of any modification of a housing benefit scheme or council tax benefit scheme under section 134(8) ($a$)  or section 139(6) ($a$)  of the Administration Act (disregard of war disablement and war widows' pensions);
%
%($c$) so much of any decision of a relevant authority as adopts a decision of a rent officer under any order made by virtue of section 122 of the [1996 c. 52. ] Housing Act 1996 (decisions of rent officers for the purposes of housing benefit);
%
%($d$) any decision of a relevant authority as to the amount of benefit to which a person is entitled in a case in which the amount is determined by the rate of benefit provided for by law; or
%
%($e$) any such other decision as may be prescribed.
%
%(3) In the case of a decision to which this paragraph applies, any person affected by the decision shall have a right to appeal to an appeal tribunal.
%
%(4) Nothing in sub-paragraph (3)  shall confer a right of appeal in relation to—
%
%($a$) a prescribed decision; or
%
%($b$) a prescribed determination embodied in or necessary to a decision.
%
%(5) Regulations under sub-paragraph (4)  shall not prescribe any decision or determination that relates to the conditions of entitlement to housing benefit or council tax benefit for which a claim has been validly made.
%
%(6) Where any amount of housing benefit or council tax benefit is determined to be recoverable under or by virtue of section 75 or 76 of the Administration Act (overpayments and excess benefits), any person from whom it has been determined that it is so recoverable shall have a right of appeal to an appeal tribunal.
%
%(7) A person with a right of appeal under this paragraph shall be given such notice of the decision in respect of which he has that right, and of that right, as may be prescribed.
%
%(8) Regulations may make provision as to the manner in which, and the time within which, appeals are to be brought.
%
%(9) In deciding an appeal under this paragraph, an appeal tribunal—
%
%($a$) need not consider any issue that is not raised by the appeal; and
%
%($b$) shall not take into account any circumstances not obtaining at the time when the decision appealed against was made.
%Redetermination etc. of appeals by tribunal
%
%7(1) This paragraph applies where an application is made to a person for leave under paragraph 8(7)($a$)  or ($c$)  to appeal from a decision of an appeal tribunal.
%
%(2) If the person considers that the decision was erroneous in point of law, he may set aside the decision and refer the case either for redetermination by the tribunal or for determination by a differently constituted tribunal.
%
%(3) If each of the principal parties to the case expresses the view that the decision was erroneous in point of law, the person shall set aside the decision and refer the case for determination by a differently constituted tribunal.
%
%(4) In this paragraph and paragraph 8 “principal parties” means—
%
%($a$) where he is the applicant for leave to appeal or the circumstances are otherwise such as may be prescribed, the Secretary of State;
%
%($b$) the relevant authority against whose decision the appeal to the appeal tribunal was brought; and
%
%($c$) the person affected by the decision against which the appeal to the appeal tribunal was brought or by the tribunal’s decision on that appeal.
%Appeal from tribunal to Commissioner
%
%8(1) Subject to the provisions of this paragraph, an appeal lies to a Commissioner from any decision of an appeal tribunal under paragraph 6 or 7 on the ground that the decision of the tribunal was erroneous in point of law.
%
%(2) An appeal lies under this paragraph at the instance of any of the following—
%
%($a$) the Secretary of State;
%
%($b$) the relevant authority against whose decision the appeal to the appeal tribunal was brought;
%
%($c$) any person affected by the decision against which the appeal to the appeal tribunal was brought or by the tribunal’s decision on that appeal.
%
%(3) If each of the principal parties to the appeal expresses the view that the decision appealed against was erroneous in point of law, the Commissioner may set aside the decision and refer the case to a tribunal with directions for its determination.
%
%(4) Where the Commissioner holds that the decision appealed against was erroneous in point of law, he shall set it aside.
%
%(5) Where under sub-paragraph (4)  the Commissioner sets aside a decision—
%
%($a$) he shall have power, if he can do so without making fresh or further findings of fact, to give the decision which he considers the tribunal should have given;
%
%($b$) he shall also have power, if he considers it expedient, to make such findings and to give such decision as he considers appropriate in the light of them; and
%
%($c$) if he does not exercise the power in paragraph ($a$)  or ($b$), he shall refer the case to a tribunal with directions for its determination.
%
%(6) Subject to any direction of the Commissioner, a reference under sub-paragraph (3)  or (5)($c$)  shall be to a differently constituted tribunal.
%
%(7) No appeal lies under this paragraph without leave; and leave for the purposes of this sub-paragraph may be given—
%
%($a$) by the person who constituted, or was the chairman of, the tribunal when the decision to be appealed against was given;
%
%($b$) subject to and in accordance with regulations, by a Commissioner; or
%
%($c$) in a prescribed case, by such person not falling within paragraph ($a$)  or ($b$)  as may be prescribed.
%
%(8) Regulations may make provision as to the manner in which, and the time within which, appeals are to be brought and applications made for leave to appeal.
%Appeal from Commissioner on point of law
%
%9(1) Subject to sub-paragraphs (2)  and (3), an appeal on a question of law shall lie to the appropriate court from any decision of a Commissioner.
%
%(2) No appeal under this paragraph shall lie from a decision except—
%
%($a$) with the leave of the Commissioner who gave the decision or, in a prescribed case, with the leave of a Commissioner selected in accordance with regulations; or
%
%($b$) if he refuses leave, with the leave of the appropriate court.
%
%(3) An application for leave under this paragraph in respect of a Commissioner’s decision may only be made by—
%
%($a$) a person who, before the proceedings before the Commissioner were begun, was entitled to appeal to the Commissioner from the decision to which the Commissioner’s decision relates;
%
%($b$) any other person who was a party to the proceedings in which the decision to which the Commissioner’s decision relates was given;
%
%($c$) any other person who is authorised by regulations to apply for leave;
%
%and regulations may make provision with respect to the manner in which, and the time within which, applications must be made to a Commissioner for leave under this paragraph, and with respect to the procedure for dealing with such applications.
%
%(4) On an application to a Commissioner for leave under this paragraph, it shall be the duty of the Commissioner to specify as the appropriate court—
%
%($a$) the Court of Appeal if it appears to him that the relevant dwelling is in England or Wales; and
%
%($b$) the Court of Session if it appears to him that the relevant dwelling is in Scotland;
%
%except that if it appears to him, having regard to the circumstances of the case and in particular to the convenience of the persons who may be parties to the proposed appeal, that he should specify a different court mentioned in paragraph ($a$)  or ($b$)  as the appropriate court, it shall be his duty to specify that court as the appropriate court.
%
%(5) In this paragraph—
%
%    “the appropriate court”, except in sub-paragraph (4) , means the court specified in pursuance of that sub-paragraph;
%
%    “the relevant dwelling”, in relation to any decision, means the dwelling by reference to which any claim or award of housing benefit or council tax benefit to which the decision relates was made. 
%
%Procedure
%
%10(1) Regulations may make for the purposes of this Schedule any such provision as is specified in Schedule 5 to the [1998 c. 14. ] Social Security Act 1998, or as would be so specified if the references to the Secretary of State in paragraph 1 of that Schedule were references to a relevant authority.
%
%(2) Regulations prescribing the procedure to be followed in cases before a Commissioner shall provide that any hearing shall be in public except in so far as the Commissioner for special reasons otherwise directs.
%
%(3) It is hereby declared that the power by regulations to prescribe procedure includes power—
%
%($a$) to make provision as to the representation of one person, at any hearing of a case, by another person whether having professional qualifications or not; and
%
%($b$) to confer on the Secretary of State a right to be represented and heard in any proceedings before a Commissioner to which he is not already a party.
%
%(4) If it appears to a Commissioner that a matter before him involves a question of fact of special difficulty, he may direct that in dealing with that matter he shall have the assistance of one or more persons appearing to him to have knowledge or experience which would be relevant in determining that question.
%
%(5) If it appears to the Chief Commissioner (or, in the case of his inability to act, to such other of the Commissioners as he may have nominated to act for the purpose) that—
%
%($a$) an application for leave under paragraph 8(7)($b$), or
%
%($b$) an appeal,
%
%falling to be heard by one of the Commissioners involves a question of law of special difficulty, he may direct that the application or appeal be dealt with, not by that Commissioner alone, but by a tribunal consisting of any three or more of the Commissioners.
%
%(6) If the decision of such a tribunal is not unanimous, the decision of the majority shall be the decision of the tribunal; and the presiding Commissioner shall have a casting vote if the votes (including his first vote) are equally divided.
%
%(7) Where a direction is given under sub-paragraph (5)($a$), paragraph 8(7)($b$)  shall have effect as if the reference to a Commissioner were a reference to such a tribunal as is mentioned in sub-paragraph (5).
%
%(8) Except so far as it may be applied in relation to England and Wales by regulations, Part I of the [1996 c. 23. ] Arbitration Act 1996 shall not apply to any proceedings under this Schedule.
%Finality of decisions
%
%11Subject to the provisions of this Schedule, any decision made in accordance with the preceding provisions of this Schedule shall be final.
%Matters arising as respects decisions
%
%12Regulations may make provision as respects matters arising—
%
%($a$) pending any decision under this Schedule of a relevant authority, an appeal tribunal or a Commissioner which relates to—
%
%(i) any claim for housing benefit or council tax benefit;
%
%(ii) any person’s entitlement to such a benefit or its receipt;
%
%or
%
%($b$) out of the revision under paragraph 3, or on appeal, of any such decision.
%Suspension in prescribed circumstances
%
%13(1) Regulations may provide for—
%
%($a$) suspending, in whole or in part, any payments of housing benefit or council tax benefit;
%
%($b$) suspending, in whole or in part, any reduction (by way of council tax benefit) in the amount that a person is or will become liable to pay in respect of council tax;
%
%($c$) the subsequent making, or restoring, in prescribed circumstances of any or all of the payments, or reductions, so suspended.
%
%(2) Regulations made under sub-paragraph (1)  may, in particular, make provision for any case where, in relation to a claim for housing benefit or council tax benefit—
%
%($a$) it appears to the relevant authority that an issue arises whether the conditions for entitlement to such a benefit are or were fulfilled;
%
%($b$) it appears to the relevant authority that an issue arises whether a decision as to an award of such a benefit should be revised (under paragraph 3) or superseded (under paragraph 4);
%
%($c$) an appeal is pending against a decision of an appeal tribunal, a Commissioner or a court; or
%
%($d$) it appears to the relevant authority, where an appeal is pending against the decision given by a Commissioner or a court in a different case, that if the appeal were to be determined in a particular way an issue would arise whether the award of housing benefit or council tax benefit in the case itself ought to be revised or superseded.
%
%(3) For the purposes of sub-paragraph (2) , an appeal against a decision is pending if—
%
%($a$) an appeal against the decision has been brought but not determined;
%
%($b$) an application for leave to appeal against the decision has been made but not determined; or
%
%($c$) the time within which—
%
%(i) an application for leave to appeal may be made, or
%
%(ii) an appeal against the decision may be brought,
%
%has not expired and the circumstances are such as may be prescribed.
%
%(4) In sub-paragraph (2)($d$)  the reference to a different case—
%
%($a$) includes a reference to a case involving a different relevant authority; but
%
%($b$) does not include a reference to a case relating to a different benefit unless the different benefit is housing benefit or council tax benefit.
%Suspension for failure to furnish information etc.
%
%14(1) The powers conferred by this paragraph are exercisable in relation to persons who fail to comply with information requirements.
%
%(2) Regulations may provide for—
%
%($a$) suspending, in whole or in part, any payments of housing benefit or council tax benefit;
%
%($b$) suspending, in whole or in part, any reduction (by way of council tax benefit) in the amount that a person is or will become liable to pay in respect of council tax;
%
%($c$) the subsequent making, or restoring, in prescribed circumstances of any or all of the payments, or any right, so suspended.
%
%(3) In this paragraph and paragraph 15 “information requirement” means—
%
%($a$) in the case of housing benefit, a requirement in pursuance of regulations made by virtue of section 5(1)(hh) of the Administration Act to furnish information or evidence needed for a determination whether a decision on an award of that benefit should be revised under paragraph 3 or superseded under paragraph 4 of this Schedule; and
%
%($b$) in the case of council tax benefit, a requirement made in pursuance of regulations under section 6(1)(hh) of the Administration Act to furnish information or evidence needed for a determination whether a decision on an award of that benefit should be so revised or superseded.
%Termination in cases of a failure to furnish information
%
%15Regulations may provide that, except in prescribed cases or circumstances—
%
%($a$) a person whose benefit has been suspended in accordance with regulations under paragraph 13 and who subsequently fails to comply with an information requirement, or
%
%($b$) a person whose benefit has been suspended in accordance with regulations under paragraph 14 for failing to comply with such a requirement,
%
%shall cease to be entitled to the benefit from a date not earlier than the date on which payments were suspended.
%Decisions involving issues that arise on appeal in other cases
%
%16(1) This paragraph applies where—
%
%($a$) a relevant decision, or a decision under paragraph 3 about the revision of an earlier decision, falls to be made in any particular case; and
%
%($b$) an appeal is pending against the decision given in another case by a Commissioner or a court.
%
%(2) A relevant authority need not make the decision while the appeal is pending if they consider it possible that the result of the appeal will be such that, if it were already determined, there would be no entitlement to benefit.
%
%(3) If a relevant authority consider it possible that the result of the appeal will be such that, if it were already determined, it would affect the decision in some other way—
%
%($a$) they need not, except in such cases or circumstances as may be prescribed, make the decision while the appeal is pending;
%
%($b$) they may, in such cases or circumstances as may be prescribed, make the decision on such basis as may be prescribed.
%
%(4) Where—
%
%($a$) a relevant authority act in accordance with sub-paragraph (3)($b$), and
%
%($b$) following the making of the determination it is appropriate for their decision to be revised,
%
%they shall then revise their decision (under paragraph 3) in accordance with that determination.
%
%(5) For the purposes of this paragraph, an appeal against a decision is pending if—
%
%($a$) an appeal against the decision has been brought but not determined;
%
%($b$) an application for leave to appeal against the decision has been made but not determined; or
%
%($c$) the time within which—
%
%(i) an application for leave to appeal may be made, or
%
%(ii) an appeal against the decision may be brought,
%
%has not expired and the circumstances are such as may be prescribed.
%
%(6) In paragraphs ($a$), ($b$)  and ($c$)  of sub-paragraph (5) , any reference to an appeal against a decision, or to an application for leave to appeal against a decision, includes a reference to—
%
%($a$) an application for judicial review of the decision under section 31 of the [1981 c. 54. ] Supreme Court Act 1981 or for leave to apply for judicial review; or
%
%($b$) an application to the supervisory jurisdiction of the Court of Session in respect of the decision.
%
%(7) In sub-paragraph (1)($b$)  the reference to another case—
%
%($a$) includes a reference to a case involving a decision made, or falling to be made, by a different relevant authority; but
%
%($b$) does not include a reference to a case relating to another benefit unless the other benefit is housing benefit or council tax benefit.
%Appeals involving issues that arise on appeal in other cases
%
%17(1) This paragraph applies where—
%
%($a$) an appeal (“appeal A”) in relation to a relevant decision (whether as originally made or as revised under paragraph 3) is made to an appeal tribunal, or from an appeal tribunal to a Commissioner; and
%
%($b$) an appeal (“appeal B”) is pending against a decision given in a different case by a Commissioner or a court.
%
%(2) If the relevant authority whose decision gave rise to appeal A consider it possible that the result of appeal B will be such that, if it were already determined, it would affect the determination of appeal A, they may serve notice requiring the tribunal or Commissioner—
%
%($a$) not to determine appeal A but to refer it to them; or
%
%($b$) to deal with the appeal in accordance with sub-paragraph (4) .
%
%(3) Where appeal A is referred to the authority under sub-paragraph (2)($a$), following the determination of appeal B and in accordance with that determination, they shall if appropriate—
%
%($a$) in a case where appeal A has not been determined by the tribunal, revise (under paragraph 3) their decision which gave rise to that appeal; or
%
%($b$) in a case where appeal A has been determined by the tribunal, make a decision (under paragraph 4) superseding the tribunal’s decision.
%
%(4) Where appeal A is to be dealt with in accordance with this sub-paragraph, the appeal tribunal or Commissioner shall either—
%
%($a$) stay appeal A until appeal B is determined; or
%
%($b$) if the tribunal or Commissioner considers it to be in the interests of the appellant to do so, determine appeal A as if—
%
%(i) appeal B had already been determined; and
%
%(ii) the issues arising on appeal B had been decided in the way that was most unfavourable to the appellant.
%
%(5) Where the appeal tribunal or Commissioner acts in accordance with sub-paragraph (4)($b$), following the determination of appeal B the relevant authority whose decision gave rise to appeal A shall, if appropriate, make a decision (under paragraph 4) superseding the decision of the tribunal or Commissioner in accordance with that determination.
%
%(6) For the purposes of this paragraph, an appeal against a decision is pending if—
%
%($a$) an appeal against the decision has been brought but not determined;
%
%($b$) an application for leave to appeal against the decision has been made but not determined; or
%
%($c$) the time within which—
%
%(i) an application for leave to appeal may be made, or
%
%(ii) an appeal against the decision may be brought,
%
%has not expired and the circumstances are such as may be prescribed.
%
%(7) In this paragraph—
%
%($a$) the reference in sub-paragraph (1)($a$)  to an appeal to a Commissioner includes a reference to an application for leave to appeal to a Commissioner;
%
%($b$) the reference in sub-paragraph (1)($b$)  to a different case—
%
%(i) includes a reference to a case involving a different relevant authority; but
%
%(ii) does not include a reference to a case relating to a different benefit unless the different benefit is housing benefit or council tax benefit; and
%
%($c$) any reference in paragraph ($a$), ($b$)  or ($c$)  of sub-paragraph (6)  to an appeal, or to an application for leave to appeal, against a decision includes a reference to—
%
%(i) an application for judicial review of the decision under section 31 of the [1981 c. 54. ] Supreme Court Act 1981 or for leave to apply for judicial review; or
%
%(ii) an application to the supervisory jurisdiction of the Court of Session in respect of the decision.
%
%(8) In sub-paragraph (4)  “the appellant” means the person who appealed or, as the case may be, first appealed against the decision mentioned in sub-paragraph (1)($a$) .
%
%(9) Regulations may make provision supplementing the provision made by this paragraph.
%Restrictions on entitlement to benefit in certain cases of error
%
%18(1) Subject to sub-paragraph (2) , this paragraph applies where—
%
%($a$) the effect of the determination, whenever made, of an appeal by virtue of this Schedule to a Commissioner or the court (“the relevant determination”) is that the relevant authority’s decision out of which the appeal arose was erroneous in point of law; and
%
%($b$) after the date of the relevant determination a decision falls to be made by that relevant authority or another relevant authority in accordance with that determination (or would, apart from this paragraph, fall to be so made)—
%
%(i) in relation to a claim for housing benefit or council tax benefit;
%
%(ii) as to whether to revise, under paragraph 3, a decision as to a person’s entitlement to such a benefit; or
%
%(iii) on an application made under paragraph 4 for a decision as to a person’s entitlement to such a benefit to be superseded.
%
%(2) This paragraph does not apply where the decision mentioned in sub-paragraph (1)($b$) —
%
%($a$) is one which, but for paragraph 16(2)  or (3)($a$), would have been made before the date of the relevant determination; or
%
%($b$) is one made in pursuance of paragraph 17(3)  or (5).
%
%(3) In so far as the decision relates to a person’s entitlement to benefit in respect of a period before the date of the relevant determination, it shall be made as if the relevant authority’s decision had been found by the Commissioner or court not to have been erroneous in point of law.
%
%(4) Sub-paragraph (1)($a$)  shall be read as including a case where—
%
%($a$) the effect of the relevant determination is that part or all of a purported regulation or order is invalid; and
%
%($b$) the error of law made by the relevant authority was to act on the basis that the purported regulation or order (or the part held to be invalid) was valid.
%
%(5) It is immaterial for the purposes of sub-paragraph (1) —
%
%($a$) where such a decision as is mentioned in paragraph ($b$)(i)  falls to be made, whether the claim was made before or after the date of the relevant determination;
%
%($b$) where such a decision as is mentioned in paragraph ($b$)(ii)  or (iii)  falls to be made on an application under paragraph 3 or (as the case may be) 4, whether the application was made before or after that date.
%
%(6) In this paragraph “the court” means—
%
%($a$) the High Court;
%
%($b$) the Court of Appeal;
%
%($c$) the Court of Session;
%
%($d$) the House of Lords; or
%
%($e$) the Court of Justice of the European Community.
%
%(7) For the purposes of this paragraph, any reference to entitlement to benefit includes a reference to entitlement—
%
%($a$) to any increase in the rate of a benefit; or
%
%($b$) to a benefit, or increase of benefit, at a particular rate.
%
%(8) The date of the relevant determination shall, in prescribed cases, be determined for the purposes of this paragraph in accordance with any regulations made for that purpose.
%
%(9) Regulations made under sub-paragraph (8)  may include provision—
%
%($a$) for a determination of a higher court to be treated as if it had been made on the date of a determination by a lower court or by a Commissioner; or
%
%($b$) for a determination of a lower court or of a Commissioner to be treated as if it had been made on the date of a determination by a higher court.
%Correction of errors and setting aside of decisions
%
%19(1) Regulations may make provision with respect to—
%
%($a$) the correction of accidental errors in any decision or record of a decision made under or by virtue of any relevant provision; and
%
%($b$) the setting aside of any such decision in a case where it appears just to set the decision aside on the ground that—
%
%(i) a document relating to the proceedings in which the decision was given was not sent to, or was not received at an appropriate time by, a party to the proceedings or a party’s representative, or was not received at an appropriate time by the body or person who gave the decision; or
%
%(ii) a party to the proceedings or a party’s representative was not present at a hearing related to the proceedings.
%
%(2) Nothing in sub-paragraph (1)  shall be construed as derogating from any power to correct errors or set aside decisions which is exercisable apart from regulations made by virtue of that sub-paragraph.
%
%(3) In this paragraph “relevant provision” means—
%
%($a$) any of the provisions of this Schedule;
%
%($b$) any of the provisions of Part VII of the [1992 c. 4. ] Social Security Contributions and Benefits Act 1992 so far as they relate to housing benefit or council tax benefit; or
%
%($c$) any of the provisions of Part VIII of the [1992 c. 4. ] Administration Act or of any regulations under section 2A of that Act, so far as the provisions or regulations relate to, or to arrangements for, housing benefit or council tax benefit.
%Regulations
%
%20(1) The power to make regulations under this Schedule shall be exercisable—
%
%($a$) in the case of regulations with respect to proceedings before the Commissioners, by the Lord Chancellor; and
%
%($b$) in any other case, by the Secretary of State;
%
%and the Lord Chancellor shall consult with the Scottish Ministers before making any regulations under this Schedule that apply to Scotland.
%
%(2) Any power conferred by this Schedule to make regulations shall include power to make different provision for different areas or different relevant authorities.
%
%(3) Subsections (3)  to (7)  of section 79 of the [1998 c. 14. ] Social Security Act 1998 (supplemental provision in connection with powers to make subordinate legislation under that Act) shall apply to any power to make regulations under this Schedule as they apply to any power to make regulations under that Act.
%
%(4) A statutory instrument containing (whether alone or with other provisions) regulations under paragraph 6(2)($e$)  or (4)  shall not be made unless a draft of the instrument has been laid before Parliament and approved by a resolution of each House.
%
%(5) A statutory instrument—
%
%($a$) which contains (whether alone or with other provisions) regulations made under this Schedule, and
%
%($b$) which is not subject to any requirement that a draft of the instrument be laid before and approved by a resolution of each House of Parliament,
%
%shall be subject to annulment in pursuance of a resolution of either House of Parliament.
%
%(6) In this paragraph the reference to regulations with respect to proceedings before the Commissioners includes a reference to regulations with respect to any such proceedings for the determination of any matter, or for leave to appeal to or from the Commissioners.
%Consequential amendments of the Administration Act
%
%21(1) In section 5(1)(hh) of the Administration Act (regulations about claims for and payments of benefit)—
%
%($a$) in sub-paragraph (i) , after “1998” there shall be inserted “or, as the case may be, under paragraph 3 of Schedule 7 to the Child Support, Pensions and Social Security Act 2000”; and
%
%($b$) in sub-paragraph (ii) , after “Act” there shall be inserted “or, as the case may be, paragraph 4 of that Schedule”.
%
%(2) In section 6(1)  of the Administration Act (regulations about claims for and payments of council tax benefit), after paragraph (h) there shall be inserted—
%
%“(hh)for requiring such person as may be prescribed in accordance with the regulations to furnish any information or evidence needed for a determination whether a decision on an award of a benefit—
%
%(i) should be revised under paragraph 3 of Schedule 7 to the Child Support, Pensions and Social Security Act 2000; or
%
%(ii) should be superseded under paragraph 4 of that Schedule;”.
%Consequential amendments of the Social Security Act 1998
%
%22(1) Section 34(4)  and (5)  and section 35 of the [1998 c. 14. ] Social Security Act 1998 (regulations for the determination of claims and reviews of housing benefit and council tax benefit and for the suspension of those benefits) shall cease to have effect.
%
%(2) In paragraph 4(1)($a$)  of Schedule 1 to that Act (supplementary provisions relating to the appeal tribunals), for “or section 20 of the Child Support Act” there shall be substituted “, section 20 of the Child Support Act or paragraph 6 of Schedule 7 to the Child Support, Pensions and Social Security Act 2000”.
%
%(3) In paragraph 3(1)  of Schedule 4 to that Act (provisions relating to the Social Security Commissioners), after “section 14 of this Act” there shall be inserted “or under paragraph 8 of Schedule 7 to the Child Support, Pensions and Social Security Act 2000”.'.
%Interpretation
%
%23(1) In this Schedule—
%
%    “the Administration Act” means the [1992 c. 5. ] Social Security Administration Act 1992;
%
%    “affected” shall be construed subject to any regulations under sub-paragraph (2) ;
%
%    “appeal tribunal” means an appeal tribunal constituted under Chapter I of Part I of the [1998 c. 14. ] Social Security Act 1998;
%
%    “the Chief Commissioner” means the Chief Social Security Commissioner;
%
%    “Commissioner” means the Chief Commissioner or any other Social Security Commissioner, and includes a tribunal of three or more Commissioners constituted under paragraph 10(5) ;
%
%    “prescribed” means prescribed by regulations under this Schedule;
%
%    “relevant authority” has the meaning given by paragraph 1(1) ;
%
%    “relevant decision” has the meaning given by paragraph 1(2). 
%
%(2) Regulations may make provision specifying the circumstances in which a person is or is not to be treated for the purposes of this Schedule as a person who is affected by any decision of a relevant authority.
%
%(3) For the purposes of this Schedule any decision that is made or falls to be made—
%
%($a$) by a person authorised to carry out any function of a relevant authority relating to housing benefit or council tax benefit, or
%
%($b$) by a person providing services relating to housing benefit or council tax benefit directly or indirectly to a relevant authority,
%
%shall be treated as a decision of the relevant authority on whose behalf the function is carried out or, as the case may be, to whom those services are provided. 

%SCHEDULE 8Declarations of status: consequential amendments
%The Births and Deaths Registration Act 1953 (c. 20)
%
%1In section 14A(1)($a$)  of the Births and Deaths Registration Act 1953 (re-registration of birth where notification of declaration of parentage given under section 56(4)  of the [1986 c. 55. ] Family Law Act 1986), for “56(4)” there shall be substituted “55A(7)  or 56(4)”.
%The Magistrates' Courts Act 1980 (c. 43)
%
%2(1) Section 65 of the Magistrates' Courts Act 1980 (meaning of family proceedings) shall be amended as follows.
%
%(2) In subsection (1)  (proceedings which are family proceedings), after paragraph (m) there shall be inserted—
%
%“(mm)section 55A of the [1986 c. 55. ] Family Law Act 1986;”.
%
%(3) In subsection (2)  (power of court to treat combined proceedings as family proceedings), in paragraph ($e$), before “section 20” there shall be inserted “proceedings under”.
%The Family Law Act 1986 (c. 55)
%
%3The Family Law Act 1986 shall be amended as follows.
%
%4In section 55 (declarations as to marital status)—
%
%($a$) in subsection (1), for “the court” there shall be substituted “the High Court or a county court”, and
%
%($b$) in subsection (3), after “made” there shall be inserted “to a court”.
%
%5In section 56 (declarations as to legitimacy or legitimation)—
%
%($a$) in subsections (1)  and (2) , for “the court” there shall be substituted “the High Court or a county court”, and
%
%($b$) in subsection (4) , after “made” there shall be inserted “by a court”.
%
%6In section 57(1)  (application to the court for declaration as to overseas adoption), for “the court” there shall be substituted “the High Court or a county court”.
%
%7In section 58 (general provisions)—
%
%($a$) in subsection (1), after “application” there shall be inserted “to a court”, and
%
%($b$) in subsection (3), for “The” there shall be substituted “A”.
%
%8In section 59 (provisions relating to the Attorney-General)—
%
%($a$) in subsections (1)  and (2) , after “an application” there shall be inserted “to a court”, and
%
%($b$) in subsection (3), after “any application” there shall be inserted “to a court”.
%The Family Law Reform Act 1987 (c. 42)
%
%9In section 23(1)  of the Family Law Reform Act 1987—
%
%($a$) in subsection (2)  to be substituted for section 20(2)  of the [1969 c. 46. ] Family Law Reform Act 1969 (report to court about scientific tests), for “person responsible for” there shall be substituted “individual”; and
%
%($b$) in subsection (2A)  to be inserted in section 20 of that Act (blood tests in proceedings under section 56 of the [1986 c. 55. ] Family Law Act 1986), for “56” there shall be substituted “55A or 56”.
%The Children Act 1989 (c. 41)
%
%10(1) Part I of Schedule 11 to the Children Act 1989 (jurisdiction) shall be amended as follows.
%
%(2) In paragraph 1(2A)  (additional proceedings which may be required to be commenced in a particular court)—
%
%($a$) for paragraph ($a$)  there shall be substituted—
%
%“($a$) under section 55A of the [1986 c. 55. ] Family Law Act 1986 (declarations of parentage); or”, and
%
%($b$) in paragraph ($b$), for “of that Act” there shall be substituted “of the [1991 c. 48. ] Child Support Act 1991”.
%
%(3) In paragraph 2(3)  (power to transfer certain proceedings)—
%
%($a$) after paragraph ($b$)  there shall be inserted—
%
%“(ba)any proceedings under section 55A of the [1986 c. 55. ] Family Law Act 1986”, and
%
%($b$) in paragraph (bb), before “section 20” there shall be inserted “any proceedings under”.
%The Child Support Act 1991 (c. 48)
%
%11The Child Support Act 1991 shall be amended as follows.
%
%12In section 26(2)  (cases where Secretary of State may make maintenance calculation despite denial of parentage), in Case C (where there has been a declaration under section 56 of the [1986 c. 55. ] Family Law Act 1986), after “section” there shall be inserted “55A or”.
%
%13For section 27 (declarations of parentage) there shall be substituted—
%“27Applications for declaration of parentage under Family Law Act 1986
%
%(1) This section applies where—
%
%($a$) an application for a maintenance calculation has been made (or is treated as having been made), or a maintenance calculation is in force, with respect to a person (“the alleged parent”) who denies that he is a parent of a child with respect to whom the application or calculation was made or treated as made;
%
%($b$) the Secretary of State is not satisfied that the case falls within one of those set out in section 26(2) ; and
%
%($c$) the Secretary of State or the person with care makes an application for a declaration under section 55A of the [1986 c. 55. ] Family Law Act 1986 as to whether or not the alleged parent is one of the child’s parents.
%
%(2) Where this section applies—
%
%($a$) if it is the person with care who makes the application, she shall be treated as having a sufficient personal interest for the purposes of subsection (3)  of that section; and
%
%($b$) if it is the Secretary of State who makes the application, that subsection shall not apply.
%
%(3) This section does not apply to Scotland.”
%
%14In section 27A(2)($b$)  (Secretary of State to recover fees for scientific tests if a court has made a declaration of parentage under section 27), for “section 27” there shall be substituted “section 55A of the [1986 c. 55. ] Family Law Act 1986”.
%The Access to Justice Act 1999 (c. 22)
%
%15In Schedule 2 to the Access to Justice Act 1999 (services which are not to be funded as part of community legal services), in paragraph 2(3), after paragraph ($d$)  there shall be inserted—
%
%“(da)under section 55A of the Family Law Act 1986 (declarations of parentage),”.

\part[Schedule 9 --- Repeals and revocations]{Schedule 9\\*Repeals and revocations}

%Part IChild support
%Chapter or number	Citation	Extent of repeal or revocation
%10 \& 11 Geo. 6 c. 24. 	The Naval Forces (Enforcement of Maintenance Liabilities) Act 1947. 	In section 1(1), paragraph (aaa).
%3 \& 4 Eliz. 2 c. 18. 	The Army Act 1955. 	In section 150A, in subsection (2) , paragraph ($b$)  and the word “or” preceding it, and in subsection (3), the words “or cancels” and “or (as the case may be) that it has been cancelled”.
%3 \& 4 Eliz. 2 c. 19. 	The Air Force Act 1955. 	In section 150A, in subsection (2) , paragraph ($b$)  and the word “or” preceding it, and in subsection (3), the words “or cancels” and “or (as the case may be) that it has been cancelled”.
%1973 c. 18. 	The Matrimonial Causes Act 1973. 	In section 29, in subsection (7), the words “or is cancelled”, “or was cancelled” and “or, as the case may be, the date with effect from which it was cancelled”; and in subsection (8) , paragraph ($b$)  and the word “and” preceding it.
%1978 c. 22. 	The Domestic Proceedings and Magistrates' Courts Act 1978. 	In section 5, in subsection (7), the words “or is cancelled”, “or was cancelled” and “or, as the case may be, the date with effect from which it was cancelled”; and in subsection (8) , paragraph ($b$)  and the word “and” preceding it.
%1989 c. 41. 	The Children Act 1989. 	In Schedule 1, in paragraph 3(7), the words “or is cancelled”, “or was cancelled” and “or, as the case may be, the date with effect from which it was cancelled”; and in paragraph 3(8) , paragraph ($b$)  and the word “and” preceding it.
%1991 c. 48. 	The Child Support Act 1991. 	In section 15(10) , the definition of “specified” and the preceding word “and”.
%		In section 17(1), the word “and” after paragraph ($b$) .
%		In section 28D(2)($a$), “lapsed or”.
%		Sections 28H and 28I.
%		Section 40(1)  and (2).
%		Section 41(3)  to (5).
%		Section 44(3) .
%		Section 46B(3) .
%		In section 54, the definitions of “assessable income”, “current assessment”, “departure direction” and “maintenance requirement”.
%		In Schedule 1, paragraph 13, and in paragraph 16, sub-paragraph (1)($d$)  and ($e$), sub-paragraphs (2)  to (9) , and in sub-paragraph (10)  the words “, or should be cancelled”.
%		Schedule 4C.
%1992 c. 5. 	The Social Security Administration Act 1992. 	In section 170(5) , in the definition of “the relevant enactments”, paragraph (ab).
%1992 c. 6. 	The Social Security (Consequential Provisions) Act 1992. 	In Schedule 2, paragraph 113. 
%1995 c. 18. 	The Jobseekers Act 1995. 	In Schedule 2, paragraph 20(2) , (4)  and (7) .
%1995 c. 34. 	The Child Support Act 1995. 	Sections 1, 2 and 3. 
%		Sections 6, 7, 8, 9, 10 and 11. 
%		Section 14(2)  and (3) .
%		Section 18(3)  and (5).
%		Section 19. 
%		Section 22. 
%		Section 24. 
%		Section 26(4)($c$) .
%		Schedules 1 and 2. 
%		In Schedule 3, paragraphs 12, 15 and 20($a$) .
%1998 c. 14. 	The Social Security Act 1998. 	Section 42. 
%		In Schedule 7, paragraphs 20, 24, 25, 28, 34, and 35; in paragraph 36, the words “(1)  and”; and paragraphs 37, 38, 39, 40, 43, 46, 48(1), (2) , (3)  and (5)($a$), ($b$)  and ($c$), 53 and 54. 
%S.I. 1998/2780 (C.66).	The Social Security Act 1998 (Commencement No. 2) Order 1998. 	Article 3(4) .
%1999 c. 10. 	The Tax Credits Act 1999. 	In Schedule 1, paragraph 6(i) .
%		In Schedule 2, paragraph 17($a$) .
%Part IIState pensions
%Chapter	Short title	Extent of repeal
%1999 c. 30. 	The Welfare Reform and Pensions Act 1999. 	In Schedule 8, paragraph 5($b$)  and the word “and” immediately preceding it.
%Part IIIOccupational and personal pension schemes
%(1) 
%Member-nominated trustees and directors
%Chapter	Short title	Extent of repeal
%1995 c. 26. 	The Pensions Act 1995. 	In section 16(1), the words “(subject to section 17)” and in paragraph ($b$), the words “, and the appropriate rules,”.
%		Section 17. 
%		In section 18(1), the words “, subject to section 19,” and in paragraph ($b$), the words “, and the appropriate rules,”.
%		Sections 19 and 20. 
%		
%
%In section 21—
%($a$) 
%
%in subsections (1)  and (2) , the words “, or the appropriate rules,”;
%($b$) 
%
%in subsection (3), the words “or rules”;
%($c$) 
%
%in subsection (4) , the words “(or further arrangements)” in paragraph ($a$), and paragraph ($b$)  and the word “and” immediately preceding it;
%($d$) 
%
%subsection (5) ;
%($e$) 
%
%in subsection (7), the words “and this section”, paragraph ($b$)  and the word “and” immediately preceding paragraph ($b$) ; and
%($f$) 
%
%in subsection (8) , paragraph ($b$)  and the word “and” immediately preceding it.
%1999 c. 30. 	The Welfare Reform and Pensions Act 1999. 	In Schedule 12, paragraphs 46 and 48 and in paragraph 49, sub-paragraph ($b$)  and the word “and” immediately preceding it.
%(2) 
%Information to be given to the authority
%Chapter	Short title	Extent of repeal
%1993 c. 48. 	The Pension Schemes Act 1993. 	In section 178($a$), the words “sections 22 to 26 of the Pensions Act 1995”.
%1995 c. 26. 	The Pensions Act 1995. 	In Schedule 3, paragraph 43. 
%(3) 
%The Pensions Ombudsman
%Chapter	Short title	Extent of repeal
%1993 c. 48. 	The Pension Schemes Act 1993. 	
%
%In section 146—
%($a$) 
%
%in subsection (1)($c$), the words “which arises” and the words from “and which” to “beneficiary, and”;
%($b$) 
%
%in subsection (1)($d$), the words “which arises”; and
%($c$) 
%
%subsection (3A) .
%1995 c. 26. 	The Pensions Act 1995. 	Section 157(7) .
%(4) 
%Guaranteed minimum for widows and widowers
%Chapter	Short title	Extent of repeal
%1993 c. 48. 	The Pension Schemes Act 1993. 	In section 17(5) , the words “Category B retirement pension,”, in the first place where they occur, and the words from “or for which” onwards.
%(5) 
%Protected rights
%Chapter	Short title	Extent of repeal
%1995 c. 26. 	The Pensions Act 1995. 	In Schedule 5, paragraph 34($a$) .
%(6) 
%Contributions equivalent premiums
%Chapter or number	Citation	Extent of repeal or revocation
%1995 c. 26. 	The Pensions Act 1995. 	In Schedule 5, paragraph 57($a$)(ii) .
%S.I. 1995/3213 (N.I. 22).	The Pensions (Northern Ireland) Order 1995. 	In Schedule 3, paragraph 49($a$)(ii) .
%(7) 
%Use of cash equivalent
%Chapter	Short title	Extent of repeal
%1993 c. 48. 	The Pension Schemes Act 1993. 	Section 95(4) .
%(8) 
%Transfer values
%Chapter	Short title	Extent of repeal
%Sub-paragraph (4)  of paragraph 8 of Schedule 5 to this Act has effect in relation to this repeal as it has effect in relation to sub-paragraph (2)  of that paragraph.
%1993.  c. 48. 	The Pension Schemes Act 1993. 	Section 98(7)($a$) .
%(9) 
%Information about contracting-out
%Chapter	Short title	Extent of repeal
%1999 c. 2. 	The Social Security Contributions (Transfer of Functions, etc.) Act 1999. 	In Schedule 1, paragraph 60. 
%(10) 
%Duties relating to statements of contributions
%Chapter	Short title	Extent of repeal
%1995 c. 26. 	The Pensions Act 1995. 	In section 49(10) , the word “and” at the end of paragraph ($a$) .
%(11) 
%Spent provisions
%Chapter	Short title	Extent of repeal
%1993 c. 48. 	The Pension Schemes Act 1993. 	Section 56(5).
%1993 c. 49. 	The Pension Schemes (Northern Ireland) Act 1993. 	Section 52(5).
%Part IVWar pensions
%Chapter	Short title	Extent of repeal
%6 \& 7 Geo. 6.  c. 39. 	The Pensions Appeal Tribunals Act 1943. 	In section 8, in subsection (1), the words from “Provided” to the end, subsection (2)  and, in subsection (3), the words from “Provided” to the end.
%12, 13 \& 14 Geo. 6.  c. 12. 	The Pensions Appeal Tribunals Act 1949. 	Section 1(2).
%		Section 2. 
%1990 c. 41. 	The Courts and Legal Services Act 1990. 	In Schedule 10, paragraph 5. 
%1995 c. 26. 	The Pensions Act 1995. 	Section 169(6) .
%Part VLoss of benefit
%Chapter	Short title	Extent of repeal
%1998 c. 14. 	The Social Security Act 1998. 	In Schedule 3, in paragraph 3, the word “or” at the end of sub-paragraph ($c$) .
%Part VIInvestigation powers
%Chapter	Short title	Extent of repeal
%1992 c. 5. 	The Social Security Administration Act 1992. 	Section 111A(2).
%		Section 112(3) .
%1993 c. 48. 	The Pension Schemes Act 1993. 	In Schedule 8, paragraph 26. 
%1995 c. 18. 	The Jobseekers Act 1995. 	Section 33. 
%		Section 34(2) , (3)  and (5)  to (7) .
%		In Schedule 2, paragraph 54. 
%1995 c. 26. 	The Pensions Act 1995. 	In Schedule 5, paragraph 15(2).
%1997 c. 27. 	The Social Security (Recovery of Benefits) Act 1997. 	In Schedule 3, paragraph 4. 
%1997 c. 47. 	The Social Security Administration (Fraud) Act 1997. 	Section 12. 
%		In Schedule 1, paragraph 4(4) .
%1999 c. 2. 	The Social Security Contributions (Transfer of Functions, etc.) Act 1999. 	In Schedule 5, paragraph 2. 
%1999 c. 10. 	The Tax Credits Act 1999. 	In Schedule 2, paragraphs 11($a$), 13($a$)  and 14($a$) .
%1999 c. 30. 	The Welfare Reform and Pensions Act 1999. 	In Schedule 8, paragraph 34(2)($a$) .
%Part VIIHousing Benefit and Council Tax Benefit
%Chapter	Short title	Extent of repeal
%1998 c. 14. 	The Social Security Act 1998. 	In section 34, subsections (4)  and (5).
%		Section 35. 

\amendment{
Sch. 9 Pts. I--VII are not yet in force.
}

\section[Part VIII --- NICs in respect of benefits in kind]{Part VIII\\*NICs in respect of benefits in kind}

\renewcommand\parthead{--- Schedule 9 Part VIII}

\subsection*{(1) 
Great Britain}

{\footnotesize
%\begin{tabulary}{\linewidth}{JJJ}
\begin{longtable}{p{45.96pt}p{134.92946pt}p{173.11526pt}}
\hline
\itshape Chapter	&\itshape Short title	&\itshape Extent of repeal\\
\hline
\endhead
\hline
\endlastfoot
1992 c. 4. &	The Social Security Contributions and Benefits Act 1992. 	&In section 1(2)($b$), the words “in respect of cars made available for private use and car fuel”.\\
&		&In Schedule 1, paragraphs 3(2)  and 8(1)($i$).\\
1998 c. 14. 	&The Social Security Act 1998. 	&Section 50(2).\\
&&		Section 52. \\
&&		In Schedule 7, paragraph 58. \\
1999 c. 2. 	&The Social Security Contributions (Transfer of Functions, etc.)\ Act 1999. 	&In section 8(1), paragraph ($j$), and in paragraph ($l$), the words “amount of interest or”.\\
&&		In Schedule 1, paragraph 19(2).\\
	&&	In Schedule 3, paragraph 10. \\
%\end{tabulary}
\end{longtable}

}

1. 
These repeals (except the repeals in section 8(1)  of the Social Security Contributions (Transfer of Functions, etc.)\ Act 1999) have effect in relation to the tax year beginning with 6th April 2000 and subsequent tax years.

2. 
The repeals in section 8(1)  of the Social Security Contributions (Transfer of Functions, etc.)\ Act 1999 have effect in accordance with section 76(7)  of this Act.

\subsection*{
(2) 
Northern Ireland}

{\footnotesize
%\begin{tabulary}{\linewidth}{JJJ}
\begin{longtable}{p{50pt}p{150.9199pt}p{153.07274pt}}
\hline
\itshape Chapter	&\itshape Short title	&\itshape Extent of repeal\\
\hline
\endhead
\hline
\endlastfoot
1992 c. 7. 	&The Social Security Contributions and Benefits (Northern Ireland) Act 1992. 	&In section 1(2)($b$), the words “in respect of cars made available for private use and car fuel”.\\
&&		In Schedule 1, paragraphs 3(2)  and 8(1)($i$).\\
\textls[75]{S.I. 1998/\hspace{0pt}1506 (N.I.} 10).	&The Social Security (Northern Ireland) Order 1998. 	&Article 47(2).\\
&&		Article 49. \\
&&		In Schedule 6, paragraph 40. \\
\textls[75]{S.I. 1999/}\hspace{0pt}671. 	&The Social Security Contributions (Transfer of Functions, etc.)\ (Northern Ireland) Order 1999. 	&In Article 7(1), sub-paragraph ($j$), \textls[25]{and in sub-paragraph ($l$), the} words “amount of interest or”.\\
&&		In Schedule 1, paragraph 22(2).\\
&&		In Schedule 3, paragraph 11. \\
%\end{tabulary}
\end{longtable}

}

1. 
These repeals (except the repeals in Article 7(1)  of the Social Security Contributions (Transfer of Functions, etc.)\ (Northern Ireland) Order 1999) have effect in relation to the tax year beginning with 6th April 2000 and subsequent tax years.

2. 
The repeals in Article 7(1)  of the Social Security Contributions (Transfer of Functions, etc.)\ (Northern Ireland) Order 1999 have effect in accordance with section 80(7)  of this Act.


%Part IXTests for determining parentage and declarations of status
%Chapter	Short title	Extent of repeal
%1968 c. 63. 	The Domestic and Appellate Proceedings (Restriction of Publicity) Act 1968. 	In section 2, subsection (1)($e$)  and, in subsection (3), the words “or ($e$)”.
%1968 c. 64. 	The Civil Evidence Act 1968. 	In section 12(5) , in the definition of “relevant proceedings”, paragraph ($d$) .
%1980 c. 43. 	The Magistrates' Courts Act 1980. 	In section 65(1)(o) and (2)($e$), the words “or section 27”.
%1986 c. 55. 	The Family Law Act 1986. 	Section 56(1)($a$) .
%		Section 58(5)($b$) .
%		Section 63. 
%1987 c. 42. 	The Family Law Reform Act 1987. 	In paragraph 19($a$)  of Schedule 2, the words from “and there” to the end.
%1989 c. 41. 	The Children Act 1989. 	Section 89. 
%		In Schedule 11, in paragraphs 1(3)(bb) and 2(3)(bb), the words from “or 27” to “parentage)”.
%1990 c. 41. 	The Courts and Legal Services Act 1990. 	In Schedule 16, paragraph 3. 
%1991 c. 48. 	The Child Support Act 1991. 	In section 26(2) , Case D.
%1995 c. 34. 	The Child Support Act 1995. 	In section 20, subsections (1)  to (4) .
%1998 c. 14. 	The Social Security Act 1998. 	In Schedule 7, paragraph 32. 
%1999 c. 22. 	The Access to Justice Act 1999. 	In Schedule 2, in paragraph 2(3)($g$), the words “or 27”.

\amendment{
Sch. 9 Pt. IX is not yet in force.
}

%\part{Explanatory Notes}
%
%\renewcommand\parthead{--- Explanatory Notes}
%
%\section{Introduction}
%
%These explanatory notes relate to the Child Support, Pensions and Social Security Act which received Royal Assent on 28th July 2000.  They have been prepared by the Department of Social Security in order to assist the reader in understanding the Act. They do not form part of the Act and have not been endorsed by Parliament.
%
%The notes need to be read in conjunction with the Act. They are not, and are not meant to be, a comprehensive description of the Act. So where a section or part of a section does not seem to require any explanation or comment, none is given.
%
%\subsection{Structure of the notes}
%
%The notes start with a brief overview of the Act as a whole, outlining the different measures and setting them in context. The notes are then divided into five parts that provide the background to the changes and detailed commentary on the sections for each of the areas covered by the Act.
%
%\subsection{Terminology}
%
%At the end of the notes there is a glossary of some of the terms that are referred to in the notes. These are marked in the text with an asterisk. Social security benefits are referred to by their common names and abbreviations (for example, “Jobseeker’s Allowance” or “JSA”, rather than by the terms that appear in the legislation (for example, “a jobseeker’s allowance”).
%
%\subsection{Financial effects}
%
%Unless otherwise indicated, figures are expressed in 1999/2000 price terms.
%
%\section{Overview}
%
%\subsection{Background to the Act}
%
%1. In March 1998, the Government set out its broad welfare reform agenda in the Green Paper entitled \emph{New ambitions for our country: A NEW CONTRACT FOR WELFARE} (Cm 3805). The central principle espoused in the Green Paper was “work for those who can, and security for those who cannot.”
%
%2. Since then, the Government has legislated for a number of reforms in the Welfare Reform and Pensions Act 1999 and has published two main documents of relevance to this Act, to take forward that broad agenda:
%\begin{itemize}
%\item    Green Paper \emph{A new contract for welfare: PARTNERSHIP IN PENSIONS} (Cm 4179), published in December 1998.  Some of the proposals for reform have already been taken forward in the Welfare Reform and Pensions Act 1999.  These include the introduction of stakeholder pensions and changes to occupational and personal pensions.
%
%\item    White Paper \emph{A new contract for welfare: CHILDREN’S RIGHTS AND PARENTS' RESPONSIBILITIES} (Cm 4349), published on 1st July 1999. 
%\end{itemize}
%
%3. In addition, the Government has reviewed the operation of the National Insurance system, the way in which the appeals system for War Pensions operates, enforcement of community punishments, and the powers held by fraud inspectors.
%
%\subsection{The measures in the Act}
%
%4. The main elements in the Act are:
%\begin{description}
%\item[Part I:]
%
%        reform of the child support system.
%
%    \item[Part II:]
%
%        reform of the State Earnings-Related Pension Scheme by way of the State Second Pension;
%
%        measures to extend sharing of pension provision on divorce and to facilitate improved pension information for individuals;
%
%        further reform of the regulation of occupational and personal pensions; and
%
%        measures to extend appeal rights for war pensioners and the introduction of new time limits.
%
%    \item[Part III:]
%
%        measures to withdraw or reduce benefit entitlement where an offender has breached the terms of a community sentence;
%
%        clarification and alignment of the powers of benefit fraud Inspectors which currently differ between different benefits and between DSS and local authority investigators; and
%
%        measures to align the arrangements for decision-making and appeals in Housing Benefit and Council Tax Benefit with those applying to other social security benefits, and to remove the discretion of Local Authorities to recover overpayments of Housing Benefit resulting from tenant fraud from a landlord (other than in cases of collusion) where the landlord has reported the alleged fraud.
%
%    \item[Part IV:]
%
%        aligning the treatment of benefits in kind for employers’ National Insurance Contributions with their treatment for Income Tax purposes;
%
%        facilitate remuneration through share option packages, enhancing employer’s flexibility in rewarding employees.
%\end{description}
%
%\section{Part 1: Child Support.}
%\subsection{Background}
%\subsubsection{The current system}
%
%5. The current system dates from 1993, established by the Child Support Act 1991 (the 1991 Act). In the preceding decade, while the number of children living in lone-parent families increased substantially, the proportion of children receiving maintenance fell – in 1989, 23\% of lone parents claiming Income Support* were receiving maintenance, compared to around 50\% in 1979.  The new child support system was intended to reverse this decline, by providing consistent rules for assessing maintenance liability, and a readily accessible means for collecting and enforcing payment that was due.
%
%6. The 1991 Act set out the structure of a maintenance formula for calculating child support liability. This formula, which took into account the income, housing costs and family responsibilities of both parents, replaced the largely discretionary decisions on maintenance taken by the courts. A system was built up around this formula, administered by the new Child Support Agency (CSA) to ensure the correct calculation of the liability, the collection of maintenance, and enforcement if payment was not forthcoming.
%
%7. Further regulations and the Child Support Act 1995 (the 1995 Act) built upon the 1991 Act. In particular, the 1995 Act introduced the Child Maintenance Bonus, intended as an incentive to encourage parents with care into work, and also introduced the departures scheme which allowed for the assessment of child support liability to take account of exceptional circumstances not recognised in the formula-based assessment.
%
%\subsubsection{The proposals for reform}
%
%8. The Government’s plans for reform of the current system are set out in the White Paper \emph{A new contract for welfare: CHILDREN’S RIGHTS AND PARENTS’ RESPONSIBILITIES} (Cm 4349) published on 1st July 1999.  Proposals were first published in July 1998 in the consultation document \emph{CHILDREN FIRST: a new approach to child support} (Cm 3992). Over 1500 written responses were received which have informed the current plans.
%
%9. The White Paper identified a number of problems with the current system.
%\begin{itemize}
%\item    While the CSA has almost 1.5 million children on its books, only around 300,000 gain financially from child support payments. Of these 300,000, only around 100,000 see the benefit of all the maintenance that is due.
%
%\item    The complexity of the current formula leads to long delays in assessing liability. This in turn makes it difficult to ensure that child support is paid regularly. Because the assessment process is complex, the CSA has less time to help parents understand what they should pay or to chase up non-payment.
%
%\item    Families living on Income Support do not gain from the payment of maintenance as their benefit is reduced by an amount equal to the maintenance paid.
%\end{itemize}
%
%10. The key changes proposed in the White Paper to address these issues were:
%\begin{itemize}
%\item    simplification of the way in which child support liability is assessed by replacing the current complex assessment formula with a simple percentage of the non-resident parent’s net income;
%
%\item    the introduction of a child maintenance premium in Income Support and income-based Jobseeker’s Allowance* that will enable families receiving these benefits to keep up to £10 a week of the maintenance paid for their children. Under current rules, maintenance reduces these benefits pound for pound;
%
%\item    strengthening the sanctions regime by building on existing powers contained in the current scheme, and introducing new measures, to improve compliance;
%
%\item    reform of the CSA’s service to its customers by increasing use of the telephone and face-to-face contact, directing a greater proportion of its resources to making sure maintenance is paid on time and sorting out disputes about liability or payment arrangements as soon as possible and, wherever possible, without the need to refer the dispute to an appeal tribunal.
%\end{itemize}
%
%11. Not all of the proposed reforms require primary legislation. For example, the child maintenance premium can be introduced through amendments to the relevant secondary legislation. And, to implement the reforms, substantial changes to the way that the CSA operates will be necessary, including the introduction of new computer systems.
%
%\subsubsection{The measures in the Act}
%
%12. The provisions in the Act, which replace the existing formula with a simpler system of rates and clarify the responsibilities of parents, cover in particular:
%\begin{itemize}
%\item    the processes for applying for child support and the way in which child support liability will be decided;
%
%\item    the rates used for calculating maintenance liability, based on 15\% of the non-resident parent’s income for one child, 20\% for two and 25\% for three or more;
%
%\item    what is to count as income;
%
%\item    clear penalties for parents who deliberately misrepresent their circumstances to the CSA --- and for those who refuse to provide the information needed to calculate liability and collect maintenance;
%
%\item    the variation of the normal rate of maintenance liability to recognise certain exceptional costs and sources of income;
%
%\item    rights to dispute and appeal decisions on child support liability and the processes by which liability will be kept up to date;
%
%\item    financial and other penalties for late and non-payment;
%
%\item    improvements to the provisions for establishing paternity;
%
%\item    transitional provisions and phasing processes to allow movement from the existing scheme to the new scheme; and
%
%\item    other matters such as:
%\begin{itemize}
%\item        the relationship between court orders for child maintenance and the child support system;
%
%\item        liability for certain non-resident parents living abroad;
%
%\item        the introduction of fees for child support services when these reach an acceptable standard; and
%
%\item        benefit penalties when a parent with care on Income Support opts out of child support arrangements without good cause.
%\end{itemize}
%\end{itemize}
%
%\subsection{Commentary on Sections}
%
%\subsubsection{Maintenance calculations and interim and default maintenance decisions}
%
%\paragraph{Section 1: Maintenance calculations and terminology}
%
%13. A central part of the Government’s reform of the child support system is a new way of working out child support liability. In place of the existing formula, which includes a wide range of income and expenses in the assessment, will be a simpler system of rates, based solely on:
%\begin{itemize}
%\item    the non-resident parent’s* net income (taking into account a restricted range of potential income sources); and
%
%\item    the number of children for whom the non-resident parent is responsible.
%\end{itemize}
%
%14. It is intended that the maintenance calculation should be based on one of three rates: a basic rate, a reduced rate or a flat rate.
%\begin{itemize}
%\item    The general rule is that there should be a \emph{basic rate} of liability based on a percentage of the non-resident parent's net weekly income. The percentages applied will depend on whether the non-resident parent is liable to pay maintenance for one, two or three or more children. Where the non-resident parent is also responsible for children living in his household (referred to as “relevant other children”), the basic rate is calculated by applying the percentages to the non-resident parent's net weekly income after this has been reduced to take account of the number of relevant other children.
%
%\item    A \emph{reduced rate}, which will be payable where the non-resident parent's net weekly income is more than £100 but less than £200. 
%
%\item    A \emph{flat rate}, which will be payable if the non-resident parent’s net weekly income is £100 or less or he receives a prescribed benefit, pension or allowance, or he or his partner receive any prescribed benefit. It is also intended that certain categories of non-resident parent (including those with a net weekly income below £5) will have a nil rate of liability.
%\end{itemize}
%
%15. The new calculation rules allow for apportionment of the liability where there is more than one person with care*. The provisions for apportionment of liability are in Schedule 1, paragraph 6.  The rates can be modified in shared care cases, that is, if the non-resident parent has any child for whom he is liable to pay child support living with him for at least one night a week. Provision for shared care is set out in Schedule 1, paragraphs 7 to 9.  Maintenance liability can also be varied in exceptional cases – the provisions for varying the liability are in sections 5 to 7 of this Act.
%
%16. Sections 4 and 7 of the 1991 Act provide that persons with care, non-resident parents and, in Scotland, qualifying children, can apply for a maintenance calculation. Section 6 of that Act (substituted by section 3) provides that parents with care who claim Income Support or income-based Jobseeker’s Allowance can be treated as having applied for a maintenance calculation. It is the duty of the Secretary of State to reach a decision on any application for which he has jurisdiction.
%
%17. Sections 4(10)  and 44 of the 1991 Act provide circumstances in which there is no jurisdiction. These provisions are amended by sections 2 and 22 of this Act.
%
%18. Section 1 provides the basis for the maintenance calculation and the rates that will be used to determine maintenance liability.
%
%19. Subsection (1)  substitutes a new section 11 (dealing with the rules for maintenance calculations) in the 1991 Act.
%
%\paragraph{New section 11: Maintenance calculations}
%
%20. This section places a duty on the Secretary of State to make a maintenance calculation. It provides, for the purposes of revision, supersession and appeal, that the outcome of the calculation is a decision about whether child support maintenance is payable.
%
%21. New section 11(1)  and (2)  provide that the Secretary of State shall deal with an application for a maintenance calculation in accordance with the Act and make a decision about whether, and how much, child maintenance is payable, or decide not to make a calculation, or make a decision under section 12. 
%
%22. New section 11(3)  to (5)  allow the Secretary of State to stop acting on an application treated as made under section 6(3)  if the parent with care ceases to fall within section 6(1) . (Section 6(3)  of the 1991 Act is substituted by section 3 of this Act. It provides that a parent with care who claims or receives Income Support or income-based Jobseeker’s Allowance can be treated as having applied for a maintenance calculation). However, if the parent with care still wants to apply for a maintenance calculation (in other words she wants it treated as though she has applied under section 4 of the 1991 Act) and there is no court order or pre-1993 written maintenance agreement* in place preventing this, then she has one month to respond to the letter telling her that the Secretary of State intends to stop acting. Where the parent with care is content for the Secretary of State to stop acting on her application, but the non-resident parent has already been contacted, then the non-resident parent must be notified. If the parent with care is herself prevented from applying under section 4 then she must be notified of this. These provisions mirror subsections (1A), (1B) and (1C) of the existing section 11. 
%
%23. New section 11(6)  provides that the amount of a maintenance calculation shall be fixed by the rates set out in Part I of Schedule 1. 
%
%24. New section 11(7)  provides for maintenance where a variation has been agreed to. New section 11(8)  refers to Part II of Schedule 1. 
%
%25. Subsection (2)  of section 1 amends the 1991 Act to replace existing child support terminology, where it appears in the 1991 Act, with the new terminology to be used in the reformed child support scheme.
%\begin{itemize}
%\item    \emph{Maintenance calculation} will replace maintenance assessment.
%
% \item   \emph{Calculation} will replace assessment wherever it occurs in connection with an assessment of maintenance.
%\end{itemize}
%
%26. Subsection (3)  introduces a new Part I of Schedule 1 to the 1991 Act.
%
%\paragraph{Schedule 1: Calculation of weekly amount of child support maintenance}
%
%27. This Schedule replaces Part I of Schedule 1 to the 1991 Act with a new provision that sets out the way that the weekly amount of child support maintenance will be calculated.
%
%\subparagraph{Paragraph 1: General rule}
%
%28. This paragraph provides the foundation on which child support liability is based.
%\begin{description}
%\item [Sub-paragraph (1) ] provides for child support liability to be calculated at the basic rate except where rules provide that a reduced rate, flat rate or nil rate liability is appropriate. These terms are explained in paragraphs 2, 3, 4 and 5. 
%
%\item[Sub-paragraph (2) ] provides for the amount payable to the parent with care to be the amount calculated using the appropriate (applicable) rate or, where there is more than one parent with care, a proportion of that amount (see paragraph 6) in either case, reduced as necessary where the non-resident parent shares the care of a qualifying child* (see paragraphs 7 to 9).
%\end{description}
%
%\subparagraph{Paragraph 2: Basic rate}
%
%29. This paragraph provides the rules for determining the basic rate for child support liability.
%
%\begin{description}
%\item[Sub-paragraph (1) ] provides for the basic rate to be calculated at a set percentage of the non-resident parent’s net income. Where the non-resident parent is liable for maintenance for one qualifying child, the basic rate is 15\%. For two children, the basic rate is 20\% and for three or more children, the basic rate is 25\%.
%
%\item[Sub-paragraph (2) ] provides for the non-resident parent’s net income to be reduced by 15\%, 20\% or 25\% before the provisions of sub-paragraph (1)  are applied, where he has one, two or three or more children living with him (relevant other children).
%\end{description}
%
%\subparagraph{Paragraph 3: Reduced rate}
%
%30. This paragraph provides the rules for determining which non-resident parents will have a liability calculated at the reduced rate.
%\begin{description}
%\item[Sub-paragraph (1) ] provides that a reduced rate is payable where the non-\hspace{0pt}resident parent’s net income is less than £200 but more than £100, unless the non-resident parent is liable for the nil rate or flat rate of liability.
%
%\item[Sub-paragraph (2) ] provides for the reduced rate (or its method of calculation) to be prescribed in regulations.  The intention is that regulations will provide for percentages to be applied to net income so that liability increases in proportion to the amount by which net income exceeds £100.
%
%{\itshape Example: Neil has one qualifying child.  His earnings are £150 and his liability is £18.   When his earnings increase to £170, his liability increases to £23, in proportion to the amount by which his earnings exceed £100. }
%\end{description}
%
%\subparagraph{Paragraph 4: Flat rate}
%
%31. This paragraph provides the rules for determining which non-resident parents will have a flat rate liability.
%\begin{description}
%\item[Sub-paragraph (1) ] provides for the flat rate to apply where the non-resident parent has net weekly income of £100 or less; or he is in receipt of a prescribed social security benefit, pension or allowance, or he or his partner (if any) is in receipt of prescribed benefits.  The flat rate is not payable in a case where a non-resident parent has a nil liability (see paragraph 5).
%
%\item[Sub-paragraph (2) ] provides for the flat rate to be payable at a different amount where the non-resident parent has one or more partners who are also liable to pay child support, and either the non-resident parent or his partner is in receipt of a prescribed benefit (intended to be Income Support or income-based Jobseeker’s Allowance).  For example, in a case where both members of a couple in receipt of Income Support are non-resident parents, it is intended to provide for the non-resident parent’s liability to be one half of the flat rate amount.
%
%\item[Sub-paragraph (3) ] provides for the prescribed social security benefits, pensions and allowances in sub-paragraph (1)($b$) to include those paid to non-resident parents under the law of countries other than those in the United Kingdom, for example a state retirement pension paid to an EU national.
%\end{description}
%
%\subparagraph{Paragraph 5: Nil rate}
%
%32. This paragraph provides that the non-resident parent will be liable for a nil rate where he has a net income of below £5 or is of a prescribed description. It is intended to prescribe full time students in advanced education and prisoners among the categories for this purpose.
%
%\subparagraph{Paragraph 6: Apportionment}
%
%33. The provisions of this paragraph deal with cases where there is more than one person with care and more than one qualifying child in respect of the same non-resident parent. In these circumstances, the maintenance liability of the non-resident parent will be apportioned between the persons with care. The non-resident parent’s liability is divided by the number of qualifying children and then shared between the parents with care in proportion to the number of qualifying children in each family.
%
%{\itshape Example: David has three qualifying children, one being cared for by Dawn and two being cared for by Rebecca.  Dawn would receive one-third of David’s maintenance liability, whilst Rebecca would receive two-thirds.}
%
%\subparagraph{Paragraph 7: Shared care – basic and reduced rate}
%
%34. The provisions of this paragraph set out the rules for adjusting maintenance liability where the non-resident parent shares the care of a qualifying child (see paragraph 15) and the maintenance liability is calculated at the basic or reduced rate.
%\begin{description}
%\item[Sub-paragraph (1) ] restricts a reduction in liability for shared care under this paragraph to maintenance payable at the basic or reduced rate.  Paragraph 8 (see below) provides for the effect of shared care on a flat-rate liability.
%
%\item[Sub-paragraphs (2)  to (4) ] provide that where the non-resident parent has overnight care of the child for at least 52 nights in total during a prescribed 12 month period, the amount payable to the parent with care of that child will be decreased by one-seventh for care on 52 to 103 nights, two sevenths for care on 104 to 155 nights, three-sevenths for care on 156 to 174 nights and one-half for care on 175 or more nights.  Where a period of 12 months is not available, paragraph 9 allows the Secretary of State to make regulations giving him flexibility to use a period other than 12 months.
%
%\item[Sub-paragraph (5) ] provides that where the parent with care is caring for more than one qualifying child of the same non-resident parent then the reduction will be the sum of the relevant fractions divided by the number of such qualifying children.  For example, where the non-resident parent shares the care of two children, one for an average of one night a week, and the other for an average of two nights a week, his liability is reduced by $\frac{3}{14}$ths.
%
%\item[Sub-paragraph (6) ] provides for the maintenance liability to be reduced by a further £7 for each qualifying child for whom care is equally shared.
%
%\item[Sub-paragraph (7) ] restricts the amount by which the provisions of this paragraph can reduce liability so that the non-resident parent cannot have a liability of less than £5.   Where there is more than one person with care and more than one qualifying child in respect of the same non-resident parent, this sum will be apportioned between the persons with care in accordance with the provisions of paragraph 6. 
%\end{description}
%
%\subparagraph{Paragraph 8: Shared care – flat rate}
%
%35. The provisions of this paragraph apply where the non-resident parent has a flat rate liability because he is in receipt of a prescribed social security benefit, pension or allowance or he or his partner are in receipt of prescribed benefits or he or his partner receive prescribed benefits and both are non-resident parents.
%\begin{description}
%\item[Sub-paragraph (2) ] provides that where a non-resident parent shares the care of a qualifying child for at least 52 nights in total during a prescribed 12 month period, his liability to the parent with care of that child will be nil.  However, his liability to any other parent with care to whom he is liable to pay maintenance will remain.
%
%{\itshape Example: a non-resident parent in receipt of income-based Jobseeker's Allowance has two children living with different parents with care.  His flat rate liability is £5 and this is apportioned amongst the parents with care at £2.50 each.  The non-resident parent shares the care of a child in the first parent with care’s household and therefore pays nothing to that parent with care.  However, his contact with the child living with the second parent with care is limited and does not meet “shared care” criteria.  He therefore pays £2.50 to the second parent with care.}
%\end{description}
%
%\subparagraph{Paragraph 9: Regulations about shared care}
%
%36. The provisions of this paragraph allow the Secretary of State to use regulations to set the parameters of what counts as shared care. Regulations will provide for what nights count for this purpose, what counts as care for these purposes and the use of periods other than 12 months to set the reduction for shared care.
%
%\subparagraph{Paragraph 10: Net weekly income}
%
%37. This paragraph enables the Secretary of State to specify in regulations the items to be taken into account in calculating the net weekly income of the non-resident parent. The intention is to take account of income tax, National Insurance contributions* and contributions to an Inland Revenue approved pension scheme. Sub-paragraph (2)  allows the Secretary of State to estimate a non-resident parent’s income, or make an assumption as to any fact, if he feels that the information he has is incomplete or not truly representative. Sub-paragraph (3)  sets a limit of £2,000 on the amount of net weekly income that is to be used in the calculation of maintenance.
%
%\subparagraph{Paragraph 10A: Regulations about rates, figures, etc}
%
%38. This paragraph provides a regulation-making power enabling the Secretary of State to adapt the percentages and amounts used to set the maintenance rates and to revise the number of nights and fractions used to adjust the maintenance calculation where care of a child is shared. It also provides that the level of earnings above which child maintenance is not payable can be revised.
%
%\subparagraph{Paragraph 10B: Regulations about income}
%
%39. This paragraph provides the Secretary of State with further regulation-making powers to enable him to define what will and will not count as income. For example, where the Secretary of State is satisfied that a non-resident parent has intentionally deprived himself of income, by, for example, working for a relative and not being paid, the regulation-making power will allow him to include that income for the purposes of calculating his maintenance liability.
%
%\subparagraph{Paragraph 10C: References to various terms}
%
%40. The provisions of this paragraph set the definitions of various terms used in this Schedule.
%\begin{description}
%    \item[Qualifying children:] Children who are living apart from one or both parents and for whom the maintenance calculation falls to be made.
%
%\item    [Relevant other children:] Children in respect of whom either the non-resident parent or his partner receives Child Benefit*, or in respect of whom certain other prescribed conditions are met.  For example, it is intended to prescribe for a child to be a relevant other child where child benefit entitlement conditions are not yet met because the child has not been resident in the United Kingdom for more than 26 weeks.
%
%\item    [A person “receives”] a benefit, pension or allowance for any week for which it is paid or due to be paid.
%
%\item    [A person’s partner:] the other member of a couple. Or, in the case of a marriage under a law which permits polygamy, another party to the marriage who is of the opposite sex and is a member of the same household.
%
%\item    [A couple:] a man and woman who are married and members of the same household or not married and living together as husband and wife.
%\end{description}
%
%\paragraph{Section 2: Applications under section 4 of the Child Support Act 1991}
%
%41. The White Paper \emph{Children Come First}, published in 1990, stated that the current child support system would be available to all parents. However, it was recognised that a staged programme of implementation would be needed. Priority would be given to those who needed child support most. The take-on of applications from parents who had existing maintenance arrangements was deferred and the jurisdiction of the Child Support Agency (CSA) in cases where either parent or the child was living abroad was specifically denied. (Section 44 of the 1991 Act currently excludes cases where the parents are not habitually resident in the United Kingdom from the CSA’s jurisdiction.)
%
%42. The phased take-on of applications from parents with existing maintenance agreements was set out in regulations (SI 1993/966). By 1995 it was clear that the CSA was not in a position to take on such a high volume of cases and an amendment was introduced. Section 4(10) , inserted by the 1995 Act, deferred applications for child support for an indefinite period where, for example, there was a written maintenance agreement in force made before 5th April 1993, or there was any maintenance order. These cases would continue to be subject to the jurisdiction of the courts.
%
%43. The term “maintenance order” is defined in section 8(11)  of the 1991 Act as an order requiring periodical payments to, or for the benefit of, a child under specified legislation. Written maintenance agreements, which are registered in Scotland in the Book of Sessions, are also treated as maintenance orders.
%
%44. This section provides for the Secretary of State to accept applications from parents who have a maintenance order made after the reforms are introduced provided that the order has been in force for at least a year. Parents with maintenance orders in force at the time that the reforms are introduced – and those with written maintenance agreements made before April 1993 – will, as now, use the courts for enforcement and variation of child maintenance liability.
%
%45. The section amends section 4(10)  of the 1991 Act, which prevents the CSA from accepting applications from parents with maintenance orders.
%
%46. Subsection (2)  amends this exclusion to refer only to maintenance orders made before a prescribed date. The Government intends to prescribe the date that the reforms come into effect for this purpose. Subsection (3)  adds a new exclusion to cover maintenance orders made after the prescribed date if they have not been in force for at least a year.
%
%47. The Government intends to use its powers to prescribe the effective dates of maintenance calculations in paragraph 11 of Schedule 1 to the 1991 Act to set the effective date of any liability resulting from an application covered by the new section 4(10) ($aa$) to two months after the date of application. This will allow both parents time to consider whether they wish to renegotiate the maintenance order before child support liability begins. The effective date of any liability is the date when the court order ceases to have effect and child support is payable.
%
%\paragraph{Section 3: Applications by persons claiming or receiving benefit}
%
%48. Currently a parent with care on Income Support or income-based Jobseeker's Allowance or any other prescribed benefit can be required to authorise the Secretary of State to take action to recover child support maintenance. She is not required to do so if there are reasonable grounds for believing that if she did there would be a risk to her, or any child living with her, suffering harm or undue distress. This is known as “good cause” for not claiming child maintenance.
%
%49. Under the reformed child support system, parents with care who get Income Support, income-based Jobseeker's Allowance or other prescribed benefits will be treated as having applied for child support, unless they specifically request the Secretary of State not to recover child support maintenance. Where a parent with care asks the Secretary of State not to pursue maintenance the Secretary of State will decide if she has good cause not to do so and if he concludes she does not, her benefit will be reduced by a prescribed amount, currently 40\% of the adult personal allowance. This benefit penalty will apply until she asks the Secretary of State to pursue maintenance or shows good cause, as now. The current definition of good cause will be retained.
%
%50. Section 19 of this Act substitutes section 46 of the 1991 Act, which provides for a benefit penalty for parents with care to whom section 6 applies, if they refuse, without good cause, to agree to child support action for their children. While the current section 6 includes a consideration of good cause before requiring the parent to apply for child support, the amendments in section 3 will allow parents to opt out of the child support process under any circumstances. The consideration of whether there is good cause for opting out will form part of the consideration of a benefit penalty, and so is placed in a substituted section 46 in section 19. 
%
%51. The intention is that, having been told by the parent with care that she wishes to opt out, the Secretary of State will ask for her reasons. The parent will have four weeks to provide reasons. If at the end of this time, it is accepted that there are reasonable grounds for believing that pursuing child support would cause harm or undue distress to the parent with care or her children, no further action will be taken. If the Secretary of State decides that there is no good cause, he will impose a benefit penalty.
%
%52. Section 3 provides that parents with care on Income Support or income-based Jobseeker's Allowance (or other prescribed benefits) will be treated as applying for child support unless they choose to opt out. This section substitutes a new section 6 for section 6 of the 1991 Act, under which the parent with care is treated as applying for child support. Section 19 substitutes section 46 of the 1991 Act in relation to failure to comply with obligations imposed by section 6. 
%
%\paragraph{New section 6: Applications by those claiming or receiving benefit}
%
%53. New section 6(1)  to (3)  provide that a parent with care who claims or who is receiving Income Support or income-based Jobseeker's Allowance may be treated as having applied for child maintenance. Subsection (1)  contains a power to prescribe other benefits for the purpose of this section; for example, should another income-related benefit be introduced in the future.
%
%54. New section 6(4)  requires the Secretary of State to notify the parent with care of this, of her ability under subsection (5)  to request him not to act, and of the power to impose a reduced benefit decision under section 46. 
%
%55. New section 6(6)  sets out that this will apply whether or not she receives the benefit in respect of that qualifying child. This provision is contained in the current section 6 (see subsection (8) ). A parent with care can claim benefit for herself and the qualifying child, but benefit for the child will not be awarded in circumstances where the child has earnings, a trust fund or settlement, or capital of more than £3000. 
%
%56. New section 6(7)  follows closely the wording in the current subsection (9) . It requires the parent with care to provide the Secretary of State with the information to enable him to identify or trace the non-resident parent so that a child support maintenance calculation can be made and payments collected. She is not required to comply with this section if she has asked the Secretary of State not to pursue child maintenance.
%
%57. New section 6(8)  provides a power to make regulations specifying the circumstances in which the requirement to supply information does not apply or will be waived. This carries forward a power (which has not been used) in the current section 6.  This power is retained because it may provide protection for parents with care in as yet unforeseen circumstances.
%
%58. New section 6(9)  allows a parent with care who is no longer entitled to a benefit to which this section relates to stop child support action. It makes it clear that, until the parent indicates that she wants child support to cease, the Secretary of State may continue to pursue maintenance.
%
%59. New section 6(10)  of the substituted section 6 requires the Secretary of State to comply with a request under subsection (9)  to cease acting. Regulations under new section 6(11)  can provide for the detail of how this will happen.
%
%60. New section 6(12)  reflects the current section 6(14) . It provides that the provisions in this section will apply even when there is a maintenance calculation already in force. For example, in situations where there is a change in the parent with care’s circumstances, and she claims good cause or makes a new application.
%
%\paragraph{Section 4: Default and interim maintenance decisions}
%
%61. There will be circumstances in which a final maintenance calculation cannot be made straightaway, for example, when sufficient details are not made available, or need to be verified. The reformed scheme will allow for maintenance to be collected:
%\begin{itemize}
%\item    where the information needed to complete a maintenance calculation (apart from information needed in relation to a variation application) is not immediately available, using a default rate; and
%
%\item    where the calculation cannot be completed because an application for a variation is outstanding, using an interim rate.
%\end{itemize}
%
%62. The system of default rates will allow the CSA to get maintenance flowing quickly where there is no information about the non-resident parent's current earnings. It is intended that these will be set at 15\%, 20\% or 25\% of average non-resident parent’s weekly earnings (currently around £200) according to the number of qualifying children.
%
%63. When the information needed to complete a proper assessment is provided, the default rate will be replaced by a new maintenance calculation. For non-co-operative non-resident parents, maintenance liability for the past will only be recalculated if the full rate is higher than the default maintenance rate. This will both provide an incentive to those non-resident parents to provide information quickly and avoid creating overpayments which have to be recovered from the parent with care.
%
%64. The interim rate will be set at the same level as the normal maintenance calculation pending a decision on the variation application. If a variation is allowed, the interim rate will be replaced, with retrospective effect, by the new rate of maintenance liability resulting from the variation.
%
%65. This section substitutes a new section 12 of the 1991 Act which provides for decisions to set liability at a default or interim rate. The section provides the power to make regulations which will define the way that these decisions are made and subsequently altered.
%
%\paragraph{New section 12: Default and interim maintenance decisions}
%
%66. New section 12(1)  provides for a default maintenance decision that will establish a maintenance liability calculated in accordance with regulations made under subsections (4)  and (5). This decision may be made where there is insufficient information (apart from information needed in relation to a variation application) to decide maintenance liability.
%
%67. Decisions on maintenance liability are covered by section 11 of the 1991 Act (as substituted by section 1 of this Act) which requires Secretary of State to make a decision on an application for a maintenance calculation, and by sections 16 and 17 of the 1991 Act which provide for the revision and supersession of maintenance decisions.
%
%68. New section 12(2)  provides for interim maintenance decisions in cases where an application for a variation has been made which has not yet been determined. Sections 28A and 28B of the 1991 Act, inserted by section 5 of this Act, provide for applications for a variation and the preliminary consideration of such applications.
%
%69. New section 12(3)  provides that the amount of child support maintenance payable by virtue of an interim maintenance decision will be fixed in accordance with Part I of Schedule 1. 
%
%70. New section 12(4)  and (5)  provide for regulations to define the way that default and interim decisions are made. The Government intends to provide by regulations that default rates will be £30 per week for one qualifying child, £40 for two children and £50 for three or more children.
%
%\subsubsection{Applications for a variation}
%
%\paragraph{Section 5: Departure from usual rules for calculating maintenance}
%
%71. The new child support rates set out in Part I of Schedule 1 to the Child Support Act 1991, substituted by Schedule 1 to this Act, are intended to provide a fair maintenance calculation in the vast majority of cases. Nevertheless, the Government recognises that there will be exceptional cases where the child support rates do not properly reflect a non-resident parent’s ability to support his children. For example, a non-resident parent may need to spend an exceptionally large amount of money keeping in touch with the children, or the net income used in working out his liability may not properly reflect the resources available to him.
%
%72. Accordingly, the Government has decided to allow for the variation (both upwards and downwards) of the rates payable under the replacement scheme in certain exceptional cases. However, the Government is concerned to avoid simply re-introducing the complexity of the existing formula by another route. The exceptional cases in which a variation will be possible will therefore be clearly defined.
%
%73. The structure of the new legislation follows the broad lines of the departures scheme which was introduced by the Child Support Act 1995.  In particular:
%\begin{itemize}\item
%    the consideration of a non-resident parent’s application for a variation may depend on his continuing regular payment of maintenance (section 28C); and
%
%\item    maintenance liability will only be varied if it is just and equitable to do so.
%\end{itemize}
%
%74. However, unlike the provision for departures, an application for a variation may be made before the maintenance calculation has been completed, and the revised legislation is drafted to deal specifically with an application made in these circumstances. Regulations made under section 28G of the 1991 Act, as inserted by section 7 of this Act, will provide the rules for handling an application for a variation made after the maintenance calculation has been completed. Where the maintenance calculation is made without taking account of the variation application, liability will initially be based on an interim maintenance decision. Section 12(2) , substituted by section 4 of this Act, provides for this decision.
%
%75. This section provides the general rules governing the application for a variation before a final maintenance calculation has been made, and how the application is to be considered and decided.
%
%76. Subsection (2)  replaces sections 28A, 28B and 28C of the 1991 Act.
%
%\paragraph{New section 28A: Applications for variation of usual rules for calculating maintenance}
%
%77. This section provides the rules governing applications for a variation. It specifies who can apply, and in what circumstances and in what manner the application can be made. The substituted Schedule 4A, introduced by section 6 of this Act, supplements this section.
%
%78. New section 28A(1), (2)  and (3)  provide that the person with care and the non-resident parent (or, in Scotland, either of them or the child) can each make an application for a variation at any time, once an application for a maintenance calculation has been made and before a maintenance calculation decision under section 11 (a normal calculation) or 12(1)  (a default decision) has been made.
%
%79. New section 28A(4)  and (5)  provide that, unlike applications for departures, applications for a variation need not be in writing unless, exceptionally, the Secretary of State considers this to be appropriate (for example, having regard to the complexity surrounding the case). When making an application, the applicant will be required to state the ground on which they are applying. The Secretary of State may impose other conditions. Where appropriate, he may, for example, require a dedicated application form to have been completed properly before he will accept that an application has been made.
%
%80. New section 28A(6)  cross-references to the substituted Schedule 4A (see section 6 and Schedule 2 of this Act) which provides additional regulation-making powers relating to the handling of variation applications.
%
%\paragraph{New section 28B: Preliminary consideration of applications}
%
%81. This section provides for a preliminary consideration of the application. This is intended to sift out at the earliest possible stage those applications which have no prospect of success.
%
%82. New section 28B(1)  provides that, having received an application, the Secretary of State may carry out a preliminary examination (known as a “preliminary sift”) to check that it merits further consideration.
%
%83. New section 28B(2)  provides, in particular, that an application from any source will be rejected where it has not been made on one or more of the recognised grounds, or where a default maintenance decision (substituted section 12(1)  of the 1991 Act) would be made. A partial list of the criteria which the Secretary of State will consider under the preliminary sift is in substituted Schedule 4B of 1991 Act and the rest will be prescribed in regulations. The intention is to sift out applications from non-resident parents in the circumstances where, for example, at the date from which any variation agreed in response to the application would take effect, they had either a nil liability, or a flat-rate liability, or a liability which has been reduced to the equivalent of the flat rate on account of any shared care adjustments. In these circumstances, the non-resident parents could not benefit from the effect of a variation.
%
%\paragraph{New section 28C: Imposition of regular payments condition}
%
%84. This section provides for the imposition of a regular payments condition. This condition requires a non-resident parent who has made an application for a variation to continue paying maintenance regularly while the application is being considered. This is intended to ensure that children receive maintenance regularly and reliably and that unnecessary debts are not built up during the variation process.
%
%85. New section 28C(1)  provides that, where the Secretary of State has made an interim decision pending the determination of a non-resident parent’s variation application, and has not rejected the application at the preliminary sift stage, he may require the parent in question to make regular, ongoing payments of maintenance as a pre-condition of having the application considered. This is called a “regular payments condition”.
%
%86. New section 28C(2)  provides that the rate may either be at the rate of the existing interim decision or at a lesser rate which might anticipate the effect of a successful variation application.
%
%87. New section 28C(3)  provides that, in these circumstances, the Secretary of State will notify all the persons with care (and child, if the applicant for the maintenance calculation) concerned, and the non-resident parent, of the imposition of the condition and the effect of failing to comply with it.
%
%88. New section 28C(4)  provides that the regular payments condition will cease to have effect either when, in response to the variation application, the Secretary of State replaces his interim maintenance decision with a decision under section 11 (whether he agrees to variation or not) or where the variation application is withdrawn.
%
%89. New section 28C(5) , (6)  and (7)  provide that, if the Secretary of State determines that the non-resident parent has failed to comply with the regular payment condition, the Secretary of State may refuse to consider the variation application and proceed to replace the existing interim decision on the basis that the variation application has failed. Regulations will provide for deciding what constitutes a “regular payment”. For example, there will need to be scope for taking some account of occasions where payment is unavoidably late, for example, where a bank fails to operate a direct debit. It is intended that where the Secretary of State is not satisfied that the regular payments condition has been met, progress on the variation application may be suspended to allow the non-resident parent the further opportunity to comply. If within the period of a further calendar month, he has still failed to do so without good reason, the application will fail. In this event, the Secretary of State will not vary the maintenance calculation and will notify all the persons with care (or child) and the non-resident parent accordingly. In these circumstances, the non-resident parent will have to make a fresh application if he again wishes to have special circumstances considered.
%
%90. Subsections (3)  and (4)  of section 5 of this Act make amendments to the wording of sections 28D and 28E of the 1991 Act (which deal with determination of applications and matters to be taken into account, respectively), substituting references to departure directions with references to variations. With respect to section 28D, the intention is that where the variation application has not failed, been withdrawn, or been rejected at any preliminary stage, the Secretary of State may elect either to determine the application himself or, exceptionally, to refer it direct to the appeal tribunal for determination. This represents no change from the options available to the Secretary of State under the departures scheme. Cases which the Secretary of State might refer to the tribunal are those which are particularly complex or contentious and which he feels unable to resolve.
%
%91. Subsection (5)  substitutes section 28F of the 1991 Act (which relates to the determination of departure applications) with equivalent wording – with some modifications – relating to the determination of applications for variations.
%
%\paragraph{New section 28F: Agreement to a variation}
%
%92. New section 28F(1)  provides that a variation may be allowed only if it has been made on one or more of the recognised grounds, and if, having regard to all the circumstances, it would be just and equitable to allow a variation in any particular case.
%
%93. New section 28F(2)  provides that, in determining whether it would be just and equitable to vary the normal rules in any particular case, the Secretary of State must have regard to the welfare of any child who would be affected by the variation, and such other factors as may be prescribed in regulations. The Secretary of State will need to consider, for example, whether any variation in the amount of child support liability would be likely to result in either parent giving up work.
%
%94. New section 28F(3)  reaffirms that an application from any source will not be agreed to where the Secretary of State has insufficient information to enable him to make a decision as to maintenance liability under section 11 of the 1991 Act, such that he has to make a default decision under section 12(1)  of that Act. The full list of the other circumstances that will automatically prevent agreement to a variation will be prescribed in regulations. In particular, the intention is to disallow applications from any source where a non-resident parent was in receipt of (or was the partner of someone in receipt of) a prescribed income-related benefit at the date from which any variation given in response to the application could take effect.
%
%95. New section 28F(4)  provides that, where the Secretary of State agrees to a variation, he has to determine the basis on which the child support maintenance is to be calculated, and proceed to make a decision under section 11 on that basis.
%
%96. New section 28F(5)  provides that where the Secretary of State has made an interim maintenance decision and subsequently makes a decision under section 11 (whether or not he agrees to a variation), the interim maintenance decision is to be treated as having been replaced by his decision under section 11.  Any appeal which has previously been lodged against the interim decision will lapse, other than in prescribed circumstances. Any outstanding activity under section 16 (revision) or section 17 (supersession) relating to the interim decision itself will be dealt with, as part of the final decision. There will be a right of appeal against the final section 11 decision.
%
%97. New section 28F(6)  requires the Secretary of State to comply with any regulations made under the powers of Part II of the substituted Schedule 4B, which is provided for by section 6 of this Act, in considering whether to agree to a variation.
%
%\paragraph{Section 6: Applications for a variation: further provisions}
%
%98. This section substitutes both Schedule 4A to the 1991 Act (which, among other things, provides additional regulation-making powers relating to the procedural handling of departure applications) and Schedule 4B to the same Act (which specifies the cases in which a departure direction may be given and the regulatory controls which govern the operation of the departures scheme) with the equivalent provisions in relation to variation applications.
%
%\paragraph{Schedule 2}
%
%99. Schedule 2 substitutes Schedules 4A and 4B to the 1991 Act.
%
%\paragraph{New Schedule 4A: Applications for a variation}
%
%100. This Schedule contains detailed provisions supplementing the rules governing applications for variations in section 28A. In particular, it provides for:
%\begin{itemize}
%\item    regulations to specify the procedure to be followed by the Secretary of State or a tribunal in considering an application (paragraph 2);
%
%\item    information to be supplied within a specified period to enable an application for a variation to be determined (paragraph 4);
%
%\item    two or more variation applications to be considered together (paragraph 5(1) );
%
%\item    a tribunal to be able to consider any variation application which has been referred to it for determination under section 28D(1)($b$)  at the same time as any appeal under section 20 connected to an interim maintenance decision (paragraph 5(3) ).
%\end{itemize}
%
%\paragraph{New Schedule 4B: Applications for a variation: the cases and controls}
%
%101. This Schedule details the cases and controls relating to variations.
%
%\subparagraph{Part I: The cases}
%
%\subparagraph{\itshape Paragraph 2: Special expenses}
%
%102. Paragraph 2 relates to the special expenses in respect of which a non-resident parent may apply for a variation of the normal rules by which maintenance liability is calculated. These will be set out in regulations.
%\begin{description}
%\item[Sub-paragraph (2) ] provides that the Secretary of State may have regard either to all or part of the expenses, or, in prescribed cases, only to that element of the expenses which exceeds a prescribed threshold.
%
%\item[Sub-paragraph (3) ] specifies some cases which may be prescribed.  The list is not intended to be exhaustive.  The non-resident parent will be able to seek a variation in recognition of one or more of the following expenses:
%\begin{itemize}
%\item    the costs incurred in keeping in contact with a qualifying child;
%
%\item    the costs attributable to the long-term illness or disability of a relevant other child;
%
%\item    the costs incurred in honouring debts which were incurred at time when both parents were living together and were for the joint benefit of both parents, or for the benefit of the child in respect of whom a maintenance calculation has been applied for (“the child concerned”), or for the benefit of any other child within a prescribed category,
%
%\item    the costs incurred in meeting the boarding school fees payable in respect of the child concerned;
%
%\item    payments of the mortgage on the former home, where the former partner continues to live in the house with a qualifying child, in the circumstances where, exceptionally, the non-resident parent no longer has any interest in the property.
%\end{itemize}
%
%\item[Sub-paragraph (4) ] provides that the definitions of “illness”, “disability” and “long term” will be prescribed in regulations.
%
%\item[Sub-paragraph (5) ] provides that the definition of “boarding school fees” and the elements of the fees that the Secretary of State may recognise, will be prescribed in regulations.  Regulations will also allow the Secretary of State to make an estimate of the fees that he may recognise, in the circumstances where the relevant amounts are not otherwise readily identifiable.
%\end{description}
%
%\subparagraph{Paragraph 3: Property or capital transfers}
%
%103. This is a feature of the departures scheme and the ground rules and calculations remain unchanged.
%
%Sub-paragraph (1)  requires there to have been a property settlement between the parties in pursuance of a court order or maintenance agreement which pre-dates 5 April 1993.   A variation may be made in respect of, for example, the equivalent weekly value of that transfer.
%
%Sub-paragraph (2)  provides that the Secretary of State will continue to take no account of transfers valued at less than a minimum figure.  This figure will, as now, be prescribed in regulations and is intended to remain at £5000. 
%Paragraph 4: Additional cases
%
%104. This paragraph provides for regulations to specify further grounds on which any person with care (or, in Scotland, a child) may apply for a variation of the rules on the calculation of liability.
%
%Sub-paragraph (2)  gives examples of such cases. The list is not intended to be exhaustive.  The person with care (or child) will be able to seek a variation in recognition of one or more of the following grounds: where the non-resident parent has assets which exceed a prescribed value (it is intended to prescribe cash or its equivalent, or property other than his normal place of residence, which exceed in total a value of £65,000); where the non-resident parent enjoys a lifestyle which is inconsistent with the income to which the Secretary of State is able to have regard in the determination of the rate of liability; where the non-resident parent is in receipt of income to which the Secretary of State would not otherwise have had regard (the intention is to prescribe for cases where the non-resident parent has a flat rate liability because he is in receipt of a prescribed social security benefit or war pension, or where he has a nil rate of liability, and has minimum additional income of £100 per week); or where the non-resident parent has unreasonably reduced the income to which the Secretary of State has had regard in the calculation of maintenance liability.
%Part II: Regulatory controls
%
%105. Paragraph 5provides additional regulation-making powers relating to variations.
%
%Sub-paragraphs (1)  and (3)  provide regulation-making powers relating to the manner in which the Secretary of State may modify the normal rules for calculating maintenance in the event of a successful variation application.  The Secretary of State will normally give effect to a variation by offsetting the expenses against, or increasing the value of, the non-resident parent’s net income prior to any further adjustment in respect of relevant children (where appropriate).  The only exception to the normal rules will apply, as now, to pre-1993 property transfers, where the equivalent weekly value of the transfer (as calculated) will be deducted from the non-resident parent’s “bottom line” liability.
%
%Sub-paragraph (2)  provides that no variation may be made other than in the circumstances prescribed.
%
%Sub-paragraphs (4)  and (5)  provide that the Secretary of State may by regulations impose a limit on the amount of special expenses which he may take into account for the purposes of a variation, and that regulations may provide for different provision with respect  to different levels of income.  The intention is that the Secretary of State will recognise expenditure on certain of the prescribed grounds only in so far as it exceeds £10 or £15 per week, depending on the non-resident parent’s net weekly income.
%
%106. Paragraph 6 provides that the Secretary of State may, by regulations, and with prescribed modifications, apply the “shared care” rules and adjustments referred to in paragraph 7 of Part I, Schedule 1 (as substituted by Schedule 1 to this Act) in cases where he has agreed to a variation of the normal rules by which the maintenance liability is calculated.
%Section 7: Variations: revision and supersession
%
%107. This section substitutes section 28G of the Child Support Act 1991.  (The terms “revision” and “supersession” refer to replacing decisions, either from the original date (revision) or a later date (supersession)).
%New section 28G: Variations: Revision and supersession
%
%108. New section 28G(1)  enables variation applications to be made when a maintenance calculation is in force.
%
%109. New section 28G(2)  provides the power by regulations to modify sections 16, 17, 20 and 28A to 28F of, and Schedules 4A and 4B to, the 1991 Act for these variation applications.
%
%110. New section 28G(3)  provides a power by regulations to permit the Secretary of State, when superseding a decision on his own initiative under section 17, to make a decision on the basis of a variation agreed to in respect of an earlier decision. This is because some variation circumstances, such as property transfers, once accepted, will continue to be relevant to liability except in flat rate or nil rate cases.
%Section 8: Revision and supersession of decisions
%
%111. In June 1999, new decision-making and appeals rules were introduced for child support. The intention of these changes was to simplify the decision-making process, to focus decisions on the outcome rather than the process, and to streamline the appeals system. However, in developing the new arrangements, the Government considered that further changes to the 1991 Act were needed to support the new system.
%
%112. In particular, the existing legislation did not always clearly provide for a decision to be made. This in turn made it difficult to frame the rules for the revision and supersession of decisions (in sections 16 and 17 of the 1991 Act) and to indicate clearly the point at issue in providing for a right of appeal (section 20).
%
%113. The new decision-making provisions were introduced by the Social Security Act 1998*. This Act substituted sections 16, 17 and 20 of the 1991 Act as well as introducing a new Schedule 4C which provided for decision-making and appeals in specific cases.
%
%114. The changes to sections 11 and 12 of the 1991 Act (introduced by sections 1 and 4 of this Act) and the new rules for variations in child support liability (sections 5 to 7 of this Act) focus more clearly on the decisions to be made. This in turn enables the revision, supersession and appeals rules to be restructured.
%
%115. This section amends section 16 of the 1991 Act by inserting a subsection (1A) to cover the additional cases of decisions to reduce benefit and decisions of appeal tribunals on variations. This replaces provisions in paragraph 1 of Schedule 4C to the 1991 Act.
%
%116. A reduced benefit decision may be imposed if a parent with care who has claimed or who is receiving Income Support or income-based Jobseeker’s Allowance requests, without good cause, not to be treated as having applied for child support, or fails to provide information or undergo a scientific test – see section 46 of the 1991 Act, as substituted by section 19 of this Act.
%
%117. An appeal tribunal can determine an application for a variation if asked to do so by the Secretary of State. This process is not the same as determining an appeal: there is no provision for the revision of decisions on appeals.
%
%118. This section also inserts a new subsection (1B) into section 16 of the 1991 Act which provides that on revision, a section 12(1)  decision may be treated as if made under section 11 of 1991 Act.
%Section 9: Decisions superseding earlier decisions
%
%119. This section amends section 17 of the 1991 Act to clarify the decisions which may be superseded.
%
%120. Subsections (1)  and (2)  amend section 17(1)  to provide for the supersession of:
%
%    a reduced benefit decision;
%
%    a decision of an appeal tribunal on a variation referral; and
%
%    a decision of a Commissioner on an appeal from a decision referred to in paragraph ($b$)  or ($d$)  (a decision of an appeal tribunal, including a decision on referral of a variation).
%
%121. Subsection (3)  substitutes subsection (4)  of section 17 of the 1991 Act with two new subsections (4)  and (4A)  which provide for the date from which a supersession takes effect. The existing section 17 provides that a supersession takes effect from the date of the decision or the date of the application unless otherwise prescribed. Regulations have been made which enable decisions to take effect from the date of the change of circumstances which leads to the supersession (the Child Support (Miscellaneous Amendments) (No. 2) Regulations 1999 (SI 1999/1047)).
%
%122. In child support, decisions normally take effect from the beginning of a maintenance period: this subsection amends section 17 to provide for this. A “maintenance period” represents the weekly unit in which maintenance liability is calculated. The first maintenance period starts on the date that the non-resident parent's liability begins: each subsequent maintenance period starts on the day after the last day of the previous one. Other periods may be prescribed for particular cases.
%Section 10: Appeals to appeal tribunals
%
%123. This section substitutes section 20 of the 1991 Act with a new provision governing the right to appeal child support decisions. Its purpose is to set out clearly the decisions which can be appealed and the circumstances in which an appeal can brought against such decisions. As now, the intention is that decisions that affect child support liability will be appealable. There will also continue to be a right of appeal against a decision to impose a reduced benefit decision. Decisions on fees and fixed penalties will also be appealable.
%New section 20: Appeals to appeal tribunals
%
%124. New section 20(1)  sets out who may appeal and the decisions that they may appeal against. An appeal may be brought by any qualifying person. Subsection (2)  provides a definition of this term. Decisions which can be appealed are:
%($a$) 
%
%a decision to make a maintenance calculation (section 11), a default or interim decision (section 12) and a superseding decision (section 17);
%($b$) 
%
%a decision not to make a maintenance calculation or supersede a decision.  The Secretary of State has no jurisdiction to make a maintenance calculation in certain circumstances (such as where the child is living abroad) and decisions which cannot be superseded include certain changes of circumstances (such as housing costs, as these will not be taken into account in the maintenance calculation);
%($c$) 
%
%a reduced benefit decision;
%($d$) 
%
%the imposition of a penalty for late payment of maintenance and the amount of the penalty; and
%($e$) 
%
%the requirement to pay fees.
%
%125. New section 20(2)  provides the definition of “qualifying person” for the purpose of subsection (1)  of this section. A qualifying person is:
%($a$) 
%
%either the person with care and the non-resident parent;
%($b$) 
%
%a child in Scotland who made the application for a maintenance calculation which led to the decision;
%($c$) 
%
%the parent with care affected by the decision to reduce benefit;
%($d$) 
%
%the parent required to make penalty payments; or
%($e$) 
%
%the person required to pay fees.
%
%126. New section 20(3)  provides that anyone with a right of appeal against a decision or imposition of a requirement must be told of this right.
%
%127. New sections 20(4)  and (5)  provide for regulations to specify how, and within what time, an appeal must be brought. As now, it is intended that there will be a one-month time limit for bringing an appeal, which can be extended at the tribunal's discretion if there was good cause for failing to appeal sooner.
%
%128. New section 20(6)  provides that the time to appeal against a decision to reduce benefit runs from the date that the reduction in benefit is notified.
%
%129. New section 20(7)  provides that the tribunal cannot consider changes in circumstances which happened after the date of the decision and need not look at any issue not raised when the decision was made. This is the same as for social security benefit appeals.
%
%130. New section 20(8)  provides for the way that a tribunal can decide the appeal if it is allowed. The tribunal can either:
%($a$) 
%
%decide the appeal itself, or
%($b$) 
%
%send the decision back to the CSA with directions as to how a new decision must be made.  This provision is needed because the tribunal will often not have all the information or computer support necessary to make a new maintenance calculation.
%Section 11: Redetermination of appeals
%
%131. The 1998 Social Security Act (the 1998 Act) introduced a new system of decision-making and appeals in child support and social security. This Act replaced the existing structure of appeal tribunals, including child support appeal tribunals, with a unified tribunal system. The legislation governing child support appeals remains separate, however, with section 20 of the 1991 Act providing the basic legislative framework.
%
%132. In social security legislation, the provisions governing appeal rights are supplemented by a provision allowing tribunals to redetermine appeals when an appeal to a Commissioner against the appeal decision has been sought. Section 13 of the 1998 Act allows a tribunal to set the decision aside if all the parties to the appeal agree that the decision is wrong in law. The appeal then goes to another tribunal to be considered again. This means that Commissioners do not have to deal with uncontested appeals.
%
%133. This provision was not carried over into child support legislation. This section corrects this omission by introducing a new section 23A in the 1991 Act. This new section mirrors section 13 of the 1998 Act.
%New section 23A: Redetermination of appeals
%
%134. This section provides for the setting aside of appeal tribunal decisions and the reconsideration of the appeal by the tribunal. It sets out the circumstances in which this can happen and the procedure to be followed.
%
%135. New section 23A(1)  provides that the section applies when there is an application for leave to appeal to a Commissioner from a decision of a tribunal on a question of law.
%
%136. New section 23A(2)  allows the person who constituted the tribunal, or otherwise a tribunal chairman to set aside the tribunal decision if he decides it was wrong on a point of law. He can then either refer it for redetermination by the same tribunal or a different one.
%
%137. New section 23A(3)  provides that a tribunal decision shall be set aside if each of the principal parties accepts that it was wrong in law. Such a case is to be referred for determination by a different tribunal.
%
%138. New section 23A(4)  defines the “principal parties” to an appeal. They are the Secretary of State, and the qualifying persons referred to in section 20(2)  of the 1991 Act (section 10). The qualifying persons are the person with care and the non-resident parent. And where the application for a maintenance calculation has been made under section 7 of the 1991 Act (a child in Scotland) the person with care, the principal parties are the non-resident parent and the child concerned. In the case of an appeal relating to financial penalties or fees, they are the person liable to make payment and in the case of a reduced benefit direction they are the person in respect of whom the benefit is payable.
%Information
%Section 12: Information required by the Secretary of State
%
%139. The power to request information in section 14(1)  of the 1991 Act is currently phrased in terms of information or evidence needed to determine an application, or a question arising in connection with an application, or needed in connection with collection or enforcement of maintenance.
%
%140. The ability to request information should not be limited to the initial decision regarding maintenance liability, but should apply in connection with any decision to be made under the Act, as well as in connection with collection and enforcement of child support or other maintenance.
%
%141. This section amends section 14 of the 1991 Act to allow the Secretary of State to require any information which he may need to make any decision or impose any condition or requirement under the Act.
%Section 13: Information – offences
%
%142. The current child support scheme can be thwarted by parents who fail to produce the information required to make a child maintenance assessment. Parents may also provide false information, which can result in an incorrect assessment of liability. When the current scheme was developed, this problem was to be addressed by applying punitive interim maintenance assessments to uncooperative non-resident parents. This sanction has proved ineffective because it is practically impossible to enforce a punitive interim maintenance assessment. The lack of effective sanctions was highlighted in the Benefit Fraud Inspectorate’s report on the Child Support Agency.
%
%143. The White Paper A new contract for welfare: CHILDREN’S RIGHTS AND PARENTS’ RESPONSIBILITIES made it clear that, for the new scheme, the Government intended to ensure that parents who sought to avoid their child support responsibilities would face effective penalties. In particular the White Paper proposed a new penalty for parents who lied to the Child Support Agency or refused to provide information.
%
%144. Section 13 introduces a new section 14A which provides for a fine of up to £1000 for anyone who provides false information or refuses to supply information to the CSA.
%New section 14A: Information – offences
%
%145. New section 14A(1)  provides whom this section applies to.
%
%Subsection (1)($a$)  specifies those who are required to provide the information necessary to trace and, in the case of a maintenance application made under section 4, identify the non-resident parent, and to assess and collect the maintenance liability.
%
%Subsection (1)($b$)  enables the Secretary of State to apply the section to other persons.  It is intended to specify for example, employers of non-resident parents and accountants.
%
%146. New section 14A(2)  introduces an offence of knowingly making a false statement or representation or knowingly providing, or allowing to be provided, information which is false.
%
%147. New section 14A(3)  introduces an offence of failure to provide information when required by the Secretary of State to do so.
%
%148. New section 14A(4)  provides that if a person has a reasonable excuse for failing to comply with the Secretary of State’s request he may use this as a defence.
%
%149. New section 14A(5)  provides that if a person is found guilty of either new offence, he will be subject on conviction to a fine of up to £1000. 
%Section 14: Inspectors
%
%150. As explained in the note on section 13, the process of deciding child support liability and collecting maintenance for children can be delayed if information is not provided. Section 13 provides for a penalty if parents lie or refuse to give information to the Child Support Agency (CSA). In other circumstances, other means of getting information may be appropriate. In some cases, a visit to an employer’s premises, or the premises from which a non-resident parent conducts his business, can yield information that would be difficult to get by other means.
%
%151. The 1991 Act contains a provision to allow child support inspectors to be appointed on a case-by-case basis to carry out visits. The legislation allows for inspectors to enter any premises which are not solely residential, to question anyone they find there and to see any documents. Obstructing an inspector carries a fine of up to £1,000. 
%
%152. In practice, however, inspectors are very rarely used. This is because inspectors cannot be appointed for a reasonable period of time: they have to be separately appointed for each case. This in turn means that the CSA cannot build up a team of trained inspectors to be used as required. Given the substantial training, which is required to make an inspector fully effective, this rule severely limits the usefulness of this provision.
%
%153. Section 14 of this Act amends section 15 of the 1991 Act to provide for inspectors to be appointed in a way that does not tie the appointment to an individual case. The section also restates the powers of inspectors to bring this child support provision in line with the more general provisions for DSS investigators set out in Part III of this Act.
%
%154. Subsection (2)  substitutes subsections (1)  to (4)  of section 15 in the 1991 Act.
%Amended section 15: Inspectors
%
%155. New section 15(1)  allows Secretary of State to appoint inspectors. This provision allows the Secretary of State to set the terms of appointment. It is intended that inspectors will be appointed for fixed periods. Normally, inspectors will work for the CSA, but on occasion other people with special qualifications will be appointed for specific tasks. It is intended, for example, to have reciprocal arrangements with inspectors in local authorities and the Benefits Agency.
%
%156. New section 15(4)  sets out the inspectors’ powers to enter at any reasonable time, either alone or accompanied, the premises defined in subsection (4A)  as being liable to inspection. In these premises, the inspector is empowered to examine and enquire as he thinks appropriate.
%
%157. New section 15(4A)  defines premises liable to inspection for the purpose of subsection (4) . These are any premises other than places used only as a person’s home in which:
%
%    a non-resident parent is working, or where he has been working (as an employee or on a self-employed basis); or
%
%    where another person holds information in a professional capacity about a non-resident parent.
%
%158. Subsection (3)  of section 14 amends subsection (6)  of section 15 to allow inspectors to obtain from the persons named in subsection (5)  (any person aged 18 or over whom the inspector finds on the premises) any information and documents which the inspector reasonably requires.
%
%159. Subsection (4)  inserts subsection (11)  in section 15, which provides that premises include:
%($a$) 
%
%permanent and moveable structures, and, if appropriate, vehicles, boats etc;
%($b$) 
%
%offshore installations such as oil-rigs; and
%($c$) 
%
%all other places occupied on a permanent or temporary basis.
%Parentage
%Section 15: Presumption of parentage in child support cases
%
%160. Most fathers who are non-resident parents acknowledge their children and accept their responsibility to them. In these cases, child support liability can be worked out without any further investigation as to paternity. However, occasionally a man may have good reason to doubt the parent with care’s statement that he is the father of the child in question. And, in some cases, men have contested paternity in order to slow down the process of collecting child maintenance.
%
%161. To allow child support to be worked out without unnecessary delay, the Secretary of State can, in specific circumstances, assume that a man is the father of a child even if he denies it. In these cases, child support liability can only be stopped if the non-resident parent proves in court that he is not in fact the child’s father.
%
%162. In England and Wales, the circumstances in which paternity can be assumed include those where a child was adopted by the man in question and also where there is a court declaration that the man is the child’s father. However, in Scotland, there is also a presumption that a man is the father of a child if he was married to the child’s mother at any time between the date of conception and the child’s birth. This section makes clear that the presumption of paternity arising from marriage, already recognised by the courts in England and Wales, can be applied for child support purposes.
%
%163. A person who is treated as a non-resident parent as a result of these presumptions can challenge his child support liability by applying to court. The provision for such applications is in secondary legislation made under section 45 of the 1991 Act. Section 83 of this Act introduces a new, simplified route specifically for the courts to determine whether or not one person is the parent of another. This will be of general application.
%
%164. This section amends section 26 of the 1991 Act to add four new cases in which child support liability can be worked out on the basis that a person who denies he or she is a parent is in fact the parent of the qualifying child.
%
%    Case A1 allows the Secretary of State to presume that a man is the father of a child living in England and Wales if the man was married to the child’s mother at any time between the date of conception and the child’s birth. This follows the existing presumption in Scottish law.
%
%    Case A2 provides a presumption that a man who is named on the child’s birth certificate is the child’s father even if he was not married to the mother. This will apply also to children registered in Northern Ireland or in Scotland.
%
%    Case A3 enables the Secretary of State to presume parentage if either:
%
%        the alleged parent has refused to take a DNA test; or
%
%        the result of a DNA test shows that he is a parent of the child but he refuses to accept it.
%
%    Case B1 provides that the alleged parent may be presumed to be the parent of the child where section 27 or 28 of the Human Fertilisation and Embryology Act 1990 applies. These sections relate to children born as a result of fertility treatment or artificial insemination. Section 27 provides that a woman who gives birth as a result of such treatment will be treated as the child’s mother unless the child is adopted. Section 28 provides that a man who is married to a woman who has received such treatment (or a man who is himself taking part in the treatment) will normally be the father of the child in law. This provision does not apply if the man did not consent to the treatment, or where the child is adopted.
%
%The following two sections appear in the Act in Part V: Miscellaneous and Supplemental.
%Section 82: Tests for determining parentage
%
%165. Part III of the Family Law Reform Act 1969 enables the court to direct the use of blood tests in order to resolve a dispute about paternity which has arisen in the course of civil proceedings.
%
%166. Regulations under the Act provide that samples may only be taken by a registered medical practitioner, or someone who has been appointed as a tester under the Act. They also prescribe the procedure for the taking of samples, set conditions for the secure despatch of samples to a tester, and prescribe the fees payable to samplers.
%
%167. Blood testing under the Act is carried out by authorised testers who are appointed by the Lord Chancellor. There is no regulation of the laboratory conditions and standards under which testers work, or the frequency with which they undertake the work. Once a person is appointed as a tester, there is no mechanism to review his or her suitability.
%
%168. This section replaces the present system of approving individual paternity testers by one based on the accreditation of laboratories. This will allow the Lord Chancellor to regulate laboratory conditions and set minimum qualifications for the individual testers.
%
%169. This section also amends the legislation to address the issue raised by a High Court judgment (re O and re J(Minors)(Blood Tests: Constraint) [2000] 2 W.L.R. 1284). Section 21(3)  of the 1969 Act provides that a blood sample may be taken from a person under the age of 16 if the person who has care and control of the child consents. In the judgement, it was held that the effect of this provision is that the court has no power to enforce a direction for the taking from a child under 16 of a blood sample to establish paternity, if the person with care of the child refuses to consent to the sample being taken.
%
%170. Subsection (2)  amends section 20 of the 1969 Act to provide for tests to be carried out by a body which has been accredited either by the Lord Chancellor or by a body appointed by him for that purpose.
%
%171. Subsection (3)  amends section 21(3)  of the 1969 Act to provide that, where the person with care and control of a child under 16 does not consent to the taking of a blood sample, the sample may be taken if the court considers that it would be in the best interests of the child to do so.
%
%172. Subsection (4)  amends section 22 of the 1969 Act, which sets out procedural matters on which the Lord Chancellor may make regulations, in two respects. First, an amendment replaces the current requirement that samples be taken by appointed individual medical practitioners with a provision enabling samples to be taken by registered medical practitioners or members of such professional bodies as may be prescribed by the regulations. Secondly, an amendment enables the Lord Chancellor to prescribe conditions which a body must meet to be eligible for accreditation.
%
%173. Subsection (5)  provides that neither this section nor anything else in the Act will affect proceedings to determine declarations of parentage which are pending when these provisions take effect.
%
%174. The Government also intends to bring section 23 of the Family Law Reform Act 1987 into force by commencement order in conjunction with these new provisions. Section 23 amended the 1969 Act to allow for other bodily samples as well as blood to be taken from the categories of people specified in the 1969 Act (the child, the mother and the putative father) and from any other party to the proceedings, to resolve a dispute about parentage.
%Section 83: Declarations of status
%
%175. Anyone who is living in England or Wales (or who has been habitually resident there for at least a year) can seek a declaration from the High Court or a county court that:
%
%    a named person is or was his parent; or
%
%    he is legitimate; or
%
%    he has or has not become a legitimated person (that is, in the same position as being born legitimate).
%
%176. Declarations may be sought, for example, to acquire nationality or citizenship, to establish rights of inheritance or to amend a birth certification. Only the “child” in question (who may in fact be an adult) is entitled to apply for such a declaration. Both the child’s parents, if they are still alive, must be joined as respondents to the proceedings.
%
%177. Section 56 of the Family Law Reform Act 1986 (as substituted by section 22 of the Family Law Reform Act 1987) provides for the declaration. The Family Proceedings Rules provide that both parents must be respondents.
%
%178. A “section 56” declaration is binding on the Crown and on all other persons. The declaration is without limit in time in the UK whether for the purpose of legal proceedings or for any other purpose. The legislation makes no provision for a declaration from the court that a named person is not the child’s parent.
%
%179. This section replaces part of the existing section 56 of the Family Law Act 1986, and also amends sections 58 and 60 of that Act. The section provides for any person to apply to a civil court for a declaration as to whether or not a person named in the application is or was the parent of another person so named. The intention of the new section is to provide a single procedure for obtaining a declaration of parentage to replace the two free-standing provisions contained in the 1986 Act and in section 27 of the Child Support Act 1991 (which is modified to take account of the new procedure), and to widen the power to make such declarations.
%
%180. Subsection (2)  inserts a new section 55A in the 1986 Act which allows for an application for a declaration that a person is or is not the parent of another person.
%New section 55A: Declarations of parentage
%
%181. New section 55A(1)  provides that any person may apply for a declaration as to whether or not a person named in the application is or was the parent of another person named in the application. The application may be made to the High Court, a county court or a magistrates’ court.
%
%182. New section 55A(2)  provides that the court can consider such an application only if either of the persons named in the application is domiciled in England and Wales on the date of the application, or has been habitually resident in England and Wales throughout the period of one year ending with that date; or if either of the persons named in the application died before the period of one year ended and was at death domiciled in England and Wales, or had been habitually resident in England and Wales for one year preceding their death.
%
%183. New section 55A(3)  and (4)  will enable any person to apply for a declaration of parentage, subject to the requirement that if the applicant is not the child or one of the alleged parents concerned, then he or she will have to show a sufficient personal interest in the determination of the application. If the court is not satisfied that this is the case, it must refuse to hear the application. A person with care must automatically be treated as having a sufficient personal interest. This requirement does not apply when the application is made by the Secretary of State.
%
%184. New section 55A(5)  provides that the court may refuse to determine an application where one of the persons named in it is a child, and it considers that to determine the matter would not be in the best interests of the child.
%
%185. New section 55A(6)  provides that where the court has refused to hear an application, it may order that the applicant requires the leave of the court to apply again for the same declaration.
%
%186. New section 55A(7)  provides for notification of a declaration of parentage to the Registrar General.
%
%187. Subsection (3)  of section 83 removes the provision of section 58 of the 1986 Act that no declaration may be made by any court that any person is or was illegitimate. This is because the effect of a declaration of parentage could be that a child is or was illegitimate, which is inconsistent with the existing provision of section 58(5)($b$) .
%
%188. Subsection (4)  provides for a right of appeal from the magistrates’ court to the High Court. This is in addition to the right of appeal to the High Court by way of case stated, that is, on the basis that the decision is wrong in law or magistrates have acted in excess of jurisdiction; or to apply to the High Court for leave to apply for judicial review. This right of appeal must be conferred expressly. Appeals to the Court of Appeal from the High Court and the county courts are governed by existing provisions in the Supreme Court Act 1981 and the County Courts Act 1984. 
%
%189. Subsection (5)  introduces Schedule 8 which provides for consequential amendments.
%
%190. Subsection (6)  provides that neither this section nor anything else in the Act will affect proceedings about declarations of parentage which are pending when these provisions take effect.
%Disqualification from driving
%Section 16: Disqualification from driving
%
%191. Currently section 40 of the Child Support Act 1991, which applies only in England and Wales, enables the Secretary of State to apply to a magistrates’ court for the issue of a warrant committing a non-resident parent to prison where distress action, garnishee proceedings or a charging order have failed to recover some, or all, of the child support maintenance outstanding.
%
%192. If the court is satisfied that there has been wilful refusal or culpable neglect, it may issue a warrant for committal to prison for a maximum period of six weeks, or suspend the sentence. It has previously been held that the term “wilful refusal or culpable neglect” means that the conduct of the non-resident parent must amount to deliberate defiance or reckless disregard. The non-resident parent may be released from prison on payment of the amount stated on the warrant or have the period reduced for part payment.
%
%193. This section provides for a disqualification order to be made in relation to holding or obtaining a driving licence as an alternative to committal. Subsections (2)  and (3)  amend section 40 (the provision for committal) and insert a new section 40B (the further provision of disqualification from driving).
%
%194. Subsection (1)  inserts a new section 39A in the 1991 Act.
%New section 39A: Commitment to prison and disqualification from driving
%
%195. New section 39A(1)  provides that this section applies where the Secretary of State has tried to obtain the amount outstanding by distress or enforcement through the county or sheriff courts.
%
%196. New section 39A(2)  provides for the courts to be able to consider either committal or disqualification from driving.
%
%197. New section 39A(3)  provides for the courts to consider:
%
%    whether a driving licence is needed by the liable person to earn a living;
%
%    the financial circumstances of the liable person; and
%
%    whether there has been wilful refusal or culpable neglect.
%
%198. New section 39A(4)  provides for the Secretary of State and the liable person to make representations to court on which penalty should be imposed.
%
%199. New section 39A(5)  defines “driving licence”.
%
%200. New section 39A(6)  defines “court”.
%
%201. Subsection (2)  of section 16 amends section 40 of the 1991 Act which provides for committal by omitting subsections (1)  and (2)  which set out the present powers of the court and what must be considered. These matters are now covered by the new section 39A(1)  and (3)  above.
%
%202. Subsection (3)  provides for a new section 40B to be inserted before section 41. 
%New section 40B: Disqualification from driving: further provision
%
%203. New section 40B(1)  provides a power for the court to disqualify the liable person from driving if the courts agree that he has wilfully refused to pay or been guilty of culpable neglect in connection with paying maintenance.
%
%(1)($a$)  provides for the disqualification order to apply for a period not exceeding two years.
%
%(1)($b$)  provides that the disqualification order may be suspended.
%
%204. New section 40B(2)  provides that the courts cannot make both a disqualification order and warrant for committal at the same time.
%
%205. New section 40B(3)  provides that the order should include the amount of the arrears included in the liability order and the court costs.
%
%206. New section 40B(4)  provides for the courts to require the liable person to produce his driving licence (defined in section 108(1)  of the Road Traffic Act 1988).
%
%207. New section 40B(5)  provides that the courts may lift the order, or substitute a shorter disqualification period, if part of the amount outstanding is paid, and must revoke the disqualification if payment is made in full before the end of the disqualification period.
%
%208. New section 40B(6)  provides for the Secretary of State to be able to give his views to the court on the amount that should be paid before the disqualification order is lifted. It also provides for the liable person to reply to the representations.
%
%209. New section 40B(7)  provides for a further application to be made to the courts if any amount remains outstanding at the end of the disqualification period.
%
%210. New section 40B(8)  provides for the court, on imposing the disqualification, to notify the Secretary of State of the fact that a disqualification order has been made, amended or lifted, and the new section 40B(9)  provides that a licence produced to the court should be sent to the Secretary of State. In practice, the notice and the licence will be sent to the DVLA.
%
%211. New section 40B(10)  provides for section 80 of the Magistrates Court Act 1980 to apply to a disqualification order, to reflect provisions currently in section 40.  This will enable a liable person to be searched in court and money found applied against the amount owing.
%
%212. New section 40B(11)  provides for regulations to be made, prescribing the way in which disqualification orders will operate, and new section 40B(12)  modifies this section in its application to Scotland.
%
%213. Subsections (4)  and (5)  of section 16 provide for references to the disqualification to be made in the Road Traffic Act 1988 and the Road Traffic Offenders Act 1988.  This will enable the police to require production of the licence if it is not given to the courts. Failure to produce the licence in these circumstances is a criminal offence punishable by a fine of up to £1,000. 
%Section 17: Civil Imprisonment: Scotland
%
%214. Subsection (1)  provides that section 40 does not apply to Scotland.
%
%215. Subsection (2)  inserts new section 40A into the 1991 Act. The new section provides the procedure for the sheriff to follow if he is satisfied that it is appropriate to commit a liable person to prison. The new section is comparable in its scope to section 40 of the 1991 Act for England and Wales (see the introduction to section 16).
%New section 40A: Commitment to Prison – Scotland
%
%216. New section 40A(1)  provides that where the sheriff is satisfied that the liable person has wilfully refused or culpably neglected to pay, the sheriff may issue a warrant for his committal to prison or may fix a term of imprisonment but postpone the committal of the liable person to prison. The sheriff may impose conditions on the postponement, for example, that the liable person makes regular payments of maintenance.
%
%217. New section 40A(2)  provides that the warrant which the sheriff issues will be in respect of the arrears of maintenance and the Secretary of State’s expenses in raising the proceedings for committal to prison. The warrant must state what the total amount is.
%
%218. New section 40A(3)  prohibits a warrant being issued in respect of a person who is under 18 years of age.
%
%219. New section 40A(4)  provides that the warrant will order the imprisonment of the liable person for a specified period but that he may be released on payment of the amount stated in the warrant – unless he is in custody for some other reason.
%
%220. New section 40A(5)  provides that the maximum period of imprisonment is 6 weeks.
%
%221. New section 40A(6)  gives the Secretary of State power in regulations to provide for the period of imprisonment to be reduced where the outstanding amount has been partly paid.
%
%222. New section 40A(7)  provides that the warrant may be directed to such person as the sheriff thinks fit.
%
%223. New section 40A(8)  gives the Court of Session power to make subordinate legislation regulating practice and procedure in the Sheriff Court in relation to civil imprisonment for child support purposes. The power will be exercised through Rules of Court made in Acts of Sederunt. The Court of Session will have power to make provision:
%
%    about the form of any warrant issued;
%
%    with respect to renewing applications where no warrant is issued or no term of imprisonment has been fixed;
%
%    that an employer’s statement about a liable person’s wages is to be sufficient evidence of the amount of those wages;
%
%    for the sheriff citing a liable person to appear before him for the purposes of an inquiry into the liable person’s behaviour and means; and, if the liable person does not obey the citation, for the sheriff issuing a citation for him to appear before the sheriff and, if necessary, warrant for his arrest;
%
%    for the sheriff issuing a warrant for the liable person’s arrest, without issuing a citation, for the purposes of enabling the inquiry into his means and conduct to be heard;
%
%    as to the execution of the warrant of arrest.
%
%Financial penalties
%Section 18: Financial Penalties
%
%224. Early attempts to implement interest charges on arrears of child support maintenance were abandoned from April 1995.  The calculations were complex and difficult to explain to clients. An alternative provision to interest was introduced by the Child Support Act 1995 but did not come into force. Neither of these provisions will have effect in the new scheme.
%
%225. Instead, a simpler, discretionary financial penalty will be introduced. The intention is for the Secretary of State to have discretion to impose a financial penalty of up to 25 per cent of the amount owed. This will be levied for each week in which payment was not made, but will not be compounded. The charge will not be child support maintenance but will be an administrative penalty payable to the Department of Social Security in recognition of the additional work involved in pursuing late or non-payment and will be paid into the Consolidated Fund.
%
%226. It is intended that the penalty will not be imposed if a missed payment is paid within a reasonable period, or the payment was missed for good reason, such as sickness, or acceptable arrangements are made to pay the missing amount and to continue to pay over an agreed period.
%
%227. It is envisaged that the penalty will rarely need to be applied, but that it will provide a useful incentive for persuading non-resident parents to meet their responsibilities.
%
%228. This section makes an amendment to section 41 and replaces section 41A of the 1991 Act. It removes the provisions on charging interest and inserts new provisions for financial penalties to be charged.
%
%229. Subsection (1)  amends section 41 of the 1991 Act to remove the charging of interest on arrears. Transitional provisions will allow the Secretary of State to continue to collect and enforce interest charges already imposed.
%
%230. Subsection (2)  substitutes section 41A of the 1991 Act with a new provision on financial penalties.
%New section 41A: Penalty payments
%
%231. New section 41A(1)  provides for regulations that allow the Secretary of State to require a non-resident parent who is late in paying child support maintenance to make a penalty payment. Regulations will further provide the way in which penalties are calculated.
%
%232. New section 41A(2)  makes the amount of a penalty payment discretionary but limits the amount to be charged to 25 per cent of the amount due for that week.
%
%233. New section 41A(3)  provides that the amount of the child support maintenance arrears remains due even when a financial penalty has been imposed. The financial penalty is not child support maintenance and is not passed on to the parent with care.
%
%234. New section 41A(4)  provides for regulations to:
%($a$) 
%
%state at what point in time a financial penalty becomes payable; and
%($b$) 
%
%allow all or part of the penalty to be waived at the discretion of the Secretary of State. This will depend on reasons given for late or non-payment and the level of co-operation in paying the arrears.
%
%235. New section 41A(5)  allows regulations on collection and enforcement to apply to penalty payments in the same way as they do to child maintenance payments. Therefore the Secretary of State will have exactly the same powers to collect and enforce penalty payments and may combine this action with action to collect and enforce child maintenance.
%
%236. New section 41A(6)  provides that any payment collected must be paid into the Consolidated Fund and is therefore not paid over to the parent with care.
%Section 19: Reduced benefit decisions
%
%237. This section replaces section 46 of the Child Support Act 1991. 
%New section 46: Reduced benefit decisions
%
%238. New section 46(1)  applies where a parent with care has asked the Secretary of State not to pursue child maintenance, or failed to provide information or refused to take a scientific test such as a DNA test. For example, where the parent with care fears violence from the non-resident parent if he were to be pursued for maintenance.
%
%239. New section 46(2)  enables the Secretary of State to require the parent with care to provide reasons why she has “good cause” either to ask the Secretary of State not to act under section 6, or to fail to give information as required by section 6, or to refuse to take a scientific test. When a parent with care is in receipt of a benefit referred to in, or prescribed for, the purposes of section 6(1)  and asks the Secretary of State not to act, or refuses to take a test, the parent with care will be interviewed. If she is unsure whether she wants to ask the Secretary of State not to act she will be given a specified period to make her decision and give her reasons.
%
%240. New section 46(3)  provides that when the specified period has expired the Secretary of State must make a decision, based on the information provided by the parent with care, on whether there are reasonable grounds for believing that she or her child(ren) would be at a risk of harm or undue distress as a consequence of the Secretary of State recovering child support maintenance from the non-resident parent, insisting on the provision of information or if she were to agree to a scientific test. The term “reduced benefit decision” will replace the term “reduced benefit direction” in the existing Act.
%
%241. “Specified” is defined in section 46(10) , which gives power to prescribe a period. The Government intends to prescribe four weeks, from the date when the parent with care is given notice asking for her reasons under section 46(2).
%
%242. New section 46(4)  provides that if the Secretary of State considers that there are reasonable grounds for believing that the parent with care or her child would be at risk of harm or undue distress, then he is to take no further action under section 46, and that she will be notified of this.
%
%243. New section 46(5)  to (10)  set out the same provisions as section 46 of the 1991 Child Support Act, but substitutes some of the existing terminology. For example, reduced benefit direction in section 46 is changed to reduced benefit decision under this legislation.
%
%244. New section 46(6)  enables the Secretary of State to require the parent to state whether she still does not wish him to act under section 6(3)  and to give her reasons.
%Miscellaneous
%Section 20: Voluntary payments
%
%245. Liability to pay child support usually begins on the day that the non-resident parent is told about the application for a maintenance calculation. However, there will usually be some delay between this date and the date that a maintenance calculation is completed. This means that arrears of maintenance build before parents know how much they should be paying. Voluntary payments made during this period can reduce the debt and provide financial support for the children while child maintenance is being worked out.
%
%246. However, at present voluntary payments are not defined and have no statutory status. The CSA follows policy guidelines in determining which payments can be set off against arrears of maintenance. The Government considers that the use of the discretion is not providing sufficient reassurance to parents that all cases are being treated in the same way. This in turn provides a disincentive to make payments for the children before the maintenance calculation is completed. The Government therefore proposes to give statutory recognition to voluntary payments.
%
%247. This section gives statutory recognition to voluntary payments by inserting a new section in the 1991 Act which establishes clearly the range of payments to be covered and allowing such payments to be offset against child support arrears and current maintenance. Subsections (2)  and (3)  of this section amend the provision for repayments of overpaid child support to cover the voluntary payments that exceed any child maintenance due.
%
%248. Subsection (1)  inserts a new section 28J in the 1991 Act.
%New section 28J: Voluntary payments
%
%249. New section 28J(1)  provides that this section applies where: a person has made an application for a maintenance calculation, or is treated as having made an application, under section 6 of the 1991 Act; the application has not yet been determined; and the non-resident parent actually makes a voluntary payment.
%
%250. Section 6 of the 1991 Act is substituted by section 3 of this Act. It provides that a parent with care who claims or receives Income Support or income-based Jobseeker’s Allowance can be treated as having applied for a maintenance calculation.
%
%251. New section 28J(2)  defines the term “voluntary payment” as:
%
%    subsection (2)($a$) : a payment on account of child maintenance which the non-resident parent expects to pay.  The payment may be based on an estimate provided to him by the Secretary of State or based on an amount he has worked out for himself as being due; and
%
%    subsection (2)($b$) : a payment which is made before the actual calculation has been notified, or the application for maintenance determined.
%
%252. New section 28J(3)  provides for regulations that will set out circumstances in which voluntary payments can be taken into account.
%
%Subsection (3)($a$)  provides for voluntary payments to be offset against the arrears which have built up before the non-resident parent was notified of the calculation.
%
%Subsection (3)($b$)  provides for the balance to be offset against future liability, to the extent that the voluntary payments exceed any outstanding debt.
%
%253. New section 28J(4)  provides for conditions to be set regarding payments and to whom they can be paid. It allows for voluntary payments to be made via the CSA, direct to the parent with care, or another specified party.
%
%254. New section 28J(5)  provides a general power for regulations about voluntary payments and, in particular, about the type of payment that can be accepted.
%
%Subsection (5)($a$)  provides for regulations to specify which payments are, and which are not, to be treated as a voluntary payment.  This will relate to all payments whether they are paid to the parent with care or any other party.  It is intended that as well as cash payments, any payment that is made for food, shelter and warmth will normally be taken into account.  However, payments in kind, that is, where the non-resident parent spends money on other items for the child, will not be taken into account.
%
%Subsection (5)($b$)  provides for regulations to specify the extent and the circumstances in which these payments can be taken into account once it is accepted that the payment is of the right type to be counted as a voluntary payment.
%
%255. Subsections (2)  to (4)  of section 20 amend section 41B of the 1991 Act which provides for the lump-sum repayment to the non-resident parent of maintenance that he has overpaid. This provision takes effect when the overpayment cannot be repaid in a reasonable time by offsetting it against future child support liability. The intention of this amendment is to treat overpayments of voluntary payments in the same way as overpayments of child support maintenance.
%
%256. Subsection (3)  provides for a new subsection to be inserted after subsection (1)  of section 41B, which allows the provisions of section 41B to apply where a voluntary payment has been made, and:
%
%    it subsequently turns out that there is no maintenance liability due at all; or
%
%    the non-resident parent has paid more, in voluntary payments, than the total of arrears.
%
%257. Subsection (4)  substitutes subsection (7)  of section 41B. The substituted subsection (7)  will provide that a payment is to be treated as being an overpayment of child support maintenance made by a non-resident parent where:
%
%    a payment was made but the maintenance calculation turns out not to be valid (for example where a person who believes himself to be the non-resident parent turns out not to be the non-resident parent or where the CSA do not have jurisdiction over a case); and
%
%    a voluntary payment has been made but there is no liability to pay child maintenance.
%
%Section 21: Recovery of child support maintenance by deduction from benefit
%
%258. In the current scheme, where a non-resident parent has no assessable income because he is in receipt of an income-related benefit, a contribution to maintenance can be deducted in certain circumstances. However, there are many exempt categories and, in practice, only around 23,000 non-resident parents make a contribution to maintenance.
%
%259. In the reformed scheme most non-resident parents in receipt of certain prescribed benefits, including income-related benefits and war pensions, will be liable to pay a minimum amount of maintenance (£5) a week. The vast majority of exemptions will be removed.
%
%260. In addition, the parent with care may make an application for a variation against a non-resident parent in receipt of certain prescribed benefits where he has, for example, earnings, an occupational pension or assets. Provision is therefore made to deduct from benefit the amount of child maintenance determined in these cases. Where arrears of maintenance have accrued, an amount may also be deducted from benefit.
%
%261. This section substitutes section 43 of the 1991 Act (contribution to maintenance by deduction from benefit) with a new section on the recovery of maintenance by deduction from benefits. It increases the range of benefits from which deductions can be made in respect of current maintenance and arrears, and includes deductions from war pensions.
%New section 43 – recovery of child support maintenance by deduction from benefit
%
%262. New section 43(1)  states that the section applies where the non-resident parent is liable to pay a flat rate of child support maintenance because he (or his partner) is receiving one of a range of prescribed benefits or a war pension. This subsection also allows regulations to prescribe additional conditions which may have to be satisfied before a deduction can be made.
%
%263. New section 43(2)  is an enabling provision which allows maintenance or arrears to be deducted from benefits, by means of regulations under subsection (1)(p) of section 5 of the Social Security Administration Act 1992*).
%
%264. New section 43(3)  provides that, for the purposes of making deductions from benefit, a war pension is to be included as a benefit.
%Section 22: Child Support jurisdiction
%
%265. This section amends section 44 of the 1991 Act to extend child support jurisdiction to non-resident parents who are not habitually resident in the United Kingdom but who are employed by a UK-based employer. This will mean that certain non-resident parents who are employed abroad will be required to pay child support for their children who live in the United Kingdom.
%
%266. Subsection (2)  amends section 44(1)  of 1991 Act (which limits child support jurisdiction) to refer to a new subsection (2A) , inserted by subsection (3) .
%
%267. Subsection (3)  adds a new subsection (2A)  which lists the cases where, even though the non-resident parent is living abroad, the CSA will have jurisdiction to calculate and collect maintenance. These cases cover people employed abroad:
%($a$) 
%
%in the civil service;
%($b$) 
%
%in the armed services;
%($c$) 
%
%by a UK-based company, the description of which will be prescribed in regulations, or
%($d$) 
%
%by a body prescribed in regulations.  These regulations are intended to be used to cover employment comparable to those listed in this subsection which are subsequently identified.
%
%268. Subsection (4)  removes the provision in subsection (3)  of section 44 to cancel a maintenance assessment when there is no longer jurisdiction to make an assessment. Cancellations in these circumstances are to become supersession decisions in the new scheme, provisions for which are in section 17 of the 1991 Act, as amended by section 9 of this Act.
%Section 23: Abolition of the child maintenance bonus
%
%269. The child maintenance bonus is a lump sum payment of up to £1,000 which can be paid to a parent with care who has been receiving Income Support (or income-based Jobseeker’s Allowance) when she leaves benefit to take up work. The payment is based on the amount of maintenance paid for the parent with care’s children during her time on benefit: it accrues at up to £5 for each week in which maintenance is paid. This allows families to see some gain from maintenance payments which reduce benefit entitlement pound for pound. The bonus is also intended as a work incentive.
%
%270. In practice, relatively few lone parents gain from the child maintenance bonus. Around 1,000 payments are made each month.
%
%271. Under the reformed scheme, the Government intends to replace the child maintenance bonus by a child maintenance premium, which will allow all families on Income Support or income-based Jobseeker’s Allowance to keep up to £10 per week of any child maintenance paid. When a parent with care transfers to the new scheme and so becomes entitled to the child maintenance premium, she will no longer be able to receive a child maintenance bonus.
%
%272. This Act contains no provision for the child maintenance premium. Existing legislation which governs Income Support and income-based Jobseeker’s Allowance already allows for regulations to provide that income can be disregarded.
%
%273. This section repeals the legislation governing the child maintenance bonus. Regulations will bring the child maintenance premium into effect for parents with care with an existing child support assessment when they are transferred to the new scheme.
%Section 24: Periodical reviews.
%
%274. When the 1991 Act was passed by Parliament, it included a provision for the periodical review of child support assessments. When an assessment had been in force for a prescribed period (initially a year, subsequently extended to two years) the Secretary of State was required by this provision to write to both parents to find out if their circumstances had changed. When all the information needed to make an assessment had been checked, a new assessment of child support liability would be made.
%
%275. In practice, this process proved difficult to operate. Parents, many of whom had been unwilling to co-operate in making the first assessment, failed to reply to requests for further information. Others were unable to provide all the information which the Secretary of State required. Since it was impossible to clear the periodical review without this information, substantial backlogs of work built up. The problem became even worse as cases where a review was stalled became due for another review.
%
%276. In June 1999, the decision-making and appeal processes in CSA were improved and streamlined. Section 16 of the 1991 Act, which provided for periodical reviews, was replaced by a provision for revision of decisions. However, transitional provisions ensured that outstanding periodical reviews could still be completed.
%
%277. There are still some 350,000 periodical reviews outstanding. The CSA has made it clear that it will complete any review where either parent requests this. There is, however, little sign that parents want past periodical reviews completed. The effect of these reviews is difficult to predict – some will increase liability, thus creating substantial debts for the non-resident parent, while others reduce liability, creating overpayments which have to be recovered from the parent with care.
%
%278. This section removes the requirement on the CSA to complete outstanding periodic reviews. This provision will come into effect when the Act receives Royal Assent.
%Section 25: Regulations
%
%279. Section 52 of the Child Support Act 1991 provides for the Parliamentary control of regulations and orders made under this Act. Many of the delegated powers in the 1991 Act require a resolution of both Houses before any regulations made under them can come into effect (the “affirmative procedure”).
%
%280. This section amends section 52 to provide for Parliamentary control of regulations made under new child support delegated powers in this Act.
%
%281. The substitutedsubsection (2)  alters the list of regulation making powers which follow the affirmative procedure.
%
%282. This subsection also amends the reference to Part I of Schedule 1 to refer specifically to the new paragraph 3(2)  (regulations prescribing how the reduced rate of liability is worked out) and 10A(1)  (regulations amending the way that liability is worked out) – see commentary on Schedule 1 above.
%
%283. New subsection (2A)  in section 52 provides that the first set of regulations under paragraph 10(1)  of Schedule 1 to the 1991 Act (regulations defining net weekly income for the maintenance calculation) will follow the affirmative procedure. Subsequent regulations will follow the negative procedure.
%Section 26: Amendments
%
%284. This section introduces Schedule 3 which makes minor and consequential amendments to the 1991 and 1995 Child Support Acts and a number of other Acts.
%Schedule 3
%
%285. Paragraphs 1 to 10 and 14 provide for changes to other Acts (ie non-child support and Social Security Acts) covering England, Wales and Scotland, to reflect changes in this Act. These Acts make reference to child maintenance, and in all of them a change is being made in terminology to refer to “maintenance calculation” in place of “maintenance assessment”.
%
%286. The Army Act 1955 and The Air Force Act 1955 allow for the deduction from pay in respect of a wife or child to such extent as is specified in the Order of Council. They set out how child maintenance will be deducted from a serviceman’s pay and are amended for child support purposes. It is being amended to reflect changes in terminology in this Act.
%
%287. The Matrimonial Causes Act 1973 provides for the duration of continuing financial provision orders in favour of children and the age limit on making such orders. Where the court has made an order, it may vary or discharge it, as well as suspend or revive any provision in the order. It is being amended to reflect changes in terminology in this Act.
%
%288. The Domestic Proceedings and Magistrates Courts Act 1978 sets out the age at which responsibility for financial provisions in favour of children ceases, and the duration of such orders. It is being amended to reflect changes in terminology in this Act.
%
%289. The Family Law (Scotland) Act 1985 is amended to reflect changes in terminology provided by section 1(2)  of this Act.
%
%290. The Insolvency Act 1986 sets out the effects of discharging a bankruptcy on various monies owed. It is amended to reflect changes in terminology in this Act.
%
%291. The Debtors (Scotland) Act 1987 is amended to provide that when a liable person is sequestrated in Scotland, it will not be possible to use a Deduction from Earnings Order under the 1991 Act to enforce a child support maintenance calculation (previously a maintenance assessment). The 1987 Act is also amended to define “maintenance order”, and to reflect the changes in terminology provided by section 1(2)  of this Act.
%
%292. The Income and Corporation Taxes Act 1988 sets out which Social Security benefits shall and shall not be charged to income tax and what payments shall not be treated as income. It is being amended to reflect changes in terminology in this Act. It is also being amended to reflect the repeal of section 24 of the Child Support Act 1995. 
%
%293. Sections 36 and 38 of The Finance Act 1988 are amended to reflect changes in terminology in this Act.
%
%294. Schedule 1 of The Children Act 1989 is amended to reflect changes in terminology provided by section 1(2)  of this Act.
%
%295. The Prisoners Earnings Act (paragraph 14) sets out the powers of the prison Governor to make deductions and impose levies on prisoners who are paid “enhanced” wages or “net weekly earnings”, including for child support purposes. It is being amended to reflect changes in terminology in this Act.
%
%296. Paragraph 11 amends the Child Support Act 1991. 
%
%Sub-paragraph (2)  replaces the term “absent parent” in the 1991 Act with the term “non-resident parent”.
%
%Sub-paragraph (3)  amends section 4 of the 1991 Act to make it clear that the information that must be provided when applying for child support includes information which enables the non-resident parent to be identified.
%
%Sub-paragraph (4)  amends section 7 of the 1991 Act, which currently gives a child in Scotland the right to apply for an assessment.
%
%Sub-paragraph (4)($a$)  will enable a child to apply for a maintenance calculation if no parent has been treated under the new section 6(3)  of the 1991 Act as having applied for a maintenance calculation with respect to the child.
%
%Sub-paragraph (4)($b$), which amends section 7(10) , alters the restrictions on when an application may be made by a qualifying child.  The effect of the amendment is that the child may not apply if there is a maintenance order in force in respect of him which was made after a prescribed date, and which has been in force for less than a year.
%
%Sub-paragraph (5)  amends section 8 of the 1991 Act, which governs the role of the courts in dealing with child maintenance.
%
%Sub-paragraph (5)($a$)  amends section 8(1)  in consequence of the changes to section 6. 
%
%Sub-paragraph (5)($b$)  inserts a reference to the new substituted section 8(3A)  in section 8. 
%
%Sub-paragraph (5)($c$)  substitutes subsection 8(3A) .  The new subsection (3A)  allows the courts to vary court orders made after the date that the child support reforms are introduced even where, in accordance with the provisions of section 2 of this Act, the Child Support Agency could accept an application for child support as a result of more than one year having passed after the order was made.  This power, like the court order itself, would cease if a maintenance calculation is made.
%
%Sub-paragraph (5)($d$)  inserts a cross-reference to the “cap” on maintenance provided by paragraph 10(3)  of Schedule 1.   This means that the courts can make top-up orders for child maintenance where the non-resident parent has net weekly income of more than £2,000 per week.
%
%Sub-paragraphs (6)  to (10)  amend sections 9, 14, 26, 27A and 28 respectively, in consequence of the changes to section 6. 
%
%Sub-paragraph (11)  amends section 28ZA of the 1991 Act which covers decisions made under section 11, 12, 16 or 17 and which involves issues that arise on appeal in other cases.  These amendments consolidate provisions that were inserted into the 1991 Act by the Social Security Act 1998. 
%
%Sub-paragraph (11) ($a$)  reproduces part of paragraph 4 of Schedule 4C to the 1991 Act by moving a provision relating to reduced benefit decisions under section 46 to section 28ZA and generalising the reference to decisions in section 28ZA(1) .  Section 28ZA already allows for section 46 decisions to be held back pending resolution of a lead case.
%
%Sub-paragraph (11) ($b$)  consolidates Schedule 4C paragraph 4 by including a reference to appeals that are pending against reduced benefit decisions to section 28ZA.
%
%Sub-paragraph (12)  amends section 28ZB of the 1991 Act which covers appeals made under section 20 of the 1991 Act which involve issues that arise on appeals in other cases.  This consolidates provisions that were inserted into the 1991 Act by the Social Security Act 1998. 
%
%Sub-paragraphs (12) ($a$)  and ($b$)  amend section 28ZB to include reduced benefit direction appeals and appeals against the imposition of fees, partly by moving these provisions from paragraph 5 of Schedule 4C.
%
%Sub-paragraph (13)  amends section 28ZC of the 1991 Act.  Section 28ZC limits the retrospective effects of decisions in certain cases of error.  For example, where an understanding of law has been overturned by a decision on appeal it cannot affect liability or other decisions for a period before the new interpretation was determined.  This consolidates provisions that were inserted into the 1991 Act by the Social Security Act 1998. 
%
%Sub-paragraphs (13)($a$)  to ($e$)  reproduce the effect of existing provisions in paragraph 6 of Schedule 4C of the 1991 Act in section 28ZC
%
%Sub-paragraph (14)  provides that sections 28H and 28I will no longer have effect.  This is because the departures scheme is to be replaced by new provisions for variations in child support liability – see sections 5 to 7. 
%
%Sub-paragraph (15) amends section 30 of 1991 Act.  Section 30 is about collection and enforcement of forms of maintenance other than child maintenance.  It amends this so as to avoid doubts about the meaning.  It clarifies that regulations can be made for both periodical payments and secured periodical payments.
%
%Sub-paragraph (16) amends section 32 of 1991 Act.  Section 32 concerns deductions from earnings orders.  This provision amends section 32 to provide for regulations to say that the non-resident parent will always retain a set percentage of his earnings after a deduction has been made for maintenance.
%
%Sub-paragraph (17) amends section 33 of 1191 Act.  This section concerns liability orders.  The amendment is intended to put beyond doubt that payments of child maintenance can only be classed as having been made if they have been paid to, or through, the person specified in, or in accordance with, regulations.
%
%Sub-paragraph (18) amends section 47 of 1991 Act which relates to fees.  Although section 47 is in force, the ability to charge fees has not been used since April 1995, pending improvements in the CSA’s performance.  The Government intends to consider charging fees again when the new system is running smoothly.  Sub-paragraph (18) inserts a new subsection (4)  in section 47 that enables payments of fees to be recovered in the same way as maintenance, for example, by a deduction from earnings order*.
%
%Sub-paragraph (19) amends section 51 of 1991 Act to reflect the more streamlined decision-making process in the new scheme.  This provision allows for regulations to set out the procedure to be followed in making a maintenance calculation (under section 11) or, when superseding an existing calculation, (under section 17).  It also covers decisions relating to the revision and supersession of maintenance calculations, default rates and interim maintenance decisions.
%
%Sub-paragraph (20) amends section 54 of 1991 Act as regards definitions.  For example, “assessable income” and “departure direction” are omitted, and “voluntary payments” are added.
%
%Sub-paragraph (21) amends section 58(9)  and (10)  of the 1991 Act (the extent provision) so that section 40 does not extend to Scotland but section 40A (as introduced by section 17 of this Act) extends only to Scotland.  Section 40 concerns commitment to prison: section 40A concerns the different provisions which apply in Scotland.
%
%Sub-paragraph (22) amends paragraphs 13, 14 and 16 of Schedule 1 of the 1991 Act which covers general provisions about maintenance assessments.
%
%Sub-paragraph (22)($a$)  repeals paragraph 13 of Schedule 1.   Paragraph 13 enables the Secretary of State to make nil assessments of child support liability.  For example, where the non-resident parent has net income of £5 or less or is a student.  Schedule 1, paragraph 5, now provides for such nil rates of liability in the new scheme.
%
%Sub-paragraph 22($b$)  amends paragraph 14 of Schedule 1 to the 1991 Act.  The purpose is to ensure that the drafting is consistent with other parts of this Act.  It reflects the new section 6 of the 1991 Act, introduced by section 3 of this Act, under which a maintenance application can be “treated as made”.  Paragraph 14 of Schedule 1 allows for two or more maintenance applications for the same child from different parents with care to be treated as the same claim.  This could arise where there is a dispute about who is the parent with care.  Paragraph 14 also allows for the replacement of an earlier maintenance calculation by a later one.
%
%Sub-paragraph (22)($c$)  removes provisions that enable the Secretary of State to cancel assessments in certain circumstances.  For example where there is no longer a person with care or where the parents are back together again.  In the new scheme, such changes will be supersession decisions using section 17 of the 1991 Act.  Schedule 1 paragraph 16(3)  enables the Secretary of State to cancel an assessment where the parent with care is no longer on benefit and requests that he does so.  This provision is being replaced by the new section 6(9)  which is inserted by section 3 of this Act.
%
%297. Paragraph 12 amends the Social Security Administration Act 1992. 
%
%298. Paragraph 13 amends the Child Support Act 1995. 
%
%Sub-paragraph (2)  amends section 18 to remove the delegated power which enabled the Secretary of State to bring parents with court orders into the CSA’s jurisdiction.  This repeal is consequential on the changes to section 4(10)  of the 1991 Act introduced by section 2. 
%
%Sub-paragraph (3)  repeals section 24 of the 1995 Act, which provides for compensation payments in respect of people receiving Family Credit or Disability Working Allowance.
%
%299. Paragraph 15amends the Social Security Act 1998. 
%
%Sub-paragraph (2)  amends Schedule 2 to the Social Security Act 1998 to reflect the terminology of the new child support scheme.  “Reduced benefit directions” will become “reduced benefit decisions”.  Appeals against reduced benefit decisions will, as now, be against the decision to make a reduced benefit decision under section 46 of the 1991 Act  (see sections 10 and 19).
%Section 27: Temporary compensation payment scheme
%
%300. Some arrears of maintenance will normally accrue after the start-date for liability but before a maintenance assessment has been made. However, in recognition of the significant backlogs that developed in the early years of the CSA, the 1995 Child Support White Paper Improving Child Support (Cm 2745) paved the way for the CSA to introduce a scheme which allowed the Agency to agree not to enforce more than six months’ worth of arrears, providing the non-resident parent met his responsibilities for a year. After a year, the Agency makes payments to the parent with care in lieu of those she would have received had the non-resident parent paid in full.
%
%301. The scheme was never intended to become a catch-all for individual cases of delay or maladministration. It was part of a strategy to tackle the backlogs, improve compliance and get the Agency on its feet. The scheme was not therefore translated into primary legislation, and authority for compensation payments was accordingly granted by HM Treasury on a non-statutory basis, with payments approved annually in the Appropriation Act.
%
%302. However, the CSA did not start to clear backlogs to the expected timescales. The scheme was expanded, to include arrears arising from delayed periodic reviews (section 16 of the 1991 Act), and change of circumstance reviews (section 17 of the 1991 Act), because it was accepted that, on balance, the lack of transparency and the complexity of the current system often produced changes in maintenance assessments which were difficult for non-resident parents to predict. Because the scheme was still non-statutory, a condition of the extension was that the Government should seek legislative powers if the arrangements needed to continue further.
%
%303. This section provides a statutory basis for continuing a scheme under which, in certain circumstances, a non-resident parent will not be required to pay the whole of the arrears of maintenance. It is intended that the circumstances should be where significant delay by the CSA has arisen under the current scheme, but only where the non-resident parent gives a commitment to meet his ongoing liabilities and pay the arrears specified in the agreement, and meets this commitment.
%
%304. Subsection (1)  provides the circumstances where this section applies. These are where the effective date of an assessment following an application or a review under the current scheme, before the introduction of the new decision-making provisions in June 1999 following the Social Security Act 1998, means that arrears have built up.
%
%305. Subsection (2)  enables the Secretary of State to apply this section to different cases of arrears from those in subsection (1), and to disapply the section to specified cases in subsection (1) .
%
%306. Subsection (3)  provides the powers for an agreement between the Secretary of State and the non-resident parent, in order for the Secretary of State not to require him to pay, and not to take action to recover, the whole of the arrears in prescribed circumstances.
%
%307. Subsection (4)  provides the Secretary of State with the power to prescribe the terms of the agreement referred to in subsection (3) .
%
%308. Subsection (5)  provides that the section will only apply to agreements made before 1st April 2002 and expiring before 1st April 2003. 
%
%309. Subsection (6)  provides that the Secretary of State has power not to seek to recover the arrears provided the non-resident parent meets the terms of that agreement.
%
%310. Subsection (7)  provides that if the non-resident parent has complied with the agreement, then when it expires the Secretary of State may make payments to the person with care, and the non-resident parent will cease to be liable for the full amount of the arrears of maintenance.
%
%311. Subsection (8)  provides if the non-resident parent defaults under the agreement he becomes liable to pay all the outstanding arrears.
%
%312. Subsection (9)  provides the Secretary of State with the power to regulate for agreements made on or after 1st April 2002.  This is subject to approval by resolution in each House of Parliament. Subsection (10)  defines “prescribed”.
%
%313. Subsections (11)  and (12)  concern the procedure for regulations under this section.
%Section 28: Pilot schemes
%
%314. This section provides a power for pilot schemes to be set up for specific elements of the child support provisions in the Act. This will enable the CSA to test discrete elements of the new scheme on a smaller scale before introducing them nationwide, or test operational provisions for limited periods of time and in limited geographical areas to establish the best way of delivering detailed aspects of the reforms.
%
%315. At present, the Government has no specific plans to pilot any of the provisions. The intention is that all cases taken on by the CSA after the reforms have been implemented will have maintenance liability worked out using the new rules. However, the Government considers it prudent to provide for the option to pilot provisions if and when it appears necessary.
%
%316. Subsection (1)  provides that regulations made under provisions inserted or substituted in the 1991 Act by this Part of this Act, or under the Schedules relating to the Child Support provision, may be made so that they have effect for a specified period up to, but not exceeding, 12 months. Subsection (2)  provides that any regulations made under the provisions of subsection (1)  will be referred to as “a pilot scheme”. Subsection (3)  allows for pilot schemes to have effect in one or more specified areas, to apply to one or more specified classes of person or to people selected by prescribed criteria or on a sampling basis.
%
%317. Subsection (4)  provides for a pilot scheme to be able to make consequential or transitional provision for the way that the pilot scheme would be wound up. Subsection (5)  provides that a pilot scheme can be replaced by a further scheme making the same, or similar provisions.
%
%318. Subsection (6)  provides that any regulations providing for a pilot scheme will need to be approved by a resolution of each House of Parliament (the affirmative procedure).
%Section 29: Interpretation, transitional provisions, savings, etc.
%
%319. The Government has stated that the new scheme will deal with new applications first. Existing cases will be transferred at a later date when the scheme has bedded in and the new rates will be phased in over time. Transitional provisions will therefore be introduced to facilitate the conversion of cases and the phasing of amounts payable.
%
%320. This wide-ranging general power introduces the ability to make regulations which will allow cases to be transferred from the existing scheme to the new scheme. The Act does not provide detail on all aspects of the new scheme or state exactly how it will work. The detail will be set out in regulations.
%
%321. It is intended that provisions will also be introduced to safeguard the way in which aspects of current liability have been calculated, and to ensure that amounts can be carried forward to the new scheme. New child support legislation will have a knock-on effect on other legislation and the ability to make consequential provisions is therefore also introduced in this section.
%
%322. Subsection (2)  provides for regulations to ensure that the new legislation can be brought into being as smoothly as possible. Such regulations may cover the transition to the new scheme, the ability to save any current provisions so that they can continue to be used in the new scheme, amending other legislation which is affected by the new scheme and making any other regulations that may be required.
%
%323. Subsection (3)  provides examples of the regulations that may be introduced. Subsection (3)($a$)  enables regulations to provide for a transitional rate of liability to be payable, including the phasing-in of the amount due when the provisions come into effect. Subsection (3)($b$)  provides that regulations may allow departure directions and any other finding in relation to a previous determination to be taken into account when determining the amount of maintenance payable.
%
%324. Subsection (4)  provides that section 175(3)  and (5)  of the Social Security Contributions and Benefits Act 1992* (the “Contributions and Benefits Act”) applies to the regulation-making power of this section, to allow for different provisions to be made for different cases and for different purposes. It also provides powers for discretion to be exercised in dealing with various matters.
%
%325. Subsections (5)  and (6)  provide that regulations will be made by Statutory Instrument, and subject to the negative procedure.
%Part II: Pensions.
%Background – State Second Pension and Occupational and Personal Pensions
%The current position
%SERPS and contracting-out arrangements
%
%326. The UK pension system combines a contributory state scheme, consisting of basic Retirement Pension* and Additional Pension, derived from the State Earnings-Related Pension Scheme (SERPS), with a private system of occupational and personal pensions*.
%
%327. All employees and self-employed people, except the very lowest paid, pay (or are treated as having paid) National Insurance contributions* (NICs). These give entitlement to the basic state pension. In addition, employees – but not the self-employed – earn entitlement to an additional, second-tier pension (SERPS), unless they choose to “contract-out” of SERPS and join a pension scheme which gives them rights in place of their SERPS entitlement.
%
%328. For those who are, or have been, members of SERPS, the amount of Additional Pension they will receive is based on the amount of their “surplus earnings” in the years since 1978 when SERPS was introduced. Surplus earnings are those between the Lower Earnings Limit* (LEL) and Upper Earnings Limit* (UEL). The surplus earnings are increased in line with average earnings to the year before a person reaches state pension age in order to maintain their value in earnings terms. The total amount is multiplied by an accrual rate of between 25% and 20% depending on the year in which the person reaches state pension age (for anyone reaching state pension age from 6 April 2009 onwards, the accrual rate is 20%). This amount is then divided by the number of years in the person’s working life since 1978 (up to a maximum of 49 for those who reach state pension age from 2027) to give the annual amount of that person’s Additional Pension derived from SERPS.
%
%329. SERPS is solely an earnings-related scheme, with higher benefits accruing to those who have had higher earnings throughout their working life. Lower earners accrue lower benefits, and it is possible to have worked and paid NICs throughout a working life and still receive a combined basic and Additional Pension which is less than is available through means-tested benefits. Those who have had breaks in their employment history for periods of caring or disability can be similarly affected.
%
%330. Employers may choose whether or not to provide an occupational pension scheme and whether to contract-out of SERPS. For those who are contracted-out of SERPS, a NI rebate is given in recognition of the fact that there is a reduced liability on the state.
%
%331. A person contracted-out of SERPS may be a member of:
%
%    a “salary-related” occupational pension scheme, where the pension received depends on the employee’s salary and service history. In such schemes employers and employees pay lower rate NICs;
%
%    a “money purchase” based occupational pension scheme. The contributions and any NI rebate are invested and, on retirement, can be used to buy an annuity. The final pension received depends on the investment performance of the scheme and the annuity rates available at retirement. Employers and employees pay a reduced rate of NICs and an age-related payment, which is increased with the age of the member, is paid to the pension scheme by the Inland Revenue; or
%
%    an “appropriate personal pension” (i.e. a contracted-out personal pension). In this case, employers and employees pay the full rate of NICs and an age-related payment is paid into their pension fund at the end of the tax year*.
%
%332. Some occupational schemes are hybrid or “mixed benefit”, combining features of salary-related and money purchase schemes. All personal pensions are provided on a money purchase basis.
%
%333. Occupational pension schemes which are contracted-out must satisfy conditions which are designed to ensure that employees in these schemes, who are benefiting from NI rebates, receive pensions from the scheme which at least equal what they would have received from SERPS.
%
%334. Prior to April 1997, in order for a salary-related scheme to contract-out, it had to promise to provide a Guaranteed Minimum Pension (GMP), which is broadly equivalent to what the SERPS entitlement would have been had the individual not contracted-out.
%
%335. From April 1997, contracted-out salary-related schemes have had to satisfy a scheme-based test (reference scheme test). This requires schemes to meet a statutory standard laid down in the Pension Schemes Act 1993.  The scheme actuary will certify that the test is met if the scheme provides benefits broadly equivalent to, or better than, those of the reference scheme.
%
%336. Employees who join occupational pension schemes which have not contracted-out of SERPS, or non-appropriate personal pension schemes, accrue SERPS benefits as well as benefits under the scheme, but they do not receive NI rebates.
%Occupational and personal pensions regulatory framework
%
%337. There is a framework designed to protect the interests of scheme members within which occupational and personal pensions have to operate. This includes provisions to ensure that those who leave schemes before retirement can transfer or preserve their accrued rights, that pensions in payment receive some protection against price increases, that schemes are properly run and assets safeguarded, and that the benefits scheme members expect to receive are secure.
%
%338. The Occupational Pensions Regulatory Authority (Opra) was established by the Pensions Act 1995* to regulate key aspects of the occupational pensions framework. The sale of personal pensions is regulated by the Financial Services Authority* (FSA).
%Recent developments
%
%339. The Government’s proposals for the reform of the pensions system in Britain were set out in the Green Paper A new contract for welfare: PARTNERSHIP IN PENSIONS (Cm 4179), published in December 1998. 
%
%340. Key principles set out in the Green Paper were that the new pensions system should:
%
%    improve pensions for low earners and carers by reforming SERPS with a more generous State Second Pension;
%
%    introduce new stakeholder pensions for moderate and higher earners; and
%
%    continue to support and strengthen the framework for occupational pension provision.
%
%341. Some of the measures to achieve the above aims were introduced in the Welfare Reform and Pensions Act 1999*. These include the legislative framework for the introduction of stakeholder pensions. It is intended to implement stakeholder pensions from April 2001.  These schemes will offer money purchase benefits to members, providing benefits related to the contributions paid by the members, together with the investment returns on those contributions (less charges). The charges will be subject to an upper limit.
%
%342. Stakeholder pension schemes will be targeted at those with moderate earnings (around £10,000 to £20,000 a year) who want to save more for retirement but who do not have access to an occupational scheme and for whom many existing personal pensions can be unsuitable or expensive. They will be set up within an approved governance structure (the arrangements for the management and oversight of a pension scheme) and meet minimum standards intended to encourage more moderate earners to save for their retirement. However, everyone will be able to pay into such a pension scheme, regardless of whether they are in work or not.
%
%343. The Welfare Reform and Pensions Act 1999 also included legislation on several detailed proposals on occupational and personal pension schemes which were contained in the consultation document Strengthening the Pensions Framework that accompanied the 1998 Green Paper.
%The measures in the Act
%State Second Pension
%
%344. The State Second Pension will reform SERPS by boosting the Additional (second tier) Pension of low and moderate earners, and by providing Additional Pension for the first time for carers and some long-term disabled people with broken work records.
%
%345. State Second Pension will be calculated by reference to the surplus in an individual’s earnings factor. An individual’s earnings factor corresponds to the whole of his earnings up to the Upper Earnings Limit and the surplus to the amount of those earnings between the Lower Earnings Limit and the Upper Earnings Limit.
%
%346. The State Second Pension regime will provide for a new Low Earnings Threshold, which will be uprated in line with increases in national average earnings. In 1999/00 terms this Low Earnings Threshold will be £9,500.  Anyone earning less than £9,500 but at or above the annual Lower Earnings Limit (£3,432 in 1999/00) will be treated for State Second Pension purposes as if they had an earnings factor of £9,500. 
%
%347. Carers, who have no earnings or earnings below the annual LEL, will be treated for State Second Pension purposes as if they had earnings of £9,500 for any year throughout which:
%
%    they receive Child Benefit* for a child under 6;
%
%    they are entitled to Invalid Care Allowance*; or
%
%    they are given Home Responsibilities Protection* because they are caring for a sick or disabled person.
%
%348. Those entitled to long-term Incapacity Benefit* or Severe Disablement Allowance* throughout a tax year will also be treated for State Second Pension purposes as if they had an earnings factor of £9,500 in that year, provided they meet a simple labour market attachment condition when they reach state pension age. This condition requires that they have worked and paid (or are treated as having paid) Class 1 National Insurance contributions for at least one tenth of their working life since 1978, when Additional Pension was introduced.
%
%349. There will be two stages to the State Second Pension. The first will be earnings-related for those earning above the Low Earnings Threshold. On the surplus in an earnings factor (actual or treated) of £9,500 (that is £9,500 less the prevailing annual Lower Earnings Limit), everyone will earn at least twice as much entitlement to Additional Pension as they did under SERPS. Where there is a surplus in the earnings factor corresponding to the amount of a person’s earnings above £9,500 but not exceeding £21,600, the accrual rate on that surplus will be half what it would have been under SERPS. This will have the effect of recouping some of the increased accrual that everyone will receive on the surplus in the earnings factor of £9,500.  However, all employees earning between the annual Lower Earnings Limit and £21,600 will receive more than they would have done under SERPS, with the largest proportionate gains going to those with the lowest earnings. Those earning £21,600 and above will receive the same as under SERPS.
%
%350. In the second stage, to be introduced when stakeholder pension schemes have become established, State Second Pension will become a flat-rate scheme for those with a significant part of their working life ahead of them (for example, those aged under 45 at the point of change). In the second stage of State Second Pension, everyone who is contracted-in to the state scheme will be treated as if they had earnings of £9,500 (or corresponding to the prevailing Low Earnings Threshold at that time), regardless of the level of their actual earnings. Qualifying carers and long-term disabled people with broken work records will continue to be treated as if they had such an earnings factor. State Second Pension will be calculated for everyone by reference to the surplus in an earnings factor of £9,500.  National Insurance rebates to those in contracted-out pension schemes will remain earnings-related.
%Contracting-out arrangements
%
%351. The Government views the contracting-out regime as central to the success of private pension provision and is keen to ensure that any changes made to the arrangements for contracting-out reflect the introduction of the State Second Pension, supporting and encouraging private pension provision.
%
%352. When individuals contract-out they do so on the basis that their pension arrangements will give them, broadly speaking, what they would have received from the state had they not contracted-out. As there is a reduced liability on the state, individuals and employers running occupational schemes receive a contracted-out rebate, which is calculated by reference to the value of the state benefit given up. The Government Actuary conducts a review every 5 years to determine the appropriate level of the rebate. The next review, due to take place later this year, will consider the level of rebates with effect from 2002. 
%
%353. As the State Second Pension is designed to boost the pension of low and moderate earners, the Government intends to change the contracting-out arrangements to ensure that members of contracted-out pension schemes are not better off contracting back in. Proposals on how the future contracting-out regime could be structured were the subject of a consultation exercise which ended on 14 January 2000.  After giving careful consideration to all the responses, the Government has decided to introduce measures which provide for:
%
%    all rebates for contracting-out into a personal pension, including a personal pension based stakeholder pension, to be calculated to reflect the enhanced 3 part accrual rate in the State Second Pension;
%
%    rebates to continue to be calculated as they are now for all occupational pension schemes, which will not be required to change their benefits;
%
%    people in all contracted-out pension arrangements on low earnings (up to £9,500) to get a top-up from the State Second Pension; and
%
%    the top-up to be extended to people on moderate earnings (up to £21,600) in contracted-out occupational pension schemes.
%
%354. The combination of these measures will ensure that low and moderate earners in contracted-out provision will also benefit from the extra help that the State Second Pension will give. This will simplify the choice of alternative pension vehicles available to them, without their having to contract back in to the state scheme to access that help.
%
%355. These proposals will also apply when State Second Pension becomes a flat-rate scheme for those who are contracted-in to the State scheme. Rebates and any state scheme top-up will continue to be earnings-related. This will prevent disruption to schemes at that stage, as well as providing an incentive for moderate earners to contract-out of the state scheme into a funded arrangement and ensuring that they will continue to be better off under State Second Pension.
%Contracted-out Personal Pension Schemes
%
%356. The rebate for contracted-out personal pension schemes will be based on the different accruals within the State Second Pension. For example, as the accrual rate for those earning at or above the annual Lower Earnings Limit but below £9,500 will double in relation to SERPS, so will the rebate. Low earners will therefore get a rebate based on a 40% rather than a 20% accrual which will be paid directly into their pension fund. For moderate earners it will be based on 40% on the first band of earnings, up to £9,500, and 10% on the second band, up to £21,600. 
%
%357. However, unlike State Second Pension, which will treat those earning at or above the annual Lower Earnings Limit but below £9,500 as if they were earning that amount, the rebate for low earners will remain based on actual earnings. To ensure that they still get the extra help from the low earner’s boost, they will get a State Second Pension “top-up”. In broad terms, the top-up will operate by calculating the State Second Pension an individual would have received had he not contracted-out (which of course includes the low earner’s boost) and taking away an amount which represents the pension derived from the rebate.
%
%358. This means that someone earning £8,000 would receive benefits from their pension scheme based on their actual earnings of £8,000 and a State scheme top-up based on £1,500 (the difference between actual earnings and £9,500). The rebate would be based on 40% of actual earnings between the Lower Earnings Limit and £8,000, and the top-up would be based on applying the 40% accrual to the £1,500. 
%Contracted-out Occupational Schemes
%
%359. The system will operate slightly differently for occupational schemes in order to cater for the Pensions Industry’s desire to avoid disruption to employers. Occupational schemes will therefore continue to have their reduced rate of National Insurance contributions calculated on the same basis as now – that is, a uniform accrual rate of 20%.
%
%360. In order to ensure that individuals in these schemes receive the extra help intended for low and moderate earners, an extended State Second Pension “top-up” will apply. It will work on the basis of calculating the State Second Pension that an individual would have received if he had not contracted-out, less a deduction equating to a pension derived from the rebate input (ie 20% of actual earnings).
%
%361. This means that for people in occupational schemes, all the extra help is delivered by way of the State scheme, rather then partly through their scheme via the rebate, and partly through the state, as is proposed for personal pension schemes.
%Reporting on the effect on the NI Fund of earnings uprating.
%
%362. This Act requires the Secretary of State to lay before Parliament a Government Actuary report which will estimate the effect on the balance in the National Insurance Fund, and the rate of National Insurance contributions needed to keep the Fund in balance, if the basic state pension were to be increased in line with earnings. The report will project figures for each year up to and including 2005/06. 
%Protection of “inherited SERPS”
%
%363. As a result of changes originally enacted in the 1986 Social Security Act, the amount of additional pension a surviving spouse could expect to inherit was due to be halved in respect of a married person who died after 5 April 2000.  The Welfare Reform and Pensions Act 1999 included provision to make regulations that would provide for, among other things, the deferral of the start of the new rule to a later year, and the setting up of a scheme to determine who may suffer future financial loss as the result of incorrect information about the impending change.
%
%364. This Act provides for the new rules to apply from October 2002 (but also provides that this date may be further postponed by regulations) and clarifies the eligibility criteria for the Inherited SERPS scheme.
%Improving pensions information
%
%365. The Act also contains a measure to improve overall pension information for individuals so that they have a clear indication of what sort of retirement income to expect and can therefore make better-informed decisions on what savings they need to make. This will:
%
%    permit state pension information to be passed to employers and pension providers so that they can issue pension statements giving details of both state and private pension rights unless individuals have indicated that they do not want the information. Because employers and pension providers will not need to gain the express consent of individuals the measure will improve the take-up of combined pension statements by employees and reduce administrative burdens; and
%
%    provide that state pension details can be passed to other third parties such as organisations which provide financial information services so that individuals who give consent can access their state pension details through these services.
%
%Improving the framework for occupational and personal pensions
%
%366. Besides the reform of the National Insurance rebate, there are four main parts to the reform of occupational pensions in the Act:
%
%    increased member involvement in schemes by requiring that all schemes must have one-third Member-Nominated Trustees (MNTs) by a process laid out in regulations or under procedures devised by the employer and approved by scheme members. The purpose is to increase confidence in the schemes, thereby encouraging more employees to join;
%
%    further protection of members’ pension rights by:
%
%        allowing the Occupational Pensions Regulatory Authority* (Opra) to make their register of disqualified trustees more accessible to the public; and
%
%        giving powers to Opra to monitor schemes which are in the process of winding up to ensure that winding up is undertaken as quickly as possible;
%
%    changes to the existing regulation-making powers to enable future regulations to require money purchase schemes to provide members with an illustration of the likely future value of their pension, thereby improving members’ appreciation of the value of their pension rights; and
%
%    measures to provide further clarification, simplification and flexibility for those operating schemes. These will:
%
%        increase the options for discharging contracted-out pension rights;
%
%        increase the options available to scheme members when they transfer their pension rights; and
%
%        allow a greater range of persons to make representations to the Pensions Ombudsman.
%
%Commentary on Sections
%Chapter I:  State Pensions
%State second pension
%Section 30: Earnings from which pension is derived
%
%367. Subsection (1)  inserts a new subsection (2A)  into section 22 of the Social Security Contributions and Benefits Act 1992* (the “Contributions and Benefits Act”) which sets out the earnings on which Additional Pension (in the State Earnings-Related Pension Scheme or SERPS) is calculated. Under State Second Pension, Additional Pension is to be calculated on those earnings on which Class 1 employee National Insurance contributions have been paid or treated as paid.
%
%368. From April 2000 employees earning below a new Primary Threshold no longer pay National Insurance contributions. Those employees with earnings between the prevailing Lower Earnings Limit (LEL) (£66 a week in 1999/00) and the new Primary Threshold* will be treated as if they had paid National Insurance contributions on those earnings. Provided their annual earnings are at least 52 times the weekly LEL (£3,432 in 1999/00), the year will qualify for contributory benefits such as basic Retirement Pension. If their earnings exceed this amount, employees will accrue entitlement to Additional Pension on the amount by which their earnings exceed 52 times the LEL.
%
%369. The self-employed do not accrue entitlement to Additional Pension. Their flat-rate Class 2 National Insurance contributions entitle them to flat-rate contributory benefits, such as basic Retirement Pension. However, someone may be both an employed earner and a self-employed earner in the course of a year. In such a case, if they are a member of SERPS as an employed earner, their Class 2 contributions are currently taken into account when calculating the amount of surplus Class 1 contributions on which their entitlement to Additional Pension is based. But if they are contracted-out of SERPS into an occupational pension or personal pension scheme, they receive a rebate of National Insurance contributions which is based solely on their Class 1 employee contributions. This section has the effect of treating members of State Second Pension in a similar way as those contracted-out. Only Class 1 earnings will be taken into account when calculating the amount of their State Second Pension entitlement.
%
%Subsection (2)($a$)  inserts new paragraph (za) into section 44(6)  which sets out how earnings factors are to be determined for State Second Pension purposes. For State Second Pension the earnings factor will be the total of the earnings on which Class 1 employee National Insurance contributions have been paid or treated as paid, unless the person concerned is treated as having an earnings factor for one of the reasons set out in section 44A of the Contributions and Benefits Act (inserted by subsection (3), see below).
%
%Subsection 2($b$)  amends section 44(6) ($a$)  to limit the current method of determining earnings factors for Additional Pension under SERPS to the period before “the first appointed year”, which is the year from which State Second Pension will take effect.
%
%370. Subsection (3)  inserts new section 44A into the Contributions and Benefits Act.
%New section 44A: Deemed earnings factors
%
%371. New section 44A(1)  provides for a person to be deemed to have an earnings factor equal to the Low Earnings Threshold when calculating entitlement to Additional Pension under the State Second Pension if they qualify in any of the ways set out in new section 44A(2). In 1999/00 terms the Low Earnings Threshold will be £9,500 (see new section 44A(5)  below).
%
%New section 44A(2)($a$)  provides for a person to be treated as if they had an earnings factor of £9,500 in a qualifying year when they had earnings at or above the level needed to make the year a qualifying one for basic pension (earnings at or above the annual LEL of £3,432 in 1999/00, which is 52 times the weekly LEL of £66 a week in 1999/00) but less than the Low Earnings Threshold (£9,500).
%
%New section 44A(2)($b$)  provides for a person to be treated as if they had an earnings factor of £9,500 in a qualifying year if Invalid Care Allowance was paid to them throughout the year.  A person can also qualify if they would have been entitled to receive Invalid Care Allowance were it not for the fact that they received another (higher) benefit, such as widows’ benefits* or Incapacity Benefit.
%
%New section 44A(2)($c$)  provides for a person with no earnings, or earnings below the LEL, to be treated as if they had an earnings factor of £9,500 in a qualifying year when they were paid Child Benefit for a child under 6, or if they satisfied certain other conditions to be specified in regulations.  It is intended that these other conditions will be the same as those specified in regulations 2(2)($b$), 3 and 4 of the Social Security Pensions (Home Responsibilities) Regulations 1994 (S.I. 1994/704).  These regulations provide for a person to be treated as being precluded from regular employment by responsibilities at home if:
%(i) 
%
%they receive Income Support*, because they are caring for a sick or disabled person (and therefore are not required to be available for work); or
%(ii) 
%
%they spend at least 35 hours a week caring for a person who receives Attendance Allowance* or the care component in Disability Living Allowance* at the middle or highest rate.
%
%New section 44A(2)($d$)  provides for a person to be treated as if they had an earnings factor of £9,500 in a qualifying year if long-term Incapacity Benefit was paid to them throughout the tax year.  A person can also qualify if they would have been entitled to receive long-term Incapacity Benefit but failed to satisfy the necessary contribution conditions for that benefit, or received another (higher) benefit, such as widows’ benefits*, or received an occupational or personal pension which reduced the amount of Incapacity Benefit to nil.  Such a person would also need to satisfy the labour market attachment test set out in new section 44A(3)  and (4)  below.
%
%372. New section 44A(3)  and (4)  detail the labour market attachment conditions for those who could qualify for State Second Pension on grounds of entitlement to long-term Incapacity Benefit. Such a person must have paid, or be treated as having paid, Class 1 employee National Insurance contributions for at least one tenth of their working life since 1978, when Additional Pension was introduced. (A full working life for state pension purposes is counted from the start of the tax year in which a person reaches 16 to the end of the tax year before the one in which they reach state pension age). For instance, someone reaching state pension age in 2005/06 would have a working life of 27 years and would need to have worked and to have paid National Insurance contributions for 3 years (1/10th of 27 years rounded to the nearest whole year) in order to receive entitlement to State Second Pension on grounds of incapacity for work. Anyone reaching state pension age after April 2024 would need 5 years (working life of 46 years, up to a maximum working life of 49 years, 1/10th rounded to the nearest whole year being 5 years). National Insurance credits* will not satisfy this condition. However, any year in which the person has worked but not actually paid Class 1 National Insurance contributions because their earnings, although above the annual LEL, were below the new Primary Threshold on which such contributions are paid, will be treated as if they had paid contributions on those earnings.
%
%373. Any year when the disabled person has been a carer and qualifies for State Second Pension by virtue of subsections (2)($b$)  or ($c$)  above will be excluded from the number of years in the working life when calculating whether they have met the labour market attachment condition. For instance, someone retiring on 6 April 2024 would have a working life of 46 years and would need 5 years (1/10th of 46 years rounded to nearest whole number) in which they had worked and paid Class 1 National Insurance contributions. But if they had received Child Benefit for a child under 6, and were treated as precluded from regular employment by responsibilities at home for those 5 years whilst the child was under that age, or if they were entitled to Invalid Care Allowance for 5 years, their working life would be reduced to 41 years and they will only need 4 years (1/10th of 41 years rounded to nearest whole number).
%
%374. New section 44A(5)  sets the Low Earnings Threshold in State Second Pension at £9,500.  This is subject to new section 148A of the Social Security Administration Act 1992* (the “Administration Act”) which provides for the Low Earnings Threshold to be increased in line with rises in national average earnings (see the note to subsection (1)  of section 33 below).
%
%375. New section 44A(6)  defines “occupational pension scheme” and “personal pension scheme” as used in the inserted new section 44A(2)($d$)(ii) .
%
%376. Subsection (4)  of section 30 provides for someone to be treated as if they had an earnings factor of £9,500 in a qualifying year if they are paid Severe Disablement Allowance (SDA) throughout the year and they meet the labour market attachment test set out in new section 44A(3)  and (4) . SDA is being withdrawn for new claimants from April 2001 but those already receiving the benefit will continue to do so.
%Section 31: Calculation
%
%377. Subsections (1)  and (2)  amend subsection (2)  of, and insert subsection (3A)  into, section 45 of the Contributions and Benefits Act, which sets out the way Additional Pension is calculated. They provide for Additional Pension to be the sum of entitlement accrued under SERPS and entitlement accrued under State Second Pension.
%
%378. Subsection (3)  introduces Schedule 4 which sets out a new Schedule 4A which is to be inserted into the Contributions and Benefits Act. It sets out the way in which State Second Pension is to be calculated for those contracted-in and those contracted-out.
%New Schedule 4A: Additional pension
%
%379. This Schedule sets out the way in which Additional Pension will be calculated under State Second Pension (“the final amount”). There are four Parts to the Schedule:
%
%    Part I sets out the calculation of the final amount;
%
%    Part II sets out the calculation of the yearly amount for someone who is not contracted-out for any part of the year;
%
%    Part III sets out the calculation of the yearly amount (if any) for someone in contracted-out employment for the whole of the year;
%
%    Part IV provides for regulations to be made in particular to provide for the calculations where someone moves between contracted-in and contracted-out employment in the course of a year, and between different types of contracted-out provision within a year.
%
%Part I: The amount
%
%Paragraph 1(1)  provides for the amount of Additional Pension under State Second Pension to be calculated by adding together the amounts (if any) for each year since the introduction of State Second Pension, and then dividing by the number of relevant years (that is, the number of years in the working life since 1978, when Additional Pension was introduced).
%
%Sub-paragraphs (2)  to  (7)  set out the method of calculating the number of years in the working life to be used in the calculation of any Additional Pension payable under State Second Pension to widows and widowers, based on their late spouse’s contributions.  This method is the same as that used under SERPS for those widowed from 6 April 1999 onwards.
%
%Sub-paragraph (8)  defines “relevant year” as the number of years in the working life since 1978 by reference to section 44(7)  of the Contributions and Benefits Act.
%Part II: Surplus earnings factors
%
%Paragraph 2(1)  provides for Part II of the Schedule to apply to any year in which a person has a surplus in his earnings factor for the year.  There is a surplus where the earnings factor for the year exceeds the Qualifying Earnings Factor (QEF), which is 52 times the weekly Lower Earnings Limit for the year.
%
%Sub-paragraphs (2) , (3)  and (4)  set out how State Second Pension will be calculated for someone who is not contracted-out of the state scheme at any time during the year in question.
%
%First, the surplus earnings factors for the years in question are to be divided into the bands shown in the appropriate table in sub-paragraph (3)  or (4) .  Secondly, the surpluses in each of the bands are to be revalued for each year to ensure that they maintain their value in earnings terms.  This revaluation is in line with the increase in national average earnings up to the year before the year in which state pension age is reached (under section 148 of the Administration Act).  Thirdly, the revalued surpluses in each band are to be multiplied by the relevant percentage in the tables in sub-paragraphs (3)  or (4) .  Finally, the totals for each band are added together to give the total for the year.
%
%This calculation differs from that used for SERPS by the application of different accrual rates to the surplus in the earnings factor falling within different bands of surplus earnings factors.  Under SERPS anyone retiring from 6 April 2009 onwards (Table 2) would have had one accrual rate of 20% on their surplus earnings factor.  Under State Second Pension the same person will have an accrual rate of 40% on the surplus in their earnings factor falling within Band 1.   Band 1 covers surpluses in the earnings factor which correspond to the amount of earnings between the Qualifying Earnings Factor and the Low Earnings Threshold of £9,500.   This band will include those low earners, carers and disabled people with broken work records who are treated as if they had an earnings factor of £9,500 in a qualifying year under section 44A(2)  of the Contributions and Benefits Act as inserted by subsection (3)  of section 30 above.
%
%Surpluses in a person’s earnings factor falling within Band 2 will have an accrual rate of 10%.   Band 2 covers twice the amount of surplus falling within Band 1, rounded to the nearest £100.   So, if the Qualifying Earnings Factor is £3,432 (52 times the 1999/00 weekly Lower Earnings Limit) and the Low Earnings Threshold is £9,500, the amount of surplus falling within Band 1 will be £6,068.   This means that the amount of surplus falling within Band 2 will be £12,136 and the upper limit of Band 2 will be £9,500 + £12,136, rounded to the nearest £100, which is £21,600.   Someone earning £21,600 will receive the same amount from State Second Pension as they would have done under SERPS, because the higher accrual rate on the surplus in their earnings factor falling within Band 1 will be entirely offset by the lower accrual rate on the surplus in their earnings factor falling within Band 2.   Those earning less than £21,600 will receive more from State Second Pension than they would have done under SERPS.
%
%Surpluses in a person’s earnings factor falling within Band 3 will have an accrual rate of 20%, that is the same as under SERPS.  Band 3 covers surpluses in the earnings factor which correspond to earnings which are above the surplus falling within Band 2 but not exceeding the Upper Earnings Limit.
%
%Someone reaching state pension age before 6 April 2009 would have had an accrual rate higher than 20% under SERPS.  This is because of the changes made in the Social Security Act 1986, which reduced the accrual rate in SERPS in stages from 25% to 20% for those retiring between 2000/01 and 2009/10 (in respect of accruals from 1988/89).  There will be similar transitional arrangements in the State Second Pension and Table 1 shows the accrual rate for those retiring before 6 April 2009.   Someone reaching state pension age before then will have accrual rates which give an extra 1% on the surplus in their earnings factor falling within Band 1, 0. 25% on the surplus in their earnings factor falling within Band 2, and 0. 5% on the surplus in their earnings factor falling within Band 3 for each year by which the year in which they reach state pension age is earlier than 2009/10.   For instance, someone reaching state pension age in the year beginning 6 April 2008 would have accrual rates of: 41% on the surplus in their earnings factor falling within Band 1 (twice what it would have been under SERPS); 10. 25% on any surplus falling within Band 2 (half what it would have been under SERPS); and 20. 5% on any surplus falling within Band 3 (the same as it would have been under SERPS).
%
%Sub-paragraph (5)  will enable regulations to be made to bring in Stage 2 of the State Second Pension for people with a significant part of their working life ahead of them.  All those coming within the scope of Stage 2 will earn entitlement to State Second Pension as if they had an earnings factor of £9,500, regardless of their actual earnings.  This means that low earners, carers and long-term disabled people with broken work records will continue to be deemed to have an earnings factor of £9,500.   However, those earning more than £9,500 will only earn entitlement to State Second Pension on the surplus in their earnings factor falling within Band 1, that is the amount between the prevailing annual Lower Earnings Limit (the Qualifying Earnings Factor) and the Low Earnings Threshold (the deemed earnings factor under new section 44A, which is £9,500 or the prevailing level at the time Stage 2 is introduced).  This will only apply to entitlement accrued after the “second appointed year”, which will be the year in which Stage 2 is introduced.  Any entitlement accrued under Stage 1 will be preserved.  It is intended that Stage 2 will not be brought in until stakeholder pension schemes have established themselves.
%
%Sub-paragraphs (6) , (7)  and (8)  define “the value of N”, “LET”, “QEF” and “2QEF”, and also the “final relevant year”.
%Part III: Contracted-out employment etc
%Paragraph 3: Introduction
%
%380. Paragraph 3 provides for Part III, and not Part II, of the Schedule to apply to any year throughout which the person is contracted-out of the state scheme and in an occupational or personal pension scheme.
%Paragraph 4: The amount
%
%381. This paragraph provides for calculation of the amount for that year (for the purposes of the calculation in paragraph 1) under State Second Pension for someone in contracted-out employment (amount C) to be the amount of additional pension they would have received if they had not been contracted-out of State Second Pension (amount A) less the amount of pension they are deemed to receive in respect of their contracted-out National Insurance rebate (amount B).
%Paragraph 5: Amount A
%
%382. Amount A is the amount a person would have received from State Second Pension if they had not been contracted-out. The calculation is the same as that in Part II with one exception - there is no provision to move to a flat-rate scheme as in the second stage of State Second Pension. This is because those who are contracted-out will continue to receive earnings-related rebates, and their top-up (if any) will be based on the earnings-related first stage of State Second Pension.
%Paragraph 6: Amount B (first case)
%
%383. Amount B (first case) applies to those who are contracted-out by an occupational salary-related, or money purchase scheme, including an employer’s occupational-based stakeholder pension scheme. It is the amount of contracted-out second pension which a person is treated as receiving in respect of their National Insurance rebate.
%
%384. A person in a contracted-out occupational scheme receives a rebate which is calculated to reflect the cost of providing benefits of an actuarial value equivalent to that of the state benefit given up. Currently, under SERPS, the benefit given up is 20% of their lifetime earnings for someone who reaches state pension age on or after 6 April 2009.  (Those reaching state pension age before 6 April 2009 have a higher accrual rate as described above, which is reflected in the amount of their rebate.)
%
%385. Under State Second Pension a person in a contracted-out occupational scheme will continue to receive a rebate calculated on this basis.
%
%386. Therefore, amount B (the amount to be deducted from what the person would have got if they had not been contracted-out of the state scheme) is 20% (or 20+N% for someone retiring before 6 April 2009) of the “assumed surplus” in their earnings factor for the year. The “assumed surplus” is defined in sub-paragraph (2)  of paragraph 8 as the surplus there would have been if the person had not been contracted-out. In effect, this is their earnings between the Lower and Upper Earnings Limits for the year. The assumed surplus is then revalued in the same way as the surplus in amount A, to ensure that the amount B for each year also maintains its value relative to average earnings up to state pension age.
%
%387. As paragraph 8(3)  makes clear, there is to be no boost to the amount of the Low Earnings Threshold when calculating amount B. So where actual earnings are between the annual Lower Earnings Limit and the Low Earnings Threshold, the rebate will be calculated on the basis of the actual earnings.
%
%388. For all those earning less than the upper limit of Band 2 (£21,600 in 1999/00 terms) there will be a surplus when amount B (first case) is deducted from amount A. This is because State Second Pension is more generous than SERPS for those earning less than the upper limit of Band 2.  Those earning below the Low Earnings Threshold benefit from the boost to that threshold, and moderate earners continue to benefit from the increased accrual rate on their Band 1 earnings until earnings reach the upper limit of Band 2.  The surplus of amount A over amount B represents the extra a person would have received if they had remained in an earnings-related State Second Pension scheme. The surplus will be paid as a top-up to the state pension when the person reaches state pension age.
%Paragraph 7: Amount B (second case)
%
%389. Amount B (second case) applies to those who are contracted-out by an appropriate personal pension, including a non-occupational stakeholder pension scheme. It is the amount of contracted-out second pension which a person is treated as receiving in respect of their National Insurance rebate.
%
%390. For those contracted-out into appropriate personal pension schemes, including stakeholder personal pension schemes, the amount of rebate paid to their scheme will be increased when State Second Pension is introduced to reflect the 3 part accrual rate in State Second Pension itself. This is reflected in Table 5 (for those reaching state pension age before 6 April 2009) and in Table 6 (for those reaching state pension age on or after 6 April 2009).
%
%391. This means that for all those earning at or above the Low Earnings Threshold, amount B (the revalued assumed surplus) will be the same as amount A (the revalued surplus calculated as if the person had been contracted-in). Where earnings are above the annual Lower Earnings Limit, but below the Low Earnings Threshold, for the year there will be a surplus when amount B (second case) is deducted from amount A. This surplus represents the extra a person would have received in State Second Pension from the low earner’s boost. The surplus will be paid as a top-up to the state pension when the person reaches state pension age.
%Paragraph 8: Interpretation
%
%392. This paragraph defines “salary related contracted-out scheme”, “money purchase contracted-out scheme”, “appropriate personal pension scheme”, “assumed surplus”, “N”, “LET”, “QEF”, “2QEF” and “final relevant year”.
%Part IV: Other cases
%
%393. Paragraph 9 provides for regulations to be made for calculating the amount of any State Second Pension in a year for those cases not covered by Parts II or III. In the main these will be cases where the person’s circumstances change during the course of the year. For instance, a person may move between contracted-in and contracted-out employment during the year. Or they may move between different categories of contracted-out employment, such as from employment with a personal pension scheme to one with an occupational pension scheme.
%
%394. In such cases it will be necessary to apportion the amounts calculated according to amounts of employment in each circumstance. The guiding principle for these regulations will be to provide a top-up from the state scheme if the person contracted-out would have received more from State Second Pension than the amount they are treated as receiving in respect of their National Insurance rebate.
%
%395. Paragraph 9 also provides for regulations to be made in “such other cases as the Secretary of State thinks fit”. This provision will be used to prescribe how the provisions regarding the Contribution Equivalent Premium and the restoration of state scheme rights will operate under, and interact with, the State Second Pension provisions. These are the provisions which deal with the necessary calculations for those who are contracted back into the State scheme because, for instance, their contracted-out employment terminates after less than 2 years. As above, this power is to be used to ensure that no one loses out because they had a period in contracted-out, rather than contracted-in, employment.
%Section 32: Calculation of Category B retirement pension
%
%396. Subsection (1)  inserts new section 46(3)  in the Contributions and Benefits Act. It corrects the method of calculating SERPS Additional Pension which someone, who (at some point in the past) had received Widowed Parent’s Allowance or Bereavement Allowance, can inherit from their deceased spouse when they reach state pension age. The correction is necessary to restore the policy intention that no-one will receive less Additional Pension with their Category B pension (based on their late spouse’s contributions) than they would have done before the introduction of the new Bereavement Benefits, which are due to come into effect in April 2001.  There is a similar provision for State Second Pension in paragraph 1(5)  – (7)  of new Schedule 4A of the Contributions and Benefits Act as inserted by Schedule 4 of this Act.
%
%397. Subsections (2)  and (3)  make consequential amendments to section 48BB(5)  of the Contributions and Benefits Act and paragraph 5 of Schedule 8 to the Welfare Reform and Pensions Act 1999. 
%Section 33: Revaluation
%
%398. Subsection (1)  inserts new section 148A after section 148 of the Administration Act.
%New section 148A: Revaluation of low earnings threshold
%
%399. New section 148A(1)  to (8)  provides for the Secretary of State to make orders to increase the amount of the Low Earnings Threshold. The figure of £9,500 is in 1999/00 terms. It will be increased to take account of rises in national average earnings both before the beginning of the first appointed year and annually thereafter. This will be done by making an order in the year before State Second Pension is introduced which takes account of increases in average earnings. In order to ensure that full years are taken into account the increase will be measured over the period from 1 October 1998 to 30 September in the year before State Second Pension is introduced. This will allow time for the order to be made to take effect from the start of the first appointed year. Subsequently, orders will be made each year to take account of increases in national average earnings. The review period will normally begin from the end of the previous period unless a change of circumstances requires a different period to be used. In all cases amounts are to be rounded to the nearest £100. 
%
%400. Subsection (2)  of section 32 sets out how section 148 of the Administration Act is to be applied to the revaluation of surpluses in a person’s earnings factors under State Second Pension. Section 148 of the Administration Act provides for earnings factors to be revalued in line with increases in national average earnings. This is done to ensure that the value of a person’s earnings over their working life is maintained in earnings terms up to state pension age when the amount of the Additional Pension is calculated. An annual order under section 148 of the Administration Act sets out the percentages to be used for this revaluation. In State Second Pension these percentages will be applied to the surpluses in a person’s earnings factors falling into each of the three Bands as set out in the new Schedule 4A to the Contributions and Benefits Act.
%
%401. Subsections (3)  and (4)  clarify how section 148 of the Administration Act applies where the surplus in earnings factors is to be revalued for SERPS. The way in which earnings factors are revalued for calculating the amount of Additional Pension (where the date of death of a spouse, or of reaching state pension age falls on or after 6 April 2000) was changed by section 128 of the Pensions Act 1995 which inserted section 44(5A)  into the Contributions and Benefits Act. The purpose of this change was to require the surpluses in a person’s earnings factors to be revalued rather than the earnings factors themselves. Subsections (3)  and (4)  make clear that where there is a requirement under section 44(5A)  of the Contributions and Benefits Act to revalue the surplus in a person’s earnings factors in order to calculate the amount of their Additional Pension under section 45 of that Act, there is no requirement also to revalue the earnings factors themselves.
%Section 34: Report of Government Actuary: rebates etc.
%
%402. This section amends the sections in the Pension Schemes Act dealing with the Government Actuary’s report on National Insurance Contribution rebates and reduced contribution rates. The Government Actuary sets out what, in his opinion, is the cost of providing benefits of an actuarial value equivalent to that of the State benefits given up. The Report is used when the Secretary of State decides what is the appropriate level of rebate or rate reduction. Unlike SERPS, not all of the State Second Pension will necessarily be given up, and the rebate calculations and the reduced contribution rates calculation need to reflect this fact. Section 34 therefore amends the sections to include a reference to the fact that, in some cases, only part of the State benefit will be given up.
%Section 35: Supplementary
%
%403. Subsection (2)  extends to State Second Pension the provision in section 21(5A)  of the Contributions and Benefits Act whereby National Insurance contributions paid or treated as paid have effect as if they had been paid on the whole of earnings up to the Upper Earnings Limit.
%
%404. Subsections (3)  and (4)  insert references to the new Schedule 4A in sections 39 and 39C of the Contributions and Benefits Act, which concern the rate of widowed mother’s allowance, widow’s pension, widowed parent’s allowance and bereavement allowance.
%
%405. Subsections (5) , (6)  and (8)  clarify the provisions in sections 44(5A)  and 45 of the Contributions and Benefits Act on calculating Additional Pension entitlement, where the date of death or reaching state pension age falls on or after 6 April 2000.  The calculation is to be based on the “adjusted”, that is, revalued, amount of the surplus in the earnings factor. This has an effect on the calculations under new Schedule 4A. (See also the note to subsections (3)  and (4)  of section 33).
%
%406. Subsections (9) , (10) , (11) , (12)  and (13)  insert references to the new Schedule 4A in sections 48A, 48B, 48BB, 48C and 51 of the Contributions and Benefits Act, which deal with Category B retirement pensions.
%
%407. Subsection (14)  inserts definitions of “first appointed year” and “second appointed year” in section 122(1)  of the Contributions and Benefits Act. State Second Pension is to begin from a date (“the first appointed year”) to be appointed by order. The precise date will depend upon the necessary operational systems to deliver State Second Pension being in place. The earliest date for implementation will be April 2002.  Additional Pension is accrued up to and including the year before the year in which state pension age is reached. So those reaching state pension age in the year beginning 6 April 2003 will be the first to have accrued any entitlement to State Second Pension.
%
%408. The “second appointed year” will also be appointed by order. This will be the date from which the flat-rate Stage 2 of State Second Pension will be introduced for those with a significant part of their working life ahead of them. It is intended that Stage 2 will not be introduced until stakeholder pension schemes have become established.
%
%409. Subsection (15) provides for the orders appointing the first or second appointed years to be made without being subject to Parliamentary control.
%Report on pensions upratingt
%Section 36: Report on cost of pension uprating in line with general earnings level
%
%410. Section 36 provides that the Government Actuary (or his deputy) shall submit a report to the Secretary of State, giving his opinion as to the effect:
%
%    on the balance in the National Insurance Fund, and
%
%    the rate of National Insurance contributions needed to keep the Fund in balance,
%
%if the basic state pension were to be increased each year in line with average earnings.
%
%411. The report will provide figures for each year up to and including 2005/06, and the Secretary of State shall lay a copy of the report before Parliament.
%
%412. The section does not stipulate when the report shall be published, but the Government’s intention is that it shall be a sister document to the next uprating report which is expected to be produced in January 2001. 
%Earnings Factors
%Section 37: Revaluation of earnings factors
%
%413. Additional pension is the earnings-related benefit element of the state retirement pension. Contributions are made via the State Earnings-Related Pension Scheme (SERPS). It is calculated on the basis of earnings factors, which are those earnings between the Lower and Upper Earnings Limits on which a person pays National Insurance contributions.
%
%414. Section 148 of the Social Security Administration Act 1992 requires the Secretary of State to ensure that the earnings factors, used for calculating additional pension under SERPS and Guaranteed Minimum Pensions in contracted-out schemes, maintain their value in relation to the general level of earnings. The annual “Revaluation of Earnings Factors Order” gives the amount by which earnings factors for each year will need to be uprated to keep them in line with increases in average earnings.
%
%415. The annual revaluation of earnings factors currently covers movements in earnings over each twelve-month period from December to December. The data underlying the order comes from the Office of National Statistics’ Average Earnings Index which is often not available in time for the start of the financial year. The intended purpose of this section, which follows consultation with representatives of the pensions industry, is to allow the Department to move the period covered by the order to September to September. This will give employers and pensions administrators access to revaluation figures at an earlier and more convenient time of year. It will also provide consistency with the annual revaluation of the low earnings threshold proposed in the State Second Pension.
%
%416. Section 37 amends section 148(2)  to enable flexibility to be used in determining the period to be considered for the purposes of the revaluation of earnings factors.
%Section 38: Modification of earnings factors
%
%417. Section 48A of the Pension Schemes Act 1993 covers the situation of people who have earnings in a single tax year both from contracted-out employment and from contracted-in employment. Only their earnings from contracted-in employment in that year will count towards the State Earnings-Related Pension Scheme (SERPS). Section 48A(5)  allows regulations to be made which modify the calculation of additional pension under SERPS (that is, the calculation under section 44(5)  of the Social Security Contributions and Benefits Act 1992) to take account of the relevant employment taking up only part of the year. The present regulations are the Social Security (Contracting-out and Qualifying Earnings Factors) Regulations 1996 which came into force on 6 April 1997. 
%
%418. However, section 128 of the Pensions Act 1995 replaced section 44(5)  with a new section 44(5A)  which applies to people who reach pensionable age after 5 April 2000 (or, in the case of widows or widowers, whose spouse dies after that date). But the regulation-making power in section 48A(5)  was not amended at the same time. As a result, there is no power at present to make an updated version of the 1996 regulations, which would refer to the new section 44(5A) . The existing regulations will not be effective after 5 April 2000.  This new section amends section 48A(5)  so as to provide the power to make a new set of regulations that will enable the calculation of additional pension using the same part-years provision to continue. However, as these regulations cannot be made until after the Act receives Royal Assent it also authorises the continuing use of the existing regulations in the meantime (subsection (3) ).
%
%419. Subsection (1)  provides the power to make regulations modifying the provisions of section 44(5A)  of the Contributions and Benefits Act so that the SERPS calculation takes account of part-year earnings from employment which is not contracted-out.
%
%420. Subsection (2)  applies the modification to the calculation of additional pension payable in relation to Widowed Parent’s Allowance, Category A Retirement Pension where the pensioner reaches pension age after 5 April 2000, and Category B Retirement Pension where the claimant has previously received Widowed Parent’s Allowance or Bereavement Allowance or where the spouse dies after 5 April 2000. 
%
%421. Subsection (3)  provides for the existing regulations to continue to have effect from 6 April 2000 until new regulations are made. To make subsection (3)  workable, subsection (4)  provides for references to section 44(5A)  of the Contributions and Benefits Act to be treated as references to section 44(5).
%
%422. Subsection (5)  enables the new regulations to include provision for reviewing the calculations made under the old method and recalculating and paying pension in accordance with the new method.
%
%423. Subsection (6)  describes the circumstances in which persons will be affected because their pensions will be calculated using the old method after 5 April 2000. 
%Preservation of rights in respect of additional pensions
%Section 39: Preservation of rights in respect of additional pensions
%
%424. Currently, widows and, in certain circumstances, widowers, may receive the full amount of their deceased spouse’s SERPS. However, changes originally enacted in the Social Security Act 1986 (but now consolidated in the Contributions and Benefits Act 1992) halved the amount of SERPS the surviving spouse could receive. The change was due to take effect in respect of married persons who died after 5 April 2000.  This change was not fully publicised, and some people were incorrectly told that they, or their widow(er) could expect to “inherit” the full amount of SERPS.
%
%425. The Government has now decided to postpone the reduction to 6 October 2002, and to set up an Inherited SERPS Scheme.
%
%426. Section 52 of the Welfare Reform and Pensions Act 1999 enables the Secretary of State to make affirmative regulations to do one or more of the following:
%
%    to provide for specified categories of widows and widowers to receive more than 50% of their spouse’s SERPS;
%
%    to postpone the 50% reduction due to come into effect from 6 April 2000 to a later year;
%
%    to set up a scheme to determine who has been misled by incorrect or incomplete information about the 50% reduction and who, or whose spouses may, as a result, suffer financial loss in the future, so as to ensure that the reduction is not applied in their case.
%
%427. Until provision for one of these options is in force, the section also ensures that widow(ers) continue to “inherit” the full amount of their spouse’s SERPS.
%
%428. Section 39 amends section 52 of the Welfare Reform and Pensions Act to provide for the 50% reduction in inherited SERPS to come into effect in respect of deaths occurring on or after 6 October 2002, but also to provide that regulations may postpone the change to a later date. It clarifies the eligibility criteria for the Inherited SERPS Scheme to ensure that people who were denied the opportunity of considering taking relevant steps to protect their spouse’s position because they received incorrect or incomplete information, can seek redress. It also allows for the regulations to make further specific provisions relating to the manner in which decisions under the scheme may be taken.
%
%429. Subsections (1)  and (2)  change the implementation date for the 50% reduction in SERPS from 6 April 2000 to 6 October 2002 where entitlement to additional pension in SERPS arises in Widowed Mother’s Allowance, Widowed Parent’s Allowance and Category B retirement pension, including where this is increased because of deferred retirement.
%
%430. Subsection (3)  provides that regulations may defer the implementation date of 6 October 2002 to a later date.
%
%431. Subsection (4)  allows for the regulations to provide that a person can be eligible for the scheme if, as a consequence of receiving incorrect or incomplete information, he did not consider either taking a step to safeguard the future financial position of his spouse that he might have taken, or refraining from taking such a step which he took but might not have taken had he received the right information.
%
%432. Subsection (5)  allows for the regulations to prescribe matters that may be relied on, or presumptions that may be made, in the making of decisions under the preserved rights scheme.
%Other provisions
%Section 40: Home responsibilities protection
%
%433. Section 40 inserts sub-paragraph (7A)  into paragraph 5 of Schedule 3 to the Contributions and Benefits Act. It provides for regulations to be made for those precluded from regular employment by responsibilities at home to supply the necessary information for this to be taken into account when assessing their pension entitlement. This happens automatically for a person who receives Child Benefit for a child under 16 in any year in which they do not meet the Qualifying Earnings Factor. Those meeting the prescribed conditions for caring for a sick or disabled person have to supply the necessary information. Currently such notifications can be made at any time up to state pension age, and awards can be backdated to 1978. 
%
%434. The Government intends to bring in regulations which will require notifications to be made by the end of the third year following the year in which the qualifying caring activity took place. This requirement will only apply to qualifying periods following the introduction of State Second Pension. It is to ensure that entitlement to State Second Pension on the grounds of caring activity is established and recorded timeously. It will also determine any years to be excluded from the requisite number of years in the calculation of the basic Retirement Pension.
%Section 41: Sharing of state scheme rights
%
%435. Although the value of a pension can currently be taken into account by the courts in reaching a financial settlement on divorce or nullity of marriage, the pension itself can only be offset against other assets or “earmarked”, ie the court can order part of a pension to be paid direct to a former spouse by a pension scheme when it comes into payment. “Earmarking” has been little used because it does not facilitate a clean break and the former spouse loses her or his intended retirement income if the ex-spouse whose pension has been “earmarked” dies first.
%
%436. By contrast, pension sharing provides a former spouse with a pension in her own right, security of income throughout retirement, and a clean break. It will also enable more couples to reach a fair settlement where the pension to be shared is the most significant asset in the marriage.
%
%437. However, pension sharing will not be compulsory: it will be an additional option for couples to consider alongside offsetting and earmarking and the Government expects that most couples will, as now, continue to offset their pension rights against other assets.
%
%438. The Government consulted on a draft Pension Sharing Bill in June 1998.  The Social Security Select Committee conducted a detailed inquiry into the draft Bill. Many of the recommendations in the Committee’s Report, published in October 1998, were subsequently adopted by the Government when legislation on pension sharing was included in the Welfare Reform and Pensions Act 1999. 
%
%439. The pension sharing provisions in the Act broadly provide for all second-tier pensions to be shared ie private and occupational pensions and SERPS (and, in due course, the State Second Pension).
%
%440. Section 41 contains sub-delegation powers to enable the Secretary of State to set out in regulations how the cash equivalent of SERPS rights is to be calculated. The regulations will give the Secretary of State the power to require that the cash equivalent shall be calculated and verified in such a manner as may be approved by the Government Actuary or by an actuary authorised by the Government Actuary to act on his behalf for that purpose. The Secretary of State will also have the power to require cash equivalents to be calculated and verified by adopting methods and making assumptions which are consistent with guidance published by the Institute of Actuaries and Faculty of Actuaries.
%
%441. Subsection (1)  substitutes Section 49(4)  of the Welfare Reform and Pensions Act 1999 to include the sub-delegation power. For consistency, subsections (2)  - (4)  of this section make equivalent changes to related provisions in sections 45B, 55A and 55B of the Social Security Contributions and Benefits Act 1992.  Section 45B of that Act is concerned with the reduction in the additional pension of the member whose pension has been shared; section 55A deals with the calculation of the additional pension acquired by the former spouse who was the beneficiary of the pension share (who receives a “shared additional pension”); and section 55B makes provision for the reduction of a shared additional pension which has itself been the subject of a pension sharing order or agreement.
%
%442. Similar provisions are already in place for other pension sharing provisions in the Welfare Reform and Pensions Act 1999 where a valuation of non-state pension rights is needed.
%Section 42: Disclosure of state pension information
%
%443. The Government indicated in the Pensions Green Paper (A new contract for welfare: PARTNERSHIP IN PENSIONS Cm 4179 December 1998) that it wished to work with employers and pension providers to develop integrated personal pension statements, combining state and private pension rights. The Government’s aim is to include details of current and projected state pension rights in annual pension or financial statements provided by employers and pension providers. The intention is to provide individuals with better information on the sort of future income they might expect to help them plan for their retirement.
%
%444. In order to comply with existing legal requirements, the Department of Social Security can only pass state pension details to employers and pension providers with the express consent of employees and scheme members.
%
%445. Employers and pension scheme providers have expressed concern that continued adherence to an express consent process would lead to a low take-up by individuals and would impose a significant administrative burden which would discourage them from providing combined forecasts.
%
%446. The measures in the Act are intended to address these concerns and the Government’s wish to ensure that individual state pension details can be made available to other third parties such as organisations which provide financial planning services if an individual wishes this to be done.
%
%447. Section 42 provides that state pension information can be passed to employers and pension scheme providers unless individuals have indicated that they do not want such information disclosed by “opting-out”. The intention is to improve significantly the take-up of combined pension statements by employees and reduce substantially the administrative burdens on employers and pension providers.
%
%448. It also provides that state pension details can be passed to third parties such as organisations which provide financial information services to help individuals identify the most appropriate pension or other saving arrangements provided their express consent has been obtained. In due course, this will enable individuals to access their state pension details electronically through the comparative financial databases which are currently being developed.
%
%449. Subsection (1)  provides that the section is to apply to state pension information held by the Secretary of State or by those providing services to the Secretary of State which are concerned with his social security functions. State pension information is defined in subsection (7) .
%
%450. Subsection (2)  provides that regulations may allow the Secretary of State to disclose or permit the disclosure of state pension information to those specified in subsection (3)  who have made an application for disclosure in the manner prescribed in regulations and in accordance with prescribed conditions.
%
%451. Subsection (3)  sets out the persons who can receive state pension information. These are the trustees and managers of occupational or personal pension schemes, employers, and appropriate third parties engaged in the provision of financial information services.
%
%452. Subsection (4)  sets out some of the conditions which must be included in regulations permitting the disclosure of state pension information. These conditions are that appropriate third parties engaged in the provision of financial information services obtain the consent to the disclosure of his state pension information by the individual concerned and in the case of the other persons referred to in subsection (3)  (the trustees and managers of occupational or personal pension schemes and employers) that either the condition as to consent or the alternative condition referred in subsection (5)  applies.
%
%453. Subsection (5)  sets out the alternative condition referred to in subsection (4)  in relation to the trustees and managers of occupational or personal pension schemes and employers. It provides that steps are to be taken to ensure that individuals are made aware of their right to opt out of the procedures for the provision of state pension details. It also provides prescribed minimum times to ensure that individuals have adequate time to consider what is intended, and opt out if they wish to do so.
%
%454. Subsection (6)  provides that for the purpose of making an application for state pension information, the applicant may disclose to the Secretary of State such information relating to an individual as is prescribed in regulations.
%
%455. Subsection (7)  sets out what constitutes state pension information relating to an individual for the purposes of the section – namely, date of birth and age at which state pension age is reached; amounts of basic and additional state pension entitlement already accrued; and projected basic and additional state pension entitlement.
%
%456. Subsection (8)  provides that regulations made under this section shall be subject to the negative Parliamentary procedure.
%
%457. Subsection (9)  provides that section 189(4)  - (6)  of the Social Security Administration Act 1992 apply to regulations made under this section. The application of section 189(4)  - (6)  is in accordance with the general rules governing subordinate legislation made under powers in that Act and will thereby enable the regulations made under the section to, for example, make different provision for different groups covered by the regulations and to make provision for incidental, supplemental and consequential matters relating to the disclosure of state pension information.
%
%458. Subsection (10)  provides that information can be supplied to the Secretary of State by the Inland Revenue in relation to functions which are conferred on him by regulations under this section.
%
%459. Subsection (11)  provides definitions of terms used in this section.
%Chapter II: Occupational and Personal Pension Schemes
%Selection of trustees and of directors of corporate trustees
%
%460. Sections 43 to 46 amend sections 16, 18 and 21 of the Pensions Act 1995* (member-nominated trustees and directors); they further provide that sections 17, 19 and 20 shall cease to have effect and introduce a new section 18A. Under the current legislation, trustees are required to implement arrangements for at least one third of the scheme trustees to be member-nominated trustees, or where the trustee is a company, for one-third of the directors to be member-nominated directors. However, the employer has the right to implement alternative arrangements that do not include any member trustees, or directors, provided the members agree. Under the new provisions, all schemes will be required to have at least one third member-nominated trustees or directors, but there will be two ways to determine the nomination and selection arrangements: a flexible nomination and selection procedure laid out in regulations, or, alternatively, by the employer proposing nomination and selection arrangements which are subsequently approved by scheme members.
%
%461. Section 16, as amended, will require trustees to ensure that arrangements are put in place for at least one-third of the trustees to be nominated and selected by scheme members. There will be two routes under which member-nominated trustees can be nominated and selected: a statutory route, the nature of which will be determined by reference to section 16 and regulations under section 16, where the trustees are responsible for the precise details of the arrangements and for their implementation; and an alternative route under section 18A (see section 45) where arrangements for the nomination and selection of the scheme trustees are proposed by the employer and implemented by the trustees. Section 16 and regulations made under section 16(9)  will apply to both routes (but section 18A(3)  allows provision different from that made by regulations under section 16(9)  for the scheme specific route). Equivalent provisions apply in relation to trustee companies.
%Section 43: Member-nominated trustees
%
%462. This section amends section 16 to provide a revised statutory framework for appointing member-nominated trustees.
%
%463. The revised provisions make no distinction between “arrangements” and “appropriate rules” so subsections (2)  to (4)  remove references to “appropriate rules” from section 16 of the Pensions Act.
%
%464. Subsection (5)  incorporates the substance of section 20(3)  of the Pensions Act into section 16.  Member-nominated trustees must serve a term of office of between three and six years and be eligible for reselection. The existing section 16(6) , which provides for the determination of the minimum number of member-nominated trustees, and for this number to be exceeded only if the employer agrees, remains unchanged.
%
%465. Subsection (6)  incorporates the substance of section 20(5)  of the Pensions Act by inserting a new subsection (6A)  into section 16.  An employer may require that a non-member can only stand for nomination as a member-nominated trustee if the employer approves.
%
%466. Section 16(7), which provides for all member-nominated trustees to have the same powers remains unchanged.
%
%467. Subsection (7)  amends section 16(8)  to enable arrangements under section 16 to provide for a trustee who changes category of membership (for example, from active to deferred) to cease to be a trustee. The requirement for a member-nominated trustee to stand down if they cease to be a member remains unchanged.
%
%468. Subsection (8)  introduces two new subsections to section 16.  The new section 16(9)  is a regulation-making power that will be used to prescribe what is meant by “nominated and selected by members”, and to further stipulate details of the arrangements the trustees are required to make for nominating and selecting member-nominated trustees. The intention is to give trustees flexibility to adopt arrangements that best suit the circumstances of the scheme, for example by dividing the membership into separate constituencies. Regulations will provide that all active and pensioner members must be given the opportunity to make nominations. The new section 16(10)  incorporates the provisions of section 17(4)  of the Pensions Act into section 16.  As now, the regulations will provide for exemptions for certain types of scheme. Schemes that are currently exempt will continue to be so.
%
%469. Subsection (9)  repeals section 17 of the Pensions Act (employer’s right to propose alternative arrangements).
%Section 44: Corporate trustees
%
%470. This section makes changes to section 18 of the Pensions Act for member-nominated directors in schemes where the trustee is a company similar to the changes in section 16 for individual trustees.
%
%471. In addition, subsection (2)($a$)  extends the scope of section 18 to include all schemes where there is a trustee company and there is no trustee of the scheme who is not a company.
%
%472. Subsection (8)  modifies section 18(8)  to ensure that the membership of different schemes will be aggregated where the trustee company is trustee for more than one scheme, unless the trustee company decides otherwise.
%
%473. Subsection (10)  repeals section 19 and 20 of the Pensions Act (employer’s right to propose alternative arrangements and meaning of appropriate rules).
%Section 45: Employer's proposals for selection of trustees or directors
%
%474. This section introduces a new section 18A. The new section makes provison for the employer to propose arrangements for nominating and selecting trustees of the scheme or directors of a corporate trustee of the scheme.
%New section 18A: Employer’s proposals for selection of trustees or directors
%
%475. New section 18A(1)($a$)  gives employers the right to propose arrangements for nominating and selecting trustees. Subsection (1)($b$)  ensures the arrangements provide for at least one third of the trustees to be member-nominated trustees, and that the other requirements of section 16(3)  to (7)  apply. Subsection (1)($d$)  requires that the proposal is approved by scheme members. Subsection (1)($d$)  also incorporates the regulation-making power similar to that contained in section 21(7)  which will enable a statutory consultation procedure for seeking member approval for the proposal to be prescribed. This will be largely the same as the current procedure that is provided in Schedule 1 to the Occupational Pension Schemes (Member-nominated Trustees and Directors) Regulations 1996, although the procedure may be tightened to reduce any opportunity for abuse. Regulations made under subsection (1)($e$)  will impose additional conditions on employers, for example to give notice to the trustees of the intention to propose arrangements. Section 18A(2)  makes the equivalent provision for trustee companies. Once approved, the trustees are charged with implementing the arrangements.
%
%476. New section 18A(3)  allows regulations governing arrangements under an employer’s proposal to provide for different nomination and selection arrangements from those made under the statutory route. For example, the employer will be able to propose that nominations for trustees are made by organisations representing members (such as Trades Unions and pensioner organisations) as well as members themselves.
%
%477. New section 18A(4)($a$)  provides the power to make regulations governing the manner and time in which trustees must implement approved arrangements. This is similar to the current power under section 21(4)($a$) . Trustees will be given six months following approval to ensure that the arrangements are made, and trustees appointed. Regulations under subsection 18A(4)($b$)  will determine when approval of section 18A arrangements cease to have effect. As now, approval will last for six years. They will also determine what happens when approval of arrangements ceases to have effect without the existing arrangements having been re-approved or fresh arrangements approved.
%
%478. New section 18A(5)  enables regulations to be made about approval of arrangements for the purpose of section 18A. Regulations under subsection 18A(5)($a$)  will give the Occupational Pensions Regulatory Authority (Opra) the discretion to treat proposals as approved in certain circumstances where there is a breach of the requirements of the approval process. Regulations under subsection 18A(5)($b$)  will provide for proposals to be treated as approved by persons who do not object. The existing section 21(8) ($b$)  allows the approval process to operate in this way. Regulations will, as now, provide for proposals to be approved if not more than 10% of those consulted object.
%
%479. New section 18A(6)  permits nominations for a member-nominated trustee or director to be made by an organisation of a prescribed description that represents the interests of members of the scheme. It also permits nominations by such organisations to be the only nominations. It is intended that regulations will prescribe that recognised Trades Unions and pensioner organisations, for example, can make such nominations.
%
%480. New section 18A(7)  disapplies the section as far as it applies to member-nominated trustees in cases where all the trustees comprise all the members, or where there is only a corporate trustee (or trustees).
%
%481. New section 18A(8)  is a regulation-making power to disapply the section for schemes of a prescribed description. This provision is required in addition to the exemptions from sections 16 and 18 because those sections impose a mandatory requirement on all trustees, whereas this section only applies if the employer chooses to propose scheme-specific arrangements. In practice, section 18A will be disapplied for the same classes of scheme that are exempt from sections 16 and 18. 
%
%482. Subsections (2)  and (3)  of section 45 are consequential amendments to, respectively, sections 68(2)  and 117(2)($c$)  of the Pensions Act.
%Section 46: Non-compliance in relation to arrangements or proposals
%
%483. This section contains various consequential amendments to section 21 of the Pensions Act 1995. 
%
%484. All references to appropriate rules are removed, as are references to sections 17 and 19 (which are repealed).
%
%485. A new subsection (2A)  has been added to section 21 to enable Opra to impose sanctions on an employer who fails to carry out the statutory consultation procedure properly. The equivalent provision is currently in sections 17(5)  and 19(5). Opra already has the power under section 21 to impose sanctions on trustees who fail (without reasonable cause, in the case of individual trustees) to comply with the requirements. Opra can prohibit a trustee, or impose a financial penalty.
%Winding-up of schemes
%
%486. These measures aim to speed the process of winding-up by introducing accountability into the winding-up process and by giving Opra a more active role in the process than at present. A consultation paper setting out proposals for speeding up the winding-up process was issued on 27 May 1999.  The comments received were taken into account.
%
%487. Scheme rules or the trust deed setting up the scheme set out the events which may trigger the cessation and winding-up of an occupational pension scheme. These generally are the employer’s insolvency, notice from the employer that he no longer wishes to sponsor the scheme, or failure by the employer to pay contributions within a specified period. It is the trustees or managers who are required to carry out the winding-up.
%
%488. Winding-up can be a time-consuming task, sometimes taking many years, particularly where the scheme records have not been well kept. During this time members may feel particularly vulnerable.
%
%489. The measures aim to ensure that a trustee is in place following the insolvency of the employer so that decisions can be made about the future of the scheme. Where winding-up has started, trustees or managers will be required to make reports to Opra if winding-up is not completed within a specified period of time and Opra will be able to direct action to speed the process along. Opra will also be able to modify scheme rules where they need to be changed to allow winding-up to proceed.
%Section 47: Information to be given to the Authority
%
%490. This section inserts three new sections into the Pensions Act 1995.  It also amends section 118 of that Act to allow these new sections to be modified by regulations to impose the duties on other people (see subsection (4) ). Sections 26A, 26B and 26C set out circumstances in which trustees or managers of schemes or scheme administrators are required to notify Opra during the insolvency of the employer.
%
%491. Subsection (1)  amends references in section 22 of the Pensions Act 1995 to include the new inserted section 26A. Subsection (2)  inserts sections 26A, 26B and 26C.
%New section 26A: Information to be given to the Authority in a s. 22 case
%
%492. New section 26A sets out the circumstances in which the trustees or persons involved in the administration of a scheme must make a report to Opra, where the scheme has to have an independent trustee during the insolvency of the employer (sections 22 and 23 of the Pensions Act 1995).
%
%493. New section 26A(1)  requires the trustees of a scheme, where the scheme has to have an independent person in place as trustee during the insolvency of the employer, to notify Opra that there appears to be no independent trustee unless they have been told by the insolvency practitioner or official receiver that he is satisfied that one of them satisfies the independence test, or they have reasonable grounds to believe that the practitioner or official receiver is satisfied that one of them does so. The notification must be made as soon as reasonably practicable.
%
%494. New section 26A(2)  places on those involved with the administration of the scheme a requirement similar to that in subsection (1)  where there are no trustees.
%
%495. New section 26A(3)  sets out the circumstances where no notification to Opra is required. These are where it appears that the insolvency practitioner or official receiver intends to appoint an independent trustee and that he will do so within a specified period.
%
%496. New section 26A(4)  removes the requirement for a report to be made under subsection (2)  by those involved with the administration of the scheme where it appears that Opra are already aware that the scheme has no trustees.
%
%497. New section 26A(5)  ensures that the requirement in subsection (1)  covers later situations where the practitioner or receiver is no longer satisfied that the independence test is met, even though he may previously have told the trustees that it was met.
%
%498. New section 26A(6)  defines whether the practitioner or receiver is satisfied as to a person’s independent status by reference to the independence test in section 23. 
%
%499. New section 26A(7)  provides that section 10 of the Pensions Act 1995 applies to trustees who fail to take reasonable steps to ensure compliance with the requirements to notify Opra regarding the independent trustee. Section 10 allows Opra to impose financial penalties.
%
%500. New section 26A(8)  provides that section 10 of the Pensions Act 1995 applies to anyone who fails to comply with the subsection (2)  requirement to notify Opra that there are no trustees.
%New section 26B: Information to be given in cases where s. 22 disapplied
%
%501. The new section 26B sets out the circumstances in which reports must be made to Opra on the insolvency of the employer where the scheme is not required to have an independent trustee (section 22 of the Pensions Act 1995).
%
%502. New section 26B(1)  requires the persons involved (if any) in the administration of a trust scheme, where there is no requirement for an independent trustee, to notify Opra where the employer of the scheme is the sole trustee and he becomes insolvent, unless they have an assurance from the employer. For multi-employer schemes this will apply only where all the employers are insolvent.
%
%503. New section 26B(2)  provides that for the purposes of this section an employer’s assurance has been received if the employer has told the persons involved in the administration of the scheme that there is no reason why the employer should not continue to act as a trustee of the scheme, he does not withdraw that statement, and the trustees of the scheme have not changed since the employer has made that statement.
%
%504. New section 26B(3)  removes the requirement for a report to be made under subsection (2)  where it appears that Opra are already aware of the situation or where the prescribed period has not elapsed, or at any other time which is prescribed.
%
%505. New section 26B(4)  provides that section 10 of the Pensions Act 1995 applies to anyone who fails to comply with the requirements in this section.
%New section 26C: Construction of ss. 26A and 26B
%
%506. The new section 26C sets out further details relating to the requirements in new sections 26A and 26B.
%
%507. New section 26C(1)  sets out who is considered to be involved in the administration of the scheme for the purpose of the requirements in sections 26A and 26B. For example, those persons who are involved in the administration of the scheme in their professional capacity, such as actuaries and auditor, the fund manager, the employer of the scheme, their employees, agents or contractors who carry out administration tasks, are not considered to be involved in the administration of the scheme.
%
%508. New section 26C(2)  provides that regulations may add to the list of those who are not considered to be involved in the administration of the scheme.
%
%509. New section 26C(3)  provides that wherever there is a requirement in section 26A or 26B to do something “as soon as reasonably practicable”, that may be replaced by time limits specified in regulations.
%
%510. Subsection (3)  of section 46 makes a consequential amendment to section 118(2)  of the Pensions Act 1995 to allow for regulations to exempt schemes from the new requirements in sections 26A to 26C.
%
%511. Subsection (4)  inserts a new section 118(3)  into that Act to allow for regulations to modify sections 26A and 26B so as to impose the notification duty on persons other than trustees and other than those involved in the administration of the scheme.
%
%512. Subsection (5)  amends the provisions in the Pension Schemes Act 1993 so that regulations may prescribe who is to be treated as a trustee for the purposes of sections 22 to 26 of the Pensions Act 1995 and the new sections inserted by this section.
%Section 48: Modification of scheme to secure winding-up
%
%513. This section inserts a new section 71A into the Pensions Act. This is to extend Opra’s existing powers to modify scheme rules, to enable winding-up to continue.
%New section 71A: Modification by Authority to secure winding-up
%
%514. New section 71A(1)  enables Opra to modify scheme rules to ensure that the scheme is properly wound up but only where the scheme is being wound up and the employer is insolvent.
%
%515. New section 71A(2)  only allows Opra to modify scheme rules where they have been asked by the trustees or managers to do so. The request cannot be made in advance. As with the modification itself, the request may be made only while the scheme is being wound up and the employer is insolvent.
%
%516. New section 71A(3)  requires that unless regulations provide otherwise, the application to Opra must be in writing.
%
%517. New section 71A(4)  allows regulations to set out the detail of the information which is contained in, or documents which must accompany, the application. The regulations may also provide for certain people to be told about the request for a modification; what the notification must contain; for the time limit in which they will have to contact Opra to make representations; and how Opra must deal with the request for modification.
%
%518. New section 71A(5)  limits Opra’s powers to modify scheme rules to the minimum necessary to enable the scheme to be wound up properly and for any modification to be restricted to those which would not have a significant adverse effect on accrued rights or benefit entitlements under the scheme.
%
%519. New section 71A(6)  makes it clear that any modification made by Opra will be as effective in law as if it had been made under scheme rules and without any requirement to obtain consent before any modification can be made.
%
%520. New section 71A(7)  allows regulations to exempt certain types of schemes in particular circumstances or for the requirements in the section to apply with modifications in particular circumstances.
%
%521. New section 71A(8)  sets out the circumstances in which an employer is to be treated as insolvent for the purpose of this section. The circumstances are those which trigger the application of section 22 of the Pensions Act 1995 (or would trigger it if that section applied to the scheme) ie. where an insolvency practitioner or official receiver takes up office. These terms are defined in section 22(3)  by reference to the Insolvency Act 1986. 
%
%522. New section 71A(9)  excludes public service pension schemes from this section.
%Section 49: Reports about winding-up
%
%523. This section introduces a number of provisions including a requirement for trustees or managers to make reports to Opra, a definition of when a scheme begins to wind up and a requirement for records to be kept of a decision to wind up a scheme.
%New section 72A: Reports to Authority about winding-up
%
%524. New section 72A(1)  introduces a requirement for trustees or managers of a scheme which began to wind up after a specified date to make regular reports to Opra about the progress of winding-up.
%
%525. New section 72A(2)  allows regulations to specify when the first report should be made to Opra. That period will be within a specified period of the date on which winding-up began, or the date on which the winding-up was brought within the section (if later).
%
%526. New section 72A (3)  sets out the timing of subsequent reports to Opra which must be made at no more than twelve-monthly intervals after the date of the previous report. If the last report was made late, the next one must still be made no later than twelve months after the last one was due.
%
%527. New section 72A(4)  allows Opra to extend the deadline for making any follow-up reports. Opra can only extend the interval by up to twelve months (under new section 72A(5) ), and can only grant the extension within the time limit, not after it. There is no similar power to extend time for the first report.
%
%528. New section 72A(6)  allows more than one extension of the deadline for the follow-up report, but the total extensions for that report must not exceed the twelve-month limit mentioned in subsection (5).
%
%529. New section 72A(7)  provides that regulations may make requirements as to the reports to Opra, including how the reports should be made, and what they must contain.
%
%530. New section 72A(8)  provides that regulations may provide for circumstances in which reports need not be made to Opra, and may vary the twelve-monthly period in which further reports must be made. It also provides that regulations may alter the periods in which follow-up reports must be made, and the period over which Opra can extend the time limit for those reports.
%
%531. New section 72A(9)  applies sections 3 and 10 of the Pensions Act 1995, so that Opra may prohibit from being a trustee someone who fails to take reasonable steps to ensure compliance, and may impose a financial penalty on a trustee or manager who fails to comply with the requirements.
%
%532. Subsection (2)  of section 49 inserts into section 124 of the Pensions Act 1995 a definition of when winding-up begins for the purposes of Part I of that Act.
%
%533. Subsection (3)  adds to the requirements in section 49 regarding records, by inserting a new section 49A. The new section 49A requires trustees or managers of an occupational pension scheme to keep written records of their decision to wind up the scheme, of decisions about when steps should start to be taken for the purposes of winding-up the scheme, and of any decision to defer winding-up. It provides that regulations may extend the requirements to any person, who although not a trustee or manager, can nevertheless make a decision to wind the scheme up. It also allows regulations to make requirements about the form and content of the record. Sanctions under sections 3 and 10 of the 1995 Act can be imposed for non-compliance. Where regulations extend the requirements to other persons, sanctions may be provided for in regulations (under section 10(3)  of that Act).
%Section 50: Directions for facilitating winding-up
%
%534. This section inserts new section 72B which allows Opra to direct that specific information should be provided, or action taken within a prescribed timescale, where a scheme has begun winding-up. It also inserts new section 72C which imposes sanctions on those not complying with Opra’s directions.
%New section 72B: Directions by Authority for facilitating winding-up
%
%535. New section 72B(1)  provides that where a scheme has begun winding-up, Opra will have power to give directions if they feel it is appropriate to do so on any of the grounds in subsection (2).
%
%536. New section 72B(2)  sets out the grounds Opra may take into account. It also allows regulations to prescribe further circumstances in which Opra may give directions.
%
%537. New section 72B(3)  limits Opra’s powers to direct to where the first report has been made, or should have been made, to Opra under new section 72A, unless regulations prescribe otherwise.
%
%538. New section 72B(4)  allows regulations to provide that in certain circumstances Opra may only give directions when asked to do so by the trustees or managers of schemes.
%
%539. New section 72B(5)  provides that a direction from Opra must be given in writing, and can be given to trustees or managers, persons involved in the administration of the scheme or persons prescribed in regulations.
%
%540. New section 72B(6)  sets out requirements that can be imposed by a direction. They include providing information to the trustees, or managers, or persons involved in the administration of the scheme, or persons prescribed in regulations (which may include Opra), and requiring other steps to be taken.
%
%541. New section 72B(7)  allows Opra to extend the time limit for persons to comply with the direction, on more than one occasion if necessary, where Opra consider it appropriate to do so.
%
%542. New section 72B(8)  allows for regulations to limit what Opra may require in their directions and sets out requirements as to when and how applications must be made for an extension to the period for complying with the direction.
%
%543. New section 72B(9)  sets out who is considered to be involved in the administration of the scheme for the purposes of these requirements. It is almost identical to new section 26C(1)  (see section 47).
%
%544. New section 72B(10)  provides that regulations may add to the list of those who are not considered to be involved in the administration of the scheme. It is identical to new section 26C(2)  (see section 47).
%New section 72C: duty to comply with directions under 72B
%
%545. New section 72C(2)  provides that section 3 of the Pensions Act 1995 (Opra may prohibit a person from being a trustee) applies to any trustee who fails to take reasonable steps to ensure compliance, and has no reasonable excuse.
%
%546. New section 72C(3)  applies section 10 of the 1995 Act (financial penalties) to any trustee or manager who fails to take reasonable steps to ensure compliance, and has no reasonable excuse.
%
%547. New section 72C(4)  applies section 10 to anyone else who fails to comply with a direction, and has no reasonable excuse.
%
%548. New section 72C(5)  provides that any duty of non-disclosure is not a reasonable excuse for failure to supply information in accordance with directions from Opra. The statutory duty to comply with directions will mean that a person complying with a direction will not be in breach of the non-disclosure duty.
%Other provisions
%Section 51: Restriction on index-linking where annuity tied to investments
%
%549. Rights which accrue from 5 April 1988 in respect of Guaranteed Minimum Pension and protected rights have to be indexed by RPI, capped at 3%. If inflation is above 3% SERPS is fully indexed.
%
%550. All rights accrued from 6 April 1997 in salary-related and money purchase occupational schemes have to be indexed at RPI, capped at 5%. Protected rights in appropriate personal pensions are also subject to the same level of indexation. Additional voluntary contributions and personal pensions are not subject to an indexation requirement.
%
%551. The Department of Social Security issued a public consultation document on 31 January 2000 seeking views on whether greater flexibility should be allowed so that members of money purchase schemes could choose to buy either an investment-linked annuity or a traditional index-linked annuity to satisfy the indexation requirements. The document was circulated widely within the pensions industry, employers and was available on the internet for other interested groups and members of the public.
%
%552. The consultation ended on 29 February. 40 responses were received, of which 34 supported the proposal for change and generally welcomed the Government's willingness to recognise innovative annuity products which are being developed by annuity providers.
%
%553. Investment linked-annuities enable the annuitant to benefit from growth in a range of underlying investments after retirement, though this goes hand in hand with a risk of possible falls in pension income if investment performance is poor. Although an investment-linked annuity will not guarantee to produce an increase in the pension each year, such annuities have performed better overall than the traditional index-linked annuity in recent years.
%
%554. The measure in the Act allows money purchase occupational pension schemes to offer their members the option of using the non-protected rights element of their accumulated pension fund accrued from April 1997 to buy an investment-linked annuity instead of an index-linked annuity. They would continue to be able to choose a traditional index-linked annuity if they wished. The section also provides for a power to prescribe the conditions which investment-based annuity products must satisfy (sub-paragraph (1)($c$) ), although it is not envisaged that this power would be used in the short term. Regulations may be considered necessary in the future, however, if investment-based products were to be designed in such a way that they provided a high starting income with little prospect for future increases.
%
%555. This section sets out the circumstances when an investment-linked annuity can be used to satisfy the indexation requirements which are currently contained in section 51(2)  of the Pensions Act 1995. 
%
%556. Subsection (1)  provides for a new section 51A to the Pensions Act 1995 to supersede the requirement to increase a pension in payment annually by the published RPI figure, capped at 5%.
%
%557. Subsection (2)  provides for the insertion of a new section 51A in the Pensions Act 1995. 
%New section 51A: Restriction on increase where annuity tied to investments
%
%558. New section 51A(1)  provides that an annual increase under section 51 is not required in respect of the element of money purchase scheme funds as described in sub-paragraphs 1($a$), ($b$)  and ($c$) .
%
%Sub-paragraph 1($a$)  stipulates that the alternative pension is payable from an investment-linked annuity.
%
%Sub-paragraph 1($b$)  prevents the inclusion of benefits in respect of protected rights.
%
%Sub-paragraph 1($c$)  provides that regulations may prescribe conditions to be satisfied for investment-linked annuity products.
%
%559. New section 51A(2)  provides for the option of an investment-linked annuity whether provided under an annuity contract or payable from the funds of money purchase schemes.
%
%560. New section 51A(3)  provides for the new rule to apply to increases after the date appointed for the new section 51A to come into force.
%Section 52: Information for members of schemes
%
%561. The Government intends to introduce amendments to existing regulations that require annual benefit statements to be sent to members of occupational and personal pension schemes with money purchase benefits.
%
%562. In addition to the existing information about contributions paid and the current value of the “pot”, they will be required to include an illustration of the likely value of the “pot” at retirement age, and the benefits it might provide, expressed in today’s prices.
%
%563. This section makes changes to section 113 of the Pension Schemes Act 1993. 
%
%564. Subsection (1)  adds a new sub-paragraph ($ca$)  to section 113(1)  of the Pension Schemes Act to permit regulations to require an annual benefit statement in a money purchase scheme to include an illustration of the future benefits that might become payable under the scheme.
%
%565. Subsection (2)  adds a new subsection (3A)  to section 113 of the Pension Schemes Act to allow the basis for calculating any forecast of future benefits to be calculated by reference to guidance notes. This will allow the Secretary of State for Social Security to delegate responsibility for deciding the method of calculation to a suitable professional body such as the Institute and Faculty of Actuaries.
%
%566. Subsection (2)  also inserts a new subsection (3B) into section 113 to provide for regulations made under that section to allow Opra to extend time limits for compliance with requirements set out in regulations, in relation to cases where schemes are being wound up.
%Section 53: Jurisdiction of the Pensions Ombudsman
%
%567. The Social Security Act 1990 created the office of Pensions Ombudsman by inserting new provisions in the Social Security Act 1975.  The functions of the Pensions Ombudsman are now contained in sections 145 to 152 of the Pension Schemes Act 1993.  His jurisdiction was extended under amendments to that Act introduced by section 157 of the Pensions Act 1995.  The Pensions Ombudsman can investigate complaints of injustice caused by maladministration and disputes of fact and law brought by members of occupational and personal pension schemes, and their spouses and dependants, against trustees, managers or employers of those schemes. Complaints can also be brought by the same people against the administrators of schemes. The Ombudsman is also able to investigate complaints and disputes from employers against trustees or managers in relation to the same scheme and vice versa for complaints (but not disputes), and investigate complaints from trustees or managers of one scheme against trustees or managers of another.
%
%568. This section extends the Pensions Ombudsman’s jurisdiction by making amendments to section 146 of the Pension Schemes Act 1993.  This section will allow a greater range of people to refer complaints or disputes to the Pensions Ombudsman.
%
%569. Subsection (1)  indicates that this section makes amendments to section 146 of the Pensions Schemes Act 1993. 
%
%570. Subsection (2)  extends the application of section 146(1)  to another type of complaint which the Pensions Ombudsman can investigate. The new section 146(1)(ba) allows the Pensions Ombudsman to investigate complaints made by the independent trustee (the trustee who is required under the Pensions Act 1995 to be in place when the sponsoring employer of a final salary occupational scheme is insolvent) alleging maladministration by the other trustees, or the former trustees, of a scheme. This will enable an independent trustee, if he believes that the actions of other trustees, or former trustees, constitute maladministration which would have a detrimental effect on the scheme members, to refer the matter to the Pensions Ombudsman.
%
%571. Subsection (3)  inserts into section 146(1), by way of new subsections (1)($e$)  to ($g$), additional types of disputes or complaints that the Pensions Ombudsman can investigate.
%
%New section 146(1)($e$)  allows trustees of the same scheme to refer disputes between themselves to the Pensions Ombudsman.  This will include “friendly” disputes where the trustees are seeking a direction as to how they should act.
%
%New section 146(1)($f$)  allows the Pensions Ombudsman to investigate a dispute between the independent trustee and other trustees, or former trustees, of the scheme.  This will mean that an independent trustee, who has concerns about the actions of the trustees or former trustees prior to his appointment, will be able to refer the matter to the Pensions Ombudsman.  At present, the independent trustee and the other trustees of the scheme are barred from referring such matters to the Pensions Ombudsman.
%
%New section 146(1)($g$)  allows a sole trustee to raise a question with the Pensions Ombudsman about the carrying out of his functions.  This will enable sole trustees to obtain a direction from the Pensions Ombudsman regarding how they should act, in the same way as trustees in “friendly” disputes can.
%
%572. Subsection (4)  inserts new subsections (1A) and (1B) into section 146. 
%
%573. New section 146(1A) prevents the Pensions Ombudsman from investigating the complaints or disputes listed in section 146(1)($c$)  to ($g$)  unless they are referred to him by particular people, as provided for in the new subsection (1A)($a$)  to ($e$) .
%
%New section 146(1A)($a$)  reproduces the effect of existing section 146(1)($c$) .  It prevents the Pensions Ombudsman from investigating a dispute between a scheme member or another beneficiary of the scheme and the trustees or employer unless it is referred to him by the member or beneficiary.  This prevents employers or trustees referring disputes with members to the Pensions Ombudsman.
%
%New section 146(1A)($b$)  prevents the Pensions Ombudsman from investigating a dispute between employers and trustees or the trustees of different schemes unless the dispute is referred to him by one of the employers or the trustees.  This removes the bar on trustees referring disputes with the scheme’s sponsoring employer to the Pensions Ombudsman, which is the unintentional effect of the current wording of section 146($d$) .
%
%New section 146(1A)($c$)  limits the Pensions Ombudsman to only investigating a dispute between the trustees of the same scheme in circumstances where half or more of the trustee board has agreed to refer it to him.  Having half or more of the trustee board agree to refer the matter will prevent a minority in the board delaying the actions of the majority.  This will also allow trustees to seek clarification of scheme rules without having to go to court.  This will be particularly useful when a scheme is winding up, as it will reduce costs on the scheme at a time when it needs to conserve its resources.
%
%New section 146(1A)($d$)  allows only the independent trustee of a scheme subject to insolvency procedures to refer a dispute to the Pensions Ombudsman and not the other trustees of the scheme.
%
%New section 146(1A)($e$)  ensures that the Pensions Ombudsman will not accept a question referred to him about the functions of the sole trustee unless it is referred to him by that sole trustee.
%
%New section 146(1B) ensures that the Pensions Ombudsman can treat a question referred to him by a sole trustee as if it were a reference to him, or determination by him, of a dispute.
%
%574. Subsection (5)  will allow members of a personal pension scheme to make complaints about actions of the employer. At present, if an employer is involved in the running of a personal pension scheme, particularly a group personal pension scheme, members of the scheme cannot refer complaints about the employer’s actions to the Pensions Ombudsman.
%
%575. Subsection (6)  makes replacement provision in respect of one of the circumstances where the Pensions Ombudsman cannot investigate. At present, if a case has gone to an employment tribunal or a court, even in error, the issue cannot then be referred to the Pensions Ombudsman. The new provision will allow the Pensions Ombudsman to accept a complaint or a dispute for investigation where the subject matter has previously gone before an employment tribunal or a court, and the case has been discontinued (unless this was on the basis of a settlement or compromise). Subsection (10)  ensures that the changes made by subsection (6)  to the Pensions Ombudsman’s jurisdiction will not apply to any cases that were referred to him before the provisions come into force.
%
%576. Subsection (7)  provides that a person entitled to a pension credit as against the trustees or managers of a scheme can be considered an actual or potential beneficiary within the meaning of section 146(7), for the purposes of making a complaint or referring a dispute to the Pensions Ombudsman. This will allow those who have an entitlement to a pension credit, but who will not become a member of the scheme awarding the credit, to make a complaint or refer a dispute to the Pensions Ombudsman.
%
%577. Subsection (8)  inserts a definition of “independent trustee” into section 146(8) . The independent trustee will be the trustee appointed as such by the insolvency practitioner under section 23(3)($b$)  of the Pensions Act 1995. 
%
%578. Subsection (9)  replaces the words “complaints and disputes” in 146(1)  with the word “matters”. This ensures that the Pensions Ombudsman can consider questions from sole trustees which could not be regarded as a dispute. It also replaces the latter part of section 146(1)($b$)  of the Pension Schemes Act 1993.  This clarifies the position regarding the identity of the scheme to which the complaint relates in cases where complaints of maladministration are made by the trustees of one scheme against the trustees of another scheme. This subsection also removes the words “which arises” from sections 146(1)($c$)  and 146(1($d$) . This will allow disputes between current and former trustees to be considered by the Pensions Ombudsman.
%Section 54: Investigations by the Pensions Ombudsman
%
%579. As a result of a Court of Appeal judgement, under the current legislation, the Pensions Ombudsman should not accept a case if the investigation of it would impact upon the interests, particularly the financial interests, of those not directly involved in the case. This is because those not directly involved in the case are currently not able to make representations to the Ombudsman and are not, therefore, bound by his determinations. This section amends sections 148, 149 and 151 of the Pension Schemes Act 1993 as amended by the Pensions Act 1995. 
%
%580. Subsection (2)  inserts new paragraphs (ba) and (bb) into section 148(5). They extend the meaning of who is a party to an investigation for the purposes of staying court proceedings. These new paragraphs allow the Pensions Ombudsman to link to a case those whose interests may be affected by the complaint or dispute or its outcome.
%
%581. Subsection (3)  replaces subsection (1)  of section 149 with a new section 149(1)  which lists the person to whom the Pensions Ombudsman is obliged to give the opportunity to comment, with regard to matters being investigated by him. The replacement subsection obliges the Pensions Ombudsman to give those who are being complained against, those who are responsible for the management of schemes to which the dispute relates, and those whose interests are, or may be, affected, the opportunity to put their point of view to him.
%
%582. Subsection (3)  also inserts two new subsections (1A) and (1B) into section 149.  Inserted subsection (1A) ensures the Pensions Ombudsman is not required to give an opportunity to make representations from someone who (as the person making the complaint or reference) has had adequate opportunity to comment or whose interests are being represented by a person appointed to do so. Inserted subsection (1B) makes clear that if a person has been appointed to represent a group, after making initial representations on his own behalf, that person should also be given the opportunity to make comments as a representative of that group.
%
%583. Subsection (4)  inserts new paragraph (ba) in section 149(3), which lists those matters that can be covered in the Pensions Ombudsman’s procedure rules. New paragraph (ba) allows rules to be made permitting the Pensions Ombudsman to appoint a person to represent a group of those who have the same interest in a complaint, for instance, such a group as all the pensioner members. It will then be this appointed person who will make representations on behalf of that group. The precise manner in which these representative persons will be appointed will be laid out in the Personal and Occupational Pension Schemes (Pensions Ombudsman) (Procedure) Rules. The procedure for selection will ensure that those nominated as representing a particular group can satisfy the Pensions Ombudsman that they are truly representative of that group and do not have a conflict of interest in the particular case.
%
%584. Subsection (5)  inserts new paragraph ($d$)  which adds an additional item in the list of items that can be included in the rules. This will enable the procedure rules to include provisions to allow the Pensions Ombudsman to order that the cost of legal expenses in a particular case can be met from the funds of the scheme. It is envisaged that such orders will be made when the case is particularly complex and involves the interests of several groups. The procedure rules may state that the order should cover only certain expenses up to a certain limit.
%
%585. Subsection (6)  inserts subsection (8)  into section 149.  This is intended to ensure that those whose interests may be affected by any determination, or any directions the Pensions Ombudsman may give in relation to the dispute, will also have the opportunity to make representations rather than only giving the opportunity to those with a direct interest in the complaint or dispute itself.
%
%586. Subsection (7)  inserts new paragraph ($c$)  into subsection (1)  of section 151.  Section 151(1)  specifies who should be given notice of the Pensions Ombudsman’s determination in a particular case. The additional provision requires the Pensions Ombudsman to issue a copy of his determination in a particular case to all those who could have commented on the allegations. Therefore, determinations will be sent to those against whom the allegations are made and to those who could have made representations to the Pensions Ombudsman. These would be either those identified by the Pensions Ombudsman as able to make representations directly to him on their own behalf, or those who are representing groups of individuals who have the same interest.
%
%587. Subsection (8)  replaces part of subsection (3)  of section 151.  Subsection (3)  specifies who will be bound by the Pensions Ombudsman’s determination. This ensures that those who have had the opportunity to comment or make representations – either individually or via an appointed person – will be bound by the Pensions Ombudsman’s determination. Those who are bound by the determination can appeal against it on a point of law to the High Court (see section 151(4) ).
%
%588. Subsection (9)  ensures that these changes to the Pensions Ombudsman’s remit will not apply to any cases that are referred to him before the provisions come into force.
%Section 55: Prohibition on different rules for overseas residents
%
%589. The Council of the European Union adopted Council Directive 98/49/EC on 29 June 1998.  Its purpose is to safeguard the occupational pension rights of employed and self-employed workers who move within the European Community, and thereby promote the free movement of workers. Occupational pension schemes in the UK already operate within the spirit of the Directive, but existing legislation does not currently oblige schemes to comply with two specific requirements of the Directive. This section is intended to ensure compliance with the Directive by:
%
%    preventing occupational pension schemes from having scheme rules which allow the accrued pension entitlement of members or beneficiaries to be altered because the member or beneficiary wants payment to be made anywhere outside the UK; and
%
%    allowing workers who work outside the UK to continue membership of their employer’s UK occupational pension scheme. Any such scheme members, and the sponsoring employer, will be able to continue to make contributions to the scheme, subject only to limitations imposed by the Inland Revenue.
%
%590. There will be two regulation-making powers to enable specific exceptions to the rules on payment of pension and the right to remain a member of a UK scheme, provided such exceptions do not contravene the terms of the Directive.
%
%591. Section 55 inserts a new section 66A in the Pensions Act 1995.  The provisions in this new section will be brought into force from a date to be established by order made by statutory instrument.
%New section 66A: Prohibition on different rules for overseas residents etc
%
%592. New section 66A(1)  provides that this section applies to an occupational pension scheme which has any rule that contravenes the requirements in subsections (2)  and (3)  in respect of scheme membership, payment of scheme benefits and the payment of contributions.
%
%593. New section 66A(2)  prevents discrimination in respect of the entitlement to pension benefits of a member or beneficiary, and prevents any discrimination in respect of the payment of those benefits according to whether or not a country outside of the United Kingdom is to be the destination of that payment. Exceptions to the application of the provisions of this subsection may be made by regulations. New subsection (4)  provides that the date from which schemes will be in contravention in respect of subsection (2)  will be from the day section 55 of the Child Support Pensions and Social Security Act 2000 is brought into force.
%
%594. New section 66A(3)  stops occupational pension schemes having a rule to prevent workers who are posted to work in a country outside of the United Kingdom from continuing to remain eligible to be members of that occupational pension scheme. Members may not be prevented from making contributions to their occupational pension scheme. That occupational pension scheme must not have a rule which prevents the scheme accepting contributions from the sponsoring employer in respect of members who are posted to work wholly or partly outside of the United Kingdom. Exceptions to the application of the provisions of this subsection may be made by regulations. New subsection (5)  provides that the date from which schemes will be in contravention in respect of subsection (2)  will be from the day section 55 of the Child Support Pensions and Social Security Act 2000 is brought into force.
%
%595. New section 66A(6)  allows for deductions such as income tax to be made from pension benefits due to members and beneficiaries, notwithstanding any discriminatory effect. Similarly, it is made clear that schemes continue to comply with the conditions for approval, exemption or tax relief given or available under the Tax Acts.
%Section 56: Miscellaneous amendments and alternative to anti-franking rules
%
%596. This section brings into force Schedule 5 which makes various amendments to the Pension Schemes Act 1993 and the Pensions Act 1995. 
%Schedule 5
%Part I: Miscellaneous Amendments
%Paragraph 1: Guaranteed minimum for widows and widowers
%
%597. These provisions amend section 17 of the Pension Schemes Act 1993 and are consequential upon the introduction of new bereavement benefits under the Welfare Reform and Pensions Act 1999*. The relevant provisions in the 1999 Act are expected to be brought into force from 5 April 2001.  Sub-paragraph (1)  inserts new subsection (4A)  in section 17. 
%
%New subsection (4A) ($a$)  provides that the scheme must provide a Guaranteed Minimum Pension (GMP) for the widow or widower for any period for which a Category B pension is payable by virtue of the earner’s contributions, or would have been payable but for the overlapping benefit provisions in section 43(1)  of the Social Security Contributions and Benefits Act 1992.  This restates the existing law.
%
%New subsection (4A) ($b$)  ensures that a GMP is payable for any period for which Widowed Parents Allowance (WPA) or Bereavement Allowance (BA) is payable to the widow or widower by virtue of the earner’s contributions.
%
%New subsection (4A) ($c$)  ensures that where a person ceases to be entitled to WPA or BA when over 45, that person will still continue to receive a GMP, provided that he or she is not cohabiting with a person of the opposite sex and provided that he or she has not remarried. Currently, a person entitled to bereavement benefits (widowed mother’s allowance or widow’s pension) when over the age of 45 continues to receive those benefits, and accordingly a GMP, until state pension age, unless he or she remarries or cohabits with a person of the opposite sex. New subsection (4A) ($c$)  thus preserves the current position as regards GMPs despite the fact that the position as regards entitlement to bereavement benefits is to change.
%
%598. Sub-paragraphs (2)  and (3)  make minor amendments designed to ensure that people whose entitlement to bereavement benefits continues under the existing law also continue to be entitled to GMPs under the existing law.
%Paragraph 2: Transfer of rights to overseas personal pension schemes
%
%599. Section 1 of the Pensions Schemes Act 1993 provides a definition of a personal pension scheme, the scope of which is limited to schemes providing benefits to, or in respect of, persons employed in Great Britain. The effect of this is to prevent the transfer of protected rights or Guaranteed Minimum Pension rights to a personal pension scheme set up and administered wholly or primarily overseas. This paragraph amends sections 20 and 28 of the Pension Schemes Act 1993 in order to permit such rights to be transferred to overseas arrangements.
%Paragraph 3: Protected rights
%
%600. Protected rights are (subject to rare exceptions) that part of a member’s fund within a personal pension or occupational money purchase scheme that is derived from the National Insurance contribution rebate.
%
%601. Section 28 of the Pension Schemes Act 1993 provides that effect may only be given to protected rights in the way specified in that section. Section 28 permits effect to be given to protected rights by way of a lump sum only in limited circumstances and, in particular, not before the member has reached age 60. 
%
%602. Paragraph 3 amends section 28 to insert a new subsection (4A)  and (4B).
%
%New section 28(4A)  provides for effect to be given to a member’s protected rights in an occupational pension scheme by way of a lump sum where the trustees or managers of the scheme are satisfied that the member, whatever his age, is terminally ill and likely to die within a year.
%
%New section 28(4B) restricts the amount payable under subsection (4A)  where the member is a married person on the date on which the lump sum becomes payable.  The balance of the protected rights will then go to provide for survivors’ benefits.  The amount payable under this subsection is restricted to no more than a half of the member’s protected rights.
%Paragraph 4: Review and alteration of rates of contribution
%
%603. This paragraph amends sections 42(1)($a$)(i)  and (3)  of the Pension Schemes Act 1993 so that the cross-references to section 41 in these sections take account of the changes made to that section by paragraph 127 of Schedule 7 to the Social Security Act 1998. 
%Paragraph 5: Contributions equivalent premiums
%
%604. Paragraph 5(1)  substitutes subsection (4)  and introduces a new subsection (4A)  in section 58 in the Pension Schemes Act 1993 to ensure that Contributions Equivalent Premiums (CEPs) continue to be equivalent to the National Insurance contribution (NIC) rebate. The CEP is the amount that a contracted-out salary related scheme is required to pay in order for someone with less than two years’ qualifying service in the scheme to be reinstated into the State Earnings Related Pension Scheme (SERPS).
%
%605. At present, section 58(4)  provides for the CEP to be the difference between the amount of Class 1 contributions payable in respect of the earner’s contracted-out employment and the amount of those contributions that would have been payable had the employment not been contracted-out. This method of calculation ensures that CEPs relating to periods prior to April 1999 are equivalent to the contracted-out rebate. Following the introduction of a new Earnings Threshold (the level of earnings at which an employer becomes liable to pay Class 1 contributions) on 6 April 1999, the existing method of calculation no longer ensures that the CEP is equivalent to the rebate. All CEPs in respect of periods after 6 April 1999 would be lower than the rebate.
%
%New section 58(4)  and 58(4A)  ensure that the CEP will be equal to the amount of the NIC rebate payable in respect of contracting-out for periods after 6 April 1999 (as it is already for periods before 6 April 1999).
%
%New section 58(4A)  provides that where trivial or fractional amounts were not included in the calculation of the rebate they are not included in the calculation of the CEP.
%
%606. Paragraph 5(2)  amends subsection (2)  of section 61 of the Pensions Schemes Act 1993 to ensure that the employee’s share of the CEP continues to be equal to the actual reduction in his primary Class 1 contributions paid throughout the period of contracting-out.
%
%607. At present, section 61(2)  provides for the employee’s share of the CEP to be based on the contracted-out rebate, which is currently equal to the actual reduction in the primary Class 1 contribution. When a new Primary Threshold (the level of earnings at which an employee will become liable to pay Class 1 contributions) is introduced on 6 April 2000, section 61(2)  will permit schemes to recover from employees more than that actual reduction in certain cases.
%
%608. Paragraph 5(3)  substitutes a new paragraph ($b$)  in section 63(1)  of the Pensions Schemes Act 1993 so that the reference to section 58 in that paragraph takes account of the changes being made by paragraph 5(1) .
%
%609. Paragraph 5(4)  ensures that the amendments made by paragraphs 5(1), 5(2)  and 5(3)  have effect in relation to any CEP payable on or after 6 April 1999. 
%Paragraph 6: Contribution equivalent premiums: Northern Ireland
%
%610. This paragraph makes corresponding provision relating to the CEP for Northern Ireland.
%Paragraph 7: Use of cash equivalent for annuity
%
%611. Where a member of a contracted-out money purchase occupational pension scheme exercises his right to take a cash equivalent transfer value of his accrued rights, section 95(4)  of the Pension Schemes Act 1993 prohibits the purchase of an annuity. A member may ask for the cash equivalent transfer value to be transferred to another suitable occupational pension scheme or an appropriate personal pension. This paragraph removes the prohibition on annuity purchase and gives the member a further option for the use of his cash equivalent transfer value.
%Paragraph 8: Transfer values where pension in payment
%
%612. Subject to limited exemptions, members of occupational pension schemes are prohibited from taking their pension before they actually retire or leave service. Inland Revenue has proposed to use their discretion so that occupational pension scheme members may receive all or part of their accrued pension while still continuing in pensionable employment. Scheme members taking up this option would lose their right to a cash equivalent transfer value, since section 98(7)  of the Pension Schemes Act 1993 removes this right if any part of a pension is in payment.
%
%613. This paragraph amends section 98(7)  so that a member will be able to take a transfer of his rights which have not come into payment. It also amends section 97(2)  to allow regulations to take account of the amount of pension already in payment when calculating a cash equivalent transfer value. The definition of pensioner member in section 124(1)  of the Pensions Act 1995 is amended so as to exclude a person with pension rights accruing as an active member of a scheme.
%Paragraph 9: Information about contracting-out
%
%614. This paragraph substitutes a new section 156 in the Pension Schemes Act 1993 to make further provision for the information which may be supplied to pension scheme administrators in the light of changes made to contracting-out arrangements by the Pensions Act 1995.  At present, section 156 allows the Secretary of State or the Inland Revenue to provide information to pension scheme administrators in connection with any Guaranteed Minimum Pension (GMP) or its calculation. As currently in force, section 156 does not apply to appropriate personal pension schemes (APPS) and specifically excludes occupational money purchase schemes (COMPS).
%New section 156: Information for purposes of contracting-out
%
%New section 156(1)  enables the Secretary of State or the Inland Revenue to provide trustees or managers of any occupational pension scheme or APPS with the information they are likely to need to enable them to discharge their obligations under the contracting-out arrangements in Part III of the Pension Schemes Act 1993.  This will include, for instance, the information which scheme administrators need to help them determine the correct level of contracting-out benefit.
%
%New section 156(2)  enables the Secretary of State or the Inland Revenue to provide the same information to other persons in categories specified in regulations and is currently provided for by section 156($b$) .
%Paragraph 10: Register of disqualified trustees
%
%615. Section 29(3)  and (4)  of the Pensions Act 1995 specify the circumstances in which Opra may disqualify a person from being a trustee of an occupational pension scheme. Section 30(7)  of the Act requires Opra to keep a register of all persons it disqualifies (the register does not cover automatic disqualifications under section 29(1) ). Opra must, where it receives a request to do so, disclose whether a person named in the request is included in the register as being disqualified in respect of the particular scheme named in the request. This means that Opra may only answer “yes” or “no” to the enquiry and cannot volunteer other information which may be relevant. There is also no requirement for the register to be open to public inspection.
%
%616. This paragraph inserts a requirement for Opra to make the register available for inspection in person by the public. It expands on the requirement for Opra to respond to requests. Opra still cannot volunteer information, but, if requested to do so, it must disclose whether a person named in the request is disqualified in respect of a scheme specified in the request or in respect of all schemes. It also allows Opra to publish, in a medium of its choosing, lists of those who appear on the register, and the fact that they are disqualified from being a trustee of all schemes, some schemes or a single scheme. The full name (including initials and titles) and date of birth must be listed if the Authority has a record of them, even if those matters are not recorded in the register itself. The schemes themselves will not be named.
%
%617. This will provide easier access to the register for those responsible for appointing trustees and will thus reduce the risk of disqualified people being appointed as trustees. A person’s name will not be published in respect of any particular disqualification until either time limits for appeals and for applications to review that disqualification decision are passed, or (where the time limit has not passed) it is unlikely that there will be an appeal or application for review, or where an appeal or review is pending.
%Paragraph 11: Conditions of payment of surplus to an employee
%
%618. This paragraph makes technical changes to section 37(4)($d$)  of the Pensions Act 1995 and will allow occupational pension schemes that are making payments from surplus funds to an employer to use their own scheme rules to make increases to pensioner members from the surplus. Occupational pension schemes which have pension funds which are surplus to liabilities are required to take steps to reduce the surplus. If the employer wants to take a refund, the scheme must first award increases to pensioner members’ pensions. These amendments will allow schemes to avoid recalculating increases which have already been granted, under scheme rules. Pensioner members will not suffer any financial loss as a result of the proposed changes.
%Paragraph 12: Duties relating to statements of contributions
%
%619. The trustees or managers of every occupational pension scheme are required to appoint an auditor to obtain audited accounts and a statement about the prompt payment of contributions under the scheme during the preceding scheme year. In an “earmarked scheme” (which is a money purchase occupational scheme under which all the benefits provided are secured by one or more contracts of insurance, or by annuity contracts which are specifically allocated to the provision of benefits to, or in respect of, individual members) the auditor is only under a statutory obligation to produce a statement about contributions.
%
%620. This paragraph replaces section 41(5)  of the Pensions Act 1995 to enable regulations to be made permitting earmarked schemes to obtain a statement about contributions from a prescribed person or body as an alternative to the scheme appointing an auditor for this purpose. The existing section 41(5)($a$)  provides for regulations to prescribe the persons who may act as auditors or actuaries. The substituted paragraph will have the same effect. The new subsection (5A)  enables regulations to be made which impose a duty on the trustees or managers of earmarked schemes to provide the person making the statement about contributions with sufficient information to enable them to do so. The new subsection (5B) allows for the imposition of civil penalties by Opra on any trustee or manager of an earmarked scheme who fails to provide the information which they are required to provide by regulations made under subsection (5A) .
%
%621. The paragraph also amends section 88 of the Pensions Act 1995.  The new subsection (5)  places a duty on the person providing the statement to report to Opra if contributions have not been paid on time during the course of the scheme year. The new subsection (6)  provides that Opra may impose a civil sanction on any person who fails to make such a report within the time limit set out in regulations.
%Part II Alternative to anti-franking rules
%
%622. Paragraphs 14 to 17 of Schedule 5 introduce a new minimum benefits test which replaces the anti-franking provisions set out in sections 87 to 92 of the Pension Schemes Act (PSA) 1993.  The existing anti-franking legislation prohibits occupational pension schemes from funding increases to Guaranteed Minimum Pensions (GMPs) from other scheme benefits. This principle is reflected in the alternative rules in paragraphs 14 to 17 and the protection is extended to rights built up on after 6 April 1997 (which replaced GMPs). Rights accrued after the end of a period of contracted-out service and late retirement enhancements will no longer be protected. The new provisions prevent schemes from offsetting their pre-6 April 1997 pensions against their post-6 April 1997 pensions, however schemes will be allowed, as at present, to fund the first increase to the GMP, required in the tax year after the one in which it comes into payment, from the scheme pension.
%Paragraph 14: Cases in which alternative applies
%
%623. The new provisions apply to all occupational pension schemes that hold GMP rights, subject to exceptions to be prescribed in regulations. All members who left pensionable service (or died) after the legislation comes into force and whose pensions (or survivors’ pensions) become payable after that date will have their benefits calculated under the new arrangements. However, the scheme managers or trustees will be able to elect, in a manner to be prescribed, to operate the new rules for members who left pensionable service before these provisions come into force. Where a scheme elects to calculate early leavers’ preserved pensions by reference to the new arrangements, that election must apply to all such deferred members and is a once and for all choice.
%Paragraph 15: Alternative rules
%
%624. The minimum benefit rules underpin a scheme’s own benefit formula. To check whether the level of pension payable to a scheme member meets the statutory minimum, the scheme administrator will undertake a notional calculation as follows:
%
%    Step 1: Calculate the member’s GMP entitlement.  For the purpose of this calculation, the amount of GMP would include increases in deferment required under section 15 of the PSA, statutory revaluation under section 16 of the PSA and increases in payment under section 109 of the PSA.
%
%    Step 2: Calculate the amount of GMP at the termination of salary-related pensionable service, exclude any increases required under sections 15 or 109 of the PSA, or early leaver revaluation required under 16(3), but include any revaluation under section 16(1)  of the PSA (section 148 revaluation).  The revaluation on the GMP, for the purpose of this step, is to be calculated up to the tax year before the one in which the member left salary-related pensionable service in the scheme, or the tax year before the one in which s/he reached State Pension Age, whichever is the earlier.  State Pension Age in this context means 60 for a woman and 65 for a man.
%
%    Step 3: Determine whether there are any benefits in excess of the GMP which derive from pre-April 1997 rights.  This can be done by deducting the amount in step 2 from the benefits that are attributable to all the pre-April 1997 rights.
%
%    Step 4: Calculate the amount of any such benefits in excess of the GMP.  The level of earnings used must not be lower than that those used to calculate the post-April 1997 benefits under step 6. 
%
%    Step 5: Revalue the pre-April 1997 excess over the GMP, in accordance with the rules in chapter II of Part IV of the PSA (Revaluation of Accrued Benefits (excluding Guaranteed Minimum Pensions)).
%
%    Step 6: Calculate any benefits accruing in the scheme after 6 April 1997. 
%
%    Step 7: Revalue post-6 April 1997 benefits, in accordance with the rules on revaluation set out in chapter II of Part IV of the PSA (Revaluation of Accrued Benefits (excluding GMPs)) and index them as required by section 51 (Indexation) of the Pensions Act 1995. 
%
%    Step 8: Add together the GMP in step 1 plus the revalued excess over the GMP calculated under step 5 and the post-April 1997 rights as calculated under step 7.   This is the minimum pension payable.
%
%Paragraph 16: Relationship between alternative rules and other rules
%
%625. The minimum benefits rule does not directly apply to the calculation of alternatives to Short Service Benefits provided under section 73(2)($b$)  of the PSA. However, the Short Service Benefit on which the alternative is based must itself be calculated in compliance with the minimum benefits test. The new test overrides scheme rules, where the two conflict. For the purposes of calculating transfer values, schemes will be treated as having these provisions within their rules. The test must be undertaken before the level of a member’s pension is adjusted to take account of commutation, forfeiture, suspension, charges, liens or set-offs. The remainder of the paragraph sets out the definitions of phrases used.
%Paragraph 17: Supplemental
%
%626. Paragraph 17 gives the Secretary of State a power to modify in regulations the provisions in paragraphs 14 to 16.  The exercise of this power is subject to negative procedures, ie subject to annulment in pursuance of a resolution in either House of Parliament.
%Chapter III: War Pensions
%Background
%The Current Position
%
%627. The war pensions scheme is long established, with most provisions originating from around the time of the First and Second World Wars. The legislation and procedures governing decision-making and appeals have not significantly changed since then. War Pensions legislation permits awards to be made in respect of any disablement (physical or mental) or death due to service. Awards vary according to the assessed level of disablement.
%Appeals
%
%628. Where a claim to a war pension is rejected, there is a right of appeal to the independent Pensions Appeal Tribunals (PAT). Most decisions about entitlement to a war pension or assessment of the level of disability are appealable. But certain decisions, such as entitlement to supplementary allowances (which can be paid in addition to a basic war pension) do not carry a statutory right of appeal. There are also groups of people, such as those who served in the inter-war years, that do not have appeal rights. Instead, War Pensions Committees (a countrywide network of statutory bodies comprised of volunteers appointed by the Secretary of State) hear these cases and can make non-binding recommendations to the Secretary of State.
%The Pensions Appeal Tribunals
%
%629. The PATs are completely separate from the appeals arrangements that apply to Social Security benefits. They are administered by the Lord Chancellor’s Department, the Scottish Courts Administration and the Northern Ireland Court Service. The current system of appeals is slow, in part due to the complexity of the schemes, with waiting times averaging two years (a year for the War Pensions Agency to prepare the papers, and a year for the PATs to list and hear the appeal) although recently performance has improved.
%
%630. The existing legislation provides for varying time limits for different types of appeal. It envisaged a 12-month time limit for some parts of the scheme and a 3-month time limit for others. For a variety of reasons, the envisaged time limits are not, however, always applied in practice. In part this is due to the fact that the PAT can hear late appeals when the appellant demonstrates that there is “reasonable excuse” for the delay in submitting the appeal. “Reasonable excuse” is not defined in the legislation and there are no PAT guidelines on its interpretation. In practice, most late applications are heard and so the time limits are not applied.
%Composition of the Pensions Appeals Tribunals
%
%631. Currently the Tribunals are composed of –
%Entitlement Appeals	a legally qualified member
%a medically qualified member
%a “service” member
%Assessment Appeals	two medically qualified members
%a “service” member.
%
%632. Currently the “service” member must be of the same gender, have held similar rank and had a similar service history to the appellant.
%The Central Advisory Committee on War Pensions
%
%633. This Committee, which is a statutory advisory body, was established in 1921 to “consider such matters as may be put before them by the Minister for their advice”. It has been required, since 1970, to include at least 12 War Pensions Committee chairmen amongst its membership. However, War Pensions Committees have reduced from 149 in 1970 to just 29 now, and are again due for reconstitution on 1st January 2001. 
%Recent Developments
%
%634. In April 1999, independent consultants working with the War Pensions Agency (WPA) published a report A Review of Decision Making and Appeals Process. The report recommended a variety of measures including the extension of appeal rights and changes to appeal time limits. The Social Security Select Committee welcomed the review. Representatives of ex-service organisations have received the report and members of the Central Advisory Committee on War Pensions were able to address this matter at their meetings, with the Parliamentary under Secretary, in June and December 1999.  The WPA has since completed a feasibility study that concluded that almost all of the proposals identified in the report were both desirable and achievable.
%The Measures in the ActExtension of Appeal Rights
%
%635. This provision will enable the Secretary of State to increase the scope of appeal rights through a power to permit the creation of new appeal rights, by affirmative regulations, and the repeal of a provision that prohibits appeals related to service before 3 September 1939. 
%
%636. The intention is to use this provision to provide appeal rights that are similar to those provided in the Social Security scheme. For example, certain war pensions supplementary allowances do not have appeal rights whereas similar social security benefits already carry a right of appeal.
%
%637. Appeals relating to the new appeal rights will be heard by the PATs. There is also a provision for these decisions to be set aside or to be appealed beyond the PAT to the High Court, as with certain other types of appeal.
%Appeal Time Limits
%
%638. All appeals will now be subject to a statutory 6-month appeal time limit, except for interim assessment appeals where the existing statutory 3-month time limit will be retained. There is also a power to make regulations providing for the grounds upon which a PAT may hear a late appeal, which is defined in section 57(2)  of this Act as being an appeal received in the 12 month period after the relevant statutory appeal time limit has expired. Transitional protection is provided in those areas where the appeal time limit is to be reduced. This will have the effect demonstrated in the example below.
%
%TYPE OF DECISION
%
%(Made before the provisions come into force)
%	EFFECT
%
%Note:
%
%Interim assessments will not be affected.  They will retain their current 3-month appeal time limit.
%
%Assume for this example only that the commencement date of new provisions is 1 July 2001, and that regulations may have been made bringing in the "late" appeal time limit.
%Entitlement decision	All decisions, regardless of the date on which they were made and notified, will have one year from 1 July 2001 in which to submit an appeal.  The time within which an appeal must be brought will therefore expire on 30 June 2002 although a “late” appeal may be brought up to 30 June 2003. 
%Final Assessment	Will retain the current 12-month appeal time limit from date of notification.  So, if notification takes place on 30 June 2001 (the last available day prior to commencement) the time within which an appeal must be brought will therefore expire on 29 June 2002, although a late appeal may be brought up to 29 June 2003. 
%Jurisdiction of Tribunals
%
%639. The tribunal will not be required to consider issues that have not been raised by the appellant or the Secretary of State. Additionally the tribunal will only take into account matters that occurred up to the date the decision that is under appeal was made. Any changes in circumstances that occur after a decision is made should be notified to the Secretary of State who may review the original decision and issue a further notification which may provide a further appeal right. These provisions are similar to those in the Social Security Act 1998*.
%Composition of Tribunals
%
%640. This section provides for a President and a Deputy President to be appointed for each part of the United Kingdom. The President will be able to issue directions and will be responsible for deciding the appropriate composition of appeal tribunals either on a case-by-case basis or according to the type of case. All tribunals will be required to include a legally qualified member. But, because of the reducing pool of people with relevant expertise, eg Civilian Defence Volunteers, the requirement for them to include a “service” member of the same gender and rank as the appellant will be removed. Instead, the Lord Chancellor will have a duty to appoint persons with knowledge or experience of service life to the pool of tribunal members and in addition the power also to appoint suitably experienced lay members to the tribunal pool.
%Composition of the Central Advisory Committee on War Pensions
%
%641. The number of War Pensions Committees continues to reduce. The Secretary of State will no longer be required to appoint twelve war pensions committee chairmen to the CAC, but may select an appropriate number provided at least one chairman is appointed.
%Commentary on Sections
%Section 57: Rights of Appeal
%
%642. This section provides the scope for the Secretary of State to give further appeal rights to war pensioners.
%
%643. Subsection (1)  inserts a new section 5A into the Pensions Appeal Tribunals Act 1943.  Section 5A provides for new appeal rights in areas where they do not currently exist.
%New section 5A: Appeals in other cases
%
%644. New section 5A(1)  provides that where a “specified decision” (defined in new section 5A(2)  below) is made in connection with a claim made under sections 1, 2 or 3 of the 1943 Act, the claimant will be notified of the decision and that decision will carry a right of appeal to the PAT. Section 1 refers to members of the naval, military or air forces, section 2 refers to mercantile mariners, pilots etc, and section 3 refers to civil defence volunteers and certain civilians.
%
%645. Sections 4 and 5 of the 1943 Act (which provide appeal rights where an award is withheld or reduced on grounds of serious negligence or misconduct, and against interim and final assessments) have been excluded, as no additional appeal rights are required in these areas.
%
%646. New section 5A(2)  defines a “specified decision” as one that is specified in regulations made by the Secretary of State. It is to be a different type of decision to that already envisaged in the current sections 1 to 5 of the 1943 Act.
%
%647. New section 5A(3)  provides that the regulations specifying the decisions will be affirmative, that is, subject to the approval of both Houses of Parliament.
%
%648. Subsection (2)  of section 57 provides for the appellant or the Secretary of State to progress the appeal, with leave of the tribunal or the nominated Judge, to the High Court if the party bringing the appeal believes the decision of the tribunal to have been erroneous in point of law. This provision already applies to appeals made under sections 1 to 4 (entitlement appeals) of the 1943 Act and will apply to a section 5A appeal.
%
%649. Subsection (3)  provides that the tribunal’s decision on section 5A appeals is capable of being set aside on a joint application of the parties, in certain circumstances. The original appeal will be heard again by the tribunal. This provision already applies to appeals made under sections 1 to 4 (entitlement appeals) of the 1943 Act.
%
%650. Subsection (4)  repeals section 1(2)  of the Pensions Appeal Tribunals Act 1949 that amended the 1943 Act by prohibiting appeals about pension matters due to service before 3rd September 1939.  This will provide appeal rights for groups such as those who served in the inter-war years.
%Section 58: Time limits for appeals
%
%651. This section amends the current provision and provides for a uniform time limit of 6 months in all areas, other than interim assessments, which retain their present 3-month time limit. The effect is to reduce the time limit in some areas and to introduce a time limit in others. This section also provides for transitional protection for decisions made before the new provision is commenced.
%
%652. Subsection (1)  section 8(1)  of the 1943 Act so that appeals against interim assessments remain subject to a 3-month time limit but in all other cases, including the “new” section 5A appeals, there will be a uniform 6-month appeal time limit from the date the decision or assessment is notified.
%
%653. Subsection (2)  inserts three subsections after section 8(3)  of the 1943 Act.
%
%654. New section 8(4)  enables the Secretary of State to make regulations revising the new appeal time limits of 6 or 3 months for submitting an appeal, either up or down. At present there are no plans to use this provision.
%
%655. New section 8(5)  enables the Secretary of State to prescribe in regulations when a PAT can admit a late appeal. In any event, the ability of the PAT to hear a late appeal is not available for those appeals made more than 12 months after the expiry of the statutory appeal time limit (3 or 6 months as appropriate). The current provision which allows late appeals if there was a reasonable excuse for the delay is repealed.
%
%656. New section 8(6)  provides that the regulations in either subsection (4)  or (5)  will be subject to the approval of both Houses of Parliament.
%
%657. The provisions in subsection (2)  (the new section 8(4) , 8(5)  and 8(6) ) will apply to all decisions and assessments from the day of commencement regardless of when the decision or assessment was made.
%
%658. Subsection (3)  provides that the new appeal time limits in subsection (1)  of this section will not apply to entitlement decisions under sections 1 to 4 of the 1943 Act, nor to final assessments under section 5(2)  made before the day of commencement. Paragraph ($a$)  refers to entitlement decisions and paragraph ($b$)  to final assessments.
%
%659. Subsection (4)  amends section 8(1)  of the 1943 Act to provide that people who have been notified of a pre-commencement entitlement decision as set out in subsection (3)($a$)  of this section will have 12 months from the commencement date of section 57(1)  of this Act in which to bring an appeal.
%
%660. Subsection (5)  provides that the time within which an appeal may be brought under section 6(1)  of the War Pensions Act 1921 (First World War claims) will be amended from 12 months to 6 months, to ensure parity with the other time limit provisions.
%
%661. Subsection (6)  provides that the amendment to section 6(1)  of the War Pensions Act 1921 does not apply to decisions made before subsection (5)  comes into force.
%Section 59: Matters relevant on appeal to Pensions Appeal Tribunal
%
%662. This section inserts a new section 5B in the 1943 Act. Section 5B will provide clarification of the jurisdiction of the tribunal. The provision is similar to that in section 12(8)  of the Social Security Act 1998. 
%
%663. New section 5B($a$)  states that it is not necessary for the tribunal to consider issues that have not been raised by the appellant or the Secretary of State in relation to the appeal. 5B($b$)  requires the tribunal to take no account of circumstances that did not exist when the decision, that is the subject of the appeal, was made.
%Section 60: Constitution and procedure of Pensions Appeal Tribunals
%
%664. This section makes provision in relation to members of the Tribunal, provides for the appointment of a President and a Deputy President for each part of the United Kingdom and gives the President power to make directions and decide the appropriate composition of appeal tribunals.
%
%665. Subsection (1)  inserts the word “allowances” into sub-paragraph (2)  of paragraph 2 of the Schedule to the 1943 Act to make it clear that expenses as well as remuneration may be paid to Tribunal members.
%
%666. Subsection (2)  enables appropriate terms of appointment to be specified for each category of Tribunal member.
%
%667. Subsection (3)  inserts paragraph 2A into the Schedule, to provide for the qualifications that tribunal members should hold. The legal and medical provisions do not differ from the current requirements but the qualifications required by other members have been both simplified and extended. The service requirement is less specific in that the tribunal member need not be of the same gender and similar rank with similar service history. Instead, the Lord Chancellor will have a duty to appoint persons with knowledge or experience of service life to the pool of tribunal members and, in addition, the power also to appoint suitably experienced lay members to the tribunal pool. The President should take account of their experience in deciding who should sit on tribunals hearing different types of cases.
%
%668. This subsection also inserts paragraph 2B, which provides for the appointment of the President and a Deputy President for England and Wales, Scotland and Northern Ireland. This section addresses who will appoint the President, the qualification required and who carries out the duties when the President is temporarily indisposed.
%
%669. Subsection (4)  replaces paragraph 3 of the Schedule to the 1943 Act with new paragraphs 3 to 3C. Paragraph 3 provides that the members of the Tribunal must always have a legally qualified member, and that the chairman must be a legally qualified member. Paragraph 3A provides powers for the President to give directions as to the composition of the tribunals in relation to particular appeals, types of appeal or appeals generally. Paragraph 3B provides the power for the President to give directions as to the practice and procedure of the Tribunals. Paragraph 3C provides powers for the Lord Chancellor etc. to exercise the power under paragraphs 3A and 3B if there is no President or Deputy President in post. There is also a power to revoke past directions.
%
%670. Subsection (5)  provides for full time members of the Pensions Appeal Tribunals to be included in Schedule 11 to the Courts and Legal Services Act 1990 which bars them from legal practice.
%Section 61: Composition of Central Advisory Committee
%
%671. This section enables the Secretary of State to appoint fewer than twelve, but at least one, chairmen of a War Pensions Committee to the Central Advisory Committee on War Pensions. The Secretary of State intends to use this provision to ensure that there is a suitable balance in the number of chairmen appointed to the CAC.
%Part III: Social Security Administration
%Loss of Benefit
%Background
%The current position
%
%672. A person who is found guilty of certain criminal offences may be the subject of a community sentence* imposed by the court. Community sentences include probation orders, community service orders and combination orders.
%
%673. If the terms of the sentence are not met, the Probation Service (or, in Scotland, the Procurator Fiscal) will refer the matter back to court. The court will decide whether the offender has breached the order, and, if so, what further penalty should be imposed for the breach, or if the order should be revoked and the offender re-sentenced for the original offence.
%
%674. About 130,000 offenders are sentenced to probation orders, community service orders and combination orders each year in England and Wales, of whom currently about 30,000 are returned to court as a result of breach proceedings. In Scotland, in 1998, about 12,400 people were sentenced to community service orders or probation orders, and around 4,000 of these were subject to proceedings for breach.
%The measures in the Act
%
%675. The measures in this Part of the Act will allow the Secretary of State for Social Security to withdraw or reduce benefit where a person fails to comply with a designated community sentence. The sanction will be for a fixed period (to be prescribed in regulations) and will commence after a court has determined that a community sentence has been breached.
%
%676. The measures in respect of Social Security benefits, which are a matter reserved to Westminster, cover Great Britain. The payment of training allowances is a devolved matter for Scotland. The provisions in this part of the Act relating to the loss of Training Allowances apply in England and Wales only, and this part of the Act provides that such allowances paid by Scottish Ministers will not be withdrawn under these measures.
%
%677. In the first instance, the measure will be piloted in separate areas in England and Wales to test the links between Social Security offices and the Probation Service within a single criminal justice system and to assess the behavioural impact on offenders. For the duration of the pilots, the sanction period will be set at four weeks and will apply in respect of probation orders, community service orders and combination orders.
%
%678. During the pilot phase, the scheme will cover people aged between 18 and 59.  The benefits that will be affected are Jobseeker’s Allowance* (JSA), Income Support* (IS) and also the JSA-equivalent element of certain training allowances* (TAs).
%
%679. For recipients of both contributory and income-based JSA, the benefit will not be payable for the period of the sanction. Housing Benefit, and any other “passported” benefit entitlements, will not be affected. For example, a JSA recipient aged 25, with no dependants or housing costs, would normally be entitled to £52. 20 per week (the April 2000 rate of personal allowance for a single adult aged over 25). If the court decided he had breached a community sentence, the full amount – £52. 20 – would be withdrawn for four weeks. Circumstances will be prescribed in which, after two weeks, he could be entitled to a reduced payment of JSA (a “hardship payment”) of £31. 30 per week. Entitlement to a hardship payment would depend on his personal circumstances, taking into account any income or capital which he may possess.
%
%680. If a JSA claimant falls into a vulnerable group, he could apply for hardship payments from the first day of the 4-week period. This is in line with the current provisions for hardship payments arising from employment condition sanctions. The term “vulnerable group” refers to the group of people specified in regulation 140(1)  of the Jobseeker’s Allowance Regulations 1996 and includes, among others, those responsible for a child or young person and those where either the claimant or partner would be entitled to a disability-related premium.
%
%681. There will be slightly different arrangements in place to deal with JSA claims made under the “joint claim” arrangements. If either of the couple are found to be in breach of a community sentence, benefit will not be payable in respect of that member of the couple. This is referred to in more detail in the commentary on sections (section 63).
%
%682. For IS claimants, the effect of the sanction will be to reduce the amount of benefit in payment, rather than to withdraw payment of the benefit in its entirety. The effect of this measure on a lone parent with one child aged under 11, receiving IS, would be to reduce benefit entitlement by 40% of the single adult rate – a reduction of £20. 88 per week at April 2000 rates. This means that for a period of 4 weeks, weekly benefit of £73. 82 would be payable, instead of the full rate of £94. 70. 
%
%683. The Secretary of State will not be able to extend the measure to other benefits or to other types of community sentence without making regulations which would be required to be passed by resolution of both Houses of Parliament (the affirmative procedure).
%Commentary on Sections
%
%684. These sections contain provisions to remove or reduce the benefit or, in England and Wales, the training allowance of offenders who have not fulfilled their responsibilities in relation to specified community sentences.
%Section 62: Loss of benefit for breach of community order.
%
%685. This section provides for benefit to be reduced (in respect of IS) or withdrawn (in respect of JSA or TAs) where an offender is in breach of a community sentence.
%
%686. The provisions are triggered after a court has determined that a community sentence has been breached.
%
%687. Subsection (1)  states that these provisions apply when a court has determined that that an offender has failed to comply with the terms of his community sentence without reasonable excuse. The Secretary of State will be notified by the Probation Service of a determination where the offender is, or becomes entitled to, a relevant benefit.
%
%688. Subsection (2)  provides that a relevant benefit shall not be payable for a “prescribed period”, which is the period for which the benefit will be withdrawn. Initially, regulations will provide for the prescribed period to be 4 weeks. Subsection (7)  (below) sets an overriding maximum period of 26 weeks.
%
%689. Subsection (3)  provides for IS to be paid at a reduced rate for the prescribed period rather than withdrawn completely. Details of the reduction will be prescribed in regulations. The broad aim is that the reduction regime will be similar to that which will apply in JSA cases where hardship is established.
%
%690. Subsection (4)  enables regulations to prescribe that JSA recipients may be eligible for a reduced rate of benefit during the prescribed period, providing they satisfy certain conditions. These conditions will be similar to the hardship provisions which currently apply in JSA. If hardship is established and the claimant satisfies the other conditions of entitlement, the claimant will be awarded a reduced payment of income-based JSA. The reduction is normally a sum equivalent to 40% of the appropriate single person's allowance, whether or not the claimant is a single person. However, if someone in the claimant’s family is seriously ill or pregnant the reduction is 20%.
%
%691. Subsection (5)  provides that payments made under section 2 of the Employment and Training Act 1973 (TAs) to participants on certain training schemes and employment programmes shall not be payable for the prescribed period except to the prescribed extent. It is intended that the element of any TA which equates with the participant’s underlying JSA entitlement will be withdrawn. Any additional premium, top-up or payment of expenses will remain payable, subject to continued participation in the scheme or programme.
%
%692. The basic element of a TA is equal to the amount of JSA which the participant would be entitled to if he were not engaged in the training scheme or employment programme. In addition, he may receive a training premium or top-up (the amount of which depends on the scheme concerned) plus payment of certain expenses. It is intended that the sanction will apply only to the basic element. For example, a young single person participating in the Voluntary Sector option of the New Deal for Young People receives an allowance consisting of a basic element of £41. 35 a week (at 2000 rates), plus a weekly top-up of £15. 38.  The sanction would mean that the £41. 35 would not be paid for 4 weeks, but the £15. 38 would remain in payment (provided the young person continued to participate in New Deal). Some types of payments under section 2 of the Employment and Training Act 1973 – such as those payable in Employment Zones – will be excluded from these provisions.
%
%693. Subsection (6)  provides for the payment of arrears in the event that a decision that an offender has breached the terms of his sentence is subsequently quashed or set aside by a Court.
%
%694. Subsection (7)  precribes a maximum length of 26 weeks for the “prescribed period”. The intention for the pilot exercises is that this period will be for 4 weeks only. Regulations will specify the date from which the restriction will commence (Section 65(2) ). This will usually be the first full benefit week after the decision to impose a restriction is made. This avoids the need to calculate part-week payments.
%
%695. Definitions of the terms used in this provision are set out in subsection (8) .
%
%696. Subsection (9)  provides that where the “relevant benefit” is a TA, then references in this section to “entitlement to benefit” also refer to cases where a TA is in payment. TAs are payable at the discretion of the Secretary of State under section 2 of the Employment and Training Act 1973: technically there is no entitlement to an allowance.
%
%697. Subsection (11)  modifies the provisions of section 62 so as to make them applicable to Scotland.
%Section 63: Loss of joint-claim Jobseekers Allowance
%
%698. This section sets out how the Community Sentence provisions will apply to a couple claiming JSA under the “joint claim” provisions.
%
%699. Schedule 7 to the Welfare Reform and Pensions Act 1999* introduced the requirement on certain couples to make joint claims for JSA. These provisions are to be commenced on 19 March 2001.  The requirement to make a joint claim for JSA will impact on couples without children, where one or both partners is in the 18-24 years age range on the date when the measure is introduced. Coverage will apply to those born on or after a certain date, so older couples without children will be included as time passes. Under joint claims, both members of the couple will have to claim JSA and both will have to meet JSA labour market conditions.
%
%700. Subsection (1)  provides for subsections (2)  and (3)  to apply where the conditions of entitlement to joint-claim JSA are or become satisfied in relation to a couple and the community sentence sanction would have applied to at least one of the couple.
%
%701. Subsection (2)  provides that no joint claim JSA will be payable where both members of the couple are subject to community sentence sanctions, or where one member is subject to a community sentence sanction and the other is already subject to a sanction pursuant to section 20A of the Jobseekers Act 1995*. Section 20A provides for sanctions to apply to members of joint claim couples where they have unreasonably caused or prolonged their own unemployment.
%
%702. Subsection (3)  provides that where only one member of the couple is subject to a community sentence sanction and the other member is not subject to certain other sanctions, the amount of JSA payable will be reduced to an amount calculated in a prescribed way and will be paid to the member of the couple who is not subject to that sanction.
%
%703. Subsection (4)  provides for hardship payments to be made in cases where both members of a joint-claim couple for JSA are sanctioned for a breach of a community sentence, or where one member is sanctioned for such a breach and the other is sanctioned for employment-related reasons imposed under section 19 of the Jobseekers Act 1995.  Regulations will prescribe the rate of payment, what information the couple need to provide and the circumstances in which such payments will be made.
%
%704. Subsection (5)  provides that the reduced amount referred to in subsection (3)  will be calculated in the same way as described in section 20A of the Jobseekers Act 1995.  The normal couple rate is £81. 95 (April 2000 rates), plus any premiums or housing costs where applicable. If both members of the couple receive a community sentence sanction the amount will be reduced to nil. If one member of the couple is subject to a sanction, the other member will be paid the equivalent of the appropriate single person’s rate (ie £41. 35 if aged 18 to 24, £52. 20 aged 25 and over).
%
%705. Subsection (6)  provides for the payment of arrears in the event that a decision that an offender has breached the terms of his sentence is subsequently quashed or set aside by a Court.
%
%706. Subsection (7)  sets a maximum length of 26 weeks for the “prescribed period”. The intention for the pilot exercises is that the period will be for 4 weeks only. Regulations will be made specifying when the period of the restriction will commence. This will usually be the first full benefit week after the decision to impose a restriction is made. This avoids the need to calculate part-time week payments.
%Section 64: Information provision
%
%707. This section concerns the provision of information. It enables regulations to be made which will allow information to be exchanged between the Probation Service (in England and Wales) and local authorities (who control probation functions in Scotland) on the one hand, and officers of the Department of Social Security and the Department for Education and Employment on the other.
%
%708. Subsection (1)  provides that a court will be required to explain to an offender that a benefit sanction will arise as a consequence of failing to comply with a community sentence.
%
%709. Subsection (2)  enables the Secretary of State to make regulations requiring the probation service to notify the DSS or DfEE, at the prescribed time and in the prescribed manner, of the following:
%
%    that an information has been laid at court relating to a breach of a community sentence;
%
%    that a Court has made a determination that the offender has failed without reasonable excuse to comply with his community sentence;
%
%    prescribed information about the offender;
%
%    any circumstances, whereby any adjustment or repayment may need to be made, as specified in sections 62(6)  and 63(6) .
%
%710. Subsection (3)  enables the High Court of Justiciary in Scotland to make Rules of Court requiring the Clerk of Court to notify the Secretary of State of the commencement and determination of court proceedings, such information about the offender as the court may specify and of any circumstances whereby any adjustment or repayment may need to be made. Subsection (11)  specifies when such proceedings are commenced.
%
%711. Subsection (4)  imposes an obligation on the Secretary of State (in practice the Benefits Agency), to notify the offender, when an information is laid or proceedings commenced, that he will suffer a loss of benefit if the court determines that he has breached his order. Subsection (5)  requires that the offender should be notified as soon as is reasonably practicable.
%
%712. Subsection (6)  allows regulations to be made relating to:
%($a$) 
%
%how a person, listed in subsection (7), uses information relating to community orders or social security benefits;
%($b$) 
%
%how people exchange information; and
%($c$) 
%
%the purposes for which a person may use the  information supplied.
%
%713. Regulations under this subsection will prescribe the manner in which the Probation Services and the Benefits Agency will exchange information and the uses to which the information can be put. The intention is to allow exchange of information on, for example, benefit receipt, identity and address, the laying of information (or the commencement of Scottish court proceedings) and the outcome of the court proceedings, between the Probation Service (or in Scotland, local authorities controlling probation functions), the Benefits Agency, the Employment Service and private sector service providers. The information exchanged will need to be sufficient to identify the offender and to ensure these provisions are properly implemented. This information will not include details of the original offence in respect of which the original Order was made. However, information for evaluation, statistical and research purposes will also be included, which may go wider than that required for pure implementation of the benefit sanction regime.
%
%714. Subsection (7)  lists those persons authorised under subsection (6)  above.
%
%715. Subsection (8)  allows regulations to be made covering how exchanged information can be used and passed on.
%
%716. Subsection (9)  provides that the explanation which the court is to give under subsection (1)  (about the consequences of failure to comply with a community order) will be treated as part of the explanation which the court is required to give to an offender under the relevant Scottish legislation, before it makes a community service order or a probation order.
%Section 65: Loss of benefit regulations
%
%717. This section contains provisions about the making of regulations by the Secretary of State.
%
%718. Subsection (1)  defines the term “prescribed” to mean prescribed by, or in accordance with, regulations made by Secretary of State.
%
%719. Subsection (2)  enables regulations made by the Secretary of State for the purpose of these provisions to determine the time from which any period prescribed in regulations is to run.
%
%720. Subsection (3)  provides for all regulations under these provisions, other than regulations referred to in subsection (4) , to be made by the negative resolution procedure.
%
%721. Subsection (4)  lists which regulations will require the affirmative resolution procedure. These are:
%a)
%
%regulations to prescribe a reduced amount of Income Support to be paid to a claimant who is subject to a sanction for breach of a community sentence;
%b)
%
%regulations prescribing a reduced amount of a joint-claim Jobseeker’s Allowance where one member of a couple is subject to a sanction;
%c)
%
%regulations prescribing the circumstances in which hardship payments are to be made , and the amount, where the sanction applies to a single claimant or where both members of a joint claim couple are subject to a sanction;
%d)
%
%any regulations specifying additional benefits to be covered by these provisions;
%e)
%
%regulations adding to the list of community orders, breach of which will result in loss of benefit.
%
%722. Subsection (5)  allows the regulation-making powers in sections 62 to 64 to be used in such a way as to make different provisions for different classes of cases, imposing conditions or creating exceptions. It also enables the regulations to include incidental, consequential and transitional provisions.
%
%723. Subsection (6)  provides that regulations under these measures can include different provision for different areas.
%
%724. Subsection (7)  gives the Secretary of State power to make consequential modifications to the Scottish criminal procedure legislation where he makes an order prescribing further descriptions of relevant community orders.
%Section 66: Appeals relating to loss of benefit
%
%725. This section amends the Social Security Act 1998 to provide that an appeal to a Social Security decision maker lies against a decision to withdraw or reduce benefit under section 62 or 63. 
%Investigation Powers
%Background
%The current position
%
%726. Section 110 of the Social Security Administration Act 1992* (the “Administration Act”) sets out provisions concerning the appointment and powers of social security fraud inspectors. Sections 110A and 110B of that Act set out provisions concerning the appointment and powers of local authority fraud inspectors in relation to Housing Benefit and Council Tax Benefit. Section 33 of the Jobseekers Act 1995* set out provisions regarding the powers of social security fraud inspectors in relation to Jobseeker’s Allowance*.
%
%727. These powers allow inspectors to elicit and inspect information. For example, the inspector may ask an employer about his employees in order to establish whether benefit fraud is occurring. The information inspector’s request could include information on earnings, tax and National Insurance and pensions, and in the case of the self-employed or landlords, their business records. Inspectors may visit those from whom they require information, or they may write to them.
%
%728. The legislation governing inspectors’ powers was introduced with the National Insurance system. It has been added to and amended over many years and has now become somewhat piecemeal and inconsistent. This has led to employers, employees and the Data Protection Registrar being unclear as to what inspectors are and are not allowed to do under these powers.
%Recent developments
%
%729. On 23 March 1999, the Government published its strategy and plans for reducing fraud and error A new contract for welfare: Safeguarding Social Security (CM 4276). The need to make effective use of inspectors’powers is an important part of successfully investigating, detecting and punishing fraud. Fraud officers’ effectiveness could be jeopardised by employers and the Data Protection Registrar continuing to express uncertainty about what the powers allow, as their uncertainties are resulting in some employers refusing to comply with this legislation.
%
%730. Discussions on the broad thrust of these measures have been held with business organisations, the TUC, local authority associations, the Office of the Data Protection Registrar and Liberty. All agreed that there was a need for clearer powers and business organisations in particular supported moves to modernise the legislation – for example, allowing more use of information technology such as e-mail.
%The measures in the Act
%
%731. The measures in the Act clarify and align inspectors’ powers in order to remove the uncertainty that currently exists and allow inspectors to operate effectively.
%
%732. The measures are contained in section 67 and Schedule 6.  They replace the current legislation.
%
%733. In the new legislation, the term “inspector” is replaced by “authorised officer”. The new powers enable the Secretary of State and Local Authorities to authorise officers to use the powers set out in new sections 109B and 109C. Section 109B sets out the powers that authorised officers have to require information by issuing a “written notice”. The powers at section 109C set out an authorised officer’s powers to enter premises in order to obtain information. The powers do not allow authorised officers to force entry to premises, forcibly search premises or compulsorily detain persons for questioning. The purposes for which the powers can be used are also set out clearly in the legislation in section 109A(2)  and 110A(2).
%
%734. If those from whom information has been requested fail to comply with authorised officers’ requests they may be prosecuted under the current section 111 of the Administration Act.
%Commentary on Sections
%Section 67: Investigation powers
%
%735. This section gives effect to Schedule 6 (social security investigation powers).
%Schedule 6
%
%736. This Schedule contains provisions to replace certain enforcement provisions in Part VI of the Administration Act.
%
%737. Sections 110, 110A and 110B of the Administration Act provide powers for social security fraud “inspectors” to obtain information by either writing to or visiting persons. Typically, the powers would be used to ask an employer for details of his employees in order to obtain information about people who may be committing benefit fraud by working whilst claiming benefit, or to obtain information from a self-employed person who may be working whilst claiming benefit. The powers set out in Paragraphs 2 and 3 of this Schedule replace these provisions. In the new sections, the term “inspector” has been replaced by “authorised officer”.
%Paragraph 2: Replacement for inspectors’ powers
%
%738. This paragraph replaces section 110 of the Administration Act with new sections 109A, 109B and 109C. Section 110 sets out the powers of inspectors who are appointed by the Secretary of State – this means those investigating fraud in all social security benefits.
%
%739. The powers in new sections 109A, 109B and 109C can be used in relation to “relevant social security legislation”. This is set out in new section 121DA inserted by paragraph 8 of this Schedule. The change to the term “relevant social security legislation” does not expand the scope of inspectors’ powers except in so far as Jobseeker’s Allowance inspections are now subject only to the new powers. Sections 33 and 34 of the Jobseekers Act 1995 are being repealed.
%New section 109A: Authorisation for investigators
%
%740. This section provides for the Secretary of State to authorise officers to exercise the powers set out in new sections 109B and 109C. It sets out who may be authorised, the purposes for which authorised officers may use the powers in new sections 109B and 109C and other particulars concerning the authorisations.
%
%741. New section 109A(1)  provides that a person who has the Secretary of State’s authorisation under section 109A may exercise any of the powers set out in new sections 109B and 109C for the purposes set out in subsection (2).
%
%742. New section 109A(2)  sets out the purposes for which authorised officers may exercise the powers in new sections 109B and 109C. The purposes are:
%($a$) 
%
%ascertaining whether a social security benefit is or was payable in an individual case;
%($b$) 
%
%investigating the circumstances of accidents, injuries or diseases giving rise to claims for Industrial Injuries Benefit and other benefits;
%($c$) 
%
%ascertaining whether the provisions of the relevant social security legislation have been, are being or are likely to be contravened (in cases involving particular individuals as well as more generally);
%($d$) 
%
%preventing, detecting and securing evidence of the commission of criminal offences in relation to the relevant social security legislation (either by particular individuals or more generally).
%
%743. In paragraph 741 above, ($a$)  and ($b$)  are linked to claims made by particular individuals; ($c$)  and ($d$)  include wider purposes where social security investigators may wish to investigate concerns more generally. They may, for instance, wish to investigate whether there is widespread contravention of the legislation in a particular workplace. In such cases, the investigator may wish to ask for lists of people. For example, they may ask for a list of all employees where an employer has a history of colluding with his staff in order that they can commit benefit fraud. The list would then be cross-referenced against benefit records.
%
%744. New section 109A(3)  sets out the requirements relating to an authorisation. The Secretary of State must have granted the authorisation for the purposes set out in subsection (2). The person given the authorisation must also fall into one of the categories set out in this subsection. The Secretary of State may authorise persons other than those in his own Department – those working for other Departments and for local authorities and their contractors. This means that the Secretary of State is able to make use of personnel and expertise in other organisations and helps promote closer working across Government and local authorities in anti-fraud work.
%
%745. New section 109A(4)  sets out further requirements regarding authorisations.
%($a$) 
%
%provides that an individual’s authorisation must be contained in a certificate that is given to him as evidence of his entitlement to exercise the powers contained in new sections 109B and 109C.
%($b$) 
%
%provides that the authorisation may contain provisions regarding the duration for which it may have effect.
%($c$) 
%
%allows the Secretary of State to restrict the powers exercisable by virtue of the authorisation.  He can prohibit the use of the powers except for particular purposes in particular circumstances or in relation to particular benefits or provisions of relevant social security legislation.
%
%($b$)  and ($c$)  are discretionary powers.
%
%746. The combination of this subsection and subsection (3)  has the effect that the Secretary of State has a range of options open to him when authorising officers. For example, the Secretary of State could authorise a local authority’s officers to conduct social security investigations in an area where he wished to promote closer working (subsection (3) ). However, he may equally (by using subsection (4) ) authorise the local authority’s investigators for the duration of only one investigation, or limit the authorisation he grants so that local authority investigators may investigate all benefits except for benefits where they may have less expertise.
%
%747. New section 109A(5)  allows the Secretary of State to withdraw an authorisation at any time.
%
%748. New section 109A(6)  concerns circumstances where an individual working for a local authority, or a person administering Housing Benefit and/or Council Tax Benefit on behalf of a local authority, is granted an authorisation under this section.
%($a$) 
%
%provides that the Secretary of State and the local authority may enter into such arrangements as they see fit with regard to the exercise of these powers.
%($b$) 
%
%allows the Secretary of State to make payments to the local authority if he considers it appropriate to do so.
%
%749. New section 109A(7)  provides that the matters on which an individual is authorised to report under section 139A of the Administration Act shall include work under this section carried out by an individual working for a local authority or for a person administering Housing Benefit and/or Council Tax Benefit on behalf of a local authority – for example, a contractor. This subsection would allow the Secretary of State to authorise the Benefit Fraud Inspectorate to inspect this work as it currently inspects local authority fraud work.
%
%750. New section 109A(8)  provides that authorised officers may exercise their powers in relation to persons employed by the Crown and conduct enquiries about persons employed by the Crown and in premises owned or occupied by the Crown.
%New section 109B: Power to require information
%
%751. Section 109B sets out powers to allow authorised officers to request information by a written notice for the purposes set out in section 109A(2). A notice shall be taken to be in writing in cases where the notice is transmitted by electronic means (such as fax or e-mail) provided that the notice is received by the recipient in a form which is legible and capable of being recorded for future reference. In this section, the term “document” includes anything in which information is recorded in electronic or any other format (see paragraph 8 of this Schedule).
%
%752. New section 109B(1)  provides that an authorised officer may request information by written notice. The authorised officer may request information from a person if he has reasonable grounds for suspecting that:
%($a$) 
%
%the person falls into one or more of the categories set out in subsection (2) ; and
%($b$) 
%
%the person has, or may have, access to, or possession of, information about a matter which is relevant to at least one of the purposes set out in section 109A(2).
%
%753. The authorised officer may require the person to provide all the information that is set out in the notice which is in the person’s possession or to which the person has access. It must also be reasonable for the officer to require the information for the pursuit of at least one of the purposes outlined in section 109A(2).
%
%754. New section 109B(2)  sets out the persons who may be required to provide information under this section. These are listed at subsections ($a$)  to ($j$) .
%
%755. New section 109B(3)  provides that a person has complied with a request when he has provided the required information within the time and in the format specified in the notice. However, the person will not be deemed to have failed to comply if he does not provide information to which he does not have access or which he does not possess (see subsection (1) ).
%
%756. New section 109B(4)  makes it clear that authorised officers may require the persons in subsection (2)  to produce and hand over the information requested, including copies or extracts of it – for example, if an employer of casual labour claims not to keep a record of his staff, the authorised officer could require him to compile one. Authorised officers’ requests must be reasonable (see subsection (1) ).
%
%757. New section 109B(5)  protects people from being required to provide information which may incriminate themselves or their spouse.
%New section 109C: Powers of entry
%
%758. Section 109C provides an authorised officer’s power to enter premises in order to obtain information for the purposes set out in section 109A(2). As in section 109B, the term “document” should be taken to include anything in which information is stored in electronic or other format (see paragraph 8).
%
%759. New section 109C(1)  provides that the authorised officer may enter premises if they are liable to inspection under this section and where it is reasonable for him to do so to exercise his powers under the new section 109C. The premises liable to inspection under this section are set out in subsection (4) . What is meant by the term “premises” is set out in paragraph 8 of this Schedule. The authorised officer may be accompanied by anyone else he considers it necessary to take with him. He may enter premises only at a reasonable time.
%
%760. New section 109C(2)  provides that, once an authorised officer has entered any premises liable to inspection under this section, he may look around those premises and conduct any enquiry there that he considers appropriate in his investigation for any of the purposes set out in section 109A(2).
%
%761. New section 109C(3)  provides that, once an authorised officer has entered any premises liable to inspection under this section, he may question anyone whom he finds on the premises, require them to provide documents that he may reasonably require for the purposes set out in section 109A(2)  and take possession of, and remove, these documents. As with section 109B(4) , the authorised officer may require the person to produce, hand over and create documents or copies and extracts of documents. Under this section, authorised officers also have the power to make their own copies of documents. What authorised officers do once they have entered the premises must be reasonable.
%
%762. New section 109C(4)  sets out the premises liable to inspection under this section.
%
%763. New section 109C(5)  states that where the authorised officer seeks to enter any premises under this section, he must, if requested to do so, produce the certificate containing his authorisation.
%
%764. New section 109C(6)  protects people from being required to provide information which may incriminate themselves or their spouse.
%Paragraph 3: Exercise of powers on behalf of local authorities
%
%765. This paragraph replaces sections 110A and 110B of the Administration Act with a new section 110A. Sections 110A and 110B of the Administration Act concern local authority benefit fraud inspectors’ powers. The powers relate to Housing Benefit and Council Tax Benefit only.
%New section 110A: Authorisation by local authorties
%
%766. New section 110A(1)  provides that a person who has the authorisation of a local authority administering Housing Benefit and Council Tax Benefit may, subject to subsection (8) , exercise any of the powers set out in new sections 109B and 109C for the purposes set out in subsection (2).
%
%767. New section 110A(2)  sets out the purposes for which a person authorised by a local authority may use the powers set out in new sections 109B and 109C. The purposes are:
%($a$) 
%
%ascertaining whether housing benefit or council tax benefit is or was payable in an individual case;
%($b$) 
%
%ascertaining whether the provisions of the relevant social security legislation regarding Housing Benefit and Council Tax Benefit have been, are being or are likely to be contravened (in cases involving particular individuals as well as more generally);
%($c$) 
%
%preventing, detecting and securing evidence of the commission of criminal offences in relation to Housing Benefit and Council Tax Benefit (either by particular individuals or more generally).
%
%768. In paragraph 766 above, ($a$)  is linked to claims made by particular individuals; ($b$)  and ($c$)  include wider purposes where investigators may wish to investigate concerns more generally. They may, for instance, wish to investigate whether there is widespread contravention of the legislation in a particular workplace. In such cases, the investigator may wish to ask for lists of people. For example, they may ask for a list of all employees where an employer has a history of colluding with his staff in order that they can commit benefit fraud. The list would then be cross-referenced against benefit records.
%
%769. New section 110A(3)  sets out the requirements for the local authority’s authorisation. The local authority must have granted the individual an authorisation for the purposes set out in subsection (2). The person given the authorisation must also fall into one of the categories set out in this subsection.
%
%770. New section 110A(4)  provides that the provisions in section 109A(4)  also apply to local authority authorisations. An authorisation must be contained in a certificate and a local authority may restrict the extent of an authorisation (see section 109A(4) ).
%
%771. New section 110A(5)  allows the Secretary of State to withdraw the authorisation at any time.
%
%772. New section 110A(6)  provides that certificates of authorisation, and withdrawals of authorisations by local authorities, must be signed by the head of paid services or the chief finance officer.
%
%773. New section 110A(7)  provides that local authorities have a duty to comply with directions given by the Secretary of State. These directions may relate to:
%($a$) 
%
%whether or not the local authority may make any authorisations for the purposes of subsection (2) ;
%($b$) 
%
%the period for which authorisations granted by the local authority shall have effect;
%($c$) 
%
%the number of authorisations a local authority may make;
%($d$) 
%
%how they should use the power to restrict authorisations in section 109A(4)($c$)  (applied by subsection (4) .
%
%774. New section 110A(8)  provides that officers authorised under this section may use the powers set out in sections 109B and 109C, with the following differences:
%($a$) 
%
%provides that the purposes for which these powers may be used are those set out in subsection (2)  of this section, not section 109A(2) , and
%($b$) 
%
%provides that the powers may be used in relation to the relevant social security legislation only in so far as it relates to Housing Benefit and Council Tax Benefit.
%
%775. New section 110A(9)  provides that an authorised officer is not restricted to enquiring about benefits administered by the local authority who has appointed him. Thus, one local authority can conduct an inspection on behalf of another – for example, if one has greater expertise or was visiting the employer anyway.
%Paragraphs 4 to 9: Consequential amendments
%
%776. These paragraphs make consequential amendments to the Administration Act.
%
%777. Paragraph 4provides for consequential amendments to section 111 of the Administration Act to ensure it applies to the relevant new sections as it currently applies to section 110. 
%
%778. Paragraph 5ensures consistency in references to the legislation to which new sections 109A, 109B, 109C and 100A apply and to which section 111A applies.
%
%779. Paragraph 6ensures consistency in references to the legislation to which new sections 109A, 109B, 109C and 100A apply and to which section 112 applies.
%
%780. Paragraph 7provides for section 113 of the Administration Act to apply in relation to the relevant social security legislation covered in the new sections, and to legislation concerning National Insurance contributions, Statutory Sick Pay and Statutory Maternity Pay.
%
%781. Paragraph 8inserts a new interpretation section 121DA in Part VI of the Administration Act.
%New section 121DA: Interpretation of Part VI
%
%782. New section 121DA(1)  sets out what is meant by the term “relevant social security legislation”.
%
%783. New section 121DA(2)  sets out the definition of an “authorised officer”.
%
%784. New section 121DA(3)  sets out what is meant by references to “document” and what is meant by notices being given in writing (see explanation of section 109B above).
%
%785. New section 121DA (4)  sets out the meaning of “premises”. Premises liable to inspection under section 109C must fall within the new section 109C(4)  and this section. This makes clear that premises include moveable structures, vehicles, offshore installations and places of any other description whether or not they are occupied as land and anyone present there is regarded as an occupier. This provision is similar to the definition of premises in the current section 33 of the Jobseekers Act 1995. 
%
%786. New section 121DA(5)  sets out what is meant by the terms “benefit”, “benefit offence” and “compensation payment”.
%
%787. New section 121DA(6)  makes clear that where a local authority has contracted out some of its Housing Benefit and Council Tax Benefit work, those working for the contractor or a sub-contractor can be authorised to act in accordance with the provisions in Part VI of the Administration Act.
%
%788. New section 121DA(7)  sets out that “subordinate legislation” has the same meaning in this section as the Interpretation Act 1978 (which means Orders in Council, orders, rules, regulations, schemes, warrants, bylaws and any other instruments made under an Act, see section 21).
%
%789. Paragraph 9makes a consequential amendment to Schedule 10 to the Administration Act.
%Housing Benefit and Council Tax Benefit
%Revisions and Appeals
%Current position
%
%790. Housing benefit and council tax benefit (HB/CTB) are income-related social security benefits which are administered by local authorities (unlike other social security benefits, which are administered by the Secretary of State). Under current legislation, a local authority may review a decision they have made in relation to a HB/CTB claim at any time if there has been a change of circumstances, or the authority considers that the decision was made on the basis of an error of fact or law. Where a person wishes to dispute a decision of a local authority on a HB/CTB claim, regulations provide for an initial, internal, review by the authority, with a right to a further review by a Review Board comprised of councillors from that local authority. A Review Board’s decision may be challenged only by way of judicial review.
%The measures in the Act
%
%791. The measures in the Act align the arrangements for decision-making in HB/CTB with those recently introduced in other social security benefits under the Social Security Act 1998*. They provide clearly defined procedures for changing decisions on benefit entitlement and other matters, and place greater emphasis on claimants’ own responsibilities for exercising their rights promptly and ensuring that information held on their claims is accurate. They also introduce a right of appeal from local authorities’ decisions on HB/CTB claims to an appeal tribunal administered by the Appeals Service agency. HB and CTB Review Boards will be abolished.
%Commentary on Sections
%Section 68: Housing benefit and council tax benefit: revisions and appeals
%
%792. Section 68 gives effect to Schedule 7.  The Schedule sets out the detail of the new provisions for decision-making and appeals in relation to local authority decisions on HB/CTB claims. The provisions of the Schedule are in the main self-contained, but the majority mirror provisions of the Social Security Act 1998 applying to decisions and appeals in the other social security benefits. The Schedule also amends the Social Security Administration Act 1992* (the “Administration Act”) so that powers concerning the requirement to provide evidence and information in relation to revising or superseding decisions are extended to HB/CTB, and makes minor amendments to Schedules 1 and 4 to the Social Security Act 1998 to include HB/CTB in provisions under those Schedules.
%
%793. In particular, the Schedule:
%
%    makes provision for the revision and supersession of local authority decisions;
%
%    introduces a right of appeal from the local authority decision to an appeal tribunal (as constituted under the Social Security Act 1998);
%
%    provides a right of appeal from a decision of the tribunal, on a point of law, to a Social Security Commissioner.
%
%794. It is intended that regulations made under the powers in this Schedule would closely mirror decision-making provisions for other social security benefits which are contained in the Social Security and Child Support (Decisions and Appeals) Regulations 1999, made under powers in the Social Security Act 1998.  The regulations would also amend, where appropriate, the Social Security Commissioners (Procedure) Regulations 1999 to take account of HB/CTB.
%Schedule 7
%Paragraph 1: Introductory
%
%795. This paragraph defines “relevant authority” and “relevant decision” for the purposes of the Schedule. The paragraph also provides that references to a relevant decision do not include a decision under paragraph 3 to revise a relevant decision.
%
%Sub-paragraph (1)  provides that “relevant authority” for the purpose of this Schedule means an authority administering HB or CTB.
%
%Sub-paragraph (2)  provides that “relevant decision” for the purposes of this Schedule means:
%a)
%
%a decision of a relevant authority on a claim for HB/CTB;
%b)
%
%decisions under paragraph 4 which supersede a decision falling within ($a$)  above within this paragraph, or within sub-paragraph (1)($b$)  of that paragraph (decision of an appeal tribunal or Commissioner).
%Paragraph 2: Decisions on claims for benefit
%
%796. This paragraph provides that, once a decision has been made on a HB/CTB claim, that claim ceases to exist. This means that if a person’s claim fails, but his circumstances subsequently change and he wishes to apply again for benefit, he must make a fresh claim.
%Paragraph 3: Revision of decisions
%
%797. This paragraph provides powers for local authorities to revise a HB/CTB decision (whether as originally made, or superseded under paragraph 4), either on application or on their own initiative. Regulations would prescribe the cases, circumstances and period in which a decision could be revised. The paragraph also provides for the date from which the decision takes effect (a revised decision would normally take effect from the same date as the decision being revised) and regulation-making powers to vary that date in certain circumstances. It further provides for the lapsing of appeals where the decision appealed against is revised.
%
%Sub-paragraph (1)  provides for the local authority to revise any HB/CTB decision, either on application by a person affected, or on their own initiative.  It enables regulations to prescribe the period within which, and the cases or circumstances in which, a decision can be revised.  In line with the regulations applying in other social security benefits, it is intended that a person will normally have one month to apply for revision of a HB/CTB decision.  The regulations would provide for an extension of the time limit for special circumstances, with an overall time limit of 13 months.
%
%Sub-paragraph (2)  provides that when revising a decision the local authority need only consider the particular issue (or issues) raised by the application for revision, or (if acting on their own initiative) which caused them to consider revising the decision.
%
%Sub-paragraph (3)  provides that a revision would take effect from the day on which the original decision took effect, unless regulations under sub-paragraph (4)  provided otherwise, or if the revision was made after a decision in another case had held that an authority’s decision was wrong in law (in which case paragraph 18 would apply).
%
%Sub-paragraph (4)  allows for regulations to prescribe cases or circumstances in which the revision shall take effect other than from the date of the original decision.  An example would be where the original decision took effect from the wrong date.
%
%Sub-paragraph (5)  provides that the period within which an application for appeal may be made (provided for in regulations made under paragraph 6(8) ) runs from the date on which the revision was made.
%
%Sub-paragraph (6)  provides that, where an appeal has been lodged against a decision which is subsequently revised, the appeal will lapse, except in the circumstances prescribed in regulations.  It is the intention that regulations would provide that an appeal continues where the revised decision is not advantageous to the claimant.
%Paragraph 4: Decisions superseding earlier decisions
%
%798. This paragraph will allow the local authority to supersede their decision (whether as originally made or as revised under paragraph 3), or, in prescribed circumstances, a decision made by an appeal tribunal or a Commissioner. A decision which supersedes an earlier decision would not normally take effect from the date of the decision being superseded. The local authority would supersede a decision in cases where a revision under paragraph 3 was inappropriate (for example, because there had been a change of circumstances, or in a case where a person did not apply for a revision within one month of the disputed decision). A new decision could be made either in response to an application or on the local authority’s own initiative.
%
%Sub-paragraph (1)  provides for a local authority to supersede their own decisions, and those of appeal tribunals or Commissioners, both on application or on their own initiative.
%
%Sub-paragraph (2)  defines the term “appropriate relevant authority” for the purposes of this paragraph as the authority which made the decision being superseded or appealed.
%
%Sub-paragraph (3)  provides that in superseding a decision the local authority need only consider the particular issue (or issues) raised by the application for superseding, or (if acting on their own initiative) which caused them to consider superseding the decision.
%
%Sub-paragraph (4)  allows regulations to prescribe cases, circumstances and procedures to enable decisions to be superseded.  In line with regulations applying to other social security benefits, it is proposed that, unless a decision falls to be revised, it will be superseded where it is wrong in fact or law.  A decision will also be superseded where there has been a relevant change of circumstances.
%
%Sub-paragraph (5)  provides that, generally, a supersession will take effect on the day on which it was made or, where applicable, the day on which the application was made, unless regulations under sub-paragraph (6)  provide otherwise, or if the revision was made after a decision in another case had held that an authority’s decision was wrong in law (in which case paragraph 18 applies).
%
%Sub-paragraph (6)  allows for regulations to be made for cases or circumstances in which a decision to supersede an earlier decision would take effect from a different date from the date it is made or the date an application is made.  In line with other social security benefits, where a decision is superseded with a decision which is to the claimant’s advantage because of a change of circumstances, it is intended that regulations would provide that the new decision may take effect from the date of change, where the change is notified within one month.  Otherwise, the new decision would take effect from the date the change is notified, or the date the decision is made where there is no notification.  Where a new decision was disadvantageous, it would always take effect from the date of change.
%
%It is also intended that regulations under sub-paragraphs (4)  and (6)  would include provision for the local authority to supersede a decision of an appeal tribunal or Commissioner where that decision was made in ignorance of, or based on a mistake as to, a material fact, and where the decision of the tribunal was based on a determination of a Commissioner or court in another case which has subsequently been overturned by a (higher) court.
%Paragraph 5: Use of experts by authorities
%
%799. This paragraph provides that a local authority may have the assistance of one or more experts, where it appears to them that a question of fact in relation to a decision requires special expertise. This provision enables the local authority to seek advice from, for example, an accountant where there was an issue surrounding income from self-employment.
%Paragraph 6: Appeal to appeal tribunal
%
%800. Paragraph 6 sets out which decisions may be appealed, by whom, when and how. It requires that a person who has a right of appeal shall be notified of that fact, and provides relevant regulation-making powers.
%
%Sub-paragraph (1)  sets out those decisions which may be appealed, namely local authority decisions (whether as originally made, or as revised under paragraph (3)  or superseded under paragraph (4) ), on claims for, or awards of, HB/CTB and other prescribed decisions (subject to the provisions of sub-paragraph 2).
%
%Sub-paragraph (2)  provides that no appeal lies against:
%($a$) 
%
%a decision terminating or reducing an award of HB/CTB made in consequence of a decision made under regulations under section 2A of the Administration Act (work-focused interviews);
%($b$) 
%
%a local authority decision on any modification of the HB/CTB schemes under section 134(8) ($a$)  or 139(6) ($a$)  of the Administration Act (disregard of war disablement and war widows’ pensions);
%($c$) 
%
%so much of any decision as adopts a decision of a rent officer under an order made by virtue of section 122 of the Housing Act 1996 (decisions of rent officers for the purpose of housing benefit);
%($d$) 
%
%a decision determined by the rate of benefit provided for by law (for example, the rate of personal allowance used in the calculation of benefit entitlement, which is set annually by the Secretary of State and approved by Parliament);
%($e$) 
%
%any other decision prescribed in regulations.
%
%It is proposed that regulations made under ($e$)  above will, in line with other benefits, cover decisions which may broadly be termed “administrative”, for example, the method of paying benefit.  These regulations would be subject to affirmative resolution.
%
%Sub-paragraph (3)  provides that a person affected by a local authority decision to which this paragraph applies has a right of appeal to an appeal tribunal (defined in paragraph 23 as an appeal tribunal constituted under Chapter 1 of Part I of the Social Security Act 1998).
%
%Sub-paragraph (4)  provides that sub-paragraph (3)  shall not confer a right of appeal against a prescribed decision, nor against determinations which are embodied in or necessary to the final decision.
%
%Sub-paragraph (5)  clarifies the nature and limitations of the regulation-making power in subsection (4) .  It is explicit that the regulations shall not include any decision which relates to the conditions of entitlement to HB/CTB for which a claim has been validly made.
%
%Sub-paragraph (6)  provides that, where there is a decision that there is an overpayment of HB or CTB which is recoverable, any person from whom the local authority decides the overpayment is recoverable shall have a right of appeal to an appeal tribunal.
%
%Sub-paragraph (7)  provides for a person with a right of appeal to be given notice of a decision and right of appeal, as prescribed in regulations.
%
%Sub-paragraph (8)  provides for regulations setting out the manner and time for making appeals.  It is intended that the regulations would provide for a time limit of one month from the date of the decision within which to ask for a revision of a decision (under paragraph 3) or lodge an appeal.  If a person asks for a revision, he would have a further month from the date on which the local authority notifies him of the result of their reconsideration of the disputed decision in which to lodge an appeal.  These time limits would bring HB/CTB into line with other social security benefits.
%
%Sub- paragraph (9)  provides that an appeal tribunal need not consider issues which are not raised on the appeal, and shall not take account of any changes of circumstances that have occurred since the appealed decision was made.
%Paragraph 7: Redetermination etc. of appeals by a tribunal
%
%801. This paragraph provides for cases to be re-heard by an appeal tribunal where the tribunal has made an error of law or where the parties to an appeal agree that the tribunal has made an error of law.
%
%Sub-paragraph (1)  provides for cases in which an appeal tribunal can redetermine an appeal as being those where an application for leave to appeal from a decision of an appeal tribunal has been made to the person who constituted, or was the chairman of, the tribunal when the decision under appeal was given (or, in prescribed cases, such other person (apart from the Commissioner) as may be prescribed).  Sub paragraph (3)  allows that person to set aside the decision under appeal, if he considers that there was an error of law, and to refer the case for redetermination by the same, or a differently constituted, tribunal.
%
%Sub-paragraph (3)  requires that a decision be set aside and redetermined by a differently constituted tribunal where each of the principal parties to the appeal express the view that there was an error of law.
%
%Sub-paragraph (4)  defines who the principal parties to appeals are for the purpose of subsection (3)  above and for paragraph 8, namely:
%($a$) 
%
%the Secretary of State, where he is the applicant for leave to appeal or in circumstances prescribed by regulations;
%($b$) 
%
%the local authority against whose decision the appeal to the appeal tribunal was brought;
%($c$) 
%
%the person affected by that decision, or by the tribunal’s decision on that appeal.
%
%It is intended that regulations under sub-paragraph 4($a$)  may provide, for example, that where a case is drawn to Secretary of State’s attention in which he believes there has been an error of law, he may draw that fact to the appeal tribunal’s attention, even though he is not an applicant for leave to appeal.
%
%Regulations made under paragraph 23 would define who is a person “affected” for the purposes of this Schedule.
%Paragraph 8: Appeal from tribunal to Commissioner
%
%802. Paragraph 8 sets out the powers of the Commissioner and who may appeal to the Commissioner, on what grounds, when and how. This paragraph mirrors provisions of section 14 of the Social Security Act 1998.  It is intended that the regulations made under that section, which are contained in the Social Security Commissioners (Procedure) Regulations 1999, would be amended where necessary to take account of HB/CTB.
%
%Sub-paragraph (1)  provides a right of appeal from any decision of an appeal tribunal to a Commissioner on the ground that the decision of the tribunal was erroneous in point of law.
%
%Sub-paragraph (2)  specifies that the persons who have a right of appeal to a Commissioner are the Secretary of State, the local authority against whose decision the appeal to the appeal tribunal was brought and a person affected by that decision.
%
%Sub-paragraph (3)  provides that a Commissioner may set aside a decision of an appeal tribunal and refer the case to another tribunal for a fresh determination, provided all the principal parties to the appeal express the view that the original decision was erroneous in point of law.
%
%Sub-paragraph (4)  provides for a Commissioner to set aside a decision of an appeal tribunal where he holds that the decision appealed against was erroneous in point of law.
%
%Sub-paragraph (5)  enables the Commissioner, where he has set aside an appeal tribunal's decision under sub-paragraph (4) : ($a$)  to give the decision he considers the tribunal should have given, if he can do so without making fresh or further findings of fact; ($b$)  to make such findings of fact and, in the light of them, to give the decision he considers appropriate; or ($c$)  to refer the case to a tribunal with directions for its determination.
%
%Sub-paragraph (6)  provides that references made under sub-paragraphs (3)  or (5)  shall be to an appeal tribunal differently constituted from the tribunal which gave the original decision, unless otherwise directed by the Commissioner.
%
%Sub-paragraph (7)  provides for leave to appeal to be given by the person who constituted, or was the chairman of, the tribunal when the decision was given, or, in prescribed cases, by another person prescribed in regulations or by a Commissioner.
%
%Sub-paragraph (8)  provides for regulations to make provision as to the manner and time in which appeals and applications for leave to appeal should be made.
%Paragraph 9: Appeal from Commissioner on a point of law
%
%803. Paragraph 9 sets out provisions in respect of appeals from the decision of the Commissioner, what may be appealed, when and how. Appeals may be made to the appropriate court in England and Wales or Scotland. This paragraph mirrors provisions of section 15 of the Social Security Act 1998.  It is intended that the regulations made under that section, which are contained in the Social Security Commissioners (Procedure) Regulations 1999, would be amended where necessary to take account of HB/CTB.
%
%Sub-paragraph (1)  allows an appeal on a point of law to the appropriate court from any decision of a Commissioner.
%
%Sub-paragraph (2)  provides that appeals can only be made with the leave of a Commissioner (either the Commissioner who gave the decision or, in prescribed cases, by another Commissioner selected in accordance with regulations) or, if the Commissioner refuses leave, with the leave of the appropriate court.
%
%Sub-paragraph (3)  specifies the persons who may apply for leave to appeal; namely, ($a$)  a person who was entitled to appeal to the Commissioner, ($b$)  any other person who was a party to the proceedings, and ($c$)  any other person authorised by regulations to apply for leave.  It also allows for regulations to make provision in relation to manner, time limit and procedure as to applications for leave to appeal.
%
%Sub-paragraph (4)  requires the Commissioner to specify, on an application for leave to appeal, the appropriate appeal court by reference to the dwelling in respect of which HB/CTB was awarded, namely, the Court of Appeal, where the dwelling is in England and Wales, and the Court of Session where the dwelling is in Scotland.  A different appeal court will be specified, irrespective of where the dwelling is situated, if necessary, having regard to the circumstances of the case and the convenience of the parties.
%
%Sub-paragraph (5)  defines terms used in this paragraph.
%Paragraph 10: Procedure
%
%804. This paragraph provides for regulations to set out such procedures relevant for the purposes of this Schedule as are specified in Schedule 5 to the Social Security Act 1998.  It also provides for Commissioners to have the assistance of experts in cases of special difficulty, and for tribunals of three or more Commissioners where there is a question of law of special difficulty. This paragraph mirrors provisions of section 16 of the Social Security Act. It is intended that regulations made under that section which are contained in the Social Security Commissioners (Procedure) Regulations 1999 would be amended where necessary to take account of HB/CTB.
%
%Sub-paragraph (1)  enables provision to be made in regulations as to procedural matters for HB/CTB decision-making and appeals.  These would reproduce the effect of provisions in regulations made under Schedule 5 to the Social Security Act 1998 in respect of other social security benefits.
%
%Sub-paragraph (2)  provides that regulations specifying procedure to be followed in cases before a Commissioner shall provide for the hearing to be heard in public, unless the Commissioner, for special reasons, otherwise directs.
%
%Sub-paragraph (3)  provides that the power to prescribe procedure includes the power to make provision ($a$)  for appellants to be represented at hearings by another person (regardless of whether that representative has professional qualifications); and ($b$)  to confer on the Secretary of State the right to be heard in any proceedings before a Commissioner to which he is not already a party.
%
%Sub-paragraph (4)  provides that, if a matter before a Commissioner involves an especially difficult question of fact, he may direct that he should have expert assistance from one or more persons with relevant knowledge or experience.
%
%Sub-paragraph (5)  provides that the Chief Commissioner (or, if he is unable to act, any other Commissioner who has been nominated to act in his place) can direct that an appeal, or an application for leave to appeal, which involves a question of law of special difficulty be heard by a tribunal of three or more Commissioners.
%
%Sub-paragraph (6)  provides that, if the decision of the tribunal of Commissioners is not unanimous, the decision of the majority shall be the decision of the tribunal.  In a case where votes are equally divided, the presiding Commissioner shall have a casting vote.
%
%Sub-paragraph (7)  ensures that the reference in paragraph 8(7)($b$)  to applications for leave to appeal being determined by a Commissioner may be construed as a reference to a tribunal of Commissioners when appropriate.
%
%Sub-paragraph (8)  provides that Part I of the Arbitration Act 1996 (which relates to arbitration pursuant to arbitration agreements) shall not apply to any proceedings under this Schedule, unless regulations provide otherwise in relation to England and Wales.
%Paragraph 11: Finality of decisions
%
%805. This paragraph provides that any decision made under the preceding provisions of this Schedule is final unless it is revised, superseded or appealed.
%Paragraph 12: Matters arising as respects decisions
%
%806. Paragraph 12 provides for regulations in respect of matters arising whilst a decision is pending, and those arising out of a revision of, or an appeal from, a decision. The intention is that regulations are to be made to allow an interim decision pending a decision by the local authority, an appeal tribunal or Commissioner which relates to a claim for or entitlement to HB/CTB. It would also enable provision to be made in regulations for matters arising out of the revision or appeal of such a decision.
%Paragraph 13: Suspension in prescribed circumstances
%
%807. This paragraph provides for regulations to be made for suspending payments of HB/CTB and the subsequent making of any payments so suspended in prescribed circumstances. It replaces sections 5(1)(n) and (o) and 6(1)(n) and (o) of the Administration Act.
%
%Sub-paragraph (1)  gives a power for regulations to: ($a$)  set out the circumstances in which payments of HB/CTB may be wholly or partially suspended; ($b$)  prescribe circumstances in which any reduction by way of CTB in a person’s council tax liability may be partly or wholly suspended; and ($c$)  prescribe the circumstances in which any or all the suspended payments are to be restored.
%
%Sub-paragraph (2)  provides that regulations under sub-paragraph (1)  above may, in particular, cover cases where ($a$)  there is a doubt as to whether the conditions of entitlement to benefit are met; ($b$)  there is a question as to whether a decision may need to be revised or superseded; ($c$)  an appeal against a decision of an appeal tribunal, a Commissioner or a Court is pending; or ($d$)  the local authority considers that an award of HB/CTB might need to be revised or superseded after a decision is given on appeal in a different case by a Commissioner or a court. Sub-paragraph (3)  defines “pending” for the purposes of this sub-paragraph.
%
%Sub-paragraph (4)  clarifies that the reference in sub-paragraph (2)($d$)  to a different case includes a case involving a different local authority, but does not include a case relating to a benefit other than HB or CB.
%Paragraph 14: Suspension for failure to furnish information etc
%
%808. This paragraph provides a regulation-making power so that payments of benefit may be suspended where a person fails to provide information needed to determine whether a decision on an award of benefit should be revised or superseded.
%
%Sub-paragraph (1)  provides that the power to suspend in this paragraph applies to failure to comply with information requirements (defined in sub-paragraph (3) ).
%
%Sub-paragraph (2)  enables regulations to: ($a$)  set out the circumstances in which payments of HB/CTB may be wholly or partially suspended; ($b$)  prescribe circumstances in which any reduction by way of CTB in a person’s council tax liability may be partly or wholly suspended; and ($c$)  prescribe the circumstances in which any or all the suspended payments are to be restored.
%
%In line with other social security benefits, it is proposed that regulations will provide that, as a general rule, payment of benefit may be suspended, in whole or in part, if the person fails to provide the requested information within one month of being asked.
%
%Sub-paragraph (3)  defines an information requirement, for the purposes of this paragraph and paragraph 15, as being a requirement under regulations made under section 5(1)(hh) or 6(1)(hh) of the Administration Act, as amended by paragraphs 21(2)  and 21(3)  of the Schedule, with respect to the provision of information and evidence needed to determine whether a decision on an award of HB/CTB should be revised or superseded.
%Paragraph 15: Termination in cases of failure to furnish information
%
%809. Paragraph 15 provides for regulations enabling a person’s entitlement to benefit to be terminated where payment has been suspended in accordance with regulations under paragraph 13 above and the person has subsequently failed to comply with an information requirement or, in the case of paragraph 14 above, the person has persisted in their failure to comply with an information requirement.
%
%810. In line with other social security benefits, in the case of suspension under paragraph 14, it is proposed that regulations would provide that benefit would be suspended until the claimant provides the requested information, up to a maximum period of one month. If at the end of a month the information is not provided then entitlement to benefit would be terminated from the date suspension commenced. In the case of a suspension under paragraph 13, it is proposed that regulations would provide a maximum period of one month following the notification of the information requirement. The one-month period may be extended if special reasons apply.
%Paragraph 16: Decisions involving issues that arise on appeal in other cases
%
%811. This paragraph makes provision for cases which turn on a point of law which is to be considered by a court on appeal in another case. The local authority may defer making a decision in such cases, or make it in prescribed cases on such basis as may be prescribed.
%
%Sub-paragraph (1)  provides that this paragraph applies where a decision falls to be made (including one revising or superseding an earlier decision) which turns on an issue of law which is being challenged in another case (the “lead case”), through the Courts.
%
%Sub-paragraph (2)  provides that the local authority need not make a decision where they consider that the outcome of the lead case might mean there would be no entitlement to benefit.
%
%Sub-paragraph (3)  provides that if the local authority considers it possible that the outcome of the lead case might affect the decision in the case in some other way, they ($a$)  need not make the decision while the appeal is pending, except in prescribed cases or circumstances; or ($b$)  may make the decision on such basis as may be prescribed.
%
%Sub-paragraph (4)  requires the local authority, where they have made a decision on the prescribed basis, to revise that decision where appropriate, once the lead case is finally decided.
%
%Sub-paragraph (5)  defines when an appeal is pending in a lead case for the purposes of this paragraph.  Regulations made under this sub-paragraph would cover cases where an appeal has not been brought, or an application for appeal has not been made, but the time limit for doing so has not expired.  This would cover the situation where the local authority has received a decision of an appeal tribunal or a Commissioner and is considering whether they should seek leave to appeal against it.
%
%Sub-paragraph (6)  provides that the reference to an appeal, or application for leave to appeal, in sub-paragraph (5)  includes an application for judicial review to the High Court or its equivalent in the Court of Session.
%
%Sub-paragraph (7)  clarifies that the reference to another case in sub-paragraph (1)  includes a case involving a decision made by a different local authority, but does not include a case relating to a benefit other than HB or CTB.
%Paragraph 17: Appeals involving issues that arise on appeal in other cases
%
%812. Paragraph 17 deals with appeals which turn on an issue of law which is being challenged in another case (the “lead case”) through the courts. It allows the local authority to require an appeal tribunal or Commissioner: to refer an appeal case to them instead of deciding it, to stay an appeal case, or to make a decision in an appeal case as if the lead case was decided in the way most unfavourable to the appellant.
%
%Sub-paragraph (1)  provides that this paragraph applies where an appeal is made to an appeal tribunal or Commissioner, and that appeal turns on an issue of law which is being challenged in the lead case through the courts.
%
%Sub-paragraph (2)  provides that the local authority may serve notice requiring the appeal tribunal or, as the case may be, the Commissioner, to refer the appeal to them, or to deal with the appeal in accordance with sub-paragraph (4)  below.
%
%Sub-paragraph (3)  provides that the local authority shall, if and as appropriate, revise or supersede the decision in a case referred to them under sub-paragraph (2)($a$), in accordance with the decision in the lead case, once that case is finally decided.
%
%Sub-paragraph (4)  provides that where the local authority issues a notice under sub-paragraph (2)($b$)  above, the appeal tribunal or Commissioner shall either ($a$)  stay the appeal or ($b$)  if they consider it to be in the interests of the appellant to do so, decide it as if the lead case were decided in the way most unfavourable to the appellant.  A decision in the latter instance would allow any benefit, which would not be affected by the decision in the lead case, to be paid.
%
%Sub-paragraph (5)  requires the local authority, if appropriate, to supersede any decision of the appeal tribunal or Commissioner under sub-paragraph (4)($b$)  above once the lead case is finally decided.
%
%Sub-paragraph (6)  sets out where an appeal is pending in a lead case for the purposes of this paragraph.  Regulations to be made under this sub-paragraph would cover cases where an appeal has not been brought, or an application for leave to appeal has not been made, but the time limit for doing so has not expired.  Regulations would cover the situation where the local authority has received a decision of a Commissioner or court and is considering whether there appears to be an error of law in the decision and whether they should seek leave to appeal against it.
%
%Sub-paragraph (7)  provides that the reference to an appeal to a Commissioner in sub-paragraph (1)($a$)  includes a reference to an application for leave to appeal to a Commissioner.  It also clarifies that the reference in sub-paragraph (1)($b$)  to a different case includes a reference to a case involving a different local authority, but does not include a case relating to a benefit other than HB or CTB.  It further provides that the reference to an appeal or an application for leave to appeal in sub-paragraph (6)  includes an application for judicial review to the High Court or its equivalent in the Court of Session.
%
%Sub-paragraph (8)  defines “appellant” for the purposes of sub-paragraph (4) .
%
%Sub-paragraph (9)  enables regulations to be made to supplement provision in this paragraph.
%Paragraph 18: Restrictions on entitlement to benefit in certain cases of error
%
%813. Paragraph 18 provides that, where the outcome of an appeal overturns an understanding of the law previously applied, with the effect that decisions in other cases are wrong, restrictions may be imposed on arrears which would otherwise fall to be paid. It also defines the terms used.
%
%Sub->paragraph (1)  identifies those cases where entitlement to benefit is restricted because an understanding of the law has been overturned by a decision on appeal (a “relevant determination”).  Sub-paragraph (2)  identifies cases to which subsection (1)  does not apply.
%
%Sub-paragraph (3)  provides that cases identified in accordance with sub-paragraph (1)  above, are to be decided, as far as the period before the appeal decision is concerned, as though that relevant determination had not overturned the earlier understanding of the law.
%
%Sub-paragraph (4)  makes clear that appeal decisions of the courts, which determine that provisions in statutory instruments are themselves unlawful, are relevant determinations for the purpose of sub-paragraph (1)($a$) .  That is to say that, even though a provision in a statutory instrument is found to be ineffective by a decision on appeal, cases will be determined on the basis that it is effective so far as the period before that determination is concerned.
%
%Sub-paragraph (5)  provides that the restriction on the payment of arrears applies regardless of whether a claim or application for revision or supersession is made before or after the date of the determination of the appeal in the lead case.
%
%Sub-paragraph (6)  defines “the court” for purposes of this paragraph.
%
%Sub-paragraph (7)  clarifies that references to entitlement to benefit also covers ($a$)  entitlement to any increase in the rate of benefit and ($b$)  also that entitlement includes a benefit (or increase) at a particular rate.
%
%Sub-paragraphs (8)  and (9)  allow regulations to be made to prescribe how the date of the relevant determination is to be determined.
%Paragraph 19: Correction of errors and setting aside of decisions
%
%814. This paragraph permits regulations to be made defining circumstances in which accidental errors in decisions may be corrected, and provides for setting aside decisions in specified cases where there is procedural irregularity.
%
%Sub-paragraph (1)($a$)  provides for regulations to be made for the correction of accidental errors in a decision or record of decision given by the local authority, an appeal tribunal or a Commissioner.  Thus “slips of the pen” (such as mis-written dates) can be corrected.  Sub-paragraph (1)($b$)  provides for regulations to be made for the appeal tribunal or Commissioner to set aside a decision and re-hear a case in the interests of justice if a procedural error has occurred in the service or receipt of documents, or if a party to the proceedings was not present at the hearing.
%
%Sub-paragraph (2)  provides that any power exercised under this provision shall not derogate from any other power to correct or set aside decisions.
%
%Sub-paragraph (3)  defines “relevant provision”.
%Paragraph 20: Regulations
%
%815. This paragraph makes provision in respect of subordinate legislation. It provides, in particular, that regulations made under paragraph 6(2)($e$)  or (4)  in respect of decisions against which no appeal lies shall be subject to affirmative resolution.
%Paragraph 21: Consequential amendments of the Administration Act
%
%816. Paragraph 21 amends sections 5 and 6 of the Administration Act to provide for regulations with respect to the provision of information and evidence needed to determine whether a decision on an award of HB/CTB should be revised or superseded.
%Paragraph 22: Consequential amendments of the Social Security Act 1998
%
%817. This paragraph provides for the repeal of sub-sections 34(4)  and (5)  (reviews of HB/CTB determinations) and section 35 (suspension of benefit in prescribed circumstances) of the Social Security Act 1998, which are replaced by the provisions of this Schedule. The paragraph also amends Schedules 1 and 4 to that Act to extend the payment of travelling and other expenses to people attending appeal tribunals and hearings before a Commissioner under the provisions of this Schedule.
%Paragraph 23: Interpretation
%
%818. This paragraph defines terms used in this Schedule, and provides for regulations to specify the circumstances in which a person is, or is not, to be considered as a person who is affected by any decision of a local authority. It also clarifies that decisions of persons authorised to carry out functions of, or providing services to, a local authority, are to be treated as decisions of that local authority for the purposes of this Schedule.
%Discretionary Payments
%Section 69: Discretionary financial assistance with housing
%
%819. Under these provisions local authorities will have discretion to provide people entitled to Housing Benefit or Council Tax Benefit (HB/CTB), or both, with additional financial assistance with their rent or Council Tax. These payments will help alleviate exceptional hardship people may incur where their rent is above that met by Housing Benefit, and to cater for unforeseen exceptional circumstances, such as non-payment of wages.
%
%820. The provisions will provide for discretionary housing payments made by the local authorities to be paid in addition to benefit entitlement. The decision to make payments will be discretionary. The local authority will decide whether to make payment and the level of each payment based on the circumstances of each individual case. The total amount a local authority will be able to spend on the discretionary scheme in any one year will be part of the framework of the scheme determined by the Secretary of State but the decision to pay on any particular case will be for the local authority. People requesting help from the discretionary scheme will be able to ask local authorities to review their decision if for whatever reason they are dissatisfied.
%
%821. Subsection (1)  provides for regulations which will enable local authorities to make discretionary payments to people in receipt of HB/CTB.
%
%Subsection (1)($a$)  and ($b$)  provide for regulations to be made conferring powers on local authorities to consider making payments to people entitled to HB/CTB who appear, to the local authority, to need further financial assistance to meet their housing costs. It is also intended that the regulations will set out the information that will be required by local authorities to make decisions, and the provision for people to ask for decisions to be reviewed. Decisions about the people to help, the financial amount, and the period for which such help shall be given, will be for the local authorities to determine.
%
%822. Subsection (2)  provides for the regulation-making powers which will set out the conditions and circumstances under which these payments may be considered.
%
%Subsection (2)($a$)  provides for regulations to be made for the circumstances under which discretionary housing payments may be made. It is intended that the regulations made under this power will enable local authorities to consider making payments to tenants (other than local authority tenants) entitled to HB, for example to alleviate exceptional hardship, and to claimants entitled to HB or CTB, for example, to cater for unforeseen exceptional circumstances.
%
%Subsection (2)($b$)  enables regulations to be made to provide that decisions about whether to help any particular persons, the amount of the payments and the period for which the help should be given, will be (subject to what follows) for local authorities.
%
%Subsection (2)($c$)  enables regulations to be made imposing a limit on the amount of the discretionary housing payment that may be made in any particular case. It is intended that the regulations will specify that the additional financial help given to any person should not exceed the eligible rent (contractual rent less ineligible charges) as prescribed by the HB scheme, or where the help is for the paying of council tax, not higher than the amount a person would otherwise be entitled to under the rules of the CTB scheme.
%
%Subsection (2)($d$)  enables regulations to be made to restrict the period for which discretionary housing payments are made.
%
%Subsection (2)($e$)  enables regulations to be made to set out the way in which people should apply for discretionary payments and the way in which local authorities should deal with those claims.  The regulations may, for example, require local authorities to deal with the claims timeously and to notify their decisions within a set time.
%
%Subsection (2)($f$)  enables regulations to be made to enable local authorities to gather such relevant information from the people as is reasonably required to enable them to make a decision.
%
%Subsection (2)($g$)  enables regulations to be made in respect of the circumstances in which discretionary housing payments could cease, for example, where the person is no longer entitled to housing benefit or council tax benefit, and will enable the recovery of overpayments.
%
%Subsection (2)(h) enables regulations to be made to provide for local authorities to review their decisions in particular circumstances.
%
%823. Subsection (3)  provides that regulations made under this section should be subject to negative procedure. Subsection (4)  provides supplementary incidental powers to be used when making regulations under the provisions of this section. Subsection (5)  enables regulations to be made for different areas or different local authorities should the need arise.
%
%824. Subsection (6)  amends section 176(1)  of the Social Security Administration Act 1992 to require consultation with representative organisations, for example, the Local Authority Associations, prior to the making of regulations under the above provisions.
%
%825. Subsection (7)  defines the terms used.
%Section 70: Grants towards the cost of discretionary housing payments
%
%826. Under these provisions, local authorities will have the power to make discretionary payments in certain circumstances. These payments will be cash-limited in order to ensure that the discretion is used prudently. Powers will allow Central Government to provide authorities with financial assistance with the scheme both in terms of the scheme’s administration and the payments made under it.
%
%827. This section makes financial provision in respect of the housing discretionary payment scheme. It provides for the receipt and distribution of money by the Secretary of State to authorities in order to enable them to make discretionary housing payments, but with a ceiling on the total which they can spend.
%
%828. Subsection (1)  makes provisions for each authority to be given central Government funding towards their discretionary housing payments, and administrative expenditure.
%
%829. Subsection (2)  makes provision in respect of the calculation and payment of the amount of housing discretionary payments by applying certain provisions of section 140B and 140C of the Administration Act.
%
%830. Subsection (3)  enables the Secretary of State, by order, to make provision for an upper limit to be set, or to impose subsidiary limits, on the amount which authorities can spend on discretionary payments in any financial year. Subsection (4)  provides that the order may specify those limits or provide for the way in which the limit should be determined. Subsection (5)  provides for the Parliamentary control of the order made under this section, by providing for the negative procedure to be followed.
%
%831. Subsection (6)  provides supplementary incidental powers to be used when making Orders under the provisions of this section. Subsection (7)  enables regulations to be made for different areas or different local authorities should the need arise.
%
%832. Subsection (8)  defines the terms used.
%Recovery of Housing Benefit
%Current position
%
%833. For tenants in the private rented sector, Housing Benefit may be paid direct to the landlord or his managing agent in prescribed circumstances. Currently approximately 56% of private sector tenants and 86% of registered social landlord tenants have their benefit paid direct to their landlord or agent.
%
%834. Tenants in receipt of Housing Benefit are required to notify the local authority of any change in their circumstances that could affect their entitlement, for example, a change of address or change in income. Landlords and agents receiving benefit direct are also required to notify the local authority of any change in their tenants’ circumstances that they might reasonably be expected to know could affect the tenants’ entitlement to Housing Benefit.
%
%835. Local authorities often face problems recovering Housing Benefit debts if tenants simply disappear. Housing Benefit legislation, therefore, allows local authorities, where benefit is paid direct, the discretion to recover an overpayment from either the tenant or the landlord or agent (as the person to whom benefit was paid). The result of this is that, in cases of tenant fraud, local authorities may, and often do, recover the debt from the landlord or his agent. Although such a recovery re-opens the tenant’s rental liability to the landlord, creating rent arrears, in practice it means that the fraudulent tenant often suffers no consequence for their action. This can act as a disincentive to landlords and their agents to report suspected fraud, when they know that they could find themselves repaying any subsequent overpayment.
%The measure in the Act
%
%836. Section 71 makes provision to amend section 75(3)  of the Social Security Administration Act 1992* (the “Administration Act”). Section 75(3)  currently provides for recovery to be made in all cases from the person to whom it was paid or such other person as may be prescribed. Landlords and agents fall under the first category as the person to whom benefit is paid. The measure will allow for exceptions to be made in regulations to the provision that overpayments are recoverable from the person to whom benefit was paid. Regulations will prescribe that where the landlord or agent has reported suspected tenant fraud, the local authority cannot recover from the landlord/agent to whom it was paid but can recover from the tenant. However, in cases where the landlord is shown to have acted maliciously or is in collusion with the tenant, local authorities will retain their discretion to recover the overpayment from either party.
%Section 71: Recovery of housing benefit
%
%837. This section replaces subsection (3)  of section 75 of the Administration Act, which provides the powers which specify from whom an overpayment of Housing Benefit may be recovered.
%
%New subsection (3)($a$)  provides that an overpayment is recoverable from the person to whom the benefit was paid, but allows for exceptions to this to be prescribed in regulations.  This will allow for regulations to provide for an exception in the case where a landlord or agent has reported suspected tenant fraud, and subsequent investigations have found no evidence to indicate the landlord had acted maliciously or had been in collusion with the fraudulent tenant.
%
%New subsection (3)($b$)  provides that in addition to the overpayment being recovered from the person to whom the benefit was paid, it may also be recovered from such other person as prescribed in regulations.  This replicates the existing power in the current subsection (3) .  Regulations currently prescribe that, in addition to the person to whom benefit was paid, recovery can also be sought from the claimant, or such other person who misrepresented or failed to disclose a material fact that resulted in the overpayment of benefit.  The intention is to continue with these provisions in the new regulations.
%Child Benefit
%Section 72: Child benefit disregards
%
%838. Child Benefit is only payable to a person who is responsible for a child in any given week. Section 143 of the Social Security Contributions and Benefits Act 1992 sets out the meaning of “person responsible for a child”. One of the necessary conditions is that the child is living with that person in that week. Various days of absence can be disregarded in determining whether the child is living with that person. Subsection (3)($c$)  prescribes that such days shall include days in which, in prescribed circumstances, the child is in residential accommodation following arrangements made under statutes specified in the current legislation, one being the Social Work (Scotland) Act 1968. 
%
%839. The intended effect of section 143(3)($c$)  and the regulation made under that power is to provide for Child Benefit to be payable in respect of children who live away from home, in residential accommodation, solely on account of a physical or mental illness or disability. The introduction of the Children (Scotland) Act 1995, which repealed parts of the Social Work (Scotland) Act 1968, had the unintended effect of excluding payment of Child Benefit in respect of a small group of disabled children in Scotland. This section restores the legislative basis for payments of Child Benefit in respect of this group, which are currently being made on an extra statutory basis.
%Social Security Advisory Committee
%Section 73: Social Security Advisory Committee
%
%840. The Social Security Advisory Committee (SSAC) gives advice and assistance to the Secretary of State in connection with his duties under the “relevant enactments”. This section amends section 170 of the Social Security Act 1992 and inserts additional provisions to be treated as relevant enactments for the purposes of scrutiny by SSAC. The same applies in Northern Ireland under provisions relating to “relevant Northern Ireland enactments”.
%
%841. Subsection (1)  lists those provisions in the Act that shall now be inserted into the Social Security Administration Act 1992 and treated as “relevant enactments” for the purposes of scrutiny by SSAC.
%
%842. Subsection (2)  provides for those provisions to be replicated in Northern Ireland, and therefore treated as relevant Northern Ireland enactments for the purposes of SSAC scrutiny.
%Part IV: National Insurance Contributions
%Background
%The current position
%
%843. Earnings-related National Insurance contributions (NICs*) are paid by both employees and employers on all cash earnings which reach a set amount. These contributions go into the National Insurance Fund to pay for contributory benefits. However, employees pay no NICs on most non-cash earnings (payments, or benefits, in kind).
%
%844. Since 1991, Class 1A NICs, which are an employer-only charge, have been liable on car and fuel benefits provided for the employee. But most other non-cash benefits are provided NIC-free. Some benefits, which are readily convertible assets, such as shares or gold bars, as well as non-cash vouchers, are already subject to Class 1 NICs paid by both employers and employees. Such benefits are set out in Schedules 1A, 1B and 1C to the Contributions Regulations 1979. 
%
%845. All benefits in kind are subject to income tax as emoluments either under section 19 or Part V of the Income and Corporation Taxes Act 1988 (ICTA).
%Recent developments
%Benefits in kind
%
%846. The Chancellor announced in the March 1999 Budget that all taxable benefits in kind not already subject to NICs would become liable to a Class 1A charge from April 2000.  This will align more closely the tax and NICs treatment of provided benefits such as private medical insurance, beneficial loans or assets transferred to the employee.
%
%847. This will implement a recommendation of the Task Force chaired by Martin Taylor, ex-Chief Executive of Barclays plc., in the report The Modernisation of Britain’s Tax and Benefit System.
%Share options
%
%848. Since 6 April 1999, gains made on share options granted after 5 April 1999, acquired by reason of employment, have been subject to Class 1 National Insurance Contributions (NICs). This does not apply, however, to share options awarded and exercised under an Inland Revenue approved scheme, nor to options where the shares, or the right to acquire shares, are not readily convertible assets, i.e. they are not readily convertible into cash.
%
%849. Whilst companies can plan for NICs on regular pay, it is more difficult for them to plan for NICs on share option gains, particularly where the share price is volatile. Under accountancy rules, companies are required to make a provision in their accounts for the anticipated NICs liability on share options, based on the market price of the shares on the date that they prepare the accounts. Employers have expressed concern that their exposure to a highly unpredictable NIC liability could put at risk their investment strategies, and even make some technically insolvent.
%The measures in the Act
%Benefits in kind
%
%850. The new measures contain provisions which are intended to make all non-cash earnings received by directors or employees earning £8,500 or more per year subject to Class 1A (employer-only) NICs, unless they are already subject to Class 1 NICs, or Class 1B NICs (if they are minor benefits which are included in a PAYE Settlement Agreement, an administrative arrangement under which an employer accounts directly to the Revenue for tax and NICs on minor or irregular payments).
%
%851. The extended Class 1A charge follows the shape of the existing charge on car and fuel benefits. The Class 1A due will be calculated using the valuation figures already required for tax purposes which an employer enters on a P11D form – a report to the Inland Revenue of the value of benefits in kind provided throughout the tax year* to each employee. This is intended to keep extra reporting requirements to a minimum.
%
%852. The new measures provide powers for Treasury Ministers to make regulations to except particular items or reduce the Class 1A liability where appropriate. The sections also give powers to prescribe penalties for failure to pay Class 1A NICs correctly and other matters, such as dates and forms of records, surrounding collection of the NICs.
%
%853. The Act also includes measures to move the liability for Class 1A NICs on benefits in kind provided by a third party from the direct employer of the recipient to the third party provider. Typically, this would arise where a manufacturer wishes to reward or incentivise the salespeople in retail outlets that sell his products, for example by providing them with holidays, goods or sometimes non-cash vouchers. The staff are likely to be employed by the retailer, not the manufacturer. Currently, the employer is liable for NICs on these benefits, possibly without his knowledge, and without any involvement in their provision. In general, the third parties have indicated their willingness to meet NICs liabilities of providing incentive awards, as they can do for tax.
%
%854. The measures in the Act provide a mechanism to do this, by moving the liability for the Class 1A NICs on provided benefits and any related tax to that third party. This new liability is to be a voluntary commitment for the tax year 2000-2001.  If the third party chooses not to pay the Class 1A NICs, the liability will rest with the employer, but in future tax years, the liability will be compulsory for the third party.
%
%855. To remove all need for the employer to be involved, regulations came into force on 6 April 2000 to make non-cash vouchers that are provided by a third party liable to Class 1A NICs, rather than Class 1 NICs as at present (SI 2000/761 in GB and 2000/758 in Northern Ireland).
%
%856. A further provision allows any emolument received from employed earner’s employment which is subject to Schedule E tax also to be treated as earnings for National Insurance. This is intended to be used to clarify the NICs treatment of items bought using a company credit card or similar form of purchase. It could also be used for tax/NICs alignment, such as to introduce the NICs regulations relating to the proposed new all-employee share plan.
%Share options
%
%857. The provision in the Act does three things. First, it allows employers and employees to reach an agreement that a secondary contributor (usually the employer) may recover from the employee some or all of the secondary NIC in respect of a gain made from a right to acquire shares. These gains usually take the form of the exercise of share options.
%
%858. Second, as an alternative, the employer may make an application for approval of an election to the Board of Inland Revenue. If approval is obtained, the employer and employee can jointly elect to transfer all or some of the liability to pay the secondary NIC on the share option gain to the employee. Elections can only be made when the Board are satisfied that the election to be used will transfer the liability, and that the accompanying arrangements will ensure that the NIC liability transferred to the employee is paid.
%
%859. Third, the provision strengthens the existing statutory bar that prevents the employer recovering any part of the secondary NIC in respect of all forms of earnings (subject to the new exception for rights to acquire shares). It extends the protection to the employee by ensuring that the person liable to pay the Class 1A (on benefits in kind) or Class 1B (on Pay As You Earn Voluntary Settlement Agreements) cannot recover his liability from the employee.
%Commentary on Sections
%
%860. Currently, employees pay primary Class 1 National Insurance contributions (NICs) on all cash earnings they receive between the Lower Earnings Limit* (LEL) and the Upper Earnings Limit* (UEL). Employers pay secondary Class 1 NICs on all cash payments above the earnings threshold (equal to the Single Person’s Tax Allowance) which they pay to employees with no UEL applying. Non-cash benefits are excluded in regulations from the computation of earnings subject to Class 1.  But some, such as bonds and gemstones, are brought back into a Class 1 NICs charge. Employers also pay Class 1A NICs – at a rate equivalent to the Class 1 secondary rate – on the value of car and fuel benefits which are provided to their employers for their private use. Employees pay no NICs on such benefits.
%Section 74: Contributions in respect of benefits in kind (Great Britain)
%
%861. This section enables all taxable benefits not yet subject to NICs to be brought within a Class 1A NICs charge.
%
%862. Subsection (1)  makes a change to section 1(2)($b$)  of the Social Security Contributions and Benefits Act 1992* (the “Contributions and Benefits Act”) to reflect the fact that the Class 1A charge will no longer apply to car and fuel benefits.
%
%863. Subsection (2)  replaces section 10 of the Contributions and Benefits Act, which sets out the existing Class 1A charge, with a new section 10. 
%New section 10: Class 1A contributions: benefits in kind etc.
%
%864. New section 10(1)  defines the circumstances when a Class 1A contribution is due. An earner receives an emolument which is chargeable to tax under Schedule E from employed earners’ employment to which Chapter II, Part V ICTA applies – i.e. the earner is a director or earns £8,500 per year or more. As all or some of the emolument received is exempted from, or not liable to, Class 1 NICs, then Class 1A NICs are due.
%
%865. New section 10(2)  and (3)  provides that the person liable to pay the Class 1A NICs (usually the employer) is also the person who is liable to pay secondary Class 1 NICs in that tax year – or would be if there were any earnings liable to Class 1 NICs.
%
%866. New section 10(4)  provides that the amount of Class 1A due is the amount of the emolument not subject to Class 1 - as per subsection (1)  - multiplied by the Class 1A rate for the tax year.
%
%867. New section 10(5)  provides that the rate for Class 1A is the same as that for secondary Class 1 NICs.
%
%868. New section 10(6)  provides that Class 1A is not due on any emoluments which have been included in a PAYE Settlement Agreement for tax and NICs purposes.
%
%869. New section 10(7)  provides that, for section 10 only, the emolument subject to Class 1A shall amount to the valuation of the benefit for tax purposes.
%
%New section 10(7)($a$)  disapplies reliefs or allowances which, for income tax purposes, the employee may be able to claim against their tax under ICTA sections 198, 201, 201AA or 332(3), which allow deductions for certain types of expenses.  These will not affect the valuation for Class 1A NICs purposes where the benefit is provided partly for business and partly for private use.
%
%New section 10(7)($b$)  excludes from Class 1A charge those benefits where the whole amount of the emolument would be deductible for tax because it has been provided wholly for business purposes.  An example would be a fax machine provided only for use when engaged in the employed earner’s employment.
%
%870. New section 10(8)  provides regulation-making powers for the Treasury to amend the effect of 10(7) . It will allow, should this be needed, the matching by regulations of any alterations to relevant tax legislation. For example, if a new ICTA section introduced a new relieving provision which needed to be included for the coherence of the Class 1A NICs scheme this could be done in regulations.
%
%871. New section 10(9)  provides regulation-making powers for the Treasury to exempt certain persons or types of emolument from Class 1A liability or reduce Class 1A liability. The Government anticipates using these powers, for example, to mirror items already exempted from Class 1 NICs, such as certain forms of training; or to reduce liability where the cost of providing a benefit is split between more than one company.
%
%872. Subsection (3)  of section 74 introduces new subsection 4(6)  of the Contributions and Benefits Act. This provides regulation-making powers for the Treasury to treat any amount, which is the value of a benefit subject to Schedule E tax, as earnings from employed earner’s employment. As such, it will maintain the existing use of the power in subsection 4(6)  in relation to the provision of conditional and convertible shares. It further allows the Treasury to prescribe the time and manner in which the earnings are to be treated as being paid.
%
%873. An example of how this power could be used is to prescribe that where an employee buys or obtains goods or services by use of a company credit card which are immediately transferred from the company to that individual the amount involved in the purchase shall be liable to Class 1 NICs. Also, it would permit future alignment of the tax and NIC treatment of payments to employed earners. For example, it is possible that this power may be used to mirror in NIC legislation the tax provisions relating to the new all-employee share plan, proposed in the 1999 Budget.
%
%874. Subsections (4)  to (7)  make consequential changes to existing legislation.
%
%875. Subsection (8)  provides that the Class 1A charge shall come into effect from 6 April 2000 to match the beginning of the tax year 2000-2001. 
%
%876. Subsection (9)  provides that statutory instruments made under any of the powers in the new section 10 may be backdated to the beginning of the tax year in which they are made. This will allow the details of the Class 1A charge to become operative from the beginning of the tax year. This provision is likely only to be used in future years to mirror a change in tax legislation in a Finance Act for that year.
%Section 75: Third party providers of benefits in kind: GB
%
%877. Section 75 concerns the provision of benefits in kind (BIKs) to employees by somebody other than their own employer – a third party. The new measure only operates where the employer has not arranged or facilitated the provision of the BIKs.
%
%878. In the case of a third party provider, the term “benefits in kind” includes non-cash vouchers, which will be moved from a Class 1 to a Class 1A NICs liability in regulations.
%
%879. The section moves the liability for Class 1A NICs on third party provided BIKs from the employer to the third party.
%
%880. On Royal Assent this provision takes retrospective effect to 6 April 2000.  In the first year the third party may choose to meet the Class 1A liability on BIKs he provides. From 6 April 2001 the Class 1A liability becomes compulsory on the third party provider.
%
%881. Subsection (1)  introduces new sections 10ZA and 10ZB into the Contributions and Benefits Act.
%New section 10ZA: Liability of third party provider of benefits in kind
%
%882. New section 10ZA(1)  lists the elements necessary for this measure to take effect. The employee or a member of his family must receive a taxable emolument that attracts a Class 1A charge under the new section 10; the BIK is provided by someone other than the employer; and the employer has not arranged or facilitated the provision.
%
%883. New section 10ZA(2)  provides that where the third party also pays a sum to meet the employee’s tax liability on the BIK, that payment also is subject to Class 1A NICs, rather than Class 1 as it would be if the employer had paid it.
%
%884. New section 10ZA(3)  moves the liability for the Class 1A NICs on the relevant benefit and any associated tax from the employer and onto the third party, except in the circumstances in subsection (4) .
%
%885. New section 10ZA(4)  provides that for the tax year commencing 6 April 2000 the third party needs to elect to pay the Class 1A NICs and notify the Revenue in writing.
%
%886. New section 10ZA(5)  gives the Treasury power to prescribe in regulations the meaning of “arranged or facilitated”.
%
%887. New section 10ZA(6)  defines members of an employee’s family as carrying the same meaning as in s.168(4)  ICTA.
%New section 10ZB: Non-cash vouchers provided by third parties
%
%888. This section applies where the third party provider is awarding non-cash vouchers, as defined in section 141 of the Income and Corporation Taxes Act 1988 (ICTA).
%
%889. New section 10ZB(2)  provides that a Class 1A NICs charge is liable on all vouchers provided to employees by third parties no matter whether they earn over the £8,500 limit set for other benefits or below that level.
%
%890. Subsection (2)  of section 75 inserts a new subsection (3A)  into 110ZA of the Social Security Administration Act 1992.  This includes the premises of third party providers in the list of premises liable to inspection by Revenue officers.
%
%891. Subsection (3)  brings new section 10ZA into force from 6 April 2000 on Royal Assent.
%
%892. Subsection (4)  allows any regulations made under the power in new section 10ZA to be retrospective back to the commencement of the tax year in which they are made.
%Section 76: Collection etc. of NICs: Great Britain
%
%893. Currently the payment of Class 1A National Insurance contributions may be made by one of two methods – via the Inland Revenue Pay As You Earn (PAYE) system, or direct to the Inland Revenue National Insurance Contributions Office. With the extension of Class 1A to all taxable benefits, there will be a new single payment and collection method. Employers will make one annual return for their Class 1A to the Accounts Office where they already send their PAYE payments, accompanied by a separate payment slip.
%
%894. This section makes minor amendments to current legislation to support the operation of the new method. The detail of the new payment method will be in regulations.
%
%895. Subsection (1)  explains that the following subsections (2)  to (5)  make amendments to Schedule 1 to the Contributions and Benefits Act.
%
%896. Subsection (2)  amends paragraph 7(2)($b$)  to include the application of section 5 of the Taxes Management Act 1970 (evidence in cases of fraudulent conduct), in relation to certain penalties. This brings the PAYE payment method (for payment mostly of Class 1 contributions) and the new Class 1A payment method into alignment.
%
%897. Subsection (3)  substitutes a new paragraph 7B(2)($e$)  to provide for regulations to determine the date from which interest is to be calculated, in cases where a Class 1A contribution is not paid by the due date. This aligns with the wording in paragraph 6, which relates to NICs collected with PAYE tax.
%
%898. Subsection (4)  inserts a new sub-paragraph (5A)  into paragraph 7B, which provides for regulations to be made which may apply provisions contained in Part X (Penalties, etc.) of the Taxes Management Act 1970. 
%
%899. Subsection (5)  inserts a new paragraph 7BA which provides for regulation-making powers to prescribe the circumstances under which a payment or repayment of contributions or interest due to a person under Schedule 1 may be offset against any other contributions liabilities which the person may have. For example, a Class 1A overpayment might be offset from a secondary Class 1 liability.
%
%900. Subsection (6)  removes paragraph (1)($j$)  from section 8 of the Social Security Contributions (Transfer of Functions etc.) Act 1999.  This removes from the list of decisions by officers of the Board of Inland Revenue the question of liability to pay interest under paragraph 7B(2)($e$)  of Schedule 1.  The application of interest to late paid Class 1A contributions will apply automatically, as it does for tax and Class 1 contributions collected with PAYE tax.
%
%901. Subsection (7)  provides that the effect of subsection (6)  is only in relation to interest which accrues on Class 1A contributions due in respect of the tax year beginning 6th April 2000 and subsequent years (meaning, therefore, Class 1A contributions due under provisions in the new section 10 of the Contributions and Benefits Act (see section 74(2)  above).
%Section 77: Liability of earner for secondary contributions: Great Britain
%
%902. This section provides for the treatment of National Insurance contributions on share option gains.
%
%903. Subsection (1)  amends Schedule 1 to the Contributions and Benefits Act 1992.  It omits sub-paragraph (2)  of paragraph 3 of Schedule 1 to that Act, which deals with prohibition on deduction or recovery of Class 1 Contributions.
%
%904. Subsection (2)  inserts new paragraphs 3A and 3B into Schedule 1 to the Contributions and Benefits Act 1992. 
%New paragraph 3A: Prohibition on recovery of employer’s contributions
%
%Sub-paragraph 3A(1)  prevents any person who is liable to pay any secondary Class 1 contributions, or any Class 1A or 1B contribution, from recovering these in any way.  This reinforces that an employer may not under any circumstances compel an employee to pay any part of the secondary liability.  It also strengthens the existing position by ensuring that the employer can recover neither of the two existing employer-only National Insurance charges, namely Class 1A (due on benefits in kind) and Class 1B (due on items included in PAYE settlement agreements).  However, this new sub-paragraph is subject to the exception allowed under new sub-paragraph 3A(2).
%
%Sub-paragraph 3A(2)  allows an exception to sub-paragraph 3A(1) .  It allows a secondary contributor to recover some or all of his secondary Class 1 liability in relation to share option gains from the employee, but only where the employee agrees to this.  Sub-paragraph 3A(3)  makes it clear that such agreements will only be allowed if they are made after the date of announcement of this measure, namely 19 May 2000. 
%New paragraph 3B: Transfer of liability to be borne by earner
%
%Sub-paragraph 3B(1)  allows a secondary contributor and an employee to make a joint election to transfer to the employee the liability for some or all of the secondary contributor’s Class 1 contributions relating to gains on share options.  For such an election to be valid, prior approval must be sought from the Inland Revenue.  This approval will be given for both the form of the election and the arrangements made for securing that the payments required to meet the liability transferred by the proposed election will be paid and paid on time.
%
%Sub-paragraph 3B(2)  makes it clear that any liability which has been transferred by such an election is treated for the purposes of this Act, the Administration Act and Part II of the Social Security Contributions (Transfer of Functions, etc.) Act 1999 as a liability falling on the earner.
%
%Sub-paragraph 3B(3)  states that an election under new sub-paragraph 3B(1)  continues in force until it ceases to have effect in accordance with its terms, it is revoked jointly by both parties or the approval of the election is withdrawn by the Inland Revenue in relation to options not yet granted.  Where more than one of these events occur the election will cease to be in force of the date of earliest of these events.
%
%Sub-paragraph 3B(4)  allows the Inland Revenue to give approval to multiple elections relating to a particular secondary contributor as well as single ones.  It can approve elections between a particular secondary contributor and particular earners or a particular class of earners, or elections made by a particular secondary contributor in certain circumstances.
%
%Sub-paragraph 3B(5)  defines the grounds on which the Inland Revenue can refuse approval of an election.  It allows the Inland Revenue to refuse approval if it appears that adequate arrangements to make sure that the liability will be met have not been made, or if the Inland Revenue feels that it does not have enough information to decide on this.
%
%Sub-paragraph 3B(6)  allows the Inland Revenue to withdraw approval if they feel that the arrangements for making sure that the liability is met are proving inadequate, or if they feel that an election they have approved is likely to result in the avoidance or non-payment of secondary Class 1 contributions.
%
%Sub-paragraph 3B(7)  states that the Inland Revenue may withdraw general approval in relation to a particular secondary contributor, or withdraw approval for a single election or multiple elections in accordance with 3B(6) .  It states that such a withdrawal of approval means that any existing election made under the approved arrangements has no effect on contributions due on any right to obtain shares obtained after the approval is withdrawn.
%
%Sub-paragraphs 3B(8)  and (9)  allow the person who applied for (or received) approval the right of appeal in the case of its refusal or withdrawal.  This appeal is to the Special Commissioners.  Where approval is withdrawn and elections made under that approval are effected in relation to future option grants, the employee also has a right of appeal.
%
%Sub-paragraph 3B(10)  makes it clear that elections cannot apply to contributions made on gains realised before the election was made.  This is, however, subject to sub-paragraph 3B(12)  below.
%
%Sub-paragraph 3B(11)  allows the Inland Revenue to make regulations in respect to elections made under sub-paragraph 3B(1) .  These regulations may in particular deal with the matters contained in such an election, the manner in which the election may be made and the manner of applications for approval.
%
%Sub-paragraph 3B(12)  provides a limited allowance for elections to apply to contributions made on gains realised before the election was made.  If an election is made within 3 months of this Act receiving Royal Assent, it may relate to such liabilities arising on or after 19 May 2000. 
%
%Sub-paragraph 3B(13)  clarifies that references to contributions on share option gains by the earner mean any secondary Class 1 contributions payable in respect of a gain treated as remuneration derived from employment under section 4(4)($a$)  of the Contributions and Benefits Act 1992. 
%
%905. Subsection (3)  amends the Contributions and Benefits Act 1992.  In section 6(4) , it replaces the words from “paragraph 3” with the words “paragraphs 3 to 3B of Schedule 1 to this Act”.
%
%906. Subsection (4)  also amends the Social Security Contributions and Benefits Act 1992.  It inserts new sub-paragraph ($ca$)  into paragraph 8(1)  of Schedule 1 to that Act. This new sub-paragraph extends the general regulations to require a secondary contributor to inform an earner to whom liability for secondary Class 1 contributions has been transferred whenever a transferred liability arises, and the amount of that liability.
%
%907. Subsection (5)  amends the Social Security Contributions (Transfer of Functions, etc.) Act 1999.  It inserts new paragraph (ia) into paragraph 8(1)  of that Act. This new paragraph extends the powers of the Inland Revenue to decide whether or not to grant, or withdraw, approval for an election under new sub-paragraph 3B(1)  above. Subsection (6)  inserts a new subsection (2A)  in section 10 and amends section 10(1) . The new subsection (2A)  provides that decisions falling within the new section 8(1)(ia) will not be covered by regulations made under section 10. 
%
%908. Subsection (7)  further amends the Social Security (Transfer of Functions, etc.) Act 1999.  It amends section 12(4)  of that Act, which deals with appeals to be heard by the General Commissioners, to include a reference to the right to appeal to the Special Commissioners given in new sub-paragraph 3B(8)  above.
%Sections 78, 79, 80 and 81
%
%909. These sections mirror the provisions in, respectively, sections 74, 75, 76 and 77, for Northern Ireland.
%Part V: Miscellaneous and Supplemental
%Miscellaneous
%
%Section 82: Tests for determining parentage andSection 83: Declarations of status are described in Part 1 of these notes covering the Child Support measures.
%Supplemental
%Section 84: Expenses
%
%910. This section authorises expenditure incurred under this Act and any increase in expenditure incurred under any Act in so far as that increase is attributable to any provision of this Act.
%Section 85: Repeals
%
%911. This section gives effect to Schedule 9, which repeals certain existing legislation as a consequence of the measures in the Act.
%
%    Part I of the Schedule (Child Support): see commentary on Part I of the Act;
%
%    Part II of the Schedule (State pensions): see commentary on Part II, Chapter I of the Act;
%
%    Part III of the Schedule (occupational and personal pension schemes): see commentary on Part II, Chapter II of the Act;
%
%    Part IV of the Schedule (War Pensions): see commentary on Part II, Chapter III of the Act;
%
%    Part V of the Schedule (loss of benefit): see commentary on sections 62 to 66 (loss of benefit for breach of community service order);
%
%    Part VI of the Schedule (Investigation Powers): see commentary on Schedule 6 (Investigation Powers);
%
%    Part VII of the Schedule (Housing Benefit and Council Tax Benefit): see commentary on sections 68 to 71;
%
%    Part VIII of the Schedule (NICs in respect of benefits in kind: Great Britain (1)  and Northern Ireland (2) ): see commentary on Part IV of the Act;
%
%    Part IX of the Schedule (Tests for determining parentage and declarations of status): see commentary on sections 82 and 83 (following section 15).
%
%Section 86: Commencement and transitional provisions
%Commencement provisions
%
%912. This Act introduces a large number of measures, not all of which will come into force at the same time. Subsections (2)  and (3)  provide for the provisions listed in subsection (1)  to be brought into force, possibly on different days and for different purposes, by order made by the Secretary of State or, in the case of provisions specified in subsection (3)($b$), by the Lord Chancellor. Subsection (4)  provides that for the measures relating to Child Support, other than section 24 (which removes the requirement for the CSA to complete periodical reviews), and the reduction and withdrawal of benefit (sections 62 to 66), this power also includes the power to pilot the measures by bringing the provisions into force on different days in different areas. Those measures which are not specified in, or excepted from, subsection (1)  will come into force on Royal Assent.
%Transitional provisions
%
%913. Subsection (5)  provides the power to make, by regulations, any necessary transitional arrangements in relation to the measures on selection of trustees and of directors of corporate trustees, and on Housing Benefit and Council Tax Benefit revisions and appeals and discretionary housing payments. Subsection (6)  provides that regulations made under subsection (5)  are to be made by negative instrument, and subsection (7)  enables the regulations to (among other things) make different provision for different classes of cases, impose conditions or create exceptions.
%Section 87: Short title and extent
%
%914. The measures in the Act will apply throughout Great Britain. This section sets out which of the provisions of the Act will extend to Northern Ireland. The provisions, specified in subsection (2) , are concerned with
%($a$) 
%
%an amendment of a provision about member nominated trustees which itself extends to Northern Ireland;
%($b$) 
%
%War Pensions;
%($c$) 
%
%consultation with the Social Security Advisory Committee about regulations relating to disclosure of state pension information, loss of benefits, Housing Benefit and Council Tax Benefit revisions and appeals and discretionary housing payments;
%($d$) 
%
%liability for Class 1A National Insurance Contributions in Northern Ireland, and collection of contributions there;
%($e$) 
%
%consequential amendments made in Schedule 3 in Acts which extend to Northern Ireland;
%($f$) 
%
%calculation of the contributions equivalent premium in Northern Ireland; and
%($g$) 
%
%this section, and the provision for expenses and commencement etc. in this Part of the Act, and repeals in Acts which apply in, or extend to, Northern Ireland.
%Schedules
%
%915. The Schedules are described in the main commentary after the introducing section, where explanation is required.
%Schedule	Subject	Commentary
%1	Substituted Part 1 of Schedule 1 to the Child Support Act 1991(calculation of child support maintenance)	Section 1
%2	Substituted Schedules 4A and 4B to the Child Support 1991 Act (variations)	Section 6
%3	Amendment of enactments relating to child support	Section 26
%4	Additional pension	Section 31
%5	Pensions: miscellaneous amendments and alternative to anti-franking rules	Section 56
%6	Social Security investigation powers	Section 67
%7	Housing Benefit and Council Tax Benefit: revisions and appeals	Section 68
%8	Declarations of status: consequential amendments	Section 82
%9	Repeals	Section 85
%Commencement
%
%Section 86(1)  and (2)  set out which provisions of the Act will be brought into force by commencement orders. All other provisions came into force on Royal Assent
%Hansard References
%Stage	Date	Hansard references
%House of Commons
%Introduction	1 December 1999	vol 340, col 313
%Second Reading	11 January 2000	vol. 342, col 150 – 249
%Committee	
%
%18 January to
%
%7 March 2000
%
%(25 sittings)
%	Standing Committee F
%Report and Third Reading	29 March 2000	vol 347, col 347 – 464
%3 April 2000	vol 347, col 644 – 781
%House of Lords
%Introduction	5 April 2000	vol 611, col 1305
%Second Reading	17 April 2000	vol 612, col 461 – 530
%Committee	8 May 2000	vol 612, col 1203 – 1360
%15 May 2000	vol 613, col 12 – 170
%22 May 2000	vol 613, col 488 – 630
%Report	22 June 2000	vol 614, col 429 – 560
%27 June 2000	vol 614, col 768 – 890
%Third Reading	19 July 2000	vol 615, col 1081 – 1123
%House of Commons
%Consideration of Lords amendments	24 July 2000	vol 354, col 792 – 824
%House of Lords
%Consideration of Commons amendments	26 July 2000	vol 616, col 421 - 433
%Royal Assent	28 July 2000	
%
%House of Commons
%
%vol 354 (Part I of II)
%
%col 1457
%
%House of Lords
%
%vol 616 col 766
%Annex Glossary of Terms:
%
%    Administration Act
%
%    The Social Security Administration Act 1992: the Act that contains most of the rules and regulation-making powers to specify how social security benefits should be claimed, paid and administered. It consolidated the existing legislation in 1992, and has been amended subsequently. See also the Contributions and Benefits Act.
%
%    Attendance Allowance (AA)
%
%    A non-contributory, tax free, non-means-tested benefit paid to meet the extra costs arising from the care needs of elderly and disabled people. Paid at two rates: higher rate (needing care day and night) and lower rate (needing care day or night).
%
%    Basic Retirement Pension (Basic Pension)
%
%    The flat rate state pension paid to people who have met the minimum contribution requirements. Married women, widows and some widowers can receive a pension based on their spouse’s contribution record.
%
%    Child Benefit
%
%    A non-contributory, non-means tested, non-taxable benefit payable for each child in a family from birth up to age 19, or to a fixed termination date related to the end of non-advanced secondary education.
%
%    Community Sentence
%
%    A community sentence means a sentence which consists of, or includes, one or more community orders.  Community orders may be imposed by the Courts on persons aged 16 or over who have been convicted of an offence. A community service order is an order requiring a person to perform unpaid work for not less than 40 hours and not more than 240 hours. A probation order requires a person to be supervised by the probation service for a period of between 6 months and 3 years. A combination order comprises elements of both community service and probation orders.
%
%    Contributions and Benefits Act
%
%    The Social Security Contributions and Benefits Act 1992: contains most of the provisions for setting out the rules for National Insurance contributions and entitlement to social security benefits (with the main exception of Jobseeker's Allowance). It consolidated the existing legislation when it was introduced in 1992, and has been amended since then. See also the Social Security Administration Act.
%
%    Deductions from earnings order
%
%    An instruction from the Secretary of State to a non-resident parent's employer to make deductions directly from his salary to pay his liability. Used where voluntary arrangements have broken down.  A non-resident parent may also choose to pay by this method.
%
%    Disability Living Allowance (DLA)
%
%    A non-contributory, tax free, non-means-tested benefit, introduced in April 1992, to meet the extra costs of care and mobility needs of people who became disabled before the age of 65.  There are two components: a care component (paid at higher, middle or lower rate) and a mobility component (paid at a higher or lower rate).
%
%    Financial Services Authority
%
%    The FSA is a statutory authority established by the Financial Services and Markets Act to regulate the UK financial services industry. It has powers to authorise financial service providers, to regulate their actions and impose disciplinary sanctions. Part of their role is to investigate complaints from individuals who believe they have been given wrong or bad advice by the company that sold them their personal pension.
%
%    Home Responsibilities Protection
%
%    Home Responsibilities Protection protects the basic retirement pension position of someone caring for a child under 16 or a sick or disabled person. It is not a National Insurance credit or benefit in its own right, but works by reducing the number of qualifying years needed for a full basic retirement pension.
%
%    Incapacity Benefit (IB)
%
%    A taxable contributory benefit introduced in April 1995 to replace Sickness and Invalidity Benefits for people who are unable to work because of illness or disability. Payable weekly at 1 of 3 rates:
%
%        a short-term lower rate: payable to those who do not qualify for Statutory Sick Pay, for the first 28 weeks of incapacity
%
%        a short-term higher rate: payable from 28 weeks to 52 weeks of incapacity
%
%        a long-term rate: payable after 52 weeks of incapacity
%
%    Income Support (IS)
%
%    An income-related benefit introduced in 1988, as successor to Supplementary Benefit, to support people not in remunerative work, whose net income is less than a minimum level set by Parliament, and determined by age, family membership and other circumstances.
%
%    Invalid Care Allowance (ICA)
%
%    A non-contributory, non-means-tested benefit for people who give up the opportunity of full-time work to provide care on a regular and substantial basis (at least 35 hours or more a week) to a severely disabled person.
%
%    Jobseekers Act 1995
%
%    The Jobseekers Act 1995: established Jobseeker's Allowance.
%
%    Jobseeker's Allowance (JSA)
%
%    A benefit introduced October 1996 to replace contributory Unemployment Benefit and income-related Income Support for all those over 18 needing financial support because of unemployment, administered jointly by Employment Service and Benefits Agency.
%
%    Lower Earnings Limit (LEL)
%
%    The level of earnings below which there is not a liability for employees to pay National Insurance contributions. It is also the level at which people secure entitlement to basic contributory benefits. Earnings above this point (and up to the Upper Earnings Limit) accrue entitlement to SERPS or to contracted-out rebates.
%
%    Maintenance agreement
%
%    An agreement for making, or for securing the making, of maintentance payments, or, in Scotland, aliment, to or for the benefit of any child.
%
%    National Insurance contributions (NICs)
%
%    Contributions payable by those in work and their employer into the National Insurance fund, which are used to pay contributory social security benefits to qualifying individuals. Self-employed people pay a lower rate but have more limited rights to benefits. Contributions are divided into six classes, which bring access to different benefit entitlements:
%
%    The Welfare Reform Act (Section 69 and Schedule 9) introduces a new Primary Threshold as the point from which employees start to pay NICs. This means that from April 2000 employees do not pay contributions on earnings below the new Primary Threshold. It is proposed that from April 2001, the Primary Threshold will be aligned with the Secondary Threshold and as such, the Income Tax Personal Allowance. However, where an employee has earnings between the prevailing Lower Earnings Limit, and the new Primary Threshold, they will be treated as if they had paid contributions on those earnings to protect their ability to build up entitlement to contributory benefits.
%
%    Class 1: Payable by employed earners on all earning between the Lower Earnings Limit and Upper Earnings Limit, and by employers on all earnings above the Earnings Threshold. Class 1 contributions give access to all National Insurance benefits, both at a flat rate and with earnings-related increases where relevant (provided that the individual meets the specific conditions of entitlement for each benefit).
%
%    Class 1A: Contributions paid by employers in respect of employees' car and fuel benefits. The charge is based on the cash equivalent of the car benefit and the car fuel benefit provided for private use of the employee.
%
%    Class 1B: Contributions paid on settlements (PAYE Settlement Agreements) made with the Inland Revenue by an employer. This class of contributions was introduced in April 1999. 
%
%    Class 2: Flat-rate contribution paid by self-employed earners. Benefits are payable at the basic rate only, and there is no entitlement to certain benefits (for example Jobseeker's Allowance).
%
%    Class 3: Flat-rate voluntary contributions payable for any period when people were not liable to pay class 1 or 2 contributions because they were not employed or were outside of the UK.
%
%    Class 4: Profit-related additional contributions payable by self-employed earners with profits above an annual threshold, up to an upper threshold equivalent to the Upper Earnings Limit for Class 1 contributors. These contributions do not give entitlement to any additional benefits.
%
%    National Insurance Credits
%
%    A Class 1 National Insurance credit is a credit of earnings for the sole purpose of assisting a person towards satisfying the contribution conditions for basic Retirement Pension, Widows’ Benefits, Incapacity Benefit or Jobseeker's Allowance.  A credit is available for weeks where a person is unable to work due to one of a number of specified contingencies, the most common of which is that he is incapable of work through illness or disability or he is unemployed, available for and actively seeking work.  Alternatively, a credit may accompany receipt of a particular benefit such as Invalid Care Allowance or Statutory Maternity Pay, or be awarded automatically to a man within five years of state pension age who has no liability to pay contributions.  Earnings cannot be credited beyond the extent needed to give a person a Qualifying Earnings Factor for that year.
%
%    Non-resident parent
%
%    “Non-resident parent” means a parent who is not living in the same household as the child in respect of whom an application for maintenance has been made (the “qualifying child”).
%
%    Occupational Pensions Regulatory Authority (Opra)
%
%    Opra is a statutory body created by the Pensions Act 1995, which is responsible for ensuring that occupational pension schemes comply with the requirements of the relevant legislation.  It has the power to investigate and can impose a range of penalties for non-compliance, including prohibition from acting as a trustee and imposition of fines.  It may also pursue criminal proceedings.
%
%    Occupational Pension Scheme
%
%    A scheme organised by an employer or on behalf of a group of employers to provide pensions and/or other benefits for, or in respect of, one or more employees on leaving service or on death or retirement.
%
%    Parent with care
%
%    "Parent with care" means a parent living in the same household as the qualifying child and who usually provides day-to-day care of the child. When the person caring for the child is not the child's legal parent, she is sometimes known as a "person with care" (PeWC).
%
%    Pensions Act 1995
%
%    This Act provides a framework of statutory obligations on employers, trustees, scheme professionals and others connected with pension schemes in order to provide greater security for scheme members.
%
%    Personal Pension
%
%    An arrangement between an individual who is self-employed, in non-pensionable employment or who is not a member of an employer's scheme, and a pension provider (such as an insurance company) which enables the individual to make provision for a pension on a money purchase basis. See also occupational pension scheme.
%
%    An Appropriate Personal Pension (APP) is a personal pension scheme that has been certified as suitable for contracting out of SERPS.  APPs are not available to the self-employed.
%
%    Primary Threshold
%
%    The Primary Threshold is the point at which employees begin to pay National Insurance contributions.  It is being increased over the 2 years beginning April 2000 to the level of the Income Tax Personal Allowance.  The Primary Threshold for 2000/01 is £76. 00 per week.
%
%    Qualifying child
%
%    A child who is living apart from one or both parents and for whom an application for child support has been made.
%
%    Severe Disablement Allowance (SDA)
%
%    A tax-free, non-contributory benefit for those incapable of work for at least 28 weeks who do not qualify for Incapacity Benefit. Severe Disablement Allowance is being withdrawn for new claimants from April 2001 but those already receiving the benefit will continue to do so
%
%    Social Security Act 1998
%
%    Provided for a new system for making decisions on cases and handling disputes and appeals. Implemented for child support in June 1999. 
%
%    Tax Year
%
%    The tax year is the period within which liability to pay Income Tax arises, and runs from 6 April in one year to 5 April the next. The tax year for 1999-2000 runs from 6 April 1999 to 5 April 2000. 
%
%    Training Allowance
%
%    A weekly allowance paid from public funds to people participating in certain courses of training, instruction or work experience provided by, or in pursuance of arrangements made with, the Secretary of State for Education and Employment under section 2 of the Employment and Training Act 1973.  A training allowance consists of a basic element equivalent to the participant’s benefit entitlement when unemployed plus, where appropriate, a training premium or top-up, reimbursement of travelling expenses, and living away from home allowance.
%
%    Upper Earnings Limit (UEL)
%
%    The level of weekly earnings above which there is no liability for employee National Insurance contributions. It sets the upper limit for the weekly earnings on which Additional Pension accrue and which qualify for contracted-out rebates. See also Lower Earnings Limit.
%
%    Widows’ Benefits
%
%    Widows’ Benefits are available to working age widows and entitlement is based on the National Insurance contribution record of the deceased husband.  They are to be replaced by Bereavement Benefits, which will be available equally to widowed men and women on the same basis – based on the late spouses’ National Insurance record.
%
%    Welfare Reform and Pensions Act 1999
%
%    Introduced a range of measures relating to Social Security benefits, pensions and National Insurance contributions.

\end{document}