\documentclass[12pt,a4paper]{article}

\newcommand\regstitle{Child Support, Pensions and Social Security Act 2000}

\newcommand\regsnumber{c.~19}

%\opt{newrules}{
\title{\regstitle}
%}

%\opt{2012rules}{
%\title{Child Maintenance and Other Payments Act 2008\\(2012 scheme version)}
%}

\author{2000 Chapter 19}

\date{Royal Assent
28th July 2000\\
%Laid before Parliament
%27th January 2000\\
%Coming into force
%19th June 2000
}

%\opt{oldrules}{\newcommand\versionyear{1993}}
%\opt{newrules}{\newcommand\versionyear{2003}}
%\opt{2012rules}{\newcommand\versionyear{2012}}

\usepackage{csa-regs}

\setlength\headheight{27.57402pt}

\renewcommand\siprefix{\relax}

\begin{document}

\maketitle

{\noindent\large
An Act to amend the law relating to child support; to amend the law relating to occupational and personal pensions and war pensions; to amend the law relating to social security benefits and social security administration; to amend the law relating to national insurance contributions; to amend Part III of the Family Law Reform Act 1969 and Part III of the Family Law Act 1986; and for connected purposes.

}

\bigskip

\lettrine{B}{e it enacted} by the Queen’s most Excellent Majesty, by and with the advice and consent of the Lords Spiritual and Temporal, and Commons, in this present Parliament assembled, and by the authority of the same, as follows:—  

\enlargethispage{\baselineskip}

{\sloppy

\tableofcontents

}

\bigskip

\setcounter{secnumdepth}{-2}

\part[Part I --- Child Support]{Part I\\*Child Support}

\renewcommand\parthead{--- Part I}

\section[\itshape Maintenance calculations and interim and default maintenance decisions]{\itshape\sloppy Maintenance calculations and interim and default maintenance decisions}

\subsection[1. Maintenance calculations and terminology]{1. Maintenance calculations and terminology}

(1) In the Child Support Act 1991 (“the 1991 Act”), for section 11 (maintenance assessments) there shall be substituted—
\begin{quotation}
\subsection*{“11. Maintenance calculations}

(1) An application for a maintenance calculation made to the Secretary of State shall be dealt with by him in accordance with the provision made by or under this Act.

(2) The Secretary of State shall (unless he decides not to make a maintenance calculation in response to the application, or makes a decision under section 12) determine the application by making a decision under this section about whether any child support maintenance is payable and, if so, how much.

(3) Where—
\begin{enumerate}\item[]
($a$) a parent is treated under section 6(3)  as having applied for a maintenance calculation; but

($b$) the Secretary of State becomes aware before determining the application that the parent has ceased to fall within section 6(1),
\end{enumerate}
he shall, subject to subsection (4), cease to treat that parent as having applied for a maintenance calculation.

(4) If it appears to the Secretary of State that subsection (10)  of section 4 would not have prevented the parent with care concerned from making an application for a maintenance calculation under that section he shall—
\begin{enumerate}\item[]
($a$) notify her of the effect of this subsection; and

($b$) if, before the end of the period of one month beginning with the day on which notice was sent to her, she asks him to do so, treat her as having applied not under section 6 but under section 4. 
\end{enumerate}

(5) Where subsection (3)  applies but subsection (4)  does not, the Secretary of State shall notify—
\begin{enumerate}\item[]
($a$) the parent with care concerned; and

($b$) the non-resident parent (or alleged non-resident parent), where it appears to him that that person is aware that the parent with care has been treated as having applied for a maintenance calculation.
\end{enumerate}

(6) The amount of child support maintenance to be fixed by a maintenance calculation shall be determined in accordance with Part I of Schedule 1 unless an application for a variation has been made and agreed.

(7) If the Secretary of State has agreed to a variation, the amount of child support maintenance to be fixed shall be determined on the basis he determines under section 28F(4).

(8) Part II of Schedule 1 makes further provision with respect to maintenance calculations.”
\end{quotation}

(2) In the 1991 Act—
\begin{enumerate}\item[]
($a$) for “maintenance assessment”, wherever it occurs, there shall be substituted “maintenance calculation”; and

($b$) for “assessment” (or any variant of that term), wherever it occurs, there shall be substituted “calculation” (or the corresponding variant) preceded, where appropriate, by “a” instead of “an”.
\end{enumerate}

(3) For Part I of Schedule 1 to the 1991 Act, there shall be substituted the Part I set out in Schedule 1 to this Act.

\amendment{
S. 1(1), (2) are in force only for new-rules cases; see the Child Support, Pensions and Social Security Act 2000 (Commencement No. 12) Order 2003 art. 3.
}


\subsection{2. Applications under section 4 of the Child Support Act 1991}

(1) In section 4 of the 1991 Act (child support maintenance), subsection (10)  shall be amended as follows.

(2) In paragraph ($a$), after “maintenance order” there shall be inserted “made before a prescribed date”.

(3) After paragraph ($a$), there shall be inserted—
\begin{quotation}
“($aa$) a maintenance order made on or after the date prescribed for the purposes of paragraph ($a$)  is in force in respect of them, but has been so for less than the period of one year beginning with the date on which it was made; or”.
\end{quotation}

\amendment{
S. 2 is in force only for new-rules cases; see the Child Support, Pensions and Social Security Act 2000 (Commencement No. 12) Order 2003 art. 3.
}

\subsection{3. Applications by persons claiming or receiving benefit}

For section 6 of the 1991 Act (applications by those receiving benefit) there shall be substituted—
\begin{quotation}
\subsection*{“6. Applications by those claiming or receiving benefit}

(1) This section applies where income support, an income-based jobseeker’s allowance or any other benefit of a prescribed kind is claimed by or in respect of, or paid to or in respect of, the parent of a qualifying child who is also a person with care of the child.

(2) In this section, that person is referred to as “the parent”.

(3) The Secretary of State may—
\begin{enumerate}\item[]
($a$) treat the parent as having applied for a maintenance calculation with respect to the qualifying child and all other children of the non-resident parent in relation to whom the parent is also a person with care; and

($b$) take action under this Act to recover from the non-resident parent, on the parent’s behalf, the child support maintenance so determined.
\end{enumerate}

(4) Before doing what is mentioned in subsection (3), the Secretary of State must notify the parent in writing of the effect of subsections (3)  and (5)  and section 46. 

(5) The Secretary of State may not act under subsection (3)  if the parent asks him not to (a request which need not be in writing).

(6) Subsection (1)  has effect regardless of whether any of the benefits mentioned there is payable with respect to any qualifying child.

(7) Unless she has made a request under subsection (5), the parent shall, so far as she reasonably can, comply with such regulations as may be made by the Secretary of State with a view to the Secretary of State’s being provided with the information which is required to enable—
\begin{enumerate}\item[]
($a$) the non-resident parent to be identified or traced;

($b$) the amount of child support maintenance payable by him to be calculated; and

($c$) that amount to be recovered from him.
\end{enumerate}

(8) The obligation to provide information which is imposed by subsection (7)—
\begin{enumerate}\item[]
($a$) does not apply in such circumstances as may be prescribed; and

($b$) may, in such circumstances as may be prescribed, be waived by the Secretary of State.
\end{enumerate}

(9) If the parent ceases to fall within subsection (1), she may ask the Secretary of State to cease acting under this section, but until then he may continue to do so.

(10) The Secretary of State must comply with any request under subsection (9)  (but subject to any regulations made under subsection (11)).

(11) The Secretary of State may by regulations make such incidental or transitional provision as he thinks appropriate with respect to cases in which he is asked under subsection (9)  to cease to act under this section.

(12) The fact that a maintenance calculation is in force with respect to a person with care does not prevent the making of a new maintenance calculation with respect to her as a result of the Secretary of State’s acting under subsection (3).”
\end{quotation}

\amendment{
S. 3 is in force only for certain cases; see the Child Support, Pensions and Social Security Act 2000 (Commencement No. 12) Order 2003 art. 4.
}

\subsection{4. Default and interim maintenance decisions}

For section 12 of the 1991 Act (interim maintenance assessments) there shall be substituted—
\begin{quotation}
\subsection*{“12. Default and interim maintenance decisions}

(1) Where the Secretary of State—
\begin{enumerate}\item[]
($a$) is required to make a maintenance calculation; or

($b$) is proposing to make a decision under section 16 or 17,
\end{enumerate}
and it appears to him that he does not have sufficient information to enable him to do so, he may make a default maintenance decision.

(2) Where an application for a variation has been made under section 28A(1)  in connection with an application for a maintenance calculation (or in connection with such an application which is treated as having been made), the Secretary of State may make an interim maintenance decision.

(3) The amount of child support maintenance fixed by an interim maintenance decision shall be determined in accordance with Part I of Schedule 1. 

(4) The Secretary of State may by regulations make provision as to default and interim maintenance decisions.

(5) The regulations may, in particular, make provision as to—
\begin{enumerate}\item[]
($a$) the procedure to be followed in making a default or an interim maintenance decision; and

($b$) a default rate of child support maintenance to apply where a default maintenance decision is made.”
\end{enumerate}
\end{quotation}

\amendment{
S. 4 is in force only for new-rules cases; see the Child Support, Pensions and Social Security Act 2000 (Commencement No. 12) Order 2003 art. 3.
}

\section{\itshape Applications for a variation}

\subsection{5. Departure from usual rules for calculating maintenance}

(1) The 1991 Act shall be amended as follows.

(2) For sections 28A to 28C (which deal respectively with applications for departure directions, their preliminary consideration, and the imposition of a regular payments condition) there shall be substituted—
\begin{quotation}
\section*{“Variations}

\subsection*{28A. Application for variation of usual rules for calculating maintenance}

(1) Where an application for a maintenance calculation is made under section 4 or 7, or treated as made under section 6, the person with care or the non-resident parent or (in the case of an application under section 7) either of them or the child concerned may apply to the Secretary of State for the rules by which the calculation is made to be varied in accordance with this Act.

(2) Such an application is referred to in this Act as an “application for a variation”.

(3) An application for a variation may be made at any time before the Secretary of State has reached a decision (under section 11 or 12(1)) on the application for a maintenance calculation (or the application treated as having been made under section 6).

(4) A person who applies for a variation—
\begin{enumerate}\item[]
($a$) need not make the application in writing unless the Secretary of State directs in any case that he must; and

($b$) must say upon what grounds the application is made.
\end{enumerate}

(5) In other respects an application for a variation is to be made in such manner as may be prescribed.

(6) Schedule 4A has effect in relation to applications for a variation.

\subsection*{28B. Preliminary consideration of applications}

(1) Where an application for a variation has been duly made to the Secretary of State, he may give it a preliminary consideration.

(2) Where he does so he may, on completing the preliminary consideration, reject the application (and proceed to make his decision on the application for a maintenance calculation without any variation) if it appears to him—
\begin{enumerate}\item[]
($a$) that there are no grounds on which he could agree to a variation;

($b$) that he has insufficient information to make a decision on the application for the maintenance calculation under section 11 (apart from any information needed in relation to the application for a variation), and therefore that his decision would be made under section 12(1); or

($c$) that other prescribed circumstances apply.
\end{enumerate}

\subsection*{28C. Imposition of regular payments condition}

(1) Where—
\begin{enumerate}\item[]
($a$) an application for a variation is made by the non-resident parent; and

($b$) the Secretary of State makes an interim maintenance decision,
\end{enumerate}
the Secretary of State may also, if he has completed his preliminary consideration (under section 28B) of the application for a variation and has not rejected it under that section, impose on the non-resident parent one of the conditions mentioned in subsection (2)  (a “regular payments condition”).

(2) The conditions are that—
\begin{enumerate}\item[]
($a$) the non-resident parent must make the payments of child support maintenance specified in the interim maintenance decision;

($b$) the non-resident parent must make such lesser payments of child support maintenance as may be determined in accordance with regulations made by the Secretary of State.
\end{enumerate}

(3) Where the Secretary of State imposes a regular payments condition, he shall give written notice of the imposition of the condition and of the effect of failure to comply with it to—
\begin{enumerate}\item[]
($a$) the non-resident parent;

($b$) all the persons with care concerned; and

($c$) if the application for the maintenance calculation was made under section 7, the child who made the application.
\end{enumerate}

(4) A regular payments condition shall cease to have effect—
\begin{enumerate}\item[]
($a$) when the Secretary of State has made a decision on the application for a maintenance calculation under section 11 (whether he agrees to a variation or not);

($b$) on the withdrawal of the application for a variation.
\end{enumerate}

(5) Where a non-resident parent has failed to comply with a regular payments condition, the Secretary of State may in prescribed circumstances refuse to consider the application for a variation, and instead reach his decision under section 11 as if no such application had been made.

(6) The question whether a non-resident parent has failed to comply with a regular payments condition is to be determined by the Secretary of State.

(7) Where the Secretary of State determines that a non-resident parent has failed to comply with a regular payments condition he shall give written notice of his determination to—
\begin{enumerate}\item[]
($a$) that parent;

($b$) all the persons with care concerned; and

($c$) if the application for the maintenance calculation was made under section 7, the child who made the application.”
\end{enumerate}
\end{quotation}

(3) In section 28D (determination of applications)—
\begin{enumerate}\item[]
($a$) for subsection (1)  there shall be substituted—
\begin{quotation}
“(1) Where an application for a variation has not failed, the Secretary of State shall, in accordance with the relevant provisions of, or made under, this Act—
\begin{enumerate}\item[]
($a$) either agree or not to a variation, and make a decision under section 11 or 12(1); or

($b$) refer the application to an appeal tribunal for the tribunal to determine what variation, if any, is to be made.”;
\end{enumerate}
\end{quotation}

($b$) in each of subsections (2)  and (3), for “an application for a departure direction” there shall be substituted “an application for a variation”; and

($c$) in subsection (2), in paragraph ($a$)  “lapsed or” shall be omitted, at the end of paragraph ($b$)  “or” shall be inserted, and after that paragraph there shall be inserted—
\begin{quotation}
“($c$) the Secretary of State has refused to consider it under section 28C(5).”
\end{quotation}
\end{enumerate}

(4) In section 28E (matters to be taken into account)—
\begin{enumerate}\item[]
($a$) in subsections (1), (3)  and (4), for “any application for a departure direction” (wherever appearing) there shall be substituted “whether to agree to a variation”; and

($b$) in subsection (4)($a$), for “a departure direction were made” there shall be substituted “the Secretary of State agreed to a variation”.
\end{enumerate}

(5) For section 28F (departure directions) there shall be substituted—
\begin{quotation}
\subsection*{“28F. Agreement to a variation}

(1) The Secretary of State may agree to a variation if—
\begin{enumerate}\item[]
($a$) he is satisfied that the case is one which falls within one or more of the cases set out in Part I of Schedule 4B or in regulations made under that Part; and

($b$) it is his opinion that, in all the circumstances of the case, it would be just and equitable to agree to a variation.
\end{enumerate}

(2) In considering whether it would be just and equitable in any case to agree to a variation, the Secretary of State—
\begin{enumerate}\item[]
($a$) must have regard, in particular, to the welfare of any child likely to be affected if he did agree to a variation; and

($b$) must, or as the case may be must not, take any prescribed factors into account, or must take them into account (or not) in prescribed circumstances.
\end{enumerate}

(3) The Secretary of State shall not agree to a variation (and shall proceed to make his decision on the application for a maintenance calculation without any variation) if he is satisfied that—
\begin{enumerate}\item[]
($a$) he has insufficient information to make a decision on the application for the maintenance calculation under section 11, and therefore that his decision would be made under section 12(1); or

($b$) other prescribed circumstances apply.
\end{enumerate}

(4) Where the Secretary of State agrees to a variation, he shall—
\begin{enumerate}\item[]
($a$) determine the basis on which the amount of child support maintenance is to be calculated in response to the application for a maintenance calculation (including an application treated as having been made); and

($b$) make a decision under section 11 on that basis.
\end{enumerate}

(5) If the Secretary of State has made an interim maintenance decision, it is to be treated as having been replaced by his decision under section 11, and except in prescribed circumstances any appeal connected with it (under section 20) shall lapse.

(6) In determining whether or not to agree to a variation, the Secretary of State shall comply with regulations made under Part II of Schedule 4B.”
\end{quotation}

\amendment{
S. 5 is in force only for new-rules cases; see the Child Support, Pensions and Social Security Act 2000 (Commencement No. 12) Order 2003 art. 3.
}

\subsection{6. Applications for a variation: further provisions}

(1) For Schedule 4A to the 1991 Act there shall be substituted the Schedule 4A set out in Part I of Schedule 2. 

(2) For Schedule 4B to that Act there shall be substituted the Schedule 4B set out in Part II of Schedule 2. 

\amendment{
S. 6 is in force only for the purpose of making regulations and Acts of Sederunt.
}

\subsection{7. Variations: revision and supersession}

For section 28G of the 1991 Act (effect and duration of departure directions) there shall be substituted—
\begin{quotation}
\subsection*{“28G. Variations: revision and supersession}

(1) An application for a variation may also be made when a maintenance calculation is in force.

(2) The Secretary of State may by regulations provide for—
\begin{enumerate}\item[]
($a$) sections 16, 17 and 20; and

($b$) sections 28A to 28F and Schedules 4A and 4B,
\end{enumerate}
to apply with prescribed modifications in relation to such applications.

(3) The Secretary of State may by regulations provide that, in prescribed cases (or except in prescribed cases), a decision under section 17 made otherwise than pursuant to an application for a variation may be made on the basis of a variation agreed to for the purposes of an earlier decision without a new application for a variation having to be made.”
\end{quotation}

\amendment{
S. 7 is in force only for new-rules cases; see the Child Support, Pensions and Social Security Act 2000 (Commencement No. 12) Order 2003 art. 3.
}

\section{\itshape Revision and supersession of decisions}

\subsection{8. Revision of decisions}

(1) Section 16 of the 1991 Act (revision of decisions) shall be amended as follows.

(2) In subsection (1), for “of the Secretary of State under section 11, 12 or 17” there shall be substituted “to which subsection (1A)  applies”.

(3) After subsection (1), there shall be inserted—
\begin{quotation}
“(1A) This subsection applies to—
\begin{enumerate}\item[]
($a$) a decision of the Secretary of State under section 11, 12 or 17;

($b$) a reduced benefit decision under section 46;

($c$) a decision of an appeal tribunal on a referral under section 28D(1)($b$).
\end{enumerate}

(1B) Where the Secretary of State revises a decision under section 12(1)—
\begin{enumerate}\item[]
($a$) he may (if appropriate) do so as if he were revising a decision under section 11; and

($b$) if he does that, his decision as revised is to be treated as one under section 11 instead of section 12(1)  (and, in particular, is to be so treated for the purposes of an appeal against it under section 20).”
\end{enumerate}
\end{quotation}

\amendment{
S. 8 is in force only for new-rules cases; see the Child Support, Pensions and Social Security Act 2000 (Commencement No. 12) Order 2003 art. 3.
}

\subsection{9. Decisions superseding earlier decisions}

(1) Section 17 of the 1991 Act (decisions superseding earlier decisions) shall be amended as follows.

(2) In subsection (1), for paragraph ($c$)  there shall be substituted—
\begin{quotation}
“($c$) any reduced benefit decision under section 46;

($d$) any decision of an appeal tribunal on a referral under section 28D(1)($b$);

($e$) any decision of a Child Support Commissioner on an appeal from such a decision as is mentioned in paragraph ($b$)  or ($d$).”
\end{quotation}

(3) For subsection (4)  there shall be substituted—
\begin{quotation}
“(4) Subject to subsection (5)  and section 28ZC, a decision under this section shall take effect as from the beginning of the maintenance period in which it is made or, where applicable, the beginning of the maintenance period in which the application was made.

(4A) In subsection (4), a “maintenance period” is (except where a different meaning is prescribed for prescribed cases) a period of seven days, the first one beginning on the effective date of the first decision made by the Secretary of State under section 11 or (if earlier) his first default or interim maintenance decision (under section 12) in relation to the non-resident parent in question, and each subsequent one beginning on the day after the last day of the previous one.”
\end{quotation}

\amendment{
S. 9 is in force only for new-rules cases; see the Child Support, Pensions and Social Security Act 2000 (Commencement No. 12) Order 2003 art. 3.
}

\section{\itshape Appeals}

\subsection{10. Appeals to appeal tribunals}

For section 20 of the 1991 Act (appeals to appeal tribunals) there shall be substituted—
\begin{quotation}
\subsection*{“20. Appeals to appeal tribunals}

(1) A qualifying person has a right of appeal to an appeal tribunal against—
\begin{enumerate}\item[]
($a$) a decision of the Secretary of State under section 11, 12 or 17 (whether as originally made or as revised under section 16);

($b$) a decision of the Secretary of State not to make a maintenance calculation under section 11 or not to supersede a decision under section 17;

($c$) a reduced benefit decision under section 46;

($d$) the imposition (by virtue of section 41A) of a requirement to make penalty payments, or their amount;

($e$) the imposition (by virtue of section 47) of a requirement to pay fees.
\end{enumerate}

(2) In subsection (1), “qualifying person” means—
\begin{enumerate}\item[]
($a$) in relation to paragraphs ($a$)  and ($b$)—
\begin{enumerate}\item[]
(i) the person with care, or non-resident parent, with respect to whom the Secretary of State made the decision, or

(ii) in a case relating to a maintenance calculation which was applied for under section 7, either of those persons or the child concerned;
\end{enumerate}

($b$) in relation to paragraph ($c$), the person in respect of whom the benefits are payable;

($c$) in relation to paragraph ($d$), the parent who has been required to make penalty payments; and

($d$) in relation to paragraph ($e$), the person required to pay fees.
\end{enumerate}

(3) A person with a right of appeal under this section shall be given such notice as may be prescribed of—
\begin{enumerate}\item[]
($a$) that right; and

($b$) the relevant decision, or the imposition of the requirement.
\end{enumerate}

(4) Regulations may make—
\begin{enumerate}\item[]
($a$) provision as to the manner in which, and the time within which, appeals are to be brought; and

($b$) such provision with respect to proceedings before appeal tribunals as the Secretary of State considers appropriate.
\end{enumerate}

(5) The regulations may in particular make any provision of a kind mentioned in Schedule 5 to the Social Security Act 1998. 

(6) No appeal lies by virtue of subsection (1)($c$)  unless the amount of the person’s benefit is reduced in accordance with the reduced benefit decision; and the time within which such an appeal may be brought runs from the date of notification of the reduction.

(7) In deciding an appeal under this section, an appeal tribunal—
\begin{enumerate}\item[]
($a$) need not consider any issue that is not raised by the appeal; and

($b$) shall not take into account any circumstances not obtaining at the time when the Secretary of State made the decision or imposed the requirement.
\end{enumerate}

(8) If an appeal under this section is allowed, the appeal tribunal may—
\begin{enumerate}\item[]
($a$) itself make such decision as it considers appropriate; or

($b$) remit the case to the Secretary of State, together with such directions (if any) as it considers appropriate.”
\end{enumerate}
\end{quotation}

\amendment{
S. 10 is in force only for new-rules cases; see the Child Support, Pensions and Social Security Act 2000 (Commencement No. 12) Order 2003 art. 3.
}

\subsection{11. Redetermination of appeals}

After section 23 of the 1991 Act there shall be inserted—
\begin{quotation}
\subsection*{“23A. Redetermination of appeals}

(1) This section applies where an application is made to a person under section 24(6)($a$)  for leave to appeal from a decision of an appeal tribunal.

(2) If the person who constituted, or was the chairman of, the appeal tribunal considers that the decision was erroneous in law, he may set aside the decision and refer the case either for redetermination by the tribunal or for determination by a differently constituted tribunal.

(3) If each of the principal parties to the case expresses the view that the decision was erroneous in point of law, the person shall set aside the decision and refer the case for determination by a differently constituted tribunal.

(4) The “principal parties” are—
\begin{enumerate}\item[]
($a$) the Secretary of State; and

($b$) those who are qualifying persons for the purposes of section 20(2)  in relation to the decision in question.”
\end{enumerate}
\end{quotation}

\section{\itshape Information}

\subsection{12. Information required by Secretary of State}

In section 14 of the 1991 Act (information required by the Secretary of State), in subsection (1), after “such an application” there shall be inserted “(or application treated as made), or needed for the making of any decision or in connection with the imposition of any condition or requirement under this Act,”.

\amendment{
S. 12 is in force only for new-rules cases; see the Child Support, Pensions and Social Security Act 2000 (Commencement No. 12) Order 2003 art. 3.
}

\subsection{13. Information—offences}

After section 14 of the 1991 Act there shall be inserted—
\begin{quotation}
\subsection*{“14A. Information—offences}

(1) This section applies to—
\begin{enumerate}\item[]
($a$) persons who are required to comply with regulations under section 4(4)  or 7(5); and

($b$) persons specified in regulations under section 14(1)($a$).
\end{enumerate}

(2) Such a person is guilty of an offence if, pursuant to a request for information under or by virtue of those regulations—
\begin{enumerate}\item[]
($a$) he makes a statement or representation which he knows to be false; or

($b$) he provides, or knowingly causes or knowingly allows to be provided, a document or other information which he knows to be false in a material particular.
\end{enumerate}

(3) Such a person is guilty of an offence if, following such a request, he fails to comply with it.

(4) It is a defence for a person charged with an offence under subsection (3)  to prove that he had a reasonable excuse for failing to comply.

(5) A person guilty of an offence under this section is liable on summary conviction to a fine not exceeding level 3 on the standard scale.”
\end{quotation}

\subsection{14. Inspectors}

(1) Section 15 of the 1991 Act (powers of inspectors) shall be amended as follows.

(2) For subsections (1)  to (4)  there shall be substituted—
\begin{quotation}
“(1) The Secretary of State may appoint, on such terms as he thinks fit, persons to act as inspectors under this section.

(2) The function of inspectors is to acquire information which the Secretary of State needs for any of the purposes of this Act.

(3) Every inspector is to be given a certificate of his appointment.

(4) An inspector has power, at any reasonable time and either alone or accompanied by such other persons as he thinks fit, to enter any premises which—
\begin{enumerate}\item[]
($a$) are liable to inspection under this section; and

($b$) are premises to which it is reasonable for him to require entry in order that he may exercise his functions under this section,
\end{enumerate}
and may there make such examination and inquiry as he considers appropriate.

(4A) Premises liable to inspection under this section are those which are not used wholly as a dwelling house and which the inspector has reasonable grounds for suspecting are—
\begin{enumerate}\item[]
($a$) premises at which a non-resident parent is or has been employed;

($b$) premises at which a non-resident parent carries out, or has carried out, a trade, profession, vocation or business;

($c$) premises at which there is information held by a person (“A”) whom the inspector has reasonable grounds for suspecting has information about a non-resident parent acquired in the course of A’s own trade, profession, vocation or business.”
\end{enumerate}
\end{quotation}

(3) In subsection (6), for the words from “any person who” to the end of paragraph ($d$)  there shall be substituted “any such person”.

(4) After subsection (10)  there shall be inserted—
\begin{quotation}
“(11) In this section, “premises” includes—
\begin{enumerate}\item[]
($a$) moveable structures and vehicles, vessels, aircraft and hovercraft;

($b$) installations that are offshore installations for the purposes of the Mineral Workings (Offshore Installations) Act 1971; and

($c$) places of all other descriptions whether or not occupied as land or otherwise,
\end{enumerate}
and references in this section to the occupier of premises are to be construed, in relation to premises that are not occupied as land, as references to any person for the time being present at the place in question.”
\end{quotation}

\section{\itshape Parentage}

\subsection{15. Presumption of parentage in child support cases}

(1) In section 26(2)  of the 1991 Act (cases in which the Secretary of State may assume a person to be the parent of a child for the purpose of making a maintenance calculation under that Act), before Case A there shall be inserted—
\begin{quotation}
\subsubsection*{“Case A1}

Where—
\begin{enumerate}\item[]
($a$) the child is habitually resident in England and Wales;

($b$) the Secretary of State is satisfied that the alleged parent was married to the child’s mother at some time in the period beginning with the conception and ending with the birth of the child; and

($c$) the child has not been adopted.
\end{enumerate}

\subsubsection*{Case A2}

Where—
\begin{enumerate}\item[]
($a$) the child is habitually resident in England and Wales;

($b$) the alleged parent has been registered as father of the child under section 10 or 10A of the Births and Deaths Registration Act 1953, or in any register kept under section 13 (register of births and still-births) or section 44 (Register of Corrections Etc) of the Registration of Births, Deaths and Marriages (Scotland) Act 1965, or under Article 14 or 18(1)($b$)(ii)  of the Births and Deaths Registration (Northern Ireland) Order 1976; and

($c$) the child has not subsequently been adopted.
\end{enumerate}

\subsubsection*{Case A3}

Where the result of a scientific test (within the meaning of section 27A) taken by the alleged parent would be relevant to determining the child’s parentage, and the alleged parent—
\begin{enumerate}\item[]
($a$) refuses to take such a test; or

($b$) has submitted to such a test, and it shows that there is no reasonable doubt that the alleged parent is a parent of the child.”
\end{enumerate}
\end{quotation}

(2) In that provision, after Case B there shall be inserted—
\begin{quotation}
\subsubsection*{“Case B1}

Where the Secretary of State is satisfied that the alleged parent is a parent of the child in question by virtue of section 27 or 28 of that Act (meaning of “mother” and of “father” respectively).”
\end{quotation}

\section{\itshape Disqualification from driving}

\subsection{16. Disqualification from driving}

(1) After section 39 of the 1991 Act there shall be inserted—
\begin{quotation}
\subsection*{\sloppy “39A. Commitment to prison and disqualification from driving}

(1) Where the Secretary of State has sought—
\begin{enumerate}\item[]
($a$) in England and Wales to levy an amount by distress under this Act; or

($b$) to recover an amount by virtue of section 36 or 38,
\end{enumerate}
and that amount, or any portion of it, remains unpaid he may apply to the court under this section.

(2) An application under this section is for whichever the court considers appropriate in all the circumstances of—
\begin{enumerate}\item[]
($a$) the issue of a warrant committing the liable person to prison; or

($b$) an order for him to be disqualified from holding or obtaining a driving licence.
\end{enumerate}

(3) On any such application the court shall (in the presence of the liable person) inquire as to—
\begin{enumerate}\item[]
($a$) whether he needs a driving licence to earn his living;

($b$) his means; and

($c$) whether there has been wilful refusal or culpable neglect on his part.
\end{enumerate}

(4) The Secretary of State may make representations to the court as to whether he thinks it more appropriate to commit the liable person to prison or to disqualify him from holding or obtaining a driving licence; and the liable person may reply to those representations.

(5) In this section and section 40B, “driving licence” means a licence to drive a motor vehicle granted under Part III of the Road Traffic Act 1988. 

(6) In this section “the court” means—
\begin{enumerate}\item[]
($a$) in England and Wales, a magistrates' court;

($b$) in Scotland, the sheriff.”
\end{enumerate}
\end{quotation}

(2) In section 40 of the 1991 Act (commitment to prison), subsections (1)  and (2)  shall be omitted.

(3) Before section 41 of the 1991 Act there shall be inserted—
\begin{quotation}
\subsection*{“40B. Disqualification from driving: further provision}
 
(1) If, but only if, the court is of the opinion that there has been wilful refusal or culpable neglect on the part of the liable person, it may—
\begin{enumerate}\item[]
($a$) order him to be disqualified, for such period specified in the order but not exceeding two years as it thinks fit, from holding or obtaining a driving licence (a “disqualification order”); or

($b$) make a disqualification order but suspend its operation until such time and on such conditions (if any) as it thinks just.
\end{enumerate}

(2) The court may not take action under both section 40 and this section.

(3) A disqualification order must state the amount in respect of which it is made, which is to be the aggregate of—
\begin{enumerate}\item[]
($a$) the amount mentioned in section 35(1), or so much of it as remains outstanding; and

($b$) an amount (determined in accordance with regulations made by the Secretary of State) in respect of the costs of the application under section 39A.
\end{enumerate}

(4) A court which makes a disqualification order shall require the person to whom it relates to produce any driving licence held by him, and its counterpart (within the meaning of section 108(1)  of the Road Traffic Act 1988).

(5) On an application by the Secretary of State or the liable person, the court—
\begin{enumerate}\item[]
($a$) may make an order substituting a shorter period of disqualification, or make an order revoking the disqualification order, if part of the amount referred to in subsection (3)  (the “amount due”) is paid to any person authorised to receive it; and

($b$) must make an order revoking the disqualification order if all of the amount due is so paid.
\end{enumerate}

(6) The Secretary of State may make representations to the court as to the amount which should be paid before it would be appropriate to make an order revoking the disqualification order under subsection (5)($a$), and the person liable may reply to those representations.

(7) The Secretary of State may make a further application under section 39A if the amount due has not been paid in full when the period of disqualification specified in the disqualification order expires.

(8) Where a court—
\begin{enumerate}\item[]
($a$) makes a disqualification order;

($b$) makes an order under subsection (5); or

($c$) allows an appeal against a disqualification order,
\end{enumerate}
it shall send notice of that fact to the Secretary of State; and the notice shall contain such particulars and be sent in such manner and to such address as the Secretary of State may determine.

(9) Where a court makes a disqualification order, it shall also send the driving licence and its counterpart, on their being produced to the court, to the Secretary of State at such address as he may determine.

(10) Section 80 of the Magistrates' Courts Act 1980 (application of money found on defaulter) shall apply in relation to a disqualification order under this section in relation to a liable person as it applies in relation to the enforcement of a sum mentioned in subsection (1)  of that section.

(11) The Secretary of State may by regulations make provision in relation to disqualification orders corresponding to the provision he may make under section 40(11).

(12) In the application to Scotland of this section—
\begin{enumerate}\item[]
($a$) in subsection (2)  for “section 40” substitute “section 40A”;

($b$) in subsection (3)  for paragraph ($a$)  substitute—
\begin{quotation}
“($a$) the appropriate amount under section 38;”;
\end{quotation}

($c$) subsection (10)  is omitted; and

($d$) for subsection (11)  substitute—
\begin{quotation}
“(11) The power of the Court of Session by Act of Sederunt to regulate the procedure and practice in civil proceedings in the sheriff court shall include power to make, in relation to disqualification orders, provision corresponding to that which may be made by virtue of section 40A(8).””
\end{quotation}
\end{enumerate}
\end{quotation}

(4) In section 164(5)  of the Road Traffic Act 1988 (power of constables to require production of driving licence etc.), after “Road Traffic Offenders Act 1988” there shall be inserted “, section 40B of the Child Support Act 1991”.

(5) In section 27(3)  of the Road Traffic Offenders Act 1988 (offence of failing to produce a licence), for the word “then,” there shall be substituted “, or if the holder of the licence does not produce it and its counterpart as required by section 40B of the Child Support Act 1991, then,”.

\subsection{17. Civil imprisonment: Scotland}

(1) In section 40 of the 1991 Act (commitment to prison), for subsections (12)  to (14)  there shall be substituted—
\begin{quotation}
“(12) This section does not apply to Scotland.”
\end{quotation}

(2) After section 40 there shall be inserted—
\begin{quotation}
\subsection*{“40A. Commitment to prison: Scotland}

(1) If, but only if, the sheriff is satisfied that there has been wilful refusal or culpable neglect on the part of the liable person he may—
\begin{enumerate}\item[]
($a$) issue a warrant for his committal to prison; or

($b$) fix a term of imprisonment and postpone the issue of the warrant until such time and on such conditions (if any) as he thinks just.
\end{enumerate}

(2) A warrant under this section—
\begin{enumerate}\item[]
($a$) shall be made in respect of an amount equal to the aggregate of—
\begin{enumerate}\item[]
(i) the appropriate amount under section 38; and

(ii) an amount (determined in accordance with regulations made by the Secretary of State) in respect of the expenses of commitment; and
\end{enumerate}

($b$) shall state that amount.
\end{enumerate}

(3) No warrant may be issued under this section against a person who is under the age of 18. 

(4) A warrant issued under this section shall order the liable person—
\begin{enumerate}\item[]
($a$) to be imprisoned for a specified period; but

($b$) to be released (unless he is in custody for some other reason) on payment of the amount stated in the warrant.
\end{enumerate}

(5) The maximum period of imprisonment which may be imposed by virtue of subsection (4)  is six weeks.

(6) The Secretary of State may by regulations make provision for the period of imprisonment specified in any warrant issued under this section to be reduced where there is part payment of the amount in respect of which the warrant was issued.

(7) A warrant issued under this section may be directed to such person as the sheriff thinks fit.

(8) The power of the Court of Session by Act of Sederunt to regulate the procedure and practice in civil proceedings in the sheriff court shall include power to make provision—
\begin{enumerate}\item[]
($a$) as to the form of any warrant issued under this section;

($b$) allowing an application under this section to be renewed where no warrant is issued or term of imprisonment is fixed;

($c$) that a statement in writing to the effect that wages of any amount have been paid to the liable person during any period, purporting to be signed by or on behalf of his employer, shall be sufficient evidence of the facts stated;

($d$) that, for the purposes of enabling an inquiry to be made as to the liable person’s conduct and means, the sheriff may issue a citation to him to appear before the sheriff and (if he does not obey) may issue a warrant for his arrest;

($e$) that for the purpose of enabling such an inquiry, the sheriff may issue a warrant for the liable person’s arrest without issuing a citation;

($f$) as to the execution of a warrant of arrest.”
\end{enumerate}
\end{quotation}

\section{\itshape Financial penalties}

\subsection{18. Financial penalties}

(1) In section 41 of the 1991 Act (arrears of child support maintenance), subsections (3)  to (5)  (which provide for the payment of interest on arrears) shall cease to have effect.

(2) For section 41A of the 1991 Act (arrears: alternative to interest payments) there shall be substituted—
\begin{quotation}
\subsection*{“41A. Penalty payments}

(1) The Secretary of State may by regulations make provision for the payment to him by non-resident parents who are in arrears with payments of child support maintenance of penalty payments determined in accordance with the regulations.

(2) The amount of a penalty payment in respect of any week may not exceed 25\% of the amount of child support maintenance payable for that week, but otherwise is to be determined by the Secretary of State.

(3) The liability of a non-resident parent to make a penalty payment does not affect his liability to pay the arrears of child support maintenance concerned.

(4) Regulations under subsection (1)  may, in particular, make provision—
\begin{enumerate}\item[]
($a$) as to the time at which a penalty payment is to be payable;

($b$) for the Secretary of State to waive a penalty payment, or part of it.
\end{enumerate}

(5) The provisions of this Act with respect to—
\begin{enumerate}\item[]
($a$) the collection of child support maintenance;

($b$) the enforcement of an obligation to pay child support maintenance,
\end{enumerate}
apply equally (with any necessary modifications) to penalty payments payable by virtue of regulations under this section.

(6) The Secretary of State shall pay penalty payments received by him into the Consolidated Fund.”
\end{quotation}

\amendment{
S. 18 is in force only for new-rules cases; see the Child Support, Pensions and Social Security Act 2000 (Commencement No. 12) Order 2003 art. 3.
}

\subsection{19. Reduced benefit decisions}

For section 46 of the 1991 Act (failure to comply with obligations imposed by section 6) there shall be substituted—
\begin{quotation}
\subsection*{“46. Reduced benefit decisions}

(1) This section applies where any person (“the parent”)—
\begin{enumerate}\item[]
($a$) has made a request under section 6(5);

($b$) fails to comply with any regulation made under section 6(7); or

($c$) having been treated as having applied for a maintenance calculation under section 6, refuses to take a scientific test (within the meaning of section 27A).
\end{enumerate}

(2) The Secretary of State may serve written notice on the parent requiring her, before the end of a specified period—
\begin{enumerate}\item[]
($a$) in a subsection (1)($a$)  case, to give him her reasons for making the request;

($b$) in a subsection (1)($b$)  case, to give him her reasons for failing to do so; or

($c$) in a subsection (1)($c$)  case, to give him her reasons for her refusal.
\end{enumerate}

(3) When the specified period has expired, the Secretary of State shall consider whether, having regard to any reasons given by the parent, there are reasonable grounds for believing that—
\begin{enumerate}\item[]
($a$) in a subsection (1)($a$)  case, if the Secretary of State were to do what is mentioned in section 6(3);

($b$) in a subsection (1)($b$)  case, if she were to be required to comply; or

($c$) in a subsection (1)($c$)  case, if she took the scientific test,
\end{enumerate}
there would be a risk of her, or of any children living with her, suffering harm or undue distress as a result of his taking such action, or her complying or taking the test.

(4) If the Secretary of State considers that there are such reasonable grounds, he shall—
\begin{enumerate}\item[]
($a$) take no further action under this section in relation to the request, the failure or the refusal in question; and

($b$) notify the parent, in writing, accordingly.
\end{enumerate}

(5) If the Secretary of State considers that there are no such reasonable grounds, he may, except in prescribed circumstances, make a reduced benefit decision with respect to the parent.

(6) In a subsection (1)($a$)  case, the Secretary of State may from time to time serve written notice on the parent requiring her, before the end of a specified period—
\begin{enumerate}\item[]
($a$) to state whether her request under section 6(5)  still stands; and

($b$) if so, to give him her reasons for maintaining her request,
\end{enumerate}
and subsections (3)  to (5)  have effect in relation to such a notice and any response to it as they have effect in relation to a notice under subsection (2)($a$)  and any response to it.

(7) Where the Secretary of State makes a reduced benefit decision he must send a copy of it to the parent.

(8) A reduced benefit decision is to take effect on such date as may be specified in the decision.

(9) Reasons given in response to a notice under subsection (2)  or (6)  need not be given in writing unless the Secretary of State directs in any case that they must.

(10) In this section—
\begin{enumerate}\item[]
($a$) “comply” means to comply with the requirement or with the regulation in question; and “complied” and “complying” are to be construed accordingly;

($b$) “reduced benefit decision” means a decision that the amount payable by way of any relevant benefit to, or in respect of, the parent concerned be reduced by such amount, and for such period, as may be prescribed;

($c$) “relevant benefit” means income support or an income-based jobseeker’s allowance or any other benefit of a kind prescribed for the purposes of section 6; and

($d$) “specified”, in relation to a notice served under this section, means specified in the notice; and the period to be specified is to be determined in accordance with regulations made by the Secretary of State.”
\end{enumerate}
\end{quotation}

\amendment{
S. 19 is in force only for certain cases; see the Child Support, Pensions and Social Security Act 2000 (Commencement No. 12) Order 2003 art. 4.
}

\section{\itshape Miscellaneous}

\subsection{20. Voluntary payments}

(1) After section 28I of the 1991 Act there shall be inserted—
\begin{quotation}
\section*{\itshape “Voluntary payments}

\subsection*{28J. Voluntary payments}

(1) This section applies where—
\begin{enumerate}\item[]
($a$) a person has applied for a maintenance calculation under section 4(1)  or 7(1), or is treated as having applied for one by virtue of section 6;

($b$) the Secretary of State has neither made a decision under section 11 or 12 on the application, nor decided not to make a maintenance calculation; and

($c$) the non-resident parent makes a voluntary payment.
\end{enumerate}

(2) A “voluntary payment” is a payment—
\begin{enumerate}\item[]
($a$) on account of child support maintenance which the non-resident parent expects to become liable to pay following the determination of the application (whether or not the amount of the payment is based on any estimate of his potential liability which the Secretary of State has agreed to give); and

($b$) made before the maintenance calculation has been notified to the non-resident parent or (as the case may be) before the Secretary of State has notified the non-resident parent that he has decided not to make a maintenance calculation.
\end{enumerate}

(3) In such circumstances and to such extent as may be prescribed—
\begin{enumerate}\item[]
($a$) the voluntary payment may be set off against arrears of child support maintenance which accrued by virtue of the maintenance calculation taking effect on a date earlier than that on which it was notified to the non-resident parent;

($b$) the amount payable under a maintenance calculation may be adjusted to take account of the voluntary payment.
\end{enumerate}

(4) A voluntary payment shall be made to the Secretary of State unless he agrees, on such conditions as he may specify, that it may be made to the person with care, or to or through another person.

(5) The Secretary of State may by regulations make provision as to voluntary payments, and the regulations may in particular—
\begin{enumerate}\item[]
($a$) prescribe what payments or descriptions of payment are, or are not, to count as “voluntary payments”;

($b$) prescribe the extent to which and circumstances in which a payment, or a payment of a prescribed description, counts.”
\end{enumerate}
\end{quotation}

(2) Section 41B of the 1991 Act (repayment of overpaid child support maintenance) shall be amended as follows.

(3) After subsection (1)  there shall be inserted—
\begin{quotation}
“(1A) This section also applies where the non-resident parent has made a voluntary payment and it appears to the Secretary of State—
\begin{enumerate}\item[]
($a$) that he is not liable to pay child support maintenance; or

($b$) that he is liable, but some or all of the payment amounts to an overpayment,
\end{enumerate}
and, in a case falling within paragraph ($b$), it also appears to him that subsection (1)($a$)  or ($b$)  applies.”
\end{quotation}

(4) For subsection (7)  there shall be substituted—
\begin{quotation}
“(7) For the purposes of this section—
\begin{enumerate}\item[]
($a$) a payment made by a person under a maintenance calculation which was not validly made; and

($b$) a voluntary payment made in the circumstances set out in subsection (1A)($a$),
\end{enumerate}
shall be treated as an overpayment of child support maintenance made by a non-resident parent.”
\end{quotation}

\amendment{
S. 20 is in force only for new-rules cases; see the Child Support, Pensions and Social Security Act 2000 (Commencement No. 12) Order 2003 art. 5.
}

\subsection{21. Recovery of child support maintenance by deduction from benefit}

For section 43 of the 1991 Act (contribution to maintenance by deduction from benefit) there shall be substituted—
\begin{quotation}
\subsection*{“43. Recovery of child support maintenance by deduction from benefit}

(1) This section applies where—
\begin{enumerate}\item[]
($a$) a non-resident parent is liable to pay a flat rate of child support maintenance (or would be so liable but for a variation having been agreed to), and that rate applies (or would have applied) because he falls within paragraph 4(1)($b$)  or ($c$)  or 4(2)  of Schedule 1; and

($b$) such conditions as may be prescribed for the purposes of this section are satisfied.
\end{enumerate}

(2) The power of the Secretary of State to make regulations under section 5 of the Social Security Administration Act 1992 by virtue of subsection (1)($p$)  (deductions from benefits) may be exercised in relation to cases to which this section applies with a view to securing that payments in respect of child support maintenance are made or that arrears of child support maintenance are recovered.

(3) For the purposes of this section, the benefits to which section 5 of the 1992 Act applies are to be taken as including war disablement pensions and war widows' pensions (within the meaning of section 150 of the Social Security Contributions and Benefits Act 1992 (interpretation)).”
\end{quotation}

\amendment{
S. 21 is in force only for new-rules cases; see the Child Support, Pensions and Social Security Act 2000 (Commencement No. 12) Order 2003 art. 3.
}

\subsection{22. Jurisdiction}

(1) Section 44 of the 1991 Act (jurisdiction) shall be amended as follows.
 
(2) In subsection (1), after “United Kingdom” there shall be inserted “, except in the case of a non-resident parent who falls within subsection (2A)”.

(3) After subsection (2)  there shall be inserted—
\begin{quotation}
“(2A) A non-resident parent falls within this subsection if he is not habitually resident in the United Kingdom, but is—
\begin{enumerate}\item[]
($a$) employed in the civil service of the Crown, including Her Majesty’s Diplomatic Service and Her Majesty’s Overseas Civil Service;

($b$) a member of the naval, military or air forces of the Crown, including any person employed by an association established for the purposes of Part XI of the Reserve Forces Act 1996;

($c$) employed by a company of a prescribed description registered under the Companies Act 1985 in England and Wales or in Scotland, or under the Companies (Northern Ireland) Order 1986; or

($d$) employed by a body of a prescribed description.”
\end{enumerate}
\end{quotation}

(4) Subsection (3)  shall cease to have effect.

\amendment{
S. 22(4) is in force only for new-rules cases; see the Child Support, Pensions and Social Security Act 2000 (Commencement No. 12) Order 2003 art. 3.
}

\subsection{23. Abolition of the child maintenance bonus}

Section 10 of the Child Support Act 1995 (which provides for the child maintenance bonus) shall cease to have effect.

\amendment{
S. 23 is in force only for new-rules cases; see the Child Support, Pensions and Social Security Act 2000 (Commencement No. 12) Order 2003 art. 6.
}

\subsection{24. Periodical reviews}

Article 3(4)  of the  Social Security Act 1998 (Commencement No.\ 2) Order 1998 (which saved section 16 of the 1991 Act for certain purposes) is revoked; and accordingly that section shall cease to have effect for all purposes.

\subsection{25. Regulations}

In section 52 of the 1991 Act (regulations and orders), for subsection (2)  there shall be substituted—
\begin{quotation}
“(2) No statutory instrument containing (whether alone or with other provisions) regulations made under—
\begin{enumerate}\item[]
($a$) section 6(1), 12(4)  (so far as the regulations make provision for the default rate of child support maintenance mentioned in section 12(5)($b$)), 28C(2)($b$), 28F(2)($b$), 30(5A), 41(2), 41A, 41B(6), 43(1), 44(2A)($d$), 46 or 47;

($b$) paragraph 3(2)  or 10A(1)  of Part I of Schedule 1; or

($c$) Schedule 4B,
\end{enumerate}
or an order made under section 45(1)  or (6), shall be made unless a draft of the instrument has been laid before Parliament and approved by a resolution of each House of Parliament.

(2A) No statutory instrument containing (whether alone or with other provisions) the first set of regulations made under paragraph 10(1)  of Part I of Schedule 1 as substituted by section 1(3)  of the Child Support, Pensions and Social Security Act 2000 shall be made unless a draft of the instrument has been laid before Parliament and approved by a resolution of each House of Parliament.”
\end{quotation}

\amendment{
S. 25 is in force only for new-rules cases; see the Child Support, Pensions and Social Security Act 2000 (Commencement No. 12) Order 2003 art. 3.
}

\subsection{26. Amendments}

Schedule 3 (amendment of enactments) shall have effect.

\subsection{27. Temporary compensation payment scheme}

(1) This section applies where—
\begin{enumerate}\item[]
($a$) a maintenance assessment is made before a prescribed date following an application for one under section 4, 6 or 7 of the 1991 Act; or

($b$) a fresh maintenance assessment has been made following either a periodic review under section 16 of the 1991 Act or a review under section 17 of that Act (as they had effect before their substitution by section 40 or 41 respectively of the Social Security Act 1998),
\end{enumerate}
and the effective date of the assessment is earlier than the date on which the assessment was made, with the result that arrears of child support maintenance have become due under the assessment.

(2) The Secretary of State may in regulations provide that this section has effect as if it were modified so as—
\begin{enumerate}\item[]
($a$) to apply to cases of arrears of child support maintenance having become due additional to those referred to in subsection (1);

($b$) not to apply to any such case as is referred to in subsection (1).
\end{enumerate}

(3) If this section applies, the Secretary of State may in prescribed circumstances agree with the absent parent, on terms specified in the agreement, that—
\begin{enumerate}\item[]
($a$) the absent parent will not be required to pay the whole of the arrears, but only some lesser amount; and

($b$) the Secretary of State will not, while the agreement is complied with, take action to recover any of the arrears.
\end{enumerate}

(4) The terms which may be specified are to be prescribed in or determined in accordance with regulations made by the Secretary of State.

(5) An agreement may be entered into only if it is made before 1st April 
%2002 
2005  % Word substituted (17.7.02) by SI 2002/1854 reg 2($a$) 
and expires before 1st April 
%2003
2006%  % Word substituted (17.7.02) by SI 2002/1854 reg 2($b$) 
. 

(6) If the absent parent enters into such an agreement, the Secretary of State may, while the absent parent complies with it, refrain from taking action under the 1991 Act to recover the arrears.

(7) Upon the expiry of the agreement, if the absent parent has complied with it—
\begin{enumerate}\item[]
($a$) he ceases to be liable to pay the arrears; and

($b$) the Secretary of State may make payments of such amounts and at such times as he may determine to the person with care.
\end{enumerate}

(8) If the absent parent fails to comply with the agreement he becomes liable to pay the full amount of any outstanding arrears (as well as any other amount payable in accordance with the assessment).

(9) The Secretary of State may by regulations provide for this section to have effect as if there were substituted for the dates in subsection (5)  such later dates as are prescribed.

(10) In this section, “prescribed” means prescribed in regulations made by the Secretary of State.

(11) Regulations under this section shall be made by statutory instrument.

(12) No statutory instrument containing regulations under subsection (9)  is to be made unless a draft of the instrument has been laid before Parliament and approved by a resolution of each House of Parliament; but otherwise a statutory instrument containing regulations under this section shall be subject to annulment in pursuance of a resolution of either House of Parliament.

\amendment{
Words substituted in s. 27(5) (17.7.02) by the Child Support (Temporary Compensation Payment Scheme) (Modification and Amendment) Regulations 2002 reg. 2.

\medskip

S. 28 is not yet in force.
}

%28Pilot schemes
%
%(1) Any regulations made under—
%
%($a$) provisions inserted or substituted in the 1991 Act by this Part of this Act (or Schedule 1, 2 or 3); and
%
%($b$) in so far as they are consequential on or supplementary to any such regulations, regulations made under any other provisions in the 1991 Act,
%
%may be made so as to have effect for a specified period not exceeding 12 months.
%
%(2) Any regulations which, by virtue of subsection (1), are to have effect for a limited period are referred to in this section as “a pilot scheme”.
%
%(3) A pilot scheme may provide that its provisions are to apply only in relation to—
%
%($a$) one or more specified areas or localities;
%
%($b$) one or more specified classes of person;
%
%($c$) persons selected by reference to prescribed criteria, or on a sampling basis.
%
%(4) A pilot scheme may make consequential or transitional provision with respect to the cessation of the scheme on the expiry of the specified period.
%
%(5) A pilot scheme (“the previous scheme”) may be replaced by a further pilot scheme making the same provision as that made by the previous scheme (apart from the specified period), or similar provision.
%
%(6) A statutory instrument containing (whether alone or with other provisions) a pilot scheme shall not be made unless a draft of the instrument has been laid before Parliament and approved by resolution of each House of Parliament.

\subsection{29. Interpretation, transitional provisions, savings, etc}

(1) In this Part, “the 1991 Act” means the Child Support Act 1991. 

(2) The Secretary of State may in regulations make such transitional and transitory provisions, and such incidental, supplementary, savings and consequential provisions, as he considers necessary or expedient in connection with the coming into force of this Part or any provision in it.

(3) The regulations may, in particular—
\begin{enumerate}\item[]
($a$) provide for the amount of child support maintenance payable by or to any person to be at a transitional rate (or more than one such rate successively) resulting from the phasing-in by way of prescribed steps of any increase or decrease in the amount payable following the coming into force of this Part or any provision in it;

($b$) provide for a departure direction or any finding in relation to a previous determination of child support maintenance to be taken into account in a decision as to the amount of child support maintenance payable by or to any person.
\end{enumerate}

(4) Section 175(3)  and (5)  of the Social Security Contributions and Benefits Act 1992 (supplemental power in relation to regulations) applies to regulations made under this section as it applies to regulations made under that Act.

(5) The power to make regulations under this section is exercisable by statutory instrument.

(6) A statutory instrument containing regulations under this section shall be subject to annulment in pursuance of a resolution of either House of Parliament.

\addtocontents{toc}{\protect\pagebreak[3]}

\part[Part II --- Pensions]{Part II\\*Pensions}

\addtocontents{toc}{\protect\nopagebreak[3]}

\section[Chapter I --- State pensions]{Chapter I\\*State pensions}

\renewcommand\parthead{--- Part II Chapter I}

\subsection{\itshape State second pension}

\subsubsection{30. Earnings from which pension derived}

(1) In section 22 of the Social Security Contributions and Benefits Act 1992 (earnings from which earnings factors are derived), after subsection (2)  there shall be inserted—
\begin{quotation}
“(2A) For the purposes specified in subsection (2)($b$)  above, in the case of the first appointed year or any subsequent tax year a person’s earnings factor shall be treated as derived only from those of his earnings on which primary Class 1 contributions have been paid or treated as paid.”
\end{quotation}

(2) In section 44 of that Act (Category A retirement pension), in subsection (6)—
\begin{enumerate}\item[]
($a$) before paragraph ($a$)  there shall be inserted—
\begin{quotation}
“($za$) where the relevant year is the first appointed year or any subsequent year, to the aggregate of his earnings factors derived from those of his earnings upon which primary Class 1 contributions have been paid or treated as paid in respect of that year;”;
\end{quotation}
and

($b$) in paragraph ($a$), after “subsequent tax year” there shall be inserted “before the first appointed year”.
\end{enumerate}

(3) After that section there shall be inserted—
\begin{quotation}
\subsection*{“44A. Deemed earnings factors}

(1) For the purposes of section 44(6)($za$)  above, if any of the conditions in subsection (2)  below is satisfied for a relevant year, a pensioner is deemed to have an earnings factor for that year which—
\begin{enumerate}\item[]
($a$) is derived from earnings on which primary Class 1 contributions were paid; and

($b$) is equal to the amount which, when added to any other earnings factors taken into account under that provision, produces an aggregate of earnings factors equal to the low earnings threshold.
\end{enumerate}

(2) The conditions referred to in subsection (1)  above are that—
\begin{enumerate}\item[]
($a$) the pensioner would, apart from this section, have an earnings factor for the year—
\begin{enumerate}\item[]
(i) equal to or greater than the qualifying earnings factor for the year; but

(ii) less than the low earnings threshold for the year;
\end{enumerate}

($b$) invalid care allowance—
\begin{enumerate}\item[]
(i) was payable to the pensioner throughout the year; or

(ii) would have been so payable but for the fact that under regulations the amount payable to him was reduced to nil because of his receipt of other benefits;
\end{enumerate}

($c$) for the purposes of paragraph 5(7)($b$)  of Schedule 3, the pensioner is taken to be precluded from regular employment by responsibilities at home throughout the year by virtue of—
\begin{enumerate}\item[]
(i) the fact that child benefit was payable to him in respect of a child under the age of six; or

(ii) his satisfying such other condition as may be prescribed;
\end{enumerate}

($d$) the pensioner is a person satisfying the requirement in subsection (3)  below to whom long-term incapacity benefit was payable throughout the year, or would have been so payable but for the fact that—
\begin{enumerate}\item[]
(i) he did not satisfy the contribution conditions in paragraph 2 of Schedule 3; or

(ii) under regulations the amount payable to him was reduced to nil because of his receipt of other benefits or of payments from an occupational pension scheme or personal pension scheme.
\end{enumerate}
\end{enumerate}

(3) The requirement referred to in subsection (2)($d$)  above is that—
\begin{enumerate}\item[]
($a$) for one or more relevant years the pensioner has paid, or (apart from this section) is treated as having paid, primary Class 1 contributions on earnings equal to or greater than the qualifying earnings factor; and

($b$) the years for which he has such a factor constitute at least one tenth of his working life.
\end{enumerate}

(4) For the purposes of subsection (3)($b$)  above—
\begin{enumerate}\item[]
($a$) a pensioner’s working life shall not include—
\begin{enumerate}\item[]
(i) any tax year before 1978--79; or

(ii) any year in which he is deemed under subsection (1)  above to have an earnings factor by virtue of fulfilling the condition in subsection (2)($b$)  or ($c$)  above; and
\end{enumerate}

($b$) the figure calculated by dividing his working life by ten shall be rounded to the nearest whole year (and any half year shall be rounded down).
\end{enumerate}

(5) The low earnings threshold for the first appointed year and subsequent tax years shall be £9,500 (but subject to section 148A of the Administration Act).

(6) In subsection (2)($d$)(ii)  above, “occupational pension scheme” and “personal pension scheme” have the meanings given by subsection (6)  of section 30DD above for the purposes of subsection (5)  of that section.”
\end{quotation}

(4) For the purposes of subsection (1)  of section 44A of the Social Security Contributions and Benefits Act 1992, a pensioner is deemed to have an earnings factor in relation to any relevant year as specified in that subsection if—
\begin{enumerate}\item[]
($a$) severe disablement allowance was payable to him throughout the year; and

($b$) he satisfies the requirement in subsection (3)  of that section.
\end{enumerate}

\subsubsection{31. Calculation}

(1) In section 45 of the Social Security Contributions and Benefits Act 1992 (calculation of additional pension in a Category A retirement pension), in subsection (2)—
\begin{enumerate}\item[]
($a$) after “shall be” there shall be inserted “the sum of the following”;

($b$) in paragraph ($b$), after “after 1987--88” there shall be inserted “but before the first appointed year”; and

($c$) after that paragraph there shall be inserted “; and
\begin{quotation}
($c$) in relation to any tax years falling within subsection (3A)  below, the weekly equivalent of the amount calculated in accordance with Schedule 4A to this Act.”
\end{quotation}
\end{enumerate}

(2) In that section the following subsection shall be inserted after subsection (3)—
\begin{quotation}
“(3A) The following tax years fall within this subsection—
\begin{enumerate}\item[]
($a$) the first appointed year;

($b$) subsequent tax years.”
\end{enumerate}
\end{quotation}

(3) After Schedule 4 to that Act there shall be inserted the Schedule set out in Schedule 4 to this Act.

\subsubsection{32. Calculation of Category B retirement pension}

(1) In section 46 of the Social Security Contributions and Benefits Act 1992 (modifications of section 45 for calculating the additional pension in certain benefits), after subsection (2)  there shall be inserted—
\begin{quotation}
“(3) For the purpose of determining the additional pension falling to be calculated under section 45 above by virtue of section 48BB below in a case where the deceased spouse died under pensionable age, the following definition shall be substituted for the definition of “N” in section 45(4)($b$)  above—
\begin{quotation}
    ““N” =
\begin{enumerate}\item[]
    ($a$) 
    the number of tax years which begin after 5th April 1978 and end before the date when the deceased spouse dies, or

    ($b$) 
    the number of tax years in the period—
\begin{enumerate}\item[]
    (i) 
    beginning with the tax year in which the deceased spouse (“S”) attained the age of 16 or, if later, 1978--79, and

    (ii) 
    ending immediately before the tax year in which S would have attained pensionable age if S had not died earlier,
\end{enumerate}
\end{enumerate}
    whichever is the smaller number.”” 
\end{quotation}
\end{quotation}

(2) In section 48BB of that Act (Category B retirement pension: entitlement by reference to benefits under section 39A or 39B), in subsection (5)  for “section 46(2)” there shall be substituted “section 46(3)”.

(3) In paragraph 5 of Schedule 8 to the Welfare Reform and Pensions Act 1999 (welfare benefits: minor and consequential amendments), sub-paragraph ($b$), and the word “and” immediately preceding it, shall be omitted.

\subsubsection{33. Revaluation}

(1) After section 148 of the Social Security Administration Act 1992 there shall be inserted—
\begin{quotation}
\subsection*{“148A. Revaluation of low earnings threshold}

(1) The Secretary of State shall in the tax year preceding the first appointed year and in each subsequent tax year review the general level of earnings obtaining in Great Britain and any changes in that level which have taken place during the review period.

(2) In this section, “the review period” means—
\begin{enumerate}\item[]
($a$) in the case of the first review under this section, the period beginning with 1st October 1998 and ending on 30th September in the tax year preceding the first appointed year; and

($b$) in the case of each subsequent review under this section, the period since—
\begin{enumerate}\item[]
(i) the end of the last period taken into account in a review under this section; or

(ii) such other date (whether earlier or later) as the Secretary of State may determine.
\end{enumerate}
\end{enumerate}

(3) If on such a review it appears to the Secretary of State that the general level of earnings has increased during the review period, he shall make an order under this section.

(4) An order under this section shall be an order directing that, for the purposes of the Contributions and Benefits Act—
\begin{enumerate}\item[]
($a$) there shall be a new low earnings threshold for the tax years after the tax year in which the review takes place; and

($b$) the amount of that threshold shall be the amount specified in subsection (5)  below—
\begin{enumerate}\item[]
(i) increased by the percentage by which the general level of earnings increased during the review period; and

(ii) rounded to the nearest £100 (taking any amount of £50 as nearest to the next whole £100).
\end{enumerate}
\end{enumerate}

(5) The amount referred to in subsection (4)($b$)  above is—
\begin{enumerate}\item[]
($a$) in the case of the first review under this section, £9,500; and

($b$) in the case of each subsequent review, the low earnings threshold for the year in which the review takes place.
\end{enumerate}

(6) This section does not require the Secretary of State to direct any increase where it appears to him that the increase would be inconsiderable.

(7) If on any review under subsection (1)  above the Secretary of State determines that he is not required to make an order under this section, he shall instead lay before each House of Parliament a report explaining his reasons for arriving at that determination.

(8) For the purposes of any review under subsection (1)  above the Secretary of State shall estimate the general level of earnings in such manner as he thinks fit.”
\end{quotation}

(2) Section 148 of the Social Security Administration Act 1992 (revaluation of earnings factors) shall have effect as if—
\begin{enumerate}\item[]
($a$) the amounts for the first appointed year and any subsequent tax year that are to be reviewed under that section,

($b$) the amounts for those years to which any directions by an order under subsection (4)  of that section are to be applied, and

($c$) accordingly, the amounts for the purpose of maintaining the value of which that section has effect,
\end{enumerate}
included the parts of the surplus in an earnings factor referred to in paragraphs 2(2)($a$), 5(2)($a$)  and 7(2)($a$)  of Schedule 4A to the Social Security Contributions and Benefits Act 1992. 

(3) Nothing in section 148 of the Social Security Administration Act 1992 shall require, or ever have required, the earnings factors used for computing a surplus in an earnings factor for any year under section 44(5A)  of the Social Security Contributions and Benefits Act 1992 to be treated as increased in any case in which that surplus, or any part of it, is itself reviewed under section 148 of the Social Security Administration Act 1992. 

(4) In section 128(3)  of the Pensions Act 1995 (revaluation of surpluses in earnings factors under section 44(5A)  of the Social Security Contributions and Benefits Act 1992), after “1992” there shall be inserted “for the purposes of section 45(1)  and (2)($a$)  and ($b$)  of that Act”.

\subsubsection{34. Report of Government Actuary: rebates etc}

In each of sections 42(1)($a$)(ii), 42B(1)($a$)  and 45A(1)($a$)  of the Pension Schemes Act 1993 (reports by Government Actuary on cost of providing benefits equivalent to benefits which are foregone) for “which, under section 48A,” there shall be substituted “(or parts of benefits) which, in accordance with section 48A below and Schedule 4A to the Social Security Contributions and Benefits Act 1992,”.

\subsubsection{35. Supplementary}

(1) The Social Security Contributions and Benefits Act 1992 shall be amended as follows.

(2) In section 21(5A)($b$)  (contribution conditions)—
\begin{enumerate}\item[]
($a$) after “22(1)($a$)” there shall be inserted “, (2A)”; and

($b$) for “44(6)($a$)” there shall be substituted “44(6)($za$)  and ($a$)”.
\end{enumerate}

(3) In section 39 (rate of widowed mother’s allowance and widow’s pension), in subsections (1), (2)  and (3), after “sections 44 to 45B” there shall be inserted “and Schedule 4A”.

(4) In section 39C (rate of widowed parent’s allowance and bereavement allowance), in subsections (1), (3)  and (4), after “sections 44 to 45A” there shall be inserted “and Schedule 4A”.

(5) In section 44 (Category A retirement pension), in subsection (5A), after “section 45” there shall be inserted “and Schedule 4A”.

(6) In that subsection, for the words from “that year,” to “surplus” there shall be substituted “that year,
\begin{quotation}
($b$) the amount of the surplus is the amount of that excess, and

($c$) for the purposes of section 45(1)  and (2)($a$)  and ($b$)  below, the adjusted amount of the surplus”.
\end{quotation}

(7) In subsection (6)  of that section, after “section 45” there shall be inserted “or Schedule 4A”.

(8) In section 45 (the additional element in a Category A retirement pension)—
\begin{enumerate}\item[]
($a$) in subsections (1)  and (2)($a$)  and ($b$), before “amount” (in each place) there shall be inserted “adjusted”; and

($b$) in subsection (6), for “the amount of any surpluses” there shall be substituted “any amount”.
\end{enumerate}

(9) In section 48A(4)  (Category B retirement pension for married person), after “sections 44 to 45B above” there shall be inserted “and Schedule 4A below”.

(10) In section 48B (Category B retirement pension for widows and widowers), in subsections (2)  and (3), after “sections 44 to 45B above” there shall be inserted “and Schedule 4A below”.

(11) In section 48BB (Category B retirement pension: entitlement by reference to benefits under section 39A or 39B), in subsections (5)  and (6), after “sections 44 to 45A above” there shall be inserted “and Schedule 4A below”.

(12) In section 48C(4)  (Category B retirement pension: general), after “sections 44 to 45B above” there shall be inserted “and Schedule 4A below”.

(13) In section 51 (Category B retirement pension for widowers), in subsections (2)  and (3), after “sections 44 to 45A above” there shall be inserted “and Schedule 4A below”.

(14) In section 122(1)  (interpretation of Parts I to VI), at the appropriate place in alphabetical order, there shall be inserted—
\begin{quotation}
““first appointed year” means such tax year, no earlier than 2002--03, as may be appointed by order, and “second appointed year” means such subsequent tax year as may be so appointed;”.
\end{quotation}

(15) In section 176 (Parliamentary control), after subsection (3)  there shall be inserted—
\begin{quotation}
“(4) Subsection (3)  above does not apply to a statutory instrument by reason only that it contains an order appointing the first or second appointed year (within the meanings given by section 122(1)  above).”
\end{quotation}

\addtocontents{toc}{\protect\pagebreak[3]}

\subsection{\itshape Report on pensions uprating}

\subsubsection{36. Report on cost of pension uprating in line with general earnings level}

The Government Actuary or the Deputy Government Actuary shall report to the Secretary of State his opinion on the effect on the level of the National Insurance Fund, and the effect which might be expected on the rates of contributions, in each year up to and including 2005--06 of annual increases in the basic pension by the percentage increase in the general level of earnings; and the Secretary of State shall lay a copy of the report before Parliament.

\subsection{\itshape Earnings factors}

\subsubsection{37. Revaluation of earnings factors}

In section 148(2)  of the Social Security Administration Act 1992 (revaluation of earnings factors), for the words from “place” to the end there shall be substituted “place—
\begin{quotation}
\begin{enumerate}\item[]\noindent
($a$) since the end of the period taken into account for the last review under this section, or

($b$) since such other date (whether earlier or later) as he may determine;
\end{enumerate}\noindent
and for the purposes of any such review the Secretary of State shall estimate the general level of earnings in such manner as he thinks fit.”
\end{quotation}

\subsubsection{38. Modification of earnings factors}

(1) In section 48A(5)  of the 1993 Act (power to modify the application of section 44(5)  of the 1992 Act where in any year a pensioner’s earnings derive only partially from contracted-out employment), after “44(5)” there shall be inserted “or (5A)”.

(2) Subsection (1)  shall have effect—
\begin{enumerate}\item[]
($a$) in relation to the application of section 44(5A)  of the 1992 Act by virtue of sections 39C(1)  and 48BB(5)  of that Act;

($b$) in relation to the application of section 44(5A)  of the 1992 Act in the circumstances described in section 128(4)  to (6)  of the 1995 Act.
\end{enumerate}

(3) In relation to the period—
\begin{enumerate}\item[]
($a$) beginning with 6th April 2000, and

($b$) ending with the day before the first regulations under section 48A(5)  of the 1993 Act (as amended by subsection (1)  above) come into force,
\end{enumerate}
the Secretary of State shall be taken to have, and to have had, power to calculate and pay relevant pensions by reference to section 44(5)  of the 1992 Act as modified by regulations under section 48A(5)  of the 1993 Act.

(4) For the purpose of applying subsection (3)  above—
\begin{enumerate}\item[]
($a$) the substitution made by section 128(1)  of the 1995 Act shall be ignored; and

($b$) references in enactments to section 44(5A)  of the 1992 Act shall (so far as necessary) be treated as references to section 44(5).
\end{enumerate}

(5) The first regulations under section 48A(5)  of the 1993 Act (as amended by subsection (1)  above) may include provision in relation to—
\begin{enumerate}\item[]
($a$) revising the calculation of a relevant pension;

($b$) paying a relevant pension in accordance with a revised calculation.
\end{enumerate}

(6) Relevant pensions are pensions which fall to be calculated—
\begin{enumerate}\item[]
($a$) in the circumstances described in section 128(4)  to (6)  of the 1995 Act; and

($b$) in relation to persons where, by virtue of section 48A(1)  of the 1993 Act, section 44(6)  of the 1992 Act has effect in any tax year as mentioned in section 48A(1)  of the 1993 Act in relation to some but not all of a person’s earnings.
\end{enumerate}

(7) For the purposes of this section—
\begin{enumerate}\item[]
($a$) the 1992 Act is the Social Security Contributions and Benefits Act 1992;

($b$) the 1993 Act is the Pension Schemes Act 1993;

($c$) the 1995 Act is the Pensions Act 1995. 
\end{enumerate}

\subsection{\itshape Preservation of rights in respect of additional pensions}

\subsubsection{39. Preservation of rights in respect of additional pensions}

(1) In the provisions of the Social Security Contributions and Benefits Act 1992 that are set out in subsection (2)  (provisions relating to additional pensions for surviving spouses)—
\begin{enumerate}\item[]
($a$) the references to 5th April 2000 (wherever occurring) shall have effect, and be deemed always to have had effect, as references to 5th October 2002; and

($b$) the references to 6th April 2000 (wherever occurring) shall have effect, and be deemed always to have had effect, as references to 6th October 2002. 
\end{enumerate}

(2) Those provisions are—
\begin{enumerate}\item[]
($a$) sections 39(3)  and 39C(4)  (widowed mother’s allowance and widowed parent’s allowance);

($b$) sections 48BB(7), 48C(3)  and 51(3)  (Category B retirement pensions); and

($c$) paragraphs 4(3), 5A(2)  and (3)  and 6(3)  and (4)  of Schedule 5 (deferred pensions).
\end{enumerate}

(3) For section 52(3)  of the Welfare Reform and Pensions Act 1999 (power to substitute a later year for references to year 2000 in prescribed provisions of the Social Security Contributions and Benefits Act 1992) there shall be substituted—
\begin{quotation}
“(3) The regulations may amend (or further amend) any prescribed provision set out in section 39(2)  of the Child Support, Pensions and Social Security Act 2000 (which sets out provisions falling within subsection (2)  of this section) so as to substitute a reference to a later date for—
\begin{enumerate}\item[]
($a$) any reference in that provision to 5th October 2002 or 6th October 2002; or

($b$) any reference to a date inserted in that provision by a substitution made by virtue of this subsection.”
\end{enumerate}
\end{quotation}

(4) After section 52(4)  of that Act of 1999 there shall be inserted—
\begin{quotation}
“(4A) The regulations may provide, for the purposes of any provision made by virtue of subsection (4), for a case in which a person who, as a consequence of receiving incorrect or incomplete information, did not give any consideration to—
\begin{enumerate}\item[]
($a$) the taking of a step which is a step he might have taken had he considered the matter on the basis of correct and complete information, or

($b$) refraining from taking a step which is a step he did take but might have refrained from taking had he considered the matter on that basis,
\end{enumerate}
to be treated as a case in which his failure to take the step, or his taking of the step he did take, was in reliance on the incorrect or incomplete information and as a case in which that step is one which he would have taken, or (as the case may be) would not have taken, had the information been correct and complete.”
\end{quotation}

(5) In section 52(6)  of that Act of 1999 (supplemental provisions of regulations relating to the scheme), after paragraph ($e$)  there shall be inserted—
\begin{quotation}
“($ea$) prescribing the matters that may be relied on, and the presumptions that may be made, in the determination of whether or not the prescribed conditions have been satisfied;”.
\end{quotation}

\subsection{\itshape Other provisions}

\subsubsection{40. Home responsibilities protection}

In paragraph 5 of Schedule 3 to the Social Security Contributions and Benefits Act 1992 (contribution conditions for entitlement to Category A and B retirement pension, widowed mother’s allowance and widow’s pension), after sub-paragraph (7)  (reduction of number of years for which contribution conditions must be satisfied) there shall be inserted—
\begin{quotation}
“(7A) Regulations may provide that a person is not to be taken for the purposes of sub-paragraph (7)($b$)  above as precluded from regular employment by responsibilities at home unless he meets the prescribed requirements as to the provision of information to the Secretary of State.”
\end{quotation}

\subsubsection{41. Sharing of state scheme rights}

(1) In section 49 of the Welfare Reform and Pensions Act 1999 (creation of state scheme pension debits and credits), for subsection (4)  there shall be substituted—
\begin{quotation}
“(4) The Secretary of State may by regulations make provision about the calculation and verification of cash equivalents for the purposes of this section.

(4A) The power conferred by subsection (4)  above includes power to provide—
\begin{enumerate}\item[]
($a$) for calculation or verification in such manner as may be approved by or on behalf of the Government Actuary, and

($b$) for things done under the regulations to be required to be done in accordance with guidance from time to time prepared by a person prescribed by the regulations.”
\end{enumerate}
\end{quotation}

(2) In section 45B of the Social Security Contributions and Benefits Act 1992 (pension sharing resulting in reduction of additional Category A retirement pension), for subsection (7)  there shall be substituted—
\begin{quotation}
“(7) The Secretary of State may by regulations make provision about the calculation and verification of cash equivalents for the purposes of this section.

(7A) The power conferred by subsection (7)  above includes power to provide—
\begin{enumerate}\item[]
($a$) for calculation or verification in such manner as may be approved by or on behalf of the Government Actuary, and

($b$) for things done under the regulations to be required to be done in accordance with guidance from time to time prepared by a person prescribed by the regulations.”
\end{enumerate}
\end{quotation}

(3) In section 55A of that Act (shared additional pension), for subsection (6)  there shall be substituted—
\begin{quotation}
“(6) The Secretary of State may by regulations make provision about the calculation and verification of cash equivalents for the purposes of this section.

(6A) The power conferred by subsection (6)  above includes power to provide—
\begin{enumerate}\item[]
($a$) for calculation or verification in such manner as may be approved by or on behalf of the Government Actuary, and

($b$) for things done under the regulations to be required to be done in accordance with guidance from time to time prepared by a person prescribed by the regulations.”
\end{enumerate}
\end{quotation}

(4) In section 55B of that Act (pension sharing resulting in reduction of shared additional pension), for subsection (7)  there shall be substituted—
\begin{quotation}
“(7) The Secretary of State may by regulations make provision about the calculation and verification of cash equivalents for the purposes of this section.

(7A) The power conferred by subsection (7)  above includes power to provide—
\begin{enumerate}\item[]
($a$) for calculation or verification in such manner as may be approved by or on behalf of the Government Actuary, and

($b$) for things done under the regulations to be required to be done in accordance with guidance from time to time prepared by a person prescribed by the regulations.”
\end{enumerate}
\end{quotation}

\subsubsection{42. Disclosure of state pension information}

(1) This section applies to any state pension information which is held in relation to any individual—
\begin{enumerate}\item[]
($a$) by the Secretary of State; or

($b$) in connection with the provision of any services provided to the Secretary of State for purposes connected with his functions relating to social security, by the person providing those services.
\end{enumerate}

(2) Regulations may confer a power on the Secretary of State to disclose, or to authorise the disclosure of, any information to which this section applies in any case in which—
\begin{enumerate}\item[]
($a$) the person to whom the disclosure is made is a person falling within subsection (3)  who has, in the prescribed manner, applied to the Secretary of State for the disclosure of the information; and

($b$) it appears to the Secretary of State that the prescribed conditions for the making of a disclosure of the information in question to that person have been satisfied.
\end{enumerate}

(3) A person falls within this subsection if—
\begin{enumerate}\item[]
($a$) he is the trustee or manager of an occupational pension scheme of which the individual to whom the information relates is a member;

($b$) he is the trustee or manager of a personal pension scheme of which that individual is a member;

($c$) he is the employer in relation to an occupational pension scheme of which that individual is a member;

($d$) he is the employer in relation to any employed earner’s employment of that individual which is not contracted-out employment; or

($e$) he is proposing to provide services to that individual in circumstances in which the provision of the services, or the proposal to do so, may involve the giving of advice or forecasts to which the information to which this section applies may be relevant.
\end{enumerate}

(4) The Secretary of State shall secure that his powers under this section are exercised so that at least the following is prescribed for the purposes of subsection (2)($b$), namely—
\begin{enumerate}\item[]
($a$) in the case of an application for information made by a person falling within paragraph ($e$)  of subsection (3), a condition that the individual to whom the information relates has consented to the making of the application and to the disclosure; and

($b$) in any other case, either that condition or the alternative condition set out in subsection (5).
\end{enumerate}

(5) The alternative condition is—
\begin{enumerate}\item[]
($a$) that such steps as may be prescribed have been taken for the purpose of ascertaining whether the individual to whom the information relates objects to the making of the application for the disclosure of information relating to him; and

($b$) that the prescribed time has elapsed without any objection by that individual.
\end{enumerate}

(6) A person applying to the Secretary of State, in accordance with regulations under this section, for the disclosure of any information relating to an individual shall be entitled, for the purpose of making the application, to make such disclosures of information relating to that individual as may be authorised by the regulations.

(7) In this section the reference, in relation to an individual, to state pension information is a reference to the following information about that individual—
\begin{enumerate}\item[]
($a$) his date of birth, and the age at which and date on which he attains pensionable age—
\begin{enumerate}\item[]
(i) for the purposes of the Pension Schemes Act 1993, in relation to any guaranteed minimum pension to which he is entitled; and

(ii) in accordance with the rules in paragraph 1 of Schedule 4 to the Pensions Act 1995;
\end{enumerate}

($b$) the amount of any basic retirement pension a present or future entitlement to which has already accrued to that individual, and the amount of any additional retirement pension such an entitlement to which has already accrued to that individual;

($c$) a projection of the amount of the basic retirement pension to which that individual is likely to become entitled, or might become entitled in particular circumstances; and

($d$) a projection of the amount of the additional retirement pension to which that individual is likely to become entitled, or might become entitled in particular circumstances.
\end{enumerate}

(8) Regulations under this section shall be made by statutory instrument, which shall be subject to annulment in pursuance of a resolution of either House of Parliament.

(9) Subsections (4)  to (6)  of section 189 of the Social Security Administration Act 1992 (supplemental and incidental powers etc.)\ shall apply in relation to any power to make regulations under this section as they apply in relation to the powers to make regulations that are conferred by that Act.

(10) For the purposes of section 121E of the Social Security Administration Act 1992 (supply of information by the Inland Revenue to the Secretary of State for the purposes of the Secretary of State’s functions relating to social security), the Secretary of State’s functions relating to social security shall be taken to include any power conferred on him by regulations under this section.

(11) In this section—
\begin{enumerate}\item[]
    “basic retirement pension” and “additional retirement pension” mean any basic or, as the case may be, additional pension under the Social Security Contributions and Benefits Act 1992;

    “contracted-out employment” has the same meaning as in the Pension Schemes Act 1993;

    “employed earner” has the same meaning as it has in Parts I to V of the Social Security Contributions and Benefits Act 1992 (by virtue of section 2(1)  of that Act);

    “employer”—
\begin{enumerate}\item[]
    ($a$) 
    in relation to any occupational pension scheme, has the same meaning as in Part I of the Pensions Act 1995; and

    ($b$) 
    in relation to employed earner’s employment, has the same meaning as in the Pension Schemes Act 1993;
\end{enumerate}

    “member”, in relation to an occupational pension scheme, has the same meaning as in Part I of the Pensions Act 1995;

    “occupational pension scheme” and “personal pension scheme” have the same meanings as in the Pension Schemes Act 1993;

    “prescribed” means prescribed by or determined in accordance with regulations;

    “regulations” means regulations made by the Secretary of State;

    “trustee” and “manager”, in relation to an occupational pension scheme, have the same meanings as in Part I of the Pensions Act 1995.  
\end{enumerate}

\section[Chapter II --- Occupational and Personal Pension Schemes]{Chapter II\\*Occupational and Personal Pension Schemes}
%Selection of trustees and of directors of corporate trustees
\renewcommand\parthead{--- Part II Chapter II}

%43Member-nominated trustees
%
%(1) Section 16 of the [1995 c. 26. ] Pensions Act 1995 (requirements for trustees to be nominated and selected by members of the scheme) shall be amended in accordance with subsections (2)  to (8)  of this section.
%
%(2) In subsection (1)  (duty of trustees to make arrangements for selection of member-nominated trustees)—
%
%($a$) the words “(subject to section 17)” and in paragraph ($b$), the words “, and the appropriate rules,” shall be omitted; and
%
%($b$) in paragraph ($a$), for “persons selected” there shall be substituted “the selection of persons nominated”.
%
%(3) In subsection (3)($a$)  (selected persons to be trustees), for “in accordance with the appropriate rules” there shall be substituted “as a member-nominated trustee”.
%
%(4) In subsection (4)  (procedure for filling vacancies unfilled because of insufficient nominations), for “the appropriate rules” there shall be substituted “regulations”.
%
%(5) In subsection (5)  (period of service as a member-nominated trustee), after “six years” there shall be inserted “but for a member-nominated trustee to be eligible for selection again at the end of any period of service as such a trustee.”
%
%(6) After subsection (6)  there shall be inserted—
%
%“(6A) The arrangements must provide that, where the employer so requires, a person who is not a qualifying member of the scheme must have the employer’s approval to qualify for selection as a member-nominated trustee.”
%
%(7) In subsection (8)  (persons ceasing to be member-nominated trustees on ceasing to be qualifying members of the scheme)—
%
%($a$) for “The arrangements must” there shall be substituted “The arrangements—
%
%($a$) must”; and
%
%($b$) at the end there shall be inserted “; and
%
%($b$) may provide for a member-nominated trustee who—
%
%(i) is a qualifying member of one of the following descriptions, that is to say, an active, deferred or pensioner member, and
%
%(ii) ceases (without ceasing to be a qualifying member) to be a qualifying member of that description,
%
%to cease, by virtue of that fact, to be a trustee.”
%
%(8) After subsection (8)  there shall be inserted—
%
%“(9) Regulations may make provision in relation to arrangements under this section—
%
%($a$) supplementing the requirements of this section as to the matters to be contained in the arrangements; and
%
%($b$) providing for the manner in which, and the time within which, persons are, for the purposes of the arrangements, to be nominated and selected as member-nominated trustees.
%
%(10) This section does not apply in the case of a trust scheme if—
%
%($a$) every member of the scheme is a trustee of the scheme and no other person is such a trustee;
%
%($b$) every trustee of the scheme is a company; or
%
%($c$) the scheme is of a prescribed description.”
%
%(9) Section 17 of that Act (exceptions to section 16 where the employer’s alternative proposals are approved) shall cease to have effect.
%44Corporate trustees
%
%(1) Section 18 of the [1995 c. 26. ] Pensions Act 1995 (requirements for member-nominated directors of trustee company) shall be amended in accordance with subsections (2)  to (9)  of this section.
%
%(2) In subsection (1)  (duty of corporate trustee to make arrangements for selection of member-nominated directors)—
%
%($a$) for the words from “and the employer” to “satisfied” there shall be substituted “and there is no trustee of the scheme who is not a company”;
%
%($b$) the words “, subject to section 19,” and in paragraph ($b$), the words “, and the appropriate rules,” shall be omitted; and
%
%($c$) in paragraph ($a$), for “persons selected” there shall be substituted “the selection of persons nominated”.
%
%(3) In subsection (3)($a$)  (selected persons to be directors), for “in accordance with the appropriate rules” there shall be substituted “as a member-nominated director”.
%
%(4) In subsection (4)  (procedure for filling vacancies unfilled because of insufficient nominations), for “the appropriate rules” there shall be substituted “regulations”.
%
%(5) In subsection (5)  (period of service as a member-nominated director), after “six years” there shall be inserted “but for a member-nominated director to be eligible for selection again at the end of any period of service as such a director.”
%
%(6) After subsection (6)  there shall be inserted—
%
%“(6A) The arrangements must provide that, where the employer so requires, a person who is not a qualifying member of the scheme must have the employer’s approval to qualify for selection as a member-nominated director.”
%
%(7) In subsection (7)  (persons ceasing to be member-nominated directors on ceasing to be qualifying members of the scheme)—
%
%($a$) for “The arrangements must” there shall be substituted “The arrangements—
%
%($a$) must”; and
%
%($b$) at the end there shall be inserted “; and
%
%($b$) may provide for a member-nominated director who—
%
%(i) is a qualifying member of one of the following descriptions, that is to say, an active, deferred or pensioner member, and
%
%(ii) ceases (without ceasing to be a qualifying member) to be a qualifying member of that description,
%
%to cease, by virtue of that fact, to be a director.”
%
%(8) For subsection (8)  (companies that are trustees of two or more different trust schemes) there shall be substituted—
%
%“(8) Where—
%
%($a$) the same company is a trustee of two or more schemes by reference to each of which this section applies to the company, and
%
%($b$) the company does not, in the prescribed manner, elect that this subsection should not apply,
%
%the preceding provisions of this section and section 21(8)  shall have effect as if those schemes were a single scheme and the members of each of the schemes were members of that single scheme.”
%
%(9) After subsection (8)  there shall be inserted—
%
%“(9) Regulations may make provision in relation to arrangements under this section—
%
%($a$) supplementing the requirements of this section as to the matters to be contained in the arrangements; and
%
%($b$) providing for the manner in which, and the time within which, persons are, for the purposes of the arrangements, to be nominated and selected as member-nominated directors.
%
%(10) This section does not apply in the case of a trust scheme if the scheme is of a prescribed description.”
%
%(10) Sections 19 and 20 of that Act (exceptions to section 18 where the employer’s alternative proposals are approved and meaning of “appropriate rules”) shall cease to have effect.
%45Employer’s proposals for selection of trustees or directors
%
%(1) After section 18 of the [1995 c. 26. ] Pensions Act 1995 there shall be inserted—
%“Further provisions about the selection of trustees and directors
%18AEmployer’s proposals for selection of trustees or directors
%
%(1) Where, in the case of any trust scheme—
%
%($a$) the employer makes proposals for the adoption of arrangements for the nomination and selection of the trustees of the scheme,
%
%($b$) the proposed arrangements comply with all the requirements of section 16 and do not contain anything inconsistent with those requirements,
%
%($c$) the proposed arrangements comply with such other requirements as may be prescribed,
%
%($d$) the proposed arrangements are approved under such procedure for obtaining the views of members of the scheme as may be prescribed, and
%
%($e$) such other conditions are satisfied as may be prescribed,
%
%the trustees of the scheme shall secure that the proposed arrangements are made and implemented.
%
%(2) Where, in the case of any company which is trustee of a trust scheme of which there is no trustee who is not a company—
%
%($a$) the employer makes proposals for the adoption of arrangements for the nomination and selection of the directors of the company,
%
%($b$) the proposed arrangements comply with all the requirements of section 18 and do not contain anything inconsistent with those requirements,
%
%($c$) the proposed arrangements comply with such other requirements as may be prescribed,
%
%($d$) the proposed arrangements are approved under such procedure for obtaining the views of members of the scheme as may be prescribed, and
%
%($e$) such other conditions are satisfied as may be prescribed,
%
%the company shall secure that the proposed arrangements are made and implemented.
%
%(3) Arrangements made and implemented under this section may include provision that is different from that for which provision is made by regulations under section 16(9)  or 18(9) .
%
%(4) Regulations may make provision—
%
%($a$) as to the manner in which, and the time within which, arrangements proposed and approved for the purposes of this section are to be implemented by the trustees of a trust scheme or by a company which is a trustee of a trust scheme; and
%
%($b$) as to what is to happen where an approval for the purposes of this section of any arrangements ceases, in accordance with regulations, to have effect.
%
%(5) Regulations about the manner in which anything is approved for the purposes of this section may provide—
%
%($a$) for it to be treated as approved in accordance with the prescribed procedure where the Authority determine that prescribed conditions have been satisfied in relation to any departures from that procedure that have occurred; and
%
%($b$) for persons who do not object to it to be treated as having approved it.
%
%(6) Regulations may provide that, for the purposes of this section and any arrangements under this section, arrangements are to be taken as complying with the requirements of section 16 or 18, and as being consistent with those requirements, notwithstanding that nominations made for the purposes of the arrangements by a person or organisation which—
%
%($a$) represents for any particular purposes the interests of persons who are comprised in the membership of the scheme in question, and
%
%($b$) is of such a description as is specified in the regulations,
%
%are to be treated under the arrangements as nominations, or as the only nominations, made by qualifying members of the scheme.
%
%(7) Provision made by or under the preceding provisions of this section with respect to member-nominated trustees does not apply in the case of a trust scheme if—
%
%($a$) every member of the scheme is a trustee of the scheme and no other person is such a trustee; or
%
%($b$) every trustee of the scheme is a company.
%
%(8) Provision made by or under the preceding provisions of this section does not apply if the scheme is of a prescribed description.”
%
%(2) In section 68(2)($b$)  of that Act (power of trustee to modify scheme), for “17(2)” there shall be substituted “18A(1)”.
%
%(3) In section 117(2)($c$)  of that Act (overriding requirements), for “17(2)” there shall be substituted “18A(1)”.
%46Non-compliance in relation to arrangements or proposals
%
%(1) In section 21 of the [1995 c. 26. ] Pensions Act 1995 (consequences for trustees of failure to implement arrangements)—
%
%($a$) in subsections (1)  and (2), the words “, or the appropriate rules,” shall be omitted;
%
%($b$) in subsections (1)  and (3), for “17(2)”, in each place, there shall be substituted “18A(1)”;
%
%($c$) in subsection (2), for “19(2)”, in each place, there shall be substituted “18A(2)”;
%
%($d$) in subsection (3), the words “or rules” shall be omitted;
%
%($e$) in subsection (4), for “17(2), 18(1)  or 19(2)” there shall be substituted “18(1)  or 18A(1)  or (2)” and the words “(or further arrangements)” in paragraph ($a$), paragraph ($b$)  and the word “and” immediately preceding it shall be omitted;
%
%($f$) subsection (5)  shall cease to have effect;
%
%($g$) in subsection (6), for “20” there shall be substituted “18A”;
%
%($h$) in subsection (7), for “16 to 20” there shall be substituted “16 and 18” and the words “and this section”, paragraph ($b$)  and the word “and” immediately preceding paragraph ($b$)  shall be omitted;
%
%(i) in subsection (8)($a$), for the words from “of the appropriate” to “given” there shall be substituted “for the purposes of section 18A of proposed arrangements must be given, in accordance with regulations under that section,”; and
%
%($j$) paragraph ($b$)  of subsection (8)  and the word “and” immediately preceding it shall be omitted.
%
%(2) In subsection (1)  of that section, after paragraph ($b$)  there shall be inserted “or
%
%($c$) regulations under section 16(9) ($b$)  have not been complied with,”.
%
%(3) In subsection (2)  of that section, after paragraph ($b$)  there shall be inserted “or
%
%($c$) regulations under section 18(9) ($b$)  have not been complied with,”.
%
%(4) After subsection (2)  of that section there shall be inserted—
%
%“(2A) Section 10 applies to an employer who has made a proposal for the purposes of section 18A but who contravenes any requirements of any regulations under section 18A relating to the submission of that proposal for approval.”
%
%(5) After subsection (6)  there shall be inserted—
%
%“(6A) In sections 16 to 18A “company” means a company within the meaning given by section 735(1)  of the [1985 c. 6. ] Companies Act 1985 or a company which may be wound up under Part V of the [1986 c. 45. ] Insolvency Act 1986 (unregistered companies).”

\amendment{
Ss. 43--46 are not yet in force.
}

\subsection{\itshape Winding-up of schemes}

\subsubsection{47. Information to be given to the Authority}

(1) In section 22(1)  and (3)  of the Pensions Act 1995 (circumstances in which provisions apply to a trust scheme the employer in relation to which has been subjected to an insolvency procedure), for “26”, in each case, there shall be substituted “26A”.

(2) After section 26 of that Act there shall be inserted—
\begin{quotation}
\subsection*{“26A. Information to be given to the Authority in a s.\ 22 case}

(1) If at any time while section 22 applies in relation to a scheme—
\begin{enumerate}\item[]
($a$) the trustees of the scheme do not include at least one person who the practitioner or official receiver has informed them is a person about whose independent status he is satisfied, and

($b$) the trustees have no other reasonable grounds for believing that their number includes at least one person about whose independent status the practitioner or official receiver is satisfied,
\end{enumerate}
it shall be the duty of the trustees, as soon as reasonably practicable after it first appears to any one or more of them as mentioned in paragraphs ($a$)  and ($b$), to give notice to the Authority that the scheme appears not to have an independent trustee.

(2) If a trust scheme is without trustees at any time while section 22 applies to it, it shall be the duty of every person involved in the administration of the scheme, as soon as reasonably practicable after it first appears to him that the scheme is without trustees, to give notice to the Authority that the scheme has no trustees.

(3) No person shall be required to give a notice under subsection (1)  or (2)  at any time when it appears to him on reasonable grounds—
\begin{enumerate}\item[]
($a$) that it is the intention of the practitioner or official receiver, for the purpose of complying with his duty under section 23(1)($b$), to make or secure the appointment of any person as a trustee of the scheme; and

($b$) that the appointment will be made within the period specified by or under section 23(2)  for the performance of that duty.
\end{enumerate}

(4) No person shall be required to give a notice under subsection (2)  at any time when it appears to him, on reasonable grounds, that the Authority are already aware that the scheme has no trustees.

(5) Where the practitioner or official receiver at any time informs the trustees of a trust scheme that he is not, or is no longer, satisfied about a person’s independent status, no account shall be taken for the purposes of subsection (1)($a$)  of any information that he was so satisfied which was given by the practitioner or official receiver to the trustees before that time.

(6) References in this section to the practitioner or official receiver being satisfied about a person’s independent status are references to his being satisfied for the purposes of section 23 that that person is an independent person.

(7) If subsection (1)  is not complied with, section 10 applies to any trustee who has failed to take all such steps as are reasonable to secure compliance.

(8) Section 10 applies to any person who fails to comply with a duty imposed on him by subsection (2).

\subsection*{26B. Information to be given in cases where s.\ 22 disapplied}

(1) Where, at any time—
\begin{enumerate}\item[]
($a$) section 22 would apply in relation to a trust scheme but for regulations under section 118,

($b$) the employer in relation to the scheme is the sole trustee of the scheme,

($c$) there are persons involved in the administration of the scheme, and

($d$) none of those persons has received an employer’s assurance relating to the scheme,
\end{enumerate}
it shall be the duty of every person who is involved in the administration of the scheme, as soon as reasonably practicable after it first appears to him as mentioned in paragraphs ($a$)  and ($b$), to give notice to the Authority that the case is one falling within paragraphs ($a$)  to ($d$).

(2) For the purposes of this section a person has received an employer’s assurance relating to a scheme if during the period while section 22 would have applied in relation to the scheme but for regulations under section 118—
\begin{enumerate}\item[]
($a$) he has been informed by the person who is the employer in relation to the scheme that there is no reason why the employer should not continue to act as a trustee of the scheme;

($b$) he has not subsequently been informed by the person who is the employer in relation to the scheme that that has ceased to be the case; and

($c$) the trustees of the scheme have not changed since he was informed as mentioned in paragraph ($a$).
\end{enumerate}

(3) No person shall be required to give a notice under subsection (1)—
\begin{enumerate}\item[]
($a$) at any time when it appears to him, on reasonable grounds, that the Authority are already aware that the case is one falling within paragraphs ($a$)  to ($d$)  of that subsection;

($b$) if a period is prescribed for the purposes of this paragraph, at any time in the prescribed period after the event by virtue of which the scheme became a scheme in relation to which section 22 would apply but for regulations under section 118; or

($c$) at any other time that is prescribed for the purposes of this subsection.
\end{enumerate}

(4) Section 10 applies to any person who fails to comply with any duty imposed on him by subsection (1).

\subsection*{26C. Construction of ss.\ 26A and 26B}

(1) In sections 26A and 26B references, in relation to a scheme, to a person involved in the administration of the scheme are (subject to subsection (2)) references to any person who is so involved otherwise than as—
\begin{enumerate}\item[]
($a$) the employer in relation to that scheme;

($b$) a trustee of the scheme;

($c$) the auditor of the scheme or its actuary;

($d$) a legal adviser of the trustees of the scheme;

($e$) a fund manager for the scheme;

($f$) a person acting on behalf of a person who is involved in the administration of the scheme;

($g$) a person providing services to a person so involved;

($h$) a person acting in his capacity as an employee of a person so involved;

($i$) a person who would fall within any of paragraphs ($f$)  to ($h$)  if persons acting in relation to the scheme in any capacity mentioned in the preceding paragraphs were treated as involved in the administration of a scheme.
\end{enumerate}

(2) In this section references, in relation to a scheme, to a person involved in the administration of the scheme do not include references to persons of a particular description if regulations provide for persons of that description to be excluded from those references.

(3) If regulations so provide in relation to any provision of section 26A or 26B, so much of that provision as requires any notice to be given as soon as reasonably practicable after a particular time shall have effect as a requirement to give that notice within such period after that time as may be prescribed.”

(3) In subsection (2)  of section 118 of that Act (powers to provide for sections 22 to 26 not to apply in the case of certain schemes), for “sections 22 to 26” there shall be substituted “some or all of the provisions of sections 22 to 26C”.

(4) After that subsection there shall be inserted—
\begin{quotation}
“(3) Regulations may modify sections 26A and 26B for the purpose of requiring prescribed persons, in addition to or instead of the persons who (apart from the regulations) would be required to provide information to the Authority under those sections, to be subject to the duties imposed by those sections.”
\end{quotation}

(5) In section 178($b$)  of the Pension Schemes Act 1993 (regulations providing for who is to be treated as a trustee of a scheme), at the end there shall be inserted “or sections 22 to 26C of the Pensions Act 1995”.
\end{quotation}

\subsubsection{48. Modification of scheme to secure winding-up}

After section 71 of the Pensions Act 1995 (effect of modification orders under section 69) there shall be inserted—
\begin{quotation}
\subsection*{“71. AModification by Authority to secure winding-up}

(1) The Authority may at any time while—
\begin{enumerate}\item[]
($a$) an occupational pension scheme is being wound up, and

($b$) the employer in relation to the scheme is subject to an insolvency procedure,
\end{enumerate}
make an order modifying that scheme with a view to ensuring that it is properly wound up.

(2) The Authority shall not make such an order except on an application made to them, at a time such as is mentioned in subsection (1), by the trustees or managers of the scheme.

(3) Except in so far as regulations otherwise provide, an application for the purposes of this section must be made in writing.

(4) Regulations may make provision—
\begin{enumerate}\item[]
($a$) for the form and manner in which an application for the purposes of this section is to be made to the Authority;

($b$) for the matters which are to be contained in such an application;

($c$) for the documents which must be attached to an application for the purposes of this section or which must otherwise be delivered to the Authority with or in connection with any such application;

($d$) for persons to be required, before such time as may be prescribed, to give such notifications of the making of an application for the purposes of this section as may be prescribed;

($e$) for the matters which are to be contained in a notification of such an application;

($f$) for persons to have the opportunity, for a prescribed period, to make representations to the Authority about the matters to which such an application relates;

($g$) for the manner in which the Authority are to deal with any such application.
\end{enumerate}

(5) The power of the Authority to make an order under this section—
\begin{enumerate}\item[]
($a$) shall be limited to what they consider to be the minimum modification necessary to enable the scheme to be properly wound up; and

($b$) shall not include power to make any modification that would have a significant adverse effect on—
\begin{enumerate}\item[]
(i) the accrued rights of any member of the scheme; or

(ii) any person’s entitlement under the scheme to receive any benefit.
\end{enumerate}
\end{enumerate}

(6) A modification of an occupational pension scheme by an order of the Authority under this section shall be as effective in law as if—
\begin{enumerate}\item[]
($a$) it had been made under powers conferred by or under the scheme;

($b$) the modification made by the order were capable of being made in exercise of such powers notwithstanding any enactment, rule of law or rule of the scheme that would have prevented their exercise for the making of that modification; and

($c$) the exercise of such powers for the making of that modification would not have been subject to any enactment, rule of law or rule of the scheme requiring the implementation of any procedure or the obtaining of any consent in connection with the making of a modification.
\end{enumerate}

(7) Regulations may provide that, in prescribed circumstances, this section—
\begin{enumerate}\item[]
($a$) does not apply in the case of occupational pension schemes of a prescribed class or description; or

($b$) in the case of occupational pension schemes of a prescribed class or description applies with prescribed modifications.
\end{enumerate}

(8) The times when an employer in relation to an occupational pension scheme shall be taken for the purposes of this section to be subject to an insolvency procedure are—
\begin{enumerate}\item[]
($a$) in the case of a trust scheme, while section 22 applies in relation to the scheme; and

($b$) in the case of a scheme that is not a trust scheme, while section 22 would apply in relation to the scheme if it were a trust scheme;
\end{enumerate}
and for the purposes of this subsection no account shall be taken of modifications or exclusions contained in any regulations under section 118. 

(9) The Authority shall not be entitled to make an order under this section in relation to a public service pension scheme.”
\end{quotation}

\subsubsection{49. Reports about winding-up}

(1) After section 72 of the Pensions Act 1995 there shall be inserted—
\begin{quotation}
\section*{\itshape “Supervision of winding-up}

\subsection*{72A. Reports to Authority about winding-up}

(1) Where—
\begin{enumerate}\item[]
($a$) an occupational pension scheme is being wound up, and

($b$) the winding-up is one beginning at a time (whether before or after the passing of this Act) by reference to which regulations provide that it is to be a winding-up to which this section applies,
\end{enumerate}
it shall be the duty of the trustees or managers, in accordance with this section, to make periodic reports in writing to the Authority about the progress of the winding-up.

(2) In the case of each winding-up, the first report to be made under this section shall be made—
\begin{enumerate}\item[]
($a$) except in a case to which paragraph ($b$)  applies—
\begin{enumerate}\item[]
(i) after the end of the prescribed period beginning with the day on which the winding-up began; and

(ii) before the end of the prescribed period that begins with the end of the period that applies for the purposes of sub-paragraph (i);
\end{enumerate}
and

($b$) in a case where the winding-up began before the coming into force of the regulations which (for the purposes of subsection (1)($b$)) prescribe the time by reference to which the winding-up is one to which this section applies, before such date as may be prescribed by those regulations.
\end{enumerate}

(3) E%
%Subject to subsection (4), e
ach subsequent report made under this section in the case of a winding-up shall be made no more than twelve months after the date which 
%(apart from any postponement under subsection (4)) 
was the latest date for the making of the previous report required to be made in the case of that winding-up.

%(4) If, in the case of any report required to be made under subsection (3), the Authority consider (whether on an application made for the purpose or otherwise) that it would be appropriate to do so, they may, at any time before the latest time for the making of that report, postpone that latest time by such period as they think fit.
%
%(5) The latest time for making a report shall not be postponed under subsection (4)  by more than twelve months.
%
%(6) Subject to the application of the limit specified in subsection (5)  to the cumulative period of the postponements, more than one postponement may be made under subsection (4)  in the case of the same report.

(7) A report under this section—
\begin{enumerate}\item[]
($a$) must contain such information and statements as may be prescribed; and

($b$) must be made in accordance with the prescribed requirements.
\end{enumerate}

(8) Regulations may—
\begin{enumerate}\item[]
($a$) provide that, in prescribed circumstances, there shall be no obligation to make a report that would otherwise fall to be made under this section%;
%
%($b$) make provision for the period within which, and the manner in which, applications may be made for a postponement under subsection (4); and
%
%($c$) modify subsections (3)  and (5)  by substituting periods of different lengths for the periods for the time being specified in those subsections%
.
\end{enumerate}

(9) If there is any failure by the trustees or managers of any scheme to comply with their duty to make a report in accordance with the requirements imposed by or under this section—
\begin{enumerate}\item[]
($a$) section 3 applies, if the scheme is a trust scheme, to any trustee who has failed to take all such steps as are reasonable to secure compliance; and

($b$) section 10 applies (irrespective of the description of scheme involved) to any trustee or manager who has failed to take all such steps.”
\end{enumerate}
\end{quotation}

(2) In section 124 of that Act (interpretation of Part I), after subsection (3)  there shall be inserted—
\begin{quotation}
“(3A) In a case of the winding-up of an occupational pension scheme in pursuance of an order of the Authority under section 11 or of an order of a court, the winding-up shall (subject to subsection (3E)) be taken for the purposes of this Part to begin—
\begin{enumerate}\item[]
($a$) if the order provides for a time to be the time when the winding-up begins, at that time; and

($b$) in any other case, at the time when the order comes into force.
\end{enumerate}

(3B) In a case of the winding-up of an occupational pension scheme in accordance with a requirement or power contained in the rules of the scheme, the winding-up shall (subject to subsections (3C)  to (3E)) be taken for the purposes of this Part to begin—
\begin{enumerate}\item[]
($a$) at the time (if any) which under those rules is the time when the winding-up begins; and

($b$) if paragraph ($a$)  does not apply, at the earliest time which is a time fixed by the trustees or managers as the time from which steps for the purposes of the winding-up are to be taken.
\end{enumerate}

(3C) Subsection (3B)  shall not require a winding-up of a scheme to be treated as having begun at any time before the end of any period during which effect is being given—
\begin{enumerate}\item[]
($a$) to a determination under section 38 that the scheme is not for the time being to be wound up; or

($b$) to a determination in accordance with the rules of the scheme to postpone the commencement of a winding-up.
\end{enumerate}

(3D) In subsection (3B)($b$)  the reference to the trustees or managers of the scheme shall have effect in relation to any scheme the rules of which provide for a determination that the scheme is to be wound up to be made by persons other than the trustees or managers as including a reference to those other persons.

(3E) Subsections (3A)  to (3D)  above do not apply for such purposes as may be prescribed.”
\end{quotation}

(3) After section 49 of that Act (other responsibilities of trustees employers etc.)\ there shall be inserted—
\begin{quotation}
\subsection*{“49A. Record of winding-up decisions}

(1) Except so far as regulations otherwise provide, the trustees or managers of an occupational pension scheme shall keep written records of—
\begin{enumerate}\item[]
($a$) any determination for the winding-up of the scheme in accordance with its rules;

($b$) decisions as to the time from which steps for the purposes of the winding-up of the scheme are to be taken;

($c$) determinations under section 38;

($d$) determinations in accordance with the rules of the scheme to postpone the commencement of a winding-up of the scheme.
\end{enumerate}

(2) For the purpose of this section—
\begin{enumerate}\item[]
($a$) the determinations and decisions of which written records must be kept under this section include determinations and decisions by persons who—
\begin{enumerate}\item[]
(i) are not trustees or managers of a scheme, but

(ii) are entitled, in accordance with the rules of a scheme, to make a determination for its winding-up;
\end{enumerate}
and

($b$) regulations may, in relation to such determinations or decisions as are mentioned in paragraph ($a$), impose obligations to keep written records on the persons making the determinations or decisions (as well as, or instead of, on the trustees or managers).
\end{enumerate}

(3) Regulations may provide for the form and content of any records that are required to be kept under this section.

(4) Section 3 applies to any trustee of a scheme who fails to take all such steps as are reasonable to secure compliance by the trustees of that scheme with the obligations imposed on them by this section.

(5) Section 10 applies to any trustee or manager of a scheme who fails to take all such steps as are reasonable to secure compliance by the trustees or managers of that scheme with those obligations.”
\end{quotation}

\amendment{
S. 49(1) is only partially in force (see the Child Support, Pensions and Social Security Act 2000 (Commencement No. 11) Order 2002 art. 3(1)($b$) --(d)).
}

\subsubsection{50. Directions for facilitating winding-up}

After the section 72A inserted in the Pensions Act 1995 by section 49 there shall be inserted—
\begin{quotation}
\subsection*{“72B. Directions by Authority for facilitating winding-up}

(1) Subject to the following provisions of this section, the Authority shall have power, at any time after the winding-up of an occupational pension scheme has begun, to give directions under this section if they consider that the giving of the direction is appropriate on any of the grounds set out in subsection (2).

(2) Those grounds are—
\begin{enumerate}\item[]
($a$) that the trustees or managers of the scheme are not taking all the steps in connection with the winding-up that the Authority consider would be being taken if the trustees or managers were acting reasonably;

($b$) that steps being taken by the trustees or managers for the purposes of the winding-up involve things being done with what the Authority consider to be unreasonable delay;

($c$) that the winding-up is being obstructed or unreasonably delayed by the failure of any person—
\begin{enumerate}\item[]
(i) to provide information to the trustees or managers;

(ii) to provide information to a person involved in the administration of the scheme;

(iii) to provide information to a person of a prescribed description; or

(iv) to take any step (other than the provision of information) that he has been asked to take by the trustees or managers;
\end{enumerate}

($d$) that the winding-up would be likely to be facilitated or accelerated by the taking by any person other than the trustees or managers of any other steps;

($e$) that in any prescribed circumstances not falling within paragraphs ($a$)  to ($d$)—
\begin{enumerate}\item[]
(i) the provision by any person of any information to the trustees or managers or to any other person, or

(ii) the taking of any other step by any person,
\end{enumerate}
would be likely to facilitate or accelerate the progress of the winding-up.
\end{enumerate}

(3) Except in prescribed circumstances, the power of the Authority to give a direction under this section in the case of a winding-up shall be exercisable only where—
\begin{enumerate}\item[]
($a$) periodic reports about the progress of the winding-up are required to be made under section 72A; and

($b$) the first report that has to be made for the purposes of that section in the case of that winding-up either has been made or should have been made.
\end{enumerate}

(4) Regulations may provide that, in prescribed circumstances, the Authority shall not give a direction on the ground set out in subsection (2)($e$)  except in response to an application made by the trustees or managers of the scheme for the giving of a direction on that ground.

(5) A direction under this section is a direction in writing given to and imposing requirements on—
\begin{enumerate}\item[]
($a$) any or all of the trustees or managers of the scheme;

($b$) a person who is involved in its administration; or

($c$) a person of a prescribed description.
\end{enumerate}

(6) The requirements that may be imposed by a direction under this section are any requirement for the person to whom it is given, within such period specified in the direction as the Authority may consider reasonable—
\begin{enumerate}\item[]
($a$) to provide the trustees or managers with all such information as may be specified or described in the direction;

($b$) to provide a person involved in the administration of the scheme with all such information as may be so specified or described;

($c$) to provide a person who is of a prescribed description with all such information as may be so specified or described;

($d$) to take such steps (other than the provision of information) as may be so specified or described.
\end{enumerate}

(7) If, at any time before the end of a period within which any step is required by a direction under this section to be taken by any person, the Authority consider (whether on an application made for the purpose or otherwise) that it would be appropriate to do so, they may extend (or further extend) that period until such time as they think fit.

(8) Regulations may—
\begin{enumerate}\item[]
($a$) impose limitations on the steps that a person may be required to take by a direction under this section;

($b$) make provision for the period within which, and the manner in which, applications may be made for a period to be extended (or further extended) under subsection (7).
\end{enumerate}

(9) In this section references, in relation to a scheme, to a person involved in the administration of the scheme are (subject to subsection (10)) references to any person who is so involved otherwise than as—
\begin{enumerate}\item[]
($a$) the employer in relation to that scheme;

($b$) a trustee or manager of the scheme;

($c$) the auditor of the scheme or its actuary;

($d$) a legal adviser of the trustees or managers of the scheme;

($e$) a fund manager for the scheme;

($f$) a person acting on behalf of a person who is involved in the administration of the scheme;

($g$) a person providing services to a person so involved;

($h$) a person acting in his capacity as an employee of a person so involved;

($i$) a person who would fall within any of paragraphs ($f$)  to ($h$)  if persons acting in relation to the scheme in any capacity mentioned in the preceding paragraphs were treated as involved in the administration of a scheme.
\end{enumerate}

(10) In this section references, in relation to a scheme, to a person involved in the administration of the scheme do not include references to persons of a particular description if regulations provide for persons of that description to be excluded from those references.

\subsection*{72C. Duty to comply with directions under s.\ 72B}

(1) It shall be the duty of any person to whom a direction is given under section 72B to comply with it.

(2) Where a direction is given under section 72B to the trustees of a trust scheme, section 3 applies to any trustee who fails, without reasonable excuse, to take all such steps as are reasonable to secure compliance with it.

(3) Section 10 applies to any trustee or manager of a scheme who fails, without reasonable excuse, to take all such steps as are reasonable to secure compliance by the trustees or managers of that scheme with any direction given to them under section 72B.

(4) Section 10 applies to any person who—
\begin{enumerate}\item[]
($a$) is a person to whom a direction under section 72B is given otherwise than in the capacity of a trustee or manager; and

($b$) without reasonable excuse, fails to comply with that direction.
\end{enumerate}

(5) For the purposes of this section it shall not be a reasonable excuse in relation to any failure to provide information in pursuance of a direction under section 72B that the provision of that information would (but for the duty imposed by subsection (1)  of this section) involve a breach by any person of a duty owed to another not to disclose that information.”
\end{quotation}

\subsection{\itshape Other provisions}

\subsubsection{51. Restriction on index-linking where annuity tied to investments}

(1) In section 51(2)  of the Pensions Act 1995 (annual increases in rate of pension), for “Subject to section 52” there shall be substituted “Subject to sections 51A and 52”.

(2) After section 51 of that Act there shall be inserted—
\begin{quotation}
\subsection*{“51A. Restriction on increase where annuity tied to investments}

(1) No increase under section 51 is required to be made, at any time on or after the relevant date, of so much of any pension under a money purchase scheme as—
\begin{enumerate}\item[]
($a$) is payable by way of an annuity the amount of which for any year after the first year of payment is determined (whether under the terms of the scheme or under the terms of the annuity contract in pursuance of which it is payable) by reference to fluctuations in the value of, or the return from, particular investments;

($b$) does not represent benefits payable in respect of the protected rights of any member of the scheme; and

($c$) satisfies such other conditions (if any) as may be prescribed.
\end{enumerate}

(2) For the purposes of this section it shall be immaterial whether the annuity in question is payable out of the funds of the scheme in question or under an annuity contract entered into for the purposes of the scheme.

(3) In this section “the relevant date” means the date appointed for the coming into force of section 51 of the Child Support, Pensions and Social Security Act 2000.”
\end{quotation}

\subsubsection{52. Information for members of schemes etc}

(1) In subsection (1)  of section 113 of the Pension Schemes Act 1993 (regulations as to information to be provided to scheme members etc.), for the word “and” at the end of paragraph ($c$)  there shall be substituted—
\begin{quotation}
“($ca$) of the pensions and other benefits an entitlement to which would be likely to accrue to the member, or be capable of being secured by him, in respect of the rights that may arise under it; and”.
\end{quotation}

(2) After subsection (3)  of that section there shall be inserted—
\begin{quotation}
“(3A) The regulations may provide for the information that must be given to be determined, in whole or part, by reference to guidance which—
\begin{enumerate}\item[]
($a$) is prepared and from time to time revised by a prescribed body; and

($b$) is for the time being approved by the Secretary of State.%
”
\end{enumerate}
%
%(3B) The regulations may, in relation to cases where a scheme is being wound up, contain—
%\begin{enumerate}\item[]
%($a$) provision conferring power on the Regulatory Authority, at times before the period expires, to extend any period specified in the regulations as the period within which a requirement imposed by the regulations must be complied with; and
%
%($b$) provision as to the contents of any application for the exercise of such a power and as to the form and manner in which, and the time within which, any such application must be made.”
%\end{enumerate}
\end{quotation}

\amendment{
The insertion of s. 113(3B)  into the Pension Schemes Act 1993 by s. 52(2) is not yet in force.
}

\subsubsection{53. Jurisdiction of the Pensions Ombudsman}

(1) Section 146 of the Pension Schemes Act 1993 (functions of the Pensions Ombudsman) shall be amended as follows.

(2) In subsection (1), after paragraph ($b$)  there shall be inserted—
\begin{quotation}
“($ba$) a complaint made to him by or on behalf of an independent trustee of a trust scheme who, in connection with any act or omission which is an act or omission either—
\begin{enumerate}\item[]
(i) of trustees of the scheme who are not independent trustees, or

(ii) of former trustees of the scheme who were not independent trustees,
\end{enumerate}
alleges maladministration of the scheme,”.
\end{quotation}

(3) In that subsection, for the words after sub-paragraph (ii)  of paragraph ($d$)  there shall be substituted—
\begin{quotation}
“and in a case falling within sub-paragraph (ii)  references in this Part to the scheme to which the reference relates are references to each of the schemes,

($e$) any dispute not falling within paragraph ($f$)  between different trustees of the same occupational pension scheme,

($f$) any dispute, in relation to a time while section 22 of the Pensions Act 1995 (schemes subject to insolvency procedures) applies in relation to an occupational pension scheme, between an independent trustee of the scheme and either—
\begin{enumerate}\item[]
\begin{sloppypar}
(i) trustees of the scheme who are not independent trustees, or
\end{sloppypar}

(ii) former trustees of the scheme who were not independent trustees, and
\end{enumerate}

($g$) any question relating, in the case of an occupational pension scheme with a sole trustee, to the carrying out of the functions of that trustee.”
\end{quotation}

(4) After that subsection there shall be inserted—
\begin{quotation}
“(1A) The Pensions Ombudsman shall not investigate or determine any dispute or question falling within subsection (1)($c$)  to ($g$)  unless it is referred to him—
\begin{enumerate}\item[]
($a$) in the case of a dispute falling within subsection (1)($c$), by or on behalf of the actual or potential beneficiary who is a party to the dispute,

($b$) in the case of a dispute falling within subsection (1)($d$), by or on behalf of any of the parties to the dispute,

($c$) in the case of a dispute falling within subsection (1)($e$), by or on behalf of at least half the trustees of the scheme,

($d$) in the case of a dispute falling within subsection (1)($f$), by or on behalf of the independent trustee who is a party to the dispute,

($e$) in the case of a question falling within subsection (1)($g$), by or on behalf of the sole trustee.
\end{enumerate}

(1B) For the purposes of this Part, any reference to or determination by the Pensions Ombudsman of a question falling within subsection (1)($g$)  shall be taken to be the reference or determination of a dispute.”
\end{quotation}

(5) In subsection (3)  (persons responsible for the management of the scheme to be the trustees and managers and employer), after “occupational pension scheme” there shall be inserted “or a personal pension scheme”.

(6) For paragraph ($a$)  of subsection (6)  (exclusion of the Ombudsman’s jurisdiction where court proceedings have been begun) there shall be substituted—
\begin{quotation}
“($a$) if, before the making of the complaint or the reference of the dispute—
\begin{enumerate}\item[]
(i) proceedings in respect of the matters which would be the subject of the investigation have been begun in any court or employment tribunal, and

(ii) those proceedings are proceedings which have not been discontinued or which have been discontinued on the basis of a settlement or compromise binding all the persons by or on whose behalf the complaint or reference is made;”.
\end{enumerate}
\end{quotation}

(7) In subsection (7)  (persons who are actual or potential beneficiaries)—
\begin{enumerate}\item[]
($a$) after paragraph ($b$)  there shall be inserted—
\begin{quotation}
“($ba$) a person who is entitled to a pension credit as against the trustees or managers of the scheme;” and
\end{quotation}

($b$) in sub-paragraph (i)  of paragraph ($c$), for “paragraph ($a$)  or ($b$)” there shall be substituted “paragraph ($a$), ($b$)  or ($ba$)”.
\end{enumerate}

(8) In subsection (8)  (interpretation) after the definition of “employer” there shall be inserted—
\begin{quotation}
““independent trustee”, in relation to a scheme, means—
\begin{enumerate}\item[]
($a$) a trustee of the scheme appointed under section 23(1)($b$)  of the Pensions Act 1995 (appointment of independent trustee by insolvency practitioner or official receiver),

($b$) a person appointed under section 7(1)  of that Act to replace a trustee falling within paragraph ($a$)  or this paragraph;”.
\end{enumerate}
\end{quotation}

(9) In subsection (1)—
\begin{enumerate}\item[]
($a$) for “complaints and disputes” there shall be substituted “matters”;

($b$) in paragraph ($b$), for the words from “is to” to the end of the paragraph there shall be substituted “are references to the other scheme referred to in that sub-paragraph”; and

($c$) in paragraphs ($c$)  and ($d$), the words “which arises”, in each place where they occur, shall be omitted.
\end{enumerate}

(10) Subsection (6)  does not have effect in relation to proceedings begun before the day appointed under section 86 for the coming into force of this section.

\subsubsection{54. Investigations by the Pensions Ombudsman}

(1) The Pension Schemes Act 1993 shall be amended as follows.

(2) In section 148(5)  (meaning of parties to an investigation for the purposes of staying proceedings), after paragraph ($b$)  there shall be inserted—
\begin{quotation}
“($ba$) any actual or potential beneficiary of the scheme whose interests are or may be affected by the matters to which the complaint or dispute relates,

($bb$) any actual or potential beneficiary of the scheme whose interests it is reasonable to suppose might be affected by—
\begin{enumerate}\item[]
(i) the Pensions Ombudsman’s determination of the complaint or dispute, or

(ii) directions that may be given by the Ombudsman in consequence of that determination,”.
\end{enumerate}
\end{quotation}

(3) For subsection (1)  of section 149 (obligation to allow persons to comment on allegations in complaint or reference) there shall be substituted—
\begin{quotation}
“(1) Where the Pensions Ombudsman proposes to conduct an investigation into a complaint made or dispute referred under this Part, he shall—
\begin{enumerate}\item[]
($a$) give every person against whom allegations are made in the complaint or reference an opportunity to comment on those allegations,

($b$) give every person responsible for the management of the scheme to which the complaint or reference relates an opportunity to make representations to him about the matters to which the complaint or dispute relates, and

($c$) give every actual or potential beneficiary of that scheme whose interests are or may be affected by the matters to which the complaint or dispute relates, an opportunity to make representations about those matters.
\end{enumerate}

(1A) Subject to subsection (1B), subsection (1)  shall not require an opportunity to make comments or representations to be given to any person if the Pensions Ombudsman is satisfied that that person is—
\begin{enumerate}\item[]
($a$) a person who, as the person or one of the persons making the complaint or reference, has had his opportunity to make comments or representations about the matters in question; or

($b$) a person whose interests in relation to the matters to which the complaint or dispute relates are being represented, in accordance with rules under this section, by a person who has been given an appropriate opportunity to make comments or representations.
\end{enumerate}

(1B) The Pensions Ombudsman shall, under subsection (1), give an opportunity to make comments and representations to a person falling within subsection (1A)($a$)  in any case in which that person is a person who, in accordance with rules, is appointed or otherwise determined, after the making of the complaint or reference, to represent the interests of other persons in relation to the matters to which the complaint or dispute relates.”
\end{quotation}

(4) In subsection (3)  of section 149 (matters as to which rules may be made), for “and” at the end of paragraph ($b$)  there shall be substituted—
\begin{quotation}
“($ba$) for the interests of all of a number of persons who—
\begin{enumerate}\item[]
(i) are actual or potential beneficiaries of the scheme to which the complaint or reference relates, and

(ii) appear to have the same interest in relation to any of the matters to which the complaint or dispute relates,
\end{enumerate}
to be represented for the purposes of the investigation by such one or more of them, or such other person, as may be appointed by the Ombudsman or otherwise determined in accordance with the rules,”.
\end{quotation}

(5) In that subsection, at the end of paragraph ($c$), there shall be inserted “and
\begin{quotation}
($d$) for the payment of legal expenses incurred by a party to an investigation (as defined in section 148(5)) out of funds held for the purposes of the scheme to which the complaint or reference relates.”
\end{quotation}

(6) After subsection (7)  of section 149 there shall be inserted—
\begin{quotation}
“(8) References in this section to the matters to which a complaint or dispute relates include references to any matter which it is reasonable to suppose might form the subject of—
\begin{enumerate}\item[]
($a$) the Pensions Ombudsman’s determination of the complaint or dispute, or

($b$) any directions that may be given by the Ombudsman in consequence of that determination.”
\end{enumerate}
\end{quotation}

(7) In subsection (1)  of section 151 (persons to be given notice of a determination by the Ombudsman), at the end of paragraph ($b$)  there shall be inserted “and
\begin{quotation}
($c$) to every other person who was required under section 149 to be given an opportunity—
\begin{enumerate}\item[]
(i) to comment on an allegation in the complaint or reference, or

(ii) to make representations about matters to which the complaint or reference relates,”.
\end{enumerate}
\end{quotation}

(8) In subsection (3)  of section 151 (persons bound by determinations), for “and” at the end of paragraph ($b$)  there shall be substituted—
\begin{quotation}
“($ba$) any person who under section 149 was given such an opportunity to make any such comment or representation as is mentioned in subsection (1)($c$)  of this section,

($bb$) any person whose interests were represented by a person falling within any of the preceding paragraphs, and”;
\end{quotation}
and, in paragraph ($c$)  of that subsection for “paragraph ($a$)  or ($b$)” there shall be substituted “any of paragraphs ($a$)  to ($bb$) ”;

(9) Nothing in any provision made by this section shall—
\begin{enumerate}\item[]
($a$) apply in relation to any complaint or reference made to the Pensions Ombudsman before the day on which this section comes into force; or

($b$) authorise the making of any provision applying in relation to any such complaint or reference.
\end{enumerate}

\amendment{
S. 54 is in force only for the purpose of making regulations and rules.
}

\subsubsection{55. Prohibition on different rules for overseas residents etc}

After section 66 of the Pensions Act 1995 there shall be inserted—
\begin{quotation}
\section*{\itshape “Treatment of overseas residents etc.}

\subsection*{66A. Prohibition on different rules for overseas residents etc}

(1) This section applies where an occupational pension scheme contains provisions contravening subsection (2)  or (3).

(2) Except so far as regulations otherwise provide, provisions of an occupational pension scheme contravene this subsection to the extent that they would (apart from this section) have an effect with respect to—
\begin{enumerate}\item[]
($a$) the entitlement of any person to benefits under the scheme, or

($b$) the payment to any person of benefits under the scheme,
\end{enumerate}
which would be different according to whether or not a place outside the United Kingdom is specified by that person as the place to which he requires payments of benefits under the scheme to be made to him.

(3) Except so far as regulations otherwise provide, provisions of an occupational pension scheme contravene this subsection to the extent that they would (apart from this section) have an effect with respect to—
\begin{enumerate}\item[]
($a$) the entitlement of any person to remain a member of the scheme,

($b$) the eligibility of any person to remain a person by or in respect of whom contributions are made towards or under the scheme, or

($c$) the making by or in respect of any person who is a member of the scheme of any contributions towards or under the scheme,
\end{enumerate}
which would be different according to whether that person works wholly in the United Kingdom or wholly or partly outside the United Kingdom.

(4) Provisions contravening subsection (2)  shall have effect, in relation to all times after the coming into force of section 55 of the Child Support, Pensions and Social Security Act 2000, as if they made the same provision in relation to a person who requires payments of benefits to be made to a place outside the United Kingdom as they make in relation to a person in whose case all payments of benefits fall to be made to a place in the United Kingdom.

(5) Provisions contravening subsection (3)  shall have effect, in relation to all times after the coming into force of section 55 of the Child Support, Pensions and Social Security Act 2000, as if they made the same provision in relation to persons working wholly or partly outside the United Kingdom as they make in relation to persons working wholly in the United Kingdom.

(6) This section—
\begin{enumerate}\item[]
($a$) shall be without prejudice to any enactment under which any amount is to be or may be deducted, or treated as deducted, from amounts payable by way of benefits under the scheme or treated as so payable; and

($b$) shall not apply in relation to so much of any provision of a scheme as is required for securing compliance with the conditions of any approval, exemption or relief given or available under the Tax Acts.”
\end{enumerate}
\end{quotation}

\subsubsection[56. Miscellaneous amendments and alternative to anti-franking rules]{\sloppy 56. Miscellaneous amendments and alternative to anti-franking rules}

Schedule 5 (which contains miscellaneous amendments of the Pension Schemes Act 1993 and the Pensions Act 1995 and makes provision for an alternative to the anti-franking rules in Part III of that Act of 1993) shall have effect.

\section[Chapter III --- War Pensions]{Chapter III\\*War Pensions}

\renewcommand\parthead{--- Part II Chapter III}

\subsection{57. Rights of appeal}

(1) After section 5 of the Pensions Appeal Tribunals Act 1943 there shall be inserted—
\begin{quotation}
\subsection*{“5A. Appeals in other cases}

(1) Where, in the case of any such claim as is referred to in section 1, 2 or 3 of this Act, the Minister makes a specified decision—
\begin{enumerate}\item[]
($a$) he shall notify the claimant of the decision, specifying the ground on which it is made, and

($b$) thereupon an appeal against the decision shall lie to the Tribunal on the issue whether the decision was rightly made on that ground.
\end{enumerate}

(2) For the purposes of subsection (1), a “specified decision” is a decision (other than a decision which is capable of being the subject of an appeal under any other provision of this Act) which is of a kind specified by the Minister in regulations made by statutory instrument.

(3) Regulations under this section shall not be made unless a draft of the regulations has been laid before, and approved by a resolution of, each House of Parliament.”
\end{quotation}

(2) In subsection (2)  of section 6 of that Act (further appeals to High Court in cases of appeals brought under sections 1 to 4), for the words from “section one” to “four” there shall be substituted “sections 1, 2, 3, 4 or 5A”.

(3) In subsection (2A)  of that section (setting aside of decisions of Tribunal on appeals under sections 1, 2, 3 or 4), for “or 4” there shall be substituted “, 4 or 5A”.

(4) Section 1(2)  of the Pensions Appeal Tribunals Act 1949 (no right of appeal against rejection of claims relating to service before 3rd September 1939) shall cease to have effect.

\subsection{58. Time limit for appeals}

(1) In section 8 of the Pensions Appeal Tribunals Act 1943 (time limit for appeals), in subsection (1)  (notice of appeal to be given within twelve months of notification of decision or assessment), for the words from “twelve months after” to “in any other case,” there shall be substituted “six months after”.

(2) After subsection (3)  of that section there shall be inserted—
\begin{quotation}
“(4) The Minister may by regulations made by statutory instrument amend subsections (1)  and (3)  so as to substitute a different number of months for any number of months specified there.

(5) The Minister may by regulations made by statutory instrument provide that the Tribunal may, in circumstances prescribed in the regulations, allow an appeal to be brought not later than twelve months after the end of any period limited by this section.

(6) Regulations under subsection (4)  or (5)  shall not be made unless a draft of the regulations has been laid before, and approved by a resolution of, each House of Parliament.”
\end{quotation}

(3) Subsection (1)  shall not have effect in relation to—
\begin{enumerate}\item[]
($a$) decisions from which an appeal lies to the Tribunal under sections 1 to 4 of the Pensions Appeal Tribunals Act 1943 and which are made before the day on which that subsection comes into force, or

($b$) decisions or assessments from which an appeal lies to the Tribunal under section 5(2)  of that Act and which are made before the day on which that subsection comes into force.
\end{enumerate}

(4) In relation to decisions falling within subsection (3)($a$)  of this section, section 8 of the Pensions Appeal Tribunals Act 1943 shall have effect as if for paragraphs ($a$)  to ($c$)  of subsection (1)  of that section there were substituted “the day on which section 58(1)  of the Child Support, Pensions and Social Security Act 2000 came into force”.

(5) In section 6(1)  of the War Pensions Act 1921 (notice of appeal to be given within twelve months of notification of rejection of claim), for “twelve” there shall be substituted “six”.

(6) Subsection (5)  shall not have effect in relation to any appeal if the decision or assessment appealed against was made before the day on which that subsection comes into force.

\subsection{59. Matters relevant on appeal to Pensions Appeal Tribunal}

Before section 6 of the Pensions Appeal Tribunals Act 1943 (constitution, jurisdiction and procedure of Pensions Appeal Tribunal), there shall be inserted—
\begin{quotation}
\subsection*{“5B. Matters relevant on appeal}

In deciding any appeal, a Pensions Appeal Tribunal—
\begin{enumerate}\item[]
($a$) need not consider any issue that is not raised by the appellant or the Minister in relation to the appeal; and

($b$) shall not take into account any circumstances not obtaining at the time when the decision appealed against was made.”
\end{enumerate}
\end{quotation}

\subsection{60. Constitution and procedure of Pensions Appeal Tribunals}

(1) In sub-paragraph (2)  of paragraph 2 of the Schedule to the Pensions Appeal Tribunals Act 1943 (remuneration for members of Pensions Appeal Tribunals), after “remuneration” there shall be inserted “and allowances”.

(2) After that sub-paragraph there shall be inserted—
\begin{quotation}
“(2A) Subject to sub-paragraphs (3)  and (4)  below, a member of such a Tribunal shall hold and vacate his office in accordance with the terms of his appointment, but shall be eligible for reappointment.”
\end{quotation}

(3) After paragraph 2 of that Schedule, there shall be inserted—
\begin{quotation}
“2A.---(1) The Lord Chancellor shall ensure that the appointments made by him under paragraph 2 above have the effect, in the case of each of the Tribunals, that the persons holding office as members of that Tribunal at all times include—
\begin{enumerate}\item[]
($a$) persons who are legally qualified;

($b$) persons who are medically qualified;

($c$) persons with knowledge or experience of service in Her Majesty’s naval, military or air forces; and

($d$) other persons.
\end{enumerate}

(2) For the purposes of this Schedule a person is legally qualified if—
\begin{enumerate}\item[]
($a$) he has a seven year general qualification within the meaning of section 71 of the Courts and Legal Services Act 1990;

($b$) he is an advocate or solicitor in Scotland of at least seven years' standing; or

($c$) he is a member of the Bar of Northern Ireland or solicitor of the Supreme Court of Northern Ireland of at least seven years' standing.
\end{enumerate}

(3) For the purposes of this Schedule a person is medically qualified if he is a duly qualified medical practitioner of at least seven years' standing.

(4) In making any appointment under paragraph 2 it shall be the duty of the Lord Chancellor to have regard to the desirability of having as members of the Tribunals persons with knowledge or experience of matters relating to the disability of persons.

\medskip

2B.---(1) A President of Pensions Appeal Tribunals and a Deputy President of Pensions Appeal Tribunals may be appointed for each part of the United Kingdom.

(2) The person entitled to appoint a person under this paragraph to be a President or Deputy President of Pensions Appeal Tribunals shall be—
\begin{enumerate}\item[]
($a$) in the case of an appointment for England and Wales, the Lord Chancellor;

($b$) in the case of an appointment for Scotland, the Lord President of the Court of Session; and

($c$) in the case of an appointment for Northern Ireland, the Lord Chief Justice of Northern Ireland.
\end{enumerate}

(3) Only legally qualified members of a Pensions Appeal Tribunal shall be eligible for appointment under this paragraph.

(4) A person shall cease to be President or Deputy President of Pensions Appeal Tribunals if he ceases to be a member of any such Tribunal.

(5) The Deputy President of Pensions Appeal Tribunals for any part of the United Kingdom shall carry out such of the functions of the President for that part of the United Kingdom as that President assigns to him.

(6) If at any time the President of Pensions Appeal Tribunals for any part of the United Kingdom is temporarily unable to carry out his functions under this Schedule, those functions shall be carried out by the Deputy President for that part of the United Kingdom.”
\end{quotation}

(4) For paragraph 3 of that Schedule (constitution of Tribunal for particular hearings) there shall be substituted—
\begin{quotation}
“3. The members of the Tribunal hearing a particular appeal shall in every case include a legally qualified member; and only a legally qualified member may preside as chairman for the hearing of any appeal.

\medskip

3A.---(1) The President of Pensions Appeal Tribunals for any part of the United Kingdom may give directions as to—
\begin{enumerate}\item[]
($a$) the number of members of the Tribunal who should hear an appeal in that part of the United Kingdom;

($b$) the extent to which the members hearing such an appeal must include—
\begin{enumerate}\item[]
(i) medically qualified persons; and

(ii) persons who are neither legally qualified nor medically qualified;
\end{enumerate}

($c$) the extent to which in the case of such an appeal the members hearing it must include persons satisfying other requirements specified by the President;

($d$) the manner of determining the members who are to serve as the chairman and members of the Tribunal for the hearing of such an appeal.
\end{enumerate}

\medskip

3B. The President of Pensions Appeal Tribunals for any part of the United Kingdom may give directions as to the practice and procedure to be followed by such Tribunals in that part of the United Kingdom.

\medskip

3C.---(1) The power to give directions under paragraphs 3A and 3B shall be exercisable in relation to a particular appeal, to a category of appeal or to appeals generally.

(2) If at any time there is, in the case of any part of the United Kingdom, neither a President of Pensions Appeal Tribunals nor a Deputy President, the power of the President to give directions under paragraphs 3A and 3B above shall be exercisable—
\begin{enumerate}\item[]
($a$) in the case of England and Wales, by the Lord Chancellor;

($b$) in the case of Scotland, by the Lord President of the Court of Session; and

($c$) in the case of Northern Ireland, by the Lord Chief Justice of Northern Ireland.
\end{enumerate}

(3) The power to give directions under paragraphs 3A and 3B above includes power to vary or revoke directions previously given.”
\end{quotation}

(5) In Schedule 11 to the Courts and Legal Services Act 1990 (judges barred from legal practice), at the end there shall be inserted “Member of a Pensions Appeal Tribunal”.

\subsection{61. Composition of central advisory committee}

(1) In section 9 of the Chronically Sick and Disabled Persons Act 1970 (central advisory committee on war pensions to include chairmen of not less than twelve of the war pensions committees), in subsection (1), for “chairmen of not less than twelve” there shall be substituted “chairman of at least one”.

(2) In section 3 of the War Pensions Act 1921 (constitution of central advisory committee), for “representatives of any committees” there shall be substituted “at least one person from one of the committees”.

\addtocontents{toc}{\protect\pagebreak[3]}

\part[Part III --- Social Security]{Part III\\*Social Security}

\renewcommand\parthead{--- Part III}

\section{\itshape Loss of benefit}

\subsection{62. Loss of benefit for breach of community order}

(1) If—
\begin{enumerate}\item[]
($a$) a court makes a determination that a person (“the offender”) has failed without reasonable excuse to comply with the requirements of a relevant community order made in respect of him,

($b$) the Secretary of State is notified in accordance with regulations under section 64 of the determination, and

($c$) the offender is a person with respect to whom the conditions for any entitlement to a relevant benefit are or become satisfied,
\end{enumerate}
then, even though those conditions are satisfied, the following restrictions shall apply in relation to the payment of that benefit in the offender’s case.

(2) Subject to subsections (3)  to (5), the relevant benefit shall not be payable in the offender’s case for the prescribed period.

(3) Where the relevant benefit is income support, the benefit shall be payable in the offender’s case for the prescribed period as if the applicable amount used for the determination under section 124(4)  of the Social Security Contributions and Benefits Act 1992 of the amount of the offender’s entitlement for that period were reduced in such manner as may be prescribed.

(4) The Secretary of State may by regulations provide that, where the relevant benefit is jobseeker’s allowance, any income-based jobseeker’s allowance shall be payable, during the whole or a part of the prescribed period, as if one or more of the following applied—
\begin{enumerate}\item[]
($a$) the rate of the allowance were such reduced rate as may be prescribed;

($b$) the allowance were payable only if there is compliance by the offender with such obligations with respect to the provision of information as may be imposed by the regulations;

($c$) the allowance were payable only if the circumstances are otherwise such as may be prescribed.
\end{enumerate}

(5) Where the relevant benefit is a payment under section 2 of the Employment and Training Act 1973 (under which training allowances are payable), that benefit shall not be payable for the prescribed period except to such extent (if any) as may be prescribed.

(6) Where the determination by a court that was made in the offender’s case is quashed or otherwise set aside by the decision of that or any other court, all such payments and other adjustments shall be made in his case as would be necessary if the restrictions imposed by or under this section in respect of that determination had not been imposed.

(7) The length of any period prescribed for the purposes of any of subsections (2)  to (5)  shall not exceed twenty-six weeks.

(8) In this section—
\begin{enumerate}\item[]
    “income-based jobseeker’s allowance” and “joint-claim jobseeker’s allowance” have the same meanings as in the Jobseekers Act 1995;

    “relevant benefit” means—
\begin{enumerate}\item[]
    ($a$) 
    income support;

    ($b$) 
    any jobseeker’s allowance other than joint-claim jobseeker’s allowance;

    ($c$) 
    any benefit under the Social Security Contributions and Benefits Act 1992 (other than income support) which is prescribed for the purposes of this section; or

    ($d$) 
    any prescribed payment under section 2 of the Employment and Training Act 1973 (under which training allowances are payable);
\end{enumerate}

%    “relevant community order” means—
%\begin{enumerate}\item[]
%%    ($a$) 
%%    a community service order;
%%
%%    ($b$) 
%%    a probation order;
%%
%%    ($c$) 
%%    a combination order;
%
%% Paras ($a$) --(c) in definition of ``relevant community order'' in s 62(8) substituted (1.4.01) by 2000 c 43 Sch 7 para 206($a$) 
%($a$) a community punishment order;
%
%($b$) a community rehabilitation order;
%
%($c$) a community punishment and rehabilitation order;
%
%    ($d$) 
%    such other description of community order within the meaning of the Powers of Criminal Courts (Sentencing) Act 2000 as may be prescribed for the purposes of this section; or
%
%    ($e$) 
%    any order falling in England and Wales to be treated as an order specified in paragraphs ($a$)  to ($d$). 
%\end{enumerate}

% ``Definition of ``relevant community order'' substituted (4.4.05) by 2003 c 44 Sch 32 para 131(2)
“relevant community order” means—
\begin{enumerate}\item[]
($a$) a community order made under section 177 of the Criminal Justice Act 2003; or

($b$) any order falling in England or Wales to be treated as such an order.
\end{enumerate}
\end{enumerate}

(9) In relation to a relevant benefit falling within paragraph ($d$)  of the definition of that expression in subsection (8), references in this section to the conditions for entitlement to that benefit being or becoming satisfied with respect to any person are references to there having been or, as the case may be, the taking of a decision to make a payment of such benefit to that person.

(10) In relation to any time before the coming into force of the Powers of Criminal Courts (Sentencing) Act 2000, the reference to that Act in subsection (8)  shall be taken to be a reference to Part I of the Criminal Justice Act 1991. 

(11) In the application to Scotland of this section—
\begin{enumerate}\item[]
($a$) in subsection (1)  after the word “excuse” insert “(or, in the case of a probation order, failed)”;

($b$) for paragraph ($b$)  of that subsection substitute—
\begin{quotation}
“($b$) the Secretary of State is notified in accordance with an Act of Adjournal made under section 64 of the determination”; and
\end{quotation}

($c$) in subsection (8)—
\begin{enumerate}\item[]
(i) in the definition of relevant benefit, paragraph ($d$)  does not apply in the case of any payment made by or on behalf of the Scottish Ministers; and

%(ii) in the definition of relevant community order, for paragraphs ($c$)  to ($e$)  substitute—
%\begin{quotation}
%“($c$) such other description of order made under the Criminal Procedure (Scotland) Act 1995 as may be prescribed for the purposes of this section; or
%
%($d$) any order falling in Scotland to be treated as an order specified in paragraphs ($a$)  to ($c$).”
%\end{quotation}

% S 62(11)(c)(ii) substituted (1.4.01) by 2000 c 43 Sch 7 para 206($b$) 
(ii) in the definition of “relevant community order”, for paragraphs ($a$) 
%to ($e$) 
and ($b$)  % Words substituted (4.4.05) by 2003 c 44 Sch 32 para 131(3)
substitute—
\begin{quotation}
“($a$) a community service order;

($b$) a probation order;

($c$) such other description of order made under the Criminal Procedure (Scotland) Act 1995 as may be prescribed for the purposes of this section; or

($d$) any order falling in Scotland to be treated as an order specified in paragraphs ($a$) to ($c$)”.”
\end{quotation}
\end{enumerate}
\end{enumerate}

\amendment{
S. 62(1)--(10) is in force for certain purposes only; see the Child Support, Pensions and Social Security Act 2000 (Commencement No. 10) Order 2001 art. 2(2)($b$).  The whole of s. 62 is in force for the purpose of making regulations.

S. 62(11)(c)(ii) substituted and paras. ($a$) --(c) in definition of ``relevant community order'' in s. 62(8) substituted (1.4.01) by the Criminal Justice and Court Services Act 2000 Sch. 7 para. 206.

Words substituted in s. 62(11)(c)(ii) and definition of ``relevant community order'' in s. 62(8) substituted (4.4.05) by the Criminal Justice Act 2003 Sch. 32 para. 131.
}

\subsection{63. Loss of joint-claim jobseeker’s allowance}

(1) Subsections (2)  and (3)  shall have effect, subject to the other provisions of this section, where—
\begin{enumerate}\item[]
($a$) the conditions for the entitlement of any joint-claim couple to a joint-claim jobseeker’s allowance are or become satisfied at any time; and

($b$) the restriction in subsection (2)  of section 62 would apply in the case of at least one of the members of the couple if the entitlement were an entitlement of that member to a relevant benefit.
\end{enumerate}

(2) The allowance shall not be payable in the couple’s case for so much of the prescribed period as is a period for which—
\begin{enumerate}\item[]
($a$) in the case of each of the members of the couple, the restriction in subsection (2)  of section 62 would apply if the entitlement were an entitlement of that member to a relevant benefit; or

%($b$) that restriction would so apply in the case of one of the members of the couple and the other member of the couple is subject to sanctions for the purposes of section 20A of the Jobseekers Act 1995 (denial or reduction of joint-claim jobseeker’s allowance).

% S 63(2)($b$)  substituted (1.4.02) by 2001 c 11 s 12(1)
($b$) that restriction would apply in the case of one of the members of the couple and the other member of the couple—
\begin{enumerate}\item[]
(i) is subject to sanctions for the purposes of section 20A of the Jobseekers Act 1995 (denial or reduction of joint-claim jobseeker’s allowance); or

(ii) is a person in whose case the restriction in subsection (2) of section 8 of the Social Security Fraud Act 2001 (loss of benefit for offenders) would apply if the entitlement were an entitlement to a sanctionable benefit (within the meaning of that section).
\end{enumerate}
\end{enumerate}

(3) For any part of the period for which subsection (2)  does not apply, the allowance—
\begin{enumerate}\item[]
($a$) shall be payable in the couple’s case as if the amount of the allowance were reduced to an amount calculated using the method prescribed for the purposes of this subsection; but

($b$) shall be payable only to the member of the couple who is not the person in relation to whom the court has made a determination.
\end{enumerate}

(4) The Secretary of State may by regulations provide in relation to cases to which subsection (2)  would otherwise apply that joint-claim jobseeker’s allowance shall be payable in a couple’s case, during the whole or a part of so much of the prescribed period as falls within paragraph ($a$)  or ($b$)  of that subsection, as if one or more of the following applied—
\begin{enumerate}\item[]
($a$) the rate of the allowance were such reduced rate as may be prescribed;

($b$) the allowance were payable only if there is compliance by each of the members of the couple with such obligations with respect to the provision of information as may be imposed by the regulations;

($c$) the allowance were payable only if the circumstances are otherwise such as may be prescribed.
\end{enumerate}

(5) Subsection (6)  of section 20A of the Jobseekers Act 1995 (calculation of reduced amount) shall apply for the purposes of subsection (3)  above as it applies for the purposes of subsection (5)  of that section.

(6) Subsection (6)  of section 62 shall apply for the purposes of this section in relation to any determination relating to one or both members of the joint-claim couple as it applies for the purposes of that section in relation to the determination relating to the offender.

(7) The length of any period prescribed for the purposes of subsection (2)  or (3)  shall not exceed twenty-six weeks.

(8) In this section—
\begin{enumerate}\item[]
    “joint-claim couple” and “joint-claim jobseeker’s allowance” have the same meanings as in the Jobseekers Act 1995; and

    “relevant benefit” has the same meaning as in section 62.  
\end{enumerate}

\amendment{
S. 63 is in force for certain purposes only (see the Child Support, Pensions and Social Security Act 2000 (Commencement No. 10) Order 2001 art. 2(2)($b$) ) and also for the purpose of making regulations.

S. 63(2)($b$)  substituted (1.4.02) by the Social Security Fraud Act 2001 s. 12(1).
}

\subsection{64. Information provision}

(1) A court in Great Britain shall, before making a relevant community order in relation to any person, explain to that person in ordinary language the consequences by virtue of sections 62 and 63 of a failure to comply with the order.

(2) The Secretary of State may by regulations require the 
%Chief Probation Officer for any area in England and Wales
chief officer of a local probation board% % Words substituted (6.4.01) by 2000 c 43 Sch 7 para 207($a$) 
, or such other person as may be prescribed, to notify the Secretary of State at the prescribed time and in the prescribed manner—
\begin{enumerate}\item[]
($a$) of the laying by 
%a person employed or appointed by a probation committee 
an officer of a local probation board  % Words substituted (6.4.01) by 2000 c 43 Sch 7 para 207($b$) 
of any information that a person has failed to comply with the requirements of a relevant community order;

($b$) of any such determination as is mentioned in section 62(1);

($c$) of such information about the offender, and in the possession of the person giving the notification, as may be prescribed; and

($d$) of any circumstances by virtue of which any payment or adjustment might fall to be made by virtue of section 62(6)  or 63(6).
\end{enumerate}

(3) The High Court of Justiciary may, by Act of Adjournal, make provision requiring the clerk of the court in which any proceedings are commenced that could result in a determination of a failure to comply with a relevant community order to notify the Secretary of State at such time and in such manner as may be specified in the Act of Adjournal of—
\begin{enumerate}\item[]
($a$) the commencement of the proceedings;

($b$) any such determination made in the proceedings;

($c$) such information about the offender as may be so specified; and

($d$) any circumstances by virtue of which any payment or adjustment might fall to be made by virtue of section 62(6)  or 63(6).
\end{enumerate}

(4) Where it appears to the Secretary of State that—
\begin{enumerate}\item[]
($a$) the laying of any information that has been laid in England and Wales, or

($b$) the commencement of any proceedings that have been commenced in Scotland,
\end{enumerate}
could result in a determination the making of which would result in the imposition by or under one or both of sections 62 and 63 of any restrictions, it shall be the duty of the Secretary of State to notify the person in whose case those restrictions would be imposed, or (as the case may be) the members of any joint-claim couple in whose case they would be imposed, of the consequences under those sections of such a determination in the case of that person, or couple.

(5) A notification required to be given by the Secretary of State under subsection (4)  must be given as soon as reasonably practicable after it first appears to the Secretary of State as mentioned in that subsection.

(6) The Secretary of State may by regulations make such provision as he thinks fit for the purposes of sections 62 to 65 of this Act about—
\begin{enumerate}\item[]
($a$) the use by a person within subsection (7)  of information relating to community orders 
(as defined by section 177 of the Criminal Justice Act 2003)  % Words inserted (4.4.05) by 2003 c 44 Sch 32 para 132
or social security;

($b$) the supply of such information by a person within that subsection to any other person (whether or not within that subsection); and

($c$) the purposes for which a person to whom such information is supplied under the regulations may use it.
\end{enumerate}

(7) The persons within this subsection are—
\begin{enumerate}\item[]
($a$) the Secretary of State;

($b$) a person providing services to the Secretary of State;

($c$) %a person employed or appointed by a probation committee 
an officer of a local probation board%  % Words substituted (6.4.01) by 2000 c 43 Sch 7 para 207($b$) 
;

($d$) a person employed by a council constituted under section 2 of the Local Government etc.\ (Scotland) Act 1994. 
\end{enumerate}

(8) Regulations under subsection (6)  may, in particular, authorise information supplied to a person under the regulations—
\begin{enumerate}\item[]
($a$) to be used for the purpose of amending or supplementing other information held by that person; and

($b$) where so used, to be supplied to any other person to whom, and used for any purpose for which, the information amended or supplemented could be supplied or used.
\end{enumerate}

(9) The explanation given to the offender by the court in pursuance of subsection (1)  shall be treated as part of the explanation required to be given to the offender for the purposes of section 228(5)  or 238(4)  of the Criminal Procedure (Scotland) Act 1995. 

(10) In this section “relevant community order” has the same meaning as in section 62
and “local probation board” means a local probation board established under section 4 of the Criminal Justice and Court Services Act 2000% Words inserted (6.4.01) by 2000 c 43 Sch 7 para 207(c)
. 

(11) For the purposes of this section proceedings that could result in such a determination as is mentioned in subsection (3)  are commenced in Scotland when, and only when, a warrant to arrest the offender or to cite the offender to appear before a court is issued under section 232(1)  or 239(4)  of the Criminal Procedure (Scotland) Act 1995. 

\amendment{
S. 64(1) is in force for certain purposes only; see the Child Support, Pensions and Social Security Act 2000 (Commencement No. 10) Order 2001 art. 2(2)($a$).  S. 64(2), (4)($a$), (5), (6), (7)($a$)  to (c), (8), (10) is in force for certain purposes only; see the Child Support, Pensions and Social Security Act 2000 (Commencement No. 10) Order 2001 art. 2(2)($b$).  The whole of s. 64 is in force for the purpose of making regulations.

Words inserted in s. 64(10) and words substituted in s. 64(2), (2)($a$), (7)(c) (6.4.01) by the Criminal Justice and Court Services Act 2000 Sch. 7 para. 207.

Words inserted in s. 64(6)(a) (4.4.05) by the Criminal Justice Act 2003 Sch. 32 para. 132.
}

\subsection{65. Loss of benefit regulations}

(1) In the loss of benefit provisions “prescribed” means prescribed by or determined in accordance with regulations made by the Secretary of State.

(2) Regulations prescribing a period for the purposes of any of the loss of benefit provisions may contain provision for determining the time from which the period is to run.

(3) Regulations under any of the loss of benefit provisions shall be made by statutory instrument which (except in the case of regulations to which subsection (4)  applies) shall be subject to annulment in pursuance of a resolution of either House of Parliament.

(4) A statutory instrument containing (whether alone or with other provisions)—
\begin{enumerate}\item[]
($a$) a provision prescribing the manner in which the applicable amount is to be reduced for the purposes of section 62(3),

($b$) a provision prescribing the manner in which an amount of joint-claim jobseeker’s allowance is to be reduced for the purposes of section 63(3)($a$),

($c$) a provision the making of which is authorised by section 62(4)  or 63(4),

($d$) a provision prescribing benefits under the Social Security Contributions and Benefits Act 1992 as benefits that are to be relevant benefits for the purposes of section 62, or

($e$) a provision that any description of order is to be a relevant community order for the purposes of that section,
\end{enumerate}
shall not be made unless a draft of the instrument has been laid before, and approved by a resolution of, each House of Parliament.

(5) Subsections (4)  to (6)  of section 189 of the Social Security Administration Act 1992 (supplemental and incidental powers etc.)\ shall apply in relation to any power to make regulations that is conferred by the loss of benefit provisions as they apply in relation to the powers to make regulations that are conferred by that Act.

(6) The provision that may be made in exercise of the powers to make regulations that are conferred by the loss of benefit provisions shall include different provision for different areas.

(7) Where regulations made under section 62(8)  prescribe a description of order made under the Criminal Procedure (Scotland) Act 1995 as a relevant community order for the purposes of that section, the regulations may make such modifications of that section as appear to the Secretary of State to be necessary in consequence of so prescribing.

(8) In this section “the loss of benefit provisions” means sections 62 to 64 of this Act.

\amendment{
S. 65(1)--(6), (8) is in force for certain purposes only; see the Child Support, Pensions and Social Security Act 2000 (Commencement No. 10) Order 2001 art. 2(2)($b$).  The whole of s. 65 is in force for the purpose of making regulations.
}

\subsection{66. Appeals relating to loss of benefit}

In paragraph 3 of Schedule 3 to the Social Security Act 1998 (decisions against which an appeal lies), after sub-paragraph ($d$)  there shall be inserted “; or
\begin{quotation}
($e$) section 62 or 63 of the Child Support, Pensions and Social Security Act 2000.”
\end{quotation}

\amendment{
S. 66 is in force for certain purposes only; see the Child Support, Pensions and Social Security Act 2000 (Commencement No. 10) Order 2001 art. 2(2)($b$).
}

\section{\itshape Investigation powers}

\subsection{67. Investigation powers}

Schedule 6 to this Act (which amends the enforcement provisions contained in Part VI of the Social Security Administration Act 1992) shall have effect.

\section{\itshape Housing benefit and council tax benefit etc.}

\subsection{68. Housing benefit and council tax benefit: revisions and appeals}

Schedule 7 (which makes provision for the revision of decisions made in connection with claims for housing benefit or council tax benefit and for appeals against such decisions) shall have effect.

\subsection{69. Discretionary financial assistance with housing}

(1) The Secretary of State may by regulations make provision conferring a power on relevant authorities to make payments by way of financial assistance (“discretionary housing payments”) to persons who—
\begin{enumerate}\item[]
($a$) are entitled to housing benefit or council tax benefit, or to both; and

($b$) appear to such an authority to require some further financial assistance (in addition to the benefit or benefits to which they are entitled) in order to meet housing costs.
\end{enumerate}

(2) Regulations under this section may include any of the following—
\begin{enumerate}\item[]
($a$) provision prescribing the circumstances in which discretionary housing payments may be made under the regulations;

($b$) provision conferring (subject to any provision made by virtue of paragraph ($c$)  or ($d$)  of this subsection or an order under section 70) a discretion on a relevant authority—
\begin{enumerate}\item[]
(i) as to whether or not to make discretionary housing payments in a particular case; and

(ii) as to the amount of the payments and the period for or in respect of which they are made;
\end{enumerate}

($c$) provision imposing a limit on the amount of the discretionary housing payment that may be made in any particular case;

($d$) provision restricting the period for or in respect of which discretionary housing payments may be made;

($e$) provision about the form and manner in which claims for discretionary housing payments are to be made and about the procedure to be followed by relevant authorities in dealing with and disposing of such claims;

($f$) provision imposing conditions on persons claiming or receiving discretionary housing payments requiring them to provide a relevant authority with such information as may be prescribed;

($g$) provision entitling a relevant authority that are making or have made a discretionary housing payment, in such circumstances as may be prescribed, to cancel the making of further such payments or to recover a payment already made;

($h$) provision requiring or authorising a relevant authority to review decisions made by the authority with respect to the making, cancellation or recovery of discretionary housing payments.
\end{enumerate}

(3) Regulations under this section shall be made by statutory instrument, which shall be subject to annulment in pursuance of a resolution of either House of Parliament.

(4) Subsections (4)  to (6)  of section 189 of the Social Security Administration Act 1992 (supplemental and incidental powers etc.)\ shall apply in relation to any power to make regulations under this section as they apply in relation to the powers to make regulations that are conferred by that Act.

(5) Any power to make regulations under this section shall include power to make different provision for different areas or different relevant authorities.

(6) In section 176(1)  of that Act (consultation with representative organisation on subordinate legislation relating to housing benefit or council tax benefit), after paragraph ($a$)  there shall be inserted—
\begin{quotation}
“($aa$) regulations under section 69 of the Child Support, Pensions and Social Security Act 2000;”.
\end{quotation}

(7) In this section—
\begin{enumerate}\item[]
    “prescribed” means prescribed by or determined in accordance with regulations made by the Secretary of State; and

    “relevant authority” means an authority administering housing benefit or council tax benefit. 
\end{enumerate}

\subsection{70. Grants towards cost of discretionary housing payments}

(1) The Secretary of State may, out of money provided by Parliament, make to a relevant authority such payments as he thinks fit in respect of—
\begin{enumerate}\item[]
($a$) the cost to that authority of the making of discretionary housing payments; and

($b$) the expenses involved in the administration by that authority of any scheme for the making of discretionary housing payments.
\end{enumerate}

(2) The following provisions, namely—
\begin{enumerate}\item[]
($a$) subsections (1), (3), (4), (5)($b$)
%, (7)($b$)   % Words repealed (18.11.03) by 2003 c 26 Sch 8 Pt I
and (8)  of section 140B of the Social Security Administration Act 1992 (calculation of amount of subsidy payable to authorities administering housing benefit or council tax benefit), and

($b$) section 140C of that Act (payment of subsidy),
\end{enumerate}
shall apply in relation to payments under this section as they apply in relation to subsidy under section 140A of that Act.

(3) The Secretary of State may by order make provision—
\begin{enumerate}\item[]
($a$) imposing a limit on the total amount of expenditure in any year that may be incurred by a relevant authority in making discretionary housing payments;

($b$) imposing subsidiary limits on the expenditure that may be incurred in any year by a relevant authority in making discretionary housing payments in the circumstances specified in the order.
\end{enumerate}

(4) An order imposing a limit by virtue of subsection (3)($a$)  or ($b$)  may fix that limit either by specifying the amount of the limit or by providing for the means by which it is to be determined.

(5) An order under this section shall be made by statutory instrument, which shall be subject to annulment in pursuance of a resolution of either House of Parliament.

(6) Subsections (4)  to (6)  of section 189 of the Social Security Administration Act 1992 (supplemental and incidental powers etc.)\ shall apply in relation to any power to make an order under this section as they apply in relation to the powers to make an order that are conferred by that Act.

(7) Any power to make an order under this section shall include power to make different provision for different areas or different relevant authorities.

(8) In this section—
\begin{enumerate}\item[]
    “discretionary housing payment” means any payment made by virtue of regulations under section 69;

    “relevant authority” means an authority administering housing benefit or council tax benefit;

    “subsidy” has the same meaning as in sections 140A to 140G of the Social Security Administration Act 1992;

    “year” means a financial year within the meaning of the Local Government Finance Act 1992.  
\end{enumerate}

\amendment{
Words repealed in s. 70(2)($a$)  (18.11.03) by the Local Government Act 2003 Sch. 8 Pt. I.
}

\subsection{71. Recovery of housing benefit}

For subsection (3)  of section 75 of the Social Security Administration Act 1992 (overpayments of housing benefit) there shall be substituted—
\begin{quotation}
“(3) An amount recoverable under this section shall be recoverable—
\begin{enumerate}\item[]
($a$) except in such circumstances as may be prescribed, from the person to whom it was paid; and

($b$) where regulations so provide, from such other person (as well as, or instead of, the person to whom it was paid) as may be prescribed.”
\end{enumerate}
\end{quotation}

\section{\itshape Child benefit}

\subsection{72. Child benefit disregards}

In section 143(3)($c$)  of the Social Security Contributions and Benefits Act 1992 (disregard of days of absence in the case of children in residential accommodation in pursuance of arrangements made under the specified enactments), for sub-paragraph (iii)  and the word “or” immediately preceding it there shall be substituted—
\begin{quotation}
“(iii) the Social Work (Scotland) Act 1968;

(iv) the National Health Service (Scotland) Act 1978;

(v) the Education (Scotland) Act 1980;

(vi) the Mental Health (Scotland) Act 1984; or

(vii) the Children (Scotland) Act 1995.”
\end{quotation}

\section{\itshape Social Security Advisory Committee}

\subsection{73. Social Security Advisory Committee}

(1) Section 170 of the Social Security Administration Act 1992 (functions of the Social Security Advisory Committee in relation to the relevant enactments and the relevant Northern Ireland enactments) shall be amended as follows.

(2) In the definition in subsection (5)  of “relevant enactments”, after paragraph ($ae$)  there shall be inserted—
\begin{quotation}
“($af$) section 42, sections 62 to 65 and sections 68 to 70 of the Child Support Pensions and Social Security Act 2000 and Schedule 7 to that Act;”.
\end{quotation}

(3) In the definition in that subsection of “relevant Northern Ireland enactments”, after paragraph ($ae$)  there shall be inserted—

\begin{quotation}
“($af$) any provisions in Northern Ireland which correspond to section 42, any of sections 62 to 65, 68 to 70 of the Child Support, Pensions and Social Security Act 2000 or Schedule 7 to that Act; and”.
\end{quotation}

\amendment{
S. 73 is partially in force: see the Child Support Pensions and Social Security Act 2000 (Commencement No. 2) Order 2000 art. 2($b$), 4.
}

\part[Part IV --- National Insurance Contributions]{Part IV\\*National Insurance Contributions}

\renewcommand\parthead{--- Part IV}

\section{\itshape Great Britain}

\subsection{74. Contributions in respect of benefits in kind: Great Britain}

(1) In section 1(2)($b$)  of the Social Security Contributions and Benefits Act 1992 (Class 1A contributions), the words “in respect of cars made available for private use and car fuel” shall be omitted.

(2) For section 10 of that Act (Class 1A contributions) there shall be substituted—
\begin{quotation}
\subsection*{“10. Class 1A contributions: benefits in kind etc}

(1) Where—
\begin{enumerate}\item[]
($a$) for any tax year an earner is chargeable to income tax under Schedule E on an amount which for the purposes of the Income Tax Acts is or falls to be treated as an emolument received by him from any employment (“the relevant employment”),

($b$) the relevant employment is both employed earner’s employment and employment to which Chapter II of Part V of the 1988 Act (employment as a director or with annual emoluments of more than £8,500) applies, and

($c$) the whole or a part of the emolument falls, for the purposes of Class 1 contributions, to be left out of account in the computation of the earnings paid to or for the benefit of the earner,
\end{enumerate}
a Class 1A contribution shall be payable for that tax year, in accordance with this section, in respect of that earner and so much of the emolument as falls to be so left out of account.

(2) Subject to section 10ZA below, a Class 1A contribution for any tax year shall be payable by—
\begin{enumerate}\item[]
($a$) the person who is liable to pay the secondary Class 1 contribution relating to the last (or only) relevant payment of earnings in that tax year in relation to which there is a liability to pay such a Class 1 contribution; or

($b$) if paragraph ($a$)  above does not apply, the person who, if the emolument in respect of which the Class 1A contribution is payable were earnings in respect of which Class 1 contributions would be payable, would be liable to pay the secondary Class 1 contribution.
\end{enumerate}

(3) In subsection (2)  above “relevant payment of earnings” means a payment which for the purposes of Class 1 contributions is a payment of earnings made to or for the benefit of the earner in respect of the relevant employment.

(4) The amount of the Class 1A contribution in respect of any emolument shall be the Class 1A percentage of so much of it as falls to be left out of account as mentioned in subsection (1)($c$)  above.

(5) In subsection (4)  above “the Class 1A percentage” means a percentage rate equal to the percentage rate specified as the secondary percentage in section 9(2)  above for the tax year in question.

(6) No Class 1A contribution shall be payable for any tax year in respect of so much of any emolument as is taken for the purposes of the making of Class 1B contributions for that year to be included in a PAYE settlement agreement.

(7) For the purposes of this section—
\begin{enumerate}\item[]
($a$) the amounts which for the purposes of the Income Tax Acts are or fall to be treated as emoluments received by an earner from any employment shall be determined (subject to paragraph ($b$)  below) disregarding sections 198, 201, 201AA and 332(3)  of the 1988 Act (deductions for expenses etc.);\ but

($b$) where an amount which is deductible in respect of any matter under any of those sections is at least equal to the whole of any corresponding amount which (but for this paragraph) would fall by reference to that matter to be included in those emoluments, the whole of the corresponding amount shall be treated as not so included.
\end{enumerate}

(8) The Treasury may by regulations—
\begin{enumerate}\item[]
($a$) modify the effect of subsection (7)  above by adding any enactment contained in the Income Tax Acts to the list of sections of the 1988 Act contained in paragraph ($a$)  of that subsection; or

($b$) make such amendments of subsection (7)  above as appear to them to be necessary or expedient in consequence of any alteration of the provisions of the Income Tax Acts relating to the charge to tax under Schedule E.
\end{enumerate}

(9) The Treasury may by regulations provide—
\begin{enumerate}\item[]
($a$) for Class 1A contributions not to be payable, in prescribed circumstances, by prescribed persons or in respect of prescribed persons or emoluments;

($b$) for reducing Class 1A contributions in prescribed circumstances.
\end{enumerate}

(10) In this section “the 1988 Act” means the Income and Corporation Taxes Act 1988.”
\end{quotation}

(3) For subsection (6)  of section 4 of that Act (power to treat emoluments in respect of share acquisitions etc.\ as earnings) there shall be substituted—
\begin{quotation}
“(6) Regulations may make provision for the purposes of this Part—
\begin{enumerate}\item[]
($a$) for treating any amount on which an employed earner is chargeable to income tax under Schedule E as remuneration derived from the earner’s employment; and

($b$) for treating any amount which in accordance with regulations under paragraph ($a$)  above constitutes remuneration as an amount of remuneration paid, at such time as may be determined in accordance with the regulations, to or for the benefit of the earner in respect of his employment.”
\end{enumerate}
\end{quotation}

(4) In paragraph 5($b$)  of Schedule 1 to that Act (power to modify section 10 for cases where a car is made available by reason of more than one employment), for “a car is made available” there shall be substituted “something is provided or made available”.

(5) In paragraph 8(1)($ia$) of that Schedule (power to provide by regulations for repayment in prescribed cases of the whole or a part of a Class 1B contribution), after “part” there shall be inserted “of a Class 1A or”.

(6) In section 120(4)  of the Social Security Administration Act 1992 (proof of previous offences relating to Class 1A contributions), for “car” there shall be substituted “amount”.

(7) In section 162(5)($c$)  of that Act (appropriate national health service allocation of Class 1A contributions), for “cash equivalents of the benefits of the cars and car fuel” there shall be substituted “emoluments”.

(8) This section shall have effect in relation to the tax year beginning with 6th April 2000 and subsequent tax years.

(9) Regulations made by statutory instrument under any power conferred by virtue of this section may be made so as to have retrospective effect in relation to any time in the tax year in which they are made (including, in the case of regulations made in the tax year in which this Act is passed, any time in that tax year before the passing of this Act).

\subsection{75. Third party providers of benefits in kind: Great Britain}

(1) After section 10 of the Social Security Contributions and Benefits Act 1992 there shall be inserted—
\begin{quotation}
\subsection*{“10ZA. Liability of third party provider of benefits in kind}

(1) This section applies, where—
\begin{enumerate}\item[]
($a$) a Class 1A contribution is payable for any tax year in respect of the whole or any part of an emolument received by an earner;

($b$) the emolument, in so far as it is one in respect of which such a contribution is payable, consists in a benefit provided for the earner or a member of his family or household;

($c$) the person providing the benefit is a person other than the person (“the relevant employer”) by whom, but for this section, the Class 1A contribution would be payable in accordance with section 10(2)  above; and

($d$) the provision of the benefit by that other person has not been arranged or facilitated by the relevant employer.
\end{enumerate}

(2) For the purposes of this Act if—
\begin{enumerate}\item[]
($a$) the person providing the benefit pays an amount for the purpose of discharging any liability of the earner to income tax for any tax year, and

($b$) the income tax in question is tax chargeable in respect of the provision of the benefit or of the making of the payment itself,
\end{enumerate}
the amount of the payment shall be treated as if it were an emolument consisting in the provision of a benefit to the earner in that tax year and falling, for the purposes of Class 1 contributions, to be left out of account in the computation of the earnings paid to or for the benefit of the earner.

(3) Subject to subsection (4)  below, the liability to pay any Class 1A contribution in respect of—
\begin{enumerate}\item[]
($a$) the benefit provided to the earner, and

($b$) any further benefit treated as so provided in accordance with subsection (2)  above,
\end{enumerate}
shall fall on the person providing the benefit, instead of on the relevant employer.

(4) Subsection (3)  above applies in the case of a Class 1A contribution for the tax year beginning with 6th April 2000 only if the person providing the benefit in question gives notice in writing to the Inland Revenue on or before 6th July 2001 that he is a person who provides benefits in respect of which a liability to Class 1A contributions is capable of falling by virtue of this section on a person other than the relevant employer.

(5) The Treasury may by regulations make provision specifying the circumstances in which a person is or is not to be treated for the purposes of this Act as having arranged or facilitated the provision of any benefit.

(6) In this section references to a member of a person’s family or household shall be construed in accordance with section 168(4)  of the Income and Corporation Taxes Act 1988. 

\subsection*{10ZB. Non-cash vouchers provided by third parties}

(1) In section 10ZA above references to the provision of a benefit include references to the provision of a non-cash voucher.

(2) Where—
\begin{enumerate}\item[]
($a$) a non-cash voucher is received by any person from employment to which Chapter II of Part V of the Income and Corporation Taxes Act 1988 does not apply, and

($b$) the case would be one in which the conditions in section 10ZA(1)($a$)  to ($d$)  above would be satisfied in relation to the provision of that voucher if that Chapter did apply to that employment,
\end{enumerate}
sections 10 and 10ZA above shall have effect in relation to the provision of that voucher, and to any such payment in respect of the provision of that voucher as is mentioned in section 10ZA(2)  above, as if that employment were employment to which that Chapter applied.

(3) In this section “non-cash voucher” has the same meaning as in section 141 of the Income and Corporation Taxes Act 1988.”
\end{quotation}

(2) After subsection (3)  of section 110ZA of the Social Security Administration Act 1992 (premises liable to inspection) there shall be inserted—
\begin{quotation}
“(3A) The references in subsection (3)  above to a trade or business include references to the administration of any scheme for the provision of benefits to persons by reason of their employment.”
\end{quotation}

(3) Subsection (1)  shall have effect in relation to the tax year beginning with 6th April 2000 and subsequent tax years.

(4) Regulations made by virtue of this section under section 10ZA(5)  of the Social Security Contributions and Benefits Act 1992 may be made so as to have retrospective effect in relation to any time in the tax year in which they are made (including, in the case of regulations made in the tax year in which this Act is passed, any time in that tax year before the passing of this Act).

\subsection{76. Collection etc.\ of NICs: Great Britain}

(1) Schedule 1 to the Social Security Contributions and Benefits Act 1992 (supplementary provisions relating to contributions) shall be amended in accordance with subsections (2)  to (5).

(2) In paragraph 7(2)($b$)  (application of sections 100 to 100D and 102 to 104 of the Taxes Management Act 1970 in relation to certain penalties), for “104” there shall be substituted “105”.

(3) For sub-paragraph (2)($e$)  of paragraph 7B (power to provide for interest to be charged on late payment in the case of payment outside the PAYE system) there shall be substituted—
\begin{quotation}
“($e$) require interest to be paid on contributions that are not paid by the due date, and provide for determining the date from which such interest is to be calculated;”.
\end{quotation}

(4) After sub-paragraph (5)  of that paragraph there shall be inserted—
\begin{quotation}
“(5A) Regulations under this paragraph may, in relation to any penalty imposed by such regulations, make provision applying (with or without modifications) any enactment applying for the purposes of income tax that is contained in Part X of the Taxes Management Act 1970 (penalties).”
\end{quotation}

(5) After that paragraph there shall be inserted—
\begin{quotation}
“7BA. The Inland Revenue may by regulations provide for amounts in respect of contributions or interest that fall to be paid or repaid in accordance with any regulations under this Schedule to be set off, or to be capable of being set off, in prescribed circumstances and to the prescribed extent, against any such liabilities under regulations under this Schedule of the person entitled to the payment or repayment as may be prescribed.”
\end{quotation}

(6) In section 8(1)  of the Social Security Contributions (Transfer of Functions, etc.)\ Act 1999 (decisions to be made by an Inland Revenue officer and appealable under section 11)—
\begin{enumerate}\item[]
($a$) paragraph ($j$)  (interest under regulations made by virtue of paragraph 7B(2)($e$)  of Schedule 1 to the Social Security Contributions and Benefits Act 1992) shall cease to have effect; and

($b$) in paragraph ($l$), for “paragraphs ($j$)  and ($k$)” there shall be substituted “paragraph ($k$)”, and the words “amount of interest or” shall be omitted.
\end{enumerate}

(7) Subsection (6)  has effect in relation to interest accruing on sums becoming due in respect of the tax year beginning with 6th April 2000 or any subsequent tax year.

\subsection{77. Liability of earner for secondary contributions: Great Britain}

(1) In paragraph 3 of Schedule 1 to the Social Security Contributions and Benefits Act 1992 (prohibition on deduction or recovery of Class 1 contributions), sub-paragraph (2)  shall be omitted.

(2) After that paragraph there shall be inserted—
\begin{quotation}
\subsection*{“Prohibition on recovery of employer’s contributions}

3A.---(1) Subject to sub-paragraph (2)  below, a person who is or has been liable to pay any secondary Class 1 or any Class 1A or Class 1B contributions shall not—
\begin{enumerate}\item[]
($a$) make, from earnings paid by him, any deduction in respect of any such contributions for which he or any other person is or has been liable;

($b$) otherwise recover any such contributions (directly or indirectly) from any person who is or has been a relevant earner; or

($c$) enter into any agreement with any person for the making of any such deduction or otherwise for the purpose of so recovering any such contributions.
\end{enumerate}

(2) Sub-paragraph (1)  above does not apply to the extent that an agreement between—
\begin{enumerate}\item[]
($a$) a secondary contributor, and

($b$) any person (“the earner”) in relation to whom the secondary contributor is, was or will be such a contributor in respect of the contributions to which the agreement relates,
\end{enumerate}
allows the secondary contributor to recover (whether by deduction or otherwise) the whole or any part of any secondary Class 1 contribution payable in respect of a gain that is treated as remuneration derived from that earner’s employment by virtue of section 4(4)($a$)  above.

(3) Sub-paragraph (2)  above does not authorise any recovery (whether by deduction or otherwise)—
\begin{enumerate}\item[]
($a$) in pursuance of any agreement entered into before 19th May 2000; or

($b$) in respect of any liability to a contribution arising before the day of the passing of the Child Support, Pensions and Social Security Act 2000. 
\end{enumerate}

(4) In this paragraph—
\begin{enumerate}\item[]
    “agreement” includes any arrangement or understanding (whether or not legally enforceable); and

    “relevant earner”, in relation to a person who is or has been liable to pay any contributions, means an earner in respect of whom he is or has been so liable. 
\end{enumerate}

\subsection*{Transfer of liability to be borne by earner}

3B.---(1) This paragraph applies where—
\begin{enumerate}\item[]
($a$) an election is jointly made by—
\begin{enumerate}\item[]
(i) a secondary contributor, and

(ii) a person (“the earner”) in relation to whom the secondary contributor is or will be such a contributor in respect of contributions on share option gains by the earner,
\end{enumerate}
for the whole or a part of any liability of the secondary contributor to contributions on any such gains to be transferred to the earner; and

($b$) the election is one in respect of which the Inland Revenue have, before it was made, given by notice to the secondary contributor their approval to both—
\begin{enumerate}\item[]
(i) the form of the election; and

(ii) the arrangements made in relation to the proposed election for securing that the liability transferred by the election will be met.
\end{enumerate}
\end{enumerate}

(2) Any liability which—
\begin{enumerate}\item[]
($a$) arises while the election is in force, and

($b$) is a liability to pay the contributions on share option gains by the earner, or the part of them, to which the election relates,
\end{enumerate}
shall be treated for the purposes of this Act, the Administration Act and Part II of the Social Security Contributions (Transfer of Functions, etc.)\ Act 1999 as a liability falling on the earner, instead of on the secondary contributor.

(3) Subject to sub-paragraph (7)($b$)  below, an election made for the purposes of sub-paragraph (1)  above shall continue in force from the time when it is made until whichever of the following first occurs, namely—
\begin{enumerate}\item[]
($a$) it ceases to have effect in accordance with its terms;

($b$) it is revoked jointly by both parties to the election;

($c$) notice is given to the earner by the secondary contributor terminating the effect of the election.
\end{enumerate}

(4) An approval given to the secondary contributor for the purposes of sub-paragraph (1)($b$)  above may be given either—
\begin{enumerate}\item[]
($a$) for an election to be made by the secondary contributor and a particular person; or

($b$) for all elections to be made, or to be made in particular circumstances, by the secondary contributor and particular persons or by the secondary contributor and persons of a particular description.
\end{enumerate}

(5) The grounds on which the Inland Revenue shall be entitled to refuse an approval for the purposes of sub-paragraph (1)($b$)  above shall include each of the following—
\begin{enumerate}\item[]
($a$) that it appears to the Inland Revenue that adequate arrangements have not been made for securing that the liabilities transferred by the proposed election or elections will be met by the person or persons to whom they would be so transferred; and

($b$) that it appears to the Inland Revenue that they do not have sufficient information to determine whether or not grounds falling within paragraph ($a$)  above exist.
\end{enumerate}

(6) If, at any time after they have given an approval for the purposes of sub-paragraph (1)($b$)  above, it appears to the Inland Revenue—
\begin{enumerate}\item[]
($a$) that the arrangements that were made or are in force for securing that liabilities transferred by elections to which the approval relates are met are proving inadequate or unsatisfactory in any respect, or

($b$) that any election to which the approval relates has resulted, or is likely to result, in the avoidance or non-payment of the whole or any part of any secondary Class 1 contributions,
\end{enumerate}
the Inland Revenue may withdraw the approval by notice to the secondary contributor.

(7) The withdrawal by the Inland Revenue of any approval given for the purposes of sub-paragraph (1)($b$)  above—
\begin{enumerate}\item[]
($a$) may be either general or confined to a particular election or to particular elections; and

($b$) shall have the effect that the election to which the withdrawal relates has no effect on contributions on share option gains in respect of any right to acquire shares obtained after—
\begin{enumerate}\item[]
(i) the date on which notice of the withdrawal of the approval is given; or

(ii) such later date as the Inland Revenue may specify in that notice.
\end{enumerate}
\end{enumerate}

(8) Where the Inland Revenue have refused or withdrawn their approval for the purposes of sub-paragraph (1)($b$)  above, the person who applied for it or, as the case may be, to whom it was given may appeal to the Special Commissioners against the Inland Revenue’s decision.

(9) On an appeal under sub-paragraph (8)  above the Special Commissioners may—
\begin{enumerate}\item[]
($a$) dismiss the appeal;

($b$) remit the decision appealed against to the Inland Revenue with a direction to make such decision as the Special Commissioners think fit; or

($c$) in the case of a decision to withdraw an approval, quash that decision and direct that that decision is to be treated as never having been made.
\end{enumerate}

(10) Subject to sub-paragraph (12)  below, an election under sub-paragraph (1)  above shall not apply to any contributions in respect of gains realised before it was made.

(11) Regulations made by the Inland Revenue may make provision with respect to the making of elections for the purposes of this paragraph and the giving of approvals for the purposes of sub-paragraph (1)($b$)  above; and any such regulations may, in particular—
\begin{enumerate}\item[]
($a$) prescribe the matters that must be contained in such an election;

($b$) provide for the manner in which such an election is to be capable of being made and of being confined to particular liabilities or the part of particular liabilities; and

($c$) provide for the making of applications for such approvals and for the manner in which those applications are to be dealt with.
\end{enumerate}

(12) Where—
\begin{enumerate}\item[]
($a$) an election is made under this paragraph before the end of the period of three months beginning with the date of the passing of the Child Support, Pensions and Social Security Act 2000, and

($b$) that election is expressed to relate to liabilities for contributions arising on or after 19th May 2000 and before the making of the election,
\end{enumerate}
this paragraph shall have effect in relation to those liabilities as if sub-paragraph (2)  above provided for them to be deemed to have fallen on the earner (instead of on the secondary contributor); and the secondary contributor shall accordingly be entitled to reimbursement from the earner for any payment made by that contributor in or towards the discharge of any of those liabilities.

(13) In this paragraph references to contributions on share option gains by the earner are references to any secondary Class 1 contributions payable in respect of a gain that is treated as remuneration derived from the earner’s employment by virtue of section 4(4)($a$)  above.

(14) In this paragraph “the Special Commissioners” means the Commissioners for the special purposes of the Income Tax Acts.”
\end{quotation}

(3) In section 6(4)  of that Act (persons by whom Class 1 contributions are payable), for the words from “paragraph 3” onwards there shall be substituted “paragraphs 3 to 3B of Schedule 1 to this Act.”

(4) In paragraph 8(1)  of Schedule 1 to that Act (general regulations), after paragraph ($c$)  there shall be inserted—
\begin{quotation}
“($ca$) for requiring a secondary contributor to notify a person to whom any of his liabilities are transferred by an election under paragraph 3B above of—
\begin{enumerate}\item[]
(i) any transferred liability that arises;

(ii) the amount of any transferred liability that arises; and

(iii) the contents of any notice of withdrawal by the Inland Revenue of any approval that relates to that election;”.
\end{enumerate}
\end{quotation}

(5) In section 8(1)  of the Social Security Contributions (Transfer of Functions, etc.)\ Act 1999 (decisions to be taken by officers of the Inland Revenue), after paragraph ($i$)  there shall be inserted—
\begin{quotation}
“($ia$) to decide whether to give or withdraw an approval for the purposes of paragraph 3B(1)($b$)  of Schedule 1 to the Social Security Contributions and Benefits Act 1992;”.
\end{quotation}

(6) In section 10 of that Act of 1999 (regulations about varying or superseding decisions), at the beginning of subsection (1)  there shall be inserted “Subject to subsection (2A)  below,”, and after subsection (2)  there shall be inserted—
\begin{quotation}
“(2A) The decisions in relation to which provision may be made by regulations under this section shall not include decisions falling within section 8(1)($ia$) above.”
\end{quotation}

(7) In section 12(4)  of that Act of 1999 (appeals to be heard by General Commissioners), after “Subject to” there shall be inserted “paragraph 3B(8)  of Schedule 1 to the Social Security Contributions and Benefits Act 1992 (which provides for appeals under that paragraph to be heard by the Special Commissioners), to”.

\section{\itshape Northern Ireland}

\subsection{78. Contributions in respect of benefits in kind: Northern Ireland}

(1) In section 1(2)($b$)  of the Social Security Contributions and Benefits (Northern Ireland) Act 1992 (Class 1A contributions), the words “in respect of cars made available for private use and car fuel” shall be omitted.

(2) For section 10 of that Act (Class 1A contributions) there shall be substituted—
\begin{quotation}
\subsection*{“10. Class 1A contributions: benefits in kind etc}

(1) Where—
\begin{enumerate}\item[]
($a$) for any tax year an earner is chargeable to income tax under Schedule E on an amount which for the purposes of the Income Tax Acts is or falls to be treated as an emolument received by him from any employment (“the relevant employment”),

($b$) the relevant employment is both employed earner’s employment and employment to which Chapter II of Part V of the 1988 Act (employment as a director or with annual emoluments of more than £8,500) applies, and

($c$) the whole or a part of the emolument falls, for the purposes of Class 1 contributions, to be left out of account in the computation of the earnings paid to or for the benefit of the earner,
\end{enumerate}
a Class 1A contribution shall be payable for that tax year, in accordance with this section, in respect of that earner and so much of the emolument as falls to be so left out of account.

(2) Subject to section 10ZA below, a Class 1A contribution for any tax year shall be payable by—
\begin{enumerate}\item[]
($a$) the person who is liable to pay the secondary Class 1 contribution relating to the last (or only) relevant payment of earnings in that tax year in relation to which there is a liability to pay such a Class 1 contribution; or

($b$) if paragraph ($a$)  above does not apply, the person who, if the emolument in respect of which the Class 1A contribution is payable were earnings in respect of which Class 1 contributions would be payable, would be liable to pay the secondary Class 1 contribution.
\end{enumerate}

(3) In subsection (2)  above “relevant payment of earnings” means a payment which for the purposes of Class 1 contributions is a payment of earnings made to or for the benefit of the earner in respect of the relevant employment.

(4) The amount of the Class 1A contribution in respect of any emolument shall be the Class 1A percentage of so much of it as falls to be left out of account as mentioned in subsection (1)($c$)  above.

(5) In subsection (4)  above “the Class 1A percentage” means a percentage rate equal to the percentage rate specified as the secondary percentage in section 9(2)  above for the tax year in question.

(6) No Class 1A contribution shall be payable for any tax year in respect of so much of any emolument as is taken for the purposes of the making of Class 1B contributions for that year to be included in a PAYE settlement agreement.

(7) For the purposes of this section—
\begin{enumerate}\item[]
($a$) the amounts which for the purposes of the Income Tax Acts are or fall to be treated as emoluments received by an earner from any employment shall be determined (subject to paragraph ($b$)  below) disregarding sections 198, 201, 201AA and 332(3)  of the 1988 Act (deductions for expenses etc.);\ but

($b$) where an amount which is deductible in respect of any matter under any of those sections is at least equal to the whole of any corresponding amount which (but for this paragraph) would fall by reference to that matter to be included in those emoluments, the whole of the corresponding amount shall be treated as not so included.
\end{enumerate}

(8) The Treasury may by regulations—
\begin{enumerate}\item[]
($a$) modify the effect of subsection (7)  above by adding any enactment contained in the Income Tax Acts to the list of sections of the 1988 Act contained in paragraph ($a$)  of that subsection; or

($b$) make such amendments of subsection (7)  above as appear to them to be necessary or expedient in consequence of any alteration of the provisions of the Income Tax Acts relating to the charge to tax under Schedule E.
\end{enumerate}

(9) The Treasury may by regulations provide—
\begin{enumerate}\item[]
($a$) for Class 1A contributions not to be payable, in prescribed circumstances, by prescribed persons or in respect of prescribed persons or emoluments;

($b$) for reducing Class 1A contributions in prescribed circumstances.
\end{enumerate}

(10) In this section “the 1988 Act” means the Income and Corporation Taxes Act 1988.”
\end{quotation}

(3) For subsection (6)  of section 4 of that Act (power to treat emoluments in respect of share acquisitions etc.\ as earnings) there shall be substituted—
\begin{quotation}
“(6) Regulations may make provision for the purposes of this Part—
\begin{enumerate}\item[]
($a$) for treating any amount on which an employed earner is chargeable to income tax under Schedule E as remuneration derived from the earner’s employment; and

($b$) for treating any amount which in accordance with regulations under paragraph ($a$)  above constitutes remuneration as an amount of remuneration paid, at such time as may be determined in accordance with the regulations, to or for the benefit of the earner in respect of his employment.”
\end{enumerate}
\end{quotation}

(4) In paragraph 5($b$)  of Schedule 1 to that Act (power to modify section 10 for cases where a car is made available by reason of more than one employment), for “a car is made available” there shall be substituted “something is provided or made available”.

(5) In paragraph 8(1)($ia$) of that Schedule (power to provide by regulations for repayment in prescribed cases of the whole or a part of a Class 1B contribution), after “part” there shall be inserted “of a Class 1A or”.

(6) In section 114(4)  of the Social Security Administration (Northern Ireland) Act 1992 (proof of previous offences relating to Class 1A contributions), for “car” there shall be substituted “amount”.

(7) In section 142(5)($c$)  of that Act (appropriate health service allocation of Class 1A contributions), for “cash equivalents of the benefits of the cars and car fuel” there shall be substituted “emoluments”.

(8) This section shall have effect in relation to the tax year beginning with 6th April 2000 and subsequent tax years.

(9) Regulations made by statutory instrument under any power conferred by virtue of this section may be made so as to have retrospective effect in relation to any time in the tax year in which they are made (including, in the case of regulations made in the tax year in which this Act is passed, any time in that tax year before the passing of this Act).

\subsection{79. Third party providers of benefits in kind: Northern Ireland}

(1) After section 10 of the Social Security Contributions and Benefits (Northern Ireland) Act 1992 there shall be inserted—
\begin{quotation}
\subsection*{“10ZA. Liability of third party provider of benefits in kind}

(1) This section applies, where—
\begin{enumerate}\item[]
($a$) a Class 1A contribution is payable for any tax year in respect of the whole or any part of an emolument received by an earner;

($b$) the emolument, in so far as it is one in respect of which such a contribution is payable, consists in a benefit provided for the earner or a member of his family or household;

($c$) the person providing the benefit is a person other than the person (“the relevant employer”) by whom, but for this section, the Class 1A contribution would be payable in accordance with section 10(2)  above; and

($d$) the provision of the benefit by that other person has not been arranged or facilitated by the relevant employer.
\end{enumerate}

(2) For the purposes of this Act if—
\begin{enumerate}\item[]
($a$) the person providing the benefit pays an amount for the purpose of discharging any liability of the earner to income tax for any tax year, and

($b$) the income tax in question is tax chargeable in respect of the provision of the benefit or of the making of the payment itself,
\end{enumerate}
the amount of the payment shall be treated as if it were an emolument consisting in the provision of a benefit to the earner in that tax year and falling, for the purposes of Class 1 contributions, to be left out of account in the computation of the earnings paid to or for the benefit of the earner.

(3) Subject to subsection (4)  below, the liability to pay any Class 1A contribution in respect of—
\begin{enumerate}\item[]
($a$) the benefit provided to the earner, and

($b$) any further benefit treated as so provided in accordance with subsection (2)  above,
\end{enumerate}
shall fall on the person providing the benefit, instead of on the relevant employer.

(4) Subsection (3)  above applies in the case of a Class 1A contribution for the tax year beginning with 6th April 2000 only if the person providing the benefit in question gives notice in writing to the Inland Revenue on or before 6th July 2001 that he is a person who provides benefits in respect of which a liability to Class 1A contributions is capable of falling by virtue of this section on a person other than the relevant employer.

(5) The Treasury may by regulations make provision specifying the circumstances in which a person is or is not to be treated for the purposes of this Act as having arranged or facilitated the provision of any benefit.

(6) In this section references to a member of a person’s family or household shall be construed in accordance with section 168(4)  of the Income and Corporation Taxes Act 1988. 

\subsection*{10ZB. Non-cash vouchers provided by third parties}

(1) In section 10ZA above references to the provision of a benefit include references to the provision of a non-cash voucher.

(2) Where—
\begin{enumerate}\item[]
($a$) a non-cash voucher is received by any person from employment to which Chapter II of Part V of the Income and Corporation Taxes Act 1988 does not apply, and

($b$) the case would be one in which the conditions in section 10ZA(1)($a$)  to ($d$)  above would be satisfied in relation to the provision of that voucher if that Chapter did apply to that employment,
\end{enumerate}
sections 10 and 10ZA above shall have effect in relation to the provision of that voucher, and to any such payment in respect of the provision of that voucher as is mentioned in section 10ZA(2)  above, as if that employment were employment to which that Chapter applied.

(3) In this section “non-cash voucher” has the same meaning as in section 141 of the Income and Corporation Taxes Act 1988.”
\end{quotation}

(2) After subsection (3)  of section 104ZA of the Social Security Administration (Northern Ireland) Act 1992 (premises liable to inspection) there shall be inserted—
\begin{quotation}
“(3A) The references in subsection (3)  above to a trade or business include references to the administration of any scheme for the provision of benefits to persons by reason of their employment.”
\end{quotation}

(3) Subsection (1)  shall have effect in relation to the tax year beginning with 6th April 2000 and subsequent tax years.

(4) Regulations made by virtue of this section under section 10ZA(5)  of the  Social Security Contributions and Benefits (Northern Ireland) Act 1992 may be made so as to have retrospective effect in relation to any time in the tax year in which they are made (including, in the case of regulations made in the tax year in which this Act is passed, any time in that tax year before the passing of this Act).

\subsection{80. Collection etc.\ of NICs: Northern Ireland.}

(1) Schedule 1 to the Social Security Contributions and Benefits (Northern Ireland) Act 1992 (supplementary provisions relating to contributions) shall be amended in accordance with subsections (2)  to (5).

(2) In paragraph 7(2)($b$)  (application of sections 100 to 100D and 102 to 104 of the Taxes Management Act 1970 in relation to certain penalties), for “104” there shall be substituted “105”.

(3) For sub-paragraph (2)($e$)  of paragraph 7B (power to provide for interest to be charged on late payment in the case of payment outside the PAYE system) there shall be substituted—
\begin{quotation}
“($e$) require interest to be paid on contributions that are not paid by the due date, and provide for determining the date from which such interest is to be calculated;”.
\end{quotation}

(4) After sub-paragraph (5)  of that paragraph there shall be inserted—
\begin{quotation}
“(5A) Regulations under this paragraph may, in relation to any penalty imposed by such regulations, make provision applying (with or without modifications) any enactment applying for the purposes of income tax that is contained in Part X of the Taxes Management Act 1970 (penalties).”
\end{quotation}

(5) After that paragraph there shall be inserted—
\begin{quotation}
“7BA. The Inland Revenue may by regulations provide for amounts in respect of contributions or interest that fall to be paid or repaid in accordance with any regulations under this Schedule to be set off, or to be capable of being set off, in prescribed circumstances and to the prescribed extent, against any such liabilities under regulations under this Schedule of the person entitled to the payment or repayment as may be prescribed.”
\end{quotation}

(6) In Article 7(1)  of the Social Security Contributions (Transfer of Functions, etc.)\ (Northern Ireland) Order 1999 (decisions to be made by an Inland Revenue officer and appealable under Article 10)—
\begin{enumerate}\item[]
($a$) sub-paragraph ($j$)  (interest under regulations made by virtue of paragraph 7B(2)($e$)  of Schedule 1 to the Social Security Contributions and Benefits (Northern Ireland) Act 1992) shall cease to have effect; and

($b$) in sub-paragraph ($l$), for “sub-paragraphs ($j$)  and ($k$)” there shall be substituted “sub-paragraph ($k$)”, and the words “amount of interest or” shall be omitted.
\end{enumerate}

(7) Subsection (6)  has effect in relation to interest accruing on sums becoming due in respect of the tax year beginning with 6th April 2000 or any subsequent tax year.

\subsection{81. Liability of earner for secondary contributions: Northern Ireland}

(1) In paragraph 3 of Schedule 1 to the Social Security Contributions and Benefits (Northern Ireland) Act 1992 (prohibition on deduction or recovery of Class 1 contributions), sub-paragraph (2)  shall be omitted.

(2) After that paragraph there shall be inserted—
\begin{quotation}
\subsection*{“Prohibition on recovery of employer’s contributions}

3A.---(1) Subject to sub-paragraph (2)  below, a person who is or has been liable to pay any secondary Class 1 or any Class 1A or Class 1B contributions shall not—
\begin{enumerate}\item[]
($a$) make, from earnings paid by him, any deduction in respect of any such contributions for which he or any other person is or has been liable;

($b$) otherwise recover any such contributions (directly or indirectly) from any person who is or has been a relevant earner; or

($c$) enter into any agreement with any person for the making of any such deduction or otherwise for the purpose of so recovering any such contributions.
\end{enumerate}

(2) Sub-paragraph (1)  above does not apply to the extent that an agreement between—
\begin{enumerate}\item[]
($a$) a secondary contributor, and

($b$) any person (“the earner”) in relation to whom the secondary contributor is, was or will be such a contributor in respect of the contributions to which the agreement relates,
\end{enumerate}
allows the secondary contributor to recover (whether by deduction or otherwise) the whole or any part of any secondary Class 1 contribution payable in respect of a gain that is treated as remuneration derived from that earner’s employment by virtue of section 4(4)($a$)  above.

(3) Sub-paragraph (2)  above does not authorise any recovery (whether by deduction or otherwise)—
\begin{enumerate}\item[]
($a$) in pursuance of any agreement entered into before 19th May 2000; or

($b$) in respect of any liability to a contribution arising before the day of the passing of the Child Support, Pensions and Social Security Act 2000. 
\end{enumerate}

(4) In this paragraph—
\begin{enumerate}\item[]
    “agreement” includes any arrangement or understanding (whether or not legally enforceable); and

    “relevant earner”, in relation to a person who is or has been liable to pay any contributions, means an earner in respect of whom he is or has been so liable. 
\end{enumerate}

\subsection*{Transfer of liability to be borne by earner}

3B.---(1) This paragraph applies where—
\begin{enumerate}\item[]
($a$) an election is jointly made by—
\begin{enumerate}\item[]
(i) a secondary contributor, and

(ii) a person (“the earner”) in relation to whom the secondary contributor is or will be such a contributor in respect of contributions on share option gains by the earner,
\end{enumerate}
for the whole or a part of any liability of the secondary contributor to contributions on any such gains to be transferred to the earner; and

($b$) the election is one in respect of which the Inland Revenue have, before it was made, given by notice to the secondary contributor their approval to both—
\begin{enumerate}\item[]
(i) the form of the election; and

(ii) the arrangements made in relation to the proposed election for securing that the liability transferred by the election will be met.
\end{enumerate}
\end{enumerate}

(2) Any liability which—
\begin{enumerate}\item[]
($a$) arises while the election is in force, and

($b$) is a liability to pay the contributions on share option gains by the earner, or the part of them, to which the election relates,
\end{enumerate}
shall be treated for the purposes of this Act, the Administration Act and Part III of the Social Security Contributions (Transfer of Functions, etc.)\ (Northern Ireland) Order 1999 as a liability falling on the earner, instead of on the secondary contributor.

(3) Subject to sub-paragraph (7)($b$)  below, an election made for the purposes of sub-paragraph (1)  above shall continue in force from the time when it is made until whichever of the following first occurs, namely—
\begin{enumerate}\item[]
($a$) it ceases to have effect in accordance with its terms;

($b$) it is revoked jointly by both parties to the election;

($c$) notice is given to the earner by the secondary contributor terminating the effect of the election.
\end{enumerate}

(4) An approval given to the secondary contributor for the purposes of sub-paragraph (1)($b$)  above may be given either—
\begin{enumerate}\item[]
($a$) for an election to be made by the secondary contributor and a particular person; or

($b$) for all elections to be made, or to be made in particular circumstances, by the secondary contributor and particular persons or by the secondary contributor and persons of a particular description.
\end{enumerate}

(5) The grounds on which the Inland Revenue shall be entitled to refuse an approval for the purposes of sub-paragraph (1)($b$)  above shall include each of the following—
\begin{enumerate}\item[]
($a$) that it appears to the Inland Revenue that adequate arrangements have not been made for securing that the liabilities transferred by the proposed election or elections will be met by the person or persons to whom they would be so transferred; and

($b$) that it appears to the Inland Revenue that they do not have sufficient information to determine whether or not grounds falling within paragraph ($a$)  above exist.
\end{enumerate}

(6) If, at any time after they have given an approval for the purposes of sub-paragraph (1)($b$)  above, it appears to the Inland Revenue—
\begin{enumerate}\item[]
($a$) that the arrangements that were made or are in force for securing that liabilities transferred by elections to which the approval relates are met are proving inadequate or unsatisfactory in any respect, or

($b$) that any election to which the approval relates has resulted, or is likely to result, in the avoidance or non-payment of the whole or any part of any secondary Class 1 contributions,
\end{enumerate}
the Inland Revenue may withdraw the approval by notice to the secondary contributor.

(7) The withdrawal by the Inland Revenue of any approval given for the purposes of sub-paragraph (1)($b$)  above—
\begin{enumerate}\item[]
($a$) may be either general or confined to a particular election or to particular elections; and

($b$) shall have the effect that the election to which the withdrawal relates has no effect on contributions on share option gains in respect of any right to acquire shares obtained after—
\begin{enumerate}\item[]
(i) the date on which notice of the withdrawal of the approval is given; or

(ii) such later date as the Inland Revenue may specify in that notice.
\end{enumerate}
\end{enumerate}

(8) Where the Inland Revenue have refused or withdrawn their approval for the purposes of sub-paragraph (1)($b$)  above, the person who applied for it or, as the case may be, to whom it was given may appeal to the Special Commissioners against the Inland Revenue’s decision.

(9) On an appeal under sub-paragraph (8)  above the Special Commissioners may—
\begin{enumerate}\item[]
($a$) dismiss the appeal;

($b$) remit the decision appealed against to the Inland Revenue with a direction to make such decision as the Special Commissioners think fit; or

($c$) in the case of a decision to withdraw an approval, quash that decision and direct that that decision is to be treated as never having been made.
\end{enumerate}

(10) Subject to sub-paragraph (12)  below, an election under sub-paragraph (1)  above shall not apply to any contributions in respect of gains realised before it was made.

(11) Regulations made by the Inland Revenue may make provision with respect to the making of elections for the purposes of this paragraph and the giving of approvals for the purposes of sub-paragraph (1)($b$)  above; and any such regulations may, in particular—
\begin{enumerate}\item[]
($a$) prescribe the matters that must be contained in such an election;

($b$) provide for the manner in which such an election is to be capable of being made and of being confined to particular liabilities or the part of particular liabilities; and

($c$) provide for the making of applications for such approvals and for the manner in which those applications are to be dealt with.
\end{enumerate}

(12) Where—
\begin{enumerate}\item[]
($a$) an election is made under this paragraph before the end of the period of three months beginning with the date of the passing of the Child Support, Pensions and Social Security Act 2000, and

($b$) that election is expressed to relate to liabilities for contributions arising on or after 19th May 2000 and before the making of the election,
\end{enumerate}
this paragraph shall have effect in relation to those liabilities as if sub-paragraph (2)  above provided for them to be deemed to have fallen on the earner (instead of on the secondary contributor); and the secondary contributor shall accordingly be entitled to reimbursement from the earner for any payment made by that contributor in or towards the discharge of any of those liabilities.

(13) In this paragraph references to contributions on share option gains by the earner are references to any secondary Class 1 contributions payable in respect of a gain that is treated as remuneration derived from the earner’s employment by virtue of section 4(4)($a$)  above.

(14) In this paragraph “the Special Commissioners” means the Commissioners for the special purposes of the Income Tax Acts.”
\end{quotation}

(3) In section 6(4)  of that Act (persons by whom Class 1 contributions are payable), for the words from “paragraph 3” onwards there shall be substituted “paragraphs 3 to 3B of Schedule 1 to this Act.”

(4) In paragraph 8(1)  of Schedule 1 to that Act (general regulations), after paragraph ($c$)  there shall be inserted—
\begin{quotation}
“($ca$) for requiring a secondary contributor to notify a person to whom any of his liabilities are transferred by an election under paragraph 3B above of—
\begin{enumerate}\item[]
(i) any transferred liability that arises;

(ii) the amount of any transferred liability that arises; and

(iii) the contents of any notice of withdrawal by the Inland Revenue of any approval that relates to that election;”.
\end{enumerate}
\end{quotation}

(5) In Article 7(1)  of the Social Security Contributions (Transfer of Functions, etc.)\ (Northern Ireland) Order 1999 (decisions to be taken by officers of the Inland Revenue), after sub-paragraph ($i$)  there shall be inserted—
\begin{quotation}
“($ia$) to decide whether to give or withdraw an approval for the purposes of paragraph 3B(1)($b$)  of Schedule 1 to the Contributions and Benefits Act;”.
\end{quotation}

(6) In Article 9 of that Order (regulations about varying or superseding decisions), at the beginning of paragraph (1)  there shall be inserted “Subject to paragraph (2A)  below,”, and after paragraph (2)  there shall be inserted—
\begin{quotation}
“(2A) The decisions in relation to which provision may be made by regulations under this Article shall not include decisions falling within Article 7(1)($ia$) of this Order.”
\end{quotation}

(7) In Article 11(4)  of that Order (appeals to be heard by General Commissioners), after “Subject to” there shall be inserted “paragraph 3B(8)  of Schedule 1 to the Contributions and Benefits Act (which provides for appeals under that paragraph to be heard by the Special Commissioners), to”.

\part[Part V --- Miscellaneous and supplemental]{Part V\\*Miscellaneous and supplemental}

\renewcommand\parthead{--- Part V}

\section{\itshape Miscellaneous}

\subsection{82. Tests for determining parentage}

(1) Part III of the Family Law Reform Act 1969 (tests for determining parentage) shall be amended in accordance with subsections (2)  to (4).

(2) In section 20 (power of the court to require tests)—
\begin{enumerate}\item[]
($a$) for subsections (1A)  and (1B)  (nomination of the person by whom tests are to be carried out) there shall be substituted—
\begin{quotation}
“(1A) Tests required by a direction under this section may only be carried out by a body which has been accredited for the purposes of this section by—
\begin{enumerate}\item[]
($a$) the Lord Chancellor, or

($b$) a body appointed by him for the purpose.”;
\end{enumerate}
\end{quotation}

($b$) in subsection (2)—
\begin{enumerate}\item[]
(i) for “person responsible for” there shall be substituted “individual”, and

(ii) after “this section” there shall be inserted “(“the tester”)”;
\end{enumerate}

($c$) in subsection (4), for “the person who made the report” there shall be substituted “the tester”; and

($d$) in subsection (5)—
\begin{enumerate}\item[]
(i) for “the person responsible for carrying out the tests taken for the purpose of giving effect to the direction, or any” there shall be substituted “the tester, or any other”,

(ii) for “that person” there shall be substituted “the tester or that other person”, and

(iii) after “and where” there shall be inserted “the tester or”.
\end{enumerate}
\end{enumerate}

(3) In section 21 (consents, etc, required for the taking of blood samples), in subsection (3), for the words “if the person who has the care and control of him consents” there shall be substituted—
\begin{quotation}
“($a$) if the person who has the care and control of him consents; or

($b$) where that person does not consent, if the court considers that it would be in his best interests for the sample to be taken.”
\end{quotation}

(4) In section 22(1)  (power of Lord Chancellor to make further provision relating to tests for determining parentage)—
\begin{enumerate}\item[]
($a$) in paragraph ($a$)  (power to provide that bodily samples are not to be taken except by such medical practitioners as may be appointed by the Lord Chancellor), for the words from “such medical practitioners” to the end there shall be substituted “registered medical practitioners or members of such professional bodies as may be prescribed by the regulations;”, and

($b$) for paragraph ($e$)  (power to provide that scientific tests are not to be carried out except by persons appointed by the Lord Chancellor) there shall be substituted—
\begin{quotation}
“($e$) prescribe conditions which a body must meet in order to be eligible for accreditation for the purposes of section 20 of this Act;”.
\end{quotation}
\end{enumerate}

(5) The amendments made by this section shall not have effect in relation to any proceedings pending at the commencement of this section.

\subsection{83. Declarations of status}

(1) Part III of the Family Law Act 1986 (declarations of status) shall be amended as follows.

(2) After section 55 there shall be inserted—
\begin{quotation}
\subsection*{“55A. Declarations of parentage}

(1) Subject to the following provisions of this section, any person may apply to the High Court, a county court or a magistrates' court for a declaration as to whether or not a person named in the application is or was the parent of another person so named.

(2) A court shall have jurisdiction to entertain an application under subsection (1)  above if, and only if, either of the persons named in it for the purposes of that subsection—
\begin{enumerate}\item[]
($a$) is domiciled in England and Wales on the date of the application, or

($b$) has been habitually resident in England and Wales throughout the period of one year ending with that date, or

($c$) died before that date and either—
\begin{enumerate}\item[]
(i) was at death domiciled in England and Wales, or

(ii) had been habitually resident in England and Wales throughout the period of one year ending with the date of death.
\end{enumerate}
\end{enumerate}

(3) Except in a case falling within subsection (4)  below, the court shall refuse to hear an application under subsection (1)  above unless it considers that the applicant has a sufficient personal interest in the determination of the application (but this is subject to section 27 of the Child Support Act 1991).

(4) The excepted cases are where the declaration sought is as to whether or not—
\begin{enumerate}\item[]
($a$) the applicant is the parent of a named person;

($b$) a named person is the parent of the applicant; or

($c$) a named person is the other parent of a named child of the applicant.
\end{enumerate}

(5) Where an application under subsection (1)  above is made and one of the persons named in it for the purposes of that subsection is a child, the court may refuse to hear the application if it considers that the determination of the application would not be in the best interests of the child.

(6) Where a court refuses to hear an application under subsection (1)  above it may order that the applicant may not apply again for the same declaration without leave of the court.

(7) Where a declaration is made by a court on an application under subsection (1)  above, the prescribed officer of the court shall notify the Registrar General, in such a manner and within such period as may be prescribed, of the making of that declaration.”
\end{quotation}

(3) Section 58(5)($b$)  (prohibition of declarations of illegitimacy) shall be omitted.

(4) After section 60(4)  there shall be inserted—
\begin{quotation}
“(5) An appeal shall lie to the High Court against—
\begin{enumerate}\item[]
($a$) the making by a magistrates' court of a declaration under section 55A above,

($b$) any refusal by a magistrates' court to make such a declaration, or

($c$) any order under subsection (6)  of that section made on such a refusal.”
\end{enumerate}
\end{quotation}

(5) Schedule 8 (which makes amendments consequential on subsection (1)) shall have effect.

(6) Nothing in this Act shall affect any proceedings pursuant to an application under—
\begin{enumerate}\item[]
($a$) section 56(1)($a$)  of the Family Law Act 1986, or

($b$) section 27 of the Child Support Act 1991,
\end{enumerate}
which are pending immediately before the commencement of this section.

\section{\itshape Supplemental}

\subsection{84. Expenses}

There shall be paid out of money provided by Parliament—
\begin{enumerate}\item[]
($a$) any expenditure incurred by the Secretary of State for or in connection with the carrying out of his functions under this Act; and

($b$) any increase attributable to this Act in the sums which are payable out of money so provided under any other Act.
\end{enumerate}

\subsection{85. Repeals}

(1) The enactments mentioned in Schedule 9 (which include some spent provisions) are hereby repealed to the extent specified in the third column of that Schedule.

(2) The repeals specified in that Schedule have effect subject to the commencement provisions and savings contained, or referred to, in the notes set out in that Schedule.

\subsection{86. Commencement and transitional provisions}

(1) This section applies to the following provisions of this Act—
\begin{enumerate}\item[]
($a$) Part I (other than section 24);

($b$) Part II (other than sections 38 and 39 and paragraphs 4 to 6, 8(1), (3)  and (4)  and 13 of Schedule 5);

($c$) Part III;

($d$) sections 82 and 83 and Schedule 8;

($e$) Parts I to VII and IX of Schedule 9. 
\end{enumerate}

(2) The provisions of this Act to which this section applies shall come into force on such day as may be appointed by order made by statutory instrument; and different days may be appointed under this section for different purposes.

(3) The power to make an order under subsection (2)  shall be exercisable—
\begin{enumerate}\item[]
($a$) except in a case falling within paragraph ($b$), by the Secretary of State; and

($b$) in the case of an order bringing into force any of the provisions of sections 82 and 83, Schedule 8 or Part IX of Schedule 9, by the Lord Chancellor.
\end{enumerate}

(4) In the case of Part I (other than section 24) and of sections 62 to 66, the power under subsection (2)  to appoint different days for different purposes includes power to appoint different days for different areas.

(5) The Secretary of State may by regulations make such transitional provision as he considers necessary or expedient in connection with the bringing into force of any of the following provisions of this Act—
\begin{enumerate}\item[]
($a$) sections 43 to 46 and section (1)  of Part III of Schedule 9;

($b$) sections 68 to 70 and Schedule 7 and Part VII of Schedule 9. 
\end{enumerate}

(6) Regulations under subsection (5)  shall be made by statutory instrument subject to annulment in pursuance of a resolution of either House of Parliament.

(7) Section 174(2)  to (4)  of the Pensions Act 1995 (supplementary provision in relation to powers to make subordinate legislation under that Act) shall apply in relation to the power to make regulations under subsection (5)  as it applies to any power to make regulations under that Act.

(8) In this section “subordinate legislation” has the same meaning as in the  Interpretation Act 1978. 

\subsection{87. Short title and extent}

(1) This Act may be cited as the Child Support, Pensions and Social Security Act 2000. 

(2) The following provisions of this Act extend to Northern Ireland—
\begin{enumerate}\item[]
($a$) so much of section 46 as amends section 21(3)  of the Pensions Act 1995;

($b$) sections 57 to 61 (except section 60(5));

($c$) section 73;

($d$) sections 78 to 81;

($e$) in Schedule 3, paragraphs 8 and 9, and in paragraph 11, sub-paragraph (2)  (and sub-paragraph (1)  so far as it relates to that sub-paragraph);

($f$) paragraph 6 of Schedule 5; and

($g$) this Part, except—
\begin{enumerate}\item[]
(i) sections 82 and 83 and Schedule 8; and

(ii) so much of this Part as gives effect to any repeal other than the repeals mentioned in subsection (3).
\end{enumerate}
\end{enumerate}

(3) The repeals mentioned in subsection (2)($g$)  (which extend to Northern Ireland) are—
\begin{enumerate}\item[]
($a$) the repeals, in Part I of Schedule 9, that relate to the Tax Credits Act 1999;

($b$) the repeals, in sections (1), (6)  and (11)  of Part III of that Schedule, that relate to—
\begin{enumerate}\item[]
(i) section 21(3)  of the Pensions Act 1995;

(ii) paragraph 49($a$)(ii)  of Schedule 3 to the Pensions (Northern Ireland) Order 1995; and

(iii) section 52(5)  of the Pension Schemes (Northern Ireland) Act 1993;
\end{enumerate}

($c$) the repeals in Part IV of that Schedule (except so far as relating to the Courts and Legal Services Act 1990); and

($d$) the repeals in section (2)  of Part VIII of that Schedule.
\end{enumerate}

(4) Subject to that, this Act does not extend to Northern Ireland.

\small

\part*{SCHEDULES}

\part[Schedule 1 --- Substituted Part I of Schedule 1 to the Child Support Act 1991]{Schedule 1\\*Substituted Part I of Schedule 1 to the Child Support Act 1991}

\renewcommand\parthead{--- Schedule 1}

\amendment{
Sch. 1 is in force only for new-rules cases; see the Child Support, Pensions and Social Security Act 2000 (Commencement No. 12) Order 2003 art. 3.
}

\begin{quotation}
\part*{\noindent “Part I\\*Calculation of weekly amount of child support maintenance}

\section*{General rule}

1.---(1) The weekly rate of child support maintenance is the basic rate unless a reduced rate, a flat rate or the nil rate applies.

(2) Unless the nil rate applies, the amount payable weekly to a person with care is—
\begin{enumerate}\item[]
($a$) the applicable rate, if paragraph 6 does not apply; or

($b$) if paragraph 6 does apply, that rate as apportioned between the persons with care in accordance with paragraph 6,
\end{enumerate}
as adjusted, in either case, by applying the rules about shared care in paragraph 7 or 8. 

\section*{Basic rate}

2.---(1) The basic rate is the following percentage of the non-resident parent’s net weekly income—
\begin{enumerate}\item[]
    15\% where he has one qualifying child;

    20\% where he has two qualifying children;

    25\% where he has three or more qualifying children. 
\end{enumerate}

(2) If the non-resident parent also has one or more relevant other children, the appropriate percentage referred to in sub-paragraph (1)  is to be applied instead to his net weekly income less—
\begin{enumerate}\item[]
    15\% where he has one relevant other child;

    20\% where he has two relevant other children;

    25\% where he has three or more relevant other children. 
\end{enumerate}

\section*{Reduced rate}

3.---(1) A reduced rate is payable if—
\begin{enumerate}\item[]
($a$) neither a flat rate nor the nil rate applies; and

($b$) the non-resident parent’s net weekly income is less than £200 but more than £100. 
\end{enumerate}

(2) The reduced rate payable shall be prescribed in, or determined in accordance with, regulations.

(3) The regulations may not prescribe, or result in, a rate of less than £5. 

\section*{Flat rate}

4.---(1) Except in a case falling within sub-paragraph (2), a flat rate of £5 is payable if the nil rate does not apply and—
\begin{enumerate}\item[]
($a$) the non-resident parent’s net weekly income is £100 or less; or

($b$) he receives any benefit, pension or allowance prescribed for the purposes of this paragraph of this sub-paragraph; or

($c$) he or his partner (if any) receives any benefit prescribed for the purposes of this paragraph of this sub-paragraph.
\end{enumerate}

(2) A flat rate of a prescribed amount is payable if the nil rate does not apply and—
\begin{enumerate}\item[]
($a$) the non-resident parent has a partner who is also a non-resident parent;

($b$) the partner is a person with respect to whom a maintenance calculation is in force; and

($c$) the non-resident parent or his partner receives any benefit prescribed under sub-paragraph (1)($c$).
\end{enumerate}

(3) The benefits, pensions and allowances which may be prescribed for the purposes of sub-paragraph (1)($b$)  include ones paid to the non-resident parent under the law of a place outside the United Kingdom.

\section*{Nil rate}

5. The rate payable is nil if the non-resident parent—
\begin{enumerate}\item[]
($a$) is of a prescribed description; or

($b$) has a net weekly income of below £5. 
\end{enumerate}

\section*{Apportionment}

6.---(1) If the non-resident parent has more than one qualifying child and in relation to them there is more than one person with care, the amount of child support maintenance payable is (subject to paragraph 7 or 8) to be determined by apportioning the rate between the persons with care.

(2) The rate of maintenance liability is to be divided by the number of qualifying children, and shared among the persons with care according to the number of qualifying children in relation to whom each is a person with care.

\section*{Shared care—basic and reduced rate}

7.---(1) This paragraph applies only if the rate of child support maintenance payable is the basic rate or a reduced rate.

(2) If the care of a qualifying child is shared between the non-resident parent and the person with care, so that the non-resident parent from time to time has care of the child overnight, the amount of child support maintenance which he would otherwise have been liable to pay the person with care, as calculated in accordance with the preceding paragraphs of this Part of this Schedule, is to be decreased in accordance with this paragraph.

(3) First, there is to be a decrease according to the number of such nights which the Secretary of State determines there to have been, or expects there to be, or both during a prescribed twelve-month period.

(4) The amount of that decrease for one child is set out in the following Table—

\medskip

{\footnotesize\noindent
%\begin{tabulary}{\linewidth}{JJ}
\begin{longtable}{ll}
\hline
\itshape Number of nights	& \itshape Fraction to subtract\\
\hline
\endhead
\hline
\endlastfoot
52 to 103	&One-seventh\\
104 to 155	&Two-sevenths\\
156 to 174	&Three-sevenths\\
175 or more	&One-half\\
%\hline
%\end{tabulary}
\end{longtable}

}

\medskip

(5) If the person with care is caring for more than one qualifying child of the non-resident parent, the applicable decrease is the sum of the appropriate fractions in the Table divided by the number of such qualifying children.

(6) If the applicable fraction is one-half in relation to any qualifying child in the care of the person with care, the total amount payable to the person with care is then to be further decreased by £7 for each such child.

(7) If the application of the preceding provisions of this paragraph would decrease the weekly amount of child support maintenance (or the aggregate of all such amounts) payable by the non-resident parent to the person with care (or all of them) to less than £5, he is instead liable to pay child support maintenance at the rate of £5 a week, apportioned (if appropriate) in accordance with paragraph 6. 

\section*{Shared care—flat rate}

8.---(1) This paragraph applies only if—
\begin{enumerate}\item[]
($a$) the rate of child support maintenance payable is a flat rate; and

($b$) that rate applies because the non-resident parent falls within paragraph 4(1)($b$)  or ($c$)  or 4(2).
\end{enumerate}

(2) If the care of a qualifying child is shared as mentioned in paragraph 7(2)  for at least 52 nights during a prescribed 12-month period, the amount of child support maintenance payable by the non-resident parent to the person with care of that child is nil.

\section*{Regulations about shared care}

9. The Secretary of State may by regulations provide—
\begin{enumerate}\item[]
($a$) for which nights are to count for the purposes of shared care under paragraphs 7 and 8, or for how it is to be determined whether a night counts;

($b$) for what counts, or does not count, as “care” for those purposes; and

($c$) for paragraph 7(3)  or 8(2)  to have effect, in prescribed circumstances, as if the period mentioned there were other than 12 months, and in such circumstances for the Table in paragraph 7(4)  (or that Table as modified pursuant to regulations made under paragraph 10A(2)($a$)), or the period mentioned in paragraph 8(2), to have effect with prescribed adjustments.
\end{enumerate}

\section*{Net weekly income}

10.---(1) For the purposes of this Schedule, net weekly income is to be determined in such manner as is provided for in regulations.

(2) The regulations may, in particular, provide for the Secretary of State to estimate any income or make an assumption as to any fact where, in his view, the information at his disposal is unreliable, insufficient, or relates to an atypical period in the life of the non-resident parent.

(3) Any amount of net weekly income (calculated as above) over £2,000 is to be ignored for the purposes of this Schedule.

\section*{Regulations about rates, figures, etc.}

10A.---(1) The Secretary of State may by regulations provide that—
\begin{enumerate}\item[]
($a$) paragraph 2 is to have effect as if different percentages were substituted for those set out there;

($b$) paragraph 3(1)  or (3), 4(1), 5, 7(7)  or 10(3)  is to have effect as if different amounts were substituted for those set out there.
\end{enumerate}

(2) The Secretary of State may by regulations provide that—
\begin{enumerate}\item[]
($a$) the Table in paragraph 7(4)  is to have effect as if different numbers of nights were set out in the first column and different fractions were substituted for those set out in the second column;

($b$) paragraph 7(6)  is to have effect as if a different amount were substituted for that mentioned there, or as if the amount were an aggregate amount and not an amount for each qualifying child, or both.
\end{enumerate}

\section*{Regulations about income}

10B. The Secretary of State may by regulations provide that, in such circumstances and to such extent as may be prescribed—
\begin{enumerate}\item[]
($a$) where the Secretary of State is satisfied that a person has intentionally deprived himself of a source of income with a view to reducing the amount of his net weekly income, his net weekly income shall be taken to include income from that source of an amount estimated by the Secretary of State;

($b$) a person is to be treated as possessing income which he does not possess;

($c$) income which a person does possess is to be disregarded.
\end{enumerate}

\section*{References to various terms}

10C.---(1) References in this Part of this Schedule to “qualifying children” are to those qualifying children with respect to whom the maintenance calculation falls to be made.

(2) References in this Part of this Schedule to “relevant other children” are to—
\begin{enumerate}\item[]
($a$) children other than qualifying children in respect of whom the non-resident parent or his partner receives child benefit under Part IX of the Social Security Contributions and Benefits Act 1992; and

($b$) such other description of children as may be prescribed.
\end{enumerate}

(3) In this Part of this Schedule, a person “receives” a benefit, pension, or allowance for any week if it is paid or due to be paid to him in respect of that week.

(4) In this Part of this Schedule, a person’s “partner” is—
\begin{enumerate}\item[]
($a$) if they are a couple, the other member of that couple;

($b$) if the person is a husband or wife by virtue of a marriage entered into under a law which permits polygamy, another party to the marriage who is of the opposite sex and is a member of the same household.
\end{enumerate}

(5) In sub-paragraph (4)($a$), “couple” means a man and a woman who are—
\begin{enumerate}\item[]
($a$) married to each other and are members of the same household; or

($b$) not married to each other but are living together as husband and wife.”.
\end{enumerate}
\end{quotation}

\part[Schedule 2 --- Substituted Schedules 4A and 4B to the 1991 Act]{Schedule 2\\*Substituted Schedules 4A and 4B to the 1991 Act}

\renewcommand\parthead{--- Schedule 2 Part I}

\amendment{
Sch. 2 is in force only for new-rules cases; see the Child Support, Pensions and Social Security Act 2000 (Commencement No. 12) Order 2003 art. 3.
}

\section[Part I --- Substituted Schedule 4A]{Part I\\*Substituted Schedule 4A}

\begin{quotation}
\part*{\noindent “S\lowercase{CHEDULE} 4A\\*Applications for a variation}

\section*{Interpretation}

1. In this Schedule, “regulations” means regulations made by the Secretary of State.

\section*{Applications for a variation}

2. Regulations may make provision—
\begin{enumerate}\item[]
($a$) as to the procedure to be followed in considering an application for a variation;

($b$) as to the procedure to be followed when an application for a variation is referred to an appeal tribunal under section 28D(1)($b$).
\end{enumerate}

\section*{Completion of preliminary consideration}

3. Regulations may provide for determining when the preliminary consideration of an application for a variation is to be taken to have been completed.

\section*{Information}

4. If any information which is required (by regulations under this Act) to be furnished to the Secretary of State in connection with an application for a variation has not been furnished within such period as may be prescribed, the Secretary of State may nevertheless proceed to consider the application.

\section*{Joint consideration of applications for a variation and appeals}

5.---(1) Regulations may provide for two or more applications for a variation with respect to the same application for a maintenance calculation to be considered together.

(2) In sub-paragraph (1), the reference to an application for a maintenance calculation includes an application treated as having been made under section 6. 

(3) An appeal tribunal considering an application for a variation under section 28D(1)($b$)  may consider it at the same time as an appeal under section 20 in connection with an interim maintenance decision, if it considers that to be appropriate.”
\end{quotation}

\section[Part II --- Substituted Schedule 4B]{Part II\\*Substituted Schedule 4B}

\renewcommand\parthead{--- Schedule 2 Part II}

\begin{quotation}
\part*{\noindent “S\lowercase{CHEDULE} 4B\\*Applications for a variation: The Cases and Controls}

\section*{Part I\\*The Cases}

\subsection*{\itshape General}

1.---(1) The cases in which a variation may be agreed are those set out in this Part of this Schedule or in regulations made under this Part.

(2) In this Schedule “applicant” means the person whose application for a variation is being considered.

\subsection*{\itshape Special expenses}

2.---(1) A variation applied for by a non-resident parent may be agreed with respect to his special expenses.

(2) In this paragraph “special expenses” means the whole, or any amount above a prescribed amount, or any prescribed part, of expenses which fall within a prescribed description of expenses.

(3) In prescribing descriptions of expenses for the purposes of this paragraph, the Secretary of State may, in particular, make provision with respect to—
\begin{enumerate}\item[]
($a$) costs incurred by a non-resident parent in maintaining contact with the child, or with any of the children, with respect to whom the application for a maintenance calculation has been made (or treated as made);

($b$) costs attributable to a long-term illness or disability of a relevant other child (within the meaning of paragraph 10C(2)  of Schedule 1);

($c$) debts of a prescribed description incurred, before the non-resident parent became a non-resident parent in relation to a child with respect to whom the maintenance calculation has been applied for (or treated as having been applied for)—
\begin{enumerate}\item[]
(i) for the joint benefit of both parents;

(ii) for the benefit of any such child; or

(iii) for the benefit of any other child falling within a prescribed category;
\end{enumerate}

($d$) boarding school fees for a child in relation to whom the application for a maintenance calculation has been made (or treated as made);

($e$) the cost to the non-resident parent of making payments in relation to a mortgage on the home he and the person with care shared, if he no longer has an interest in it, and she and a child in relation to whom the application for a maintenance calculation has been made (or treated as made) still live there.
\end{enumerate}

(4) For the purposes of sub-paragraph (3)($b$)—
\begin{enumerate}\item[]
($a$) “disability” and “illness” have such meaning as may be prescribed; and

($b$) the question whether an illness or disability is long-term shall be determined in accordance with regulations made by the Secretary of State.
\end{enumerate}

(5) For the purposes of sub-paragraph (3)($d$), the Secretary of State may prescribe—
\begin{enumerate}\item[]
($a$) the meaning of “boarding school fees”; and

($b$) components of such fees (whether or not itemised as such) which are, or are not, to be taken into account,
\end{enumerate}
and may provide for estimating any such component.

\subsection*{\itshape Property or capital transfers}

3.---(1) A variation may be agreed in the circumstances set out in sub-paragraph (2)  if before 5th April 1993—
\begin{enumerate}\item[]
($a$) a court order of a prescribed kind was in force with respect to the non-resident parent and either the person with care with respect to the application for the maintenance calculation or the child, or any of the children, with respect to whom that application was made; or

($b$) an agreement of a prescribed kind between the non-resident parent and any of those persons was in force.
\end{enumerate}

(2) The circumstances are that in consequence of one or more transfers of property of a prescribed kind and exceeding (singly or in aggregate) a prescribed minimum value—
\begin{enumerate}\item[]
($a$) the amount payable by the non-resident parent by way of maintenance was less than would have been the case had that transfer or those transfers not been made; or

($b$) no amount was payable by the non-resident parent by way of maintenance.
\end{enumerate}

(3) For the purposes of sub-paragraph (2), “maintenance” means periodical payments of maintenance made (otherwise than under this Act) with respect to the child, or any of the children, with respect to whom the application for a maintenance calculation has been made.

\subsection*{Additional cases}

4.---(1) The Secretary of State may by regulations prescribe other cases in which a variation may be agreed.

(2) Regulations under this paragraph may, for example, make provision with respect to cases where—
\begin{enumerate}\item[]
($a$) the non-resident parent has assets which exceed a prescribed value;

($b$) a person’s lifestyle is inconsistent with his income for the purposes of a calculation made under Part I of Schedule 1;

($c$) a person has income which is not taken into account in such a calculation;

($d$) a person has unreasonably reduced the income which is taken into account in such a calculation.
\end{enumerate}

\section*{Part II\\*Regulatory Controls}

5.---(1) The Secretary of State may by regulations make provision with respect to the variations from the usual rules for calculating maintenance which may be allowed when a variation is agreed.

(2) No variations may be made other than those which are permitted by the regulations.

(3) Regulations under this paragraph may, in particular, make provision for a variation to result in—
\begin{enumerate}\item[]
($a$) a person’s being treated as having more, or less, income than would be taken into account without the variation in a calculation under Part I of Schedule 1;

($b$) a person’s being treated as liable to pay a higher, or a lower, amount of child support maintenance than would result without the variation from a calculation under that Part.
\end{enumerate}

(4) Regulations may provide for the amount of any special expenses to be taken into account in a case falling within paragraph 2, for the purposes of a variation, not to exceed such amount as may be prescribed or as may be determined in accordance with the regulations.

(5) Any regulations under this paragraph may in particular make different provision with respect to different levels of income.

\medskip

6. The Secretary of State may by regulations provide for the application, in connection with child support maintenance payable following a variation, of paragraph 7(2)  to (7)  of Schedule 1 (subject to any prescribed modifications).”
\end{quotation}

\part[Schedule 3 --- Amendment of enactments relating to child support]{Schedule 3\\*Amendment of enactments relating to child support}

\renewcommand\parthead{--- Schedule 3}

\section*{\itshape The Army Act 1955}

1.---(1) Section 150A of the Army Act 1955 (enforcement of maintenance assessment by deductions from pay) shall be amended as follows.

(2) In subsections (1), (2)($a$), (3)($a$)  (twice) and (4), for “maintenance assessment” there shall be substituted “maintenance calculation”.

(3) In subsection (3)  (twice), for “the assessment” there shall be substituted “the calculation”.

\amendment{
Para. 1 is in force only for new-rules cases; see the Child Support, Pensions and Social Security Act 2000 (Commencement No. 12) Order 2003 art. 3.
}

\section*{\itshape The Air Force Act 1955}

2.---(1) Section 150A of the Air Force Act 1955 (enforcement of maintenance assessment by deductions from pay) shall be amended as follows.

(2) In subsections (1), (2)($a$), (3)($a$)  (twice) and (4), for “maintenance assessment” there shall be substituted “maintenance calculation”.

(3) In subsection (3)  (twice), for “the assessment” there shall be substituted “the calculation”.

\amendment{
Para. 2 is in force only for new-rules cases; see the Child Support, Pensions and Social Security Act 2000 (Commencement No. 12) Order 2003 art. 3.
}

\section*{\itshape The Matrimonial Causes Act 1973}

3.---(1) The Matrimonial Causes Act 1973 shall be amended as follows.

(2) In section 29 (duration of continuing financial provision orders in favour of children, and age limit on making certain orders in their favour)—
\begin{enumerate}\item[]
($a$) in subsections (5)($a$), (7)  (three times) and (8)($a$), for “maintenance assessment” there shall be substituted “maintenance calculation”;

($b$) in subsections (5)($a$)  and ($b$)(ii)  and (6)($b$), for “current assessment” there shall be substituted “current calculation”;

($c$) in subsection (6)($b$), for “maintenance assessments” there shall be substituted “maintenance calculations”; and

($d$) in subsection (6)($b$), for “those assessments” there shall be substituted “those calculations”.
\end{enumerate}

(3) In section 31 (variation, discharge, etc, of certain orders for financial relief)—
\begin{enumerate}\item[]
($a$) in subsections (11)($c$)  and (12)($a$)  and ($c$), for “maintenance assessment” there shall be substituted “maintenance calculation”; and

($b$) in subsection (11) (twice), for “the assessment” there shall be substituted “the calculation”.
\end{enumerate}

(4) In section 52 (interpretation), in subsection (1), for “maintenance assessment” there shall be substituted “maintenance calculation”.

\amendment{
Para. 3 is in force only for new-rules cases; see the Child Support, Pensions and Social Security Act 2000 (Commencement No. 12) Order 2003 art. 3.
}

\section*{\itshape The Domestic Proceedings and Magistrates Courts Act 1978}

4.---(1) The Domestic Proceedings and Magistrates Courts Act 1978 shall be amended as follows.

(2) In section 5 (age limit on making orders for financial provision for children and duration of such orders)—
\begin{enumerate}\item[]
($a$) in subsections (5)($a$), (7)  (three times) and (8)($a$), for “maintenance assessment” there shall be substituted “maintenance calculation”;

($b$) in subsections (5)($a$)  and ($b$)(ii)  and (6)($b$), for “current assessment” there shall be substituted “current calculation”; and

($c$) in subsection (6)($b$), for “those assessments” there shall be substituted “those calculations”.
\end{enumerate}

(3) In section 20 (variation, revival and revocation of orders for periodical payments)—
\begin{enumerate}\item[]
($a$) in subsections (9A)($c$)  and (9B)($a$)  and ($c$), for “maintenance assessment” there shall be substituted “maintenance calculation”; and

($b$) in subsection (9A) (three times), for “the assessment” there shall be substituted “the calculation”.
\end{enumerate}

(4) In section 88 (interpretation), in subsection (1), for “maintenance assessment” there shall be substituted “maintenance calculation”.

\amendment{
Para. 4 is in force only for new-rules cases; see the Child Support, Pensions and Social Security Act 2000 (Commencement No. 12) Order 2003 art. 3.
}

\section*{\itshape The Family Law (Scotland) Act 1985}

5.---(1) The Family Law (Scotland) Act 1985 shall be amended as follows.

(2) In section 5 (variation and recall of decrees of aliment), in subsection (1A), for “maintenance assessment” there shall be substituted “maintenance calculation”.

(3) In section 7 (agreements about aliment), in subsection (2A), for “maintenance assessment” there shall be substituted “maintenance calculation”.

(4) In section 13 (orders for periodical allowance), in subsection (4A), for “maintenance assessment” there shall be substituted “maintenance calculation”.

(5) In section 16 (agreements about financial provision), in subsection (3)($d$), for “maintenance assessment” there shall be substituted “maintenance calculation”.

(6) In section 27 (interpretation), in subsection (1), for “maintenance assessment” there shall be substituted “maintenance calculation”.

\amendment{
Para. 5 is in force only for new-rules cases; see the Child Support, Pensions and Social Security Act 2000 (Commencement No. 12) Order 2003 art. 3.
}

\section*{\itshape The Insolvency Act 1986}

6. In section 281 of the Insolvency Act 1986 (effect of discharge on a bankrupt), in subsection (5)($b$), for “maintenance assessment” there shall be substituted “maintenance calculation”.

\amendment{
Para. 6 is in force only for new-rules cases; see the Child Support, Pensions and Social Security Act 2000 (Commencement No. 12) Order 2003 art. 3.
}

\section*{\itshape The Debtors (Scotland) Act 1987}

7.---(1) The Debtors (Scotland) Act 1987 shall be amended as follows.

(2) In section 72 (effect of sequestration on diligence against earnings), in subsection (4A), for “maintenance assessment” there shall be substituted “maintenance calculation”.

(3) In section 106 (interpretation), in the definition of “maintenance order”, in paragraph ($j$), for “maintenance assessment” there shall be substituted “maintenance calculation”.

\amendment{
Para. 7 is in force only for new-rules cases; see the Child Support, Pensions and Social Security Act 2000 (Commencement No. 12) Order 2003 art. 3.
}

\section*{\itshape The Income and Corporation Taxes Act 1988}

8.---(1) The Income and Corporation Taxes Act 1988 shall be amended as follows.

(2) In section 347B (qualifying maintenance payments)—
\begin{enumerate}\item[]
($a$) in subsections (8)  and (9)($a$)  and ($c$), for “maintenance assessment” there shall be substituted “maintenance calculation”;

($b$) in subsection (9)($b$)  and ($c$), for “the assessment” there shall be substituted “the calculation”; and

($c$) for subsection (11)  there shall be substituted—
\begin{quotation}
“(11) In this section “maintenance calculation” means a maintenance calculation made under the Child Support Act 1991 or a maintenance assessment made under the Child Support (Northern Ireland) Order 1991.”
\end{quotation}
\end{enumerate}

(3) In section 617 (social security benefits and contributions), in subsection (2)($ae$), for “section 24 of the Child Support Act 1995 or under any corresponding enactment” there shall be substituted “any enactment corresponding to section 24 of the Child Support Act 1995”.

\amendment{
Para. 8 is in force only for new-rules cases; see the Child Support, Pensions and Social Security Act 2000 (Commencement No. 12) Order 2003 art. 3.
}

\section*{\itshape The Finance Act 1988}

9. In the Finance Act 1988, in each of subsection (5A)  of section 36 (annual payments) and subsection (8A) of section 38 (maintenance payments under existing obligations: 1989--90 onwards), for “maintenance assessment made” there shall be substituted “maintenance calculation or maintenance assessment made respectively”.

\amendment{
Para. 9 is in force only for new-rules cases; see the Child Support, Pensions and Social Security Act 2000 (Commencement No. 12) Order 2003 art. 3.
}

\section*{\itshape The Children Act 1989}

10.---(1) Schedule 1 to the Children Act 1989 (financial provision for children) shall be amended as follows.

(2) In paragraph 3—
\begin{enumerate}\item[]
($a$) in sub-paragraph (5)($a$), (7)  (three times) and (8)($a$), for “maintenance assessment” there shall be substituted “maintenance calculation”;

($b$) in sub-paragraph (5)($a$)  and ($b$)(ii)  and (6)($b$), for “current assessment” there shall be substituted “current calculation”;

($c$) in sub-paragraph (6)($b$), for “maintenance assessments” there shall be substituted “maintenance calculations”; and

($d$) in sub-paragraph (6)($b$), for “those assessments” there shall be substituted “those calculations”.
\end{enumerate}

(3) In paragraph 6—
\begin{enumerate}\item[]
($a$) in sub-paragraph (9)  (three times), for “the assessment” there shall be substituted “the calculation”; and

($b$) in sub-paragraph (9)($c$), for “maintenance assessment” there shall be substituted “maintenance calculation”.
\end{enumerate}

(4) In paragraph 16(3), for “maintenance assessment” there shall be substituted “maintenance calculation”.

\amendment{
Para. 10 is in force only for new-rules cases; see the Child Support, Pensions and Social Security Act 2000 (Commencement No. 12) Order 2003 art. 3.
}

\section*{\itshape The Child Support Act 1991}

11.---(1) The 1991 Act shall be amended as follows.

(2) For “absent parent” (or any variant of that expression), wherever it occurs, there shall be substituted “non-resident parent” (or the corresponding variant) preceded, where appropriate, by “a” instead of “an”.

(3) In section 4 (child support maintenance)—
\begin{enumerate}\item[]
($a$) in subsection (4)($a$), after “be” there shall be inserted “identified or”; and

($b$) in subsection (9), after “an application” there shall be inserted “treated as made”.
\end{enumerate}

(4) In section 7 (right of a child in Scotland to apply for assessment)—
\begin{enumerate}\item[]
($a$) in subsection (1), for paragraph ($b$)  there shall be substituted—
\begin{quotation}
“($b$) no parent has been treated under section 6(3)  as having applied for a maintenance calculation with respect to the child.”; and
\end{quotation}

($b$) in subsection (10)—
\begin{enumerate}\item[]
(i) after “qualifying child if” there shall be inserted “($a$)”;

(ii) after “maintenance order” there shall be inserted “made before a prescribed date”%
; and

(iii) at the end there shall be inserted “or
\begin{quotation}
($b$) a maintenance order made on or after the date prescribed for the purposes of paragraph ($a$)  is in force in respect of them, but has been so for less than the period of one year beginning with the date on which it was made.”%
.
\end{quotation}
\end{enumerate}
\end{enumerate}

(5) In section 8 (role of the courts with respect to maintenance for children)—
\begin{enumerate}\item[]
($a$) in subsection (1), after “duly made” there shall be inserted “(or treated as made)”;

($b$) in subsection (3), at the beginning insert “Except as provided in subsection (3A),”;

($c$) for subsection (3A)  there shall be substituted—
\begin{quotation}
“(3A) Unless a maintenance calculation has been made with respect to the child concerned, subsection (3)  does not prevent a court from varying a maintenance order in relation to that child and the non-resident parent concerned—
\begin{enumerate}\item[]
($a$) if the maintenance order was made on or after the date prescribed for the purposes of section 4(10)($a$)  or 7(10)($a$); or

($b$) where the order was made before then, in any case in which section 4(10)  or 7(10)  prevents the making of an application for a maintenance calculation with respect to or by that child.”; and
\end{enumerate}
\end{quotation}

($d$) in subsection (6), for paragraph ($b$)  there shall be substituted—
\begin{quotation}
“($b$) the non-resident parent’s net weekly income exceeds the figure referred to in paragraph 10(3)  of Schedule 1 (as it has effect from time to time pursuant to regulations made under paragraph 10A(1)($b$)); and”.
\end{quotation}
\end{enumerate}

(6) In section 9 (agreements about maintenance), in subsection (6), for paragraphs ($a$)  and ($b$)  there shall be substituted—
\begin{quotation}
“($a$) no parent has been treated under section 6(3)  as having applied for a maintenance calculation with respect to the child; or

($b$) a parent has been so treated but no maintenance calculation has been made,”.
\end{quotation}

(7) In section 14 (information required by Secretary of State), in subsection (1), after “any application” there shall be inserted “made or treated as made”.

(8) In section 26 (disputes about parentage), in subsection (1), after “made” there shall be inserted “or treated as made”.

(9) In section 27A (recovery of fees for scientific tests)—
\begin{enumerate}\item[]
($a$) in subsection (1)($a$), after “made” there shall be inserted “or treated as made”; and

($b$) in subsection (1)($b$), after “made” there shall be inserted “or, as the case may be, treated as made”.
\end{enumerate}

(10) In section 28 (power of the Secretary of State to bring or defend actions of declarator), in subsection (1)($a$)—
\begin{enumerate}\item[]
($a$) after “made”, where it first occurs, there shall be inserted “or treated as made”; and

($b$) for “or assessment was made” there shall be substituted “was made or treated as made or the calculation was made”.
\end{enumerate}

(11) In section 28ZA (decisions involving issues that arise on appeal in other cases), in subsection (1)—
\begin{enumerate}\item[]
($a$) in paragraph ($a$), for the words “in relation to a maintenance assessment” there shall be substituted “or with respect to a reduced benefit decision under section 46”; and

($b$) for paragraph ($b$)  there shall be substituted—

\begin{quotation}
“($b$) an appeal is pending against a decision given in relation to a different matter by a Child Support Commissioner or a court.”
\end{quotation}
\end{enumerate}

(12) In section 28ZB (appeals involving issues that arise on appeal in other cases)—
\begin{enumerate}\item[]
($a$) in subsection (1), for paragraph ($a$)  there shall be substituted—
\begin{quotation}
“($a$) an appeal (“appeal A”) in relation to a decision or the imposition of a requirement falling within section 20(1)  is made to an appeal tribunal, or from an appeal tribunal to a Child Support Commissioner;”; and
\end{quotation}

($b$) in subsection (4), for the words “or assessment” there shall be substituted “or the imposition of the requirement”.
\end{enumerate}

(13) In section 28ZC (restrictions on liability in certain cases of error)—
\begin{enumerate}\item[]
($a$) in subsection (1)($b$)(i), at the end there shall be inserted “or one treated as having been so made, or under section 46 as to the reduction of benefit”;

($b$) in subsection (1)($b$)(ii), for the words from “a decision” to the end there shall be substituted “any decision (made after the commencement date) referred to in section 16(1A)”;

($c$) in subsection (1)($b$)(iii), for the words from “a decision” to the end there shall be substituted “any decision (made after the commencement date) referred to in section 17(1)”;

($d$) in subsection (3), after “liability” there shall be inserted “or the reduction of a person’s benefit”; and

($e$) in subsection (6), in the definition of “adjudicating authority”, at the end there shall be inserted “or, in the case of a decision made on a referral under section 28D(1)($b$), an appeal tribunal”.
\end{enumerate}

(14) Sections 28H (departure directions: decisions and appeals) and 28I (transitional provisions relating to departure directions) shall cease to have effect.

(15) In section 30 (collection and enforcement of certain forms of maintenance), for subsection (2)  there shall be substituted—
\begin{quotation}
“(2) The Secretary of State may, except in prescribed cases, arrange for the collection of any periodical payments, or secured periodical payments, of a prescribed kind which are payable for the benefit of a child even though he is not arranging for the collection of child support maintenance with respect to that child.”.
\end{quotation}

(16) In section 32 (regulations about deduction from earnings orders), in subsection (2), after paragraph ($b$)  there shall be inserted—
\begin{quotation}
“($bb$) for the amount or amounts which are to be deducted from the liable person’s earnings not to exceed a prescribed proportion of his earnings (as determined by the employer);”.
\end{quotation}

(17) In section 33 (liability orders), after subsection (5)  there shall be inserted—
\begin{quotation}
“(6) Where regulations have been made under section 29(3)($a$)—
\begin{enumerate}\item[]
($a$) the liable person fails to make a payment (for the purposes of subsection (1)($a$)  of this section); and

($b$) a payment is not paid (for the purposes of subsection (3)),
\end{enumerate}
unless the payment is made to, or through, the person specified in or by virtue of those regulations for the case of the liable person in question.”
\end{quotation}

(18) In section 47 (fees), after subsection (3)  there shall be inserted—
\begin{quotation}
“(4) The provisions of this Act with respect to—
\begin{enumerate}\item[]
($a$) the collection of child support maintenance;

($b$) the enforcement of any obligation to pay child support maintenance,
\end{enumerate}
shall apply equally (with any necessary modifications) to fees payable by virtue of regulations made under this section.”
\end{quotation}

(19) In section 51 (supplementary power to make regulations), in subsection (2)—
\begin{enumerate}\item[]
($a$) for paragraph ($a$)(ii)  and (iii)  there shall be substituted—
\begin{quotation}
“(ii) the making of decisions under section 11;

(iii) the making of decisions under section 16 or 17;”; and
\end{quotation}

($b$) for paragraph ($b$)  there shall be substituted—
\begin{quotation}
“($b$) extending the categories of case to which section 16, 17 or 20 applies;”.
\end{quotation}
\end{enumerate}

(20) In section 54 (interpretation)—
\begin{enumerate}\item[]
($a$) in the definition of “application for a departure direction”, for “departure direction” there shall be substituted “variation”, and after “28A” there shall be inserted “or 28G”;

($b$) after the definition of “deduction from earnings order” there shall be inserted—
\begin{quotation}
““default maintenance decision” has the meaning given in section 12;”;
\end{quotation}

($c$) in the definition of “interim maintenance assessment”, for the word “assessment” there shall be substituted the word “decision”;

($d$) for the definition of “maintenance assessment” there shall be substituted—
\begin{quotation}
““maintenance calculation” means a calculation of maintenance made under this Act and, except in prescribed circumstances, includes a default maintenance decision and an interim maintenance decision;”;
\end{quotation}

($e$) the definitions of “assessable income”, “current assessment”, “departure direction” and “maintenance requirement” shall be omitted; and

($f$) after the definition of “qualifying child” there shall be inserted—
\begin{quotation}
““voluntary payment” has the meaning given in section 28J.”.
\end{quotation}
\end{enumerate}

(21) In section 58 (short title, commencement and extent)—
\begin{enumerate}\item[]
($a$) in subsection (9), after “35” there shall be inserted “, 40”; and

($b$) in subsection (10), after “28” there shall be inserted “, 40A”.
\end{enumerate}

(22) In Schedule 1 (maintenance assessments)—
\begin{enumerate}\item[]
($a$) paragraph 13 (which relates to assessments under which the amount payable is nil) shall cease to have effect;

($b$) in paragraph 14 (which provides for consolidated applications and assessments), the existing text shall be sub-paragraph (1)  of that paragraph, and after that sub-paragraph there shall be inserted—
\begin{quotation}
“(2) In sub-paragraph (1), the references (however expressed) to applications for maintenance calculations include references to applications treated as made.”; and
\end{quotation}

($c$) in paragraph 16 (which is about the termination of assessments)—
\begin{enumerate}\item[]
(i) in sub-paragraph (1), paragraphs ($d$)  and ($e$)  shall cease to have effect,

(ii) sub-paragraphs (2)  to (9)  shall cease to have effect; and

(iii) in sub-paragraph (10), the words “, or should be cancelled” shall cease to have effect.
\end{enumerate}
\end{enumerate}

\amendment{
Para. 11(2)--(14) and (16)--(22) are in force only for new-rules cases; see the Child Support, Pensions and Social Security Act 2000 (Commencement No. 12) Order 2003 art. 3.  }

\section*{\itshape The Social Security Administration Act 1992}

12. In section 7A of the Social Security Administration Act 1992 (sharing of functions as regards certain claims and information), in subsection (6)($a$)—
\begin{enumerate}\item[]
($a$) after “application” there shall be inserted “(or an application treated as having been made)”; and

($b$) for “maintenance assessment” there shall be substituted “maintenance calculation”.
\end{enumerate}

\amendment{
Para. 12 is in force only for new-rules cases; see the Child Support, Pensions and Social Security Act 2000 (Commencement No. 12) Order 2003 art. 3. 
}

\section*{\itshape The Child Support Act 1995}

13.---(1) The Child Support Act 1995 shall be amended as follows.

(2) In section 18 (deferral of right to apply for maintenance assessment), subsection (5)  (which enables the Secretary of State by order to repeal any of the provisions of section 18) shall cease to have effect.

(3) Section 24 (which provides for the making of regulations under which compensation could be paid for a reduction in child support maintenance attributable to changes in child support legislation, and which is now spent) shall cease to have effect.

\amendment{
Para. 13 is in force only for new-rules cases; see the Child Support, Pensions and Social Security Act 2000 (Commencement No. 12) Order 2003 art. 3. 
}

\section*{\itshape Prisoners' Earnings Act 1996}

14. In section 1 of the Prisoners' Earnings Act 1996 (power to make deductions and impose levies), in subsection (4), in paragraph ($d$)  of the definition of “net weekly earnings”, for “maintenance assessment” there shall be substituted “maintenance calculation”.

\amendment{
Para. 14 is in force only for new-rules cases; see the Child Support, Pensions and Social Security Act 2000 (Commencement No. 12) Order 2003 art. 3. 
}

\section*{\itshape The Social Security Act 1998}

15.---(1) The Social Security Act 1998 shall be amended as follows.

(2) In Schedule 2 (decisions against which no appeal lies), for paragraph 8 and the heading preceding it there shall be substituted—
\begin{quotation}
\subsection*{\itshape “Reduction in accordance with reduced benefit decision}

8. A decision to reduce the amount of a person’s benefit in accordance with a reduced benefit decision (within the meaning of section 46 of the Child Support Act).”.
\end{quotation}

\amendment{
Para. 15 is in force only for new-rules cases; see the Child Support, Pensions and Social Security Act 2000 (Commencement No. 12) Order 2003 art. 3. 
}

\part[Schedule 4 --- Additional pension]{Schedule 4\\*Additional pension}

\renewcommand\parthead{--- Schedule 4}

\noindent
The Schedule to be inserted after Schedule 4 to the Social Security Contributions and Benefits Act 1992 is as follows—
\begin{quotation}
\part*{\noindent “Schedule 4A\\*Additional pension}

\section*{Part I\\*The amount}

1.---(1) The amount referred to in section 45(2)($c$)  above is to be calculated as follows—
\begin{enumerate}\item[]
($a$) take for each tax year concerned the amount for the year which is found under the following provisions of this Schedule;

($b$) add the amounts together;

($c$) divide the sum of the amounts by the number of relevant years;

($d$) the resulting amount is the amount referred to in section 45(2)($c$)  above, except that if the resulting amount is a negative one the amount so referred to is nil.
\end{enumerate}

(2) For the purpose of applying sub-paragraph (1)  above in the determination of the rate of any additional pension by virtue of section 39(1), 39C(1), 48A(4)  or 48B(2)  above, in a case where the deceased spouse died under pensionable age, the divisor used for the purposes of sub-paragraph (1)($c$)  above shall be whichever is the smaller of the alternative numbers referred to below (instead of the number of relevant years).

(3) The first alternative number is the number of tax years which begin after 5th April 1978 and end before the date when the entitlement to the additional pension commences.

(4) The second alternative number is the number of tax years in the period—
\begin{enumerate}\item[]
($a$) beginning with the tax year in which the deceased spouse attained the age of 16 or, if later, 1978--79; and

($b$) ending immediately before the tax year in which the deceased spouse would have attained pensionable age if he had not died earlier.
\end{enumerate}

(5) For the purpose of applying sub-paragraph (1)  above in the determination of the rate of any additional pension by virtue of section 48BB(5)  above, in a case where the deceased spouse died under pensionable age, the divisor used for the purposes of sub-paragraph (1)($c$)  above shall be whichever is the smaller of the alternative numbers referred to below (instead of the number of relevant years).

(6) The first alternative number is the number of tax years which begin after 5th April 1978 and end before the date when the deceased spouse dies.

(7) The second alternative number is the number of tax years in the period—
\begin{enumerate}\item[]
($a$) beginning with the tax year in which the deceased spouse attained the age of 16 or, if later, 1978--79; and

($b$) ending immediately before the tax year in which the deceased spouse would have attained pensionable age if he had not died earlier.
\end{enumerate}

(8) In this paragraph “relevant year” has the same meaning as in section 44 above.

\section*{Part II\\*Surplus earnings factor}

2.---(1) This Part of this Schedule applies if for the tax year concerned there is a surplus in the pensioner’s earnings factor.

(2) The amount for the year is to be found as follows—
\begin{enumerate}\item[]
($a$) calculate the part of the surplus for that year falling into each of the bands specified in the appropriate table below;

($b$) multiply the amount of each such part in accordance with the last order under section 148 of the Administration Act to come into force before the end of the final relevant year;

($c$) multiply each amount found under paragraph ($b$)  above by the percentage specified in the appropriate table in relation to the appropriate band;

($d$) add together the amounts calculated under paragraph ($c$)  above.
\end{enumerate}

(3) The appropriate table for persons attaining pensionable age after the end of the first appointed year but before 6th April 2009 is as follows—

\medskip

\noindent\textsc{Table 1}

{\noindent\footnotesize
%\begin{tabulary}{\linewidth}{JJJ}
\begin{longtable}{lll}
\hline
&\itshape Amount of surplus	&\itshape Percentage\\
\hline
\endhead
\hline
\endlastfoot
Band 1. 	&Not exceeding $LET$	&$40 + 2N$\\
Band 2. 	&Exceeding $LET$ but not exceeding $3LET - 2QEF$	&$10 + \frac{N}{2}$\\
Band 3. 	&Exceeding $3LET - 2QEF$	&$20 + N$\\
%\hline
%\end{tabulary}
\end{longtable}

}

\medskip

(4) The appropriate table for persons attaining pensionable age on or after 6th April 2009 is as follows—

\medskip

\noindent\textsc{Table 2}

{\noindent\footnotesize
%\begin{tabulary}{\linewidth}{JJJ}
\begin{longtable}{lll}
\hline
&\itshape Amount of surplus	&\itshape Percentage\\
\hline
\endhead
\hline
\endlastfoot
Band 1. 	&Not exceeding $LET$	&40\\
Band 2. 	&Exceeding $LET$ but not exceeding $3LET - 2QEF$	&10\\
Band 3. 	&Exceeding $3LET - 2QEF$	&20\\
%\hline
%\end{tabulary}
\end{longtable}

}

(5) Regulations may provide, in relation to persons attaining pensionable age after such date as may be prescribed, that the amount found under this Part of this Schedule for the second appointed year or any subsequent tax year is to be calculated using only so much of the surplus in the pensioner’s earnings factor for that year as falls into Band 1 in the table in sub-paragraph (4)  above.

(6) For the purposes of the tables in this paragraph—
\begin{enumerate}\item[]
($a$) the value of $N$ is 0.5 for each tax year by which the tax year in which the pensioner attained pensionable age precedes 2009--10;

($b$) “$LET$” means the low earnings threshold for that year as specified in section 44A above;

($c$) “$QEF$” means the qualifying earnings factor for the tax year concerned.
\end{enumerate}

(7) In the calculation of “$2QEF$” the amount produced by doubling $QEF$ shall be rounded to the nearest whole £100 (taking any amount of £50 as nearest to the previous whole £100).

(8) In this paragraph “final relevant year” has the same meaning as in section 44 above.

\section*{Part III\\*Contracted-out employment etc}
\subsection*{\itshape Introduction}

3.---(1) This Part of this Schedule applies if the following condition is satisfied in relation to each tax week in the tax year concerned.

(2) The condition is that any earnings paid to or for the benefit of the pensioner in the tax week in respect of employment were in respect of employment qualifying him for a pension provided by a salary related contracted-out scheme or by a money purchase contracted-out scheme or by an appropriate personal pension scheme.

(3) If the condition is satisfied in relation to one or more tax weeks in the tax year 
concerned, Part II of this Schedule does not apply in relation to the year.

\subsection*{\itshape The amount}

4. The amount for the year is amount C where—
\begin{enumerate}\item[]
($a$) amount C is equal to amount A minus amount B, and

($b$) amounts A and B are calculated as follows.
\end{enumerate}

\subsection*{\itshape Amount A}

5.---(1) Amount A is to be calculated as follows.

(2) If there is an assumed surplus in the pensioner’s earnings factor for the year—
\begin{enumerate}\item[]
($a$) calculate the part of the surplus for that year falling into each of the bands specified in the appropriate table below;

($b$) multiply the amount of each such part in accordance with the last order under section 148 of the Administration Act to come into force before the end of the final relevant year;

($c$) multiply each amount found under paragraph ($b$)  above by the percentage specified in the appropriate table in relation to the appropriate band;

($d$) add together the amounts calculated under paragraph ($c$)  above.
\end{enumerate}

(3) The appropriate table for persons attaining pensionable age after the end of the first appointed year but before 6th April 2009 is as follows—

\pagebreak[3]

\noindent\textsc{Table 3}

{\noindent\footnotesize
%\begin{tabulary}{\linewidth}{JJJ}
\begin{longtable}{lll}
\hline
&\itshape Amount of surplus	&\itshape Percentage\\
\hline
\endhead
\hline
\endlastfoot
Band 1. 	&Not exceeding $LET$	&$40 + 2N$\\
Band 2. 	&Exceeding $LET$ but not exceeding $3LET - 2QEF$	&$10 + \frac{N}{2}$\\
Band 3. 	&Exceeding $3LET - 2QEF$	&$20 + N$\\
%\hline
%\end{tabulary}
\end{longtable}

}

(4) The appropriate table for persons attaining pensionable age on or after 6th April 2009 is as follows—

\medskip

\noindent\textsc{Table 4}

{\noindent\footnotesize
%\begin{tabulary}{\linewidth}{JJJ}
\begin{longtable}{lll}
\hline
&\itshape Amount of surplus	&\itshape Percentage\\
\hline
\endhead
\hline
\endlastfoot
Band 1. 	&Not exceeding $LET$	&40\\
Band 2. 	&Exceeding $LET$ but not exceeding $3LET - 2QEF$	&10\\
Band 3. 	&Exceeding $3LET - 2QEF$	&20\\
%\hline
%\end{tabulary}
\end{longtable}

}

\subsection*{\itshape Amount B (first case)}

6.---(1) Amount B is to be calculated in accordance with this paragraph if the pensioner’s employment was entirely employment qualifying him for a pension provided by a salary related contracted-out scheme or by a money purchase contracted-out scheme.

(2) If there is an assumed surplus in the pensioner’s earnings factor for the year—
\begin{enumerate}\item[]
($a$) multiply the amount of the assumed surplus in accordance with the last order under section 148 of the Administration Act to come into force before the end of the final relevant year;

($b$) multiply the amount found under paragraph ($a$)  above by the percentage specified in sub-paragraph (3)  below.
\end{enumerate}

(3) The percentage is—
\begin{enumerate}\item[]
($a$) $20 + N$ if the person attained pensionable age after the end of the first appointed year but before 6th April 2009;

($b$) 20 if the person attained pensionable age on or after 6th April 2009. 
\end{enumerate}

\subsection*{\itshape Amount B (second case)}

7.---(1) Amount B is to be calculated in accordance with this paragraph if the pensioner’s employment was entirely employment qualifying him for a pension provided by an appropriate personal pension scheme.

(2) If there is an assumed surplus in the pensioner’s earnings factor for the year—
\begin{enumerate}\item[]
($a$) calculate the part of the surplus for that year falling into each of the bands specified in the appropriate table below;

($b$) multiply the amount of each such part in accordance with the last order under section 148 of the Administration Act to come into force before the end of the final relevant year;

($c$) multiply each amount found under paragraph ($b$)  above by the percentage specified in the appropriate table in relation to the appropriate band;

($d$) add together the amounts calculated under paragraph ($c$)  above.
\end{enumerate}

(3) The appropriate table for persons attaining pensionable age after the end of the first appointed year but before 6th April 2009 is as follows—

\pagebreak[3]

\noindent\textsc{Table 5}

{\noindent\footnotesize
%\begin{tabulary}{\linewidth}{JJJ}
\begin{longtable}{lll}
\hline
&\itshape Amount of surplus	&\itshape Percentage\\
\hline
\endhead
\hline
\endlastfoot
Band 1. 	&Not exceeding $LET$	&$40 + 2N$\\
Band 2. 	&Exceeding $LET$ but not exceeding $3LET - 2QEF$	&$10 + \frac{N}{2}$\\
Band 3. 	&Exceeding $3LET - 2QEF$	&$20 + N$\\
%\hline
%\end{tabulary}
\end{longtable}

}

(4) The appropriate table for persons attaining pensionable age on or after 6th April 2009 is as follows—

\noindent\textsc{Table 6}

{\noindent\footnotesize
%\begin{tabulary}{\linewidth}{JJJ}
\begin{longtable}{lll}
\hline
&\itshape Amount of surplus	&\itshape Percentage\\
\hline
\endhead
\hline
\endlastfoot
Band 1. 	&Not exceeding $LET$	&40\\
Band 2. 	&Exceeding $LET$ but not exceeding $3LET - 2QEF$	&10\\
Band 3. 	&Exceeding $3LET - 2QEF$	&20\\
%\hline
%\end{tabulary}
\end{longtable}

}

\subsection*{\itshape Interpretation}

8.---(1) In this Part of this Schedule “salary related contracted-out scheme”, “money purchase contracted-out scheme” and “appropriate personal pension scheme” have the same meanings as in the Pension Schemes Act 1993. 

(2) For the purposes of this Part of this Schedule the assumed surplus in the pensioner’s earnings factor for the year is the surplus there would be in that factor for the year if section 48A(1)  of the Pension Schemes Act 1993 (no primary Class 1 contributions deemed to be paid) did not apply in relation to any tax week falling in the year.

(3) Section 44A above shall be ignored in applying section 44(6)  above for the purpose of calculating amount B.

(4) For the purposes of this Part of this Schedule—
\begin{enumerate}\item[]
($a$) the value of $N$ is 0.5 for each tax year by which the tax year in which the pensioner attained pensionable age precedes 2009--10;

($b$) “$LET$” means the low earnings threshold for that year as specified in section 44A above;

($c$) “$QEF$” is the qualifying earnings factor for the tax year concerned.
\end{enumerate}

(5) In the calculation of “$2QEF$” the amount produced by doubling $QEF$ shall be rounded to the nearest whole £100 (taking any amount of £50 as nearest to the previous whole £100).

(6) In this Part of this Schedule “final relevant year” has the same meaning as in section 44 above.

\section*{Part IV\\*Other cases}

9. The Secretary of State may make regulations containing provisions for finding the amount for a tax year in—
\begin{enumerate}\item[]
($a$) cases where the circumstances relating to the pensioner change in the course of the year;

($b$) such other cases as the Secretary of State thinks fit.”
\end{enumerate}
\end{quotation}

\part[Schedule 5 --- Pensions: miscellaneous amendments and alternative to anti-franking rules]{Schedule 5\\*Pensions: miscellaneous amendments and alternative to anti-franking rules}

\section[Part I --- Miscellaneous amendments]{Part I\\*Miscellaneous amendments}

\renewcommand\parthead{--- Schedule 5 Part I}

\subsection*{Guaranteed minimum for widows and widowers}

1.---(1) In section 17 of the 1993 Act (guaranteed minimum for widow or widower), after subsection (4)  there shall be inserted—
\begin{quotation}
“(4A) The scheme must provide for the widow or widower’s pension to be payable to the widow or widower—
\begin{enumerate}\item[]
($a$) for any period for which a Category B retirement pension is payable to the widow or widower by virtue of the earner’s contributions or would be so payable but for section 43(1)  of the Social Security Contributions and Benefits Act 1992 (persons entitled to more than one retirement pension);

($b$) for any period for which widowed parent’s allowance or bereavement allowance is payable to the widow or widower by virtue of the earner’s contributions; and

($c$) in the case of a widow or widower whose entitlement by virtue of the earner’s contributions to a widowed parent’s allowance or bereavement allowance has come to an end at a time after the widow or widower attained the age of 45, for so much of the period beginning with the time when the entitlement came to an end as neither—
\begin{enumerate}\item[]
(i) comprises a period during which the widow or widower and a person of the opposite sex are living together as husband and wife; nor

(ii) falls after the time of any remarriage by the widow or widower.”
\end{enumerate}
\end{enumerate}
\end{quotation}

(2) In subsection (5)  of that section—
\begin{enumerate}\item[]
($a$) for “must provide” there shall be substituted “must also make provision”;

($b$) the words “Category B retirement pension,”, in the first place where they occur, and the words from “or for which” onwards shall be omitted.
\end{enumerate}

(3) In subsection (6)  of that section, for “must provide” there shall be substituted “must also make provision”.

\subsection*{Transfer of rights to overseas personal pension schemes}

2.---(1) In section 20(1)  of the 1993 Act (power to make provision for transfer of rights relating to guaranteed minimum pensions to an occupational or a personal pension scheme)—
\begin{enumerate}\item[]
($a$) in paragraph ($a$), for “or to a personal pension scheme” there shall be substituted “, to a personal pension scheme or to an overseas arrangement”; and

($b$) in paragraph ($b$), for “or a personal pension scheme” there shall be substituted “, a personal pension scheme or an overseas arrangement”.
\end{enumerate}

(2) In section 28(2)($b$)  of that Act (effect may be given to protected rights by a transfer to an occupational or personal pension scheme)—
\begin{enumerate}\item[]
($a$) in sub-paragraph (i), for “or to a personal pension scheme” there shall be substituted “, to a personal pension scheme or to an overseas arrangement”; and

($b$) in sub-paragraph (ii), for “or to an occupational pension scheme” there shall be substituted “, to an occupational pension scheme or to an overseas arrangement”.
\end{enumerate}

(3) In section 181(1)  of that Act (interpretation), there shall be inserted, at the appropriate place in the alphabetical order—
\begin{quotation}
““overseas arrangement” means a scheme or arrangement which—
\begin{enumerate}\item[]
($a$) has effect, or is capable of having effect, so as to provide benefits on termination of employment or on death or retirement to or in respect of earners;

($b$) is administered wholly or primarily outside Great Britain;

($c$) is not an appropriate scheme; and

($d$) is not an occupational pension scheme;”.
\end{enumerate}
\end{quotation}

\subsection*{Protected rights}

3.---(1) Section 28 of the 1993 Act (ways of giving effect to protected rights) shall be amended as follows.

(2) In subsection (4)  (giving effect to protected rights at or after retirement age), for paragraph ($d$)  there shall be substituted—
\begin{quotation}
“($d$) the amount of the lump sum is equal to the value on that date of the protected rights to which effect is being given.”
\end{quotation}

(3) After that subsection there shall be inserted—
\begin{quotation}
“(4A) Subject to subsection (4B), in the case of an occupational pension scheme, effect may be given to protected rights by the provision of a lump sum if—
\begin{enumerate}\item[]
($a$) the trustees or managers of the scheme are satisfied that the member is terminally ill and likely to die within the period of twelve months beginning with the date on which the lump sum becomes payable; and

($b$) the amount of the lump sum is equal to the value on that date of the protected rights to which effect is being given.
\end{enumerate}

(4B) The value of the protected rights to which effect may be given under subsection (4A)  in a case in which the member is a married person on the date on which the lump sum becomes payable shall not exceed one half of the value on that date of all the member’s protected rights.”
\end{quotation}

(4) In subsections (3)  and (5), for “or (4)”, in each case, there shall be substituted “, (4)  or (4A)”.

\subsection*{Review and alteration of rates of contribution}

4. In section 42(1)($a$)(i)  and (3)  of the 1993 Act (review of percentages mentioned in section 41), for “41(1A)($a$)  and ($b$)” there shall be substituted “41(1A)  and (1B)”.

\subsection*{Contributions equivalent premiums: Great Britain}

5.---(1) For subsection (4)  of section 58 of the 1993 Act (calculation of contributions equivalent premiums) there shall be substituted—
\begin{quotation}
“(4) Subject to subsection (4A), the amount of the contributions equivalent premium shall be equal to the sum of the following amounts—
\begin{enumerate}\item[]
($a$) the amount of every reduction made under section 41 (as from time to time in force) in the amount of Class 1 contributions payable in respect of the earner’s employment in employment which was contracted-out by reference to the scheme; and

($b$) the total amount by which the reductions falling within paragraph ($a$)  would have been larger if the amount of the contributions falling to be reduced had in each case been at least equal to the amount of the reduction of those contributions provided for by section 41. 
\end{enumerate}

(4A) The amounts brought into account in accordance with subsection (4)($b$)  shall not include any amount which, by virtue of regulations made under section 41(1D) so as to avoid the payment of trivial or fractional amounts, is an amount that was not payable by the Inland Revenue to the secondary contributor.”
\end{quotation}

(2) In section 61(2)  of that Act (recovery of amount of premium attributable to primary Class 1 contributions), after “attributable to” there shall be inserted “any actual reductions of”.

(3) In section 63(1)  of that Act (amounts to be certified by the Inland Revenue), for paragraph ($b$)  there shall be substituted—
\begin{quotation}
“($b$) the sum of the amounts specified in section 58(4);”.
\end{quotation}

(4) This paragraph shall have effect, and be deemed to have had effect, in relation to any contributions equivalent premium payable on or after 6th April 1999. 

\subsection*{Contributions equivalent premiums: Northern Ireland}

6.---(1) For subsection (4)  of section 54 of the Pension Schemes (Northern Ireland) Act 1993 (calculation of contributions equivalent premiums) there shall be substituted—
\begin{quotation}
“(4) Subject to subsection (4A), the amount of the contributions equivalent premium shall be equal to the sum of the following amounts—
\begin{enumerate}\item[]
($a$) the amount of every reduction made under section 37 (as from time to time in force) in the amount of Class 1 contributions payable in respect of the earner’s employment in employment which was contracted-out by reference to the scheme; and

($b$) the total amount by which the reductions falling within paragraph ($a$)  would have been larger if the amount of the contributions falling to be reduced had in each case been at least equal to the amount of the reduction of those contributions provided for by section 37. 
\end{enumerate}

(4A) The amounts brought into account in accordance with subsection (4)($b$)  shall not include any amount which, by virtue of regulations made under section 37(1D) so as to avoid the payment of trivial or fractional amounts, is an amount that was not payable by the Inland Revenue to the secondary contributor.”
\end{quotation}

(2) In section 57(2)  of that Act (recovery of amount of premium attributable to primary Class 1 contributions), after “attributable to” there shall be inserted “any actual reductions of”.

(3) In section 59(1)  of that Act (amounts to be certified by the Inland Revenue), for paragraph ($b$)  there shall be substituted—
\begin{quotation}
“($b$) the sum of the amounts specified in section 54(4);”.
\end{quotation}

(4) This paragraph shall have effect, and be deemed to have had effect, in relation to any contributions equivalent premium payable on or after 6th April 1999. 

\subsection*{Use of cash equivalent for annuity}

7. Section 95(4)  of the 1993 Act (cash equivalent of rights under a money purchase contracted-out scheme not to be used for purchase of annuity) shall cease to have effect.

\subsection*{Transfer values where pension in payment}

8.---(1) In section 97(2)  of the 1993 Act (regulations about calculation of cash equivalents), for the “and” at the end of paragraph ($a$)  there shall be substituted—
\begin{quotation}
“($aa$) for a cash equivalent, including a guaranteed cash equivalent, to be reduced so as to take account of the extent (if any) to which an entitlement has arisen under the scheme to the present payment of the whole or any part of—
\begin{enumerate}\item[]
(i) any pension; or

(ii) any benefit in lieu of pension;
\end{enumerate}
and”.
\end{quotation}

(2) In section 98(7)  of that Act (loss of right to cash equivalent)—
\begin{enumerate}\item[]
($a$) after “right” there shall be inserted “if”; and

($b$) paragraph ($a$)  (loss of right on the whole or any part of a pension becoming payable) shall cease to have effect.
\end{enumerate}

(3) In section 124(1)  of the 1995 Act (interpretation), in the definition of “pensioner member”, after “other benefits” there shall be inserted “and who is not an active member of the scheme”.

(4) Sub-paragraph (2)  has effect in relation to any case in which the whole or any part of a pension or other benefit becomes payable on or after the coming into force of that sub-paragraph.

\subsection*{Information about contracting-out}

9. For section 156 of the 1993 Act (provision of information as to guaranteed minimum pensions) there shall be substituted—
\begin{quotation}
\subsection*{“156. Information for purposes of contracting-out}

(1) The Secretary of State or the Inland Revenue may give to the trustees or managers of an occupational pension scheme or appropriate scheme such information as appears to the Secretary of State or Inland Revenue appropriate to give to them for the purpose of enabling them to comply with their obligations under Part III.

(2) The Secretary of State or Inland Revenue may also give to such persons as may be prescribed any information that they could give under subsection (1)  to trustees or managers of a scheme.”
\end{quotation}

\subsection*{Register of disqualified trustees}

10.---(1) In subsection (7)  of section 30 of the 1995 Act (disclosure of contents of register of disqualified trustees), for the words from “and” onwards there shall be substituted “but the arrangements made by the Authority for the register must secure that the contents of the register are not disclosed or otherwise made available to members of the public except in accordance with section 30A.”

(2) After that subsection there shall be inserted—
\begin{quotation}
“(8) Nothing in subsection (7)  requires the Authority to exclude any matter from a report published under section 103.”
\end{quotation}

(3) After that section there shall be inserted—
\begin{quotation}
\subsection*{\sloppy “30A. Accessibility of register of disqualified trustees}

(1) The Authority shall make arrangements that secure that the disqualification register is open, during the normal working hours of the Authority, for inspection in person and without notice at—
\begin{enumerate}\item[]
($a$) the principal office used by them for the carrying out of their functions under this Act; and

($b$) such other offices (if any) of theirs as they consider to be places where it would be reasonable for a copy of the register to be kept open for inspection.
\end{enumerate}

(2) If a request is made to the Authority—
\begin{enumerate}\item[]
($a$) to state whether a particular person identified in the request is a person appearing in the disqualification register as disqualified in respect of a scheme specified in the request, or

($b$) to state whether a particular person identified in the request is a person appearing in that register as disqualified in respect of all trust schemes,
\end{enumerate}
it shall be the duty of the Authority promptly to comply with the request in such manner as they consider reasonable.

(3) The Authority may, in such manner as they think fit, publish a summary of the disqualification register if (subject to subsections (6)  to (8)) the summary—
\begin{enumerate}\item[]
($a$) contains all the information described in subsection (4);

($b$) arranges that information in the manner described in subsection (5);

($c$) does not (except by identifying a person as disqualified in respect of all trust schemes) identify any of the schemes in respect of which persons named in the summary are disqualified; and

($d$) does not disclose any other information contained in the register.
\end{enumerate}

(4) That information is—
\begin{enumerate}\item[]
($a$) the full names and titles, so far as the Authority have a record of them, of all the persons appearing in the register as persons who are disqualified;

($b$) the dates of birth of such of those persons as are persons whose dates of birth are matters of which the Authority have a record; and

($c$) in the case of each person whose name is included in the published summary, whether that person appears in the register—
\begin{enumerate}\item[]
(i) as disqualified in respect of only one scheme;

(ii) as disqualified in respect of two or more schemes but not in respect of all trust schemes; or

(iii) as disqualified in respect of all trust schemes.
\end{enumerate}
\end{enumerate}

(5) For the purposes of paragraph ($c$)  of subsection (4), the information contained in the published summary must be arranged in three separate lists, one for each of the descriptions of disqualification specified in the three sub-paragraphs of that paragraph.

(6) The Authority shall ensure, in the case of any published summary, that a person is not identified in the summary as a disqualified person if it appears to them that the determination by virtue of which that person appears in the register—
\begin{enumerate}\item[]
($a$) is the subject of any pending review, appeal or legal proceedings which could result in that person’s removal from the register; or

($b$) is a determination which might still become the subject of any such review, appeal or proceedings.
\end{enumerate}

(7) The Authority shall ensure, in the case of any published summary, that the particulars relating to a person do not appear in a particular list mentioned in subsection (5)  if it appears to them that a determination by virtue of which that person’s particulars would appear in that list—
\begin{enumerate}\item[]
($a$) is the subject of any pending review, appeal or legal proceedings which could result in such a revocation or other overturning of a disqualification of that person as would require his particulars to appear in a different list; or

($b$) is a determination which might still become the subject of any such review, appeal or proceedings.
\end{enumerate}

(8) Where subsection (7)  prevents a person’s particulars from being included in a particular list in the published summary, they shall be included, instead, in the list in which they would have been included if the disqualification to which the review, appeal or proceedings relate had already been revoked or otherwise overturned.

(9) For the purposes of this section a determination is one which might still become the subject of a review, appeal or proceedings if, and only if, in the case of that determination—
\begin{enumerate}\item[]
($a$) the time for the making of an application for a review, or for the bringing of an appeal or other proceedings, has not expired; and

($b$) there is a reasonable likelihood that such an application might yet be made, or that such an appeal or such proceedings might yet be brought.
\end{enumerate}

(10) In this section—
\begin{enumerate}\item[]
    “the disqualification register” means the register kept by the Authority under section 30(7); and

    “name”, in relation to a person any of whose names is recorded by the Authority as an initial, means that initial.” 
\end{enumerate}
\end{quotation}

\subsection*{Conditions of payment of surplus to employer}

11.---(1) Section 37 of the 1995 Act (payment of surplus to employer) shall be amended as follows.

(2) For paragraph ($d$)  of subsection (4)  (conditions of payment of surplus) there shall be substituted—
\begin{quotation}
“($d$) the annual rates of the pensions under the scheme are increased, at intervals of not more than twelve months, by at least the relevant percentage,”.
\end{quotation}

(3) After subsection (5)  there shall be inserted—
\begin{quotation}
“(5A) For the purposes of subsection (4)($d$), the relevant percentage is the percentage which, for the purposes of the increases of the annual rates of the pensions under the scheme—
\begin{enumerate}\item[]
($a$) falls to be computed by reference to a period which, except in the case of the first increase—
\begin{enumerate}\item[]
(i) begins with the end of the period by reference to which the last preceding increase was made; and

(ii) ends with a date which falls after the date of the last preceding increase;
\end{enumerate}
and

($b$) is equal to whichever is the lesser of—
\begin{enumerate}\item[]
(i) the percentage increase in the retail prices index over the period by reference to which the increase is made; and

(ii) the equivalent over that period of 5 per cent.\ per annum.”
\end{enumerate}
\end{enumerate}
\end{quotation}

(4) In subsection (6)  (interpretation of section), for the words from the beginning to the end of paragraph ($a$)  there shall be substituted—
\begin{quotation}
“(6) In this section—
\begin{enumerate}\item[]
($a$) “annual rate” has the same meaning as in section 54, and”.
\end{enumerate}
\end{quotation}

(5) The preceding provisions of this paragraph have effect in relation to payments made to an employer at any time after the commencement of this paragraph.

\subsection*{Duties relating to statements of contributions}

12.---(1) In section 41 of the 1995 Act (provision of documents for members), for subsection (5)  there shall be substituted—
\begin{quotation}
“(5) Regulations may in the case of occupational pension schemes provide for—
\begin{enumerate}\item[]
($a$) prescribed persons,

($b$) persons with prescribed qualifications or experience, or

($c$) persons approved by the Secretary of State,
\end{enumerate}
to act for the purposes of subsection (2)  instead of scheme auditors or actuaries.

(5A) Regulations may impose duties on the trustees or managers of an occupational pension scheme to disclose information to, and make documents available to, a person acting under subsection (5).

(5B) If any duty imposed under subsection (5A)  is not complied with, sections 3 and 10 apply to any trustee, and section 10 applies to any manager, who has failed to take all such steps as are reasonable to secure compliance.”
\end{quotation}

%(2) In section 49 of that Act, in subsection (9)  (duties in event of employer’s failure to pay contributions in prescribed period), after paragraph ($b$)  there shall be inserted “; and
%
%($c$) except in prescribed circumstances, any person acting instead of an auditor for the purposes of section 41(2)($b$)  in relation to the scheme must give notice of the failure, within the prescribed period, to the Authority.”
%
%(3) In that section, there shall be inserted after subsection (10)—
%
%“(10A)Section 10 applies to a person who fails to comply with subsection (9) ($c$).”
%
%(4) In section 88 of that Act (payment schedule to money purchase schemes: supplementary), after subsection (4)  there shall be inserted—
%
%“(5) Except in prescribed circumstances, any person acting instead of an auditor for the purposes of section 41(2)($b$)  in relation to an occupational pension scheme to which section 87 applies must, where any amounts payable in accordance with the payment schedule have not been paid on or before the due date, give notice of that fact, within the prescribed period, to the Authority.
%
%(6) Section 10 applies to a person so acting who fails to comply with subsection (5).”

\amendment{
Para. 12(2)--(6) are not yet in force.
}

\subsection*{Interpretation of Part I}

13. In this Part of this Schedule—
\begin{enumerate}\item[]
    “the 1993 Act” means the Pension Schemes Act 1993; and

    “the 1995 Act” means the Pensions Act 1995.  
\end{enumerate}

\section[Part II --- Alternative to anti-franking rules]{Part II\\*Alternative to anti-franking rules}

\renewcommand\parthead{--- Schedule 5 Part II}

\amendment{
Sch. 5 Pt. II is in force only for the purpose of making regulations and orders.
}

\subsection*{Cases in which alternative applies}

14.---(1) Subject to the following provisions of this paragraph, this Part of this Schedule applies, instead of Chapter III of Part IV of the 1993 Act (anti-franking rules), in the case of a person (“the pensioner”) who is entitled to benefits under any occupational pension scheme if the benefits to which he is entitled under the scheme include a guaranteed minimum pension.

(2) This Part of this Schedule does not apply in the pensioner’s case, instead of Chapter III of Part IV of the 1993 Act, unless—
\begin{enumerate}\item[]
($a$) the pensioner is a member of the scheme who, in relation to that scheme, left pensionable service after the coming into force of this Part of this Schedule;

($b$) the pensioner is the widow or widower of a member of the scheme whose pensionable service ended (by death or otherwise) after the coming into force of this Part of this Schedule; or

($c$) sub-paragraph (3)  applies to the benefits to which the pensioner is entitled under the scheme.
\end{enumerate}

(3) This sub-paragraph applies to the benefits to which the pensioner is entitled under the scheme if—
\begin{enumerate}\item[]
($a$) the time at which the benefits first become payable is after the coming into force of this Part of this Schedule;

($b$) the benefits do not first become payable in respect of the death of a member of the scheme to whom benefits had already become payable under the scheme before the coming into force of this Part of this Schedule; and

($c$) the trustees or managers of the scheme have elected, in the prescribed manner, that this Part of this Schedule should apply to benefits first becoming payable under the scheme after the coming into force of this Part of this Schedule.
\end{enumerate}

(4) This Part of this Schedule does not apply in the pensioner’s case (and, accordingly, Chapter III of Part IV of the 1993 Act does) if the scheme is a scheme of a prescribed description, unless the trustees or managers of the scheme have elected, in the prescribed manner, that this Part of this Schedule should apply in the case of the scheme.

(5) An election for the purposes of any provision of this paragraph—
\begin{enumerate}\item[]
($a$) shall not be exercisable differently in relation to different members of the scheme; and

($b$) once exercised, shall be irrevocable.
\end{enumerate}

\subsection*{Alternative rules}

15.---(1) Where this Part of this Schedule applies in the pensioner’s case, the amount of the benefits to which he is entitled under the scheme shall not be less than the amount of the benefits to which he would have been entitled under the scheme if his entitlement fell to be calculated by the method set out in sub-paragraph (2).

(2) That method is as follows—
\begin{enumerate}\item[]
    Step 1: compute the amount of any benefits consisting in the guaranteed minimum pension to which the pensioner is entitled;

    Step 2: compute what would have been the amount of those benefits on the assumptions set out in sub-paragraph (3);

    Step 3: determine the extent (if any) to which attributing an amount of benefits equal to the amount computed in accordance with Step 2 to rights accruing before 6th April 1997 would leave any such rights unused;

    Step 4: compute, in accordance with sub-paragraph (4), the amount of such of the benefits to which the pensioner is entitled under the scheme as are attributable to rights accruing before 6th April 1997 (if any) which, applying the determination in Step 3, would be left unused after the attribution of the amount mentioned in that Step to rights so accruing;

    Step 5: compute the amount resulting, on the required assumption, from the application of the statutory revaluations and increases in the case of the benefits computed in accordance with Step 4;

    Step 6: compute, in accordance with sub-paragraph (4), the amount of such of the benefits to which the pensioner is entitled under the scheme as are attributable to rights accruing on or after 6th April 1997;

    Step 7: compute the amount resulting, on the required assumption, from the application of the statutory revaluations and increases in the case of the benefits computed in accordance with Step 6;

    Step 8: aggregate the results of Steps 1, 5 and 7 to give the minimum benefits required by sub-paragraph (1). 
\end{enumerate}

(3) The assumptions referred to in Step 2 in sub-paragraph (2)  are—
\begin{enumerate}\item[]
($a$) that no increases are required to be made in accordance with section 15 or 109 of the 1993 Act (deferment increases and indexation);

($b$) that increases in accordance with section 16(1)  of that Act (revaluation of earnings factors for early leavers) of any earner’s earnings factors are to be calculated as if references to the final relevant year were references to whichever is the earlier of—
\begin{enumerate}\item[]
(i) the final relevant tax year; and

(ii) the tax year immediately preceding that in which the member in question left service that qualified him for salary-related benefits under the scheme; and
\end{enumerate}

($c$) that no increases in accordance with any provision included in the scheme by virtue of section 16(3)  of that Act (increases of weekly equivalent for person leaving contracted-out service before final relevant year) are to be made for any year after the tax year immediately preceding that in which the member in question left service that qualified him for salary-related benefits under the scheme.
\end{enumerate}

(4) For the purposes of Steps 4 and 6 in sub-paragraph (2)—
\begin{enumerate}\item[]
($a$) if (apart from this sub-paragraph) there would be a difference between the two Steps in the level of salary taken as the level by reference to which any salary-related benefits are to be computed, the level used for Step 4 must be no lower than that used for Step 6; and

($b$) statutory revaluations and increases shall not be attributed to rights accruing at any time.
\end{enumerate}

(5) For the purposes of Steps 5 and 7 in sub-paragraph (2), the required assumption is that the benefits in whose case the statutory revaluations and increases are applied comprise a whole pension deriving from the rights to which they are taken to be attributable for the purposes of Step 4 or, as the case may be, Step 6. 

(6) Subject to sub-paragraph (7), references in this paragraph to the statutory revaluations and increases are references to—
\begin{enumerate}\item[]
($a$) the revaluations required to be made in accordance with Chapter II of Part IV of the 1993 Act (revaluation of accrued benefits); and

($b$) the increases required to be made by virtue of section 51 of the 1995 Act (indexation).
\end{enumerate}

(7) For the purpose of applying the statutory revaluations and increases for the purposes of Steps 5 and 7 in sub-paragraph (2)—
\begin{enumerate}\item[]
($a$) money may be used in a way allowed by section 110(1)  of the 1993 Act (use of money to pay guaranteed minimum pension increase for subsequent year); and

($b$) any deductions authorised by section 53(1)  or (2)  of the 1995 Act (permitted deductions from statutory increases) may be made.
\end{enumerate}

(8) In this paragraph “the pensioner” has the meaning given by paragraph 14. 

(9) Any reference in this paragraph to a provision of the 1993 Act includes a reference to any enactment re-enacted in that provision.

\subsection*{Relationship between alternative rules and other rules}

16.---(1) Paragraph 15 shall not apply to benefits consisting in an alternative to a short service benefit provided for under section 73(2)($b$)  of the 1993 Act, except to the extent that—
\begin{enumerate}\item[]
($a$) that paragraph would apply for the computation of the short service benefit to which those benefits are an alternative; and

($b$) the amount of any of the alternative benefits falls to be computed wholly or partly by reference to the value of what would have been the short service benefit.
\end{enumerate}

(2) Section 94 of the 1993 Act (right to cash equivalent) shall have effect as if the provisions of this Part of this Schedule were included for the purposes of that section in the applicable rules.

(3) Subject to sub-paragraph (4), the preceding provisions of this Part of this Schedule override any provision of an occupational pension scheme with which they are inconsistent except a provision which, under subsection (3)  of section 129 of the 1993 Act, is a protected provision for the purposes of subsection (2)  of that section.

(4) The preceding provisions of this Part of this Schedule shall be without prejudice to any person’s entitlement to exercise—
\begin{enumerate}\item[]
($a$) any right of commutation, forfeiture or surrender of the whole or any part of any benefits computed in accordance with this Part of this Schedule;

($b$) any charge or lien on the whole or any part of any such benefits; or

($c$) any right of set-off against the whole or any part of any such benefits;
\end{enumerate}
and, accordingly, the computations to be done under paragraph 15 shall be done disregarding anything falling within any of paragraphs ($a$)  to ($c$).

\subsection*{Supplemental}

17.---(1) In this Part of this Schedule references to rights accruing to a member of a scheme before 6th April 1997 include references—
\begin{enumerate}\item[]
($a$) in relation to salary-related benefits, to rights accruing at any time in respect of service before that date; and

($b$) in relation to benefits of any description, to rights that derive from any transfer of accrued rights or transfer payment and represent rights accruing under any other scheme before that date;
\end{enumerate}
and a reference in this Part of this Schedule to rights accruing on or after that date shall be construed accordingly.

(2) For the purposes of this Part of this Schedule rights to money purchase benefits that are attributable to payments in respect of employment are rights accruing before 6th April 1997 in so far only as that employment was employment carried on before that date; and a reference in this Part of this Schedule to rights accruing on or after that date shall be construed accordingly.

(3) In this Part of this Schedule—
\begin{enumerate}\item[]
    “the 1993 Act” means the Pension Schemes Act 1993;

    “the 1995 Act” means the Pensions Act 1995; and

    “salary-related benefits” means benefits that are not money purchase benefits. 
\end{enumerate}

(4) Expressions defined for the purposes of the 1993 Act have the same meanings in this Part of this Schedule as they have in that Act.

(5) Any power of the Secretary of State to make regulations under this Part of this Schedule shall be exercisable by statutory instrument subject to annulment in pursuance of a resolution of either House of Parliament.

(6) The Secretary of State may by order make such modifications of paragraphs 14 to 16 as he considers appropriate.

(7) An order under sub-paragraph (6)  shall be made by statutory instrument subject to annulment in pursuance of a resolution of either House of Parliament.

(8) Subsections (2)  to (5)  of section 182 of the 1993 Act (supplemental provision in connection with powers to make subordinate legislation under that Act) shall apply—
\begin{enumerate}\item[]
($a$) to any power of the Secretary of State to make regulations under this Part of this Schedule, and

($b$) to the power of the Secretary of State to make an order under sub-paragraph (6),
\end{enumerate}
as they apply to his powers to make regulations and orders under that Act.

(9) In section 178($a$)  of the 1993 Act (regulations providing for who is to be treated as a manager of a scheme), for the words from “or Part III” to “1999” there shall be substituted “, Part III or IV of the Welfare Reform and Pensions Act 1999 or Part II of Schedule 5 to the Child Support, Pensions and Social Security Act 2000”. 

\part[Schedule 6 --- Social security investigation powers]{Schedule 6\\*Social security investigation powers}

\renewcommand\parthead{--- Schedule 6}

\section*{Preliminary}

1. Part VI of the Social Security Administration Act 1992 (enforcement) shall be amended as follows.

\section*{Replacement for inspector’s powers}

2. The following sections shall be substituted for section 110 (appointment and powers of inspectors)—
\begin{quotation}
\subsection*{“109A. Authorisations for investigators}

(1) An individual who for the time being has the Secretary of State’s authorisation for the purposes of this Part shall be entitled, for any one or more of the purposes mentioned in subsection (2)  below, to exercise any of the powers which are conferred on an authorised officer by sections 109B and 109C below.

(2) Those purposes are—
\begin{enumerate}\item[]
($a$) ascertaining in relation to any case whether a benefit is or was payable in that case in accordance with any provision of the relevant social security legislation;

($b$) investigating the circumstances in which any accident, injury or disease which has given rise, or may give rise, to a claim for—
\begin{enumerate}\item[]
(i) industrial injuries benefit, or

(ii) any benefit under any provision of the relevant social security legislation,
\end{enumerate}
occurred or may have occurred, or was or may have been received or contracted;

($c$) ascertaining whether provisions of the relevant social security legislation are being, have been or are likely to be contravened (whether by particular persons or more generally);

($d$) preventing, detecting and securing evidence of the commission (whether by particular persons or more generally) of benefit offences.
\end{enumerate}

(3) An individual has the Secretary of State’s authorisation for the purposes of this Part if, and only if, the Secretary of State has granted him an authorisation for those purposes and he is—
\begin{enumerate}\item[]
($a$) an official of a Government department;

($b$) an individual employed by an authority administering housing benefit or council tax benefit;

($c$) an individual employed by an authority or joint committee that carries out functions relating to housing benefit or council tax benefit on behalf of the authority administering that benefit; or

($d$) an individual employed by a person authorised by or on behalf of any such authority or joint committee as is mentioned in paragraph ($b$)  or ($c$)  above to carry out functions relating to housing benefit or council tax benefit for that authority or committee.
\end{enumerate}

(4) An authorisation granted for the purposes of this Part to an individual of any of the descriptions mentioned in subsection (3)  above—
\begin{enumerate}\item[]
($a$) must be contained in a certificate provided to that individual as evidence of his entitlement to exercise powers conferred by this Part;

($b$) may contain provision as to the period for which the authorisation is to have effect; and

($c$) may restrict the powers exercisable by virtue of the authorisation so as to prohibit their exercise except for particular purposes, in particular circumstances or in relation to particular benefits or particular provisions of the relevant social security legislation.
\end{enumerate}

(5) An authorisation granted under this section may be withdrawn at any time by the Secretary of State.

(6) Where the Secretary of State grants an authorisation for the purposes of this Part to an individual employed by a local authority, or to an individual employed by a person who carries out functions relating to housing benefit or council tax benefit on behalf of a local authority—
\begin{enumerate}\item[]
($a$) the Secretary of State and the local authority shall enter into such arrangements (if any) as they consider appropriate with respect to the carrying out of functions conferred on that individual by or in connection with the authorisation granted to him; and

($b$) the Secretary of State may make to the local authority such payments (if any) as he thinks fit in respect of the carrying out by that individual of any such functions.
\end{enumerate}

(7) The matters on which a person may be authorised to consider and report to the Secretary of State under section 139A below shall be taken to include the carrying out by any such individual as is mentioned in subsection (3)($b$)  to ($d$)  above of any functions conferred on that individual by virtue of any grant by the Secretary of State of an authorisation for the purposes of this Part.

(8) The powers conferred by sections 109B and 109C below shall be exercisable in relation to persons holding office under the Crown and persons in the service of the Crown, and in relation to premises owned or occupied by the Crown, as they are exercisable in relation to other persons and premises.

\subsection*{109B. Power to require information}

(1) An authorised officer who has reasonable grounds for suspecting that a person—
\begin{enumerate}\item[]
($a$) is a person falling within subsection (2)  below, and

($b$) has or may have possession of or access to any information about any matter that is relevant for any one or more of the purposes mentioned in section 109A(2)  above,
\end{enumerate}
may, by written notice, require that person to provide all such information described in the notice as is information of which he has possession, or to which he has access, and which it is reasonable for the authorised officer to require for a purpose so mentioned.

(2) The persons who fall within this subsection are—
\begin{enumerate}\item[]
($a$) any person who is or has been an employer or employee within the meaning of any provision made by or under the Contributions and Benefits Act;

($b$) any person who is or has been a self-employed earner within the meaning of any such provision;

($c$) any person who by virtue of any provision made by or under that Act falls, or has fallen, to be treated for the purposes of any such provision as a person within paragraph ($a$)  or ($b$)  above;

($d$) any person who is carrying on, or has carried on, any business involving the supply of goods for sale to the ultimate consumers by individuals not carrying on retail businesses from retail premises;

($e$) any person who is carrying on, or has carried on, any business involving the supply of goods or services by the use of work done or services performed by persons other than employees of his;

($f$) any person who is carrying on, or has carried on, an agency or other business for the introduction or supply, to persons requiring them, of persons available to do work or to perform services;

($g$) any local authority acting in their capacity as an authority responsible for the granting of any licence;

($h$) any person who is or has been a trustee or manager of a personal or occupational pension scheme;

($i$) any person who is or has been liable to make a compensation payment or a payment to the Secretary of State under section 6 of the Social Security (Recovery of Benefits) Act 1997 (payments in respect of recoverable benefits); and

($j$) the servants and agents of any such person as is specified in any of paragraphs ($a$)  to ($i$)  above.
\end{enumerate}

(3) The obligation of a person to provide information in accordance with a notice under this section shall be discharged only by the provision of that information, at such reasonable time and in such form as may be specified in the notice, to the authorised officer who—
\begin{enumerate}\item[]
($a$) is identified by or in accordance with the terms of the notice; or

($b$) has been identified, since the giving of the notice, by a further written notice given by the authorised officer who imposed the original requirement or another authorised officer.
\end{enumerate}

(4) The power of an authorised officer under this section to require the provision of information shall include a power to require the production and delivery up and (if necessary) creation of, or of copies of or extracts from, any such documents containing the information as may be specified or described in the notice imposing the requirement.

(5) No one shall be required under this section to provide any information (whether in documentary form or otherwise) that tends to incriminate either himself or, in the case of a person who is married, his spouse.

\subsection*{109C. Powers of entry}

(1) An authorised officer shall be entitled, at any reasonable time and either alone or accompanied by such other persons as he thinks fit, to enter any premises which—
\begin{enumerate}\item[]
($a$) are liable to inspection under this section; and

($b$) are premises to which it is reasonable for him to require entry in order to exercise the powers conferred by this section.
\end{enumerate}

(2) An authorised officer who has entered any premises liable to inspection under this section may—
\begin{enumerate}\item[]
($a$) make such an examination of those premises, and

($b$) conduct any such inquiry there,
\end{enumerate}
as appears to him appropriate for any one or more of the purposes mentioned in section 109A(2)  above.

(3) An authorised officer who has entered any premises liable to inspection under this section may—
\begin{enumerate}\item[]
($a$) question any person whom he finds there;

($b$) require any person whom he finds there to do any one or more of the following—
\begin{enumerate}\item[]
(i) to provide him with such information,

(ii) to produce and deliver up and (if necessary) create such documents or such copies of, or extracts from, documents,
\end{enumerate}
as he may reasonably require for any one or more of the purposes mentioned in section 109A(2)  above; and

($c$) take possession of and either remove or make his own copies of any such documents as appear to him to contain information that is relevant for any of those purposes.
\end{enumerate}

(4) The premises liable to inspection under this section are any premises (including premises consisting in the whole or a part of a dwelling house) which an authorised officer has reasonable grounds for suspecting are—
\begin{enumerate}\item[]
($a$) premises which are a person’s place of employment;

($b$) premises from which a trade or business is being carried on or where documents relating to a trade or business are kept by the person carrying it on or by another person on his behalf;

($c$) premises from which a personal or occupational pension scheme is being administered or where documents relating to the administration of such a scheme are kept by the person administering the scheme or by another person on his behalf;

($d$) premises where a person who is the compensator in relation to any such accident, injury or disease as is referred to in section 109A(2)($b$)  above is to be found;

($e$) premises where a person on whose behalf any such compensator has made, may have made or may make a compensation payment is to be found.
\end{enumerate}

(5) An authorised officer applying for admission to any premises in accordance with this section shall, if required to do so, produce the certificate containing his authorisation for the purposes of this Part.

(6) Subsection (5)  of section 109B applies for the purposes of this section as it applies for the purposes of that section.”
\end{quotation}

\section*{Exercise of powers on behalf of local authorities}

3. For sections 110A and 110B (inspectors appointed by local authorities etc.\ for the purposes of housing benefit or council tax benefit), there shall be substituted—
\begin{quotation}
\subsection*{“110A. Authorisations by local authorities}

(1) An individual who for the time being has the authorisation for the purposes of this Part of an authority administering housing benefit or council tax benefit (“a local authority authorisation”) shall be entitled, for any one or more of the purposes mentioned in subsection (2)  below, to exercise any of the powers which, subject to subsection (8)  below, are conferred on an authorised officer by sections 109B and 109C above.

(2) Those purposes are—
\begin{enumerate}\item[]
($a$) ascertaining in relation to any case whether housing benefit or council tax benefit is or was payable in that case;

($b$) ascertaining whether provisions of the relevant social security legislation that relate to housing benefit or council tax benefit are being, have been or are likely to be contravened (whether by particular persons or more generally);

($c$) preventing, detecting and securing evidence of the commission (whether by particular persons or more generally) of benefit offences relating to housing benefit or council tax benefit.
\end{enumerate}

(3) An individual has the authorisation for the purposes of this Part of an authority administering housing benefit or council tax benefit if, and only if, that authority have granted him an authorisation for those purposes and he is—
\begin{enumerate}\item[]
($a$) an individual employed by that authority;

($b$) an individual employed by another authority or joint committee that carries out functions relating to housing benefit or council tax benefit on behalf of that authority;

($c$) an individual employed by a person authorised by or on behalf of—
\begin{enumerate}\item[]
(i) the authority in question,

(ii) any such authority or joint committee as is mentioned in paragraph ($b$)  above,
\end{enumerate}
to carry out functions relating to housing benefit or council tax benefit for that authority or committee;

($d$) an official of a Government department.
\end{enumerate}

(4) Subsection (4)  of section 109A above shall apply in relation to a local authority authorisation as it applies in relation to an authorisation under that section.

(5) A local authority authorisation may be withdrawn at any time by the authority that granted it or by the Secretary of State.

(6) The certificate or other instrument containing the grant or withdrawal by any local authority of any local authority authorisation must be issued under the hand of either—
\begin{enumerate}\item[]
($a$) the officer designated under section 4 of the Local Government and Housing Act 1989 as the head of the authority’s paid service; or

($b$) the officer who is the authority’s chief finance officer (within the meaning of section 5 of that Act).
\end{enumerate}

(7) It shall be the duty of any authority with power to grant local authority authorisations to comply with any directions of the Secretary of State as to—
\begin{enumerate}\item[]
($a$) whether or not such authorisations are to be granted by that authority;

($b$) the period for which authorisations granted by that authority are to have effect;

($c$) the number of persons who may be granted authorisations by that authority at any one time; and

($d$) the restrictions to be contained by virtue of subsection (4)  above in the authorisations granted by that authority for those purposes.
\end{enumerate}

(8) The powers conferred by sections 109B and 109C above shall have effect in the case of an individual who is an authorised officer by virtue of this section as if those sections had effect—
\begin{enumerate}\item[]
($a$) with the substitution for every reference to the purposes mentioned in section 109A(2)  above of a reference to the purposes mentioned in subsection (2)  above; and

($b$) with the substitution for every reference to the relevant social security legislation of a reference to so much of it as relates to housing benefit or council tax benefit.
\end{enumerate}

(9) Nothing in this section conferring any power on an authorised officer in relation to housing benefit or council tax benefit shall require that power to be exercised only in relation to cases in which the authority administering the benefit is the authority by whom that officer’s authorisation was granted.”
\end{quotation}

\section*{Consequential amendments}

4. In section 111 (delay and obstruction of inspector)—
\begin{enumerate}\item[]
($a$) in subsection (3), for “section 110(5)” there shall be substituted “an authorisation granted under section 109A or 110A”; and

($b$) in subsection (4)—
\begin{enumerate}\item[]
(i) for “section 110(5)  above any power conferred by section 110 above” there shall be substituted “an authorisation granted under section 109A or 110A above, any power conferred by section 109B or 109C above”; and

(ii) for the words “section 110”, where they occur at the end of the subsection, there shall be substituted “sections 109B and 109C”.
\end{enumerate}
\end{enumerate}

\medskip

5. In section 111A(1)  (dishonest representations), before “social security legislation” there shall be inserted “relevant”.

\medskip

6. In section 112(1)  (false representations), before “social security legislation” there shall be inserted “relevant”.

\medskip

7.---(1) In subsection (1)  of section 113 (breach of regulations)—
\begin{enumerate}\item[]
($a$) for “Acts to which section 110 above applies” there shall be substituted “legislation to which this section applies”;

($b$) for the words “that Act”, in the first place where they occur, there shall be substituted “that legislation”; and

($c$) for the words “that Act”, where they occur in paragraph ($b$), there shall be substituted “any enactment contained in the legislation in question”.
\end{enumerate}

(2) After that subsection there shall be inserted—
\begin{quotation}
“(1A) The legislation to which this section applies is—
\begin{enumerate}\item[]
($a$) the relevant social security legislation; and

($b$) the enactments specified in section 121DA(1)  so far as relating to contributions, statutory sick pay or statutory maternity pay.”
\end{enumerate}
\end{quotation}

\medskip

8. After section 121D (but still in Part VI) there shall be inserted—
\begin{quotation}
\subsection*{“121DA. Interpretation of Part VI}

(1) In this Part “the relevant social security legislation” means the provisions of any of the following, except so far as relating to contributions, working families' tax credit, disabled person’s tax credit, statutory sick pay or statutory maternity pay, that is to say—
\begin{enumerate}\item[]
($a$) the Contributions and Benefits Act;

($b$) this Act;

($c$) the Pensions Act, except Part III;

($d$) section 4 of the Social Security (Incapacity for Work) Act 1994;

($e$) the Jobseekers Act 1995;

($f$) the Social Security (Recovery of Benefits) Act 1997;

($g$) Parts I and IV of the Social Security Act 1998;

($h$) Part V of the Welfare Reform and Pensions Act 1999;

($i$) the Social Security Pensions Act 1975;

($j$) the Social Security Act 1973;

($k$) any subordinate legislation made, or having effect as if made, under any enactment specified in paragraphs ($a$)  to ($j$)  above.
\end{enumerate}

(2) In this Part “authorised officer” means a person acting in accordance with any authorisation for the purposes of this Part which is for the time being in force in relation to him.

(3) For the purposes of this Part—
\begin{enumerate}\item[]
($a$) references to a document include references to anything in which information is recorded in electronic or any other form;

($b$) the requirement that a notice given by an authorised officer be in writing shall be taken to be satisfied in any case where the contents of the notice—
\begin{enumerate}\item[]
(i) are transmitted to the recipient of the notice by electronic means; and

(ii) are received by him in a form that is legible and capable of being recorded for future reference.
\end{enumerate}
\end{enumerate}

(4) In this Part “premises” includes—
\begin{enumerate}\item[]
($a$) moveable structures and vehicles, vessels, aircraft and hovercraft;

($b$) installations that are offshore installations for the purposes of the Mineral Workings (Offshore Installations) Act 1971; and

($c$) places of all other descriptions whether or not occupied as land or otherwise;
\end{enumerate}
and references in this Part to the occupier of any premises shall be construed, in relation to premises that are not occupied as land, as references to any person for the time being present at the place in question.

(5) In this Part—
\begin{enumerate}\item[]
    “benefit” includes any allowance, payment, credit or loan;

    “benefit offence” means a criminal offence committed in connection with a claim for benefit under a provision of the relevant social security legislation, or in connection with the receipt or payment of such a benefit; and

    “compensation payment” has the same meaning as in the Social Security (Recovery of Benefits) Act 1997.  
\end{enumerate}

(6) In this Part—
\begin{enumerate}\item[]
($a$) any reference to a person authorised to carry out any function relating to housing benefit or council tax benefit shall include a reference to a person providing services relating to the benefit directly or indirectly to an authority administering it; and

($b$) any reference to the carrying out of a function relating to such a benefit shall include a reference to the provision of any services relating to it.
\end{enumerate}

(7) In this section “subordinate legislation” has the same meaning as in the Interpretation Act 1978.”
\end{quotation}

\medskip

9. In paragraph 5 of Schedule 10 to the Social Security Administration Act 1992 (transitional provisions for supplementary benefit), for the words before sub-paragraph ($a$)  there shall be substituted “Part VI of this Act shall have effect as if the following Acts were included in the Acts comprised in the relevant social security legislation”.

\part[Schedule 7 --- Housing benefit and council tax benefit: revisions and appeals]{Schedule 7\\*Housing benefit and council tax benefit: revisions and appeals}

\renewcommand\parthead{--- Schedule 7}

\section*{Introductory}

1.---(1) In this Schedule “relevant authority” means an authority administering housing benefit or council tax benefit.

(2) In this Schedule “relevant decision” means any of the following—
\begin{enumerate}\item[]
($a$) a decision of a relevant authority on a claim for housing benefit or council tax benefit;

($b$) any decision under paragraph 4 of this Schedule which supersedes a decision falling within paragraph ($a$), within this paragraph or within paragraph ($b$)  of sub-paragraph (1)  of that paragraph;
\end{enumerate}
but references in this Schedule to a relevant decision do not include references to a decision under paragraph 3 to revise a relevant decision.

\section*{Decisions on claims for benefit}

2. Where at any time a claim for housing benefit or council tax benefit is decided by a relevant authority—
\begin{enumerate}\item[]
($a$) the claim shall not be regarded as subsisting after that time; and

($b$) accordingly, the claimant shall not (without making a further claim) be entitled to the benefit on the basis of circumstances not obtaining at that time.
\end{enumerate}

\section*{Revision of decisions}

3.---(1) Any relevant decision may be revised or further revised by the relevant authority which made the decision—
\begin{enumerate}\item[]
($a$) either within the prescribed period or in prescribed cases or circumstances; and

($b$) either on an application made for the purpose by a person affected by the decision or on their own initiative;
\end{enumerate}
and regulations may prescribe the procedure by which a decision of a relevant authority may be so revised.

(2) In making a decision under sub-paragraph (1), the relevant authority need not consider any issue that is not raised by the application or, as the case may be, did not cause them to act on their own initiative.

(3) Subject to sub-paragraphs (4)  and (5)  and paragraph 18, a revision under this paragraph shall take effect as from the date on which the original decision took (or was to take) effect.

(4) Regulations may provide that, in prescribed cases or circumstances, a revision under this paragraph shall take effect as from such other date as may be prescribed.

(5) Where a decision is revised under this paragraph, for the purposes of any rule as to the time allowed for bringing an appeal, the decision shall be regarded as made on the date on which it is so revised.

(6) Except in prescribed circumstances, an appeal against a decision of the relevant authority shall lapse if the decision is revised under this paragraph before the appeal is determined.

\section*{Decisions superseding earlier decisions}

4.---(1) Subject to sub-paragraph (4), the following, namely—
\begin{enumerate}\item[]
($a$) any relevant decision (whether as originally made or as revised under paragraph 3), and

($b$) any decision under this Schedule of an appeal tribunal or a Commissioner,
\end{enumerate}
may be superseded by a decision made by the appropriate relevant authority, either on an application made for the purpose by a person affected by the decision or on their own initiative.

(2) In this paragraph “the appropriate relevant authority” means the authority which made the decision being superseded, the decision appealed against to the tribunal or, as the case may be, the decision to which the decision being appealed against to the Commissioner relates.

(3) In making a decision under sub-paragraph (1), the relevant authority need not consider any issue that is not raised by the application or, as the case may be, did not cause them to act on their own initiative.

(4) Regulations may prescribe the cases and circumstances in which, and the procedure by which, a decision may be made under this paragraph.

(5) Subject to sub-paragraph (6)  and paragraph 18, a decision under this paragraph shall take effect as from the date on which it is made or, where applicable, the date on which the application was made.

(6) Regulations may provide that, in prescribed cases or circumstances, a decision under this paragraph shall take effect as from such other date as may be prescribed.

\section*{Use of experts by relevant authorities}

5. Where it appears to a relevant authority that a matter in relation to which a relevant decision falls to be made by them involves a question of fact requiring special expertise, they may direct that, in dealing with that matter, they shall have the assistance of one or more persons appearing to them to have knowledge or experience which would be relevant in determining that question.

\section*{Appeal to appeal tribunal}

6.---(1) Subject to sub-paragraph (2), this paragraph applies to any relevant decision (whether as originally made or as revised under paragraph 3) of a relevant authority which—
\begin{enumerate}\item[]
($a$) is made on a claim for, or on an award of, housing benefit or council tax benefit; or

($b$) does not fall within paragraph ($a$)  but is of a prescribed description.
\end{enumerate}

(2) This paragraph does not apply to—
\begin{enumerate}\item[]
($a$) any decision terminating or reducing the amount of a person’s housing benefit or council tax benefit that is made in consequence of any decision made under regulations under section 2A of the Administration Act (work-focused interviews);

($b$) any decision of a relevant authority as to the application or operation of any modification of a housing benefit scheme or council tax benefit scheme under section 134(8)($a$)  or section 139(6)($a$)  of the Administration Act (disregard of war disablement and war widows' pensions);

($c$) so much of any decision of a relevant authority as adopts a decision of a rent officer under any order made by virtue of section 122 of the Housing Act 1996 (decisions of rent officers for the purposes of housing benefit);

($d$) any decision of a relevant authority as to the amount of benefit to which a person is entitled in a case in which the amount is determined by the rate of benefit provided for by law; or

($e$) any such other decision as may be prescribed.
\end{enumerate}

(3) In the case of a decision to which this paragraph applies, any person affected by the decision shall have a right to appeal to an appeal tribunal.

(4) Nothing in sub-paragraph (3)  shall confer a right of appeal in relation to—
\begin{enumerate}\item[]
($a$) a prescribed decision; or

($b$) a prescribed determination embodied in or necessary to a decision.
\end{enumerate}

(5) Regulations under sub-paragraph (4)  shall not prescribe any decision or determination that relates to the conditions of entitlement to housing benefit or council tax benefit for which a claim has been validly made.

(6) Where any amount of housing benefit or council tax benefit is determined to be recoverable under or by virtue of section 75 or 76 of the Administration Act (overpayments and excess benefits), any person from whom it has been determined that it is so recoverable shall have a right of appeal to an appeal tribunal.

(7) A person with a right of appeal under this paragraph shall be given such notice of the decision in respect of which he has that right, and of that right, as may be prescribed.

(8) Regulations may make provision as to the manner in which, and the time within which, appeals are to be brought.

(9) In deciding an appeal under this paragraph, an appeal tribunal—
\begin{enumerate}\item[]
($a$) need not consider any issue that is not raised by the appeal; and

($b$) shall not take into account any circumstances not obtaining at the time when the decision appealed against was made.
\end{enumerate}

\section*{Redetermination etc.\ of appeals by tribunal}

7.---(1) This paragraph applies where an application is made to a person for leave under paragraph 8(7)($a$)  or ($c$)  to appeal from a decision of an appeal tribunal.

(2) If the person considers that the decision was erroneous in point of law, he may set aside the decision and refer the case either for redetermination by the tribunal or for determination by a differently constituted tribunal.

(3) If each of the principal parties to the case expresses the view that the decision was erroneous in point of law, the person shall set aside the decision and refer the case for determination by a differently constituted tribunal.

(4) In this paragraph and paragraph 8 “principal parties” means—
\begin{enumerate}\item[]
($a$) where he is the applicant for leave to appeal or the circumstances are otherwise such as may be prescribed, the Secretary of State;

($b$) the relevant authority against whose decision the appeal to the appeal tribunal was brought; and

($c$) the person affected by the decision against which the appeal to the appeal tribunal was brought or by the tribunal’s decision on that appeal.
\end{enumerate}

\section*{Appeal from tribunal to Commissioner}

8.---(1) Subject to the provisions of this paragraph, an appeal lies to a Commissioner from any decision of an appeal tribunal under paragraph 6 or 7 on the ground that the decision of the tribunal was erroneous in point of law.

(2) An appeal lies under this paragraph at the instance of any of the following—
\begin{enumerate}\item[]
($a$) the Secretary of State;

($b$) the relevant authority against whose decision the appeal to the appeal tribunal was brought;

($c$) any person affected by the decision against which the appeal to the appeal tribunal was brought or by the tribunal’s decision on that appeal.
\end{enumerate}

(3) If each of the principal parties to the appeal expresses the view that the decision appealed against was erroneous in point of law, the Commissioner may set aside the decision and refer the case to a tribunal with directions for its determination.

(4) Where the Commissioner holds that the decision appealed against was erroneous in point of law, he shall set it aside.

(5) Where under sub-paragraph (4)  the Commissioner sets aside a decision—
\begin{enumerate}\item[]
($a$) he shall have power, if he can do so without making fresh or further findings of fact, to give the decision which he considers the tribunal should have given;

($b$) he shall also have power, if he considers it expedient, to make such findings and to give such decision as he considers appropriate in the light of them; and

($c$) if he does not exercise the power in paragraph ($a$)  or ($b$), he shall refer the case to a tribunal with directions for its determination.
\end{enumerate}

(6) Subject to any direction of the Commissioner, a reference under sub-paragraph (3)  or (5)($c$)  shall be to a differently constituted tribunal.

(7) No appeal lies under this paragraph without leave; and leave for the purposes of this sub-paragraph may be given—
\begin{enumerate}\item[]
($a$) by the person who constituted, or was the chairman of, the tribunal when the decision to be appealed against was given;

($b$) subject to and in accordance with regulations, by a Commissioner; or

($c$) in a prescribed case, by such person not falling within paragraph ($a$)  or ($b$)  as may be prescribed.
\end{enumerate}

(8) Regulations may make provision as to the manner in which, and the time within which, appeals are to be brought and applications made for leave to appeal.

\section*{Appeal from Commissioner on point of law}

9.---(1) Subject to sub-paragraphs (2)  and (3), an appeal on a question of law shall lie to the appropriate court from any decision of a Commissioner.

(2) No appeal under this paragraph shall lie from a decision except—
\begin{enumerate}\item[]
($a$) with the leave of the Commissioner who gave the decision or, in a prescribed case, with the leave of a Commissioner selected in accordance with regulations; or

($b$) if he refuses leave, with the leave of the appropriate court.
\end{enumerate}

(3) An application for leave under this paragraph in respect of a Commissioner’s decision may only be made by—
\begin{enumerate}\item[]
($a$) a person who, before the proceedings before the Commissioner were begun, was entitled to appeal to the Commissioner from the decision to which the Commissioner’s decision relates;

($b$) any other person who was a party to the proceedings in which the decision to which the Commissioner’s decision relates was given;

($c$) any other person who is authorised by regulations to apply for leave;
\end{enumerate}
and regulations may make provision with respect to the manner in which, and the time within which, applications must be made to a Commissioner for leave under this paragraph, and with respect to the procedure for dealing with such applications.

(4) On an application to a Commissioner for leave under this paragraph, it shall be the duty of the Commissioner to specify as the appropriate court—
\begin{enumerate}\item[]
($a$) the Court of Appeal if it appears to him that the relevant dwelling is in England or Wales; and

($b$) the Court of Session if it appears to him that the relevant dwelling is in Scotland;
\end{enumerate}
except that if it appears to him, having regard to the circumstances of the case and in particular to the convenience of the persons who may be parties to the proposed appeal, that he should specify a different court mentioned in paragraph ($a$)  or ($b$)  as the appropriate court, it shall be his duty to specify that court as the appropriate court.

(5) In this paragraph—
\begin{enumerate}\item[]
    “the appropriate court”, except in sub-paragraph (4), means the court specified in pursuance of that sub-paragraph;

    “the relevant dwelling”, in relation to any decision, means the dwelling by reference to which any claim or award of housing benefit or council tax benefit to which the decision relates was made. 
\end{enumerate}

\section*{Procedure}

10.---(1) Regulations may make for the purposes of this Schedule any such provision as is specified in Schedule 5 to the Social Security Act 1998, or as would be so specified if the references to the Secretary of State in paragraph 1 of that Schedule were references to a relevant authority.

(2) Regulations prescribing the procedure to be followed in cases before a Commissioner shall provide that any hearing shall be in public except in so far as the Commissioner for special reasons otherwise directs.

(3) It is hereby declared that the power by regulations to prescribe procedure includes power—
\begin{enumerate}\item[]
($a$) to make provision as to the representation of one person, at any hearing of a case, by another person whether having professional qualifications or not; and

($b$) to confer on the Secretary of State a right to be represented and heard in any proceedings before a Commissioner to which he is not already a party.
\end{enumerate}

(4) If it appears to a Commissioner that a matter before him involves a question of fact of special difficulty, he may direct that in dealing with that matter he shall have the assistance of one or more persons appearing to him to have knowledge or experience which would be relevant in determining that question.

(5) If it appears to the Chief Commissioner (or, in the case of his inability to act, to such other of the Commissioners as he may have nominated to act for the purpose) that—
\begin{enumerate}\item[]
($a$) an application for leave under paragraph 8(7)($b$), or

($b$) an appeal,
\end{enumerate}
falling to be heard by one of the Commissioners involves a question of law of special difficulty, he may direct that the application or appeal be dealt with, not by that Commissioner alone, but by a tribunal consisting of any three or more of the Commissioners.

(6) If the decision of such a tribunal is not unanimous, the decision of the majority shall be the decision of the tribunal; and the presiding Commissioner shall have a casting vote if the votes (including his first vote) are equally divided.

(7) Where a direction is given under sub-paragraph (5)($a$), paragraph 8(7)($b$)  shall have effect as if the reference to a Commissioner were a reference to such a tribunal as is mentioned in sub-paragraph (5).

(8) Except so far as it may be applied in relation to England and Wales by regulations, Part I of the Arbitration Act 1996 shall not apply to any proceedings under this Schedule.

\section*{Finality of decisions}

11. Subject to the provisions of this Schedule, any decision made in accordance with the preceding provisions of this Schedule shall be final.

\section*{Matters arising as respects decisions}

12. Regulations may make provision as respects matters arising—
\begin{enumerate}\item[]
($a$) pending any decision under this Schedule of a relevant authority, an appeal tribunal or a Commissioner which relates to—
\begin{enumerate}\item[]
(i) any claim for housing benefit or council tax benefit;

(ii) any person’s entitlement to such a benefit or its receipt;
\end{enumerate}
or

($b$) out of the revision under paragraph 3, or on appeal, of any such decision.
\end{enumerate}

\section*{Suspension in prescribed circumstances}

13.---(1) Regulations may provide for—
\begin{enumerate}\item[]
($a$) suspending, in whole or in part, any payments of housing benefit or council tax benefit;

($b$) suspending, in whole or in part, any reduction (by way of council tax benefit) in the amount that a person is or will become liable to pay in respect of council tax;

($c$) the subsequent making, or restoring, in prescribed circumstances of any or all of the payments, or reductions, so suspended.
\end{enumerate}

(2) Regulations made under sub-paragraph (1)  may, in particular, make provision for any case where, in relation to a claim for housing benefit or council tax benefit—
\begin{enumerate}\item[]
($a$) it appears to the relevant authority that an issue arises whether the conditions for entitlement to such a benefit are or were fulfilled;

($b$) it appears to the relevant authority that an issue arises whether a decision as to an award of such a benefit should be revised (under paragraph 3) or superseded (under paragraph 4);

($c$) an appeal is pending against a decision of an appeal tribunal, a Commissioner or a court; or

($d$) it appears to the relevant authority, where an appeal is pending against the decision given by a Commissioner or a court in a different case, that if the appeal were to be determined in a particular way an issue would arise whether the award of housing benefit or council tax benefit in the case itself ought to be revised or superseded.
\end{enumerate}

(3) For the purposes of sub-paragraph (2), an appeal against a decision is pending if—
\begin{enumerate}\item[]
($a$) an appeal against the decision has been brought but not determined;

($b$) an application for leave to appeal against the decision has been made but not determined; or

($c$) the time within which—
\begin{enumerate}\item[]
(i) an application for leave to appeal may be made, or

(ii) an appeal against the decision may be brought,
\end{enumerate}
has not expired and the circumstances are such as may be prescribed.
\end{enumerate}

(4) In sub-paragraph (2)($d$)  the reference to a different case—
\begin{enumerate}\item[]
($a$) includes a reference to a case involving a different relevant authority; but

($b$) does not include a reference to a case relating to a different benefit unless the different benefit is housing benefit or council tax benefit.
\end{enumerate}

\section*{Suspension for failure to furnish information etc.}

14.---(1) The powers conferred by this paragraph are exercisable in relation to persons who fail to comply with information requirements.

(2) Regulations may provide for—
\begin{enumerate}\item[]
($a$) suspending, in whole or in part, any payments of housing benefit or council tax benefit;

($b$) suspending, in whole or in part, any reduction (by way of council tax benefit) in the amount that a person is or will become liable to pay in respect of council tax;

($c$) the subsequent making, or restoring, in prescribed circumstances of any or all of the payments, or any right, so suspended.
\end{enumerate}

(3) In this paragraph and paragraph 15 “information requirement” means—
\begin{enumerate}\item[]
($a$) in the case of housing benefit, a requirement in pursuance of regulations made by virtue of section 5(1)($hh$)  of the Administration Act to furnish information or evidence needed for a determination whether a decision on an award of that benefit should be revised under paragraph 3 or superseded under paragraph 4 of this Schedule; and

($b$) in the case of council tax benefit, a requirement made in pursuance of regulations under section 6(1)($hh$)  of the Administration Act to furnish information or evidence needed for a determination whether a decision on an award of that benefit should be so revised or superseded.
\end{enumerate}

\section*{Termination in cases of a failure to furnish information}

15. Regulations may provide that, except in prescribed cases or circumstances—
\begin{enumerate}\item[]
($a$) a person whose benefit has been suspended in accordance with regulations under paragraph 13 and who subsequently fails to comply with an information requirement, or

($b$) a person whose benefit has been suspended in accordance with regulations under paragraph 14 for failing to comply with such a requirement,
\end{enumerate}
shall cease to be entitled to the benefit from a date not earlier than the date on which payments were suspended.

\section*{Decisions involving issues that arise on appeal in other cases}

16.---(1) This paragraph applies where—
\begin{enumerate}\item[]
($a$) a relevant decision, or a decision under paragraph 3 about the revision of an earlier decision, falls to be made in any particular case; and

($b$) an appeal is pending against the decision given in another case by a Commissioner or a court.
\end{enumerate}

(2) A relevant authority need not make the decision while the appeal is pending if they consider it possible that the result of the appeal will be such that, if it were already determined, there would be no entitlement to benefit.

(3) If a relevant authority consider it possible that the result of the appeal will be such that, if it were already determined, it would affect the decision in some other way—
\begin{enumerate}\item[]
($a$) they need not, except in such cases or circumstances as may be prescribed, make the decision while the appeal is pending;

($b$) they may, in such cases or circumstances as may be prescribed, make the decision on such basis as may be prescribed.
\end{enumerate}

(4) Where—
\begin{enumerate}\item[]
($a$) a relevant authority act in accordance with sub-paragraph (3)($b$), and

($b$) following the making of the determination it is appropriate for their decision to be revised,
\end{enumerate}
they shall then revise their decision (under paragraph 3) in accordance with that determination.

(5) For the purposes of this paragraph, an appeal against a decision is pending if—
\begin{enumerate}\item[]
($a$) an appeal against the decision has been brought but not determined;

($b$) an application for leave to appeal against the decision has been made but not determined; or

($c$) the time within which—
\begin{enumerate}\item[]
(i) an application for leave to appeal may be made, or

(ii) an appeal against the decision may be brought,
\end{enumerate}
has not expired and the circumstances are such as may be prescribed.
\end{enumerate}

(6) In paragraphs ($a$), ($b$)  and ($c$)  of sub-paragraph (5), any reference to an appeal against a decision, or to an application for leave to appeal against a decision, includes a reference to—
\begin{enumerate}\item[]
($a$) an application for judicial review of the decision under section 31 of the Supreme Court Act 1981 or for leave to apply for judicial review; or

($b$) an application to the supervisory jurisdiction of the Court of Session in respect of the decision.
\end{enumerate}

(7) In sub-paragraph (1)($b$)  the reference to another case—
\begin{enumerate}\item[]
($a$) includes a reference to a case involving a decision made, or falling to be made, by a different relevant authority; but

($b$) does not include a reference to a case relating to another benefit unless the other benefit is housing benefit or council tax benefit.
\end{enumerate}

%Appeals involving issues that arise on appeal in other cases
%
%17(1) This paragraph applies where—
%
%($a$) an appeal (“appeal A”) in relation to a relevant decision (whether as originally made or as revised under paragraph 3) is made to an appeal tribunal, or from an appeal tribunal to a Commissioner; and
%
%($b$) an appeal (“appeal B”) is pending against a decision given in a different case by a Commissioner or a court.
%
%(2) If the relevant authority whose decision gave rise to appeal A consider it possible that the result of appeal B will be such that, if it were already determined, it would affect the determination of appeal A, they may serve notice requiring the tribunal or Commissioner—
%
%($a$) not to determine appeal A but to refer it to them; or
%
%($b$) to deal with the appeal in accordance with sub-paragraph (4).
%
%(3) Where appeal A is referred to the authority under sub-paragraph (2)($a$), following the determination of appeal B and in accordance with that determination, they shall if appropriate—
%
%($a$) in a case where appeal A has not been determined by the tribunal, revise (under paragraph 3) their decision which gave rise to that appeal; or
%
%($b$) in a case where appeal A has been determined by the tribunal, make a decision (under paragraph 4) superseding the tribunal’s decision.
%
%(4) Where appeal A is to be dealt with in accordance with this sub-paragraph, the appeal tribunal or Commissioner shall either—
%
%($a$) stay appeal A until appeal B is determined; or
%
%($b$) if the tribunal or Commissioner considers it to be in the interests of the appellant to do so, determine appeal A as if—
%
%(i) appeal B had already been determined; and
%
%(ii) the issues arising on appeal B had been decided in the way that was most unfavourable to the appellant.
%
%(5) Where the appeal tribunal or Commissioner acts in accordance with sub-paragraph (4)($b$), following the determination of appeal B the relevant authority whose decision gave rise to appeal A shall, if appropriate, make a decision (under paragraph 4) superseding the decision of the tribunal or Commissioner in accordance with that determination.
%
%(6) For the purposes of this paragraph, an appeal against a decision is pending if—
%
%($a$) an appeal against the decision has been brought but not determined;
%
%($b$) an application for leave to appeal against the decision has been made but not determined; or
%
%($c$) the time within which—
%
%(i) an application for leave to appeal may be made, or
%
%(ii) an appeal against the decision may be brought,
%
%has not expired and the circumstances are such as may be prescribed.
%
%(7) In this paragraph—
%
%($a$) the reference in sub-paragraph (1)($a$)  to an appeal to a Commissioner includes a reference to an application for leave to appeal to a Commissioner;
%
%($b$) the reference in sub-paragraph (1)($b$)  to a different case—
%
%(i) includes a reference to a case involving a different relevant authority; but
%
%(ii) does not include a reference to a case relating to a different benefit unless the different benefit is housing benefit or council tax benefit; and
%
%($c$) any reference in paragraph ($a$), ($b$)  or ($c$)  of sub-paragraph (6)  to an appeal, or to an application for leave to appeal, against a decision includes a reference to—
%
%(i) an application for judicial review of the decision under section 31 of the [1981 c. 54. ] Supreme Court Act 1981 or for leave to apply for judicial review; or
%
%(ii) an application to the supervisory jurisdiction of the Court of Session in respect of the decision.
%
%(8) In sub-paragraph (4)  “the appellant” means the person who appealed or, as the case may be, first appealed against the decision mentioned in sub-paragraph (1)($a$).
%
%(9) Regulations may make provision supplementing the provision made by this paragraph.

\amendment{
Para. 17 is not yet in force.
}

\section*{Restrictions on entitlement to benefit in certain cases of error}

18.---(1) Subject to sub-paragraph (2), this paragraph applies where—
\begin{enumerate}\item[]
($a$) the effect of the determination, whenever made, of an appeal by virtue of this Schedule to a Commissioner or the court (“the relevant determination”) is that the relevant authority’s decision out of which the appeal arose was erroneous in point of law; and

($b$) after the date of the relevant determination a decision falls to be made by that relevant authority or another relevant authority in accordance with that determination (or would, apart from this paragraph, fall to be so made)—
\begin{enumerate}\item[]
(i) in relation to a claim for housing benefit or council tax benefit;

(ii) as to whether to revise, under paragraph 3, a decision as to a person’s entitlement to such a benefit; or

(iii) on an application made under paragraph 4 for a decision as to a person’s entitlement to such a benefit to be superseded.
\end{enumerate}
\end{enumerate}

(2) This paragraph does not apply where the decision mentioned in sub-paragraph (1)($b$)—
\begin{enumerate}\item[]
($a$) is one which, but for paragraph 16(2)  or (3)($a$), would have been made before the date of the relevant determination%; or
%
%($b$) is one made in pursuance of paragraph 17(3)  or (5)%
.
\end{enumerate}

(3) In so far as the decision relates to a person’s entitlement to benefit in respect of a period before the date of the relevant determination, it shall be made as if the relevant authority’s decision had been found by the Commissioner or court not to have been erroneous in point of law.

(4) Sub-paragraph (1)($a$)  shall be read as including a case where—
\begin{enumerate}\item[]
($a$) the effect of the relevant determination is that part or all of a purported regulation or order is invalid; and

($b$) the error of law made by the relevant authority was to act on the basis that the purported regulation or order (or the part held to be invalid) was valid.
\end{enumerate}

(5) It is immaterial for the purposes of sub-paragraph (1)—
\begin{enumerate}\item[]
($a$) where such a decision as is mentioned in paragraph ($b$)(i)  falls to be made, whether the claim was made before or after the date of the relevant determination;

($b$) where such a decision as is mentioned in paragraph ($b$)(ii)  or (iii)  falls to be made on an application under paragraph 3 or (as the case may be) 4, whether the application was made before or after that date.
\end{enumerate}

(6) In this paragraph “the court” means—
\begin{enumerate}\item[]
($a$) the High Court;

($b$) the Court of Appeal;

($c$) the Court of Session;

($d$) the House of Lords; or

($e$) the Court of Justice of the European Community.
\end{enumerate}

(7) For the purposes of this paragraph, any reference to entitlement to benefit includes a reference to entitlement—
\begin{enumerate}\item[]
($a$) to any increase in the rate of a benefit; or

($b$) to a benefit, or increase of benefit, at a particular rate.
\end{enumerate}

(8) The date of the relevant determination shall, in prescribed cases, be determined for the purposes of this paragraph in accordance with any regulations made for that purpose.

(9) Regulations made under sub-paragraph (8)  may include provision—
\begin{enumerate}\item[]
($a$) for a determination of a higher court to be treated as if it had been made on the date of a determination by a lower court or by a Commissioner; or

($b$) for a determination of a lower court or of a Commissioner to be treated as if it had been made on the date of a determination by a higher court.
\end{enumerate}

\amendment{
Para. 18(2)($b$)  is not yet in force.
}

\section*{Correction of errors and setting aside of decisions}

19.---(1) Regulations may make provision with respect to—
\begin{enumerate}\item[]
($a$) the correction of accidental errors in any decision or record of a decision made under or by virtue of any relevant provision; and

($b$) the setting aside of any such decision in a case where it appears just to set the decision aside on the ground that—
\begin{enumerate}\item[]
(i) a document relating to the proceedings in which the decision was given was not sent to, or was not received at an appropriate time by, a party to the proceedings or a party’s representative, or was not received at an appropriate time by the body or person who gave the decision; or

(ii) a party to the proceedings or a party’s representative was not present at a hearing related to the proceedings.
\end{enumerate}
\end{enumerate}

(2) Nothing in sub-paragraph (1)  shall be construed as derogating from any power to correct errors or set aside decisions which is exercisable apart from regulations made by virtue of that sub-paragraph.

(3) In this paragraph “relevant provision” means—
\begin{enumerate}\item[]
($a$) any of the provisions of this Schedule;

($b$) any of the provisions of Part VII of the Social Security Contributions and Benefits Act 1992 so far as they relate to housing benefit or council tax benefit; or

($c$) any of the provisions of Part VIII of the Administration Act or of any regulations under section 2A of that Act, so far as the provisions or regulations relate to, or to arrangements for, housing benefit or council tax benefit.
\end{enumerate}

\section*{Regulations}

20.---(1) The power to make regulations under this Schedule shall be exercisable—
\begin{enumerate}\item[]
($a$) in the case of regulations with respect to proceedings before the Commissioners, by the Lord Chancellor; and

($b$) in any other case, by the Secretary of State;
\end{enumerate}
and the Lord Chancellor shall consult with the Scottish Ministers before making any regulations under this Schedule that apply to Scotland.

(2) Any power conferred by this Schedule to make regulations shall include power to make different provision for different areas or different relevant authorities.

(3) Subsections (3)  to (7)  of section 79 of the Social Security Act 1998 (supplemental provision in connection with powers to make subordinate legislation under that Act) shall apply to any power to make regulations under this Schedule as they apply to any power to make regulations under that Act.

(4) A statutory instrument containing (whether alone or with other provisions) regulations under paragraph 6(2)($e$)  or (4)  shall not be made unless a draft of the instrument has been laid before Parliament and approved by a resolution of each House.

(5) A statutory instrument—
\begin{enumerate}\item[]
($a$) which contains (whether alone or with other provisions) regulations made under this Schedule, and

($b$) which is not subject to any requirement that a draft of the instrument be laid before and approved by a resolution of each House of Parliament,
\end{enumerate}
shall be subject to annulment in pursuance of a resolution of either House of Parliament.

(6) In this paragraph the reference to regulations with respect to proceedings before the Commissioners includes a reference to regulations with respect to any such proceedings for the determination of any matter, or for leave to appeal to or from the Commissioners.

\section*{\sloppy Consequential amendments of the Administration Act}

21.---(1) In section 5(1)($hh$)  of the Administration Act (regulations about claims for and payments of benefit)—
\begin{enumerate}\item[]
($a$) in sub-paragraph (i), after “1998” there shall be inserted “or, as the case may be, under paragraph 3 of Schedule 7 to the Child Support, Pensions and Social Security Act 2000”; and

($b$) in sub-paragraph (ii), after “Act” there shall be inserted “or, as the case may be, paragraph 4 of that Schedule”.
\end{enumerate}

(2) In section 6(1)  of the Administration Act (regulations about claims for and payments of council tax benefit), after paragraph ($h$)  there shall be inserted—
\begin{quotation}
“($hh$) for requiring such person as may be prescribed in accordance with the regulations to furnish any information or evidence needed for a determination whether a decision on an award of a benefit—
\begin{enumerate}\item[]
(i) should be revised under paragraph 3 of Schedule 7 to the Child Support, Pensions and Social Security Act 2000; or

(ii) should be superseded under paragraph 4 of that Schedule;”.
\end{enumerate}
\end{quotation}

\section*{Consequential amendments of the Social Security Act 1998}

22.---(1) Section 34(4)  and (5)  and section 35 of the Social Security Act 1998 (regulations for the determination of claims and reviews of housing benefit and council tax benefit and for the suspension of those benefits) shall cease to have effect.

(2) In paragraph 4(1)($a$)  of Schedule 1 to that Act (supplementary provisions relating to the appeal tribunals), for “or section 20 of the Child Support Act” there shall be substituted “, section 20 of the Child Support Act or paragraph 6 of Schedule 7 to the Child Support, Pensions and Social Security Act 2000”.

(3) In paragraph 3(1)  of Schedule 4 to that Act (provisions relating to the Social Security Commissioners), after “section 14 of this Act” there shall be inserted “or under paragraph 8 of Schedule 7 to the Child Support, Pensions and Social Security Act 2000”.'.

\amendment{
Para. 22(1) is not yet in force for certain purposes; see the Child Support, Pensions and Social Security Act 2000 (Commencement No. 8) Order 2001 art. 2(3).
}

\section*{Interpretation}

23.---(1) In this Schedule—
\begin{enumerate}\item[]
    “the Administration Act” means the Social Security Administration Act 1992;

    “affected” shall be construed subject to any regulations under sub-\hspace{0pt}paragraph (2);

    “appeal tribunal” means an appeal tribunal constituted under Chapter I of Part I of the Social Security Act 1998;

    “the Chief Commissioner” means the Chief Social Security Commissioner;

    “Commissioner” means the Chief Commissioner or any other Social Security Commissioner, and includes a tribunal of three or more Commissioners constituted under paragraph 10(5);

    “prescribed” means prescribed by regulations under this Schedule;

    “relevant authority” has the meaning given by paragraph 1(1);

    “relevant decision” has the meaning given by paragraph 1(2). 
\end{enumerate}

(2) Regulations may make provision specifying the circumstances in which a person is or is not to be treated for the purposes of this Schedule as a person who is affected by any decision of a relevant authority.

(3) For the purposes of this Schedule any decision that is made or falls to be made—
\begin{enumerate}\item[]
($a$) by a person authorised to carry out any function of a relevant authority relating to housing benefit or council tax benefit, or

($b$) by a person providing services relating to housing benefit or council tax benefit directly or indirectly to a relevant authority,
\end{enumerate}
shall be treated as a decision of the relevant authority on whose behalf the function is carried out or, as the case may be, to whom those services are provided. 

\part[Schedule 8 --- Declarations of status: consequential amendments]{Schedule 8\\*Declarations of status: consequential amendments}

\renewcommand\parthead{--- Schedule 8}

\section*{\itshape The Births and Deaths Registration Act 1953}

1. In section 14A(1)($a$)  of the Births and Deaths Registration Act 1953 (re-registration of birth where notification of declaration of parentage given under section 56(4)  of the Family Law Act 1986), for “56(4)” there shall be substituted “55A(7)  or 56(4)”.

\section*{\itshape The Magistrates' Courts Act 1980}

2.---(1) Section 65 of the Magistrates' Courts Act 1980 (meaning of family proceedings) shall be amended as follows.

(2) In subsection (1)  (proceedings which are family proceedings), after paragraph ($m$) there shall be inserted—
\begin{quotation}
“($mm$) section 55A of the Family Law Act 1986;”.
\end{quotation}

(3) In subsection (2)  (power of court to treat combined proceedings as family proceedings), in paragraph ($e$), before “section 20” there shall be inserted “proceedings under”.

\section*{\itshape The Family Law Act 1986}

3. The Family Law Act 1986 shall be amended as follows.

\medskip

4. In section 55 (declarations as to marital status)—
\begin{enumerate}\item[]
($a$) in subsection (1), for “the court” there shall be substituted “the High Court or a county court”, and

($b$) in subsection (3), after “made” there shall be inserted “to a court”.
\end{enumerate}

\medskip

5. In section 56 (declarations as to legitimacy or legitimation)—
\begin{enumerate}\item[]
($a$) in subsections (1)  and (2), for “the court” there shall be substituted “the High Court or a county court”, and

($b$) in subsection (4), after “made” there shall be inserted “by a court”.
\end{enumerate}

\medskip

6. In section 57(1)  (application to the court for declaration as to overseas adoption), for “the court” there shall be substituted “the High Court or a county court”.

\medskip

7. In section 58 (general provisions)—
\begin{enumerate}\item[]
($a$) in subsection (1), after “application” there shall be inserted “to a court”, and

($b$) in subsection (3), for “The” there shall be substituted “A”.
\end{enumerate}

\medskip

8. In section 59 (provisions relating to the Attorney-General)—
\begin{enumerate}\item[]
($a$) in subsections (1)  and (2), after “an application” there shall be inserted “to a court”, and

($b$) in subsection (3), after “any application” there shall be inserted “to a court”.
\end{enumerate}

\section*{\itshape The Family Law Reform Act 1987}

9. In section 23(1)  of the Family Law Reform Act 1987—
\begin{enumerate}\item[]
($a$) in subsection (2)  to be substituted for section 20(2)  of the Family Law Reform Act 1969 (report to court about scientific tests), for “person responsible for” there shall be substituted “individual”; and

($b$) in subsection (2A)  to be inserted in section 20 of that Act (blood tests in proceedings under section 56 of the Family Law Act 1986), for “56” there shall be substituted “55A or 56”.
\end{enumerate}

\section*{\itshape The Children Act 1989}

10.---(1) Part I of Schedule 11 to the Children Act 1989 (jurisdiction) shall be amended as follows.

(2) In paragraph 1(2A)  (additional proceedings which may be required to be commenced in a particular court)—
\begin{enumerate}\item[]
($a$) for paragraph ($a$)  there shall be substituted—
\begin{quotation}
“($a$) under section 55A of the Family Law Act 1986 (declarations of parentage); or”, and
\end{quotation}

($b$) in paragraph ($b$), for “of that Act” there shall be substituted “of the Child Support Act 1991”.
\end{enumerate}

(3) In paragraph 2(3)  (power to transfer certain proceedings)—
\begin{enumerate}\item[]
($a$) after paragraph ($b$)  there shall be inserted—
\begin{quotation}
“($ba$) any proceedings under section 55A of the Family Law Act 1986”, and
\end{quotation}

($b$) in paragraph ($bb$), before “section 20” there shall be inserted “any proceedings under”.
\end{enumerate}

\section*{\itshape The Child Support Act 1991}

11. The Child Support Act 1991 shall be amended as follows.

\medskip

12. In section 26(2)  (cases where Secretary of State may make maintenance calculation despite denial of parentage), in Case C (where there has been a declaration under section 56 of the Family Law Act 1986), after “section” there shall be inserted “55A or”.

\medskip

13. For section 27 (declarations of parentage) there shall be substituted—
\begin{quotation}
\subsection*{“27. Applications for declaration of parentage under Family Law Act 1986}

(1) This section applies where—
\begin{enumerate}\item[]
($a$) an application for a maintenance calculation has been made (or is treated as having been made), or a maintenance calculation is in force, with respect to a person (“the alleged parent”) who denies that he is a parent of a child with respect to whom the application or calculation was made or treated as made;

($b$) the Secretary of State is not satisfied that the case falls within one of those set out in section 26(2); and

($c$) the Secretary of State or the person with care makes an application for a declaration under section 55A of the Family Law Act 1986 as to whether or not the alleged parent is one of the child’s parents.
\end{enumerate}

(2) Where this section applies—
\begin{enumerate}\item[]
($a$) if it is the person with care who makes the application, she shall be treated as having a sufficient personal interest for the purposes of subsection (3)  of that section; and

($b$) if it is the Secretary of State who makes the application, that subsection shall not apply.
\end{enumerate}

(3) This section does not apply to Scotland.”
\end{quotation}

\medskip

14. In section 27A(2)($b$)  (Secretary of State to recover fees for scientific tests if a court has made a declaration of parentage under section 27), for “section 27” there shall be substituted “section 55A of the Family Law Act 1986”.

\section*{\itshape The Access to Justice Act 1999}

15. In Schedule 2 to the Access to Justice Act 1999 (services which are not to be funded as part of community legal services), in paragraph 2(3), after paragraph ($d$)  there shall be inserted—
\begin{quotation}
“($da$) under section 55A of the Family Law Act 1986 (declarations of parentage),”.
\end{quotation}

\part[Schedule 9 --- Repeals and revocations]{Schedule 9\\*Repeals and revocations}

\renewcommand\parthead{--- Schedule 9}

\section[Part I --- Child support]{Part I\\*Child support}

\renewcommand\parthead{--- Schedule 9 Part I}

{\footnotesize
%\begin{tabulary}{\linewidth}{JJJ}
\begin{longtable}{p{50pt}p{83.27403pt}p{220.72266pt}}
\hline
\itshape Chapter	&\itshape Short title	&\itshape Extent of repeal\\
\hline
\endhead
\hline
\endlastfoot
10 \& 11 Geo.\ 6 c.\ 24. 	&The Naval Forces (Enforcement of Maintenance Liabilities) Act 1947. 	&In section 1(1), paragraph ($aaa$).\\
3 \& 4 Eliz.\ 2 c.\ 18. 	&The Army Act 1955. 	&In section 150A, in subsection (2), paragraph ($b$)  and the word “or” preceding it, and in subsection (3), the words “or cancels” and “or (as the case may be) that it has been cancelled”.\\
3 \& 4 Eliz.\ 2 c.\ 19. 	&The Air Force Act 1955. 	&In section 150A, in subsection (2), paragraph ($b$)  and the word “or” preceding it, and in subsection (3), the words “or cancels” and “or (as the case may be) that it has been cancelled”.\\
1973 c.\ 18. 	&The Matrimonial Causes Act 1973. 	&In section 29, in subsection (7), the words “or is cancelled”, “or was cancelled” and “or, as the case may be, the date with effect from which it was cancelled”; and in subsection (8), paragraph ($b$)  and the word “and” preceding it.\\
1978 c.\ 22. 	&The Domestic Proceedings and Magistrates' Courts Act 1978. 	&In section 5, in subsection (7), the words “or is cancelled”, “or was cancelled” and “or, as the case may be, the date with effect from which it was cancelled”; and in subsection (8), paragraph ($b$)  and the word “and” preceding it.\\
1989 c.\ 41. 	&The Children Act 1989. 	&In Schedule 1, in paragraph 3(7), the words “or is cancelled”, “or was cancelled” and “or, as the case may be, the date with effect from which it was cancelled”; and in paragraph 3(8), paragraph ($b$)  and the word “and” preceding it.\\
1991 c.\ 48. 	&The Child Support Act 1991. 	&In section 15(10), the definition of “specified” and the preceding word “and”.\\
		&&In section 17(1), the word “and” after paragraph ($b$).\\
		&&In section 28D(2)($a$), “lapsed or”.\\
		&&Sections 28H and 28I.\\
		&&Section 40(1)  and (2).\\
		&&Section 41(3)  to (5).\\
		&&Section 44(3).\\
		&&Section 46B(3).\\
		&&In section 54, the definitions of “assessable income”, “current assessment”, “departure direction” and “maintenance requirement”.\\
		&&In Schedule 1, paragraph 13, and in paragraph 16, sub-paragraph (1)($d$)  and ($e$), sub-paragraphs (2)  to (9), and in sub-paragraph (10)  the words “, or should be cancelled”.\\
		&&Schedule 4C.\\
1992 c.\ 5. 	&The Social Security Administration Act 1992. 	&In section 170(5), in the definition of “the relevant enactments”, paragraph ($ab$).\\
1992 c.\ 6. 	&The Social Security (Consequential Provisions) Act 1992. 	&In Schedule 2, paragraph 113. \\
1995 c.\ 18. 	&The Jobseekers Act 1995. 	&In Schedule 2, paragraph 20(2), (4)  and (7).\\
1995 c. 34. 	&The Child Support Act 1995. 	&Sections 1, 2 and 3. \newline
%		&&
Sections 6, 7, 8, 9%
, 10 
and 11. \\
		&&Section 14(2)  and (3).\\
		&&Section 18(3)  and (5).\\
		&&Section 19. \\
		&&Section 22. \\
		&&Section 24. \\
		&&Section 26(4)($c$).\\
		&&Schedules 1 and 2. \\
		&&In Schedule 3, paragraphs 12, 15 and 20($a$).\\
1998 c.\ 14. 	&The Social Security Act 1998. 	&Section 42. \newline
%		&&
In Schedule 7, paragraphs 20, 24, 25, 
28, 34, and 35; in paragraph 36, the words “(1)  and”; and paragraphs 37, 38, 39, 40, 43, 46, 48(1), (2), (3)  and (5)($a$), ($b$)  and ($c$), 53 and 54%
.\\ 
S.I.\ 1998/\hspace{0pt}2780 (C.66).	&The Social Security Act 1998 (Commencement No.\ 2) Order 1998. 	&Article 3(4).\\
1999 c.\ 10. 	&The Tax Credits Act 1999. 	&In Schedule 1, paragraph 6($i$).\newline
%		&&
In Schedule 2, paragraph 17($a$).\\
%\end{tabulary}
\end{longtable}

}


\amendment{
Most of Sch. 9 Pt. I is in force only for new-rules cases; see the Child Support, Pensions and Social Security Act 2000 (Commencement No. 12) Order 2003 art. 3, 6.  The repeals in s. 15(10), 40(1), (2) of the Child Support Act 1991, s. 24 of the Child Support Act 1995, Sch. 7 para. 28 of the Social Security Act 1998 and in the Social Security Act 1998 (Commencement No. 2) Order 1998 apply to both old- and new-rules cases.
}

\section[Part II --- State pensions]{Part II\\*State pensions}

\renewcommand\parthead{--- Schedule 9 Part II}

{\footnotesize
\begin{tabulary}{\linewidth}{JJJ}
%\begin{longtable}{p{50pt}p{66.85179pt}p{237.15175pt}}
\hline
\itshape Chapter	&\itshape Short title	&\itshape Extent of repeal\\
\hline
1999 c. 30. 	&The Welfare Reform and Pensions Act 1999. 	&In Schedule 8, paragraph 5($b$)  and the word “and” immediately preceding it.\\
\hline
\end{tabulary}

}

\section[Part III --- Occupational and personal pension schemes]{Part III\\*Occupational and personal pension schemes}

\renewcommand\parthead{--- Schedule 9 Part III}

\amendment{
Sch. 9 Pt. III s. (1) is not yet in force.
}

%(1) 
%Member-nominated trustees and directors
%Chapter	Short title	Extent of repeal
%1995 c. 26. 	The Pensions Act 1995. 	In section 16(1), the words “(subject to section 17)” and in paragraph ($b$), the words “, and the appropriate rules,”.
%		Section 17. 
%		In section 18(1), the words “, subject to section 19,” and in paragraph ($b$), the words “, and the appropriate rules,”.
%		Sections 19 and 20. 
%		
%
%In section 21—
%($a$) 
%
%in subsections (1)  and (2), the words “, or the appropriate rules,”;
%($b$) 
%
%in subsection (3), the words “or rules”;
%($c$) 
%
%in subsection (4), the words “(or further arrangements)” in paragraph ($a$), and paragraph ($b$)  and the word “and” immediately preceding it;
%($d$) 
%
%subsection (5);
%($e$) 
%
%in subsection (7), the words “and this section”, paragraph ($b$)  and the word “and” immediately preceding paragraph ($b$); and
%($f$) 
%
%in subsection (8), paragraph ($b$)  and the word “and” immediately preceding it.
%1999 c. 30. 	The Welfare Reform and Pensions Act 1999. 	In Schedule 12, paragraphs 46 and 48 and in paragraph 49, sub-paragraph ($b$)  and the word “and” immediately preceding it.

\subsection*{(2) 
Information to be given to the authority}

{\footnotesize
\begin{tabulary}{\linewidth}{JJJ}
%\begin{longtable}{p{50pt}p{66.85179pt}p{237.15175pt}}
\hline
\itshape Chapter	&\itshape Short title	&\itshape Extent of repeal\\
\hline
1993 c. 48. 	&The Pension Schemes Act 1993. 	&In section 178($a$), the words “sections 22 to 26 of the Pensions Act 1995”.\\
1995 c. 26. 	&The Pensions Act 1995. 	&In Schedule 3, paragraph 43.\\ 
\hline
\end{tabulary}

}

\subsection*{(3) 
The Pensions Ombudsman}

{\footnotesize
\begin{tabulary}{\linewidth}{JJJ}
%\begin{longtable}{p{50pt}p{66.85179pt}p{237.15175pt}}
\hline
\itshape Chapter	&\itshape Short title	&\itshape Extent of repeal\\
\hline
1993 c. 48. 	&The Pension Schemes Act 1993. 	
&In section 146—\newline\hspace*{1em}($a$) 
in subsection (1)($c$), the words “which arises” and the \hspace*{1em}words from “and which” to “beneficiary, and”;\\&&\hspace*{1em}($b$) 
in subsection (1)($d$), the words “which arises”; and\\&&\hspace*{1em}($c$) 
subsection (3A).\\
%1995 c. 26. 	&The Pensions Act 1995. 	&Section 157(7).\\
\hline
\end{tabulary}

}

\amendment{
The repeal in this section relating to the Pensions Act 1995 is not yet in force.
}

\subsection*{(4) 
Guaranteed minimum for widows and widowers}

{\footnotesize
%\begin{tabulary}{\linewidth}{JJJ}
\begin{longtable}{p{45.96pt}p{55.78545pt}p{252.25719pt}}
\hline
\itshape Chapter	&\itshape Short title	&\itshape Extent of repeal\\
\hline
\endhead
\hline
\endlastfoot
1993 c. 48. 	&The Pension Schemes Act 1993. 	&In section 17(5), the words “Category B retirement pension,”, in the first place where they occur, and the words from “or for which” onwards.\\
%\hline
%\end{tabulary}
\end{longtable}

}

\subsection*{(5) 
Protected rights}

{\footnotesize
%\begin{tabulary}{\linewidth}{JJJ}
\begin{longtable}{lll}
\hline
\itshape Chapter	&\itshape Short title	&\itshape Extent of repeal\\
\hline
\endhead
\hline
\endlastfoot
1995 c. 26. 	&The Pensions Act 1995. 	&In Schedule 5, paragraph 34($a$).\\
%\hline
%\end{tabulary}
\end{longtable}

}

\enlargethispage{\baselineskip}

\subsection*{(6) 
Contributions equivalent premiums}

{\footnotesize
%\begin{tabulary}{\linewidth}{JJJ}
\begin{longtable}{p{81.02374pt}p{155.22356pt}p{117.7467pt}}
\hline
\itshape Chapter or number	&\itshape Short title	&\itshape Extent of repeal or revocation\\
\hline
\endhead
\hline
\endlastfoot
1995 c. 26. 	&The Pensions Act 1995. 	&In Schedule 5, paragraph 57($a$)(ii).\\
S.I. 1995/3213 (N.I. 22).	&The Pensions (Northern Ireland) Order 1995. 	&In Schedule 3, paragraph 49($a$)(ii).\\
%\hline
%\end{tabulary}
\end{longtable}

}

\enlargethispage{-\baselineskip}

\subsection*{(7) 
Use of cash equivalent}

{\footnotesize
%\begin{tabulary}{\linewidth}{JJJ}
\begin{longtable}{lll}
\hline
\itshape Chapter	&\itshape Short title	&\itshape Extent of repeal\\
\hline
\endhead
\hline
\endlastfoot
1993 c. 48. 	&The Pension Schemes Act 1993. 	&Section 95(4).\\
%\hline
%\end{tabulary}
\end{longtable}

}

\enlargethispage{-\baselineskip}

\subsection*{(8) 
Transfer values}

{\footnotesize
%\begin{tabulary}{\linewidth}{JJJ}
\begin{longtable}{lll}
\hline
\itshape Chapter	&\itshape Short title	&\itshape Extent of repeal\\
\hline
\endhead
\hline
\endlastfoot
1993  c. 48. 	&The Pension Schemes Act 1993. 	&Section 98(7)($a$).\\
%\hline
%\end{tabulary}
\end{longtable}

}

Sub-paragraph (4)  of paragraph 8 of Schedule 5 to this Act has effect in relation to this repeal as it has effect in relation to sub-paragraph (2)  of that paragraph.

\subsection*{(9) 
Information about contracting-out}

{\footnotesize
%\begin{tabulary}{\linewidth}{JJJ}
\begin{longtable}{p{40.96pt}p{223.44916pt}p{89.59044pt}}
\hline
\itshape Chapter	&\itshape Short title	&\itshape Extent of repeal\\
\hline
\endhead
\hline
\endlastfoot
1999 c. 2. 	&The Social Security Contributions (Transfer of Functions, etc.)\ Act 1999. 	&In Schedule 1, paragraph 60. \\
%\hline
%\end{tabulary}
\end{longtable}

}

\amendment{
Sch. 9 Pt. III s. (10) is not yet in force.
}

%(10) 
%Duties relating to statements of contributions
%Chapter	Short title	Extent of repeal
%1995 c. 26. 	The Pensions Act 1995. 	In section 49(10), the word “and” at the end of paragraph ($a$).

\subsection*{(11) 
Spent provisions}

{\footnotesize\centering
\begin{tabulary}{\linewidth}{JJJ}
%\begin{longtable}{p{50pt}p{66.85179pt}p{237.15175pt}}
\hline
\itshape Chapter	&\itshape Short title	&\itshape Extent of repeal\\
\hline
1993 c. 48. 	&The Pension Schemes Act 1993. 	&Section 56(5).\\
1993 c. 49. 	&The Pension Schemes (Northern Ireland) Act 1993. 	&Section 52(5).\\
\hline
\end{tabulary}

}

\section[Part IV --- War pensions]{Part IV\\*War pensions}

\renewcommand\parthead{--- Schedule 9 Part IV}

{\footnotesize
%\begin{tabulary}{\linewidth}{JJJ}
\begin{longtable}{p{50pt}p{66.85179pt}p{237.15175pt}}
\hline
\itshape Chapter	&\itshape Short title	&\itshape Extent of repeal\\
\hline
\endhead
\hline
\endlastfoot
6 \& 7 Geo.\ 6.\  c.\ 39. 	& The Pensions Appeal Tribunals Act 1943. &	In section 8, in subsection (1), the words from “Provided” to the end, subsection (2)  and, in subsection (3), the words from “Provided” to the end.\\
12, 13 \& 14 Geo.\ 6.\  c.\ 12. 	&The Pensions Appeal Tribunals Act 1949. 	&Section 1(2).\newline Section 2. \\
1990 c. 41. 	&The Courts and Legal Services Act 1990. &	In Schedule 10, paragraph 5. \\
1995 c. 26. 	&The Pensions Act 1995. 	&Section 169(6).\\
%\end{tabulary}
\end{longtable}

}

%Part VLoss of benefit
%Chapter	Short title	Extent of repeal
%1998 c. 14. 	The Social Security Act 1998. 	In Schedule 3, in paragraph 3, the word “or” at the end of sub-paragraph ($c$).

\amendment{
Sch. 9 Pt. V is not yet in force.
}

\section[Part VI --- Investigation powers]{Part VI\\*Investigation powers}

\renewcommand\parthead{--- Schedule 9 Part VI}

{\footnotesize
%\begin{tabulary}{\linewidth}{JJJ}
\begin{longtable}{p{43.5645pt}p{185.35036pt}p{125.09244pt}}
\hline
\itshape Chapter	&\itshape Short title	&\itshape Extent of repeal\\
\hline
\endhead
\hline
\endlastfoot
1992 c. 5. 	&The Social Security Administration Act 1992. 	&Section 111A(2).\\
		&&Section 112(3).\\
1993 c. 48. 	&The Pension Schemes Act 1993. 	&In Schedule 8, paragraph 26. \\
1995 c. 18. 	&The Jobseekers Act 1995. 	&Section 33.\\ 
		&&Section 34(2), (3)  and (5)  to (7).\\
		&&In Schedule 2, paragraph 54.\\ 
1995 c. 26. 	&The Pensions Act 1995. 	&In Schedule 5, paragraph 15(2).\\
1997 c. 27. 	&The Social Security (Recovery of Benefits) Act 1997. 	&In Schedule 3, paragraph 4. \\
1997 c. 47. 	&The Social Security Administration (Fraud) Act 1997. 	&Section 12. \newline In Schedule 1, paragraph 4(4).\\
1999 c. 2. 	&The Social Security Contributions (Transfer of Functions, etc.)\ Act 1999. 	&In Schedule 5, paragraph 2. \\
1999 c. 10. 	&The Tax Credits Act 1999. 	&In Schedule 2, paragraphs 11($a$), 13($a$)  and 14($a$).\\
1999 c. 30. 	&The Welfare Reform and Pensions Act 1999. 	&In Schedule 8, paragraph 34(2)($a$).\\
%\hline
%\end{tabulary}
\end{longtable}

}

\section[Part VII --- Housing Benefit and Council Tax Benefit]{Part VII\\*Housing Benefit and Council Tax Benefit}

\renewcommand\parthead{--- Schedule 9 Part VII}

{\footnotesize\centering
\begin{tabulary}{\linewidth}{JJJ}
%\begin{longtable}{p{45.96pt}p{134.92946pt}p{173.11526pt}}
\hline
\itshape Chapter	&\itshape Short title	&\itshape Extent of repeal\\
\hline
%\endhead
%\hline
%\endlastfoot
1998 c. 14. 	&The Social Security Act 1998. 	&In section 34, subsections (4)  and (5).\\
&&		Section 35.\\ 
\hline
\end{tabulary}
%\end{longtable}

}


\amendment{
Sch. 9 Pt. VII is not yet in force for certain purposes; see the Child Support, Pensions and Social Security Act 2000 (Commencement No. 8) Order 2001 art. 2(3).
}

\section[Part VIII --- NICs in respect of benefits in kind]{Part VIII\\*NICs in respect of benefits in kind}

\renewcommand\parthead{--- Schedule 9 Part VIII}

\subsection*{(1) 
Great Britain}

{\footnotesize
%\begin{tabulary}{\linewidth}{JJJ}
\begin{longtable}{p{45.96pt}p{134.92946pt}p{173.11526pt}}
\hline
\itshape Chapter	&\itshape Short title	&\itshape Extent of repeal\\
\hline
\endhead
\hline
\endlastfoot
1992 c. 4. &	The Social Security Contributions and Benefits Act 1992. 	&In section 1(2)($b$), the words “in respect of cars made available for private use and car fuel”.\\
&		&In Schedule 1, paragraphs 3(2)  and 8(1)($i$).\\
1998 c. 14. 	&The Social Security Act 1998. 	&Section 50(2).\\
&&		Section 52. \\
&&		In Schedule 7, paragraph 58. \\
1999 c. 2. 	&The Social Security Contributions (Transfer of Functions, etc.)\ Act 1999. 	&In section 8(1), paragraph ($j$), and in paragraph ($l$), the words “amount of interest or”.\\
&&		In Schedule 1, paragraph 19(2).\\
	&&	In Schedule 3, paragraph 10. \\
%\end{tabulary}
\end{longtable}

}

1. 
These repeals (except the repeals in section 8(1)  of the Social Security Contributions (Transfer of Functions, etc.)\ Act 1999) have effect in relation to the tax year beginning with 6th April 2000 and subsequent tax years.

2. 
The repeals in section 8(1)  of the Social Security Contributions (Transfer of Functions, etc.)\ Act 1999 have effect in accordance with section 76(7)  of this Act.

\pagebreak[3]

\subsection*{
(2) 
Northern Ireland}

{\footnotesize
%\begin{tabulary}{\linewidth}{JJJ}
\begin{longtable}{p{50pt}p{150.9199pt}p{153.07274pt}}
\hline
\itshape Chapter	&\itshape Short title	&\itshape Extent of repeal\\
\hline
\endhead
\hline
\endlastfoot
1992 c. 7. 	&The Social Security Contributions and Benefits (Northern Ireland) Act 1992. 	&In section 1(2)($b$), the words “in respect of cars made available for private use and car fuel”.\\
&&		In Schedule 1, paragraphs 3(2)  and 8(1)($i$).\\
S.I. 1998/\hspace{0pt}1506 (N.I. 10).	&The Social Security (Northern Ireland) Order 1998. 	&Article 47(2).\newline Article 49. \newline In Schedule 6, paragraph 40. \\
S.I. 1999/\hspace{0pt}671. 	&The Social Security Contributions (Transfer of Functions, etc.)\ (Northern Ireland) Order 1999. 	&In Article 7(1), sub-paragraph ($j$), and in sub-paragraph ($l$), the words “amount of interest or”.\\
&&		In Schedule 1, paragraph 22(2).\\
&&		In Schedule 3, paragraph 11. \\
%\end{tabulary}
\end{longtable}

}

1. 
These repeals (except the repeals in Article 7(1)  of the Social Security Contributions (Transfer of Functions, etc.)\ (Northern Ireland) Order 1999) have effect in relation to the tax year beginning with 6th April 2000 and subsequent tax years.

2. 
The repeals in Article 7(1)  of the Social Security Contributions (Transfer of Functions, etc.)\ (Northern Ireland) Order 1999 have effect in accordance with section 80(7)  of this Act.

\section[Part IX --- Tests for determining parentage and declarations of status]{Part IX\\*Tests for determining parentage and declarations of status}

{\footnotesize
%\begin{tabulary}{\linewidth}{JJJ}
\begin{longtable}{p{43.5645pt}p{132.50633pt}p{177.92844pt}}
\hline
\itshape Chapter	&\itshape Short title	&\itshape Extent of repeal\\
\hline
\endhead
\hline
\endlastfoot
1968 c. 63. 	&The Domestic and Appellate Proceedings (Restriction of Publicity) Act 1968. 	&In section 2, subsection (1)($e$)  and, in subsection (3), the words “or ($e$)”.\\
1968 c. 64. 	&The Civil Evidence Act 1968. 	&In section 12(5), in the definition of “relevant proceedings”, paragraph ($d$).\\
1980 c. 43. 	&The Magistrates' Courts Act 1980. 	&In section 65(1)($o$)  and (2)($e$), the words “or section 27”.\\
1986 c. 55. 	&The Family Law Act 1986. 	&Section 56(1)($a$).\newline
		Section 58(5)($b$).\newline
		Section 63. \\
1987 c. 42. 	&The Family Law Reform Act 1987. 	&In paragraph 19($a$)  of Schedule 2, the words from “and there” to the end.\\
1989 c. 41. 	&The Children Act 1989. 	&Section 89. \newline
		In Schedule 11, in paragraphs 1(3)($bb$)  and 2(3)($bb$), the words from “or 27” to “parentage)”.\\
1990 c. 41. 	&The Courts and Legal Services Act 1990. 	&In Schedule 16, paragraph 3. \\
1991 c. 48. 	&The Child Support Act 1991. 	&In section 26(2), Case D.\\
1995 c. 34. 	&The Child Support Act 1995. 	&In section 20, subsections (1)  to (4).\\
1998 c. 14. 	&The Social Security Act 1998. 	&In Schedule 7, paragraph 32. \\
1999 c. 22. 	&The Access to Justice Act 1999. 	&In Schedule 2, in paragraph 2(3)($g$), the words “or 27”.\\
%\end{tabulary}
\end{longtable}

}

\end{document}