\documentclass[12pt,a4paper]{article}

\newcommand\regstitle{The Child Support (Miscellaneous Amendments) (No.~2) Regulations 2008}

\newcommand\regsnumber{2008/2544}

%\opt{newrules}{
\title{\regstitle}
%}

%\opt{2012rules}{
%\title{Child Maintenance and~Other Payments Act 2008\\(2012 scheme version)}
%}

\author{S.I.\ 2008 No.\ 2544}

\date{Made
26th September 2008\\
Laid before Parliament
1st October 2008\\
Coming into force
27th October 2008
}

%\opt{oldrules}{\newcommand\versionyear{1993}}
%\opt{newrules}{\newcommand\versionyear{2003}}
%\opt{2012rules}{\newcommand\versionyear{2012}}

\usepackage{csa-regs}

\setlength\headheight{42.11603pt}

%\hbadness=10000

\begin{document}

\maketitle

\noindent
The Secretary of State for~Work and~Pensions, in exercise of the powers conferred by sections~29(2), (4), (6) and~(7), 32(2)($bb$), 51(1) and~(2)($b$),~52(4) and~54 of, and~paragraphs~10(1) and~(2) and~11 of Schedule~1 to the Child Support Act 1991\footnote{1991 c.~48. Subsections (4) to (7) of section~29 are inserted into that section~by section~20 of the Child Maintenance and Other Payments Act 2008 (c.~6). Section 32(2)($bb$) is inserted into section~32 by paragraph~11(16) of Schedule~3 to the Child Support, Pensions and Social Security Act 2000 (c.~19). Section~54 is cited for~the meaning given to the word “prescribed”.} and~section~29 of the Child Support, Pensions and~Social Security Act 2000\footnote{2000 c.~19.} makes the following Regulations: 

{\sloppy

\tableofcontents

}

\bigskip

\setcounter{secnumdepth}{-2}

\subsection[1. Citation, commencement and~interpretation]{Citation, commencement and~interpretation}

1.---(1)  These Regulations may be cited as the Child Support (Miscellaneous Amendments) (No.~2) Regulations 2008 and~shall come into force on 27th October 2008.

(2) In these Regulations—
\begin{enumerate}\item[]
“the Act” means the Child Support Act 1991;

“the Collection and~Enforcement Regulations” means the Child Support (Collection and~Enforcement) Regulations 1992\footnote{S.I.~1992/1989.};

“the Maintenance Assessment Procedure Regulations” means the Child Support (Maintenance Assessment Procedure) Regulations 1992\footnote{S.I.~1992/1813.};

“the Decisions and~Appeals Regulations” means the Social Security and~Child Support (Decisions and~Appeals) Regulations 1999\footnote{S.I.~1999/991.};

“the Maintenance Calculation Procedure Regulations” means the Child Support (Maintenance Calculation Procedure) Regulations 2000\footnote{S.I.~2001/157.};

“the Maintenance Calculations and~Special Cases Regulations” means the Child Support (Maintenance Calculations and~Special Cases) Regulations 2000\footnote{S.I.~2001/155.}; and

“the Transitional Provisions Regulations” means the Child Support (Transitional Provisions) Regulations 2000\footnote{S.I.~2000/3186.}.
\end{enumerate}

\subsection[2. Amendment of the Collection and~Enforcement Regulations]{Amendment of the Collection and~Enforcement Regulations}

2.---(1)  The Collection and~Enforcement Regulations are amended in accordance with the following paragraphs.

(2) In regulation~3 (method of payment)—
\begin{enumerate}\item[]
($a$) in paragraph~(1)—
\begin{enumerate}\item[]
(i) omit sub-paragraph~($h$);

(ii) after sub-paragraph~($h$), add—
\begin{quotation}
“($i$) by deduction from earnings order.”;
\end{quotation}
\end{enumerate}

($b$) in paragraph~(1A)\footnote{Paragraph (1A) was inserted into regulation~3 by regulation~2 of S.I.~2001/162 and substituted by regulation~3 of S.I.~2006/1520.} omit sub-paragraph~($c$);

($c$) after paragraph~(2), add—
\begin{quotation}
“(3) Where the Secretary of State is considering specifying a deduction from earnings order by virtue of paragraph~(1)($i$), that method of payment is not to be used in any case where there is good reason not to use it.

(4) For~the purposes of paragraph~(3) the matters which are to be taken into account in determining whether there is good reason not to use that method of payment are whether the making of a deduction from earnings order is likely to result in the disclosure of the parentage of a child and~the impact of that disclosure on—
\begin{enumerate}\item[]
($a$) the liable person’s employment;

($b$) any relationship between the liable person and~a third party.
\end{enumerate}

(5) For~the purposes of paragraph~(3) the circumstances in which good reason not to use that method of payment is to be regarded as existing are—
\begin{enumerate}\item[]
($a$) a member of the liable person’s or~parent with care’s family is employed by the same relevant employer as the liable person;

($b$) that family member’s employment requires knowledge of the relevant employer’s functions in giving effect to the deduction from earnings order; and

($c$) as a consequence of these circumstances the liable person’s employment status or~family relationships may be adversely affected by the use of a deduction from earnings order as a method of payment.
\end{enumerate}

(6) For~the purposes of paragraph~(3) the matters which are not to be taken into account in determining whether there is good reason not to use that method of payment are—
\begin{enumerate}\item[]
($a$) the liable person’s preference for~a different method of payment;

($b$) the liable person’s preference for~a relevant employer not to be informed about that parent’s maintenance liability;

($c$) that a third party would become aware of the liable person’s maintenance liability,
\end{enumerate}
unless they are relevant to any matter falling within paragraph~(4) or~circumstance falling within paragraph~(5).

(7) Where the Secretary of State is considering specifying the method of payment set out in paragraph~(1)($i$)  and~decides that there is no good reason not to use it, that method is not to be specified until—
\begin{enumerate}\item[]
($a$) the time within which an appeal against that decision may ordinarily be brought (including any period during which a further appeal may ordinarily be brought) has ended; or

($b$) if an appeal is brought on the grounds set out in regulation~22(3A), the time at which proceedings on the appeal (including any proceedings on a further appeal) have been concluded.
\end{enumerate}

(8) Nothing in this regulation~is to prevent the Secretary of State exercising his powers under section~31 of the Act to make a deduction from earnings order where the Secretary of State considers it is appropriate in the circumstances of the case, unless he has specified a deduction from earnings order as a method of payment by virtue of paragraph~(1)($i$).

(9) In this regulation—
\begin{enumerate}\item[]
“couple” means—
\begin{enumerate}\item[]
($a$) 
a man and~woman who are married to each other and~are members of the same household;

($b$) 
a man and~woman who are not married to each other but are living together as husband~and~wife;

($c$) 
two people of the same sex who are civil partners of each other and~are members of the same household; or

($d$) 
two people of the same sex who are not civil partners of each other but are living together as if they were civil partners,
\end{enumerate}
and~for~the purposes of paragraph~($d$), two people of the same sex are to be regarded as living together as if they were civil partners if, but only if, they would be regarded as living together as husband~and~wife were they instead two people of the opposite sex;

“family” means partner, parent, parent-in-law, son, son-in-law, daughter, daughter-in-law, step-parent, step-son, step-daughter, brother, sister, grand-parent, grand-child, uncle, aunt, nephew, niece, or~if any of the preceding persons is one member of a couple, the other member of that couple;

“partner” means where a person is a member of a couple the other member of that couple; and

“relevant employer” means the employer of a liable person in respect of whom the order under section~31 of the Act would be made but for~paragraph~(3).”.
\end{enumerate}
\end{quotation}
\end{enumerate}

(3) In regulation~11 (protected earnings rate)\footnote{The relevant amending instruments are S.I.~1995/1045, S.I.~1996/1945 and S.I.~1999/1510.}—
\begin{enumerate}\item[]
($a$) in paragraph~(2), for~“paragraph~(3) or~paragraph~(4)” substitute “paragraph~(3), paragraph~(4) or~paragraph~(5)”;

($b$) after paragraph~(4), add—
\begin{quotation}
“(5) This paragraph~applies where the liable person—
\begin{enumerate}\item[]
($a$) has more than one employer; and

($b$) the Secretary of State makes an order under section~31 of the Act (“an order”) against that person in respect of more than one employer.
\end{enumerate}

(6) Where paragraph~(5) applies, the protected earnings rate for~each order is to be divided proportionately between the earnings of the liable person with each employer in accordance with paragraph~(7).

(7) The amount to be specified as the protected earnings rate in respect of any period in an order is an amount equal to the percentage of the liable person’s exempt income which is the same as the amounts earned with an employer, as a percentage of the total earnings with the employers.

(8) Any reference to an “employer” in paragraphs~(6) and~(7) is to be construed as a reference to an employer subject to an order made in respect of a liable person.”.
\end{quotation}
\end{enumerate}

(4) In regulation~22 (appeals against deduction from earnings orders)—
\begin{enumerate}\item[]
\begin{sloppypar}
($a$) at the beginning of paragraph~(2), insert “Subject to paragraph~(2A),”;
\end{sloppypar}

($b$) after paragraph~(2), insert—
\begin{quotation}
“(2A) Any appeal against a decision of the Secretary of State that the exclusion required by regulation~3(3) does not apply is—
\begin{enumerate}\item[]
($a$) where the liable person is resident in the United Kingdom, to be made within 28 days of the date on which that decision is given or~sent to the liable person;

($b$) where the liable person is not resident in the United Kingdom, to be made within 56 days of the date on which that decision is given or~sent to the liable person.”;
\end{enumerate}
\end{quotation}

\begin{sloppypar}
($c$) at the beginning of paragraph~(3), insert “Subject to paragraph~(3A),”;
\end{sloppypar}

($d$) after paragraph~(3), insert—
\begin{quotation}
“(3A) Where the Secretary of State is considering specifying a deduction from earnings order as a method of payment under regulation~3(1)($i$)  an appeal may also be made against a decision of the Secretary of State that the exclusion required by regulation~3(3) does not apply.”;
\end{quotation}

($e$) at the beginning of paragraph~(4), insert “Subject to paragraph~(5),”; and

($f$) after paragraph~(4), add—
\begin{quotation}
“(5) Where an appeal is brought on the grounds set out in paragraph~(3A), and~the court, or~as the case may be, the sheriff, is satisfied that the appeal should be allowed the court or~the sheriff is to refer the case to the Secretary of State for~him to specify whichever of the methods of payment set out in regulation~3(1) he considers to be appropriate in the circumstances.”.
\end{quotation}
\end{enumerate}

\subsection[3. Amendment of the Decisions and~Appeals Regulations]{Amendment of the Decisions and~Appeals Regulations}

3.  In regulation~3A of the Decisions and~Appeals Regulations (revision of child support decisions)\footnote{Regulation~3A was inserted by regulation~5 of S.I.~2000/3185.} after paragraph~(7), add—
\begin{quotation}
“(8) Subject to paragraph~(9), section~16 of the Child Support Act shall apply in relation to any decision of the Secretary of State not to make a maintenance calculation, as it applies in relation to any decision of the Secretary of State under sections~11, 12 or~17 of that Act, or~the determination of an appeal tribunal on a referral under section~28D(1)($b$)  of that Act.

(9) Paragraph (8) shall not apply to any decision not to make a maintenance calculation where the Secretary of State makes a decision under section~12 of the Child Support Act.”.
\end{quotation}

\subsection[4. Amendment of the Maintenance Assessment Procedure Regulations]{Amendment of the Maintenance Assessment Procedure Regulations}

4.  In regulation~30A\footnote{Regulation~30A was inserted by regulation~33 of S.I.~1995/3261.} of the Maintenance Assessment Procedure Regulations, after paragraph~(7) insert—
\begin{quotation}
“(8) The effective date of a new maintenance assessment, where the circumstances set out in paragraph~(9) apply, shall be—
\begin{enumerate}\item[]
($a$) on, or~on one of the 6 days immediately following, the effective date as it would have been but for~this paragraph; and

($b$) on the same day of the week as the day on which the maintenance period in respect of the previous maintenance assessment, as defined in paragraph~(9)($b$), began.
\end{enumerate}

(9) The circumstances referred to in paragraph~(8) are where—
\begin{enumerate}\item[]
($a$) a maintenance assessment (“the previous maintenance assessment”) has been in force in relation to the absent parent, whether or~not in respect of the same parent with care; and

($b$) the previous maintenance assessment is no longer in force when the decision as to the maintenance assessment is made.”.
\end{enumerate}
\end{quotation}

\subsection[5. Amendment of the Maintenance Calculation Procedure Regulations]{Amendment of the Maintenance Calculation Procedure Regulations}

5.---(1)  Part~VII of the Maintenance Calculation Procedure Regulations\footnote{The relevant amending instruments are S.I.~2002/1204, S.I.~2003/328 and S.I.~2006/1520.} is amended as follows.

(2) In regulation~25(1), after “29” insert “,~29B”.

(3) In regulation~26(1), for~“regulation~28” substitute “regulations~28 and~29B”.

(4) At the beginning of regulation~28, insert “Subject to regulation~29B,”.

(5) At the beginning of regulation~29(1), insert “Subject to regulation~29B,”.

(6) After regulation~29A insert—
\begin{quotation}
\subsection*{“Effective date where there has been a previous maintenance calculation}

29B.---(1)  This regulation~applies where—
\begin{enumerate}\item[]
($a$) a maintenance calculation (“the previous maintenance calculation”) has been in force in relation to the non-resident parent, whether or~not in respect of the same parent with care; and

($b$) the previous maintenance calculation is no longer in force when the decision as to the maintenance calculation is made.
\end{enumerate}

(2) Where this regulation~applies, the effective date of the maintenance calculation shall be—
\begin{enumerate}\item[]
($a$) on, or~on one of the 6 days immediately following, the effective date as it would have been but for~this regulation; and

($b$) on the same day of the week as the day on which the maintenance period in respect of the previous maintenance calculation began.”.
\end{enumerate}
\end{quotation}

(7) In regulation~31, after paragraph~(1C) insert—
\begin{quotation}
“(1D) Where a maintenance assessment has been in force in relation to a non-resident parent, regulation~29B shall apply as if references to a maintenance calculation having been in force were to a maintenance assessment having been in force.”.
\end{quotation}

\subsection[6. Amendment of the Maintenance Calculations and~Special Cases Regulations]{Amendment of the Maintenance Calculations and~Special Cases Regulations}

6.  In paragraph~6(3) of the Schedule~to the Maintenance Calculations and~Special Cases Regulations, omit “made in anticipation of the calculation of profits”.

\subsection[7. Amendment of the Transitional Provisions Regulations]{Amendment of the Transitional Provisions Regulations}

7.  In regulation~15(3A) of the Transitional Provisions Regulations (case conversion date)\footnote{Paragraph~(3A) was inserted by regulation~9 of S.I.~2003/328.}, for~paragraph~($b$)  substitute—
\begin{quotation}
“($b$) A is the non-resident parent in relation to the maintenance calculation and~B is the absent parent in relation to the maintenance assessment.”.
\end{quotation}

\bigskip

Signed 
by authority of the 
Secretary of State for~Work and~Pensions.
%I concur

{\raggedleft
\emph{Stephen C.~Timms}\\*
Minister
%Parliamentary Under-Secretary 
of State,\\*Department for~Work and~Pensions

}

26th September 2008

\small

\part{Explanatory Note}

\renewcommand\parthead{— Explanatory Note}

\subsection*{(This note is not part of the Regulations)}

The powers exercised to make these Regulations are those contained in the Child Support Act 1991 (c.~48) (“the 1991 Act”). Some of those powers are conferred by provisions of the 1991 Act prior~to the amendments made to that Act by the Child Support, Pensions and~Social Security Act 2000 (c.~19) (“the 2000 Act”), some of which amendments are not fully in force, and~relate to the child support scheme which was in force prior~to 3rd March 2003 and~which remains in force for~the purposes of certain cases (“the old scheme”). Other powers are conferred by provisions of the 1991 Act as amended by the 2000 Act, which relate to the child support scheme provided for~by those amendments, which came into force for~the purposes of specified categories of cases on 3rd March 2003 (see the Child Support, Pensions and~Social Security Act 2000 (Commencement No.~12) Order 2003) (“the current scheme”). These Regulations also exercise powers inserted into the 1991 Act by the Child Maintenance and~Other Payments Act 2008 (c.~6).

Regulation 2 amends the Child Support (Collection and~Enforcement) Regulations 1992 (“the Collection and~Enforcement Regulations”).

Regulation 2(2) amends regulation~3 of the Collection and~Enforcement Regulations to make provision omitting a voluntary deduction from earnings arrangement from the methods of payment which may be specified by the Secretary of State, for~the Secretary of State to specify a deduction from earnings order (“an order”) as a method of payment and~paragraph~(2)($b$)  makes a consequential amendment. Paragraph (2)($c$)  inserts new paragraphs~(3) to (9) in regulation~3 of those Regulations providing for~that method not to be used where there is good reason not to use it, the matters which are, and~are not, to be taken into account and~the circumstances which are to be regarded as existing, when determining whether there is good reason not to use that method of payment, the period during which that method may not be specified and~makes provision which allows the Secretary of State to make an order where it considers it is appropriate in the circumstances of the case, unless it has specified an order as a method of payment. These amendments apply to the old scheme and~the current scheme.

Regulation 2(3) amends regulation~11 of the Collection and~Enforcement Regulations to insert new paragraphs~(5) to (8). These paragraphs~set out how the protected earnings rate is to be calculated where a liable person has more than one employer and~the Secretary of State makes a deduction from earnings order against that person in respect of more than one employer. These amendments apply to the old scheme.

Regulation 2(4) amends regulation~22 of the Collection and~Enforcement Regulations inserting new paragraphs~(2A), (3A) and~(5). Inserted paragraph~(2A) provides that such an appeal is to be made within 28 days of the date on which that decision is given or~sent to the liable person, inserted paragraph~(3A) provides that a liable person may appeal against the decision of the Secretary of State that the exclusion required by regulation~3(3) (that a deduction from earnings order is not to be used as a method of payment where there is good reason not to use it) does not apply and~inserted paragraph~(5) provides that on an appeal where the magistrates’ court or~the sheriff is satisfied that the appeal should be allowed it is to refer the case to the Secretary of State to specify the method of payment. These amendments apply to the old scheme and~the current scheme.

Regulation 3 amends regulation~3A of the Social Security and~Child Support (Decisions and~Appeals) Regulations 1999 to add new paragraphs~(8) and~(9). Inserted paragraph~(8) extends the categories of case to which section~16 of the 1991 Act applies to a decision of the Secretary of State not to make a maintenance calculation. This amendment applies to the current scheme.

Regulation 4 amends regulation~30A of the Child Support (Maintenance Assessment Procedure) Regulations 1992 to insert new paragraphs~(8) and~(9). The effect of this is to enable the effective date of a maintenance calculation to be aligned with the first day of the maintenance period of a maintenance calculation or~assessment that has previously been, but has ceased to be, in force in relation to the non-resident parent. This amendment applies to the old scheme.

Regulation 5 amends the Child Support (Maintenance Calculation Procedure) Regulations 2000 to insert regulation~29B. This has the same effect as regulation~4 but in relation to the current scheme.

Regulation 6 amends paragraph~6(3) of the Schedule~to the Child Support (Maintenance Calculations and~Special Cases) Regulations 2000 to omit the words “made in anticipation of the calculation of profits” from that paragraph~so that, where a person receives a commission or~bonus as part of their earnings, that commission or~bonus does not have to be made in anticipation of the calculation of profits to be taken into account under that paragraph~for~the purposes of calculating net income. This amendment applies to the current scheme.

Regulation 7 amends the Child Support (Transitional Provisions) Regulations 2000 (“the Transitional Provisions Regulations”) by removing sub-paragraph~($b$)(ii)  of regulation~15(3A). This means that child support cases will no longer be converted in accordance with regulation~15(2) of the Transitional Provisions Regulations if a maintenance calculation is made with respect to a person with care who is living with someone who is a person with care in relation to a maintenance assessment. This amendment is consequential upon the repeal of section~10 of the Child Support Act 1995, abolishing the Child Maintenance Bonus.

A full impact assessment has not been produced for~this instrument as it has only a negligible impact on the private and~voluntary sectors. 

\end{document}