\documentclass[12pt,a4paper]{article}

\newcommand\regstitle{The Social Security Amendment (Residential Care and~Nursing Homes) Regulations 2002}

\newcommand\regsnumber{2002/398}

%\opt{newrules}{
\title{\regstitle}
%}

%\opt{2012rules}{
%\title{Child Maintenance and~Other Payments Act 2008\\(2012 scheme version)}
%}

\author{S.I.\ 2002 No.\ 398}

\date{Made
21st February 2002\\
Laid before Parliament
27th February 2002\\
Coming into~force
8th April 2002
}

%\opt{oldrules}{\newcommand\versionyear{1993}}
%\opt{newrules}{\newcommand\versionyear{2003}}
%\opt{2012rules}{\newcommand\versionyear{2012}}

\usepackage{csa-regs}

\setlength\headheight{42.11603pt}

%\hbadness=10000

\begin{document}

\maketitle

\noindent
The Secretary of State for Work and~Pensions, in exercise of the powers conferred upon him by sections~123(1)($a$), 135(1), 137(1) and~175(1), (3) and~(4) of the Social Security Contributions and~Benefits Act 1992\footnote{1992 c.~4; section 137(1) is an interpretation provision and is cited because of the meaning ascribed to the word “prescribed”.}, sections~5(1)($p$)  and~189(4) of the Social Security Administration Act 1992\footnote{1992 c.~5.}, sections~4(5),~35(1) and~36(1) and~(2) of the Jobseekers Act 1995\footnote{1995 c.~18; section 35(1) is an interpretation provision and is cited because of the meaning ascribed to the words “prescribed” and “regulations”.} and~sections~10(6) and~84 of the Social Security Act 1998\footnote{1998 c.~14; section 84 is an interpretation provision and is cited because of the meaning ascribed to the word “prescribe”.} and~of all other powers enabling him in that behalf, after agreement by the Social Security Advisory Committee that proposals in respect of these Regulations should not be referred to it\footnote{See sections 170 and 173(1)($b$)  of the Social Security Administration Act 1992; paragraph~67 of Schedule 2 to the Jobseekers Act 1995 added that Act to the list of “relevant enactments” in respect of which regulations must normally be referred to the Committee.}, hereby makes the following Regulations: 

{\sloppy

\tableofcontents

}

\bigskip

\setcounter{secnumdepth}{-2}

\subsection[1. Citation and~commencement]{Citation and~commencement}

1.  These Regulations may be cited as the Social Security Amendment (Residential Care and~Nursing Homes) Regulations 2002 and~shall come into force on 8th April 2002.

\subsection[2. Amendment of the Social Security (Claims and~Payments) Regulations 1987]{Amendment of the Social Security (Claims and~Payments) Regulations 1987}

2.---(1)  Schedule 9 to the Social Security (Claims and~Payments) Regulations 1987\footnote{S.I.~1987/1968; the relevant amending instruments are S.I.~1991/2284, 1992/3147, 1993/2113 and 1996/1460.} (deductions from benefit and~direct payment to third parties) shall be amended in accordance with the following paragraphs of this regulation.

(2) In the definition of “hostel” in paragraph~1(1), for “19(3)” there shall be substituted “2(1)”.

(3) In paragraph~4 (miscellaneous accommodation costs)—
\begin{enumerate}\item[]
($a$) in sub-paragraph~(1)—
\begin{enumerate}\item[]
(i) in paragraph~($a$), for “19(3)” there shall be substituted “2(1)”;

(ii) in paragraph~($b$), the words “Schedule 4 (persons in residential care and~nursing homes) or” and~“Schedule 4 (applicable amounts of persons in residential care and~nursing homes) or” shall be omitted;
\end{enumerate}

($b$) in sub-paragraph~(2)—
\begin{enumerate}\item[]
(i) paragraph~($a$)  shall be omitted;

(ii) in paragraph~($ab$)—
\begin{enumerate}\item[]
($aa$) for the words from “does not have a preserved right” to “1948” there shall be substituted the words “is not in residential accommodation for the purposes of the Income Support Regulations or, as the case may be, for the purposes of the Jobseeker’s Allowance Regulations,”;

($bb$) for the words from “will equal the aggregate of the amounts” to the end of the paragraph~there shall be substituted the words “will equal the amount prescribed in respect of personal expenses in paragraph~13(1) of Schedule 7 to the Income Support Regulations or, as the case may be, in paragraph~15(1) of Schedule 5 or paragraph~9(1) of Schedule 5A to the Jobseeker’s Allowance Regulations;”.
\end{enumerate}
\end{enumerate}
\end{enumerate}

\subsection[3. Amendment of the Social Security and~Child Support (Decisions and~Appeals) Regulations 1999]{Amendment of the Social Security and~Child Support (Decisions and~Appeals) Regulations 1999}

3.  In Schedule 3A to the Social Security and~Child Support (Decisions and~Appeals) Regulations 1999\footnote{S.I.~1999/991; Schedule 3A was inserted by S.I.~2000/1596.} (date on which change of circumstances takes effect in certain cases where a claimant is in receipt of income support or jobseeker’s allowance)—
\begin{enumerate}\item[]
($a$) after paragraph~3($a$), there shall be inserted the following sub-paragraph—
\begin{quotation}
“($aa$) income support is being paid from 8th April 2002 to persons who, immediately before that day, had a preserved right for the purposes of the Income Support Regulations;”;
\end{quotation}

($b$) after paragraph~8($a$), there shall be inserted the following sub-paragraph—
\begin{quotation}
“($aa$) jobseeker’s allowance is being paid from 8th April 2002 to persons who, immediately before that day, had a preserved right for the purposes of the Jobseeker’s Allowance Regulations;”.
\end{quotation}
\end{enumerate}

\subsection[4. Amendment of the Social Security Amendment (Residential Care and~Nursing Homes) Regulations 2001]{Amendment of the Social Security Amendment (Residential Care and~Nursing Homes) Regulations 2001}

4.---(1)  The Schedule to the Social Security Amendment (Residential Care and~Nursing Homes) Regulations 2001\footnote{S.I.~2001/3767.} shall be amended in accordance with the following paragraphs of this regulation.

(2) In Part I—
\begin{enumerate}\item[]
($a$) in paragraph~12—
\begin{enumerate}\item[]
(i) before sub-paragraph~($a$), there shall be inserted the following sub-paragraph—
\begin{quotation}
“($za$) in paragraph~(1)($a$), “($b$),” shall be omitted;”;
\end{quotation}

(ii) after sub-paragraph~($b$), there shall be inserted the following sub-paragraph—
\begin{quotation}
“($bb$) in paragraph~(1)($d$), “($b$)  or” shall be omitted;”;
\end{quotation}
\end{enumerate}

($b$) for paragraph~14($b$)(ii), there shall be substituted the following—
\begin{quotation}
“(ii) in paragraph~($a$), the words “pursuant to sub-\hspace{0pt}paragraph~(4A)” shall be omitted;”;
\end{quotation}

($c$) after paragraph~17($h$), there shall be added the following sub-paragraph—
\begin{quotation}
“($i$) in column (2) of paragraph~19A\footnote{Paragraph 19A was inserted into Schedule 7 to the Income Support (General) Regulations 1987 (S.I.~1987/1967) by S.I.~2001/488.}, in sub-paragraph~(2), “or 4” shall be omitted.”.
\end{quotation}
\end{enumerate}

(3) In Part II—
\begin{enumerate}\item[]
($a$) in paragraph~15—
\begin{enumerate}\item[]
(i) before sub-paragraph~($a$), there shall be inserted the following sub-paragraph—
\begin{quotation}
“($za$) in paragraph~(1)($a$), “($b$),” shall be omitted;”;
\end{quotation}

(ii) after sub-paragraph~($b$), there shall be inserted the following sub-paragraph—
\begin{quotation}
“($bb$) in paragraph~(1)($d$), “($b$)  or” shall be omitted;”;
\end{quotation}
\end{enumerate}

($b$) in paragraph~16—
\begin{enumerate}\item[]
(i) for “148” there shall be substituted “148A”;

(ii) before sub-paragraph~($a$), there shall be inserted the following sub-paragraph—
\begin{quotation}
“($za$) in paragraph~(1)($a$), “($b$),” shall be omitted;”;
\end{quotation}

(iii) after sub-paragraph~($b$), there shall be inserted the following sub-paragraph—
\begin{quotation}
“($bb$) in paragraph~(1)($d$), “($b$)  or” shall be omitted;”;
\end{quotation}
\end{enumerate}

($c$) for paragraph~18($b$)(ii), there shall be substituted the following—
\begin{quotation}
“(ii) in paragraph~($a$), the words “pursuant to sub-\hspace{0pt}paragraph~(5)” shall be omitted;”;
\end{quotation}

($d$) in paragraph~20($b$), for the word “both” there shall be substituted the word “all”.
\end{enumerate}

\bigskip

Signed 
by authority of the 
Secretary of State for~Work and~Pensions.
%I concur
%By authority of the Lord Chancellor

{\raggedleft
\emph{Maria Eagle}\\*
%Secretary
%Minister
Parliamentary Under-Secretary 
of State,\\*Department 
for~Work and~Pensions

}

21st February 2002

\small

\part{Explanatory Note}

\renewcommand\parthead{— Explanatory Note}

\subsection*{(This note is not part of the Regulations)}

Regulations 2, 3 and~4(2)($c$)  of these Regulations respectively amend the Social Security (Claims and~Payments) Regulations 1987 (S.I.~1987/1968), the Social Security and~Child Support (Decisions and~Appeals) Regulations 1999 (S.I.~1999/991) and~the Social Security Amendment (Residential Care and~Nursing Homes) Regulations 2001 (S.I.~2001/3767) (“the Preserved Rights Regulations”) so as to make further provision and~consequential amendments in connection with the cessation, from 8th April 2002, of the payment of special amounts which are applicable to recipients of income support and~jobseeker’s allowance who are in residential care homes and~nursing homes. Regulation 4(2)($a$)  and~($b$)  and~(3) makes further technical amendments to the Preserved Rights Regulations.

These Regulations do not impose a charge on business. 

\end{document}