\documentclass[12pt,a4paper]{article}

\newcommand\regstitle{The Child Support Maintenance (Changes to Basic Rate Calculation and Minimum Amount of Liability) Regulations 2012}

\newcommand\regsnumber{2012/2678}

%\opt{newrules}{
\title{\regstitle}
%}

%\opt{2012rules}{
%\title{Child Maintenance and~Other Payments Act 2008\\(2012 scheme version)}
%}

\author{S.I.\ 2012 No.\ 2678}

\date{Made
23rd October 2012\\
%Laid before Parliament
%8th November 2012\\
Coming into force
in accordance with regulation 1
}

%\opt{oldrules}{\newcommand\versionyear{1993}}
%\opt{newrules}{\newcommand\versionyear{2003}}
%\opt{2012rules}{\newcommand\versionyear{2012}}

\usepackage{csa-regs}

\setlength\headheight{42.11603pt}

%\hbadness=10000

\begin{document}

\maketitle

\enlargethispage{\baselineskip}

\noindent
The Secretary of State makes the following Regulations in exercise of the powers conferred by section 52(4) of, and paragraph 10A(1) of Schedule 1 to, the Child Support Act 1991\footnote{1991 c.~48. Schedule 1 was substituted by the Child Support, Pensions and Social Security Act 2000 (c.~19), section 1(3); paragraph 10A(1) was amended by the Child Maintenance and Other Payments Act 2008 (c.~6) (”the 2008 Act”), Schedule 7, paragraph 1(30).}:

A draft of this instrument was laid before and approved by a resolution of each House of Parliament in accordance with section 52(2)\footnote{Section 52(2) was substituted by the Child Support, Pensions and Social Security Act 2000 (c.~19), section 25.} of that Act. 

{\sloppy

\tableofcontents

}

\bigskip

\setcounter{secnumdepth}{-2}

\subsection[1. Citation and commencement]{Citation and commencement}

1.---(1)  These Regulations may be cited as the Child Support Maintenance (Changes to Basic Rate Calculation and Minimum Amount of Liability) Regulations 2012 and shall come into force as set out below.

(2) This regulation comes into force on the day on which paragraph 3 of Schedule 4 to the Child Maintenance and Other Payments Act 2008\footnote{2008 c.~6.} comes into force for the first time.

(3) Regulation 2 (change to calculation of the basic rate) comes into force in relation to a particular case on the day on which paragraph 3 of Schedule~4 to the Child Maintenance and Other Payments Act 2008 comes into force in relation to that type of case, immediately after that paragraph comes into force.\looseness=-1

(4) Regulation 3 (minimum amount of liability where non-resident parent party to other maintenance arrangement) comes into force in relation to a particular case on the day on which paragraph 5 of Schedule 4 to the Child Maintenance and Other Payments Act 2008 comes into force in relation to that type of case, immediately after that paragraph comes into force.

\subsection[2. Change to calculation of the basic rate]{Change to calculation of the basic rate}

2.  In Schedule 1 to the Child Support Act 1991, sub-paragraph (3) of paragraph 2\footnote{1991 c.~48. Paragraph 2 was substituted by the 2008 Act, Schedule 4, paragraph 3.} shall have effect as if, for “12\%”, “16\%” and “19\%” in that sub-paragraph, there were substituted “11\%”, “14\%” and “16\%” respectively.

\subsection[3. Minimum amount of liability where non-resident parent party to other maintenance arrangement]{Minimum amount of liability where non-resident parent party to other maintenance arrangement}

3.  In Schedule 1 to the Child Support Act 1991, sub-paragraph (2) of paragraph 5A\footnote{Paragraph 5A was inserted by the 2008 Act, Schedule 4, paragraph 5.} shall have effect as if for “£7” in that sub-paragraph there were substituted “£5”. 

\bigskip

\pagebreak[3]

Signed 
by authority of the 
Secretary of State for~Work and~Pensions.
%I concur
%By authority of the Lord Chancellor

{\raggedleft
\emph{Steve Webb}\\*
%Secretary
Minister
%Parliamentary Under-Secretary 
of State\\*Department 
for~Work and~Pensions

}

23rd October 2012

\small

\part{Explanatory Note}

\renewcommand\parthead{— Explanatory Note}

\subsection*{(This note is not part of the Regulations)}

These Regulations modify the provisions in Schedule 1 to the Child Support Act 1991 (c.~48) (“the 1991 Act”) relating to the calculation of the basic rate of maintenance and the minimum amount of liability where the non-resident parent is party to another maintenance arrangement.

Paragraph 3 of Schedule 4 to the Child Maintenance and Other Payments Act 2008 (c.~6) (“the 2008 Act”) substitutes a new paragraph 2 of Schedule 1 to the 1991 Act. Under paragraph 2, the basic rate is calculated by applying a percentage to the non-resident parent’s gross weekly income. For the purposes of this calculation, paragraph 2(3) provides that, where the non-resident parent has one or more relevant other children, the non-resident parent’s weekly income is reduced by a given percentage.

Where paragraph 3 comes into force in relation to a particular case, regulation 2 provides that paragraph 2(3) of Schedule 1 to the 1991 Act has effect in relation to that case as if, for the percentage reductions of 12\% (1 relevant other child), 16\% (2 relevant other children) and 19\% (3 or more relevant other children), there were substituted the percentages of 11\%, 14\% and 16\% respectively.

Under paragraph 4 of Schedule 1 to the 1991 Act, the flat rate of maintenance is £5 a week and, under other provisions of Schedule 1, the minimum amount of liability for various purposes is also £5 a week.

Paragraph 5 of Schedule 4 to the 2008 Act inserts a new paragraph 5A into Schedule 1 of the 1991 Act. Paragraph 5A provides that, where a non-resident parent is party to another maintenance agreement, the minimum amount of liability is £7 a week (however, it is proposed that, on the initial commencement of paragraph 5 of Schedule 4, the flat rate of maintenance and the minimum amount of liability will be £5 a week).

Where paragraph 5 comes into force in relation to a particular case, regulation 3 provides that paragraph 5A(2) of Schedule 1 to the 1991 Act has effect in relation to that case as if for the figure of £7 there were substituted the figure of £5. The net effect is that the minimum amount of liability under paragraph 5A is £5 a week.

A full impact assessment has not been published for this instrument as it has no impact on the private sector or civil society organisations. 

\end{document}