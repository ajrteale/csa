\documentclass[12pt,a4paper]{article}

\newcommand\regstitle{The Social Security and Child Support (Miscellaneous Amendments) Regulations 1997}

\newcommand\regsnumber{1997/827}

%\opt{newrules}{
\title{\regstitle}
%}

%\opt{2012rules}{
%\title{Child Maintenance and Other Payments Act 2008\\(2012 scheme version)}
%}

\author{S.I. 1997 No. 827}

\date{Made 13th March 1997\\Laid before Parliament 17th March 1997\\Coming into force 7th April 1997}

%\opt{oldrules}{\newcommand\versionyear{1993}}
%\opt{newrules}{\newcommand\versionyear{2003}}
%\opt{2012rules}{\newcommand\versionyear{2012}}

\usepackage{csa-regs}

\begin{document}

\maketitle

\noindent
The Secretary of State for Education and Employment, in relation to regulations 2 and 3 of these Regulations and the Secretary of State for Social Security, in relation to the remainder of these Regulations, in exercise of the powers conferred on them by sections 4(5), 20(4) and (6), 35(1) and 36(1) to (3) of the Jobseekers Act 1995\footnote{\frenchspacing 1995 c. 18; section 35(1) is an interpretation provision and is cited because of the meanings ascribed to the words “prescribed” and “regulations”.}, sections 123(1)($a$), 124(1)($e$), 135(1), 137(1) and 175(1) and (3) of the Social Security Contributions and Benefits Act 1992\footnote{\frenchspacing 1992 c. 4; section 124(1)($e$) was inserted by the Jobseekers Act 1995 (c. 18), Schedule 2, paragraph 30(5). Section 137(1) is cited because of the meaning ascribed to the word “prescribed”.}, sections 24 and 30 of the Criminal Justice Act 1991\footnote{\frenchspacing 1991 c. 53.}, sections 5(1)($p$), 15A(2)($a$), 189 and 191 of the Social Security Administration Act 1992\footnote{\frenchspacing 1992 c. 5; section 191 is an interpretation provision and is cited because of the meaning ascribed to the word “prescribe”. Section 15A was inserted by the Social Security (Mortgage Interest Payments) Act 1992 (c. 33).}, sections 14(3), 97(5), 113 and 116(1) of, paragraphs 1 and 6 of Schedule 4 to and paragraph 6 of Schedule 8 to, the Local Government Finance Act 1992\footnote{\frenchspacing 1992 c. 14.} and sections 21, 52(4) and 54($b$) of the Child Support Act 1991\footnote{\frenchspacing 1991 c. 48; section 54 is cited because of the meaning ascribed to the word “prescribed”.}, and of all other powers enabling each of them in that behalf, after agreement by the Social Security Advisory Committee that proposals in respect of regulations 2 to 7 of these Regulations should not be referred to it\footnote{\frenchspacing \emph{See} the Social Security Administration Act 1992 (c. 5), sections 170 and 173(1)($b$) and the Jobseekers Act 1995 (c. 18), Schedule 2, paragraph 67($a$).}, and after consultation with the Council on Tribunals in respect of regulation 9\footnote{\frenchspacing \emph{See} section 8 of the Tribunals and Inquiries Act 1992 (c. 53).}, hereby make the following Regulations:

{\sloppy

\tableofcontents

}

\bigskip

\setcounter{secnumdepth}{-2}

\subsection[1. Citation, commencement and interpretation]{Citation, commencement and interpretation}

1.—(1) These Regulations may be cited as the Social Security and Child Support (Miscellaneous Amendments) Regulations 1997 and shall come into force on 7th April 1997.

(2) In these Regulations, “the Jobseeker’s Allowance Regulations” means the Jobseeker’s Allowance Regulations 1996\footnote{\frenchspacing S.I. 1996/207; to which there are amendments which are not relevant to these Regulations.} and “the Income Support Regulations” means the Income Support (General) Regulations 1987\footnote{\frenchspacing S.I. 1987/1967.}.

\subsection[2. Amendment of regulation 63 of the Jobseeker’s Allowance Regulations]{Amendment of regulation 63 of the Jobseeker’s Allowance Regulations}

2.—(1) Regulation 63 of the Jobseeker’s Allowance Regulations (reduced payments under section 17) shall be amended in accordance with the following paragraphs of this regulation.

(2) In paragraph (1), after the words “reduced by” there shall be inserted the words “, if he is a single person or a lone parent,” and for the words “in accordance with paragraph 1 of Schedule 1” there shall be substituted the words—
\begin{quotation}
“in accordance with paragraph 1(1) or 1(2) of Schedule 1 (as the case may be) or, if he is a member of a couple, a sum equal to 40\% of the amount which would have been applicable in his case if he had been a single person determined in accordance with paragraph 1(1) of Schedule 1”.
\end{quotation}

(3) In paragraph (3), after the words “shall be” there shall be inserted the words “if he is a single person or a lone parent” and at the end there shall be inserted the words—
\begin{quotation}
“determined in accordance with paragraph 1(1) or 1(2) of Schedule 1 (as the case may be) or, if he is a member of a couple, of 20\% of the amount which would have been applicable in his case if he had been a single person determined in accordance with paragraph 1(1) of Schedule 1”.
\end{quotation}

\subsection[3. Amendment of regulation 68 of the Jobseeker’s Allowance Regulations]{Amendment of regulation 68 of the Jobseeker’s Allowance Regulations}

3.—(1) Regulation 68 of the Jobseeker’s Allowance Regulations (reduced amount of allowance) shall be amended in accordance with the following paragraphs of this regulation.

(2) In paragraph (1), after the words “reduced by”, there shall be inserted the words “, if he is a single person or a lone parent,” and for the words “in accordance with paragraph 1 of Schedule 1” there shall be substituted the words—
\begin{quotation}
“in accordance with paragraph 1(1) or 1(2) of Schedule 1 (as the case may be) or, if he is a member of a couple, a sum equal to 40\% of the amount which would have been applicable in his case if he had been a single person determined in accordance with paragraph 1(1) of Schedule 1”.
\end{quotation}

(3) In paragraph (2), after the words “reduced by” there shall be inserted the words “, if he is a single person or a lone parent,” and for the words “in accordance with paragraph 1 of Schedule 1” there shall be substituted the words—
\begin{quotation}
“in accordance with paragraph 1(1) or 1(2) of Schedule 1 (as the case may be) or, if he is a member of a couple, a sum equal to 20\% of the amount which would have been applicable in his case if he had been a single person determined in accordance with paragraph 1(1) of Schedule 1”.
\end{quotation}

\subsection[4. Amendment of Schedule 2 to the Jobseeker’s Allowance Regulations]{Amendment of Schedule 2 to the Jobseeker’s Allowance Regulations}

4.—(1) Schedule 2 to the Jobseeker’s Allowance Regulations (housing costs) shall be amended in accordance with the following paragraphs of this regulation.

(2) For sub-paragraph (3) of paragraph 13 (linking rules) there shall be substituted the following sub-paragraph—
\begin{quotation}
“(3) For the purposes of this Schedule, where a claimant has ceased to be entitled to a jobseeker’s allowance because he or his partner is participating in arrangements for training made under section 2 of the Employment and Training Act 1973\footnote{\frenchspacing 1973 c. 50; section 2 was substituted by the Employment Act 1988 (c. 19), section 25(1) and repealed in part by the Employment Act 1989 (c. 38), section 29(4), Schedule 7, Part I.} or attending a course at an employment rehabilitation centre established under that section, he shall be treated as if he had been in receipt of a jobseeker’s allowance for the period during which he or his partner was participating in such a course.”.
\end{quotation}

(3) In paragraph 17(7)($d$) (non-dependant deductions), for the words “a jobseeker’s allowance” there shall be substituted the words “an income-based jobseeker’s allowance”.

\subsection[5. Amendment of Schedule 1B to the Income Support Regulations]{Amendment of Schedule 1B to the Income Support Regulations}

5.  In paragraph 16(1) of Schedule 1B to the Income Support Regulations\footnote{\frenchspacing Schedule 1B was inserted by S.I. 1996/206.} (certain persons aged 50 or over who are within the prescribed categories of persons) after the words “that date” on the first occasion where those words occur, there shall be inserted the words “was in receipt of income support and”.

\subsection[6. Amendment of Schedule 3 to the Income Support Regulations]{Amendment of Schedule 3 to the Income Support Regulations}

6.  In paragraph 18(7)($d$) of Schedule 3 to the Income Support Regulations\footnote{\frenchspacing Schedule 3 was substituted by S.I. 1995/1613.} (non-dependant deductions in relation to housing costs), after the words “income support” there shall be inserted the words “or an income-based jobseeker’s allowance”.

\subsection[7. Amendment of the Social Security (Claims and Payments) Regulations 1987]{Amendment of the Social Security (Claims and Payments) Regulations 1987}

7.—(1) The Social Security (Claims and Payments) Regulations 1987\footnote{\frenchspacing S.I. 1987/1968.} shall be amended in accordance with the following paragraphs of this regulation.

(2) In Schedule 9\footnote{\frenchspacing The relevant amending interests to Schedule 9 are S.I. 1991/2284, 1992/1026, 1993/495 and 1996/481.} (deductions from benefit and direct payments to third parties)—
\begin{enumerate}\item[]
($a$) in paragraph 8(1), the words “and paragraph 3(5) of Schedule 9A” shall be omitted;

($b$) paragraph 8(3) shall be omitted;

($c$) in paragraph 9(1A)($b$), the words “Schedule 9A” shall be omitted; and

($d$) in paragraph 9(1B), head ($za$) shall be omitted.
\end{enumerate}

(3) In Schedule 9A\footnote{\frenchspacing Schedule 9A was inserted by S.I. 1992/1026; the relevant amendments are S.I. 1995/1613 and 1996/1460.} (deductions of Mortgage Interest and Payments to auditing lenders)—
\begin{enumerate}\item[]
($a$) in paragraph 1, the words “5 per cent.\ of the personal allowance” to “paragraph 1(1)($e$) of Schedule 1 to the Jobseeker’s Allowance Regulations” shall be omitted;

($b$) in paragraph 3, sub-paragraphs (5) and (6) (arrears of mortgage interest) shall be omitted; and

($c$) in paragraph 4, the word “plus” at the end of sub-paragraph (2)($b$) and sub-paragraph (2)($c$) shall be omitted.
\end{enumerate}

\subsection[8. Consequential amendments to the amendments made in regulation 7(3)($b$)]{Consequential amendments to the amendments made in regulation 7(3)($b$)}

8.—(1) In regulation 4(2)($b$) of the Fines (Deductions from Income Support) Regulations 1992\footnote{\frenchspacing S.I. 1992/2182; regulation 4 was substituted by S.I. 1993/495.} (sufficient entitlement for deduction), the words “paragraph 3(5) of Schedule 9A to the Claims and Payments Regulations” shall be omitted.

(2) In regulation 5(2)($b$) of the Council Tax (Deductions from Income Support) Regulations 1993\footnote{\frenchspacing S.I. 1993/494.} (sufficient entitlement for deduction), the words “and paragraph 3(5) of Schedule 9A” shall be omitted.

\subsection[9. Amendment of the Child Support Appeal Tribunals (Procedure) Regulations 1992]{Amendment of the Child Support Appeal Tribunals (Procedure) Regulations 1992}

9.  In paragraph (2A) of regulation 3 of the Child Support Appeal Tribunals (Procedure) Regulations 1992\footnote{\frenchspacing S.I. 1992/2641; regulation 3(2A) was inserted by S.I. 1996/3196.} (making an appeal or application and time limits) for “\textsc{\lowercase{FY8 3ZZ}}” there shall be substituted “\textsc{\lowercase{FY1 1GJ}}”.

\bigskip

Signed in connection with regulations 2 and 3 of these Regulations by authority of the Secretary of State for Education and Employment.

{\raggedleft
\emph{Eric Forth}\\*Minister of State,\\*Department for Education and Employment

}

13th March 1997 

\bigskip

Signed 
in connection with the remainder of these Regulations
by authority of the Secretary of State for Social Security.

{\raggedleft
\emph{A. J. B. Mitchell}\\*Parliamentary Under-Secretary of State,\\*Department of Social Security

}

13th March 1997

\small

\part{Explanatory Note}

\renewcommand\parthead{--- Explanatory Note}

\subsection*{(This note is not part of the Regulations)}

These Regulations include amendments to the Jobseeker’s Allowance Regulations 1996 (S.I.\ 1996/207), the Income Support (General) Regulations 1987 (S.I.\ 1987/1967) and the Social Security (Claims and Payments) Regulations 1987 (S.I.\ 1987/1968).

  In particular, these Regulations—
\begin{itemize}
\item clarify the rules relating to the reduction of amounts of jobseeker’s allowance in the case of young persons who commit certain acts, or who fail to do certain acts, by virtue of which, their jobseeker’s allowance falls to be reduced (regulations 2 and 3);
\item clarify the linking rules in jobseeker’s allowance where a person is participating in training or in similar arrangements (regulation 4(2));
\item clarify the position in relation to persons aged 50 or over who are in a prescribed category of person for the purpose of entitlement to income support (regulation 5);
\item provide that non-dependant deductions may not be made in respect of housing costs in both income support and jobseeker’s allowance where a non-dependant is on an income-based jobseeker’s allowance (regulations 4(3) and 6);
\item provide that certain direct payments for mortgage interest arrears can no longer be made (regulation 7(3)($b$)) (regulations 7(2), (3)($a$) and ($c$) and 8 make amendments which are consequential on that amendment).
\end{itemize}

  These Regulations also amend the Child Support Appeal Tribunals (Procedure) Regulations 1992 (S.I.\ 1992/2641) to facilitate arrangements made with the Royal Mail for lodging notices of appeal (regulation 9).

  These Regulations do not impose a charge on businesses.

\end{document}