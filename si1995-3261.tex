\documentclass[a4paper]{article}

\usepackage[welsh,english]{babel}

\usepackage[utf8]{inputenc}
\usepackage[T1]{fontenc}

\usepackage{textcomp}

%\usepackage[2012rules]{optional}

\usepackage[osf]{mathpazo}

\usepackage{perpage}

\usepackage[perpage,para,symbol]{footmisc}

%\opt{newrules}{
\title{The Child Support (Miscellaneous Amendments) (No.\ 2) Regulations 1995}
%}

%\opt{2012rules}{
%\title{Child Maintenance and Other Payments Act 2008\\(2012 scheme version)}
%}

\author{S.I. 1995 No. 3261}

\date{Made 15th December 1995\\Coming into force:\\Regulations 1, 23, 48 and 56 18th December 1995\\Remainder 22nd January 1996}

%\opt{oldrules}{\newcommand\versionyear{1993}}
%\opt{newrules}{\newcommand\versionyear{2003}}
%\opt{2012rules}{\newcommand\versionyear{2012}}

\usepackage{fancyhdr}
\pagestyle{fancy}
\fancyhead[L]{}
\fancyhead[C]{\itshape The Child Support (Miscellaneous Amendments) (No. 2) Regulations 1995 (S.I.~1995/3261) \parthead%\phantom{...}% (\versionyear{} scheme version)
}
\fancyhead[R]{}
\fancyfoot[C]{\thepage}
\newcommand{\parthead}{}

\usepackage{array}
\usepackage{multirow}
\usepackage[debugshow]{tabulary}
\usepackage{longtable}
\usepackage{multicol}
\usepackage{lettrine}

\usepackage[colorlinks=true]{hyperref}
\usepackage{microtype}

\hyphenation{Aw-dur-dod}
\hyphenation{bank-rupt-cy}
\hyphenation{Ec-cles-ton}
\hyphenation{Eux-ton}
\hyphenation{Hogh-ton}
\hyphenation{Pres-ton}
\hyphenation{Pru-den-tial}
\hyphenation{Riv-ing-ton}

\newcolumntype{x}[1]
	{>{\raggedright}p{#1}}
\newcommand{\tn}{\tabularnewline}
\setlength\tymin{50pt}

\newcommand\amendment[1]{\subsubsection*{Notes}{\itshape\frenchspacing\footnotesize #1 \par}}

\setlength\headheight{22.87003pt}

\begin{document}

\maketitle

\noindent
Whereas a draft of this instrument was laid before Parliament in accordance with section 52(2) of the Child Support Act 1991\footnote{\frenchspacing 1991 c. 48.} and approved by a resolution of each House of Parliament:

 Now, therefore, the Secretary of State for Social Security, in exercise of the powers conferred by sections 12(2) and (3), 14(1), (1A) and (3), 16, 17, 18, 32(1), 41(2), 41B(3) and (6), 42(1), 46(5) and (11), 51, 52, 54 and 56(3) of, and paragraphs 5(1) and (2), 6(2), 8 and 11 of Schedule 1 to, the Child Support Act 1991\footnote{\frenchspacing Section 54 is cited because of the meaning ascribed to the word “prescribed”.}, section 18(7) of the Child Support Act 1995\footnote{\frenchspacing 1995 c. 34.} and of all other powers enabling him in that behalf, hereby makes the following Regulations:

{\sloppy

\tableofcontents

}

\setcounter{secnumdepth}{-2}

\subsection[1. Citation, commencement and interpretation]{Citation, commencement and interpretation}

1.—(1) These Regulations may be cited as the Child Support (Miscellaneous Amendments) (No.\ 2) Regulations 1995.

(2) This regulation, regulation 23, regulation 48 and regulation 56 of these Regulations shall come into force on 18th December 1995 and all other regulations shall come into force on 22nd January 1996.

(3) In these Regulations—
\begin{enumerate}\item[]
“the Arrears Regulations” means the Child Support (Arrears, Interest and Adjustment of Maintenance Assessments) Regulations 1992\footnote{\frenchspacing S.I. 1992/1816. Regulation 13 was amended by S.I. 1993/913 and S.I. 1995/1045.};

“the Child Support Amendment Regulations” means the Child Support and Income Support (Amendment) Regulations 1995\footnote{\frenchspacing S.I. 1995/1045.};

“the Collection and Enforcement Regulations” means the Child Support (Collection and Enforcement) Regulations 1992\footnote{\frenchspacing S.I. 1992/1989. Regulation 9 was amended by S.I. 1995/1045.};

“the Information, Evidence and Disclosure Regulations” means the Child Support (Information, Evidence and Disclosure) Regulations 1992\footnote{\frenchspacing S.I. 1992/1812. Regulation 2 was amended by S.I. 1995/123 and S.I. 1995/1045. Regulation 3 was amended and regulation 9A inserted by S.I. 1995/1045.};

“the Maintenance Arrangements and Jurisdiction Regulations” means the Child Support (Maintenance Arrangements and Jurisdiction) Regulations 1992\footnote{\frenchspacing S.I. 1992/2645. Regulation 1 was amended by S.I. 1995/1045 and regulation 3 by S.I. 1995/123 and S.I. 1995/1045.};

“the Maintenance Assessment Procedure Regulations” means the Child Support (Maintenance Assessment Procedure) Regulations 1992\footnote{\frenchspacing S.I. 1992/1813. Regulation 8 was amended by S.I. 1993/913, S.I. 1995/123 and S.I. 1995/1045 and regulation 9 by S.I. 1993/913 and S.I. 1995/1045. Regulation 10 was amended by S.I. 1994/227, S.I. 1995/123 and S.I. 1995/1045. Regulation 12 was amended by S.I. 1993/913 and regulation 14 by S.I. 1995/1045. Regulations 17 and 19 were amended by S.I. 1993/913 and S.I. 1995/1045. Regulation 30 was amended by S.I. 1995/123 and S.I. 1995/1045. Regulation 31 was amended by S.I. 1994/227, S.I. 1995/123 and S.I. 1995/1045.};

“the Maintenance Assessments and Special Cases Regulations” means the Child Support (Maintenance Assessments and Special Cases) Regulations 1992\footnote{\frenchspacing S.I. 1992/1815. Regulations 1, 9, 15, 22 and Schedule 2 were amended by S.I. 1993/913 and S.I. 1995/1045. Regulation 2 was amended by S.I. 1995/1045 and regulation 11 by S.I. 1994/227 and S.I. 1995/1045. Schedule 3 was amended by S.I. 1993/913, S.I. 1994/227 and S.I. 1995/1045.};

“the Miscellaneous Amendments Regulations” means the Child Support (Miscellaneous Amendments and Transitional Provisions) Regulations 1994\footnote{\frenchspacing S.I. 1994/227. Regulation 7 was amended by S.I. 1995/1045.};

“the Northern Ireland Regulations” means the Child Support (Northern Ireland Reciprocal Arrangements) Regulations 1993\footnote{\frenchspacing S.I. 1993/584.}.
\end{enumerate}

\subsection[2. Substitution of regulation 8 of the Arrears Regulations]{Substitution of regulation 8 of the Arrears Regulations}

2.  For regulation 8 of the Arrears Regulations (retention of arrears), there shall be substituted the following regulation—
\begin{quotation}
\subsection*{“Retention of recovered arrears of child support maintenance by the Secretary of State}

8.—(1) This regulation applies where—
\begin{enumerate}\item[]
(i) the Secretary of State recovers arrears from an absent parent under section 41 of the Act; and

(ii) income support is paid to or in respect of the person with care or was paid to or in respect of that person at the date or dates upon which the payment or payments of child support maintenance referred to in paragraph (2) should have been made.
\end{enumerate}

(2) Where paragraph (1) applies, the Secretary of State may retain such amount of those arrears as is equal to the difference between the amount of income support that was paid to or in respect of the person with care and the amount of income support that he is satisfied would have been paid had the absent parent paid, by the due dates, the amounts due under the child support maintenance assessment in force or to be taken to have been in force by virtue of the provisions of section 41(2A) of the Act.”.
\end{quotation}

\subsection[3. Insertion of regulation 10A into the Arrears Regulations]{Insertion of regulation 10A into the Arrears Regulations}

3.  After regulation 10 of the Arrears Regulations (adjustment of amounts), there shall be inserted the following regulation—
\begin{quotation}
\subsection*{\sloppy “Reimbursement of a repayment of overpaid child maintenance}

10A.—(1) The Secretary of State may require a relevant person to repay the whole or a part of any payment by way of reimbursement made to an absent parent under section 41B(2) of the Act where the overpayment referred to in section 41B(1) of the Act arose—
\begin{enumerate}\item[]
($a$) in respect of the amount payable under a maintenance assessment calculated in accordance with Part I of Schedule 1 to the Act and where income support, family credit or disability working allowance was not in payment to that person at any time during the period in which that overpayment occurred or at the date or dates on which the payment by way of reimbursement was made; or

($b$) in respect of the amount payable under an interim maintenance assessment and that amount has not been varied under regulation 8D(1) of the Maintenance Assessment Procedure Regulations following the making of a maintenance assessment calculated in accordance with Part I of Schedule 1 to the Act.
\end{enumerate}

(2) In a case falling within section 4 or 7 of the Act, where the circumstances set out in section 41B(6) apply, the Secretary of State may retain out of the child support maintenance collected by him in accordance with section 29 of the Act such sums as cover the amount of any payment by way of reimbursement required by him from the relevant person under section 41B(3) of the Act.”.
\end{quotation}

\subsection[4. Amendment of regulation 12 of the Arrears Regulations]{Amendment of regulation 12 of the Arrears Regulations}

4.  In paragraph (3) of regulation 12 of the Arrears Regulations (review of adjustments), for “19(1)” there shall be substituted “19(2)”.

\subsection[5. Amendment of regulation 13 of the Arrears Regulations]{Amendment of regulation 13 of the Arrears Regulations}

5.  For paragraph (6) of regulation 13 of the Arrears Regulations (procedure and notifications on applications and reviews under regulation 12) there shall be substituted the following paragraphs—
\begin{quotation}
“(6) A notification under paragraphs (2), (4) and (5), and under paragraph (3) following a review under regulation 12(1), shall include information as to the provisions of section 20 of the Act.

(7) A notification under paragraph (3) following a review under regulation 12(3) shall include information as to the provisions of section 18 of the Act.”.
\end{quotation}

\subsection[6. Amendment of regulation 9 of the Collection and Enforcement Regulations]{Amendment of regulation 9 of the Collection and Enforcement Regulations}

6.  After paragraph ($c$) of regulation 9 of the Collection and Enforcement Regulations (deduction from earnings orders) there shall be inserted the following paragraph—
\begin{quotation}
“($cc$) where known, the liable person’s national insurance number;”.
\end{quotation}

\subsection[7. Amendment of regulation 2 of the Information, Evidence and Disclosure Regulations]{Amendment of regulation 2 of the Information, Evidence and Disclosure Regulations}

7.—(1) Regulation 2 of the Information, Evidence and Disclosure Regulations (duty to furnish information) shall be amended in accordance with the following provisions of this regulation.

(2) In paragraph (1)—
\begin{enumerate}\item[]
(i) after the words “maintenance assessment” there shall be inserted the words “or for a review of a maintenance assessment”;

(ii) after the word “Act” there shall be inserted the words “or a child support officer is conducting or proposing to conduct a review under section 19 of the Act,”; and

(iii) after the words “Secretary of State” there shall be inserted the words “or a child support officer”.
\end{enumerate}

(3) In sub-paragraph ($b$) of paragraph (2), for the words “assessment has been made who” there shall be substituted the words “assessment has been made, or in relation to whom a maintenance assessement has been made in respect of which a child support officer is conducting or proposing to conduct a review and that person”.

(4) In sub-paragraph ($c$) and ($cc$) of paragraph (2), after the words “application for a maintenance assessment has been made” there shall be inserted the words “or in relation to whom a maintenance assessment has been made and a child support officer is conducting or proposing to conduct a review of that assessment”.

\subsection[8. Amendment of regulation 3 of the Information, Evidence and Disclosure Regulations]{Amendment of regulation 3 of the Information, Evidence and Disclosure Regulations}

8.  In regulation 3(1) of the Information, Evidence and Disclosure Regulations (purposes for which evidence required), after the words “the Secretary of State” there shall be inserted the words “or a child support officer”.

\subsection[9. Insertion of regulation 3A into the Information, Evidence and Disclosure Regulations]{Insertion of regulation 3A into the Information, Evidence and Disclosure Regulations}

9.  After regulation 3 of the Information, Evidence and Disclosure Regulations there shall be inserted the following regulation—
\begin{quotation}
\subsection*{“Contents of request for information or evidence}

3A.  Any request by the Secretary of State or a child support officer in accordance with regulations 2 and 3 for the provision of information or evidence in connection with a review or proposed review under section 16, 17, 18 or 19 of the Act shall set out the possible consequences of failure to provide such information or evidence.”.
\end{quotation}

\subsection[10. Substitution of regulation 5 of the Information, Evidence and Disclosure Regulations]{Substitution of regulation 5 of the Information, Evidence and Disclosure Regulations}

10.  For regulation 5 of the Information, Evidence and Disclosure Regulations (time for furnishing information), there shall be substituted the following regulation—
\begin{quotation}
\subsection*{“Time within which information or evidence is to be furnished}

5.—(1) Subject to paragraph (2) and the provisions of regulations 2(5), 6(1) and 17(5) of the Maintenance Assessment Procedure Regulations, information or evidence furnished in accordance with regulations 2 and 3 shall be furnished as soon as is reasonably practicable in the particular circumstances of the case.

(2) Where an application for a review has been made under section 17 or 18 of the Act, or a child support officer is proposing to conduct or is conducting a review under section 19 of the Act, and the Secretary of State or a child support officer has after 22nd January 1996 requested any person to provide information or evidence in accordance with the provisions of regulations 2 and 3 in connection with that review or proposed review, that person shall furnish such information or evidence within 14 days of that request being made.”.
\end{quotation}

\subsection[11. Amendment of regulation 9A of the Information, Evidence and Disclosure Regulations]{Amendment of regulation 9A of the Information, Evidence and Disclosure Regulations}

11.—(1) Regulation 9A of the Information, Evidence and Disclosure Regulations (disclosure of information to other persons), shall be amended in accordance with the following provisions of this regulation.

(2) The word “or” at the end of sub-paragraph ($c$) of paragraph (1) shall be omitted.

(3) After sub-paragraph ($d$) of paragraph (1), there shall be added the following sub-paragraphs—
\begin{quotation}
“($e$) why a decision has been made not to arrange for, or to cease, collection of any child support maintenance under section 29 of the Act;

($f$) why a particular method of enforcement, under section 31, 33, 35, 36, 38 or 40 of the Act of an amount due under a maintenance assessment has been adopted in a particular case; or

($g$) why a decision has been made not to enforce, or to cease to enforce, under section 31 or 33 of the Act the amount due under a maintenance assessment.”.
\end{quotation}

\subsection[12. Substitution of regulation 10 of the Information, Evidence and Disclosure Regulations]{Substitution of regulation 10 of the Information, Evidence and Disclosure Regulations}

12.  For regulation 10 of the Information, Evidence and Disclosure Regulations (disclosure of information to the Secretary of State) there shall be substituted the following regulations—
\begin{quotation}
\subsection*{“Disclosure of information by the Secretary of State}

10.—(1) The Secretary of State may disclose to a child support officer, in the circumstances set out in paragraph (2), information held by him for the purposes of the Act which has been provided by or in relation to a person in connection with an application for a maintenance assessment or in connection with an assessment which is or has been in force.

(2) The information referred to in paragraph (1) may be disclosed for use in connection with any other application for a maintenance assessment or in connection with a review of a maintenance assessment which is or has been in force, in respect of which the person referred to in paragraph (1) is the absent parent, alleged absent parent or person with care.

\subsection*{Disclosure of information by a child support officer}

10A.—(1) A child support officer may disclose to another child support officer or to the Secretary of State, in the circumstances set out in paragraph (2), information held by him for the purposes of the Act which has been provided by or in relation to a person in connection with an application for a maintenance assessment, in connection with a review of a maintenance assessment, or otherwise in connection with an assessment which is or has been in force.

(2) The information referrred to in paragraph (1) may be disclosed for use in connection with any other application for a maintenance assessment, or in connection with any other review of a maintenance assessment, which is or has been in force, in respect of which the person referred to in paragraph (1) is the absent parent, alleged absent parent or person with care.”.
\end{quotation}

\subsection[13. Amendment of regulation 3 of the Maintenance Arrangements and Jurisdiction Regulations]{Amendment of regulation 3 of the Maintenance Arrangements and Jurisdiction Regulations}

13.  At the beginning of paragraphs (5) and (8) of regulation 3 of the Maintenance Arrangements and Jurisdiction Regulations (maintenance assessments and court orders), there shall be inserted the words “Subject to regulation 33(7) of the Maintenance Assessment Procedure Regulations,”.

\subsection[14. Addition of regulation 9 to the Maintenance Arrangements and Jurisdiction Regulations]{Addition of regulation 9 to the Maintenance Arrangements and Jurisdiction Regulations}

14.  After regulation 8 of the Maintenance Arrangements and Jurisdiction Regulations (assessments and orders made in error), there shall be added the following regulation—
\begin{quotation}
\subsection*{“Cases in which application may be made under section 4 or 7 of the Act}

9.  The provisions of section 4(10) or 7(10) of the Act\footnote{\frenchspacing Sections 4(10) and 7(10) were inserted by section 18(7) of the Child Support Act 1995.} shall not apply to prevent an application being made under those sections after 22nd January 1996 where a decision has been made by the relevant court either that it has no power to vary or that it has no power to enforce a maintenance order in a particular case.”.
\end{quotation}

\subsection[15. Amendment of regulation 1 of the Maintenance Assessment Procedure Regulations]{\sloppy Amendment of regulation 1 of the Maintenance Assessment Procedure Regulations}

15.  In paragraph (3) of regulation 1 of the Maintenance Assessment Procedure Regulations (citation, commencement interpretation)—
\begin{enumerate}\item[]
(i) after the words “a direction is “suspended” if”, the word “either” shall be omitted;

(ii) the word “or” at the end of sub-paragraph ($a$) of the definition of “suspended” shall be omitted;

(iii) the word “or” shall be added at the end of sub-paragraph ($b$) of that definition; and

(iv) after sub-paragraph ($b$), there shall be added the following sub-paragraph—
\begin{quotation}
“($c$) at the time that the direction is given one or more of the deductions set out in regulation 40A is being made from the income support payable to or in respect of the parent concerned,”.
\end{quotation}
\end{enumerate}

\subsection[16. Substitution of regulation 8 of the Maintenance Assessment Procedure Regulations]{\sloppy Substitution of regulation 8 of the Maintenance Assessment Procedure Regulations}

16.  For regulation 8 of the Maintenance Assessment Procedure Regulations (amount and duration of an interim maintenance assessment) there shall be substituted the following regulations—
\begin{quotation}
\subsection*{“Categories of interim maintenance assessment}

8.—(1) Where a child support officer serves notice under section 12(4) of the Act of his intention to make an interim maintenance assessment, he shall not make that interim assessment before the end of a period of 14 days, commencing with the date that notice was given or sent.

(2) There shall be four categories of interim maintenance assessment, Category A, Category B, Category C, and Category D interim maintenance assessments.

(3) An interim maintenance assessment made by a child support officer shall be—
\begin{enumerate}\item[]
($a$) a Category A interim maintenance assessment, where any information, other than information referred to in sub-paragraph ($b$), that is required by him to enable him to make an assessment in accordance with the provisions of Part I of Schedule 1 to the Act has not been provided by that absent parent, and that parent has that information in his possession or can reasonably be expected to acquire it;

($b$) a Category B interim maintenance assessment, where the information that is required by him as to the income of the partner or other member of the family of the absent parent or parent with care for the purposes of the calculation of the income of that partner or other member of the family under regulation 9(2), 10, 11(2) or 12(1) of the Maintenance Assessments and Special Cases Regulations—
\begin{enumerate}\item[]
(i) has not been provided by that partner or other member of the family, and that partner or other member of the family has that information in his possession or can reasonably be expected to acquire it; or

(ii) has been provided by that partner or other member of the family to the absent parent or parent with care, but the absent parent or parent with care has not provided it to the Secretary of State or the child support officer;
\end{enumerate}

($c$) a Category C interim maintenance assessment where—
\begin{enumerate}\item[]
(i) the absent parent is a self-employed earner as defined in regulation 1(2) of the Maintenance Assessments and Special Cases Regulations; and

(ii) the absent parent is currently unable to provide, but has indicated that he expects within a reasonable time to be able to provide, information to enable a child support officer to determine the earnings of that absent parent in accordance with paragraphs 3 to 5 of Schedule 1 to the Maintenance Assessments and Special Cases Regulations; and

(iii) no maintenance order as defined in section 8(11) of the Act or written maintenance agreement as defined in section 9(1) of the Act is in force with respect to children in respect of whom the Category C interim maintenance assessment would be made; or
\end{enumerate}

($d$) a Category D interim maintenance assessment where it appears to a child support officer, on the basis of information available to him as to the income of the absent parent, that the amount of any maintenance assessment made in accordance with Part I of Schedule 1 to the Act applicable to that absent parent may be higher than the amount of a Category A interim maintenance assessment in force in respect of him.
\end{enumerate}

(4) In this regulation and in regulation 8A, “family” and “partner” have the same meanings as in the Maintenance Assessments and Special Cases Regulations.

\subsection*{Amount of an interim maintenance assessment}

8A.—(1) The amount of child support maintenance fixed by a Category A interim maintenance assessment shall be 1.5 multiplied by the amount of the maintenance requirement in respect of the qualifying child or qualifying children concerned calculated in accordance with the provisions of paragraph 1 of Schedule 1 to the Act, and paragraphs 2 to 9 of that Schedule shall not apply to Category A interim maintenance assessments.

(2) Subject to paragraph (5), the amount of child support maintenance fixed by a Category B interim maintenance assessment shall be determined in accordance with paragraphs (3) and (4).

(3) Where a child support officer is unable to determine the exempt income—
\begin{enumerate}\item[]
($a$) of an absent parent under regulation 9 of the Maintenance Assessments and Special Cases Regulations because he is unable to determine whether regulation 9(2) of those Regulations applies;

($b$) of a parent with care under regulation 10 of those Regulations because he is unable to determine whether regulation 9(2) of those Regulations, as modified by and applied by regulation 10 of those Regulations applies,
\end{enumerate}
the amount of the Category B interim maintenance assessment shall be the maintenance assessment calculated in accordance with Part I of Schedule 1 to the Act on the assumption that—
\begin{enumerate}\item[]
(i) in a case falling within sub-paragraph ($a$), regulation 9(2) of those Regulations does apply;

(ii) in a case falling within sub-paragraph ($b$), regulation 9(2) of those Regulations as modified by and applied by regulation 10 of those Regulations does apply.
\end{enumerate}

(4) Where the disposable income of an absent parent, calculated in accordance with regulation 12(1)($a$) of the Maintenance Assessments and Special Cases Regulations, would, without taking account of the income of any member of his family, bring him within the provisions of paragraph 6 of Schedule 1 to the Act (protected income), and a child support officer is unable to ascertain the disposable income of the other members of his family, the amount of the Category B interim maintenance assessment shall be the maintenance assessment calculated in accordance with Part I of Schedule 1 to the Act on the assumption that the provisions of paragraph 6 of Schedule 1 to the Act do not apply to the absent parent.

(5) Where the application of the provisions of paragraph (3) or (4) would result in the amount of a Category B interim maintenance assessment being more than 30 per centum of the net income of the absent parent as calculated in accordance with regulation 7 of the Maintenance Assessments and Special Cases Regulations, those provisions shall not apply to that absent parent and instead, the amount of that Category B interim maintenance assessment shall be 30 per centum of his net income as so calculated and where that calculation results in a fraction of a penny, that fraction shall be disregarded.

(6) The amount of child support maintenance fixed by a Category C interim maintenance assessment shall be £30.00 but a child support officer may set a lower amount, including a nil amount, if he thinks it reasonable to do so in all the circumstances of the case.

(7) Paragraph 6 of Schedule 1 to the Act shall not apply to Category C interim maintenance assessments.

(8) A child support officer shall notify the person with care where he is considering setting a lower amount for a Category C interim maintenance assessment in accordance with paragraph (6) and shall take into account any relevant representations made by that person with care in deciding the amount of that Category C interim maintenance assessment.

(9) The amount of child support maintenance fixed by a Category D interim maintenance assessment shall be calculated or estimated by applying to the absent parent’s income, in so far as the child support officer is able to determine it at the time of the making of that Category D interim maintenance assessment, the provisions of Part I of Schedule 1 to the Act and regulations made under it, subject to the modification that—
\begin{enumerate}\item[]
($a$) paragraphs 6 and 8 of that Schedule shall not apply;

($b$) only paragraphs (1)($a$) and (5) of regulation 9 of the Maintenance Assessments and Special Cases Regulations shall apply; and

($c$) heads ($b$) and ($c$) of sub-paragraph (3) of paragraph 1 of Schedule 1 to the Maintenance Assessments and Special Cases Regulations shall not apply.
\end{enumerate}

(10) Where the absent parent referred to in paragraph (9) is an employed earner as defined in regulation 1 of the Maintenance Assessments and Special Cases Regulations and the child support officer is unable to calculate the net income of that absent parent, his net income shall be estimated under the provisions of paragraph (2A)($a$) and ($b$) of that regulation.

\subsection*{Review of an interim maintenance assessment}

8B.—(1) Subject to paragraph (4), where a child support officer—
\begin{enumerate}\item[]
($a$) makes a Category A interim maintenance assessment following a review of a Category A interim maintenance assessment under section 16 of the Act; or

($b$) makes a Category D interim maintenance assessment following a review of a Category D maintenance assessment under section 16 of the Act,
\end{enumerate}
the effective date of that assessment shall be 104 weeks after the effective date of the previous interim maintenance assessment, disregarding any previous interim maintenance assessment made following a review under section 19 of the Act.

(2) Subject to paragraph (4), where a child support officer reviews a Category A or Category D interim maintenance assessment under section 19(1)($c$)\footnote{\frenchspacing Section 19 was substituted by section 15 of the Child Support Act 1995.} of the Act on the grounds that it is defective because of a mistake as to its effective date or for reasons which include a mistake as to its effective date, the effective date of a Category A or Category D interim maintenance assessment made following such a review shall be the correct effective date applicable to the interim maintenance assessment being reviewed, as determined in accordance with paragraph (1), regulation 8C(1) or regulation 3(5) of the Maintenance Arrangements and Jurisdiction Regulations, as the case may be.

(3) Subject to paragraph (4), where a child support officer reviews a Category A or Category D interim maintenance assessment under section 19(1)($c$) of the Act on the grounds that it is defective for reasons which do not include a mistake as to its effective date, the effective date of a Category A or Category D interim maintenance assessment made following such a review shall be the same as the effective date of the interim maintenance assessment that has been reviewed.

(4) Where the effective date of a Category A interim maintenance assessment made following a review under section 16 or 19(1)($c$) of the Act would by virtue of the provisions of paragraphs (1) to (3) be earlier than 16th February 1995, the effective date of that assessment shall be 16th February 1995.

\subsection*{Effective date of an interim maintenance assessment}

8C.—(1) Except where regulation 3(5) of the Maintenance Arrangements and Jurisdiction Regulations (effective date of maintenance assessment where court order in force), regulation 9(9) or 33(7) or paragraph (2) applies, the effective date of an interim maintenance assessment shall be—
\begin{enumerate}\item[]
($a$) in respect of a Category A interim maintenance assessment, subject to regulations 8B, 9(2) and (3) and sub-paragraph ($d$), such date, being not earlier than the first and not later than the seventh day following the date upon which that interim maintenance assessment was made, as falls on the same day of the week as the date determined in accordance with regulation 30(2)($a$)(ii) or ($b$)(ii) as the case may be;

($b$) in respect of a Category B interim maintenance assessment made after 22nd January 1996, subject to sub-paragraph ($d$) and to regulations 31 to 31C, the date specified in regulation 30(2)($a$)(ii) or ($b$)(ii) as the case may be;

($c$) in respect of a Category C interim maintenance assessment, subject to sub-paragraph ($d$) and regulations 31 to 31C, the date set out in sub-paragraph ($a$);

($d$) in respect of a Category A, Category B or Category C interim maintenance assessment, where the application of the provisions of sub-paragraph ($a$), ($b$) or ($c$) would otherwise set an effective date for an interim maintenance assessment earlier than the end of a period of eight weeks from the date upon which—
\begin{enumerate}\item[]
(i) the maintenance enquiry form referred to in regulation 30(2)($a$)(i) was given or sent to an absent parent; or

(ii) the application made by an absent parent referred to in regulation 30(2)($b$)(i) was received by the Secretary of State,
\end{enumerate}
in circumstances where that absent parent has complied with the provisions of regulation 30(2)($a$)(i) or ($b$)(i) or paragraph (2A) of that regulation applies, the date determined in accordance with regulation 30(2)($a$)(i) or ($b$)(i).
\end{enumerate}

(2) The effective date of an interim maintenance assessment made under section 12(1)($b$) or ($c$) of the Act\footnote{\frenchspacing Section 12(1)($b$) and ($c$) were inserted by section 11 of the Child Support Act 1995.} shall, subject to regulations 8B, 9(2), (3) and (9), or 33(7), and, as regards Category B and Category C interim maintenance assessments to regulations 31 to 31C, be such date, not earlier than the first and not later than the seventh day following the date upon which that interim maintenance assessment was made, as falls on the same day of the week as the effective date of the maintenance assessment calculated in accordance with Part I of Schedule 1 to the Act which is being reviewed.

(3) In cases where the effective date of an interim maintenance assessment is determined under paragraph (1), regulation 8B or 9(2), (3) or (9), where a maintenance assessment, except a maintenance assessment falling within regulation 8D(7), is made after an interim maintenance assessment has been in force, child support maintenance calculated in accordance with Part I of Schedule 1 to the Act shall be payable in respect of the period preceding that during which the interim maintenance assessment was in force.

(4) The child support maintenance payable under the provisions of paragraph (3) shall be payable in respect of the period between the effective date of the assessment (or, where separate assessments are made for different periods under paragraph 15 of Schedule 1 to the Act, the effective date of the assessment in respect of the earliest such period) and the effective date of the interim maintenance assessment.

\subsection*{Miscellaneous provisions in relation to interim maintenance assessments}

8D.—(1) Subject to paragraph (2), where a maintenance assessment calculated in accordance with Part I of Schedule 1 to the Act is made following an interim maintenance assessment, the amount of child support maintenance assessment, the amount of child support maintenance payable in respect of the period after 18th April 1995, during which that interim maintenance assessment was in force shall be that fixed by the maintenance assessment.

(2) Paragraph (1) shall not apply where a maintenance assessment calculated in accordance with Part I of Schedule 1 to the Act falls within paragraph (7).

(3) Subject to regulations 9(13) and 9A(6), for the purposes of sections 17, 18 and 19(1)($a$), ($b$) and ($e$) and (6), of the Act, a maintenance assessment shall not include a Category A or Category D interim maintenance assessment.

(4) The provisions of regulations 29, 31 to 31C, 32, 33(5) and 55 shall not apply to a Category A or Category D interim maintenance assessment.

(5) Subject to paragraph (6) and regulation 9(15), an interim maintenance assessment shall cease to have effect on the first day of the maintenance period during which the Secretary of State receives the information which enables a child support officer to make the maintenance assessment or assessments in relation to the same absent parent, person with care, and qualifying child or qualifying children, calculated in accordance with Part I of Schedule 1 to the Act.

(6) Subject to regulation 9(15), where a child support officer has insufficient information or evidence to enable him to make a maintenance assessment calculated in accordance with Part I of Schedule 1 to the Act for the whole of the period beginning with the effective date applicable to a particular case, an interim maintenance assessment made in that case shall cease to have effect—
\begin{enumerate}\item[]
($a$) on 18th April 1995 where by that date the Secretary of State has received the information or evidence set out in paragraph (7); or

($b$) on the first day of the maintenance period after 18th April 1995 in which the Secretary of State has received that information or evidence.
\end{enumerate}

(7) The information or evidence referred to in paragraph (6) is information or evidence enabling a child support officer to make a maintenance assessment calculated in accordance with Part I of Schedule 1 to the Act, for a period beginning after the effective date applicable to that case, in respect of the absent parent, parent with care and qualifying child or qualifying children in respect of whom the interim maintenance assessment referred to in paragraph (6) was made.

(8) For the purposes of paragraph (6), the Secretary of State shall be treated as having received the information or evidence which has caused the interim maintenance assessment to cease to have effect on the first day upon which the absent parent in question became entitled to income support.”.
\end{quotation}

\subsection[17. Substitution of regulation 9 of the Maintenance Assessment Procedure Regulations]{\sloppy Substitution of regulation 9 of the Maintenance Assessment Procedure Regulations}

17.  For regulation 9 of the Maintenance Assessment Procedure Regulations (cancellation of an interim maintenance assessment) there shall be substituted the following regulations—
\begin{quotation}
\subsection*{“Cancellation of an interim maintenance assessment}

9.—(1) Where a child support officer is satisfied that there was unavoidable delay by the absent parent in—
\begin{enumerate}\item[]
(i) completing and returning a maintenance enquiry form under the provisions of regulation 6(1);

(ii) providing information or evidence that is required by the Secretary of State for the determination of an application for a maintenance assessment; or

(iii) providing information or evidence that is required by a child support officer to enable him to conduct or complete a review under section 16, 17, 18 or 19 of the Act,
\end{enumerate}
he may cancel an interim maintenance assessment which is in force.

(2) Where a child support officer cancels a Category A, Category B or Category D interim maintenance assessment in accordance with the provisions of paragraph (1), and he is satisfied that there was unavoidable delay for only part of the period during which that assessment was in force, and that another Category A, Category B or Category D interim maintenance assessment should be made, the effective date of that other Category A, or Category D interim maintenance assessment shall, subject to paragraph (3), be the first day of the maintenance period following the date upon which, in the opinion of the child support officer, the delay became avoidable and the effective date of that other Category B interim maintenance assessment made after 22nd January 1996 shall be the date set out in regulation 8C(1)($b$).

(3) Where the Category A or Category B interim maintenance assessment cancelled in accordance with the provisions of paragraph (1) was made prior to 18th April 1995 and the effective date of any new Category A or Category B interim maintenance assessment would, by virtue of paragraph (2), be prior to 18th April 1995, the effective date of that new Category A or Category B interim maintenance assessment shall be the first day of the maintenance period which begins on or after 18th April 1995.

(4) Where in respect of any Category A or Category B interim maintenance assessment in force before 18th April 1995 the delay referred to in paragraph (1) became avoidable before 18th April 1995, that Category A or Category B interim maintenance assessment may not be cancelled with effect from a date earlier than the date the delay became avoidable.

(5) Subject to paragraph (1), where a child support officer is satisfied that it would be appropriate to make an interim maintenance assessment the Category of which is different from that of the interim maintenance assessment in force, he may cancel the interim maintenance assessment which is in force with effect from—
\begin{enumerate}\item[]
(i) subject to sub-paragraph (ii), whichever is the later of the first day of the maintenance period in which he becomes so satisfied or the first day of the maintenance period which begins on or after 18th April 1995; or

(ii) where he is satisfied that the interim maintenance assessment in force should be replaced by a Category B interim maintenance assessment, whichever is the later of the effective date of the interim maintenance assessment in force or 22nd January 1996.
\end{enumerate}

(6) Where an interim maintenance assessment is cancelled under the provisions of paragraph (5)(ii) and that interim maintenance assessment was made immediately following a previous interim maintenance assessment, a child support officer shall also cancel that previous interim maintenance assessment with effect from the effective date of that previous interim maintenance assessment or 22nd January 1996 whichever is the later.

(7) Where an interim maintenance assessment has been cancelled in the circumstances set out in paragraph (5)(ii) or (6), payments made under that interim maintenance assessment shall be treated as payments made under the Category B interim maintenance assessment which replaces it.

(8) In paragraph (5), “Category” in relation to an interim maintenance assessment means Category A, Category B, Category C or Category D, as the case may be.

(9) Where a child support officer makes an interim maintenance assessment following the cancellation of an interim maintenance assessment in accordance with paragraph (5), the effective date of the fresh interim maintenance assessment shall be—
\begin{enumerate}\item[]
(i) subject to sub-paragraph (ii), the date upon which that cancellation took effect;

(ii) where the fresh interim maintenance assessment is a Category B interim maintenance assessment, subject to paragraphs (10) and (11), the date determined in accordance with regulation 8C(1)($b$) or 22nd January 1996, whichever is later.
\end{enumerate}

(10) Where paragraph (9)(ii) applies and the interim maintenance assessment cancelled in accordance with paragraph (5) caused a court order to cease to have effect in accordance with regulation 3(6) of the Maintenance Arrangements and Jurisdiction Regulations, the effective date of the Category B interim maintenance assessment referred to in paragraph (9)(ii) shall be the date upon which that cancellation took effect.

(11) Where paragraphs (6) and (9)(ii) apply and the interim maintenance assessment cancelled in accordance with paragraph (6) caused a court order to cease to have effect in accordance with regulation 3(6) of the Maintenance Arrangements and Jurisdiction Regulations, the effective date of the Category B interim maintenance assessment referred to in paragraph 9(ii) shall be the date upon which that cancellation in accordance with paragraph (6) took effect.

(12) A child support officer may cancel an interim maintenance assessment which is in force with effect from such date as he considers appropriate in all the circumstances on the grounds that—
\begin{enumerate}\item[]
($a$) there was a material procedural error in connection with the making of the assessment; or

($b$) he is satisfied that he did not, or has subsequently ceased to have jurisdiction to make that interim maintenance assessment.
\end{enumerate}

(13) Where a child support officer has cancelled an interim maintenance assessment under paragraph (12), a relevant person may apply to the Secretary of State for a review of that cancellation under section 18(3) of the Act and the provisions of section 18(5) to (8) shall apply to that review.

(14) Where, following a review under section 18(3) of the Act, a child support officer sets aside the cancellation of the interim maintenance assessment which has been cancelled under paragraph (12), the effective date of the reinstated interim maintenance assessment shall be the date on which the cancelled interim maintenance assessment ceased to have effect or 22nd January 1996 whichever is the later.

(15) An interim maintenance assessment in force which is made under section 12(1)($b$) or ($c$) of the Act shall be cancelled by a child support officer with effect from the effective date of that interim maintenance assessment as soon as is reasonably practicable after he has received the information or evidence which enables him to carry out or to complete a review under section 16, 17, 18 or 19 of the Act.

(16) Where an interim maintenance assessment has been cancelled under paragraph (15), payments made under it shall be treated as payments made under the maintenance assessment being reviewed under section 16, 17, 18 or 19 of the Act or under any maintenance assessment made following the review which replaces for the relevant period the maintenance assessment being reviewed.

\subsection*{\sloppy \textls[25]{Application for cancellation of an interim mainte\-}nance assessment}

9A.—(1) An absent parent with respect to whom a Category A or Category D interim maintenance assessment is in force may apply to a child support officer for that interim maintenance assessment to be cancelled.

(2) Any application made under paragraph (1) shall be in writing, and shall include the a statement of the grounds for the application.

(3) A child support officer who receives an application under provisions of paragraph (1), shall—
\begin{enumerate}\item[]
($a$) decide whether the interim maintenance assessment is to be cancelled and, if so, the date with effect from which it is to be cancelled;

($b$) in any case where he does cancel an interim maintenance assessment, decide whether it is appropriate for a maintenance assessment to be made in accordance with the provisions of Part I of Schedule 1 to the Act;

($c$) in any case where he has decided that it is appropriate for a maintenance assessment to be made in accordance with the provisions of Part I of Schedule 1 to the Act, make that assessment.
\end{enumerate}

(4) Where a child support officer has made a decision under paragraph (3), he shall immediately notify the applicant, so far as that is reasonably practicable, and shall give the reasons for his decision in writing.

(5) A notification under paragraph (4) shall include information as to the provisions of sections 18 and 20 of the Act and regulation 24(1) and, where an assessment is made in accordance with the provisions of Part I of Schedule 1 to the Act, the provisions of sections 16 and 17 of the Act.

(6) Where a child support officer has made a decision following an application under paragraph (1), the absent parent may apply to the Secretary of State for a review of that decision and, subject to the modification set out in paragraph (7), the provisions of section 18(5) to (8) of the Act shall apply to such a review.

(7) The modification referred to in paragraph (6) is that section 18(6) of the Act shall have effect as if for “the refusal, assessment or cancellation in question” there is substituted “the decision following an application under regulation 9A(1) of the Child Support (Maintenance Assessment Procedure) Regulations 1992”.

(8) Regulations 10, 11, 24 and 25 shall apply to reviews under paragraph (6).”.
\end{quotation}

\subsection[18. Amendment of regulation 10 of the Maintenance Assessment Procedure Regulations]{Amendment of regulation 10 of the Maintenance Assessment Procedure Regulations}

18.—(1) Regulation 10 of the Maintenance Assessment Procedure Regulations (notifications) shall be amended in accordance with the following provisions of this regulation.

(2) At the beginning of sub-paragraph ($b$) of paragraph (1), there shall be inserted the words “makes a new interim maintenance assessment under section 12 of the Act or”, and after the words“regulation 8” in that paragraph there shall be added the words “or 9”.

(3) For sub-paragraph ($c$) of paragraph (2), there shall be substituted the following sub-paragraph—
\begin{quotation}
“($c$) the net and assessable income of the absent parent and, where relevant, the amount determined under regulation 9(1)($b$) of the Maintenance Assessments and Special Cases Regulations (housing costs);”.
\end{quotation}

(4) After sub-paragraph ($c$) of paragraph (2), there shall be inserted the following sub-paragraph—
\begin{quotation}
“($cc$) where relevant, the absent parent’s protected income level and the amount of the maintenance assessment before the adjustment in respect of protected income specified in paragraph 6(2) of Schedule 1 to the Act was carried out;”.
\end{quotation}

(5) For sub-paragraph ($d$) of paragraph (2), there shall be substituted the following sub-paragraph—
\begin{quotation}
“($d$) the net and assessable income of the parent with care, and, where relevant, an amount in relation to housing costs determined in the manner specified in regulation 10 of the Maintenance Assessments and Special Cases Regulations (calculation of exempt income of parent with care);”.
\end{quotation}

(6) After sub-paragraph ($f$) of paragraph (2), there shall be added the following sub-paragraph—
\begin{quotation}
“($h$)\footnote{\frenchspacing Where the provisions of Part II of the Schedule to S.I. 1992/2644 (C. 83) are applied a further item (sub-paragraph ($g$)) is to be included in paragraph (2) by virtue of paragraph 10 of that Schedule.} any amount determined in accordance with Schedule 3A or 3B to the Maintenance Assessments and Special Cases Regulations (qualifying transfer of property and travel costs).”.
\end{quotation}

\subsection[19. Amendment of regulation 12 of the Maintenance Assessment Procedure Regulations]{Amendment of regulation 12 of the Maintenance Assessment Procedure Regulations}

19.  For paragraph (1) of regulation 12 of the Maintenance Assessment Procedure Regulations (notification of refusal to make new or fresh maintenance assessment) there shall be substituted the following paragraph—
\begin{quotation}
“(1) Where a child support officer—
\begin{enumerate}\item[]
($a$) refuses an application for a maintenance assessment under the Act;

($b$) refuses to make a fresh assessment following a review under section 17 of the Act;

($c$) refuses to make an assessment or a fresh assessment following a review under section 18 of the Act; or

($d$) decides not to make a maintenance assessment or a fresh assessment under section 19 of the Act,
\end{enumerate}
he shall immediately notify the following persons, so far as that is reasonably practicable—
\begin{enumerate}\item[]
(i) where an application for a maintenance assessment under section 4 or 6 of the Act is refused, the applicant;

(ii) where an application under section 7 of the Act is refused, the applicant child and the other relevant persons who have been notified of the application;

(iii) where there is a refusal to make a fresh assessment following a review under section 17 or 18(2) of the Act, or a child support officer has decided not to make a fresh assessment following a review under section 19(1)($c$) of the Act, the relevant persons; or

(iv) where there is a refusal to make an assessment following a review under section 18(1)($a$) of the Act, or a child support officer has decided not to make an assessment following a review under section 19(1)($a$) of the Act, the applicant for that assessment,
\end{enumerate}
and shall give in writing the reasons for his refusal.”.
\end{quotation}

\subsection[20. Amendment of regulation 14 of the Maintenance Assessment Procedure Regulations]{Amendment of regulation 14 of the Maintenance Assessment Procedure Regulations}

20.—(1) Regulation 14 of the Maintenance Assessment Procedure Regulations (notification of cancellation of assessment) shall be amended in accordance with the following provisions of this regulation.

(2) In paragraph (1), for the words “regulation 9” there shall be substituted the words “regulation 9A”.

(3) In paragraph (2), for “31(8)” there shall be substituted “31A(8)”.

\subsection[21. Amendment of regulation 15 of the Maintenance Assessment Procedure Regulations]{Amendment of regulation 15 of the Maintenance Assessment Procedure Regulations}

21.—(1) Regulation 15 of the Maintenance Assessment Procedure Regulations (notification of refusal to reinstate assessment) shall be amended in accordance with the following provisions of this regulation.

(2) In paragraph (1) after the words “assessment that has been cancelled” there shall be inserted the words “or following a review under section 19(1)($d$) of the Act decides not to reinstate a cancelled maintenance assessment”.

(3) For paragraph (2) there shall be substituted the following paragraph—
\begin{quotation}
“(2) A notification under paragraph (1) shall, where the review is carried out under section 18(3) of the Act, include information as to the provisions of section 20 of the Act and, where the review is carried out under section 19(1)($d$) of the Act, except where that review is of the cancellation of a Category A or Category D interim maintenance assessment, the provisions of section 18 of the Act and regulations 24(1) and 31A(8).”.
\end{quotation}

\subsection[22. Insertion of regulation 15A into the Maintenance Assessment Procedure Regulations]{Insertion of regulation 15A into the Maintenance Assessment Procedure Regulations}

22.  After regulation 15 of the Maintenance Assessment Procedure Regulations, there shall be inserted the following regulation—
\begin{quotation}
\subsection*{“Notification of reinstatement of a maintenance assessment}

15A.—(1) Where a child support officer, following a review under section 18(3) or 19(1)($d$) of the Act, has decided that the cancellation of a maintenance assessment should be set aside, he shall immediately notify the relevant persons, so far as that is reasonably practicable, and shall give in writing reasons for the setting aside of the cancellation and, if applicable, the date with effect from which the maintenance assessment is reinstated.

(2) A notification under paragraph (1) shall, where the review is carried out under section 18(3) of the Act, include information as to the provisions of section 20 of the Act.”.
\end{quotation}

\subsection[23. Insertion of regulation 16A into the Maintenance Assessment Procedure Regulations]{Insertion of regulation 16A into the Maintenance Assessment Procedure Regulations}

23.  After regulation 16 of the Maintenance Assessment Procedure Regulations (notification of ceasing to be a child), there shall be inserted the following regulation—
\begin{quotation}
\subsection*{“Notification that an appeal has lapsed}

16A.  Where a case falls within section 20A(1) of the Act and the appeal that has been brought under section 20 of the Act lapses under the provisions of section 20A(2) of the Act a child support officer shall, so far as that is reasonably practicable, notify the relevant persons that that appeal has lapsed.”.
\end{quotation}

\subsection[24. Amendment of regulation 17 of the Maintenance Assessment Procedure Regulations]{Amendment of regulation 17 of the Maintenance Assessment Procedure Regulations}

24.—(1) Regulation 17 of the Maintenance Assessment Procedure Regulations (periodical reviews) shall be amended in accordance with the following provisions of this regulation.

(2) In paragraph (1), for the words “regulation 18(1)” there shall be substituted the words “regulation 18” and at the end of sub-paragraph ($c$), there shall be added the words “where before 22nd January 1996 a child support officer decided, in accordance with section 17(3) of the Act, to proceed with a review,”.

(3) For paragraph (2) there shall be substituted the following paragraph—
\begin{quotation}
“(2) Where a maintenance assessment in force is a fresh assessment, following—
\begin{enumerate}\item[]
($a$) a review under section 17 of the Act where, after 22nd January 1996, a child support officer decided, in accordance with section 17(3) of the Act, to proceed with that review; or

($b$) a review under section 18 or 19 of the Act,
\end{enumerate}
that assessment shall be reviewed by a child support officer under section 16 of the Act after it has been in force for a period of—
\begin{enumerate}\item[]
(i) in a case where the effective date of the assessment that has been reviewed was on or before 18 April 1994, 52 weeks;

(ii) in a case where the effective date of the assessment that has been reviewed was after 18th April 1994, 104 weeks,
\end{enumerate}
less, in either case, the period between the effective date of the assessment that has been reviewed and the effective date of the fresh assessment following that review.”.
\end{quotation}

(4) At the end of paragraph (5) there shall be added “and shall set out the possible consequences of failure to provide that information or evidence.”.

(5) In paragraph (7)($b$), the words “or 17” shall be omitted.

\subsection[25. Substitution of regulation 18 of the Maintenance Assessment Procedure Regulations]{Substitution of regulation 18 of the Maintenance Assessment Procedure Regulations}

25.  For regulation 18 of the Maintenance Assessment Procedure Regulations (review under section 17 treated as review under section 16), there shall be substituted the following regulation—
\begin{quotation}
\subsection*{“Review under section 16 of the Act to be substituted for review under section 17 of the Act}

18.  Where after 22nd January 1996 a child support officer considers that he is likely to be required under section 17(3) of the Act to make one or more fresh maintenance assessments if he conducts a review under that section and the application for that review was received by the Secretary of State not earlier than 8 weeks prior to the date upon which the next review of the maintenance assessment in force is due under the provisions of section 16 of the Act, the child support officer shall carry out a review under section 16 of the Act instead of the review under section 17 of the Act for which application has been made.”.
\end{quotation}

\subsection[26. Amendment of regulation 19 of the Maintenance Assessment Procedure Regulations]{Amendment of regulation 19 of the Maintenance Assessment Procedure Regulations}

26.—(1) Regulation 19 of the Maintenance Assessment Procedure Regulations (change of circumstances reviews) shall be amended in accordance with the following provisions of this regulation.

(2) For paragraphs (2) and (3) there shall be substituted the following paragraphs—
\begin{quotation}
“(2) Any application made under section 17 of the Act after 22nd January 1996 shall be in writing and shall give details of the change of circumstances in respect of which a review is sought.

(3) Where a child support officer conducts the review in respect of which notification has been given in accordance with paragraph (1), he shall take into account any information in relation to a change of circumstances notified to him in writing by a relevant person.”.
\end{quotation}

(3) Paragraphs (4) and (4A) shall be omitted.

\subsection[27. Amendment of regulation 20 of the Maintenance Assessment Procedure Regulations]{Amendment of regulation 20 of the Maintenance Assessment Procedure Regulations}

27.—(1) Regulation 20 of the Maintenance Assessment Procedure Regulations (fresh assessments on change of circumstances review), shall be amended in accordance with the following provisions of this regulation.

(2) In paragraph (1)—
\begin{enumerate}\item[]
($a$) after the words “who has completed a review” there shall be inserted the words “of an original assessment”;

($b$) for the words “fixed by the assessment” there shall be substituted the words “fixed by that assessment”;

($c$) the words “currently in force” shall be omitted; and

($d$) after the words “as a result of the review,” there shall be inserted the words “of that assessment”.
\end{enumerate}

(3) In paragraph (2)—
\begin{enumerate}\item[]
($a$) after the words “who has completed a review” there shall be inserted the words “of an original assessment”;

($b$) after the words “as a result of the review” there shall be inserted the words “of that assessment”; and

($c$) after the words “would apply to that” there shall be inserted the word “fresh”.
\end{enumerate}

(4) In paragraph (3)—
\begin{enumerate}\item[]
($a$) after the words “who has completed a review” there shall be inserted the words “of an original assessment”;

($b$) after the words “were a fresh assessment to be made as a result of the review” there shall be inserted the words “of that assessment”;

($c$) after the words “the children in respect of whom that” there shall be inserted the word “fresh”; and

($d$) after the words “if a fresh assessment were to be made as a result of the review” there shall be inserted the words “of that original assessment”.
\end{enumerate}

\subsection[28. Amendment of regulation 21 of the Maintenance Assessment Procedure Regulations]{Amendment of regulation 21 of the Maintenance Assessment Procedure Regulations}

\begin{sloppypar}\noindent
28.—(1) Regulation 21 of the Maintenance Assessment Procedure Regulations (change of circumstances reviews: special cases—regulation 22), shall be amended in accordance with the following provisions of this regulation.
\end{sloppypar}

(2) In paragraph (2)—
\begin{enumerate}\item[]
($a$) after the words “the aggregate amount of child support maintenance fixed by the” there shall be inserted the word “original”;

($b$) the words “currently in force” shall be omitted; and

($c$) after the words “as a result of the review” there shall be inserted the words “of those original assessments”.
\end{enumerate}

(3) In paragraph (3), for the words “each fresh assessment” there shall be substituted the words “a review of each original assessment”.

\subsection[29. Amendment of regulation 22 of the Maintenance Assessment Procedure Regulations]{Amendment of regulation 22 of the Maintenance Assessment Procedure Regulations}

29.  In paragraph (2) of regulation 22 of the Maintenance Assessment Procedure Regulations (change of circumstances reviews: special cases—regulation 23), for the words “each fresh assessment” there shall be substituted the words “a review of each original assessment”.

\subsection[30. Amendment of regulation 27 of the Maintenance Assessment Procedure Regulations]{Amendment of regulation 27 of the Maintenance Assessment Procedure Regulations}

30.  In regulation 27 of the Maintenance Assessment Procedure Regulations (review under section 18(1)($b$) of the Act), after the words “completed a review” there shall be inserted the words “of an original assessment as defined in section 17(1) of the Act”.

\subsection[31. Substitution of regulation 28 of the Maintenance Assessment Procedure Regulations]{Substitution of regulation 28 of the Maintenance Assessment Procedure Regulations}

31.  For regulation 28 of the Maintenance Assessment Procedure Regulations (reviews under section 19 of the Act), there shall be substituted the following regulation—
\begin{quotation}
\subsection*{“Reviews conducted under section 19(1)($b$) of the Act}

28.  Where a child support officer has completed a review under section 19(1)($b$) of the Act of an original assessment as defined in section 17(1) of the Act regulations 20 to 22 shall apply in relation to any fresh assessment following that review.”.
\end{quotation}

\subsection[32. Amendment of regulation 30 of the Maintenance Assessment Procedure Regulations]{Amendment of regulation 30 of the Maintenance Assessment Procedure Regulations}

32.—(1) Regulation 30 of the Maintenance Assessment Procedure Regulations (effective dates of new assessments) shall be amended in accordance with the following provisions of this regulation.

(2) In paragraph (1), for the words “regulation 8(3) (interim maintenance assessments)” there shall be substituted the words “regulations 8C (effective dates of interim maintenance assessments), 30A (effective dates in particular cases), 33(7) (maintenance periods)”.

(3) After paragraph (2)($b$), there shall be added the following sub-\hspace{0pt}paragraph—
\begin{quotation}
“($c$) in a case where the application for a maintenance assessment is an application in relation to which the provisions of regulation 3 have been applied, the date an effective maintenance application form is received by the Secretary of State.”.
\end{quotation}

\subsection[33. Insertion of regulation 30A into the Maintenance Assessment Procedure Regulations]{Insertion of regulation 30A into the Maintenance Assessment Procedure Regulations}

33.  After regulation 30 of the Maintenance Assessment Procedure Regulations, there shall be inserted the following regulation—
\begin{quotation}
\subsection*{“Effective dates of new maintenance assessments in particular cases}

30A.—(1) Subject to regulation 33(7), where a new maintenance assessment is made in accordance with Part I of Schedule 1 to the Act following an interim maintenance assessment which has ceased to have effect in the circumstances set out in regulation 8D(6), the effective date of that maintenance assessment shall be the date upon which that interim maintenance assessment ceased to have effect in accordance with that regulation.

(2) Where a child support officer receives the information or evidence to enable him to make a maintenance assessment calculated in accordance with Part I of Schedule 1 to the Act for a period prior to the date upon which an interim maintenance assessment has ceased to have effect in the circumstances set out in regulation 8D(6), that maintenance assessment shall, subject to regulation 33(7), have effect for the period from the date set by regulation 3(7) of the Maintenance Arrangements and Jurisdiction Regulations or regulation 30(2)($a$) or ($b$), as the case may be, to the effective date of the maintenance assessment referred to in paragraph (1).”.
\end{quotation}

\subsection[34. Substitution of regulation 31 of the Maintenance Assessment Procedure Regulations]{Substitution of regulation 31 of the Maintenance Assessment Procedure Regulations}

34.  For regulation 31 of the Maintenance Assessment Procedure Regulations (effective dates on review), there shall be substituted the following regulations—
\begin{quotation}
\subsection*{“Effective dates of maintenance assessments following a review under section 16 or 17 of the Act}

31.—(1) Subject to paragraph (2), where a fresh maintenance assessment is made following a review under section 16 of the Act, the effective date of that assessment shall be 104 weeks after the effective date of the previous assessment disregarding any previous assessment made following a review made under section 17 of the Act, where after 22nd January 1996 a child support officer decided, in accordance with section 17(3) of the Act, to proceed with a review, or under section, 18 or 19 of the Act or any interim maintenance assessment made under section 12(1)($b$) or ($c$) of the Act.

(2) Where a fresh maintenance assessment is made following a review under section 16 of the Act in the circumstances set out in regulation 18, the effective date of that fresh maintenance assessment shall be the date determined under paragraph (3).

(3) Subject to paragraphs (4), (5) and (6), where an application is made under section 17 of the Act for a review of a maintenance assessment in force, and a fresh maintenance assessment is made in accordance with the provisions of regulation 20, 21 or 22, the effective date of that assessment shall be the first day of the maintenance period in which the application is received.

(4) Where an application is made under section 17 of the Act for a review of a maintenance assessment in force following the death of a qualifying child and a fresh maintenance assessment is made in accordance with the provisions of regulation 20, 21 or 22, the effective date of that assessment shall be the first day of the maintenance period during the course of which that child died.

(5) Where a child support officer has carried out a review of an original assessment under section 17(4A) of the Act, the effective date of any fresh assessment made under section 17(6) of the Act shall be the date determined under paragraph (3).

(6) Where a fresh maintenance assessment is made under section 17(7) of the Act following a review of a subsequent assessment, the effective date of that fresh assessment shall be the effective date of that subsequent assessment.

\subsection*{Effective dates of maintenance assessments following a review under section 18 of the Act}

31A.—(1) Where, following a review under section 18(1)($a$) of the Act, a maintenance assessment is made following a refusal to make a maintenance assessment, the effective date of that assessment shall be the effective date of the assessment that would have been made if the application for a maintenance assessment had not been refused.

(2) Subject to paragraphs (3) to (6) and to regulation 31C, where an application is made under section 18(2) of the Act for a review of a maintenance assessment in force at the time of that application, the effective date of a fresh assessment (if one is made) following such a review shall be—
\begin{enumerate}\item[]
($a$) where the application is received by the Secretary of State within 28 days of the date of notification of that assessment, or on a later date but the Secretary of State is satisfied that there was unavoidable delay, the effective date as determined by the child support officer dealing with the review;

($b$) subject to sub-paragraph ($a$), where the application is received by the Secretary of State later than 28 days after the date of notification of that assessment, the first day of the maintenance period in which the application is received.
\end{enumerate}

(3) Subject to paragraph (5), where an application is made under section 18(2) of the Act for a review of a maintenance assessment in force following notification being given to the relevant person that the child support officer does not propose to review the assessment in consequence of the coming into force of the provisions mentioned in paragraph (4), the effective date of a fresh assessment (if one is made) following such a review shall be—
\begin{enumerate}\item[]
($a$) where the application is received within 28 days of the Secretary of State notifying the relevant person of the child support officer’s decision, or on a later date where the Secretary of State is satisfied that there was unavoidable delay, the effective date as determined by the child support officer dealing with the review;

($b$) subject to sub-paragraph ($a$), where the application is received by the Secretary of State later than 28 days after the date of the notification of the child support officer’s decision, the first day of the maintenance period in which the application is received.
\end{enumerate}

(4) Paragraph (3) applies to the following provisions of the Child Support and Income Support (Amendment) Regulations 1995—
\begin{enumerate}\item[]
($a$) regulation 44(2);

($b$) regulation 45;

($c$) regulation 46(2)($d$) and ($e$);

($d$) regulation 51.
\end{enumerate}

(5) Where the application made under section 18(2) of the Act is made following notification being given to the relevant person that the child support officer has determined that the amount to be allowed in the computation of the relevant person’s exempt income in accordance with Schedule 3A to the Maintenance Assessments and Special Cases Regulations is nil by reason of the failure of the relevant person to furnish within a reasonable time the evidence required by paragraph 2 of that Schedule—
\begin{enumerate}\item[]
($a$) where the Secretary of State is satisfied that there was good cause for the delay in furnishing the evidence the effective date of any assessment made in consequence of the review shall be the effective date which would have been applicable to the assessment had the evidence been furnished timeously;

($b$) where the Secretary of State is not satisfied that there was good cause for the delay, the effective date of any revised assessment shall be the first day of the maintenance period in which the relevant person provides that evidence.
\end{enumerate}

(6) The effective date of any fresh maintenance assessment, made following a review under section 18(6A) of the Act of a maintenance assessment made after the original assessment, shall be the effective date of the maintenance assessment which has been reviewed.

(7) Where, an application is made under section 18(1)($b$) of the Act, for a review of a refusal of an application under section 17 of the Act for the review of a maintenance assessment, the effective date of a fresh maintenance assessment (if one is made) shall be the date determined under regulation 31(3).

(8) Where, following a review under section 18(3) of the Act, a cancelled maintenance assessment is reinstated, the effective date of the reinstated assessment shall be the date on which the cancelled assessment ceased to have effect.

\subsection*{Effective dates of maintenance assessments following a review under section 19 of the Act}

31B.—(1) Where a maintenance assessment is made following a review under section 19(1)($a$) of the Act of a refusal to make a maintenance assessment, the effective date of that maintenance assessment shall be the date determined under regulation 31A(1).

(2) Where a fresh maintenance assessment is made, following a review under section 19(1)($b$) of the Act of a refusal of an application under section 17 of the Act for review of a maintenance assessment, the effective date of that fresh maintenance assessment shall be the date determined under regulation 31(3) to (6).

(3) Subject to paragraph (5) and regulation 31C, where a child support officer has carried out a review of a maintenance assessment on the grounds set out in section 19(2) of the Act, the effective date of any fresh maintenance assessment made following that review shall be the effective date as determined by the child support officer dealing with the review.

(4) Subject to paragraph (5) and regulation 31C, where a child support officer has carried out a review of a maintenance assessment on the grounds set out in section 19(6) of the Act, the effective date of any fresh assessment made following such review shall be the first day of the maintenance period in which the child support officer suspected that he might be required to make one or more fresh maintenance assessments if an application under section 17 of the Act were made.

(5) Where a fresh maintenance assessment is made under section 19 of the Act following the death of a qualifying child, the effective date of that assessment shall be the first day of the maintenance period during which that child died.

\subsection*{Provisions as to effective dates of maintenance assessments in specific cases}

31C.—(1) Where there has been a misrepresentation or failure to disclose a material fact on the part of the person with care or absent parent in connection with an application for a maintenance assessment under the Act, a review under section 16 of the Act, or with information or evidence requested by a child support officer on a review under section 17, 18 or 19 of the Act and that misrepresentation or failure has resulted in an incorrect assessment or a series of incorrect assessments, the effective date of a fresh assessment (or of a fresh assessment in relation to the earliest relevant period) following discovery of the misrepresentation or failure shall be the effective date of the incorrect assessment or the first incorrect assessment, as the case may be.

(2) Where a fresh maintenance assessment is made on a review under section 18 or 19 of the Act by reason of an assessment having been made in ignorance of a material fact or having been based on a mistake as to a material fact and that ignorance or mistake, as the case may be, is attributable to an operational or administrative error on the part of the Secretary of State or of a child support officer, the effective date of that assessment shall be the effective date of the assessment that has been reviewed.

(3) Where a child support officer on a review under section 18 or 19 of the Act is satisfied that a maintenance assessment which is or has been in force is defective by reason of a mistake as to the effective date of that assessment, the effective date of a fresh assessment shall be that determined in accordance with paragraph (1) or (2), regulations 8C(1), 30 to 31B, 33(7), or in accordance with regulation 3(5), (7) or (8) of the Maintenance Arrangements and Jurisdiction Regulations, as the case may be.”.
\end{quotation}

\subsection[35. Insertion of regulation 32B into the Maintenance Assessment Procedure Regulations]{Insertion of regulation 32B into the Maintenance Assessment Procedure Regulations}

35.—(1) After regulation 32 of the Maintenance Assessment Procedure Regulations (cancellation), there shall be inserted the following regulation—
\begin{quotation}
\subsection*{“Notification of intention to cancel a maintenance assessment under paragraph 16(4A) of Schedule 1 to the Act}

32B.—(1) A child support officer shall, if it is reasonably practicable to do so, give written notice to the relevant persons of his intention to cancel a maintenance assessment under paragraph 16(4A) of Schedule 1 to the Act.

(2) Where a notice under paragraph (1) has been given, a child support officer shall not cancel that maintenance assessment before the end of a period of 14 days commencing with the date that notice was given or sent.”.
\end{quotation}

\subsection[36. Amendment of regulation 33 of the Maintenance Assessment Procedure Regulations]{Amendment of regulation 33 of the Maintenance Assessment Procedure Regulations}

36.—(1) Regulation 33 of the Maintenance Assessment Procedure Regulations shall be amended in accordance with the following provisions of this regulation.

(2) In paragraph (6), after the words “the earlier maintenance assessment,” there shall be inserted the words “except where regulation 3(7) of the Maintenance Arrangements and Jurisdiction Regulations or paragraph (8) applies,”.

(3) After paragraph (6) of regulation 33 of the Maintenance Assessment Procedure Regulations (maintenance periods), there shall be added the following paragraphs—
\begin{quotation}
“(7) Subject to regulation 3(7) of the Maintenance Arrangements and Jurisdiction Regulations and to paragraph (8), the effective date of a maintenance assessment made in response to an application falling within paragraph (6) shall be the date upon which the first maintenance period in relation to that application commences in accordance with that paragraph.

(8) The first maintenance period in relation to a maintenance assessment which is made in response to an application falling within paragraph (6) and which immediately follows an interim maintenance assessment shall commence on the effective date of that interim maintenance assessment or 22nd January 1996 whichever is the later, and the effective date of that maintenance assessment shall be the date upon which that first maintenance period commences.”.
\end{quotation}

\subsection[37. Insertion of regulation 35A into the Maintenance Assessment Procedure Regulations]{Insertion of regulation 35A into the Maintenance Assessment Procedure Regulations}

37.  After regulation 35 of the Maintenance Assessment Procedure Regulations (periods for compliance), there shall be inserted the following regulation—
\begin{quotation}
\subsection*{“Circumstances in which a reduced benefit direction shall not be given}

35A.  A child support officer shall not after 22nd January 1996 give a reduced benefit direction where—
\begin{enumerate}\item[]
($a$) income support is paid to or in respect of the parent in question and the applicable amount of the claimant for income support includes one or more of the amounts set out in paragraph 15(3), (4) or (6) of Part IV of Schedule 2 to the Income Support (General) Regulations 1987\footnote{\frenchspacing S.I. 1987/1967. Part IV of Schedule 2 was substituted by S.I. 1995/559.}; or

($b$) an amount equal to one or more of the amounts specified in sub-paragraph ($a$) is included, by virtue of regulation 9 of the Maintenance Assessments and Special Cases Regulations, in the exempt income of the parent in question and family credit or disability working allowance is paid to or in respect of that parent.”.
\end{enumerate}
\end{quotation}

\subsection[38. Insertion of regulation 40A into the Maintenance Assessment Procedure Regulations]{Insertion of regulation 40A into the Maintenance Assessment Procedure Regulations}

38.  After regulation 40 of the Maintenance Assessment Procedure Regulations (suspension), there shall be inserted the following regulation—
\begin{quotation}
\subsection*{“Suspension of a reduced benefit direction where certain deductions are being made from income support}

40A.—(1) A reduced benefit direction made after 22nd January 1996 shall be suspended where, on the date it is given, one or more of the deductions specified in paragraph (2) are being made from income support paid to or in respect of the parent concerned.

(2) The deductions relevant for the purposes of paragraph (1) are—
\begin{enumerate}\item[]
(i) deductions in respect of arrears of housing costs, fuel or water charges under paragraph 3, 5, 6 or 7 of Schedule 9 to the Social Security (Claims and Payments) Regulations 1987\footnote{\frenchspacing S.I. 1987/1968. Paragraph 3 of Schedule 9 was amended by S.I. 1988/522, S.I. 1992/1026 and S.I. 1992/2595 and paragraph 5 by S.I. 1988/522, S.I. 1991/2284 and S.I. 1992/2595. Paragraph 6 was amended by S.I. 1988/522, S.I. 1991/2284, S.I. 1992/2595 and S.I. 1994/2319. Paragraph 7 was amended by S.I. 1992/2595 and S.I. 1994/2319.};

(ii) deductions in respect of overpaid benefit under regulation 15, 16 or 17 of the Social Security (Payments on Account, Overpayments and Recovery) Regulations 1988\footnote{\frenchspacing S.I. 1988/664. Regulations 15, 16 and 17 were amended by S.I. 1988/688 and S.I. 1991/2742.};

(iii) deductions in respect of arrears of Community Charge liability under regulation 2 or 4 of the Community Charges (Deductions from Income Support) (No.\ 2) Regulations 1990\footnote{\frenchspacing S.I. 1990/545. Regulation 2 was amended by S.I. 1992/1026 and S.I. 1993/2113.};

(iv) deductions in respect of arrears of Council Tax liability under regulation 5 or 7 of the Council Tax (Deductions from Income Support) Regulations 1993\footnote{\frenchspacing S.I. 1993/494.};

(v) deductions in respect of fines under regulation 4 of the Fines (Deductions from Income Support) Regulations 1992\footnote{\frenchspacing S.I. 1992/2182. Regulation 4 was substituted by S.I. 1993/495.};

(vi) deductions in respect of social fund awards under section 78(1) to (3) of the Social Security Administration Act 1992\footnote{\frenchspacing 1992 c. 5.}.
\end{enumerate}

(3) When income support payable to or in respect of the parent concerned is no longer subject to the deductions relevant for the purposes of paragraph (1), the reduced benefit direction shall cease to be suspended at the end of a period of 14 days after notification has been served under regulation 49A.”.
\end{quotation}

\subsection[39. Insertion of regulation 49A into the Maintenance Assessment Procedure Regulations]{Insertion of regulation 49A into the Maintenance Assessment Procedure Regulations}

39.  After regulation 49 of the Maintenance Assessment Procedure Regulations (notice of termination), there shall be inserted the following regulation—
\begin{quotation}
\subsection*{“Notice of termination of suspension of a reduced benefit direction}

49A.—(1) Where the deductions relevant for the purposes of regulation 40A cease to be made, a child support officer shall, so far as is reasonably practicable, serve on the parent concerned notice of the date from which the suspension of the reduced benefit direction shall cease.

(2) The adjudication officer shall be served with a copy of any notice served under paragraph (1).”.
\end{quotation}

\subsection[40. Amendment of regulation 1 of the Maintenance Assessments and Special Cases Regulations]{Amendment of regulation 1 of the Maintenance Assessments and Special Cases Regulations}

40.—(1) Regulation 1 of the Maintenance Assessments and Special Cases Regulations (citation, commencement and interpretation) shall be amended in accordance with the following provisions of this regulation.

(2) In paragraph (2)—
\begin{enumerate}\item[]
($a$) for sub-paragraph (ii) in the definition of “day to day care” there shall be substituted the following sub-paragraphs—
\begin{quotation}
“(ii) in relation to an application for child support maintenance, “relevant week” shall have the meaning ascribed to it in head (ii) of sub-paragraph ($a$) of the definition of “relevant week” in this paragraph;

(iii) in relation to a review of a maintenance assessment under section 16 of the Act “relevant week” means the period of 7 days immediately preceding whichever is the later of the date on which a request is made to an absent parent or to a person with care for information or evidence under regulation 17(5) of the Maintenance Assessment Procedure Regulations; or

(iv) in relation to a review under section 17, 18(1)($a$), (1)($b$), (2) or (6A) or 19(1)($a$) to ($c$) or (6) of the Act, “relevant week” shall have the meaning ascribed to it in sub-paragraph ($a$), ($c$), ($d$), ($e$) or ($f$), as the case may be, of the definition of “relevant week” in this paragraph.”;
\end{quotation}

($b$) in sub-paragraph ($a$) in the definition of “relevant week”, after the words “in relation to an application for child support maintenance” there shall be inserted the words “or a review under section 18(1)($a$) or 19(1)($a$) of the Act”;

($c$) for sub-paragraph ($b$) in the definition of “relevant week”, there shall be substituted the following sub-paragraphs—
\begin{quotation}
“($b$) in relation to a review of an assessment under section 16 of the Act, the period of 7 days immediately preceding the date on which a request for information or evidence under regulation 17(5) of the Maintenance Assessment Procedure Regulations is made;

($c$) in relation to a review under section 17 of the Act, the period of 7 days immediately preceding the date on which the application for review is received by the Secretary of State;

($d$) in relation to a review under section 18(1)($b$) or 19(1)($b$) of the Act, the period of 7 days immediately preceding the date on which application for the review under section 17 of the Act was received by the Secretary of State;

($e$) in relation to a review under section18(2), (6A) or 19(1)($c$) of the Act, the relevant week which was applicable for the purposes of the making of the maintenance assessment which is being reviewed; or

($f$) in relation to a review under section 19(6) of the Act, the period of 7 days immediately preceding the date on which, in the circumstances referred to in that sub-section, the child support officer first suspected that it would be appropriate to make one or more fresh assessments.”.
\end{quotation}
\end{enumerate}

(3) In paragraph (2A), after the word “week” in head (ii) of sub-paragraph ($e$) there shall be inserted the words “but no deduction shall be made in respect of the portion (if any) of the bonus or commission which, if added to estimated income, would cause such income to exceed the upper earnings limit for Class 1 contributions as provided for in section 5(1)($b$) of the Contributions and Benefits Act”.

\subsection[41. Amendment of regulation 2 of the Maintenance Assessments and Special Cases Regulations]{Amendment of regulation 2 of the Maintenance Assessments and Special Cases Regulations}

41.  In paragraph (2) of regulation 2 of the Maintenance Assessments and Special Cases Regulations (calculation or estimation of amounts), for the words “regulation 8(2C)” there shall be substituted the words “regulation 8A(4)”.

\subsection[42. Amendment of regulation 9 of the Maintenance Assessments and Special Cases Regulations]{Amendment of regulation 9 of the Maintenance Assessments and Special Cases Regulations}

42.  In sub-paragraph ($h$) of paragraph (1) of regulation 9 of the Maintenance Assessments and Special Cases Regulations (exempt income), after the words “that home” there shall be added the words “but where a local authority has determined that the absent parent in question or his partner is entitled to housing benefit in respect of fees for that accommodation or that home, the net amount of such fees after deduction of housing benefit”.

\subsection[43. Amendment of regulation 11 of the Maintenance Assessments and Special Cases Regulations]{Amendment of regulation 11 of the Maintenance Assessments and Special Cases Regulations}

43.—(1) Regulation 11 of the Maintenance Assessments and Special Cases Regulations (protected income) shall be amended in accordance with the following provisions of this regulation.

\begin{sloppypar}
(2) In sub-paragraph ($b$) of paragraph (1), for the words “regulation 15(10)($a$)” there shall be substituted the words “regulation 15(4)”.
\end{sloppypar}

(3) In sub-paragraph ($i$) of paragraph (1), after the words “that home” there shall be added the words “but where housing benefit is paid to the absent parent in question or his partner in respect of fees for that accommodation or that home the net amount of such fees after deduction of housing benefit”.

(4) The word “and” at the end of paragraph (2)($a$)(iii) shall be omitted.

(5) After head (iii) of sub-paragraph ($a$) of paragraph (2), there shall be added the following heads—
\begin{quotation}
“(iv) paragraph 27 of Schedule 2 shall apply as though the reference to paragraph 3(2) and (4) of Schedule 3 were omitted;

(v) there shall be deducted the amount of any maintenance which is being paid in respect of a child by the absent parent or his partner under an order requiring such payment made by a court outside Great Britain; and”.
\end{quotation}

\subsection[44. Amendment of regulation 15 of the Maintenance Assessments and Special Cases Regulations]{Amendment of regulation 15 of the Maintenance Assessments and Special Cases Regulations}

44.  Regulation 15 of the Maintenance Assessments and Special Cases Regulations (amount of housing costs) shall be amended by substituting “(4)” for “(10)” at the beginning of paragraph (10).

\subsection[45. Amendment of regulation 22 of the Maintenance Assessments and Special Cases Regulations]{Amendment of regulation 22 of the Maintenance Assessments and Special Cases Regulations}

45.—(1) Regulation 22 of the Maintenance Assessments and Special Cases Regulations (multiple applications relating to an absent parent) shall be amended in accordance with the following provisions of this regulation.

(2) For paragraph (1) there shall be substituted the following paragraph—
\begin{quotation}
“(1) Where an application for a maintenance assessment has been made in respect of an absent parent and—
\begin{enumerate}\item[]
($a$) at least one other application for a maintenance assessment has been made in relation to the same absent parent (or a person who is treated as an absent parent by regulation 20(2)) but to different children; or

($b$) at least one maintenance assessment is in force in relation to the same absent parent or a person who is treated as an absent parent by regulation 20(2) but to different children,
\end{enumerate}
that case shall be treated as a special case for the purposes of the Act.”.
\end{quotation}

(3) In paragraph (2), for the words “paragraph (1)” there shall be substituted the words “paragraph (1)($a$)” and after the word “applies” there shall be inserted the words “or in respect of the application made in circumstances where paragraph (1)($b$) applies”.

(4) After paragraph (2) there shall be inserted the following paragraph—
\begin{quotation}
“(2A) Where paragraph (1)($b$) applies, and a maintenance assessment has been made in respect of the application referred to in paragraph (1), each maintenance assessment in force at the time of that assessment shall be reduced using the formula for calculation of assessable income set out in paragraph (2) and each reduction shall take effect on the date specified in regulation 33(7) of the Maintenance Assessment Procedure Regulations.”.
\end{quotation}

\subsection[46. Amendment of Schedule 2 to the Maintenance Assessments and Special Cases Regulations]{Amendment of Schedule 2 to the Maintenance Assessments and Special Cases Regulations}

46.  In paragraph 27 of Schedule 2 to the Maintenance Assessments and Special Cases Regulations (disregards), for the words “the total of—” to the end of that paragraph there shall be substituted the words “the total of the amount of the payments set out in paragraphs 1($b$), 3(2) and (4) of Schedule 3 as modified, where applicable, by regulation 18.”.

\subsection[47. Amendment of Schedule 3 to the Maintenance Assessments and Special Cases Regulations]{Amendment of Schedule 3 to the Maintenance Assessments and Special Cases Regulations}

47.—(1) Schedule 3 to the Maintenance Assessments and Special Cases Regulations (eligible housing costs) shall be amended in accordance with the following provisions of this regulation.

(2) After sub-paragraph (2) of paragraph 3, there shall be inserted the following sub-paragraph—
\begin{quotation}
“(2A) Where an absent parent or as the case may be a parent with care has entered into a loan for repairs or improvements of a kind referred to in paragraph 1($d$) and that parent makes periodical payments of an amount provided for in accordance with the terms of that loan to reduce the amount of that loan, the amount of those payments shall be eligible to be taken into account as housing costs of that parent.”.
\end{quotation}

(3) Paragraph 6 shall be amended as follows—
\begin{enumerate}\item[]
(i) for sub-paragraph ($a$) there shall be substituted the following sub-paragraph—
\begin{quotation}
\begin{sloppypar}
“($a$) where the costs are inclusive of ineligible service charges within the meaning of paragraph 1($a$)(i) of Schedule 1 to the Housing Benefit (General) Regulations 1987\footnote{\frenchspacing S.I. 1987/1971.} (ineligible service charges), the amounts specified as ineligible in paragraph 1A of that Schedule;”;
\end{sloppypar}
\end{quotation}

(ii) sub-paragraph ($aa$) shall be omitted;

(iii) in sub-paragraph ($b$) the word “and” shall be omitted;

(iv) after sub-paragraph ($c$) there shall be added the following—
\begin{quotation}
“and

($d$) where the costs are inclusive of charges, other than those which are not to be included by virtue of sub-paragraphs ($a$) to ($c$), that part of those charges which exceeds the greater of the following amounts—
\begin{enumerate}\item[]
(i) the total of the charges other than those which are ineligible service charges within the meaning of paragraph 1 of Schedule 1 to the Housing Benefit Regulations (housing costs);

(ii) 25 per centum of the total amount of eligible housing costs,
\end{enumerate}
and for the purposes of this sub-paragraph, where the amount of those charges is not separately identifiable, that amount shall be such amount as is reasonably attributable to those charges.”.
\end{quotation}

\end{enumerate}

\subsection[48. Amendment of Schedule 3A to the Maintenance Assessments and Special Cases Regulations]{Amendment of Schedule 3A to the Maintenance Assessments and Special Cases Regulations}

48.—(1) In paragraph 8 of Schedule 3A to the Maintenance Assessments and Special Cases Regulations (compensating transfers), for the words “The value of” there shall be substituted the words “Subject to paragraph 8A, the value of”.

(2) After paragraph 8, there shall be inserted the following paragraph—
\begin{quotation}
“8A.—(1) This paragraph applies where—
\begin{enumerate}\item[]
($a$) the property which is the subject of a compensating transfer is or includes cash or deposits as defined in paragraph 5(i);

($b$) that property was acquired by the parent with care after the relevant date;

($c$) the absent parent has no legal interest in that property;

($d$) if that property is or includes cash obtained by a mortgage or charge, that mortgage or charge was executed by the parent with care after the relevant date and was of property to the whole of which she is legally entitled; and

($e$) the effect of the compensating transfer is that the parent with care or a relevant child is beneficially entitled (subject to any mortgage or charge) to the whole of the absent parent’s legal estate in the land which is the subject of the qualifying transfer.
\end{enumerate}

(2) Where sub-paragraph (1) applies, the qualifying value of the compensating transfer shall be the amount of the cash or deposits transferred pursuant to the court order or written maintenance agreement referred to in head ($a$) of the definition of “qualifying transfer” in paragraph 1(1).”.
\end{quotation}

\subsection[49. Amendment of Schedule 3B to the Maintenance Assessments and Special Cases Regulations]{Amendment of Schedule 3B to the Maintenance Assessments and Special Cases Regulations}

49.  For paragraph 17 of Schedule 3B to the Maintenance Assessments and Special Cases Regulations (travelling costs) there shall be substituted the following paragraphs—
\begin{quotation}
“17.  Subject to paragraph 17A, there shall be calculated, or if that is impracticable estimated, for each pair of places referred to in paragraph 16 between which straight-line distances are required to be calculated or estimated, the number of journeys which the relevant person makes between them during a period comprising a whole number of weeks which appears to the child support officer to be representative of the normal working pattern of the relevant person.

\medskip

17A.  For the purposes of the calculation required by paragraph 17, there shall be disregarded—
\begin{enumerate}\item[]
($a$) any pair of journeys between the same work place and his home where the first journey is from his work place to his home and the time which elapses between the start of the first journey and the conclusion of the second is not more than two hours; and

($b$) any journey in respect of which—
\begin{enumerate}\item[]
(i) the travelling costs are borne wholly or in part by the relevant employer; or

(ii) the relevant employer provides transport for any part of the journey for the use of the relevant person.”.
\end{enumerate}
\end{enumerate}
\end{quotation}

\subsection[50. Amendment of regulation 64 of the Child Support Amendment Regulations]{Amendment of regulation 64 of the Child Support Amendment Regulations}

50.—(1) Regulation 64 of the Child Support Amendment Regulations (transitional provisions) shall be amended in accordance with the following provisions of this regulation.

(2) In paragraph (1) for the words “or on that date there is in force a decision of a child support officer under section 43 of the Act (contribution to maintenance by deduction from benefit) and that decision or” there shall be substituted the word “and”.

(3) In paragraph (2), sub-paragraph ($g$) shall be omitted.

\subsection[51. Amendment of regulation 7 of the Miscellaneous Amendments Regulations]{Amendment of regulation 7 of the Miscellaneous Amendments Regulations}

51.—(1) Regulation 7 of the Miscellaneous Amendments Regulations (scope) shall be amended in accordance with the following provisions of this regulation.

(2) In sub-paragraph ($a$) of paragraph (2), for the words “regulation 8(1B)” there shall be substituted the words “regulation 8(3)” and the word “or” shall be omitted.

(3) At the end of sub-paragraph ($b$) of paragraph (2), there shall be added the word “or”.

(4) After sub-paragraph ($b$) of paragraph (2), there shall be added the following sub-paragraph—
\begin{quotation}
“($c$) a maintenance assessment calculated in accordance with Part I of Schedule 1 to the Act which is made following a Category A or Category D interim maintenance assessment within the meaning of regulation 8 of the Procedure Regulations where that Category A or Category D interim maintenance assessment is made after 22nd January 1996.”.
\end{quotation}

\subsection[52. Amendment of regulation 11 of the Miscellaneous Amendments Regulations]{\sloppy Amendment of regulation 11 of the Miscellaneous Amendments Regulations}

52.—(1) Regulation 11 of the Miscellaneous Amendments Regulations (reviews) shall be amended in accordance with the following provisions of this regulation.

(2) In paragraph (1), after the words “under that assessment is” there shall be inserted the words “or was”.

(3) In paragraph (4), for “regulation 31” there shall be substituted “regulations 31 to 31C”.

\subsection[53. Amendment of regulation 2 of the Northern Ireland Regulations]{\sloppy Amendment of regulation 2 of the Northern Ireland Regulations}

53.  In paragraph (1) of regulation 2 of the Northern Ireland Regulations (adaptation), after the words “Schedule 1” there shall be inserted the words “as amended by the Exchange of Letters set out in Schedule 1A”.

\subsection[54. Insertion of Schedule 1A into the Northern Ireland Regulations]{Insertion of Schedule 1A into the Northern Ireland Regulations}

54.  After Schedule 1 to the Northern Ireland Regulations (Memorandum), there shall be inserted, as Schedule 1A, the Schedule set out in the Schedule to these Regulations.

\subsection[55. Amendment of Schedule 2 to the Northern Ireland Regulations]{Amendment of Schedule 2 to the Northern Ireland Regulations}

55.  In Schedule 2 to the Northern Ireland Regulations (adaptation), after the entry relating to section 15 there shall be inserted the following entry—
\begin{quotation}
\noindent
\begin{tabular}{lll}
\hline
“Sections 27 and 28 & Article 28 & Declaration of parentage”.\\
\hline
\end{tabular}
\end{quotation}

\subsection[56. Reviews consequent on amendments made by these Regulations]{\sloppy Reviews consequent on amendments made by these Regulations}

56.  Where a child support officer makes a fresh assessment following a review under section 19(6) of the Act in consequence of the coming into force of regulation 48, the effective date of that fresh assessment shall be the first day of the maintenance period following 18th December 1995 and regulations 20 to 23, 27 and 28 of the Maintenance Assessment Procedure Regulations shall not apply to that assessment.

\subsection[57. Transitional and consequential provisions]{Transitional and consequential provisions}

57.—(1) The provisions set out in paragraph (2) shall not apply to a maintenance assessment in force on January 1996 until it is reviewed under section 16, 17 or 18 of the Act.

(2) The provisions referred to in paragraph (1) are—
\begin{enumerate}\item[]
($a$) regulation 40(3);

($b$) regulation 42;

($c$) regulation 43(3);

($d$) head (v) of sub-paragraph ($a$) of paragraph (2) of regulation 11 of the Maintenance Assessments and Special Cases Regulations as inserted by regulation 43(5) of these Regulations;

($e$) regulation 46; and

($f$) regulation 47.
\end{enumerate}

(3) Where a review is carried out wholly or partly in consequence of one or more of the provisions set out in 
%regulation 39, 41, 42(3) or (5), 45, 46 or 48, 
regulation 40, 42, 43(3) or (5), 46, 47 or 49,  % Words substituted (22.1.96) by SI 1995/3265 reg 4
and the amount of any fresh assessment made following that review is different from the amount of any fresh assessment that would have been made had those provisions not been in force, the effective date of that fresh assessment shall not be earlier than 22nd January 1996.

(4) The provisions of paragraph (2) inserted into regulation 19 of the Maintenance Assessment Procedure Regulations by regulation 26(2) of these Regulations and the provisions of paragraph (3) of regulation 26 shall not apply to a review which before 22nd January 1996, a child support officer has decided to conduct.

\amendment{
Words substituted in reg. 57(3) (22.1.96) by the Child Support (Miscellaneous Amendments) (No. 3) Regulations 1995 reg. 4.
}

\bigskip

Signed by authority of the Secretary of State for Social Security.

{\raggedleft
\emph{A. J. B. Mitchell}\\*Parliamentary Under-Secretary of State,\\*Department of Social Security

}

15th December 1995

\clearpage

\part[Schedule --- Schedule to be inserted into the Northern Ireland Regulations as Schedule 1A to those Regulations]{S C H E D U L E\\*Schedule to be inserted into the Northern Ireland Regulations as Schedule 1A to those Regulations}

\renewcommand\parthead{--- Schedule}

\begin{quotation}
\part*{Schedule 1A\\*Exchange of letters amending the Memorandum of Arrangements relating to the provision made for child support maintenance in the United Kingdom}

\section*{\sloppy No. 1\\*The Secretary of State for Social Security and the Department of Health and Social Services for Northern Ireland}

7th November 1995

  Sir,

  I have the honour to refer to the Memorandum of Arrangements relating to the provision made for Child Support Maintenance between the Secretary of State for Social Security of the one part and the Department of Health and Social Services for Northern Ireland of the other part which came in to effect on 5 April 1993 (which in this letter is referred to as “the Principal Memorandum”) and to recent discussions between the Department of Social Security and the Department of Health and Social Services for Northern Ireland concerning the need to amend the Principal Memorandum so as to make further provision in relation to child support matters.

  I now have the honour to propose the following amendments to the Principal Memorandum:

  After paragraph (4) of Article 5 there shall be inserted:—
\begin{quotation}
 “(5) Subject to paragraph (7), where an application for a maintenance assessment is made under the provisions for one territory in relation to an absent parent, a person treated as such, or an alleged absent parent who resides in the other territory, that application shall be dealt with in, and in accordance with the provision made for, the territory in which the person with care resides.

(6) Subject to paragraph (7), where an application for a maintenance assessment is made under section 7 of the Act by a qualifying child, the application shall be dealt with in, and in accordance with the provision made for, the territory in which the person with care of that child resides.

(7) Where paragraphs (5) and (6) apply, the determining authority shall, in determining the amount of child support maintenance to be fixed by any maintenance assessment, take into account in calculating that amount, any provisions which would otherwise have been applicable to that calculation had the assessment been made in accordance with the provision made for the other territory.”.
\end{quotation}

  After Part VI there shall be inserted the following Part:—
\begin{quotation}
 \part*{“Part VIA\\Parentage}

12A.  Where a person with care resides in one territory and an alleged parent who denies that he is one of the parents of a child in respect of whom an application for a maintenance assessment has been made resides in the other territory:—
\begin{enumerate}\item[]
($a$) The person with care or the Secretary of State may apply for a declaration as to whether or not the alleged parent is one of the child’s parents, under Article 28 of the Order;

($b$) The person with care or the Department of Health and Social Services may apply for such a declaration under section 27 of the Act; and

($c$) The Department of Health and Social Services may bring an action for declarator of parentage under the provisions of section 28 of the Act.”.
\end{enumerate}
\end{quotation}

  If the foregoing proposals are acceptable to you, I have the honour to propose that this letter and your reply to that effect shall constitute a Memorandum of Arrangements between us which shall come into effect on 21st January 1996.

  \emph{Andrew Mitchell}

  For the Secretary of State for Social Security

\section*{\sloppy No. 2\\*The Department of Health and Social Services for Northern Ireland to the Secretary of State for Social Security}

8th November 1995

  Sir

  I refer to your letter of 7th November 1995 which reads as follows:

\begin{quotation}
  “I have the honour to refer to the Memorandum of Arrangements relating to the provision made for Child Support Maintenance between the Secretary of State for Social Security of the one part and the Department of Health and Social Services for Northern Ireland of the other part which came in to effect on 5 April 1993 (which in this letter is referred to as “the Principal Memorandum”) and to recent discussions between the Department of Social Security and the Department of Health and Social Services for Northern Ireland concerning the need to amend the Principal Memorandum so as to make further provision in relation to child support matters.

\begin{sloppypar}
  I now have the honour to propose the following amendments to the Principal Memorandum:
\end{sloppypar}

\begin{sloppypar}
  After paragraph (4) of Article 5 there shall be inserted:—
\end{sloppypar}
\begin{quotation}
 “(5) Subject to paragraph (7), where an application for a maintenance assessment is made under the provisions for one territory in relation to an absent parent, a person treated as such, or an alleged absent parent who resides in the other territory, that application shall be dealt with in, and in accordance with the provision made for, the territory in which the person with care resides.

(6) Subject to paragraph (7), where an application for a maintenance assessment is made under section 7 of the Act by a qualifying child, the application shall be dealt with in, and in accordance with the provision made for, the territory in which the person with care of that child resides.

\begin{sloppypar}
(7) Where paragraphs (5) and (6) apply, the determining authority shall, in determining the amount of child support maintenance to be fixed by any maintenance assessment, take into account in calculating that amount, any provisions which would otherwise have been applicable to that calculation had the assessment been made in accordance with the provision made for the other territory.”.
\end{sloppypar}
\end{quotation}

  After Part VI there shall be inserted the following Part:—
\begin{quotation}
 \part*{“Part VIA\\Parentage}

12A.  Where a person with care resides in one territory and an alleged parent who denies that he is one of the parents of a child in respect of whom an application for a maintenance assessment has been made resides in the other territory:—
\begin{enumerate}\item[]
\begin{sloppypar}
($a$) The person with care or the Secretary of State may apply for a declaration as to whether or not the alleged parent is one of the child’s parents, under Article 28 of the Order;
\end{sloppypar}

($b$) The person with care or the Department of Health and Social Services may apply for such a declaration under section 27 of the Act; and

($c$) The Department of Health and Social Services may bring an action for declarator of parentage under the provisions of section 28 of the Act.”.”
\end{enumerate}
\end{quotation}
\end{quotation}

  I have the honour to confirm that the foregoing proposals are acceptable to the Department of Health and Social Services for Northern Ireland and agree that your letter and this reply shall constitute a Memorandum of Arrangements between us which shall come into effect on 21st January 1996.

  Sealed with the Official Seal of the Department of Health and Social Services for Northern Ireland on the 8th day of November 1995.

  \emph{F. A. Elliott}

  Permanent Secretary.
\end{quotation}

\part{Explanatory Note}

\renewcommand\parthead{--- Explanatory Note}

\subsection*{(This note is not part of the Regulations)}

 These Regulations amend various regulations made under the Child Support Act 1991 (“the Act”).

  The Child Support (Arrears, Interest and Adjustment of Maintenance Assessments) Regulations 1992 are amended to make provision for the circumstances in which a parent with care must reimburse the Secretary of State for overpayments of maintenance which he has repaid to the absent parent as provided for in section 41A of the Act, which was inserted by section 23 of the Child Support Act 1995 (regulation 3).

  The Child Support (Information, Evidence and Disclosure) Regulations 1992 are amended to apply to the provision of information on reviews (regulations 7 and 8); to set the time within which certain information is to be supplied (regulation 10); to extend the circumstances in which information can be given (regulation 11); and to make provision for disclosure of information by the Secretary of State to a child support officer and by a child support officer to the Secretary of State (regulation 12).

  The Child Support (Maintenance Arrangements and Jurisdiction) Regulations 1992 are amended to allow an application for a maintenance assessment to be made notwithstanding that a court order is in existence, where the court has decided that it has no power to vary or enforce that order (regulation 14).

 The Child Support (Maintenance Assessment Procedure) Regulations 1992 are amended in the following respects—
\begin{enumerate}\item[]
 ($a$) regulations 8 and 9 are divided into a number of regulations to make them more comprehensible. Amendments have also been made to make provision for the effective date of Category B interim maintenance assessments generally to have the same effective date as would be applicable to a full maintenance assessment in that case; for effective dates of interim maintenance assessments made where an absent parent has failed to provide information required on review; for an interim maintenance assessment to cease to have effect where a child support officer receives information as to an absent parent’s circumstances for part but not the whole of the period since the maintenance enquiry form was sent; and in some circumstances for review of a cancellation of an interim maintenance assessment (regulations 16 and 17);

 ($b$) regulation 16A is inserted to make provision for notification of lapsing of an appeal under section 20A of the Act (regulation 23);

 ($c$) regulation 18 is substituted to provide that where an application for a review under section 17 of the Act is received less than 8 weeks before a periodical review under section 16 of the Act is due to take place, the periodical review rather than the review under section 17 shall be done (regulation 25);

 ($d$) regulation 19 is amended to make new provision for reviews under section 17 of the Act to take account of the amendment of that section. The regulation provides that a child support officer must take account of matters which are brought to his attention by the parties (regulation 26);

 ($e$) regulation 30A is inserted to provide for effective dates of new assessments which relate to part only of the period after the maintenance enquiry form was sent and also for the effective date of a subsequent assessment made when all relevant information is available for the whole of the relevant period (regulation 33);

 ($f$) regulation 31 is divided into a number of regulations to make it more comprehensible and some amendments are made to effective dates of assessments made on review, in particular, on a review under the new provisions of section 19 of the Act (regulation 34);

 ($g$) regulations 35A and 40A are inserted to make provision for the circumstances in which a reduced benefit direction should not be given or will be suspended (regulations 37 and 38).
\end{enumerate}

  The Child Support (Maintenance Assessments and Special Cases) Regulations 1992 are amended to make further provision for the definition of “relevant week” for the purposes of reviews and the definition of “day to day care” (regulation 40); to make clear that where housing costs consist of fees paid for residential care, the amount of such fees, for the purposes of exempt and protected income, shall be net of any housing benefit (regulations 42 and 43); to make provision for adjustment of existing maintenance assessments where a new application is made in multiple application cases (regulation 45); and to make provision for the value of a compensating transfer made out of assets belonging to the parent with care alone (regulation 48).

  Other amendments made are of a minor, technical or consequential nature.

  These Regulations do not impose any costs on business.

\end{document}