\documentclass[12pt,a4paper]{article}

\newcommand\regstitle{The Child Support Management of Payments and Arrears (Amendment) Regulations 2012}

\newcommand\regsnumber{2012/3002}

\title{\regstitle}

\author{S.I.\ 2012 No.\ 3002}

\date{Made
28th November 2012\\
%Laid before Parliament
%8th November 2012\\
Coming into force
in accordance with regulation 1
}

%\opt{oldrules}{\newcommand\versionyear{1993}}
%\opt{newrules}{\newcommand\versionyear{2003}}
%\opt{2012rules}{\newcommand\versionyear{2012}}

\usepackage{csa-regs}

\setlength\headheight{42.11603pt}

%\hbadness=10000

\begin{document}

\maketitle

\enlargethispage{\baselineskip}

\noindent
The Secretary of State for Work and Pensions makes the following Regulations in exercise of the powers conferred by sections 14(3), 41D(2) and (3), 41E, 51(1) and 52(4) of the Child Support Act 1991\footnote{1991 c.~48 (“the 1991 Act”). Sections 41D and 41E of the 1991 Act were inserted by sections 32 and 33 respectively, of the Child Maintenance and Other Payments Act 2008 (c.~6) (“the 2008 Act”) and amended by the Public Bodies (Child Maintenance and Enforcement Commission: Abolition and Transfer of Functions) Order 2012 (S.I.~2012/2007) (“the 2012 Order”). Section 51 of the 1991 Act was amended by section 1(2) of the Child Support, Pensions and Social Security Act (c.~19).}.

A draft of this instrument was laid before and approved by a resolution of each House of Parliament in accordance with section 52(2)($a$)  and (2A)($b$)\footnote{Section 52(2)($a$)  of the 1991 Act was amended by section 57 of, and paragraph 22 of Schedule 7 to, the 2008 Act. Section 52(2A) of the 1991 Act was substituted by section 57 of, and paragraph 23 of Schedule 7 to, the 2008 Act.} of that Act. 

{\sloppy

\tableofcontents

}

\bigskip

\setcounter{secnumdepth}{-2}

\subsection[1. Commencement and citation]{Commencement and citation}

1.  These Regulations may be cited as the Child Support Management of Payments and Arrears (Amendment) Regulations 2012 and come into force on the day on which sections 32 (power to accept part payment of arrears in full and final satisfaction) and 33 (power to write off arrears) of the Child Maintenance and Other Payments Act 2008\footnote{2008 c.~6.} come into force.

\subsection[2. Amendment of the Child Support (Management of Payments and Arrears) Regulations 2009]{Amendment of the Child Support (Management of Payments and Arrears) Regulations 2009}

2.---(1)  The Child Support (Management of Payments and Arrears) Regulations 2009\footnote{S.I.~2009/3151.} are amended as follows.

(2) In regulation 2(1) (interpretation), after the definition of “the AIMA Regulations” insert—
\begin{quotation}
““child in Scotland” means a child who has made an application for a maintenance calculation under section 7 of the 1991 Act;”.
\end{quotation}

(3) After Part~IV (recovery from estates) insert—
\begin{quotation}
\section*{“Part IVA\\*Part payment of arrears in full and final satisfaction}

\subsection*{Interpretation of this Part}

13A.  In this Part—
\begin{enumerate}\item[]
“appropriate person” means the person from whom the appropriate consent is required under section 41D(5) or (6) of the 1991 Act.
\end{enumerate}

\subsection*{\sloppy Amounts owed to different persons to be treat\-ed separately}

13B.  Where the arrears of child support maintenance for which a person is liable comprise amounts that have accrued in respect of—
\begin{enumerate}\item[]
($a$) separate applications for a maintenance calculation; or

($b$) one application but would, if recovered, be payable to different persons,
\end{enumerate}
those amounts are to be treated as separate amounts of arrears for the purpose of exercising the power under section 41D(1) of the 1991 Act.

\subsection*{Appropriate consent}

13C.---(1)  The Secretary of State may not exercise the power under section 41D(1) of the 1991 Act without the appropriate consent (as provided for in subsections (5) to (7) of section 41D), unless one of the following conditions applies—
\begin{enumerate}\item[]
($a$) that the Secretary of State would be entitled to retain the whole of the arrears under section 41(2) of the 1991 Act if it recovered them; or

($b$) that the Secretary of State would be entitled to retain part of the arrears under section 41(2) of that Act if it recovered them, and the part of the arrears that the Secretary of State would not be entitled to retain is equal to or less than the payment accepted under section 41D(1) of that Act.
\end{enumerate}

(2) Where the consent of any appropriate person is required, the Secretary of State must make available such information and guidance as the Secretary of State thinks appropriate for the purpose of helping that person decide whether to give that consent.

\subsection*{Agreement}

13D.---(1)  Where the Secretary of State proposes to exercise the power under section 41D(1) of the 1991 Act, the Secretary of State must prepare a written agreement.

(2) The agreement must—
\begin{enumerate}\item[]
($a$) name the non-resident parent, and where the consent of any appropriate person is required, the name of that person;

($b$) specify the amount of arrears to which the agreement relates and the period of liability to which those arrears relate;

($c$) state the amount that is agreed will be paid in satisfaction of those arrears;

($d$) state the method of payment and to whom payment will be made; and

($e$) state the day by which payment is to be made.
\end{enumerate}

(3) The Secretary of State must send the non-resident parent and, where applicable, the appropriate person, a copy of the agreement.

(4) The agreement does not take effect until—
\begin{enumerate}\item[]
($a$) the non-resident parent has agreed in writing to its terms; and

($b$) where applicable, the appropriate person has given to the Secretary of State their consent in writing.
\end{enumerate}

\subsection*{Where payment is received}

13E.---(1)  Unless the non-resident parent fails to comply with the terms of the agreement, the Secretary of State must not take action to recover any of the arrears to which the agreement relates.

(2) Where the non-resident parent has made full payment in accordance with the agreement all remaining liability in respect of the arrears of child support maintenance to which the agreement relates is extinguished.

(3) Where the non-resident parent fails to make any payment or only makes part payment or otherwise fails to adhere to the terms of the agreement, the non-resident parent remains liable to pay the full amount of any outstanding arrears to which the agreement relates and the Secretary of State may arrange to recover any of those outstanding arrears in accordance with the 1991 Act.

(4) Nothing in these Regulations prevents the Secretary of State from entering into a new agreement with the non-resident parent in respect of any of the arrears to which the previous agreement relates provided that the new agreement complies with the requirements set out in regulation 13D.

(5) Where the Secretary of State enters into a new agreement with the non-resident parent in respect of any of the arrears to which a previous agreement related, the previous agreement ceases to have effect on the coming into effect of that new agreement.

\section*{Part IVB\\*Write off of arrears}

\subsection*{\sloppy Amounts owed to different persons to be treat\-ed separately}

13F.  Where the arrears of child support maintenance for which a person is liable comprise amounts that have accrued in respect of—
\begin{enumerate}\item[]
($a$) separate applications for a maintenance calculation; or

($b$) one application, but would, if recovered, be payable to different persons,
\end{enumerate}
those amounts are to be treated as separate amounts of arrears for the purpose of exercising the power under section 41E(1) of the 1991 Act.

\subsection*{Circumstances in which the Secretary of State may exercise the power in section 41E of the 1991 Act}

13G.  The circumstances of the case specified for the purposes of section 41E(1)($a$)  of the 1991 Act are that—
\begin{enumerate}\item[]
($a$) the person with care has requested under section 4(5) of that Act that the Secretary of State ceases to act in respect of the arrears;

($b$) a child in Scotland has requested under section 7(6) of that Act that the Secretary of State ceases to act in respect of the arrears;

($c$) the person with care, or (in Scotland) the child, has died;

($d$) the non-resident parent died before 25 January 2010 or there is no further action that can be taken with regard to recovery of the arrears from the non-resident parent’s estate under Part~IV;

($e$) the arrears relate to liability for child support maintenance for any period in respect of which an interim maintenance assessment was in force between 5 April 1993 and 18 April 1995; or

($f$) the non-resident parent has been informed by the Secretary of State that no further action would ever be taken to recover those arrears.
\end{enumerate}

\subsection*{Secretary of State required to give notice}

13H.---(1)  Where the Secretary of State is considering exercising the powers under section 41E(1) of the 1991 Act, the Secretary of State must send written notice to the person with care or, where relevant, a child in Scotland and the non-resident parent.

(2) The requirement in paragraph (1) does not apply where the person in question cannot be traced or has died.

(3) The notice must—
\begin{enumerate}\item[]
($a$) specify the person with care or, where relevant, a child in Scotland, in respect of whom liability in respect of arrears of child support maintenance has accrued;

($b$) specify the amount of the arrears and the period of liability to which the arrears relate;

($c$) state why it appears to the Secretary of State that it would be unfair or inappropriate to enforce liability in respect of the arrears;

($d$) advise the person that they may make representations, within 30 days of receiving the notice, to the Secretary of State as to whether the liability in respect of the arrears should be extinguished; and

($e$) explain the effect of any decision to extinguish liability in respect of any arrears of child support maintenance under section 41E(1) of the 1991 Act.
\end{enumerate}

(4) If no representations are received by the Secretary of State within 30 days of the notice being received by the person with care or, where relevant, a child in Scotland and the non-resident parent, the Secretary of State may make the decision to extinguish the arrears.

(5) For the purposes of this regulation, where the Secretary of State sends any written notice by post to a person’s last known or notified address that document is treated as having been received by that person on the second day following the day on which it is posted.

\subsection*{Secretary of State to take account of the parties’ views}

13I.  Where the Secretary of State receives representations within the 30 day period referred to in regulation 13H(3)($d$), the Secretary of State must take account of those representations in making a decision under section 41E(1) of the 1991 Act.

\subsection*{Notification of decision to write off}

13J.---(1)  On making a decision under section 41E(1) of the 1991 Act, the Secretary of State must send written notification to the non-resident parent and the person with care or, where relevant, a child in Scotland, of that decision.

(2) The requirement in paragraph (1) does not apply where the person in question cannot be traced or has died.”.
\end{quotation}

\subsection[3. Amendment of the Child Support Information Regulations 2008]{Amendment of the Child Support Information Regulations 2008}

3.---(1)  Regulation 13(1) of the Child Support Information Regulations 2008\footnote{S.I.~2008/2551.} is amended as follows.

(2) After sub-paragraph ($f$), omit “or”.

(3) After sub-paragraph ($g$), insert—
\begin{quotation}
“($h$) why it was decided, in relation to any arrears of child support maintenance, not to accept payment in part in satisfaction of liability for the whole under section 41D(1) of the 1991 Act; or

($i$) why it was decided not to extinguish liability in respect of arrears of child support maintenance under section 41E(1) of the 1991 Act.”. 
\end{quotation}

\bigskip

\pagebreak[3]

Signed 
by authority of the 
Secretary of State for~Work and~Pensions.
%I concur
%By authority of the Lord Chancellor

{\raggedleft
\emph{Steve Webb}\\*
%Secretary
Minister
%Parliamentary Under-Secretary 
of State\\*Department 
for~Work and~Pensions

}

28th November 2012

\small

\part{Explanatory Note}

\renewcommand\parthead{— Explanatory Note}

\subsection*{(This note is not part of the Regulations)}

These Regulations amend the Child Support (Management of Payments and Arrears) Regulations 2009 (“the 2009 Regulations”) and come into force on the day on which sections 32 and 33 of the Child Maintenance and Other Payments Act 2008 (c.~6) come into force.

Regulation 2(3) of these Regulations inserts Parts IVA and IVB into the 2009 Regulations. Part~IVA makes provision in relation to the power to accept part payment of arrears in satisfaction of any arrears of child support maintenance under section 41D of the Child Support Act 1991 (“the 1991 Act”). Part~IVB makes provision in relation to the power to extinguish liability in respect of arrears of child support maintenance under section 41E(1) of the 1991 Act.

In the inserted Part~IVA of the 2009 Regulations, regulation 13B provides that where arrears of child support maintenance for which a person is liable comprise different amounts owed to different persons they are to be treated as separate amounts of arrears for the purpose of exercising the power under section 41D(1) of the 1991 Act.

Regulation 13C provides that the Secretary of State may not exercise the power contained in section 41D(1) of the 1991 Act without the appropriate consent unless certain conditions apply. The meaning of appropriate consent is set out in subsections (5) and (6) of section 41D of the 1991 Act.

Regulation 13D provides that where it is proposed to accept an offer of a part payment of arrears, the Secretary of State must set out the terms of the agreement in writing and send it to the non-resident parent, and where applicable, the appropriate person (as defined in regulation 13A). This regulation also provides that the non-resident parent must have agreed to the terms in writing, and where applicable, the appropriate person, must have given written consent to the Secretary of State.

Regulation 13E provides that while a non-resident parent is complying with the agreement, the Secretary of State must not take steps to recover the outstanding arrears to which the agreement relates. Once full payment has been made in accordance with the terms of the agreement then all remaining liability in respect of the arrears of child support maintenance to which the agreement relates is extinguished.

Where the agreement has not been adhered to, the Secretary of State may recover all outstanding arrears. However the Secretary of State may enter into a new agreement with the non-resident parent in respect of any arrears to which the previous agreement relates provided the new agreement complies with the requirements set out in regulations 13C and 13D.

In the inserted Part~IVB of the 2009 Regulations, regulation 13F provides that where arrears of child support maintenance for which a person is liable comprise different amounts owed to different persons they are to be treated as separate amounts of arrears for the purpose of exercising the power under section 41E(1) of the 1991 Act.

Regulation 13G specifies the circumstances in which the Secretary of State may exercise the power in section 41E(1) of the 1991 Act.

Regulation 13H sets out the requirement to notify the parties before making a decision. It also provides that if no representations are received from the parties within 30 days of the notice being received by the relevant parties, the Secretary of State may make the decision to extinguish the arrears.

Regulation 13I provides that where the Secretary of State receives representations in response to a notice given under regulation 13H, the Secretary of State must take account of those representations in making a decision under section 41E of the 1991 Act.

Regulation 13J provides that on making a decision to write off arrears under section 41E(1) of the 1991 Act, the Secretary of State must send written notice to the non-resident parent and the person with care or, where relevant, a child in Scotland, of that decision.

Regulation 3 of these Regulations makes amendments to the Child Support Information Regulations 2008. These amendments provide for the disclosure by the Secretary of State of information held for the purposes of the 1991 Act relating to one party to a maintenance calculation to another party to that calculation. Any disclosure is permitted where, in the opinion of the Secretary of State, such information is essential to inform the party to whom it would be given as to why it was decided not to exercise power under either section 41D(1) (power to accept part payments in full and final satisfaction) or section 41E(1) (power to write off arrears) of the 1991 Act.

A full regulatory impact assessment has not been produced for this instrument as no impact on the private or voluntary sectors is foreseen. 

\end{document}