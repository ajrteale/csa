\documentclass[12pt,a4paper]{article}

\newcommand\regstitle{The Child Maintenance and Other Payments Act 2008 (Commencement No.~11 and Transitional Provisions) Order 2013}

\newcommand\regsnumber{2013/1860}

\title{\regstitle}

\author{S.I.\ 2013 No.\ 1860 (C.~79)}

\date{Made
%3rd July 2013\\
%%Laid before Parliament
%%27th June 2013\\
%Coming into force
23rd July 2013
}

%\opt{oldrules}{\newcommand\versionyear{1993}}
%\opt{newrules}{\newcommand\versionyear{2003}}
%\opt{2012rules}{\newcommand\versionyear{2012}}

\usepackage{csa-regs}

\setlength\headheight{42.11603pt}

%\hbadness=10000

\begin{document}

\maketitle

\enlargethispage{\baselineskip}

\noindent
The Secretary of State for Work and Pensions makes the following Order in exercise of the powers conferred by section 62(3) and (4) of the Child Maintenance and Other Payments Act 2008\footnote{2008 c.~6.}: 

{\sloppy

\tableofcontents

}

\bigskip

\setcounter{secnumdepth}{-2}

\subsection[1. Citation and interpretation]{Citation and interpretation}

1.---(1)  This Order may be cited as the Child Maintenance and Other Payments Act 2008 (Commencement No.~11 and Transitional Provisions) Order 2013.

(2) In this Order—
\begin{enumerate}\item[]
“1991 Act” means the Child Support Act 1991\footnote{1991 c.~48.};

“2000 Act” means the Child Support, Pensions and Social Security Act 2000\footnote{2000 c.~19. The relevant amendments to the Child Support Act 1991 (“the 1991 Act”) were made by sections 1 and 26 of, and paragraph 11(1), (2) and (20) of Schedule 3 to, the Child Support, Pensions and Social Security Act 2000 (“the 2000 Act”).};

“2008 Act” means the Child Maintenance and Other Payments Act 2008;

“2012 Regulations” means the Child Support Maintenance Calculation Regulations 2012\footnote{S.I.~2012/2677.};

“new calculation rules” means Part~I of Schedule 1 to the 1991 Act as amended by the provisions specified in article 2.
\end{enumerate}

(3) In this Order, subject to paragraph (5)—
\begin{enumerate}\item[]
“maintenance calculation”, “non-resident parent”, “person with care” and “qualifying child” have the meanings given in the 1991 Act\footnote{The definition of “maintenance calculation” was substituted for the definition of “maintenance assessment” in section 54 of the 1991 Act by section 26 of, and paragraph 11(1) and (20)($d$)  of Schedule 3 to, the 2000 Act. The term “non-resident parent” was substituted for the term “absent parent” by section 26 of, and paragraph 11(1) and (2) of Schedule 3 to, the 2000 Act. The definition of “qualifying child” in section 3(1) of the 1991 Act was amended by section 26 of, and paragraph 11(1) and (2) of Schedule 3 to, the 2000 Act.};

“absent parent” and “maintenance assessment” have the meanings given in the 1991 Act before its amendment by the 2000 Act.
\end{enumerate}

(4) In this Order, a reference to an existing case is to a case in which there is—
\begin{enumerate}\item[]
($a$) a maintenance assessment in force;

($b$) a maintenance calculation, made otherwise than in accordance with the new calculation rules, in force;

($c$) an application for a maintenance assessment which has been made but not determined; or

($d$) an application for a maintenance calculation, which falls to be made otherwise than in accordance with the new calculation rules, which has been made but not determined.
\end{enumerate}

(5) In this Order—
\begin{enumerate}\item[]
($a$) a reference to a non-resident parent includes reference to a person who is—
\begin{enumerate}\item[]
(i) alleged to be the non-resident parent for the purposes of an application for child support maintenance under the 1991 Act, or

(ii) treated as the non-resident parent for the purposes of the 1991 Act; and
\end{enumerate}

($b$) a reference to an absent parent includes reference to a person who is—
\begin{enumerate}\item[]
(i) alleged to be the absent parent for the purposes of an application for child support maintenance under the 1991 Act, or

(ii) treated as the absent parent for the purposes of the 1991 Act.
\end{enumerate}
\end{enumerate}

\subsection[2. Appointed day for the coming into effect of the new calculation rules]{Appointed day for the coming into effect of the new calculation rules}

2.  The following provisions of the 2008 Act come into force, in so far as those provisions are not already in force, on 29th July 2013 for the purposes of those types of cases falling within article 3—
\begin{enumerate}\item[]
($a$) section 16 (changes to the calculation of maintenance) and paragraph~1 of Schedule 4 (introductory), so far as relating to the paragraphs referred to in paragraph ($b$);

($b$) paragraphs 2, 3 and 5 to 10 of Schedule 4 (changes to the calculation of maintenance);

($c$) sections 17 (power to regulate supersession) and 18 (determination of applications for a variation);

($d$) section 57(1) and paragraph 1(1) of Schedule 7 (minor and consequential amendments), so far as relating to the paragraph referred to in paragraph ($e$);

($e$) paragraph 1(2) and (29) of Schedule 7;

($f$) section 58 (repeals), so far as relating to the entries referred to in paragraph ($g$); and

($g$) in Schedule 8 (repeals), the entries relating to—
\begin{enumerate}\item[]
(i) Schedule 1 (maintenance calculations) to the 1991 Act, and

(ii) Schedule 24 (social security, child support and tax credits) to the Civil Partnership Act 2004\footnote{2004 c.~33.}.
\end{enumerate}
\end{enumerate}

\subsection[3. Cases to which the new calculation rules apply]{Cases to which the new calculation rules apply}

3.---(1)  The types of cases falling within this article, for the purposes of article~2, are those cases satisfying any of paragraphs (2) to (4).

(2) A case satisfies this paragraph where—
\begin{enumerate}\item[]
($a$) an application under section 4 or 7 of the 1991 Act\footnote{Section 4 was amended by section 18(1) of the Child Support Act 1995 (c.~34), paragraph 19 of Schedule 7, and Schedule 8, to the Social Security Act 1998 (c.~14) (“the 1998 Act”), sections 1(2) and 2(1) to (3) of, and paragraph 11(1) to (3) of Schedule 3 to, the 2000 Act, section 35(1) of, and Schedule 8 to, the Child Maintenance and Other Payments Act 2008 (c.~6) (“the 2008 Act”) and S.I.~2012/2007. Section 7 was amended by paragraph 21 of Schedule 7, and Schedule 8, to the 1998 Act, section 1(2) of, and paragraph 11(1),(2) and (4) of Schedule 3 to, the 2000 Act, section 35(2) of the 2008 Act and S.I.~2012/2007.} is made to the Secretary of State on or after 29th July 2013;

($b$) that application is made in respect of two or three qualifying children with the same person with care and the same non-resident parent; and

($c$) subject to paragraph (5), there is no existing case which has both the same person with care and the same non-resident parent referred to in sub-paragraph ($b$).
\end{enumerate}

(3) A case satisfies this paragraph where it is an existing case and—
\begin{enumerate}\item[]
($a$) the non-resident parent in a case falling within paragraph (2) is also the non-resident parent or absent parent in relation to the existing case; and

($b$) the person with care in relation to the existing case is not the person with care in relation to the case falling within paragraph (2).
\end{enumerate}

(4) A case satisfies this paragraph where it is an existing case and—
\begin{enumerate}\item[]
($a$) the non-resident parent or absent parent (“$\mathcal{A}$”) is a partner of a non-resident parent in a case falling within paragraph (2) (“$\mathcal{B}$”); and

($b$) $\mathcal{A}$ or $\mathcal{B}$ is in receipt of a prescribed benefit.
\end{enumerate}

(5) Where—
\begin{enumerate}\item[]
($a$) the applicant in relation to an existing case makes a request to the Secretary of State under section 4(5) or 7(6) of the 1991 Act to cease acting; and

($b$) a further application is made under section 4 or 7 of the 1991 Act in relation to the same qualifying child, person with care and non-resident parent on or after 29th July 2013, but before the expiry of 13 weeks from the date of cessation of action by the Secretary of State,
\end{enumerate}
the case is to be treated as an existing case (and so is not a case that satisfies paragraph (2)).

(6) For the purposes of paragraphs (2)($a$)  and (5)($b$), the date an application is made is—
\begin{enumerate}\item[]
($a$) where made by telephone, the date it is made; and

($b$) where made by post, the date of receipt by the Secretary of State.
\end{enumerate}

(7) For the purposes of paragraph (4)—
\begin{enumerate}\item[]
“partner” has the meaning given in paragraph 10C(4) (references to various terms) of Schedule 1 to the 1991 Act as amended by the 2000 Act\footnote{Part~I of Schedule 1 to the 1991 Act was substituted by section 1(3) of, and Schedule 1 to, the 2000 Act.};

“prescribed benefit” means a benefit prescribed, or treated as prescribed, for the purposes of paragraph 4(1)($c$)  (flat rate) of Schedule 1 to the 1991 Act as amended by the 2000 Act.
\end{enumerate}

(8) For the purposes of paragraph (5)($b$), the date of cessation of action by the Secretary of State is—
\begin{enumerate}\item[]
($a$) where there is a maintenance assessment or maintenance calculation in force, the date on which the liability under that assessment or calculation ends as a result of the request to cease acting; and

($b$) where there is an application still to be determined, the date notified to the person with care as the date on which the Secretary of State has ceased acting.
\end{enumerate}

\subsection[4. Appointed day for coming into force of a repeal within the 2008 Act]{Appointed day for coming into force of a repeal within the 2008 Act}

4.  In Schedule 8 (repeals) of the 2008 Act, the entry relating to section~28 (pilot schemes) of the 2000 Act comes into force on 29th July 2013.

\subsection[5. Transitional provision for existing cases]{Transitional provision for existing cases}

5.  Where a case falls within article 3(3) or (4), the provisions of the 1991 Act continue to apply—
\begin{enumerate}\item[]
($a$) as they were in force immediately before the coming into force of the provisions in article 2 in relation to that case;

($b$) until the maintenance calculation made in response to the application referred to in article 3(2)($a$)  takes effect.
\end{enumerate}

\subsection[6. Amendment of the Child Maintenance and Other Payments Act 2008 (Commencement No.~10 and Transitional Provisions) Order 2012]{Amendment of the Child Maintenance and Other Payments Act 2008 (Commencement No.~10 and Transitional Provisions) Order 2012}

6.  With effect from 29th July 2013, Article 6 of the Child Maintenance and Other Payments Act 2008 (Commencement No.~10 and Transitional Provisions) Order 2012\footnote{S.I.~2012/3042.} is omitted.

\subsection[7. Transitional provision when making the maintenance calculation]{Transitional provision when making the maintenance calculation}

7.  For the period beginning on 29th July 2013 and ending on the date on which the new calculation rules come into force for all purposes—
\begin{enumerate}\item[]
($a$) regulation 34(2) of the 2012 Regulations (the general rule for determining gross weekly income) shall be read as if after paragraph ($b$)  there were inserted—
\begin{quotation}
“or

($c$) the Secretary of State is unable, for whatever reason, to request or obtain the required information from HMRC.”;
\end{quotation}

($b$) regulation 42(1)($a$)  of the 2012 Regulations (estimate of current income where insufficient information available) shall be read as if after “34(2)($b$)  (historic income nil or not available)” there were inserted “or~($c$)  (Secretary of State unable to request or obtain information from HMRC).”;

($c$) regulation 69(5) of the 2012 Regulations (non-resident parent with unearned income) shall be read as if after paragraph ($b$)  there were inserted—
\begin{quotation}
“or

($c$) the Secretary of State is unable, for whatever reason, to request or obtain the information from HMRC.”.
\end{quotation}
\end{enumerate}

\bigskip

\pagebreak[3]

Signed 
by authority of the 
Secretary of State for~Work and~Pensions.
%I concur
%By authority of the Lord Chancellor

{\raggedleft
\emph{Freud}\\*
%Secretary
%Minister
Parliamentary Under-Secretary 
of State\\*Department 
for~Work and~Pensions

}

23rd July 2013

\small

\part{Explanatory Note}

\renewcommand\parthead{— Explanatory Note}

\subsection*{(This note is not part of the Regulations)}

This Order brings into force provisions of the Child Maintenance and Other Payments Act 2008 (c.~6) (“the 2008 Act”) for the purpose of applying new rules for calculating child support maintenance to certain applications made on or after 29th July 2013 and certain cases linked to those applications.

The 2008 Act amends the statutory scheme for calculation, collection and enforcement of child support maintenance, as originally set out in the Child Support Act 1991 (c.~48) (“the 1991 Act”) and amended by the Child Support, Pensions and Social Security Act 2000 (c.~19) (“the 2000 Act”). The amendments made by the 2000 Act were brought into force by the Child Support, Pensions and Social Security Act 2000 (Commencement No.~12) Order 2003 (S.I.~2003/192) for new applications after 3rd March 2003 and for existing cases related to such applications. However, the original provisions of the 1991 Act remained in force for a substantial number of cases, effectively resulting in two separate schemes. The 2008 Act makes further amendments to the rules for calculating child support maintenance. These provisions together constitute a third scheme (“the new calculation rules”). The Child Maintenance and Other Payments Act 2008 (Commencement No.~10 and Transitional Provisions) Order 2012 (S.I.~2012/3042) (“the 2012 Order”) brought the majority of the amendments made by the 2008 Act into force for new applications made on or after 10th December 2012 where there were four or more children with the same person with care and non-resident parent and no existing case with the same person with care and non-resident parent and for certain existing cases related to such applications.

Article 2 brings into force the majority of the amendments made to the new calculation rules on 29th July 2013 for the purposes of certain types of cases, which are set out in article 3.

Article 3 provides that the cases to which the new calculation rules will apply are those new applications made on or after 29th July 2013 which relate to two or three qualifying children with both the same person with care and non-resident parent, where there is no existing case with the same person with care and the same non-resident parent. For the purposes of considering whether there is an existing case, article 3(5) makes provision so that any case voluntarily closed in the thirteen weeks preceding the new application will still be considered an existing case and prevent the new calculation rules applying.

The new calculation rules will also apply to any existing case in which the non-resident parent named in the new application is also the non-resident parent and there is a different parent with care (article 3(3)). The new calculation rules will also apply to any existing case in which the non-resident parent is the partner of a non-resident parent named in a new application, and either of those non-resident parents claims a prescribed benefit (article 3(4)).

Article 4 brings into force a repeal relating to pilot schemes on 29th July 2013.

Article 5 makes transitional provision so that the new calculation rules only apply to the existing case from the date the calculation made in response to the new application takes effect.

Article 6 omits a transitional provision in relation to the maintenance calculation from the 2012 Order, with effect from 29th July 2013.

Article 7 makes transitional provision in relation to the mainteance calculation. During the period beginning on 29th July 2013 and ending when the new scheme rules are commenced for all purposes, three provisions of the Child Support Maintenance Calculation Regulations 2012 are to be read as if additional words were inserted. The effect of this is as follows. Article 7(1)($a$)  allows a non-resident parent’s gross weekly income to be calculated on the basis of current income if the Secretary of State is unable to request or obtain information from HMRC. Article 7(1)($b$)  allows the Secretary of State to estimate income where the Secretary of State is unable to request or obtain information from HMRC and the current income information available is insufficient or unreliable. Article 7(1)($c$)  allows the Secretary of State, in cases where the Secretary of State is unable to request or obtain information from HMRC, to determine the amount of a non-resident parent’s unearned income by reference to the most recent tax year, based, as far as possible, on information that would be required to be provided in a self-assessment tax return. 

\end{document}