\documentclass[12pt,a4paper]{article}

\newcommand\regstitle{The Child Support (Management of Payments and Arrears) Regulations 2009}

\newcommand\regsnumber{2009/3151}

%\opt{newrules}{
\title{\regstitle}
%}

%\opt{2012rules}{
%\title{Child Maintenance and~Other Payments Act 2008\\(2012 scheme version)}
%}

\author{S.I.\ 2009 No.\ 3151}

\date{Made
30th November 2009\\
Laid before Parliament
4th Deceember 2009\\
Coming into~force
25th January 2010
}

%\opt{oldrules}{\newcommand\versionyear{1993}}
%\opt{newrules}{\newcommand\versionyear{2003}}
%\opt{2012rules}{\newcommand\versionyear{2012}}

\usepackage{csa-regs}

\setlength\headheight{27.61603pt}

%\hbadness=10000

\begin{document}

\maketitle

\noindent
The Secretary of State for Work and Pensions, in exercise of the powers conferred by sections 14(3), 28J(3), 41C(1), 43A, 51(1) and (2)($d$), ($e$)  and ($f$), 52(4) and 54 of the Child Support Act 1991\footnote{1991 c.~48. Section 14(3) was amended by section~86(1) of, and Schedule 7 to, the Social Security Act 1998 (c.~14) and section~13(4) of, and Schedule 3 to, the Child Maintenance and Other Payments Act 2008 (c.~6) (“the 2008 Act”). Section 28J was inserted by section~20(1) of the Child Support, Pensions and Social Security Act 2000 (c.~19) (“the 2000 Act”). Section 41C was inserted by section~31 of the 2008 Act. Section 43A was inserted by section~38 of the 2008 Act. Section 51 was amended by section~1(2) of the 2000 Act. Section 54 is cited for the definition of “prescribed”.}, makes the following Regulations: 

{\sloppy

\tableofcontents

}

\bigskip

\setcounter{secnumdepth}{-2}

\section[Part I --- General]{Part I\\*General}

\renewcommand\parthead{--- Part I}

\subsection[1. Citation and commencement]{Citation and commencement}

1.  These Regulations may be cited as the Child Support (Management of Payments and Arrears) Regulations 2009 and come into force on 25th January 2010.

\subsection[2. Interpretation]{Interpretation}

2.---(1)  In these Regulations—
\begin{enumerate}\item[]
“the 1991 Act” means the Child Support Act 1991;

“a 1993 scheme case” means a case in respect of which the provisions of the Child Support, Pensions and Social Security Act 2000\footnote{2000 c.~19.} have not been brought into force in accordance with article 3 of the Child Support, Pensions and Social Security Act 2000 (Commencement No.~12) Order 2003\footnote{S.I.~2003/192 (C.~11).};

“the AIMA Regulations” means the Child Support (Arrears, Interest and Adjustment of Maintenance Assessments) Regulations 1992\footnote{S.I.~1992/1816.};

% Definition of ``child in Scotland'' inserted (10.12.12) by SI 2012/3002 reg 2(2)
“child in Scotland” means a child who has made an application for a maintenance calculation under section 7 of the 1991 Act;

“the Decisions and Appeals Regulations” means the Social Security and Child Support (Decisions and Appeals) Regulations 1999\footnote{S.I.~1999/991.};

“non-resident parent” includes a person treated as a non-resident parent by virtue of regulations made under section~42 of the 1991 Act;

“relevant person” means—
\begin{enumerate}\item[]
($a$) 
a person with care;

($b$) 
a non-resident parent;

($c$) 
where the application for a maintenance calculation is made by a child under section~7 of the 1991 Act, that child,
\end{enumerate}
in respect of whom a maintenance calculation is or has been in force.
\end{enumerate}

(2) In the application of these Regulations to a 1993 scheme case, any reference to expressions in the 1991 Act (including “non-resident parent” and “maintenance calculation”) or to regulations made under that Act are to be read with the necessary modifications.

\amendment{
Definition of ``child in Scotland'' inserted (10.12.12) by the Child Support Management of Payments and Arrears (Amendment) Regulations 2012 reg.~2(2).
}

\subsection[3. Arrears notices]{Arrears notices}

3.---(1)  This regulation applies to a case where—
\begin{enumerate}\item[]
($a$) the 
%Commission 
Secretary of State  % Words substituted (1.8.12) by SI 2012/2007 Sch para 121(2)(a)
is arranging for the collection of child support maintenance under section~29 of the 1991 Act; and

($b$) the non-resident parent has failed to make one or more payments of child support maintenance due.
\end{enumerate}

(2) Where the 
%Commission 
Secretary of State  % Words substituted (1.8.12) by SI 2012/2007 Sch para 121(2)(b)
is considering taking action with regard to a case falling within paragraph~(1) 
%it 
the Secretary of State  % Words substituted (1.8.12) by SI 2012/2007 Sch para 121(2)(b)
must serve a notice on the non-resident parent.

(3) The notice must—
\begin{enumerate}\item[]
%($a$) itemize the payments of child support maintenance due and not paid;

% Reg 3(3)(a) substituted (30.4.12) by SI 2012/712
($a$) include the amount of all outstanding arrears of child support maintenance due and not paid;

($b$) set out in general terms the provisions as to arrears contained in this regulation and regulation 8 of the AIMA Regulations\footnote{Regulation 8 was substituted by S.I.~1995/3261 and amended by S.I.~1996/1345 and 2001/162. References to the Secretary of State in that regulation are treated as references to the Commission by virtue of paragraph~55(3) of Schedule 3 to the 2008 Act, as the function of the Secretary of State was transferred to the Commission by section~13 of that Act.}; and

($c$) request the non-resident parent make payment of all outstanding arrears.
\end{enumerate}

(4) Where a notice has been served under paragraph~(2), no duty to serve a further notice under that paragraph~arises in relation to further arrears unless those further arrears have arisen after an intervening continuous period of not less than 12 weeks during the course of which all payments of child support maintenance due from the non-resident parent have been paid on time in accordance with regulations made under section~29 of the 1991 Act.

\amendment{
Reg. 3(3)(a) substituted (30.4.12) by the Child Support (Miscellaneous Amendments) Regulations 2012 reg.~3.

Words substituted in reg.~3(1)(a), (2) (1.8.12) by the Public Bodies (Child Maintenance and Enforcement Commission: Abolition and Transfer of Functions) Order 2012 Sch. para.~121(2).
}

\subsection[4. Attribution of payments]{Attribution of payments}

4.  Where a maintenance calculation is or has been in force and there are arrears of child support maintenance, the 
%Commission 
Secretary of State  % Words substituted (1.8.12) by SI 2012/2007 Sch para 121(3)
may attribute any payment of child support maintenance made by a non-resident parent to child support maintenance due as 
%it 
the Secretary of State  % Words substituted (1.8.12) by SI 2012/2007 Sch para 121(3)
thinks fit.

\amendment{
Words substituted in reg.~4 (1.8.12) by the Public Bodies (Child Maintenance and Enforcement Commission: Abolition and Transfer of Functions) Order 2012 Sch. para.~121(3).
}

\section[Part II --- Set off]{Part II\\*Set off}

\renewcommand\parthead{--- Part II}

\subsection[5. Set off of liabilities to pay child support maintenance]{Set off of liabilities to pay child support maintenance}

5.---(1)  The circumstances prescribed for the purposes of section~41C(1)($a$)  of the 1991 Act, in which the 
%Commission 
Secretary of State  % Words substituted (1.8.12) by SI 2012/2007 Sch para 121(4)
may set off liabilities to pay child support maintenance, are set out in paragraph~(2).

(2) The 
%Commission 
Secretary of State  % Words substituted (1.8.12) by SI 2012/2007 Sch para 121(4)
may set off the liability to pay child support maintenance of one person (“$\mathcal{A}$”) against the liability to pay child support maintenance of another person (“$\mathcal{B}$”) where—
\begin{enumerate}\item[]
($a$) $\mathcal{A}$ is liable to pay child support maintenance under a maintenance calculation, whether that calculation is current or no longer in force, in relation to which $\mathcal{B}$ is the person with care; and

($b$) $\mathcal{B}$ is liable to pay child support maintenance under a maintenance calculation, whether that calculation is current or no longer in force, in relation to which $\mathcal{A}$ is the person with care.
\end{enumerate}

(3) There shall be no set off in relation to any amount which if paid could be retained under section~41 of the 1991 Act.

\amendment{
Words substituted in reg.~5(1), (2) (1.8.12) by the Public Bodies (Child Maintenance and Enforcement Commission: Abolition and Transfer of Functions) Order 2012 Sch. para.~121(4).
}

\subsection[6. Set off of payments against child support maintenance liability]{Set off of payments against child support maintenance liability}

6.---(1)  The circumstances prescribed for the purposes of section~41C(1)($b$)  of the 1991 Act, in which the 
%Commission 
Secretary of State  % Words substituted (1.8.12) by SI 2012/2007 Sch para 121(5)
may set off a payment against a person’s liability to pay child support maintenance, are set out in paragraph~(2).

(2) The 
%Commission 
Secretary of State  % Words substituted (1.8.12) by SI 2012/2007 Sch para 121(5)
may set off a payment against a non-resident parent’s liability to pay child support maintenance where—
\begin{enumerate}\item[]
($a$) the payment falls within paragraph~(3); and

($b$) the person with care agreed to the making of the payment.
\end{enumerate}

(3) A payment is of a prescribed description for the purposes of section~41C(1)($b$)  of the 1991 Act if it was made by the non-resident parent in respect of—
\begin{enumerate}\item[]
($a$) a mortgage or loan taken out on the security of the property which is the qualifying child’s home where that mortgage or loan was taken out to facilitate the purchase of, or to pay for essential repairs or improvements to, that property;

($b$) rent on the property which is the qualifying child’s home;

($c$) mains-supplied gas, water or electricity charges at the qualifying child’s home;

($d$) council tax payable by the person with care in relation to the qualifying child’s home;

($e$) essential repairs to the heating system in the qualifying child’s home; or

($f$) repairs which are essential to maintain the fabric of the qualifying child’s home.
\end{enumerate}

\amendment{
Words substituted in reg.~6(1), (2) (1.8.12) by the Public Bodies (Child Maintenance and Enforcement Commission: Abolition and Transfer of Functions) Order 2012 Sch. para.~121(5).
}

\subsection[7. Application of set off]{Application of set off}

7.---(1)  In setting off a person’s liability for child support maintenance under this Part, the 
%Commission 
Secretary of State  % Words substituted (1.8.12) by SI 2012/2007 Sch para 121(6)(a)
may apply the amount to be set off to reduce any arrears of child support maintenance due under any current maintenance calculation, or any previous maintenance calculation made in respect of the same relevant persons.

(2) Where there are no arrears of child support maintenance due, or an amount remains to be set off after the application of paragraph~(1), the 
%Commission 
Secretary of State  % Words substituted (1.8.12) by SI 2012/2007 Sch para 121(6)(b)
may adjust the amount payable in relation to the current maintenance calculation by such amount as 
%it 
the Secretary of State  % Words substituted (1.8.12) by SI 2012/2007 Sch para 121(6)(b)
considers appropriate in all the circumstances of the case, having regard in particular to—
\begin{enumerate}\item[]
($a$) the circumstances of the relevant persons; and

($b$) the amount to be set off and the period over which it would be reasonable to adjust the amount payable to set off that amount.
\end{enumerate}

(3) An adjustment of the amount payable in relation to the current maintenance calculation under paragraph~(2) may reduce the amount payable to nil.

\amendment{
Words substituted in reg.~7(1), (2) (1.8.12) by the Public Bodies (Child Maintenance and Enforcement Commission: Abolition and Transfer of Functions) Order 2012 Sch. para.~121(6).
}

\section[Part III --- Overpayments and voluntary payments]{Part III\\*Overpayments and voluntary payments}

\renewcommand\parthead{--- Part III}

\subsection[8. Application of overpayments]{Application of overpayments}

8.---(1)  Where for any reason, including the retrospective effect of a maintenance calculation, there has been an overpayment of child support maintenance, the 
%Commission 
Secretary of State  % Words substituted (1.8.12) by SI 2012/2007 Sch para 121(7)(a)
may apply the amount overpaid to reduce any arrears of child support maintenance due under any previous maintenance calculation in respect of the same relevant persons.

(2) Where there is no previous maintenance calculation, or an amount of the overpayment remains after the application of paragraph~(1), the 
%Commission 
Secretary of State  % Words substituted (1.8.12) by SI 2012/2007 Sch para 121(7)(b)
may adjust the amount payable in relation to the current maintenance calculation by such amount as 
%it 
the Secretary of State  % Words substituted (1.8.12) by SI 2012/2007 Sch para 121(7)(b)
considers appropriate in all the circumstances of the case, having regard in particular to—
\begin{enumerate}\item[]
($a$) the circumstances of the relevant persons; and

($b$) the amount of the overpayment and the period over which it would be reasonable to adjust the amount payable for the overpayment to be rectified.
\end{enumerate}

(3) An adjustment of the amount payable in relation to the current maintenance calculation under paragraph~(2) may reduce the amount payable to nil.

\amendment{
Words substituted in reg.~8(1), (2) (1.8.12) by the Public Bodies (Child Maintenance and Enforcement Commission: Abolition and Transfer of Functions) Order 2012 Sch. para.~121(7).
}

\subsection[9. Application of voluntary payments]{Application of voluntary payments}

9.---(1)  Where there has been a voluntary payment\footnote{“Voluntary payment” is defined in section~54 of the 1991 Act, by reference to section~28J of that Act. The definition was inserted by section~26 of, and Schedule 3 to, the 2000 Act.} the 
%Commission 
Secretary of State  % Words substituted (1.8.12) by SI 2012/2007 Sch para 121(8)(a)
may apply the amount of the voluntary payment to reduce any arrears of child support maintenance due under any previous maintenance calculation in respect of the same relevant persons.

(2) Where there is no previous maintenance calculation, or an amount of the voluntary payment remains after the application of paragraph~(1), the 
%Commission 
Secretary of State  % Words substituted (1.8.12) by SI 2012/2007 Sch para 121(8)(b)
may adjust the amount payable in relation to the current maintenance calculation by such amount as 
%it 
the Secretary of State  % Words substituted (1.8.12) by SI 2012/2007 Sch para 121(8)(b)
considers appropriate in all the circumstances of the case, having regard in particular to—
\begin{enumerate}\item[]
($a$) the circumstances of the relevant persons; and

($b$) the amount of the voluntary payment and the period over which it would be reasonable to adjust the amount payable for the voluntary payment to be taken into account.
\end{enumerate}

(3) An adjustment of the amount payable in relation to the current maintenance calculation under paragraph~(2) may reduce the amount payable to nil.

\amendment{
Words substituted in reg.~9(1), (2) (1.8.12) by the Public Bodies (Child Maintenance and Enforcement Commission: Abolition and Transfer of Functions) Order 2012 Sch. para.~121(8).
}

\section[Part IV --- Recovery from estates]{Part IV\\*Recovery from estates}

\renewcommand\parthead{--- Part IV}

\subsection[10. Application and interpretation]{Application and interpretation}

10.---(1)  This Part applies in relation to the estate of a person who dies on or after the day on which these Regulations come into force.

(2) In this Part, “child support maintenance” means child support maintenance for the collection of which the 
%Commission 
Secretary of State  % Words substituted (1.8.12) by SI 2012/2007 Sch para 121(9)
is authorised to make arrangements.

\amendment{
Words substituted in reg.~10(2) (1.8.12) by the Public Bodies (Child Maintenance and Enforcement Commission: Abolition and Transfer of Functions) Order 2012 Sch. para.~121(9).
}

\subsection[11. Recovery of arrears from a deceased person’s estate]{Recovery of arrears from a deceased person’s estate}

11.  Arrears of child support maintenance for which a deceased person was liable immediately before death are a debt payable by the deceased’s executor or administrator out of the deceased’s estate to the 
%Commission 
Secretary of State%  % Words substituted (1.8.12) by SI 2012/2007 Sch para 121(10)
.

\amendment{
Words substituted in reg.~11 (1.8.12) by the Public Bodies (Child Maintenance and Enforcement Commission: Abolition and Transfer of Functions) Order 2012 Sch. para.~121(10).
}

\subsection[12. Appeals and other proceedings]{Appeals and other proceedings}

12.---(1)  The deceased’s executor or administrator has the same rights, subject to the same procedures and time limits, as the deceased person had immediately before death to institute, continue or withdraw any proceedings under the 1991 Act, whether by appeal or otherwise.

% Reg 12(2) omitted (28.10.13) by SI 2013/2380 reg 7(4)
%(2) Regulation 34 of the Decisions and Appeals Regulations shall apply to a case where the non-resident parent is the deceased party to the proceedings as if for paragraphs (1) and (2) there were substituted the following paragraph—
%\begin{quotation}
%“(1) In any proceedings, on the death of a non-resident parent, the 
%%Commission 
%Secretary of State  % Words substituted (1.8.12) by SI 2012/2007 Sch para 121(11)
%must appoint the deceased’s executor or administrator to proceed with the appeal in place of the deceased, unless there is no such person in which circumstances 
%%it 
%the Secretary of State  % Words substituted (1.8.12) by SI 2012/2007 Sch para 121(11)
%may appoint such person as 
%%it 
%the Secretary of State  % Words substituted (1.8.12) by SI 2012/2007 Sch para 121(11)
%thinks fit to proceed with the appeal.”.
%\end{quotation}

\amendment{
Words substituted in reg.~12(2) (1.8.12) by the Public Bodies (Child Maintenance and Enforcement Commission: Abolition and Transfer of Functions) Order 2012 Sch. para.~121(11).

Reg. 12(2) omitted (28.10.13) by the Social Security, Child Support, Vaccine Damage and Other Payments (Decisions and Appeals) (Amendment) Regulations 2013 reg.~7(4) (subject to transitional provisions in reg.~8(1)).
}

\subsection[13. Disclosure of information]{Disclosure of information}

13.---(1)  The 
%Commission 
Secretary of State  % Words substituted (1.8.12) by SI 2012/2007 Sch para 121(12)
may disclose information held for the purposes of the 1991 Act to the deceased’s executor or administrator where, in the opinion of the 
%Commission 
Secretary of State%  % Words substituted (1.8.12) by SI 2012/2007 Sch para 121(11)
, such information is essential to enable the executor or administrator to administer the deceased’s estate, including, where necessary, to institute, continue or withdraw proceedings under the 1991 Act.

(2) Any application for information under this regulation shall be made to the 
%Commission 
Secretary of State  % Words substituted (1.8.12) by SI 2012/2007 Sch para 121(11)
in writing setting out the reasons for the application.

(3) Except where a person gives written permission to the 
%Commission 
Secretary of State  % Words substituted (1.8.12) by SI 2012/2007 Sch para 121(11)
that the information mentioned in sub-paragraphs ($a$)  and ($b$)  in relation to that person may be disclosed to other persons, any information disclosed under paragraph~(1) must not contain—
\begin{enumerate}\item[]
($a$) the address of any person, except that of the recipient of the information in question and the office of the officer concerned who is exercising functions of the 
%Commission 
Secretary of State  % Words substituted (1.8.12) by SI 2012/2007 Sch para 121(11)
under the 1991 Act, or any other information the use of which could reasonably be expected to lead to any such person being located;

($b$) any other information the use of which could reasonably be expected to lead to any person, other than a party to the maintenance calculation, being identified.
\end{enumerate}

\amendment{
Words substituted in reg.~13 (1.8.12) by the Public Bodies (Child Maintenance and Enforcement Commission: Abolition and Transfer of Functions) Order 2012 Sch. para.~121(12).
}

% Parts IVA, IVB inserted (10.12.12) by SI 2012/3002 reg 2(3)
\section[Part IVA --- Part payment of arrears in full and final satisfaction]{Part IVA\\*Part payment of arrears in full and final satisfaction}

\renewcommand\parthead{--- Part IVA}

\amendment{
Pt.~IVA inserted (10.12.12) by the Child Support Management of Payments and Arrears (Amendment) Regulations 2012 reg.~2(3).
}

\subsection[13A. Interpretation of this Part]{Interpretation of this Part}

13A.  In this Part—
\begin{enumerate}\item[]
“appropriate person” means the person from whom the appropriate consent is required under section 41D(5) or (6) of the 1991 Act.
\end{enumerate}

\subsection[13B. Amounts owed to different persons to be treated separately]{Amounts owed to different persons to be treated separately}

13B.  Where the arrears of child support maintenance for which a person is liable comprise amounts that have accrued in respect of—
\begin{enumerate}\item[]
($a$) separate applications for a maintenance calculation; or

($b$) one application but would, if recovered, be payable to different persons,
\end{enumerate}
those amounts are to be treated as separate amounts of arrears for the purpose of exercising the power under section 41D(1) of the 1991 Act.

\subsection[13C. Appropriate consent]{Appropriate consent}

13C.---(1)  The Secretary of State may not exercise the power under section~41D(1) of the 1991 Act without the appropriate consent (as provided for in subsections (5) to (7) of section 41D), unless one of the following conditions applies—
\begin{enumerate}\item[]
($a$) that the Secretary of State would be entitled to retain the whole of the arrears under section 41(2) of the 1991 Act if it recovered them; or

($b$) that the Secretary of State would be entitled to retain part of the arrears under section 41(2) of that Act if it recovered them, and the part of the arrears that the Secretary of State would not be entitled to retain is equal to or less than the payment accepted under section~41D(1) of that Act.
\end{enumerate}

(2) Where the consent of any appropriate person is required, the Secretary of State must make available such information and guidance as the Secretary of State thinks appropriate for the purpose of helping that person decide whether to give that consent.

\subsection[13D. Agreement]{Agreement}

13D.---(1)  Where the Secretary of State proposes to exercise the power under section 41D(1) of the 1991 Act, the Secretary of State must prepare a written agreement.

(2) The agreement must—
\begin{enumerate}\item[]
($a$) name the non-resident parent, and where the consent of any appropriate person is required, the name of that person;

($b$) specify the amount of arrears to which the agreement relates and the period of liability to which those arrears relate;

($c$) state the amount that is agreed will be paid in satisfaction of those arrears;

($d$) state the method of payment and to whom payment will be made; and

($e$) state the day by which payment is to be made.
\end{enumerate}

(3) The Secretary of State must send the non-resident parent and, where applicable, the appropriate person, a copy of the agreement.

(4) The agreement does not take effect until—
\begin{enumerate}\item[]
($a$) the non-resident parent has agreed in writing to its terms; and

($b$) where applicable, the appropriate person has given to the Secretary of State their consent in writing.
\end{enumerate}

\subsection[13E. Where payment is received]{Where payment is received}

13E.---(1)  Unless the non-resident parent fails to comply with the terms of the agreement, the Secretary of State must not take action to recover any of the arrears to which the agreement relates.

(2) Where the non-resident parent has made full payment in accordance with the agreement all remaining liability in respect of the arrears of child support maintenance to which the agreement relates is extinguished.

(3) Where the non-resident parent fails to make any payment or only makes part payment or otherwise fails to adhere to the terms of the agreement, the non-resident parent remains liable to pay the full amount of any outstanding arrears to which the agreement relates and the Secretary of State may arrange to recover any of those outstanding arrears in accordance with the 1991 Act.

(4) Nothing in these Regulations prevents the Secretary of State from entering into a new agreement with the non-resident parent in respect of any of the arrears to which the previous agreement relates provided that the new agreement complies with the requirements set out in regulation 13D.

(5) Where the Secretary of State enters into a new agreement with the non-resident parent in respect of any of the arrears to which a previous agreement related, the previous agreement ceases to have effect on the coming into effect of that new agreement.

\section[Part IVB --- Write off of arrears]{Part IVB\\*Write off of arrears}

\renewcommand\parthead{--- Part IVB}

\amendment{
Pt.~IVB inserted (10.12.12) by the Child Support Management of Payments and Arrears (Amendment) Regulations 2012 reg.~2(3).
}

\subsection[13F. Amounts owed to different persons to be treated separately]{Amounts owed to different persons to be treated separately}

13F.  Where the arrears of child support maintenance for which a person is liable comprise amounts that have accrued in respect of—
\begin{enumerate}\item[]
($a$) separate applications for a maintenance calculation; or

($b$) one application, but would, if recovered, be payable to different persons,
\end{enumerate}
those amounts are to be treated as separate amounts of arrears for the purpose of exercising the power under section 41E(1) of the 1991 Act.

\subsection[13G. Circumstances in which the Secretary of State may exercise the power in section 41E of the 1991 Act]{Circumstances in which the Secretary of State may exercise the power in section 41E of the 1991 Act}

\begin{sloppypar}
13G.  The circumstances of the case specified for the purposes of section~41E(1)($a$)  of the 1991 Act are that—
\end{sloppypar}
\begin{enumerate}\item[]
($a$) the person with care has requested under section 4(5) of that Act that the Secretary of State ceases to act in respect of the arrears;

($b$) a child in Scotland has requested under section 7(6) of that Act that the Secretary of State ceases to act in respect of the arrears;

($c$) the person with care, or (in Scotland) the child, has died;

($d$) the non-resident parent died before 25 January 2010 or there is no further action that can be taken with regard to recovery of the arrears from the non-resident parent’s estate under Part~IV;

($e$) the arrears relate to liability for child support maintenance for any period in respect of which an interim maintenance assessment was in force between 5 April 1993 and 18 April 1995; or

($f$) the non-resident parent has been informed by the Secretary of State that no further action would ever be taken to recover those arrears.
\end{enumerate}

\subsection[13H. Secretary of State required to give notice]{Secretary of State required to give notice}

13H.---(1)  Where the Secretary of State is considering exercising the powers under section 41E(1) of the 1991 Act, the Secretary of State must send written notice to the person with care or, where relevant, a child in Scotland and the non-resident parent.

(2) The requirement in paragraph (1) does not apply where the person in question cannot be traced or has died.

(3) The notice must—
\begin{enumerate}\item[]
($a$) specify the person with care or, where relevant, a child in Scotland, in respect of whom liability in respect of arrears of child support maintenance has accrued;

($b$) specify the amount of the arrears and the period of liability to which the arrears relate;

($c$) state why it appears to the Secretary of State that it would be unfair or inappropriate to enforce liability in respect of the arrears;

($d$) advise the person that they may make representations, within 30 days of receiving the notice, to the Secretary of State as to whether the liability in respect of the arrears should be extinguished; and

($e$) explain the effect of any decision to extinguish liability in respect of any arrears of child support maintenance under section 41E(1) of the 1991 Act.
\end{enumerate}

(4) If no representations are received by the Secretary of State within 30 days of the notice being received by the person with care or, where relevant, a child in Scotland and the non-resident parent, the Secretary of State may make the decision to extinguish the arrears.

(5) For the purposes of this regulation, where the Secretary of State sends any written notice by post to a person’s last known or notified address that document is treated as having been received by that person on the second day following the day on which it is posted.

\subsection[13I. Secretary of State to take account of the parties’ views]{Secretary of State to take account of the parties’ views}

13I.  Where the Secretary of State receives representations within the 30 day period referred to in regulation 13H(3)($d$), the Secretary of State must take account of those representations in making a decision under section 41E(1) of the 1991 Act.

\subsection[13J. Notification of decision to write off]{Notification of decision to write off}

13J.---(1)  On making a decision under section 41E(1) of the 1991 Act, the Secretary of State must send written notification to the non-resident parent and the person with care or, where relevant, a child in Scotland, of that decision.

(2) The requirement in paragraph (1) does not apply where the person in question cannot be traced or has died.

\section[Part V --- Revocations and savings]{Part V\\*Revocations and savings}

\renewcommand\parthead{--- Part V}

\subsection[14. Revocations]{Revocations}

14.  The Regulations specified in the Schedule are revoked to the extent specified.

\subsection[15. Savings]{Savings}

15.---(1)  Where before these Regulations come into force, an adjustment has been made under regulation 10(1) of the AIMA Regulations in a 1993 scheme case, regulations 10(2) and (3) and 11 to 17\footnote{Regulation 10(1) was substituted by S.I.~1995/1045 and amended by S.I.~1999/1510. Regulation 10(2) and (3) was amended by S.I.~1999/1510. Regulation 11 was substituted by S.I.~1995/1045 and amended by S.I.~1999/1510. Regulations 12 to 17 were substituted by S.I.~1999/1510 and regulation 14 was amended by S.I.~2008/2683.} of those Regulations continue to apply to that case for the purposes of—
\begin{enumerate}\item[]
($a$) making and determining any appeal against the adjustment;

($b$) making and determining any application for a revision of the adjustment;

($c$) determining any application for a supersession made before these regulations come into force.
\end{enumerate}

(2) Where before these Regulations come into force, an adjustment has been made under regulation 10(1) or (3A)\footnote{Regulation 10(1) was substituted by S.I.~1995/1045 and amended by S.I.~1999/1510 and 2001/162, in relation to cases other than 1993 scheme cases. Regulations 10(3A) was inserted by S.I.~2001/162.} of the AIMA Regulations in a case other than a 1993 scheme case, regulation 30A\footnote{Regulation 30A was inserted by S.I.~2000/3185 and amended by S.I.~2008/2683 and S.I.~2009/396.} of the Decisions and Appeals Regulations continues to apply to that case for the purposes of making and determining any appeal against the adjustment. 

\bigskip

\pagebreak[3]

Signed 
by authority of the 
Secretary of State for~Work and~Pensions.
%I concur
%By authority of the Lord Chancellor

{\raggedleft
\emph{Helen Goodman}\\*
%Secretary
%Minister
Parliamentary Under-Secretary 
of State\\*Department 
for~Work and~Pensions

}

30th November 2009

\small

\clearpage

\part[Schedule --- Revocations]{Schedule\\*Revocations}

\renewcommand\parthead{--- Schedule}

%\begin{tabulary}{\linewidth}{JlJ}
\begin{longtable}{p{149.63034pt}p{72.78448pt}p{104.93481pt}}
\hline
\itshape Regulations revoked	& \itshape References	& \itshape Extent of revocation\\
\hline
\endhead
\hline
\endlastfoot
Child Support (Arrears, Interest and Adjustment of Maintenance Assessments) Regulations 1992	&S.I.~1992/1816	&Regulations 2 to 7, 9, 10 and 11 to 17\footnote{Regulations 10(2) and (3) and 11 to 17 were revoked by regulation 14 of S.I.~2000/3185, but it has not come into force in relation to 1993 scheme cases as sections 16, 17 and 20 of the 1991 Act, as amended by the 2000 Act, have not come into force in relation to those cases (see regulation 1(1) of S.I.~2000/3185). Regulations 3, 4, 6 and 7 were omitted by regulation 5(3)($e$)  of S.I.~2001/162, but it has not come into force in relation to 1993 scheme cases as relevant provisions of the 2000 Act have not come into force in relation to those cases (see regulation 1(3) of S.I.~2001/162).}.\\
Child Support (Miscellaneous Amendments) Regulations 1993	&S.I.~1993/913	&Regulations 35 to 40.\\
Child Support and Income Support (Amendment) Regulations 1995	&S.I.~1995/1045	&Regulations 7 to 11.\\
Social Security and Child Support (Decisions and Appeals) Regulations 1999	&S.I.~1999/991	&Regulation 30A.\\
Child Support (Decisions and Appeals) (Amendment) Regulations 2000	&S.I.~2000/3185	
&
Regulation 10, insofar as it inserts regulation 15D in S.I.~1999/991\footnote{Regulation 10 inserts regulation 15D into S.I.~1999/991, but it has not yet come into force in relation to 1993 scheme cases as sections 16, 17 and 20 of the 1991 Act, as amended by the 2000 Act, have not come into force in relation to those cases (see regulation 1(1) of S.I.~2000/3185). S.I.~2009/396 omits regulation 15D of the Decisions and Appeals Regulations in relation to those cases in relation to which that regulation has already come into force.}.
\\
&&Regulation 12.\\
Child Support (Collection and Enforcement and Miscellaneous Amendments) Regulations 2000	&S.I.~2001/162	&Regulation 5(3)($b$), ($c$)  and ($e$)  and (4)($d$).\\
Child Support (Miscellaneous Amendments) Regulations 2009	&S.I.~2009/396	&\hbadness=10000 Regulations 3 and 4(15).\\
%\end{tabulary}
\end{longtable}

\clearpage

\part{Explanatory Note}

\renewcommand\parthead{— Explanatory Note}

\subsection*{(This note is not part of the Regulations)}

These Regulations are made under powers in the Child Support Act 1991 (c.~19) (“the 1991 Act”) and come into force on 25th January 2010. They are, in part, consolidating regulations which revoke and re-enact some provisions of the Child Support (Arrears, Interest and Adjustment of Maintenance Assessments) Regulations 1992 (S.I.~1992/1816) (“the AIMA Regulations”), with some changes.

Regulations 3 and 4 re-enact regulations 2 and 9 of the AIMA Regulations. Regulation 3 requires the Commission to serve a notice on a non-resident parent where it is considering taking action in relation to arrears of child support maintenance due. Regulation 4 allows the Commission to attribute any payment of child support maintenance made by the non-resident parent to child support maintenance due as it thinks fit.

Regulations 5 and 6 set out the circumstances in which the Commission may set off an amount against a person’s liability to pay child support maintenance. There are 2 situations in which set off may occur. Regulation 5 provides that where the parent with care and the non-resident parent each owes child support maintenance to the other, the Commission may set off one person’s liability against the other person’s liability. Regulation 6 allows prescribed payments made by a non-resident parent to be set off against their liability. Regulation 7 makes provision as to how any amount should be set off against that liability.

Regulations 8 and 9 provide for the adjustment of arrears and amounts of child support maintenance payable to take account of overpayments and voluntary payments. They re-enact regulation 10 of the AIMA regulations, with a change which allows the amount payable to be reduced to nil.

Regulation 10 limits the application of Part 4 to those cases where the Commission is authorised to collect child support maintenance and the person dies on or after the date these Regulations come into force.

Regulation 11 provides that arrears of child support maintenance owed by a deceased person immediately before death are a debt payable by the deceased’s executor or administrator out of the deceased’s estate.

Regulation 12(1) provides for the executor or administrator of the estate to have the same rights as the deceased person prior to death to institute, continue or withdraw proceedings under the 1991 Act, whether by way of appeal or otherwise. Regulation 12(2) modifies regulation 34 of the Social Security and Child Support (Decisions and Appeals) Regulations 1999 (S.I.~1999/991) (“the Decisions and Appeals Regulations”) so that the Commission must appoint a deceased non-resident parent’s executor or administrator to proceed with any appeal, unless there is no such person in which case the Commission may appoint such person as it thinks fit.

Regulation 13 makes provision for the disclosure of information to the deceased’s executor or administrator where it is essential, in the Commission’s opinion, for the proper administration of the estate, including the bringing, continuing or withdrawing of proceedings under the 1991 Act.

Regulation 14, and the Schedule, revokes various provisions in the AIMA Regulations and related provisions in the Decisions and Appeal Regulations, some of which deal with the appeal of decisions to adjust the amount payable to take account of an overpayment or voluntary payment.

Regulation 15 saves the relevant provisions for specified purposes where the decision to adjust the amount payable was made before the coming into force of these Regulations.

A full impact assessment has not been published for this instrument as it has no impact on the private or voluntary sectors. 

\end{document}