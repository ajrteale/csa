\documentclass[12pt,a4paper]{article}

\newcommand\regstitle{The Child Support (Temporary Compensation Payment Scheme) Regulations 2000}

\newcommand\regsnumber{2000/3174}

%\opt{newrules}{
\title{\regstitle}
%}

%\opt{2012rules}{
%\title{Child Maintenance and Other Payments Act 2008\\(2012 scheme version)}
%}

\author{S.I. 2000 No. 3174}

\date{Made 30th November 2000\\Laid before Parliament 6th December 2000\\Coming into force 31st January 2001}

%\opt{oldrules}{\newcommand\versionyear{1993}}
%\opt{newrules}{\newcommand\versionyear{2003}}
%\opt{2012rules}{\newcommand\versionyear{2012}}

\usepackage{csa-regs}

\setlength\headheight{27.57402pt}

\begin{document}

\maketitle

\noindent
The Secretary of State for Social Security, in exercise of the powers conferred upon him by section~27(1) to (4) and (10) of the Child Support, Pensions and Social Security Act 2000\footnote{\frenchspacing 2000 c. 19.} and of all other powers enabling him in that behalf, hereby makes the following Regulations: 

{\sloppy

\tableofcontents

}

\bigskip

\setcounter{secnumdepth}{-2}

\subsection[1. Citation, commencement and interpretation]{Citation, commencement and interpretation}

1.---(1)  These Regulations may be cited as the Child Support (Temporary Compensation Payment Scheme) Regulations 2000 and shall come into force on 31st January 2001.

\pagebreak[3]

(2) In these Regulations, unless the context otherwise requires—
\begin{enumerate}\item[]
“the 2000 Act” means the Child Support, Pensions and Social Security Act 2000;

“the Child Support Act” means the Child Support Act 1991 before its amendment by the 2000 Act\footnote{\frenchspacing 1991 c. 48.}; and

“the Social Security Act” means the Social Security Act 1998\footnote{\frenchspacing 1998 c. 14.}.
\end{enumerate}

\subsection[2. Application of the Regulations]{Application of the Regulations}

2.---(1)  For the purposes of section~27(2) of the 2000 Act, section~27 shall have effect as if it were modified so as to apply to cases of arrears of child support maintenance which have become due under a fresh maintenance assessment made in the following circumstances:
\begin{enumerate}\item[]
($a$) where the Secretary of State has given a departure direction under section~28F of the Child Support Act and—
\begin{enumerate}\item[]
(i) the revised amount is higher than the current amount; and

(ii) the effective date of the fresh maintenance assessment is a date before 1st June 1999; or
\end{enumerate}

($b$) following a review under section~18 of the Child Support Act (reviews of decisions of child support officers) or a review under section~19 of that Act (reviews at instigation of child support officers) (as those provisions had effect before their substitution by section~41 of the Social Security Act),
\end{enumerate}
and the effective date of the assessment is earlier than the date on which the assessment was made; or
\begin{enumerate}\item[]
($c$) following an appeal to 
%a child support appeal tribunal 
the First-tier Tribunal  % Words substituted (3.11.08) by SI 2008/2683 Sch 1 para 141
under section~20 of the Child Support Act (as it had effect before its substitution by section~42 of the Social Security Act) against a decision of a child support officer.
\end{enumerate}

(2) In this regulation—
\begin{enumerate}\item[]
“current amount” means the amount of child support maintenance fixed by the current assessment; and

“revised amount” means the amount of child support maintenance fixed by the fresh maintenance assessment as a result of the departure direction given by the Secretary of State.
\end{enumerate}

\amendment{
Words substituted in reg.~2(1)(c) (3.11.08) by the Tribunals, Courts and Enforcement Act 2007 (Transitional and Consequential Provisions) Order 2008 Sch.~1 para.~141.
}

\subsection[3. Prescribed date]{Prescribed date}

3.  For the purposes of section~27(1)($a$)  of the 2000 Act, the prescribed date is 1st April 
%2002
2005%  % Word substituted (17.7.02) by SI 2002/1854 reg 3
.

\amendment{
Word substituted in reg. 3 (17.7.02) by the Child Support (Temporary Compensation Payment Scheme) (Modification and Amendment) Regulations 2002 reg. 3.
}

\subsection[4. Prescribed circumstances]{Prescribed circumstances}

4.---(1)  In relation to cases of arrears which have become due under a maintenance assessment falling within section~27(1)($a$)  of the 2000 Act or a fresh maintenance assessment falling within section~27(1)($b$)  of the 2000 Act or regulation 2(1), the prescribed circumstances for the purposes of section~27(3) of the 2000 Act are that—
\begin{enumerate}\item[]
($a$) more than 6 months of arrears of child support maintenance have become due under the maintenance assessment;

($b$) at least 3 months of those arrears are due to unreasonable delay due to an act or omission by the Secretary of State or a child support officer as the case may be;

($c$) the Secretary of State is authorised under section~29(1) of the Child Support Act to arrange for the collection of child support maintenance payable in accordance with the maintenance assessment;

($d$) the Secretary of State is satisfied that the absent parent is, at the time the agreement is made, making such payments as are required of him in accordance with regulations made under section~29(3)($b$)  or ($c$)  of the Child Support Act;

($e$) where the absent parent is liable to make child support maintenance payments under a different maintenance assessment, there are no existing arrears in relation to any of them at the time the agreement is made, except for those arrears that the Secretary of State is satisfied have arisen through no fault of the absent parent; and

($f$) in relation to cases under section~27(1)($b$)  of the 2000 Act or regulation 2(1), the absent parent has paid any arrears which he has been required to pay in relation to the maintenance assessment, or has done so except in relation to—
\begin{enumerate}\item[]
(i) arrears of at least 3 months which are due to unreasonable delay due to an act or omission of the Secretary of State or a child support officer as the case may be; or

(ii) any other arrears that the Secretary of State is satisfied have arisen through no fault of the absent parent.
\end{enumerate}
\end{enumerate}

(2) In this regulation “agreement” means an agreement under section~27 of the 2000 Act.

\subsection[5. Terms of the agreement]{Terms of the agreement}

5.---(1)  For the purposes of section~27(4) of the 2000 Act, the terms which may be specified in the agreement are—
\begin{enumerate}\item[]
($a$) the period of the agreement;

($b$) payment of the child support maintenance payable in accordance with the maintenance assessment and, where relevant, the arrears, by whichever of the following methods the Secretary of State specifies as being appropriate in the circumstances—
\begin{enumerate}\item[]
(i) by standing order;

(ii) by any other method which requires one person to give his authority for payments to be made from an account of his to an account of another’s on specific dates during the period for which the authority is in force and without the need for further authority from him;

(iii) by an arrangement whereby one person gives his authority for payments to be made from an account of his, or on his behalf, to another person or to an account of that other person;

(iv) by cheque or postal order;

(v) in cash;

(vi) by debit card;

(vii) where the Secretary of State has made a deduction from earnings order under section~31 of the Child Support Act—
\begin{enumerate}\item[]
($aa$) by cheque;

($bb$) by automated credit transfer; or

($cc$) by such other method as the Secretary of State may specify;
\end{enumerate}
\end{enumerate}

($c$) the amount of the arrears that the absent parent is required to pay (which shall include at least the last 6 months of the arrears due under the maintenance assessment);

($d$) the day and interval by reference to which payments of the arrears are to be made by the absent parent; and

($e$) the confirmation by the Secretary of State that he will not, while the agreement is complied with, take action to recover any of the arrears.
\end{enumerate}

(2) In this regulation “debit card” means a card, operating as a substitute for a cheque, that can be used to obtain cash or to make a payment at a point of sale whereby the card holder’s bank or building society account is debited without deferment of payment. 

\bigskip

Signed 
by authority of the Secretary of State for Social Security.

{\raggedleft
\emph{P.~Hollis}\\*Parliamentary Under-Secretary of State,\\*Department of Social Security

}

30th November 2000

\small

\part{Explanatory Note}

\renewcommand\parthead{--- Explanatory Note}

\subsection*{(This note is not part of the Regulations)}

These Regulations provide for a temporary compensation scheme made under section~27 of the Child Support, Pensions and Social Security Act 2000 (“the Act”) in certain cases where there has been a delay in the making of a maintenance assessment under the Child Support Act 1991 leading to arrears of child support maintenance.

Regulation 1 contains interpretation provisions.

Regulation 2 sets out the cases additional to those in the Act to which the scheme will apply.

Regulation 3 provides the prescribed date for the purposes of section~27(1)($a$)  of the Act.

Regulation 4 prescribes the circumstances in which the Secretary of State may agree that the absent parent will not be required to pay the whole of the arrears due under a maintenance assessment and the Secretary of State will not seek to recover any of the arrears.

Regulation 5 prescribes the terms of the agreement when the scheme shall apply.

The impact on business of these Regulations was covered in the Regulatory Impact Assessment (RIA) for the 2000 Act in accordance with which, and in consequence of which, these Regulations are made. A copy of that RIA has been placed in the libraries of both Houses of Parliament and can be obtained from the Department of Social Security, Regulatory Impact Unit, Adelphi, 1--11 John Adam Street, London \textsc{\lowercase{WC2N 6HT}}.  

\end{document}