\documentclass[12pt,a4paper]{article}

\newcommand\regstitle{The Social Security and Child Support (Decisions and Appeals) Amendment Regulations 2000}

\newcommand\regsnumber{2000/119}

%\opt{newrules}{
\title{\regstitle}
%}

%\opt{2012rules}{
%\title{Child Maintenance and Other Payments Act 2008\\(2012 scheme version)}
%}

\author{S.I. 2000 No. 119}

\date{Made
20th January 2000\\
Laid before Parliament
27th January 2000\\
Coming into force
17th February 2000}

%\opt{oldrules}{\newcommand\versionyear{1993}}
%\opt{newrules}{\newcommand\versionyear{2003}}
%\opt{2012rules}{\newcommand\versionyear{2012}}

\usepackage{csa-regs}

\setlength\headheight{27.57402pt}

\begin{document}

\maketitle

\noindent
The Secretary of State for Social Security, in exercise of the powers conferred upon him by sections 10(6), 79(1), (3) and (4) and 84 of the Social Security Act 1998\footnote{\frenchspacing 1998 c. 14. Section 84 is cited because of the meaning ascribed to the word “prescribe”.} and of all other powers enabling him in that behalf, after agreement by the Social Security Advisory Committee that proposals to make these Regulations should not be referred to it\footnote{\frenchspacing See sections 170 and 173(1)($b$)  of the Social Security Administration Act 1992 (c. 5); section 10 of the Social Security Act 1998 is a relevant enactment for the purposes of section 170 by virtue of the amendment of section 170(5) of the 1992 Act by the Social Security Act 1998, Schedule 7, paragraph 104($a$).}, hereby makes the following Regulations: 

{\sloppy

\tableofcontents

}

\bigskip

\setcounter{secnumdepth}{-2}

\subsection[1. Citation, commencement and interpretation]{Citation, commencement and interpretation}

1.---(1)  These Regulations may be cited as the Social Security and Child Support (Decisions and Appeals) Amendment Regulations 2000 and shall come into force on 17th February 2000.

(2) In these Regulations, “the principal Regulations” means the Social Security and Child Support (Decisions and Appeals) Regulations 1999\footnote{\frenchspacing S.I. 1999/991; relevant amending instruments are S.I. 1999/1623; 2677 and 3178 (C. 81).}.

\subsection[2. Amendment of regulation 7 of the principal Regulations]{Amendment of regulation 7 of the principal Regulations}

2.  For paragraph (9) of regulation 7 of the principal Regulations (date from which a decision superseded under section 10 of the Social Security Act 1998 takes effect) there shall be substituted the following paragraph—
\begin{quotation}
“(9) A decision relating to attendance allowance or disability living allowance which is advantageous to the claimant and which is made under section 10 on the basis of a relevant change of circumstances shall take effect from—
\begin{enumerate}\item[]
($a$) where the decision is made on the Secretary of State’s own initiative, the date of that decision;

($b$) where—
\begin{enumerate}\item[]
(i) the change is relevant to the question of entitlement to a particular rate of benefit; and

(ii) the claimant notifies the change before a date one month after he satisfied the conditions of entitlement to that rate or within such longer period as may be allowed under regulation 8,
\end{enumerate}
the first pay day (as specified in Schedule 6 to the Claims and Payments Regulations\footnote{\frenchspacing S.I. 1987/1968; the relevant amending instrument is S.I. 1991/2741.}) after he satisfied those conditions;

($c$) where—
\begin{enumerate}\item[]
(i) the change is relevant to the question of whether benefit is payable; and

(ii) the claimant notifies the change before a date one month after the change or within such longer period as may be allowed under regulation 8,
\end{enumerate}
the first pay day (as specified in Schedule 6 to the Claims and Payments Regulations) after the change occurred; or

($d$) in any other case, the date of the application for the superseding decision.”.
\end{enumerate}
\end{quotation}

\subsection[3. Amendment of regulation 8 of the principal Regulations]{Amendment of regulation 8 of the principal Regulations}

3.  In regulation 8 of the principal Regulations (effective date for the late notification of change of circumstances)—
\begin{enumerate}\item[]
($a$) in paragraph (1) after the words “regulation 7(2)” there shall be inserted the words “and (9)”; and

($b$) in paragraphs (2) and (5) after the words “regulation 7(2)” there shall be inserted the words “or (9)”. 
\end{enumerate}

\bigskip

Signed 
by authority of the Secretary of State for Social Security.

{\raggedleft
\emph{Angela Eagle
}\\*Parliamentary Under-Secretary of State,\\*Department of Social Security

}

20th January 2000

\small

\part{Explanatory Note}

\renewcommand\parthead{--- Explanatory Note}

\subsection*{(This note is not part of the Regulations)}

These Regulations amend the Social Security and Child Support (Decisions and Appeals) Regulations 1999 (S.I.\ 1999/991) (“the principal Regulations”).

Regulation 2 amends regulation 7(9) of the principal Regulations which makes provision in relation to attendance allowance cases and disability living allowance cases for the effective date of a supersession (under section 10 of the Social Security Act 1998 (c.\ 14)) to a claimant’s advantage where there has been a relevant change of circumstances.

Regulation 3 amends regulation 8 of the principal Regulations to include a reference to regulation 7(9).

These Regulations do not impose a charge on business. 

\end{document}