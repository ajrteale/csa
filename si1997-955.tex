\documentclass[12pt,a4paper]{article}

\newcommand\regstitle{The Social Security (Adjudication) and Commissioners Procedure and Child Support Commissioners (Procedure) Amendment Regulations 1997}

\newcommand\regsnumber{1997/955}

%\opt{newrules}{
\title{\regstitle}
%}

%\opt{2012rules}{
%\title{Child Maintenance and Other Payments Act 2008\\(2012 scheme version)}
%}

\author{S.I. 1997 No. 955}

\date{Made 18th March 1997\\Laid before Parliament 20th March 1997\\Coming into force 28th April 1997}

%\opt{oldrules}{\newcommand\versionyear{1993}}
%\opt{newrules}{\newcommand\versionyear{2003}}
%\opt{2012rules}{\newcommand\versionyear{2012}}

\usepackage{csa-regs}

\setlength\headheight{42.07402pt}

\begin{document}

\maketitle

\noindent
The Lord Chancellor, in relation to regulations 7 to 13, and regulation 1 in so far as it relates to those regulations, in exercise of the powers conferred upon him by sections 22(3) and (4), 24(6) and (7) and 52 of the Child Support Act 1991\footnote{\frenchspacing 1991 c. 48.} and sections 23(10), 34(4), 48(3) and 189(2) of the Social Security Administration Act 1992\footnote{\frenchspacing 1992 c. 5.} and of all other powers enabling him in that behalf, after consultation with the Lord Advocate; and the Secretary of State for Social Security, in relation to regulations 2 to 6, and regulation 1 in so far as it relates to those regulations, in exercise of the powers conferred upon him by sections 23(10), 34(4) and 48(3) of the Social Security Administration Act 1992 and of all other powers enabling him in that behalf; after consultation with the Council on Tribunals in accordance with section 8(1) of the Tribunals and Inquiries Act 1992\footnote{\frenchspacing 1992 c. 53.}, hereby make the following Regulations:

{\sloppy

\tableofcontents

}

\bigskip

\setcounter{secnumdepth}{-2}

\subsection[1. Citation, commencement and interpretation]{Citation, commencement and interpretation}

1.—(1) These Regulations may be cited as the Social Security (Adjudication) and Commissioners Procedure and Child Support Commissioners (Procedure) Amendment Regulations 1997 and shall come into force on 28th April 1997.

(2) In these Regulations—
\begin{enumerate}\item[]
“the Adjudication Regulations” means the Social Security (Adjudication) Regulations 1995\footnote{\frenchspacing S.I. 1995/1801. The relevant amending instruments are S.I. 1996/1518 and 1996/2450.};

“the Child Support Commissioners Procedure Regulations” means the Child Support Commissioners (Procedure) Regulations 1992\footnote{\frenchspacing S.I. 1992/2640.};

“the Commissioners Procedure Regulations” means the Social Security Commissioners Procedure Regulations 1987\footnote{\frenchspacing S.I. 1987/214. The relevant amending instrument is S.I. 1992/1121.}.
\end{enumerate}

\subsection[2. Amendment of regulation 1 of the Adjudication Regulations]{Amendment of regulation 1 of the Adjudication Regulations}

2.  In paragraph (2) of regulation 1 of the Adjudication Regulations (citation, commencement and interpretation) there shall be inserted after the definition of “disability question” the following definition—
\begin{quotation}
““full statement of the tribunal’s decision” means the statement referred to in regulations 23(3A), 29(6A) and 38(5A);”.
\end{quotation}

\subsection[3. Amendment of regulation 24 of the Adjudication Regulations]{Amendment of regulation 24 of the Adjudication Regulations}

3.—(1) Regulation 24 of the Adjudication Regulations (application for leave to appeal to a Commissioner from an appeal tribunal) shall be amended in accordance with the following provisions of this regulation.

(2) For paragraph (1) there shall be substituted the following paragraph—
\begin{quotation}
“(1) Subject to the following provisions of this regulation, an application to the chairman of an appeal tribunal for leave to appeal to a Commissioner from a decision of an appeal tribunal shall—
\begin{enumerate}\item[]
($a$) be made in accordance with regulation 3 and Schedule 2; and

($b$) have annexed to it a copy of the full statement of the tribunal’s decision.”.
\end{enumerate}
\end{quotation}

(3) In paragraph (3) the words “made under paragraph (1)($a$) shall be recorded in the record of the proceedings of the tribunal, and on an application under paragraph (1)($b$)” shall be omitted.

\subsection[4. Amendment of regulation 32 of the Adjudication Regulations]{Amendment of regulation 32 of the Adjudication Regulations}

4.  For paragraph (1) of regulation 32 of the Adjudication Regulations (application for leave to appeal to a Commissioner from a disability appeal tribunal) there shall be substituted the following paragraph—
\begin{quotation}
“(1) Subject to the following provisions of this regulation, an application to the chairman of a disability appeal tribunal for leave to appeal to a Commissioner from a decision of a disability appeal tribunal shall—
\begin{enumerate}\item[]
($a$) be made in accordance with regulation 3 and Schedule 2; and

($b$) have annexed to it a copy of the full statement of the tribunal’s decision.”.
\end{enumerate}
\end{quotation}

\subsection[5. Amendment of regulation 39 of the Adjudication Regulations]{Amendment of regulation 39 of the Adjudication Regulations}

5.  For paragraph (1) of regulation 39 of the Adjudication Regulations (application for leave to appeal to a Commissioner from a medical appeal tribunal) there shall be substituted the following paragraph—
\begin{quotation}
“(1) Subject to the following provisions of this regulation, an application to the chairman of a medical appeal tribunal for leave to appeal to a Commissioner from a decision of a medical appeal tribunal shall—
\begin{enumerate}\item[]
($a$) be made in accordance with regulation 3 and Schedule 2; and

($b$) have annexed to it a copy of the full statement of the tribunal’s decision.”.
\end{enumerate}
\end{quotation}

\subsection[6. Amendment of Schedule 2 to the Adjudication Regulations]{Amendment of Schedule 2 to the Adjudication Regulations}

6.—(1) Schedule 2 to the Adjudication Regulations (time limits for making applications, appeals or references) shall be amended in accordance with the following provisions of this regulation.

(2) In paragraph 7, in column (3), for the words “a copy of the record of the decision was given to the applicant” there shall be substituted the words “a copy of the full statement of the tribunal’s decision was given or sent to the applicant”.

(3) In paragraph 8, in column (3), for the words “a copy of the record of the decision was given to the applicant” there shall be substituted the words “a copy of the full statement of the tribunal’s decision was given or sent to the applicant”.

(4) In paragraph 9, in column (3), for the words “a copy of the record of the decision was given to the applicant” there shall be substituted the words “a copy of the full statement of the tribunal’s decision was given or sent to the applicant”.

\subsection[7. Amendment of regulation 2 of the Commissioners Procedure Regulations]{Amendment of regulation 2 of the Commissioners Procedure Regulations}

7.  In regulation 2 of the Commissioners Procedure Regulations (interpretation) there shall be inserted after the definition of “forfeiture rule question” the following definition—
\begin{quotation}
““full statement of the tribunal’s decision” has the same meaning as in the Social Security (Adjudication) Regulations 1995\footnote{\frenchspacing S.I. 1995/1801. The definition of “full statement of the tribunal’s decision” was inserted by the Social Security (Adjudication) and Commissioners Procedure and Child Support Commissioners (Procedure) Amendment Regulations 1997.};”.
\end{quotation}

\subsection[8. Amendment of regulation 4 of the Commissioners Procedure Regulations]{Amendment of regulation 4 of the Commissioners Procedure Regulations}

8.—(1) Regulation 4 of the Commissioners Procedure Regulations (notice of application to a Commissioner for leave to appeal) shall be amended in accordance with the following provisions of this regulation.

(2) In paragraph (1), for the words “a copy of the decision against which leave to appeal is being sought” there shall be substituted the words “a copy of the full statement of the tribunal’s decision against which leave to appeal is being sought”.

(3) In paragraph (2), the words “, otherwise than by a decision recorded in the record of the proceedings of the tribunal,” shall be omitted.

\subsection[9. Amendment of regulation 6 of the Commissioners Procedure Regulations]{Amendment of regulation 6 of the Commissioners Procedure Regulations}

9.  In regulation 6 of the Commissioners Procedure Regulations (notice of appeal) there shall be substituted for the word “decision” the words “full statement of the tribunal’s decision”.

\subsection[10. Amendment of regulation 1 of the Child Support Commissioners Procedure Regulations]{Amendment of regulation 1 of the Child Support Commissioners Procedure Regulations}

10.  In paragraph (2) of regulation 1 of the Child Support Commissioners Procedure Regulations (citation, commencement and interpretation) after the definition of “Commissioner” there shall be inserted the following definition—
\begin{quotation}
““full statement of the tribunal’s decision” means the statement referred to in regulation 13(3A) of the Child Support Appeal Tribunals (Procedure) Regulations 1992\footnote{\frenchspacing S.I. 1992/2641. Paragraph (3A) of regulation 13 was inserted by S.I. 1996/182, substituted by S.I. 1996/2450 and amended by S.I. 1996/2907.};”.
\end{quotation}

\subsection[11. Amendment of regulation 2 of the Child Support Commissioners Procedure Regulations]{Amendment of regulation 2 of the Child Support Commissioners Procedure Regulations}

11.  In sub-paragraph ($a$) of paragraph (1) of regulation 2 of the Child Support Commissioners Procedure Regulations (application to the chairman of an appeal tribunal or to a Commissioner for leave to appeal to a Commissioner) for the words “notice of the decision of the tribunal” there shall be substituted the words “the full statement of the tribunal’s decision”.

\subsection[12. Amendment of regulation 3 of the Child Support Commissioners Procedure Regulations]{Amendment of regulation 3 of the Child Support Commissioners Procedure Regulations}

12.  In paragraph (1) of regulation 3 of the Child Support Commissioners Procedure Regulations (notice of application for leave to appeal to a Commissioner) for the words “a copy of the decision against which leave to appeal is being sought” there shall be substituted the words “a copy of the full statement of the tribunal’s decision against which leave to appeal is being sought”.

\subsection[13. Amendment of regulation 5 of the Child Support Commissioners Procedure Regulations]{Amendment of regulation 5 of the Child Support Commissioners Procedure Regulations}

13.  In regulation 5(1) of the Child Support Commissioners Procedure Regulations (notice of appeal) for the word “decision” there shall be substituted the words “full statement of the tribunal’s decision”.

\bigskip

Signed 
by authority of the Secretary of State for Social Security.

{\raggedleft
\emph{Roger Evans}\\*Parliamentary Under-Secretary of State,\\*Department of Social Security

}

18th March 1997

\bigskip

%Signed in connection with regulations 2 and 3 of these Regulations by authority of the Secretary of State for Education and Employment.

{\raggedleft
\emph{Mackay of Clashfern, C.}%\\*Minister of State,\\*Department for Education and Employment

}

18th March 1997 

\small

\part{Explanatory Note}

\renewcommand\parthead{--- Explanatory Note}

\subsection*{(This note is not part of the Regulations)}

These Regulations amend the provisions of the Social Security (Adjudication) Regulations 1995, the Social Security Commissioners Procedure Regulations 1987 and the Child Support Commissioners (Procedure) Regulations 1992 in so far as they relate to applications for leave to appeal against a tribunal decision.

  These Regulations require any application for leave to appeal to a Commissioner against the decision of a tribunal to be accompanied by a copy of the tribunal’s statement of the reasons for its decision and of its findings on questions of fact material thereto. This requirement applies equally whether the application is made to the chairman of a tribunal or to a Commissioner, and whether the tribunal in question is a social security appeal tribunal, a disability appeal tribunal or a medical appeal tribunal (regulations 3 to 5 and 8). A notice of appeal must also be accompanied by a copy of this statement (regulation 9).

  The same changes are made in relation to child support by regulations 12 and 13.

  The Regulations also amend the time limit for making an application to the chairman of a social security appeal tribunal, a disability appeal tribunal, a medical appeal tribunal or a child support appeal tribunal for leave to appeal to a Commissioner so that it begins with the date on which the tribunal’s statement of the reasons for its decision and of its findings on questions of fact material thereto is given or sent to the applicant (regulations 6 and 11).

  The Regulations reflect the changes made in the Social Security (Adjudication) and Child Support Amendment (No.\ 2) Regulations 1996 (S.I.\ 1996/2450), which amended the provisions on the form and promulgation of tribunal decisions.

  The Regulations do not impose any costs on business.


\end{document}