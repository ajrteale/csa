\documentclass[12pt,a4paper]{article}

\newcommand\regstitle{The Social Security and Child Support (Decisions and Appeals), Vaccine Damage Payments and Jobseeker's Allowance (Amendment) Regulations 1999}

\newcommand\regsnumber{1999/2677}

%\opt{newrules}{
\title{\regstitle}
%}

%\opt{2012rules}{
%\title{Child Maintenance and Other Payments Act 2008\\(2012 scheme version)}
%}

\author{S.I. 1999 No. 2677}

\date{Made
27th September 1999\\
Laid before Parliament
27th September 1999\\
Coming into force
18th October 1999}

%\opt{oldrules}{\newcommand\versionyear{1993}}
%\opt{newrules}{\newcommand\versionyear{2003}}
%\opt{2012rules}{\newcommand\versionyear{2012}}

\usepackage{csa-regs}

\setlength\headheight{42.07402pt}

\begin{document}

\maketitle

\noindent
The Secretary of State for Social Security, in exercise of the powers conferred by sections 3A(1), (3) and (4), 4(2) and 8(2)($b$) and (3)($a$) of the Vaccine Damage Payments Act 1979\footnote{\frenchspacing 1979 c. 17; section 3A was inserted by section 45 of the Social Security Act 1998 (c. 14) and section 4 was substituted by section 46 of that Act.}, sections 19(2) and 36(1) of the Jobseekers Act 1995\footnote{\frenchspacing 1995 c. 18.}, and sections 9(1), 10(3) and (6), 12(7), 16(1) and 79(1), (3) and (4) of, and paragraph 1($a$) of Schedule 5 to, the Social Security Act 1998\footnote{\frenchspacing 1998 c. 14.}, and of all other powers enabling him in that behalf, after consultation with the Social Security Advisory Committee\footnote{\frenchspacing \emph{See} section 172(1) of the Social Security Administration Act 1992 (c. 5) (“the 1992 Act”); sections 9, 10, 12 and 16 of the Social Security Act 1998 are relevant enactments for the purposes of that provision by virtue of the amendment of section 170(5) of the 1992 Act by the Social Security Act 1998, Schedule 7, paragraph 104($a$) and the Jobseekers Act 1995 is a relevant enactment by virtue of the amendment of section 170(5) of the 1992 Act by the Jobseekers Act 1995, Schedule 2, paragraph 67.} (except in relation to the provisions concerned with vaccine damage payments), by this Instrument hereby makes the following Regulations: 

{\sloppy

\tableofcontents

}

\bigskip

\setcounter{secnumdepth}{-2}

\section{Part I}

\renewcommand\parthead{--- Part I}

\subsection[1. Citation, commencement and interpretation]{Citation, commencement and interpretation}

1.---(1)  These Regulations may be cited as the Social Security and Child Support (Decisions and Appeals), Vaccine Damage Payments and Jobseeker’s Allowance (Amendment) Regulations 1999 and shall come into force on 18th October 1999.

(2) In Part II below any reference to a regulation or a Part is to a regulation or a Part of the Vaccine Damage Payments Regulations 1979\footnote{\frenchspacing S.I. 1979/432, to which there are amendments not relevant to these Regulations.}.

(3) In Part IV below any reference to a regulation is a reference to a regulation in the Social Security and Child Support (Decisions and Appeals) Regulations 1999\footnote{\frenchspacing S.I. 1999/991; the relevant amending instruments are S.I. 1999/1623 and 1662 (C. 47).}.

\section[Part II --- Amendment of the Vaccine Damage Payments Regulations 1979]{\sloppy Part II\\*Amendment of the Vaccine Damage Payments Regulations 1979}

\renewcommand\parthead{--- Part II}

\subsection[2. Amendment of regulation 1]{Amendment of regulation 1}

2.  In regulation 1(2) (citation, commencement and interpretation) the definitions of “the President” and “tribunal” shall be omitted.

\subsection[3. Amendment of regulation 4]{Amendment of regulation 4}

3.  In regulation 4(1) (obligations of disabled person), for the words “a tribunal” there shall be susbstituted the words “an appeal tribunal”.

\subsection[4. Substitution of Part IV]{Substitution of Part IV}

4.  For Part IV (reconsideration) there shall be substituted the following Part—

\begin{quotation}
\section*{``Part IV\\*Decisions reversing earlier decisions}

\subsection*{Decisions reversing earlier decisions made by the Secretary of State or appeal tribunals}

11.---(1)  The Secretary of State may make a decision under section 3A(1) of the Act which reverses a decision of his, made under section 3 of the Act, or of an appeal tribunal, made under section 4 of the Act—
\begin{enumerate}\item[]
($a$) pursuant to an application in the circumstances described in paragraph (2) below; or

($b$) except where paragraph (3) applies, on his own initiative.
\end{enumerate}

(2) The circumstances referred to in paragraph (1)($a$) above are—
\begin{enumerate}\item[]
($a$) the application is made in writing and contains an explanation as to why the applicant believes the decision in respect of which the application is made to be wrong; and

($b$) where the application is in respect of a decision of the Secretary of State, the application is made within six years of the date on which notification of that decision was given; or

($c$) where the application is in respect of a decision of an appeal tribunal, the application is made before whichever is the later of—
\begin{enumerate}\item[]
(i) the date two years after the date on which notification of that decision was given; or

(ii) the date six years after the date on which notification of the decision of the Secretary of State which was appealed was given.
\end{enumerate}
\end{enumerate}

(3) This paragraph applies where—
\begin{enumerate}\item[]
($a$) less than 21 days have elapsed since notice under regulation 12 below was given; or

($b$) more than six years have elapsed since the date on which notification of that decision was given except where it appears to the Secretary of State that a payment was made in consequence of a misrepresentation or failure to disclose any material fact.
\end{enumerate}

(4) Where the Secretary of State has made a decision under section 3A(1) of the Act, he shall notify—
\begin{enumerate}\item[]
($a$) the disabled person (if he is alive) to whom the decision relates; and

($b$) if the disabled person is not a claimant, the claimant who made the claim in respect of that disabled person,
\end{enumerate}
of that decision and the reasons for it.

\subsection*{Procedure by which a decision may be made under section 3A of the Act on the Secretary of State’s own initiative}

12.  Where the Secretary of State on his own initiative proposes to make a decision under section 3A of the Act reversing a decision (“the original decision”) of his or of an appeal tribunal he shall give notice in writing of his proposal to—
\begin{enumerate}\item[]
($a$) the disabled person (if he is alive) to whom the original decision relates; and

($b$) the claimant in relation to the original decision where he is not the disabled person.”.
\end{enumerate}
\end{quotation}

\section[Part III --- Amendment of the Jobseeker's Allowance Regulations 1996]{Part III\\*Amendment of the Jobseeker's Allowance Regulations 1996}

\renewcommand\parthead{--- Part III}

\subsection[5. Amendment of the Jobseeker’s Allowance Regulations 1996]{Amendment of the Jobseeker’s Allowance Regulations 1996}

5.---(1)  In regulation 69 (prescribed period for the purposes of section 19(2) of the Jobseekers Act) of the Jobseeker’s Allowance Regulations 1996\footnote{\frenchspacing S.I. 1996/207; regulation 69 was amended by S.I. 1997/2863 and modified by S.I. 1998/2825.} (“the 1996 Regulations”) the words “shall begin on the first day of the week following the date on which a jobseeker’s allowance is determined not to be payable and” shall be omitted.

(2) Regulation 69 of the 1996 Regulations shall be renumbered “regulation 69(1)” and after paragraph (1) of regulation 69 there shall be added the following paragraph—
\begin{quotation}
“(2) The prescribed period for the purposes of section 19(2) shall begin—
\begin{enumerate}\item[]
\begin{sloppypar}
($a$) where, in accordance with regulation 26A(1) of the Claims and Payments Regulations\footnote{\frenchspacing S.I. 1987/1968; regulation 26A was inserted by S.I. 1996/1460.}, a jobseeker’s allowance is paid otherwise than fortnightly in arrears, on the day following the end of the last benefit week in respect of which that allowance was paid; and
\end{sloppypar}

($b$) in any other case, on the first day of the benefit week following the date on which a jobseeker’s allowance is determined not to be payable.”.
\end{enumerate}
\end{quotation}

\section[Part IV --- Amendment of the Social Security and Child Support (Decisions and Appeals) Regulations 1999]{Part IV\\*Amendment of the Social Security and Child Support (Decisions and Appeals) Regulations 1999}

\renewcommand\parthead{--- Part IV}

\subsection[6. Amendment of regulation 3]{Amendment of regulation 3}

6.  In regulation 3\footnote{\frenchspacing Regulation 3 was amended by S.I. 1999/1623 and 1662 (C. 47).} (revision of decisions)—
\begin{enumerate}\item[]
($a$) for sub-paragraph ($a$) of paragraph (1) there shall be substituted the following sub-paragraph—
\begin{quotation}
“($a$) he commences action leading to the revision within one month of the date of—
\begin{enumerate}\item[]
(i) notification of the original decision; or

(ii) the making of an appeal under section 12 provided that the appeal is made within the time prescribed in regulation 31 or, in a case to which regulation 32 applies, the time prescribed in that regulation; or”; and
\end{enumerate}
\end{quotation}

($b$) for paragraph (9) there shall be substituted the following paragraph—
\begin{quotation}
“(9) Paragraph (1) shall not apply in respect of—
\begin{enumerate}\item[]
($a$) a relevant change of circumstances which occurred since the decision was made or where the Secretary of State has evidence or information which indicates that a relevant change of circumstances will occur; nor

($b$) a decision which relates to an attendance allowance or a disability living allowance where the person is terminally ill, within the meaning of section 66(2)($a$) of the Contributions and Benefit Act, unless an application for revision which contains an express statement that the person is terminally ill is made either by—
\begin{enumerate}\item[]
(i) the person himself; or

(ii) any other person purporting to act on his behalf whether or not that other person is acting with his knowledge or authority,
\end{enumerate}
but where such an application is received a decision may be so revised notwithstanding that no claim under section 66(1) or, as the case may be, 72(5) or 73(12) of that Act has been made.”.
\end{enumerate}
\end{quotation}
\end{enumerate}

\subsection[7. Amendment of regulation 6]{Amendment of regulation 6}

7.  In regulation 6 (supersession of decisions)—
\begin{enumerate}\item[]
($a$) for sub-paragraph ($f$)  of paragraph (2) there shall be substituted the following sub-paragraph—
\begin{quotation}
“($f$) is a decision that a jobseeker’s allowance is payable to a claimant where that allowance ceases to be payable by virtue of section 19(1) of the Jobseekers Act;”;
\end{quotation}
    and 

($b$) after sub-paragraph ($b$) of paragraph (6) there shall be added the following sub-paragraph—
\begin{quotation}
“($c$) the fact that a person has become terminally ill, within the meaning of section 66(2)($a$) of the Contributions and Benefits Act, unless an application for supersession which contains an express statement that the person is terminally ill is made either by—
\begin{enumerate}\item[]
(i) the person himself; or

(ii) any other person purporting to act on his behalf whether or not that other person is acting with his knowledge or authority;
\end{enumerate}
and where such an application is received a decision may be so superseded nothwithstanding that no claim under section 66(1) or, as the case may be, 72(5) or 73(12) of that Act has been made.”.
\end{quotation}
\end{enumerate}

\subsection[8. Amendment of regulation 7]{Amendment of regulation 7}

8.  For paragraph (8) of regulation 7\footnote{\frenchspacing Regulations 6 and 7 were amended by S.I. 1999/1623.} (date from which a decision superseded under section 10 takes effect) there shall be substituted the following paragraph—
\begin{quotation}
“(8) A decision to which regulation 6(2)($f$)  applies shall take effect—
\begin{enumerate}\item[]
($a$) where section 19(2) of the Jobseekers Act applies, as from the beginning of the period specified in regulation 69 of the Jobseeker’s Allowance Regulations; or

($b$) where section 19(3) of the Jobseekers Act applies, as from the beginning of the period determined in accordance with that subsection.”.
\end{enumerate}
\end{quotation}

\subsection[9. Amendment of regulation 33]{Amendment of regulation 33}

9.  In regulation 33(1)($a$)(i) (making of appeals and applications), after the word “under” there shall be inserted the words “section 4(1) of the Vaccine Damage Payments Act,”.

\subsection[10. Amendment of regulation 53]{Amendment of regulation 53}

10.  In regulation 53(3) (decisions of appeal tribunals), for sub-paragraph ($b$) there shall be substituted the following sub-paragraph—
\begin{quotation}
“($b$) except in the case of an appeal under the Vaccine Damage Payments Act, the conditions governing appeals to a Commissioner.”.
\end{quotation}

\bigskip

Signed 
by authority of the Secretary of State for Social Security.

{\raggedleft
\emph{Jeff Rooker
}\\*Minister of State,\\*Department of Social Security

}

27th September 1999

\small

\part{Explanatory Note}

\renewcommand\parthead{--- Explanatory Note}

\subsection*{(This note is not part of the Regulations)}

These Regulations amend the Vaccine Damage Payments Regulations 1979 (S.I.\ 1979/432) (“the 1979 Regulations”), the Jobseeker’s Allowance Regulations 1996 (S.I.\ 1996/207) (“the 1996 Regulations”) and the Social Security and Child Support (Decisions and Appeals) Regulations 1999 (S.I.\ 1999/991) (“the 1999 Regulations”).

Part II amends the 1979 Regulations in consequence of the commencement of sections 45 to 47 of the Social Security Act 1998 (“the 1998 Act”). Part IV (reconsideration) of the 1979 Regulations is replaced with a new Part IV (decisions reversing earlier decisions) which provides for the circumstances in which a decision of the Secretary of State or of an appeal tribunal may be reversed by a decision of the Secretary of State. The new Part IV also provides for the procedure to be followed in connection with the reversal of a decision. Minor amendments are also made to the 1979 Regulations to reflect the new procedures introduced by the 1998 Act.

Part III makes an amendment to the 1996 Regulations as to when benefit ceases to be payable under section 19 (circumstances in which a jobseeker’s allowance is not payable) of the Jobseekers Act 1995 (c.\ 18) (“the 1995 Act”) in cases where benefit is paid otherwise than fortnightly in arrears.

Part IV contains amendments to the 1999 Regulations. Provision is made—
\begin{itemize}
\item    for the revision of decisions within one month of an appeal being made (regulation 6);

\item    to prevent revision or supersession of a decision in relation to an attendance allowance or a disability living allowance on the ground that a person is terminally ill unless an application for revision or supersession is made expressly on that ground (regulations 6 and 7);

\item    in connection with the supersession of decisions in relation to jobseeker’s allowance where a sanction is imposed under section 19 of the 1995 Act (regulation 8). 
\end{itemize}

Minor amendments are made by regulations 9 and 10 (so that provision is made explicitly in relation to the making of appeals against decisions of the Secretary of State under the Vaccine Damage Payments Act 1979 (1979 c.\ 17) and appeal tribunal decisions on such appeals).

These Regulations do not impose a charge on business. 

\end{document}