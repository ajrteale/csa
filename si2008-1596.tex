\documentclass[12pt,a4paper]{article}

\newcommand\regstitle{The Social Security (Recovery of Benefits) (Lump Sum Payments) Regulations 2008}

\newcommand\regsnumber{2008/1596}

%\opt{newrules}{
\title{\regstitle}
%}

%\opt{2012rules}{
%\title{Child Maintenance and~Other Payments Act 2008\\(2012 scheme version)}
%}

\author{S.I.\ 2008 No.\ 1596}

\date{Made
18th June 2008\\
Laid before Parliament
25th June 2008\\
Coming into~force
1st October 2008
}

%\opt{oldrules}{\newcommand\versionyear{1993}}
%\opt{newrules}{\newcommand\versionyear{2003}}
%\opt{2012rules}{\newcommand\versionyear{2012}}

\usepackage{csa-regs}

\setlength\headheight{42.11603pt}

%\hbadness=10000

\begin{document}

\maketitle

\begin{sloppypar}
\noindent
The Secretary of State for Work and~Pensions makes the following Regulations in exercise of the powers conferred by section~189(4) and~(6) of the Social Security Administration Act 1992\footnote{1992 c.~5.}, sections~1A, 14(2), (3) and~(4),~18,~19,~21(3),~23(1),~(2) and~(7) and~29 of, and~paragraphs 4 and~8 of Schedule 1 to, the Social Security (Recovery of Benefits) Act 1997\footnote{1997 c.~27. Section 1A was inserted by section~54 of the Child Maintenance and Other Payments Act 2008 (c.~6) and section~29 is cited for the meaning ascribed to the word “prescribed”.}, section~79(6) of the Social Security Act 1998\footnote{1998 c.~14.} and~section~53 of the Child Maintenance and~Other Payments Act 2008\footnote{2008 c.~6.}, which contains only regulations~made by virtue of, or consequential on sections~54 and~57(2) of the Child Maintenance and~Other Payments Act 2008 and~which are made before the end of a period of 6 months beginning with the coming into force of those sections\footnote{\emph{See} section~173(5)($b$)  of the Social Security Administration Act 1992 (c.~5).}: 
\end{sloppypar}

{\sloppy

\tableofcontents

}

\bigskip

\setcounter{secnumdepth}{-2}

\section[Part~I --- General]{Part~I\\*General}

\renewcommand\parthead{--- Part~I}

\subsection[1. Citation, commencement and~interpretation]{Citation, commencement and~interpretation}

1.---(1)  These Regulations may be cited as the Social Security (Recovery of Benefits) (Lump Sum Payments) Regulations 2008 and~shall come into force on 1st October 2008.

(2) In these Regulations—
\begin{enumerate}\item[]
“the Act” means the Social Security (Recovery of Benefits) Act 1997;

“compensator” means a person making a compensation payment;

“Compensation Recovery Unit” means the Compensation Recovery Unit of the Department for Work and~Pensions at Durham House, Washington, Tyne and~Wear, \textsc{\lowercase{NE38 7SF}};

“lump sum payments” are payments to which section~1A(2) of the Act applies, except in relation to regulation~18(1)($b$);

“recoverable benefits” has the same meaning as in section~1(4)($c$)  of the Act;

“recoverable lump sum payments” means any lump sum payments which are recoverable by virtue of regulation~4.
\end{enumerate}

\subsection[2. Application of the Act]{Application of the Act}

2.---(1)  The provisions of the Act specified in paragraph~(2) apply for the purposes of these Regulations with the modifications, where appropriate, prescribed in Schedule 1.

(2) The specified provisions are—
\begin{enumerate}\item[]
($a$) section~1(3) (cases in which this Act applies);

($b$) sections~10 to 14 (reviews and~appeals);

($c$) sections~15 and~17 (courts);

($d$) sections~18 and~19 (reduction of compensation: complex cases);

($e$) sections~20 to 23 (miscellaneous);

($f$) sections~26 and~27 (provisions relating to Northern Ireland);

($g$) sections~28 to 31 (general);

($h$) section~33 (consequential amendments and~repeals);

($i$) section~34(1) and~(3) (short title and~extent);

($j$) Schedule 1 (compensation payments---exempted payments and~power to disregard small payments).
\end{enumerate}

\subsection[3. Consequential amendments]{Consequential amendments}

3.  The consequential amendments set out in Schedule 2 apply for the purposes of these Regulations.

\subsection[4. Recovery of lump sum payments]{Recovery of lump sum payments}

4.---(1)  The Secretary of State may recover the amount of a payment to which section~1A(2) of the Act applies (“a lump sum payment”) where—
\begin{enumerate}\item[]
($a$) a compensation payment in consequence of a disease is made to or in respect of—
\begin{enumerate}\item[]
(i) a person (“$\mathcal{P}$”); or

(ii) a dependant of $\mathcal{P}$,
\end{enumerate}
to whom, or in respect of whom, a lump sum payment has been, or is likely to be, made; and

($b$) the compensation payment is made in consequence of the same disease as the lump sum payment.
\end{enumerate}

(2) In paragraph~(1), references to a payment made in consequence of a disease—
\begin{enumerate}\item[]
($a$) are references to a payment made by or on behalf of a person who is, or is alleged to be, liable to any extent in respect of the disease; but

($b$) do not include references to a payment mentioned in Part~I of Schedule 1 to the Act.
\end{enumerate}

\subsection[5. Application of these Regulations to a dependant of $\mathcal{P}$]{Application of these Regulations to a dependant of $\mathbf{P}$}

5.---(1)  Subject to paragraph~(2), in these Regulations and~any provision of the Act as modified any reference to $\mathcal{P}$ is to be construed as if it included a reference to a dependant of $\mathcal{P}$ where that dependant is the person to whom, or in respect of whom, a lump sum payment is made.

(2) Paragraph (1) does not apply in relation to regulations~4, 10(7) and~12(7) and~sections~15 and~23(2) of, and~paragraphs 3($a$)  and~5(1) of Part~I of Schedule 1 to, the Act.

\subsection[6. Compensation payments to which these Regulations apply]{Compensation payments to which these Regulations apply}

6.  These Regulations apply in relation to compensation payments made on or after the day on which section~54 of the Child Maintenance and~Other Payments Act 2008 comes into force.

\subsection[7. Exempted trusts and~payments]{Exempted trusts and~payments}

7.---(1)  The following trusts are prescribed for the purposes of paragraph~4 of Schedule 1 to the Act—
\begin{enumerate}\item[]
($a$) the Macfarlane Trust established on 10th March 1988 partly out of funds provided by the Secretary of State to the H\ae{}mophilia Society for the relief of poverty or distress among those suffering from h\ae{}mophilia;

($b$) the Macfarlane (Special Payments) Trust established on 29th January 1990 partly out of funds provided by the Secretary of State, for the benefit of certain persons suffering from h\ae{}mophilia;

($c$) the Macfarlane (Special Payments) (No.~2) Trust established on 3rd May 1991 partly out of funds provided by the Secretary of State, for the benefit of certain persons suffering from h\ae{}mophilia and~other beneficiaries;

($d$) the Eileen Trust established on 29th March 1993 out of funds provided by the Secretary of State, for the benefit of persons eligible for payment in accordance with its provisions;

($e$) a trust established out of funds provided by the Secretary of State in respect of persons who suffered, or who are suffering, from variant Creutzfeldt-Jakob disease for the benefit of persons eligible for interim payments in accordance with its provisions;

($f$) a trust established out of funds provided by the Secretary of State in respect of persons who suffered, or who are suffering, from variant Creutzfeldt-Jakob disease for the benefit of persons eligible for payments, other than interim payments, in accordance with its provisions.
\end{enumerate}

(2) The following payments are prescribed for the purposes of paragraph~8 of Schedule 1 to the Act—
\begin{enumerate}\item[]
($a$) any payment made under the Vaccine Damage Payments Act 1979\footnote{1979 c.~17.} to or in respect of $\mathcal{P}$;

($b$) any award of compensation made to or in respect of $\mathcal{P}$ under the Criminal Injuries Compensation Act 1995\footnote{1995 c.~53.} or by the Criminal Injuries Compensation Board under the Criminal Injuries Compensation Scheme 1990 or any earlier scheme or under the Criminal Injuries Compensation (Northern Ireland) Order 2002\footnote{S.I.~2002/796 (N.I.~1).};

($c$) any payment made to $\mathcal{P}$ in respect of sensorineural hearing loss where the loss is less than 50 decibels in one or both ears;

($d$) any contractual amount paid to $\mathcal{P}$ by an employer of $\mathcal{P}$ in respect of a period of incapacity for work;

($e$) any payment made under the National Health Service (Injury Benefits) Regulations 1995\footnote{S.I.~1995/866.}, the National Health Service (Scotland) (Injury Benefits) Regulations 1998\footnote{S.I.~1998/1594 (S.~84).} or the Health and~Personal Social Services (Injury Benefits) Regulations (Northern Ireland) 2001\footnote{S.R.~2001 No.~367.};

($f$) any payment made by or on behalf of the Secretary of State for the benefit of persons eligible for payment in accordance with the provisions of a scheme established by the Secretary of State on 24th April 1992 or, in Scotland, on 10th April 1992;

($g$) any payment made from the Skipton Fund, the ex-gratia payment scheme administered by the Skipton Fund Limited, incorporated on 25th March 2004, for the benefit of certain persons suffering from hepatitis C and~other persons eligible for payment in accordance with the scheme’s provisions;

($h$) any payment made from the London Bombings Relief Charitable Fund, the company limited by guarantee (number 5505072) and~registered charity of that name established on 11th July 2005 for the purpose of (amongst other things) relieving sickness, disability or financial need of victims (including families or dependants of victims) of the terrorist attacks carried out in London on 7th July 2005.
\end{enumerate}

\section[Part II --- Certificates]{Part II\\*Certificates}

\renewcommand\parthead{--- Part II}

\subsection[8. Applications for certificates]{Applications for certificates}

8.---(1)  Before making a compensation payment the compensator must apply to the Secretary of State for a certificate.

(2) Where the compensator applies for a certificate, the Secretary of State must—
\begin{enumerate}\item[]
($a$) send to the compensator a written acknowledgment of receipt of the application; and

($b$) issue the certificate before the end of the period of 4 weeks.
\end{enumerate}

(3) An application for a certificate is to be treated for the purposes of the Act as received by the Secretary of State on the day on which it is received by the Compensation Recovery Unit, or if the application is received after normal business hours, or on a day which is not a normal business day at that office, on the next such day.

\subsection[9. Information contained in certificates]{Information contained in certificates}

9.---(1)  Subject to paragraph~(2), a certificate must specify—
\begin{enumerate}\item[]
($a$) the amounts;

($b$) which of the type of payments referred to in section~1A(2) of the Act applies; and

($c$) the dates,
\end{enumerate}
of any lump sum payments which have been, or are likely to have been paid.

(2) Where the type of payment is an extra-statutory payment the certificate may specify that type of payment as if it were a payment to which section~1A(2)($a$)  applies.

(3) The Secretary of State may estimate, in such manner as the Secretary of State thinks fit the amount of the lump sum payments specified in the certificate.

(4) Where the Secretary of State issues a certificate, the information contained in that certificate must be provided to—
\begin{enumerate}\item[]
($a$) the person who appears to the Secretary of State to be $\mathcal{P}$; or

($b$) any person who the Secretary of State thinks will receive a compensation payment in respect of $\mathcal{P}$.
\end{enumerate}

(5) A person to whom a certificate is issued or who is provided with information under 
%paragraph~(3) 
paragraph (4)  % Words substituted (1.10.08) by SI 2008/2365 reg 6(2)
is entitled to particulars of the manner in which any amount, type of payment or date specified in the certificate has been determined, if that person applies to the Secretary of State for those particulars.

\amendment{
Words substituted in reg.~9(5) (1.10.08) by the Social Security (Miscellaneous Amendments) (No.~3) Regulations 2008 reg.~6(2).
}

\section[Part III --- Liability of person paying compensation]{Part III\\*Liability of person paying compensation}

\renewcommand\parthead{--- Part III}

\subsection[10. Liability to pay Secretary of State amount of lump sum payments]{Liability to pay Secretary of State amount of lump sum payments}

10.---(1)  A person who makes a compensation payment in any case is liable to pay the Secretary of State an amount equal to the total amount of—
\begin{enumerate}\item[]
($a$) in a case to which paragraph~(2) applies, the recoverable lump sum payments; or

($b$) in a case to which paragraph~(3) applies, the compensation payment.
\end{enumerate}

(2) Paragraph (1)($a$)  applies to a case where—
\begin{enumerate}\item[]
($a$) the compensation payment is equal to, or more than, any recoverable lump sum payments; or

($b$) a dependant is a beneficiary of part of a compensation payment made in respect of $\mathcal{P}$, that part of the compensation payment is equal to, or more than, any recoverable lump sum payments which have been made to that dependant.
\end{enumerate}

(3) Paragraph (1)($b$)  applies to a case where—
\begin{enumerate}\item[]
($a$) the compensation payment; or

($b$) a dependant is a beneficiary of part of a compensation payment made in respect of $\mathcal{P}$, and~recoverable lump sum payments have been made to that dependant, the share of the compensation payment,
\end{enumerate}
is less than the lump sum payments.

(4) The liability referred to in paragraph~(1) arises—
\begin{enumerate}\item[]
($a$) immediately before the compensation payment or, if there is more than one, the first of them is made;

($b$) prior to any liability to pay the Secretary of State an amount equal to the total amount of the recoverable benefits payable under section~6 of the Act.
\end{enumerate}

(5) No amount becomes payable under this regulation~before the end of the period of 14 days following the day on which the liability arises.

(6) Subject to paragraph~(4), an amount becomes payable under this regulation~at the end of the period of 14 days beginning with the day on which a certificate is first issued showing that the amount of recoverable lump sum payment to which it relates has been or is likely to have been paid.

(7) In the case of a lump sum payment which has been made to a dependant of $\mathcal{P}$, this regulation~applies only to the extent to which the compensator is making any payment—
\begin{enumerate}\item[]
\begin{enumerate}
\item[($a$)] (i) under the Fatal Accidents Act 1976\footnote{1976 c.~30.};

(ii) to the extent that it is made in respect of a liability arising by virtue of section~1 of the Damages (Scotland) Act 1976\footnote{1976 c.~13. Section 1 was amended by the Administration of Justice Act 1982 (c.~53), section~14(1), the International Transport Conventions Act 1983 (c.~14), section~3(6) and paragraph~2 of Schedule 1 and the Damages (Scotland) Act 1993 (c.~5), section~1.}; or

(iii) under the Fatal Accidents (Northern Ireland) Order 1977\footnote{S.I.~1977/1251 (N.I.~18).},
\end{enumerate}
to that dependant; or

($b$) in respect of $\mathcal{P}$, and~that dependant is an intended beneficiary of part or all of that payment.
\end{enumerate}

\subsection[11. Recovery of payment due under regulation~10]{Recovery of payment due under regulation~10}

11.---(1)  This regulation~applies where a compensator has made a compensation payment but—
\begin{enumerate}\item[]
($a$) has not applied for a certificate; or

($b$) has not made a payment to the Secretary of State under regulation~10 before the end of the period allowed under that regulation.
\end{enumerate}

(2) The Secretary of State may—
\begin{enumerate}\item[]
($a$) issue the compensator who made the compensation payment with a certificate, if none has been issued; or

($b$) issue that compensator with a copy of the certificate or (if more than one has been issued) the most recent one,
\end{enumerate}
and~(in either case) issue that compensator with a demand~that payment of any amount due under regulation~10 be made immediately.

(3) The Secretary of State may, in accordance with paragraphs (4) and~(5), recover the amount for which a demand~for payment is made under paragraph~(2) from the compensator who made the compensation payment.

(4) If the compensator who made the compensation payment resides or carries on business in England~and~Wales and~a county court so orders, any amount recoverable under paragraph~(3) is recoverable by execution issued from the county court or otherwise as if it were payable under an order of that court.

(5) If the compensator who made the payment resides or carries on business in Scotland, any amount recoverable under paragraph~(3) may be enforced in like manner as an extract registered decree arbitral bearing a warrant for execution issued by the sheriff court of any sheriffdom in Scotland.

(6) A document bearing a certificate which—
\begin{enumerate}\item[]
($a$) is signed by a person authorised to do so by the Secretary of State; and

($b$) states that the document, apart from the certificate, is a record of the amount recoverable under paragraph~(3),
\end{enumerate}
is conclusive evidence that that amount is so recoverable.

(7) A certificate under paragraph~(6) purporting to be signed by a person authorised to do so by the Secretary of State is to be treated as so signed unless the contrary is proved.

\section[Part IV --- Reduction of compensation payment]{Part IV\\*Reduction of compensation payment}

\renewcommand\parthead{--- Part IV}

\subsection[12. Reduction of compensation payment]{Reduction of compensation payment}

12.---(1)  This regulation~applies in a case where, in relation to any compensation payment in consequence of a disease made to, or in respect of $\mathcal{P}$, a lump sum payment has been, or is likely to be made to, or in respect of $\mathcal{P}$.

(2) In such a case, any claim of a person to receive the compensation payment is to be treated for all purposes as discharged if—
\begin{enumerate}\item[]
($a$) that person is paid the amount (if any) of the compensation payment calculated in accordance with this regulation; and

($b$) if the amount of the compensation payment so calculated is nil, that person is given a statement saying so by the compensator who (apart from this regulation) would have paid the gross amount of the compensation payment.
\end{enumerate}

(3) For an award of compensation for which paragraph~(1) is satisfied, so much of the gross amount of the compensation payment as is equal to the amount of the lump sum payment is to be reduced (to nil, if necessary) by deducting the amount of the recoverable lump sum payment.

(4) Paragraph (3) is to have effect as if a requirement to reduce a payment by deducting an amount which exceeds that payment were a requirement to reduce that payment to nil.

(5) The amount of the compensation payment calculated in accordance with this regulation~is—
\begin{enumerate}\item[]
($a$) the gross amount of the compensation payment;
\end{enumerate}
less
\begin{enumerate}\item[]
($b$) the reductions made under paragraph~(3),
\end{enumerate}
(and, accordingly, the amount may be nil).

(6) The reduction specified in paragraph~(3) is to be attributed to the heads of compensation in the following order—
\begin{enumerate}\item[]
($a$) damages for non-pecuniary loss;

($b$) damages for pecuniary loss,
\end{enumerate}
and, the reduction is to be made before any reduction in respect of recoverable benefits under section~8 of the Act.

(7) Where the lump sum payment has been made to a dependant of $\mathcal{P}$, the reduction specified in paragraph~(3) may be attributed—
\begin{enumerate}\item[]
($a$) to any damages awarded to that dependant—
\begin{enumerate}\item[]
(i) under the Fatal Accidents Act 1976;

(ii) to the extent that they are made in respect of a liability arising by virtue of section~1 of the Damages (Scotland) Act 1976; or

(iii) under the Fatal Accidents (Northern Ireland) Order 1977,
\end{enumerate}
other than those paid for funeral expenses;

($b$) to any part of a compensation payment paid in respect of $\mathcal{P}$, where that dependant is an intended beneficiary of part or all of that compensation.
\end{enumerate}

\subsection[13. Regulation 12: supplementary]{Regulation 12: supplementary}

13.---(1)  A compensator who makes a compensation payment calculated in accordance with regulation~12 must inform the person to whom the payment is made—
\begin{enumerate}\item[]
($a$) that the payment has been so calculated; and

($b$) of the date for payment by reference to which the calculation has been made.
\end{enumerate}

(2) If the amount of a compensation payment calculated in accordance with regulation~12 is nil, a compensator giving a statement saying so is to be treated for the purposes of these Regulations as making a payment within regulation~4(1)($a$)  on the day on which the statement is given.

(3) Where a compensator—
\begin{enumerate}\item[]
($a$) makes a compensation payment calculated in accordance with regulation~12; and

($b$) if the amount of the compensation payment so calculated is nil, gives a statement saying so,
\end{enumerate}
the compensator is to be treated, for the purpose of determining any rights and~liabilities in respect of contribution or indemnity, as having paid the gross amount of the compensation payment.

(4) For the purposes of these Regulations—
\begin{enumerate}\item[]
($a$) the gross amount of the compensation payment is the amount of the compensation payment apart from regulation~12; and

($b$) the amount of any recoverable lump sum payment is the amount determined in accordance with the certificate.
\end{enumerate}

\subsection[14. Reduction of compensation: complex cases]{Reduction of compensation: complex cases}

14.---(1)  This regulation~applies where—
\begin{enumerate}\item[]
($a$) a compensation payment in the form of a lump sum (an “earlier payment”) has been made to or in respect of $\mathcal{P}$; and

($b$) subsequently another such payment (a “later payment”) is made to or in respect of the same $\mathcal{P}$ in consequence of the same disease.
\end{enumerate}

(2) In determining the liability under regulation~10(1) arising in connection with the making of the later payment, the amount referred to in that regulation~is to be reduced by any amount paid in satisfaction of that liability as it arose in connection with the earlier payment.

(3) Where—
\begin{enumerate}\item[]
($a$) a payment made in satisfaction of the liability under regulation~10(1) arising in connection with an earlier payment is not reflected in the certificate in force at the time of a later payment; and

($b$) in consequence, the aggregate of payments made in satisfaction of the liability exceeds what it would have been had that payment been so reflected,
\end{enumerate}
the Secretary of State is to pay the compensator who made the later payment an amount equal to the excess.

(4) Where—
\begin{enumerate}\item[]
($a$) a compensator receives a payment under paragraph~(3); and

($b$) the amount of the compensation payment made by that compensator was calculated under regulation~12,
\end{enumerate}
then the compensation payment is to be recalculated under regulation~12, and~the compensator must pay the amount of the increase (if any) to the person to whom the compensation payment was made.

(5) Where both the earlier payment and~the later payment are made by the same compensator, that compensator may—
\begin{enumerate}\item[]
($a$) aggregate the gross amounts of the payments made;

($b$) calculate what would have been the reduction made under regulation~12(3) if that aggregate amount had been paid at the date of the last payment on the basis that—
\begin{enumerate}\item[]
(i) the aggregate amount is to be taken to be the gross amount; and

(ii) the amount of any recoverable lump sum payment is to be taken to be the amount determined in accordance with the most recent certificate;
\end{enumerate}

($c$) deduct from that reduction calculated under sub-paragraph~($b$)  the amount of the reduction under regulation~12(3) from any earlier payment; and

($d$) deduct from the latest gross payment the net reduction calculated under sub-paragraph~($c$)  (and~accordingly the latest payment may be nil).
\end{enumerate}

(6) Where a refund is made under paragraph~(3), the Secretary of State is to send the compensator (with the refund) and~the person to whom the compensation payment was made a statement showing—
\begin{enumerate}\item[]
($a$) the total amount that has already been paid by that compensator to the Secretary of State;

($b$) the amount that ought to have been paid by that compensator; and

($c$) the amount to be repaid to that compensator by the Secretary of State.
\end{enumerate}

(7) Where the reduction of a compensation payment is recalculated by virtue of paragraph~(4) or (5) the compensator must give notice of the calculation to $\mathcal{P}$.

\section[Part V --- Miscellaneous]{Part V\\*Miscellaneous}

\renewcommand\parthead{--- Part V}

\subsection[15. Information to be provided by the compensator]{Information to be provided by the compensator}

15.  The following information is prescribed for the purposes of sections~21(3)($a$)  and~23(1) of the Act—
\begin{enumerate}\item[]
($a$) the full name and~address of $\mathcal{P}$;

($b$) where known, the date of birth or national insurance number of $\mathcal{P}$, or both if both are known; and

($c$) the nature of the disease.
\end{enumerate}

\subsection[16. Information to be provided by $\mathcal{P}$]{Information to be provided by $\mathbf{P}$}

16.  The following information is prescribed for the purposes of section~23(2) of the Act—
\begin{enumerate}\item[]
($a$) whether $\mathcal{P}$ has claimed or may claim a compensation payment, and~if so, the full name and~address of the person against whom the claim was or may be made;

($b$) the amount of any compensation payment and~the date on which it was made;

($c$) the amount of the lump sum payment claimed, the type of that payment and~the date on which it was paid.
\end{enumerate}

\subsection[17. Provision of information]{Provision of information}

17.  A person required to give information to the Secretary of State under regulation~15 or 16 is to do so by sending it to the Compensation Recovery Unit not later than 14 days after—
\begin{enumerate}\item[]
($a$) where the person is one to whom regulation~15 applies, the date on which the compensator receives a claim for compensation from $\mathcal{P}$ in respect of the disease;

($b$) where the person is one to whom regulation~16 applies, the date on which the Secretary of State requests the information from $\mathcal{P}$.
\end{enumerate}

\subsection[18. Periodical payments]{Periodical payments}

18.---(1)  This regulation~applies where in final settlement of $\mathcal{P}$’s claim, an agreement is entered into—
\begin{enumerate}\item[]
($a$) for the making of periodical payments (whether of an income or capital nature); or

($b$) for the making of such payments and~lump sum payments,
\end{enumerate}
and, those payments would fall to be treated for the purposes of the Act as compensation payments.

(2) Where this regulation~applies—
\begin{enumerate}\item[]
($a$) the compensator in question is to be taken to have made a single compensation payment on the day of settlement;

($b$) the total of the payments due to be made under the agreement referred to in paragraph~(1) are to be taken to be a compensation payment for the purposes of the Act; and

($c$) that single compensation payment is a payment from which lump sum payments may be recovered under these Regulations.
\end{enumerate}

(3) In any case where—
\begin{enumerate}\item[]
($a$) the person making the periodical payments (“the secondary party”) does so in pursuance of arrangements entered into with another (“the primary party”) (as in a case where the primary party purchases an annuity for $\mathcal{P}$ from the secondary party); and

($b$) apart from those arrangements, the primary party would have been regarded as the compensator,
\end{enumerate}
then for the purposes of the Act, the primary party is to be regarded as the compensator and~the secondary party is not to be so regarded.

(4) In this regulation—
\begin{enumerate}\item[]
“the day of settlement” means—
\begin{enumerate}\item[]
($a$) 
if the agreement referred to in paragraph~(1) is approved by a court, the day on which that approval is given; and

($b$) 
in any other case, the day on which the agreement is entered into;
\end{enumerate}

“a single compensation payment” means the total amount of the payments due to be made under the agreement referred to in paragraph~(1).
\end{enumerate}

\subsection[19. Adjustments]{Adjustments}

19.---(1)  Where the conditions specified in subsection~(1) and~paragraphs ($a$)  and~($b$)  of subsection~(2) of section~14 of the Act are satisfied, the Secretary of State is to pay the difference between the amount that has been paid and~the amount that ought to have been paid to the compensator.

(2) Where the conditions specified in subsection~(1) and~paragraphs ($a$)  and~($b$)  of subsection~(3) of section~14 of the Act are satisfied, the compensator is to pay the difference between the amount that has been paid and~the amount that ought to have been paid to the Secretary of State.

(3) Where the Secretary of State is making a refund under paragraph~(1), or demanding a payment of a further amount under paragraph~(2), the Secretary of State is to send to the compensator (with the refund or demand) and~to the person to whom the compensation payment was made a statement showing—
\begin{enumerate}\item[]
($a$) the total amount that has already been paid to the Secretary of State;

($b$) the amount that ought to have been paid; and

($c$) the difference, and~whether a repayment by the Secretary of State or a further payment by the compensator to the Secretary of State is required.
\end{enumerate}

(4) This paragraph~applies where—
\begin{enumerate}\item[]
($a$) the amount of the compensation payment by the compensator was calculated under regulation~12; and

($b$) the Secretary of State has made a payment under paragraph~(1).
\end{enumerate}

(5) Where paragraph~(4) applies, the amount of the compensation payment is to be recalculated under regulation~12 to take account of the fresh certificate and~the compensator must pay the amount of the increase (if any) to the person to whom the compensation payment was made.

(6) This paragraph~applies where—
\begin{enumerate}\item[]
($a$) the amount of the compensation payment made by the compensator was calculated under regulation~12;

($b$) the compensator has made a payment under paragraph~(2); and

($c$) the fresh certificate issued after the review or appeal was required as a result of $\mathcal{P}$ or such other person to whom the compensation payment was made supplying to the compensator information, knowing it to be incorrect or insufficient, with the intent of enhancing the compensation payment calculated under regulation~12, and~the compensator supplying that information to the Secretary of State without knowing it to be incorrect or insufficient.
\end{enumerate}

(7) Where paragraph~(6) applies, the compensator may recalculate the compensation payment under regulation~12 to take account of the fresh certificate and~may require the repayment of the difference (if any) between the payment made and~the payment as so recalculated by the person to whom the compensator made the compensation payment. 

\bigskip

Signed 
by authority of the 
Secretary of State for~Work and~Pensions.
%I concur
%By authority of the Lord Chancellor

{\raggedleft
\emph{William D McKenzie}\\*
%Secretary
%Minister
Parliamentary Under-Secretary 
of State,\\*Department 
for~Work and~Pensions
%Ministry of Justice
%Two of the Commissioners of Inland~Revenue

}

18th June 2008


\small

\part[Schedule 1 --- Modification of certain provisions of the Act]{Schedule 1\\*Modification of certain provisions of the Act}

\renewcommand\parthead{--- Schedule 1}

1.  This Schedule applies to any case to which regulation~4 applies.

\medskip

2.  Where this Schedule applies, section~1 (cases in which this Act applies) is to apply as if in subsection~(3), for “Subsection~(1)($a$)” there were substituted “Section 1A(1)($a$)”.

\medskip

3.  Where this Schedule applies, section~10 (review of certificates of recoverable benefits)\footnote{Section 10 was amended by the Social Security Act 1998 (c.~14), section~86(1) and paragraph~149 of Schedule 7.} is to apply as if in—
\begin{enumerate}\item[]
($a$) the heading and~in subsection~(1), there were omitted “of recoverable benefits” in each place it occurs;

($b$) subsection~(3), for “benefits” there were substituted “lump sum payments, except where that certificate has been reviewed under regulation~9ZA(1)($e$)  of the Social Security and~Child Support (Decisions and~Appeals) Regulations 1999 (review of certificates),”.
\end{enumerate}

\medskip

4.  Where this Schedule applies, section~11 (appeals against certificates of recoverable benefits)\footnote{Section 11 was amended by the Social Security Act 1998, section~86(1) and paragraph~150 of Schedule 7 and by the Constitutional Reform Act 2005 (c.~4), section~59(5) and paragraph~1(2) of Schedule 11 and subsection~(6) of section~11 was repealed by the Social Security Act 1998, Schedule 8.} is to apply as if in—
\begin{enumerate}\item[]
($a$) the heading and~in subsections~(1) and~(2)($a$), there were omitted “of recoverable benefits” in each place it occurs;

($b$) subsection~(1)($a$), there were omitted “, rate or period”;

($c$) subsection~(1)($b$)—
\begin{enumerate}\item[]
(i) for “listed benefits” there were substituted “lump sum payments”;

(ii) there were omitted “accident, injury or”;
\end{enumerate}

($d$) subsection~(1)($c$)—
\begin{enumerate}\item[]
(i) for “listed benefits” there were substituted “lump sum payments”;

(ii) for “the injured person during the relevant period” there were substituted “$\mathcal{P}$”;
\end{enumerate}

($e$) subsection~(1)($d$), for “1(1)($a$)” there were substituted “1A(1)($a$)”;

($f$) subsection~(2)($aa$)  for “section~7(2)($a$)” there were substituted “regulation~11(2)($a$)  of the Lump Sum Payments Regulations”;

($g$) subsection~(2)($b$), for “section~8) the injured person” there were substituted “regulation~12 of the Lump Sum Payments Regulations) $\mathcal{P}$”;

($h$) subsection~(3), for “section~6” there were substituted “regulation~10 of the Lump Sum Payments Regulations”.
\end{enumerate}

\medskip

5.  Where this Schedule applies, section~12 (reference of questions to medical appeal tribunal)\footnote{Section 12 was amended by the Social Security Act 1998, section~86(1) and paragraph~151 of Schedule 7 and subsections (6) to (8) of section~12 were repealed by the Social Security Act 1998, Schedule 8.} is to apply as if in—
\begin{enumerate}\item[]
($a$) the heading for “questions to medical appeal tribunal” there were substituted “appeal to appeal tribunal”;

($b$) subsection~(3), there were omitted “accident, injury or”;

($c$) subsection~(4)($a$), for “amounts, rates and~periods” there were substituted “amount, type and~date of payments”;

($d$) subsections~(4)($a$)  and~($c$), there were omitted “of recoverable benefits” in each place it occurs.
\end{enumerate}

\medskip

%6.  Where this Schedule applies, section~13 (appeal to Social Security Commissioner)\footnote{Section 13 was amended by the Social Security Act 1998, section 86(1) and (2) and paragraph~152 of Schedule 7 and subsection (4) of section 14 was repealed by the Social Security Act 1998, Schedule 8.} is to apply as if in—
%\begin{enumerate}\item[]
%($a$) subsection~(2)($b$), there were omitted “of recoverable benefits”;
%
%($b$) subsection~(2)($c$)  for “section~8) the injured person” there were substituted “regulation~12 of the Lump Sum Payments Regulations) $\mathcal{P}$”.
%\end{enumerate}

% Para 6 substituted (1.10.08) by SI 2008/2365 reg 6(3)
6.  Where this Schedule applies, section 13 (appeal to Social Security Commissioner) is to apply as if in—
\begin{enumerate}\item[]
($a$) subsection (2)($b$)  there were omitted “of recoverable benefits”;

($b$) subsection (2)($bb$)  for “section 7(2)($a$)” there were substituted “regulation~11(2)($a$)  of the Lump Sum Payment Regulations”; and

($c$) subsection (2)($c$)  for “section 8) the injured person” there were substituted “regulation 12 of the Lump Sum Payments Regulations).
\end{enumerate}

\amendment{
Para.~6 substituted (1.10.08) by the Social Security (Miscellaneous Amendments) (No.~3) Regulations 2008 reg.~6(3).
}

\medskip

7.  Where this Schedule applies, section~14 (reviews and~appeals: supplementary) is to apply as if in—
\begin{enumerate}\item[]
($a$) subsection~(1), there were omitted “of recoverable benefits”;

($b$) subsections~(2) and~(3), for “section~6” there were substituted “regulation~10 of the Lump Sum Payments Regulations” in each place it occurs;

($c$) subsection~(4), for “section~8” there were substituted “regulation~12 of the Lump Sum Payments Regulations”.
\end{enumerate}

\medskip

8.  Where this Schedule applies, for section~15 (court orders) is to apply as if there were substituted—
\begin{quotation}
“15.---(1)  This section~applies where a court makes an order for a compensation payment to be made in a case where a compensation payment is to be made to a dependant of P—
\begin{enumerate}\item[]
($a$) under the Fatal Accidents Act 1976 (c.~30);

($b$) to the extent that it is made in respect of a liability arising by virtue of section~1 of the Damages (Scotland) Act 1976 (c.~13);

($c$) under the Fatal Accidents (Northern Ireland) Order 1977 (S.I.~1977/1251 (N.I.~18)); or

($d$) in respect of $\mathcal{P}$, where that dependant is an intended beneficiary of part or all of that compensation,
\end{enumerate}
and~a lump sum payment has been made to that dependant, unless the order is made with the consent of that dependant and~the person by whom the payment is to be made.

(2) The court must specify in the order the amount of the payment made—
\begin{enumerate}\item[]
($a$) under the Fatal Accidents Act 1976;

($b$) to the extent that it is made in respect of a liability arising by virtue of section~1 of the Damages (Scotland) Act 1976;

($c$) under the Fatal Accidents (Northern Ireland) Order 1977; or

($d$) in respect of $\mathcal{P}$, where a dependant of $\mathcal{P}$ is an intended beneficiary of part or all of that compensation,
\end{enumerate}
which is attributable to each or any dependant of $\mathcal{P}$ who has received a lump sum   payment.”.
\end{quotation}

\medskip

9.  Where this Schedule applies, section~17 (benefits irrelevant to assessment of damages) is to apply as if—
\begin{enumerate}\item[]
($a$) in the heading for “benefits” there were substituted “lump sum payments”;

($b$) there were omitted “accident, injury or”;

($c$) for “listed benefits” there were substituted “lump sum payments”.
\end{enumerate}

\medskip

10.  Where this Schedule applies, section~18 (lump sum and~periodical payments) is to apply as if—
\begin{enumerate}\item[]
($a$) in subsection~(1)—
\begin{enumerate}\item[]
(i) for “the injured person” there were substituted “P”;

(ii) there were omitted “accident, injury or”;
\end{enumerate}

($b$) in subsection~(2), for “section~8” there were substituted “regulation~12 of the Lump Sum Payments Regulations”;

($c$) for subsection~(3) there were substituted—
\begin{quotation}
“(3) For the purposes of subsection~(2), the regulations~may provide for—
\begin{enumerate}\item[]
($a$) the gross amounts of the compensation payments to be aggregated and~for the aggregate amount to be the gross amount of the compensation payment for the purposes of regulation~12 of the Lump Sum Payments Regulations; and

($b$) for the amount of any lump sum payment to be taken to be the amount determined in accordance with the most recent certificate.”;
\end{enumerate}
\end{quotation}

($d$) in subsection~(4), for “the injured person’s” there were substituted “$\mathcal{P}$’s”;

($e$) in subsection~(5), there were omitted paragraph~($a$).
\end{enumerate}

\medskip

11.  Where this Schedule applies, section~19 (payments by more than one person) is to apply as if in—
\begin{enumerate}\item[]
($a$) subsection~(1)—
\begin{enumerate}\item[]
(i) for “injured person” there were substituted “P”;

(ii) there were omitted “accident, injury or”;
\end{enumerate}

($b$) subsection~(2)—
\begin{enumerate}\item[]
(i) for “section~6” there were substituted “regulation~10 of the Lump Sum Payments Regulations”;

(ii) for “benefits” there were substituted “lump sum payments”;
\end{enumerate}

($c$) subsection~(3)—
\begin{enumerate}\item[]
(i) in paragraph~($a$), for “benefits” there were substituted “lump sum payments”;

(ii) in paragraph~($b$), for “section~8” there were substituted “regulation~12 of the Lump Sum Payments Regulations”.
\end{enumerate}
\end{enumerate}

\medskip

12.  Where this Schedule applies, section~20 (amounts overpaid under section~6) is to apply as if in—
\begin{enumerate}\item[]
($a$) the heading and~in subsection~(1), for “section~6” there were substituted “regulation~10 of Lump Sum Payments Regulations” in each place it occurs;

($b$) subsection~(4)($a$), for “section~8” there were substituted “regulation~12 of the Lump Sum Payments Regulations”.
\end{enumerate}

\medskip

13.  Where this Schedule applies, section~21 (compensation payments to be disregarded) is to apply as if in—
\begin{enumerate}\item[]
($a$) subsections~(1) and~(5)($a$), for “sections~6 and~8” there were substituted “regulations~10 and~12 of the Lump Sum Payments Regulations” in each place it occurs;

($b$) subsection~(2)($a$), there were omitted “of recoverable benefits”;

($c$) subsection~(3)($a$)—
\begin{enumerate}\item[]
(i) for “the injured person” there were substituted “$\mathcal{P}$”;

(ii) there were omitted “accident, injury or”;
\end{enumerate}

($d$) subsection~(4), for “section~4” there were substituted “regulation~8 of the Lump Sum Payments Regulations”;

($e$) subsection~(5)($b$), for “section~6” there were substituted “regulation~10 of the Lump Sum Payments Regulations”.
\end{enumerate}

\medskip

14.  Where this Schedule applies, section~22(1) (liability of insurers) is to apply as if—
\begin{enumerate}\item[]
($a$) in paragraph~($a$), there were omitted “accident, injury or”;

($b$) for “section~6” there were substituted “regulation~10 of the Lump Sum Payments Regulations”.
\end{enumerate}

\medskip

15.  Where this Schedule applies, section~23 (provision of information) is to apply as if—
\begin{enumerate}\item[]
($a$) in subsection~(1), for—
\begin{enumerate}\item[]
(i) “any accident, injury or” there were substituted “a”;

(ii) “any person (“the injured person”)” there were substituted “$\mathcal{P}$”;

(iii) “the injured person” there were substituted “$\mathcal{P}$”;
\end{enumerate}

($b$) in subsection~(1)($a$), there were omitted “accident, injury or”;

($c$) for subsection~(2), there were substituted—
\begin{quotation}
“(2) Where $\mathcal{P}$ or a dependant of $\mathcal{P}$, receives or claims a lump sum payment which is or is likely to be paid in respect of the disease suffered by $\mathcal{P}$, the prescribed information about the disease must be given to the Secretary of State by $\mathcal{P}$ or a dependant of $\mathcal{P}$, as the case may be.”;
\end{quotation}

($d$) in subsection~(3), for “listed benefit” there were substituted “lump sum payment”;

($e$) in subsection~(4)—
\begin{enumerate}\item[]
(i) for “any accident, injury or” there were substituted “a”;

(ii) there were omitted “, or any damage to property,”;
\end{enumerate}

($f$) there were omitted subsections~(5), (6) and~(8).
\end{enumerate}

\medskip

16.  Where this Schedule applies, section~26 (residence of the injured person---Northern Ireland) is to apply as if—
\begin{enumerate}\item[]
($a$) in subsections~(1)($a$)  and~($b$)(i), (2)($a$), ($b$)  and~($c$)  and~(3)($d$)(ii), there were omitted “of recoverable benefits” in each place it occurs;

($b$) in subsections~(1)($c$)(ii)  and~(2)($c$)(i), for “section~6” there were substituted “regulation~10 of the Lump Sum Payments Regulations”;

($c$) in subsections~(1) and~(2), for “injured person’s address” there were substituted “address of $\mathcal{P}$”;

($d$) for subsection~(3)($a$), there were substituted—
\begin{quotation}
“($a$) “the address of $\mathcal{P}$” is the address first notified in writing to the person making the payment by or on behalf of $\mathcal{P}$ as the residence of $\mathcal{P}$ (or if $\mathcal{P}$ had died, by or on behalf of the person entitled to receive the compensation payment as the last residence of $\mathcal{P}$),”;
\end{quotation}

($e$) in subsection~(3)($d$)(i)  and~the heading to this section, for “the injured person” there were substituted “$\mathcal{P}$” in each place it occurs.
\end{enumerate}

\medskip

17.  Where this Schedule applies, section~27 (jurisdiction of courts---Northern Ireland) is to apply as if in—
\begin{enumerate}\item[]
($a$) subsections~(1) and~(2), for “section~7” there were substituted “regulation~11 of the Lump Sum Payments Regulations” in each place it occurs;

($b$) subsection~(3)($a$)(i), for—
\begin{enumerate}\item[]
(i) “the injured person” the first time it occurs, there were substituted “$\mathcal{P}$”;

(ii) “the injured person or, if he” there were substituted “$\mathcal{P}$ or, if $\mathcal{P}$”.
\end{enumerate}
\end{enumerate}

\medskip

18.  Where this Schedule applies, section~29 (general interpretation)\footnote{Section 29 was amended by the Social Security Act 1998, section 86(1) and paragraph~153 of Schedule 7.} is to apply as if—
\begin{enumerate}\item[]
($a$) there were omitted the following definitions—
\begin{enumerate}\item[]
(i) “benefit”;

(ii) “compensation scheme for motor accidents”;

(iii) “listed benefit”;
\end{enumerate}

($b$) in the appropriate place, there were inserted the following definitions—
\begin{enumerate}\item[]
(i) “certificate” means a certificate which includes amounts in respect of recoverable benefits and~of recoverable lump sum payments, including where any of those amounts are nil;

(ii) “$\mathcal{P}$” is to be construed in accordance with regulation~5 of the Lump Sum Payments Regulations;

(iii) “recoverable lump sum payments” means any lump sum payments which are recoverable by virtue of regulation~4 of the Lump Sum Payments Regulations;

(iv) “the Lump Sum Payments Regulations” means the Social Security (Recovery of Benefits) (Lump Sum Payments) Regulations 2008.
\end{enumerate}
\end{enumerate}

\medskip

19.  Where this Schedule applies, Part~I of Schedule 1 (compensation payments---exempted payments)\footnote{Part I of Schedule 1 was amended by the Powers of Criminal Courts (Sentencing) Act 2000 (c.~6), section 165 and paragraph 181 of Schedule 9 and S.I.~2001/3649, article 358.} is to apply as if—
\begin{enumerate}\item[]
($a$) in paragraph~2 and~3($a$), for “the injured person” there were substituted “$\mathcal{P}$” in each place it occurs;

($b$) in paragraph~3($a$)  and~($b$)  there were omitted “accident, injury or” in each place it occurs;

($c$) for paragraph~5(1) there were substituted—
\begin{quotation}
“(1) Any payment made to $\mathcal{P}$ or a dependant of $\mathcal{P}$ by an insurer under the terms of any contract of insurance entered into between $\mathcal{P}$ and~the insurer before the date on which $\mathcal{P}$ or a dependant of $\mathcal{P}$ first claims a lump sum payment in consequence of the disease in question suffered by $\mathcal{P}$.”;
\end{quotation}

($d$) in paragraph~6 for “an accident, injury or” there were substituted “a”.
\end{enumerate}

\medskip

20.  Where this Schedule applies, paragraph~9 of Part~II of Schedule 1 (compensation payments---power to disregard small payments) is to apply as if in—
\begin{enumerate}\item[]
($a$) sub-paragraph~(1), for “sections~6 and~8” there were substituted “regulations~10 and~12 of the Lump Sum Payments Regulations”;

($b$) sub-paragraph~(3)($a$)—
\begin{enumerate}\item[]
(i) for “injured person” there were substituted “$\mathcal{P}$”;

(ii) there were omitted “accident, injury or”.
\end{enumerate}
\end{enumerate}

\part[Schedule 2 --- Consequential amendments]{Schedule 2\\*Consequential amendments}

\renewcommand\parthead{--- Schedule 2}

1.  The Social Security and~Child Support (Decisions and~Appeals) Regulations 1999\footnote{S.I.~1999/991.} are amended as follows—
\begin{enumerate}\item[]
($a$) in regulation~1(3) (interpretation), after the definition of “referral” insert—
\begin{quotation}
““the Lump Sum Payments Regulations” means the Social Security (Recovery of Benefits) (Lump Sum Payments) Regulations 2008;”;
\end{quotation}

($b$) after regulation~9 (certificates of recoverable benefits), insert—
\begin{quotation}
\subsection*{“Review of certificates}

9ZA.---(1)  A certificate may be reviewed under section~10 of the 1997 Act where the Secretary of State is satisfied that—
\begin{enumerate}\item[]
($a$) a mistake (whether in the computation of the amount specified or otherwise) occurred in the preparation of the certificate;

($b$) the lump sum payment recovered from a compensator who makes a compensation payment (as defined in section~1A(5) of the 1997 Act) is in excess of the amount due to the Secretary of State;

($c$) incorrect or insufficient information was supplied to the Secretary of State by the compensator who applied for the certificate and~in consequence the amount of lump sum payment specified in the certificate was less than it would have been had the information supplied been correct or sufficient;

($d$) a ground for appeal is satisfied under section~11 of the 1997 Act or an appeal has been made under that section; or

($e$) a certificate has been issued and, for any reason, a recoverable lump sum payment was not included in that certificate.
\end{enumerate}

(2) In this regulation~and~regulations~1(3) in paragraph~($b$)  of the definition of “party to the proceedings”, 29, 31, 33, 36(2)($a$)(ii)  and~58(1), where applicable—
\begin{enumerate}\item[]
($a$) any reference to the 1997 Act is to be construed so as to include a reference to that Act as applied by regulation~2 of the Lump Sum Payments Regulations and, where applicable, as modified by Schedule 1 to those Regulations;

%($b$) “certificate” has the same meaning as in regulation~1(2) of the Lump Sum Payments Regulations;

% Reg 9ZA(2)(b) substituted (1.10.08) by SI 2008/2365 reg 6(4)
($b$) “certificate” means a certificate of recoverable lump sum payments, including where any of the amounts is nil;

($c$) “lump sum payment” is a payment to which section~1A(2) of the 1997 Act  applies;

($d$) “$\mathcal{P}$” is to be construed in accordance with regulations~4(1)($a$)(i)  and~5 of the Lump Sum Payments Regulations.”;
\end{enumerate}
\end{quotation}

($c$) in regulation~29 (further particulars required relating to certificate of recoverable benefits appeals or applications)—
\begin{enumerate}\item[]
(i) in the heading to that regulation~and~paragraphs (1), (1)($a$)  and~(6), after “recoverable benefits” insert “or, as the case may be, recoverable lump sum payments”;

(ii) in paragraph~(2), after “1997 Act” insert “or, in the case of lump sum payments, 
%regulation~14 
regulation 13  % Words substituted (1.10.08) by SI 2008/2365 reg 6(5)
of the Lump Sum Payments Regulations”;
\end{enumerate}

($d$) in regulation~31 (time within which an appeal is to be brought)—
\begin{enumerate}\item[]
(i) in paragraph~(3), after “recoverable benefits” insert “or, as the case may be, recoverable lump sum payments”;

(ii) in paragraph~(3)($a$)  after “1997 Act” add “or, in the case of lump sum payments, regulation~10 of the Lump Sum Payments Regulations”;

(iii) for paragraph~(3)($c$)  substitute—
\begin{quotation}
“($c$) where an agreement is made under which an earlier compensation payment is treated as having been made in final discharge of a claim made by or in respect of—
\begin{enumerate}\item[]
(i) an injured person, arising out of the accident, injury or disease; or

(ii) $\mathcal{P}$, arising out of the disease,
\end{enumerate}
not later than one month after the date of that agreement.”;
\end{quotation}
\end{enumerate}

%($e$) in regulation~33(2)($a$)  (making of appeals and~applications), after “recoverable benefits” insert “or, as the case may be, recoverable lump sum payments”.

% Para 1(e) substituted (1.10.08) by SI 2008/2365 reg 6(6)
($e$) in regulation 33 (making of appeals and applications)—
\begin{enumerate}\item[]
(i) in paragraph (1)($d$)  after “recoverable benefits” insert “,~recoverable lump sum payments”;

(ii) in paragraph (2)($a$)  after “recoverable benefits” insert “or, as the case maybe, recoverable lump sum payments”.
\end{enumerate}
\end{enumerate}

\amendment{
Inserted reg.~9ZA(2)(b) in para.~1(b) substituted, words substituted in para.~1(c)(ii) and para.~1(e) substituted (1.10.08) by the Social Security (Miscellaneous Amendments) (No.~3) Regulations 2008 reg.~6(4)--(6).
}

\part{Explanatory Note}

\renewcommand\parthead{— Explanatory Note}

\subsection*{(This note is not part of the Regulations)}

These Regulations make provision for the recovery of lump sum payments being payments to which section~1A(2) of the Social Security (Recovery of Benefits) Act 1997 (c.~27) (“the 1997 Act”) applies. Section 1A was inserted into the 1997 Act by section~54 of the Child Maintenance and~Other Payments Act 2008 (c.~6) (“the 2008 Act”).

Regulation 2 sets out the provisions of the 1997 Act which apply for the purposes of these Regulations and~introduces Schedule 1 which modifies certain of those provisions.

Regulation 3 makes consequential amendments to the Social Security and~Child Support (Decisions and~Appeals) Regulations 1999 (S.I.~1999/991).

Regulation 4 sets out the circumstances in which the Secretary of State may recover lump sum payments.

Regulation 5 provides for the application of these Regulations to the dependant of a person with a disease to which these Regulations apply where that dependant is the person to whom, or in respect of whom, a lump sum payment is made.

Regulation 6 provides that these Regulations apply to a compensation payment made on or after the day on which section~54 of the 2008 Act comes into force.

Regulation 7 sets out the trusts and~payments which are prescribed for the purposes of paragraphs 4 and~8, respectively, of Schedule 1 to the 1997 Act as being exempted payments.

Regulation 8 sets out the procedure for applications by a compensator for a certificate, including the date on which an application for a certificate is to be treated as received by the Secretary of State and~the 4 week time limit for the Secretary of State to issue the certificate.

Regulation 9 sets out the information which much be contained in a certificate, the person to whom information must be provided, the right of that person to require more detailed particulars of the information contained in the certificate and~provides for the Secretary of State to estimate the amount of lump sum payments specified in the certificate.

Regulation 10 sets out the liability to pay the Secretary of State in respect of any recoverable lump sum payment.

Regulation 11 provides for the procedure to be followed where the compensator has not applied for a certificate and~no payment has been made to the Secretary of State within the time limit set out in regulation~10.

Regulations 12 and~13 make provision for the reduction of a compensation payment by the amount of the lump sum payment or where the lump sum payment is equal to, or greater than, the compensation payment, reducing that payment to nil. This regulation~also provides for the attribution of the reduction and~reductions to be made in respect of compensation payments made where a lump sum payment has been made to a dependant.

Regulation 14 sets out the procedure for making a reduction in a case where two compensation payments are made at different times to the same person in consequence of the same disease.

Regulations 15 to 17 set out the requirements and~time limits in respect of the provision of information by the person or dependant who has been paid a lump sum and~the compensator.

Regulation 18 makes provision for the recovery of lump sum payments where the compensation payments are periodical payments. This regulation~provides that where there is an agreement to make periodical payments, the whole of the compensation due to be paid under such an agreement is a compensation payment from which lump sum payments may be recovered under these Regulations.

Regulation 19 provides for the making of adjustments to the amount due to be paid to the Secretary of State by the compensator where in consequence of a review or an appeal that amount has either been increased or decreased.

An assessment of the impact of these Regulations on business and~the voluntary sector is included in the impact assessment that accompanied the Child Maintenance and~Other Payments Bill. Copies of that assessment are available in the libraries of both Houses of Parliament, and~also may be obtained from the Better Regulations Unit of the Department for Work and~Pensions, level 4, the Adelphi, 1--11 John Adam Street, London \textsc{\lowercase{WC2N 6HT}}, or from the Department for Work and~Pensions website: \url{http://www.dwp.gov.uk/childmaintenance/pdfs/cm-bill-rial.pdf}

\end{document}