\documentclass[12pt,a4paper]{article}

\newcommand\regstitle{The Child Support (Miscellaneous and~Consequential Amendments) Regulations 2009}

\newcommand\regsnumber{2009/736}

%\opt{newrules}{
\title{\regstitle}
%}

%\opt{2012rules}{
%\title{Child Maintenance and~Other Payments Act 2008\\(2012 scheme version)}
%}

\author{S.I.\ 2009 No.\ 736}

\date{Made
18th March 2009\\
%Laid before Parliament
%4th March 2009\\
Coming into~force
6th April 2009
}

%\opt{oldrules}{\newcommand\versionyear{1993}}
%\opt{newrules}{\newcommand\versionyear{2003}}
%\opt{2012rules}{\newcommand\versionyear{2012}}

\usepackage{csa-regs}

\setlength\headheight{27.61603pt}

%\hbadness=10000

\begin{document}

\maketitle

\noindent
The Secretary of State for Work and~Pensions makes the following Regulations in exercise of the powers conferred by sections 52(4) and~54 of, and~paragraphs~5(1) and~(2), 6(4) and~9($d$)  of Schedule 1 and~paragraphs~4(1) and~5(1) of Schedule 4B to, the Child Support Act 1991\footnote{1991 c.~48; section~54 is cited for the meaning ascribed to the word “prescribed”. Schedule 4B was inserted by section~6(2) of, and~Schedule 2 to, the Child Support Act 1995 (c.~34) and~substituted by section~6(2) of, and~Part 2 of Schedule 2 to, the Child Support, Pensions and~Social Security Act 2000 (c.~19). These Regulations exercise (in regulation 2) powers conferred by paragraph~5(1) of that Schedule prior to its substitution by section~6(2) of, and~Part II of Schedule 2 to, the Child Support, Pensions and~Social Security Act 2000, and~(in regulation 4) powers conferred by paragraph~4(1) of that Schedule as so substituted.}.

A draft of this instrument was laid before and~approved by a resolution of each House of Parliament in accordance with section~52(2)($c$)  of that Act. 

{\sloppy

\tableofcontents

}

\bigskip

\setcounter{secnumdepth}{-2}

\subsection[1. Commencement, citation and~interpretation]{Commencement, citation and~interpretation}

1.---(1)  These Regulations may be cited as the Child Support (Miscellaneous and~Consequential Amendments) Regulations 2009 and~shall come into force on 6th April 2009.

(2) In these Regulations—
\begin{enumerate}\item[]
“the Departure Direction Regulations” means the Child Support Departure Direction and~Consequential Amendments Regulations 1996\footnote{S.I.~1996/2907, which is revoked, with savings, by S.I.~2001/156.};

“the Maintenance Assessments and~Special Cases Regulations” means the Child Support (Maintenance Assessments and~Special Cases) Regulations 1992\footnote{S.I.~1992/1815, to which there are amendments not relevant to these Regulations.}; and

“the Variations Regulations” means the Child Support (Variations) Regulations 2000\footnote{S.I.~2001/156, relevant amending instrument is S.I.~2005/785.}.
\end{enumerate}

\subsection[2. Amendments to the Departure Direction Regulations]{Amendments to the Departure Direction Regulations}

2.---(1)  The Departure Direction Regulations are amended as follows.

(2) In regulation 1(2) (citation, commencement and~interpretation), after the definition of “Arrears Regulations” insert—
\begin{quotation}
““the Commission” means the Child Maintenance and~Enforcement Commission;”.
\end{quotation}

(3) For regulation 24 of the Departure Direction Regulations (diversion of income), substitute—
\begin{quotation}
“24.---(1)  A case shall constitute a case for the purposes of paragraph~5(1) of Schedule 4B to the Act where—
\begin{enumerate}\item[]
($a$) the non-applicant (“$A$”) has the ability to control the amount of income that—
\begin{enumerate}\item[]
(i) $A$ receives, or

(ii) is taken into account as $A$’s assessable income,
\end{enumerate}
including earnings from employment or self-employment and~dividends from shares, whether or not the whole of that income is derived from the company or business from which those earnings are derived; and

($b$) the Commission is satisfied that $A$ has unreasonably reduced the amount of $A$’s income which would otherwise fall to be taken into account under regulation 7 or 8 of the Maintenance Assessments and~Special Cases Regulations by diverting it to other persons or for purposes other than the provision of such income for $A$.
\end{enumerate}

(2) In this regulation “assessable income” means the amount calculated in accordance with paragraph~5(1) to (3) of Schedule 1 to the Act and~regulations made for the purposes of that paragraph.”.
\end{quotation}

\subsection[3. Amendment of the Maintenance Assessments and~Special Cases Regulations]{Amendment of the Maintenance Assessments and~Special Cases Regulations}

3.---(1)  The Maintenance Assessments and~Special Cases Regulations are amended as follows.

(2) In regulation 1(2) (citation, commencement and~interpretation)—
\begin{enumerate}\item[]
($a$) after the definition of “patient” insert—
\begin{quotation}
““pensionable age” has the meaning given by the rules in paragraph~1 of Schedule 4 to the Pensions Act 1995\footnote{1995 c.~26; paragraph~1 of Schedule 4 was amended by: paragraph~39 of Schedule 2 to the State Pension Credit Act 2002  (c.~16); paragraph~13 of Schedule 3 to the Welfare Reform Act 2007 (c.~5); paragraph~4 of Schedule 3 to the Pensions Act 2007 (c.~22).};”;
\end{quotation}

($b$) after the definition of “profit-related pay” insert—
\begin{quotation}
““qualifying age for state pension credit” means—
\begin{enumerate}\item[]
($a$) 
in the case of a woman, pensionable age; or

($b$) 
in the case of a man, the age which is pensionable age in the case of a woman born on the same day as the man;”.
\end{enumerate}
\end{quotation}
\end{enumerate}

(3) In regulation 9(1)($d$)  (exempt income: calculation or estimation of $E$), for “aged less than 60” substitute “who had not attained the qualifying age for state pension credit”.

(4) In regulation 18(2)($a$)(iii) (excessive housing costs), for “were aged less than 60” substitute “had not attained the qualifying age for state pension credit”.

(5) In Schedule 2, after paragraph~15 insert—
\begin{quotation}
“15A.  A payment made by the Secretary of State under section~2 of the Employment and~Training Act 1973\footnote{1973 c.~50; section~2 was substituted by section~25(1) of the Employment Act 1988 (c.~19).} by way of In-Work Credit, Better Off In-Work Credit or Return to Work Credit.”.
\end{quotation}

\subsection[4. Amendments to the Variations Regulations]{Amendments to the Variations Regulations}

4.---(1)  The Variations Regulations are amended as follows.

(2) In regulation 1(2) (citation, commencement and~interpretation), after the definition of “capped amount” insert—
\begin{quotation}
““the Commission” means the Child Maintenance and~Enforcement Commission;”.
\end{quotation}

(3) In regulation 19 of the Variations Regulations (income not taken into account and~diversion of income), for paragraph~(4) substitute—
\begin{quotation}
“(4) A case shall constitute a case for the purposes of paragraph~4(1) of Schedule 4B to the Act where—
\begin{enumerate}\item[]
($a$) the non-resident parent (“$P$”) has the ability to control the amount of income that—
\begin{enumerate}\item[]
(i) $P$ receives, or

(ii) is taken into account as $P$’s net weekly income,
\end{enumerate}
including earnings from employment or self-employment, whether or not the whole of that income is derived from the company or business from which those earnings are derived; and

($b$) the Commission is satisfied that $P$ has unreasonably reduced the amount of $P$’s income which would otherwise fall to be taken into account under the Maintenance Calculations and~Special Cases Regulations or paragraph~(1A) by diverting it to other persons or for purposes other than the provision of such income for $P$.
\end{enumerate}

(4A) In paragraph~(4), “net weekly income” has the same meaning as in the Maintenance Calculations and~Special Cases Regulations.”.
\end{quotation}

\bigskip

\pagebreak[3]

Signed 
by authority of the 
Secretary of State for~Work and~Pensions.
%I concur
%By authority of the Lord Chancellor

{\raggedleft
\emph{Kitty Ussher}\\*
%Secretary
%Minister
Parliamentary Under-Secretary 
of State\\*Department 
for~Work and~Pensions

}

18th March 2009

\small

\part{Explanatory Note}

\renewcommand\parthead{— Explanatory Note}

\subsection*{(This note is not part of the Regulations)}

The powers exercised to make these Regulations are those contained in the Child Support Act 1991 (c.~48) (“the 1991 Act”). Some of those powers are conferred by provisions of the 1991 Act prior to the amendments made to that Act by the Child Support, Pensions and~Social Security Act 2000 (c.~19) (“the 2000 Act”), some of which amendments are not fully in force, and~relate to the child support scheme which was in force prior to 3rd March 2003 and~which remains in force for the purposes of certain cases (“the old scheme”). Other powers are conferred by provisions of the 1991 Act as amended by the 2000 Act, which relate to the child support scheme provided for by those amendments, which came into force for the purposes of specified categories of cases on 3rd March 2003 (see the Child Support, Pensions and~Social Security Act 2000 (Commencement No.~12) Order 2003) (“the current scheme”).

These Regulations are made in respect of functions of the Child Maintenance and~Enforcement Commission (“the Commission”). Section 13 of the Child Maintenance and~Other Payments Act 2008 (c. 6) transfers functions from the Secretary of State to the Commission and~that section~was brought into force by the Child Maintenance and~Other Payments Act 2008 (Commencement No.~4 and~Transitional Provision) Order 2008 on 1st November 2008.

Regulation~2 amends the Child Support Departure Direction and~Consequential Amendments Regulations 1996 (“the Departure Direction Regulations”). This regulation substitutes regulation 24 of the Departure Direction Regulations (diversion of income) to extend the ground in that regulation for a departure direction for diversion of income to a case where the non-applicant has the ability to control the amount of income which is taken into account as assessable income received from employment or self-employment and~dividends from shares and~the Commission is satisfied that this is income which would otherwise fall to be taken into account under the Child Support (Maintenance Assessments and~Special Cases) Regulations 1992 (“the 1992 Regulations”). These amendments affect the old scheme.

Regulation~3 amends the 1992 Regulations. Regulation~3(2) inserts definitions for “pensionable age” and~“qualifying age for state pension credit” into regulation 1 of the 1992 Regulations. Regulation~3(3) and~(4) amends regulations 9 and~18 of the 1992 Regulations, replacing the references to age 60 with references to the qualifying age for state pension credit. The Pensions Act 1995 (c.~26) allows for the eventual equalisation of state pension age between men and~women. Regulation~9(1)($d$)  (exemption of disability premium for the purposes of calculating the income of the absent parent) and~regulation 18(2)($a$)(iii) (no restriction on housing costs to apply to a person entitled to disability premium) of the 1992 Regulations are amended to reflect the equalisation of state pension age.

Regulation~3(5) inserts paragraph~15A into Schedule 2 to the 1992 Regulations. This paragraph~prescribes, for the purpose of disregarding amounts when calculating parents’ net income, any payment of In-Work Credit, Better Off In-Work Credit or Return to Work Credit under section~2 of the Employment and~Training Act 1973 (c.~50). The amendments to the 1992 Regulations affect the old scheme.

Regulation~4 amends the Child Support (Variations) Regulations 2000 (“the Variations Regulations”). This regulation substitutes regulation 19(4) of the Variations Regulations (income not taken into account and~diversion of income) to extend the ground in that paragraph~for a variation for diversion of income to a case where the non-resident parent has the ability to control the amount of income which is taken into account as net weekly income received from employment or self-employment and~the Commission is satisfied that this is income which would otherwise fall to be taken into account under the Child Support (Maintenance Calculations and~Special Cases) Regulations 2000 or under paragraph~(1A) of regulation 19 of the Variations Regulations. These amendments affect the current scheme.

A full impact assessment has not been produced for this instrument as it has no impact on the private or voluntary sectors. 

\end{document}