\documentclass[12pt,a4paper]{article}

\newcommand\regstitle{The Social Security and Child Support (Decisions and Appeals) Regulations 1999}

\newcommand\regsnumber{1999/991}

\usepackage[newrules]{optional}

\opt{oldrules}{
\title{\regstitle\\(1993 scheme version)}
}

\opt{newrules}{
\title{\regstitle}
}

%\opt{2012rules}{
%\title{Child Maintenance and Other Payments Act 2008\\(2012 scheme version)}
%}

\author{S.I. 1999 No. 991}

\date{Made 26th March 1999\\Coming into force in accordance with regulation 1(2)}

\opt{oldrules}{\newcommand\versionyear{1993}}
\opt{newrules}{\newcommand\versionyear{2003}}
%\opt{2012rules}{\newcommand\versionyear{2012}}

\usepackage{csa-regs}

\setlength\headheight{27.61603pt}

\begin{document}

\maketitle

\noindent
Whereas a draft of this Instrument was laid before Parliament in accordance with section 80(1) of the Social Security Act 1998\footnote{\frenchspacing 1998 c. 14.} and approved by resolution of each House of Parliament;

 Now, therefore, the Secretary of State for Social Security, in exercise of powers set out in Schedule 1 to this Instrument and of all other powers enabling him in that behalf, with the concurrence of the Lord Chancellor in so far as the Regulations are made under section 6(3) of the Social Security Act 1998, by this Instrument, which contains only regulations made by virtue of, or consequential upon, those provisions of the Social Security Act 1998 and which is made before the end of the period of six months beginning with the coming into force of those provisions\footnote{\frenchspacing See section 173(5)($b$) of the Social Security Administration Act 1992 (c. 5).}, after consultation with the Council on Tribunals in accordance with section 8 of the Tribunals and Inquiries Act 1992\footnote{\frenchspacing 1992 c. 53.}, hereby makes the following Regulations:\looseness=1

{\sloppy

\tableofcontents

}

\bigskip

\setcounter{secnumdepth}{-2}

\section[Part I --- General]{Part I\\*General}

\renewcommand\parthead{--- Part I}

\subsection[1. Citation, commencement% 
%and interpretation
, application and interpretation   % Words substituted by SI 2013/381 reg 55(2)
]{Citation, commencement% 
%and interpretation
, application and interpretation   % Words substituted by SI 2013/381 reg 55(2)
}

1.—(1) These Regulations may be cited as the Social Security and Child Support (Decisions and Appeals) Regulations 1999.

(2) These Regulations shall come into force—
\begin{enumerate}\item[]
($a$) in so far as they relate to child support and for the purposes of this regulation and regulation 2 on 1st June 1999;

($b$) in so far as they relate to—
\begin{enumerate}\item[]
(i) industrial injuries benefit, guardian’s allowance and child benefit; and\looseness=-1

(ii) a decision made under the Pension Schemes Act 1993\footnote{\frenchspacing 1993 c. 48; section 170 was substituted by paragraph 131 of Schedule 7 to the Social Security Act 1998.} by virtue of section 170(2) of that Act;
\end{enumerate}
on 5th July 1999;

($c$) in so far as they relate to retirement pension, widow’s benefit, incapacity benefit, severe disablement allowance and maternity allowance, on 6th September 1999;

($d$) in so far as they relate to
%family credit and disability working allowance, 
working families' tax credit and disabled person’s tax credit,  % Words substituted (5.10.99) by SI 1999/2570 reg 3
on 5th October 1999;

\pagebreak[3]

($e$) in so far as they relate to attendance allowance, disability living allowance, invalid care allowance, jobseeker’s allowance, credits of contributions or earnings, home responsibilities protection and vaccine damage payments, on 18th October 1999; and

($f$) for all remaining purposes, on 29th November 1999.
\end{enumerate}

% Reg 1(2A), (2B) inserted by SI 2013/381 reg 55(3)
(2A) In so far as these Regulations relate to—
\begin{enumerate}\item[]
($a$) an employment and support allowance payable under the Welfare Reform Act, they apply only in so far as the Act has effect apart from the amendments made by Schedule 3 and Part I of Schedule 14 to the Welfare Reform Act 2012 (“the 2012 Act”) (removing references to an income-related allowance);

($b$) a jobseeker’s allowance payable under the Jobseekers Act 1995, they apply only in so far as the Act has effect apart from the amendments made by Part I of Schedule 14 to the 2012 Act (removing references to an income-based allowance).
\end{enumerate}

(2B) These Regulations do not apply to universal credit (within the meaning of Part I of the Welfare Reform Act 2012) or personal independence payment (within the meaning of Part IV of that Act).

(3) In these Regulations, unless the context otherwise requires—
\begin{enumerate}\item[]
“the Act” means the Social Security Act 1998;

“the 1997 Act” means the Social Security (Recovery of Benefits) Act 1997\footnote{\frenchspacing 1997 c. 27.};

% Definition of ``the Arrears, Interest and Adjustment of Maintenance Assessments Regulations'' inserted (3.3.03 for new-rules cases) by SI 2000/3185 reg 2(a)
[\emph{2003 scheme}] “the Arrears, Interest and Adjustment of Maintenance Assessments Regulations” means the Child Support (Arrears, Interest and Adjustment of Maintenance Assessments) Regulations 1992\footnote{\frenchspacing S.I. 1992/1816. The relevant amending instruments are S.I. 1995/1045 and S.I. 1999/1501.};

% Definition of ``assessed income period''' inserted (7.4.03) by SI 2002/3019 reg 16(a)
“assessed income period” is to be construed in accordance with sections~6 and 9 of the State Pension Credit Act;

“the Claims and Payments Regulations” means the Social Security (Claims and Payments) Regulations 1987\footnote{\frenchspacing S.I. 1987/1968.};

“appeal” means an appeal to 
%an appeal tribunal
the First-tier Tribunal%  % Words substituted (3.11.08) by SI 2008/2683 Sch 1 para 96(a)
;

% Definitions of ``bereavement allowance'', ``bereavement benefit'', ``bereavement payment'' inserted by SI 2016/1145 reg 4(2)(a)
“bereavement allowance” means an allowance under section 39B of the Contributions and Benefits Act;

“bereavement benefit” means—
\begin{enumerate}\item[]
($a$)
a bereavement allowance;

($b$)
a bereavement payment; or

($c$)
a widowed parent’s allowance;
\end{enumerate}

“bereavement payment” means a bereavement payment under section 36 of the Contributions and Benefits Act;

% Definition of ``the Board'' inserted (5.7.99) by SI 1999/1662 reg 3(2)(a)(i)
“the Board” means the Commissioners 
%of Inland Revenue%
for Her Majesty’s Revenue and Customs%  % Words inserted by SI 2016/1145 reg 4(2)(b)
;

“claimant” means—
\begin{enumerate}\item[]
($a$) any person who is a claimant for the purposes of section 191 of the Administration Act% 
%or section 35(1) of the Jobseekers Act 
, section 35(1) of the Jobseekers Act% 
%or section 17(1) of the State Pension Credit Act  % Words substituted (7.4.03) by SI 2002/3019 reg 16(b)
, section 17(1) of the State Pension Credit Act or section 24(1) of the Welfare Reform Act  % Words substituted (27.7.08) by SI 2008/1554 reg 30(a)
or any other person from whom benefit is alleged to be recoverable; and

($b$) any person subject to a decision of 
%the Secretary of State 
an officer of the Board  % Words substituted (5.7.99) by SI 1999/1662 reg 3(2)(a)(ii)
under the Pension Schemes Act 1993\footnote{\frenchspacing 1993 c. 48.};
\end{enumerate}

% Definition of ``clerk to the appeal tribunal'' omitted (3.11.08) by SI 2008/2683 Sch 1 para 96(e)(i)
%“clerk to the appeal tribunal” means a clerk assigned to the appeal tribunal in accordance with regulation 37;

% Definition of ```the Commission'' inserted (1.11.08) by SI 2008/2656 reg 4(2), omitted (1.8.12) by SI 2012/2007 Sch para 113(2)(a)
%“the Commission” means the Child Maintenance and Enforcement Commission;

% Definition of ``contribution-based jobseeker's allowance'' inserted by SI 2016/1145 reg 4(2)(c)
“contribution-based jobseeker’s allowance” means a contribution-based jobseeker’s allowance under Part I of the Jobseekers Act;

% Definition of  ``contributory employment and support allowance'' inserted (27.7.08) by SI 2008/1554 reg 30(b)
“contributory employment and support allowance” means a contributory allowance under Part I of the Welfare Reform Act;

% Definition of ``couple'' inserted (5.12.05) by SI 2005/2878 reg 8(2)(a)
%“couple” [\emph{in Scotland}] means—
%\begin{enumerate}\item[]
%($a$) 
%a man and woman who are married to each other and are members of the same household;
%
%($b$) 
%a man and woman who are not married to each other but are living together as husband and wife;
%
%($c$) 
%two people of the same sex who are civil partners of each other and are members of the same household; or
%
%($d$) 
%two people of the same sex who are not civil partners of each other but are living together as if they were civil partners,
%\end{enumerate}
%and for the purposes of paragraph ($d$), two people of the same sex are to be regarded as living together as if they were civil partners if, but only if, they would be regarded as living together as husband and wife were they instead two people of the opposite sex; 

% Definition of ``couple'' substituted by SI 2014/107 Sch 1 para 26 (England and Wales) and SI 2014/3229 Sch 6 para 18 (Scotland)
“couple” means—
\begin{enumerate}\item[]
($a$) 
two people who are married to, or civil partners of, each other and are members of the same household; or

($b$) 
two people who are not married to, or civil partners of, each other but are living together as a married couple;
\end{enumerate}

“the date of notification” means—
\begin{enumerate}\item[]
($a$) the date that notification of a decision of the Secretary of State 
or an officer of the Board  % Words inserted (5.10.99) by SI 1999/2570 reg 4(a)
is treated as having been given or sent in accordance with regulation 2($b$); 
%or  % Word omitted by SI 2011/1498 reg 5(2)(a)

($b$) in the case of a social fund payment arising in accordance with regulations made under section 138(2) of the Contributions and Benefits Act—
\begin{enumerate}\item[]
(i) the date seven days after the date on which the Secretary of State makes his decision to make a payment to a person to meet expenses for heating;

(ii) where a person collects the instrument of payment at a post office, the date the instrument is collected;

(iii) where an instrument of payment is sent to a post office for collection but is not collected and a replacement instrument is issued, the date on which the replacement instrument is issued; or

(iv) where a person questions his failure to be awarded a payment for expenses for heating, the date on which the notification of the Secretary of State’s decision given in response to that question is issued; or
\end{enumerate}

% Para (c) inserted by SI 2011/1498 reg 5(2)(b)
($c$) where notification of a decision of the Secretary of State is sent by means of an electronic communication (within the meaning given in section 15(1) of the Electronic Communications Act 2000), the date on which the notification is sent;
\end{enumerate}

% Definition of ``the Deferral of Retirement Pensions etc. Regulations'' inserted (6.4.06) by SI 2005/2677 reg 9(2)(b)
“the Deferral of Retirement Pensions etc.\ Regulations” means the Social Security (Deferral of Retirement Pensions, Shared Additional Pension and Graduated Retirement Benefit) (Miscellaneous Provisions) Regulations 2005\footnote{S.I. 2005/2677.};

% Definition of ``designated authority'' inserted in reg 1(3) by SI 2000/897 Sch 6 para 2(a)
%“designated authority” has the meaning it has in regulation 2(1) of the Work-focused Interviews Regulations;

% Definition of ``designated authority'' substituted (30.9.02) by SI 2002/1703 Sch 2 para 6(a)(i)
“designated authority” means—
\begin{enumerate}\item[]
    ($a$)     the Secretary of State;

    ($b$) 
    a person providing services to the Secretary of State;

    ($c$) 
    a local authority;

    ($d$) 
    a person providing services to, or authorised to exercise any functions of, any such authority;
\end{enumerate}

% Definitions of ``the Employment and Support Allowance Regulations'', ``failure determination'' inserted (27.7.08) by SI 2008/1554 reg 30(c)
“the Employment and Support Allowance Regulations” means the Employment and Support Allowance Regulations 2008\footnote{S.I. 2008/794.};

%“failure determination” means a determination by the Secretary of State under regulation 63(1) of the Employment and Support Allowance Regulations that a claimant has failed to satisfy the requirement of regulation 47 or 54 of those Regulations (requirement to take part in a work-focused health related assessment or a work-focused interview);

% Definition of ``failure determination'' substituted by SI 2011/1349 reg 21
“failure determination” means a determination by the Secretary of State under regulation 61(2) of the Employment and Support Allowance Regulations or regulation 8(2) of the Employment and Support Allowance (Work-Related Activity) Regulations 2011 that a claimant has failed to satisfy a requirement of regulation 54 of the Employment and Support Allowance Regulations (requirement to take part in a work-focused interview) or regulation 3 of the Employment and Support Allowance (Work-Related Activity) Regulations 2011 (requirement to undertake work-related activity).

% Definition of ``family'' inserted in reg 1(3) by SI 2000/1596 reg 14(a)
“family” has the same meaning as in section 137 of the Contributions and Benefits Act;

% Definition of ``financially qualified panel member'' omitted (3.11.08) by SI 2008/2683 Sch 1 para 96(e)(ii)
%“financially qualified panel member” means a panel member who satisfies the requirements of paragraph 4 of Schedule 3;

% Definition of ``the Graduated Retirement Benefit Regulations'' inserted (6.4.06) by SI 2005/2677 reg 9(2)(a)
“the Graduated Retirement Benefit Regulations” means the Social Security (Graduated Retirement Benefit) Regulations 2005\footnote{S.I. 2005/454.};

% Definition of ``income-related employment and support allowance'' inserted (27.7.08) by SI 2008/1554 reg 30(d)
“income-related employment and support allowance” means an income-related allowance under Part I of the Welfare Reform Act;

“the Income Support Regulations” means the Income Support (General) Regulations 1987\footnote{\frenchspacing S.I. 1987/1967.};

% Definition of ``Income Support Work-Related Activity Regulations'' inserted by SI 2014/1097 reg 12(2)
“Income Support Work-Related Activity Regulations” means the Income Support (Work-Related Activity) and Miscellaneous Amendments Regulations 2014;

“the Jobseeker’s Allowance Regulations” means the Jobseeker’s Allowance Regulations 1996\footnote{\frenchspacing S.I. 1996/207.};

% Definitions of ``a joint-claim couple'', ``a joint-claim jobseeker's allowance'' inserted (19.3.01) by SI 2001/518 reg 4(a)
“a joint-claim couple” has the same meaning as in section 1(4) of the Jobseekers Act 1995;

    “a joint-claim jobseeker’s allowance” has the same meaning as in section~1(4) of the Jobseekers Act 1995;

% Definition of ``legally qualified panel member'' omitted (3.11.08) by SI 2008/2683 Sch 1 para 96(e)(iii)
%“legally qualified panel member” means a panel member who satisfies the requirements of paragraph 1 of Schedule 3;

% Definition of ``limited capability for work'' inserted (27.7.08) by SI 2008/1554 reg 30(e)
“limited capability for work” has the same meaning as in section 1(4) of the Welfare Reform Act;

% Definition of ``the Breach of Community Order Regulations'' inserted (15.10.01) by SI 2001/1711 reg 2(2)(a), omitted by SI 2010/424 reg 4(2)
%“the Breach of Community Order Regulations” means the Social Security (Breach of Community Order) Regulations 2001\footnote{S.I. 2001/1395.};

% Definitions inserted (3.3.03 for new-rules cases) by SI 2000/3185 reg 2(b)
[\emph{2003 scheme}] “the Maintenance Calculation Procedure Regulations” means the Child Support (Maintenance Calculation Procedure) Regulations 2000\footnote{\frenchspacing S.I. 2001/157.};

[\emph{2003 scheme}] “the Maintenance Calculations and Special Cases Regulations” means the Child Support (Maintenance Calculations and Special Cases) Regulations 2000\footnote{\frenchspacing S.I. 2001/155.};

% Definition of ``medically qualified panel member'' omitted (3.11.08) by SI 2008/2683 Sch 1 para 96(e)(iv)
%“medically qualified panel member” means a panel member who satisfies the requirements of paragraph 2 of Schedule 3;

% Definition of ``misconceived appeal'' omitted (21.12.04) by SI 2004/3368 reg 2(2)
%“misconceived appeal” means an appeal which is—
%\begin{enumerate}\item[]
%($a$) frivolous or vexatious; or
%
%($b$) obviously unsustainable and has no prospect of success,
%\end{enumerate}
%other than an out of jurisdiction appeal;

%“official error” means an error made by an officer of the Department of Social Security% 
%, the Board  % Words inserted (5.7.99) by SI 1999/1662 reg 3(2)(a)(iii)
%or the Department for Education and Employment acting as such which no person outside 
%%either Department 
%any of those Departments  % Words substituted (5.7.99) by SI 1999/1662 reg 3(2)(a)(iii)
%caused or to which no person outside 
%%either Department 
%any of those Departments  % Words substituted (5.7.99) by SI 1999/1662 reg 3(2)(a)(iii)
%materially contributed;

%% Definition of ``official error'' substituted (3.4.00) by SI 2000/897 Sch 6 para 2(b)
%“official error” means an error made by—
%\begin{enumerate}\item[]
%    ($a$) 
%    an officer of the Department of Social Security, the Board or the Department for Education and Employment acting as such which no person outside any of those Departments caused or to which no person outside any of those Departments materially contributed;
%
%%    ($b$) 
%%    a person employed by a designated authority acting on behalf of the authority, which no person outside that authority caused or to which no person outside that authority materially contributed;
%%\end{enumerate}
%
%% Sub-para (b) substituted (19.6.00) by SI 2000/1596 reg 14(b)
%($b$) a person employed by a designated authority acting on behalf of the authority, which no person outside that authority caused or to which no person outside that authority materially contributed,
%\end{enumerate}
%but excludes any error of law which is only shown to have been an error by virtue of a subsequent decision of a Commissioner or the court;

% Definition of ``official error'' substituted (20.5.02) by SI 2002/1379 reg 2(a)
    “official error” means an error made by—
\begin{enumerate}\item[]
    ($a$) 
    an officer of the Department for Work and Pensions
%, the Commission  % Words inserted (1.11.08) by SI 2008/2656 reg 4(3)(a), omitted (1.8.12) by SI 2012/2007 Sch para 113(2)(b)
 or the Board acting as such which no person outside the Department
%, the Commission  % Words inserted (1.11.08) by SI 2008/2656 reg 4(3)(b), omitted (1.8.12) by SI 2012/2007 Sch para 113(2)(b)
 or the Inland Revenue caused or to which no person outside the Department
%, the Commission  % Words inserted (1.11.08) by SI 2008/2656 reg 4(3)(b), omitted (1.8.12) by SI 2012/2007 Sch para 113(2)(b)
 or the Inland Revenue materially contributed;
 
   ($b$) 
    a person employed by a designated authority acting on behalf of the authority, which no person outside that authority caused or to which no person outside that authority materially contributed,
\end{enumerate}
    but excludes any error of law which is shown to have been an error by virtue of a subsequent decision of 
%a Commissioner 
the Upper Tribunal  % Words substituted (3.11.08) by SI 2008/2683 Sch 1 para 96(b)
or the court;

%“out of jurisdiction appeal” means an appeal brought against a decision which is specified in Schedule 2 to the Act or a decision prescribed in regulation 27 (decisions against which no appeal lies);

% Definition of ``out of jurisdiction appeal'' substituted (5.5.03) by SI 2003/1050 reg 3(1), omitted (3.11.08) by SI 2008/2683 Sch 1 para 96(e)(v)
%“out of jurisdiction appeal” means an appeal brought against a decision which is specified in—
%\begin{enumerate}\item[]
%($a$) 
%Schedule 2 to the Act or a decision prescribed in regulation 27 (decision against which no appeal lies); or
%
%($b$) 
%paragraph 6(2) of Schedule 7 to the Child Support, Pensions and Social Security Act 2000 (appeal to appeal tribunal) or a decision prescribed in regulation 16 of the Housing Benefit and Council Tax Benefit (Decisions and Appeals) Regulations 2001 (decision against which no appeal lies);
%\end{enumerate}

% Definition of ``panel'' omitted (3.11.08) by SI 2008/2683 Sch 1 para 96(e)(vi)
%“panel” means the panel constituted under section 6;

% Definition of ``panel member'' omitted (3.11.08) by SI 2008/2683 Sch 1 para 96(e)(vii)
%“panel member” means a person appointed to the panel;

% Definition of ``panel member with a disability qualification'' omitted (3.11.08) by SI 2008/2683 Sch 1 para 96(e)(viii)
%“panel member with a disability qualification” means a panel member who satisfies the requirements of paragraph 5 of Schedule 3;

% Definition of ``partner'' inserted (20.5.02) by SI 2002/1379 reg 2(b)
“partner” means—
\begin{enumerate}\item[]
    ($a$) 
    where a person is a member of 
%a married couple or an unmarried couple
a couple%  % Words substituted (5.12.05) by SI 2005/2878 reg 8(2)(b)
, the other member of that couple; or

    ($b$) 
    where a person is polygamously married to two or more members of his household, any such member;
\end{enumerate}

“party to the proceedings” means the Secretary of State 
%or where the proceedings relate to child support, the Commission  % Words inserted (1.11.08) by SI 2008/2656 reg 4(4), omitted (1.8.12) by SI 2012/2007 Sch para 113(2)(c)
or, as the case may be, the Board or an officer of the Board,  % Words inserted (5.10.99) by SI 1999/2570 reg 4(b)
and any other person—
\begin{enumerate}\item[]
($a$) who is one of the principal parties for the purposes of sections~13 and 14;

($b$) [\emph{1993 scheme}] who has a right of appeal to 
%an appeal tribunal 
the First-tier Tribunal  % Words substituted (3.11.08) by SI 2008/2683 Sch 1 para 96(a)
under section 11(2) of the 1997 Act\footnote{\frenchspacing Section 11(2) is amended by paragraph 150(2) of Schedule 7 to the Social Security Act 1998.}, section 20 of the Child Support Act as extended by paragraph 3 of Schedule 4C to that Act\footnote{\frenchspacing Schedule 4C of the Child Support Act 1991 is inserted by paragraph 54 of Schedule 7 to the Social Security Act 1998.}, section 2B(6) of the Administration Act\footnote{Section 2B was inserted by section 57 of the Welfare Reform and Pensions Act 1999 (c. 30) and amended by section 53 of, and paragraphs 8 and 9 of Schedule 7 to, and section 54 of, and Schedule 8 to, the Employment Act 2002 (c. 22).}  % Words inserted (12.4.04) by SI 2003/1886 reg 15(2)(a)
or section 12(2);

($b$) [\emph{2003 and 2012 schemes}] who has a right of appeal to 
%an appeal tribunal 
the First-tier Tribunal  % Words substituted (3.11.08) by SI 2008/2683 Sch 1 para 96(a)
under section 11(2) of the 1997 Act, section~20 of the Child Support Act%
%as extended by paragraph 3 of Schedule 4C to that Act\footnote{\frenchspacing Schedule 4C of the Child Support Act 1991 is inserted by paragraph 54 of Schedule 7 to the Social Security Act 1998.}  % Words omitted (3.3.03 for new-rules cases only) by SI 2000/3185 reg 2(c)
, section 2B(6) of the Administration Act  % Words inserted (12.4.04) by SI 2003/1886 reg 15(2)(a)
or section 12(2);
\end{enumerate}

% Definition of ``President'' omitted (3.11.08) by SI 2008/2683 Sch 1 para 96(e)(ix)
%“President” means the President of appeal tribunals appointed under section 5;

[\emph{1993 scheme}] “referral” means a referral of an application for a departure direction to 
%an appeal tribunal 
the First-tier Tribunal  % Words substituted (3.11.08) by SI 2008/2683 Sch 1 para 96(d)
under section 28D(1)($b$) of the Child Support Act\footnote{\frenchspacing Section 28D was inserted by section 4 of the Child Support Act 1995 (c. 34).};

[\emph{2003 scheme}] “referral” means a referral of an application for a 
%departure direction 
variation  %Words substituted (3.3.03 for new-rules cases only) by SI 2000/3185 reg 2(d)
to 
%an appeal tribunal 
the First-tier Tribunal  % Words substituted (3.11.08) by SI 2008/2683 Sch 1 para 96(d)
under section 28D(1)($b$) of the Child Support Act;

% Definition of ``relevant other child'' inserted (4.7.11) by SI 2011/1464 reg 2(2)
[\emph{1993 and 2003 schemes}] “relevant other child” is to be interpreted by reference to paragraph~10C(2) of Schedule 1 to the Child Support Act;

% Definition of ``the Lump Sum Payments Regulations'' inserted (1.10.08) by SI 2008/1596 Sch 2 para 1(a)
“the Lump Sum Payments Regulations” means the Social Security (Recovery of Benefits) (Lump Sum Payments) Regulations 2008\footnote{S.I. 2008/1596.};

% Definition of ``relevant person'' inserted (3.3.03 for new-rules cases only) by SI 2000/3185 reg 2(e)
\begin{sloppypar}
[\emph{2003 scheme}] except where otherwise provided “relevant person” means—
\end{sloppypar}
\begin{enumerate}\item[]
    ($a$) 
    a person with care;

    ($b$) 
    a non-resident parent;

    ($c$) 
    a parent who is treated as a non-resident parent under regulation~8 of the Maintenance Calculations and Special Cases Regulations;

    ($d$) 
    a child, where the application for a maintenance calculation is made by that child under section 7 of the Child Support Act,
\end{enumerate}
    in respect of whom a maintenance calculation has been applied for, 
%or has been treated as applied for under section 6(3) of that Act,  % Words omitted (27.10.08) by SI 2008/2543 reg 4(2)
or is or has been in force;

% Definition of ``relevant credit'' inserted (19.6.00) by SI 2000/1596 reg 14(c)
“relevant credit” means a credit of contributions or earnings resulting from a decision in accordance with regulations made under section~22(5) of the Contributions and Benefits Act;

% Definition of ``shared additional pension'' inserted by SI 2015/1985 reg 18(2)
“shared additional pension” means a shared additional pension under section 55A or 55AA of the Contributions and Benefits Act;

% Definitions of ``state pension credit'', ``State Pension Credit Act'', ``State Pension Credit Regulations'' inserted (7.4.03) by SI 2002/3019 reg 16(c)
“state pension credit” means the benefit payable under the State Pension Credit Act;

“State Pension Credit Act” means the State Pension Credit Act 2002\footnote{2002 c. 16.};

“State Pension Credit Regulations” means the State Pension Credit Regulations 2002\footnote{S.I. 2002/1792.};

% Definition of ``tax credit'' inserted (5.10.99) by SI 1999/2570 reg 4(c)
“tax credit” means working families' tax credit or disabled person’s tax credit, construing those terms in accordance with section 1(1) of the Tax Credits Act 1999;

% Definition of ``the Transfer Act'' inserted (5.7.99) by SI 1999/1670 reg 2(2)
“the Transfer Act” means the Social Security Contributions (Transfer of Functions, etc.)\ Act 1999\footnote{\frenchspacing 1999 c. 2.};

% Definition of ``the Variations Regulations'' inserted (3.3.03 for new-rules cases only) by SI 2000/3185 reg 2(e)
[\emph{2003 scheme}] “the Variations Regulations” means the Child Support (Variations) Regulations 2000\footnote{\frenchspacing S.I. 2001/156.};

% Definition of ``the Welfare Reform Act'' inserted (27.7.08) by SI 2008/1554 reg 30(f)
“the Welfare Reform Act” means the Welfare Reform Act 2007\footnote{2007 c. 5.};

% Definition of ``widowed parent's allowance'' inserted by SI 2016/1145 reg 4(2)(d)
“widowed parent’s allowance” means an allowance under section 39A of the Contributions and Benefits Act;

% Definitions of ``work-focused interview'' and ``the Work-focused Interviews Regulations'' inserted (3.4.00) by SI 2000/897 Sch 6 para 2(c)
%“work-focused interview” has the meaning it has in regulation 3 of the Work-focused Interviews Regulations;

% Definition of ``work-focused interview'' substituted (30.9.02) by SI 2002/1703 Sch 2 para 6(a)(ii)
%“work-focused interview” means an interview which a person is required to take part in under the Social Security (Jobcentre Plus Interviews) Regulations 2002\footnote{S.I. 2002/1703.}
%or under the Social Security (Jobcentre Plus Interviews for Partners) Regulations 2003\footnote{S.I. 2003/1886.}%  % Words inserted (12.4.04) by SI 2003/1886 reg 15(2)(b)
%.

% Definition of ``work-focused interview'' substituted (26.4.04) by SI 2004/959 reg 24(2)
“work-focused interview” means an interview in which a person is required to take part in accordance with regulations made under section~2A or 2AA of the Administration Act.

% Definition of ``the Work-focused Interviews Regulations'' omitted (30.9.02) by SI 2002/1703 Sch 2 para 6(a)(iii)
%    “the Work-focused Interviews Regulations” means the Social Security (Work-focused Interviews) Regulations 2000\footnote{\frenchspacing S.I. 2000/897.}.
\end{enumerate}

% Reg 1(3A) inserted (5.7.99) by SI 1999/1662 reg 3(2)(b)
(3A) In these Regulations as they relate to any decision made under the Pension Schemes Act 1993 by virtue of section 170(2) of that Act, any reference to the Secretary of State is to be construed as if it were a reference to an officer of the Board.

(4) In these Regulations, unless the context otherwise requires, a reference—
\begin{enumerate}\item[]
($a$) to a numbered section is to the section of the Act bearing that number;

($b$) to a numbered Part is to the Part of these Regulations bearing that number;

($c$) to a numbered regulation or Schedule is to the regulation in, or Schedule to, these Regulations bearing that number;

($d$) in a regulation or Schedule to a numbered paragraph is to the paragraph in that regulation or Schedule bearing that number;

($e$) in a paragraph to a lettered or numbered sub-paragraph is to the sub-paragraph in that paragraph bearing that letter or number.
\end{enumerate}

\amendment{
Definition of ``the Transfer Act'' inserted in reg. 1(3) (5.7.99) by the Social Security and Child Support (Decisions and Appeals) Amendment (No. 3) Regulations 1999 reg. 2(2).

Words inserted in definition of ``official error'' in reg. 1(3), words substituted in definitions of ``claimant'' and ``official error'' in reg. 1(3), definition of ``the Board'' inserted in reg. 1(3) and reg. 1(3A) inserted (5.7.99) by the Social Security Contributions (Transfer of Functions, etc.) Act 1999 (Commencement No. 2 and Consequential and Transitional Provisions) Order 1999 art. 3(2) (subject to transitional provisions in art. 4).

Words inserted in para. (a) of the definition of ``the date of notification'' in reg. 1(3), words inserted in definition of ``party to the proceedings'' in reg. 1(3), words substituted in reg. 1(2)(d) and definition of ``tax credit'' inserted in reg. 1(3) (5.10.99) by the Tax Credits (Decisions and Appeals) (Amendment) Regulations 1999 regs. 3, 4.

Definitions of ``designated authority'', ``work-focused interview'' and ``the Work-focused Interviews Regulations'' inserted in reg. 1(3) and definition of ``official error'' substituted in reg. 1(3) (3.4.00) by the Social Security (Work-focused Interviews) Regulations 2000 Sch. 6 para. 2.

Definitions of ``family'' and ``relevant credit'' inserted in reg. 1(3) and sub-para. (b) of definition of ``official error'' substituted in reg. 1(3) (19.6.00) by the Social Security and Child Support (Miscellaneous Amendments) Regulations 2000 reg. 14.

Definitions of ``a joint-claim couple'' and ``a joint-claim jobseeker's allowance'' inserted in reg. 1(3) (19.3.01) by the Social Security Amendment (Joint Claims) Regulations 2001 reg. 4(a).

Definition of ```the Breach of Community Order Regulations'' inserted in reg. 1(3) (15.10.01) by the Social Security (Breach of Community Order) (Consequential Amendments) Regulations 2001 reg. 2(2)(a).

Definition of ``partner'' inserted in reg. 1(3) and definition of ``official error'' substituted in reg. 1(3) (20.5.02) by the Social Security and Child Support (Decisions and Appeals) (Miscellaneous Amendments) Regulations 2002 reg. 2.

\looseness=1
Definitions of ``designated authority'', ``work-focused interview'' in reg. 1(3) substituted and definition of ``the Work-focused Interviews Regulations'' in reg. 1(3) omitted (30.9.02) by the Social Security (Jobcentre Plus Interviews) Regulations 2002 Sch. 2 para. 6(a).

Word substituted in the definition of ``referral'', words omitted in para. (b) of the definition of ``party to the proceedings'' and definitions of ``the Arrears, Interest and Adjustment of Maintenance Assessments Regulations'', ``the Maintenance Calculation Procedure Regulations'', ``the Maintenance Calculations and Special Cases Regulations'', ``relevant person'' and ``the Variations Regulations'' inserted in reg. 1(3) (3.3.03 for new-rules cases only) by the Child Support (Decisions and Appeals) (Amendment) Regulations 2000 reg. 2 (subject to reg. 1(2)).

Words substituted in definition of ``claimant'' in reg. 1(3) and definitions of ``assessed income period'', ``state pension credit'', ``State Pension Credit Act'' and ``State Pension Credit Regulations'' inserted in reg. 1(3) (7.4.03) by the State Pension Credit (Consequential, Transitional and Miscellaneous Provisions) Regulations 2002 reg. 16.

Words ``regulation 25 of the Child Benefit and Guardian's Allowance (Decisions and Appeals) Regulations 2003'' substituted for ``regulation 27'' in the definition of ``out of jurisdiction appeal'' in reg. 1(3) (7.4.03) so far as relating to child benefit or guardian's allowance by the Child Benefit and Guardian’s Allowance (Decisions and Appeals) Regulations 2003 reg. 36.

Definition of ``out of jurisdiction appeal'' substituted in reg. 1(3) (5.5.03) by the Social Security and Child Support (Miscellaneous Amendments) Regulations 2003 reg. 3(1).

Words inserted in para. (b) of definition of ``party to the proceedings'' and in definition of ``work-focused interviews'' in reg. 1(3) (12.4.04) by the Social Security (Jobcentre Plus Interviews for Partners) Regulations 2003 reg. 15(2).

Definition of ``work-focused interview'' in reg. 1(3) substituted (26.4.04) by the Social Security (Working Neighbourhoods) Regulations 2004 reg. 24(2).

Definition of ``misconceived appeal'' in reg. 1(3) omitted (21.12.04) by the Social Security, Child Support and Tax Credits (Decisions and Appeals) Amendment Regulations 2004 reg. 2(2).

Words substituted in definition of ``partner'' in reg. 1(3) and definition of ``couple'' inserted in reg. 1(3) (5.12.05) by the Social Security (Civil Partnership) (Consequential Amendments) Regulations 2005 reg. 8(2).

Definitions of ``the Deferral of Retirement Pensions etc. Regulations'', ``the Graduated Retirement Benefit Regulations'' inserted in reg. 1(3) (6.4.06) by the Social Security (Deferral of Retirement Pensions, Shared Additional Pension and Graduated Retirement Benefit) (Miscellaneous Provisions) Regulations 2005 reg. 9(2).

Words substituted in the definition of ``claimant'' in reg. 1(3) and definitions of ``contributory employment and support allowance'', ``the Employment and Support Allowance Regulations'', ``failure determination'', ``income-related employment and support allowance'', ``limited capability for work'', ``the Welfare Reform Act'' inserted in reg. 1(3) (27.7.08) by the Employment and Support Allowance (Consequential Provisions) (No. 2) Regulations 2008 reg. 30.

Definition of ``the Lump Sum Payments Regulations'' inserted in reg. 1(3) (1.10.08) by the Social Security (Recovery of Benefits) (Lump Sum Payments) Regulations 2008 Sch. 2 para. 1(a).

Words omitted in definition of ``relevant person'' in reg. 1(3) (27.10.08) by the Child Support (Consequential Provisions) Regulations 2008 reg. 4(2).

Words inserted in definitions of ``official error'', ``party to the proceedings'' in reg. 1(3) and definition of ``the Commission'' inserted in reg. 1(3) (1.11.08) by the Child Support (Consequential Provisions) (No. 2) Regulations 2008 reg. 4.

Words substituted in definition of ``appeal'', ``official error'', ``party to the proceedings'', ``referral'' in reg. 1(3) and definitions of ``clerk to the appeal tribunal'', ``financially qualified panel member'', ``legally qualified panel member'', ``medically qualified panel member'', ``out of jurisdiction appeal'', ``panel'', ``panel member'', ``panel member with a disability qualification'', ``President'' in reg. 1(3) omitted (3.11.08) by the Tribunals, Courts and Enforcement Act 2007 (Transitional and Consequential Provisions) Order 2008 Sch. 1 para. 96.

Definition of ``the Breach of Community Order Regulations'' in reg. 1(3) omitted (22.3.10) by the Welfare Reform Act 2009 (Section 26) (Consequential Amendments) Regulations 2010 reg. 4(2).

Definition of ``failure determination'' in reg. 1(3) substituted (1.6.11) by the Employment and Support Allowance (Work-Related Activity) Regulations 2011 reg. 21.

Para. (c) inserted in definition of ``the date of notification'' in reg. 1(3) (20.6.11) by the Social Security (Electronic Communications) Order 2011 reg. 5.

Definition of ``relevant other child'' inserted in reg. 1(3) (4.7.11) by the Child Support (Miscellaneous Amendments) Regulations 2011 reg. 2(2).

Words omitted in definitions of ``official error'', ``party to the proceedings'' and definition of ``the Commission'' omitted in reg. 1(3) (1.8.12) by the Public Bodies (Child Maintenance and Enforcement Commission: Abolition and Transfer of Functions) Order 2012 Sch. para. 113(2).

\looseness=1
Definitions of ``the Arrears, Interest and Adjustment of Maintenance Assessments Regulations'', ``the Maintenance Calculation Procedure Regulations'', ``the Maintenance Calculations and Special Cases Regulations'', ``relevant other child'', ``relevant person'' and ``Variations Regulations'' in reg. 1(3) omitted (10.12.12 for 2012 scheme cases only) by the Child Support (Meaning of Child and New Calculation Rules) (Consequential and Miscellaneous Amendment) Regulations 2012 reg. 6(2).

Words substituted in heading and reg. 1(2A), (2B) inserted (8.4.13 for personal independence payment, 29.4.13 otherwise) by the Universal Credit, Personal Independence Payment, Jobseeker's Allowance and Employment and Support Allowance (Decisions and Appeals) Regulations 2013 reg. 55.

Definition of ``couple'' in reg.~1(2) substituted (13.3.14 for England and Wales only) by the Marriage (Same Sex Couples) Act 2013 (Consequential Provisions) Order 2014 Sch. 1 para. 26.

Definition of ``Income Support Work-Related Activity Regulations'' inserted in reg. 1(3) (28.4.14) by the Income Support (Work-Related Activity) and Miscellaneous Amendments Regulations 2014 reg. 12(2).

Definition of ``couple'' in reg. 3(9) substituted (16.12.14 in Scotland only) by the Marriage and Civil Partnership (Scotland) Act 2014 and Civil Partnership Act 2004 (Consequential Provisions and Modifications) Order 2014 Sch. 6 para. 18.

Definition of ``shared additional pension'' inserted in reg. 1(3) (6.4.16) by the Pensions Act 2014 (Consequential, Supplementary and Incidental Amendments) Order 2015 art.~18(2).

Definitions of ``bereavement allowance'', ``bereavement benefit'', ``bereavement payment'', ``contribution-based jobseeker's allowance'', ``widowed parent's allowance'' inserted in reg. 1(3) and definition of ``the Board'' in reg. 1(3) amended (1.1.17) by the Social Security (Credits, and Crediting and Treatment of Contributions) (Consequential and Miscellaneous Amendments) Regulations 2016 reg. 4(2).
}

\subsection[2. Service of notices or documents]{Service of notices or documents}

2.  [\emph{1993 scheme}] Where, by any provision of the Act or of these Regulations—

2.  [\emph{2003 and 2012 schemes}] 2.  Where, by any provision of the Act%
% Words inserted (3.3.03 for new-rules cases only) by SI 2000/3185 reg 3
, of the Child Support Act
or of these Regulations—

\begin{enumerate}\item[]
\looseness=1
($a$) any notice or other document is required to be given or sent 
%to the clerk to the appeal tribunal or  % Words omitted (3.11.08) by SI 2008/2683 Sch 1 para 97(a)(i)
to an officer authorised by the Secretary of State
or to an officer of the Board,  % Words inserted (5.10.99) by SI 1999/2570 reg 5(a)(i)
that notice or document shall be treated as having been so given or sent on the day that it is received 
%by the clerk to the appeal tribunal or  % Words omitted (3.11.08) by SI 2008/2683 Sch 1 para 97(a)(ii)
by an officer authorised by the Secretary of State
or by an officer of the Board,  % Words inserted (5.10.99) by SI 1999/2570 reg 5(a)(ii)
as the case may be, and

($b$) any notice (including notification of a decision of the Secretary of State
or of an officer of the Board% % Words inserted (14.2.00) by SI 2000/127 reg 2(b)
) or other document is required to be given or sent to any person other than 
%the clerk to the appeal tribunal 
%%or to an officer 
%or  % Words omitted (3.11.08) by SI 2008/2683 Sch 1 para 97(b)
an officer  % Words substituted (5.10.99) by SI 1999/2570 reg 5(b)(ii)
authorised by the Secretary of State
or an officer of the Board,  % Words inserted (5.10.99) by SI 1999/2570 reg 5(a)(ii)
as the case may be, that notice or document shall, if sent by post to that person’s last known address, be treated as having been given or sent on the day that it was posted.\end{enumerate}

\amendment{
Words inserted in reg. 2(a), (b) and words substituted in reg. 2(b) (5.10.99) by the Tax Credits (Decisions and Appeals) (Amendment) Regulations 1999 reg. 5.

Words inserted in reg. 2(b) (14.2.00) by the Tax Credits (Decisions and Appeals) (Amendment) Regulations 1999 reg. 5(b)(i) as inserted by the Tax Credits (Decisions and Appeals) (Amendment) Regulations 2000 reg. 2.

Words inserted in reg. 2 (3.3.03 for new-rules cases only) by the Child Support (Decisions and Appeals) (Amendment) Regulations 2000 reg. 3 (subject to reg. 1(2)).

Words omitted in reg. 2(a), (b) (3.11.08) by the Tribunals, Courts and Enforcement Act 2007 (Transitional and Consequential Provisions) Order 2008 Sch. 1 para. 97.
}

\addtocontents{toc}{\protect\pagebreak[3]}

%Words add to heading to Part II (3.3.03 for new-rules cases only) by SI 2000/3185 reg 4
\section[Part II --- Revisions, supersessions and other matters social security and child support]{Part II\\*Revisions, supersessions and other matters social security \emph{and child support}}

\amendment{
Words added to heading of Pt. II (3.3.03 for new-rules cases only) by the Child Support (Decisions and Appeals) (Amendment) Regulations 2000 reg. 4 (subject to reg. 1(2)).

Pt. II revoked (7.4.03) so far as relating to child benefit or guardian's allowance by the Child Benefit and Guardian’s Allowance (Decisions and Appeals) Regulations 2003 reg. 34(a).

This Part is generally applicable only to 2003 scheme cases; versions for 1993 scheme cases are generally not shown. 
}

\subsection[Chapter I --- Revisions]{Chapter I\\*Revisions}

\subsubsection[3. Revision of decisions]{Revision of decisions}

\renewcommand\parthead{--- Part II Chapter I}

3.—(1) Subject to the following provisions of this regulation, any decision of the Secretary of State 
or the Board or an officer of the Board  % Words inserted (5.10.99) by SI 1999/2570 reg 6(2)
under section 8 or 10 (“the original decision”) may be revised by him 
or them  % Words inserted (5.10.99) by SI 1999/2570 reg 6(3)
if—
\begin{enumerate}\item[]
%($a$) he 
%or they  % Words inserted (5.10.99) by SI 1999/2570 reg 6(3)
%commences action leading to the revision within one month of the date of notification of the original decision; or

%% Reg 3(1)(a) substituted (18.10.99) by SI 1999/2677 reg 6(a)
%($a$) he commences action leading to the revision within one month of the date of—
%\begin{enumerate}\item[]
%(i) notification of the original decision; or
%
%(ii) the making of an appeal under section 12 provided that the appeal is made within the time prescribed in regulation 31 or, in a case to which regulation 32 applies, the time prescribed in that regulation; or
%\end{enumerate}
%
%($b$) an application for a revision is received by the Secretary of State 
%or the Board or an officer of the Board  % Words inserted (5.10.99) by SI 1999/2570 reg 6(2)
%at the appropriate office—
%\begin{enumerate}\item[]
%(i) within one month of the date of notification of the original decision,
%
%(ii) where a written statement is requested under paragraph (1)($b$) of regulation 28, within 14 days of the expiry of the period specified in head (i), or
%
%(iii) within such longer period of time as may be allowed under regulation 4.
%\end{enumerate}

% Reg 3(1)(a), (b) substituted (20.5.02) by SI 2002/1379 reg 3(a)
($a$) he or they commence action leading to revision within one month of the date of notification of the original decision; or

($b$) an application for a revision is received by the Secretary of State or the Board or an officer of the Board at the appropriate office—
\begin{enumerate}\item[]
(i) subject to regulation 9A(3), within one month of the date of notification of the original decision;

(ii) where a written statement is requested under 
paragraph (3)($b$)  of regulation 3ZA or  % Words inserted (28.10.13) by SI 2013/2380 reg 4(2)
paragraph (1)($b$)  of regulation 28 and is provided within the period specified in head (i), within 14 days of the expiry of that period;

(iii) where a written statement is requested under 
paragraph (3)($b$)  of regulation 3ZA or  % Words inserted (28.10.13) by SI 2013/2380 reg 4(2)
paragraph (1)($b$)  of regulation 28 and is provided after the period specified in head (i), within 14 days of the date on which the statement is provided; or

(iv) within such longer period as may be allowed under regulation~4.
\end{enumerate}
\end{enumerate}

(2) Where the Secretary of State 
or the Board or an officer of the Board  % Words inserted (5.10.99) by SI 1999/2570 reg 6(2)
requires further evidence or information from the applicant in order to consider all the issues raised by an application under paragraph (1)($b$) (“the original application”), he 
or they  % Words inserted (5.10.99) by SI 1999/2570 reg 6(3)
shall notify the applicant that further evidence or information is required and the decision may be revised—
\begin{enumerate}\item[]
($a$) where the applicant provides further relevant evidence or information within one month of the date of notification or such longer period of time as the Secretary of State 
or the Board or an officer of the Board  % Words inserted (5.10.99) by SI 1999/2570 reg 6(2)
may allow; or

($b$) where the applicant does not provide such evidence or information within the time allowed under sub-paragraph ($a$), on the basis of the original application.
\end{enumerate}

(3) In the case of a payment out of the social fund in respect of maternity or funeral expenses, a decision under section 8 may be revised where the application is made—
\begin{enumerate}\item[]
($a$) within one month of the date of notification of the decision, or if later

($b$) within the time prescribed for claiming such a payment under regulation 19 of, and Schedule 4 to, the Claims and Payments Regulations\footnote{\frenchspacing \emph{See} in particular paragraphs 8 and 9 of Schedule 4 to the Social Security (Claims and Payments) Regulations 1987 (S.I. 1987/1968).}, or

($c$) within such longer period of time as may be allowed under regulation~4.
\end{enumerate}

(4) In the case of a decision made under the Pension Schemes Act 1993\footnote{\frenchspacing 1993 c. 48; section 170 was substituted by paragraph 131 of Schedule 7 to the Social Security Act 1998.} by virtue of section 170(2) of that Act, the decision may be revised at any time by 
%the Secretary of State 
an officer of the Board  % Words substituted (5.7.99) by SI 1999/1662 reg 3(3)(a)
where it contains an error.

% Reg 3(4A) inserted (20.5.02) by SI 2002/1379 reg 3(b)
(4A) Where there is an appeal against an original decision (within the meaning of paragraph (1)) within the time prescribed 
%in regulation 31, or in a case to which regulation 32 applies within the time prescribed in that regulation, 
by Tribunal Procedure Rules  % Words substituted (3.11.08) by SI 2008/2683 Sch 1 para 98(a)
but the appeal has not been determined, the original decision may be revised at any time.

(5) A decision of the 
%Secretary of State 
Board or an officer of the Board  % Words inserted (5.10.99) by SI 1999/2570 reg 6(4)(a)
under section 8 or~10—
\begin{enumerate}\item[]
($a$) 
except where paragraph (5ZA) applies,  % Words inserted (1.10.07) by SI 2007/2582 reg 3(2)
which arose from an official error; or

($b$) 
%except in the case of a disability benefit decision or an incapacity benefit decision where there has been an incapacity determination (whether before or after the decision)  % Words inserted (5.7.99) by SI 1999/1623 reg 2(a)
except in a case to which sub-paragraph ($c$)  or ($d$)  applies,  % Words substituted (24.9.07) by SI 2007/2470 reg 3(2)
where the decision was made in ignorance of, or was based upon a mistake as to, some material fact and as a result of that ignorance of or mistake as to that fact, the decision was more advantageous to the claimant than it would otherwise have been but for that ignorance or mistake;

% Reg 3(5)(b) substituted (5.10.99 for certain purposes only, see SI 1999/2570 reg 1(2)) by SI 1999/2570 reg 6(4)(b)
%($b$) which was made in ignorance of, or was based on a mistake as to, some material fact;

% Reg 3(5)(c) inserted (5.7.99) by SI 1999/1623 reg 2(b)
($c$) 
subject to sub-paragraph ($d$),  % Words inserted (24.9.07) by SI 2007/2470 reg 3(3)
where the decision is a disability benefit decision, or is an incapacity benefit decision where there has been an incapacity determination 
or is an employment and support allowance decision where there has been a limited capability for work determination  % Words inserted (27.7.08) by SI 2008/1554 reg 31(2)(a)(i)
(whether before or after the decision), which was made in ignorance or, or was based upon a mistake as to, some material fact in relation to a disability determination embodied in or necessary to the disability benefit decision%
%, or the incapacity determination
, the incapacity determination or the limited capability for work determination%  % Words substituted (27.7.08) by SI 2008/1554 reg 31(2)(a)(ii)
, and---
\begin{enumerate}\item[]
(i) as a result of that ignorance of or mistake as to that fact the decision was more advantageous to the claimant than it would otherwise have been but for that ignorance or mistake and,

(ii) the Secretary of State is satisfied that at the time the decision was made the claimant or payee knew or could reasonably have been expected at the time the decision was made to know of the fact in question and that it was relevant to the decision;
\end{enumerate}

% Reg 3(5)(d) added (24.9.07) by SI 2007/2470 reg 3(4)
($d$) where the decision 
is an employment and support allowance decision,  % Words inserted (27.7.08) by SI 2008/1554 reg 31(2)(b)(i)
is a disability benefit decision, or is an incapacity benefit decision, which was made in ignorance of, or was based upon a mistake as to, some material fact not in relation to the 
limited capability for work determination,  % Words inserted (27.7.08) by SI 2008/1554 reg 31(2)(b)(ii)
incapacity or disability determination embodied in or necessary to 
the employment and support allowance decision,  % Words substituted (27.7.08) by SI 2008/1554 reg 31(2)(b)(iii)
the incapacity benefit decision or disability benefit decision, and as a result of that ignorance of, or mistake as to that fact, the decision was more advantageous to the claimant than it would otherwise have been but for the ignorance or mistake,
\end{enumerate}
may be revised 
%at any time by the Secretary of State.
by the Board or an officer of the Board at any time not later than the end of the period of six years immediately following the date of the decision or, where ignorance of the material fact referred to in sub-paragraph~($b$)  was caused by the fraudulent or negligent conduct of the claimant, not later than the end of the period of twenty years immediately following the date of the decision.  % Words substituted (5.10.99) by SI 1999/2570 reg 6(4)(c)

% Reg 3(5ZA)--(5ZC) inserted (1.10.07) by SI 2007/2582 reg 3(3)
(5ZA) This paragraph applies where—
\begin{enumerate}\item[]
($a$) the decision which would otherwise fall to be revised is a decision to award a benefit specified in paragraph (5ZB), whether or not the award has already been put in payment;

($b$) that award was based on the satisfaction by a person of the contribution conditions, in whole or in part, by virtue of credits of earnings for incapacity for work or approved training in the tax years from 1993--94 to 2007--08;

($c$) the official error derives from the failure to transpose correctly information relating to those credits from the Department for Work and Pensions’ Pension Strategy Computer System to Her Majesty’s Revenue and Customs’ computer system (\textsc{\lowercase{NIRS2}}) or from related clerical procedures; and

($d$) that error has resulted in an award to the claimant which is more advantageous to him than if the error had not been made.
\end{enumerate}

(5ZB) The specified benefits are—
\begin{enumerate}\item[]
($a$) bereavement allowance;

($b$) contribution-based jobseeker’s allowance;

($c$) incapacity benefit;

($d$) retirement pension;

($e$) widowed mother’s allowance;

($f$) widowed parent’s allowance; 
%and  % Word omitted (27.7.08) by SI 2008/1554 reg 31(3)(a)

($g$) widow’s pension;
% 
% Reg 3(5ZB)(h) added (27.7.08) by SI 2008/1554 reg 31(3)(b)
and

($h$) contributory employment and support allowance.
\end{enumerate}

(5ZC) In paragraph (5ZA)($b$), “tax year” has the meaning ascribed to it by section 122(1) of the Contributions and Benefits Act.

% Reg 3(5A) inserted (20.5.02) by SI 2002/1379 reg 3(c)
(5A) Where—
\begin{enumerate}\item[]
($a$) the Secretary of State or the Board or an officer of the Board, as the case may be, makes a decision under section 8 or 10, or that decision is revised under section 9, in respect of a claim or award (“decision $\mathcal{A}$”) and the claimant appeals against decision $\mathcal{A}$;

($b$) decision $\mathcal{A}$ is superseded or the claimant makes a further claim which is decided (“decision $\mathcal{B}$”) after the claimant made the appeal but before the appeal results in a decision by 
%an appeal tribunal 
the First-tier Tribunal  % Words substituted (3.11.08) by SI 2008/2683 Sch 1 para 98(b)
(“decision~$\mathcal{C}$”); and\looseness=-1

($c$) the Secretary of State or the Board or an officer of the Board, as the case may be, would have made decision $\mathcal{B}$ differently if he or they had been aware of decision $\mathcal{C}$ at the time he or they made decision $\mathcal{B}$,
\end{enumerate}
decision $\mathcal{B}$ may be revised at any time.

% Reg 3(5B) inserted (24.9.07) by SI 2007/2470 reg 3(5)
(5B) A decision by the Secretary of State under section 8 or 10 awarding incapacity benefit may be revised at any time if—
\begin{enumerate}\item[]
\begin{sloppypar}
($a$) it incorporates a determination that the condition in regulation~28(2)($b$)  of the Social Security (Incapacity for Work) (General) Regulations 1995\footnote{S.I. 1995/311. Regulation 28(2)($b$) was amended by S.I. 1995/987 and S.I. 1996/3207.} (conditions for treating a person as incapable of work until the personal capability assessment is carried out) is satisfied;
\end{sloppypar}

($b$) the condition referred to in sub-paragraph ($a$)  was not satisfied at the time when the further claim was first determined; and\looseness=-1

($c$) there is a period before the award which falls to be decided.\looseness=-1
\end{enumerate}

% Reg 3(5C), (5D) inserted (27.7.08) by SI 2008/1554 reg 31(4)
(5C) A decision of the Secretary of State under section 10 made in consequence of a failure determination may be revised at any time if it contained an error to which the claimant did 
not materially contribute.

% Reg 3(5D) omitted by SI 2009/1490 reg 3(2)(a)
%(5D) A decision by the Secretary of State under section 8 or 10 awarding employment and support allowance may be revised at any time if—
%\begin{enumerate}\item[]
%($a$) it incorporates a determination that the condition in regulation 30 of the Employment and Support Allowance Regulations is satisfied;
%
%($b$) the condition referred to in sub-paragraph ($a$)  was not satisfied at the time when the claim was first determined; and
%
%($c$) there is a period before the award which falls to be decided.
%\end{enumerate}

% Reg 3(5D) inserted by SI 2011/2425 reg 12(a)
(5D) A decision by the Secretary of State under section 8 or 10 awarding an employment and support allowance may be revised at any time if—
\begin{enumerate}\item[]
($a$) it incorporates a determination that the conditions in regulation~30 of the Employment and Support Allowance Regulations are satisfied;

($b$) the condition referred to in sub-paragraph ($a$) was not satisfied at the time when the claim was made; and

($c$) there is a period before the award which falls to be decided.
\end{enumerate}

% Reg 3(5E), (5F) inserted by SI 2010/840 reg 7(2)
(5E) A decision under section 8 or 10 awarding an employment and support allowance may be revised if—
\begin{enumerate}\item[]
($a$) the decision of the Secretary of State awarding an employment and support allowance was made on the basis that the claimant had made and was pursuing an appeal against a decision of the Secretary of State that the claimant did not have limited capability for work (“the original decision”); and

($b$) the appeal to the First-tier Tribunal in relation to the original decision is successful.
\end{enumerate}

(5F) A decision under section 8 or 10 awarding an employment and support allowance may be revised if—
\begin{enumerate}\item[]
($a$) the person’s current period of limited capability for work is treated as a continuation of another such period under regulation 145(1) 
%and~(2)  % Words omitted by SI 2012/919 reg 2
of the Employment and Support Allowance Regulations; and

($b$) regulation 7(1)($b$) of those Regulations applies.
\end{enumerate}

% Reg 3(5G), (5H) inserted by SI 2011/2425 reg 12(b)
(5G) Where—
\begin{enumerate}\item[]
($a$) a person’s entitlement to an employment and support allowance is terminated because of a decision which embodies a determination that the person does not have limited capability for work;

($b$) the person appeals that decision to the First-tier Tribunal;

($c$) before or after that decision is appealed by the person, that person claims and there is a decision to award---
\begin{enumerate}\item[]
(i) income support, or

(ii) jobseeker’s allowance; and
\end{enumerate}

($d$) the decision referred to in sub-paragraph ($a$) is successfully appealed,
\end{enumerate}
the decision to award income support or jobseeker’s allowance may be revised.

(5H) Where—
\begin{enumerate}\item[]
($a$) a conversion decision within the meaning of regulation 5(2)($b$) of the Employment and Support Allowance (Transitional Provisions, Housing Benefit and Council Tax Benefit) (Existing Awards) (No.~2) Regulations 2010\footnote{S.I.~2010/1907, as amended by S.I.~2010/2430.} (deciding whether an existing award qualifies for conversion) is made in respect of a person;

($b$) the person appeals that decision to the First-tier Tribunal;

($c$) before or after that decision is appealed by the person, that person claims and there is a decision to award–
\begin{enumerate}\item[]
(i) income support, or

(ii) jobseeker’s allowance; and
\end{enumerate}

($d$) the decision referred to in sub-paragraph ($a$) is successfully appealed,
\end{enumerate}
the decision to award income support or jobseeker’s allowance may be revised.

% Reg 3(5I) inserted by SI 2012/913 reg 5
(5I) Where—
\begin{enumerate}\item[]
($a$) a decision to terminate a person’s entitlement to a contributory employment and support allowance is made because of section 1A of the Welfare Reform Act (duration of contributory allowance); and

($b$) it is subsequently determined, in relation to the period of entitlement before that decision, that the person had or is treated as having had limited capability for work-related activity,
\end{enumerate}
the decision to terminate that entitlement may be revised.

% Reg 3(5J) inserted by SI 2015/339 reg 7(2)
(5J) A decision by the Secretary of State under section 8 awarding an employment and support allowance may be revised at any time where—
\begin{enumerate}\item[]
($a$) it is made immediately following the last day of a period for which the claimant was treated as capable of work or as not having limited capability for work under regulation 55ZA of the Jobseeker’s Allowance Regulations or regulation 46A of the Jobseeker’s Allowance Regulations 2013 (extended period of sickness) and that period lasted 13 weeks; and

($b$) it is not a decision which embodies a determination that the claimant is treated as having limited capability for work under regulation 30 of the Employment and Support Allowance Regulations (conditions for treating a claimant as having limited capability for work until a determination about limited capability for work has been made).
\end{enumerate}

%(6) A decision of the Secretary of State under section 8 or 10 that a jobseeker’s allowance is not payable to a claimant for any period in accordance with 
%regulation 27A of the Jobseeker’s Allowance Regulations or  % Words inserted by SI 2010/509 reg 3(2)
%section 19 
%or 20A\footnote{Section 20A was inserted by paragraph 13 of Schedule 7 to the Welfare Reform and Pensions Act 1999 (c.\ 30).}  % Words inserted (19.3.01) by SI 2000/1982 reg 5(a)
%of the Jobseekers Act% 
%, or with regulations made under section 17A of that Act  % Words inserted by SI 2011/688 reg 18(a)
%may be revised at any time by the Secretary of State.

% Reg 3(6) substituted by SI 2012/2568 reg 6(2)
(6) A decision of the Secretary of State under section 8 or 10 that a jobseeker’s allowance is reduced in accordance with section 19 or 19A of the Jobseeker’s Act 
%, or with regulations made under section 17A of that Act  % Words inserted by SI 2010/1222 reg 20(a) (expires 21.11.13)
or regulation 69B of the Jobseeker’s Allowance Regulations may be revised at any time by the Secretary of State.

% Reg 3(6A) inserted (3.4.00) by SI 2000/897 Sch 6 para 3(a)
(6A) A relevant decision within the meaning of section 2B(2) 
or (2A)  % Words inserted (12.4.04) by SI 2003/1886 reg 15(3)
of the Administration Act\footnote{\frenchspacing Section 2B was inserted by Section 57 of the Welfare Reform Act 1999 (c. 30).} may be revised at any time if it contains an error.

% Reg 3(6B) inserted by SI 2012/2575 reg 4
(6B) A decision of the Secretary of State under section 8 or 10 awarding a jobseeker’s allowance may be revised where the Secretary of State makes a decision under regulation 69B (the period of a reduction under section~19B: claimants ceasing to be available for employment etc.)\ of the Jobseeker’s Allowance Regulations\footnote{S.I.~1996/207. Regulations 69B and 70 were inserted by S.I.~2012/2568.} (“the JSA Regulations”) that the amount of the award is to be reduced in accordance with regulations 69B and 70 of the JSA Regulations.

%(7) A decision under section 8 or 10 may be revised where—
%\begin{enumerate}\item[]
%($a$) the Secretary of State 
%or the Board or an officer of the Board  % Words inserted (5.10.99) by SI 1999/2570 reg 6(2)
%awards entitlement to a relevant benefit; and
%
%($b$) on the date that entitlement arises, the claimant or a member of his family is entitled to another relevant benefit or to an increase in the rate of another benefit.
%\end{enumerate}

% Reg 3(7) substituted (19.6.00) by SI 2000/1596 reg 15
%(7) A decision under section 8 or 10 may be revised where—
%\begin{enumerate}\item[]
%($a$) the Secretary of State, appeal tribunal or Commissioner has awarded entitlement to a relevant benefit; and
%
%($b$) on the date that entitlement arises, the claimant or a member of his family becomes entitled to, and is paid, another relevant benefit or an increase in the rate of another relevant benefit.
%\end{enumerate}

% Reg 3(7) substituted (2.4.02) by SI 2002/428 reg 4(2)
(7) Where—
\begin{enumerate}\item[]
($a$) the Secretary of State or an officer of the Board makes a decision under section 8 or 10 awarding a relevant benefit to a claimant (“the original award”); and

($b$) an award of another relevant benefit or of an increase in the rate of another relevant benefit is made to the claimant or a member of his family for a period which includes the date on which the original award took effect,
\end{enumerate}
the Secretary of State or an officer of the Board, as the case may require, may revise the original award.

% Reg 3(7ZA) inserted (18.3.05) by SI 2005/337 reg 2(2)(a)
(7ZA) Where—
\begin{enumerate}\item[]
($a$) the Secretary of State makes a decision under section 8 or 10 awarding income support% 
%or state pension credit 
, 
income-based jobseeker's allowance,  % Words inserted by SI 2009/1490 reg 3(2)(b)(i)
state pension credit or an income-related employment and support allowance  % Words substituted (27.7.08) by SI 2008/1554 reg 31(5)(a)
to a claimant (“the original award”);

($b$) the claimant has a non-dependant within the meaning of regulation~3 of the Income Support Regulations% 
, regulation 2 of the Jobseeker’s Allowance Regulations  % Words inserted by SI 2009/1490 reg 3(2)(b)(ii)
or a person residing with him within the meaning of paragraph 1(1)($a$)(ii), ($b$)(ii)  or ($c$)(iii)  of Schedule I to the State Pension Credit Regulations 
or regulation 71 of the Employment and Support Allowance Regulations  % Words inserted (27.7.08) by SI 2008/1554 reg 31(5)(b)
(“the non-dependant”);

($c$) but for the non-dependant—
\begin{enumerate}\item[]
(i) a severe disability premium would be applicable to the claimant under regulation 17(1)($d$)  of the Income Support Regulations%
, regulation 83($e$) or 86A($c$) of the Jobseeker’s Allowance Regulations  % Words inserted by SI 2009/1490 reg 3(2)(b)(iii)
or regulation 67 of the Employment and Support Allowance Regulations%  % Words inserted (27.7.08) by SI 2008/1554 reg 31(5)(c)
; or

(ii)  an additional amount would be applicable to the claimant as a severe disabled person under regulation 6(4) of the State Pension Credit Regulations; and
\end{enumerate}

($d$) after the original award the non-dependant is awarded benefit which—
\begin{enumerate}\item[]
(i) is for a period which includes the date on which the original award took effect; and

(ii)  is such that a severe disability premium becomes applicable to the claimant under paragraph 13(3)($a$)  of Schedule 2 to the Income Support Regulations%
, paragraph 15(4)($a$) or 20I(3)($a$) of Schedule~1 to the Jobseeker’s Allowance Regulations%  % Words inserted by SI 2009/1490 reg 3(2)(b)(iv)
, paragraph~6(4)($a$)  of Schedule 4 to the Employment and Support Allowance Regulations  % Words inserted (27.7.08) by SI 2008/1554 reg 31(5)(d) 
or an additional amount for severe disability becomes applicable to him under paragraph 2(2)($a$)  of Schedule I to the State Pension Credit Regulations,
\end{enumerate}
\end{enumerate}
the Secretary of State may revise the original award.

% Reg 3(7A) inserted (20.5.02) by SI 2002/1379 reg 3(d)
(7A) Where a decision as to a claimant’s entitlement to a disablement pension under section 103 of the Contributions and Benefits Act is revised by the Secretary of State, or changed on appeal, a decision of the Secretary of State as to the claimant’s entitlement to reduced earnings allowance under paragraph 11 or 12 of Schedule 7 to that Act may be revised at any time provided that the revised decision is more advantageous to the claimant than the original decision.

% Reg 3(7B), (7C) inserted (18.3.05) by SI 2005/337 reg 2(2)(b)
(7B) A decision under regulation 22A\footnote{Regulation 22A was inserted by S.I. 1996/206 and was amended by S.I. 1999/2422 and 3109, 2000/590 and 2001/3767.} of the Income Support Regulations (reduction in applicable amount where the claimant is appealing against a decision which embodies a determination that he is not incapable of work) may be revised if the appeal is successful
or lapses%  % Words added (10.4.06) by SI 2006/832 reg 5(2)(a)
.

(7C) Where a person’s entitlement to income support is terminated because of a determination that he is not incapable of work and 
the decision which embodies that determination is revised or  % Words added (10.4.06) by SI 2006/832 reg 5(2)(b)(i)
he subsequently appeals the decision 
%that embodies 
which embodies  % Words substituted (10.4.06) by SI 2006/832 reg 5(2)(b)(ii)
that determination and is entitled to income support under regulation 22A of the Income Support Regulations, the decision to terminate entitlement may be revised.

% Reg 3(7CC) inserted by SI 2009/1490 reg 3(2)(c)
(7CC)  Where—
\begin{enumerate}\item[]
($a$) a person’s entitlement to income support is terminated because of a determination that the person is not incapable of work;

($b$) the person subsequently claims and is awarded jobseeker’s allowance; and

($c$) the decision which embodies the determination that the person is not incapable of work is revised or successfully appealed,
\end{enumerate}
the Secretary of State may revise the decisions to terminate income support entitlement and to award jobseeker’s allowance.

% Reg 3(7CD) inserted by SI 2014/1097 reg 12(3)
(7CD) A decision of the Secretary of State under section 10 of the Act made in consequence of a determination under regulation 6(2) of the Income Support Work-Related Activity Regulations that a claimant has% 
, without showing good cause,  % Words inserted by SI 2017/1015 reg 9
failed to satisfy a requirement of regulation 2 of those Regulations (requirement to undertake work-related activity) may be revised at any time if it contained an error to which the claimant did not materially contribute.

% Reg 3(7D), (7E) inserted (6.4.06) by SI 2005/2677 reg 9(3)
(7D) Where—
\begin{enumerate}\item[]
($a$) a person elects for an increase of—
\begin{enumerate}\item[]
(i) a Category A or Category B retirement pension in accordance with paragraph A1 or 3C of Schedule 5 to the Contributions and Benefits Act\footnote{Paragraphs A1 and 3C are inserted respectively by paragraphs 4 and 9 of Schedule 11 to the Pensions Act 2004 (c. 35).} (pension increase or lump sum where entitlement to retirement pension is deferred);

(ii) a shared additional pension in accordance with paragraph 1 of Schedule 5A to that Act\footnote{Schedule 5A is inserted by paragraph 15 of Schedule 11 to the Pensions Act 2004.} (pension increase or lump sum where entitlement to shared additional pension is deferred); or, as the case may be,

(iii) graduated retirement benefit in accordance with paragraph 12 or 17 of Schedule 1 to the Graduated Retirement Benefit Regulations (further provisions replacing section 36(4) of the National Insurance Act 1965: increases of graduated retirement benefit and lump sums);\end{enumerate}

($b$) the Secretary of State decides that the person or his partner is entitled to state pension credit and takes into account the increase of pension or benefit in making or superseding that decision; and

($c$) the person’s election for an increase is subsequently changed in favour of a lump sum in accordance with regulation 5 of the Deferral of Retirement Pensions etc.\ Regulations or, as the case may be, paragraph~20D of Schedule 1 to the Graduated Retirement Benefit Regulations\footnote{Paragraph 20D is inserted by S.I. 2005/2677.},
\end{enumerate}
the Secretary of State may revise the state pension credit decision.

% Reg 3(7DA), (7DB) inserted by SI 2015/1985 reg 18(3)
(7DA) The Secretary of State may revise the state pension credit decision where—
\begin{enumerate}\item[]
($a$) a person chooses under—
\begin{enumerate}\item[]
(i) section 8(2) of the Pensions Act 2014 (choice of lump sum or survivor’s pension under section 9 in certain cases) to be paid a state pension under section 9 of that Act (survivor’s pension based on inheritance of deferred old state pension); or

(ii) regulations made under section 10 of the Pensions Act 2014 (inheritance of graduated retirement benefit) which make provision corresponding or similar to section 8(2) to be paid a state pension under regulations made under section 10 which make provision corresponding or similar to section 9 of that Act;
\end{enumerate}

($b$) the Secretary of State—
\begin{enumerate}\item[]
(i) decides that the person or their partner is entitled to state pension credit; and

(ii) takes into account the state pension mentioned in sub-paragraph ($a$) in making or superseding that decision; and
\end{enumerate}

\begin{sloppypar}
($c$) the person’s choice for a state pension mentioned in sub-paragraph~($a$) is subsequently altered in favour of a lump sum in accordance with—
\end{sloppypar}
\begin{enumerate}\item[]
(i) regulation 6 of the State Pension Regulations 2015 (changing a choice of lump sum or survivor’s pension); or

(ii) regulations made under section 10 of the Pensions Act 2014 which make provision corresponding or similar to regulation 6 of the State Pension Regulations 2015.
\end{enumerate}
\end{enumerate}

(7DB) The Secretary of State may revise an award of a state pension under Part I of the Pensions Act 2014 where—
\begin{enumerate}\item[]
($a$) the person makes a choice under—
\begin{enumerate}\item[]
(i) section 8(2) of the Pensions Act 2014; or

(ii) regulations under section 10 of that Act which make provision corresponding or similar to section 8(2); and
\end{enumerate}

($b$) the person subsequently alters their choice in accordance with—
\begin{enumerate}\item[]
(i) regulation 6 of the State Pension Regulations 2015; or

(ii) regulations under section 10 of the Pensions Act 2014 which make provision corresponding or similar to regulation 6 of the State Pension Regulations 2015.
\end{enumerate}
\end{enumerate}

(7E) Where—
\begin{enumerate}\item[]
($a$) a person is awarded a Category A or Category B retirement pension, shared additional pension or, as the case may be, graduated retirement benefit;

($b$) an election is made, or treated as made, in respect of the award in accordance with paragraph A1 or 3C of Schedule 5 or paragraph 1 of Schedule 5A to the Contributions and Benefits Act or, as the case may be, in accordance with paragraph 12 or 17 of Schedule 1 to the Graduated Retirement Benefit Regulations; and

($c$) the election is subsequently changed in accordance with regulation 5 of the Deferral of Retirement Pensions etc.\ Regulations or, as the case may be, paragraph 20D of Schedule 1 to the Graduated Retirement Benefit Regulations,
\end{enumerate}
the Secretary of State may revise the award.

% Reg 3(7EA), (7EB) inserted by SI 2012/824 reg 4(2)
(7EA) The Secretary of State may revise a decision made under regulation~18(1) that a person ceases to be entitled to a benefit specified in paragraph~(7EB).

(7EB) Those benefits are—
\begin{enumerate}\item[]
($a$) a Category A or Category B retirement pension;

($b$) a shared additional pension;

($c$) graduated retirement benefit;

% Reg 3(7EB)(d) inserted by SI 2015/1985 reg 18(4)
($d$) a state pension under Part I of the Pensions Act 2014.
\end{enumerate}

% Reg 3(7F) inserted (10.4.06) by SI 2005/832 reg 5(2)(c)
(7F) A decision under regulation 17(1)($d$)  of the Income Support Regulations that a person is no longer entitled to a disability premium because of a determination that he is not incapable of work may be revised where the decision which embodies that determination is revised or his appeal against the decision is successful.

(8) A decision of the Secretary of State 
or the Board or an officer of the Board  % Words inserted (5.10.99) by SI 1999/2570 reg 6(2)
which is specified in Schedule 2 to the Act or is prescribed in regulation~27 (decisions against which no appeal lies) may be revised at any time.

% Reg 3(8A) inserted (15.10.01) by SI 2001/1711 reg 2(2)(b), omitted by SI 2010/424 reg 4(3)
%(8A) Where a court makes a determination which results in a restriction being imposed pursuant to section 62 or 63 of the Child Support, Pensions and Social Security Act 2000 (loss of benefit provisions) and that determination is quashed or set aside by that or any other court, a decision of the Secretary of State under section 8(1)($a$)  or 10 made in accordance with regulation 6(2)($i$) may be revised at any time.

% Reg 3(8B) inserted (1.4.02) by SI 2002/490 reg 8(a)
%(8B) Where a court convicts a person of an offence, that conviction results in a restriction being imposed under section 7, 8 or 9 of the Social Security Fraud Act 2001 (loss of benefit provisions) and that conviction is quashed or set aside by that or any other court, a decision of the Secretary of State under section 8(1)($a$)  or 10 made in accordance with regulation 6(2)($j$)  or ($k$)  may be revised at any time.

% Reg 3(8B) substituted by SI 2010/1160 reg 3(2)
(8B) Where—
\begin{enumerate}\item[]
($a$) a restriction is imposed on a person under section 6B, 7, 8 or 9 of the Social Security Fraud Act 2001 (loss of benefit provisions) as result of the person—
\begin{enumerate}\item[]
(i) being convicted of an offence by a court; or

(ii) agreeing to pay a penalty as an alternative to prosecution under section 115A of the Administration Act or section 109A of the Social Security Administration (Northern Ireland) Act 1992, and
\end{enumerate}

($b$) that conviction is quashed or set aside by that or any other court, or the person withdraws his agreement to pay a penalty,
\end{enumerate}
a decision of the Secretary of State made under section 8(1)($a$) or made under section 10 in accordance with regulation 6(2)($j$) or ($k$) may be revised at any time.

% Reg 3(8C) inserted by SI 2008/2667 reg 3(2)
(8C) A decision made under section 8 or 10 (“the original decision”) may be revised at any time—
\begin{enumerate}\item[]
($a$) where, on or after the date of the original decision—
\begin{enumerate}\item[]
(i) a late paid contribution is treated as paid under regulation 5\footnote{Regulation 5 was amended by S.I.~2002/2366 and is amended by 2008/1554 with effect from 27 October 2008.} of the Social Security (Crediting and Treatment of Contributions and National Insurance Numbers) Regulations 2001\footnote{S.I.~2001/769.} (treatment of late paid contributions where no consent, connivance or negligence by the primary contributor) on a date which falls on or before the date on which the original decision was made;

(ii) a direction is given under regulation 6\footnote{Regulation 6 was amended by S.I.~2002/2366.} of those Regulations (treatment of contributions paid late through ignorance or error) that a late contribution shall be treated as paid on a date which falls on or before the date on which the original decision was made; or

(iii) an unpaid contribution is treated as paid under regulation~60\footnote{Regulation 60 was amended by S.I.~2002/2366 and 2007/1056.} of the Social Security (Contributions) Regulations 2001\footnote{S.I.~2001/1004.} (treatment of unpaid contributions where no consent, connivance or negligence by the primary contributor) on a date which falls on or before the date on which the original decision was made; and
\end{enumerate}

($b$) where any of paragraphs (i), (ii) or (iii) apply, either an award of benefit would have been made or the amount of benefit awarded would have been different.
\end{enumerate}

% Reg 3(8D) inserted by SI 2009/659 reg 2
(8D) A decision made under section 8 or 10 may be revised at any time where, by virtue of regulation 6C (treatment of Class 3 contributions paid under section 13A of the Act) of the Social Security (Crediting and Treatment of Contributions, and National Insurance Numbers) Regulations 2001, a contribution is treated as paid on a date which falls on or before the date on which the decision was made.

% Reg 3(8E)--(8K) inserted by SI 2016/1145 reg 4(3)(a)
(8E) A decision in relation to a claim for a contribution-based jobseeker’s allowance or a contributory employment and support allowance may be revised at any time where—
\begin{enumerate}\item[]
($a$) on or after the date of the decision a contribution is treated as paid as set out in regulation 7A of the Social Security (Crediting and Treatment of Contributions, and National Insurance Numbers) Regulations 2001 (treatment of Class 2 contributions paid on or before the due date); and

($b$) by virtue of the contribution being so treated, the person satisfies the contribution conditions of entitlement listed in column 2 of the table in paragraph (8G) in relation to a contribution-based jobseeker’s allowance or a contributory employment and support allowance.
\end{enumerate}

(8F) A decision to award a benefit listed in column 1 of the table in paragraph (8G) may be revised at any time where, on or after the date of the decision—
\begin{enumerate}\item[]
($a$) any of the circumstances set out in paragraph (8H) occur; and

($b$) by virtue of the circumstance occurring, the person ceases to satisfy the contribution conditions of entitlement listed in the corresponding entry in column 2 of that table.
\end{enumerate}

(8G) The table referred to in paragraphs (8E) and (8F) is as follows—

\noindent
%\begin{tabulary}{\linewidth}{JJ}
\begin{longtable}{p{149.14655pt}p{216.85005pt}}
\hline
\itshape 1. Benefit	&\itshape 2. Contribution conditions of entitlement\\
\hline
\endhead
\hline
\endlastfoot
Contribution-based jobseeker’s allowance	&the conditions set out in section 2(1)($a$) and ($b$) of the Jobseekers Act\\
Contributory employment and support allowance	&the first and second conditions set out in paragraphs 1(1) and 2(1) of Schedule 1 to the Welfare Reform Act\\
Bereavement allowance	&the contribution conditions set out in paragraph 5(2) and (3) of Schedule 3 to the Contributions and Benefits Act\\
Widowed parent’s allowance	&the contribution conditions set out in paragraph 5(2) and (3) of Schedule 3 to the Contributions and Benefits Act\\
Bereavement payment	&the contribution condition specified in paragraph 4(1) of Schedule 3 to the Contributions and Benefits Act\\
Category A or Category B retirement pension under Part II of the Contributions and Benefits Act	&the contribution conditions set out in paragraph 5(2) and (3) or, as the case may be, 5A(2) of Schedule 3 to the Contributions and Benefits Act\\
State pension under Part I of the Pensions Act 2014	&the conditions of entitlement to a state pension in section 2(1)($b$) or, as the case may be, 2(2)($b$) or 4(1)($b$) and ($c$) of the Pensions Act 2014\\
%\end{tabulary}
\end{longtable}

(8H) The circumstances are—
\begin{enumerate}\item[]
($a$) a Class 2 contribution is repaid to a person in consequence of an amendment or correction of the person’s relevant profits under section 9ZA or 9ZB of the Taxes Management Act 1970 (amendment or correction of return by taxpayer or officer of the Board); or

($b$) a Class 2 contribution is returned to a person under regulation 52 of the Social Security (Contributions) Regulations 2001 (contributions paid in error); or

($c$) a Class 1 or Class 2 contribution paid by a person to Her Majesty’s Revenue and Customs under section 223 of the Finance Act 2014 (accelerated payment in respect of notice given while tax enquiry is in progress) is repaid to the person.
\end{enumerate}

(8I) A decision to award a benefit specified in paragraph (8K) may be revised at any time where, on or after the date of the decision—
\begin{enumerate}\item[]
($a$) any of the circumstances set out in paragraph (8H) occur; and

($b$) by virtue of the circumstances occurring, the decision was more advantageous to the claimant than it would otherwise have been.
\end{enumerate}

(8J) A decision to award a benefit specified in paragraph (8K), or a decision that that benefit is not payable, may be revised at any time where, on or after the date of the decision, a contribution is treated as paid by the relevant day by virtue of regulation 7(1) of the Social Security (Crediting and Treatment of Contributions, and National Insurance Numbers) Regulations 2001 (treatment for the purpose of any contributory benefit of contributions paid under certain provisions relating to the payment and collection of contributions).

(8K) The benefits specified in this paragraph are—
\begin{enumerate}\item[]
($a$) a bereavement benefit;

($b$) a Category A or Category B retirement pension under Part II of the Contributions and Benefits Act;

($c$) a state pension under Part I of the Pensions Act 2014.
\end{enumerate}

%(9) Paragraph (1) shall not apply in respect of a relevant change of circumstances which occurred since the decision was made or where the Secretary of State 
%or the Board or an officer of the Board  % Words inserted (5.10.99) by SI 1999/2570 reg 6(2)
%has evidence or information which indicates that a relevant change of circumstances will occur.

% Reg 3(9) substituted (18.10.99) by SI 1999/2677 reg 6(b)
(9) Paragraph (1) shall not apply in respect of—
\begin{enumerate}\item[]
($a$) a relevant change of circumstances which occurred since the decision 
%was made 
had effect  % Words substituted (5.5.03) by SI 2003/1050 reg 3(2)
or, in the case of an advance award under regulation 13\footnote{Regulation 13 was amended by S.I. 1991/2284 and 2741, 1992/247, 1994/2319, 1999/2422, 2572 and 3178 and 2002/3019.}, 13A or 13C\footnote{Regulations 13A and 13C were inserted by S.I. 1991/2741 and amended by S.I. 1999/2860 and 3178.} of the Claims and Payments Regulations, since the decision was made,  % Words inserted (18.3.05) by SI 2005/337 reg 2(2)(c)
or where the Secretary of State has evidence or information which indicates that a relevant change of circumstances will occur; 
%nor  % Word omitted (27.7.08) by SI 2008/1554 reg 31(6)(a)

($b$) a decision which relates to an attendance allowance or a disability living allowance where the person is terminally ill, within the meaning of section 66(2)($a$) of the Contributions and Benefit Act, unless an application for revision which contains an express statement that the person is terminally ill is made either by—
\begin{enumerate}\item[]
(i) the person himself; or

(ii) any other person purporting to act on his behalf whether or not that other person is acting with his knowledge or authority,
\end{enumerate}
but where such an application is received a decision may be so revised notwithstanding that no claim under section 66(1) or, as the case may be, 72(5) or 73(12) of that Act has been made;
%
% Reg 3(9)(c) inserted (27.7.08) by SI 2008/1554 reg 31(6)(b)
nor

($c$) a decision which relates to an employment and support allowance where the claimant is terminally ill, within the meaning of regulation~2(1) of the Employment and Support Allowance Regulations unless the claimant makes an application which contains an express statement that he is terminally ill and where such an application is made, the decision may be revised.
\end{enumerate}

(10) The Secretary of State 
or the Board  % Words inserted (5.10.99) by SI 1999/2570 reg 6(5)
may treat an application for a supersession as an application for a revision.

(11) In this regulation and regulation 7, “appropriate office” means---
\begin{enumerate}\item[]
($a$) the office of the 
%Department of Social Security or the Department for Education and Employment 
Department for Work and Pensions  % Words substituted (20.5.02) by SI 2002/1379 reg 3(e)(i)
the address of which is indicated on the notification of the original decision; or

($b$) in the case of a person who has claimed jobseeker’s allowance, the office specified by the Secretary of State in accordance with regulation~23 of the Jobseeker’s Allowance Regulations%
% Reg 3(11)(c), (d) inserted (5.7.99) by SI 1999/1662 reg 3(3)(b)
; or

    ($c$) 
    in the case of a contributions decision which falls within Part II of Schedule 3 to the Act\footnote{\frenchspacing Schedule 3 is amended by paragraph 36 of Schedule 7 to the Social Security Contributions (Transfer of Functions, etc.) Act 1999 (c. 2) (“the Act”).}, any National Insurance Contributions office of the Board or any office of the 
%Department of Social Security
Department for Work and Pensions%  % Words substituted (20.5.02) by SI 2002/1379 reg 3(e)(ii)
; or

    ($d$) 
    in the case of a decision made under the Pension Schemes Act 1993 by virtue of section 170(2) of that Act\footnote{1993 c. 48; section 170 has been amended by paragraph 42 of Schedule 3, paragraph 70 of Schedule 5, and Part III of Schedule 7 to the Pensions Act 1995 (c. 26). A new section 170 is substituted by paragraph 131 of Schedule 7 to the Social Security Act 1998 (c. 14) and amended by section 16(2) of the Act.}, any National Insurance Contributions office of the Board;
% Reg 3(11)(e) added (5.10.99) by SI 1999/2570 reg 6(6)
or

($e$) in the case of a person who has claimed working families' tax credit or disabled person’s tax credit, a Tax Credits Office, the address of which is indicated on the notification of the original decision;
% Reg 3(11)(f) added (3.4.00) by SI 2000/897 Sch 6 para 3(b)
or

%($f$) in the case of a relevant person within the meaning of regulation 2(2) of the Work-focused Interviews Regulations, an office of any designated authority which displays the \textsc{\lowercase{ONE}} logo.

% Reg 3(11)(f) substituted (30.9.02) by SI 2002/1703 Sch 2 para 6(b)
\looseness=-1
($f$) in the case of a person who is, or would be, required to take part in a work-focused interview, an office of the Department for Work and Pensions which is designated by the Secretary of State as a Jobcentre Plus Office or an office of a designated authority which displays the \textsc{\lowercase{ONE}} logo.
\end{enumerate}

% Reg 3(12) inserted by SI 2016/1145 reg 4(3)(b)
(12) In this regulation—
\begin{enumerate}\item[]
“relevant day” has the meaning given in regulation 7(3)($b$) of the Social Security (Crediting and Treatment of Contributions, and National Insurance Numbers) Regulations 2001;

“relevant profits” has the meaning given in section 11(3) of the Contributions and Benefits Act.
\end{enumerate}

\amendment{
Words inserted in reg. 3(5)(b) and reg. 3(5)(c) inserted (5.7.99) by the Social Security and Child Support (Decisions and Appeals) Amendment (No. 2) Regulations 1999 reg. 2.

Words substituted in reg. 3(4) and reg. 3(11)(c), (d) inserted (5.7.99) by the Social Security Contributions (Transfer of Functions, etc.) Act 1999 (Commencement No. 2 and Consequential and Transitional Provisions) Order 1999 art. 3(3) (subject to transitional provisions in art. 4).

Words inserted in reg. 3(1), (2), (7)--(9), (10), words substituted in reg. 3(5), reg. 3(11)(e) inserted and reg. 3(5)(b) substituted (5.10.99 for certain purposes only, see reg. 1(2)) by the Tax Credits (Decisions and Appeals) (Amendment) Regulations 1999 reg. 6.

Reg. 3(1)(a), (9) substituted (18.10.99) by the Social Security and Child Support (Decisions and Appeals), Vaccine Damage Payments and Jobseeker's Allowance (Amendment) Regulations 1999 reg. 6.

Reg. 3(6A), (11)(f) inserted (3.4.00) by the Social Security (Work-focused Interviews) Regulations 2000 Sch. 6 para. 3.

Reg. 3(7) substituted (19.6.00) by the Social Security and Child Support (Miscellaneous Amendments) Regulations 2000 reg. 15.

Words inserted in reg. 3(6) (19.3.01) by the Social Security (Joint Claims: Consequential Amendments) Regulations 2000 reg. 5(a).

Reg. 3(8A) inserted (15.10.01) by the Social Security (Breach of Community Order) (Consequential Amendments) Regulations 2001 reg. 2(2)(b).

Reg. 3(8B) inserted (1.4.02) by the Social Security (Loss of Benefit) (Consequential Amendments) Regulations 2002 reg. 8(a).

Reg. 3(7) substituted (2.4.02) by the Social Security (Claims and Payments and Miscellaneous Amendments) Regulations 2002 reg. 4(2).

Words substituted in reg. 3(11)(a), (c), reg. 3(4A), (5A), (7A) inserted and reg. 3(1)(a), (b) substituted (20.5.02) by the Social Security and Child Support (Decisions and Appeals) (Miscellaneous Amendments) Regulations 2002 reg. 3.

Reg. 3(11)(f) substituted (30.9.02) by the Social Security (Jobcentre Plus Interviews) Regulations 2002 Sch. 2 para. 6(b).

Words substituted in reg. 3(9)(a) (5.5.03) by the Social Security and Child Support (Miscellaneous Amendments) Regulations 2003 reg. 3(2).

Words inserted in reg. 3(6A) (12.4.04) by the Social Security (Jobcentre Plus Interviews for Partners) Regulations 2003 reg. 15(3).

Words inserted in reg. 3(9)(a) and reg. 3(7ZA), (7B), (7C) inserted (18.3.05) by the Social Security, Child Support and Tax Credits (Miscellaneous Amendments) Regulations 2005 reg. 2(2).

Reg. 3(7D), (7E) inserted (6.4.06) by the Social Security (Deferral of Retirement Pensions, Shared Additional Pension and Graduated Retirement Benefit) (Miscellaneous Provisions) Regulations 2005 reg. 9(3).

Words inserted in reg. 3(7B), (7C), words substituted in reg. 3(7C) and reg. 3(7F) inserted (10.4.06) by the Social Security (Miscellaneous Amendments (No. 2) Regulations 2006 reg. 5(2).

Words inserted in reg. 3(5)(c), words substituted in reg. 3(5)(b) and reg. 3(5)(d), (5B) inserted (24.9.07) by the Social Security (Miscellaneous Amendments) (No. 4) Regulations 2007 reg. 3(2)--(5).

Words inserted in reg. 3(5)(a) and reg. 3(5ZA)--(5ZC) inserted (1.10.07) by the Social Security (National Insurance Credits) Amendment Regulations reg. 3.

Words inserted in reg. 3(5)(c), (d), (7ZA)(b), (c)(i), (d)(ii) words substituted in reg. 3(5)(c), (7ZA)(a) and reg. 3(5ZB)(h), (5C), (5D), (9)(c) added (27.7.08) by the Employment and Support Allowance (Consequential Provisions) (No. 2) Regulations 2008 reg. 31.

Reg. 3(8C) inserted (30.10.08) by the Social Security (Miscellaneous Amendments) (No. 5) Regulations 2008 reg. 3(2).

Words substituted in reg. 3(4A), (5A)(b) (3.11.08) by the Tribunals, Courts and Enforcement Act 2007 (Transitional and Consequential Provisions) Order 2008 Sch. 1 para. 98.

Reg. 3(8D) inserted (6.4.09) by the Social Security (Additional Class 3 National Insurance Contributions) Amendment Regulations 2009 reg. 2.

Reg. 3(5D) omitted, words inserted in reg. 3(7ZA) and reg. 3(7CC) inserted (13.7.09) by the Social Security (Miscellaneous Amendments) (No. 2) Regulations 2009 reg. 3(2).

Reg. 3(8A) omitted (22.3.10) by the Welfare Reform Act 2009 (Section 26) (Consequential Amendments) Regulations 2010 reg. 4(3).

Reg. 3(8B) substituted (1.4.10) by the Social Security (Loss of Benefit) Amendment Regulations 2010 reg. 3(2).

Words inserted in reg. 3(6) (6.4.10) by the Jobseeker’s Allowance (Sanctions for Failure to Attend) Regulations 2010 reg. 3(2).

Reg. 3(5E), (5F) inserted (28.6.10) by the Social Security (Miscellaneous Amendments) (No. 3) Regulations 2010 reg. 7(2).

Words inserted in reg. 3(6) (22.11.10--21.11.13) by the Jobseeker’s Allowance (Work for Your Benefit Pilot Scheme) Regulations 2010 reg. 20(a).

Words inserted in reg. 3(6) (25.4.11) by the Jobseeker’s Allowance (Mandatory Work Activity Scheme) Regulations 2011 reg. 18(a).

Reg. 3(5D), (5G), (5H) inserted (31.10.11) by the Social Security (Miscellaneous Amendments) (No. 3) Regulations 2011 reg. 12.

Reg. 3(7EA), (7EB) inserted (17.4.12) by the Social Security (Suspension of Payment of Benefits and Miscellaneous Amendments) Regulations 2012 reg. 4(2).

Reg. 3(5I) inserted (1.5.12) by the Employment and Support Allowance (Duration of Contributory Allowance) (Consequential Amendments) Regulations 2012 reg. 5.

Words omitted in reg. 3(5F)(a) (1.5.12) by the Employment and Support Allowance (Amendment of Linking Rules) Regulations 2012 reg. 2.

Reg. 3(6) substituted (22.10.12) by the Jobseeker’s Allowance (Sanctions) (Amendment) Regulations 2012 reg. 6(2).

Reg. 3(6B) inserted (5.11.12) by the Social Security (Miscellaneous Amendments) (No.~2) Regulations 2012 reg. 4.

Words inserted in reg.~3(1)(b) (28.10.13) by the Social Security, Child Support, Vaccine Damage and Other Payments (Decisions and Appeals) (Amendment) Regulations 2013 reg.~4(2).

Reg. 3(7CD) inserted (28.4.14) by the Income Support (Work-Related Activity) and Miscellaneous Amendments Regulations 2014 reg. 12(3).

Reg. 3(5J) inserted (30.3.15) by the Jobseeker’s Allowance (Extended Period of Sickness) Amendment Regulations 2015 reg. 7(2).

Reg. 3(7DA), (7DB), (7EB)(d) inserted (6.4.16) by the Pensions Act 2014 (Consequential, Supplementary and Incidental Amendments) Order 2015 art.~18(3), (4).

Reg. 3(8E)--(8K), (12) inserted (1.1.17) by the Social Security (Credits, and Crediting and Treatment of Contributions) (Consequential and Miscellaneous Amendments) Regulations 2016 reg. 4(3).

Reg. 3(7CD) amended (16.11.17) by the Social Security (Miscellaneous Amendments No.~4) Regulations 2017 reg. 9.
}

% Reg 3ZA inserted (28.10.13) by SI 2013/2380 reg 4(3)
\subsubsection[3ZA. Consideration of revision before appeal]{Consideration of revision before appeal}

3ZA.---(1)  This regulation applies in a case where—
\begin{enumerate}\item[]
($a$) the Secretary of State gives a person written notice of a decision under section 8 or 10 of the Act (whether as originally made or as revised under section 9 of that Act); and

($b$) that notice includes a statement to the effect that there is a right of appeal in relation to the decision only if the Secretary of State has considered an application for a revision of the decision.
\end{enumerate}

(2) In a case to which this regulation applies, a person has a right of appeal under section 12(2) of the Act in relation to the decision only if the Secretary of State has considered on an application whether to revise the decision under section 9 of the Act.

(3) The notice referred to in paragraph (1) must inform the person—
\begin{enumerate}\item[]
($a$) of the time limit specified in regulation 3(1) or (3) for making an application for a revision; and

($b$) that, where the notice does not include a statement of the reasons for the decision (“written reasons”), he may, within one month of the date of notification of the decision, request that the Secretary of State provide him with written reasons.
\end{enumerate}

(4) Where written reasons are requested under paragraph (3)($b$), the Secretary of State must provide them within 14 days of receipt of the request or as soon as practicable afterwards.

(5) Where, as the result of paragraph (2), there is no right of appeal against a decision, the Secretary of State may treat any purported appeal as an application for a revision under section 9 of the Act.

\amendment{
Reg. 3ZA inserted (28.10.13) by the Social Security, Child Support, Vaccine Damage and Other Payments (Decisions and Appeals) (Amendment) Regulations 2013 reg.~4(3).
}

% Reg 3A inserted (3.3.03 for new-rules cases only) by SI 2000/3185 reg 5
\subsubsection[3A. Revision of child support decisions]{Revision of child support decisions\\*\emph{2003 scheme only}}

3A.---(1)  Subject to paragraph (2), any decision as defined in paragraph (3) may be revised under section 16 of the Child Support Act by the 
%Secretary of State
%Commission%  % Words substituted (6.4.09) by SI 2009/396 reg 4(2)
Secretary of State%  % Words substituted (1.8.12) by SI 2012/2007 Sch para 113(3)(a)(i)
—
\begin{enumerate}\item[]
%($a$) if 
%%he 
%%it  % Word substituted (6.4.09) by SI 2009/396 reg 4(2)(b)
%the Secretary of State  % Words substituted (1.8.12) by SI 2012/2007 Sch para 113(3)(a)(ii)
%receives an application for the revision of a decision either—
%\begin{enumerate}\item[]
%(i) under section 16; or
%
%(ii) by way of an application under section 28G,
%\end{enumerate}
%of the Child Support Act, within one month of the date of notification of the decision or within such longer time as may be allowed under regulation 4;

% Reg 3A(1)(a) substituted by SI 2015/338 reg 6(2)
($a$) if the Secretary of State receives an application for the revision of a decision under either section 16 or section 28G of the Child Support Act—
\begin{enumerate}\item[]
(i) within one month of the date of notification of the decision;

(ii) within one month of the date on which notice of the correction is given under regulation 9B(3) (correction of accidental errors in child support decisions); or

(iii) within such longer time as may be allowed under regulation 4;
\end{enumerate}

($b$) if—
\begin{enumerate}\item[]
(i) 
%he 
%it  % Word substituted (6.4.09) by SI 2009/396 reg 4(2)(b)
the Secretary of State  % Words substituted (1.8.12) by SI 2012/2007 Sch para 113(3)(a)(ii)
notifies the person who applied for a decision to be revised within the period specified in sub-paragraph ($a$), that the application is unsuccessful because the 
%Secretary of State
%Commission  % Words substituted (6.4.09) by SI 2009/396 reg 4(2)
Secretary of State  % Words substituted (1.8.12) by SI 2012/2007 Sch para 113(3)(a)(i)
is not in possession of all of the information or evidence needed to make a decision; and

(ii) that person reapplies for the decision to be revised within one month of the notification described in head (i)  above, or such longer period as the 
%Secretary of State
%Commission  % Words substituted (6.4.09) by SI 2009/396 reg 4(2)
Secretary of State  % Words substituted (1.8.12) by SI 2012/2007 Sch para 113(3)(a)(i)
is satisfied is reasonable in the circumstances of the case, and provides in that application sufficient information or evidence to enable a decision to be made;
\end{enumerate}

($c$) if 
%he 
%it  % Word substituted (6.4.09) by SI 2009/396 reg 4(2)(b)
the Secretary of State  % Words substituted (1.8.12) by SI 2012/2007 Sch para 113(3)(a)(ii)
is satisfied that the decision was erroneous due to a misrepresentation of, or failure to disclose, a material fact and that the decision was more advantageous to the person who misrepresented or failed to disclose that fact than it would have been but for that error;

% Reg 3A(1)(cc) inserted (3.3.03) by SI 2002/1204 reg 2(2)(a)(i)
($cc$) if an appeal is made under section 20 of the Child Support Act against a decision within the time prescribed 
%in regulation 31, or in a case to which regulation 32 applies within the time prescribed in that regulation, 
by Tribunal Procedure Rules  % Words substituted (3.11.08) by SI 2008/2683 Sch 1 para 99(a)
but the appeal has not been determined;

($d$) if 
%he 
%it  % Word substituted (6.4.09) by SI 2009/396 reg 4(2)(b)
the Secretary of State  % Words substituted (1.8.12) by SI 2012/2007 Sch para 113(3)(a)(iii)
commences action leading to the revision of the decision within one month of the date of notification of the decision; or

($e$) if the decision arose from an official error%.
%
% Reg 3A(1)(f) inserted (3.3.03) by SI 2002/1204 reg 2(2)(a)(ii)
; or

    ($f$) 
    if the grounds for revision are that a person with respect to whom a maintenance calculation was made was not, at the time the calculation was made, a parent of a child to whom the calculation relates.
\end{enumerate}

(2) Paragraph (1)($a$)  to ($d$)  shall not apply in respect of a change of circumstances which—
\begin{enumerate}\item[]
($a$) occurred since the date on which the decision had effect; or

($b$) according to information or evidence which the 
%Secretary of State
%Commission  % Words substituted (6.4.09) by SI 2009/396 reg 4(2)
Secretary of State  % Words substituted (1.8.12) by SI 2012/2007 Sch para 113(3)(b)
has, is expected to occur.
\end{enumerate}

%(3) Subject to paragraph (7), in paragraphs (1) and (2) and in regulation 4(3) “decision” means a decision of the Secretary of State under sections 11, 12 or 46 of the Child Support Act, or a determination of an appeal tribunal on a referral under section 28D(1)($b$)  of that Act, or any supersession of a decision under section 17 of that Act.

% Reg 3A(3) substituted (3.3.03) by SI 2002/1204 reg 2(2)(b)
(3) In paragraphs (1), (2) and (5A) and in regulation 4(3) “decision” means a decision of the 
%Secretary of State
%Commission  % Words substituted (6.4.09) by SI 2009/396 reg 4(2)
Secretary of State  % Words substituted (1.8.12) by SI 2012/2007 Sch para 113(3)(b)
under section %
%11, 12 or 46 
11 or 12  % Words substituted (27.10.08) by SI 2008/2543 reg 4(3)(a)
of the Child Support Act, or a determination of 
%an appeal tribunal 
the First-tier Tribunal  % Words substituted (3.11.08) by SI 2008/2683 Sch 1 para 98(b)
on a referral under section 28D(1)($b$)  of that Act, or any supersession of a decision under section~17 of that Act, whether as originally made or as revised under section~16 of that Act.

(4) A decision made under section 12(2) of the Child Support Act may be revised at any time before it is replaced by a decision under section 11 of that Act.

(5) Where the 
%Secretary of State
%Commission  % Words substituted (6.4.09) by SI 2009/396 reg 4(2)
Secretary of State  % Words substituted (1.8.12) by SI 2012/2007 Sch para 113(3)(b)
revises a decision made under section~12(1) of the Child Support Act in accordance with section 16(1B) of that Act, that decision may be revised under section 16 of that Act at any time.

% Reg 3A(5A) inserted (3.3.03) by SI 2002/1204 reg 2(2)(c)
(5A) Where—
\begin{enumerate}\item[]
($a$) the 
%Secretary of State
%Commission  % Words substituted (6.4.09) by SI 2009/396 reg 4(2)
Secretary of State  % Words substituted (1.8.12) by SI 2012/2007 Sch para 113(3)(c)(i)
makes a decision (“decision $\mathcal{A}$”) and there is an appeal;

($b$) there is a further decision in relation to the appellant (“decision~$\mathcal{B}$”) after the appeal but before the appeal results in a decision by 
%an appeal tribunal 
the First-tier Tribunal  % Words substituted (3.11.08) by SI 2008/2683 Sch 1 para 98(b)
(“decision $\mathcal{C}$”); and

($c$) the 
%Secretary of State
%Commission  % Words substituted (6.4.09) by SI 2009/396 reg 4(2)
Secretary of State  % Words substituted (1.8.12) by SI 2012/2007 Sch para 113(3)(c)(i)
would have made decision $\mathcal{B}$ differently if 
%he 
%it  % Word substituted (6.4.09) by SI 2009/396 reg 4(2)(b)
%had been  % Words omitted (1.8.12) by SI 2012/2007 Sch para 113(3)(c)(ii)
aware of decision $\mathcal{C}$ at the time 
%he made 
of making  % Words substituted (1.8.12) by SI 2012/2007 Sch para 113(3)(c)(ii)
decision $\mathcal{B}$,
\end{enumerate}
decision $\mathcal{B}$ may be revised at any time.

% Reg 3A(6), (7) omitted (6.4.09) by SI 2009/396 reg 4(2)(c)
%(6) Section 16 of the Child Support Act shall apply in relation to any decision of the Secretary of State—
%\begin{enumerate}\item[]
%($a$) under section 41A or 47 of the Child Support Act; or
%
%($b$) that an adjustment shall cease or with respect to the adjustment of amounts payable under maintenance calculations for the purpose of taking account of overpayments of child support maintenance and voluntary payments,
%\end{enumerate}
%as it applies in relation to any decision of the Secretary of State under sections 11, 
%%12, 17 or 46 
%12 or 17  % Words substituted (27.10.08) by SI 2008/2543 reg 4(3)(b)
%of that Act, or the determination of 
%%an appeal tribunal 
%the First-tier Tribunal  % Words substituted (3.11.08) by SI 2008/2683 Sch 1 para 98(b)
%on a referral under section 28D(1)($b$)  of that Act.
%
%(7) In paragraph (6)($b$)  and in regulations 6A(9), 6B(4)($d$)  and 30A “voluntary payments” means the same as in the definition in section 28J of the Child Support Act and Regulations made under that section.

% Reg 3A(8), (9) added (27.10.08) by SI 2008/2544 reg 3
(8) Subject to paragraph (9), section 16 of the Child Support Act shall apply in relation to any decision of the 
%Secretary of State
%Commission  % Words substituted (6.4.09) by SI 2009/396 reg 4(2)
Secretary of State  % Words substituted (1.8.12) by SI 2012/2007 Sch para 113(3)(d)
not to make a maintenance calculation, as it applies in relation to any decision of the 
%Secretary of State
%Commission  % Words substituted (6.4.09) by SI 2009/396 reg 4(2)
Secretary of State  % Words substituted (1.8.12) by SI 2012/2007 Sch para 113(3)(d)
under sections 11, 12 or 17 of that Act, or the determination of an appeal tribunal on a referral under section 28D(1)($b$)  of that Act.

(9) Paragraph (8) shall not apply to any decision not to make a maintenance calculation where the 
%Secretary of State
%Commission  % Words substituted (6.4.09) by SI 2009/396 reg 4(2)
Secretary of State  % Words substituted (1.8.12) by SI 2012/2007 Sch para 113(3)(d)
makes a decision under section 12 of the Child Support Act.

\amendment{
Reg. 3A inserted (3.3.03 for new-rules cases only) by the Child Support (Decisions and Appeals) (Amendment) Regulations 2000 reg. 5 (subject to reg. 1(2)).

Reg. 3A(1)(cc), (f), (5A) inserted and reg. 3A(3) substituted (3.3.03 for new-rules cases only) by the Child Support (Miscellaneous Amendments) Regulations 2002 reg. 2(2).

Words substituted in reg. 3A(3), (6) (27.10.08) by the Child Support (Consequential Provisions) Regulations 2008 reg. 4(3).

Reg. 3A(8), (9) added (27.10.08) by the Child Support (Miscellaneous Amendments) (No. 2) Regulations 2008 reg. 3.

Words substituted in reg. 3A(1)(cc), (3), (5A)(b), (6) (3.11.08) by the Tribunals, Courts and Enforcement Act 2007 (Transitional and Consequential Provisions) Order 2008 Sch. 1 para. 99.

Words substituted in reg. 3A and reg. 3A(6), (7) omitted (6.4.09) by the Child Support (Miscellaneous Amendments) Regulations 2009 reg. 4(2).

Words substituted in reg. 3A(1), (2)(b), (3), (5), (5A), (8), (9) and words omitted in reg. 3A(5A) (1.8.12) by the Public Bodies (Child Maintenance and Enforcement Commission: Abolition and Transfer of Functions) Order 2012 Sch. para. 113(3).

Reg. 3A omitted (10.12.12 for 2012 scheme cases only) by the Child Support (Meaning of Child and New Calculation Rules) (Consequential and Miscellaneous Amendment) Regulations 2012 reg. 6(3).

Reg. 3A(1)(a) substituted (23.3.15) by the Child Support (Miscellaneous and Consequential Amendments) Regulations 2015 reg. 6(2).
}

% Reg 3B inserted (28.10.13) by SI 2013/2380 reg 4(4)
\subsubsection[3B. Consideration of revision before appeal in relation to certain child support decisions]{Consideration of revision before appeal in relation to certain child support decisions}

3B.---(1)  This regulation applies in a case where—
\begin{enumerate}\item[]
($a$) the Secretary of State gives a person written notice of a decision; and

($b$) that notice includes a statement to the effect that there is a right of appeal against the decision only if the Secretary of State has considered an application for a revision of the decision.
\end{enumerate}

(2) In a case to which this regulation applies, a person has a right of appeal under section 20 of the Child Support Act 1991 (as substituted by section~10 of the Child Support, Pensions and Social Security Act 2000) against the decision only if the Secretary of State has considered on an application whether to revise the decision under section 16 of that Act.

(3) The notice referred to in paragraph (1) must inform the person of the time limit specified in regulation 3A(1)($a$)  for making an application for a revision.

(4) Where, as the result of paragraph (2), there is no right of appeal against a decision, the Secretary of State may treat any purported appeal as an application for a revision under section 16 of that Act.

(5) In this regulation “decision” means a decision mentioned in section~20(1)($a$)  or ($b$)  of the Child Support Act 1991 (as substituted by section~10 of the Child Support, Pensions and Social Security Act 2000).

\amendment{
Reg. 3B inserted (28.10.13) by the Social Security, Child Support, Vaccine Damage and Other Payments (Decisions and Appeals) (Amendment) Regulations 2013 reg.~4(4).
}

\subsubsection[4. Late application for a revision]{Late application for a revision}

4.—(1) The time limit for making an application for a revision specified in regulation 3(1) or (3) 
or 3A(1)($a$)  % Words inserted (3.3.03 for new-rules cases only) by SI 2000/3185 reg 6(a)
may be extended where the conditions specified in the following provisions of this regulation are satisfied.

(2) An application for an extension of time shall be made by 
the relevant person,  % Words inserted (3.3.03 for new-rules cases only) by SI 2000/3185 reg 6(b)
the claimant or a person acting on his behalf.

(3) An application shall—
\begin{enumerate}\item[]
($a$) contain particulars of the grounds on which the extension of time is sought and shall contain sufficient details of the decision which it is sought to have revised to enable that decision to be identified; and

($b$) be made within 13 months of the date of notification of the decision which it is sought to have revised%
, but if the applicant has requested a statement of the reasons in accordance with 
regulation~3ZA(3)($b$)  or  % Words inserted (28.10.13) by SI 2013/2380 reg 4(5)(a)
regulation 28(1)($b$)  the 13 month period shall be extended by—
\begin{enumerate}\item[]
(i) if the statement is provided within one month of the notification, an additional 14 days; or

(ii) if it is provided after the elapse of a period after the one month ends, the length of that period and an additional 14 days.
\end{enumerate}  % Words inserted (18.3.05) by SI 2005/337 reg 2(3)
\end{enumerate}

(4) An application for an extension of time shall not be granted unless the applicant satisfies the Secretary of State%
, the Commission
or the Board or an officer of the Board  % Words inserted (5.10.99) by SI 1999/2570 reg 7
that—
\begin{enumerate}\item[]
($a$) it is reasonable to grant the application;

($b$) the application for revision has merit%
, except in a case to which regulation 3ZA or 3B applies%  % Words inserted (28.10.13) by SI 2013/2380 reg 4(5)(b)
; and

($c$) special circumstances are relevant to the application and as a result of those special circumstances it was not practicable for the application to be made within the time limit specified in regulation 3 
or 3A%  % Words inserted (3.3.03 for new-rules cases only) by SI 2000/3185 reg 6(c)
.
\end{enumerate}

(5) In determining whether it is reasonable to grant an application, the Secretary of State%
, the Commission
or the Board or an officer of the Board  % Words inserted (5.10.99) by SI 1999/2570 reg 7
shall have regard to the principle that the greater the amount of time that has elapsed between the expiration of the time specified in regulation 3(1) and (3) 
and regulation 3A(1)($a$)  % Words inserted (3.3.03 for new-rules cases only) by SI 2000/3185 reg 6(d)
for applying for a revision and the making of the application for an extension of time, the more compelling should be the special circumstances on which the application is based.

(6) In determining whether it is reasonable to grant the application for an extension of time%
, except in a case to which regulation 3ZA or 3B applies%  % Words inserted (28.10.13) by SI 2013/2380 reg 4(5)(c)
, no account shall be taken of the following—
\begin{enumerate}\item[]
($a$) that the applicant or any person acting for him was unaware of or misunderstood the law applicable to his case (including ignorance or misunderstanding of the time limits imposed by these Regulations); or

($b$) that 
%a Commissioner\opt{oldrules}{ }% 
%% Words inserted (3.3.03 for new-rules cases only) by SI 2000/3185 reg 6(a)
%\opt{newrules,2012rules}{%
%, a Child Support Commissioner
%}%
the Upper Tribunal  % Words substituted (3.11.08) by SI 2008/2683 Sch 1 para 100
or a court has taken a different view of the law from that previously understood and applied.
\end{enumerate}

(7) An application under this regulation for an extension of time which has been refused may not be renewed.

\amendment{
Words inserted in reg. 4(4), (5) (5.10.99) by the Tax Credits (Decisions and Appeals) (Amendment) Regulations 1999 reg. 7.

Words inserted in reg. 4(1), (2), (4)(c), (5) and (6)(b) (3.3.03 for new-rules cases only) by the Child Support (Decisions and Appeals) (Amendment) Regulations 2000 reg. 6 (subject to reg. 1(2)).

Words inserted in reg. 4(3)(b) (18.3.05) by the Social Security, Child Support and Tax Credits (Miscellaneous Amendments) Regulations 2005 reg. 2(3).

Words substituted in reg. 4(6)(b) (3.11.08) by the Tribunals, Courts and Enforcement Act 2007 (Transitional and Consequential Provisions) Order 2008 Sch. 1 para. 100.

Words inserted in reg. 4 (6.4.09 for new-rules cases only) by the Child Support (Miscellaneous Amendments) Regulations 2009 reg. 4(3).

Words omitted in reg. 4(1), (2), (4)(c), (5) (10.12.12 for 2012 scheme cases only) by the Child Support (Meaning of Child and New Calculation Rules) (Consequential and Miscellaneous Amendment) Regulations 2012 reg. 6(4).

Words inserted in reg. 4(3)(b), (4)(b), (6) (28.10.13) by the Social Security, Child Support, Vaccine Damage and Other Payments (Decisions and Appeals) (Amendment) Regulations 2013 reg.~4(5).
}

\subsubsection[5. Date from which a decision revised under section 9 takes effect]{Date from which a decision revised under section 9 takes effect}

5.%
---(1)  % Reg 5 renumbered as reg 5(1) (27.9.04) by SI 2004/2283 reg 3
  Where, on a revision under section 9, the Secretary of State
or the Board or an officer of the Board  % Words inserted (5.10.99) by SI 1999/2570 reg 8
decides that the date from which the decision under section 8 or 10 (“the original decision”) took effect was erroneous, the decision under section 9 shall take effect on the date from which the original decision would have taken effect had the error not been made.

% Reg 5(2) inserted (27.9.04) by SI 2004/2283 reg 3
(2) Where—
\begin{enumerate}\item[]
($a$) a person attains pensionable age, claims a retirement pension after the prescribed time for claiming and the Secretary of State decides (“the original decision”) that he is not entitled because—
\begin{enumerate}\item[]
(i) in the case of a Category A retirement pension, the person has not satisfied the contribution conditions; or

(ii) in the case of a Category B retirement pension, the person’s spouse 
or civil partner  % Words inserted (5.12.05) by SI 2005/2878 reg 8(3)
has not satisfied the contribution conditions;
\end{enumerate}

($b$) in accordance with regulation 50A of the Social Security (Contributions) Regulations 2001\footnote{S.I. 2001/1004; regulation 50A was inserted by S.I. 2004/1362.} (Class 3 contributions: tax years 1996--97 to 2001--02) the Board subsequently accepts Class 3 contributions paid after the due date by the claimant or, as the case may be, the spouse
or civil partner%  % Words inserted (5.12.05) by SI 2005/2878 reg 8(3)
;

($c$) in accordance with regulation 6A of the Social Security (Crediting and Treatment of Contributions, and National Insurance Numbers) Regulations 2001\footnote{S.I. 2001/769; regulation 6A was inserted by S.I. 2004/1361.} the contributions are treated as paid on a date earlier than the date on which they were paid; and

($d$) the Secretary of State revises the original decision in accordance with regulation 11A(4)($a$),
\end{enumerate}
the revised decision shall take effect from—
\begin{enumerate}\item[]
(i) 1st October 1998; or

(ii) the date on which the claimant attained pensionable age in the case of a Category A pension, or, in the case of a Category B pension, the date on which the claimant’s spouse or civil partner  % Words inserted (5.12.05) by SI 2005/2878 reg 8(3)
attained pensionable age,
\end{enumerate}
whichever is later.

\amendment{
Words inserted in reg. 5 (5.10.99) by the Tax Credits (Decisions and Appeals) (Amendment) Regulations 1999 reg. 8.

Reg. 5 renumbered as reg. 5(1) and reg. 5(2) inserted (27.9.04) by the Social Security (Retirement Pensions) Amendment Regulations 2004 reg. 3.

Words inserted in reg. 5(2)(a)(ii), (b), (ii) (5.12.05) by the Social Security (Civil Partnership) (Consequential Amendments) Regulations 2005 reg. 8(3).
}

% Reg 5A inserted (3.3.03 for new-rules cases only) by SI 2000/3185 reg 7
\subsubsection[5A. Date from which a decision revised under section 16 of the Child Support Act takes effect]{Date from which a decision revised under section 16 of the Child Support Act takes effect\\*\emph{2003 scheme cases only}}

5A.---(1)  Where the date from which a decision took effect is found to be erroneous on a revision under section 16 of the Child Support Act, the revision shall take effect from the date on which the decision revised would have taken effect had the error not been made.

% Reg 5A(2), (3) omitted (12.7.06) by SI 2006/1520
%(2) Where the Secretary of State considers it appropriate to revise a decision under section 12(1) of the Child Support Act as if he were revising a decision under section 11 of that Act, the revision shall take effect from the first day of the maintenance period in which the information required to make a maintenance calculation was provided, except where—
%\begin{enumerate}\item[]
%($a$) the non-resident parent satisfies the Secretary of State—
%\begin{enumerate}\item[]
%(i) that he used his best endeavours to obtain the information required by the Secretary of State; and
%
%(ii) the failure to provide the information was not his fault; or
%\end{enumerate}
%
%%($b$) the decision which is treated as being made under section 11 of the Child Support Act is at a higher rate than the rate of liability which had been imposed by the decision made under section 12(1) of that Act.
%
%% Reg 5A(2)(b) substituted (3.3.03 for new-rules cases only) by SI 2003/129 reg 2(a)
%($b$) the total amount of child support maintenance which would be fixed for the relevant period by the decision which is treated as a decision made under section 11 of the Child Support Act is greater than the total amount of child support maintenance which was fixed for that period by the decision made under section 12(1) of the Child Support Act (whether as originally made, or as revised or superseded under sections 16 or 17 of that Act respectively, or decided on appeal).
%\end{enumerate}
%
%% Reg 5A(3) added (3.3.03 for new-rules cases only) by SI 2003/129 reg 2(b)
%(3) For the purposes of paragraph (2)($b$), “the relevant period” means the period during which the decision under section 12(1) of the Child Support Act (whether as originally made, or as revised or superseded under sections 16 or 17 of that Act respectively, or decided on appeal) applied.

\amendment{
Reg. 5A inserted (3.3.03 for new-rules cases only) by the Child Support (Decisions and Appeals) (Amendment) Regulations 2000 reg. 7 (subject to reg. 1(2)).

Reg. 5A(3) added and reg. 5A(2)(b) substituted (3.3.03 for new-rules cases only) by the Child Support (Decisions and Appeals) (Amendment) Regulations 2003 reg. 2.

Reg. 5A(2), (3) omitted (12.7.06) by the Child Support (Miscellaneous Amendments) Regulations 2006 reg. 4.

Reg. 5A omitted (10.12.12 for 2012 scheme cases only) by the Child Support (Meaning of Child and New Calculation Rules) (Consequential and Miscellaneous Amendment) Regulations 2012 reg. 6(3).
}

\subsection[Chapter II --- Supersessions]{Chapter II\\*Supersessions}

\subsubsection[6. Supersession of decisions]{Supersession of decisions}

\renewcommand\parthead{--- Part II Chapter II}

6.—(1) Subject to the following provisions of this regulation, for the purposes of section 10, the cases and circumstances in which a decision may be superseded under that section are set out in paragraphs (2) to (4).

(2) A decision under section 10 may be made on the Secretary of State’s 
or the Board's  % Words inserted (5.10.99) by SI 1999/2570 reg 9(2)(a)
own initiative or on an application made for the purpose on the basis that the decision to be superseded—
\begin{enumerate}\item[]
($a$) is one in respect of which—
\begin{enumerate}\item[]
(i) there has been a relevant change of circumstances since the decision 
%was made
had effect  % Words substituted (5.5.03) by SI 2003/1050 reg 3(3)(a)
or, in the case of an advance award under regulation 13\footnote{Regulation 13 was amended by S.I. 1991/2284 and 2741, 1992/247, 1994/2319, 1999/2422, 2572 and 3178 and 2002/3019.}, 13A or 13C\footnote{Regulations 13A and 13C were inserted by S.I. 1991/2741 and amended by S.I. 1999/2860 and 3178.} of the Claims and Payments Regulations
or regulation 146 of the Employment and Support Allowance Regulations%  % Words inserted (27.7.08) by SI 2008/1554 reg 32(2)(a)
, since the decision was made%  % Words inserted (18.3.05) by SI 2005/337 reg 2(4)(a)
; or

(ii) it is anticipated that a relevant change of circumstances will occur;
\end{enumerate}

($b$) is a decision of the Secretary of State 
or the Board or an officer of the Board  % Words inserted (5.10.99) by SI 1999/2570 reg 9(2)(b)(i)
other than a decision to which sub-paragraph ($d$) refers and—
\begin{enumerate}\item[]
(i) the decision was erroneous in point of law, or it was made in ignorance of, or was based upon a mistake as to, some material fact; and

(ii) an application for a supersession was received by the Secretary of State
or the Board,  % Words inserted (5.10.99) by SI 1999/2570 reg 9(2)(b)(ii)
or the decision by the Secretary of State 
or the Board  % Words inserted (5.10.99) by SI 1999/2570 reg 9(2)(b)(ii)
to act on his 
or their  % Words inserted (5.10.99) by SI 1999/2570 reg 9(2)(b)(iii)
own initiative was taken, more than one month after the date of notification of the decision which is to be superseded or after the expiry of such longer period of time as may have been allowed under regulation 4;
\end{enumerate}

%($c$) is a decision of an appeal tribunal or of a Commissioner that was made in ignorance of, or was based upon a mistake as to, some material fact;

% Reg 6(2)(c) substituted (5.5.03) by SI 2003/1050 reg 3(3)(b)
($c$) is a decision of 
%an appeal tribunal or of a Commissioner
%the First-tier Tribunal or of the Upper Tribunal%  % Words substituted (3.11.08) by SI 2008/2683 Sch 1 para 101(a)
an appeal tribunal, the First-tier Tribunal, the Upper Tribunal or of a Commissioner%  % Words substituted (3.11.08) by SI 2012/1267 reg 4(2)(a)
—
\begin{enumerate}\item[]
(i) that was made in ignorance of, or was based upon a mistake as to, some material fact; or

(ii) that was made in accordance with section 26(4)($b$), in a case where section 26(5) applies;
\end{enumerate}

($d$) is a decision which is specified in Schedule 2 to the Act or is prescribed in regulation 27 (decisions against which no appeal lies); 
%or % Word omitted (5.5.03) by SI 2003/1050 reg 3(3)(c)

%($e$) is a decision where—
%\begin{enumerate}\item[]
%(i) the Secretary of State 
%or the Board or an officer of the Board  % Words inserted (5.10.99) by SI 1999/2570 reg 9(2)(c)
%has awarded a relevant benefit; and
%
%(ii) on a date after the date that entitlement arises, the claimant or a member of his family becomes entitled to another relevant benefit or to an increase in the rate of another such benefit;
%\end{enumerate}

% Reg 6(2)(e) substituted (19.6.00) by SI 2000/1596 reg 16
($e$) is a decision where—
\begin{enumerate}\item[]
(i) the claimant has been awarded entitlement to a relevant benefit; and

(ii) 
%on a date after that entitlement arises
subsequent to the first day of the period to which that entitlement relates%  % Words substituted (2.4.02) by SI 2002/428 reg 4(3)(a)
, the claimant or a member of his family becomes entitled to
%, and is paid,  % Words omitted (2.4.02) by SI 2002/428 reg 4(3)(b)
another relevant benefit or an increase in the rate of another relevant benefit;
\end{enumerate}

% Reg 6(2)(ee) inserted (18.3.05) by SI 2005/337 reg 2(4)(b)
($ee$) is an original award within the meaning of regulation 3(7ZA) and sub-paragraphs ($a$)  to ($c$)  and ($d$)(ii)  of regulation 3(7ZA) apply but not sub-paragraph ($d$)(i);

%($f$) is a decision of the Secretary of State that a jobseekers allowance is payable to a claimant where the Secretary of State subsequently determines that the allowance is not payable in accordance with section 19 of the Jobseekers Act;

%% Reg 6(2)(f) substituted (18.10.99) by SI 1999/2677 reg 7(a)
%($f$) is a decision that a jobseeker’s allowance is payable to a claimant where that allowance ceases to be payable by virtue of 
%regulation 27A of the Jobseeker’s Allowance Regulations or  % Words inserted by SI 2010/509 reg 3(3)
%section 19(1) of the Jobseekers Act
%or ceases to be payable or is reduced by virtue of section 20A(5) of that Act% % Words inserted (19.3.01) by SI 2000/1982 reg 5(b)
%;
%
%% Reg 6(2)(fa) inserted temporarily by SI 2010/1222 reg 20(b), permanently by SI 2011/688 reg 18(b)
%($fa$) is a decision that a jobseeker’s allowance is payable to a claimant where that allowance ceases to be payable or is reduced by virtue of regulations made under section 17A of the Jobseekers Act;

% Reg 6(2)(f), (fa) substituted by SI 2012/2568 reg 6(3)
($f$) is a decision that a jobseeker’s allowance is payable at the full rate to which the claimant would be entitled in the absence of any reduction where the award is reduced under section 19 of the Jobseekers Act;

($fa$) is a decision that a jobseeker’s allowance is payable at the full rate to which the claimant would be entitled in the absence of any reduction where the award is reduced under section 19A of the Jobseekers Act;

% Reg 6(2)(g) inserted (5.7.99) by SI 1999/1623 reg 3
\looseness=-1
($g$) is an incapacity benefit decision where there has been an incapacity determination (whether before or after the decision) and where, since the decision was made, the Secretary of State has received medical evidence following an examination in accordance with regulation 8 of the Social Security (Incapacity for Work) (General) Regulations 1995\footnote{\frenchspacing S.I. 1995/311; relevant amending instruments are S.I. 1995/987, 1996/3207 and 1997/1009.} from a 
%doctor 
health care professional  % Words substituted by SI 2008/2667 reg 3(3)(a)
referred to in paragraph (1) of that regulation;
% Reg 6(2)(h) added (3.4.00) by SI 2000/897 Sch 6 para 4
%and  % Word omitted (5.5.03) by SI 2003/1050 reg 3(3)(d)

($h$) is one in respect of a person who—
\begin{enumerate}\item[]
(i) is subsequently the subject of a separate decision or determination as to whether or not he took part in a work-focused interview;
\looseness=-1

(ii) had been held not to have taken part in a work-focused interview but who had, subsequent to the decision to be superseded, attained 
%the age of 60 
pensionable age  % Words substituted by SI 2010/563 reg 2(a)
or ceased to reside in an area in which there is a requirement to take part in a work-focused interview
or, in the case of a partner who was required to take part in a work-focused interview 
%under the Social Security (Jobcentre Plus Interviews for Partners) Regulations 2003, ceased to be a partner for the purposes of those Regulations or is no longer a partner to whom those Regulations apply%  % Words added (12.4.04) by SI 2003/1886 reg 15(4)
in accordance with regulations made under section 2AA of the Administration Act, ceased to be a partner for the purposes of those regulations or is no longer a partner to whom the requirement to take part in a work-focused interview under those regulations applies  % Words substituted (26.4.04) by SI 2004/959 reg 24(3)
(and in this head “pensionable age” has the meaning given by the rules in paragraph 1 of Schedule 4 to the Pensions Act 1995\footnote{1995 c.~26.}, save that a man born before 6th April 1955 is treated as attaining pensionable age when a woman born on the same day as the man would attain pensionable age)%  % Words inserted by SI 2010/563 reg 2(b)
;
\end{enumerate}

% Reg 6(2)(i) added (15.10.01) by SI 2001/1711 reg 2(2)(c), omitted by SI 2010/424 reg 4(4)
%($i$) is a decision of the Secretary of State that a jobseeker’s allowance% 
%%or income support 
%, income support or an employment and support allowance  % Words substituted (27.7.08) by SI 2008/1554 reg 32(2)(b)
%is payable to a claimant where the Secretary of State is notified that a court has made a determination which results in a restriction being imposed pursuant to section 62 or 63 of the Child Support, Pensions and Social Security Act 2000;

% Reg 6(2)(j), (k) added (1.4.02) by SI 2002/490 reg 8(b)
($j$) is a decision of the Secretary of State that a sanctionable benefit is payable to a claimant where that benefit ceases to be payable or falls to be reduced under section 
6B,  % Words inserted by SI 2010/1160 reg 3(3)(a)
7 or 9 of the Social Security Fraud Act 2001 and for this purpose “sanctionable benefit” has the 
%same meaning as in section 7 
meaning given in section 6A  % Words substituted by SI 2010/1160 reg 3(3)(b)
of that Act;

($k$) is a decision of the Secretary of State that a joint-claim jobseeker’s allowance is payable where that allowance ceases to be payable or falls to be reduced under section 8 of the Social Security Fraud Act 2001;

% Reg 6(2)(l) added (7.4.03) by SI 2002/3019 reg 17(a)
($l$) is a relevant decision for the purposes of section 6 of the State Pension Credit Act and—
\begin{enumerate}\item[]
(i) on making that decision, the Secretary of State specified a period as the assessed income period; and

(ii) that period has ended or is about to end;
\end{enumerate}

% Reg 6(2)(m) added (6.10.03) by SI 2003/2274 reg 5(2)
($m$) is a relevant decision for the purposes of section 6 of the State Pension Credit Act in a case where—
\begin{enumerate}\item[]
(i) the information and evidence required under regulation 32(6)($a$)  of the Claims and Payments Regulations has not been provided in accordance with the time limits set out in regulation 32(6)($c$)  of those Regulations;

(ii) the Secretary of State was prevented from specifying a new assessed income period under regulation 10(1) of the State Pension Credit Regulations; and

\begin{sloppypar}
(iii) the information and evidence required under regulation 32(6)($a$)  of the Claims and Payments Regulations has since been provided;\looseness=-1
\end{sloppypar}
\end{enumerate}

% Reg 6(2)(n) added (18.3.05) by SI 2005/337 reg 2(4)(c)
($n$) is a decision by 
an appeal tribunal or  % Words inserted (3.11.08) by SI 2012/1267 reg 4(2)(b)(i)
%an appeal tribunal 
the First-tier Tribunal  % Words substituted (3.11.08) by SI 2008/2683 Sch 1 para 101(b)(i)
confirming a decision by the Secretary of State terminating a claimant’s entitlement to income support because he no longer falls within the category of person specified in paragraph 7 of Schedule 1B to the Income Support Regulations (persons incapable of work) and a further 
%appeal tribunal 
%decision of the First-tier Tribunal  % Words substituted (3.11.08) by SI 2008/2683 Sch 1 para 101(b)(ii)
decision of an appeal tribunal or the First-tier Tribunal  % Words substituted (3.11.08) by SI 2012/1267 reg 4(2)(b)(ii)
subsequently determines that he is incapable of work;

% Reg 6(2)(o) inserted (6.4.06) by SI 2005/2677 reg 9(4)
($o$) is a decision that a person is entitled to state pension credit and—\looseness=1
\begin{enumerate}\item[]
(i) the person or his partner makes, or is treated as having made, an election for a lump sum in accordance with—
\begin{enumerate}\item[]
($aa$) paragraph A1 or 3C of Schedule 5 to the Contributions and Benefits Act\footnote{Paragraphs A1 and 3C are inserted respectively by paragraphs 4 and 9 of Schedule 11 to the Pensions Act 2004 (c. 35).};

($bb$) paragraph 1 of Schedule 5A to that Act\footnote{Schedule 5A is inserted by paragraph 15 of Schedule 11 to the Pensions Act 2004.}; or, as the case may be,

($cc$) paragraph 12 or 17 of Schedule 1 to the Graduated Retirement Benefit Regulations;
\end{enumerate}
or

(ii) such a lump sum is repaid in consequence of an application to change an election for a lump sum in accordance with regulation~5 of the Deferral of Retirement Pensions etc.\ Regulations or, as the case may be, paragraph 20D of Schedule 1 to the Graduated Retirement Benefit Regulations;
\end{enumerate}

% Reg 6(2)(oa) inserted by SI 2015/1985 reg 18(5)
($oa$) is a decision that a person is entitled to state pension credit and—
\begin{enumerate}\item[]
(i) the person—
\begin{enumerate}\item[]
($aa$) chooses under section 8(2) of the Pensions Act 2014, or under regulations under section 10 of that Act which make provision corresponding or similar to section 8(2), to be paid a lump sum; or

($bb$) is entitled to a lump sum under section 8(4) of the Pensions Act 2014, or under regulations under section 10 of that Act which make provision corresponding or similar to section 8(4), because the person has failed to choose within the period mentioned in section 8(3); or
\end{enumerate}

(ii) such a lump sum is repaid in consequence of an application—
\begin{enumerate}\item[]
($aa$) to alter the choice mentioned in paragraph (i)($aa$) in accordance with regulation 6 of the State Pension Regulations 2015 or regulations made under section 10 of the Pensions Act 2014 which make provision corresponding or similar to regulation 6 of the State Pension Regulations 2015; or

($bb$) to make a late choice in accordance with regulation 4(4) of the State Pension Regulations 2015 (when a choice of lump sum or survivor’s pension may be made) or regulations made under section 10 of the Pensions Act 2014 which make provision corresponding or similar to regulation 4(4) of the State Pension Regulations 2015;
\end{enumerate}
\end{enumerate}

% Reg 6(2)(p)--(r) inserted (27.7.08) by SI 2008/1554 reg 32(2)(c)
($p$) is a decision awarding employment and support allowance where there has been a failure determination;

($q$) is a decision made in consequence of a failure determination where the reduction ceases to have effect under of regulation 64 of the Employment and Support Allowance Regulations;

%($r$) is an employment and support allowance decision where, since the decision was made, the Secretary of State has received medical evidence from a health care professional approved by the Secretary of State for the purposes of regulation 23 or 38 of the Employment and Support Allowance Regulations;

% Reg 6(2)(r) substituted by SI 2010/840 reg 7(3)
($r$) is an employment and support allowance decision where, since the decision was made, the Secretary of State has—
\begin{enumerate}\item[]
(i) received medical evidence from a health care professional approved by the Secretary of State, or

(ii) made a determination that the claimant is to be treated as having limited capability for work in accordance with regulation~20,~25,~26 or 33(2) of the Employment and Support Allowance Regulations;
\end{enumerate}

% Reg 6(2)(s) inserted by SI 2008/2667 reg 3(3)(b)
($s$) is a decision where on or after the date on which the decision was made, a late or unpaid contribution is treated as paid under—
\begin{enumerate}\item[]
(i) regulation 5 of the Social Security (Crediting and Treatment of Contributions and National Insurance Numbers) Regulations 2001 (treatment of late paid contributions where no consent, connivance or negligence by the primary contributor) on a date which falls on or before the date on which the original decision was made;

(ii) regulation 6 of those Regulations (treatment of contributions paid late through ignorance or error) on a date which falls on or before the date on which the original decision was made; or

(iii) regulation 60 of the Social Security (Contributions) Regulations 2001 (treatment of unpaid contributions where no consent, connivance or negligence by the primary contributor) on a date which falls on or before the date on which the original decision was made;
\end{enumerate}

% Reg 6(2)(sa) inserted by SI 2016/1145 reg 4(4)
($sa$) is a decision where on or after the date on which the decision was made, a late contribution is treated as paid by virtue of regulation 4 of the Social Security (Crediting and Treatment of Contributions, and National Insurance Numbers) Regulations 2001 for the purposes of entitlement to—
\begin{enumerate}\item[]
(i) a bereavement benefit;

(ii) a Category A or Category B retirement pension under Part II of the Contributions and Benefits Act; or

(iii) a state pension under Part I of the Pensions Act 2014;
\end{enumerate}

% Reg 6(2)(t), (u) inserted by SI 2014/1097 reg 12(4)
($t$) is a decision awarding income support where there has been a determination by the Secretary of State under regulation 6(2) of the Income Support Work-Related Activity Regulations that a person has failed to undertake work-related activity;

($u$) is a decision made in consequence of a determination by the Secretary of State that a person has failed to undertake work-related activity where a reduction under regulation 8(1) of the Income Support Work-Related Activity Regulations ceases to have effect by virtue of regulation~9 of those Regulations.
\end{enumerate}

(3) A decision which may be revised under regulation 3 may not be superseded under this regulation except where—
\begin{enumerate}\item[]
($a$) circumstances arise in which the Secretary of State 
or the Board or an officer of the Board  % Words inserted (5.10.99) by SI 1999/2570 reg 9(3)
may revise that decision under regulation~3; and

\looseness=1
($b$) further circumstances arise in relation to that decision which are not specified in regulation 3 but are specified in paragraph (2) or~(4).
\end{enumerate}

(4) Where the Secretary of State requires 
or the Board require  % Words inserted (5.10.99) by SI 1999/2570 reg 9(4)(a)
further evidence or information from the applicant in order to consider all the issues raised by an application under paragraph (2) (“the original application”), he 
or they  % Words inserted (5.10.99) by SI 1999/2570 reg 9(4)(b)
shall notify the applicant that further evidence or information is required and the decision may be superseded—
\begin{enumerate}\item[]
($a$) where the applicant provides further relevant evidence or information within one month of the date of notification or such longer period of time as the Secretary of State 
or the Board  % Words inserted (5.10.99) by SI 1999/2570 reg 9(4)(c)
may allow; or

($b$) where the applicant does not provide such evidence or information within the time allowed under sub-paragraph ($a$), on the basis of the original application.
\end{enumerate}

(5) The Secretary of State 
or the Board  % Words inserted (5.10.99) by SI 1999/2570 reg 9(5)
may treat an application for a revision or a notification of a change of circumstances as an application for a supersession.

(6) The following events are not relevant changes of circumstances for the purposes of paragraph (2)—
\begin{enumerate}\item[]
($a$) the repayment of a loan to which regulation 66A of the Income Support Regulations\footnote{\frenchspacing Regulation 66A was inserted by S.I. 1990/1549; relevant amending instruments are S.I. 1991/236, S.I. 1991/1559 and S.I. 1996/462.}% 
, regulation 137 of the Employment and Support Allowance Regulations  % Words inserted (27.7.08) by SI 2008/1554 reg 32(3)
or regulation 136 of the Jobseeker’s Allowance Regulations applies;

% Reg 6(6)(b) omitted (18.3.05) by SI 2005/337 reg 2(4)(d)
%($b$) the absence from a nursing home or residential care home for a period of less than one week of a resident who is—
%\begin{enumerate}\item[]
%(i) in receipt of income support or a jobseeker’s allowance; and
%
%(ii) not a claimant to whom Part II of Schedule 4 to the Income Support Regulations applies;
%\end{enumerate}

% Reg 6(6)(c) inserted (18.10.99) by SI 1999/2677 reg 7(b)
($c$) the fact that a person has become terminally ill, within the meaning of section 66(2)($a$) of the Contributions and Benefits Act, unless an application for supersession which contains an express statement that the person is terminally ill is made either by—
\begin{enumerate}\item[]
(i) the person himself; or

(ii) any other person purporting to act on his behalf whether or not that other person is acting with his knowledge or authority;
\end{enumerate}
and where such an application is received a decision may be so superseded nothwithstanding that no claim under section 66(1) or, as the case may be, 72(5) or 73(12) of that Act has been made.
\end{enumerate}

(7) In paragraph (6)($b$), “nursing home” and “residential care home” have the same meanings as they have in regulation 19 of the Income Support Regulations.

% Reg 6(8) added (7.4.03) by SI 2002/3019 reg 17(b)
(8) In relation to the assessed income period, the only change of circumstance relevant for the purposes of paragraph (2)($a$)  is that the assessed income period ends in accordance with section 9(4) of the State Pension Credit Act or the regulations made under section 9(5) of that Act.

\amendment{
Reg. 6(2)(g) inserted (5.7.99) by the Social Security and Child Support (Decisions and Appeals) Amendment (No. 2) Regulations 1999 reg. 3.

Words inserted in reg. 6(2), (3), (4), (5) (5.10.99) by the Tax Credits (Decisions and Appeals) (Amendment) Regulations 1999 reg. 9.

Reg. 6(6)(c) inserted and reg. 6(2)(f) substituted (18.10.99) by the Social Security and Child Support (Decisions and Appeals), Vaccine Damage Payments and Jobseeker's Allowance (Amendment) Regulations 1999 reg. 7.

Reg. 6(2)(h) added (3.4.00) by the Social Security (Work-focused Interviews) Regulations 2000 Sch. 6 para. 4.

Reg. 6(2)(e) substituted (19.6.00) by the Social Security and Child Support (Miscellaneous Amendments) Regulations 2000 reg. 16.

Words inserted in reg. 6(2)(f) (19.3.01) by the Social Security (Joint Claims: Consequential Amendments) Regulations 2000 reg. 5(b).

Reg. 6(2)(i) added (15.10.01) by the Social Security (Breach of Community Order) (Consequential Amendments) Regulations 2001 reg. 2(2)(c).

Reg. 6(2)(j), (k) added (1.4.02) by the Social Security (Loss of Benefit) (Consequential Amendments) Regulations 2002 reg. 8(b).

Words substituted and omitted in reg. 6(2)(e)(ii) (2.4.02) by the Social Security (Claims and Payments and Miscellaneous Amendments) Regulations 2002 reg. 4(3).

Reg. 6(2)(l), (7) added (7.4.03) by the State Pension Credit (Consequential, Transitional and Miscellaneous Provisions) Regulations 2002 reg. 17.

Words substituted in reg. 6(2)(a)(i), words omitted after reg. 6(2)(d), (g) and reg. 6(2)(c) substituted (5.5.03) by the Social Security and Child Support (Miscellaneous Amendments) Regulations 2003 reg. 3(3).

Reg. 6(2)(m) added (6.10.03) by the State Pension Credit (Consequential, Transitional and Miscellaneous Provisions) Amendment Regulations 2003 reg. 5(2).

Words added to reg. 6(2)(h)(ii) (12.4.04) by the Social Security (Jobcentre Plus Interviews for Partners) Regulations 2003 reg. 15(4).

Words substituted in reg. 6(2)(h)(ii) (26.4.04) by the Social Security (Working Neighbourhoods) Regulations 2004 reg. 24(3).

Words inserted in reg. 6(2)(a)(i), reg. 6(2)(ee) inserted, reg. 6(2)(n) added and reg. 6(6)(b) omitted (18.3.05) by the Social Security, Child Support and Tax Credits (Miscellaneous Amendments) Regulations 2005 reg. 2(4).

Reg. 6(2)(o) inserted (6.4.06) by the Social Security (Deferral of Retirement Pensions, Shared Additional Pension and Graduated Retirement Benefit) (Miscellaneous Provisions) Regulations 2005 reg. 9(4).

\begin{sloppypar}
Words inserted in reg. 6(2)(a)(i), (6)(a), words substituted in reg. 6(2)(i) and reg. 6(2)(p)--(r) added (27.7.08) by the Employment and Support Allowance (Consequential Provisions) (No. 2) Regulations 2008 reg. 32.
\end{sloppypar}

Words substituted in reg. 6(2)(g) and reg. 6(2)(s) inserted (30.10.08) by the Social Security (Miscellaneous Amendments) (No. 5) Regulations 2008 reg. 3(3).

\looseness=-1
Words substituted in reg. 6(2)(c), (n) (3.11.08) by the Tribunals, Courts and Enforcement Act 2007 (Transitional and Consequential Provisions) Order 2008 Sch. 1 para. 101.

Words inserted in reg. 6(2)(n) and words substituted in reg. 6(2)(c), (n) (3.11.08) by the Social Security and Child Support (Supersession of Appeal Decisions) Regulations 2012 reg. 4(2).

Reg. 6(2)(i) omitted (22.3.10) by the Welfare Reform Act 2009 (Section 26) (Consequential Amendments) Regulations 2010 reg. 4(4).

Words inserted and substituted in reg. 6(2)(j) (1.4.10) by the Social Security (Loss of Benefit) Amendment Regulations 2010 reg. 3(3).

Words inserted in reg. 6(2)(f) (6.4.10) by the Jobseeker’s Allowance (Sanctions for Failure to Attend) Regulations 2010 reg. 3(3).

Words substituted and inserted in reg. 6(2)(h)(ii) (6.4.10) by the Social Security (Work-focused Interviews etc.) (Equalisation of State Pension Age) Amendment Regulations 2010 reg. 2.

Reg. 6(2)(r) substituted (28.6.10) by the Social Security (Miscellaneous Amendments) (No. 3) Regulations 2010 reg. 7(3).

Reg. 6(2)(fa) inserted temporarily (22.11.10--21.11.13) by the Jobseeker’s Allowance (Work for Your Benefit Pilot Scheme) Regulations 2010 reg. 20(b).

Reg. 6(2)(fa) inserted (25.4.11) by the Jobseeker’s Allowance (Mandatory Work Activity Scheme) Regulations 2011 reg. 18(b).

Reg. 6(2)(f), (fa) substituted (22.10.12) by the Jobseeker’s Allowance (Sanctions) (Amendment) Regulations 2012 reg. 6(3).

Reg. 6(2)(t), (u) inserted (28.4.14) by the Income Support (Work-Related Activity) and Miscellaneous Amendments Regulations 2014 reg. 12(4).

Reg. 6(2)(oa) inserted (6.4.16) by the Pensions Act 2014 (Consequential, Supplementary and Incidental Amendments) Order 2015 art.~18(5).

Reg. 6(2)(sa) inserted (1.1.17) by the Social Security (Credits, and Crediting and Treatment of Contributions) (Consequential and Miscellaneous Amendments) Regulations 2016 reg. 4(4).
}

% Regs 6A, 6B inserted (3.3.03 for new-rules cases only) by SI 2000/3185 reg 8
%
%\subsubsection[6A. Supersession of child support decisions]{Supersession of child support decisions}
%
%6A.---(1)  Subject to paragraphs (7) and (8), the cases and circumstances in which a decision (“a superseding decision”) may be made by the Secretary of State for the purposes of section 17 of the Child Support Act are set out in paragraphs (2) to (6).
%
%(2) A decision may be superseded by a decision made by the Secretary of State acting on his own initiative where—
%\begin{enumerate}\item[]
%($a$) there has been a relevant change of circumstances since the decision had effect; or
%
%($b$) the decision was made in ignorance of, or was based upon a mistake as to, some material fact.
%\end{enumerate}
%
%(3) Subject to regulation 6B, a decision may be superseded by a decision made by the Secretary of State where—
%\begin{enumerate}\item[]
%($a$) an application is made on the basis that—
%\begin{enumerate}\item[]
%(i) there has been a change of circumstances since the date from which the decision had effect; or
%
%(ii) it is expected that a change of circumstances will occur; and
%\end{enumerate}
%
%($b$) the Secretary of State is satisfied that the change of circumstances is or would be relevant.
%\end{enumerate}
%
%(4) A decision may be superseded by a decision made by the Secretary of State where—
%\begin{enumerate}\item[]
%($a$) an application is made on the basis that the decision was made in ignorance of, or was based upon a mistake as to, a fact; and
%
%($b$) the Secretary of State is satisfied that the fact is or would be material.
%\end{enumerate}
%
%% Reg 6A(4A) inserted (5.5.03 for new-rules cases only) by SI 2003/1050 reg 3(4)
%(4A) A decision may be superseded by a decision made by the Secretary of State—
%\begin{enumerate}\item[]
%($a$) where an application is made on the basis that; or
%
%($b$) acting on his own initiative where,
%\end{enumerate}
%the decision to be superseded is a decision of 
%%an appeal tribunal or of a Commissioner 
%%the First-tier Tribunal or of the Upper Tribunal  % Words substituted (3.11.08) by SI 2008/2683 Sch 1 para 102
%an appeal tribunal, the First-tier Tribunal, the Upper Tribunal or of a Child Support Commisisoner  % Words substituted (3.11.08) by SI 2012/1267 reg 4(3)
%that was made in accordance with section 28ZB(4)($b$)  of the Child Support Act\footnote{Section 28ZB was inserted by the Social Security Act 1998 (c.\ 14), section 43.}, in a case where section 28ZB(5) of that Act applies.
%
%(5) A decision, other than a decision made on appeal, may be superseded by a decision made by the Secretary of State—
%\begin{enumerate}\item[]
%($a$) acting on his own initiative, where he is satisfied that the decision was erroneous in point of law; or
%
%($b$) where an application is made on the basis that the decision was erroneous in point of law.
%\end{enumerate}
%
%(6) A decision may be superseded by a decision made by the Secretary of State where he receives an application for the supersession of a decision by way of an application made under section 28G of the Child Support Act.
%
%(7) The cases and circumstances in which a decision may be superseded shall not include any case or circumstance in which a decision may be revised.
%
%(8) Paragraphs (2) to (6) shall not apply in respect of a decision to refuse an application for a maintenance calculation.
%
%(9) For the purposes of section 17 of the Child Support Act, paragraphs (2) to (6) shall apply in relation to any decision of the Secretary of State that an adjustment shall cease or with respect to the adjustment of amounts payable under a maintenance calculation for the purpose of taking account of overpayments of child support maintenance and voluntary payments, whether as originally made or as revised under section 16 of that Act.
%
%\amendment{
%Reg. 6A inserted (3.3.03 for new-rules cases only) by the Child Support (Decisions and Appeals) (Amendment) Regulations 2000 reg. 8 (subject to reg. 1(2)).
%
%Reg. 6A(4A) inserted (5.5.03 for new-rules cases only) by the Social Security and Child Support (Miscellaneous Amendments) Regulations 2003 reg. 3(4).
%
%Words substituted in reg. 6A(4A) (3.11.08) by the Tribunals, Courts and Enforcement Act 2007 (Transitional and Consequential Provisions) Order 2008 Sch. 1 para. 102.

%Words substituted in reg. 6A(4A) (3.11.08) by the Social Security and Child Support (Supersession of Appeal Decisions) Regulations 2012 reg. 4(3).
%}

% Reg 6A substituted (6.4.09) by SI 2009/396 reg 4(4)
\subsubsection[6A. Supersession of child support decisions]{Supersession of child support decisions\\*\emph{2003 scheme only}}

6A.---(1)  This regulation and regulation 6B set out the circumstances in which a decision may be made by the 
%Commission 
Secretary of State  % Words substituted (1.8.12) by SI 2012/2007 Sch para 113(4)(a)
under section 17 of the Child Support Act (decisions superseding earlier decisions).

(2) A decision may be superseded by a decision of the 
%Commission
Secretary of State%  % Words substituted (1.8.12) by SI 2012/2007 Sch para 113(4)(b)
, on an application or acting under 
%its 
the Secretary of State's  % Words substituted (1.8.12) by SI 2012/2007 Sch para 113(4)(b)
own initiative, where—
\begin{enumerate}\item[]
($a$) there has been a relevant change of circumstances since the decision had effect or it is expected that a relevant change of circumstances will occur;

($b$) the decision was made in ignorance of, or was based on a mistake as to, some material fact; or

($c$) the decision was wrong in law (unless it was a decision made on appeal).
\end{enumerate}

(3) The circumstances in which a decision may be superseded include where the relevant change of circumstances causes the maintenance calculation to cease by virtue of paragraph 16 of Schedule 1 to the Child Support Act or where the 
%Commission 
Secretary of State  % Words substituted (1.8.12) by SI 2012/2007 Sch para 113(4)(c)
no longer has jurisdiction by virtue of section 44 of that Act.

(4) A decision may be superseded by a decision of the 
%Commission 
Secretary of State  % Words substituted (1.8.12) by SI 2012/2007 Sch para 113(4)(c)
where the 
%Commission 
Secretary of State  % Words substituted (1.8.12) by SI 2012/2007 Sch para 113(4)(c)
receives an application for a variation of the decision under section 28G of the Child Support Act.

(5) A decision may not be superseded in circumstances where it may be revised.

(6) A decision to refuse an application for a maintenance calculation may not be superseded.

\amendment{
Reg. 6A substituted (6.4.09) by the Child Support (Miscellaneous Amendments) Regulations 2009 reg. 4(4).

Words substituted in reg. 6A(1), (2), (3), (4) (1.8.12) by the Public Bodies (Child Maintenance and Enforcement Commission: Abolition and Transfer of Functions) Order 2012 Sch. para. 113(4).

Reg. 6A omitted (10.12.12 for 2012 scheme cases only) by the Child Support (Meaning of Child and New Calculation Rules) (Consequential and Miscellaneous Amendment) Regulations 2012 reg. 6(3).
}

\subsubsection[6B. Circumstances in which a child support decision may not be superseded]{Circumstances in which a child support decision may not be superseded\\*\emph{2003 scheme only}}

6B.---(1)  Except as provided in paragraph (4), and subject to paragraph (3), a decision of the 
%Secretary of State
%Commission%  % Words substituted (6.4.09) by SI 2009/296 reg 4(5)(a)
Secretary of State%  % Words substituted (1.8.12) by SI 2012/2007 Sch para 113(5)
, 
%appeal tribunal or Child Support Commissioner
%the First-tier Tribunal or the Upper Tribunal%  % Words substituted (3.11.08) by SI 2008/2683 Sch 1 para 103
an appeal tribunal, the First-tier Tribunal, the Upper Tribunal or a Child Support Commissioner%  % Words substituted (3.11.08) by SI 2012/1267 reg 4(4)
, on an application made under regulation %
%6A(3)
6A(2)($a$)%  % Reference substituted (6.4.09) by SI 2009/296 reg 4(5)(b)
, shall not be superseded where the difference between—
\begin{enumerate}\item[]
($a$) the non-resident parent’s net income figure fixed for the purposes of the maintenance calculation in force in accordance with Part I of Schedule 1 to the Child Support Act; and

($b$) the non-resident parent’s net income figure which would be fixed in accordance with a superseding decision,
\end{enumerate}
is less than 5\% of the figure in sub-paragraph ($a$).

(2) In paragraph (1) “superseding decision” means a decision which would supersede the decision subject to the application made under regulation %
%6A(3)
6A(2)($a$)  % Reference substituted (6.4.09) by SI 2009/296 reg 4(5)(b)
but for the application of this regulation.

%(3) Where the application for a supersession is made on more than one ground this regulation shall only apply to the ground relating to the net income of the non-resident parent.

% Reg 6B(3) substituted (16.9.04 for new-rules cases only) by SI 2004/2415 reg 2(2)(a)
(3) Where the application for a supersession is made on more than one ground, if those grounds which do not relate to the net income of the non-resident parent lead to a superseding decision this regulation shall not apply to the ground relating to the net income of that parent.

(4) This regulation shall not apply to a decision under regulation %
%6A(3)
6A(2)($a$)  % Reference substituted (6.4.09) by SI 2009/296 reg 4(5)(b)
where—
\begin{enumerate}\item[]
($a$) the superseding decision is made in consequence of the determination of an application made under section 28G of the Child Support Act;

($b$) the superseding decision affects a variation ground in a decision made under section 11 or 17 of the Child Support Act, whether as originally made or as revised under section 16 of that Act;

($c$) the decision being superseded was made under section 12(2) of the Child Support Act, or was a decision under section 17 of that Act superseding an interim maintenance decision, whether as originally made or as revised under section 16 of that Act;
%
% Reg 6B(4)(d) omitted (6.4.09) by SI 2009/396 reg 4(5)(c)
%($d$) the decision being superseded was a decision that an adjustment shall cease or with respect to the adjustment of amounts payable under maintenance calculations for the purpose of taking account of overpayments of child support maintenance and voluntary payments or was a decision under section 17 of the Child Support Act superseding that decision, whether as originally made or as revised under section 16 of that Act; 
%or  % Word omitted (4.7.11) by SI 2011/1464 reg 2(3)(a)

($e$) the superseding decision takes effect from the dates prescribed in 
%regulation 7B(1) to (3)
%%, (19)  % Words omitted (3.3.03) by SI 2002/1204 reg 2(3)
%or (20)
paragraph 4 of Schedule 3D%  % Words substituted (6.4.09) by SI 2009/396 reg 4(5)(d)
% 
% Reg 6B(4)(f) addd (4.7.11) by SI 2011/1464 reg 2(3)(b)
; or

($f$) a decision is superseded and in relation to that superseding decision a maintenance calculation is made to which paragraph 15 of Schedule 1 to the Child Support Act applies.
\end{enumerate}

% Reg 6B(5) added (16.9.04) by SI 2004/2415 reg 2(2)(b)
(5) Where an application has been made to which paragraph (1) applied (“application $\mathcal{A}$”) and a further application (“application $\mathcal{B}$”) is made for a supersession on a ground other than one relating to the net income of the non-resident parent, the 
%Secretary of State
%Commission  % Words substituted (6.4.09) by SI 2009/296 reg 4(5)(a)
Secretary of State  % Words substituted (1.8.12) by SI 2012/2007 Sch para 113(4)(c)
may make a superseding decision on the basis that application $\mathcal{A}$ was made at the same time as application $\mathcal{B}$.

\amendment{
Reg. 6B inserted (3.3.03 for new-rules cases only) by the Child Support (Decisions and Appeals) (Amendment) Regulations 2000 reg. 8 (subject to reg. 1(2)).

Words omitted in reg. 6B(4)(e) (3.3.03 for new-rules cases only) by the Child Support (Miscellaneous Amendments) Regulations 2002 reg. 2(3).

Reg. 6B(5) added and 6B(3) substituted (16.9.04 for new-rules cases only) by the Child Support (Miscellaneous Amendments) Regulations 2004 reg. 2(2).

Words substituted in reg. 6B(1) (3.11.08) by the Tribunals, Courts and Enforcement Act 2007 (Transitional and Consequential Provisions) Order 2008 Sch. 1 para. 103.

Words substituted in reg. 6B(1) (3.11.08) by the Social Security and Child Support (Supersession of Appeal Decisions) Regulations 2012 reg. 4(4).

Words substituted in reg. 6B and reg. 6B(4)(d) omitted (6.4.09) by the Child Support (Miscellaneous Amendments) Regulations 2009 reg. 4(5).

Reg. 6B(4)(f) added (4.7.11) by the Child Support (Miscellaneous Amendments) Regulations 2011 reg. 2(3).

Words substituted in reg. 6B(1), (5) (1.8.12) by the Public Bodies (Child Maintenance and Enforcement Commission: Abolition and Transfer of Functions) Order 2012 Sch. para. 113(5).

Reg. 6B omitted (10.12.12 for 2012 scheme cases only) by the Child Support (Meaning of Child and New Calculation Rules) (Consequential and Miscellaneous Amendment) Regulations 2012 reg. 6(3).
}

\subsubsection[7. Date from which a decision superseded under section 10 takes effect]{Date from which a decision superseded under section 10 takes effect}

7.—%(1) This regulation contains exceptions to the provisions of section 10(5) as to the date from which a decision under section 10 which supersedes an earlier decision is to take effect.
%
% Reg 7(1) substituted (29.11.99) by SI 1999/3178 Sch 19 para 1(a)
(1) This regulation---
\begin{enumerate}\item[]
%($a$) is, except for paragraph (2)($b$), subject to regulations 26\footnote{\frenchspacing Regulation 26 was amended by S.I. 1988/522, 1989/136 and 1993/1113.} (income support) and 26A\footnote{\frenchspacing Regulation 26A was inserted by S.I. 1996/1460 and amended by S.I. 1998/1174.} (jobseeker’s allowance) of, and paragraph 7\footnote{\frenchspacing Paragraph 7 was substituted by S.I. 1990/2208 and amended by S.I. 1991/387, 1992/247 and 1998/1174.} (date from which superseding decision on ground of change of circumstances takes effect) of Schedule 7 to, the Claims and Payments Regulations; and

% Reg 7(1)(a) subtituted (19.6.00) by SI 2000/1596 reg 17(a)
%($a$) is, except for paragraph (2)($b$), subject to Schedule 3A; and

% Reg 7(1)(a) substituted (7.4.03) by SI 2002/3019 reg 18(a)
($a$) is, except for 
%paragraph (2)($b$)
paragraphs (2)($b$)%
%  % Words inserted (27.7.08) by SI 2008/1554 reg 33(2)(a)
, ($bb$)  % Words inserted by SI 2008/2667 reg 3(4)(a)
or ($be$), (29) and (30)%  % Words substituted (5.5.03) by SI 2003/1050 reg 3(5)(a)
, subject to Schedules 3A% 
%and 3B
, 3B and 3C%  % Words substituted (27.7.08) by SI 2008/1554 reg 33(2)(b)
; and

($b$) contains exceptions to the provisions of section 10(5) as to the date from which a decision under section 10 which supersedes an earlier decision is to take effect.
\end{enumerate}

(2) Where a decision under section 10 is made on the ground that there has been, or it is anticipated that there will be, a relevant change of circumstances since the decision 
%was made
had effect  % Words substituted (5.5.03) by SI 2003/1050 reg 3(5)(b)
or, in the case of an advance award, since the decision was made%  % Words inserted (18.3.05) by SI 2005/337 reg 2(5)(a)
, the decision under section 10 shall take effect—
\begin{enumerate}\item[]
%($a$) where the decision is advantageous to the claimant and the change was notified to an appropriate office within one month of the change occurring or within such longer period as may be allowed under regulation 8 for the claimant’s failure to notify the change on an earlier date—
%\begin{enumerate}\item[]
%(i) subject to head (ii), from the date the change occurred or, where the change does not have effect until a later date, from the first date on which such effect occurs;
%
%(ii) in a case where the date a change of circumstances is to take effect falls to be determined in accordance with regulation 26 or 26A\footnote{\frenchspacing Regulation 26A was inserted by the Social Security (Claims and Payments) (Jobseeker’s Allowance Consequential Amendments) Regulations 1996.} of the Claims and Payments Regulations, the date so determined;
%\end{enumerate}

% Reg 7(2)(a) substituted (29.11.99) by SI 1999/3178 Sch 19 para 1(b)(i)
($a$) from the date the change occurred or, where the change does not have effect until a later date, from the first date on which such effect occurs where---
\begin{enumerate}\item[]
(i) the decision is advantageous to the claimant; and

(ii) the change was notified to an appropriate office within one month of the change occurring or within such longer period as may be allowed under regulation 8 for the claimant’s failure to notify the change on an earlier date;
\end{enumerate}

($b$) where the decision is advantageous to the claimant and the change was notified to an appropriate office more than one month after the change occurred or after the expiry of any such longer period as may have been allowed under regulation 8—
\begin{enumerate}\item[]
(i) in the case of a claimant who is in receipt of income support% 
%or a jobseeker’s allowance 
, jobseeker's allowance% 
%or state pension credit  % Words substituted (7.4.03) by SI 2002/3019 reg 18(b)
, state pension credit or an employment and support allowance  % Words substituted (27.7.08) by SI 2008/1554 reg 33(3)(a)
and benefit is paid in arrears, from the beginning of the benefit week in which the notification was made;

(ii) in the case of a claimant who is in receipt of income support% 
%or a jobseeker’s allowance 
, jobseeker's allowance or state pension credit  % Words substituted (7.4.03) by SI 2002/3019 reg 18(b)
and benefit is paid in advance and the date of notification is the first day of a benefit week from that date and otherwise, from the beginning of the benefit week following the week in which the notification was made; or

(iii) in any other case, the date of notification of the relevant change of circumstances; or
\end{enumerate}

% Reg 7(2)(bb) inserted (19.6.00) by SI 2000/1596 reg 17(b)
%($bb$) where the decision is advantageous to the claimant and is made on the Secretary of State’s own initiative, from the date on which the Secretary of State commenced action with a view to supersession;

% Reg 7(2)(bb) substituted by SI 2008/2667 reg 3(4)(b)
($bb$) where the decision is advantageous to the claimant and is made on the Secretary of State’s own initiative—
\begin{enumerate}\item[]
(i) except where paragraph (ii) applies, from the beginning of the benefit week in which the Secretary of State commenced action with a view to supersession; or

(ii) in the case of a claimant who is in receipt of income support, jobseeker’s allowance or state pension credit where benefit is paid in advance and the Secretary of State commenced action with a view to supersession on a day which was not the first day of the benefit week, from the beginning of the benefit week following the week in which the Secretary of State commenced such action;
\end{enumerate}

% Reg 7(2)(bc) inserted (2.10.06) by SI 2006/2377 reg 3(2)
%($bc$) 
%subject to 
%%sub-\hspace{0pt}paragraph ($bd$)
%paragraph (2A)%  % Words substituted (19.5.08) by SI 2008/1042 reg 2(a)
%,  % Words inserted (24.9.07) by SI 2007/2470 reg 3(6)
%where the decision is advantageous to the claimant and is made in connection with the cessation of payment of a carer’s allowance, the day after the last day for which that allowance was paid;

% Reg 7(2)(bc) substituted by SI 2008/2667 reg 3(4)(c)
($bc$) where—
\begin{enumerate}\item[]
(i) the claimant is a disabled person or a disabled person’s partner;

(ii) the decision is advantageous to the claimant; and

(iii) the decision is made in connection with the cessation of payment of a carer’s allowance relating to that disabled person,
\end{enumerate}
the day after the last day for which carer’s allowance was paid to a person other than the claimant or the claimant’s partner;

% Reg 7(2)(bd) inserted (24.9.07) by SI 2007/2470 reg 3(7), omitted (19.5.08) by SI 2008/1042 reg 2(b)
%($bd$) sub-paragraph ($bc$)  shall only apply to the disabled person whose benefit is affected by the cessation of payment of carer’s allowance;

% Reg 7(2)(be) inserted (27.7.08) by SI 2008/1554 reg 33(3)(b)
($be$) in the case of a claimant who is in receipt of an employment and support allowance and the claimant makes an application which contains an express statement that he is terminally ill within the meaning of regulation 2(1) of the Employment and Support Allowance Regulations, from the date the claimant became terminally ill;

($c$) where the decision is not advantageous to the claimant—
\begin{enumerate}\item[]
% Reg 7(2)(c)(i) omitted (29.11.99) by SI 1999/3178 Sch 19 para 1(b)(ii)
%(i) in a case where the date a change of circumstances is to take effect falls to be determined in accordance with regulation 26 or 26A of the Claims and Payments Regulations, the date so determined; or

%(ii) in any other case, from the date of the change.

% Reg 7(2)(c)(ii), (iii) substituted for reg 7(2)(c)(ii) (5.7.99) by SI 1999/1623 reg 4
(ii) in the case of a disability benefit decision, or an incapacity benefit decision where there has been an incapacity determination 
or an employment and support allowance decision where there has been a limited capability for work determination  % Words inserted by SI 2009/1490 reg 3(3)
(whether before or after the decision), where the Secretary of State is satisfied that in relation to a disability determination embodied in or necessary to the disability benefit decision, or the incapacity determination
or an employment and support allowance decision where there has been a limited capability for work determination%  % Words inserted by SI 2009/1490 reg 3(3)
, the claimant or payee failed to notify an appropriate office of a change of circumstances which regulations under the Administration Act required him to notify, and the claimant or payee, as the case may be, knew or could reasonably have been expected to know that the change of circumstances should have been notified---
\begin{enumerate}\item[]
($aa$) from the date on which the claimant or payee, as the case may be, ought to have notified the change of circumstances, or

($bb$) if more than one change has taken place between the date from which the decision to be superseded took effect and the date of the superseding decision, from the date on which the first change ought to have been notified, or
\end{enumerate}

% Reg 7(2)(c)(iii) omitted (10.4.06) by SI 2006/832 reg 5(3)(a)(i)
%(iii) in any other case, except in the case of a decision which supersedes a disability benefit decision, or an incapacity benefit decision where there has been an incapacity determination (whether before or after the decision), from the date of the change.

% Reg 7(2)(c)(iv), (v) added (10.4.06) by SI 2006/832 reg 5(3)(a)(ii)
(iv) in the case of a disability benefit decision, where the change of circumstances is not in relation to the disability determination embodied in or necessary to the disability benefit decision, from the date of the change; or

(v) in any other case, except in the case of a decision which supersedes a disability benefit decision, from the date of the change.
\end{enumerate}
\end{enumerate}

% Reg 7(2A) inserted (19.5.08) by SI 2008/1042 reg 2(c), omitted by SI 2008/2667 reg 3(4)(d)
%(2A) Paragraph (2)($bc$)  shall only apply to the disabled person whose benefit is affected by the cessation of payment of carer’s allowance.

%(3) For the purposes of paragraphs (2) and (8) “benefit week” has the same meaning as in regulation 2(1) of the Income Support Regulations or, as the case may be, regulation 1(3) of the Jobseeker’s Regulations
%or regulation 1(2) of the State Pension Credit Regulations.  % Words added (7.4.03) by SI 2002/3019 reg 18(c)

% Reg 7(3) substituted (27.7.08) by SI 2008/1554 reg 33(4)
(3) For the purposes of paragraphs (2) and (8) “benefit week” has the same meaning, as the case may be, as in—
\begin{enumerate}\item[]
($a$) regulation 2(1) of the Income Support Regulations;

($b$) regulation 1(3) of the Jobseeker’s Allowance Regulations;

($c$) regulation 1(2) of the State Pension Credit Regulations; or

($d$) regulation 2(1) of the Employment and Support Allowance Regulations.
\end{enumerate}

(4) In paragraph (2) a decision which is to the advantage of the claimant includes a decision specified in regulation 30(2)($a$) to ($f$).

%(5) Where the Secretary of State supersedes 
%or the Board supersede  % Words inserted (5.10.99) by SI 1999/2570 reg 10
%a decision made by an appeal tribunal or a Commissioner on the grounds specified in regulation 6(2)($c$) (grounds of ignorance of, or mistake as to, a material fact), the decision under section 10 shall take effect—
%\begin{enumerate}\item[]
%($a$) in a case where, as a result of that ignorance of or mistake as to some material fact, the decision was more advantageous to the claimant than it would otherwise have been but for that ignorance or mistake, from the date on which the decision of the appeal tribunal or the Commissioner took, or was to take effect; or
%
%($b$) in any other case, from the date of the decision under section 10.
%\end{enumerate}

% Reg 7(5) substituted (19.6.00) by SI 2000/1596 reg 17(c)
(5) Where the Secretary of State supersedes a decision made by 
%an appeal tribunal or a Commissioner 
%the First-tier Tribunal or the Upper Tribunal  % Words substituted (3.11.08) by SI 2008/2683 Sch 1 para 104(a)(i)
an appeal tribunal, the First-tier Tribunal, the Upper Tribunal or a Commissioner  % Words substituted (3.11.08) by SI 2012/1267 reg 4(5)(a)
on the grounds specified in regulation 6(2)($c$)%
(i)  % Word inserted (5.5.03) by SI 2003/1050 reg 3(5)(c)
(ignorance of, or mistake as to, a material fact), the decision under section 10 shall take effect, in a case where, as a result of that ignorance of or mistake as to material fact, the decision to be superseded was more advantageous to the claimant than it would otherwise have been and which either—
\begin{enumerate}\item[]
($a$) does not relate to a disability benefit decision or an incapacity benefit decision where there has been an incapacity determination; or

($b$) relates to a disability benefit decision or an incapacity benefit decision where there has been an incapacity determination, and the Secretary of State is satisfied that at the time the decision was made the claimant or payee knew or could reasonably have been expected to know of the fact in question and that it was relevant to the decision,
\end{enumerate}
from the date on which the decision of 
%the appeal tribunal or the Commissioner 
%the First-tier Tribunal or the Upper Tribunal  % Words substituted (3.11.08) by SI 2008/2683 Sch 1 para 104(a)(ii)
an appeal tribunal, the First-tier Tribunal, the Upper Tribunal or a Commissioner  % Words substituted (3.11.08) by SI 2012/1267 reg 4(5)(a)
took, or was to take, effect.

(6) Any decision made under section 10 in consequence of a decision which is a relevant determination for the purposes of section 27 shall take effect as from the date of the relevant determination.

% Reg 7(6A) inserted (18.3.05) by SI 2005/337 reg 2(5)(b)
(6A) Where—
\begin{enumerate}\item[]
($a$) there is a decision which is a relevant determination for the purposes of section 27 and the Secretary of State makes a benefit decision of the kind specified in section 27(1)($b$);

($b$) there is an appeal against the determination;

($c$) after the benefit decision payment is suspended in accordance with regulation 16(1) and (3)($b$)(ii); and

($d$) on appeal a court, within the meaning of section 27, reverses the determination in whole or in part,
\end{enumerate}
a consequential decision by the Secretary of State under section 10 which supersedes his earlier decision under sub-paragraph ($a$)  shall take effect from the date on which the earlier decision took effect.

%(7) A decision to which regulation 6(2)($e$) applies may be made so as to take effect as from the date on which the decision which has been superseded had effect, or at any time thereafter which is reasonable in the particular circumstances of the case.

% Reg 7(7) substituted (19.6.00) by SI 2000/1596 reg 17(d)
%(7) A decision which falls to be superseded under regulation 6(2)($e$)  shall be superseded as from the date on which the claimant or member of his family becomes entitled to and receives the relevant benefit or increase in benefit referred to in regulation 6(2)($e$)(ii).

% Reg 7(7) substituted (2.4.02) by SI 2002/428 reg 4(4)
%(7) A decision which is superseded in accordance with regulation 6(2)($e$)  
%or ($ee$)  % Words inserted (18.3.05) by SI 2005/337 reg 2(5)(c)
%shall be superseded from the date on which entitlement arises to the other relevant benefit referred to in regulation 6(2)($e$)(ii) 
%or ($ee$)  % Words inserted (18.3.05) by SI 2005/337 reg 2(5)(c)
%or to an increase in the rate of that other relevant benefit.

% Reg 7(7) substituted (10.4.06) by SI 2006/832 reg 5(3)(b)
(7) A decision which is superseded in accordance with regulation 6(2)($e$)  or ($ee$)  shall be superseded—
\begin{enumerate}\item[]
($a$) subject to sub-paragraph ($b$), from the date on which entitlement arises to the other relevant benefit referred to in regulation 6(2)($e$)(ii)  or ($ee$)  or to an increase in the rate of that other relevant benefit; or

($b$) where the claimant or his partner—
\begin{enumerate}\item[]
(i) is not a severely disabled person for the purposes of section~135(5) of the Contributions and Benefits Act (the applicable amount) or section 2(7) of the State Pension Credit Act (guarantee credit)
or paragraph 6 of Schedule 4 to the Employment and Support Allowance Regulations;  % Words inserted (27.7.08) by SI 2008/1554 reg 33(5)(a)

(ii) by virtue of his having—
\begin{enumerate}\item[]
($aa$) a non-dependant as defined by regulation 3 of the Income Support Regulations%
, regulation 2 of the Jobseeker’s Allowance Regulations  % Words inserted by SI 2012/757 reg 17
or regulation 71 of the Employment and Support Allowance Regulations%  % Words inserted (27.7.08) by SI 2008/1554 reg 33(5)(b)
; or

($bb$) a person residing with him for the purposes of paragraph~1 of Schedule 1 to the State Pension Credit Regulations whose presence may not be ignored in accordance with paragraph 2 of that Schedule,
\end{enumerate}
at the date the superseded decision would, but for this sub-paragraph, have had effect,
\end{enumerate}
from the date on which the claimant or his partner ceased to have a non-dependant or person residing with him or from the date on which the presence of that person was first ignored.
\end{enumerate}

\begin{sloppypar}
% Reg 7(7A) inserted (6.4.06) by SI 2005/2677 reg 9(5)
(7A) Where a decision is superseded in accordance with regulation~6(2)($o$)
or ($oa$)%  % Words inserted by SI 2015/1985 art 18(6)
, the superseding decision shall take effect from the day on which a lump sum, or a payment on account of a lump sum, is paid or repaid if that day is the first day of the benefit week but, if it is not, from the next following such day.
\end{sloppypar}

%(8) A decision to which regulation 6(2)($f$) applies may be so as to take effect from—
%\begin{enumerate}\item[]
%($a$) except where sub-paragraph ($b$) applies, the date immediately following the end of benefit week in which the decision under section 10 was made; or
%
%($b$) where in accordance with regulation 26A(1) of the Claims and Payments Regulations, a jobseeker’s allowance is paid otherwise than fortnightly in arrears, and notwithstanding the provisions of regulation 69 of the Jobseeker’s Allowance Regulations, from the day immediately following the end of the last benefit week in respect of which a jobseeker’s allowance was paid.
%\end{enumerate}

% Reg 7(8) substituted (18.10.99) by SI 1999/2677 reg 8
%(8) A decision to which regulation 6(2)($f$)  applies shall take effect—
%\begin{enumerate}\item[]
%% Reg 7(8)(za) inserted by SI 2010/509 reg 3(4)
%($za$) where regulation 27A of the Jobseeker’s Allowance Regulations applies, as from the beginning of the period specified in regulation 27B of those Regulations;
%
%($a$) where section 19(2) 
%or 20A(3)  % Words inserted (19.3.01) by SI 2000/1982 reg 5(c)(i)
%of the Jobseekers Act applies, as from the beginning of the period specified in regulation 69 of the Jobseeker’s Allowance Regulations; or
%
%($b$) where section 19(3) 
%or 20A(4)  % Words inserted (19.3.01) by SI 2000/1982 reg 5(c)(i)
%of the Jobseekers Act applies, as from the beginning of the period determined in accordance with that subsection.
%\end{enumerate}

% Reg 7(8) substituted by SI 2012/2568 reg 6(4)(a)
(8) A decision to which regulation 6(2)($f$) applies shall take effect from the beginning of the period specified in regulation 69(6) of the Jobseeker’s Allowance Regulations.

% Reg 7(8ZA) inserted temporarily by SI 2010/1222 reg 20(c) (expired 21.11.13)
%(8ZA) A decision to which regulation 6(2)($fa$) applies shall take effect as from the beginning of the period specified in regulation 8(11) of the Jobseeker’s Allowance (Work for Your Benefit Pilot Scheme) Regulations 2010.

% Reg 7(8ZA) inserted by SI 2011/688 reg 18(c)
%(8ZA) A decision to which regulation 6(2)($fa$) applies shall take effect on the day specified in regulation 8(6)($a$) or ($b$) of the Jobseeker’s Allowance (Mandatory Work Activity Scheme) Regulations 2011.

% Reg 7(8ZA) substituted by SI 2012/2568 reg 6(4)(b)
(8ZA) A decision to which regulation 6(2)($fa$) applies shall take effect from the beginning of the period specified in regulation 69A(3) of the Jobseeker’s Allowance Regulations.

% Reg 7(8ZB) inserted by SI 2011/917 reg 17, omitted by SI 2012/2568 reg 6(4)(c)
%(8ZB) A decision to which regulation 6(2)($fa$) applies shall take effect on the day specified in regulation 8(9)($a$) or ($b$) of the Jobseeker’s Allowance (Employment, Skills and Enterprise Scheme) Regulations 2011.

\begin{sloppypar}
% Reg 7(8A) inserted by SI 2008/2667 reg 3(4)(e)
(8A) Where a decision is superseded in accordance with regulation~6(2)($s$), the superseding decision shall take effect from the date on which the late or unpaid contribution is treated as paid.
\end{sloppypar}

%(9) In any case relating to attendance allowance or disability living allowance in which the decision was made under section 10 on the grounds of a relevant change of circumstances by virtue of regulation 6(2)($a$)(i) and the decision is advantageous to the claimant, the decision shall take effect as from whichever is the later of—
%\begin{enumerate}\item[]
%($a$) the date declared by the Secretary of State to be the date on which the change of circumstances occurred;
%
%($b$) where more than one change has occurred between the date of the decision to be superseded (“the original decision”) and the date of the application, or, as the case may be, the date the Secretary of State determines on his own initiative to supersede the original decision, the date declared by the Secretary of State to be the date on which the most recent change took effect; or
%
%($c$) where the claimant notifies the change within one month of the date he first satisfies the conditions in, for the period of time specified in, section 65(1)($b$) of the Contributions and Benefits Act or, as the case may be, section 72(2)($a$) or 73(9)($a$) of that Act, following the change or most recent change of circumstances which gave rise to the decision under section 10, the first pay day (as specified in Schedule 6 to the Claims and Payments Regulations) after the requirement is first satisfied.
%\end{enumerate}

% Reg 7(9) substituted (17.2.00) by SI 2000/119 reg 2
(9) 
Except where paragraph (9A) applies,  % Words inserted by SI 2011/2426 reg 2(a)
a decision relating to attendance allowance or disability living allowance which is advantageous to the claimant and which is made under section 10 on the basis of a relevant change of circumstances shall take effect from—
\begin{enumerate}\item[]
%($a$) where the decision is made on the Secretary of State’s own initiative, 
%%the date of that decision;
%the date on which the Secretary of State commenced action with a view to supersession;  % Words substituted (19.6.00) by SI 2000/1596 reg 17(e)

% Reg 7(9)(a) substituted (5.5.03) by SI 2003/1050 reg 3(5)(d)
($a$) where the decision is made on the Secretary of State’s own initiative—
\begin{enumerate}\item[]
(i) the date on which the Secretary of State commenced action with a view to supersession; or

(ii) subject to paragraph (30), in a case where the relevant circumstances are that there has been a change in the legislation in relation to attendance allowance or disability living allowance, the date on which that change in the legislation had effect;
\end{enumerate}

($b$) where—
\begin{enumerate}\item[]
(i) the change is relevant to the question of entitlement to a particular rate of benefit; and

(ii) the claimant notifies the change before a date one month after he satisfied the conditions of entitlement to that rate or within such longer period as may be allowed under regulation 8,
\end{enumerate}
the 
%first pay day (as specified in Schedule 6 to the Claims and Payments Regulations\footnote{\frenchspacing S.I. 1987/1968; the relevant amending instrument is S.I. 1991/2741.}) after 
date on which  % Words substituted by SI 2008/2667 reg 3(4)(f)
he satisfied those conditions;

($c$) where—
\begin{enumerate}\item[]
(i) the change is relevant to the question of whether benefit is payable; and

(ii) the claimant notifies the change before a date one month after the change or within such longer period as may be allowed under regulation 8,
\end{enumerate}
the 
%first pay day (as specified in Schedule 6 to the Claims and Payments Regulations) after 
date on which  % Words substituted by SI 2008/2667 reg 3(4)(f)
the change occurred; or

($d$) in any other case, the date of the application for the superseding decision.
\end{enumerate}

% Reg 7(9A) inserted by SI 2011/2426 reg 2(b)
(9A) Where—
\begin{enumerate}\item[]
($a$) on or after 8th March 2001, the claimant had an award of attendance allowance, carer’s allowance, or the care component of disability living allowance;

($b$) the Secretary of State made a superseding decision in accordance with regulation 6(2)($a$) to end that award on the ground that there had been, or it was anticipated that there would be, a relevant change of circumstances as a result of the claimant moving, or planning to move, from Great Britain to an EEA state or Switzerland; and

($c$) the Secretary of State supersedes that decision in accordance with regulation 6(2)($b$)(i) on the ground that it was erroneous in point of law,
\end{enumerate}
the superseding decision referred to in sub-paragraph ($c$) shall take effect from 18th October 2007.

(10) A decision as to an award of incapacity benefit, which is made under section 10 because section 30B(4) of the Contributions and Benefits Act\footnote{\frenchspacing Section 30B was inserted by section 2(1) of the Social Security (Incapacity for Work) Act 1994 (c. 18).} applies to the claimant, shall take effect as from the date on which he became entitled to the highest rate of the care component of disability living allowance.

(11) A decision as to an award of incapacity benefit or severe disablement allowance, which is made under section 10 because the claimant is to be treated as incapable of work under regulation 10 of the Social Security (Incapacity for Work) (General) Regulations 1995\footnote{\frenchspacing S.I. 1995/311; relevant amending instruments are S.I. 1995/987, S.I. 1996/3207 and S.I. 1997/1009.} (certain persons with a severe condition to be treated as incapable of work), shall take effect as from the date he is to be treated as incapable of work.

(12) Where this paragraph applies, a decision under section 10 may be made so as to take effect as from such date not more than eight weeks before—
\begin{enumerate}\item[]
($a$) the application for supersession; or

($b$) where no application is made, the date on which the decision under section 10 is made,
\end{enumerate}
as is reasonable in the particular circumstances of the case.

(13) Paragraph (12) applies where—
\begin{enumerate}\item[]
($a$) the effect of a decision under section 10 is that there is to be included in a claimant’s applicable amount an amount in respect of a loan which qualifies under—
\begin{enumerate}\item[]
(i) paragraph 15 or 16 of Schedule 3 to the Income Support Regulations; or

(ii) paragraph 14 or 15 of Schedule 2 to the Jobseeker’s Allowance Regulations;
% and
or

    (iii) 
    paragraph 11 or 12 of Schedule II to the State Pension Credit Regulations; 
%and
or  % Word substituted (27.7.08) by SI 2008/1554 reg 33(6)(a)
% Words substituted (7.4.03) by SI 2003/3019 reg 18(d)

% Reg 7(13)(a)(iv) inserted (27.7.08) by SI 2008/1554 reg 33(6)(b)
(iv) paragraph 16 or 17 of Schedule 6 to the Employment and Support Allowance Regulations; and
\end{enumerate}

($b$) that decision could not have been made earlier because information necessary to make that decision, requested otherwise than in accordance with paragraph 10(3)($b$) of Schedule 9A to the Claims and Payments Regulations\footnote{\frenchspacing Schedule 9A was inserted by S.I. 1992/1026.} (annual requests for information), had not been supplied to the Secretary of State by the lender.
\end{enumerate}

(14) Subject to paragraph (23), where a claimant is in receipt of income support and his applicable amount includes an amount determined in accordance with Schedule 3 to the Income Support Regulations (housing costs), and there is a reduction in the amount of eligible capital owing in connection with a loan which qualifies under paragraph 15 or 16 of that Schedule, a decision made under section 10 shall take effect—
\begin{enumerate}\item[]
($a$) on the first anniversary of the date on which the claimant’s housing costs were first met under that Schedule; or

($b$) where the reduction in eligible capital occurred after the first anniversary of the date referred to in sub-paragraph ($a$), on the next anniversary of that date following the date of the reduction.
\end{enumerate}

(15) Where a claimant is in receipt of income support and payments made to that claimant which fall within paragraph 29 or 30(1)($a$) to ($c$) of Schedule 9 to the Income Support Regulations have been disregarded in relation to any decision under section 8 or 10 and there is a change in the amount of interest payable—
\begin{enumerate}\item[]
($a$) on a loan qualifying under paragraph 15 or 16 of Schedule 3 to those Regulations to which those payments relate; or

($b$) on a loan not so qualifying which is secured on the dwelling occupied as the home to which those payments relate,
\end{enumerate}
a decision under section 10 which is made as a result of that change in the amount of interest payable shall take effect on whichever of the dates referred to in paragraph (16) is appropriate in the claimant’s case.

(16) The date on which a decision under section 10 takes effect for the purposes of paragraph (15) is—
\begin{enumerate}\item[]
($a$) the date on which the claimant’s housing costs are first met under paragraph 6(1)($a$), 8(1)($a$) or 9(2)($a$) of Schedule 3 to the Income Support Regulations; or

($b$) where the change in the amount of interest payable occurred after the date referred to in sub-paragraph ($a$), on the date of the next alteration in the standard rate following the date of that change.
\end{enumerate}

(17) In paragraph (16), “standard rate” has the same meaning as it has in paragraph 1(2) of Schedule 3 to the Income Support Regulations.

% Reg 7(17A) inserted (7.4.03) by SI 2002/3019 reg 18(e)
(17A) For the purposes of state pension credit—
\begin{enumerate}\item[]
($a$) paragraph (14) shall apply as if the reference to—
\begin{enumerate}\item[]
(i) “income support and his applicable amount” was a reference to “state pension credit and his appropriate minimum guarantee”;

(ii) “Schedule 3 to the Income Support Regulations” was a reference to “Schedule II to the State Pension Credit Regulations”; and

(iii) “paragraph 15 or 16” was a reference to “paragraph 11 or~12”;
\end{enumerate}

($b$) paragraphs (15) to (17) shall not apply.
\end{enumerate}

% Reg 7(17B), (17C) inserted (7.4.03) by SI 2002/3197 reg 6(a)
%(17B) Subject to paragraph (23), where a claimant who is in receipt of state pension credit or his partner is aged 65 or over, the claimant’s appropriate minimum guarantee includes an amount determined in accordance with Schedule II to the State Pension Credit Regulations and there is a change of circumstances referred to in paragraph (17C), a decision made under section 10 shall take effect—
%\begin{enumerate}\item[]
%($a$) on the first anniversary of the date on which the claimant’s housing costs were first met under that Schedule; or
%
%($b$) where the change occurred after the first anniversary of the date referred to in sub-paragraph ($a$), on the next anniversary of that date following the date of the change.
%\end{enumerate}
%
%(17C) Paragraph (17B) applies in a case where a non-dependant commences residing with the claimant or there is an increase in a non-dependant’s income.

% Reg 7(17B), (17C) substituted (5.4.04) by SI 2004/647 reg 2(a)
(17B) Paragraph (17C) applies where—
\begin{enumerate}\item[]
($a$) a claimant is awarded state pension credit;

($b$) the claimant or his partner is aged 65 or over;

($c$) his appropriate minimum guarantee (as defined by the State Pension Credit Act) includes housing costs determined in accordance with Schedule II to the State Pension Credit Regulations; and

($d$) after the date from which sub-paragraph ($c$)  applies—
\begin{enumerate}\item[]
(i) a non-dependant (as defined in that Schedule) begins to reside with the claimant; or

(ii) 
%a non-dependant’s income increases and this affects 
there has been a change of circumstances in respect of a non-dependant and this reduces  % Words substituted by SI 2004/2327 reg 4
the applicable amount of the claimant’s housing costs.
\end{enumerate}
\end{enumerate}

(17C) In the circumstances specified in paragraph (17B) a decision made under section 10 shall take effect—
\begin{enumerate}\item[]
($a$) where there is more than one change of the kind specified in paragraph (17B)($d$)  in respect of the same non-dependant within the same 26 week period, 26 weeks after the date on which the first such change occurred; and

($b$) in any other circumstances, 26 weeks after the date on which a change specified in paragraph (17B)($d$)  occurred.
\end{enumerate}

% Reg 7(17D)--(17H) inserted (27.7.08) by SI 2008/1554 reg 33(7)
(17D) Except in a case where paragraph (23) applies, where a claimant is in receipt of an employment and support allowance and his applicable amount includes an amount determined in accordance with Schedule 6 to the Employment and Support Allowance Regulations (housing costs), and there is a reduction in the amount of eligible capital owing in connection with a loan which qualifies under paragraph 16 or 17 of that Schedule, a decision made under section 10 shall take effect—
\begin{enumerate}\item[]
($a$) on the first anniversary of the date on which the claimant’s housing costs were first met under that Schedule; or

($b$) where the reduction in eligible capital occurred after the first anniversary of the date referred to in sub-paragraph ($a$), on the next anniversary of that date following the date of the reduction.
\end{enumerate}

(17E) Where a claimant is in receipt of an employment and support allowance and payments made to that claimant which fall within paragraph 31 or 32(1)($a$)  to ($c$)  of Schedule 8 to the Employment and Support Allowance Regulations have been disregarded in relation to any decision under section 8 or 10 and there is a change in the amount of interest payable—
\begin{enumerate}\item[]
($a$) on a loan qualifying under paragraph 16 or 17 of Schedule 6 to those Regulations to which those payments relate; or

($b$) on a loan not so qualifying which is secured on the dwelling occupied as the home to which those payments relate,
\end{enumerate}
a decision under section 10 which is made as a result of that change in the amount of interest payable shall take effect on whichever of the dates referred to in paragraph (17F) is appropriate in the claimant’s case.

(17F) The date on which a decision under section 10 takes effect for the purposes of paragraph (17E) is—
\begin{enumerate}\item[]
($a$) the date on which the claimant’s housing costs are first met under paragraph 8(1)($a$), 9(1)($a$)  or 10(2)($a$)  of Schedule 6 to the Employment and Support Allowance Regulations; or

($b$) where the change in the amount of interest payable occurred after the date referred to in sub-paragraph ($a$), on the date of the next alteration in the standard rate following the date of that change.
\end{enumerate}

(17G) In paragraph (17F) “standard rate” has the same meaning as it has in paragraph 13(2) of Schedule 6 to the Employment and Support Allowance Regulations.

\begin{sloppypar}
(17H) Where the decision is superseded in accordance with regulation~6(2)($a$)(i)  and the relevant circumstances are that the claimant has a non-dependant who has become entitled to main phase employment and support allowance, the superseding decision shall take effect from the date the main phase employment and support allowance is first paid to the non-dependant.
\end{sloppypar}

(18) Subject to paragraph (24) and, except in a case to which paragraph~(23) applies, where a claimant is in receipt of a jobseeker’s allowance and his applicable amount includes an amount determined in accordance with Schedule 2 to the Jobseeker’s Allowance Regulations (housing costs), and there is a reduction in the amount of eligible capital owing in connection with a loan which qualifies under paragraph 14 or 15 of that Schedule, a decision under section 10 made as a result of that reduction shall take effect—
\begin{enumerate}\item[]
($a$) on the first anniversary of the date on which the claimant’s housing costs were first met under that Schedule; or

($b$) where the reduction in eligible capital occurred after the first anniversary of the date referred to in sub-paragraph ($a$), on the next anniversary of that date following the date of the reduction.
\end{enumerate}

(19) Where a claimant is in receipt of a jobseeker’s allowance and payments made to that claimant which fall within paragraph 30 or 31(1)($a$) to ($c$) of Schedule 7 to the Jobseeker’s Allowance Regulations have been disregarded in relation to any decision under section 8 or 10 and there is a change in the amount of interest payable—
\begin{enumerate}\item[]
($a$) on a loan qualifying under paragraph 14 or 15 of Schedule 2 to those Regulations to which those payments relate; or

($b$) on a loan not so qualifying which is secured on the dwelling occupied as the home to which those payments relate,
\end{enumerate}
any decision under section 10 which is made as a result of that change in the amount of interest payable shall take effect on whichever of the dates referred to in paragraph (20) is appropriate in the claimant’s case.

(20) The date on which a decision under section 10 takes effect for the purposes of paragraph (19) is—
\begin{enumerate}\item[]
($a$) the date on which the claimant’s housing costs are first met under paragraph 6(1)($a$), 7(1)($a$) or 8(2)($a$) of Schedule 2 to the Jobseeker’s Allowance Regulations; or

($b$) where the changes in the amount of interest payable occurred after the date referred to in sub-paragraph ($a$), on the date of the next alteration in the standard rate following the date of that change.
\end{enumerate}

(21) In paragraph (20), “standard rate” has the same meaning as it has in paragraph 1(2) of Schedule 2 to the Jobseeker’s Allowance Regulations.

(22) Where—
\begin{enumerate}\item[]
($a$) a claimant was paid benefit in respect of 6th October 1996 in accordance with an award of income support;

($b$) that claimant’s applicable amount includes an amount determined in accordance with Schedule 3 to the Income Support Regulations (housing costs);

($c$) that claimant is treated as having been awarded a jobseeker’s allowance by virtue of regulation 7 of the Jobseeker’s Allowance (Transitional Provisions) Regulations 1996\footnote{\frenchspacing S.I. 1996/2567.} (jobseeker’s allowance to replace income support and unemployment benefit); and

($d$) a decision is made under section 10 in consequence of a reduction in the amount of eligible capital owing in connection with a loan which qualifies under paragraph 15 or 16 of Schedule 3 to the Income Support Regulations,
\end{enumerate}
the decision under section 10 referred to in sub-paragraph ($d$) shall take effect on the next anniversary of the date on which housing costs were first met which occurs after the reduction.

%(23) Where, in any case to which paragraph (14) or (18) applies, a claimant has been continuously in receipt of, or treated as having been continuously in receipt of income support or a jobseeker’s allowance, or one of those benefits followed by the other, and he or his partner continues to receive either benefit, the anniversary to which those paragraphs refer shall be the anniversary of the earliest date on which benefit (whether income support or a jobseeker’s allowance) in respect of those mortgage interest costs became payable.

% Reg 7(23) substituted (7.4.03) by SI 2002/3197 reg 6(b)
(23) Where, in any case to which paragraph (14), (17A)%
%, (17B)  % Words omitted (5.4.04) by SI 2004/647 reg 2(b)
, (17D)  % Words inserted (27.7.08) by SI 2008/1554 reg 33(8)(a)
or (18) applies, a claimant has been continuously in receipt of, or treated as having been continuously in receipt of income support, a jobseeker’s allowance% 
, an employment and support allowance  % Words inserted (27.7.08) by SI 2008/1554 reg 33(8)(b)
or state pension credit, or one of those benefits followed by the other, and he or his partner continues to receive any of those benefits, the anniversary to which those paragraphs refer shall be—
\begin{enumerate}\item[]
($a$) in the case of income support% 
%or jobseeker’s allowance
, jobseeker’s allowance or employment and support allowance%  % Words substituted (27.7.08) by SI 2008/1554 reg 33(8)(c)
, the anniversary of the earliest date on which benefit in respect of those mortgage interest costs became payable;

($b$) in the case of state pension credit, the relevant anniversary date determined in accordance with paragraph 7 of Schedule II to the State Pension Credit Regulations.
\end{enumerate}

(24) Where—
\begin{enumerate}\item[]
($a$) it has been determined that the amount of a jobseeker’s allowance payable to a young person is to be reduced under regulation~63 of the Jobseeker’s Allowance Regulations because paragraph~(1)($b$)(iii),~($c$),~($d$), ($e$) or ($f$) of that regulation (reduced payments under section 17 of the Jobseekers Act) applied in his case; and

($b$) the decision made in consequence of sub-paragraph ($a$) falls to be superseded by a decision under section 10 because the Secretary of State has subsequently issued a certificate under section 17(4) of the Jobseekers Act with respect to the failure in question,
\end{enumerate}
the decision under section 10 shall take effect as from the same date as the decision made in consequence of sub-paragraph ($a$) has effect.

% Reg 7(25), (26) added (3.4.00) by SI 2000/897 Sch 6 para 5
%(25) In a case where a decision (“the first decision”) has been made that a person failed without good cause to take part in a work-focused interview, the decision under section 10 shall take effect as from the first day of the benefit week to commence for that person following the date of the first decision.

% Reg 7(25) substituted (12.4.04) by SI 2003/1886 reg 15(5)
(25) In a case where a decision (“the first decision”) has been made that a person failed without good cause to take part in a work-focused interview, the decision under section 10 shall take effect as from—
\begin{enumerate}\item[]
($a$) the first day of the benefit week to commence for that person following the date of the first decision; or

($b$) in a case where a partner has failed without good cause to take part in a work-focused interview 
%under the Social Security (Jobcentre Plus Interviews for Partners) Regulations 2003
in accordance with regulations made under section 2AA of the Administration Act\footnote{1992 c. 5. Section 2AA was inserted by section 49 of the Employment Act 2002 (c. 22).}%  % Words substituted (26.4.04) by SI 2004/959 reg 24(4)(a)
—
\begin{enumerate}\item[]
(i) the first day of the benefit week to commence for the claimant 
%(as defined in regulation 2(1) of those Regulations) 
(meaning the person who has been awarded a benefit within section~2AA(2) of the Administration Act at a higher rate referable to that partner)  % Words substituted (26.4.04) by SI 2004/959 reg 24(4)(b)
following the date of the first decision; or

\looseness=-1
(ii) if that date arises five days or less after the day on which the first decision was made, as from the first day of the second benefit week to commence for the claimant following the date of the first decision.
\end{enumerate}
\end{enumerate}

(26) In paragraph (25), “benefit week” means any period of 7 days corresponding to the week in respect of which the relevant social security benefit is due to be paid.

% Reg 7(27) added (15.10.01) by SI 2001/1711 reg 2(2)(d), omitted by SI 2010/424 reg 4(5)
%(27) A decision to which regulation 6(2)($i$) applies shall take effect from the beginning of the period specified—
%\begin{enumerate}\item[]
%($a$) subject to sub-paragraphs ($d$)  and ($e$), in relation to a jobseeker’s allowance—
%\begin{enumerate}\item[]
%(i) in regulation 3(1)($a$)  of the Breach of Community Order Regulations;
%
%(ii) in regulation 3(1)($b$)  of those Regulations;
%\end{enumerate}
%
%($b$) subject to sub-paragraphs ($d$)  and ($e$), in relation to income support—
%\begin{enumerate}\item[]
%(i) in regulation 3(3)($a$)  of the Breach of Community Order Regulations;
%
%(ii) in regulation 3(3)($b$)  of those Regulations;
%\end{enumerate}
%
%($c$) subject to sub-paragraphs ($d$)  and ($e$), in relation to a joint-claim jobseeker’s allowance—
%\begin{enumerate}\item[]
%(i) in regulation 3(4)($a$)  of the Breach of Community Order Regulations;
%
%(ii) in regulation 3(4)($b$)  of those Regulations;
%\end{enumerate}
%
%($d$) in regulation 3(5) of the Breach of Community Order Regulations;
%
%($e$) in regulation 3(6) of the Breach of Community Order Regulations.
%\end{enumerate}

% Reg 7(28) added (1.4.02) by SI 2002/490 reg 8(c)
(28) A decision to which regulation 6(2)($j$)  or ($k$)  applies shall take effect from the first day of the disqualification period prescribed for the purposes of section 
6B or  % Words inserted by SI 2010/1160 reg 3(4)
7 of the Social Security Fraud Act 2001\footnote{The beginning of the disqualification period for the purposes of section 7 is prescribed in regulation 2 of the Social Security (Loss of Benefit) Regulations 2001 (S.I. 2001/4022).}.

% Reg 7(29) added (7.4.03) by SI 2002/3019 reg 18(f)
(29) 
%A 
Subject to paragraphs (29A) and (29B), a  % Word substituted (6.10.03) by SI 2003/2274 reg 5(3)(a)
decision to which regulation~6(2)($l$) (state pension credit) refers shall take effect from the day following the day on which the assessed income period ends if that day is the first day of the claimant’s benefit week, but if it is not, from the next following such day.

% Reg 7(29A)--(29C) added (6.10.03) by SI 2003/2274 reg 5(3)(b)
(29A) A decision to which regulation 6(2)($l$)  applies, where—
\begin{enumerate}\item[]
($a$) the decision is advantageous to the claimant; and

($b$) the information and evidence required under regulation 32(1) of the Claims and Payments Regulations has not been provided within the period allowed under that regulation,
\end{enumerate}
shall take effect from the day the information and evidence required under that regulation is provided if that day is the first day of the claimant’s benefit week, but, if it is not, from the next following such day.

(29B) A decision to which regulation 6(2)($l$)  applies, where—
\begin{enumerate}\item[]
($a$) the decision is disadvantageous to the claimant; and

($b$) the information and evidence required under regulation 32(1) of the Claims and Payments Regulations has not been provided within the period allowed under that regulation,
\end{enumerate}
shall take effect from the day after the period allowed under that regulation expired.

(29C) Except where there is a change of circumstances during the period in which the Secretary of State was prevented from specifying a new assessed income period under regulation 10(1) of the State Pension Credit Regulations, a decision to which regulation 6(2)($m$)  applies shall take effect from the day on which the information and evidence required under regulation 32(6)($a$)  of the Claims and Payments Regulations was provided.

% Reg 7(30)--(33) added (5.5.03) by SI 2003/1050 reg 3(5)(e)
\sloppyword{
(30) Where a decision is superseded in accordance with regulation~6(2)($a$)(i)  and the relevant circumstances are that there has been a change in the legislation in relation to a relevant benefit, the decision under section 10 shall take effect from the date on which that change in the legislation had effect.
}

% Reg 7(30A) inserted by SI 2010/510 reg 4(2)
\begin{sloppypar}
(30A) Where a decision is superseded in accordance with regulation~6(2)($a$)(ii) and the relevant change of circumstances is the coming into force of a change in the legislation in relation to a relevant benefit, the decision under section~10 shall take effect from the date on which that change in the legislation takes effect.
\end{sloppypar}

\sloppyword{
(31) Where a decision is superseded in accordance with regulation~6(2)($a$)(ii)  and the relevant circumstances are that---
}
\begin{enumerate}\item[]
($a$) a personal capability assessment has been carried out in the case of a person to whom section 171C(4) of the Contributions and Benefits Act\footnote{1992 c.\ 4. Section 171C was inserted by the Social Security (Incapacity for Work) Act 1994 (c.\ 18), section 5 and substituted by the Welfare Reform and Pensions Act 1999 (c.\ 30), section 16.} applies; and

($b$) the own occupation test remains applicable to him under section~171B(3) of that Act\footnote{Section 171B was inserted by the Social Security (Incapacity for Work) Act 1994 (c.\ 18), section 5.},
\end{enumerate}
the decision under section 10 shall take effect on the day 
%immediately following the day  % Words omitted by SI 2008/2667 reg 3(4)(g)
on which the own occupation test is no longer applicable to that person.

(32) For the purposes of paragraph (31)—
\begin{enumerate}\item[]
($a$) “personal capability assessment” has the same meaning as in regulation 24 of the Social Security (Incapacity for Work) (General) Regulations 1995\footnote{S.I. 1995/311, the relevant amending instrument is S.I. 1999/3109.};

($b$) “own occupation test” has the same meaning as in section 171B(2) of the Contributions and Benefits Act.
\end{enumerate}

(33) A decision to which regulation 6(2)($c$)(ii)  applies shall take effect from the date on which 
%the 
%%appeal tribunal or the Commissioner’s 
%First-tier Tribunal or the Upper Tribunal's  % Words substituted (3.11.08) by SI 2008/2683 Sch 1 para 104(b)(i)
%decision 
the decision of the appeal tribunal, the First-tier Tribunal, the Upper Tribunal or the Commissioner  % Words substituted (3.11.08) by SI 2012/1267 reg 4(5)(b)(i)
would have taken effect had it been decided in accordance with the determination of the 
%Commissioner 
Upper Tribunal  % Words substituted (3.11.08) by SI 2008/2683 Sch 1 para 104(b)(ii)
or the Commissioner  % Words inserted (3.11.08) by SI 2012/1267 reg 4(5)(b)(ii)
or the court in the appeal referred to in section~26(1)($b$).

\begin{sloppypar}
% Reg 7(34) added (18.3.05) by SI 2005/337 reg 2(5)(d)
(34) A decision which supersedes a decision specified in regulation~6(2)($n$)  shall take effect from the effective date of the Secretary of State’s decision to terminate income support which was confirmed by the decision specified in regulation 6(2)($n$).
\end{sloppypar}

% Reg 7(35)--(39) added (27.7.08) by SI 2008/1554 reg 33(9)
% Reg 7(35) omitted by SI 2012/2756 reg 8(2)
%(35) A decision made in accordance with regulation 6(2)($p$), where the failure determination was made before the 13th week of entitlement, shall take effect from the first day of the benefit week following that week.

%(36) A decision made in accordance with regulation 6(2)($p$)  where paragraph (35) does not apply shall take effect from the first day of the benefit week in which the failure determination was made.

% Reg 7(36) substituted by SI 2012/2756 reg 8(3)
(36) A decision made in accordance with regulation 6(2)($p$) shall take effect—
\begin{enumerate}\item[]
($a$) on the first day of the benefit week in which the failure determination was made where, on the date of that determination, the claimant has not been paid an employment and support allowance since the failure to which that determination relates; or

($b$) in any other case, on the first day of the benefit week after the end of the benefit week in respect of which the claimant was last paid an employment and support allowance.
\end{enumerate}

(37) A decision made in accordance with regulation 6(2)($q$)  shall take effect from the first day of the benefit week in which the reduction mentioned in that sub-paragraph ceased to have effect.

%(38) A decision made in accordance with regulation 6(2)($r$)  that embodies a determination that the claimant has limited capability for work which is the first such determination shall take effect from the beginning of the 14th week of entitlement.

% Reg 7(38) substituted by SI 2010/840 reg 7(4)(a)
(38) A decision made in accordance with regulation 6(2)($r$) that embodies a determination that the claimant has—
\begin{enumerate}\item[]
($a$) limited capability for work; or

($b$) limited capability for work-related activity; or

($c$) limited capability for work and limited capability for work-related activity,
\end{enumerate}
which is the first such determination shall take effect from 
%the beginning of the 14th week of entitlement
the day after the last day of the relevant period as defined in regulation 4(4) of the Employment and Support Allowance Regulations%  % Words substituted by SI 2015/339 reg 7(3)
.

(39) A decision made in accordance with regulation 6(2)($r$), following an application by the claimant, that embodies a determination that the claimant has limited capability for work-related activity shall take effect from the date of the application.

% Reg 7(40) added by SI 2010/840 reg 7(4)(b)
(40) A decision made in accordance with regulation 6(2)($r$) that embodies a determination that the claimant has—
\begin{enumerate}\item[]
($a$) limited capability for work; or

($b$) limited capability for work-related activity; or

($c$) limited capability for work and limited capability for work-related activity,
\end{enumerate}
where regulation 5 of the Employment and Support Allowance Regulations (assessment phase---previous claimants) applies shall take effect from the beginning of the 14th week of the person’s continuous period of limited capability for work.

% Reg 7(41), (42) inserted by SI 2014/1097 reg 12(5)
(41) A decision made in accordance with regulation 6(2)($t$) shall take effect from the first day of the next benefit week following the day on which the determination mentioned in that sub-paragraph was made.

(42) A decision made in accordance with regulation 6(2)($u$) shall take effect from the first day of the benefit week in which the reduction mentioned in that sub-paragraph ceased to have effect.

% Reg 7(43) inserted by SI 2016/1145 reg 4(4)
(43) Where the decision is superseded in accordance with regulation 6(2)($sa$), the superseding decision shall take effect from the date on which the contributions are treated as paid in accordance with regulation 4(7) of the Social Security (Crediting and Treatment of Contributions, and National Insurance Numbers) Regulations 2001 for the purposes of entitlement to—
\begin{enumerate}\item[]
(i) a bereavement benefit;

(ii) a Category A or Category B retirement pension under Part II of the Contributions and Benefits Act; or

(iii) a state pension under Part I of the Pensions Act 2014.
\end{enumerate}

\amendment{
Reg. 7(2)(c)(ii), (iii) substituted for reg. 7(2)(c)(ii) (5.7.99) by the Social Security and Child Support (Decisions and Appeals) Amendment (No. 2) Regulations 1999 reg. 4.

Words inserted in reg. 7(5) (5.10.99) by the Tax Credits (Decisions and Appeals) (Amendment) Regulations 1999 reg. 10.

Reg. 7(8) substituted (18.10.99) by the Social Security and Child Support (Decisions and Appeals), Vaccine Damage Payments and Jobseeker's Allowance (Amendment) Regulations 1999 reg. 8.

Reg. 7(1), (2)(a) substituted and reg. 7(2)(c)(i) omitted (29.11.99) by the Social Security Act 1998 (Commencement No. 12 and Consequential and Transitional Provisions) Order 1999 Sch. 19 para. 1.

Reg. 7(9) substituted (17.2.00) by the Social Security and Child Support (Decisions and Appeals) Amendment Regulations 2000 reg. 2.

Reg. 7(25), (26) added (3.4.00) by the Social Security (Work-focused Interviews) Regulations 2000 Sch. 6 para. 5.

Words substituted in reg. 7(9)(a), reg. 7(2)(bb) inserted and reg. 7(1)(a), (5), (7) substituted (19.6.00) by the Social Security and Child Support (Miscellaneous Amendments) Regulations 2000 reg. 17.

Words inserted in reg. 7(8)(a), (b) (19.3.01) by the Social Security (Joint Claims: Consequential Amendments) Regulations 2000 reg. 5(c).

Reg. 7(27) added (15.10.01) by the Social Security (Breach of Community Order) (Consequential Amendments) Regulations 2001 reg. 2(2)(d).

Reg. 7(28) added (1.4.02) by the Social Security (Loss of Benefit) (Consequential Amendments) Regulations 2002 reg. 8(c).

Reg. 7(7) substituted (2.4.02) by the Social Security (Claims and Payments and Miscellaneous Amendments) Regulations 2002 reg. 4(4).

Words added to reg. 7(3), words substituted in reg. 7(2)(b)(i), (ii), reg. 7(13)(a)(iii), (17A), (29) inserted and reg. 7(1)(a) substituted (7.4.03) by the State Pension Credit (Consequential, Transitional and Miscellaneous Provisions) Regulations 2002 reg. 18.

Reg. 7(17B), (17C) inserted and reg. 7(23) substituted (7.4.03) by the State Pension Credit (Consequential, Transitional and Miscellaneous Provisions) (No. 2) Regulations 2002 reg. 6.

Words inserted in reg. 7(5), words substituted in reg. 7(1)(a), (2), reg. 7(30)--(33) added and reg. 7(9)(a) substituted (5.5.03) by the Social Security and Child Support (Miscellaneous Amendments) Regulations 2003 reg. 3(5).

Word substituted in reg. 7(29) and reg. 7(29A)-(29C) added (6.10.03) by the State Pension Credit (Consequential, Transitional and Miscellaneous Provisions) Amendment Regulations 2003 reg. 5(3).

Reg. 7(17B), (17C) substituted and word omitted in reg. 7(23) (5.4.04) by the State Pension Credit (Miscellaneous Amendments) Regulations 2004 reg. 2.

Reg. 7(25) substituted (12.4.04) by the Social Security (Jobcentre Plus Interviews for Partners) Regulations 2003 reg. 15(5).

Words substituted in reg. 7(25)(b) (26.4.04) by the Social Security (Working Neighbourhoods) Regulations 2004 reg. 24(4).

Words inserted in reg. 7(2), (7), reg. 7(6A) inserted and reg. 7(34) added (18.3.05) by the Social Security, Child Support and Tax Credits (Miscellaneous Amendments) Regulations 2005 reg. 2(5).

Words substituted in reg. 7(17B)(d)(ii) (4.4.05) by the Social Security (Housing Benefit, Council Tax Benefit, State Pension Credit and Miscellaneous Amendments) Regulations 2004 reg. 4.

Reg. 7(7A) inserted (6.4.06) by the Social Security (Deferral of Retirement Pensions, Shared Additional Pension and Graduated Retirement Benefit) (Miscellaneous Provisions) Regulations 2005 reg. 9(5).

Reg. 7(2)(c)(iv), (v) inserted, reg. 7(7) substituted and reg. 7(2)(c)(iii) omitted (10.4.06) by the Social Security (Miscellaneous Amendments) (No. 2) Regulations 2006 reg. 5(3).

Reg. 7(2)(bc) inserted (2.10.06) by the Social Security (Miscellaneous Amendments) (No. 3) Regulations 2006 reg. 3(2).

Words inserted in reg. 7(2)(bc) and reg. 7(2)(bd) added (24.9.07) by the Social Security (Miscellaneous Amendments) (No. 4) Regulations 2007 reg. 3(6), (7).

\begin{sloppypar}
Words inserted in reg. 7(1)(a), (7)(b)(i), (ii)(aa), (23)(a), words substituted in reg. 7(1)(a), (2)(b)(i), (23), reg. 7(2)(be), (13)(a)(iv), (17D)--(17H), (35)--(39) inserted and reg. 7(3) substituted (27.7.08) by the Employment and Support Allowance (Consequential Provisions) (No. 2) Regulations 2008 reg. 33.
\end{sloppypar}

Words substituted in reg. 7(2)(bc), reg. 7(2A) inserted and reg. 7(2)(bd) omitted (19.5.08) by the Social Security (Miscellaneous Amendments) (No. 2) Regulations 2008 reg. 2.

Words substituted in reg. 7(1)(a), reg. 7(2)(bb), (bc) substituted, reg. 7(2A) omitted, reg. 7(8A) inserted, words substituted in reg. 7(9)(b), (c) and words omitted in reg. 7(31) (30.10.08) by the Social Security (Miscellaneous Amendments) (No. 5) Regulations 2008 reg. 3(4).

Words substituted in reg. 7(5), (33) (3.11.08) by the Tribunals, Courts and Enforcement Act 2007 (Transitional and Consequential Provisions) Order 2008 Sch. 1 para. 104.

Words inserted in reg. 7(33) and words substituted in reg. 7(5), (33) (3.11.08) by the Social Security and Child Support (Supersession of Appeal Decisions) Regulations 2012 reg. 4(5).

Words inserted in reg. 7(2)(c)(ii) (13.7.09) by the Social Security (Miscellaneous Amendments) (No. 2) Regulations 2009 reg. 3(3).

Reg. 7(27) omitted (22.3.10) by the Welfare Reform Act 2009 (Section 26) (Consequential Amendments) Regulations 2010 reg. 4(5).

Words inserted in reg. 7(28) (1.4.10) by the Social Security (Loss of Benefit) Amendment Regulations 2010 reg. 3(4).

Reg. 7(8)(za) inserted (6.4.10) by the Jobseeker’s Allowance (Sanctions for Failure to Attend) Regulations 2010 reg. 3(4).

Reg. 7(30A) inserted (6.4.10) by the Social Security (Miscellaneous Amendments) Regulations 2010 reg. 4(2).

Reg. 7(38) substituted and reg. 7(40) added (28.6.10) by the Social Security (Miscellaneous Amendments) (No. 3) Regulations 2010 reg. 7(4).

Reg. 7(8ZA) inserted temporarily (22.11.10--21.11.13) by the Jobseeker’s Allowance (Work for Your Benefit Pilot Scheme) Regulations 2010 reg. 20(c).

Reg. 7(8ZA) inserted (25.4.11) by the Jobseeker’s Allowance (Mandatory Work Activity Scheme) Regulations 2011 reg. 18(c).

Reg. 7(8ZB) inserted (20.5.11) by the Jobseeker’s Allowance (Mandatory Work Activity Scheme) Regulations 2011 reg. 17.

Words inserted in reg. 7(9) and reg. 7(9A) inserted (31.10.11) by the Social Security (Disability Living Allowance, Attendance Allowance and Carer's Allowance) (Miscellaneous Amendments) Regulations 2011 reg. 2.

Words inserted in reg. 7(7)(b)(ii)(aa) (1.4.12) by the Social Security (Miscellaneous Amendments) Regulations 2012 reg. 17.

Reg. 7(8), (8ZA) substituted and reg. 7(8ZB) omitted (22.10.12) by the Jobseeker’s Allowance (Sanctions) (Amendment) Regulations 2012 reg. 6(4).

Reg. 7(35) omitted and reg. 7(36) substituted (3.12.12) by the Employment and Support Allowance (Sanctions) (Amendment) Regulations 2012 reg. 8.

Reg. 7(41), (42) inserted (28.4.14) by the Income Support (Work-Related Activity) and Miscellaneous Amendments Regulations 2014 reg. 12(5).

Words in reg. 7(38) substituted (30.3.15) by the Jobseeker’s Allowance (Extended Period of Sickness) Amendment Regulations 2015 reg. 7(3).

Words inserted in reg. 7(7A) (6.4.16) by the Pensions Act 2014 (Consequential, Supplementary and Incidental Amendments) Order 2015 art.~18(6).

Reg. 7(43) inserted (1.1.17) by the Social Security (Credits, and Crediting and Treatment of Contributions) (Consequential and Miscellaneous Amendments) Regulations 2016 reg. 4(5).
}

% Reg 7A inserted (5.7.99) by SI 1999/1623 reg 5
%\subsubsection[7A. Definitions for the purposes of regulations 3(5)($c$), 6(2)($g$)% 
%%and 7(2)($c$)---
%, 7(2)($c$) and (5)  % Words substituted (19.6.00) by SI 2000/1596 reg 18(a)
%and ancillary provisions]{Definitions for the purposes of regulations 3(5)($c$), 6(2)($g$)% 
%%and 7(2)($c$)---
%, 7(2)($c$) and (5)  % Words substituted (19.6.00) by SI 2000/1596 reg 18(a)
%and ancillary provisions}

% Heading substituted (27.7.08) by SI 2008/1554 reg 34(2)
\subsubsection[7A. Definitions for the purposes of Chapters I and II]{Definitions for the purposes of Chapters I and II}

7A.---(1)  For the purposes of regulations 3(5)($c$), 6(2)($g$)% 
%and 7(2)($c$)---
, 6(2)($r$)%  % Words inserted by SI 2010/840 reg 7(5)
, 7(2)($c$) and (5)---  % Words substituted (19.6.00) by SI 2000/1596 reg 18(a)
\begin{enumerate}\item[]
“disability benefit decision” means a decision to award a relevant benefit embodied in or necessary to which is a disability determination,

“disability determination” means–
\begin{enumerate}\item[]
($a$)
in the case of a decision as to an award of an attendance allowance or a disability living allowance, whether the person satisfies any of the conditions in section 64, 72(1) or 73(1) to (3), as the case may be, of the Contributions and Benefits Act,

($b$)
in the case of a decision as to an award of severe disablement allowance, whether the person is disabled for the purpose of section~68 of the Contributions and Benefits Act, or

\looseness=1
($c$)
in the case of a decision as to an award of industrial injuries benefit, whether the existence or extent of any disablement is sufficient for the purposes of section 103 or 108 of the Contributions and Benefits Act or for the benefit to be paid at the rate which was in payment immediately prior to that decision;
\end{enumerate}

\looseness=1
% Definition of ``employment and support allowance decision'' inserted (27.7.08) by SI 2008/1554 reg 34(3)(a)
“employment and support allowance decision” means a decision to award a relevant benefit or relevant credit embodied in or necessary to which is a determination that a person has or is to be treated as having limited capability for work under Part I of the Welfare Reform Act;

“incapacity benefit decision” means a decision to award a relevant benefit 
or relevant credit  % Words inserted (19.6.00) by SI 2000/1596 reg 18(b)
embodied in or necessary to which is a determination that a person is or is to be treated as incapable of work under Part~XIIA of the Contributions and Benefits Act
or an award of long term incapacity benefit under regulation 17(1) (transitional awards of long-term incapacity benefit) of the Social Security (Incapacity Benefit) (Transitional) Regulations 1995\footnote{S.I. 1995/310.}%  % Words added (24.9.07) by SI 2007/2470 reg 3(8)
,

\looseness=-1
“incapacity determination” means a determination whether a person is incapable of work by applying the 
%all work test 
personal capability assessment  % Words substituted (19.6.00) by SI 2000/1596 reg 18(c)
in regulation 24 of the Social Security (Incapacity for Work) (General) Regulations 1995 or whether a person is to be treated as incapable of work in accordance with regulation 10 (certain persons with a severe condition to be treated as incapable of work) or 27 (exceptional circumstances) of those Regulations, and

% Definition of ``limited capability for work determination'' inserted (27.7.08) by SI 2008/1554 reg 34(3)(b)
“limited capability for work determination” means a determination whether a person has limited capability for work by applying the test of limited capability for work or whether a person is to be treated as having limited capability for work in accordance with regulation~20 of the Employment and Support Allowance Regulations;

\sloppyword{%
``payee'' means a person to whom a benefit referred to in paragraph~($a$),~($b$) or ($c$) of the definition of ``disability determination'', or a benefit referred to in the definition of ``incapacity benefit decision'' 
or ``employment and support allowance decision''  % Words inserted (27.7.08) by SI 2008/1554 reg 34(3)(c)
is payable.}
\end{enumerate}

(2) Where a person’s receipt of or entitlement to a benefit (“the first benefit”) is a condition of his being entitled to any other benefit, allowance or advantage (“a second benefit”) and a decision is revised under regulation~3(5)($c$) or a superseding decision is made under regulation 6(2) to which regulation 7(2)($c$)(ii) applies, the effect of which is that the first benefit ceases to be payable, or becomes payable at a lower rate than was in payment immediately prior to that revision or supersession, a consequent decision as to his entitlement to the second benefit shall take effect from the date of the change in his entitlement to the first benefit.

\amendment{
Reg. 7A inserted (5.7.99) by the Social Security and Child Support (Decisions and Appeals) Amendment (No. 2) Regulations 1999 reg. 5.

Words inserted in definition of ``incapacity benefit decision'' in reg. 7A(1) and words substituted in reg. 7A(1) and heading (19.6.00) by the Social Security and Child Support (Miscellaneous Amendments) Regulations 2000 reg. 18.

Words added to definition of ``incapacity benefit decision'' in reg. 7A(1) (24.9.07) by the Social Security (Miscellaneous Amendments) (No. 4) Regulations 2007 reg. 3(8).

Words inserted in definition of ``payee'' in reg. 7A(1), definitions of ``employment and support allowance decision'', ``limited capability for work determination'' inserted in reg. 7A(1) and heading substituted (27.7.08) by the Employment and Support Allowance (Consequential Provisions) (No. 2) Regulations 2008 reg. 34.

Words inserted in reg. 7A(1) (28.6.10) by the Social Security (Miscellaneous Amendments) (No. 3) Regulations 2010 reg. 7(5).
}

% Regs 7B, 7C inserted (3.3.03 for new-rules cases only) by SI 2000/3185 reg 9
%\subsubsection[7B. Date from which a decision superseded under section 17 of the Child Support Act takes effect]{Date from which a decision superseded under section 17 of the Child Support Act takes effect}
%
%7B.---(1)  Subject to paragraphs (17) to (22), where a decision is superseded by a decision made by the Secretary of State in a case to which regulation 6A(2)($a$)  applies on the basis of information or evidence which was also the basis of a decision made under section 8, 9 or 10 of the Act, the decision under section 17 of the Child Support Act shall take effect from the first day of the maintenance period in which that information or evidence was first brought to the attention of an officer exercising the functions of the Secretary of State under the Child Support Act (“the officer”).
%
%% Reg 7B(1A) inserted (3.3.03) by SI 2002/1204 reg 2(4)(a)
%(1A) Where a decision is superseded by a decision made by the Secretary of State in a case to which regulation 6A(2)($a$)  or (3) applies and the relevant circumstance is that—
%\begin{enumerate}\item[]
%($a$) paragraph 4(2) of Schedule 1 to the Child Support Act applies, the decision shall take effect from the first day of the maintenance period on or after—
%\begin{enumerate}\item[]
%(i) the date on which the non-resident parent becomes the partner of a non-resident parent; or
%
%(ii) where a maintenance calculation is first made in respect of the non-resident parent’s partner, the date on which that calculation takes effect for the purposes of the Child Support Act; or
%\end{enumerate}
%
%($b$) paragraph 4(2) of Schedule 1 to the Child Support Act ceases to apply, the decision shall take effect from the first day of the maintenance period on or after the date on which—
%\begin{enumerate}\item[]
%(i) the non-resident parent or his partner ceases to be a non-resident parent; or
%
%(ii) the non-resident parent ceases to be the partner of a non-resident parent.
%\end{enumerate}
%\end{enumerate}
%
%(2) Where a decision is superseded by a decision made by the Secretary of State in a case to which regulation 6A(3)($a$)  applies and the relevant circumstance is that the non-resident parent or his partner has notified the officer that he or his partner had made a claim for a relevant benefit and, where the relevant benefit is payable, that the officer was notified within one month of notification of the award, the decision shall take effect from the first day of the maintenance period in which—
%\begin{enumerate}\item[]
%($a$) the non-resident parent or his partner notified the officer that he or his partner had made a claim for a relevant benefit, where entitlement to that benefit commences on or before the date of notification; or
%
%($b$) entitlement to the relevant benefit commences, where that entitlement commenced after the date of notification.
%\end{enumerate}
%
%(3) Where a decision is superseded by a decision made by the Secretary of State in a case to which regulation 6A(4) applies and the material fact is that the non-resident parent or his partner has notified the officer that he or his partner had made a claim for a relevant benefit before the Secretary of State notified him of an application for a maintenance calculation in accordance with regulation 5 of the Maintenance Calculation Procedure Regulations (notice of an application for a maintenance calculation) and, where the relevant benefit is payable, that the officer was notified within one month of notification of the award, the decision shall take effect from the first day of the maintenance period in which—
%\begin{enumerate}\item[]
%($a$) the non-resident parent or his partner notified the officer that he or his partner had made a claim for a relevant benefit, where entitlement to that benefit commences on or before the date of notification; or
%
%($b$) entitlement to the relevant benefit commences, where that entitlement commenced after the date of notification.
%\end{enumerate}
%
%(4) Subject to paragraphs (17) to (22), where the superseding decision is made in a case to which regulation 6A(3)($a$)(i)  applies and that decision supersedes one which has been made under section 12(2) of the Child Support Act, the decision shall take effect from the first day of the maintenance period in which the change of circumstances occurred.
%
%(5) Where the superseding decision is made in a case to which regulation 6A(3)($a$)(ii)  applies, the decision shall take effect from the first day of the maintenance period in which the change of circumstances is expected to occur.
%
%(6) Where the superseding decision is made in a case to which regulation 6A(6) applies and the relevant circumstance is that a ground for a variation is expected to occur, the decision shall take effect from the first day of the maintenance period in which the ground for the variation is expected to occur.
%
%(7) Except in a case to which paragraph (1) applies, where the superseding decision is made in a case to which regulation 7C applies, that decision shall take effect from the first day of the maintenance period which includes the date which is 28 days after the date on which the Secretary of State gave notice to the relevant persons under that regulation.
%
%(8) For the purposes of paragraph (7)—
%\begin{enumerate}\item[]
%($a$) where the relevant persons are notified on different dates, the period of 28 days shall be counted from the date of the latest notification;
%
%($b$) notification includes oral and written notification;
%
%($c$) where a person is notified in more than one way, the date on which he is notified is the date on which he was first given notification; and
%
%($d$) the date of written notification is the date on which it was given or sent to the person.
%\end{enumerate}
%
%(9) Where—
%\begin{enumerate}\item[]
%($a$) a decision made by 
%%an appeal tribunal or by a Child Support Commissioner 
%the First-tier Tribunal or the Upper Tribunal  % Words substituted (3.11.08) by SI 2008/2683 Sch 1 para 105(a)(i)
%is superseded on the ground that it was erroneous due to a misrepresentation of, or that there was a failure to disclose, a material fact; and
%
%($b$) the Secretary of State is satisfied that the decision was more advantageous to the person who misrepresented or failed to disclose that fact than it would otherwise have been but for that error,
%\end{enumerate}
%the superseding decision shall take effect from the date on which the decision of 
%%the appeal tribunal 
%the First-tier Tribunal  % Words substituted (3.11.08) by SI 2008/2683 Sch 1 para 105(a)(ii)
%or, as the case may be, 
%%the Child Support Commissioner 
%the Upper Tribunal  % Words substituted (3.11.08) by SI 2008/2683 Sch 1 para 105(a)(iii)
%took, or was to take, effect.
%
%(10) Any decision made under section 17 of the Child Support Act in consequence of a determination which is a relevant determination for the purposes of section 28ZC of that Act\footnote{\frenchspacing Section 28ZC was inserted by section 44 of the Social Security Act 1998 (c. 14).} shall take effect from the date of the relevant determination.
%
%% Reg 7B(11)--(16) omitted (27.10.08) by SI 2008/2543 reg 4(4)(a)
%%(11) Where a decision with respect to a reduced benefit decision is superseded because the decision ceases to be in force in accordance with regulation 16($a$)  of the Maintenance Calculation Procedure Regulations (termination of a reduced benefit decision), the superseding decision shall have effect—
%%\begin{enumerate}\item[]
%%($a$) where the decision is in operation immediately before it ceases to be in force, from the last day of the benefit week during the course of which the parent concerned falls within the provisions of section 46(1) of the Child Support Act; or
%%
%%($b$) where the decision is suspended immediately before it ceases to be in force, from the date on which the parent concerned falls within the provisions of section 46(1) of that Act.
%%\end{enumerate}
%%
%%(12) Where a decision with respect to a reduced benefit decision is superseded because the decision ceases to be in force in accordance with regulation 16($b$)  of the Maintenance Calculation Procedure Regulations, the superseding decision shall have effect—
%%\begin{enumerate}\item[]
%%($a$) where the decision is in operation immediately before it ceases to be in force, from the last day of the benefit week during the course of which the parent concerned complied with the obligations imposed by section 46(6)($b$)  of the Child Support Act; or
%%
%%($b$) where the decision is suspended immediately before it ceases to be in force, from the date on which the parent concerned complied with the obligations imposed by section 46(6)($b$)  of the Child Support Act.
%%\end{enumerate}
%%
%%(13) Where a decision with respect to a reduced benefit decision is superseded because the decision ceases to be in force in accordance with regulation 16($c$)  of the Maintenance Calculation Procedure Regulations, the superseding decision shall have effect from the last day of the benefit week in which entitlement to benefit ceased.
%%
%%(14) Where a decision with respect to a reduced benefit decision is superseded because the decision ceases to be in force in accordance with regulation 16($d$)  of the Maintenance Calculation Procedure Regulations, the superseding decision shall have effect—
%%\begin{enumerate}\item[]
%%($a$) where the decision is in operation immediately before it ceases to be in force, from the last day of the benefit week during the course of which the Secretary of State is supplied with information that enables him to make the calculation; or
%%
%%($b$) where the decision is suspended immediately before it ceases to be in force, from the date on which the Secretary of State is supplied with information that enables him to make the calculation.
%%\end{enumerate}
%%
%%(15) Where a decision with respect to a reduced benefit decision is superseded because the decision ceases to be in force in accordance with regulation 17(1) of the Maintenance Calculation Procedure Regulations (reduced benefit decisions where there is an additional qualifying child), the superseding decision shall have effect from—
%%\begin{enumerate}\item[]
%%($a$) the last day of the benefit week preceding the benefit week which includes, in accordance with the provisions of regulation 11(3) of the Maintenance Calculation Procedure Regulations (amount of and period of reduction of relevant benefit under a reduced benefit decision), the first day on which the further decision comes into operation; or
%%
%%($b$) the first day on which the further decision would come into operation but for the provisions of regulation 14 of the Maintenance Calculation Procedure Regulations (suspension of a reduced benefit decision when a modified applicable amount is payable (income support)) or 15 (suspension of a reduced benefit decision when a modified applicable amount is payable (income-based jobseeker’s allowance)) of those Regulations.
%%\end{enumerate}
%%
%%(16) Where a decision with respect to a reduced benefit decision is superseded because the decision ceases to be in force in accordance with regulation 18(2) of the Maintenance Calculation Procedure Regulations (suspension and termination of a reduced benefit decision where the sole qualifying child ceases to be a child or where the parent concerned ceases to be a person with care), the superseding decision shall have effect from the last day of the benefit week which includes the day on which the child ceases to be a child within the meaning of section 55 of the Child Support Act as supplemented by Schedule 1 to those Regulations, or the parent ceases to be the person with care.
%
%(17) Where a superseding decision is made in a case to which regulation 6A(2)($a$)  or (3) applies and the relevant circumstance is the death of a qualifying child or a qualifying child ceasing to be a qualifying child, the decision shall take effect from the first day of the maintenance period in which the change occurred.
%
%% Reg 7B(17A)--(17C) inserted (3.3.03) by SI 2003/328 reg 3
%(17A) Where a superseding decision is made in a case to which regulation 6A(2)($a$)  or (3) applies, and the relevant circumstance is that a person has ceased to be a person with care in relation to a qualifying child in respect of whom the maintenance calculation was made, the decision shall take effect from the first day of the maintenance period in which that person ceased to be a person with care in relation to that qualifying child.
%
%(17B) Where a superseding decision is made in a case to which regulation 6A(3) applies, and the relevant circumstance is that there is a further qualifying child in respect of the non-resident parent and the person with care to whom the maintenance calculation being superseded relates, the superseding decision shall take effect from—
%\begin{enumerate}\item[]
%($a$) subject to sub-paragraph ($b$), the first day of the maintenance period in respect of the maintenance calculation in force, following—
%\begin{enumerate}\item[]
%(i) where an effective application is made under section 17(1) of the Child Support Act by the non-resident parent, the date on which that application is made; or
%
%(ii) where the application made under section 17(1) of that Act is made by the person with care, or, where a maintenance calculation has been made in response to an application by a child under section 7 of that Act, by the child, the date of notification to the non-resident parent of that application;
%\end{enumerate}
%
%($b$) the first day of the maintenance period in respect of the maintenance calculation in force where the date set out in head (i)  or (ii)  falls on the first day of that maintenance period.
%\end{enumerate}
%
%(17C) For the purposes of paragraph (17B)—
%\begin{enumerate}\item[]
%($a$) in head (i)  of sub-paragraph ($a$), an application is effective if, were it an application for a maintenance calculation, it would comply with regulation 3(1) of the Maintenance Calculation Procedure Regulations;
%
%($b$) in head (ii)  of sub-paragraph ($a$), notification to the non-resident parent shall take the same form in respect of an application for a supersession as it would in regulation 5 of the Maintenance Calculation Procedure Regulations, in respect of an application for a maintenance calculation.
%\end{enumerate}
%
%(18) Where a superseding decision is made in a case to which regulation 6A(2)($a$)  or (3) applies and the relevant circumstance is that the non-resident parent, person with care or the qualifying child has moved out of the jurisdiction, the decision shall take effect from the first day of the maintenance period in which the non-resident parent, person with care or qualifying child leaves the jurisdiction and jurisdiction is within the meaning of section 44 of the Child Support Act.
%
%% Reg 7B(19) omitted (3.3.03) by SI 2002/1204 reg 2(4)(b)
%%(19) Where a superseding decision is made in a case to which regulation 6A(2)($a$)  or (3) applies and the relevant circumstance is that the maintenance calculation has been made in response to an application which is treated as made under section 6 of the Child Support Act and—
%%\begin{enumerate}\item[]
%%($a$) the person on whose application the calculation was made (“the applicant”) asks the Secretary of State to cease acting; and
%%
%%($b$) the Secretary of State is satisfied that the applicant has ceased to fall within section 6(1) of that Act,
%%\end{enumerate}
%%the decision shall take effect from the first day of the maintenance period after the applicant asks the Secretary of State to cease acting.
%
%(20) Where a superseding decision is made in a case to which regulation 6A(2)($a$)  or (3) applies and the relevant circumstance is that both the non-resident parent and the person with care with respect to whom a maintenance calculation was made request the Secretary of State to decide that the maintenance calculation shall cease and he is satisfied that they are living together, the decision shall take effect from the first day of the maintenance period in which the later of the two requests was made.
%
%(21) Where a superseding decision is made in a case to which regulation 6A(2)($a$)  or (3) applies and the relevant circumstance is that—
%\begin{enumerate}\item[]
%($a$) an application for a maintenance calculation is made under section 4 or 7 of the Child Support Act
%%, or treated as made under section 6(3) of that Act,  % Words omitted (27.10.08) by SI 2008/2543 reg 4(4)(b)
%in respect of a non-resident parent; and
%
%($b$) before the decision as to a maintenance calculation is made at least one other maintenance calculation is in force with respect to the same non-resident parent but to a different person with care and a different child,
%\end{enumerate}
%the effective date of the maintenance calculation made in respect of the application shall be a date which is not later than 7 days after the date of notification to the non-resident parent and which is the day on which a maintenance period in respect of the maintenance calculation in force begins.
%
%(22) Where a superseding decision is made in a case to which regulation 6A(3) applies and in relation to that decision a maintenance calculation is made to which paragraph 15 of Schedule 1 to the Child Support Act applies, the effective date of the calculation or calculations shall be the beginning of the maintenance period in which the change of circumstance to which the calculation or calculations relates occurred or is expected to occur and where that change occurred before the date of the application for the supersession and was notified after that date, the date of that application.
%
%% Reg 7B(22A) inserted (5.5.03) by SI 2003/1050 reg 3(6)
%(22A) Where a superseding decision is made in a case to which regulation 6A(4A) applies the decision shall take effect from the first day of the maintenance period following the date 
%%the appeal tribunal or the Commissioner’s 
%First-tier Tribunal or the Upper Tribunal’s  % Words substituted (3.11.08) by SI 2008/2683 Sch 1 para 105(b)(i)
%decision would have taken effect had it been decided in accordance with the determination 
%%of the Commissioner 
%of the Upper Tribunal  % Words substituted (3.11.08) by SI 2008/2683 Sch 1 para 105(b)(ii)
%or the court in the appeal referred to in section 28ZB(1)($b$)  of the Child Support Act.
%
%(23) In this regulation—
%\begin{enumerate}\item[]
%    “benefit week” in relation to income support has the same meaning as in regulation 2(1) of the Income Support Regulations%
%, in relation to employment and support allowance has the same meaning as in regulation 2(1) of the Employment and Support Allowance Regulations%  % Words inserted (27.7.08) by SI 2008/1554 reg 35
%, and in relation to jobseeker’s allowance has the same meaning as in regulation 1(3) of the Jobseeker’s Allowance Regulations;
%
%    “partner” has the same meaning as in regulation 2 of the Income Support Regulations; and
%
%    “relevant benefit” means a benefit which is prescribed in regulation 4 of the Maintenance Calculations and Special Cases Regulations for the purposes of paragraph 4(1)($b$)  of Part I of Schedule 1 to the Child Support Act, and child benefit as referred to in paragraph 10C(2)($a$)  of Part I of Schedule 1 to that Act. 
%\end{enumerate}
%
%\amendment{
%Reg. 7B inserted (3.3.03 for new-rules cases only) by the Child Support (Decisions and Appeals) (Amendment) Regulations 2000 reg. 9 (subject to reg. 1(2)).
%
%Reg. 7B(1A) inserted and reg. 7B(19) omitted (3.3.03 for new-rules cases only) by the Child Support (Miscellaneous Amendments) Regulations 2002 reg. 2(4).
%
%Reg. 7B(17A)--(17C) inserted (3.3.03 for new-rules cases only) by the Child Support (Miscellaneous Amendments) Regulations 2003 reg. 3.
%
%Reg. 7B(22A) inserted (5.5.03 for new-rules cases only) by the Social Security and Child Support (Miscellaneous Amendments) Regulations 2003 reg. 3(6).
%
%Words substituted in definition of ``benefit week'' in reg. 7B(23) (27.7.08) by the Employment and Support Allowance (Consequential Provisions) (No. 2) Regulations 2008 reg. 35.
%
%Words omitted in reg. 7B(21)(a) and reg. 7B(11)--(16) omitted (27.10.08) by the Child Support (Consequential Provisions) Regulations 2008 reg. 4(4).
%
%Words substituted in reg. 7B(9), (22A) (3.11.08) by the Tribunals, Courts and Enforcement Act 2007 (Transitional and Consequential Provisions) Order 2008 Sch. 1 para. 105.
%}

% Reg 7B substituted (6.4.09) by SI 2009/396 reg 4(6)
\subsubsection[7B. Effective date of a supersession decision]{Effective date of a supersession decision\\*\emph{2003 scheme only}}

7B.  Schedule 3D provides for cases and circumstances in which a supersession decision takes effect from a date other than the date specified in section 17(4) of the Child Support   Act.

\amendment{
Reg. 7B substituted (6.4.09) by the Child Support (Miscellaneous Amendments) Regulations 2009 reg. 4(6).

Reg. 7B omitted (10.12.12 for 2012 scheme cases only) by the Child Support (Meaning of Child and New Calculation Rules) (Consequential and Miscellaneous Amendment) Regulations 2012 reg. 6(3).
}

\subsubsection[7C. Procedure where the 
%Secretary of State 
%Commission  % Word substituted (6.4.09) by SI 2009/396 reg 4(7)	
Secretary of State  % Words substituted (1.8.12) by SI 2012/2007 Sch para 113(6)
proposes to supersede a decision under section 17 of the Child Support Act on 
%his 
%its  % Word substituted (6.4.09) by SI 2009/396 reg 4(7)
the Secretary of State's  % Words substituted (1.8.12) by SI 2012/2007 Sch para 113(6)
own initiative]{Procedure where the 
%Secretary of State 
%Commission  % Word substituted (6.4.09) by SI 2009/396 reg 4(7)	
Secretary of State  % Words substituted (1.8.12) by SI 2012/2007 Sch para 113(6)
proposes to supersede a decision under section 17 of the Child Support Act on 
%his 
%its  % Word substituted (6.4.09) by SI 2009/396 reg 4(7)
the Secretary of State's  % Words substituted (1.8.12) by SI 2012/2007 Sch para 113(6)
own initiative\\*\emph{2003 scheme only}}

7C.  Where the 
%Secretary of State 
%Commission  % Word substituted (6.4.09) by SI 2009/396 reg 4(7)	
Secretary of State  % Words substituted (1.8.12) by SI 2012/2007 Sch para 113(6)
on 
%his 
%its  % Word substituted (6.4.09) by SI 2009/396 reg 4(7)
the Secretary of State's  % Words substituted (1.8.12) by SI 2012/2007 Sch para 113(6)
own initiative proposes to make a decision superseding a decision 
%he 
%it  % Word substituted (6.4.09) by SI 2009/396 reg 4(7)
the Secretary of State  % Words substituted (1.8.12) by SI 2012/2007 Sch para 113(6)
shall notify the relevant persons who could be materially affected by the decision of that intention.

\amendment{
Reg. 7C inserted (3.3.03 for new-rules cases only) by the Child Support (Decisions and Appeals) (Amendment) Regulations 2000 reg. 9 (subject to reg. 1(2)).

Words substituted in reg. 7C (6.4.09) by the Child Support (Miscellaneous Amendments) Regulations 2009 reg. 4(7).

Words substituted in reg. 7C (1.8.12) by the Public Bodies (Child Maintenance and Enforcement Commission: Abolition and Transfer of Functions) Order 2012 Sch. para. 113(6).

Reg. 7C omitted (10.12.12 for 2012 scheme cases only) by the Child Support (Meaning of Child and New Calculation Rules) (Consequential and Miscellaneous Amendment) Regulations 2012 reg. 6(3).
}

\subsubsection[8. Effective date for late notifications of change of circumstances]{Effective date for late notifications of change of circumstances}

8.—(1) For the purposes of regulation 7(2)
and (9)% Words inserted (17.2.00) by SI 2000/119 reg 3(a)
, a longer period of time may be allowed for the notification of a change of circumstances in so far as it affects the effective date of the change where the conditions specified in the following provisions of this regulation are satisfied.

(2) An application for the purposes of regulation 7(2) 
or (9)  % Words inserted (17.2.00) by SI 2000/119 reg 3(b)
shall be made by the claimant or a person acting on his behalf.

(3) The application referred to in paragraph (2) shall—
\begin{enumerate}\item[]
($a$) contain particulars of the relevant change of circumstances and the reasons for the failure to notify the change of circumstances on an earlier date; and

%($b$) be made within 13 months of the date the change occurred.

% Reg 8(3)(b) substituted by SI 2010/510 reg 4(3)
($b$) be made—
\begin{enumerate}\item[]
(i) within 13 months of the date the change occurred; or

\begin{sloppypar}
(ii) in the case of an application for the purposes of regulation~7(9)($b$), within 13 months of the date on which the claimant satisfied the conditions of entitlement to the particular rate of benefit.
\end{sloppypar}
\end{enumerate}
\end{enumerate}

(4) An application under this regulation shall not be granted unless the Secretary of State is satisfied 
or the Board are satisfied  % Words inserted (5.10.99) by SI 1999/2570 reg 11(a)
that—
\begin{enumerate}\item[]
($a$) it is reasonable to grant the application;

($b$) the change of circumstances notified by the applicant is relevant to the decision which is to be superseded; and

($c$) special circumstances are relevant to the application and as a result of those special circumstances it was not practicable for the applicant to notify the change of circumstances within one month of the change occurring.
\end{enumerate}

(5) In determining whether it is reasonable to grant the application, the Secretary of State 
or the Board  % Words inserted (5.10.99) by SI 1999/2570 reg 11(b)
shall have regard to the principle that the greater the amount of time that has elapsed between the date one month after the change of circumstances occurred and the date the application for the purposes of regulation 7(2) 
or (9)  % Words inserted (17.2.00) by SI 2000/119 reg 3(b)
is made, the more compelling should be the special circumstances on which the application is based.

(6) In determining whether it is reasonable to grant an application, no account shall be taken of the following—
\begin{enumerate}\item[]
($a$) that the applicant or any person acting for him was unaware of, or misunderstood, the law applicable to his case (including ignorance or misunderstanding of the time limits imposed by these Regulations); or

($b$) that 
%a Commissioner 
the Upper Tribunal  % Words substituted (3.11.08) by SI 2008/2683 Sch 1 para 106
or a court has taken a different view of the law from that previously understood and applied.
\end{enumerate}

(7) An application under this regulation which has been refused may not be renewed.

\amendment{
Words inserted in reg. 8(4), (5) (5.10.99) by the Tax Credits (Decisions and Appeals) (Amendment) Regulations 1999 reg. 11.

Words inserted in reg. 8(1), (2), (5) (17.2.00) by the Social Security and Child Support (Decisions and Appeals) Amendment Regulations 2000 reg. 3.

Words substituted in reg. 8(6)(b) (3.11.08) by the Tribunals, Courts and Enforcement Act 2007 (Transitional and Consequential Provisions) Order 2008 Sch. 1 para. 106.

Reg. 8(3)(b) substituted (6.4.10) by the Social Security (Miscellaneous Amendments) Regulations 2010 reg. 4(3).
}

\subsection[Chapter III --- Other matters]{Chapter III\\*Other matters}

\subsubsection[9. Certificates of recoverable benefits]{Certificates of recoverable benefits}

\renewcommand\parthead{--- Part II Chapter III}

9.  A certificate of recoverable benefits may be reviewed under section 10 of the 1997 Act\footnote{\frenchspacing Section 10 was amended by paragraph 149 of Schedule 7 to the Social Security Act 1998.} where the Secretary of State is satisfied that—
\begin{enumerate}\item[]
($a$) a mistake (whether in computation of the amount specified or otherwise) occurred in the preparation of the certificate;

($b$) the benefit recovered from a person who makes a compensation payment (as defined in section 1 of the 1997 Act) is in excess of the amount due to the Secretary of State;

($c$) incorrect or insufficient information was supplied to the Secretary of State by the person who applied for the certificate and in consequence the amount of benefit specified in the certificate was less than it would have been had the information supplied been correct or sufficient; or

($d$) a ground for appeal is satisfied under section 11 of the 1997 Act\footnote{\frenchspacing Section 11 was amended by paragraph 150 of Schedule 7 to the Social Security Act 1998.}.
\end{enumerate}

% Reg 9ZA inserted (1.10.08) by SI 2008/1596 Sch 2 para 1(b) as amended by SI 2008/2365 reg 6(4)
\subsubsection[9ZA. Review of certificates]{Review of certificates}

9ZA.---(1)  A certificate may be reviewed under section 10 of the 1997 Act where the Secretary of State is satisfied that—
\begin{enumerate}\item[]
($a$) a mistake (whether in the computation of the amount specified or otherwise) occurred in the preparation of the certificate;

($b$) the lump sum payment recovered from a compensator who makes a compensation payment (as defined in section 1A(5) of the 1997 Act) is in excess of the amount due to the Secretary of State;

($c$) incorrect or insufficient information was supplied to the Secretary of State by the compensator who applied for the certificate and in consequence the amount of lump sum payment specified in the certificate was less than it would have been had the information supplied been correct or sufficient;

($d$) a ground for appeal is satisfied under section 11 of the 1997 Act or an appeal has been made under that section; or

($e$) a certificate has been issued and, for any reason, a recoverable lump sum payment was not included in that certificate.
\end{enumerate}

(2) In this regulation and regulations 1(3) in paragraph ($b$)  of the definition of “party to the proceedings”, 
%29, 31, 33, 36(2)($a$)(ii)  and 58(1)
29% 
%and 33%  % Words omitted (28.10.13) by SI 2013/2380 reg 4(10)(a)
% Words substituted (3.11.08) by SI 2008/2683 Sch 1 para 107
, where applicable—
\begin{enumerate}\item[]
($a$) any reference to the 1997 Act is to be construed so as to include a reference to that Act as applied by regulation 2 of the Lump Sum Payments Regulations and, where applicable, as modified by Schedule 1 to those Regulations;

%($b$) “certificate” has the same meaning as in regulation 1(2) of the Lump Sum Payments Regulations;

% Reg 9ZA(2)(b) substituted (1.10.08) by SI 2008/2365 reg 6(4)
($b$) “certificate” means a certificate of recoverable lump sum payments, including where any of the amounts is nil;

($c$) “lump sum payment” is a payment to which section 1A(2) of the 1997 Act  applies;

($d$) “$\mathcal{P}$” is to be construed in accordance with regulations 4(1)($a$)(i)  and 5 of the Lump Sum Payments Regulations.
\end{enumerate}

\amendment{
Reg. 9ZA inserted (1.10.08) by the Social Security (Recovery of Benefits) (Lump Sum Payments) Regulations 2008 Sch. 2 para. 1(b) as amended by the Social Security (Miscellaneous Amendments) (No. 3) Regulations 2008 reg. 6(4).

Words substituted in reg. 9ZA(2) (3.11.08) by the Tribunals, Courts and Enforcement Act 2007 (Transitional and Consequential Provisions) Order 2008 Sch. 1 para. 107.

Words omitted in reg. 9ZA(2) (28.10.13) by the Social Security, Child Support, Vaccine Damage and Other Payments (Decisions and Appeals) (Amendment) Regulations 2013 reg.~4(10)(a) (subject to transitional provisions in reg. 8).
}

% Reg 9ZA inserted (28.10.13) by SI 2013/2380 reg 4(6)
\subsubsection[9ZB. Consideration of review before appeal]{Consideration of review before appeal}

9ZB.---(1)  This regulation applies in a case where—
\begin{enumerate}\item[]
($a$) the Secretary of State has issued a certificate of recoverable benefits or certificate of recoverable lump sum payments; and

($b$) that certificate is accompanied by a notice to the effect that there is a right of appeal in relation to the certificate only if the Secretary of State has considered an application for review of the certificate.
\end{enumerate}

(2) In a case to which this regulation applies, a person has a right of appeal under section 11 of the 1997 Act against the certificate only if the Secretary of State has considered an application for review of the certificate under section 10 of that Act.

\amendment{
Reg. 9ZB inserted (28.10.13) by the Social Security, Child Support, Vaccine Damage and Other Payments (Decisions and Appeals) (Amendment) Regulations 2013 reg.~4(6).
}

% Reg 9A inserted (20.5.02) by SI 2002/1379 reg 4
\subsubsection[9A. Correction of accidental errors]{Correction of accidental errors}

9A.---(1)  Accidental errors in a decision of the Secretary of State or an officer of the Board under a relevant enactment within the meaning of section 28(3), or in any record of such a decision, may be corrected by the Secretary of State or an officer of the Board, as the case may be, at any time.

(2) A correction made to, or to the record of, a decision shall be deemed to be part of the decision, or of that record, and the Secretary of State or an officer of the Board shall give a written notice of the correction as soon as practicable to the claimant.

(3) In calculating the time within which an application can be made under regulation 3(1)($b$)  for a decision to be revised
%, or the time within which an appeal may be brought under regulation 31(1),  % Words omitted (3.11.08) by SI 2008/2683 Sch 1 para 108
there shall be disregarded any day falling before the day on which notice was given of a correction of the decision or to the record thereof under paragraph (2).

\amendment{
Reg. 9A inserted (20.5.02) by the Social Security and Child Support (Decisions and Appeals) (Miscellaneous Amendments) Regulations 2002 reg. 4.

Words omitted in reg. 9A(3) (3.11.08) by the Tribunals, Courts and Enforcement Act 2007 (Transitional and Consequential Provisions) Order 2008 Sch. 1 para. 108.
}

% Reg 9B inserted by SI 2015/338 reg 6(3)
\subsubsection[9B. Correction of accidental errors in Child Support Decisions]{Correction of accidental errors in Child Support Decisions}

9B.—(1) An accidental error in a decision of the Secretary of State made under the Child Support Act 1991, or in any record of such a decision, may be corrected by the Secretary of State at any time.

(2) Such a correction is to be treated as part of that decision or of that record.

(3) The Secretary of State must give written notice of the correction as soon as practicable to the persons to whom notice of the decision was required to be given.

(4) In calculating the time within which an application may be made under regulation 3A(1)($a$)  (revision of child support decisions) for a decision to be revised, no account is to be taken of any day falling before the day on which notice of any correction was given.

\amendment{
Reg. 9B inserted (23.3.15) by the Child Support (Miscellaneous and Consequential Amendments) Regulations 2015 reg. 6(3).
}

%\subsubsection[10. Effect of a determination as to capacity for work]{Effect of a determination as to capacity for work}
%
%10.  A determination (including a determination made following a change of circumstances) whether a person is, or is to be treated as, capable or incapable of work which is embodied in or necessary to a decision under Chapter II of Part I of the Act or on which such a decision is based shall be conclusive for the purposes of any further such decision.

% Reg 10 substituted (27.7.08) by SI 2008/1554 reg 36
\subsubsection[10. Effect of determination as to capacity or capability for work]{Effect of determination as to capacity or capability for work}

10.---(1)  This regulation applies to a determination whether a person—
\begin{enumerate}\item[]
($a$) is capable or incapable of work;

($b$) is to be treated as capable or incapable of work;

($c$) has or does not have limited capability for work; or

($d$) is to be treated as having or not having limited capability for work.
\end{enumerate}

(2) A determination (including a determination made following a change of circumstances) as set out in paragraph (1) which is embodied in or necessary to a decision under Chapter II of Part I of the Act or on which such a decision is based shall be conclusive for the purposes of any further decision.

\amendment{
Reg. 10 substituted (27.7.08) by the Employment and Support Allowance (Consequential Provisions) (No. 2) Regulations 2008 reg. 36.
}

\subsubsection[11. Secretary of State to determine certain matters]{Secretary of State to determine certain matters}

11.  Where, in relation to a determination for any purpose to which Part XIIA of the Contributions and Benefits Act 
or Part I of the Welfare Reform Act  % Words inserted (27.7.08) by SI 2008/1554 reg 37(a)
applies, an issue arises as to—
\begin{enumerate}\item[]
($a$) whether a person is, or is to be treated as, capable or incapable of work in respect of any period; or

% Reg 11(aa) inserted (27.7.08) by SI 2008/1554 reg 37(b)
($aa$) whether a person is, or is to be treated as, having or not having limited capability for work; or

($b$) whether a person is terminally ill,
\end{enumerate}
that issue shall be determined by the Secretary of State, notwithstanding that other matters fall to be determined by another authority.

\amendment{
Words inserted in reg. 11 and reg. 11(aa) inserted (27.7.08) by the Employment and Support Allowance (Consequential Provisions) (No. 2) Regulations 2008 reg. 37.
}

%Reg 11A inserted (5.7.99) by SI 1999/1670 reg 2(3)
\subsubsection[11A. Issues for decision by officers of Inland Revenue]{Issues for decision by officers of Inland Revenue}

11A.---(1)  Where, on consideration of any claim or other matter, it appears to the Secretary of State that an issue arises which, by virtue of section 8 of the Transfer Act, falls to be decided by an officer of the Board, he shall refer that issue to the Board.

(2) Where—
\begin{enumerate}\item[]
($a$) the Secretary of State has decided any claim or other matter on an assumption of facts—
\begin{enumerate}\item[]
(i) as to which there appeared to him to be no dispute, but

(ii) concerning which, had an issue arisen, that issue would have fallen, by virtue of section 8 of the Transfer Act, to be decided by an officer of the Board; and
\end{enumerate}

($b$) an application for revision or an application for supersession 
or an appeal  % Words inserted (20.5.02) by SI 2002/1379 reg 5(a)
is made in relation to the decision of that claim or other matter; and

($c$) it appears to the Secretary of State on 
%consideration of the application 
receipt of the application or appeal  % Words substituted (20.5.02) by SI 2002/1379 reg 5(b)
that such an issue arises,
\end{enumerate}
he shall refer that issue to the Board.

(3) Pending the final decision of any issue which has been referred to the Board in accordance with paragraph (1) or (2) above, the Secretary of State may—
\begin{enumerate}\item[]
($a$) determine any other issue arising on consideration of the claim or other matter or, as the case may be, of the application,

($b$) seek a preliminary opinion of the Board on the issue referred and decide the claim or other matter or, as the case may be, the application in accordance with that opinion on that issue; or

($c$) defer making any decision on the claim or other matter or, as the case may be, the application.
\end{enumerate}

(4) On receipt by the Secretary of State of the final decision of an issue which has been referred to the Board in accordance with paragraph (1) or (2) above, the Secretary of State shall—
\begin{enumerate}\item[]
($a$) in a case to which paragraph (3)($b$) above applies—
\begin{enumerate}\item[]
(i) consider whether the decision ought to be revised under section 9 or superseded under section 10, and

(ii) if so, revise it, or, as the case may be, make a further decision which supersedes it; or
\end{enumerate}

($b$) in a case to which paragraph (3)($a$) or ($c$) above applies, decide the claim or other matter or, as the case may be, the application,
\end{enumerate}
in accordance with the final decision of the issue so referred.

(5) In paragraphs (3) and (4) above “final decision” means the decision of an officer of the Board under section 8 of the Transfer Act or the determination of any appeal in relation to that decision.

\amendment{
Reg. 11A inserted (5.7.99) by the Social Security and Child Support (Decisions and Appeals) Amendment (No. 3) Regulations 1999 reg. 2(3).

Words inserted in reg. 11A(2)(b) and words substituted in reg. 11A(2)(c) (20.5.02) by the Social Security and Child Support (Decisions and Appeals) (Miscellaneous Amendments) Regulations 2002 reg. 5.
}

\subsubsection[12. Decision of the Secretary of State relating to industrial injuries benefit]{Decision of the Secretary of State relating to industrial injuries benefit}

12.—(1) This regulation applies where, for the purpose of a decision of the Secretary of State relating to a claim for industrial injuries benefit under Part V of the Contributions and Benefits Act an issue to be decided is—
\begin{enumerate}\item[]
($a$) the extent of a personal injury for the purposes of section 94 of that Act;

($b$) whether the claimant has a disease prescribed for the purposes of section 108 of that Act or the extent of any disablement resulting from such a disease; or

($c$) whether the claimant has a disablement for the purposes of section 103 of that Act or the extent of any such disablement.
\end{enumerate}

(2) In connection with making a decision to which this regulation applies, the Secretary of State may refer an issue, together with any relevant evidence or information available to him, including any evidence or information provided by or on behalf of the claimant, to a 
%medical practitioner 
health care professional approved by the Secretary of State  % Words substituted (3.7.07) by SI 2007/1626 reg 4(2)(a)
who has experience in such of the issues specified in paragraph (1) as are relevant to the decision, for such report as appears to the Secretary of State to be necessary for the purpose of providing him with information for use in making the decision.

(3) In making a decision to which this regulation applies, the Secretary of State shall have regard to (among other factors)—
\begin{enumerate}\item[]
($a$) all relevant medical reports provided to him in connection with that decision; and

($b$) the experience, in such of the issues specified in paragraph (1) as are relevant to the decision, of any 
%medical practitioner 
health care professional  % Words substituted (3.7.07) by SI 2007/1626 reg 4(2)(b)
who has provided a report, including a 
%medical practitioner 
health care professional approved by the Secretary of State  % Words substituted (3.7.07) by SI 2007/1626 reg 4(2)(b)
who has provided a report following an examination required by the Secretary of State under section 19.
\end{enumerate}

\amendment{
Words substituted in reg. 12(2), (3)(b) (3.7.07) by the Social Security (Miscellaneous Amendments) (No. 2) Regulations 2007 reg. 4(2).
}

% Reg 12A inserted (19.6.00) by SI 2000/1596 reg 19
\subsubsection[12A. Recrudescence of a prescribed disease]{Recrudescence of a prescribed disease}

12A.---(1)  This regulation applies to a decision made under sections 108 to 110 of the Contributions and Benefits Act where a disease is subsequently treated as a recrudescence under regulation 7 of the Social Security (Industrial Injuries) (Prescribed Diseases) Regulations 1985\footnote{\frenchspacing S.I. 1985/967.}.

(2) Where this regulation applies Chapter II of Part I of the Act shall apply as if section 8(2) did not apply.

\amendment{
Reg. 12A inserted (19.6.00) by the Social Security and Child Support (Miscellaneous Amendments) Regulations 2000 reg. 19.
}

\subsubsection[13. Income support and social fund determinations on incomplete evidence]{\sloppy Income support and social fund determinations on incomplete evidence}

13.—(1) Where, for the purpose of a decision under section 8 or 10—
\begin{enumerate}\item[]
%($a$) a determination falls to be made by the Secretary of State as to what housing costs are to be included in a claimant’s applicable amount by virtue of regulation 17(1)($e$) or 18(1)($f$) of, and Schedule 3 to, the Income Support Regulations; and

% Reg 13(1)(a) substituted (7.4.03) by SI 2002/3019 reg 19(a)(i)
($a$) a determination falls to be made by the Secretary of State as to what housing costs are to be included in—
\begin{enumerate}\item[]
(i) a claimant’s applicable amount by virtue of regulation 17(1)($e$)  or 18(1)($f$)  of, and Schedule 3 to, the Income Support Regulations; 
%or  % Word omitted (27.7.08) by SI 2008/1554 reg 38(a)(i)

(ii) a claimant’s appropriate minimum guarantee by virtue of regulation 6(6)($c$)  and Schedule II to the State Pension Credit Regulations; 
%and
%
% Reg 13(1)(a)(iii) inserted by way of substitution (27.7.08) by SI 2008/1554 reg 38(a)(ii)
or

(iii) a claimant’s applicable amount under regulation 67(1)($c$)  or 68(1)($d$)  of the Employment and Support Allowance Regulations; and
\end{enumerate}

($b$) it appears to the Secretary of State that he is not in possession of all of the evidence or information which is relevant for the purposes of such a determination,
\end{enumerate}
he shall make the determination on the assumption that the housing costs to be included in the claimant’s 
%applicable amount are those 
applicable amount or, as the case may be, appropriate minimum guarantee are those  % Words substituted (7.4.03) by SI 2002/3019 reg 19(a)(ii)
that can be immediately determined.

(2) Where, for the purpose of a decision under section 8 or 10—
\begin{enumerate}\item[]
($a$) a determination falls to be made by the Secretary of State as to whether—
\begin{enumerate}\item[]
(i) in relation to any person, the applicable amount falls to be reduced or disregarded to any extent by virtue of section 126(3) of the Contributions and Benefits Act (persons affected by trade disputes);

(ii) for the purposes of regulation 12 of the Income Support Regulations, a person is by virtue of that regulation to be treated as receiving relevant education; 
%or  % Word omitted (27.7.08) by SI 2008/1554 reg 38(b)(i)

(iii) in relation to any claimant, the applicable amount includes severe disability premium by virtue of regulation 17(1)($d$) or 18(1)($e$), and paragraph 13 of Schedule 2 to, the Income Support Regulations; 
%and
%
% Reg 13(2)(a)(iv) inserted by way of substitution (27.7.08) by SI 2008/1554 reg 38(b)(ii)
or

(iv) in relation to any claimant, the applicable amount includes the severe disability premium by virtue of regulation 67(1) or 68(1) of, and paragraph 6 of Schedule 4 to, the Employment and Support Allowance Regulations; and
\end{enumerate}

($b$) it appears to the Secretary of State that he is not in possession of all of the evidence or information which is relevant for the purposes of such a determination,
\end{enumerate}
he shall make the determination on the assumption that the relevant evidence or information which is not in his possession is adverse to the claimant.

% Reg 13(3) added (7.4.03) by SI 2002/3019 reg 19(b)
(3) Where, for the purposes of a decision under section 8 or 10—
\begin{enumerate}\item[]
($a$) a determination falls to be made by the Secretary of State as to whether a claimant’s appropriate minimum guarantee includes an additional amount in accordance with regulation 6(4) of, and paragraph 1 of Schedule I to, the State Pension Credit Regulations; and

($b$) it appears to the Secretary of State that he is not in possession of all the evidence or information which is relevant for the purpose of such a determination,
\end{enumerate}
he shall make the determination on the assumption that the relevant evidence or information which is not in his possession is adverse to the claimant.

\amendment{
Words substituted in reg. 13(1), reg. 13(3) added and reg. 13(1)(a) substituted (7.4.03) by the State Pension Credit (Consequential, Transitional and Miscellaneous Provisions) Regulations 2002 reg. 19.

Reg. 13(1)(a)(iii), (2)(a)(iv) inserted (27.7.08) by the Employment and Support Allowance (Consequential Provisions) (No. 2) Regulations 2008 reg. 38.
}

% Reg 13A inserted (6.4.06) by SI 2005/2677 reg 9(6)
\subsubsection[13A. Retirement pension after period of deferment]{Retirement pension after period of deferment}

13A.---(1)  This regulation applies where—
\begin{enumerate}\item[]
($a$) a person claims a Category A or Category B retirement pension, shared additional pension or, as the case may be, graduated retirement benefit;

($b$) an election is required by, as the case may be—
\begin{enumerate}\item[]
(i) paragraph A1 or 3C of Schedule 5 to the Contributions and Benefits Act (pension increase or lump sum where entitlement to retirement pension is deferred);

(ii) paragraph 1 of Schedule 5A to that Act (pension increase or lump sum where entitlement to shared additional pension is deferred); or, as the case may be,\looseness=1

(iii) paragraph 12 or 17 of Schedule 1 to the Graduated Retirement Benefit Regulations (further provisions replacing section 36(4) of the National Insurance Act 1965: increases of graduated retirement benefit and lump sums); and\looseness=-1 
\end{enumerate}

($c$) no election is made when the claim is made.
\end{enumerate}

(2) In the circumstances specified in paragraph (1) the Secretary of State may decide the claim before any election is made, or is treated as made, for an increase or lump sum.

(3) When an election is made, or is treated as made, the Secretary of State shall revise the decision which he made in pursuance of paragraph~(2).\looseness=1

\amendment{
Reg. 13A inserted (6.4.06) by the Social Security (Deferral of Retirement Pensions, Shared Additional Pension and Graduated Retirement Benefit) (Miscellaneous Provisions) Regulations 2005 reg. 9(5).
}

% Reg 13B inserted by SI 2015/1985 art 18(7)
\subsubsection[13B. State pension under Part I of the Pensions Act 2014 after period of deferment]{State pension under Part I of the Pensions Act 2014 after period of deferment}

13B.—(1) This regulation applies where—
\begin{enumerate}\item[]
($a$) a person claims a state pension under Part I of the Pensions Act 2014;

($b$) the person may make a choice under—
\begin{enumerate}\item[]
(i) section 8(2) of the Pensions Act 2014; or

(ii) regulations made under section 10 of that Act which make provision corresponding or similar to section 8(2); and
\end{enumerate}

($c$) the person does not make such a choice when the claim is made.
\end{enumerate}

(2) The Secretary of State may decide the claim before paragraph (4) applies.

(3) The Secretary of State may revise a decision under paragraph (2) where paragraph (4) applies.

(4) This paragraph applies where the person—
\begin{enumerate}\item[]
($a$) makes a choice mentioned in paragraph (1)($b$); or

($b$) becomes entitled to a lump sum under section 8(4) of the Pensions Act 2014, or under regulations made under section 10 of that Act which make provision corresponding or similar to section 8(4), because the person has failed to choose within the period mentioned in section 8(3).
\end{enumerate}

\amendment{
Reg. 13B inserted (6.4.16) by the Pensions Act 2014 (Consequential, Supplementary and Incidental Amendments) Order 2015 art.~18(7).
}


\subsubsection[14. Effect of alteration in the component rates of income support and jobseeker’s allowance]{\sloppy Effect of alteration in the component rates of income support and jobseeker’s allowance}

14.—(1) Section 159 of the Administration Act (effect of alteration in the component rates of income support) shall not apply to any award of income support in force in favour of a person where there is applicable to that person—%\looseness=-1
\begin{enumerate}\item[]
($a$) any amount determined in accordance with regulation 17(2) to (7) of the Income Support Regulations; or

($b$) any protected sum determined in accordance with Schedule 3A or~3B of those Regulations\footnote{\frenchspacing Schedule 3A was inserted by S.I. 1988/1445; Schedule 3B was inserted by S.I. 1989/534.}; or

($c$) any transitional addition, personal expenses addition or special transitional addition applicable under Part II of the Income Support (Transitional) Regulations 1987\footnote{\frenchspacing S.I. 1987/1969.} (transitional protection).
\end{enumerate}

(2) Where section 159 of the Administration Act does not apply to an award of income support by virtue of paragraph (1), a decision under section~10 may be made in respect of that award for the sole purpose of giving effect to any change made by an order under section 150 of the Administration Act.\looseness=1

(3) Section 159A of the Administration Act\footnote{\frenchspacing S.~159A was inserted by section 24 of the Jobseekers Act 1995 (c.~18).\looseness=-1} (effect of alterations in the component rates of jobseeker’s allowance) shall not apply to any award of a jobseeker’s allowance in force in favour of a person where there is applicable to that person any amount determined in accordance with regulation 87 of the Jobseeker’s Allowance Regulations.

(4) Where section 159A of the Administration Act does not apply to an award of a jobseeker’s allowance by virtue of paragraph (3), a decision under section 10 may be made in respect of that award for the sole purpose of giving effect to any change made by an order under section 150 of the Administration Act.\looseness=-1

% Reg 14(5), (6) added (7.4.03) by SI 2002/3019 reg 20
(5) Section 159B of the Administration Act\footnote{Section 159B was inserted by the State Pension Credit Act 2002 (c.\ 16), section 14 and Schedule 2, paragraph 17.} (effect of alterations affecting state pension credit) shall not apply to any award of state pension credit in favour of a person where in relation to that person the appropriate minimum guarantee includes an amount determined under paragraph 6 of Part III of Schedule I to the State Pension Credit Regulations.

(6) Where section 159B of the Administration Act does not apply to an award of state pension credit by virtue of paragraph (5), a decision under section 10 may be made in respect of that award for the sole purpose of giving effect to any change made to an award under section 150 of the Administration Act.

\amendment{
Reg. 14(5), (6) added (7.4.03) by the State Pension Credit (Consequential, Transitional and Miscellaneous Provisions) Regulations 2002 reg. 20.
}

% Reg 14A inserted (20.5.02) by SI 2002/1379 reg 6
\subsubsection[14A. Termination of award of income support%
% or jobseeker’s allowance
, jobseeker’s allowance or employment and support allowance%  % Words substituted (27.7.08) by SI 2008/1554 reg 39(2)
]{Termination of award of income support%
% or jobseeker’s allowance
, jobseeker’s allowance or employment and support allowance%  % Words substituted (27.7.08) by SI 2008/1554 reg 39(2)
}

14A.---(1)  This regulation applies in a case where an award of income support% 
%or a jobseeker’s allowance 
, a jobseeker’s allowance or an employment and support allowance  % Words substituted (27.7.08) by SI 2008/1554 reg 39(3)(a)
(“the existing benefit”) exists in favour of a person and, if that award did not exist and a claim was made by that person or his partner for 
an employment and support allowance,  % Words inserted (27.7.08) by SI 2008/1554 reg 39(3)(b)
a jobseeker’s allowance or, as the case may be, income support (“the alternative benefit”), an award of the alternative benefit would be made on that claim.

(2) In a case to which this regulation applies, if a claim for the alternative benefit is made the Secretary of State may bring to an end the award of the existing benefit if he is satisfied that an award of the alternative benefit will be made on that claim.

(3) Where, under paragraph (2), the Secretary of State brings an award of the existing benefit to an end he shall do so with effect from the day immediately preceding the first day on which an award of the alternative benefit takes effect.

(4) Where an award of a jobseeker’s allowance is made in accordance with the provisions of this regulation, paragraph 4 of Schedule 1 to the Jobseekers Act (waiting days) shall not apply.

% Reg 14A(5) added (27.7.08) by SI 2008/1554 reg 39(4)
(5) Where an award of an employment and support allowance is made in accordance with the provisions of this regulation, paragraph 2 of Schedule 2 to the Welfare Reform Act (waiting days) shall not apply.

\amendment{
Reg. 14A inserted (20.5.02) by the Social Security and Child Support (Decisions and Appeals) (Miscellaneous Amendments) Regulations 2002 reg. 6.

Words inserted in reg. 14A(1), words substituted in reg. 14A(1) and heading and reg. 14A(5) added (27.7.08) by the Employment and Support Allowance (Consequential Provisions) (No. 2) Regulations 2008 reg. 39.
}

\subsubsection[15. Jobseeker’s allowance determinations on incomplete evidence]{Jobseeker’s allowance determinations on incomplete evidence}

15.  Where, for the purpose of a decision under section 8 or 10—
\begin{enumerate}\item[]
($a$) a determination falls to be made by the Secretary of State as to whether—
\begin{enumerate}\item[]
(i) in relation to any person, the applicable amount falls to be reduced or disregarded to any extent by virtue of section 15 of the Jobseekers Act (persons affected by trade disputes); or

(ii) for the purposes of regulation 54(2) to (4) of the Jobseeker’s Allowance Regulations (relevant education), a person is by virtue of that regulation, to be treated as receiving relevant education; and
\end{enumerate}

($b$) it appears to the Secretary of State that he is not in possession of all of the evidence or information which is relevant for the purposes of such a determination,
\end{enumerate}
he shall make the determination on the assumption that the relevant evidence or information which is not in his possession is adverse to the claimant.

% Regs 15A--15D inserted (3.3.03 for new-rules cases only) by SI 2000/3185 reg 10
\subsubsection[15A. Provision of information]{Provision of information\\*\emph{2003 scheme only}}

15A.---(1)  Where the 
%Secretary of State 
%Commission  % Words substituted (6.4.09) by SI 2009/396 reg 4(8)
Secretary of State  % Words substituted (1.8.12) by SI 2012/2007 Sch para 113(7)
has received an application under section 16 or 17 of the Child Support Act in connection with a previously determined variation which has effect on the maintenance calculation in force, 
%he 
%it  % Word substituted (6.4.09) by SI 2009/396 reg 4(8)
the Secretary of State  % Words substituted (1.8.12) by SI 2012/2007 Sch para 113(7)
may request further information or evidence from the applicant to enable a decision on that application to be made and any such information or evidence shall be provided within one month of the date of notification of the request, or such longer period as the 
%Secretary of State 
%Commission  % Words substituted (6.4.09) by SI 2009/396 reg 4(8)
Secretary of State  % Words substituted (1.8.12) by SI 2012/2007 Sch para 113(7)
is satisfied is reasonable in the circumstances of the case.

(2) Where any information or evidence requested in accordance with paragraph (1) is not provided within the time limit specified in that paragraph, the 
%Secretary of State 
%Commission  % Words substituted (6.4.09) by SI 2009/396 reg 4(8)
Secretary of State  % Words substituted (1.8.12) by SI 2012/2007 Sch para 113(7)
may, where 
%he 
%it  % Word substituted (6.4.09) by SI 2009/396 reg 4(8)
the Secretary of State  % Words substituted (1.8.12) by SI 2012/2007 Sch para 113(7)
is able to do so, proceed to make the decision in the absence of that information or evidence.

\amendment{
Reg. 15A inserted (3.3.03 for new-rules cases only) by the Child Support (Decisions and Appeals) (Amendment) Regulations 2000 reg. 10 (subject to reg. 1(2)).

Words substituted in reg. 15A (6.4.09) by the Child Support (Miscellaneous Amendments) Regulations 2009 reg. 4(8).

Words substituted in reg. 15A (1.8.12) by the Public Bodies (Child Maintenance and Enforcement Commission: Abolition and Transfer of Functions) Order 2012 Sch. para. 113(7).

Reg. 15A omitted (10.12.12 for 2012 scheme cases only) by the Child Support (Meaning of Child and New Calculation Rules) (Consequential and Miscellaneous Amendment) Regulations 2012 reg. 6(3).
}

\subsubsection[15B. Procedure in relation to an application made under section 16 or 17 of the Child Support Act in connection with a previously determined variation]{Procedure in relation to an application made under section 16 or 17 of the Child Support Act in connection with a previously determined variation\\*\emph{2003 scheme only}}

15B.---(1)  Subject to paragraph (3), where the 
%Secretary of State 
%Commission  % Words substituted (6.4.09) by SI 2009/396 reg 4(9)
Secretary of State  % Words substituted (1.8.12) by SI 2012/2007 Sch para 113(8)(a)
has received an application under section 16 or 17 of the Child Support Act in connection with a previously determined variation which has effect on the maintenance calculation in force, 
%he
%it%  % Word substituted (6.4.09) by SI 2009/396 reg 4(9)
the Secretary of State%  % Words substituted (1.8.12) by SI 2012/2007 Sch para 113(8)(a)
—
\begin{enumerate}\item[]
($a$) shall give notice of the application to the relevant persons, other than the applicant, informing them of the grounds on which the application has been made and any relevant information or evidence the applicant has given, except information or evidence falling within paragraph (2);

($b$) may invite representations, which need not be in writing but shall be in writing if in any case 
%he 
%it  % Word substituted (6.4.09) by SI 2009/396 reg 4(9)
the Secretary of State  % Words substituted (1.8.12) by SI 2012/2007 Sch para 113(8)(a)
so directs, from the relevant persons other than the applicant on any matter relating to that application, to be submitted to the 
%Secretary of State 
%Commission  % Words substituted (6.4.09) by SI 2009/396 reg 4(9)
Secretary of State  % Words substituted (1.8.12) by SI 2012/2007 Sch para 113(8)(a)
within 14 days of notification or such longer period as the 
%Secretary of State 
%Commission  % Words substituted (6.4.09) by SI 2009/396 reg 4(9)
Secretary of State  % Words substituted (1.8.12) by SI 2012/2007 Sch para 113(8)(a)
is satisfied is reasonable in the circumstances of the case; and

($c$) shall set out the provisions of paragraphs (2)($b$)  and ($c$), (4) and (5) in relation to such representations.
\end{enumerate}

(2) The information or evidence referred to in paragraphs (1)($a$), (4)($a$)  and (7), is—
\begin{enumerate}\item[]
($a$) details of the nature of the long-term illness or disability of the relevant other child which forms the basis of a variation application on the ground in regulation 11 of the Variations Regulations (special expenses—illness or disability of relevant other child) where the applicant requests they should not be disclosed and the 
%Secretary of State 
%Commission  % Words substituted (6.4.09) by SI 2009/396 reg 4(9)
Secretary of State  % Words substituted (1.8.12) by SI 2012/2007 Sch para 113(8)(b)
is satisfied that disclosure is not necessary in order to be able to determine the application;

($b$) medical evidence or medical advice which has not been disclosed to the applicant or a relevant person and which the 
%Secretary of State 
%Commission  % Words substituted (6.4.09) by SI 2009/396 reg 4(9)
Secretary of State  % Words substituted (1.8.12) by SI 2012/2007 Sch para 113(8)(b)
considers would be harmful to the health of the applicant or that relevant person if disclosed to him;

($c$) the address of a relevant person or qualifying child, or any other information which could reasonably be expected to lead to that person or child being located, where the 
%Secretary of State 
%Commission  % Words substituted (6.4.09) by SI 2009/396 reg 4(9)
Secretary of State  % Words substituted (1.8.12) by SI 2012/2007 Sch para 113(8)(b)
considers that there would be a risk of harm or undue distress to that person or that child or any other children living with that person if the address or information were disclosed.
\end{enumerate}

(3) The 
%Secretary of State 
%Commission  % Words substituted (6.4.09) by SI 2009/396 reg 4(9)
Secretary of State  % Words substituted (1.8.12) by SI 2012/2007 Sch para 113(8)(c)(i)
need not act in accordance with paragraph (1) if—
\begin{enumerate}\item[]
($a$) 
%he 
%it  % Word substituted (6.4.09) by SI 2009/396 reg 4(9)
%is satisfied on the information or evidence available to him, that 
%%he 
%it  % Word substituted (6.4.09) by SI 2009/396 reg 4(9)
%will not agree to a variation of the maintenance calculation in force
satisfied on the information or evidence available that a variation of the maintenance calculation in force will not be agreed%  % Words substituted (1.8.12) by SI 2012/2007 Sch para 113(8)(c)(ii)(aa)
, but if, on further consideration 
%he 
%it  % Word substituted (6.4.09) by SI 2009/396 reg 4(9)
the Secretary of State  % Words substituted (1.8.12) by SI 2012/2007 Sch para 113(8)(c)(ii)(bb)
is minded to do so 
%he 
%it  % Word substituted (6.4.09) by SI 2009/396 reg 4(9)
the Secretary of State  % Words substituted (1.8.12) by SI 2012/2007 Sch para 113(8)(c)(ii)(bb)
shall, before doing so, comply with the provisions of this regulation; and

($b$) were the application to succeed, the decision as revised or superseded would be less advantageous to the applicant than the decision before it was so revised or superseded.
\end{enumerate}

(4) Where the 
%Secretary of State 
%Commission  % Words substituted (6.4.09) by SI 2009/396 reg 4(9)
Secretary of State  % Words substituted (1.8.12) by SI 2012/2007 Sch para 113(8)(d)(i)
receives representations from the relevant persons 
%he
%it%  % Word substituted (6.4.09) by SI 2009/396 reg 4(9)
the Secretary of State%  % Words substituted (1.8.12) by SI 2012/2007 Sch para 113(8)(d)(ii)
—
\begin{enumerate}\item[]
($a$) may, if 
%he 
%it  % Word substituted (6.4.09) by SI 2009/396 reg 4(9)
the Secretary of State  % Words substituted (1.8.12) by SI 2012/2007 Sch para 113(8)(d)(iii)(aa)
considers it reasonable to do so, send a copy of the representations concerned (excluding material falling within paragraph (2) above) to the applicant and invite any comments 
%he may have 
to be provided  % Words substituted (1.8.12) by SI 2012/2007 Sch para 113(8)(d)(iii)(bb)
within 14 days or such longer period as the 
%Secretary of State 
%Commission  % Words substituted (6.4.09) by SI 2009/396 reg 4(9)
Secretary of State  % Words substituted (1.8.12) by SI 2012/2007 Sch para 113(8)(c)(i)
is satisfied is reasonable in the circumstances of the case; and

($b$) where the 
%Secretary of State 
%Commission  % Words substituted (6.4.09) by SI 2009/396 reg 4(9)
Secretary of State  % Words substituted (1.8.12) by SI 2012/2007 Sch para 113(8)(c)(i)
acts under sub-paragraph ($a$), shall not proceed to make a decision in response to the application until 
%he 
%it  % Word substituted (6.4.09) by SI 2009/396 reg 4(9)
the Secretary of State  % Words substituted (1.8.12) by SI 2012/2007 Sch para 113(8)(d)(iv)
has received such comments or the period referred to in sub-paragraph ($a$)  has expired.
\end{enumerate}

(5) Where the 
%Secretary of State 
%Commission  % Words substituted (6.4.09) by SI 2009/396 reg 4(9)
Secretary of State  % Words substituted (1.8.12) by SI 2012/2007 Sch para 113(8)(e)
has not received representations from the relevant persons notified in accordance with paragraph (1) within the time limit specified in sub-paragraph ($b$)  of that paragraph, 
%he 
%it  % Word substituted (6.4.09) by SI 2009/396 reg 4(9)
the Secretary of State  % Words substituted (1.8.12) by SI 2012/2007 Sch para 113(8)(e)
may proceed to make a decision under section 16 or 17 of the Child Support Act in response to the application, in their absence.

(6) In considering an application for a revision or supersession the 
%Secretary of State 
%Commission  % Words substituted (6.4.09) by SI 2009/396 reg 4(9)
Secretary of State  % Words substituted (1.8.12) by SI 2012/2007 Sch para 113(8)(e)
shall take into account any representations received at the date upon which 
%he 
%it  % Word substituted (6.4.09) by SI 2009/396 reg 4(9)
the Secretary of State  % Words substituted (1.8.12) by SI 2012/2007 Sch para 113(8)(e)
makes a decision under section 16 or 17 of the Child Support Act, from the relevant persons including any representations received in connection with the application in accordance with paragraphs (1)($b$), (4)($a$)  and (7).

(7) Where any information or evidence requested by the 
%Secretary of State 
%Commission  % Words substituted (6.4.09) by SI 2009/396 reg 4(9)
Secretary of State  % Words substituted (1.8.12) by SI 2012/2007 Sch para 113(8)(f)
under regulation 15A is received after notification has been given under paragraph (1), 
%he 
%it  % Word substituted (6.4.09) by SI 2009/396 reg 4(9)
the Secretary of State  % Words substituted (1.8.12) by SI 2012/2007 Sch para 113(8)(f)
may, if 
%he 
%it  % Word substituted (6.4.09) by SI 2009/396 reg 4(9)
the Secretary of State  % Words substituted (1.8.12) by SI 2012/2007 Sch para 113(8)(f)
considers it reasonable to do so and except where such information or evidence falls within paragraph (2), send a copy of such information or evidence to the relevant persons and may invite them to submit representations, which need not be in writing unless the 
%Secretary of State 
%Commission  % Words substituted (6.4.09) by SI 2009/396 reg 4(9)
Secretary of State  % Words substituted (1.8.12) by SI 2012/2007 Sch para 113(8)(f)
so directs in any particular case, on that information or evidence.

(8) Where the 
%Secretary of State 
%Commission  % Words substituted (6.4.09) by SI 2009/396 reg 4(9)
Secretary of State  % Words substituted (1.8.12) by SI 2012/2007 Sch para 113(8)(g)
is considering making a decision under section 16 or 17 of the Child Support Act in accordance with this regulation, 
%he 
%it  % Word substituted (6.4.09) by SI 2009/396 reg 4(9)
the Secretary of State  % Words substituted (1.8.12) by SI 2012/2007 Sch para 113(8)(g)
shall apply the factors to be taken into account for the purposes of section 28F of the Child Support Act set out in regulation 21 of the Variations Regulations (factors to be taken into account and not to be taken into account) as factors to be taken into account and not to be taken into account when considering making a decision under this regulation.

(9) In this regulation “relevant person” means—
\begin{enumerate}\item[]
($a$) a non-resident parent, or a person treated as a non-resident parent under regulation 8 of the Maintenance Calculations and Special Cases Regulations (persons treated as non-resident parents), whose liability to pay child support maintenance may be affected by any variation agreed;

($b$) a person with care, or a child to whom section 7 of the Child Support Act applies, where the amount of child support maintenance payable by virtue of a calculation relevant to that person with care or in respect of that child may be affected by any variation agreed.
\end{enumerate}

\amendment{
Reg. 15B inserted (3.3.03 for new-rules cases only) by the Child Support (Decisions and Appeals) (Amendment) Regulations 2000 reg. 10 (subject to reg. 1(2)).

Words substituted in reg. 15B (6.4.09) by the Child Support (Miscellaneous Amendments) Regulations 2009 reg. 4(9).

Words substituted in reg. 15B(1), (2), (3), (4), (5), (6), (7), (8) (1.8.12) by the Public Bodies (Child Maintenance and Enforcement Commission: Abolition and Transfer of Functions) Order 2012 Sch. para. 113(8).

Reg. 15B omitted (10.12.12 for 2012 scheme cases only) by the Child Support (Meaning of Child and New Calculation Rules) (Consequential and Miscellaneous Amendment) Regulations 2012 reg. 6(3).
}

\subsubsection[15C. Notification of a decision made under section 16 or 17 of the Child Support Act]{Notification of a decision made under section 16 or 17 of the Child Support Act\\*\emph{2003 scheme only}}

15C.---(1)  Subject to paragraphs (2) and (5) to (11), a notification of a decision made following the revision or supersession of a decision made under section 11, 12 or 17 of the Child Support Act, whether as originally made or as revised under section 16 of that Act, shall set out, in relation to the decision in question—
\begin{enumerate}\item[]
($a$) the effective date of the maintenance calculation;

($b$) where relevant, the non-resident parent’s net weekly income;

($c$) the number of qualifying children;

($d$) the number of relevant other children;

($e$) the weekly rate;

($f$) the amounts calculated in accordance with Part I of Schedule 1 to the Child Support Act and, where there has been agreement to a variation or a variation has otherwise been taken into account, the Variations Regulations;

($g$) where the weekly rate is adjusted by apportionment or shared care or both, the amount calculated in accordance with paragraph 6, 7 or 8, as the case may be, of Part I of Schedule 1 to the Child Support Act; and

($h$) where the amount of child support maintenance which the non-resident parent is liable to pay is decreased in accordance with regulation 9 of the Maintenance Calculations and Special Cases Regulations (care provided in part by local authority) or 11 (non-resident parent liable to pay maintenance under a maintenance order) of those Regulations, the adjustment calculated in accordance with that regulation.
\end{enumerate}

(2) A notification of a revision or supersession of a maintenance calculation made under section 12(1) of the Child Support Act shall set out the effective date of the maintenance calculation, the default rate, the number of qualifying children on which the rate is based and whether any apportionment has been applied under regulation 7 of the Maintenance Calculation Procedure Regulations (default rate) and shall state the nature of the information required to enable a decision under section 11 of that Act to be made by way of section 16 of that Act.

(3) Except where a person gives written permission to the 
%Secretary of State 
%Commission  % Words substituted (6.4.09) by SI 2009/396 reg 4(10)
Secretary of State  % Words substituted (1.8.12) by SI 2012/2007 Sch para 113(9)(a)
that the information in relation to him, mentioned in sub-paragraphs ($a$)  and ($b$), may be conveyed to other persons, any document given or sent under the provisions of paragraph (1) or (2) shall not contain—
\begin{enumerate}\item[]
($a$) the address of any person other than the recipient of the document in question (other than the address of the office of the officer concerned who is exercising functions of the 
%Secretary of State 
%Commission  % Words substituted (6.4.09) by SI 2009/396 reg 4(10)
Secretary of State  % Words substituted (1.8.12) by SI 2012/2007 Sch para 113(9)(a)
under the Child Support Act) or any other information the use of which could reasonably be expected to lead to any such person being located;

($b$) any other information the use of which could reasonably be expected to lead to any person, other than a qualifying child or a relevant person, being identified.
\end{enumerate}

(4) Where a decision as to the revision or supersession of a decision made under section 11, 12 or 17 of the Child Support Act, whether as originally made or as revised under section 16 of that Act, is made under section 16 or 17 of that Act, a notification under paragraph (1) or (2) shall include information as to the provisions of sections 16, 17 and 20 of that Act.

(5) Where the 
%Secretary of State 
%Commission  % Words substituted (6.4.09) by SI 2009/396 reg 4(10)
Secretary of State  % Words substituted (1.8.12) by SI 2012/2007 Sch para 113(9)(b)
makes a decision that a maintenance calculation shall cease to have effect—
\begin{enumerate}\item[]
($a$) 
%he 
%it  % Word substituted (6.4.09) by SI 2009/396 reg 4(10)
the Secretary of State  % Words substituted (1.8.12) by SI 2012/2007 Sch para 113(9)(b)
shall immediately notify the non-resident parent and person with care, so far as that is reasonably practicable;

($b$) where a decision has been superseded in a case where a child under section 7 of the Child Support Act ceases to be a child for the purposes of that Act, 
%he 
%it  % Word substituted (6.4.09) by SI 2009/396 reg 4(10)
the Secretary of State  % Words substituted (1.8.12) by SI 2012/2007 Sch para 113(9)(b)
shall immediately notify the persons in sub-paragraph ($a$)  and the other qualifying children within the meaning of section 7 of that Act; and

($c$) any notice under sub-paragraphs ($a$)  and ($b$)  shall specify the date with effect from which that decision took effect.
\end{enumerate}

% Reg 15C(6)--(8) omitted (6.4.09) by SI 2009/396 reg 4(10)(b)
%(6) Where the Secretary of State, under the provisions of section 16 or 17 of the Child Support Act, has made a decision that an adjustment shall cease, or adjusted the amount payable under a maintenance calculation, he shall immediately notify the relevant persons, so far as that is reasonably practicable, that the adjustment has ceased or of the amount and period of the adjustment, and the amount payable during the period of the adjustment.
%
%(7) Where the Secretary of State has made a decision under section 16 of the Child Support Act, revising a decision under section 41A or 47 of that Act, he shall immediately notify the relevant persons so far as that is reasonably practicable, of the amount of child support maintenance payable, the amount of arrears, the amount of the penalty payment or fees to be paid, as the case may be, the method of payment and the day by which payment is to be made.
%
%(8) Where the non-resident parent appeals against a decision made by the Secretary of State under section 41A or 47 of the Child Support Act and the Secretary of State makes a decision under section 16 of that Act, before the appeal is decided he shall notify the relevant persons, so far as that is reasonably practicable of either the new amount of the penalty payment or the fee to be paid or that the amount is no longer payable, the method of payment and the day by which payment is to be made.

(9) Paragraphs (1) to (3) shall not apply where the 
%Secretary of State 
%Commission  % Words substituted (6.4.09) by SI 2009/396 reg 4(10)
Secretary of State  % Words substituted (1.8.12) by SI 2012/2007 Sch para 113(9)(b)
has decided not to supersede a decision under section 17 of the Child Support Act, and 
%he 
the Secretary of State  % Words substituted (1.8.12) by SI 2012/2007 Sch para 113(9)(b)
shall, so far as that is reasonably practicable, notify the relevant persons of that decision.

(10) A notification under paragraphs (6) to (9) shall include information as to the provisions of sections 16, 17 and 20 of the Child Support Act.

(11) Where paragraph (9) applies, and the 
%Secretary of State 
%Commission  % Words substituted (6.4.09) by SI 2009/396 reg 4(10)
Secretary of State  % Words substituted (1.8.12) by SI 2012/2007 Sch para 113(9)(b)
decides not to supersede under regulation 6B, 
%he 
%it  % Word substituted (6.4.09) by SI 2009/396 reg 4(10)
the Secretary of State  % Words substituted (1.8.12) by SI 2012/2007 Sch para 113(9)(b)
shall notify the relevant person, in relation to the decision in question of—
\begin{enumerate}\item[]
($a$) the fact that regulation 6B applies to the decision;

($b$) the non-resident parent’s net income figure fixed for the purposes of the maintenance calculation in force in accordance with Part I of Schedule 1 to the Child Support Act;

($c$) the non-resident parent’s net income figure provided by that parent to the 
%Secretary of State 
%Commission  % Words substituted (6.4.09) by SI 2009/396 reg 4(10)
Secretary of State  % Words substituted (1.8.12) by SI 2012/2007 Sch para 113(9)(b)
with the application for supersession under regulation 6A(3);

($d$) the decision of the 
%Secretary of State 
%Commission  % Words substituted (6.4.09) by SI 2009/396 reg 4(10)
Secretary of State  % Words substituted (1.8.12) by SI 2012/2007 Sch para 113(9)(b)
not to supersede; and

($e$) the right to appeal against the decision under section 20 of the Child Support Act.
\end{enumerate}

(12) Where an appeal lapses in accordance with section 16(6) or 28F(5) of the Child Support Act, the Secretary of State shall, so far as that is reasonably practicable, notify the relevant persons that the appeal has lapsed.

\amendment{
Reg. 15C inserted (3.3.03 for new-rules cases only) by the Child Support (Decisions and Appeals) (Amendment) Regulations 2000 reg. 10 (subject to reg. 1(2)).

Words substituted in reg. 15C and reg. 15C(6)--(8) omitted (6.4.09) by the Child Support (Miscellaneous Amendments) Regulations 2009 reg. 4(10).

Words substituted in reg. 15C(3), (5), (9), (11), (12) (1.8.12) by the Public Bodies (Child Maintenance and Enforcement Commission: Abolition and Transfer of Functions) Order 2012 Sch. para. 113(9).

Reg. 15C omitted (10.12.12 for 2012 scheme cases only) by the Child Support (Meaning of Child and New Calculation Rules) (Consequential and Miscellaneous Amendment) Regulations 2012 reg. 6(3).

\medskip

Reg. 15D omitted (6.4.09) by the Child Support (Miscellaneous Amendments) Regulations 2009 reg. 4(11).
}

% Reg 15D omitted (6.4.09) by SI 2009/396 reg 4(11)
%\subsubsection[15D. Procedure in relation to the adjustment of the amount payable under a maintenance calculation]{Procedure in relation to the adjustment of the amount payable under a maintenance calculation}
%
%15D.---(1)  Where the Secretary of State has adjusted the amount payable under a maintenance calculation under the provisions of regulation 10(1) and (3A) of the Arrears, Interest and Adjustment of Maintenance Assessments Regulations and that maintenance calculation is subsequently replaced by a fresh maintenance calculation made by virtue of a revision under section 16 of the Child Support Act or of a decision under section 17 of that Act superseding an earlier decision, that adjustment shall, subject to paragraph (2), continue to apply to the amount payable under that fresh maintenance calculation unless the Secretary of State is satisfied that such adjustment would not be appropriate in all the circumstances of the case.
%\looseness=-1
%
%(2) Where the Secretary of State is satisfied that the adjustment referred to in paragraph (1) would not be appropriate, he may make a decision under section 17 of the Child Support Act, superseding an earlier decision making an adjustment, and—
%\begin{enumerate}\item[]
%($a$) the adjustment shall cease; or
%
%($b$) he may adjust the amount payable under that fresh maintenance calculation,
%\end{enumerate}
%as he sees fit, having regard to the matters specified in regulation 10(1)($b$)(i)  to (iii)  of the Arrears, Interest and Adjustment of Maintenance Assessments Regulations.
%\looseness=-1
%
%\amendment{
%Reg. 15D inserted (3.3.03 for new-rules cases only) by the Child Support (Decisions and Appeals) (Amendment) Regulations 2000 reg. 10 (subject to reg. 1(2)).
%}

\section[Part III --- Suspension, termination and other matters]{Part III\\*Suspension, termination and other matters}

\amendment{
Pt. III revoked (7.4.03) so far as relating to child benefit or guardian's allowance by the Child Benefit and Guardian’s Allowance (Decisions and Appeals) Regulations 2003 reg. 34(a).
}

\subsection[Chapter I --- Suspension and termination]{Chapter I\\*Suspension and termination}

\subsubsection[16. Suspension in prescribed cases]{Suspension in prescribed cases}

\renewcommand\parthead{--- Part III Chapter I}

16.—(1) Subject to paragraph (2), the Secretary of State 
or the Board  % Words inserted (5.10.99) by SI 1999/2570 reg 12(2)
may suspend payment of a relevant benefit, in whole or in part, in the circumstances prescribed in paragraph (3).

(2) The Secretary of State shall suspend payment of a jobseeker’s allowance in the circumstances prescribed in paragraph (3)($a$)(i) or (ii) where the issue or one of the issues is whether a person, who has claimed a jobseeker’s allowance, is or was available for employment or whether he is or was actively seeking employment.

(3) The prescribed circumstances are that—
\begin{enumerate}\item[]
($a$) it appears to the Secretary of State 
or the Board  % Words inserted (5.10.99) by SI 1999/2570 reg 12(3)(a)(i)
that—
\begin{enumerate}\item[]
(i) an issue arises whether the conditions for entitlement to a relevant benefit are or were fulfilled;

(ii) an issue arises whether a decision as to an award of a relevant benefit should be revised under section 9 or superseded under section 10;

(iii) an issue arises whether any amount paid or payable to a person by way of, or in connection with a claim for, a relevant benefit is recoverable under section 71 (overpayments), 71A (recovery of jobseeker’s allowance: severe hardship cases\footnote{\frenchspacing Section 71A was inserted by section 18 of the Jobseekers Act 1995 (c. 18).}) or 74 (income support and other payments) of the Administration Act or regulations made under any of those sections; or

(iv) the last address notified to him 
or them % Words inserted (5.10.99) by SI 1999/2570 reg 12(3)(a)(ii)
of a person who is in receipt of a relevant benefit is not the address at which that person is residing; or
\end{enumerate}

($b$) an appeal is pending against—
\begin{enumerate}\item[]
(i) a decision of 
%an appeal tribunal, a Commissioner 
the First-tier Tribunal, the Upper Tribunal  % Words substituted (3.11.08) by SI 2008/2683 Sch 1 para 109(2)(a)
or a court;

(ii) a decision given in a different case by 
%a Commissioner 
the Upper Tribunal  % Words substituted (3.11.08) by SI 2008/2683 Sch 1 para 109(2)(b)
or a court, and it appears to the Secretary of State 
or the Board  % Words inserted (5.10.99) by SI 1999/2570 reg 12(3)(b)
that, if the appeal were to be determined in a particular way, an issue would arise as to whether the award of a relevant benefit (whether the same benefit or not) in the case itself ought to be revised or superseded.
\end{enumerate}
\end{enumerate}

%(4) For the purposes of section 21(3)($c$) an appeal is pending where the Secretary of State certifies 
%or the Board certify  % Words inserted (5.10.99) by SI 1999/2570 reg 12(4)(a)
%in writing that he proposes
%or they propose—  % Words inserted (5.10.99) by SI 1999/2570 reg 12(4)(b)
%\begin{enumerate}\item[]
%($a$) to make a request under regulation 53(4) for a statement of reasons for a decision of an appeal tribunal;
%
%($b$) to bring an appeal against the decision; or
%
%($c$) to bring an appeal against a decision in a different case and, if that appeal were to be allowed, an issue would arise as to whether the award of a relevant benefit (whether the same benefit or not) in the case itself ought to be revised or superseded.
%\end{enumerate}

% Reg 16(4) substituted (19.6.00) by SI 2000/1596 reg 20
(4) For the purposes of section 21(3)($c$)  an appeal is pending where a decision of 
%an appeal tribunal, a Commissioner 
the First-tier Tribunal, the Upper Tribunal  % Words substituted (3.11.08) by SI 2008/2683 Sch 1 para 109(3)(a)
or a court has been made and the Secretary of State—
\begin{enumerate}\item[]
($a$) is awaiting receipt of that decision or (in the case of 
%an appeal tribunal decision
a decision of the First-tier Tribunal%  % Words substituted (3.11.08) by SI 2008/2683 Sch 1 para 109(3)(b)
) is considering whether to apply for a statement of the reasons for it, or has applied for such a statement and is awaiting receipt thereof; or

($b$) has received that decision or (in the case of 
%an appeal tribunal decision
a decision of the First-tier Tribunal%  % Words substituted (3.11.08) by SI 2008/2683 Sch 1 para 109(3)(c)(i)
) the statement of the reasons for it, and is considering whether to apply for 
%leave 
permission  % Word substituted (3.11.08) by SI 2008/2683 Sch 1 para 109(3)(c)(ii)
to appeal, or, where 
%leave 
permission  % Word substituted (3.11.08) by SI 2008/2683 Sch 1 para 109(3)(c)(ii)
to appeal has been granted, is considering whether to appeal;
\end{enumerate}
and the Secretary of State shall give written notice of his proposal to make a request for a statement of the reasons for a tribunal decision, to apply for leave to appeal, or to appeal, as soon as reasonably practicable.

\amendment{
Words inserted in reg. 16(1), (3)
%, (4) 
(5.10.99) by the Tax Credits (Decisions and Appeals) (Amendment) Regulations 1999 reg. 12.

Reg. 16(4) substituted (19.6.00) by the Social Security and Child Support (Miscellaneous Amendments) Regulations 2000 reg. 20.

Words substituted in reg. 16(3)(b)(i), (ii), (4) (3.11.08) by the Tribunals, Courts and Enforcement Act 2007 (Transitional and Consequential Provisions) Order 2008 Sch. 1 para. 109.
}

\subsubsection[17. Provision of information or evidence]{Provision of information or evidence}

17.—(1) This regulation applies where the Secretary of State requires information or evidence for a determination whether a decision awarding a relevant benefit should be—
\begin{enumerate}\item[]
($a$) revised under section 9; or

($b$) superseded under section 10.
\end{enumerate}

(2) For the purposes of paragraph (1), the following persons must satisfy the requirements of paragraph (4)—
\begin{enumerate}\item[]
($a$) a person in respect of whom payment of a benefit has been suspended in the circumstances prescribed in regulation 16(3)($a$);

($b$) a person who has made an application for a decision of the Secretary of State to be revised or superseded;

%($c$) a person who fails to comply with the provisions of regulation 32(1) of the Claims and Payments Regulations in so far as they relate to documents, information or facts required by the Secretary of State;

% Reg 17(2)(c) substituted by SI 2012/824 reg 4(3)
($c$) a person from whom the Secretary of State requires information or evidence under regulation 32(1) of the Claims and Payments Regulations;

($ca$) a person from whom the Secretary of State requires documents, certificates or other evidence under regulation 24(5) or (5A) of the Jobseeker’s Allowance Regulations;

($d$) a person who qualifies for income support by virtue of paragraph 7 of Schedule 1B to the Income Support Regulations\footnote{\frenchspacing Schedule 1B was inserted by S.I. 1996/206.};

($e$) a person whose entitlement to benefit is conditional upon his being, or being treated as, incapable of work;

% Reg 17(2)(f) added (27.7.08) by SI 2008/1554 reg 40
($f$) a person whose entitlement to an employment and support allowance is conditional on his having, or being treated as having, limited capability for work.
\end{enumerate}

(3) The Secretary of State shall notify any person to whom paragraph (2) refers of the requirements of this regulation.

(4) A person to whom paragraph (2) refers must either—
\begin{enumerate}\item[]
($a$) supply the information or evidence within—
\begin{enumerate}\item[]
%(i) a period of one month beginning with the date on which the notification under paragraph (3) was sent to him; or

% Reg 17(4)(a)(i) substituted by SI 2012/824 reg 4(3)(b)
(i) a period of 14 days beginning with the date on which the notification under paragraph (3) was sent to him or such longer period as the Secretary of State allows in that notification; or

(ii) such longer period as he satisfies the Secretary of State is necessary in order to enable him to comply with the requirement; or
\end{enumerate}

($b$) satisfy the Secretary of State within the 
%period of time specified in 
period applicable under  % Words substituted by SI 2014/824 reg 4(3)(c)
sub-paragraph ($a$)(i) that either—
\begin{enumerate}\item[]
(i) the information or evidence required of him does not exist; or

(ii) that it is not possible for him to obtain it.
\end{enumerate}
\end{enumerate}

% Reg 17(4A) inserted by SI 2014/824 reg 4(3)(d)
(4A) In relation to a person to whom paragraph (2)($ca$) refers, paragraph~(4)($a$)(i) has effect as if for “14 days” there were substituted “7 days”.

(5) The Secretary of State may suspend the payment of a relevant benefit, in whole or in part, to any person to whom paragraph (2)($b$) to 
%($e$) 
($f$)  % Words substituted by SI 2010/840 reg 7(6)
applies who fails to satisfy the requirements of paragraph (4).

(6) In this regulation, “evidence” includes evidence which a person is required to provide in accordance with regulation 2 of the Social Security (Medical Evidence) Regulations 1976\footnote{\frenchspacing S.I. 1976/615; relevant amending instruments are S.I. 1982/699, 1992/247 and 1994/2975.}.

\amendment{
Reg. 17(2)(f) added (27.7.08) by the Employment and Support Allowance (Consequential Provisions) (No. 2) Regulations 2008 reg. 40.

Words substituted in reg. 17(5) (28.6.10) by the Social Security (Miscellaneous Amendments) (No. 3) Regulations 2010 reg. 7(6).

Reg. 17(2)(c), (ca) substituted for reg. 17(2)(c), reg. 17(4)(a)(i) substituted, words  substituted in reg. 17(4)(b) and reg. 17(4A) inserted (17.4.12) by the Social Security (Suspension of Payment of Benefits and Miscellaneous Amendments) Regulations 2012 reg. 4(3).
}

\subsubsection[18. Termination in cases of failure to furnish information or evidence]{Termination in cases of failure to furnish information or evidence}

18.—(1) Subject to paragraphs (2), (3) and (4), the Secretary of State shall decide that where a person—
\begin{enumerate}\item[]
($a$) whose benefit has been suspended in accordance with regulation 16 and who subsequently fails to comply with an information requirement made in pursuance of regulation 17; or

($b$) whose benefit has been suspended in accordance with regulation 17(5),
\end{enumerate}
that person shall cease to be entitled to that benefit from the date on which payment was suspended except where entitlement to benefit ceases on an earlier date other than under this regulation.

(2) Paragraph (1)($a$) shall not apply where not more than one month has elapsed since the information requirement was made in pursuance of regulation 17.

(3) Paragraph (1)($b$) shall not apply where not more than one month has elapsed since the first payment was suspended in accordance with regulation 17.

(4) Paragraph (1) shall not apply where benefit has been suspended in part under regulation 16 or, as the case may be, regulation 17.

% Regs 17, 18 substituted (5.10.99 for tax credit purposes only) by SI 1999/2570 reg 13
%\subsubsection[17. Provision of information or evidence]{Provision of information or evidence}
%
%17.---(1)  This regulation applies where the Board require information or evidence for a determination whether a decision awarding tax credit should be—
%\begin{enumerate}\item[]
%($a$) revised under section 9; or
%
%($b$) superseded under section 10.
%\end{enumerate}
%
%(2) The relevant person shall furnish such certificates, documents, information and evidence as may be required by the Board for the purposes of paragraph (1), and shall do so within one month of being required to do so or such longer period as the Board may consider reasonable.
%
%(3) In paragraph (2) “the relevant person” means any of the following—
%\begin{enumerate}\item[]
%($a$) the claimant concerned;
%
%($b$) where the tax credit could have been claimed by either of two partners or where entitlement to or the amount of the tax credit was affected or liable to be affected by the circumstances of either partner, the partner other than the claimant;
%
%($c$) the employer of the claimant or, where sub-paragraph ($b$)  applies, the employer of the partner other than the claimant.
%\end{enumerate}
%
%(4) Where the claimant or any partner of the claimant is aged not less than 60 and is a member of, or a person deriving entitlement to a pension under, a personal pension scheme, or is a party to, or a person deriving entitlement to a pension under, a retirement annuity contract, the claimant shall, where the Board so require and within one month of being required to do so or such longer period as the Board may consider reasonable, furnish the following information—
%\begin{enumerate}\item[]
%($a$) the name and address of the pension fund holder;
%
%($b$) such other information, including any reference number or policy number, as is needed to enable the personal pension scheme or retirement annuity contract to be identified.
%\end{enumerate}
%
%\pagebreak[3]
%
%(5) A pension fund holder to whom paragraph (4) applies shall, where the Board so require and within one month of being required to do so or such longer period as the Board may consider reasonable, provide the Board with the information specified in paragraph (6).
%
%(6) The information referred to in this paragraph is—
%\begin{enumerate}\item[]
%($a$) where the purchase of an annuity under a personal pension scheme has been deferred, the amount of any income which is being withdrawn from the personal pension scheme;
%
%($b$) in the case of—
%\begin{enumerate}\item[]
%(i) a personal pension scheme where income withdrawal is available, the maximum amount of income which may be withdrawn from the scheme; or
%
%(ii) a personal pension scheme where income withdrawal is not available, or a retirement annuity contract, the maximum amount of income which might be withdrawn from the fund if the fund were held under a personal pension scheme where income withdrawal was available,
%\end{enumerate}
%calculated by or on behalf of the pension fund holder by means of tables prepared from time to time by the Government Actuary which are appropriate for this purpose.
%\end{enumerate}
%
%(7) Every person providing childcare in respect of which a claimant to whom regulation 46A of the Family Credit (General) Regulations 1987\footnote{\frenchspacing S.I. 1987/1973; the regulation 13A inserted in S.I. 1987/1973 by S.I. 1994/1924 and amended by S.I. 1995/516, 1996/2545 and 1997/2793 was renumbered 46A and further amended by S.I. 1999/2487.} applies is incurring relevant childcare charges (within the meaning of that regulation), including a person providing childcare on behalf of a school, local authority, childcare scheme or establishment within paragraph (2)($b$), ($c$)  or ($d$)  of that regulation, shall furnish such certificates, documents, information and evidence as may be required by the Board for the purposes of paragraph (1), and shall do so within one month of being required to do so or such longer period as the Board may consider reasonable.
%
%\amendment{
%Reg. 17 substituted (5.10.99) by the Tax Credits (Decisions and Appeals) (Amendment) Regulations 1999 reg. 13.
%}
%
%\subsubsection[18. Suspension and termination in cases of failure to furnish information or evidence]{Suspension and termination in cases of failure to furnish information or evidence}
%
%18.---(1)  Where a claimant—
%\begin{enumerate}\item[]
%($a$) is required by the Board under regulation 17 to furnish information, or evidence, and
%
%($b$) fails to do so within the period specified by the Board in accordance with that regulation (“the suspension period”),
%\end{enumerate}
%the Board may, subject to paragraphs (3) and (4), decide to suspend payment of tax credit to or on behalf of the claimant in whole or in part.
%
%(2) Where either—
%\begin{enumerate}\item[]
%($a$) a claimant whose benefit has been suspended in whole or in part in accordance with regulation 16 subsequently fails to comply with a requirement for information or evidence made under regulation 17, within the suspension period, or within the period of one month immediately following the suspension period; or
%
%\begin{enumerate}\item[($b$)] (i) a claimant has been required by the Board under regulation 17 to furnish information or evidence,
%
%(ii) the claimant has failed to do so within the suspension period and within the period of one month immediately following the suspension period, and
%
%(iii) the Board have suspended payment of tax credit to or on behalf of the claimant in whole or in part in accordance with paragraph (1) of this regulation,
%\end{enumerate}
%\end{enumerate}
%the Board may, subject to paragraphs (3) to (5), decide that the claimant shall cease to be entitled to payment of tax credit with effect from a date not earlier than the date on which payment of tax credit was suspended.
%
%(3) No decision shall be taken by the Board pursuant to paragraph (1) or (2) where—
%\begin{enumerate}\item[]
%($a$) the failure to furnish information has been remedied; or
%
%($b$) the Board have allowed a further period of time (in addition to the suspension period or the period of one month referred to in paragraph (2)($a$)  or ($b$)(ii)) within which the claimant is required \pagebreak[3] to furnish the information and the claimant has furnished the information within that further period.
%\end{enumerate}
%
%(4) For the purposes of paragraphs (1) and (2), a claimant shall be deemed not to have failed to furnish information within the suspension period or within the period of one month referred to in paragraph (2)($a$)  or ($b$)(ii)  if he had a reasonable excuse and that excuse has not ceased; and, where that excuse has ceased, he shall be deemed not to have failed to furnish information within either of those periods for those purposes if he furnished the information without unreasonable delay after the excuse had ceased.
%
%(5) No decision shall be taken by the Board pursuant to paragraph (2) unless payment of the whole of the relevant tax credit to or on behalf of the claimant has been suspended, under regulation 16 or 17 or both of those regulations.
%
%\amendment{
%Reg. 18 substituted (5.10.99) by the Tax Credits (Decisions and Appeals) (Amendment) Regulations 1999 reg. 13.
%}

\subsubsection[19. Suspension and termination for failure to submit to medical examination]{Suspension and termination for failure to submit to medical examination}

19.—(1) Except where regulation 8 of the Social Security (Incapacity for Work) (General) Regulations 1995\footnote{\frenchspacing S.I. 1995/311.} 
%applies  % Word omitted (27.7.08) by SI 2008/1554 reg 41(a)
(where a question arises as to whether a person is capable of work)
or regulation 23 of the Employment and Support Allowance Regulations (where a question arises whether a person has limited capability for work) applies%  % Words inserted (27.7.08) by SI 2008/1554 reg 41(b)
, the Secretary of State 
or the Board  % Words inserted (5.10.99) by SI 1999/2570 reg 14(2)(a)
may require a person to submit to a medical examination by a 
%medical practitioner 
health care professional approved by the Secretary of State  % Words substituted (3.7.07) by SI 2007/1626 reg 4(3)
where that person is in receipt of a relevant benefit, and either—
\begin{enumerate}\item[]
($a$) the Secretary of State considers 
or the Board consider  % Words inserted (5.10.99) by SI 1999/2570 reg 14(2)(b)(i)
it necessary to satisfy himself 
or themselves  % Words inserted (5.10.99) by SI 1999/2570 reg 14(2)(b)(ii)
as to the correctness of the award of the benefit, or of the rate at which it was awarded; or

\looseness=-1
($b$) that person applies for a revision or supersession of the award and the Secretary of State considers 
or the Board consider  % Words inserted (5.10.99) by SI 1999/2570 reg 14(2)(c)(i)
that the examination is necessary for the purpose of making his 
or their  % Words inserted (5.10.99) by SI 1999/2570 reg 14(2)(c)(ii)
decision.
\end{enumerate}

(2) The Secretary of State 
or the Board  % Words inserted (5.10.99) by SI 1999/2570 reg 12(3)(b)
may suspend payment of a relevant benefit in whole or in part, to a person who fails, without good cause, on two consecutive occasions to submit to a medical examination in accordance with requirements under paragraph (1) except where entitlement to benefit is suspended on an earlier date other than under this regulation.

(3) Subject to paragraph (4), the Secretary of State 
or the Board  % Words inserted (5.10.99) by SI 1999/2570 reg 12(3)(b)
may determine that the entitlement to a relevant benefit of a person, in respect of whom payment of such a benefit has been suspended under paragraph~(2), shall cease from a date not earlier than the date on which payment was suspended except where entitlement to benefit ceases on an earlier date other than under this regulation.

(4) Paragraph (3) shall not apply where not more than one month has elapsed since the first payment was suspended under paragraph (2).

\amendment{
Words inserted in reg. 19(1), (2), (3) (5.10.99) by the Tax Credits (Decisions and Appeals) (Amendment) Regulations 1999 reg. 14.

Words substituted in reg. 19(1) (3.7.07) by the Social Security (Miscellaneous Amendments) (No. 2) Regulations 2007 reg. 4(3).

Words inserted and omitted in reg. 19(1) (27.7.08) by the Employment and Support Allowance (Consequential Provisions) (No. 2) Regulations 2008 reg. 41.
}

\subsubsection[20. Making of payments which have been suspended]{Making of payments which have been suspended}

20.—(1) Subject to paragraphs (2) and (3), payment of a benefit suspended in accordance with regulation 16 
or 17  % Words inserted (5.7.99) by SI 1999/1623 reg 6(a)
shall be made where—
\begin{enumerate}\item[]
($a$) in a case to which regulation 16(2) or (3)($a$)(i) to (iii) applies, the Secretary of State is satisfied 
or the Board are satisfied  % Words inserted (5.10.99) by SI 1999/2570 reg 15(2)
that the benefit suspended is properly payable and no outstanding issues remain to be resolved;

($b$) in a case to which regulation 16(3)($a$)(iv) applies, the Secretary of State is satisfied 
or the Board are satisfied  % Words inserted (5.10.99) by SI 1999/2570 reg 15(2)
that he has 
or they have  % Words inserted (5.10.99) by SI 1999/2570 reg 15(3)
been notified of the address at which the person is residing;

% Reg 20(1)(c) omitted (19.6.00) by SI 2000/1596 reg 21(a)
%($c$) in a case to which regulation 16(3)($b$) applies, an appeal is no longer pending and the benefit suspended remains payable following the determination of the appeal;

% Reg 20(1)(d) inserted (5.7.99) by SI 1999/1623 reg 6(b)
%($d$) in a case to which regulation 17(5) applies, the Secretary of State is satisfied that the benefit suspended is properly payable and the requirements of regulation 17(4) have been satisfied.

% Reg 20(1)(d) substituted (5.10.99) by SI 1999/2570 reg 15(4)
($d$) in a case to which regulation 18(1) applies, the Board are satisfied that the benefit suspended is properly payable and the requirements of regulation 17(2), (4), (5) or (7) have been satisfied.
\end{enumerate}

%(2) Where regulation 16(4)($a$) applies, payment of a benefit suspended shall be made if, within one month of the date on which he 
%or they  % Words inserted (5.10.99) by SI 1999/2570 reg 15(5)(a)
%received a copy of the tribunal’s decision, the Secretary of State has not% 
%, or the Board have not,  % Words inserted (5.10.99) by SI 1999/2570 reg 15(5)(b)
%notified the claimant in writing that he has% 
%, or they have,  % Words inserted (5.10.99) by SI 1999/2570 reg 15(5)(c)
%requested, pursuant to regulation 53(4), a statement of the reasons for the decision.
%
%(3) Where regulation 16(4)($b$) or ($c$) applies, payment of a benefit suspended shall be made if the Secretary of State fails 
%or the Board fail  % Words inserted (5.10.99) by SI 1999/2570 reg 15(6)(a)
%to notify the claimant in writing, within one month of the date on which the Secretary of State receives 
%or the Board receive  % Words inserted (5.10.99) by SI 1999/2570 reg 15(6)(b)
%the reasons in writing for the decision on appeal which was pending for the purposes of regulation 16(3)($b$), that an appeal or, as the case may be, an application for leave to appeal has been made against the decision.

% Reg 20(2), (3) substituted (19.6.00) by SI 2000/1596 reg 21(b)
(2) Where regulation 16(3)($b$)(i)  applies, payment of a benefit suspended shall be made if the Secretary of State—
\begin{enumerate}\item[]
($a$) does not, in the case of a decision of 
%an appeal tribunal
the First-tier Tribunal%  % Words substituted (3.11.08) by SI 2008/2683 Sch 1 para 110(2)(a)(i)
, apply for a statement of the reasons for that decision within the period 
%of one month specified in regulation 53(4)
specified under Tribunal Procedure Rules%  % Words substituted (3.11.08) by SI 2008/2683 Sch 1 para 110(2)(a)(ii)
;

($b$) does not, in the case of a decision of 
%an appeal tribunal, a Commissioner 
the First-tier Tribunal, the Upper Tribunal  % Words substituted (3.11.08) by SI 2008/2683 Sch 1 para 110(2)(b)(i)
or a court, make an application for 
%leave 
permission  % Words substituted (3.11.08) by SI 2008/2683 Sch 1 para 110(2)(b)(ii)
to appeal and (where 
%leave 
permission  % Words substituted (3.11.08) by SI 2008/2683 Sch 1 para 110(2)(b)(ii)
to appeal is granted) make the appeal within the time prescribed for the making of such applications and appeals;

($c$) withdraws an application for 
%leave 
permission  % Words substituted (3.11.08) by SI 2008/2683 Sch 1 para 110(2)(c)
to appeal or the appeal; or

($d$) is refused 
%leave 
permission  % Words substituted (3.11.08) by SI 2008/2683 Sch 1 para 110(2)(c)
to appeal, in circumstances where it is not open to him to renew the application for 
%leave 
permission  % Words substituted (3.11.08) by SI 2008/2683 Sch 1 para 110(2)(c)
or to make a further application for 
%leave 
permission  % Words substituted (3.11.08) by SI 2008/2683 Sch 1 para 110(2)(c)
to appeal.
\end{enumerate}

(3) Where regulation 16(3)($b$)(ii)  applies, payment of a benefit suspended shall be made if the Secretary of State, in relation to the decision of 
%a Commissioner 
the Upper Tribunal  % Words substituted (3.11.08) by SI 2008/2683 Sch 1 para 110(3)(a)
or the court in a different case—
\begin{enumerate}\item[]
($a$) does not make an application for 
%leave 
permission  % Words substituted (3.11.08) by SI 2008/2683 Sch 1 para 110(2)(c)
to appeal and (where 
%leave 
permission  % Words substituted (3.11.08) by SI 2008/2683 Sch 1 para 110(2)(c)
to appeal is granted) make the appeal within the time prescribed for the making of such applications and appeals;

($b$) withdraws an application for 
%leave 
permission  % Words substituted (3.11.08) by SI 2008/2683 Sch 1 para 110(2)(c)
to appeal or the appeal; or

($c$) is refused 
%leave 
permission  % Words substituted (3.11.08) by SI 2008/2683 Sch 1 para 110(2)(c)
to appeal, in circumstances where it is not open to him to renew the application for 
%leave 
permission  % Words substituted (3.11.08) by SI 2008/2683 Sch 1 para 110(2)(c)
or to make a further application for 
%leave 
permission  % Words substituted (3.11.08) by SI 2008/2683 Sch 1 para 110(2)(c)
to appeal.
\end{enumerate}

(4) Payment of benefit which has been suspended in accordance with regulation 19 for failure to submit to a medical examination shall be made where the Secretary of State is satisfied 
or the Board are satisfied  % Words inserted (5.10.99) by SI 1999/2570 reg 15(2)
that it is no longer necessary for the person referred to in that regulation to submit to a medical examination.

\amendment{
Words inserted in reg. 20(1) and reg. 20(1)(d) inserted (5.7.99) by the Social Security and Child Support (Decisions and Appeals) Amendment (No. 2) Regulations 1999 reg. 6.

Words inserted in reg. 20(1)(a), (b), 
%(2), (3), 
(4) and reg. 20(1)(d) substituted (5.10.99) by the Tax Credits (Decisions and Appeals) (Amendment) Regulations 1999 reg. 15.

Reg. 20(2), (3) substituted and reg. 20(1)(c) omitted (19.6.00) by the Social Security and Child Support (Miscellaneous Amendments) Regulations 2000 reg. 21.

Words substituted in reg. 20(2)(a), (b), (c), (d), (3) (3.11.08) by the Tribunals, Courts and Enforcement Act 2007 (Transitional and Consequential Provisions) Order 2008 Sch. 1 para. 110.
}

\subsection[Chapter II --- Other matters]{Chapter II\\*Other matters}

\subsubsection[21. Decisions involving issues that arise on appeal in other cases]{Decisions involving issues that arise on appeal in other cases}

\renewcommand\parthead{--- Part III Chapter II}

21.—(1) For the purposes of section 25(3)($b$) (prescribed cases and circumstances in which a decision may be made on a prescribed basis) a case which satisfies the condition in paragraph (2) is a prescribed case.

(2) The condition is that the claimant would be entitled to the benefit to which the decision which falls to be made relates, even if the appeal in the other case referred to in section 25(1)($b$) were decided in a way which is the most unfavourable to him.

(3) For the purposes of section 25(3)($b$), the prescribed basis on which the Secretary of State 
or the Board  % Words inserted (5.10.99) by SI 1999/2570 reg 16(2)
may make the decision is as if—
\begin{enumerate}\item[]
($a$) the appeal in the other case which is referred to in section 25(1)($b$) had already been determined; and

($b$) that appeal had been decided in a way which is the most unfavourable to the claimant.
\end{enumerate}

(4) The circumstance prescribed under section 25(5)($c$), where an appeal is pending against a decision for the purposes of that section, even though an appeal against the decision has not been brought (or, as the case may be, an application for 
%leave 
permission  % Words substituted (3.11.08) by SI 2008/2683 Sch 1 para 110(2)(c)
to appeal against the decision has not been made) but the time for doing so has not yet expired, is where the Secretary of State
or the Board—  % Words inserted (5.10.99) by SI 1999/2570 reg 16(3)(a)
\begin{enumerate}\item[]
($a$) certifies in writing that he is% 
, or certify in writing that they are,  % Words inserted (5.10.99) by SI 1999/2570 reg 16(3)(b)
considering appealing against that decision; and

($b$) considers%
, or consider,  % Words inserted (5.10.99) by SI 1999/2570 reg 16(3)(c)
that, if such an appeal were to be determined in a particular way—
\begin{enumerate}\item[]
(i) there would be no entitlement to benefit in a case to which section 25(1)($a$) refers; or

(ii) the appeal would affect the decision in that case in some other way.
\end{enumerate}
\end{enumerate}

\amendment{
Words inserted in reg. 21(3), (4) (5.10.99) by the Tax Credits (Decisions and Appeals) (Amendment) Regulations 1999 reg. 16.

Words substituted in reg. 21(4) (3.11.08) by the Tribunals, Courts and Enforcement Act 2007 (Transitional and Consequential Provisions) Order 2008 Sch. 1 para. 111.
}

\subsubsection[22. Appeals involving issues that arise in other cases]{Appeals involving issues that arise in other cases}

22.  The circumstance prescribed under section 26(6)($c$), where an appeal is pending against a decision in the case described in section 26(1)($b$) even though an appeal against the decision has not been brought (or, as the case may be, an application for 
%leave 
permission  % Words substituted (3.11.08) by SI 2008/2683 Sch 1 para 110(2)(c)
to appeal against the decision has not been made) but the time for doing so has not yet expired, is where the Secretary of State
or the Board—  % Words inserted (5.10.99) by SI 1999/2570 reg 17(a)
\begin{enumerate}\item[]
($a$) certifies in writing that he is% 
, or certify in writing that they are,  % Words inserted (5.10.99) by SI 1999/2570 reg 17(b) 
considering appealing against that decision; and

($b$) considers%
, or consider,  % Words inserted (5.10.99) by SI 1999/2570 reg 17(c)
that, if such an appeal were already determined, it would affect the determination of the appeal described in section 26(1)($a$).
\end{enumerate}

\amendment{
Words inserted in reg. 22 (5.10.99) by the Tax Credits (Decisions and Appeals) (Amendment) Regulations 1999 reg. 17.

Words substituted in reg. 22 (3.11.08) by the Tribunals, Courts and Enforcement Act 2007 (Transitional and Consequential Provisions) Order 2008 Sch. 1 para. 112.
}

\subsubsection[23. Child support decisions involving issues that arise on appeal in other cases]{Child support decisions involving issues that arise on appeal in other cases}

23.—(1) For the purposes of section 28ZA(2)($b$) of the Child Support Act\footnote{\frenchspacing Section 28ZA was inserted by section 43 of the Social Security Act 1998.} (prescribed cases and circumstances in which a decision may be made on a prescribed basis), a case which satisfies either of the conditions in paragraph~(2) is a prescribed case.

(2) The conditions referred to in paragraph (1) are that—
\begin{enumerate}\item[]
($a$) if a decision were not made on the basis prescribed in paragraph~(3), the parent with care would become entitled to income support if a claim were made, or to an increased amount of that benefit;

($b$) [\emph{1993 scheme}] the absent parent is an employed earner or a self-employed earner.

($b$) [\emph{2003 scheme}] the 
%absent parent
non-resident parent  % Words substituted for new-rules cases by SI 2001/158 reg 4(2)
is an employed earner or a self-employed earner.
\end{enumerate}

(3) For the purposes of section 28ZA(2)($b$) of the Child Support Act, the prescribed basis on which the 
%\opt{oldrules}{
Secretary of State %}%
%\opt{newrules,2012rules}{Commission }%  % Words substituted (6.4.09 for new-rules cases only) by SI 2009/396 reg 4(12), reversed (1.8.12) by SI 2012/2007 Sch para 113(10)(a)
may make the decision is as if—
\begin{enumerate}\item[]
($a$) [\emph{1993 scheme}] the appeal in relation to the different maintenance assessment, which is referred to in section 28ZA(1)($b$) of that Act had already been determined; and

($a$) [\emph{2003 scheme}] the appeal in relation to the different maintenance 
%\opt{oldrules}{assessment}%
calculation%  % Word substituted for new-rules cases by SI 2001/158 reg 4(3)
, which is referred to in section 28ZA(1)($b$) of that Act had already been determined; and

($b$) that appeal had been decided in a way that was the most unfavourable to the applicant for the decision mentioned in section 28ZA(1)($a$) of that Act.
\end{enumerate}

(4) The circumstances prescribed under section 28ZA(4)($c$) of the Child Support Act (where an appeal is pending against a decision for the purposes of that section, even though an appeal against the decision has not been brought or, as the case may be, an application for 
%leave 
permission  % Words substituted (3.11.08) by SI 2008/2683 Sch 1 para 113
to appeal against the decision has not been made but the time for doing so has not expired), are that the 
%\opt{oldrules}{
Secretary of State%}%
%\opt{newrules,2012rules}{Commission}%  % Words substituted (6.4.09 for new-rules cases only) by SI 2009/396 reg 4(12), reversed (1.8.12) by SI 2012/2007 Sch para 113(10)(b)
—
\begin{enumerate}\item[]
($a$) certifies in writing that 
%he
the Secretary of State  % Words substituted (1.8.12) by SI 2012/2007 Sch para 113(10)(b)
is considering appealing against that decision; and

($b$) 
%\opt{oldrules}{he }%
%\opt{newrules,2012rules}{it }%  % Words substituted (6.4.09 for new-rules cases only) by SI 2009/396 reg 4(12)
the Secretary of State  % Words substituted (1.8.12) by SI 2012/2007 Sch para 113(10)(b)
considers that, if such an appeal were to be determined in a particular way—
\begin{enumerate}\item[]
(i) there would be no liability for child support maintenance, or

(ii) such liability would be less than would be the case were an appeal not made.
\end{enumerate}
\end{enumerate}

(5) In this regulation—
\begin{enumerate}\item[]
[\emph{1993 scheme}] “absent parent” and “parent with care” have the same meaning as in section 54 of the Child Support Act;

[\emph{2003 scheme}] “%
%absent parent
non-resident parent%  % Words substituted for new-rules cases by SI 2001/158 reg 4(2)
” and “parent with care” have the same meaning as in section 54 of the Child Support Act;

“employed earner” and “self-employed earner” have the same meaning as in section 2(1) of the Contributions and Benefits Act.
\end{enumerate}

\amendment{
Words substituted in reg. 23(2)(b), (3)(a), (5) for new-rules cases only by the Child Support (Consequential Amendments and Transitional Provisions) Regulations 2001 reg. 4(2), (3).

Words substituted in reg. 23(4) (3.11.08) by the Tribunals, Courts and Enforcement Act 2007 (Transitional and Consequential Provisions) Order 2008 Sch. 1 para. 113.

Words substituted in reg. 23 (6.4.09 for new-rules cases only) by the Child Support (Miscellaneous Amendments) Regulations 2009 reg. 4(12).

Words substituted in reg. 23(3), (4) (1.8.12) by the Public Bodies (Child Maintenance and Enforcement Commission: Abolition and Transfer of Functions) Order 2012 Sch. para. 113(10).

Reg. 23 omitted (10.12.12 for 2012 scheme cases only) by the Child Support (Meaning of Child and New Calculation Rules) (Consequential and Miscellaneous Amendment) Regulations 2012 reg. 6(3).
}

\subsubsection[24. Child support appeals involving issues that arise in other cases]{Child support appeals involving issues that arise in other cases}

24.  The circumstances prescribed under section 28ZB(6)($c$) of the Child Support Act\footnote{\frenchspacing Section 28ZB was inserted by section 43 of the Social Security Act 1998.}, where an appeal is pending against a decision in the case described in section 28ZB(1)($b$) even though an appeal against the decision has not been brought (or, as the case may be, an application for 
%leave 
permission  % Words substituted (3.11.08) by SI 2008/2683 Sch 1 para 114
to appeal against the decision has not been made), is where the 
%\opt{oldrules}{
Secretary of State%}%
%\opt{newrules,2012rules}{Commission}%  % Words substituted (6.4.09 for new-rules cases only) by SI 2009/396 reg 4(12), reversed (1.8.12) by SI 2012/2007 Sch para 113(11)(a)
—
\begin{enumerate}\item[]
($a$) certifies in writing that 
%\opt{oldrules}{he }%
%\opt{newrules,2012rules}{it }%  % Words substituted (6.4.09 for new-rules cases only) by SI 2009/396 reg 4(12)
the Secretary of State  % Words substituted (1.8.12) by SI 2012/2007 Sch para 113(11)(b)
is considering appealing against that decision, and

\begin{sloppypar}
($b$) considers that, if such an appeal were already determined, it would affect the determination of the appeal described in section~28ZB(1)($a$).
\end{sloppypar}
\end{enumerate}

\amendment{
Words substituted in reg. 24 (3.11.08) by the Tribunals, Courts and Enforcement Act 2007 (Transitional and Consequential Provisions) Order 2008 Sch. 1 para. 114.

Words substituted in reg. 24 (6.4.09 for new-rules cases only) by the Child Support (Miscellaneous Amendments) Regulations 2009 reg. 4(13).

Words substituted in reg. 24 (1.8.12) by the Public Bodies (Child Maintenance and Enforcement Commission: Abolition and Transfer of Functions) Order 2012 Sch. para. 113(11).

Reg. 24 omitted (10.12.12 for 2012 scheme cases only) by the Child Support (Meaning of Child and New Calculation Rules) (Consequential and Miscellaneous Amendment) Regulations 2012 reg. 6(3).
}

\section[Part IV --- Rights of appeal and procedure for bringing appeals]{\sloppy\hbadness=2513 Part IV\\*Rights of appeal and procedure for bringing appeals}

\amendment{
Pt. IV revoked (7.4.03) so far as relating to child benefit or guardian's allowance by the Child Benefit and Guardian’s Allowance (Decisions and Appeals) Regulations 2003 reg. 34(a).
}

\subsection[Chapter I --- General appeals matters not including child support appeals]{Chapter I\\*General appeals matters not including child support appeals}

\renewcommand\parthead{--- Part IV Chapter I}

\subsubsection[25. Other persons with a right of appeal]{Other persons with a right of appeal}

25.  For the purposes of 
%section 12(2)($b$)
section 12(2)%  % Words substituted (21.12.04) by SI 2004/3368 reg 2(3)
, but subject to regulation 3ZA,%  % Words inserted (28.10.13) by SI 2013/2380 reg 4(7)
, the following other persons have a right to appeal to 
%an appeal tribunal
the First-tier Tribunal%  % Words substituted (3.11.08) by SI 2008/2683 Sch 1 para 115
—
\begin{enumerate}\item[]
% Reg 25(ai)--(aiii) inserted (20.5.02) by SI 2002/1379 reg 7
($ai$) any person who has been appointed by the Secretary of State or the Board under regulation 30(1)\footnote{Regulation 30(1) was amended by S.I. 1999/2572.} of the Claims and Payments Regulations (payments on death) to proceed with the claim of a person who has made a claim for benefit and subsequently died;

($aii$) any person who is appointed by the Secretary of State to claim benefit on behalf of a deceased person and who claims the benefit under regulation 30(5) and (6)\footnote{Regulation 30(5) was amended by S.I. 1988/1725, 1990/2208, 1991/2741, 1996/1460 and 1999/2572.} of the Claims and Payments Regulations;

($aiii$) any person who is appointed by the Secretary of State to make a claim for reduced earnings allowance or disablement benefit in the name of a person who has died and who claims under regulation 30(6A) and (6B)\footnote{Paragraphs (6A) and (6B) were inserted by S.I. 1990/2208.} of the Claims and Payments Regulations;

($a$) any person appointed by the Secretary of State 
or the Board  % Words inserted (5.10.99) by SI 1999/2570 reg 18
under regulation 33(1) of the Claims and Payments Regulations (persons unable to act) to act on behalf of another;

($b$) any person claiming attendance allowance or disability living allowance on behalf of another under section 66(2)($b$) of the Contriburions and Benefits Act or, as the case may be, section 76(3) of that Act (claims on behalf of terminally ill persons);

($c$) in relation to a pension scheme, any person who, for the purposes of Part X of the Pension Schemes Act 1993\footnote{\frenchspacing 1993 c. 48.}, is an employer, member, trustee or manager by virtue of section 146(8) of that Act.
\end{enumerate}

\amendment{
Words inserted in reg. 25(a) (5.10.99) by the Tax Credits (Decisions and Appeals) (Amendment) Regulations 1999 reg. 18.

Reg. 25A(ai)--(aiii) inserted (20.5.02) by the Social Security and Child Support (Decisions and Appeals) (Miscellaneous Amendments) Regulations 2002 reg. 7.

Words substituted in reg. 25 (21.12.04) by the Social Security, Child Support and Tax Credits (Decisions and Appeals) Amendment Regulations 2004 reg. 2(3).

Words substituted in reg. 25 (3.11.08) by the Tribunals, Courts and Enforcement Act 2007 (Transitional and Consequential Provisions) Order 2008 Sch. 1 para. 115.

\looseness=-1
Words inserted in reg. 25 (28.10.13) by the Social Security, Child Support, Vaccine Damage and Other Payments (Decisions and Appeals) (Amendment) Regulations 2013 reg.~4(7).
}

\subsubsection[26. Decisions against which an appeal lies]{Decisions against which an appeal lies}

26.  
Subject to regulation 3ZA,  % Words inserted (28.10.13) by SI 2013/2380 reg 4(8)
an appeal shall lie to
%an appeal tribunal
the First-tier Tribunal  % Words substituted (3.11.08) by SI 2008/2683 Sch 1 para 115
against a decision made by the Secretary of State
or an officer of the Board—  % Words inserted (5.10.99) by SI 1999/2570 reg 18
\begin{enumerate}\item[]
($a$) as to whether a person is entitled to a relevant benefit for which no claim is required by virtue of regulation 3 of the Claims and Payments Regulations\footnote{\frenchspacing The relevant amending instruments are S.I. 1989/136, S.I. 1994/2943 and S.I. 1996/1460.}; or

($b$) as to whether a payment be made out of the social fund to a person to meet expenses for heating by virtue of regulations made under section~138(2) of the Contributions and Benefits Act (payments out of the social fund);
% Reg 26(c) inserted (19.6.00) by SI 2000/1596 reg 22
or

($c$) under Schedule 6 to the Contributions and Benefits Act (assessment of extent of disablement) in relation to sections 103 (disablement benefit) and 108 (prescribed diseases) of that Act for the purposes of industrial injuries benefit under Part V of that Act%
% Reg 26(d) inserted (19.3.01) by SI 2001/518 reg 4(b)
; or

    ($d$) 
    under section 59 of, and Schedule 7 to, the Welfare Reform and Pensions Act 1999\footnote{1999 c. 30.} (couples to make joint-claim for jobseeker’s allowance) where one member of the couple is working and the Secretary of State has decided that both members of the couple are not engaged in remunerative work%
% Reg 26(e) inserted (6.4.10) by SI 2009/2715 reg 2
; or

($e$) under, or by virtue of regulations made under, section 23A (contributions credits for relevant parents and carers) of the Contributions and Benefits Act.\footnote{1992 c. 4. Section 23A was inserted into the Social Security Contributions and Benefits Act 1992 (c. 4) by section 3 of the Pensions Act 2007 (c. 22).}.
\end{enumerate}

\amendment{
Words inserted in reg. 26 (5.10.99) by the Tax Credits (Decisions and Appeals) (Amendment) Regulations 1999 reg. 19.

Reg. 26(c) inserted (19.6.00) by the Social Security and Child Support (Miscellaneous Amendments) Regulations 2000 reg. 22.

Reg. 26(d) inserted (19.3.01) by the Social Security Amendment (Joint Claims) Regulations 2001 reg. 4(b).

Words substituted in reg. 26 (3.11.08) by the Tribunals, Courts and Enforcement Act 2007 (Transitional and Consequential Provisions) Order 2008 Sch. 1 para. 116.

Reg. 26(e) inserted (6.4.10) by the Pensions Act 2007 (Supplementary Provision) Order 2009 art. 2.

Words inserted in reg. 26 (28.10.13) by the Social Security, Child Support, Vaccine Damage and Other Payments (Decisions and Appeals) (Amendment) Regulations 2013 reg.~4(8).
}

\subsubsection[27. Decisions against which no appeal lies]{Decisions against which no appeal lies}

27.—(1) No appeal lies to 
%an appeal tribunal 
the First-tier Tribunal  % Words substituted (3.11.08) by SI 2008/2683 Sch 1 para 117(a)
against a decision set out in Schedule 2.

(2) In paragraph (1) and Schedule 2, “decision” includes determinations embodied in or necessary to a decision.

% Reg 27(3) omitted (3.11.08) by SI 2008/2683 Sch 1 para 117(b)
%(3) An appeal made against a decision specified in paragraph (1) may be struck out in accordance with regulation 46.

\amendment{
Words substituted in reg. 27(1) and reg. 27(3) omitted  (3.11.08) by the Tribunals, Courts and Enforcement Act 2007 (Transitional and Consequential Provisions) Order 2008 Sch. 1 para. 117.
}

\subsubsection[28. Notice of decision against which appeal lies]{Notice of decision against which appeal lies}

28.—(1) A person with a right of appeal under the Act or these Regulations against any decision of the Secretary of State 
or the Board or an officer of the Board  % Words inserted (5.10.99) by SI 1999/2570 reg 20
shall—
\begin{enumerate}\item[]
($a$) be given written notice of the decision against which the appeal lies;

($b$) be informed that, in a case where that written notice does not include a statement of the reasons for that decision, he may, within one month of the date of notification of that decision, request that the Secretary of State 
or the Board or an officer of the Board  % Words inserted (5.10.99) by SI 1999/2570 reg 20
provide him with a written statement of the reasons for that decision; and

($c$) be given written notice of his right of appeal against that decision.
\end{enumerate}

(2) Where a written statement of the reasons for the decision is not included in the written notice of the decision and is requested under paragraph (1)($b$), the Secretary of State \looseness=-1
or the Board or an officer of the Board  % Words inserted (5.10.99) by SI 1999/2570 reg 20
shall provide that statement within 14 days of receipt of the request
or as soon as practicable afterwards.  % Words inserted (18.3.05) by SI 2005/337 reg 2(6)

\amendment{
Words inserted in reg. 28 (5.10.99) by the Tax Credits (Decisions and Appeals) (Amendment) Regulations 1999 reg. 20.

Words inserted in reg. 28(2) (18.3.05) by the Social Security, Child Support and Tax Credits (Miscellaneous Amendments) Regulations 2005 reg. 2(6).
}

\subsubsection[29. Further particulars required relating to certificate of recoverable benefits 
or, as the case may be, recoverable lump sum payments  % Words inserted (1.10.08) by SI 2008/1596 Sch 2 para 1(c)(i)
appeals% 
%or applications  % Words omitted (3.11.08) by SI 2008/2683 Sch 1 para 118(a)
]{Further particulars required relating to certificate of recoverable benefits 
or, as the case may be, recoverable lump sum payments  % Words inserted (1.10.08) by SI 2008/1596 Sch 2 para 1(c)(i)
appeals% 
%or applications  % Words omitted (3.11.08) by SI 2008/2683 Sch 1 para 118(a)
}

29.—%
% Reg 29(1), (2) omitted (3.11.08) by SI 2008/2683 Sch 1 para 118(b)
% Reg 29(3)--(5) omitted (28.10.13) by SI 2013/2380 reg 4(10)(b) (subject to transitional provisions in reg 8)
%(1) An appeal or application under the 1997 Act relating to a certificate of recoverable benefits 
%or, as the case may be, recoverable lump sum payments  % Words inserted (1.10.08) by SI 2008/1596 Sch 2 para 1(c)(i)
%shall, in addition to any requirements imposed by regulations, include also the following particulars—
%\begin{enumerate}\item[]
%($a$) in the case of an appeal, the date of the certificate of recoverable benefits 
%or, as the case may be, recoverable lump sum payments  % Words inserted (1.10.08) by SI 2008/1596 Sch 2 para 1(c)(i)
%or the decision by the Secretary of State on review against which the appeal is brought, the question under section 11 of the 1997 Act to which the appeal relates and a summary of the arguments relied upon by the appellant to support his contention that the certificate is wrong;
%
%($b$) in the case of an application for an extension of time under regulation 32, in relation to the appeal which it is proposed to bring, the particulars required under sub-paragraph ($a$) together with particulars of the special circumstances on which the application is based.
%\end{enumerate}
%
%(2) Where the appeal or the application for an extension of time is made by a person to whom a compensation payment has been made, a copy of the statement given to that person under section 9 of the 1997 Act 
%or, in the case of lump sum payments, 
%%regulation 14 
%regulation 13  % Words substituted (1.10.08) by SI 2008/2365 reg 6(5)
%of the Lump Sum Payments Regulations% Words inserted (1.10.08) by SI 2008/1596 Sch 2 para 1(c)(ii)
%or if that statement was not in writing, a written summary of it, shall be sent with that appeal or application.
%
%(3) Where it appears to the Secretary of State that an appeal or application does not contain the further particulars required under paragraph (1) or is not accompanied by a written statement or summary as required under paragraph (2) he may direct the appellant or applicant to provide such particulars or such a statement or summary.
%
% Reg 29(3) substituted (3.11.08) by SI 2008/2683 Sch 1 para 118(c)
%(3) Where it appears to the Secretary of State that a notice of appeal in respect of an appeal under the 1997 Act relating to a certificate of recoverable benefits or, as the case may be, recoverable lump sum payments does not contain the particulars required, the Secretary of State may direct the appellant to provide such particulars.
%
%(4) Where paragraph (3) applies, the time specified for making the appeal 
%%or application  % Words omitted (3.11.08) by SI 2008/2683 Sch 1 para 118(d)
%may be extended by such period, not exceeding 14 days from the date of the Secretary of State’s direction under paragraph (3), as the Secretary of State may determine.
%
%(5) Where further particulars 
%%or a written statement or summary  % Words omitted (3.11.08) by SI 2008/2683 Sch 1 para 118(e)
%are required under paragraph (3) they shall be sent to or delivered to the Compensation Recovery Unit of the 
%%Department of Social Security 
%Department for Work and Pensions  % Words substituted (20.5.02) by SI 2002/1379 reg 8
%at 
%%Reyrolle Building, Hebburn, Tyne and Wear, \textsc{\lowercase{NE31 1XB}} 
%Durham House, Washington, Tyne and Wear, \textsc{\lowercase{NE38~7SF}}  % Words substituted (4.12.00) by SI 2000/3030 reg 3
%within such period as the Secretary of State may direct.
%
%(6) The Secretary of State may treat any appeal relating to the certificate of recoverable benefits 
%or, as the case may be, recoverable lump sum payments  % Words inserted (1.10.08) by SI 2008/1596 Sch 2 para 1(c)(i)
%as an application for review under section 10 of the 1997 Act.
%
% Reg 29(6) substituted (28.10.13) by SI 2013/2380 reg 4(9)
(6) The Secretary of State may treat any—
\begin{enumerate}\item[]
\begin{sloppypar}
($a$) purported appeal (where, as the result of regulation~9ZB(2) (consideration of review before appeal), there is no right of appeal);
\end{sloppypar}

($b$) appeal relating to the certificate of recoverable benefits; or

($c$) appeal relating to the certificate of recoverable lump sum payments,
\end{enumerate}
as an application for review under section 10 of the 1997 Act.

\amendment{
Words substituted in reg. 29(5) (4.12.00) by the Social Security (Recovery of Benefits) (Miscellaneous Amendments) Regulations 2000 reg. 3.

Words substituted in reg. 29(5) (20.5.02) by the Social Security and Child Support (Decisions and Appeals) (Miscellaneous Amendments) Regulations 2002 reg. 8.

Words inserted in reg. 29(1), (1)(a), (2), (6) and heading (1.10.08) by the Social Security (Recovery of Benefits) (Lump Sum Payments) Regulations 2008 Sch. 2 para. 1(c) as amended by the Social Security (Miscellaneous Amendments) (No. 3) Regulations 2008 reg. 6(5).

Words omitted in reg. 29(4), (5) and heading, reg. 29(3) substituted and reg. 29(1), (2) omitted (3.11.08) by the Tribunals, Courts and Enforcement Act 2007 (Transitional and Consequential Provisions) Order 2008 Sch. 1 para. 118.

Reg. 29(6) substituted and reg. 29(3)--(5) omitted (28.10.13) by the Social Security, Child Support, Vaccine Damage and Other Payments (Decisions and Appeals) (Amendment) Regulations 2013 reg.~4(9), (10)(b) (subject to transitional provisions in reg. 8).
}

\addtocontents{toc}{\protect\pagebreak[3]}

\subsection[Chapter II --- General appeals matters including child support appeals]{Chapter II\\*General appeals matters including child support appeals}

\renewcommand\parthead{--- Part IV Chapter II}

\subsubsection[30. Appeal against a decision which has been revised --- \emph{1993 scheme version}]{Appeal against a decision which has been revised\\*\emph{1993 scheme version}}

30.—(1) An appeal against a decision of the Secretary of State
%\opt{newrules,2012rules}{, the Commission}  % Words inserted (6.4.09) by SI 2009/396 reg 4(14) , omitted (1.8.12) by SI 2012/2007 Sch para 113(12)
or the Board or an officer of the Board  % Words inserted (5.10.99) by SI 1999/2570 reg 21(1)(a)
shall not lapse where the decision is revised under section 16 of the Child Support Act or section 9 before the appeal is determined and the decision as revised is not more advantageous to the appellant than the decision before it was revised.

(2) Decisions which are more advantageous for the purposes of this regulation include decisions where—
\begin{enumerate}\item[]
($a$) any relevant benefit paid to the appellant is greater or is awarded for a longer period in consequence of the decision made under section 9;

($b$) it would have resulted in the amount of relevant benefit in payment being greater but for the operation of any provision of the Administration Act or the Contributions and Benefits Act restricting or suspending the payment of, or disqualifying a claimant from receiving, some or all of the benefit;

($c$) as a result of the decision, a denial or disqualification for the receiving of any relevant benefit, is lifted, wholly or in part;

($d$) it reverses a decision to pay benefit to a third party;

% Reg 30(2)(dd) added (18.3.05) by SI 2005/337 reg 2(7)
($dd$) it reverses a decision under section 29(2) that an accident is not an industrial accident;

($e$) in consequence of the revised decision, benefit paid is not recoverable under section 71, 71A or 74 of the Administration Act\footnote{\frenchspacing Section 71A was inserted by section 18 of the Jobseekers Act 1995 (c. 18).} or regulations made under any of those sections, or the amount so recoverable is reduced; or

($f$) a financial gain accrued or will accrue to the appellant in consequence of the decision.
\end{enumerate}

(3) Where a decision as revised under section 16 of the Child Support Act or under section 9 is not more advantageous to the appellant than the decision before it was revised, the appeal shall be treated as though it had been brought against the decision as revised.

(4) The appellant shall have a period of one month from the date of notification of the decision as revised to make further representations as to the appeal.

(5) After the expiration of the period specified in paragraph (4), or within that period if the appellant consents in writing, the appeal to the 
%appeal tribunal 
First-tier Tribunal  % Words substituted (3.11.08) by SI 2008/2683 Sch 1 para 119
shall proceed except where, in the light of the further representations from the appellant, the Secretary of State
%\opt{newrules,2012rules}{, the Commission}  % Words inserted (6.4.09) by SI 2009/396 reg 4(14), omitted (1.8.12) by SI 2012/2007 Sch para 113(12)
or the Board or an officer of the Board  % Words inserted (5.10.99) by SI 1999/2570 reg 21(1)(a)
further revises his%
, or revise their,  % Words inserted (5.10.99) by SI 1999/2570 reg 21(1)(b)
decision and that decision is more advantageous to the appellant than the decision before it was revised.

\amendment{
Words inserted in reg. 30(1), (5) (5.10.99) by the Tax Credits (Decisions and Appeals) (Amendment) Regulations 1999 reg. 21.

Reg. 30(2)(dd) added (18.3.05) by the Social Security, Child Support and Tax Credits (Miscellaneous Amendments) Regulations 2005 reg. 2(7).

Words substituted in reg. 30(5) (3.11.08) by the Tribunals, Courts and Enforcement Act 2007 (Transitional and Consequential Provisions) Order 2008 Sch. 1 para. 119.

Words omitted in reg. 30(1), (5) (1.8.12) by the Public Bodies (Child Maintenance and Enforcement Commission: Abolition and Transfer of Functions) Order 2012 Sch. para. 113(12).
}

%Words inserted in heading to reg 30 (3.3.03 for new-rules cases only) by SI 2000/3185 reg 11(a)
\subsubsection[30. Appeal against a decision which has been replaced or revised --- \emph{2003 scheme version}]{Appeal against a decision which has been replaced or revised\\*\emph{2003 scheme version}}

30.—(1) An appeal against a decision of the Secretary of State
%\opt{newrules,2012rules}{, the Commission}  % Words inserted (6.4.09) by SI 2009/396 reg 4(14) , omitted (1.8.12) by SI 2012/2007 Sch para 113(12)
or the Board or an officer of the Board  % Words inserted (5.10.99) by SI 1999/2570 reg 21(1)(a)
shall not lapse where the decision 
%is revised under section 16 of the Child Support Act 
is treated as replaced by a decision under section 11 of the Child Support Act by section 28F(5) of that Act, or is revised under section 16 of that Act  % Words substituted (3.3.03 for new-rules cases only) by SI 2000/3185 reg 11(b)(i)
or section 9 before the appeal is determined and the decision as 
replaced or  % Words inserted (3.3.03 for new-rules cases only) by SI 2000/3185 reg 11(b)(ii)
revised is not more advantageous to the appellant than the decision before it was 
replaced or  % Words inserted (3.3.03 for new-rules cases only) by SI 2000/3185 reg 11(b)(iii)
revised.

(2) Decisions which are more advantageous for the purposes of this regulation include decisions where—
\begin{enumerate}\item[]
($a$) any relevant benefit paid to the appellant is greater or is awarded for a longer period in consequence of the decision made under section 9;

($b$) it would have resulted in the amount of relevant benefit in payment being greater but for the operation of any provision of the Administration Act or the Contributions and Benefits Act restricting or suspending the payment of, or disqualifying a claimant from receiving, some or all of the benefit;

($c$) as a result of the decision, a denial or disqualification for the receiving of any relevant benefit, is lifted, wholly or in part;

($d$) it reverses a decision to pay benefit to a third party;

% Reg 30(2)(dd) added (18.3.05) by SI 2005/337 reg 2(7)
($dd$) it reverses a decision under section 29(2) that an accident is not an industrial accident;

($e$) in consequence of the revised decision, benefit paid is not recoverable under section 71, 71A or 74 of the Administration Act\footnote{\frenchspacing Section 71A was inserted by section 18 of the Jobseekers Act 1995 (c. 18).} or regulations made under any of those sections, or the amount so recoverable is reduced; or

($f$) a financial gain accrued or will accrue to the appellant in consequence of the decision.
\end{enumerate}

(3) Where a decision as 
%revised under section 16 of the Child Support Act 
replaced under section 28F(5) of the Child Support Act or revised under section 16 of that Act  % Words substituted (3.3.03 for new-rules cases only) by SI 2000/3185 reg 11(c)(i)
or under section 9 is not more advantageous to the appellant than the decision before it was 
replaced or  % Words inserted (3.3.03 for new-rules cases only) by SI 2000/3185 reg 11(c)(ii)
revised, the appeal shall be treated as though it had been brought against the decision as 
replaced or  % Words inserted (3.3.03 for new-rules cases only) by SI 2000/3185 reg 11(c)(iii)
revised.

(4) The appellant shall have a period of one month from the date of notification of the decision as 
replaced or  % Words inserted (3.3.03 for new-rules cases only) by SI 2000/3185 reg 11(d)
revised to make further representations as to the appeal.

(5) After the expiration of the period specified in paragraph (4), or within that period if the appellant consents in writing, the appeal to the 
%appeal tribunal 
First-tier Tribunal  % Words substituted (3.11.08) by SI 2008/2683 Sch 1 para 119
shall proceed except where, in the light of the further representations from the appellant, the Secretary of State
%\opt{newrules,2012rules}{, the Commission}  % Words inserted (6.4.09) by SI 2009/396 reg 4(14), omitted (1.8.12) by SI 2012/2007 Sch para 113(12)
or the Board or an officer of the Board  % Words inserted (5.10.99) by SI 1999/2570 reg 21(1)(a)
further revises his%
, or revise their,  % Words inserted (5.10.99) by SI 1999/2570 reg 21(1)(b)
decision and that decision is more advantageous to the appellant than the decision before it was 
replaced or  % Words inserted (3.3.03 for new-rules cases only) by SI 2000/3185 reg 11(e)
revised.

\amendment{
Words inserted in reg. 30(1), (5) (5.10.99) by the Tax Credits (Decisions and Appeals) (Amendment) Regulations 1999 reg. 21.

Words inserted in reg. 30(1), (3), (4), (5) and heading and words substituted in reg. 30(1), (3) (3.3.03 for new-rules cases only) by the Child Support (Decisions and Appeals) (Amendment) Regulations 2000 reg. 11 (subject to reg. 1(2)).

Reg. 30(2)(dd) added (18.3.05) by the Social Security, Child Support and Tax Credits (Miscellaneous Amendments) Regulations 2005 reg. 2(7).

Words substituted in reg. 30(5) (3.11.08) by the Tribunals, Courts and Enforcement Act 2007 (Transitional and Consequential Provisions) Order 2008 Sch. 1 para. 119.

Words inserted in reg. 30 (6.4.09 for new-rules cases only) by the Child Support (Miscellaneous Amendments) Regulations 2009 reg. 4(14).

Words omitted in reg. 30(1), (5) (1.8.12) by the Public Bodies (Child Maintenance and Enforcement Commission: Abolition and Transfer of Functions) Order 2012 Sch. para. 113(12).
}

%Words inserted in heading to reg 30 (3.3.03 for new-rules cases only) by SI 2000/3185 reg 11(a)
\subsubsection[30. Appeal against a decision which has been 
%replaced or  % Words omitted (10.12.12 for 2012 scheme cases only) by SI 2012/2785 reg 6(5)
revised --- \emph{2012 scheme version}]{Appeal against a decision which has been 
%replaced or  % Words omitted (10.12.12 for 2012 scheme cases only) by SI 2012/2785 reg 6(5)
revised\\*\emph{2012 scheme version}}

30.—%
%(1) An appeal against a decision of the Secretary of State
%%\opt{newrules,2012rules}{, the Commission}  % Words inserted (6.4.09) by SI 2009/396 reg 4(14) , omitted (1.8.12) by SI 2012/2007 Sch para 113(12)
%or the Board or an officer of the Board  % Words inserted (5.10.99) by SI 1999/2570 reg 21(1)(a)
%shall not lapse where the decision 
%%is revised under section 16 of the Child Support Act 
%is treated as replaced by a decision under section 11 of the Child Support Act by section 28F(5) of that Act, or is revised under section 16 of that Act  % Words substituted (3.3.03 for new-rules cases only) by SI 2000/3185 reg 11(b)(i)
%or section 9 before the appeal is determined and the decision as 
%replaced or  % Words inserted (3.3.03 for new-rules cases only) by SI 2000/3185 reg 11(b)(ii)
%revised is not more advantageous to the appellant than the decision before it was 
%replaced or  % Words inserted (3.3.03 for new-rules cases only) by SI 2000/3185 reg 11(b)(iii)
%revised.
%
% Reg 30(1) substituted (10.12.12 for 2012 scheme cases only) by SI 2012/2785 reg 6(6)(a)
(1) An appeal against a decision of the Secretary of State or the Board or an officer of the Board shall not lapse where—
\begin{enumerate}\item[]
($a$) the decision is revised under section 9 before the appeal is determined; and

($b$) the decision as revised is not more advantageous to the appellant than the decision before it was revised.
\end{enumerate}

(2) Decisions which are more advantageous for the purposes of this regulation include decisions where—
\begin{enumerate}\item[]
($a$) any relevant benefit paid to the appellant is greater or is awarded for a longer period in consequence of the decision made under section 9;

($b$) it would have resulted in the amount of relevant benefit in payment being greater but for the operation of any provision of the Administration Act or the Contributions and Benefits Act restricting or suspending the payment of, or disqualifying a claimant from receiving, some or all of the benefit;

($c$) as a result of the decision, a denial or disqualification for the receiving of any relevant benefit, is lifted, wholly or in part;

($d$) it reverses a decision to pay benefit to a third party;

% Reg 30(2)(dd) added (18.3.05) by SI 2005/337 reg 2(7)
($dd$) it reverses a decision under section 29(2) that an accident is not an industrial accident;

($e$) in consequence of the revised decision, benefit paid is not recoverable under section 71, 71A or 74 of the Administration Act\footnote{\frenchspacing Section 71A was inserted by section 18 of the Jobseekers Act 1995 (c. 18).} or regulations made under any of those sections, or the amount so recoverable is reduced; or

($f$) a financial gain accrued or will accrue to the appellant in consequence of the decision.
\end{enumerate}

%(3) Where a decision as 
%%revised under section 16 of the Child Support Act 
%replaced under section 28F(5) of the Child Support Act or revised under section 16 of that Act  % Words substituted (3.3.03 for new-rules cases only) by SI 2000/3185 reg 11(c)(i)
%or under section 9 is not more advantageous to the appellant than the decision before it was 
%replaced or  % Words inserted (3.3.03 for new-rules cases only) by SI 2000/3185 reg 11(c)(ii)
%revised, the appeal shall be treated as though it had been brought against the decision as 
%replaced or  % Words inserted (3.3.03 for new-rules cases only) by SI 2000/3185 reg 11(c)(iii)
%revised.

% Reg 30(3) substituted (10.12.12 for 2012 scheme cases only) by SI 2012/2785 reg 6(6)(b)
(3) Where a decision as revised under section 9 is not more advantageous to the appellant than the decision before it was revised, the appeal shall be treated as though it had been brought against the decision as revised.

(4) The appellant shall have a period of one month from the date of notification of the decision as 
%replaced or  % Words inserted (3.3.03 for new-rules cases only) by SI 2000/3185 reg 11(d), omitted (10.12.12 for 2012 scheme cases only) by SI 2012/2785 reg 6(6)(c)
revised to make further representations as to the appeal.

(5) After the expiration of the period specified in paragraph (4), or within that period if the appellant consents in writing, the appeal to the 
%appeal tribunal 
First-tier Tribunal  % Words substituted (3.11.08) by SI 2008/2683 Sch 1 para 119
shall proceed except where, in the light of the further representations from the appellant, the Secretary of State
%\opt{newrules,2012rules}{, the Commission}  % Words inserted (6.4.09) by SI 2009/396 reg 4(14), omitted (1.8.12) by SI 2012/2007 Sch para 113(12)
or the Board or an officer of the Board  % Words inserted (5.10.99) by SI 1999/2570 reg 21(1)(a)
further revises his%
, or revise their,  % Words inserted (5.10.99) by SI 1999/2570 reg 21(1)(b)
decision and that decision is more advantageous to the appellant than the decision before it was 
%replaced or  % Words inserted (3.3.03 for new-rules cases only) by SI 2000/3185 reg 11(e), omitted (10.12.12 for 2012 scheme cases only) by SI 2012/2785 reg 6(6)(c)
revised.\looseness=1

\amendment{
Words inserted in reg. 30(1), (5) (5.10.99) by the Tax Credits (Decisions and Appeals) (Amendment) Regulations 1999 reg. 21.

Words inserted in reg. 30(1), (3), (4), (5) and heading and words substituted in reg. 30(1), (3) (3.3.03 for new-rules cases only) by the Child Support (Decisions and Appeals) (Amendment) Regulations 2000 reg. 11 (subject to reg. 1(2)).

Reg. 30(2)(dd) added (18.3.05) by the Social Security, Child Support and Tax Credits (Miscellaneous Amendments) Regulations 2005 reg. 2(7).

Words substituted in reg. 30(5) (3.11.08) by the Tribunals, Courts and Enforcement Act 2007 (Transitional and Consequential Provisions) Order 2008 Sch. 1 para. 119.

Words inserted in reg. 30 (6.4.09 for new-rules cases only) by the Child Support (Miscellaneous Amendments) Regulations 2009 reg. 4(14).

Words omitted in reg. 30(1), (5) (1.8.12) by the Public Bodies (Child Maintenance and Enforcement Commission: Abolition and Transfer of Functions) Order 2012 Sch. para. 113(12).\looseness=-1

Words omitted in reg. 30(4), (5) and heading and reg. 30(1), (3) substituted (10.12.12 for 2012 scheme cases only) by the Child Support (Meaning of Child and New Calculation Rules) (Consequential and Miscellaneous Amendment) Regulations 2012 reg. 6(5), (6).

\medskip

Reg. 30A revoked (25.1.10) by the Child Support (Management of Payments and Arrears) Regulations 2009 Sch.


\medskip

Reg. 31 omitted (3.11.08) by the Tribunals, Courts and Enforcement Act 2007 (Transitional and Consequential Provisions) Order 2008 Sch. 1 para. 121.

\medskip

Regs. 32--34 omitted (28.10.13) by the Social Security, Child Support, Vaccine Damage and Other Payments (Decisions and Appeals) (Amendment) Regulations 2013 reg.~4(10)(c)--(e) (subject to transitional provisions in reg. 8).
}

% Reg 30A inserted (3.3.03 for new-rules cases only) by SI 2000/3185 reg 12, omitted (25.1.10) by SI 2009/3151 Sch
%\opt{newrules,2012rules}{
%\subsubsection[30A. Appeals to 
%%appeal tribunals 
%the First-tier Tribunal  % Words substituted (3.11.08) by SI 2008/2683 Sch 1 para 120
%in child support cases]{Appeals to 
%%appeal tribunals 
%the First-tier Tribunal  % Words substituted (3.11.08) by SI 2008/2683 Sch 1 para 120
%in child support cases}
%
%30A.  Section 20 of the Child Support Act shall apply to any decision 
%%of the Secretary of State that an adjustment shall cease or with respect to the adjustment of amounts payable under a maintenance calculation for the purpose of taking account of overpayments of child support maintenance and voluntary payments, or a decision under section 17 of that Act, whether as originally made or as revised under section 16 of that Act.
%of the Commission with respect to the adjustment of amounts payable under a maintenance calculation for the purpose of taking account of overpayments of child support maintenance or voluntary payments\footnote{“Voluntary payment” is defined in section 54 of the Child Support Act 1991 (c. 48).}.  % Words substituted (6.4.09) by SI 2009/396 reg 4(15)
%
%\amendment{
%Reg. 30A inserted (3.3.03 for new-rules cases only) by the Child Support (Decisions and Appeals) (Amendment) Regulations 2000 reg. 12 (subject to reg. 1(2)).
%
%Words substituted in heading of reg. 30A (3.11.08) by the Tribunals, Courts and Enforcement Act 2007 (Transitional and Consequential Provisions) Order 2008 Sch. 1 para. 120.
%
%Words substituted in reg. 30A (6.4.09) by the Child Support (Miscellaneous Amendments) Regulations 2009 reg. 4(15).
%
%\medskip
%
%Reg. 31 omitted (3.11.08) by the Tribunals, Courts and Enforcement Act 2007 (Transitional and Consequential Provisions) Order 2008 Sch. 1 para. 121.
%}
%}

% Reg 31 omitted (3.11.08) by SI 2008/2683 Sch 1 para 121
%\subsubsection[31. Time within which an appeal is to be brought]{Time within which an appeal is to be brought}
%
%31.—(1) Where an appeal lies from a decision of the Secretary of State 
%or the Board or an officer of the Board  % Words inserted (5.10.99) by SI 1999/2570 reg 22(2)
%to an appeal tribunal, except in the case of a decision of the Secretary of State under section 3 or 3A of the Vaccine Damage Payments Act, the time within which that appeal must be brought is, subject to the following provisions of this Part—
%\begin{enumerate}\item[]
%%($a$) within one month of the date of notification of the decision against which the appeal is brought; or
%%
%%($b$) where a written statement of reasons for that decision is requested, within 14 days of the expiry of the period specified in sub-paragraph ($a$).
%
%% Reg 31(1)(a), (b), (c) substituted for reg. 31(1)(a), (b) (20.5.02) by SI 2002/1379 reg 9
%($a$) subject to regulation 9A(3), within one month of the date of notification of the decision against which the appeal is brought;
%
%($b$) where a written statement of the reasons for that decision is requested and provided within the period specified in sub-paragraph ($a$), within 14 days of the expiry of that period; or
%
%($c$) where a written statement of the reasons for that decision is requested but is not provided within the period specified in sub-paragraph ($a$), within 14 days of the date on which the statement is provided.
%\end{enumerate}
%
%(2) Where the Secretary of State
%or the Board or an officer of the Board—  % Words inserted (5.10.99) by SI 1999/2570 reg 22(3)
%\begin{enumerate}\item[]
%($a$) revises, or following an application for a revision under regulation 3(1) or (3)%
%\opt{newrules,2012rules}{% or 3A(1)
%, 3A(1) or regulation 17(1)($a$)  of the Child Support (Maintenance Assessment Procedure) Regulations 1992\footnote{S.I. 1992/1813; regulation 17(1)($a$) was made under section 16 of the Child Support Act 1991 (c.\ 48) before that section was amended by the Child Support, Pensions and Social Security Act 2000 (c.\ 19), section 8.}%  % Words substituted (18.3.05) by SI 2005/337 reg 2(8)
%}  % Words inserted (3.3.03) for new-rules cases only by SI 2002/1204 reg 2(5)
%does not revise, a decision under section 16 of the Child Support Act or under section 9, or
%
%($b$) supersedes a decision under section 17 of the Child Support Act or under section 10,
%\end{enumerate}
%the period of one month specified in paragraph (1) shall begin to run from the date of notification of the revision or supersession of the decision, or following an application for a revision under regulation 3(1) or (3)% 
%\opt{newrules,2012rules}{% or 3A(1)
%, 3A(1) or regulation 17(1)($a$)  of the Child Support (Maintenance Assessment Procedure) Regulations 1992%  % Words substituted (18.3.05) by SI 2005/337 reg 2(8)
%}%  % Words inserted (3.3.03) for new-rules cases only by SI 2002/1204 reg 2(5)
%, the date the Secretary of State 
%or the Board or an officer of the Board  % Words inserted (5.10.99) by SI 1999/2570 reg 22(3)
%issues a notice that he is 
%or they are  % Words inserted (5.10.99) by SI 1999/2570 reg 22(4)
%not revising the decision.
%
%\begin{sloppypar}
%(3) An appeal against a certificate of recoverable benefits 
%or, as the case may be, recoverable lump sum payments  % Words inserted (1.10.08) by SI 2008/1596 Sch 2 para 1(d)(i)
%must be brought—
%\end{sloppypar}
%\begin{enumerate}\item[]
%($a$) not later than one month after the date a person making a compensation payment discharges his liability under section 6 of the 1997 Act
%or, in the case of lump sum payments, regulation 10 of the Lump Sum Payments Regulations%  % Words inserted (1.10.08) by SI 2008/1596 Sch 2 para 1(d)(ii)
%;
%
%($b$) where the certificate is reviewed by the Secretary of State 
%or the Board or an officer of the Board  % Words inserted (5.10.99) by SI 1999/2570 reg 22(3)
%in accordance with regulations made under section 11(5)($c$) of the 1997 Act, not later than one month after the date the certificate is confirmed, or, as the case may be, a fresh certificate is issued; or
%
%%($c$) where an agreement is made under which an earlier compensation payment is treated as having been made in final discharge of a claim made by or in respect of an injured person and arising out of the accident, injury or disease, not later than one month after the date of that agreement.
%
%% Words inserted (1.10.08) by SI 2008/1596 Sch 2 para 1(d)(iii)
%($c$) where an agreement is made under which an earlier compensation payment is treated as having been made in final discharge of a claim made by or in respect of—
%\begin{enumerate}\item[]
%(i) an injured person, arising out of the accident, injury or disease; or
%
%(ii) $P$, arising out of the disease,
%\end{enumerate}
%not later than one month after the date of that agreement.
%\end{enumerate}
%
%(4) Where a dispute arises as to whether an appeal was brought within the time limit specified in this regulation, the dispute shall be referred to, and be determined by, a legally qualified panel member.
%
%(5) The time limit specified in this regulation for bringing an appeal may be extended in accordance with regulation 32.
%
%\amendment{
%Words inserted in reg. 31(1), (2), (3) (5.10.99) by the Tax Credits (Decisions and Appeals) (Amendment) Regulations 1999 reg. 22.
%
%Reg. 31(1)(a), (b), (c) substituted for reg. 31(1)(a), (b) (20.5.02) by the Social Security and Child Support (Decisions and Appeals) (Miscellaneous Amendments) Regulations 2002 reg. 9.
%
%\opt{newrules}{
%Words inserted in reg. 31(2) (3.3.03 for new-rules cases only) by the Child Support (Miscellaneous Amendments) Regulations 2002 reg. 2(5).
%
%Under the Child Support Appeals (Jurisdiction of Courts) Order 2002 art. 5, this regulation applies in England and Wales to appeals brought under that Order as if ``an appeal tribunal'' were read as ``a court'' and as if ``legally qualified panel member'' and ``panel member'' were read as ``justices' clerk or the court''.
%
%Under the Child Support Appeals (Jurisdiction of Courts) (Scotland) Order 2003 art. 5, this regulation applies in Scotland to appeals brought under that Order as if ``an appeal tribunal'' were read as ``a court'' and as if ``legally qualified panel member'' and ``panel member'' were read as ``the court''.
%}
%
%Words substituted in reg. 31(2) (18.3.05 for new-rules cases only) by the Social Security, Child Support and Tax Credits (Miscellaneous Amendments) Regulations 2005 reg. 2(8).
%
%Words inserted in reg. 31(3) and reg. 31(3)(c) substituted (1.10.08) by the Social Security (Recovery of Benefits) (Lump Sum Payments) Regulations 2008 Sch. 2 para. 1(d).
%}

% Reg 32 omitted (28.10.13) by SI 2013/2380 reg 4(10)(c)
%\subsubsection[32. Late appeals]{Late appeals}
%
%32.—%
%%(1) The time within which an appeal must be brought may be extended where the conditions specified in paragraphs (2) to (8) are satisfied, but no appeal shall in any event be brought more than one year after the expiration of the last day for appealing under regulation 31.
%%
%%(2) An application for an extension of time under this regulation shall be made in accordance with regulation 33 and shall be determined by a legally qualified panel member%.
%%, except that where the Secretary of State or the Board, as the case may be, consider that the conditions in paragraphs (4)($b$)  to (8) are satisfied, the Secretary of State or the Board, as the case may be, may grant the application.  % Words added (20.5.02) by SI 2002/1379 reg 10(a)
%%
%%(3) An application under this regulation shall contain particulars of the grounds on which the extension of time is sought, including details of any relevant special circumstances for the purposes of paragraph (4).
%%
%% Reg 32(1), (2) substituted for reg 32(1)--(3) (3.11.08) by SI 2008/2683 Sch 1 para 122(a)
%(1) Where a dispute arises as to whether an appeal was brought within the time specified under Tribunal Procedure Rules the dispute shall be referred to, and determined by, the First-tier Tribunal.
%
%(2) The Secretary of State
%%, the Commission  % Words omitted (1.8.12) by SI 2012/2007 Sch para 113(13)
%or the Board, as the case may be, may treat a late appeal as made in time in accordance with Tribunal Procedure Rules if the conditions in paragraphs (4) to (8) are satisfied.
%
%%(4) An application for an extension of time shall not be granted unless the panel member is satisfied that—
%%\begin{enumerate}\item[]
%%($a$) if the application is granted there are reasonable prospects that the appeal will be successful;
%%
%%($b$) it is in the interests of justice for the application to be granted.
%%\end{enumerate}
%
%% Reg 32(4) substituted (20.5.02) by SI 2002/1379 reg 10(b)
%%(4) An application for an extension of time shall not be granted unless—
%%\begin{enumerate}\item[]
%%($a$) the panel member is satisfied that, if the application is granted, there are reasonable prospects that the appeal will be successful; or
%%
%%($b$) the panel member, the Secretary of State or the Board, as the case may be, are satisfied that it is in the interests of justice for the application to be granted.
%%\end{enumerate}
%
%% Reg 32(4) substituted (3.11.08) by SI 2008/2683 Sch 1 para 122(b)
%(4) An appeal may be treated as made in time if the Secretary of State
%%, the Commission  % Words omitted (1.8.12) by SI 2012/2007 Sch para 113(13)
%or the Board, as the case may be, is satisfied that it is in the interests of justice.
%
%(5) For the purposes of paragraph (4) it is not in the interests of justice to 
%%grant an application unless the panel member% 
%treat the appeal as made in time unless   % Words substituted (3.11.08) by SI 2008/2683 Sch 1 para 122(c)(i)
%%, 
%the Secretary of State or the Board, as the case may be,  % Words inserted (20.5.02) by SI 2002/1379 reg 10(c)(i)
%is satisfied that—
%\begin{enumerate}\item[]
%($a$) the special circumstances specified in paragraph (6) are relevant% 
%%to the application  % Words omitted (3.11.08) by SI 2008/2683 Sch 1 para 122(c)(ii)
%; or
%
%($b$) some other special circumstances exist which are wholly exceptional and relevant% 
%%to the application  % Words omitted (3.11.08) by SI 2008/2683 Sch 1 para 122(c)(ii)
%,
%\end{enumerate}
%and as a result of those special circumstances, it was not practicable for the 
%%application to be made 
%appeal to be made  % Words substituted (20.5.02) by SI 2002/1379 reg 10(c)(ii)
%within the time limit specified in 
%%regulation 31
%Tribunal Procedure Rules%  % Words substituted (3.11.08) by SI 2008/2683 Sch 1 para 122(c)(iii)
%.
%
%(6) For the purposes of paragraph (5)($a$), the special circumstances are that—
%\begin{enumerate}\item[]
%($a$) the 
%%applicant 
%appellant  % Word substituted (3.11.08) by SI 2008/2683 Sch 1 para 122(d)
%or a 
%%spouse 
%partner  % Word substituted (20.5.02) by SI 2002/1379 reg 10(d)
%or dependant of the 
%%applicant 
%appellant  % Word substituted (3.11.08) by SI 2008/2683 Sch 1 para 122(d)
%has died or suffered serious illness;
%
%($b$) the 
%%applicant 
%appellant  % Word substituted (3.11.08) by SI 2008/2683 Sch 1 para 122(d)
%is not resident in the United Kingdom; or
%
%($c$) normal postal services were disrupted.
%\end{enumerate}
%
%(7) In determining whether it is in the interests of justice to 
%%grant the application
%treat the appeal as made in time%  % Words substituted (3.11.08) by SI 2008/2683 Sch 1 para 122(e)(i)
%, 
%%the panel member shall have regard 
%regard shall be had  % Words substituted (20.5.02) by SI 2002/1379 reg 10(e)
%to the principle that the greater the amount of time that has elapsed between the expiration of the time 
%%within which the appeal is to be brought under regulation 31 and the making of the application for an extension of time, the more compelling should be the special circumstances on which the application is based.
%limit under Tribunal Procedure Rules and the submission of the notice of appeal, the more compelling should be the special circumstances.  % Words substituted (3.11.08) by SI 2008/2683 Sch 1 para 122(e)(ii)
%
%(8) In determining whether it is in the interests of justice to 
%%grant an application
%treat the appeal as made in time%  % Words substituted (3.11.08) by SI 2008/2683 Sch 1 para 122(f)(i)
%, no account shall be taken of the following—
%\begin{enumerate}\item[]
%($a$) that the applicant or any person acting for him was unaware of or misunderstood the law applicable to his case (including ignorance or misunderstanding of the time limits imposed by 
%%these Regulations
%Tribunal Procedure Rules%  % Words substituted (3.11.08) by SI 2008/2683 Sch 1 para 122(f)(ii)
%); or
%
%($b$) that 
%%a Commissioner 
%the Upper Tribunal  % Words substituted (3.11.08) by SI 2008/2683 Sch 1 para 122(f)(iii)
%or a court has taken a different view of the law from that previously understood and applied.
%\end{enumerate}
%
%% Reg 32(9)--(11) omitted (3.11.08) by SI 2008/2683 Sch 1 para 122(g)
%%(9) An application under this regulation for an extension of time which has been refused may not be renewed.
%%
%%(10) The panel member who determines an application under this regulation shall record a summary of his decision in such written form as has been approved by the President.
%%
%%(11) As soon as practicable after the decision is made a copy of the decision shall be sent or given to every party to the proceedings.
%
%\amendment{
%Words inserted in reg. 32(2), (5), words substituted in reg. 32(5), (6)(a), (7) and reg. 32(4) substituted (20.5.02) by the Social Security and Child Support (Decisions and Appeals) (Miscellaneous Amendments) Regulations 2002 reg. 10.
%
%Under the Child Support Appeals (Jurisdiction of Courts) Order 2002 art. 5, this regulation applies in England and Wales to 2003 scheme appeals brought under that Order as if ``an appeal tribunal'' were read as ``a court'', as if ``legally qualified panel member'' and ``panel member'' were read as ``justices' clerk or the court'', and as if in reg. 32(10) ``such written form as has been approved by the President'' were read as ``written form''.
%
%Under the Child Support Appeals (Jurisdiction of Courts) (Scotland) Order 2003 art. 5, this regulation applies in Scotland to 2003 scheme appeals brought under that Order as if ``an appeal tribunal'' were read as ``a court'', as if ``legally qualified panel member'' and ``panel member'' were read as ``the court'', and as if in reg. 32(10) ``such written form as has been approved by the President'' were read as ``written form''.
%
%Words substituted in reg. 32(5)--(8), words omitted in reg. 32(5)(a), (b), reg. 32(1), (2) substituted for reg. 32(1)--(3), reg. 32(4) substituted and reg. 32(9)--(11) omitted (3.11.08) by the Tribunals, Courts and Enforcement Act 2007 (Transitional and Consequential Provisions) Order 2008 Sch. 1 para. 122.
%
%Words omitted in reg. 32(2), (4) (1.8.12) by the Public Bodies (Child Maintenance and Enforcement Commission: Abolition and Transfer of Functions) Order 2012 Sch. para. 113(13).
%}

% Reg 33 omitted (28.10.13) by SI 2013/2380 reg 4(10)(d)
%\subsubsection[33. 
%%Making of appeals and applications
%Notice of appeal --- \emph{1993 scheme version}%  % Heading substituted (3.11.08) by SI 2008/2683 Sch 1 para 123(2)
%]{%
%%Making of appeals and applications
%Notice of appeal\\*\emph{1993 scheme version}%  % Heading substituted (3.11.08) by SI 2008/2683 Sch 1 para 123(2)
%}
%
%33.—%
%% Reg 33(1) omitted (3.11.08) by SI 2008/2683 Sch 1 para 123(3)
%%(1) An appeal, or an application for an extension of time for making an appeal to an appeal tribunal shall be in writing either on a form approved for the purpose by the Secretary of State 
%%or the Board   % Words inserted (5.10.99) by SI 1999/2570 reg 23(2)(a)
%%or in such other format as the Secretary of State accepts 
%%or the Board accept   % Words inserted (5.10.99) by SI 1999/2570 reg 23(2)(b)
%%as sufficient for the purpose and shall—
%%\begin{enumerate}\item[]
%%($a$) be signed by—
%%\begin{enumerate}\item[]
%%(i) the person who, under 
%%section 4(1) of the Vaccine Damage Payments Act,  % Words inserted (18.10.99) by SI 1999/2677 reg 9
%%section 20 of the Child Support Act% 
%%\opt{oldrules}{ as extended by paragraph 3 of Schedule 4C to that Act}%  Words omitted for new-rules cases by SI 2001/158 reg 4(4)
%%, section 11(2) of the 1997 Act or section 12(2), has a right of appeal; or
%%
%%(ii) where the person in head (i) has provided written authority to a representative to act on his behalf, by that representative;
%%\end{enumerate}
%%
%%($b$) be sent or delivered to an appropriate office;
%%
%%($c$) contain particulars of the grounds on which it is made; and
%%
%%($d$) contain sufficient particulars of the decision, the certificate of recoverable benefits or the subject of the application, as the case may be, to enable that decision, certificate or subject of the application to be identified.
%%\end{enumerate}
%%
%(2) 
%%In this regulation, “an appropriate office” means
%A notice of appeal made in accordance with Tribunal Procedure Rules and on a form approved by the Secretary of State
%%, the Commission  % Words omitted (1.8.12) by SI 2012/2007 Sch para 113(14)
%or the Board, as the case may be, or in such other format as the Secretary of State
%%, the Commission  % Words omitted (1.8.12) by SI 2012/2007 Sch para 113(14)
%or the Board, as the case may be, accepts, is to be sent or delivered to the following appropriate office%  % Words substituted (3.11.08) by SI 2008/2683 Sch 1 para 123(4)
%—
%\begin{enumerate}\item[]
%($a$) in the case of an appeal under the 1997 Act against a certificate of recoverable benefits
%or, as the case may be, recoverable lump sum payments%  % Words inserted (1.10.08) by SI 2008/1596 Sch 2 para 1(e)
%, the Compensation Recovery Unit of the 
%%Department of Social Security 
%Department for Work and Pensions  % Words substituted (20.5.02) by SI 2002/1379 reg 11(a)(i)
%at 
%%Reyrolle Building, Hebburn, Tyne and Wear, \textsc{\lowercase{NE31 1XB}}
%Durham House, Washington, Tyne and Wear, \textsc{\lowercase{NE38 7SF}}%  % Words substituted (4.12.00) by SI 2000/3030 reg 4
%;
%
%($b$) in the case of an appeal against a decision relating to a jobseeker’s allowance, an office of the 
%%Department of Social Security or of the Department for Education and Employment
%Department for Work and Pensions the address of which was indicated on the notification of the decision which is subject to appeal%  % Words substituted (20.5.02) by SI 2002/1379 reg 11(a)(ii)
%;
%
%($c$) in the case of a contributions decision which falls within Part II of Schedule 3 to the Act, any National Insurance Contributions office
%of the Board, or any office of the 
%%Department of Social Security 
%Department for Work and Pensions%  % Words substituted (20.5.02) by SI 2002/1379 reg 11(a)(i)
%;  % Words inserted (5.7.99) by SI 1999/1662 reg 3(4)(a)
%
%% Reg 33(2)(cc) inserted (5.7.99) by SI 1999/1662 reg 3(4)(b)
%($cc$) in the case of a decision made under the Pension Schemes Act 1993 by virtue of section 170(2) of that Act, any National Insurance Contributions office of the Board;
%
%($d$) in the case of an appeal under section 20 of the Child Support Act as extended by paragraph 3 of Schedule 4C to that Act, an office of the Child Support Agency; %and  % Word omitted (5.10.99) by SI 1999/2570 reg 23(3)(a)
%
%% Reg 33(2)(dd) inserted (5.10.99) by SI 1999/2570 reg 23(3)(b)
%($dd$) in the case of an appeal against a decision relating to working families' tax credit or disabled person’s tax credit, a Tax Credits Office of the Board; and
%
%% Reg 33(2)(ddd) inserted (3.4.00) by SI 2000/897 Sch 6 para 6
%($ddd$) in a case where the decision appealed against was a decision arising from a claim to a designated office, an office of a designated authority;
%
%($e$) in any other case, an office of the 
%%Department of Social Security
%Department for Work and Pensions the address of which was indicated on the notification of the decision which is subject to appeal%  % Words substituted (20.5.02) by SI 2002/1379 reg 11(a)(iii)
%.
%\end{enumerate}
%
%%(3) A form which is not completed in accordance with the instructions on the form—
%%\begin{enumerate}\item[]
%%($a$) except where paragraph (4) applies, does not satisfy the requirements of paragraph (1), and
%%
%%($b$) may be returned by the Secretary of State 
%%or the Board   % Words inserted (5.10.99) by SI 1999/2570 reg 23(4)
%%to the sender for completion in accordance with those instructions.
%%\end{enumerate}
%
%% Reg 33(3) substituted (3.11.08) by SI 2008/2683 Sch 1 para 123(5)
%(3) Except where paragraph (4) applies, where a form does not contain the information required under Tribunal Procedure Rules the form may be returned by the Secretary of State
%%, the Commission  % Words omitted (1.8.12) by SI 2012/2007 Sch para 113(14)
%or the Board to the sender for completion in accordance with the Tribunal Procedure Rules.
%
%(4) Where the Secretary of State is satisfied 
%or the Board are satisfied   % Words inserted (5.10.99) by SI 1999/2570 reg 23(5)(a)
%that the form, although not completed in accordance with the instructions on it, includes sufficient information to enable the appeal 
%%or application  % Words omitted (3.11.08) by SI 2008/2683 Sch 1 para 123(6)(a)
%to proceed, he 
%or they   % Words inserted (5.10.99) by SI 1999/2570 reg 23(5)(b)
%may treat the form as satisfying the requirements of 
%%paragraph (1)
%Tribunal Procedure Rules%  % Words substituted (3.11.08) by SI 2008/2683 Sch 1 para 123(6)(b)
%.
%
%(5) Where 
%%an appeal or application 
%a notice of appeal  % Words substituted (3.11.08) by SI 2008/2683 Sch 1 para 123(7)(a)
%is made in writing otherwise than on the approved form (“the letter”), and the letter includes sufficient information to enable the appeal 
%%or application  % Words omitted (3.11.08) by SI 2008/2683 Sch 1 para 123(7)(b)
%to proceed, the Secretary of State 
%or the Board   % Words inserted (5.10.99) by SI 1999/2570 reg 23(4)
%may treat the letter as satisfying the requirements of 
%%paragraph (1)
%Tribunal Procedure Rules%  % Words substituted (3.11.08) by SI 2008/2683 Sch 1 para 123(7)(c)
%.
%
%(6) Where the letter does not include sufficient information to enable the appeal %or application  % Words omitted (3.11.08) by SI 2008/2683 Sch 1 para 123(7)(b)
%to proceed, the Secretary of State 
%or the Board   % Words inserted (5.10.99) by SI 1999/2570 reg 23(4)
%may request further information in writing (“further particulars”) from the person who wrote the letter.
%
%%(7) Where a person to whom a form is returned or from whom further particulars are requested duly completes and returns the form or sends the further particulars and the form or particulars (as the case may be) are received by the Secretary of State 
%%or the Board   % Words inserted (5.10.99) by SI 1999/2570 reg 23(4)
%%within—
%%\begin{enumerate}\item[]
%%($a$) 14 days of the date on which the form was returned to him by the Secretary of State
%%or the Board,   % Words inserted (5.10.99) by SI 1999/2570 reg 23(4)
%%
%%($b$) 14 days of the date on which the Secretary of State’s 
%%or the Board's   % Words inserted (5.10.99) by SI 1999/2570 reg 23(6)
%%request was made (“the date of request”), or
%%
%%($c$) such longer period as the Secretary of State 
%%or the Board   % Words inserted (5.10.99) by SI 1999/2570 reg 23(4)
%%may direct,
%%\end{enumerate}
%%the time for making the appeal shall be extended by 14 days from the date the form was returned, the date of request or the date of the Secretary of State’s 
%%or the Board 's  % Words inserted (5.10.99) by SI 1999/2570 reg 23(6)
%%direction, as the case may be.
%
%% Reg 33(7) substituted (20.5.02) by SI 2002/1379 reg 11(b)
%(7) Where a person to whom a form is returned, or from whom further particulars are requested, duly completes and returns the form or sends the further particulars, if the form or particulars, as the case may be, are received by the Secretary of State or the Board within—
%\begin{enumerate}\item[]
%($a$) 14 days of the date on which the form was returned to him by the Secretary of State or the Board, the time for making the appeal shall be extended by 14 days from the date on which the form was returned;
%
%($b$) 14 days of the date on which the Secretary of State’s or the Board’s request was made, the time for making the appeal shall be extended by 14 days from the date of the request; or
%
%($c$) such longer period as the Secretary of State or the Board may direct, the time for making the appeal shall be extended by a period equal to that longer period directed by the Secretary of State or the Board.
%\end{enumerate}
%
%(8) Where a person to whom a form is returned or from whom further particulars are requested does not complete and return the form or send further particulars within the period of time specified in paragraph (7)—
%\begin{enumerate}\item[]
%($a$) the Secretary of State 
%or the Board   % Words inserted (5.10.99) by SI 1999/2570 reg 23(4)
%shall forward a copy of the form, or as the case may be, the letter, together with any other relevant documents or evidence to 
%%a legally qualified panel member
%the First-tier Tribunal%  % Words substituted (3.11.08) by SI 2008/2683 Sch 1 para 123(9)(a)
%, and
%
%($b$) the 
%%panel member 
%First-tier Tribunal  % Words substituted (3.11.08) by SI 2008/2683 Sch 1 para 123(9)(b)(i)
%shall determine whether the form or the letter satisfies the requirement of 
%%paragraph (1), and shall inform the appellant or applicant and the Secretary of State 
%%or the Board   % Words inserted (5.10.99) by SI 1999/2570 reg 23(4)
%%of his determination.
%Tribunal Procedure Rules.  % Words substituted (3.11.08) by SI 2008/2683 Sch 1 para 123(9)(b)(ii)
%\end{enumerate}
%
%(9) Where—
%\begin{enumerate}\item[]
%($a$) a form is duly completed and returned or further particulars are sent after the expiry of the period of time allowed in accordance with paragraph (7), and
%
%($b$) no decision has been made under paragraph (8) at the time the form or the further particulars are received by the Secretary of State
%or the Board,   % Words inserted (5.10.99) by SI 1999/2570 reg 23(4)
%\end{enumerate}
%that form or further particulars shall also be forwarded to the 
%%legally qualified panel member who 
%First-tier Tribunal which  % Words substituted (3.11.08) by SI 2008/2683 Sch 1 para 123(10)
%shall take into account any further information or evidence set out in the form or further particulars.
%
%% Reg 33(10) added (19.6.00) by SI 2000/1596 reg 23
%%(10) The Secretary of State may discontinue action on an appeal where the appeal has not been forwarded to the clerk to an appeal tribunal or to a legally qualified panel member and the appellant has given written notice that he does not wish the appeal to continue.
%
%% Reg 33(10) substituted (20.5.02) by SI 2002/1379 reg 11(c)
%(10) The Secretary of State or the Board may discontinue action on an appeal where the
%notice of  % Words inserted (3.11.08) by SI 2008/2683 Sch 1 para 123(11)(a)
%appeal has not been forwarded to the 
%%clerk to an appeal tribunal or to a legally qualified panel member 
%First-tier Tribunal  % Words substituted (3.11.08) by SI 2008/2683 Sch 1 para 123(11)(b)
%and the appellant or an authorised representative of the appellant has given written notice that he does not wish the appeal to continue.
%
%\amendment{
%Words inserted in reg. 33(2)(c) and reg. 33(2)(cc) inserted (5.7.99) by the Social Security Contributions (Transfer of Functions, etc.) Act 1999 (Commencement No. 2 and Consequential and Transitional Provisions) Order 1999 art. 3(4) (subject to transitional provisions in art. 4).
%
%Words inserted in reg. 33(1), (3), (4), (5)--(9), word omitted in reg. 33(2)(d) and reg. 33(2)(dd) inserted (5.10.99) by the Tax Credits (Decisions and Appeals) (Amendment) Regulations 1999 reg. 23.
%
%Words inserted in reg. 33(1)(a)(i) (18.10.99) by the Social Security and Child Support (Decisions and Appeals), Vaccine Damage Payments and Jobseeker's Allowance (Amendment) Regulations 1999 reg. 9.
%
%Reg. 33(2)(ddd) inserted (3.4.00) by the Social Security (Work-focused Interviews) Regulations 2000 Sch. 6 para. 6.
%
%Reg. 33(10) added (19.6.00) by the Social Security and Child Support (Miscellaneous Amendments) Regulations 2000 reg. 23.
%
%Words substituted in reg. 33(2)(a) (4.12.00) by the Social Security (Recovery of Benefits) (Miscellaneous Amendments) Regulations 2000 reg. 4.
%
%Words substituted in reg. 33(2)(a), (b), (c), (e) and reg. 33(7), (10) substituted (20.5.02) by the Social Security and Child Support (Decisions and Appeals) (Miscellaneous Amendments) Regulations 2002 reg. 11.
%
%Words inserted in reg. 33(1)(d), (2)(a) (1.10.08) by the Social Security (Recovery of Benefits) (Lump Sum Payments) Regulations 2008 Sch. 2 para. 1(e) as amended by the Social Security (Miscellaneous Amendments) (No. 3) Regulations 2008 reg. 6(6).
%
%Words inserted in reg. 33(10), words substituted in reg. 33(2), (4), (5), (8)(a), (b), (9), (10), words in reg. 33(4), (5), (6) omitted, reg. 33(3) and heading substituted and reg. 33(1) omitted (3.11.08) by the Tribunals, Courts and Enforcement Act 2007 (Transitional and Consequential Provisions) Order 2008 Sch. 1 para. 123.
%
%Words omitted in reg. 33(2), (3) (1.8.12) by the Public Bodies (Child Maintenance and Enforcement Commission: Abolition and Transfer of Functions) Order 2012 Sch. para. 113(14).
%
%Reg. 33(2)(d) omitted (10.12.12 for 2012 scheme cases only) by the Child Support (Meaning of Child and New Calculation Rules) (Consequential and Miscellaneous Amendment) Regulations 2012 reg. 6(7).
%}
%
%\subsubsection[33. 
%%Making of appeals and applications
%Notice of appeal --- \emph{2003 scheme version}  % Heading substituted (3.11.08) by SI 2008/2683 Sch 1 para 123(2)
%]{%
%%Making of appeals and applications
%Notice of appeal\\*\emph{2003 scheme version}  % Heading substituted (3.11.08) by SI 2008/2683 Sch 1 para 123(2)
%}
%
%33.—%
%% Reg 33(1) omitted (3.11.08) by SI 2008/2683 Sch 1 para 123(3)
%%(1) An appeal, or an application for an extension of time for making an appeal to an appeal tribunal shall be in writing either on a form approved for the purpose by the Secretary of State 
%%or the Board   % Words inserted (5.10.99) by SI 1999/2570 reg 23(2)(a)
%%or in such other format as the Secretary of State accepts 
%%or the Board accept   % Words inserted (5.10.99) by SI 1999/2570 reg 23(2)(b)
%%as sufficient for the purpose and shall—
%%\begin{enumerate}\item[]
%%($a$) be signed by—
%%\begin{enumerate}\item[]
%%(i) the person who, under 
%%section 4(1) of the Vaccine Damage Payments Act,  % Words inserted (18.10.99) by SI 1999/2677 reg 9
%%section 20 of the Child Support Act% 
%%\opt{oldrules}{ as extended by paragraph 3 of Schedule 4C to that Act}%  Words omitted for new-rules cases by SI 2001/158 reg 4(4)
%%, section 11(2) of the 1997 Act or section 12(2), has a right of appeal; or
%%
%%(ii) where the person in head (i) has provided written authority to a representative to act on his behalf, by that representative;
%%\end{enumerate}
%%
%%($b$) be sent or delivered to an appropriate office;
%%
%%($c$) contain particulars of the grounds on which it is made; and
%%
%%($d$) contain sufficient particulars of the decision, the certificate of recoverable benefits or the subject of the application, as the case may be, to enable that decision, certificate or subject of the application to be identified.
%%\end{enumerate}
%%
%(2) 
%%In this regulation, “an appropriate office” means
%A notice of appeal made in accordance with Tribunal Procedure Rules and on a form approved by the Secretary of State
%%, the Commission  % Words omitted (1.8.12) by SI 2012/2007 Sch para 113(14)
%or the Board, as the case may be, or in such other format as the Secretary of State
%%, the Commission  % Words omitted (1.8.12) by SI 2012/2007 Sch para 113(14)
%or the Board, as the case may be, accepts, is to be sent or delivered to the following appropriate office%  % Words substituted (3.11.08) by SI 2008/2683 Sch 1 para 123(4)
%—
%\begin{enumerate}\item[]
%($a$) in the case of an appeal under the 1997 Act against a certificate of recoverable benefits
%or, as the case may be, recoverable lump sum payments%  % Words inserted (1.10.08) by SI 2008/1596 Sch 2 para 1(e)
%, the Compensation Recovery Unit of the 
%%Department of Social Security 
%Department for Work and Pensions  % Words substituted (20.5.02) by SI 2002/1379 reg 11(a)(i)
%at 
%%Reyrolle Building, Hebburn, Tyne and Wear, \textsc{\lowercase{NE31 1XB}}
%Durham House, Washington, Tyne and Wear, \textsc{\lowercase{NE38 7SF}}%  % Words substituted (4.12.00) by SI 2000/3030 reg 4
%;
%
%($b$) in the case of an appeal against a decision relating to a jobseeker’s allowance, an office of the 
%%Department of Social Security or of the Department for Education and Employment
%Department for Work and Pensions the address of which was indicated on the notification of the decision which is subject to appeal%  % Words substituted (20.5.02) by SI 2002/1379 reg 11(a)(ii)
%;
%
%($c$) in the case of a contributions decision which falls within Part II of Schedule 3 to the Act, any National Insurance Contributions office
%of the Board, or any office of the 
%%Department of Social Security 
%Department for Work and Pensions%  % Words substituted (20.5.02) by SI 2002/1379 reg 11(a)(i)
%;  % Words inserted (5.7.99) by SI 1999/1662 reg 3(4)(a)
%
%% Reg 33(2)(cc) inserted (5.7.99) by SI 1999/1662 reg 3(4)(b)
%($cc$) in the case of a decision made under the Pension Schemes Act 1993 by virtue of section 170(2) of that Act, any National Insurance Contributions office of the Board;
%
%($d$) in the case of an appeal under section 20 of the Child Support Act% 
%%as extended by paragraph 3 of Schedule 4C to that Act%  %  Words omitted for new-rules cases by SI 2001/158 reg 4(4)
%, an office of the Child Support Agency; %and  % Word omitted (5.10.99) by SI 1999/2570 reg 23(3)(a)
%
%% Reg 33(2)(dd) inserted (5.10.99) by SI 1999/2570 reg 23(3)(b)
%($dd$) in the case of an appeal against a decision relating to working families' tax credit or disabled person’s tax credit, a Tax Credits Office of the Board; and
%
%% Reg 33(2)(ddd) inserted (3.4.00) by SI 2000/897 Sch 6 para 6
%($ddd$) in a case where the decision appealed against was a decision arising from a claim to a designated office, an office of a designated authority;
%
%($e$) in any other case, an office of the 
%%Department of Social Security
%Department for Work and Pensions the address of which was indicated on the notification of the decision which is subject to appeal%  % Words substituted (20.5.02) by SI 2002/1379 reg 11(a)(iii)
%.
%\end{enumerate}
%
%%(3) A form which is not completed in accordance with the instructions on the form—
%%\begin{enumerate}\item[]
%%($a$) except where paragraph (4) applies, does not satisfy the requirements of paragraph (1), and
%%
%%($b$) may be returned by the Secretary of State 
%%or the Board   % Words inserted (5.10.99) by SI 1999/2570 reg 23(4)
%%to the sender for completion in accordance with those instructions.
%%\end{enumerate}
%
%% Reg 33(3) substituted (3.11.08) by SI 2008/2683 Sch 1 para 123(5)
%(3) Except where paragraph (4) applies, where a form does not contain the information required under Tribunal Procedure Rules the form may be returned by the Secretary of State
%%, the Commission  % Words omitted (1.8.12) by SI 2012/2007 Sch para 113(14)
%or the Board to the sender for completion in accordance with the Tribunal Procedure Rules.
%
%(4) Where the Secretary of State is satisfied 
%or the Board are satisfied   % Words inserted (5.10.99) by SI 1999/2570 reg 23(5)(a)
%that the form, although not completed in accordance with the instructions on it, includes sufficient information to enable the appeal 
%%or application  % Words omitted (3.11.08) by SI 2008/2683 Sch 1 para 123(6)(a)
%to proceed, he 
%or they   % Words inserted (5.10.99) by SI 1999/2570 reg 23(5)(b)
%may treat the form as satisfying the requirements of 
%%paragraph (1)
%Tribunal Procedure Rules%  % Words substituted (3.11.08) by SI 2008/2683 Sch 1 para 123(6)(b)
%.
%
%(5) Where 
%%an appeal or application 
%a notice of appeal  % Words substituted (3.11.08) by SI 2008/2683 Sch 1 para 123(7)(a)
%is made in writing otherwise than on the approved form (“the letter”), and the letter includes sufficient information to enable the appeal 
%%or application  % Words omitted (3.11.08) by SI 2008/2683 Sch 1 para 123(7)(b)
%to proceed, the Secretary of State 
%or the Board   % Words inserted (5.10.99) by SI 1999/2570 reg 23(4)
%may treat the letter as satisfying the requirements of 
%%paragraph (1)
%Tribunal Procedure Rules%  % Words substituted (3.11.08) by SI 2008/2683 Sch 1 para 123(7)(c)
%.
%
%(6) Where the letter does not include sufficient information to enable the appeal %or application  % Words omitted (3.11.08) by SI 2008/2683 Sch 1 para 123(7)(b)
%to proceed, the Secretary of State 
%or the Board   % Words inserted (5.10.99) by SI 1999/2570 reg 23(4)
%may request further information in writing (“further particulars”) from the person who wrote the letter.
%
%%(7) Where a person to whom a form is returned or from whom further particulars are requested duly completes and returns the form or sends the further particulars and the form or particulars (as the case may be) are received by the Secretary of State 
%%or the Board   % Words inserted (5.10.99) by SI 1999/2570 reg 23(4)
%%within—
%%\begin{enumerate}\item[]
%%($a$) 14 days of the date on which the form was returned to him by the Secretary of State
%%or the Board,   % Words inserted (5.10.99) by SI 1999/2570 reg 23(4)
%%
%%($b$) 14 days of the date on which the Secretary of State’s 
%%or the Board's   % Words inserted (5.10.99) by SI 1999/2570 reg 23(6)
%%request was made (“the date of request”), or
%%
%%($c$) such longer period as the Secretary of State 
%%or the Board   % Words inserted (5.10.99) by SI 1999/2570 reg 23(4)
%%may direct,
%%\end{enumerate}
%%the time for making the appeal shall be extended by 14 days from the date the form was returned, the date of request or the date of the Secretary of State’s 
%%or the Board 's  % Words inserted (5.10.99) by SI 1999/2570 reg 23(6)
%%direction, as the case may be.
%
%% Reg 33(7) substituted (20.5.02) by SI 2002/1379 reg 11(b)
%(7) Where a person to whom a form is returned, or from whom further particulars are requested, duly completes and returns the form or sends the further particulars, if the form or particulars, as the case may be, are received by the Secretary of State or the Board within—
%\begin{enumerate}\item[]
%($a$) 14 days of the date on which the form was returned to him by the Secretary of State or the Board, the time for making the appeal shall be extended by 14 days from the date on which the form was returned;
%
%($b$) 14 days of the date on which the Secretary of State’s or the Board’s request was made, the time for making the appeal shall be extended by 14 days from the date of the request; or
%
%($c$) such longer period as the Secretary of State or the Board may direct, the time for making the appeal shall be extended by a period equal to that longer period directed by the Secretary of State or the Board.
%\end{enumerate}
%
%(8) Where a person to whom a form is returned or from whom further particulars are requested does not complete and return the form or send further particulars within the period of time specified in paragraph (7)—
%\begin{enumerate}\item[]
%($a$) the Secretary of State 
%or the Board   % Words inserted (5.10.99) by SI 1999/2570 reg 23(4)
%shall forward a copy of the form, or as the case may be, the letter, together with any other relevant documents or evidence to 
%%a legally qualified panel member
%the First-tier Tribunal%  % Words substituted (3.11.08) by SI 2008/2683 Sch 1 para 123(9)(a)
%, and
%
%($b$) the 
%%panel member 
%First-tier Tribunal  % Words substituted (3.11.08) by SI 2008/2683 Sch 1 para 123(9)(b)(i)
%shall determine whether the form or the letter satisfies the requirement of 
%%paragraph (1), and shall inform the appellant or applicant and the Secretary of State 
%%or the Board   % Words inserted (5.10.99) by SI 1999/2570 reg 23(4)
%%of his determination.
%Tribunal Procedure Rules.  % Words substituted (3.11.08) by SI 2008/2683 Sch 1 para 123(9)(b)(ii)
%\end{enumerate}
%
%(9) Where—
%\begin{enumerate}\item[]
%($a$) a form is duly completed and returned or further particulars are sent after the expiry of the period of time allowed in accordance with paragraph (7), and
%
%($b$) no decision has been made under paragraph (8) at the time the form or the further particulars are received by the Secretary of State
%or the Board,   % Words inserted (5.10.99) by SI 1999/2570 reg 23(4)
%\end{enumerate}
%that form or further particulars shall also be forwarded to the 
%%legally qualified panel member who 
%First-tier Tribunal which  % Words substituted (3.11.08) by SI 2008/2683 Sch 1 para 123(10)
%shall take into account any further information or evidence set out in the form or further particulars.
%
%% Reg 33(10) added (19.6.00) by SI 2000/1596 reg 23
%%(10) The Secretary of State may discontinue action on an appeal where the appeal has not been forwarded to the clerk to an appeal tribunal or to a legally qualified panel member and the appellant has given written notice that he does not wish the appeal to continue.
%
%% Reg 33(10) substituted (20.5.02) by SI 2002/1379 reg 11(c)
%(10) The Secretary of State or the Board may discontinue action on an appeal where the
%notice of  % Words inserted (3.11.08) by SI 2008/2683 Sch 1 para 123(11)(a)
%appeal has not been forwarded to the 
%%clerk to an appeal tribunal or to a legally qualified panel member 
%First-tier Tribunal  % Words substituted (3.11.08) by SI 2008/2683 Sch 1 para 123(11)(b)
%and the appellant or an authorised representative of the appellant has given written notice that he does not wish the appeal to continue.
%
%\amendment{
%Words inserted in reg. 33(2)(c) and reg. 33(2)(cc) inserted (5.7.99) by the Social Security Contributions (Transfer of Functions, etc.) Act 1999 (Commencement No. 2 and Consequential and Transitional Provisions) Order 1999 art. 3(4) (subject to transitional provisions in art. 4).
%
%Words inserted in reg. 33(1), (3), (4), (5)--(9), word omitted in reg. 33(2)(d) and reg. 33(2)(dd) inserted (5.10.99) by the Tax Credits (Decisions and Appeals) (Amendment) Regulations 1999 reg. 23.
%
%Words inserted in reg. 33(1)(a)(i) (18.10.99) by the Social Security and Child Support (Decisions and Appeals), Vaccine Damage Payments and Jobseeker's Allowance (Amendment) Regulations 1999 reg. 9.
%
%Reg. 33(2)(ddd) inserted (3.4.00) by the Social Security (Work-focused Interviews) Regulations 2000 Sch. 6 para. 6.
%
%Reg. 33(10) added (19.6.00) by the Social Security and Child Support (Miscellaneous Amendments) Regulations 2000 reg. 23.
%
%Words substituted in reg. 33(2)(a) (4.12.00) by the Social Security (Recovery of Benefits) (Miscellaneous Amendments) Regulations 2000 reg. 4.
%
%Words substituted in reg. 33(2)(a), (b), (c), (e) and reg. 33(7), (10) substituted (20.5.02) by the Social Security and Child Support (Decisions and Appeals) (Miscellaneous Amendments) Regulations 2002 reg. 11.
%
%Words omitted in reg. 33(1)(a)(i), (2)(d) (3.3.03) for new-rules cases only by the Child Support (Consequential Amendments and Transitional Provisions) Regulations 2001 reg. 4(4).
%
%Words inserted in reg. 33(1)(d), (2)(a) (1.10.08) by the Social Security (Recovery of Benefits) (Lump Sum Payments) Regulations 2008 Sch. 2 para. 1(e) as amended by the Social Security (Miscellaneous Amendments) (No. 3) Regulations 2008 reg. 6(6).
%
%Words inserted in reg. 33(10), words substituted in reg. 33(2), (4), (5), (8)(a), (b), (9), (10), words in reg. 33(4), (5), (6) omitted, reg. 33(3) and heading substituted and reg. 33(1) omitted (3.11.08) by the Tribunals, Courts and Enforcement Act 2007 (Transitional and Consequential Provisions) Order 2008 Sch. 1 para. 123.
%
%Words omitted in reg. 33(2), (3) (1.8.12) by the Public Bodies (Child Maintenance and Enforcement Commission: Abolition and Transfer of Functions) Order 2012 Sch. para. 113(14).
%}

% Reg 34 omitted (28.10.13) by SI 2013/2380 reg 4(10)(e)
%\subsubsection[34. Death of a party to an appeal]{Death of a party to an appeal}
%
%34.—(1) In any proceedings, on the death of a party to those proceedings (other than the Secretary of State
%or the Board% Words inserted (5.10.99) by SI 1999/2570 reg 24(a)
%), the Secretary of State 
%or the Board   % Words inserted (5.10.99) by SI 1999/2570 reg 24(a)
%may appoint such person as he thinks 
%or they think   % Words inserted (5.10.99) by SI 1999/2570 reg 24(b)
%fit to proceed with the appeal in the place of such deceased party.
%
%(2) A grant of probate, confirmation or letters of administration to the estate of the deceased party, whenever taken out, shall have no effect on an appointment made under paragraph (1).
%
%(3) Where a person appointed under paragraph (1) has, prior to the date of such appointment, taken any action in relation to the appeal on behalf of the deceased party, the effective date of appointment by the Secretary of State 
%or the Board   % Words inserted (5.10.99) by SI 1999/2570 reg 24(a)
%shall be the day immediately prior to the first day on which such action was taken.
%
%\amendment{
%Words inserted in reg. 34(1), (3) (5.10.99) by the Tax Credits (Decisions and Appeals) (Amendment) Regulations 1999 reg. 24.
%}

\section[Part V --- Appeal tribunals for social security, contracting out of pensions, vaccine damage and child support]{Part V\\*Appeal tribunals for social security, contracting out of pensions, vaccine damage and child support}

\amendment{
Chap. I (regs. 35--37) omitted (3.11.08) by the Tribunals, Courts and Enforcement Act 2007 (Transitional and Consequential Provisions) Order 2008 Sch. 1 para. 124.
}

% Regs 35-38 omitted (3.11.08) by SI 2008/2683 Sch 1 para 124
%\subsection[Chapter I --- The panel and appeal tribunals]{Chapter I\\*The panel and appeal tribunals}
%
%\subsubsection[35. Persons appointed to the panel]{Persons appointed to the panel}
%
%\renewcommand\parthead{--- Part V Chapter I}
%
%35.  For the purposes of section 6(3), the panel shall include persons with the qualifications specified in Schedule 3.
%
%\subsubsection[36. Composition of appeal tribunals]{Composition of appeal tribunals}
%
%36.—(1) Subject to the following provisions of this regulation, an appeal tribunal
%%, including an appeal tribunal determining a misconceived appeal as a preliminary issue in accordance with regulation 48,  % Words omitted (19.6.00) by SI 2000/1596 reg 24(a)
%shall consist of a legally qualified panel member.
%
%%(2) Subject to paragraphs (3), (4) and (5), an appeal tribunal shall consist of a medically qualified panel member and a legally qualified panel member where—
%%\begin{enumerate}\item[]
%%($a$) the issue, or one of the issues raised on the appeal relates to—
%%\begin{enumerate}\item[]
%%(i) incapacity benefit under section 30A of the Contributions and Benefits Act\footnote{\frenchspacing Section 30A was inserted by section 1 of the Social Security (Incapacity for Work) Act 1994 (c. 18).};
%%
%%(ii) industrial injuries benefit under Part V of that Act; or
%%
%%(iii) severe disablement allowance under section 68 of that Act;
%%\end{enumerate}
%%
%%($b$) the appeal is made under section 11(1)($b$) of the 1997 Act; or
%%
%%($c$) the appeal is made under section 4 of the Vaccine Damage Payments Act.
%%\end{enumerate}
%
%% Reg 36(2) substituted (1.6.99) by SI 1999/1466 reg 2(a)
%(2) Subject to 
%%paragraphs (3) to (5), 
%paragraphs (3) to (5)
%%, (8) and (9)  % Words substituted (19.6.00) by SI 2000/1596 reg 24(b)
%and (8)%  % Words substituted (21.12.04) by SI 2004/3368 reg 2(4)(a)
%, an appeal tribunal shall consist of a legally qualified panel member and—
%\begin{enumerate}\item[]
%($a$) a medically qualified panel member where—
%\begin{enumerate}\item[]
%(i) the issue, or one of the issues, raised on the appeal is whether the 
%%all work test 
%personal capability assessment  % Words substituted (19.6.00) by SI 2000/1596 reg 24(c)
%is satisfied; or
%
%(ii) the appeal is made under section 11(1)($b$) of the 1997 Act; or
%\end{enumerate}
%
%($b$) one medically qualified panel member or two such members or one medically qualified panel member and an additional member drawn from the panel for the purposes described in paragraph (5) below where—
%\begin{enumerate}\item[]
%(i) the issue, or one of the issues, raised on the appeal 
%(not being an appeal where the only issue is whether there should be a declaration of an industrial accident under section 29(2))  % Words inserted (19.6.00) by SI 2000/1596 reg 24(d)
%relates to either industrial injuries benefit under Part V of the Contributions and Benefits Act or severe disablement allowance under section 68 of that Act; or
%
%(ii) the appeal is made under section 4 of the Vaccine Damage Payments Act.
%\end{enumerate}
%\end{enumerate}
%
%(3) An appeal tribunal shall consist of a financially qualified panel member and a legally qualified panel member where—
%\begin{enumerate}\item[]
%($a$) the issue raised, or one of the issues raised on appeal or referral, relates to child support or a relevant benefit; and
%
%($b$) the appeal or referral may require consideration by members of the appeal tribunal of issues which are, in the opinion of the President, difficult and which relate to—
%\begin{enumerate}\item[]
%(i) profit and loss accounts, revenue accounts or balance sheets relating to any enterprise;
%
%(ii) an income and expenditure account in the case of an enterprise not trading for profit; or
%
%(iii) the accounts of any trust fund.
%\end{enumerate}
%\end{enumerate}
%
%(4) Where the composition of an appeal tribunal would fall to be prescribed under both paragraphs (2) and (3), it shall consist of a medically qualified panel member, a financially qualified panel member and a legally qualified panel member.
%
%(5) Where the composition of an appeal tribunal is prescribed under 
%%paragraphs (1), (2) or 
%paragraph (1), (2)($a$)
%  %or  % Words substituted (1.6.99) by SI 1999/1466 reg 2(b)
%  %(3), 
%%    , (3) or (9)  % Words substituted (19.6.00) by SI 2000/1596 reg 24(e)
%or (3)  % Words substituted (21.12.04) by SI 2004/3368 reg 2(4)(b)
%the President may determine that the appeal tribunal shall include such an additional member drawn from the panel constituted under section 6 as he considers appropriate for the purposes of providing further experience for that additional member or for assisting the President in the monitoring of standards of decision making by panel members.
%\looseness=-1
%
%(6) An appeal tribunal shall consist of a legally qualified panel member, a medically qualified panel member and a panel member with a disability qualification in any appeal which relates to an attendance allowance or a disability living allowance under Part III of the Contributions and Benefits Act or 
%%a disability working allowance 
%a disabled person’s tax credit  % Words substituted (19.6.00) by SI 2000/1596 reg 24(f)
%under section 129 of that Act.
%
%% Reg 36(7) added (1.6.99) by SI 1999/1466 reg 2(b)
%(7) In paragraph (2)($a$)(i) above, “%
%%all work test
%personal capability assessment% Words substituted (19.6.00) by SI 2000/1596 reg 24(g)
%” has the meaning it bears in regulation 2(1) of the Social Security (Incapacity for Work) (General) Regulations 1995\footnote{\frenchspacing S.I. 1995/311.}.
%
%% Reg 36(8), (9) added (19.6.00) by SI 2000/1596 reg 24(h)
%(8) A person shall not act as a medically qualified panel member of an appeal tribunal in any appeal if he has at any time advised or prepared a report upon any person whose medical condition is relevant to the issue in the appeal, or has at any time regularly attended such a person.
%
%% Reg 36(9) omitted (21.12.04) by SI 2004/3368 reg 2(4)(c)
%%(9) Subject to paragraph (5), an appeal tribunal determining a misconceived appeal as a preliminary issue in accordance with regulation 48 shall consist of a legally qualified panel member.
%
%\amendment{
%Words substituted in reg. 36(5), reg. 36(7) added and reg. 36(2) substituted (1.6.99) by the Social Security and Child Support (Decisions and Appeals) (Amendment) Regulations 1999 reg. 2.
%
%Words inserted in reg. 36(2)(b)(i), words substituted in reg. 36(2), (5), (6), (7), words omitted in reg. 36(1) and reg. 36(8), (9) added (19.6.00) by the Social Security and Child Support (Miscellaneous Amendments) Regulations 2000 reg. 24.
%
%Words substituted in reg. 36(2), (5) and reg. 36(9) omitted (21.12.04) by the Social Security, Child Support and Tax Credits (Decisions and Appeals) Amendment Regulations 2004 reg. 2(4).
%}
%
%\subsubsection[37. Assignment of clerks to appeal tribunals: function of clerks]{Assignment of clerks to appeal tribunals: function of clerks}
%
%37.  The Secretary of State shall assign a clerk to service each appeal tribunal and the clerk so assigned shall be responsible for summoning members of the panel constituted under section 6 to serve on the tribunal.

\subsection[Chapter II --- Procedure in connection with determination of appeals and referrals]{Chapter II\\*Procedure in connection with determination of appeals and referrals}

\renewcommand\parthead{--- Part V Chapter II}

\amendment{
Reg. 38 omitted (3.11.08) by the Tribunals, Courts and Enforcement Act 2007 (Transitional and Consequential Provisions) Order 2008 Sch. 1 para. 124.
}

%\subsubsection[38. Consideration and determination of appeals and referrals]{Consideration and determination of appeals and referrals}
%
%38.—(1) The procedure in connection with the consideration and determination of an appeal or a referral shall, subject to the following provisions of these Regulations, be such as a legally qualified panel member shall determine.
%
%(2) A legally qualified panel member may give directions requiring a party to the proceedings to comply with any provision of these Regulations and may at any stage of the proceedings, either of his own motion or on a written application made to the clerk to the appeal tribunal by any party to the proceedings, give such directions as he may consider necessary or desirable for the just, effective and efficient conduct of the proceedings and may direct any party to the proceedings to provide such particulars or to produce such documents as may be reasonably required.
%
%(3) Where a clerk to the appeal tribunal is authorised to take steps in relation to the procedure of the tribunal he may give directions requiring any party to the proceedings to comply with any provision of these Regulations.

% Reg 38A inserted (5.7.99) by SI 1999/1670 reg 2(4)
\subsubsection[38A. Appeals raising issues for decision by officers of Inland Revenue]{Appeals raising issues for decision by officers of Inland Revenue}

38A.---(1)  Where
%, on consideration of any appeal, it appears to an appeal tribunal 
a person has appealed to 
%an appeal tribunal and it appears to the appeal tribunal, or a legally qualified panel member,  % Words substituted (20.5.02) by SI 2002/1379 reg 12(a)
the First-tier Tribunal and it appears to the First-tier Tribunal  % Words substituted (3.11.08) by SI 2008/2683 Sch 1 para 125(a)(i)
that an issue arises which, by virtue of section 8 of the Transfer Act, falls to be decided by an officer of the Board, that tribunal 
%or legally qualified panel member, as the case may be,  % Words inserted (20.5.02) by SI 2002/1379 reg 12(b), omitted (3.11.08) by SI 2008/2683 Sch 1 para 125(a)(ii)
shall—
\begin{enumerate}\item[]
($a$) refer the appeal to the Secretary of State pending the decision of that issue by an officer of the Board; and

($b$) require the Secretary of State to refer that issue to the Board;
\end{enumerate}
and the Secretary of State shall refer that issue accordingly.

(2) Pending the final decision of any issue which has been referred to the Board in accordance with paragraph (1) above, the Secretary of State may revise the decision under appeal, or make a further decision superseding that decision, in accordance with his determination of any issue other than one which has been so referred.

(3) On receipt by the Secretary of State of the final decision of an issue which has been referred in accordance with paragraph (1) above, he shall consider whether the decision under appeal ought to be revised under section 9 or superseded under section 10, and—
\begin{enumerate}\item[]
($a$) if so, revise it or, as the case may be, make a further decision which supersedes it; or

($b$) if not, forward the appeal to the 
%appeal tribunal 
First-tier Tribunal  % Words substituted (3.11.08) by SI 2008/2683 Sch 1 para 125(b)
which shall determine the appeal in accordance with the final decision of the issue so referred.
\end{enumerate}

(4) In paragraphs (2) and (3) above, “final decision” has the same meaning as in regulation 11A(3) and (4).

\amendment{
Reg. 38A inserted (5.7.99) by the Social Security and Child Support (Decisions and Appeals) Amendment (No. 3) Regulations 1999 reg. 2(4).

Words inserted and substituted in reg. 38A(1) (20.5.02) by the Social Security and Child Support (Decisions and Appeals) (Miscellaneous Amendments) Regulations 2002 reg. 12.

Words substituted in reg. 38A(1), (3)(b) and words omitted in reg. 38A(1) (3.11.08) by the Tribunals, Courts and Enforcement Act 2007 (Transitional and Consequential Provisions) Order 2008 Sch. 1 para. 125.

\medskip

Regs. 39--47 omitted (3.11.08) by the Tribunals, Courts and Enforcement Act 2007 (Transitional and Consequential Provisions) Order 2008 Sch. 1 para. 126.

\medskip

Reg. 48 omitted (21.12.04) by the Social Security, Child Support and Tax Credits (Decisions and Appeals) Amendment Regulations 2004 reg. 2(8).

\medskip

Regs. 49--52 omitted (3.11.08) by the Tribunals, Courts and Enforcement Act 2007 (Transitional and Consequential Provisions) Order 2008 Sch. 1 para. 126.
}

% Regs 39-47 omitted (3.11.08) by SI 2008/2683 Sch 1 para 126
%\subsubsection[39. Directions concerning oral hearings]{Directions concerning oral hearings}
%
%39.—(1) Where an appeal or a referral is made to an appeal tribunal, the clerk to the appeal tribunal shall direct the appellant and any other party to the proceedings to notify the clerk to the appeal tribunal in writing whether he wishes to have an oral hearing of the appeal or whether he is content for the appeal or referral to proceed without an oral hearing.
%
%(2) Except in the case of a referral, a direction under paragraph (1) shall include a statement informing the appellant that, if he does not respond in writing to the direction within the period specified in paragraph (3), the appeal may be struck out in accordance with regulation 46.
%
%(3) A notification given in accordance with paragraph (1) must be received by the clerk to the appeal tribunal within 14 days of the date of issue of the direction of the clerk to the appeal tribunal under paragraph (1) or within such longer period as the clerk to the appeal tribunal may direct.
%
%(4) Where a party to the proceedings notifies the clerk to the appeal tribunal in accordance with paragraph (3) that he wishes to have an oral hearing of the appeal or referral, the appeal tribunal shall hold an oral hearing.

% Reg 39(1)--(4) and heading substituted (21.12.04) by SI 2004/3368 reg 2(5)
%\subsubsection[39. Choice of hearing]{Choice of hearing}
%
%39.---(1)  Where an appeal or a referral is made to an appeal tribunal the appellant and any other party to the proceedings shall notify the clerk to the appeal tribunal, on a form approved by the Secretary of State, whether he wishes to have an oral hearing of the appeal or whether he is content for the appeal or referral to proceed without an oral hearing.
%
%(2) Except in the case of a referral, the form shall include a statement informing the appellant that, if he does not notify the clerk to the appeal tribunal as required by paragraph (1) within the period specified in paragraph (3), the appeal may be struck out in accordance with regulation 46(1).
%
%(3) Notification in accordance with paragraph (1)—
%\begin{enumerate}\item[]
%($a$) if given by the appellant or a party to the proceedings other than the Secretary of State, must be sent or given to the clerk to the appeal tribunal within 14 days of the date on which the form is issued to him; or
%
%($b$) if given by the Secretary of State, must be sent or given to the clerk—
%\begin{enumerate}\item[]
%(i) in the case of an appeal, within 14 days of the date on which the form is issued to the appellant; or
%
%(ii) in the case of a referral, on the date of referral,
%\end{enumerate}
%or within such longer period as the clerk may direct.
%\end{enumerate}
%
%(4) Where an oral hearing is requested in accordance with paragraphs (1) and (3) the appeal tribunal shall hold an oral hearing unless the appeal is struck out under regulation 46(1).
%
%(5) The chairman, or in the case of an appeal tribunal which has only one member, that member, may of his own motion direct that an oral hearing of the appeal or referral be held if he is satisfied that such a hearing is necessary to enable the appeal tribunal to reach a decision.
%
%\amendment{
%Reg. 39(1)--(4) and heading substituted (21.12.04) by the Social Security, Child Support and Tax Credits (Decisions and Appeals) Amendment Regulations 2004 reg. 2(5).
%}
%
%\subsubsection[40. Withdrawal of appeal or referral]{Withdrawal of appeal or referral}
%
%40.—(1) An appeal may be withdrawn by the appellant or an authorised representative of the appellant and a referral may be withdrawn by the Secretary of State, 
%the Board or an officer of the Board,   % Words inserted (5.10.99) by SI 1999/2570 reg 24(a)
%as the case may be, either—
%\begin{enumerate}\item[]
%($a$) at an oral hearing; or
%
%($b$) at any other time before the appeal or referral is determined, by giving notice in writing of withdrawal to the clerk to the appeal tribunal.
%\end{enumerate}
%
%(2) If an appeal or a referral is withdrawn (as the case may be) in accordance with paragraph (1)($a$), the clerk to the appeal tribunal shall send a notice in writing to any party to the proceedings who is not present when the appeal or referral is withdrawn, informing him that the appeal or referral (as the case may be) has been withdrawn.
%
%(3) If an appeal or a referral is withdrawn (as the case may be) in accordance with paragraph (1)($b$), the clerk to the appeal tribunal shall send a notice in writing to every party to the proceedings informing them that the appeal or referral (as the case may be) has been withdrawn.
%
%\amendment{
%Words inserted in reg. 40(1) (5.10.99) by the Tax Credits (Decisions and Appeals) (Amendment) Regulations 1999 reg. 25.
%}
%
%\subsubsection[41. Medical examination required by appeal tribunal]
%{Medical examination required by appeal tribunal}
%
%41.  For the purposes of section 20(2) (medical examination required by appeal tribunal) the prescribed condition which must be satisfied is that the issue, or one of the issues, raised on the appeal—
%\begin{enumerate}\item[]
%($a$) is whether the claimant satisfies the conditions for entitlement to—
%\begin{enumerate}\item[]
%(i) the care component of a disability living allowance specified in section 72(1) and (2) of the Contributions and Benefits Act;
%
%(ii) the mobility component of a disability living allowance specified in section 73(1), (8) and (9) of that Act;
%
%(iii) an attendance allowance specified in section 64 and 65(1) of that Act;
%
%(iv) a 
%%disability working allowance 
%disabled person's tax credit  % Words substituted (5.10.99) by SI 1999/2570 reg 26
%specified in section 129(1)($b$) of that Act;
%
%% Reg 41(a)(v) omitted (5.7.99) by SI 1999/1670 reg 2(5)(a)
%%(v) incapacity benefit under section 30A of that Act; or
%
%(vi) severe disablement allowance under section 68 of that Act;
%\end{enumerate}
%
%($b$) relates to the period throughout which the claimant is likely to satisfy the conditions for entitlement to an attendance allowance or a disability living allowance;
%
%($c$) is the rate at which an attendance allowance is payable;
%
%($d$) is the rate at which the care component or the mobility component of a disability living allowance is payable;
%
%% Reg 41(dd) inserted (5.7.99) by SI 1999/1670 reg 2(5)(b)
%($dd$) is whether a person is incapable of work for the purposes of the Contributions and Benefits Act;
%
%% Reg 41(e) omitted (5.7.99) by SI 1999/1670 reg 2(5)(c)
%%($e$) relates to either statutory sick pay or statutory maternity pay and the appeal is made by the employer concerned;
%
%($f$) relates to the extent of a person’s disablement and its assessment in accordance with Schedule 6 to the Contributions and Benefits Act;
%
%($g$) is whether the claimant suffers a loss of physical or mental faculty as a result of the relevant accident for the purposes of section 103 of the Contributions and Benefits Act;
%
%($h$) relates to any disease or injury prescribed for the purposes of section 108 of the Contributions and Benefits Act; or
%
%($i$) relates to any payment arising under, or by virtue of a scheme having effect under, section 111 of, and Schedule 8 to, the Contributions and Benefits Act (workmen’s compensation).
%\end{enumerate}
%
%\amendment{
%Reg. 41(dd) inserted and reg. 41(a)(v), (e) omitted (5.7.99) by the Social Security and Child Support (Decisions and Appeals) Amendment (No. 3) Regulations 1999 reg. 2(5).
%
%Words substituted in reg. 41(a)(iv) (5.10.99) by the Tax Credits (Decisions and Appeals) (Amendment) Regulations 1999 reg. 26.
%}
%
%\subsubsection[42. Non-disclosure of medical advice or evidence]{Non-disclosure of medical advice or evidence}
%
%42.—(1) Where, in connection with 
%%the consideration and determination of  % Words omitted (19.6.00) by SI 2000/1596 reg 25(a)(i)
%an appeal or referral there is 
%%before an appeal tribunal  % Words omitted (19.6.00) by SI 2000/1596 reg 25(a)(i)
%medical advice or medical evidence relating to a person which has not been disclosed to him and in the opinion of 
%%the chairman, or in the case of an appeal tribunal which has only one member, in the opinion of that member, 
%a legally qualified panel member  % Words substituted (19.6.00) by SI 2000/1596 reg 25(a)(ii)
%the disclosure to that person of that advice or evidence would be harmful to his health, such advice or evidence shall not be required to be disclosed to that person.
%
%(2) Advice or evidence such as is mentioned in paragraph (1) shall not be disclosed to any person acting for or representing the person to whom it relates or, in a case where a claim for benefit is made by reference to the disability of a person other than the claimant and the advice or evidence relates to that other person, shall not be disclosed to the claimant or any person acting for or representing him, unless 
%%the chairman, or in the case of an appeal tribunal which has only one member, that member, 
%a legally qualified panel member  % Words substituted (19.6.00) by SI 2000/1596 reg 25(b)
%is satisfied that it is in the interests of the person to whom the advice or evidence relates to do so.
%
%(3) A tribunal shall not be precluded from taking into account for the purposes of the determination advice or evidence which has not been disclosed to a person under the provisions of paragraph (1) or (2).
%
%\amendment{
%Words substituted in reg. 42(1), (2) and words omitted in reg. 42(1) (19.6.00) by the Social Security and Child Support (Miscellaneous Amendments) Regulations 2000 reg. 25.
%}
%
%\subsubsection[43. Summoning of witnesses and administration of oaths]{Summoning of witnesses and administration of oaths}
%
%43.—(1) A chairman, or in the case of an appeal tribunal which has only one member, that member, may by summons, or in Scotland, by citation, require any person in Great Britain to attend as a witness at a hearing of an appeal, application or referral at such time and place as shall be specified in the summons or citation and, subject to paragraph (2), at the hearing to answer any question or produce any documents in his custody or under his control which relate to any matter in question in the appeal, application or referral but—
%\begin{enumerate}\item[]
%($a$) no person shall be required to attend in obedience to such summons or citation unless he has been given at least 14 days' notice of the hearing or, if less than 14 days' notice is given, he has informed the tribunal that the notice given is sufficient; and
%
%($b$) no person shall be required to attend and give evidence or to produce any document in obedience to such summons or citation unless the necessary expenses of attendance are paid or tendered to him.
%\end{enumerate}
%
%(2) No person shall be compelled to give any evidence or produce any document or other material that he could not be compelled to give or produce on a trial of an action in a court of law in that part of Great Britain where the hearing takes place.
%
%(3) In exercising the powers conferred by this regulation, the chairman, or in the case of an appeal tribunal which has only one member, that member, shall take into account the need to protect any matter that relates to intimate personal or financial circumstances, is commercially sensitive, consists of information communicated or obtained in confidence or concerns national security.
%
%(4) Every summons or citation issued under this regulation shall contain a statement to the effect that the person in question may apply in writing to a chairman to vary or set aside the summons or citation.
%
%(5) A chairman, or in the case of an appeal tribunal which has only one member, that member, may require any witness, including a witness summoned under the powers conferred by this regulation, to give evidence on oath or affirmation and for that purpose there may be administered an oath or affirmation in due form.
%
%\subsubsection[44. Confidentiality in child support appeals or referrals]{Confidentiality in child support appeals or referrals}
%
%44.—(1) In the circumstances specified in paragraph (2), for the purposes of paragraph 7 of Schedule 1 to the Act (President to secure confidentiality), in a child support appeal or referral, the prescribed material is—
%\begin{enumerate}\item[]
%($a$) the address of the 
%\opt{oldrules}{absent parent}%
%\opt{newrules,2012rules}{non-resident parent}%  % Words substituted for new-rules cases only by SI 2001/158 reg 4(2)
%; the parent with care; the child; a parent of the child or any other person with care of the child; or
%
%($b$) any information the use of which could reasonably be expected to lead to the location of any person specified in paragraph ($a$).
%\end{enumerate}
%
%(2) Except where the appeal is brought against a reduced benefit 
%\opt{oldrules}{direction}%
%\opt{newrules,2012rules}{decision}  % Words substituted for new-rules cases only by SI 2001/158 reg 4(5)
%within the meaning of 
%\opt{oldrules}{section 46(11)}%
%\opt{newrules,2012rules}{section 46(10)($b$)}  % Words substituted for new-rules cases only by SI 2001/158 reg 4(5)
%of the Child Support Act\opt{oldrules}{\footnote{\frenchspacing Section 46(11) is amended by paragraph 43 of Schedule 7 to the Social Security Act 1998.}}, paragraph (1) applies where in response to an enquiry from the Secretary of State, the 
%\opt{oldrules}{absent parent}%
%\opt{newrules,2012rules}{non-resident parent}  % Words substituted for new-rules cases only by SI 2001/158 reg 4(2)
%or, as the case may be, the parent with care, has within 14 days of issue of that enquiry notified the Secretary of State that he would like the information specified in paragraph (1) which relates to him to remain confidential.
%
%(3) In this regulation, the expressions “%
%\opt{oldrules}{absent parent}%
%\opt{newrules,2012rules}{non-resident parent}%  % Words substituted for new-rules cases only by SI 2001/158 reg 4(2)
%” and “parent with care” have the meanings those expressions bear in section 54 of the Child Support Act.
%
%\opt{newrules,2012rules}{
%\amendment{
%Words substituted in reg. 44(1)(a), (2), (3) for new-rules cases only by the Child Support (Consequential Amendments and Transitional Provisions) Regulations 2001 reg. 4(2), (5).
%}
%}
%
%
%\opt{oldrules}{
%\subsubsection[45. Consideration of more than one appeal under section 20 of the Child Support Act]{Consideration of more than one appeal under section 20 of the Child Support Act}
%
%45.  An appeal tribunal which is considering an appeal under section 20 of the Child Support Act in respect of a departure direction\footnote{\frenchspacing Section 20 of the Child Support Act 1991 as extended by Schedule 4C to that Act applies to an appeal against a departure direction by virtue of section 28H of the Act as substituted by paragraph 39 of Schedule 7 to the Social Security Act 1998.} which relates to a maintenance assessment may, if it considers it appropriate to do so, consider at the same time any appeal under that section in respect of another departure direction which relates to the same maintenance assessment.
%}
%
%% Reg 45 substituted (3.3.03 for new rules-cases only) by SI 2000/3185 reg 13
%\opt{newrules,2012rules}{
%\subsubsection[45. Procedure following a referral under section 28D(1)($b$)  of the Child Support Act]{Procedure following a referral under section 28D(1)($b$)  of the Child Support Act}
%
%45.---(1)  On a referral under section 28D(1)($b$)  of the Child Support Act an appeal tribunal may—
%\begin{enumerate}\item[]
%($a$) consider two or more applications for a variation with respect to the same application for a maintenance calculation together; or
%
%($b$) consider two or more applications for a variation with respect to the same maintenance calculation together.
%\end{enumerate}
%
%(2) In this regulation “maintenance calculation” means a decision under section 11 or 17 of the Child Support Act, as calculated in accordance with Part I of Schedule 1 to that Act, whether as originally made or as revised under section 16 of that Act.
%
%\amendment{
%Reg. 45 substituted (3.3.03 for new-rules cases only) by the Child Support (Decisions and Appeals) (Amendment) Regulations 2000 reg. 13 (subject to reg. 1(2) and saving provisions in reg. 15).
%}
%}
%
%\subsection[Chapter III --- Striking out appeals]{Chapter III\\*Striking out appeals}
%
%\subsubsection[46. Appeals which may be struck out]{Appeals which may be struck out}
%
%\renewcommand\parthead{--- Part V Chapter III}
%
%46.—(1) Subject to paragraphs (2) and (3), an appeal may be struck out by the clerk to the appeal tribunal—
%\begin{enumerate}\item[]
%($a$) where it is an out of jurisdiction appeal and the appellant has been notified by the Secretary of State that an appeal brought against such a decision may be struck out;
%
%($b$) for want of prosecution including an appeal not made within the time specified in these Regulations; 
%%or  % Word omitted (21.12.04) by SI 2004/3368 reg 2(6)(a)(i)
%
%($c$) 
%%subject to regulation 39(4),  % Words omitted (21.12.04) by SI 2004/3368 reg 2(6)(a)(ii)
%for failure of the appellant to comply with a direction given under these Regulations where the appellant has been notified that failure to comply with the direction could result in the appeal being struck out%
%; or  % Words added (21.12.04) by SI 2004/3368 reg 2(6)(a)(ii)
%
%% Reg 46(1)(d) added (21.12.04) by SI 2004/3368 reg 2(6)(a)(iii)
%($d$) for failure of the appellant to notify the clerk to the appeal tribunal, in accordance with regulation 39, whether or not he wishes to have an oral hearing of his appeal.
%\end{enumerate}
%
%(2) Where the clerk to the appeal tribunal determines to strike out the appeal, he shall notify the appellant that his appeal has been struck out and of the procedure for reinstatement of the appeal as specified in regulation 47.
%
%(3) The clerk to the appeal tribunal may refer any matter for determination under this regulation to a legally qualified panel member for decision by the panel member rather than the clerk to the appeal tribunal.
%
%% Reg 46(4) omitted (21.12.04) by SI 2004/3368 reg 2(6)(b)
%%(4) Subject to regulation 48, a misconceived appeal may be struck out by a legally qualified panel member but such an appeal shall not be struck out unless the appellant has been given notice of—
%%\begin{enumerate}\item[]
%%($a$) the intention to strike out the appeal,
%%
%%($b$) the ground on which the intention to strike out is based, and
%%
%%($c$) the requirement to notify the clerk to the appeal tribunal in writing of the matters specified in regulation 48(1)($a$) or ($b$) and that failure to comply with this requirement may result in the appeal being struck out.
%%\end{enumerate}
%
%\amendment{
%Words omitted in reg. 46(1)(c), reg. 46(1)(d) added and reg. 46(4) omitted (21.12.04) by the Social Security, Child Support and Tax Credits (Decisions and Appeals) Amendment Regulations 2004 reg. 2(6).
%}
%
%\subsubsection[47. Reinstatement of struck out appeals]{Reinstatement of struck out appeals}
%
%47.---%
%% Reg 47(1) inserted (20.5.02) by SI 2002/1379 reg 13
%(1) The clerk to the appeal tribunal may reinstate an appeal which has been struck out in accordance with regulation 
%%46(1)($c$)  
%46(1)($d$)  % Reference substituted (21.12.0
%where—
%\begin{enumerate}\item[]
%($a$) the appellant has made representations to him or, as the case may be, further representations in support of his appeal with reasons why he considers that his appeal should not have been struck out;
%
%($b$) the representations are made in writing within one month of the order to strike out the appeal being issued; and
%
%($c$) the clerk is satisfied in the light of those representations that there are reasonable grounds for reinstating the appeal
%\end{enumerate}
%but if the clerk is not satisfied that there are reasonable grounds for reinstatement a legally qualified panel member shall consider whether the appeal should be reinstated in accordance with paragraph (2).
%
%(2) % Reg 47 renumbered as reg 47(2) (20.5.02) by SI 2002/1379 reg 13
%A legally qualified panel member may reinstate an appeal which has been struck out in accordance with regulation 46 
%%or regulation 48(2) 
%%or 48  % Words substituted (19.6.00) by SI 2000/1596 reg 26, omitted (21.12.04) by SI 2004/3368 reg 2(7)(b)(i)
%where—
%\begin{enumerate}\item[]
%($a$) the appellant has made representations, or as the case may be, further representations in support of his appeal with reasons why he considers that his appeal should not have been struck out, to the clerk to the appeal tribunal, in writing within one month of the order to strike out the appeal being issued, and the panel member is satisfied in the light of those representations that there are reasonable grounds for reinstating the appeal;
%
%% Reg 47(2)(b) omitted (21.12.04) by SI 2004/3368 reg 2(7)(b)(ii)
%%($b$) the panel member is satisfied that the appellant did not receive the notification required under regulation 46(4);
%
%($c$) the panel member is satisfied that the appeal is not an appeal which may be struck out under regulation 46; or
%
%($d$) the panel member is satisfied that notwithstanding that the appeal is one which may be struck out under regulation 46, it is not in the interests of justice for the appeal to be struck out.
%\end{enumerate}
%
%\amendment{
%Words substituted in reg. 47 (19.6.00) by the Social Security and Child Support (Miscellaneous Amendments) Regulations 2000 reg. 26.
%
%Reg. 47 renumbered as reg. 47(2) and reg. 47(1) inserted (20.5.02) by the Social Security and Child Support (Decisions and Appeals) (Miscellaneous Amendments) Regulations 2002 reg. 13.
%
%Words substituted in reg. 47(1), words omitted in reg. 47(2) and reg. 47(2)(b) omitted (21.12.04) by the Social Security, Child Support and Tax Credits (Decisions and Appeals) Amendment Regulations 2004 reg. 2(7).
%
%\medskip
%
%Reg. 48 omitted (21.12.04) by the Social Security, Child Support and Tax Credits (Decisions and Appeals) Amendment Regulations 2004 reg. 2(8).
%}

% Reg 48 omitted (21.12.04) by SI 2004/3368 reg 2(8)
%\subsubsection[48. Misconceived appeals]{Misconceived appeals}
%
%48.—(1) Where the appellant has been given notice under regulation 46(4) of intention to strike out an appeal on the ground that it is a misconceived appeal that person must within 14 days of the issue of such notice notify the clerk to the appeal tribunal in writing that—
%\begin{enumerate}\item[]
%($a$) he wishes the question of whether his appeal is misconceived to be determined by an appeal tribunal as a preliminary issue at an oral hearing, or
%
%($b$) he is content for an appeal tribunal to consider the question of whether his appeal is misconceived as a preliminary issue without an oral hearing and make representations in writing to the clerk to the appeal tribunal as to why he considers that the appeal is not misconceived.
%\end{enumerate}
%
%(2) Where the appellant fails to notify or to make representations to the clerk to the appeal tribunal in writing as required in paragraph (1) within the period specified in that paragraph, a legally qualified panel member may strike out the appeal.
%
%(3) Where the appellant notifies the clerk to the appeal tribunal under paragraph (1) within the period specified in that paragraph that he wishes an appeal tribunal to determine the question of whether his appeal is misconceived as a preliminary issue at an oral hearing, the appeal tribunal shall hold an oral hearing for that preliminary issue.
%
%(4) Where the appeal tribunal determine as a preliminary issue that the appeal is a misconceived appeal, the appeal shall be struck out and the clerk to the appeal tribunal shall notify the appellant that the appeal is struck out.
%
%(5) Where the appeal tribunal determine as a preliminary issue that the appeal is not a misconceived appeal—
%\begin{enumerate}\item[]
%($a$) the appeal tribunal shall refer the appeal and all the supporting documentation to the Secretary of State together with a statement of the reasons why the appeal tribunal considers that the appeal is not misconceived;
%
%($b$) the clerk to the appeal tribunal shall notify the appellant of the referral of the appeal to the Secretary of State and send the appellant a copy of the reasons why the appeal tribunal considers that the appeal is not misconceived;
%
%($c$) the Secretary of State may revise or supersede the decision against which the appeal is brought; and
%
%($d$) if the Secretary of State does not revise or supersede the decision against the appeal is brought in the appellant’s favour, the Secretary of State shall refer the appeal for determination by an appeal tribunal.
%\end{enumerate}
%
%(6) Chapter IV of this Part shall apply to an oral hearing held under this regulation.

% Regs 49-58 omitted (3.11.08) by SI 2008/2683 Sch 1 para 126
%
%\subsection[Chapter IV --- Oral hearings]{Chapter IV\\*Oral hearings}
%
%\renewcommand\parthead{--- Part V Chapter IV}
%
%\subsubsection[49. Procedure at oral hearings]{Procedure at oral hearings}
%
%49.—(1) Subject to the following provisions of this Part, the procedure for an oral hearing shall be such as the chairman, or in the case of an appeal tribunal which has only one member, such as that member, shall determine.
%
%(2) Except where paragraph (3) applies, not less than 14 days notice (beginning with the day on which the notice is given and ending on the day before the hearing of the appeal is to take place) of the time and place of any oral hearing of an appeal shall be given to every party to the proceedings, and if such notice has not been given to a person to whom it should have been given under the provisions of this paragraph the hearing may proceed only with the consent of that person.
%
%(3) Any party to the proceedings may waive his right to receive not less than 14 days notice of the time and place of any oral hearing by giving notice to the clerk to the appeal tribunal.
%
%(4) If a party to the proceedings to whom notice has been given under paragraph (2) fails to appear at the hearing the chairman, or in the case of an appeal tribunal which has only one member, that member, may, having regard to all the circumstances including any explanation offered for the absence, proceed with the hearing notwithstanding his absence, or give such directions with a view to the determination of the appeal as he may think proper.
%
%(5) If a party to the proceedings has waived his right to be given notice under paragraph (2) the chairman, or in the case of an appeal tribunal which has only one member, that member, may proceed with the hearing notwithstanding his absence.
%
%%(6) Any oral hearing shall be in public except—
%%\begin{enumerate}\item[]
%%($a$) where the appellant requests a private hearing, or
%%
%%($b$) where the chairman, or in the case of an appeal tribunal which has only one member, that member, is satisfied that intimate personal or financial circumstances may have to be disclosed or that considerations of national security are involved, in which case the hearing shall be in private.
%%\end{enumerate}
%
%% Reg 49(6) substituted (20.5.02) by SI 2002/1379 reg 14(a)
%(6) An oral hearing shall be in public except where the chairman, or in the case of an appeal tribunal which has only one member, that member, is satisfied that it is necessary to hold the hearing, or part of the hearing, in private—
%\begin{enumerate}\item[]
%($a$) in the interests of national security, morals, public order or children;
%
%($b$) for the protection of the private or family life of one or more parties to the proceedings; or
%
%($c$) in special circumstances, because publicity would prejudice the interests of justice.
%\end{enumerate}
%
%%(7) Any party to the proceedings shall be entitled to be present and be heard at an oral hearing.
%
%% Reg 49(7) substituted (20.5.02) by SI 2002/1379 reg 14(b)
%(7) At an oral hearing—
%\begin{enumerate}\item[]
%($a$) any party to the proceedings shall be entitled to be present and be heard; and
%
%($b$) the following persons may be present by means of a live television link—
%\begin{enumerate}\item[]
%(i) a party to the proceedings or his representative or both; or
%
%(ii) where an appeal tribunal consists of more than one member, a tribunal member other than the chairman,
%\end{enumerate}
%provided that the person who constitutes or is the chairman of the tribunal gives permission%
%% and the appellant consents  % Words omitted (18.3.05) by SI 2005/337 reg 2(9)
%.
%\end{enumerate}
%
%(8) A person who has the right to be heard at a hearing may be accompanied and may be represented by another person whether having professional qualifications or not and, for the purposes of the proceedings at the hearing, any such representative shall have all the rights and powers to which the person whom he represents is entitled.
%
%(9) The following persons shall also be entitled to be present at an oral hearing (whether or not it is otherwise in private) but shall take no part in the proceedings—
%\begin{enumerate}\item[]
%($a$) the President;
%
%($b$) any person undergoing training as a chairman or 
%%panel   % Word omitted (20.5.02) by SI 2002/1379 reg 14(c)(i)
%member of an appeal tribunal or as a clerk to an appeal tribunal;
%
%($c$) any person acting on behalf of the President in the training or supervision of panel members or in the monitoring of standards of decision-making by panel members;
%
%($d$) with the leave of the chairman, or in the case of an appeal tribunal which has only one member, with the leave of that member, 
%%and the consent of every party to the proceedings actually present,  % Words omitted (20.5.02) by SI 2002/1379 reg 14(c)(ii)
%any other person; and
%
%($e$) a member of the Council on Tribunals or of the Scottish Committee of the Council on Tribunals.
%\end{enumerate}
%
%%(10) Nothing in paragraph (9) affects the rights of any person mentioned in sub-paragraphs ($a$) and ($b$) of that paragraph at any oral hearing where he is sitting as a member of the tribunal or acting as its clerk, and nothing in this regulation prevents the presence at an oral hearing of any witness
%%or of any person whom the chairman, or in the case of an appeal tribunal which has only one member, that member, permits to be present in order to assist the clerk.  % Words added (19.6.00) by SI 2000/1596 reg 27
%
%% Reg 49(10) substituted (20.5.02) by SI 2002/1379 reg 14(d)
%(10) Nothing in paragraph (9) affects the rights of—
%\begin{enumerate}\item[]
%($a$) any person mentioned in sub-paragraphs ($a$)  and ($b$)  of that paragraph where he is sitting as a member of a tribunal or acting as its clerk; or
%
%($b$) the clerk to the tribunal,
%\end{enumerate}
%and nothing in this regulation prevents the presence at an oral hearing of any witness or of any person whom the chairman, or in the case of an appeal tribunal which has only one member, that member, permits to be present in order to assist the appeal tribunal or the clerk.
%
%(11) Any person entitled to be heard at an oral hearing may address the tribunal, may give evidence, may call witnesses and may put questions directly to any other person called as a witness.
%
%(12) For the purpose of arriving at its decision an appeal tribunal shall, and for the purpose of discussing any question of procedure may, notwithstanding anything contained in these Regulations, order all persons not being members of the tribunal, other than the person acting as clerk to the appeal tribunal, to withdraw from the hearing except that—
%\begin{enumerate}\item[]
%($a$) a member of the Council on Tribunals or of the Scottish Committee of the Council on Tribunals, the President or any person mentioned in paragraph (9)($c$); and
%
%($b$) with the leave of the chairman, or in the case of an appeal tribunal which has only one member, with the leave of that member, any person mentioned in paragraph (9)($b$) or ($d$),
%\end{enumerate}
%may remain present at any such sitting.
%
%% Reg 49(13) added (20.5.02) by SI 2002/1379 reg 14(e)
%(13) In this regulation “live television link” means a live television link or other facilities which allow a person who is not physically present at an oral hearing to see and hear proceedings and be seen and heard by those physically present.
%
%\amendment{
%Words added to reg. 49(10) (19.6.00) by the Social Security and Child Support (Miscellaneous Amendments) Regulations 2000 reg. 27.
%
%Words omitted in reg. 49(9)(b), (d), reg. 49(13) added and reg. 49(6), (7), (10) substituted (20.5.02) by the Social Security and Child Support (Decisions and Appeals) (Miscellaneous Amendments) Regulations 2002 reg. 14.
%
%Words omitted in reg. 49(7)(b) (18.3.05) by the Social Security, Child Support and Tax Credits (Miscellaneous Amendments) Regulations 2005 reg. 2(9).
%}
%
%\subsubsection[50. Manner of providing expert assistance]{Manner of providing expert assistance}
%
%50.—(1) Where an appeal tribunal require one or more experts to provide assistance to it in dealing with a question of fact of special difficulty under section 7(4), such an expert shall, if the chairman, or in the case of a tribunal with only one member, that member, so requests, attend at the hearing and give evidence and if the chairman or member sitting alone considers it appropriate, the expert shall enquire into and provide a written report on the question.
%
%(2) A copy of any written report received from an expert in accordance with paragraph (1) shall be supplied to every party to the proceedings.
%
%\subsubsection[51. Postponement and adjournment]{Postponement and adjournment}
%
%51.—(1) Where a person to whom notice of an oral hearing is given wishes to request a postponement of that hearing he shall do so in writing to the clerk to the appeal tribunal stating his reasons for the request, and the clerk to the appeal tribunal may grant or refuse the request as he thinks fit or may pass the request to a legally qualified panel member who may grant or refuse the request as he thinks fit.
%
%(2) Where the clerk to the appeal tribunal or the panel member, as the case may be, refuses a request to postpone the hearing he shall—
%\begin{enumerate}\item[]
%($a$) notify in writing the person making the request of the refusal; and
%
%($b$) place before the appeal tribunal at the hearing both the request for the postponement and notification of its refusal.
%\end{enumerate}
%
%(3) A panel member or the clerk to the appeal tribunal may of his own motion at any time before the beginning of the hearing postpone the hearing.
%
%(4) An oral hearing may be adjourned by the appeal tribunal at any time on the application of any party to the proceedings or of its own motion.
%
%% Reg 51(5) omitted (20.5.02) by SI 2002/1379 reg 15
%%(5) Where a hearing has been adjourned and it is not practicable, or would cause undue delay, for it to be resumed before a tribunal consisting of the same member or members, the appeal or referral shall be heard by a differently constituted tribunal and the proceedings shall be by way of a complete rehearing.
%
%\amendment{
%Reg. 51(5) omitted (20.5.02) by the Social Security and Child Support (Decisions and Appeals) (Miscellaneous Amendments) Regulations 2002 reg. 15.
%}
%
%\subsubsection[52. Physical examinations at oral hearings]{Physical examinations at oral hearings}
%
%52.  For the purposes of section 20(3) an appeal tribunal may not carry out a physical examination except in a case which relates to—
%\begin{enumerate}\item[]
%($a$) the extent of a person’s disablement and its assessment in accordance with section 68(6) of, and Schedule 6 to, the Contributions and Benefits Act;
%
%($b$) the extent of a person’s disablement and its assessment in accordance with section 103 of that Act;
%
%($c$) diseases or injuries prescribed for the purposes of section 108 of that Act.
%\end{enumerate}

\subsection[Chapter V --- Decisions of appeal tribunals and related matters]{Chapter V\\*Decisions of appeal tribunals and related matters}

\renewcommand\parthead{--- Part V Chapter V}

\amendment{
Regs. 53--57 omitted (3.11.08) by the Tribunals, Courts and Enforcement Act 2007 (Transitional and Consequential Provisions) Order 2008 Sch. 1 para. 126.
}

%\subsection{\itshape Appeal tribunal decisions}
%
%\subsubsection[53. Decisions of appeal tribunals]{Decisions of appeal tribunals}
%
%53.—(1) Every decision of an appeal tribunal shall be recorded in summary by the chairman, or in the case of an appeal tribunal which has only one member, by that member.
%
%(2) The decision notice specified in paragraph (1) shall be in such written form as shall have been approved by the President and shall be signed by the chairman, or in the case of an appeal tribunal which has only one member, by that member.
%
%(3) As soon as may be practicable after an appeal or referral has been decided by an appeal tribunal, a copy of the decision notice 
%%prepared in accordance with paragraph (1) and (2)  % Words omitted (18.3.05) by SI 2005/337 reg 2(10)(a)
%shall be sent or given to every party to the proceedings who shall also be informed of—
%\begin{enumerate}\item[]
%($a$) his right under paragraph (4); and
%
%%($b$) the conditions governing appeals to a Commissioner.
%
%% Reg 53(3)(b) substituted (18.10.99) by SI 1999/2677 reg 10
%($b$) except in the case of an appeal under the Vaccine Damage Payments Act, the conditions governing appeals to a Commissioner.
%\end{enumerate}
%
%%(4) A party to the proceedings may apply in writing to the chairman, or in the case of a tribunal with only one member, to that member, for 
%%%a copy of  % Words omitted (19.6.00) by SI 2000/1596 reg 28(a)
%%a statement of the reasons for the tribunal’s decision within one month of the sending or giving of the decision notice to every party to the proceedings or within such longer period as may be allowed in accordance with regulation 54
%%and following that application the chairman or, as the case may be, that member shall record a statement of the reasons and a copy of that statement shall be sent or given to every party to the proceedings as soon as may be practicable.  % Words added (19.6.00) by SI 2000/1596 reg 28(b)
%
%% Reg 53(4) substituted (20.5.02) by SI 2002/1379 reg 16
%(4) 
%Subject to paragraph (4A),  % Words inserted (18.3.05) by SI 2005/337 reg 2(10)(b)
%a party to the proceedings may apply in writing to the clerk to the appeal tribunal for a statement of the reasons for the tribunal’s decision within one month of the sending or giving of the decision notice to every party to the proceedings or within such longer period as may be allowed in accordance with regulation 54 and following that application the chairman, or in the case of a tribunal with only one member, that member shall record a statement of the reasons and a copy of that statement shall be given to every party to the proceedings as soon as may be practicable.
%
%% Reg 53(4A) inserted (18.3.05) by SI 2005/337 reg 2(10)(c)
%(4A) Where—
%\begin{enumerate}\item[]
%($a$) the decision notice is corrected in accordance with regulation 56; or
%
%($b$) an application under regulation 57 for the decision to be set aside is refused for reasons other than a refusal to extend the time for making the application,
%\end{enumerate}
%the period specified in paragraph (4) shall run from the date on which notice of the correction or the refusal of the application for setting aside is sent to the applicant.
%
%(5) If the decision is not unanimous, the decision notice specified in paragraph (1) shall record that one of the members dissented and the statement of reasons referred to in paragraph (4) shall include the reasons given by the dissenting member for dissenting.
%
%\amendment{
%Reg. 53(3)(b) substituted (18.10.99) by the Social Security and Child Support (Decisions and Appeals), Vaccine Damage Payments and Jobseeker's Allowance (Amendment) Regulations 1999 reg. 10.
%
%Words inserted in and added to reg. 53(4) (19.6.00) by the Social Security and Child Support (Miscellaneous Amendments) Regulations 2000 reg. 28.
%
%Reg. 53(4) substituted (20.5.02) by the Social Security and Child Support (Decisions and Appeals) (Miscellaneous Amendments) Regulations 2002 reg. 16.
%
%Words inserted in reg. 53(4), words omitted in reg. 53(3) and reg. 53(4A) inserted (18.3.05) by the Social Security, Child Support and Tax Credits (Miscellaneous Amendments) Regulations 2005 reg. 2(10).
%}
%
%\subsubsection[54. Late applications for a statement of reasons of tribunal decision]{Late applications for a statement of reasons of tribunal decision}
%
%54.—(1) The time for making an application for 
%%a copy of  % Words omitted (19.6.00) by SI 2000/1596 reg 29(a)
%the statement of the reasons for a tribunal’s decision may be extended where the conditions specified in paragraphs (2) to (8) are satisfied, but% 
%, subject to 
%  %paragraph (13)
%  regulation 53(4A)%  % Words substituted (18.3.05) by SI 2005/337 reg 2(11)(a)
%,  % Words inserted (19.6.00) by SI 2000/1596 reg 29(a)
%no application shall in any event be brought more than three months after the date of the sending or giving of the notice of the decision of the appeal tribunal.
%
%(2) An application for an extension of time under this regulation shall be made in writing and shall be determined by a legally qualified panel member.
%
%(3) An application under this regulation shall contain particulars of the grounds on which the extension of time is sought, including details of any relevant special circumstances for the purposes of paragraph (4).
%
%(4) The application for an extension of time shall not be granted unless the panel member is satisfied that it is in the interests of justice for the application to be granted.
%
%(5) For the purposes of paragraph (4) it is not in the interests of justice to grant the application unless the panel member is satisfied that—
%\begin{enumerate}\item[]
%($a$) the special circumstances specified in paragraph (6) are relevant to the application; or
%
%($b$) some other special circumstances are relevant to the application,
%\end{enumerate}
%and as a result of those special circumstances it was not practicable for the application to be made within the time limit specified in regulation 53(4).
%
%(6) For the purposes of paragraph (5)($a$), the special circumstances are that—
%\begin{enumerate}\item[]
%($a$) the applicant or a 
%%spouse 
%partner  % Word substituted (20.5.02) by SI 2002/1379 reg 17(a)
%or dependant of the applicant has died or suffered serious illness;
%
%($b$) the applicant is not resident in the United Kingdom; or
%
%($c$) normal postal services were adversely disrupted.
%\end{enumerate}
%
%(7) In determining whether it is in the interests of justice to grant the application, the panel member shall have regard to the principle that the greater the amount of time that has elapsed between the expiration of the time within which the application for a copy of the statement of reasons for a tribunal’s decision is to be made and the making of the application for an extension of time, the more compelling should be the special circumstances on which the application is based.
%
%(8) In determining whether it is in the interests of justice to grant the application, no account shall be taken of the following—
%\begin{enumerate}\item[]
%($a$) that the person making the application or any person acting for him was unaware of, or misunderstood, the law applicable to his case (including ignorance or misunderstanding of the time limits imposed by these Regulations); or
%
%($b$) that a Commissioner or a court has taken a different view of the law from that previously understood and applied.
%\end{enumerate}
%
%(9) An application under this regulation for an extension of time which has been refused may not be renewed.
%
%(10) The panel member who determines the application shall record a summary of his 
%%decision 
%determination  % Word substituted (20.5.02) by SI 2002/1370 reg 17(b)
%in such written form as has been approved by the President.
%
%(11) As soon as practicable after the 
%%decision 
%determination  % Word substituted (20.5.02) by SI 2002/1370 reg 17(b)
%is made 
%%a copy 
%notice  % Words substituted (20.5.02) by SI 2002/1370 reg 17(c)
%of the 
%%decision 
%determination  % Word substituted (20.5.02) by SI 2002/1370 reg 17(b)
%shall be sent or given to every party to the proceedings.
%
%(12) Any person who under paragraph (11) receives 
%%a copy 
%notice  % Words substituted (20.5.02) by SI 2002/1370 reg 17(d)
%of the 
%%decision 
%determination  % Word substituted (20.5.02) by SI 2002/1370 reg 17(b)
%may, within one month of the 
%%decision 
%determination  % Word substituted (20.5.02) by SI 2002/1370 reg 17(b)
%being sent to him, apply in writing for a copy of the reasons for that 
%%decision 
%determination  % Word substituted (20.5.02) by SI 2002/1370 reg 17(b)
%and a copy shall be supplied to him.
%
%% Reg 54(13) added (19.6.00) by SI 2000/1596 reg 29(b)
%%(13) In calculating the time specified for applying in writing for a statement of the reasons for the tribunal’s decision there shall be disregarded any day which falls before the day on which notice was given of—
%%\begin{enumerate}\item[]
%%($a$) a correction of a decision or the record thereof pursuant to regulation 56; or
%%
%%($b$) a determination that a decision shall not be set aside following an application made under regulation 57.
%%\end{enumerate}
%
%% Reg 54(13) substituted (20.5.02) by SI 2002/1379 reg 17(e), omitted (18.3.05) by SI 2005/337 reg 2(11)(b)
%%(13) In calculating the time specified for applying in writing for a statement of the reasons for the tribunal’s decision there shall be disregarded any day which falls before the day on which notice was given of—
%%\begin{enumerate}\item[]
%%($a$) a correction of a decision or the record thereof pursuant to regulation 56; or
%%
%%($b$) a determination that a decision shall not be set aside following an application made under regulation 57, except where the decision was not set aside because of a refusal to extend the time for applying.
%%\end{enumerate}
%
%\amendment{
%Words inserted and omitted in reg. 54(1) and reg. 54(13) added (19.6.00) by the Social Security and Child Support (Miscellaneous Amendments) Regulations 2000 reg. 29.
%
%Words substituted in reg. 54(6)(a), (10), (11), (12) and reg. 54(13) substituted (20.5.02) by the Social Security and Child Support (Decisions and Appeals) (Miscellaneous Amendments) Regulations 2002 reg. 17.
%
%Words substituted in reg. 54(1) and reg. 54(13) omitted (18.3.05) by the Social Security, Child Support and Tax Credits (Miscellaneous Amendments) Regulations 2005 reg. 2(11).
%}
%
%\subsubsection[55. Record of tribunal proceedings]{Record of tribunal proceedings}
%
%55.—(1) A record of the proceedings at an oral hearing, which is sufficient to indicate the evidence taken, shall be made by the chairman, or in the case of an appeal tribunal which has only one member, by that member, in such medium as he may direct.
%
%%(2) Such record shall be preserved by the clerk to the appeal tribunal for six months from the date of the decision made by the appeal tribunal to which the record relates and any party to the proceedings may within that period apply in writing for a copy of that record and a copy shall be supplied to him.
%
%% Reg 55(2)--(4) substituted for reg. 55(2) (18.3.05) by SI 2005/337 reg 2(12)
%(2) The clerk to the appeal tribunal shall preserve—
%\begin{enumerate}\item[]
%($a$) the record of proceedings;
%
%($b$) the decision notice; and
%
%($c$) any statement of the reasons for the tribunal’s decision,
%\end{enumerate}
%for the period specified in paragraph (3).
%
%(3) That period is six months from the date of—
%\begin{enumerate}\item[]
%($a$) the decision made by the appeal tribunal;
%
%($b$) any statement of reasons for the tribunal’s decision;
%
%($c$) any correction of the decision in accordance with regulation 56;
%
%($d$) any refusal to set aside the decision in accordance with regulation 57; or
%
%($e$) any determination of an application under regulation 58 for leave to appeal against the decision,
%\end{enumerate}
%or until the date on which those documents are sent to the office of the Social Security and Child Support Commissioners in connection with an appeal against the decision or an application to a Commissioner for leave to appeal, if that occurs within the six months.
%
%(4) Any party to the proceedings may within the time specified in paragraph (3) apply in writing for a copy of the record of proceedings and a copy shall be supplied to him.
%
%\amendment{
%Reg. 55(2)--(4) substituted for reg. 55(2) (18.3.05) by the Social Security, Child Support and Tax Credits (Miscellaneous Amendments) Regulations 2005 reg. 2(12).
%}
%
%\subsubsection[56. Correction of accidental errors]{Correction of accidental errors}
%
%56.—(1) The clerk to the appeal tribunal
%%, or where the clerk refers the matter to a legally qualified panel member, that member, 
%or a legally qualified panel member  % Words substituted (19.6.00) by SI 2000/1596 reg 30
%may at any time correct accidental errors in 
%%any decision, or the record of any such decision, 
%the notice of any decision  % Words substituted (18.3.05) by SI 2005/337 reg 2(13)(a)
%of an appeal tribunal made under a relevant enactment, the Child Support Act or the Vaccine Damage Payments Act.
%
%%(2) A correction made to, or to the record of, a decision shall be deemed to be part of the decision or record of that decision and written notice of it shall be given as soon as practicable to every party to the proceedings.
%
%% Reg 56(2) substituted (18.3.05) by SI 2005/337 reg 2(13)(b)
%(2) A correction made to a decision notice shall be deemed to be part of the decision notice and written notice of the correction shall be given as soon as practicable to every party to the proceedings.
%
%(3) In this regulation and regulation 57, “relevant enactment” has the same meaning as in section 28(3).
%
%\amendment{
%Words substituted in reg. 56(1) (19.6.00) by the Social Security and Child Support (Miscellaneous Amendments) Regulations 2000 reg. 30.
%
%Words substiuted in reg. 56(1) and reg. 56(2) substituted (18.3.05) by the Social Security, Child Support and Tax Credits (Miscellaneous Amendments) Regulations 2005 reg. 2(13).
%}
%
%\subsubsection[57. Setting aside decisions on certain grounds]{Setting aside decisions on certain grounds}
%
%57.—(1) On an application made by a party to the proceedings, a decision of an appeal tribunal made under a relevant enactment, the Child Support Act or the Vaccine Damage Payments Act, may be set aside by a legally qualified panel member in a case where it appears just to set the decision aside on the ground that—
%\begin{enumerate}\item[]
%($a$) a document relating to the proceedings in which the decision was made was not sent to, or was not received at an appropriate time by, a party to the proceedings or the party’s representative or was not received at an appropriate time by the person who made the decision;
%
%($b$) a party to the proceedings in which the decision was made or the party’s representative was not present at a hearing relating to the proceedings.
%\end{enumerate}
%
%(2) In determining whether it is just to set aside a decision on the ground set out in paragraph (1)($b$), the panel member shall determine whether the party making the application gave notice that he wished to have an oral hearing, and if that party did not give such notice the decision shall not be set aside unless 
%%the chairman, or in the case of an appeal tribunal which has only one member, unless  % Words omitted (20.5.02) by SI 2002/1379 reg 18(a)
%that member is satisfied that the interests of justice manifestly so require.
%
%%(3) An application under this regulation shall be made in accordance with regulations 31 to 33.
%
%% Reg 57(3) substituted (19.6.00) by SI 2000/1596 reg 31(a)
%%(3) An application under this regulation shall—
%%\begin{enumerate}\item[]
%%($a$) be made within one month of the date on which—
%%\begin{enumerate}\item[]
%%(i) a copy of the decision notice is sent or given to the parties to the proceedings in accordance with regulation 53(3); or
%%
%%(ii) the statement of the reasons for the decision is given or sent in accordance with regulation 53(4),
%%\end{enumerate}
%%whichever is the later;
%%
%%($b$) be in writing and signed by a party to the proceedings or, where the party has provided written authority to a representative to act on his behalf, that representative;
%%
%%($c$) contain particulars of the grounds on which it is made; and
%%
%%($d$) be sent to the clerk to the appeal tribunal.
%%\end{enumerate}
%
%% Reg 57(3) substituted (20.5.02) by SI 2002/1379 reg 18(b)
%(3) An application under this regulation shall—
%\begin{enumerate}\item[]
%($a$) be made within one month of the date on which—
%\begin{enumerate}\item[]
%(i) a copy of the decision notice is sent or given to the parties to the proceedings in accordance with regulation 53(3); or
%
%(ii) the statement of the reasons for the decision is given or sent in accordance with regulation 53(4),
%\end{enumerate}
%whichever is later;
%
%($b$) be in writing and signed by a party to the proceedings or, where the party has provided written authority to a representative to act on his behalf, that representative;
%
%($c$) contain particulars of the grounds on which it is made; and
%
%($d$) be sent to the clerk to the appeal tribunal.
%\end{enumerate}
%
%(4) Where an application to set aside a decision is entertained under paragraph (1), every party to the proceedings shall be sent a copy of the application and shall be afforded a reasonable opportunity of making representations on it before the application is determined.
%
%% Reg 57(4A) inserted (18.3.05) by SI 2005/337 reg 2(14)
%(4A) Where a legally qualified panel member refuses to set aside a decision he may treat the application to set aside the decision as an application under regulation 53(4) for a statement of the reasons for the tribunal’s decision, subject to the time limits set out in regulation 53(4) and (4A).
%
%(5) Notice in writing of a determination on an application to set aside a decision shall be sent or given to every party to the proceedings as soon as may be practicable and the notice shall contain a statement giving the reasons for the determination.
%
%% Reg 57(6)--(12) added (19.6.00) by SI 2000/1596 reg 31(b)
%%(6) The time within which an application under this regulation must be made may be extended by a period not exceeding one year where the conditions specified in paragraphs (7) to (11) are satisfied.
%%
%%(7) An application for an extension of time shall be made in accordance with paragraph (3)($b$)  to ($d$), shall include details of any relevant special circumstances for the purposes of paragraph (9) and shall be determined by a legally qualified panel member.
%%
%%(8) An application for an extension of time shall not be granted unless the panel member is satisfied that—
%%\begin{enumerate}\item[]
%%($a$) if the application is granted there are reasonable prospects that the application to set aside will be successful; and
%%
%%($b$) it is in the interests of justice for the application for an extension of time to be granted.
%%\end{enumerate}
%%
%%(9) For the purposes of paragraph (8) it is not in the interests of justice to grant an application for an extension of time unless the panel member is satisfied that—
%%\begin{enumerate}\item[]
%%($a$) the special circumstances specified in paragraph (10) are relevant to that application; or
%%
%%($b$) some other special circumstances exist which are wholly exceptional and relevant to that application,
%%\end{enumerate}
%%and as a result of those special circumstances, it was not practicable for the application to set aside to be made within the time limit specified in paragraph (3)($a$).
%%
%%(10) For the purposes of paragraph (9)($a$)  the special circumstances are that—
%%\begin{enumerate}\item[]
%%($a$) the applicant or a spouse or dependant of the applicant has died or suffered serious illness;
%%
%%($b$) the applicant is not resident in the United Kingdom; or
%%
%%($c$) normal postal services were disrupted.
%%\end{enumerate}
%%
%%(11) In determining whether it is in the interests of justice to grant an application for an extension of time, the panel member shall have regard to the principle that the greater the amount of time that has elapsed between the expiry of the time within which the application to set aside is to be made and the making of the application for an extension of time, the more compelling should be the special circumstances on which the application for an extension is based.
%%
%%(12) An application under this regulation for an extension of time which has been refused may not be renewed.
%
%% Reg 57(6)--(12) substituted (20.5.02) by SI 2002/1379 reg 18(c)
%(6) The time within which an application under this regulation must be made may be extended by a period not exceeding one year where the conditions specified in paragraphs (7) to (11) are satisfied.
%
%(7) An application for an extension of time shall be made in accordance with paragraph (3)($b$)  to ($d$), shall include details of any relevant special circumstances for the purposes of paragraph (9) and shall be determined by a legally qualified panel member.
%
%(8) An application for an extension of time shall not be granted unless the panel member is satisfied that—
%\begin{enumerate}\item[]
%\begin{sloppypar}
%($a$) if the application is granted there are reasonable prospects that the application to set aside will be successful; and
%\end{sloppypar}
%
%($b$) it is in the interests of justice for the application for an extension of time to be granted.
%\end{enumerate}
%
%(9) For the purposes of paragraph (8) it is not in the interests of justice to grant an application for an extension of time unless the panel member is satisfied that—
%\begin{enumerate}\item[]
%($a$) the special circumstances specified in paragraph (10) are relevant to that application; or
%
%($b$) some other special circumstances exist which are wholly exceptional and relevant to that application,
%\end{enumerate}
%and as a result of those special circumstances, it was not practicable for the application to set aside to be made within the time limit specified in paragraph (3)($a$).
%
%(10) For the purposes of paragraph (9)($a$)  the special circumstances are that—
%\begin{enumerate}\item[]
%($a$) the applicant or a partner or dependant of the applicant has died or suffered serious illness;
%
%($b$) the applicant is not resident in the United Kingdom; or
%
%($c$) normal postal services were disrupted.
%\end{enumerate}
%
%(11) In determining whether it is in the interests of justice to grant an application for an extension of time, the panel member shall have regard to the principle that the greater the amount of time that has elapsed between the expiry of the time within which the application to set aside is to be made and the making of the application for an extension of time, the more compelling should be the special circumstances on which the application for an extension is based.
%
%(12) An application under this regulation for an extension of time which has been refused may not be renewed.
%
%\amendment{
%Reg. 57(6)--(12) added and reg. 57(3) substituted (19.6.00) by the Social Security and Child Support (Miscellaneous Amendments) Regulations 2000 reg. 31.
%
%Words omitted in reg. 57(2) and reg. 57(3), (6)--(12) substituted (20.5.02) by the Social Security and Child Support (Decisions and Appeals) (Miscellaneous Amendments) Regulations 2002 reg. 18.
%
%Reg. 57(4A) inserted (18.3.05) by the Social Security, Child Support and Tax Credits (Miscellaneous Amendments) Regulations 2005 reg. 2(14).
%}
%
%% Regs 57A, 57B inserted (19.6.00) by SI 2000/1596 reg 32
%%\subsubsection[57A. Provisions common to regulations 56 and 57]{Provisions common to regulations 56 and 57}
%%
%%57A.---(1)  In calculating any time specified for appealing to a Commissioner from a decision of an appeal tribunal there shall be disregarded any day falling before the day on which notice was given of a correction of a decision or the record thereof pursuant to regulation 56 or on which notice is given of a determination that a decision shall not be set aside following an application made under regulation 57, as the case may be.
%%
%%(2) There shall be no appeal against a correction made under regulation 56 or a refusal to make such a correction or against a determination given under regulation 57.
%%
%%(3) Nothing in this Chapter shall be construed as derogating from any power to correct errors or set aside decisions which is exercisable apart from these Regulations.
%
%% Reg 57A substituted (20.5.02) by SI 2002/1379 reg 19
%\subsubsection[57A. Provisions common to regulations 56 and 57]{Provisions common to regulations 56 and 57}
%
%57A.---%
%% Reg 57A(1) omitted (18.3.05) by SI 2005/337 reg 2(15)
%%(1)  In calculating any time specified for appealing to a Commissioner from a decision of an appeal tribunal there shall be disregarded any day falling before the day on which notice was given of—
%%\begin{enumerate}\item[]
%%($a$) a correction of a decision or the record thereof pursuant to regulation 56; or
%%
%%($b$) a determination that a decision shall not be set aside following an application made under regulation 57, except where the decision was not set aside because of a refusal to extend the time for applying.
%%\end{enumerate}
%%
%(2) There shall be no appeal against a correction made under regulation 56 or a refusal to make such a correction or against a determination made under regulation 57.
%
%(3) Nothing in this Chapter shall be construed as derogating from any power to correct errors or set aside decisions which is exercisable apart from these Regulations.
%
%\amendment{
%Reg. 57A inserted (19.6.00) by the Social Security and Child Support (Miscellaneous Amendments) Regulations 2000 reg. 32.
%
%Reg. 57A substituted (20.5.02) by the Social Security and Child Support (Decisions and Appeals) (Miscellaneous Amendments) Regulations 2002 reg. 19.
%
%Reg. 57A(1) omitted (18.3.05) by the Social Security, Child Support and Tax Credits (Miscellaneous Amendments) Regulations 2005 reg. 2(15).
%}
%
%% Reg 57AA inserted (18.3.05) by SI 2005/337 reg 2(16)
%\subsubsection[57AA. Service of decision notice by electronic mail]{Service of decision notice by electronic mail}
%
%57AA.  For the purposes of the time limits in regulations 53 to 57, a properly addressed copy of a decision notice sent by electronic mail is effective from the date it is sent.
%
%\amendment{
%Reg. 57AA inserted (18.3.05) by the Social Security, Child Support and Tax Credits (Miscellaneous Amendments) Regulations 2005 reg. 2(16).
%}
%
%%\subsubsection[57B. Interpretation of Chapter V]{Interpretation of Chapter V}
%%
%%57B.  In Chapter V, except in regulation 58, “Commissioner” includes Child Support Commissioner.
%
%% Reg 57B substituted (18.3.05) by SI 2005/337 reg 2(17)
%\subsubsection[57B. Interpretation of Chapter V]{Interpretation of Chapter V}
%
%57B.---(1)  In Chapter V, except in regulations 58 and 58A—
%\begin{enumerate}\item[]
%“Commissioner” includes Child Support Commissioner;
%
%“decision” includes a determination on a referral.
%\end{enumerate}
%
%(2) In Chapter V—
%\begin{enumerate}\item[]
%“decision notice” has the meaning given in regulation 53(1) and (2).
%\end{enumerate}
%
%\amendment{
%Reg. 57B inserted (19.6.00) by the Social Security and Child Support (Miscellaneous Amendments) Regulations 2000 reg. 32.
%
%Reg. 57B substituted (18.3.05) by the Social Security, Child Support and Tax Credits (Miscellaneous Amendments) Regulations 2005 reg. 2(17).
%}
%
%\addtocontents{toc}{\protect\pagebreak[3]}

\subsection{\itshape Applications for leave to appeal to a Commissioner (not including child support)}

\amendment{
Reg. 58 omitted (3.11.08) by the Tribunals, Courts and Enforcement Act 2007 (Transitional and Consequential Provisions) Order 2008 Sch. 1 para. 126.
}

%\subsubsection[58. Application for leave to appeal to a Commissioner from an appeal tribunal]{Application for leave to appeal to a Commissioner from an appeal tribunal}
%
%58.—(1) 
%Subject to paragraph (1A),  % Words inserted (18.3.05) by SI 2005/337 reg 2(18)(a)(i)
%an application for leave to appeal to a Commissioner from a decision of an appeal tribunal under 
%section 13 of the 1997 Act or under  % Words inserted (20.5.02) by SI 2002/1379 reg 20(a)(i)
%section 12 or 13 shall—
%\begin{enumerate}\item[]
%($a$) be 
%%made within the period of one month commencing on the date the applicant is sent 
%sent to the clerk to the appeal tribunal within the period of one month of the date of the applicant being sent  % Words substituted (20.5.02) by SI 2002/1379 reg 20(a)(ii)
%a written statement of the reasons for the decision against which leave to appeal is sought; and
%
%%($b$) have annexed to it a copy of that written statement of the reasons for the decision.
%
%% Reg 58(1)(b)--(e) substituted for reg 58(1)(b) (18.3.05) by SI 2005/337 reg 2(18)(a)(ii)
%($b$) be in writing and signed by the applicant or, where he has given written authority to a representative to make the application on his behalf, by that representative;
%
%($c$) contain particulars of the grounds on which the applicant intends to rely;
%
%($d$) contain sufficient particulars of the decision of the appeal tribunal to enable the decision to be identified; and
%
%($e$) if the application is made late, contain the grounds for seeking late acceptance.
%\end{enumerate}
%
%% Reg 58(1A) inserted (18.3.05) by SI 2005/337 reg 2(18)(b)
%(1A) Where after the written statement of the reasons for the decision has been sent to the parties to the proceedings—
%\begin{enumerate}\item[]
%($a$) the decision notice is corrected in accordance with regulation 56; or
%
%($b$) an application under regulation 57 for the decision to be set aside is refused for reasons other than a refusal to extend the time for making the application,
%\end{enumerate}
%the period specified in paragraph (1)($a$)  shall run from the date on which notice of the correction or the refusal of the application for setting aside is sent to the applicant.
%
%(2) Where an application for leave to appeal to a Commissioner is made by the Secretary of State
%or the Board,  % Words inserted (5.10.99) by SI 1999/2570 reg 27
%the clerk to an appeal tribunal shall, as soon as may be practicable, send a copy of the application to every other party to the proceedings.
%
%% Reg 58(3) omitted (20.5.02) by SI 2002/1379 reg 20(b)
%%(3) Any party to the proceedings who is sent a copy of an application for leave to appeal in accordance with paragraph (2) may make representations in writing within one month of the date the application is sent.
%
%%(4) A person determining an application for leave to appeal to a Commissioner, shall take into account any further representations received from the applicant before the determination is made, and shall record his decision in writing and send a copy to every party to the proceedings.
%
%% Reg 58(4) substituted (20.5.02) by SI 2002/1379 reg 20(c)
%(4) A person determining an application for leave to appeal to a Commissioner shall record his determination in writing and send a copy to every party to the proceedings.
%
%(5) Where there has been a failure to apply for leave to appeal within the period of time specified in paragraph (1)($a$) 
%or (1A)  % Words inserted (18.3.05) by SI 2005/337 reg 2(18)(c)
%but an application is made within one year of the last date for making an application within that period, a legally qualified panel member may, if for special reasons he thinks fit, accept and proceed to consider and determine the application.
%
%%(6) Where in any case it is impracticable, or it would be likely to cause undue delay for an application for leave to appeal against a decision of an appeal tribunal to be determined by the person who was the chairman, or in the case of an appeal tribunal which has only one member, the member, of that tribunal, 
%%or if the President considers it necessary or expedient for the purpose of supervising panel members or in the monitoring of decision-making by panel members,  % Words inserted (19.6.00) by SI 2000/1596 reg 33
%%the application shall be determined by a legally qualified panel member.
%
%% Reg 58(6) substituted (20.5.02) by SI 2002/1379 reg 20(d)
%(6) Where an application for leave to appeal against a decision of an appeal tribunal is made—
%\begin{enumerate}\item[]
%($a$) if the person who constituted, or was the chairman of, the appeal tribunal when the decision was given was a fee-paid legally qualified panel member, the application may be determined by a salaried legally qualified panel member; or
%
%($b$) if it is impracticable, or it would be likely to cause undue delay, for the application to be determined by whoever constituted, or was the chairman of, the appeal tribunal when the decision was given, the application may be determined by another legally qualified panel member.
%\end{enumerate}
%
%\amendment{
%Words inserted in reg. 58(2) (5.10.99) by the Tax Credits (Decisions and Appeals) (Amendment) Regulations 1999 reg. 27.
%
%Words inserted in reg. 58(6) (19.6.00) by the Social Security and Child Support (Miscellaneous Amendments) Regulations 2000 reg. 33.
%
%Words inserted in reg. 58(1), words substituted in reg. 58(1)(a), reg. 58(4), (6) substituted and reg. 58(3) omitted (20.5.02) by the Social Security and Child Support (Decisions and Appeals) (Miscellaneous Amendments) Regulations 2002 reg. 20.
%
%Words inserted in reg. 58(1), (5), reg. 58(1A) inserted and reg. 58(1)(b)--(e) substituted for reg. 58(1)(b) (18.3.05) by the Social Security, Child Support and Tax Credits (Miscellaneous Amendments) Regulations 2005 reg. 2(18).
%}

% Reg 58A inserted (18.3.05) by SI 2005/337 reg 2(19)
\subsubsection[58A. Appeal to 
%a Commissioner 
the Upper Tribunal  % Words substituted (3.11.08) by SI 2008/2683 Sch 1 para 127(a)
by a partner]{Appeal to 
%a Commissioner 
the Upper Tribunal  % Words substituted (3.11.08) by SI 2008/2683 Sch 1 para 127(a)
by a partner}

58A.  A partner within the meaning of section 2AA(7) of the Administration Act\footnote{Section 2AA was inserted by the Employment Act 2002 (c.\ 22), section 49.} (full entitlement to certain benefits conditional on work-focused interview for partner) may appeal to 
%a Commissioner 
the Upper Tribunal  % Words substituted (3.11.08) by SI 2008/2683 Sch 1 para 127(a)
under section 14 from a decision of 
%an appeal tribunal 
the First-tier Tribunal  % Words substituted (3.11.08) by SI 2008/2683 Sch 1 para 127(b)
in respect of a decision specified in section 2B(2A) and (6)\footnote{Section 2B was inserted by the Welfare Reform and Pensions Act 1999 (c.\ 30), section 57; subsection (2A) was inserted by the Employment Act 2002, Schedule 7, paragraph 9(4).} of the Administration Act.

\amendment{
Reg. 58A inserted (18.3.05) by the Social Security, Child Support and Tax Credits (Miscellaneous Amendments) Regulations 2005 reg. 2(19).

\looseness=-1
Words substituted in reg. 58A and heading (3.11.08) by the Tribunals, Courts and Enforcement Act 2007 (Transitional and Consequential Provisions) Order 2008 Sch. 1 para. 127.
}

\section[Part VI --- Revocations]{Part VI\\*Revocations}

\subsection[59. Revocations]{Revocations}

\renewcommand\parthead{--- Part VI}

59.—(1) The Regulations listed in column (2) of Schedule 4 are hereby revoked to the extent specified in column (3) of that Schedule.

(2) Notwithstanding their revocation for particular purposes, the Regulations listed in column (2) of Schedule 4 shall continue to have full effect up to and including 28th November 1999 in relation to any benefit to which these Regulations do not apply for the time being by virtue of regulation 1(2).

(3) So much of any document as refers expressly or by implication to any regulation revoked by paragraph (1) shall, in so far as the context permits, for the purposes of these Regulations be treated as referring to the corresponding provision of these Regulations.

\bigskip

Signed 
by authority of the Secretary of State for Social Security.

{\raggedleft
\emph{Angela Eagle}\\*Parliamentary Under-Secretary of State,\\*Department of Social Security

}

26th March 1999

\bigskip

I concur

{\raggedleft
\emph{Irvine of Lairg}\\*Lord Chancellor

}

26th March 1999

\small

\part[Schedule 1 --- Provisions conferring powers exercised in making these Regulations]{Schedule 1\\*Provisions conferring powers exercised in making these Regulations}

\renewcommand\parthead{--- Schedule 1}

{\footnotesize\hbadness=10000
%\begin{tabulary}{\linewidth}{JJJ}
\begin{longtable}{p{150pt}p{102pt}p{102pt}}
\hline
\itshape Column (1) & & \itshape Column (2)\\
\itshape Provision & & \itshape Relevant Amendments\\
\hline
\endhead
\hline
\endlastfoot
Vaccine Damage Payments Act 1979\footnote{\frenchspacing 1979 c. 17.}&Section 4(2) and (3)&The Act, Section 46.\\
&Section 7A(1)&The Act, Section 47.\\
Child Support Act 1991\footnote{\frenchspacing 1991 c. 48.}&Section 16(6)&The Act, Section 40.\\
&Section 20(5) and (6)&The Act, Section 42.\\
&Section 28ZA(2)($b$) and (4)($c$)&The Act, Section 43.\\
&Section 28ZB(6)($c$)&The Act, Section 43.\\
&Section 28ZC(7)&The Act, Section 44.\\
&Section 28ZD(1) and~ 2)&The Act, Section 44.\\
&Section 46B&The Act, Schedule 7, paragraph 44.\\
&Section 51(2)&The Act, Schedule 7, paragraph 46.\\
&Schedule 4A, paragraph 8&The Act, Schedule 7, paragraph 53.\\
Social Security Administration Act 1992\footnote{\frenchspacing 1992 c. 5.}&Section 5(1)($hh$)&The Act, Section 74.\\
&Section 159&The Act, Schedule 7, paragraph 95.\\
&Section 159A&The Act, Schedule 7, paragraph 96.\\
Pension Schemes Act 1993\footnote{\frenchspacing 1993 c. 48.}&Section 170(3)&The Act, Schedule 7, paragraph 131.\\
Social Security (Recovery of Benefits) Act 1997\footnote{\frenchspacing 1997 c. 27.}&Section 10&The Act, Schedule 7, paragraph 149.\\
&Section 11(5)\\
Social Security Act 1998\footnote{\frenchspacing 1998 c. 14.}&Section 6(3)\\
&Section 7(6)\\
&Section 9(1), (4) and (6)\\
&Section 10(3) and (6)\\
&Section 11(1)\\
&Section 12(2) and (3), (6) and (7)\\
&Section 14(10)($a$) and (11)\\
&Section 16(1) and Schedule 5\\
&Section 17\\
&Section 18(1)\\
&Section 20\\
&Section 21(1) to (3)\\
&Section 22\\
&Section 23\\
&Section 24\\
&Section 25(3)($b$) and (5)($c$)\\
&Section 26(6)($c$)\\
&Section 28(1)\\
&Section 31(2)\\
&Section 79(1) and (3) to (7)\\
&Section 84\\
&Schedule 1, paragraphs 7, 11 and 12\\
&Schedule 2, paragraph~9\\
&Schedule 3, paragraphs 1, 4 and 9\\
%\end{tabulary}
\end{longtable}

}

\vfill

\part[Schedule 2 --- Decisions against which no appeal lies]{Schedule 2\\*Decisions against which no appeal lies}

\renewcommand\parthead{--- Schedule 2}

\amendment{
Sch. 2 revoked (7.4.03) so far as relating to child benefit or guardian's allowance by the Child Benefit and Guardian’s Allowance (Decisions and Appeals) Regulations 2003 reg. 34(a).
}

\subsection*{\itshape Child Benefit}

1.  A decision of the Secretary of State as to whether an educational establishment be recognised for the purposes of Part IX of the Contributions and Benefits Act.

\medskip

2.  A decision of the Secretary of State to recognise education provided otherwise than at a recognised educational establishment.

\medskip

3.  A decision of the Secretary of State made in accordance with the discretion conferred upon him by the following provisions of the Child Benefit (Residence and Persons Abroad) Regulations 1976\footnote{\frenchspacing S.I. 1976/963; the relevant amending instrument is S.I. 1976/1758.}—
\begin{enumerate}\item[]
($a$) regulation 2(2)($c$)(iii) (decision relating to a child’s temporary absence abroad);

($b$) regulation 7(3) (certain days of absence abroad disregarded).
\end{enumerate}

\medskip

4.  A decision of the Secretary of State made in accordance with the discretion conferred upon him by regulation 2(1) or (3) of the Child Benefit (General) Regulations 1976\footnote{\frenchspacing S.I. 1976/965; the relevant amending instrument is S.I. 1976/1758.} (provisions relating to contributions and expenses in respect of a child).\looseness=-1

\subsection*{\itshape Claims and Payments}

%5.  A decision of the Secretary of State under the Claims and Payments Regulations except a decision under—
%\begin{enumerate}\item[]
%($a$) regulation 19 as to the time for claiming benefit;
%
%% Para 5(b) omitted (19.6.00) by SI 2000/1596 reg 34(a)
%%($b$) regulation 37AB\footnote{\frenchspacing Regulation 37AB was inserted by S.I. 1994/2319.} as to the payment of withheld benefit;
%
%($c$) regulation 38 as to the extinguishment of the right to payment of sums by way of benefit where payment is not obtained within the prescribed period; and
%
%($d$) the following provisions of Schedule 9 and regulation 35(1) in so far as it relates to them—
%\begin{enumerate}\item[]
%(i) paragraph 3 relating to the amount deductible by way of housing costs;
%
%(ii) paragraph 4 relating to the amount of miscellaneous housing costs payable direct to a third party;
%
%(iii) paragraph 4A\footnote{\frenchspacing Paragraph (4A) was inserted by S.I. 1991/2284 and amended by S.I. 1992/2595.} relating to the direct payment to a third party of benefit payable to or in respect of persons resident in hostels;
%
%(iv) paragraph 5 relating to payments of benefit direct to the claimant’s or his partner’s landlord;
%
%(v) paragraph 6 relating to the payment out of benefit of fuel costs;
%
%(vi) paragraph 7 relating to the payment out of benefit of water charges;
%
%(vii) paragraph 7A\footnote{\frenchspacing Paragraph 7A was inserted by S.I. 1993/478, substituted by S.I. 1993/2113 and then amended by S.I. 1996/481.} in connection with amounts payable in place of child support maintenance;
%
%(viii) paragraph 7B\footnote{\frenchspacing Paragraph 7B was inserted by S.I. 1996/2344.} as to whether an amount in respect of arrears of child support maintenance is to be deducted from a person’s jobseeker’s allowance; and
%
%(ix) paragraph 9(3) as to the priority between liabilities for items of gas and electricity.
%\end{enumerate}
%\end{enumerate}

% Para 5 substituted (20.5.02) by SI 2002/1379 reg 21(a)
5.  A decision, being a decision of the Secretary of State unless specified below as a decision of the Board, under the following provisions of the Claims and Payments Regulations—
\begin{enumerate}\item[]
%($a$) regulation 4\footnote{Regulation 4 was amended by S.I. 1991/2741, 1992/247, 1996/1460 and 2431, 1997/793, 1999/2572 and 3108 and 2000/636, 897 and 1982.} (decision of the Secretary of State or the Board as to making a claim for benefit);

% Para 5(a) substituted (21.12.04) by SI 2004/3368 reg 2(9)(a)
($a$) regulation 4(3) or (3B)\footnote{Paragraph (3) was amended by S.I. 1996/2431 and paragraph (3B) was inserted by S.I. 1996/1460.} (which partner should make a claim for income support or jobseeker’s allowance);

% Para 5(aa) inserted (27.7.08) by SI 2008/1554 reg 42(a)
($aa$) regulation 4I (which partner should make a claim for an employment and support allowance);

% Para 5(b) omitted (21.12.04) by SI 2004/3368 reg 2(9)(b)
%($b$) regulation 4A(3)\footnote{Regulation 4A was inserted by S.I. 1999/3108 and amended by S.I. 2000/897.} (sufficiency of a claim at office displaying the \textsc{\lowercase{ONE}} logo if claim not on approved form);

% Para 5(bb) inserted (18.6.03) by SI 2003/1581 reg 2(a)
%($bb$) regulation 4D (making a claim for state pension credit) or 4E\footnote{Regulations 4D and 4E were inserted by S.I. 2002/3019.} (making a claim before attaining the qualifying age);

% Para 5(bb) substituted (21.12.04) by SI 2004/3368 reg 2(9)(c)
($bb$) regulation 4D(7)\footnote{Regulation 4D was inserted by S.I. 2002/3019.} (which partner should make a claim for state pension credit);

% Para 5(c), (d), (e) omitted (21.12.04) by SI 2004/3368 reg 2(9)(b)
%($c$) regulation 6(4AA)\footnote{Paragraph (4AA) was substituted by S.I. 2000/1820.} (accepting properly completed claim for jobseeker’s allowance after claimant attends a place to make claim);
%
%($d$) regulation 6(4AB)\footnote{Paragraph (4AB) was substituted by S.I. 1997/793.} (accepting properly completed claim for jobseeker’s allowance up to one month after notification of intent to claim);
%
%($e$) regulation 6(8) and (9)\footnote{Paragraphs (8) and (9) were added by S.I. 1991/2741 and amended by S.I. 1993/2113 and 2319.} (specifying time for properly completing and submitting claim for disability living allowance or attendance allowance);

($f$) regulation 7\footnote{Regulation 7 was amended by S.I. 1995/2303, 1996/1460 and 1999/3108 and 2572.} (decision by the Secretary of State or the Board as to evidence and information required);

($g$) regulation 9\footnote{Regulation 9 was amended by S.I. 1992/247, 1996/1803 and 1999/2572.} and Schedule 1 (decision by the Secretary of State or the Board as to interchange of claims with claims for other benefits);

($h$) regulation 11\footnote{Regulation 11 was amended by S.I. 1994/2943 and 1997/793.} (treating claim for maternity allowance as claim for incapacity benefit
or employment and support allowance%  % Words inserted (27.7.08) by SI 2008/1554 reg 42(b)
);

($i$) regulation 15(7)\footnote{Reg.~15(7) was amended by S.I.~1989/1642.} (approving form of particulars required for determination of retirement pension questions in advance of claim);

($j$) regulations 20 to 24\footnote{Regs.~20 to 24 were amended by S.I.~1991/2741, 1992/247, 1993/1113, 1994/2319, 2943 and 3196, 1996/672, 1460 and 2306, 1999/2358 and 2572 and 2000/1982 and 3120.} (decision by the Secretary of State or the Board as to the time or manner of payments);

($k$) regulation 25(1)\footnote{Reg.~25(1) was amended by S.I.~1991/2741 and 1996/1436.} (intervals of payment of attendance allowance and disability living allowance where claimant is expected to return to hospital);

($l$) regulation 26\footnote{Reg.~26 was amended by S.I.~1988/522, 1989/136, 1993/1113, 1999/3178 and 2000/1596.} (manner and time of payment of income support);

($m$) regulation 26A\footnote{Reg.~26A was inserted by S.I.~1996/1460 and amended by S.I.~2000/1596.} (time and intervals of payment of jobseeker’s allowance);

% Para 5(mm) inserted (18.6.03) by SI 2003/1581 reg 2(b)
($mm$) regulation 26B\footnote{Reg.~26B was inserted by S.I.~2002/3019.} (payment of state pension credit);

% Para 5(mn) inserted (27.7.08) by SI 2008/1554 reg 42(c)
($mn$) regulation 26C (manner and time of payment of employment and support allowance);

($n$) regulation 27(1) and (1A)\footnote{Reg.~27 was substituted by S.I.~1991/2741 and amended by S.I.~1993/2113, 1994/3196 and 1999/2572.} (decision by the Board as to manner and time of payment of tax credits);

($o$) regulation 30\footnote{Reg.~30 was amended by S.I.~1988/1725, 1990/\hspace{0pt}2208, 1991/2741, 1993/2113, 1994/2319, 1996/1460, 1999/2572 and 2000/1982.} (decision by the Secretary of State or the Board as to claims or payments after death of claimant);

($p$) regulation 30A\footnote{Reg.~30A was inserted by S.I.~2001/518.} (payment of arrears of joint-claim jobseeker’s allowance where nominated person can no longer be traced);

($q$) regulation 31\footnote{Reg.~31 was amended by S.I.~1999/3108 and 3178.} (time and manner of payments of industrial injuries gratuities);%\looseness=-1

($r$) regulation 32\footnote{Reg.~32 was amended by S.I.~1992/2595, 1995/2303, 1996/1460 and 1999/2572.} (decision by the Secretary of State or the Board as to information to be given when obtaining payment of benefit);

($s$) regulation 33\footnote{Reg.~33 was amended by S.I.~1991/2741 and 1999/2572.} (appointments by the Secretary of State or the Board where person unable to act);

($t$) regulation 34\footnote{Reg.~34 was amended by S.I.~1992/2595, 1999/2572 and 2000/1982.} (decision by the Secretary of State or the Board as to paying another person on the beneficiary’s behalf);

($u$) regulation 34A(1)\footnote{Reg.~34A was inserted by S.I.~1992/1026.} (payment, out of benefit, of mortgage interest to qualifying lender);

($v$) regulation 35(2)\footnote{Reg.~35 was substituted by S.I.~1988/522 and para.~(2) was amended by S.I.~1988/1725.} (payment to third person of maternity expenses or expenses for heating in cold weather);

($w$) regulation 36\footnote{Reg.~36 was amended by S.I.~1999/2358 and 2572.} (decision by the Secretary of State or the Board to pay partner as alternative payee);

($x$) regulation 38\footnote{Reg.~38 was amended by S.I.~1989/1686, 1993/2113, 1996/672, 1999/1958, 2422, 2572 and 3178.} (decision by the Secretary of State or the Board as to the extinguishment of right to payment of sums by way of benefit where payment not obtained within the prescribed period, except a decision under paragraph (2A) (payment request after expiration of prescribed period));
%\looseness=1

($y$) regulations 42 to 46\footnote{Regs.~42 to 46 were amended by S.I. 1991/2741 and reg.~44 was amended by S.I.~1990/2208.} (mobility component of disability living allowance and disability living allowance for children);

($z$) regulation 47(2) and (3)\footnote{Reg.~47 was substituted by S.I.~1994/3196 and amended by S.I.~1999/2572.} (return of instruments of payment etc.\ to the Secretary of State or the Board).
\end{enumerate}

\amendment{
%Para. 5(b) omitted (19.6.00) by the Social Security and Child Support (Miscellaneous Amendments) Regulations 2000 reg. 34(a).

Para. 5 substituted (20.5.02) by the Social Security and Child Support (Decisions and Appeals) (Miscellaneous Amendments) Regulations 2002 reg. 21(a).

Para. 5(bb), (mm) inserted (18.6.03) by the State Pension Credit (Decisions and Appeals-Amendments) Regulations 2003 reg. 2.

Para. 5(a), (bb) substituted and para. 5(b), (c), (d), (e) omitted (21.12.04) by the Social Security, Child Support and Tax Credits (Decisions and Appeals) Amendment Regulations 2004 reg. 2(9).\looseness=-1

Words added to para. 5(h) and para. 5(aa), (mn) inserted (27.7.08) by the Employment and Support Allowance (Consequential Provisions) (No. 2) Regulations 2008 reg. 42.
}

\subsection*{\itshape [until 5th April 2016] Contracted Out Pension Schemes}

\subsection*{\itshape [from 6th April 2016] 
%Contracted Out Pension Schemes
Schemes that were Contracted-out Pension Schemes  % Heading substituted by SI 2016/200 art 15
}

6.  A decision of the Secretary of State under section 109 of the Pension Schemes Act 1993\footnote{\frenchspacing 1993 c. 48.} or any Order made under it (annual increase of guaranteed minimum pensions).\looseness=-1

\amendment{
Heading to para. 6 substituted (6.4.16) by the Pensions Act 2014 (Abolition of Contracting-out for Salary Related Pension Schemes) (Consequential Amendments and Savings) Order 2016 art. 15.
}

\subsection*{\itshape Decisions depending on other cases}

7.  A decision of the Secretary of State under section 25 or 26 (decisions and appeals depending on other cases).

\subsection*{\itshape Deductions}

8.  A decision which falls to be made by the Secretary of State under the Fines (Deductions from Income Support) Regulations 1992\footnote{\frenchspacing S.I. 1992/2182.}, other than 
%one falling within regulation 4 of those Regulations.
a decision whether benefit is sufficient for a deduction to be made.  % Words substituted (29.11.99) by SI 1999/3178 Sch 19 para 2

\amendment{
Words substituted in para. 8 (29.11.99) by the Social Security Act 1998 (Commencement No. 12 and Consequential and Transitional Provisions) Order 1999 Sch. 19 para. 2.
}

\medskip

9.—(1) Except in relation to a decision to which sub-paragraph (2) applies, any decision of the Secretary of State under the Community Charges (Deductions from Income Support) (No.\ 2) Regulations 1990\footnote{\frenchspacing S.I. 1990/545.}, the Community Charges (Deductions from Income Support) (Scotland) Regulations 1989\footnote{\frenchspacing S.I. 1989/507.} or the Council Tax (Deductions from Income Support) Regulations 1993\footnote{\frenchspacing S.I. 1993/494.}.%\looseness=-1

(2) This sub-paragraph applies to a decision—
\begin{enumerate}\item[]
($a$) whether there is an outstanding sum due of the amount sought to be deducted;

($b$) whether benefit is sufficient for a deduction to be made; and

($c$) on the priority to be given to any deduction.
\end{enumerate}

\subsection*{\itshape European Community Regulations}

10.  An authorization given by the Secretary of State in accordance with article 22(1) or 55(1) of Council Regulation \textsc{\lowercase{(EEC)}} No. 1408/71\footnote{\frenchspacing \emph{See} Council Regulation \textsc{\lowercase{(EEC)}} No. 1408/71, \textsc{\lowercase{O.J.}} No. \textsc{\lowercase{L149/2, 5.7.71 (O.J./S.E.~1971(II)}} p. 416).} on the application of social security schemes to employed persons, to self-employed persons and to members of their families moving within the Community.

\subsection*{\itshape Expenses}

11.  A decision of the Secretary of State whether to pay expenses to any person under section 180 of the Administration Act.

\subsection*{\itshape Guardian’s Allowance}

12.  A decision of the Secretary of State relating to the giving of a notice under regulation 5(8) of the Social Security (Guardian’s Allowance) Regulations 1975\footnote{\frenchspacing S.I. 1975/515.} (children whose surviving parents are in prison or legal custody).

\subsection*{\itshape Income Support}

13.  A decision of the Secretary of State 
%which embodies a determination  % Words omitted (19.6.00) by SI 2000/1596 reg 34(b)
made in accordance with paragraph~(1) or~(2) of regulation 13 (income support and social fund determinations on incomplete evidence).

\amendment{
Words omitted in para. 13 (19.6.00) by the Social Security and Child Support (Miscellaneous Amendments) Regulations 2000 reg. 34(b).
}

% Para 13A inserted (7.4.03) by SI 2002/3019 reg 21
\subsection*{\itshape State pension credit}

13A.  A decision of the Secretary of State made in accordance with paragraph (1) or~(3) of regulation 13 in relation to state pension credit (determination on incomplete evidence).

\amendment{
Para. 13A inserted (7.4.03) by the State Pension Credit (Consequential, Transitional and Miscellaneous Provisions) Regulations 2002 reg. 21.
}

\subsection*{\itshape Industrial Injuries Benefit}

14.  A decision of the Secretary of State relating to the question whether—
\begin{enumerate}\item[]
($a$) disablement pension be increased under section 104 of the Contributions and Benefits Act (constant attendance); or

($b$) disablement pension be further increased under section 105 of the Contributions and Benefits Act (exceptionally severe disablement);
\end{enumerate}
and if an increase is to be granted or renewed, the period for which and the amount at which it is payable.

\medskip

\looseness=-1
15.  A decision of the Secretary of State under regulation 2(2) of the Social Security (Industrial Injuries and Diseases) Miscellaneous Provisions Regulations 1986\footnote{\frenchspacing S.I. 1986/1561.} as to the length of any period of interruption of education which is to be disregarded.

\medskip

16.  A decision of the Secretary of State to approve or not to approve a person undertaking work for the purposes of regulation 17 of the Social Security (General Benefit) Regulations 1982\footnote{\frenchspacing S.I. 1982/1408; the relevant amending instruments are S.I. 1983/186 and S.I. 1986/1561.}.

\medskip

17.  A decision of the Secretary of State as to how the limitations under Part~VI of Schedule 7 to the Contributions and Benefits Act on the benefit payable in respect of any death are to be applied in the circumstances of any case.

\subsection*{\itshape Invalid Vehicle Scheme}

18.  A decision of the Secretary of State relating to the issue of certificates under regulation 13 of, and Schedule 2 to, the Social Security (Disability Living Allowance) Regulations 1991\footnote{\frenchspacing S.I. 1991/2890.}.

\subsection*{\itshape Jobseeker’s Allowance}

19.—(1) A decision of the Secretary of State under Chapter IV of Part II of the Jobseeker’s Allowance Regulations as to the day and the time a claimant is to attend at a job centre.

(2) A decision of the Secretary of State as to the day of the week on which a claimant is required to provide a signed declaration under regulation 24(10) of the Jobseeker’s Allowance Regulations.

(3) A decision of the Secretary of State 
%which embodies a determination  % Words omitted (19.6.00) by SI 2000/1596 reg 34(c)
made in accordance with regulation 15 (Jobseeker’s allowance determinations on incomplete evidence).

\amendment{
Words omitted in para. 19(3) (19.6.00) by the Social Security and Child Support (Miscellaneous Amendments) Regulations 2000 reg. 34(c).

\medskip
%}
%
%% Para 19A inserted (20.5.02) by SI 2002/1379 reg 21(b), omitted by SI 2010/424 reg 4(6)
%\subsection*{\itshape Loss of Benefit for Breach of Community Order}
%
%\looseness=-1
%19A.  A decision of the Secretary of State that a relevant benefit shall not be payable or shall be reduced in accordance with a determination of a court made under section 62(1) of the Child Support, Pensions and Social Security Act 2000\footnote{2000 c.\ 19.} where the only ground of appeal is that the court’s determination was made in error.
%
%\amendment{
Para. 19A inserted (20.5.02) by the Social Security and Child Support (Decisions and Appeals) (Miscellaneous Amendments) Regulations 2002 reg. 21(b).

Para. 19A omitted (22.3.10) by the Welfare Reform Act 2009 (Section 26) (Consequential Amendments) Regulations 2010 reg. 4(6).
}

\subsection*{\itshape Payments on Account, Overpayments and Recovery}

20.  A decision of the Secretary of State under the Social Security (Payments on account, Overpayments and Recovery) Regulations 1988\footnote{\frenchspacing S.I. 1988/664; the relevant amending instruments are S.I. 1988/668, 1991/2742, 1993/650 and 1996/1345.}, except a decision of the Secretary of State under the following provisions of those Regulations—
\begin{enumerate}\item[]
% Para 20(2)(a), (b) omitted by SI 2013/383 reg 20(1)(a)
%($a$) regulation 3(1)($a$) to offset any interim payment made in anticipation of an award of benefit;
%
%($b$) regulation 4(1) as to the overpayment of an interim payment;

($c$) regulation 5 as to the offsetting of a prior payment against a subsequent award;

($d$) regulation 11(1) as to whether a payment in excess of entitlement has been credited to a bank or other account;

($e$) regulation 13 as to the sums to be deducted in calculating recoverable amounts;

($f$) regulation 14(1) as to the treatment of capital to be reduced;

($g$) regulation 19 determining a claimant’s protected earnings; and

($h$) regulation 24 whether a determination as to a claimant’s protected earnings is revised or superseded.
\end{enumerate}

\amendment{
Para. 20(2)(a), (b) omitted (1.4.13) by the Social Security (Payments on Account of Benefit) Regulations 2013 reg. 20(1)(a) subject to a saving in reg. 20(2).
}

\medskip

% Para 20A inserted by SI 2013/383 reg 20(1)(b)
20A.  A decision of the Secretary of State under the Social Security (Payments on Account of Benefit) Regulations 2013 except a decision under regulation 10 of those Regulations.

\amendment{
Para. 20A inserted (1.4.13) by the Social Security (Payments on Account of Benefit) Regulations 2013 reg. 20(1)(b).
}

\subsection*{\itshape Persons Abroad}

21.  A decision of the Secretary of State made under—
\begin{enumerate}\item[]
($a$) regulation 2(1)($a$) of the Social Security Benefit (Persons Abroad) Regulations 1975\footnote{\frenchspacing S.I. 1975/563; the relevant amending instruments are S.I. 1976/409, 1977/342 and 1679, 1979/463 and 1432, 1981/1157, 1982/388 and 1738, 1983/186, 1984/1303, 1986/1545 and 1561, 1988/435, 1989/1642, 1990/40 and 621, 1991/2742, 1992/1700 and 2595, 1994/268 and 1832, 1995/829 and 1996/207 and 1345.} whether to certify that it is consistent with the proper administration of the Contributions and Benefits Act that a disqualification under section 113(1)($a$) of that Act should not apply;

\looseness=1
($b$) regulation 9(4) or (5) of those Regulations whether to allow a person to avoid disqualification for receiving benefit during a period of temporary absence from Great Britain longer than that specified in the regulation.
\end{enumerate}

\subsection*{\itshape Reciprocal Agreements}

22.  A decision of the Secretary of State made in accordance with an Order made under section 179 of the Administration Act (reciprocal agreements with countries outside the United Kingdom).

\subsection*{\itshape Social Fund Awards}

23.  A decision of the Secretary of State under section 78 of the Administration Act relating to the recovery of social fund awards.

\subsection*{\itshape Suspension}

24.  A decision of the Secretary of State relating to the suspension of a relevant benefit or to the payment of such a benefit which has been suspended under Part~III.

\subsection*{\itshape Up-rating}

25.  A decision of the Secretary of State relating to the up-rating of benefits under Part X of the Administration Act.

\medskip

% Para 26 inserted (3.4.00) by SI 2000/897 Sch 6 para 7
26.  Any decision treated as a decision of the Secretary of State whether or not to waive or defer a work-focused interview.

\amendment{
Para. 26 inserted (3.4.00) by the Social Security (Work-focused Interviews) Regulations 2000 Sch. 6 para. 7.
}

% Para 27 inserted (1.4.02) by SI 2001/4022 reg 21
%\subsection*{\itshape Loss of benefit}
%
%\looseness=-1
%27.  A decision of the Secretary of State that a sanctionable benefit as defined in section 7(8) of the Social Security Fraud Act 2001 is not payable, or is to be reduced, pursuant to section 7, 8 or 9 of that Act as a result of convictions for one or more benefit offences in each of two separate sets of proceedings, one offence being committed within 
%%3 years 
%5 years  % Words substituted (1.4.08) by SI 2008/787 art 3(2)
%of conviction for another, where the only ground of appeal is that any of the convictions was erroneous.
%
%\amendment{
%\looseness=-1
%Para. 27 inserted (1.4.02) by the Social Security (Loss of Benefit) Regulations 2001 reg. 21.
%
%Words substituted in para. 27 (1.4.08) by the Welfare Reform Act 2007 (Commencement No. 6 and Consequential Provisions) Order 2008 art. 3(2).
%}

% Para 27 substituted by SI 2010/1160 reg 3(5)
\subsection*{\itshape Loss of Benefit}

27.—(1) In the circumstances referred to in sub-paragraph (2), a decision of the Secretary of State that a sanctionable benefit as defined in section 6A(1) of the Social Security Fraud Act 2001 is not payable (or is to be reduced) pursuant to section 6B, 7, 8 or 9 of that Act as a result of—
\begin{enumerate}\item[]
\looseness=1
($a$) a conviction for one or more benefit offences in one set of proceedings;

($b$) an agreement to pay a penalty under section 115A of the Administration Act (penalty as alternative to prosecution) or section 109A of the Social Security Administration (Northern Ireland) Act 1992 (the corresponding provision for Northern Ireland) in relation to a benefit offence;

($c$) a caution in respect of one or more benefit offences; or

($d$) a conviction for one or more benefit offences in each of two sets of proceedings, the later offence or offences being committed within the period of 5 years after the date of any of the convictions for a benefit offence in the earlier proceedings.
\end{enumerate}

(2) The circumstances are that the only ground of appeal is that any of the convictions was erroneous, or that the offender (as defined in section 6B(1) of the Social Security Fraud Act 2001) did not commit the benefit offence in respect of which there has been an agreement to pay a penalty or a caution has been accepted.

\amendment{
Para. 27 substituted (1.4.10) by the Social Security (Loss of Benefit) Amendment Regulations 2010 reg. 3(5).

\medskip

Sch. 3 omitted (3.11.08) by the Tribunals, Courts and Enforcement Act 2007 (Transitional and Consequential Provisions) Order 2008 Sch. 1 para. 128.
}

% Sch 3 omitted (3.11.08) by SI 2008/2683 Sch 1 para 128
%\part[Schedule 3 --- Qualifications of persons appointed to the panel]{Schedule 3\\*Qualifications of persons appointed to the panel}
%
%\subsection*{\itshape Legal Qualifications}
%
%\renewcommand\parthead{--- Schedule 3}
%
%1.  Persons who—
%\begin{enumerate}\item[]
%%($a$) have a general qualification (construed in accordance with section 71 of the Courts and Legal Services Act 1990\footnote{\frenchspacing 1990 c. 41.}); or
%
%% Para 1(a) substituted (20.8.08) by SI 2008/1957 reg 2
%($a$) are solicitors of the Senior Courts of England and Wales, barristers in England and Wales or who have a qualification that is specified in an order made under section 7(6A) of the Social Security Act 1998\footnote{1998 c. 14; section 7(6A) was substituted by the Tribunals, Courts and Enforcement Act 2007 (c. 15), section 50 and Schedule 10, Part I, paragraph 29(1) and (4).}; or
%
%($b$) are advocates or solicitors in Scotland.
%\end{enumerate}
%
%\amendment{
%Para. 1(a) substituted (20.8.08) by the Social Security and Child Support (Decisions and Appeals) (Amendment) Regulations 2008 reg. 2 (subject to transitional provisions in reg. 3).
%}
%
%\subsection*{\itshape Medical Qualifications}
%
%2.  
%%Fully   % Word omitted (18.3.05) by SI 2005/337 reg 2(20)(a)(i)
%Registered medical practitioners, where—
%\begin{enumerate}\item[]
%%($a$) the practitioner’s name appears on a medical specialist register maintained in any \textsc{\lowercase{EEA}} State in accordance with the Medical Directive; or
%
%% Para 2(a) substituted (18.3.05) by SI 2005/337 reg 2(20)(a)(ii)
%($a$) the practitioner is a citizen of an \textsc{\lowercase{EEA}} state and his name appears on a medical specialist register maintained in an \textsc{\lowercase{EEA}} state in accordance with the Medical Directive, or he is a Swiss citizen with equivalent qualifications; or
%
%($b$) the practitioner holds a vocational training certificate or a certificate of acquired rights in an \textsc{\lowercase{EEA}} State other than the United Kingdom which must in his case be recognised in the United Kingdom by virtue of the Medical Directive (whether or not as read with the \textsc{\lowercase{EEA}} Agreement) or by virtue of an enforceable community right; or
%
%%($c$) the practitioner does not satisfy the requirements of sub-paragraphs ($a$) or ($b$) above, but has not less than 10 years experience in clinical practice, or as a medical analyst or research worker in disciplines which are the same or similar to those undertaken by practitioners to whom those sub-paragraphs apply.
%
%% Para 2(c) substituted (18.3.05) by SI 2005/337 reg 2(20)(a)(iii)
%($c$) the practitioner does not satisfy the requirements of sub-paragraph ($a$)  or ($b$)  above, but has not less than 10 years experience in clinical practice, or as a medical disability analyst in disciplines which are the same or similar to those undertaken by practitioners to whom those sub-paragraphs apply.
%\end{enumerate}
%
%\amendment{
%Word omitted in para. 2 and para. 2(a), (c) substituted (18.3.05) by the Social Security, Child Support and Tax Credits (Miscellaneous Amendments) Regulations 2005 reg. 2(20)(a).
%}
%
%\medskip
%
%3.  In paragraph 2 above and in this paragraph—
%\begin{enumerate}\item[]
%“\textsc{\lowercase{EEA}} Agreement” means the Agreement of the European Economic Area signed at Oporto on 2nd May 1992 as adjusted by the Protocol signed at Brussels on 17th March 1993\footnote{\frenchspacing Cm. 2183 and OJ No. \textsc{\lowercase{L1}}, 3.1.1994, p.572.};
%
%“\textsc{\lowercase{EEA}} State” means a state which is a contracting party to the \textsc{\lowercase{EEA}} Agreement;
%
%“Medical Directive” means Council Directive 93/16/\textsc{\lowercase{EEC}} of 5th April 1993 to facilitate the free movement of doctors and the mutual recognitions of their diplomas, certificates and other evidence of formal qualifications\footnote{\frenchspacing OJ. No. \textsc{\lowercase{L165}}, 7.7.1993 page 1.}, as amended by Council Directive 97/50/\textsc{\lowercase{EC}} of 6th October 1997\footnote{\frenchspacing OJ. No. \textsc{\lowercase{L921}}, 24.10.1997, page 35.}%
%, or any directive which replaces Directive 93/16/\textsc{\lowercase{EEC}}%  % Words inserted (18.3.05) by SI 2005/337 reg 2(20)(b) 
%;
%
%“Vocational training certificate” means a diploma, certificate or other evidence of formal qualifications awarded on completion of a course of specific training in general medical practice and referred to in article 30 of the Medical Directive.
%\end{enumerate}
%
%\amendment{
%Words inserted in definition of ``Medical Directive'' in para. 3 (18.3.05) by the Social Security, Child Support and Tax Credits (Miscellaneous Amendments) Regulations 2005 reg. 2(20)(b).
%}
%
%\subsection*{\itshape Financial Qualifications}
%
%4.  Accountants who are members of—
%\begin{enumerate}\item[]
%($a$) the Institute of Chartered Accountants in England and Wales;
%
%($b$) the Institute of Chartered Accountants in Scotland;
%
%($c$) the Institute of Chartered Accountants in Ireland;
%
%% Para 4(cc) inserted (20.5.02) by SI 2002/1379 reg 22
%($cc$) the Institute of Certified Public Accountants in Ireland;
%
%($d$) the Association of Chartered Certified Accountants;
%
%($e$) the Chartered Institute of Management Accountants; or
%
%($f$) the Chartered Institute of Public Finance and Accountancy.
%\end{enumerate}
%
%\amendment{
%Para. 4(cc) inserted (20.5.02) by the Social Security and Child Support (Decisions and Appeals) (Miscellaneous Amendments) Regulations 2002 reg. 22.
%}
%
%\subsection*{\itshape Disability Qualifications}
%
%5.  Persons, other than registered medical practitioners, who are experienced in dealing with the needs of disabled persons—
%\begin{enumerate}\item[]
%($a$) in a professional or voluntary capacity; or
%
%($b$) because they are themselves disabled.
%\end{enumerate}

% Sch 3A inserted (19.6.00) by SI 2000/1596 reg 35
\part[Schedule 3A --- Date 
%on which change of circumstances takes effect in certain cases 
from which superseding decision takes effect  % Words substituted (18.3.05) by SI 2005/337 reg 2(21)(a)
where a claimant is in receipt of income support or jobseeker’s allowance]{Schedule 3A\\*Date 
%on which change of circumstances takes effect in certain cases 
from which superseding decision takes effect  % Words substituted (18.3.05) by SI 2005/337 reg 2(21)(a)
where a claimant is in receipt of income support or jobseeker’s allowance}

\renewcommand\parthead{--- Schedule 3A}

\amendment{
Words substituted in heading to Sch. 3A (18.3.05) by the Social Security, Child Support and Tax Credits (Miscellaneous Amendments) Regulations 2005 reg. 2(21)(a).
}

\subsection*{\itshape Income Support}

1.  Subject to paragraphs 2 to 6, where the amount of income support payable under an award is changed by a superseding decision made on the ground of a change of circumstances, that superseding decision shall take effect—
\begin{enumerate}\item[]
($a$) where income support is paid in arrears, from the first day of the benefit week in which the relevant change of circumstances occurs or is expected to occur; or

($b$) where income support is paid in advance, from the date of the relevant change of circumstances, or the day on which the relevant change of circumstances is expected to occur, if either of those days is the first day of the benefit week and otherwise from the next following such day,
\end{enumerate}
and for the purposes of this paragraph any period of residence in temporary accommodation under arrangements for training made under section 2 of the Employment and Training Act 1973\footnote{\frenchspacing 1973 c. 50.} or section 2 of the Enterprise and New Towns (Scotland) Act 1990\footnote{\frenchspacing 1990 c. 35.} for a period which is expected to last for seven days or less shall not be regarded as a change of circumstances.

\medskip

2.  In the cases set out in paragraph 3, the superseding decision shall take effect from the day on which the relevant change of circumstances occurs or is expected to occur.

\medskip

3.  The cases referred to in paragraph 2 are where—
\begin{enumerate}\item[]
($a$) income support is paid in arrears and entitlement ends, or is expected to end, for a reason other than that the claimant no longer satisfies the provisions of section 124(1)($b$)  of the Contributions and Benefits Act\footnote{\frenchspacing 1992 c. 4.};

% Para 3(aa) inserted (8.4.02) by SI 2002/398 reg 3(a)
($aa$) income support is being paid from 8th April 2002 to persons who, immediately before that day, had a preserved right for the purposes of the Income Support Regulations;

($b$) a child or young person referred to in regulation 16(6) of the Income Support Regulations\footnote{\frenchspacing S.I. 1987/1967.} (child in care of local authority or detained in custody) lives, or is expected to live, with the claimant for part only of the benefit week;

% Para 3(c) omitted (18.3.05) by SI 2005/337 reg 2(21)(b)
%($c$) a claimant or his partner (as defined in regulation 2(1) of the Income Support Regulations) enters, or is expected to enter, a nursing home or a residential care home (as defined in regulation 19(3) of those Regulations) or residential accommodation (as defined in regulation 21(3)($a$)  to ($d$)  of those Regulations) for a period of not more than 8 weeks;

($d$) a person referred to in paragraph 1, 2, 3 or 18 of Schedule 7 to the Income Support Regulations—
\begin{enumerate}\item[]
(i) ceases, or is expected to cease, to be a patient; or

(ii) a member of his family ceases, or is expected to cease, to be a patient,
\end{enumerate}
in either case for a period of less than a week;

($e$) a person referred to in paragraph 8 of Schedule 7 to the Income Support Regulations—
\begin{enumerate}\item[]
(i) ceases to be a prisoner; or

(ii) becomes a prisoner;
\end{enumerate}

($f$) a person to whom section 126 of the Contributions and Benefits Act (trade disputes) applies—
\begin{enumerate}\item[]
(i) becomes incapable of work by reason of disease or bodily or mental disablement; or

(ii) enters the maternity period (as defined in section 126(2) of that Act) or the day is known on which that person is expected to enter the maternity period;
\end{enumerate}

($g$) during the currency of the claim, a claimant makes a claim for a relevant social security benefit—
\begin{enumerate}\item[]
(i) the result of which is that his benefit week changes; or

(ii) under regulation 13 of the Claims and Payment Regulations and an award of that benefit on the relevant day for the purposes of that regulation means that his benefit week is expected to change;
\end{enumerate}

% Para 3(h) added (2.10.06) by SI 2006/2377 reg 3(3)
($h$) regulation 9 of the Social Security (Disability Living Allowance) Regulations 1991\footnote{S.I. 1991/2890.} (persons in certain accommodation other than hospitals) applies, or ceases to apply, to the claimant for a period of less than one week%
%
% Para 3(i) inserted by SI 2013/388 Sch para 21
; or

($i$) regulations under section 86(1) (hospital in-patients) of the Welfare Reform Act 2012 apply, or cease to apply, to the claimant for a period of less than one week.
\end{enumerate}

\amendment{
Para. 3(aa) inserted (8.4.02) by the Social Security Amendment (Residential Care and Nursing Homes) Regulations 2002 reg. 3(a).

Para. 3(c) omitted (18.3.05) by the Social Security, Child Support and Tax Credits (Miscellaneous Amendments) Regulations 2005 reg. 2(21)(b).

Para. 3(h) added (2.10.06) by the Social Security (Miscellaneous Amendments) (No. 3) Regulations 2006 reg. 3(3).

Para 3(i) inserted (8.4.13) by the Personal Independence Payment (Supplementary Provisions and Consequential Amendments) Regulations 2013 Sch. para. 21.
}

\medskip

4.  A superseding decision made in consequence of a payment of income being treated as paid on a particular day under regulation 31(1)($b$)% 
%or (2) 
, (2) or (3)  % Words substituted by SI 2003/1731 reg 5
or 39C(3) of the Income Support Regulations (date on which income is treated as paid) shall take effect from the day on which that payment is treated as paid.

\amendment{
Words substituted in para. 4 (8.8.03) by the Social Security (Working Tax Credit and Child Tax Credit) (Consequential Amendments) (No. 3) Regulations 2003 reg. 5.
}

\medskip

5.  Where—
\begin{enumerate}\item[]
($a$) it is decided upon supersession on the ground of a relevant change of circumstances 
or change specified in paragraphs 12 and 13  % Words inserted (18.3.05) by SI 2005/337 reg 2(21)(c)(i)
that the amount of income support is, or is to be, reduced; and

($b$) the Secretary of State certifies that it is impracticable for a superseding decision to take effect from the day prescribed in 
paragraph 12 or  % Words inserted (18.3.05) by SI 2005/337 reg 2(21)(c)(ii)
the preceding paragraphs of this Schedule (other than where paragraph 3($g$)  or 4 applies),
\end{enumerate}
that superseding decision shall take effect—
\begin{enumerate}\item[]
(i) where the relevant change has occurred, from the first day of the benefit week following that in which that superseding decision is made; or

(ii) where the relevant change is expected to occur, from the first day of the benefit week following that in which that change of circumstances is expected to occur.
\end{enumerate}

\amendment{
Words inserted in para. 5(a), (b) (18.3.05) by the Social Security, Child Support and Tax Credits (Miscellaneous Amendments) Regulations 2005 reg. 2(21)(c).
}

\medskip

6.  Where—
\begin{enumerate}\item[]
($a$) a superseding decision (“the former supersession”) was made on the ground of a relevant change of circumstances in the cases set out in paragraphs 3($b$)  to ($g$); and

($b$) that superseding decision is itself superseded by a subsequent decision because the circumstances which gave rise to the former supersession cease to apply (“the second change”), that subsequent decision shall take effect from the date of the second change.
\end{enumerate}

\subsection*{\itshape Jobseeker’s Allowance}

7.  Subject to paragraphs 8 to 11, where a decision in respect of a claim for jobseeker’s allowance is superseded on the ground that there has been or there is expected to be, a relevant change of circumstances, the supersession shall take effect from the first day of the benefit week (as defined in regulation 1(3) of the Jobseeker’s Allowance Regulations) in which that relevant change of circumstances occurs or is expected to occur.

\medskip

8.  Where the relevant change of circumstances giving rise to the supersession is that—
\begin{enumerate}\item[]
($a$) entitlement to jobseeker’s allowance ends, or is expected to end, for a reason other than that the claimant no longer satisfies the provisions of section 3(1)($a$)  
or 3A(1)($a$)   % Words inserted (19.3.01) by SI 2001/518 reg 4(c)(i)
of the Jobseekers Act\footnote{\frenchspacing 1995 c. 18.}; or

% Para 8(aa) inserted (8.4.02) by SI 2002/398 reg 3(b)
($aa$) jobseeker’s allowance is being paid from 8th April 2002 to persons who, immediately before that day, had a preserved right for the purposes of the Jobseeker’s Allowance Regulations;

($b$) a child or young person who is normally in the care of a local authority or who is detained in custody lives, or is expected to live, with the claimant for a part only of the benefit week; or

% Para 8(c) omitted (18.3.05) by SI 2005/337 reg 2(21)(b)
%($c$) the claimant or his partner enters, or is expected to enter, a nursing home or residential care home for a period of not more than 8 weeks; or

($d$) the partner of the claimant or a member of his family ceases, or is expected to cease, to be a hospital in-patient for a period of less than a week%
% Para 8(e) inserted (19.3.01) by SI 2001/518 reg 4(c)(ii)
; or

    ($e$) 
    a joint-claim couple ceases to be 
%a married or an unmarried couple
a couple%  % Words substituted (5.12.05) by SI 2005/2878 reg 8(4)
,
\end{enumerate}
the supersession shall take effect from the date that the relevant change of circumstances occurs or is expected to occur.

\amendment{
Words inserted in para. 8(a) and para. 8(e) inserted (19.3.01) by the Social Security Amendment (Joint Claims) Regulations 2001 reg. 4(c).

Para. 8(aa) inserted (8.4.02) by the Social Security Amendment (Residential Care and Nursing Homes) Regulations 2002 reg. 3(b).

Para. 8(c) omitted (18.3.05) by the Social Security, Child Support and Tax Credits (Miscellaneous Amendments) Regulations 2005 reg. 2(21)(b).

Words sustituted in para. 8(e) (5.12.05) by the Social Security (Civil Partnership) (Consequential Amendments) Regulations 2005 reg. 8(4).
}

\medskip

9.  Where the relevant change of circumstances giving rise to a supersession is any of those specified in paragraph 8, and, in consequence of those circumstances ceasing to apply, a further superseding decision is made, that further superseding decision shall take effect from the date that those circumstances ceased to apply.

\medskip

10.  Where, under the provisions of regulation 96 or 102C(3) of the Jobseeker’s Allowance Regulations\footnote{\frenchspacing S.I. 1996/207; the relevant amending instrument is S.I. 1998/1174.}, income is treated as paid on a certain date and that payment gives rise, or is expected to give rise, to a relevant change of circumstance resulting in a supersession, that supersession shall take effect from that date.

\medskip

11.  Where a relevant change of circumstances 
or change specified in paragraphs 12 and 13  % Words inserted (18.3.05) by SI 2005/337 reg 2(21)(d)(i)
occurs which results, or is expected to result, in a reduced award of jobseeker’s allowance then, if the Secretary of State is of the opinion that it is impracticable for a supersession to take effect in accordance with 
paragraph 12 or % % Words inserted (18.3.05) by SI 2005/337 reg 2(21)(c)(ii)
the preceding paragraphs of this Schedule, the supersession shall take effect from the first day of the benefit week following that in which the relevant change of circumstances occurs.

\amendment{
Words inserted in para. 11 (18.3.05) by the Social Security, Child Support and Tax Credits (Miscellaneous Amendments) Regulations 2005 reg. 2(21)(d).
}

\subsection*{\sloppy\itshape Changes other than changes of circumstances}

12.  Where an amount of income support or jobseeker’s allowance payable under an award is changed by a superseding decision specified in paragraph 13 the superseding decision shall take effect—
\begin{enumerate}\item[]
($a$) in the case of a change in respect of income support, from the day specified in paragraph 1($a$)  or ($b$)  for a change of circumstances; and

($b$) in the case of a change in respect of jobseeker’s allowance, from the day specified in paragraph 7 for a change of circumstances.
\end{enumerate}

\medskip

13.  The following are superseding decisions for the purposes of paragraph 12—
\begin{enumerate}\item[]
($a$) a decision which supersedes a decision specified in regulation 6(2)($b$)  to ($ee$); and

($b$) a superseding decision which would, but for paragraph 12, take effect from a date specified in regulation 7(5) to (7), (12) to (16), (18) to (20), (22), (24) and (33).
\end{enumerate}

\amendment{
Paras. 12, 13 added (18.3.05) by the Social Security, Child Support and Tax Credits (Miscellaneous Amendments) Regulations 2005 reg. 2(21)(e).
}

% Sch 3B inserted (7.4.03) by SI 2002/3019 reg 22
\part[Schedule 3B --- Date on which change of circumstances takes effect where claimant entitled to state pension credit]{Schedule 3B\\*Date on which change of circumstances takes effect where claimant entitled to state pension credit}

\renewcommand\parthead{--- Schedule 3B}

\amendment{
Sch. 3B inserted (7.4.03) by the State Pension Credit (Consequential, Transitional and Miscellaneous Provisions) Regulations 2002 reg. 21.
}

\medskip

1.  Where the amount of state pension credit payable under an award is changed by a superseding decision made on the ground that there has been a relevant change of circumstances, that superseding decision shall take effect from the following days—
\begin{enumerate}\item[]
($a$) for the purpose only of determining the day on which an assessed income period begins under section 9 of the State Pension Credit Act, from the day following the day on which the last previous assessed income period ended; and

%($b$) except as provided in the following paragraphs, from the day that change occurs or is expected to occur if either of those days is the first day of a benefit week but if it is not from the next following such day.

% Para 1(b) substituted by SI 2011/674 reg 8(a)
($b$) except as provided in the following paragraphs---
\begin{enumerate}\item[]
(i) where state pension credit is paid in advance, from the day that change occurs or is expected to occur if either of those days is the first day of a benefit week but if it is not from the next following such day;

(ii) where state pension credit is paid in arrears, from the first day of the benefit week in which that change occurs or is expected to occur.
\end{enumerate}
\end{enumerate}

\amendment{
Para. 1(b) substituted (11.4.11) by the Social Security (Miscellaneous Amendments) Regulations 2011 reg. 8(a).
}

\medskip

%2.  Subject to paragraph 3, where the relevant change is that the claimant’s income (other than deemed income from capital) has changed
%or that the claimant becomes entitled to disability living allowance (middle or higher rate care component) or to attendance allowance%  % Words inserted by SI 2011/674 reg 8(b)
%, the superseding decision shall take effect on the first day of the benefit week in which that change occurs or if that is not practicable in the circumstances of the case, on the first day of the next following benefit week.
%
%\amendment{
%Words inserted in para. 2 (11.4.11) by the Social Security (Miscellaneous Amendments) Regulations 2011 reg. 8(b).
%}

% Para 2 substituted by SI 2013/443 reg 5(a)
2.  Subject to paragraph 3, where the relevant change is that—
\begin{enumerate}\item[]
($a$) the claimant’s income or the income of the claimant’s partner (other than deemed income from capital) has changed;

($b$) the claimant or the claimant’s partner becomes entitled to—
\begin{enumerate}\item[]
(i) disability living allowance (middle or higher rate care component); 
%or  % Word omitted by SI 2013/443 reg 5(b)(i)

(ii) attendance allowance; 
%or  % Word omitted by SI 2013/591 Sch para 15(2)(a)

% Para 2(b)(iii) inserted by SI 2013/443 reg 5(b)(i)
(iii) personal independence payment (standard or enhanced rate daily living component under section 78 of the Welfare Reform Act 2012); or

% Para 2(b)(iv) inserted by SI 2013/591 Sch para 15(2)(b)
(iv) armed forces independence payment under the Armed Forces and Reserve Forces (Compensation Scheme) Order 2011; or
\end{enumerate}

($c$) the claimant or the claimant’s partner again receives 
%either of the allowances 
any of the allowances or payments  % Words substituted by SI 2013/443 reg 5(b)(ii)
mentioned in sub-paragraph ($b$) above immediately after the end of the period specified in paragraph 2A,
\end{enumerate}
the superseding decision shall take effect on the first day of the benefit week in which that change occurs or if that is not practicable in the circumstances of the case, on the first day of the next following benefit week.

\amendment{
Para. 2 substituted (2.4.13) by the Social Security (Miscellaneous Amendments) Regulations 2013 reg. 5(a).

Para. 2(b)(iii) inserted and words substituted in para. 2(c) (8.4.13) by the Social Security (Miscellaneous Amendments) Regulations 2013 reg. 5(b).

Para. 2(b)(iv) inserted (8.4.13) by the Armed Forces and Reserve Forces Compensation Scheme (Consequential Provisions: Subordinate Legislation) Order 2013 Sch. para. 14.
}

\medskip

% Para 2A inserted by SI 2013/443 reg 5(c)
2A.  A period specified for the purposes of paragraph 2 is a period when the claimant or the claimant’s partner is maintained free of charge while undergoing medical or other treatment as an in-patient in—
\begin{enumerate}\item[]
($a$) a hospital or similar institution under—
\begin{enumerate}\item[]
(i) the National Health Service Act 2006\footnote{2006 c.~41.};

(ii) the National Health Service (Wales) Act 2006\footnote{2006 c.~42.}; or

(iii) the National Health Service (Scotland) Act 1978\footnote{1978 c.~29.}; or
\end{enumerate}

($b$) a hospital or similar institution maintained or administered by the Defence Council.
\end{enumerate}

\amendment{
Para. 2A inserted (2.4.13) by the Social Security (Miscellaneous Amendments) Regulations 2013 reg. 5(c).
}

\medskip

3.  Paragraph 2 shall not apply where the only relevant change is that working tax credit under the Tax Credits Act 2002\footnote{2002 c.\ 21.} becomes payable or becomes payable at a higher rate.

\medskip

4.  A superseding decision shall take effect from the day the change of circumstances occurs or is expected to occur if—
\begin{enumerate}\item[]
($a$) the person ceases to be or becomes a prisoner, and for this purpose “prisoner” has the same meaning as in regulation 1(2) of the State Pension Credit Regulations; or

($b$) whilst entitled to state pension credit a claimant is awarded another social security benefit and in consequence of that award his benefit week changes or is expected to change.
\end{enumerate}

\medskip

%5.  In a case where the relevant change of circumstances is that the claimant ceased for one or more days to be a patient, the superseding decision shall take effect from the first day of the benefit week in which the change occurred.

% Para 5 substituted (6.10.03) by SI 2003/2274 reg 5(4)
5.  In a case where the relevant circumstance is that the claimant ceased to be a patient, if he becomes a patient again in the same benefit week, the superseding decision in respect of ceasing to be a patient shall take effect from the first day of the week in which the change occured.

\amendment{
Para. 5 substituted (6.10.03) by the State Pension Credit (Consequential, Transitional and Miscellaneous Provisions) Amendment Regulations 2003 reg. 5(4).
}

\medskip

6.  In paragraph 5, “patient” means a person (other than a prisoner) who is regarded as receiving free in-patient treatment within the meaning of the 
%Social Security (Hospital In-Patients) Regulations 1975\footnote{S.I. 1975/555.}
Social Security (Hospital In-Patients) Regulations 2005\footnote{S.I. 2005/3360.}%  % Words substituted (24.9.07) by SI 2007/2470 reg 3(9)
.

\amendment{
Words substituted in para. 6 (24.9.07) by the Social Security (Miscellaneous Amendments) (No. 4) Regulations 2007 reg. 3(9).
}

\medskip

7.  
%Where 
Subject to 
%paragraphs 9 and 10
paragraph 8A%  % Words substituted by SI 2011/674 reg 8(c)
, where  % Words substituted by SI 2010/510 reg 4(4)(a)
an amount of state pension credit payable under an award is changed by a superseding decision specified in paragraph~8 the superseding decision shall take effect from the day specified in paragraph~1($b$).

\amendment{
Para. 7 added (10.4.06) by the Social Security (Miscellaneous Amendments (No. 2) Regulations 2006 reg. 5(4).

Words substituted in para. 7 (6.4.10) by the Social Security (Miscellaneous Amendments) Regulations 2010 reg. 4(4)(a).

Words substituted in para. 7 (11.4.11) by the Social Security (Miscellaneous Amendments) Regulations 2011 reg. 8(c).
}

\medskip

8.  The following are superseding decisions for the purposes of paragraph~7—
\begin{enumerate}\item[]
($a$) a decision which supersedes a decision specified in regulation 6(2)($b$)  to~($ee$)  and ($m$); and

($b$) a superseding decision which would, but for paragraphs 2 and 7, take effect from a date specified in regulation 7(5) to (7), (12) to (16) and~(29C).
\end{enumerate}

\amendment{
Para. 8 added (10.4.06) by the Social Security (Miscellaneous Amendments (No. 2) Regulations 2006 reg. 5(4).

\medskip

Paras. 9--11 added (6.4.10) by the Social Security (Miscellaneous Amendments) Regulations 2010 reg. 4(4)(b).

Para. 8A substituted for paras. 9, 10 (11.4.11) by the Social Security (Miscellaneous Amendments) Regulations 2011 reg. 8(d).
}

\medskip

% Paras 9--11 added by SI 2010/510 reg 4(4)(b)
%9.  Where state pension credit is paid in arrears and the relevant change of circumstances results in the award of that benefit being terminated, the superseding decision shall take effect on the first day of the benefit week next following the benefit week in which that change occurs or is expected to occur.
%
%\medskip
%
%10.  Where state pension credit is paid in advance and the relevant change of circumstances is the death of the claimant, the superseding decision shall take effect on the first day of the benefit week next following the date of death.

% Para 8A substituted for paras 9, 10 by SI 2011/674 reg 8(d)
8A.  Where the relevant change of circumstances is the death of the claimant, the superseding decision shall take effect on the first day of the benefit week next following the date of death.

\medskip

11.  In this Schedule, “benefit week” means—
\begin{enumerate}\item[]
($a$) where state pension credit is paid in advance, the period of 7 days beginning on the day on which, in the claimant’s case, that benefit is payable;

($b$) where state pension credit is paid in arrears, the period of 7 days ending on the day on which, in the claimant’s case, that benefit is payable.
\end{enumerate}

\vfill

% Sch 3C inserted (27.7.08) by SI 2008/1554 reg 43
\part[Schedule 3C --- Date from which change of circumstances takes effect where claimant entitled to employment and support allowance]{Schedule 3C\\*Date from which change of circumstances takes effect where claimant entitled to employment and support allowance}

\renewcommand\parthead{--- Schedule 3C}

\amendment{
Sch. 3C inserted (27.7.08) by the Employment and Support Allowance (Consequential Provisions) (No. 2) Regulations 2008 reg. 43.
}

\medskip

\noindent
1.  Subject to paragraphs 2 to 7, where the amount of an employment and support allowance payable under an award is changed by a superseding decision made on the ground of a change of circumstances, that superseding decision shall take effect from the first day of the benefit week in which the relevant change of circumstances occurs or is expected to occur.

\medskip

2.  In the cases set out in paragraph 3, the superseding decision shall take effect from the day on which the relevant change of circumstances occurs or is expected to occur.

\medskip

3.  The cases referred to in paragraph 2 are where—
\begin{enumerate}\item[]
($a$) entitlement ends, or is expected to end, for a reason other than that the claimant no longer satisfies the provisions of paragraph 6(1)($a$)  of Schedule 1 to the Welfare Reform Act;

($b$) a child or young person referred to in regulation 156(6)($d$)  or ($h$)  of the Employment and Support Allowance Regulations (child in care of local authority or detained in custody) lives, or is expected to live, with the claimant for part only of the benefit week;

($c$) a person referred to in paragraph 12 of Schedule 5 to the Employment and Support Allowance Regulations—
\begin{enumerate}\item[]
(i) ceases, or is expected to cease, to be a patient; or

(ii) a member of the person’s family ceases, or is expected to cease, to be a patient,
\end{enumerate}
in either case for a period of less than a week;

($d$) a person referred to in paragraph 3 of Schedule 5 to the Employment and Support Allowance Regulations—
\begin{enumerate}\item[]
(i) ceases to be a prisoner; or

(ii) becomes a prisoner;
\end{enumerate}

($e$) during the currency of the claim a claimant makes a claim for a relevant social security benefit—
\begin{enumerate}\item[]
(i) the result of which is that his benefit week changes; or\looseness=-1

(ii) in accordance with regulation 13 of the Claims and Payments Regulations and an award of that benefit on the relevant day for the purposes of that regulation means that his benefit week is expected to change;
\end{enumerate}

% Para 3(f), (g) inserted by SI 2013/2536 reg 8
($f$) regulation 9 of the Social Security (Disability Living Allowance) Regulations 1991 (persons in care homes) applies, or ceases to apply, to the claimant for a period of less than one week; or

($g$) regulations under section 85(1) of the Welfare Reform Act 2012 (care home residents) apply, or cease to apply, to the claimant for a period of less than one week.
\end{enumerate}

\amendment{
Para. 3(f), (g) inserted (29.10.13) by the Social Security (Miscellaneous Amendments) (No. 3) Regulations 2013 reg. 8.
}

\medskip

4.  A superseding decision made in consequence of a payment of income being treated as paid on a particular day under regulation 93 of the Employment and Support Allowance Regulations (date on which income is treated as paid) shall take effect from the day on which that payment is treated as paid.

\medskip

5.  Where—
\begin{enumerate}\item[]
($a$) it is decided upon supersession on the ground of a relevant change of circumstances or change specified in paragraphs 9 and 10 that the amount of an employment and support allowance is, or is to be, reduced; and

($b$) the Secretary of State certifies that it is impracticable for a superseding decision to take effect from the day prescribed in paragraph 9 or the preceding paragraphs of this Schedule (other than where paragraph 3($e$)  or 4 applies),
\end{enumerate}
that superseding decision shall take effect—
\begin{enumerate}\item[]
(i) where the relevant change has occurred, from the first day of the benefit week following that in which that superseding decision is made; or

(ii) where the relevant change is expected to occur, from the first day of the benefit week following that in which that change of circumstances is expected to occur.
\end{enumerate}

\medskip

6.  Where—
\begin{enumerate}\item[]
($a$) a superseding decision (“the former supersession”) was made on the ground of a relevant change of circumstances in the cases set out in paragraph 3($b$)  to ($e$); and

($b$) that superseding decision is itself superseded by a subsequent decision because the circumstances which gave rise to the former supersession cease to apply (“the second change”),
\end{enumerate}
that subsequent decision shall take effect from the date of the second change.

\medskip

7.  In the case of an employment and support allowance decision where there has been a limited capability for work determination, where—
\begin{enumerate}\item[]
($a$) the Secretary of State is satisfied that, in relation to a limited capability for work determination, the claimant or payee failed to notify an appropriate office of a change of circumstances which regulations under the Administration Act required him to notify; and

($b$) the claimant or payee, as the case may be, could reasonably have been expected to know that the change of circumstances should have been notified,
\end{enumerate}
the superseding decision shall take effect—
\begin{enumerate}\item[]
(i) from the date on which the claimant or payee, as the case may be, ought to have notified the change of circumstances; or\looseness=-1

(ii) if more than one change has taken place between the date from which the decision to be superseded took effect and the date of the superseding decision, from the date on which the first change ought to have been notified.
\end{enumerate}

\section*{\itshape Changes other than changes of circumstances}

8.  Where—
\begin{enumerate}\item[]
($a$) the Secretary of State supersedes a decision made by an appeal tribunal or a Commissioner on the grounds specified in regulation 6(2)($c$)(i)  (ignorance of, or mistake as to, a material fact);\looseness=-1

($b$) the decision to be superseded was more advantageous to the claimant because of the ignorance or mistake than it would otherwise have been; and

($c$) the material fact—
\begin{enumerate}\item[]
(i) does not relate to the limited capability for work determination embodied in or necessary to the decision; or\looseness=-1

(ii) relates to a limited capability for work determination embodied in or necessary to the decision and the Secretary of State is satisfied that at the time the decision was made the claimant or payee, as the case may be, knew or could reasonably have been expected to know of it and that it was relevant,
\end{enumerate}
\end{enumerate}
the superseding decision shall take effect from the first day of the benefit week in which the decision of the appeal tribunal or the Commissioner took effect or was to take effect.

\medskip

9.  Where an amount of an employment and support allowance payable under an award is changed by a superseding decision specified in paragraph 10 the superseding decision shall take effect from the day specified in paragraph 1 for a change of circumstances.\looseness=-1

\medskip

10.  The following are superseding decisions for the purposes of paragraph 9—
\begin{enumerate}\item[]
($a$) a decision which supersedes a decision specified in regulation 6(2)($b$)  and~($d$)  to ($ee$); and

\begin{sloppypar}
($b$) a superseding decision which would, but for paragraph 9, take effect from a date specified in regulation 7(6), (7), (12), (13), (17D) to (17F), and (33).
\end{sloppypar}
\end{enumerate}

% Sch 3D inserted (6.4.09 for new-rules cases only) by SI 2009/396 reg 4(16), omitted (10.12.12 for 2012 scheme cases) by SI 2012/2785 reg 6(8)
\part[Schedule 3D --- Effective dates for supersession of child support decisions]{Schedule 3D\\*Effective dates for supersession of child support decisions\\*\emph{2003 scheme only}}

\renewcommand\parthead{--- Schedule 3D}

\amendment{
Sch. 3D inserted (6.4.09 for new-rules cases only) by the Child Support (Miscellaneous Amendments) Regulations 2009 reg. 4(16).

Sch. 3D omitted (10.12.12 for 2012 scheme cases only) by the Child Support (Meaning of Child and New Calculation Rules) (Consequential and Miscellaneous Amendment) Regulations 2012 reg. 6(8).
}

\medskip

1.  This Schedule sets out the exceptions to the general rule in section 17(4) of the Child Support Act (that is the rule that a supersession decision takes effect from the beginning of the maintenance period in which it is made or, where applicable, the beginning of the maintenance period in which an application for a supersession is made).\looseness=-1

\section*{\itshape Expected change}

2.  Where the ground for the supersession decision is that a relevant change of circumstances is expected to occur or that a ground for a variation is expected to occur, the decision takes effect from the beginning of the maintenance period in which that change or that ground is expected to occur.

\section*{\itshape Decision backdated to when the change occurred}

3.  Where the ground for the supersession decision is that a relevant change of circumstances of the following kind has occurred, the decision takes effect from the beginning of the maintenance period in which the change occurred—
\begin{enumerate}\item[]
($a$) a qualifying child dies or ceases to be a qualifying child;

% Para 3(aa) inserted (4.7.11) by SI 2011/1464 reg 2(4)(a)(i)
($aa$) a relevant other child dies or ceases to be a relevant other child;

($b$) the person with care ceases to be a person with care in relation to a qualifying child;

($c$) the person with care, the non-resident parent or a qualifying child ceases to be habitually resident in the United Kingdom; 
%or  % Word omitted (4.7.11) by SI 2011/1464 reg 2(4)(a)(ii)

($d$) paragraph 4(2) of Schedule 1 to the Child Support Act (flat rate for a non-resident parent whose partner is a non-resident parent) begins or ceases to apply%
%
% Para 3(e) added (4.7.11) by SI 2011/1464 reg 2(4)(a)(iii)
; or

($e$) the non-resident parent begins or ceases to receive a benefit mentioned in regulation 4(1) of the Maintenance Calculations and Special Cases Regulations (flat rate) or begins or ceases to be a person who receives, or whose partner receives, a benefit referred to in regulation 4(2) of those Regulations.
\end{enumerate}

\amendment{
Para. 3(aa), (e) inserted (4.7.11) by the Child Support (Miscellaneous Amendments) Regulations 2011 reg. 2(4)(a).
}

\medskip

% Para 3A inserted (4.7.11) by SI 2011/1464 reg 2(4)(b)
3A.  In paragraph 3, the reference to the day on which a person begins or ceases to receive a benefit is to the day on which entitlement to the benefit commences or ends, as the case may be.

\amendment{
Para. 3A inserted (4.7.11) by the Child Support (Miscellaneous Amendments) Regulations 2011 reg. 2(4)(b).

\medskip

Para. 4 omitted (4.7.11) by the Child Support (Miscellaneous Amendments) Regulations 2011 reg. 2(4)(c).
}

% Para 4 omitted (4.7.11) by SI 2011/1464 reg 2(4)(c)
%\section*{\itshape Non-resident parent or partner on or off benefit}
%
%4.  Where a supersession decision is made by the Commission acting on its own initiative on the basis of information or evidence which was also the basis of a decision made by the Secretary of State under section 8, 9 or 10 of the Act (decisions on claims for benefits), the decision takes effect from the beginning of the maintenance period in which that information is brought to the attention of the Commission.

\section*{\itshape New qualifying child}

5.  Paragraphs 6 and 7 apply where the ground for the supersession is that there is a new qualifying child in relation to the non-resident parent.

\medskip

6.  Where there is a new qualifying child in relation to the same person with care—
\begin{enumerate}\item[]
($a$) if the application is made by the non-resident parent, the decision takes effect from the beginning of the maintenance period in which the application is made; and

($b$) if the application is made by the person with care the decision takes effect from the beginning of the maintenance period in which notification of the application is given to the non-resident parent.
\end{enumerate}

\medskip

7.  Where there is a new qualifying child in relation to a different person with care and an application for a maintenance calculation has been made under section 4 or section 7 of the Child Support Act, the decision takes effect from the beginning of the maintenance period in which notification of the calculation is given to the non-resident parent.

\section*{\itshape\sloppy Series of changes waiting to be actioned}

8.  Where a decision is superseded on application and, in relation to that decision, a maintenance calculation is made to which paragraph 15 of Schedule 1 to the Child Support Act applies, the effective date of the calculation or calculations is the beginning of the maintenance period in which the change of circumstances to which the calculation relates occurred or is expected to occur and where it occurred before the date of the application for the supersession and was notified after that date, 
%the date of that application
the beginning of the maintenance period in which that application was made%  % Words substituted (4.7.11) by SI 2011/1464 reg 2(4)(d)
.%\looseness=-1

\amendment{
Words substituted in para. 8 (4.7.11) by the Child Support (Miscellaneous Amendments) Regulations 2011 reg. 2(4)(d).
}

\section*{\itshape Own initiative decision}

9.  Unless paragraph 4 applies, where a decision is superseded in a case where the 
%Commission 
Secretary of State  % Words substituted (1.8.12) by SI 2012/2007 Sch para 113(15)
is required to give notice under regulation 7C, the decision takes effect from the first day of the maintenance period which includes the date which is 28 days after the date on which the 
%Commission 
Secretary of State  % Words substituted (1.8.12) by SI 2012/2007 Sch para 113(15)
has given notice (oral or written) to the relevant persons under that regulation.

\amendment{
Words substituted in para. 9 (1.8.12) by the Public Bodies (Child Maintenance and Enforcement Commission: Abolition and Transfer of Functions) Order 2012 Sch. para. 113(15).
}

\section*{\itshape\sloppy\hbadness=10000 Supersession of tribunal decision made pending outcome of a related appeal}

10.  Where, in accordance with section 28ZB(5) of the Child Support Act (appeals involving issues that arise on appeal in other cases), the 
%Commission 
Secretary of State  % Words substituted (1.8.12) by SI 2012/2007 Sch para 113(15)
makes a decision superseding the decision of the First-tier Tribunal or the Upper Tribunal, the superseding decision takes effect from the beginning of the maintenance period following the date on which the decision of the First-tier Tribunal or, as the case may be, the Upper Tribunal would have taken effect had it been decided in accordance with the determination of the Upper Tribunal or the court in the appeal referred to in section 28ZB(1)($b$).\looseness=-1

\amendment{
Words substituted in para. 10 (1.8.12) by the Public Bodies (Child Maintenance and Enforcement Commission: Abolition and Transfer of Functions) Order 2012 Sch. para. 113(15).
}

\section*{\itshape\sloppy\hbadness=10000  Supersession of tribunal decision made in error due to misrepresentation etc.}

11.  Where—
\begin{enumerate}\item[]
($a$) a decision made by 
%the First-tier Tribunal or the Upper Tribunal 
an appeal tribunal, the First-tier Tribunal, the Upper Tribunal or of a Child Support Commissioner  % Words substituted (3.11.08) by SI 2012/1267 reg 4(6)(a)
is superseded on the ground that it was erroneous due to misrepresentation of, or that there was a failure to disclose, a material fact; and

($b$) the 
%Commission 
Secretary of State  % Words substituted (1.8.12) by SI 2012/2007 Sch para 113(15)
is satisfied that the decision was more advantageous to the person who misrepresented or failed to disclose that fact than it would otherwise have been but for that error,
\end{enumerate}
the superseding decision takes effect from the date on which the decision of 
%the First-tier Tribunal or, as the case may be, the Upper Tribunal 
an appeal tribunal, the First-tier Tribunal, the Upper Tribunal or a Child Support Commissioner (as the case may be)  % Words substituted (3.11.08) by SI 2012/1267 reg 4(6)(b)
took, or was to take, effect.

\amendment{
Words substituted in para. 11 (3.11.08) by the Social Security and Child Support (Supersession of Appeal Decisions) Regulations 2012 reg. 4(6).

Words substituted in para. 11 (1.8.12) by the Public Bodies (Child Maintenance and Enforcement Commission: Abolition and Transfer of Functions) Order 2012 Sch. para. 113(15).
}

\section*{\itshape\sloppy\hbadness=10000 Supersession of look alike case where law reinterpreted by the Upper Tribunal or a court}

12.  Any decision made under section 17 of the Child Support Act in consequence of a determination which is a relevant determination for the purposes of section 28ZC (cases of error) of that Act takes effect from the date of the relevant determination.

\part[Schedule 4 --- Revocations]{Schedule 4\\*Revocations}

\renewcommand\parthead{--- Schedule 4}

{\footnotesize \hbadness=10000
%\begin{tabulary}{\linewidth}{JJJ}
\begin{longtable}{p{50pt}p{188pt}p{116pt}}
\hline
\itshape Column 1 & \itshape Column 2 & \itshape Column 3\\
\itshape Statutory\newline Instrument\newline Number & \itshape Statutory Instrument & \itshape Provision Revoked\\
\hline
\endhead
\hline
\endlastfoot
1979/432&The Vaccine Damage Payments Regulations 1979&Part III\\
1992/2641&The Child Support Appeal Tribunals (Procedure) Regulations 1992&The whole Regulations\\
1995/311&The Social Security (Incapacity for Work) (General) Regulations 1995&Regulations 19 and 20 to 22\\
1995/1801&The Social Security (Adjudication) Regulations 1995&The whole Regulations\\
1996/182&The Social Security (Adjudication) and Child Support Amendment Regulations 1996&Regulation 2\\
1996/425&The Social Security (Industrial Injuries and Diseases) (Miscellaneous Amendments) Regulations 1996&Regulation 2\\
1996/1518&The Social Security (Adjudication) Amendment Regulations 1996&The whole Regulations\\
1996/2306&The Social Security (Claims and Payments and Adjudication) Amendment Regulations 1996&Regulations 8 and 9\\
1996/2450&The Social Security (Adjudication) and Child Support Amendment (No.\ 2) Regulations 1996&Regulations 2 to 13\\
1996/2659&The Social Security (Adjudication) Amendment (No.\ 2) Regulations 1996&The whole Regulations\\
1997/65&The Income-Related Benefits and Jobseeker’s Allowance (Miscellaneous Amendments) Regulations 1997&Regulation 16\\
1997/793&The Social Security (Miscellaneous Amendments) (No. 2) Regulations 1997&Regulations 1(2)($a$) and 8 to 17\\
1997/810&The Social Security (Industrial Injuries) (Miscellaneous Amendments) Regulations 1997&Regulations 2, 3 and 4\\
%1997/995
1997/955  % Entry amended (5.7.99) by SI 1999/1623 reg 6
&The Social Security (Adjudication) and Commissioners Procedure and Child Support Commissioners (Procedure) Amendment Regulations 1997&In regulation 1(2), the definition of “the Adjudication Regulations” and regulations 2 to 6\\
1997/1839&The Social Security (Attendance Allowance and Disability Living Allowance) (Miscellaneous Amendments) Regulations 1997&In regulation 1(2) the definition of “the Adjudication Regulations” and regulation 4\\
1997/2237&The Social Security (Recovery of Benefits) (Appeals) Regulations 1997&The whole Regulations\\
1997/2305&The Social Security (Miscellaneous Amendments) (No.\ 4) Regulations 1997&Regulation 4\\
%\end{tabulary}
\end{longtable}

}

\amendment{
Entry ``1997/995'' in col. 1 of Sch. 4 substituted (5.7.99) by the Social Security and Child Support (Decisions and Appeals) Amendment (No. 2) Regulations 1999 reg. 7.
}

\part{Explanatory Note}

\renewcommand\parthead{--- Explanatory Note}

\subsection*{(This note is not part of the Regulations)}

 These Regulations are made by virtue of, or in consequence of, provisions in the Social Security Act 1998 (c.\ 14) (“the Act”) and supplement changes introduced by that Act to the decision-making process for social security and related matters. The Regulations also provide for the procedural rules and other requirements of a new unified appeals system introduced by the Act for social security, contracting out of pensions, child support and vaccine damage.

  The Regulations are made before the end of the period of six months beginning with the coming into force of the relevant provisions in the Act and are therefore exempted from the requirement in section 172(1) of the Social Security Administration Act 1992 (c. 5) to refer proposals to make these Regulations to the Social Security Advisory Committee and are made without reference to that Committee. The Regulations are made after consultation with the Council on Tribunals in accordance with section 8 of the Tribunals and Inquiries Act 1992 (c. 53).

  Part I of the Regulations contains provisions relating to commencement, citation and interpretation as well as service of notices or documents.

  Part II of the Regulations makes provision for decisions by the Secretary of State in social security and related matters. Chapters I and II provide for the circumstances in which the Secretary of State may revise or supersede decisions, when such decisions take effect and related procedural rules. Chapter III contains miscellaneous provisions relating to decisions of the Secretary of State in social security, including certain relevant requirements formerly contained in the Social Security (Adjudication) Regulations 1995 and other related regulations which are now revoked. It also includes provision in relation to industrial injuries benefits for the Secretary of State to seek advice from a medical practitioner.

  Part III of the Regulations makes provision for the suspension and termination of benefit and for dealing with decisions where there are related appeals or decisions.

  Part IV concerns rights of appeal and the procedure for bringing appeals. Chapter I makes provision for additional persons to have a right of appeal, for decisions (other than those in the Act) against which there is no right of appeal and decisions where there is a right of appeal. It also provides procedural rules for bringing appeals.

  Part V makes provision for appeal tribunals for social security, contracting out of pensions, vaccine damage and child support. Chapter I provides for the panel from which appeal tribunal members are drawn to include people with specified qualifications, for requirements relating to the composition of appeal tribunals and the assignment of clerks to tribunals. Chapters II to V of that Part provide for further matters relating to appeals and referrals. Chapter II makes provision for procedural requirements in the determination of appeals and referrals, including withdrawal of appeals or referrals, medical examinations and witnesses. Chapter III provides for the striking out of certain appeals and related procedures. Chapter IV provides for procedures at oral hearings and Chapter V makes provision relating to the decisions and reports of appeal tribunals and applications for leave to a Commissioner.

  Part VI and Schedule 4 provide for revocations.

  These Regulations do not impose a charge on business.

\end{document}