\documentclass[12pt,a4paper]{article}

\newcommand\regstitle{The Social Security and Child Support (Decisions and Appeals) Regulations 1999}

\newcommand\regsnumber{1999/991}

%\opt{newrules}{
\title{\regstitle}
%}

%\opt{2012rules}{
%\title{Child Maintenance and Other Payments Act 2008\\(2012 scheme version)}
%}

\author{S.I. 1999 No. 991}

\date{Made 26th March 1999\\Coming into force in accordance with regulation 1(2)}

%\opt{oldrules}{\newcommand\versionyear{1993}}
%\opt{newrules}{\newcommand\versionyear{2003}}
%\opt{2012rules}{\newcommand\versionyear{2012}}

\usepackage{csa-regs}

\setlength\headheight{27.57402pt}

\begin{document}

\maketitle

\noindent
Whereas a draft of this Instrument was laid before Parliament in accordance with section 80(1) of the Social Security Act 1998\footnote{\frenchspacing 1998 c. 14.} and approved by resolution of each House of Parliament;

 Now, therefore, the Secretary of State for Social Security, in exercise of powers set out in Schedule 1 to this Instrument and of all other powers enabling him in that behalf, with the concurrence of the Lord Chancellor in so far as the Regulations are made under section 6(3) of the Social Security Act 1998, by this Instrument, which contains only regulations made by virtue of, or consequential upon, those provisions of the Social Security Act 1998 and which is made before the end of the period of six months beginning with the coming into force of those provisions\footnote{\frenchspacing See section 173(5)($b$) of the Social Security Administration Act 1992 (c. 5).}, after consultation with the Council on Tribunals in accordance with section 8 of the Tribunals and Inquiries Act 1992\footnote{\frenchspacing 1992 c. 53.}, hereby makes the following Regulations:

{\sloppy

\tableofcontents

}

\bigskip

\setcounter{secnumdepth}{-2}

\section[Part I --- General]{Part I\\*General}

\renewcommand\parthead{--- Part I}

\subsection[1. Citation, commencement and interpretation]{Citation, commencement and interpretation}

1.—(1) These Regulations may be cited as the Social Security and Child Support (Decisions and Appeals) Regulations 1999.

(2) These Regulations shall come into force—
\begin{enumerate}\item[]
($a$) in so far as they relate to child support and for the purposes of this regulation and regulation 2 on 1st June 1999;

($b$) in so far as they relate to—
\begin{enumerate}\item[]
(i) industrial injuries benefit, guardian’s allowance and child benefit; and

(ii) a decision made under the Pension Schemes Act 1993\footnote{\frenchspacing 1993 c. 48; section 170 was substituted by paragraph 131 of Schedule 7 to the Social Security Act 1998.} by virtue of section 170(2) of that Act;
\end{enumerate}
on 5th July 1999;

($c$) in so far as they relate to retirement pension, widow’s benefit, incapacity benefit, severe disablement allowance and maternity allowance, on 6th September 1999;

($d$) in so far as they relate to family credit and disability working allowance, on 5th October 1999;

($e$) in so far as they relate to attendance allowance, disability living allowance, invalid care allowance, jobseeker’s allowance, credits of contributions or earnings, home responsibilities protection and vaccine damage payments, on 18th October 1999; and

($f$) for all remaining purposes, on 29th November 1999.
\end{enumerate}

(3) In these Regulations, unless the context otherwise requires—
\begin{enumerate}\item[]
“the Act” means the Social Security Act 1998;

“the 1997 Act” means the Social Security (Recovery of Benefits) Act 1997\footnote{\frenchspacing 1997 c. 27.};

“the Claims and Payments Regulations” means the Social Security (Claims and Payments) Regulations 1987\footnote{\frenchspacing S.I. 1987/1968.};

“appeal” means an appeal to an appeal tribunal;

“claimant” means—
\begin{enumerate}\item[]
($a$) any person who is a claimant for the purposes of section 191 of the Administration Act or section 35(1) of the Jobseekers Act or any other person from whom benefit is alleged to be recoverable; and

($b$) any person subject to a decision of the Secretary of State under the Pension Schemes Act 1993\footnote{\frenchspacing 1993 c. 48.};
\end{enumerate}

“clerk to the appeal tribunal” means a clerk assigned to the appeal tribunal in accordance with regulation 37;

“the date of notification” means—
\begin{enumerate}\item[]
($a$) the date that notification of a decision of the Secretary of State is treated as having been given or sent in accordance with regulation 2($b$); or

($b$) in the case of a social fund payment arising in accordance with regulations made under section 138(2) of the Contributions and Benefits Act—
\begin{enumerate}\item[]
(i) the date seven days after the date on which the Secretary of State makes his decision to make a payment to a person to meet expenses for heating;

(ii) where a person collects the instrument of payment at a post office, the date the instrument is collected;

(iii) where an instrument of payment is sent to a post office for collection but is not collected and a replacement instrument is issued, the date on which the replacement instrument is issued; or

(iv) where a person questions his failure to be awarded a payment for expenses for heating, the date on which the notification of the Secretary of State’s decision given in response to that question is issued;
\end{enumerate}
\end{enumerate}

“financially qualified panel member” means a panel member who satisfies the requirements of paragraph 4 of Schedule 3;

“the Income Support Regulations” means the Income Support (General) Regulations 1987\footnote{\frenchspacing S.I. 1987/1967.};

“the Jobseeker’s Allowance Regulations” means the Jobseeker’s Allowance Regulations 1996\footnote{\frenchspacing S.I. 1996/207.};

“legally qualified panel member” means a panel member who satisfies the requirements of paragraph 1 of Schedule 3;

“medically qualified panel member” means a panel member who satisfies the requirements of paragraph 2 of Schedule 3;

“misconceived appeal” means an appeal which is—
\begin{enumerate}\item[]
($a$) frivolous or vexatious; or

($b$) obviously unsustainable and has no prospect of success,
\end{enumerate}
other than an out of jurisdiction appeal;

“official error” means an error made by an officer of the Department of Social Security or the Department for Education and Employment acting as such which no person outside either Department caused or to which no person outside either Department materially contributed;

“out of jurisdiction appeal” means an appeal brought against a decision which is specified in Schedule 2 to the Act or a decision prescribed in regulation 27 (decisions against which no appeal lies);

“panel” means the panel constituted under section 6;

“panel member” means a person appointed to the panel;

“panel member with a disability qualification” means a panel member who satisfies the requirements of paragraph 5 of Schedule 3;

“party to the proceedings” means the Secretary of State and any other person—
\begin{enumerate}\item[]
($a$) who is one of the principal parties for the purposes of sections 13 and 14;

($b$) who has a right of appeal to an appeal tribunal under section 11(2) of the 1997 Act\footnote{\frenchspacing Section 11(2) is amended by paragraph 150(2) of Schedule 7 to the Social Security Act 1998.}, section 20 of the Child Support Act as extended by paragraph 3 of Schedule 4C to that Act\footnote{\frenchspacing Schedule 4C of the Child Support Act 1991 is inserted by paragraph 54 of Schedule 7 to the Social Security Act 1998.} or section 12(2);
\end{enumerate}

“President” means the President of appeal tribunals appointed under section 5;

“referral” means a referral of an application for a departure direction to an appeal tribunal under section 28D(1)($b$) of the Child Support Act\footnote{\frenchspacing Section 28D was inserted by section 4 of the Child Support Act 1995 (c. 34).};

% Definition of ``the Transfer Act'' inserted (5.7.99) by SI 1999/1670 reg 2(2)
“the Transfer Act” means the Social Security Contributions (Transfer of Functions, etc.)\ Act 1999\footnote{\frenchspacing 1999 c. 2.}.
\end{enumerate}

(4) In these Regulations, unless the context otherwise requires, a reference—
\begin{enumerate}\item[]
($a$) to a numbered section is to the section of the Act bearing that number;

($b$) to a numbered Part is to the Part of these Regulations bearing that number;

($c$) to a numbered regulation or Schedule is to the regulation in, or Schedule to, these Regulations bearing that number;

($d$) in a regulation or Schedule to a numbered paragraph is to the paragraph in that regulation or Schedule bearing that number;

($e$) in a paragraph to a lettered or numbered sub-paragraph is to the sub-paragraph in that paragraph bearing that letter or number.
\end{enumerate}

\amendment{
Definition of ``the Transfer Act'' inserted in reg. 1(3) (5.7.99) by the Social Security and Child Support (Decisions and Appeals) Amendment (No. 3) Regulations 1999 reg. 2(2).
}

\subsection[2. Service of notices or documents]{Service of notices or documents}

2.  Where, by any provision of the Act or of these Regulations—
\begin{enumerate}\item[]
($a$) any notice or other document is required to be given or sent to the clerk to the appeal tribunal or to an officer authorised by the Secretary of State, that notice or document shall be treated as having been so given or sent on the day that it is received by the clerk to the appeal tribunal or by an officer authorised by the Secretary of State, as the case may be, and

($b$) any notice (including notification of a decision of the Secretary of State) or other document is required to be given or sent to any person other than the clerk to the appeal tribunal or to an officer authorised by the Secretary of State, as the case may be, that notice or document shall, if sent by post to that person’s last known address, be treated as having been given or sent on the day that it was posted.
\end{enumerate}

\section[Part II --- Revisions, supersessions and other matters Social Security]{Part II --- Revisions, supersessions and other matters Social Security}

\subsection[Chapter I --- Revisions]{Chapter I --- Revisions}

\subsubsection[3. Revision of decisions]{Revision of decisions}

\renewcommand\parthead{--- Part II Chapter I}

3.—(1) Subject to the following provisions of this regulation, any decision of the Secretary of State under section 8 or 10 (“the original decision”) may be revised by him if—
\begin{enumerate}\item[]
%($a$) he commences action leading to the revision within one month of the date of notification of the original decision; or

% Reg 3(1)(a) substituted (18.10.99) by SI 1999/2677 reg 6(a)
($a$) he commences action leading to the revision within one month of the date of—
\begin{enumerate}\item[]
(i) notification of the original decision; or

(ii) the making of an appeal under section 12 provided that the appeal is made within the time prescribed in regulation 31 or, in a case to which regulation 32 applies, the time prescribed in that regulation; or
\end{enumerate}

($b$) an application for a revision is received by the Secretary of State at the appropriate office—
\begin{enumerate}\item[]
(i) within one month of the date of notification of the original decision,

(ii) where a written statement is requested under paragraph (1)($b$) of regulation 28, within 14 days of the expiry of the period specified in head (i), or

(iii) within such longer period of time as may be allowed under regulation 4.
\end{enumerate}
\end{enumerate}

(2) Where the Secretary of State requires further evidence or information from the applicant in order to consider all the issues raised by an application under paragraph (1)($b$) (“the original application”), he shall notify the applicant that further evidence or information is required and the decision may be revised—
\begin{enumerate}\item[]
($a$) where the applicant provides further relevant evidence or information within one month of the date of notification or such longer period of time as the Secretary of State may allow; or

($b$) where the applicant does not provide such evidence or information within the time allowed under sub-paragraph ($a$), on the basis of the original application.
\end{enumerate}

(3) In the case of a payment out of the social fund in respect of maternity or funeral expenses, a decision under section 8 may be revised where the application is made—
\begin{enumerate}\item[]
($a$) within one month of the date of notification of the decision, or if later

($b$) within the time prescribed for claiming such a payment under regulation 19 of, and Schedule 4 to, the Claims and Payments Regulations\footnote{\frenchspacing \emph{See} in particular paragraphs 8 and 9 of Schedule 4 to the Social Security (Claims and Payments) Regulations 1987 (S.I. 1987/1968).}, or

($c$) within such longer period of time as may be allowed under regulation 4.
\end{enumerate}

(4) In the case of a decision made under the Pension Schemes Act 1993\footnote{\frenchspacing 1993 c. 48; section 170 was substituted by paragraph 131 of Schedule 7 to the Social Security Act 1998.} by virtue of section 170(2) of that Act, the decision may be revised at any time by the Secretary of State where it contains an error.

(5) A decision of the Secretary of State under section 8 or 10—
\begin{enumerate}\item[]
($a$) which arose from an official error; or

($b$) 
except in the case of a disability benefit decision or an incapacity benefit decision where there has been an incapacity determination (whether before or after the decision)  % Words inserted (5.7.99) by SI 1999/1623 reg 2(a)
where the decision was made in ignorance of, or was based upon a mistake as to, some material fact and as a result of that ignorance of or mistake as to that fact, the decision was more advantageous to the claimant than it would otherwise have been but for that ignorance or mistake;

% Reg 3(5)(c) inserted (5.7.99) by SI 1999/1623 reg 2(b)
($c$) where the decision is a disability benefit decision, or is an incapacity benefit decision where there has been an incapacity determination (whether before or after the decision), which was made in ignorance or, or was based upon a mistake as to, some material fact in relation to a disability determination embodied in or necessary to the disability benefit decision, or the incapacity determination, and---
\begin{enumerate}\item[]
(i) as a result of that ignorance of or mistake as to that fact the decision was more advantageous to the claimant than it would otherwise have been but for that ignorance or mistake and,

(ii) the Secretary of State is satisfied that at the time the decision was made the claimant or payee knew or could reasonably have been expected at the time the decision was made to know of the fact in question and that it was relevant to the decision,
\end{enumerate}
\end{enumerate}
may be revised at any time by the Secretary of State.

(6) A decision of the Secretary of State under section 8 or 10 that a jobseeker’s allowance is not payable to a claimant for any period in accordance with section 19 of the Jobseekers Act may be revised at any time by the Secretary of State.

(7) A decision under section 8 or 10 may be revised where—
\begin{enumerate}\item[]
($a$) the Secretary of State awards entitlement to a relevant benefit; and
($b$) on the date that entitlement arises, the claimant or a member of his family is entitled to another relevant benefit or to an increase in the rate of another benefit.
\end{enumerate}

(8) A decision of the Secretary of State which is specified in Schedule 2 to the Act or is prescribed in regulation 27 (decisions against which no appeal lies) may be revised at any time.

%(9) Paragraph (1) shall not apply in respect of a relevant change of circumstances which occurred since the decision was made or where the Secretary of State has evidence or information which indicates that a relevant change of circumstances will occur.

% Reg 3(9) substituted (18.10.99) by SI 1999/2677 reg 6(b)
(9) Paragraph (1) shall not apply in respect of—
\begin{enumerate}\item[]
($a$) a relevant change of circumstances which occurred since the decision was made or where the Secretary of State has evidence or information which indicates that a relevant change of circumstances will occur; nor

($b$) a decision which relates to an attendance allowance or a disability living allowance where the person is terminally ill, within the meaning of section 66(2)($a$) of the Contributions and Benefit Act, unless an application for revision which contains an express statement that the person is terminally ill is made either by—
\begin{enumerate}\item[]
(i) the person himself; or

(ii) any other person purporting to act on his behalf whether or not that other person is acting with his knowledge or authority,
\end{enumerate}
but where such an application is received a decision may be so revised notwithstanding that no claim under section 66(1) or, as the case may be, 72(5) or 73(12) of that Act has been made.
\end{enumerate}

(10) The Secretary of State may treat an application for a supersession as an application for a revision.

(11) In this regulation and regulation 7, “appropriate office” means---
\begin{enumerate}\item[]
($a$) the office of the Department of Social Security or the Department for Education and Employment the address of which is indicated on the notification of the original decision; or

($b$) in the case of a person who has claimed jobseeker’s allowance, the office specified by the Secretary of State in accordance with regulation 23 of the Jobseeker’s Allowance Regulations.
\end{enumerate}

\amendment{
Words inserted in reg. 3(5)(b) and reg. 3(5)(c) inserted (5.7.99) by the Social Security and Child Support (Decisions and Appeals) Amendment (No. 2) Regulations 1999 reg. 2.

Reg. 3(1)(a), (9) substituted (18.10.99) by the Social Security and Child Support (Decisions and Appeals), Vaccine Damage Payments and Jobseeker's Allowance (Amendment) Regulations 1999 reg. 6.
}

\subsubsection[4. Late application for a revision]{Late application for a revision}

4.—(1) The time limit for making an application for a revision specified in regulation 3(1) or (3) may be extended where the conditions specified in the following provisions of this regulation are satisfied.

(2) An application for an extension of time shall be made by the claimant or a person acting on his behalf.

(3) An application shall—
\begin{enumerate}\item[]
($a$) contain particulars of the grounds on which the extension of time is sought and shall contain sufficient details of the decision which it is sought to have revised to enable that decision to be identified; and

($b$) be made within 13 months of the date of notification of the decision which it is sought to have revised.
\end{enumerate}

(4) An application for an extension of time shall not be granted unless the applicant satisfies the Secretary of State that—
\begin{enumerate}\item[]
($a$) it is reasonable to grant the application;

($b$) the application for revision has merit; and

($c$) special circumstances are relevant to the application and as a result of those special circumstances it was not practicable for the application to be made within the time limit specified in regulation 3.
\end{enumerate}

(5) In determining whether it is reasonable to grant an application, the Secretary of State shall have regard to the principle that the greater the amount of time that has elapsed between the expiration of the time specified in regulation 3(1) and (3) for applying for a revision and the making of the application for an extension of time, the more compelling should be the special circumstances on which the application is based.

(6) In determining whether it is reasonable to grant the application for an extension of time, no account shall be taken of the following—
\begin{enumerate}\item[]
($a$) that the applicant or any person acting for him was unaware of or misunderstood the law applicable to his case (including ignorance or misunderstanding of the time limits imposed by these Regulations); or

($b$) that a Commissioner or a court has taken a different view of the law from that previously understood and applied.
\end{enumerate}

(7) An application under this regulation for an extension of time which has been refused may not be renewed.

\subsubsection[5. Date from which a decision revised under section 9 takes effect]{Date from which a decision revised under section 9 takes effect}

5.  Where, on a revision under section 9, the Secretary of State decides that the date from which the decision under section 8 or 10 (“the original decision”) took effect was erroneous, the decision under section 9 shall take effect on the date from which the original decision would have taken effect had the error not been made.

\subsection[Chapter II --- Supersessions]{Chapter II\\*Supersessions}

\subsubsection[6. Supersession of decisions]{Supersession of decisions}

\renewcommand\parthead{--- Part II Chapter II}

6.—(1) Subject to the following provisions of this regulation, for the purposes of section 10, the cases and circumstances in which a decision may be superseded under that section are set out in paragraphs (2) to (4).

(2) A decision under section 10 may be made on the Secretary of State’s own initiative or on an application made for the purpose on the basis that the decision to be superseded—
\begin{enumerate}\item[]
($a$) is one in respect of which—
\begin{enumerate}\item[]
(i) there has been a relevant change of circumstances since the decision was made; or

(ii) it is anticipated that a relevant change of circumstances will occur;
\end{enumerate}

($b$) is a decision of the Secretary of State other than a decision to which sub-paragraph ($d$) refers and—
\begin{enumerate}\item[]
(i) the decision was erroneous in point of law, or it was made in ignorance of, or was based upon a mistake as to, some material fact; and

(ii) an application for a supersession was received by the Secretary of State, or the decision by the Secretary of State to act on his own initiative was taken, more than one month after the date of notification of the decision which is to be superseded or after the expiry of such longer period of time as may have been allowed under regulation 4;
\end{enumerate}

($c$) is a decision of an appeal tribunal or of a Commissioner that was made in ignorance of, or was based upon a mistake as to, some material fact;

($d$) is a decision which is specified in Schedule 2 to the Act or is prescribed in regulation 27 (decisions against which no appeal lies); or

($e$) is a decision where—
\begin{enumerate}\item[]
(i) the Secretary of State has awarded a relevant benefit; and

(ii) on a date after the date that entitlement arises, the claimant or a member of his family becomes entitled to another relevant benefit or to an increase in the rate of another such benefit;
\end{enumerate}

%($f$) is a decision of the Secretary of State that a jobseekers allowance is payable to a claimant where the Secretary of State subsequently determines that the allowance is not payable in accordance with section 19 of the Jobseekers Act;

% Reg 6(2)(f) substituted (18.10.99) by SI 1999/2677 reg 7(a)
($f$) is a decision that a jobseeker’s allowance is payable to a claimant where that allowance ceases to be payable by virtue of section 19(1) of the Jobseekers Act;

% Reg 6(2)(g) inserted (5.7.99) by SI 1999/1623 reg 3
($g$) is an incapacity benefit decision where there has been an incapacity determination (whether before or after the decision) and where, since the decision was made, the Secretary of State has received medical evidence following an examination in accordance with regulation 8 of the Social Security (Incapacity for Work) (General) Regulations 1995\footnote{\frenchspacing S.I. 1995/311; relevant amending instruments are S.I. 1995/987, 1996/3207 and 1997/1009.} from a doctor referred to in paragraph (1) of that regulation.
\end{enumerate}

(3) A decision which may be revised under regulation 3 may not be superseded under this regulation except where—
\begin{enumerate}\item[]
($a$) circumstances arise in which the Secretary of State may revise that decision under regulation 3; and

($b$) further circumstances arise in relation to that decision which are not specified in regulation 3 but are specified in paragraph (2) or (4).
\end{enumerate}

(4) Where the Secretary of State requires further evidence or information from the applicant in order to consider all the issues raised by an application under paragraph (2) (“the original application”), he shall notify the applicant that further evidence or information is required and the decision may be superseded—
\begin{enumerate}\item[]
($a$) where the applicant provides further relevant evidence or information within one month of the date of notification or such longer period of time as the Secretary of State may allow; or

($b$) where the applicant does not provide such evidence or information within the time allowed under sub-paragraph ($a$), on the basis of the original application.
\end{enumerate}

(5) The Secretary of State may treat an application for a revision or a notification of a change of circumstances as an application for a supersession.

(6) The following events are not relevant changes of circumstances for the purposes of paragraph (2)—
\begin{enumerate}\item[]
($a$) the repayment of a loan to which regulation 66A of the Income Support Regulations\footnote{\frenchspacing Regulation 66A was inserted by S.I. 1990/1549; relevant amending instruments are S.I. 1991/236, S.I. 1991/1559 and S.I. 1996/462.} or regulation 136 of the Jobseeker’s Allowance Regulations applies;

($b$) the absence from a nursing home or residential care home for a period of less than one week of a resident who is—
\begin{enumerate}\item[]
(i) in receipt of income support or a jobseeker’s allowance; and

(ii) not a claimant to whom Part II of Schedule 4 to the Income Support Regulations applies;
\end{enumerate}

% Reg 6(6)(c) inserted (18.10.99) by SI 1999/2677 reg 7(b)
($c$) the fact that a person has become terminally ill, within the meaning of section 66(2)($a$) of the Contributions and Benefits Act, unless an application for supersession which contains an express statement that the person is terminally ill is made either by—
\begin{enumerate}\item[]
(i) the person himself; or

(ii) any other person purporting to act on his behalf whether or not that other person is acting with his knowledge or authority;
\end{enumerate}
and where such an application is received a decision may be so superseded nothwithstanding that no claim under section 66(1) or, as the case may be, 72(5) or 73(12) of that Act has been made.
\end{enumerate}

(7) In paragraph (6)($b$), “nursing home” and “residential care home” have the same meanings as they have in regulation 19 of the Income Support Regulations.

\amendment{
Reg. 6(2)(g) inserted (5.7.99) by the Social Security and Child Support (Decisions and Appeals) Amendment (No. 2) Regulations 1999 reg. 3.

Reg. 6(6)(c) inserted and reg. 6(2)(f) substituted (18.10.99) by the Social Security and Child Support (Decisions and Appeals), Vaccine Damage Payments and Jobseeker's Allowance (Amendment) Regulations 1999 reg. 7.
}

\subsubsection[7. Date from which a decision superseded under section 10 takes effect]{Date from which a decision superseded under section 10 takes effect}

7.—%(1) This regulation contains exceptions to the provisions of section 10(5) as to the date from which a decision under section 10 which supersedes an earlier decision is to take effect.
%
% Reg 7(1) substituted (29.11.99) by SI 1999/3178 Sch 19 para 1(a)
(1) This regulation---
\begin{enumerate}\item[]
($a$) is, except for paragraph (2)($b$), subject to regulations 26\footnote{\frenchspacing Regulation 26 was amended by S.I. 1988/522, 1989/136 and 1993/1113.} (income support) and 26A\footnote{\frenchspacing Regulation 26A was inserted by S.I. 1996/1460 and amended by S.I. 1998/1174.} (jobseeker’s allowance) of, and paragraph 7\footnote{\frenchspacing Paragraph 7 was substituted by S.I. 1990/2208 and amended by S.I. 1991/387, 1992/247 and 1998/1174.} (date from which superseding decision on ground of change of circumstances takes effect) of Schedule 7 to, the Claims and Payments Regulations; and

($b$) contains exceptions to the provisions of section 10(5) as to the date from which a decision under section 10 which supersedes an earlier decision is to take effect.
\end{enumerate}

(2) Where a decision under section 10 is made on the ground that there has been, or it is anticipated that there will be, a relevant change of circumstances since the decision was made, the decision under section 10 shall take effect—
\begin{enumerate}\item[]
%($a$) where the decision is advantageous to the claimant and the change was notified to an appropriate office within one month of the change occurring or within such longer period as may be allowed under regulation 8 for the claimant’s failure to notify the change on an earlier date—
%\begin{enumerate}\item[]
%(i) subject to head (ii), from the date the change occurred or, where the change does not have effect until a later date, from the first date on which such effect occurs;
%
%(ii) in a case where the date a change of circumstances is to take effect falls to be determined in accordance with regulation 26 or 26A\footnote{\frenchspacing Regulation 26A was inserted by the Social Security (Claims and Payments) (Jobseeker’s Allowance Consequential Amendments) Regulations 1996.} of the Claims and Payments Regulations, the date so determined;
%\end{enumerate}

% Reg 7(2)(a) substituted (29.11.99) by SI 1999/3178 Sch 19 para 1(b)(i)
($a$) from the date the change occurred or, where the change does not have effect until a later date, from the first date on which such effect occurs where---
\begin{enumerate}\item[]
(i) the decision is advantageous to the claimant; and

(ii) the change was notified to an appropriate office within one month of the change occurring or within such longer period as may be allowed under regulation 8 for the claimant’s failure to notify the change on an earlier date;
\end{enumerate}

($b$) where the decision is advantageous to the claimant and the change was notified to an appropriate office more than one month after the change occurred or after the expiry of any such longer period as may have been allowed under regulation 8—
\begin{enumerate}\item[]
(i) in the case of a claimant who is in receipt of income support or a jobseeker’s allowance and benefit is paid in arrears, from the beginning of the benefit week in which the notification was made;

(ii) in the case of a claimant who is in receipt of income support or a jobseeker’s allowance and benefit is paid in advance and the date of notification is the first day of a benefit week from that date and otherwise, from the beginning of the benefit week following the week in which the notification was made; or

(iii) in any other case, the date of notification of the relevant change of circumstances; or
\end{enumerate}

($c$) where the decision is not advantageous to the claimant—
\begin{enumerate}\item[]
% Reg 7(2)(c)(i) omitted (29.11.99) by SI 1999/3178 Sch 19 para 1(b)(ii)
%(i) in a case where the date a change of circumstances is to take effect falls to be determined in accordance with regulation 26 or 26A of the Claims and Payments Regulations, the date so determined; or

%(ii) in any other case, from the date of the change.

% Reg 7(2)(c)(ii), (iii) substituted for reg 7(2)(c)(ii) (5.7.99) by SI 1999/1623 reg 4
(ii) in the case of a disability benefit decision, or an incapacity benefit decision where there has been an incapacity determination (whether before or after the decision), where the Secretary of State is satisfied that in relation to a disability determination embodied in or necessary to the disability benefit decision, or the incapacity determination, the claimant or payee failed to notify an appropriate office of a change of circumstances which regulations under the Administration Act required him to notify, and the claimant or payee, as the case may be, knew or could reasonably have been expected to know that the change of circumstances should have been notified---
\begin{enumerate}\item[]
($aa$) from the date on which the claimant or payee, as the case may be, ought to have notified the change of circumstances, or

($bb$) if more than one change has taken place between the date from which the decision to be superseded took effect and the date of the superseding decision, from the date on which the first change ought to have been notified, or
\end{enumerate}

(iii) in any other case, except in the case of a decision which supersedes a disability benefit decision, or an incapacity benefit decision where there has been an incapacity determination (whether before or after the decision), from the date of the change.
\end{enumerate}
\end{enumerate}

(3) For the purposes of paragraphs (2) and (8) “benefit week” has the same meaning as in regulation 2(1) of the Income Support Regulations or, as the case may be, regulation 1(3) of the Jobseeker’s Regulations.

(4) In paragraph (2) a decision which is to the advantage of the claimant includes a decision specified in regulation 30(2)($a$) to ($f$).

(5) Where the Secretary of State supersedes a decision made by an appeal tribunal or a Commissioner on the grounds specified in regulation 6(2)($c$) (grounds of ignorance of, or mistake as to, a material fact), the decision under section 10 shall take effect—
\begin{enumerate}\item[]
($a$) in a case where, as a result of that ignorance of or mistake as to some material fact, the decision was more advantageous to the claimant than it would otherwise have been but for that ignorance or mistake, from the date on which the decision of the appeal tribunal or the Commissioner took, or was to take effect; or

($b$) in any other case, from the date of the decision under section 10.
\end{enumerate}

(6) Any decision made under section 10 in consequence of a decision which is a relevant determination for the purposes of section 27 shall take effect as from the date of the relevant determination.

(7) A decision to which regulation 6(2)($e$) applies may be made so as to take effect as from the date on which the decision which has been superseded had effect, or at any time thereafter which is reasonable in the particular circumstances of the case.

%(8) A decision to which regulation 6(2)($f$) applies may be so as to take effect from—
%\begin{enumerate}\item[]
%($a$) except where sub-paragraph ($b$) applies, the date immediately following the end of benefit week in which the decision under section 10 was made; or
%
%($b$) where in accordance with regulation 26A(1) of the Claims and Payments Regulations, a jobseeker’s allowance is paid otherwise than fortnightly in arrears, and notwithstanding the provisions of regulation 69 of the Jobseeker’s Allowance Regulations, from the day immediately following the end of the last benefit week in respect of which a jobseeker’s allowance was paid.
%\end{enumerate}

% Reg 7(8) substituted (18.10.99) by SI 1999/2677 reg 8
(8) A decision to which regulation 6(2)($f$)  applies shall take effect—
\begin{enumerate}\item[]
($a$) where section 19(2) of the Jobseekers Act applies, as from the beginning of the period specified in regulation 69 of the Jobseeker’s Allowance Regulations; or

($b$) where section 19(3) of the Jobseekers Act applies, as from the beginning of the period determined in accordance with that subsection.
\end{enumerate}

%(9) In any case relating to attendance allowance or disability living allowance in which the decision was made under section 10 on the grounds of a relevant change of circumstances by virtue of regulation 6(2)($a$)(i) and the decision is advantageous to the claimant, the decision shall take effect as from whichever is the later of—
%\begin{enumerate}\item[]
%($a$) the date declared by the Secretary of State to be the date on which the change of circumstances occurred;
%
%($b$) where more than one change has occurred between the date of the decision to be superseded (“the original decision”) and the date of the application, or, as the case may be, the date the Secretary of State determines on his own initiative to supersede the original decision, the date declared by the Secretary of State to be the date on which the most recent change took effect; or
%
%($c$) where the claimant notifies the change within one month of the date he first satisfies the conditions in, for the period of time specified in, section 65(1)($b$) of the Contributions and Benefits Act or, as the case may be, section 72(2)($a$) or 73(9)($a$) of that Act, following the change or most recent change of circumstances which gave rise to the decision under section 10, the first pay day (as specified in Schedule 6 to the Claims and Payments Regulations) after the requirement is first satisfied.
%\end{enumerate}

% Reg 7(9) substituted (17.2.00) by SI 2000/119 reg 2
(9) A decision relating to attendance allowance or disability living allowance which is advantageous to the claimant and which is made under section 10 on the basis of a relevant change of circumstances shall take effect from—
\begin{enumerate}\item[]
($a$) where the decision is made on the Secretary of State’s own initiative, the date of that decision;

($b$) where—
\begin{enumerate}\item[]
(i) the change is relevant to the question of entitlement to a particular rate of benefit; and

(ii) the claimant notifies the change before a date one month after he satisfied the conditions of entitlement to that rate or within such longer period as may be allowed under regulation 8,
\end{enumerate}
the first pay day (as specified in Schedule 6 to the Claims and Payments Regulations\footnote{\frenchspacing S.I. 1987/1968; the relevant amending instrument is S.I. 1991/2741.}) after he satisfied those conditions;

($c$) where—
\begin{enumerate}\item[]
(i) the change is relevant to the question of whether benefit is payable; and

(ii) the claimant notifies the change before a date one month after the change or within such longer period as may be allowed under regulation 8,
\end{enumerate}
the first pay day (as specified in Schedule 6 to the Claims and Payments Regulations) after the change occurred; or

($d$) in any other case, the date of the application for the superseding decision.
\end{enumerate}

(10) A decision as to an award of incapacity benefit, which is made under section 10 because section 30B(4) of the Contributions and Benefits Act\footnote{\frenchspacing Section 30B was inserted by section 2(1) of the Social Security (Incapacity for Work) Act 1994 (c. 18).} applies to the claimant, shall take effect as from the date on which he became entitled to the highest rate of the care component of disability living allowance.

(11) A decision as to an award of incapacity benefit or severe disablement allowance, which is made under section 10 because the claimant is to be treated as incapable of work under regulation 10 of the Social Security (Incapacity for Work) (General) Regulations 1995\footnote{\frenchspacing S.I. 1995/311; relevant amending instruments are S.I. 1995/987, S.I. 1996/3207 and S.I. 1997/1009.} (certain persons with a severe condition to be treated as incapable of work), shall take effect as from the date he is to be treated as incapable of work.

(12) Where this paragraph applies, a decision under section 10 may be made so as to take effect as from such date not more than eight weeks before—
\begin{enumerate}\item[]
($a$) the application for supersession; or

($b$) where no application is made, the date on which the decision under section 10 is made,
\end{enumerate}
as is reasonable in the particular circumstances of the case.

(13) Paragraph (12) applies where—
\begin{enumerate}\item[]
($a$) the effect of a decision under section 10 is that there is to be included in a claimant’s applicable amount an amount in respect of a loan which qualifies under—
\begin{enumerate}\item[]
(i) paragraph 15 or 16 of Schedule 3 to the Income Support Regulations; or

(ii) paragraph 14 or 15 of Schedule 2 to the Jobseeker’s Allowance Regulations; and
\end{enumerate}

($b$) that decision could not have been made earlier because information necessary to make that decision, requested otherwise than in accordance with paragraph 10(3)($b$) of Schedule 9A to the Claims and Payments Regulations\footnote{\frenchspacing Schedule 9A was inserted by S.I. 1992/1026.} (annual requests for information), had not been supplied to the Secretary of State by the lender.
\end{enumerate}

(14) Subject to paragraph (23), where a claimant is in receipt of income support and his applicable amount includes an amount determined in accordance with Schedule 3 to the Income Support Regulations (housing costs), and there is a reduction in the amount of eligible capital owing in connection with a loan which qualifies under paragraph 15 or 16 of that Schedule, a decision made under section 10 shall take effect—
\begin{enumerate}\item[]
($a$) on the first anniversary of the date on which the claimant’s housing costs were first met under that Schedule; or

($b$) where the reduction in eligible capital occurred after the first anniversary of the date referred to in sub-paragraph ($a$), on the next anniversary of that date following the date of the reduction.
\end{enumerate}

(15) Where a claimant is in receipt of income support and payments made to that claimant which fall within paragraph 29 or 30(1)($a$) to ($c$) of Schedule 9 to the Income Support Regulations have been disregarded in relation to any decision under section 8 or 10 and there is a change in the amount of interest payable—
\begin{enumerate}\item[]
($a$) on a loan qualifying under paragraph 15 or 16 of Schedule 3 to those Regulations to which those payments relate; or

($b$) on a loan not so qualifying which is secured on the dwelling occupied as the home to which those payments relate,
\end{enumerate}
a decision under section 10 which is made as a result of that change in the amount of interest payable shall take effect on whichever of the dates referred to in paragraph (16) is appropriate in the claimant’s case.

(16) The date on which a decision under section 10 takes effect for the purposes of paragraph (15) is—
\begin{enumerate}\item[]
($a$) the date on which the claimant’s housing costs are first met under paragraph 6(1)($a$), 8(1)($a$) or 9(2)($a$) of Schedule 3 to the Income Support Regulations; or

($b$) where the change in the amount of interest payable occurred after the date referred to in sub-paragraph ($a$), on the date of the next alteration in the standard rate following the date of that change.
\end{enumerate}

(17) In paragraph (16), “standard rate” has the same meaning as it has in paragraph 1(2) of Schedule 3 to the Income Support Regulations.

(18) Subject to paragraph (24) and, except in a case to which paragraph (23) applies, where a claimant is in receipt of a jobseeker’s allowance and his applicable amount includes an amount determined in accordance with Schedule 2 to the Jobseeker’s Allowance Regulations (housing costs), and there is a reduction in the amount of eligible capital owing in connection with a loan which qualifies under paragraph 14 or 15 of that Schedule, a decision under section 10 made as a result of that reduction shall take effect—
\begin{enumerate}\item[]
($a$) on the first anniversary of the date on which the claimant’s housing costs were first met under that Schedule; or

($b$) where the reduction in eligible capital occurred after the first anniversary of the date referred to in sub-paragraph ($a$), on the next anniversary of that date following the date of the reduction.
\end{enumerate}

(19) Where a claimant is in receipt of a jobseeker’s allowance and payments made to that claimant which fall within paragraph 30 or 31(1)($a$) to ($c$) of Schedule 7 to the Jobseeker’s Allowance Regulations have been disregarded in relation to any decision under section 8 or 10 and there is a change in the amount of interest payable—
\begin{enumerate}\item[]
($a$) on a loan qualifying under paragraph 14 or 15 of Schedule 2 to those Regulations to which those payments relate; or

($b$) on a loan not so qualifying which is secured on the dwelling occupied as the home to which those payments relate,
\end{enumerate}
any decision under section 10 which is made as a result of that change in the amount of interest payable shall take effect on whichever of the dates referred to in paragraph (20) is appropriate in the claimant’s case.

(20) The date on which a decision under section 10 takes effect for the purposes of paragraph (19) is—
\begin{enumerate}\item[]
($a$) the date on which the claimant’s housing costs are first met under paragraph 6(1)($a$), 7(1)($a$) or 8(2)($a$) of Schedule 2 to the Jobseeker’s Allowance Regulations; or

($b$) where the changes in the amount of interest payable occurred after the date referred to in sub-paragraph ($a$), on the date of the next alteration in the standard rate following the date of that change.
\end{enumerate}

(21) In paragraph (20), “standard rate” has the same meaning as it has in paragraph 1(2) of Schedule 2 to the Jobseeker’s Allowance Regulations.

(22) Where—
\begin{enumerate}\item[]
($a$) a claimant was paid benefit in respect of 6th October 1996 in accordance with an award of income support;

($b$) that claimant’s applicable amount includes an amount determined in accordance with Schedule 3 to the Income Support Regulations (housing costs);

($c$) that claimant is treated as having been awarded a jobseeker’s allowance by virtue of regulation 7 of the Jobseeker’s Allowance (Transitional Provisions) Regulations 1996\footnote{\frenchspacing S.I. 1996/2567.} (jobseeker’s allowance to replace income support and unemployment benefit); and

($d$) a decision is made under section 10 in consequence of a reduction in the amount of eligible capital owing in connection with a loan which qualifies under paragraph 15 or 16 of Schedule 3 to the Income Support Regulations,
\end{enumerate}
the decision under section 10 referred to in sub-paragraph ($d$) shall take effect on the next anniversary of the date on which housing costs were first met which occurs after the reduction.

(23) Where, in any case to which paragraph (14) or (18) applies, a claimant has been continuously in receipt of, or treated as having been continuously in receipt of income support or a jobseeker’s allowance, or one of those benefits followed by the other, and he or his partner continues to receive either benefit, the anniversary to which those paragraphs refer shall be the anniversary of the earliest date on which benefit (whether income support or a jobseeker’s allowance) in respect of those mortgage interest costs became payable.

(24) Where—
\begin{enumerate}\item[]
($a$) it has been determined that the amount of a jobseeker’s allowance payable to a young person is to be reduced under regulation 63 of the Jobseeker’s Allowance Regulations because paragraph (1)($b$)(iii), ($c$), ($d$), ($e$) or ($f$) of that regulation (reduced payments under section 17 of the Jobseekers Act) applied in his case; and

($b$) the decision made in consequence of sub-paragraph ($a$) falls to be superseded by a decision under section 10 because the Secretary of State has subsequently issued a certificate under section 17(4) of the Jobseekers Act with respect to the failure in question,
\end{enumerate}
the decision under section 10 shall take effect as from the same date as the decision made in consequence of sub-paragraph ($a$) has effect.

\amendment{
Reg. 7(2)(c)(ii), (iii) substituted for reg. 7(2)(c)(ii) (5.7.99) by the Social Security and Child Support (Decisions and Appeals) Amendment (No. 2) Regulations 1999 reg. 4.

Reg. 7(8) substituted (18.10.99) by the Social Security and Child Support (Decisions and Appeals), Vaccine Damage Payments and Jobseeker's Allowance (Amendment) Regulations 1999 reg. 8.

Reg. 7(1), (2)(a) substituted and reg. 7(2)(c)(i) omitted (29.11.99) by the Social Security Act 1998 (Commencement No. 12 and Consequential and Transitional Provisions) Order 1999 Sch. 19 para. 1.

Reg. 7(9) substituted (17.2.00) by the Social Security and Child Support (Decisions and Appeals) Amendment Regulations 2000 reg. 2.
}

% Reg 7A inserted (5.7.99) by SI 1999/1623 reg 5
\subsubsection[7A. Definitions for the purposes of regulations 3(5)($c$), 6(2)($g$) and 7(2)($c$) and ancillary provisions]{Definitions for the purposes of regulations 3(5)($c$), 6(2)($g$) and 7(2)($c$) and ancillary provisions}

7A.---(1)  For the purposes of regulations 3(5)($c$), 6(2)($g$) and 7(2)($c$)---
\begin{enumerate}\item[]
“disability benefit decision” means a decision to award a relevant benefit embodied in or necessary to which is a disability determination,

“disability determination” means–
\begin{enumerate}\item[]
($a$)
in the case of a decision as to an award of an attendance allowance or a disability living allowance, whether the person satisfies any of the conditions in section 64, 72(1) or 73(1) to (3), as the case may be, of the Contributions and Benefits Act,

($b$)
in the case of a decision as to an award of severe disablement allowance, whether the person is disabled for the purpose of section 68 of the Contributions and Benefits Act, or

($c$)
in the case of a decision as to an award of industrial injuries benefit, whether the existence or extent of any disablement is sufficient for the purposes of section 103 or 108 of the Contributions and Benefits Act or for the benefit to be paid at the rate which was in payment immediately prior to that decision;
\end{enumerate}

“incapacity benefit decision” means a decision to award a relevant benefit embodied in or necessary to which is a determination that a person is or is to be treated as incapable of work under Part XIIA of the Contributions and Benefits Act,

\begin{sloppypar}
“incapacity determination” means a determination whether a person is incapable of work by applying the all work test in regulation 24 of the Social Security (Incapacity for Work) (General) Regulations 1995 or whether a person is to be treated as incapable of work in accordance with regulation 10 (certain persons with a severe condition to be treated as incapable of work) or 27 (exceptional circumstances) of those Regulations, and
\end{sloppypar}

“payee” means a person to whom a benefit referred to in paragraph ($a$), ($b$) or ($c$) of the definition of “disability determination”, or a benefit referred to in the definition of “incapacity benefit decision” is payable.
\end{enumerate}

(2) Where a person’s receipt of or entitlement to a benefit (“the first benefit”) is a condition of his being entitled to any other benefit, allowance or advantage (“a second benefit”) and a decision is revised under regulation 3(5)($c$) or a superseding decision is made under regulation 6(2) to which regulation 7(2)($c$)(ii) applies, the effect of which is that the first benefit ceases to be payable, or becomes payable at a lower rate than was in payment immediately prior to that revision or supersession, a consequent decision as to his entitlement to the second benefit shall take effect from the date of the change in his entitlement to the first benefit.

\amendment{
Reg. 7A inserted (5.7.99) by the Social Security and Child Support (Decisions and Appeals) Amendment (No. 2) Regulations 1999 reg. 5.
}

\subsubsection[8. Effective date for late notifications of change of circumstances]{Effective date for late notifications of change of circumstances}

8.—(1) For the purposes of regulation 7(2)
and (9)% Words inserted (17.2.00) by SI 2000/119 reg 3(a)
, a longer period of time may be allowed for the notification of a change of circumstances in so far as it affects the effective date of the change where the conditions specified in the following provisions of this regulation are satisfied.

(2) An application for the purposes of regulation 7(2) 
or (9)  % Words inserted (17.2.00) by SI 2000/119 reg 3(b)
shall be made by the claimant or a person acting on his behalf.

(3) The application referred to in paragraph (2) shall—
\begin{enumerate}\item[]
($a$) contain particulars of the relevant change of circumstances and the reasons for the failure to notify the change of circumstances on an earlier date; and

($b$) be made within 13 months of the date the change occurred.
\end{enumerate}

(4) An application under this regulation shall not be granted unless the Secretary of State is satisfied that—
\begin{enumerate}\item[]
($a$) it is reasonable to grant the application;

($b$) the change of circumstances notified by the applicant is relevant to the decision which is to be superseded; and

($c$) special circumstances are relevant to the application and as a result of those special circumstances it was not practicable for the applicant to notify the change of circumstances within one month of the change occurring.
\end{enumerate}

(5) In determining whether it is reasonable to grant the application, the Secretary of State shall have regard to the principle that the greater the amount of time that has elapsed between the date one month after the change of circumstances occurred and the date the application for the purposes of regulation 7(2) 
or (9)  % Words inserted (17.2.00) by SI 2000/119 reg 3(b)
is made, the more compelling should be the special circumstances on which the application is based.

(6) In determining whether it is reasonable to grant an application, no account shall be taken of the following—
\begin{enumerate}\item[]
($a$) that the applicant or any person acting for him was unaware of, or misunderstood, the law applicable to his case (including ignorance or misunderstanding of the time limits imposed by these Regulations); or

($b$) that a Commissioner or a court has taken a different view of the law from that previously understood and applied.
\end{enumerate}

(7) An application under this regulation which has been refused may not be renewed.

\amendment{
Words inserted in reg. 8(1), (2), (5) (17.2.00) by the Social Security and Child Support (Decisions and Appeals) Amendment Regulations 2000 reg. 3.
}

\subsection[Chapter III --- Other matters]{Chapter III\\*Other matters}

\subsubsection[9. Certificates of recoverable benefits]{Certificates of recoverable benefits}

\renewcommand\parthead{--- Part II Chapter III}

9.  A certificate of recoverable benefits may be reviewed under section 10 of the 1997 Act\footnote{\frenchspacing Section 10 was amended by paragraph 149 of Schedule 7 to the Social Security Act 1998.} where the Secretary of State is satisfied that—
\begin{enumerate}\item[]
($a$) a mistake (whether in computation of the amount specified or otherwise) occurred in the preparation of the certificate;

($b$) the benefit recovered from a person who makes a compensation payment (as defined in section 1 of the 1997 Act) is in excess of the amount due to the Secretary of State;

($c$) incorrect or insufficient information was supplied to the Secretary of State by the person who applied for the certificate and in consequence the amount of benefit specified in the certificate was less than it would have been had the information supplied been correct or sufficient; or

($d$) a ground for appeal is satisfied under section 11 of the 1997 Act\footnote{\frenchspacing Section 11 was amended by paragraph 150 of Schedule 7 to the Social Security Act 1998.}.
\end{enumerate}

\subsubsection[10. Effect of a determination as to capacity for work]{Effect of a determination as to capacity for work}

10.  A determination (including a determination made following a change of circumstances) whether a person is, or is to be treated as, capable or incapable of work which is embodied in or necessary to a decision under Chapter II of Part I of the Act or on which such a decision is based shall be conclusive for the purposes of any further such decision.

\subsubsection[11. Secretary of State to determine certain matters]{Secretary of State to determine certain matters}

11.  Where, in relation to a determination for any purpose to which Part XIIA of the Contributions and Benefits Act applies, an issue arises as to—
\begin{enumerate}\item[]
($a$) whether a person is, or is to be treated as, capable or incapable of work in respect of any period; or

($b$) whether a person is terminally ill,
\end{enumerate}
that issue shall be determined by the Secretary of State, notwithstanding that other matters fall to be determined by another authority.

%Reg 11A inserted (5.7.99) by SI 1999/1670 reg 2(3)
\subsubsection[11A. Issues for decision by officers of Inland Revenue]{Issues for decision by officers of Inland Revenue}

11A.---(1)  Where, on consideration of any claim or other matter, it appears to the Secretary of State that an issue arises which, by virtue of section 8 of the Transfer Act, falls to be decided by an officer of the Board, he shall refer that issue to the Board.

(2) Where—
\begin{enumerate}\item[]
($a$) the Secretary of State has decided any claim or other matter on an assumption of facts—
\begin{enumerate}\item[]
(i) as to which there appeared to him to be no dispute, but

(ii) concerning which, had an issue arisen, that issue would have fallen, by virtue of section 8 of the Transfer Act, to be decided by an officer of the Board; and
\end{enumerate}

($b$) an application for revision or an application for supersession is made in relation to the decision of that claim or other matter; and

($c$) it appears to the Secretary of State on consideration of the application that such an issue arises,
\end{enumerate}
he shall refer that issue to the Board.

(3) Pending the final decision of any issue which has been referred to the Board in accordance with paragraph (1) or (2) above, the Secretary of State may—
\begin{enumerate}\item[]
($a$) determine any other issue arising on consideration of the claim or other matter or, as the case may be, of the application,

($b$) seek a preliminary opinion of the Board on the issue referred and decide the claim or other matter or, as the case may be, the application in accordance with that opinion on that issue; or

($c$) defer making any decision on the claim or other matter or, as the case may be, the application.
\end{enumerate}

(4) On receipt by the Secretary of State of the final decision of an issue which has been referred to the Board in accordance with paragraph (1) or (2) above, the Secretary of State shall—
\begin{enumerate}\item[]
($a$) in a case to which paragraph (3)($b$) above applies—
\begin{enumerate}\item[]
(i) consider whether the decision ought to be revised under section 9 or superseded under section 10, and

(ii) if so, revise it, or, as the case may be, make a further decision which supersedes it; or
\end{enumerate}

($b$) in a case to which paragraph (3)($a$) or ($c$) above applies, decide the claim or other matter or, as the case may be, the application,
\end{enumerate}
in accordance with the final decision of the issue so referred.

(5) In paragraphs (3) and (4) above “final decision” means the decision of an officer of the Board under section 8 of the Transfer Act or the determination of any appeal in relation to that decision.

\amendment{
Reg. 11A inserted (5.7.99) by the Social Security and Child Support (Decisions and Appeals) Amendment (No. 3) Regulations 1999 reg. 2(3).
}

\subsubsection[12. Decision of the Secretary of State relating to industrial injuries benefit]{Decision of the Secretary of State relating to industrial injuries benefit}

12.—(1) This regulation applies where, for the purpose of a decision of the Secretary of State relating to a claim for industrial injuries benefit under Part V of the Contributions and Benefits Act an issue to be decided is—
\begin{enumerate}\item[]
($a$) the extent of a personal injury for the purposes of section 94 of that Act;

($b$) whether the claimant has a disease prescribed for the purposes of section 108 of that Act or the extent of any disablement resulting from such a disease; or

($c$) whether the claimant has a disablement for the purposes of section 103 of that Act or the extent of any such disablement.
\end{enumerate}

(2) In connection with making a decision to which this regulation applies, the Secretary of State may refer an issue, together with any relevant evidence or information available to him, including any evidence or information provided by or on behalf of the claimant, to a medical practitioner who has experience in such of the issues specified in paragraph (1) as are relevant to the decision, for such report as appears to the Secretary of State to be necessary for the purpose of providing him with information for use in making the decision.

(3) In making a decision to which this regulation applies, the Secretary of State shall have regard to (among other factors)—
\begin{enumerate}\item[]
($a$) all relevant medical reports provided to him in connection with that decision; and

($b$) the experience, in such of the issues specified in paragraph (1) as are relevant to the decision, of any medical practitioner who has provided a report, including a medical practitioner who has provided a report following an examination required by the Secretary of State under section 19.
\end{enumerate}

\subsubsection[13. Income support and social fund determinations on incomplete evidence]{\sloppy Income support and social fund determinations on incomplete evidence}

13.—(1) Where, for the purpose of a decision under section 8 or 10—
\begin{enumerate}\item[]
($a$) a determination falls to be made by the Secretary of State as to what housing costs are to be included in a claimant’s applicable amount by virtue of regulation 17(1)($e$) or 18(1)($f$) of, and Schedule 3 to, the Income Support Regulations; and

($b$) it appears to the Secretary of State that he is not in possession of all of the evidence or information which is relevant for the purposes of such a determination,
\end{enumerate}
he shall make the determination on the assumption that the housing costs to be included in the claimant’s applicable amount are those that can be immediately determined.

(2) Where, for the purpose of a decision under section 8 or 10—
\begin{enumerate}\item[]
($a$) a determination falls to be made by the Secretary of State as to whether—
\begin{enumerate}\item[]
(i) in relation to any person, the applicable amount falls to be reduced or disregarded to any extent by virtue of section 126(3) of the Contributions and Benefits Act (persons affected by trade disputes);

(ii) for the purposes of regulation 12 of the Income Support Regulations, a person is by virtue of that regulation to be treated as receiving relevant education; or

(iii) in relation to any claimant, the applicable amount includes severe disability premium by virtue of regulation 17(1)($d$) or 18(1)($e$), and paragraph 13 of Schedule 2 to, the Income Support Regulations; and
\end{enumerate}

($b$) it appears to the Secretary of State that he is not in possession of all of the evidence or information which is relevant for the purposes of such a determination,
\end{enumerate}
he shall make the determination on the assumption that the relevant evidence or information which is not in his possession is adverse to the claimant.

\subsubsection[14. Effect of alteration in the component rates of income support and jobseeker’s allowance]{Effect of alteration in the component rates of income support and jobseeker’s allowance}

14.—(1) Section 159 of the Administration Act (effect of alteration in the component rates of income support) shall not apply to any award of income support in force in favour of a person where there is applicable to that person—
\begin{enumerate}\item[]
($a$) any amount determined in accordance with regulation 17(2) to (7) of the Income Support Regulations; or

($b$) any protected sum determined in accordance with Schedule 3A or 3B of those Regulations\footnote{\frenchspacing Schedule 3A was inserted by S.I. 1988/1445; Schedule 3B was inserted by S.I. 1989/534.}; or

($c$) any transitional addition, personal expenses addition or special transitional addition applicable under Part II of the Income Support (Transitional) Regulations 1987\footnote{\frenchspacing S.I. 1987/1969.} (transitional protection).
\end{enumerate}

(2) Where section 159 of the Administration Act does not apply to an award of income support by virtue of paragraph (1), a decision under section 10 may be made in respect of that award for the sole purpose of giving effect to any change made by an order under section 150 of the Administration Act.

(3) Section 159A of the Administration Act\footnote{\frenchspacing Section 159A was inserted by section 24 of the Jobseekers Act 1995 (c. 18).} (effect of alterations in the component rates of jobseeker’s allowance) shall not apply to any award of a jobseeker’s allowance in force in favour of a person where there is applicable to that person any amount determined in accordance with regulation 87 of the Jobseeker’s Allowance Regulations.

(4) Where section 159A of the Administration Act does not apply to an award of a jobseeker’s allowance by virtue of paragraph (3), a decision under section 10 may be made in respect of that award for the sole purpose of giving effect to any change made by an order under section 150 of the Administration Act.

\subsubsection[15. Jobseeker’s allowance determinations on incomplete evidence]{Jobseeker’s allowance determinations on incomplete evidence}

15.  Where, for the purpose of a decision under section 8 or 10—
\begin{enumerate}\item[]
($a$) a determination falls to be made by the Secretary of State as to whether—
\begin{enumerate}\item[]
(i) in relation to any person, the applicable amount falls to be reduced or disregarded to any extent by virtue of section 15 of the Jobseekers Act (persons affected by trade disputes); or

(ii) for the purposes of regulation 54(2) to (4) of the Jobseeker’s Allowance Regulations (relevant education), a person is by virtue of that regulation, to be treated as receiving relevant education; and
\end{enumerate}

($b$) it appears to the Secretary of State that he is not in possession of all of the evidence or information which is relevant for the purposes of such a determination,
\end{enumerate}
he shall make the determination on the assumption that the relevant evidence or information which is not in his possession is adverse to the claimant.

\section[Part III --- Suspension, termination and other matters]{Part III\\*Suspension, termination and other matters}

\subsection[Chapter I --- Suspension and termination]{Chapter I\\*Suspension and termination}

\subsubsection[16. Suspension in prescribed cases]{Suspension in prescribed cases}

\renewcommand\parthead{--- Part III Chapter I}

16.—(1) Subject to paragraph (2), the Secretary of State may suspend payment of a relevant benefit, in whole or in part, in the circumstances prescribed in paragraph (3).

(2) The Secretary of State shall suspend payment of a jobseeker’s allowance in the circumstances prescribed in paragraph (3)($a$)(i) or (ii) where the issue or one of the issues is whether a person, who has claimed a jobseeker’s allowance, is or was available for employment or whether he is or was actively seeking employment.

(3) The prescribed circumstances are that—
\begin{enumerate}\item[]
($a$) it appears to the Secretary of State that—
\begin{enumerate}\item[]
(i) an issue arises whether the conditions for entitlement to a relevant benefit are or were fulfilled;

(ii) an issue arises whether a decision as to an award of a relevant benefit should be revised under section 9 or superseded under section 10;

(iii) an issue arises whether any amount paid or payable to a person by way of, or in connection with a claim for, a relevant benefit is recoverable under section 71 (overpayments), 71A (recovery of jobseeker’s allowance: severe hardship cases\footnote{\frenchspacing Section 71A was inserted by section 18 of the Jobseekers Act 1995 (c. 18).}) or 74 (income support and other payments) of the Administration Act or regulations made under any of those sections; or

(iv) the last address notified to him of a person who is in receipt of a relevant benefit is not the address at which that person is residing; or
\end{enumerate}

($b$) an appeal is pending against—
\begin{enumerate}\item[]
(i) a decision of an appeal tribunal, a Commissioner or a court;

(ii) a decision given in a different case by a Commissioner or a court, and it appears to the Secretary of State that, if the appeal were to be determined in a particular way, an issue would arise as to whether the award of a relevant benefit (whether the same benefit or not) in the case itself ought to be revised or superseded.
\end{enumerate}
\end{enumerate}

(4) For the purposes of section 21(3)($c$) an appeal is pending where the Secretary of State certifies in writing that he proposes—
\begin{enumerate}\item[]
($a$) to make a request under regulation 53(4) for a statement of reasons for a decision of an appeal tribunal;

($b$) to bring an appeal against the decision; or

($c$) to bring an appeal against a decision in a different case and, if that appeal were to be allowed, an issue would arise as to whether the award of a relevant benefit (whether the same benefit or not) in the case itself ought to be revised or superseded.
\end{enumerate}

\subsubsection[17. Provision of information or evidence]{Provision of information or evidence}

17.—(1) This regulation applies where the Secretary of State requires information or evidence for a determination whether a decision awarding a relevant benefit should be—
\begin{enumerate}\item[]
($a$) revised under section 9; or

($b$) superseded under section 10.
\end{enumerate}

(2) For the purposes of paragraph (1), the following persons must satisfy the requirements of paragraph (4)—
\begin{enumerate}\item[]
($a$) a person in respect of whom payment of a benefit has been suspended in the circumstances prescribed in regulation 16(3)($a$);

($b$) a person who has made an application for a decision of the Secretary of State to be revised or superseded;

($c$) a person who fails to comply with the provisions of regulation 32(1) of the Claims and Payments Regulations in so far as they relate to documents, information or facts required by the Secretary of State;

($d$) a person who qualifies for income support by virtue of paragraph 7 of Schedule 1B to the Income Support Regulations\footnote{\frenchspacing Schedule 1B was inserted by S.I. 1996/206.};

($e$) a person whose entitlement to benefit is conditional upon his being, or being treated as, incapable of work.
\end{enumerate}

(3) The Secretary of State shall notify any person to whom paragraph (2) refers of the requirements of this regulation.

(4) A person to whom paragraph (2) refers must either—
\begin{enumerate}\item[]
($a$) supply the information or evidence within—
\begin{enumerate}\item[]
(i) a period of one month beginning with the date on which the notification under paragraph (3) was sent to him; or

(ii) such longer period as he satisfies the Secretary of State is necessary in order to enable him to comply with the requirement; or
\end{enumerate}

($b$) satisfy the Secretary of State within the period of time specified in sub-paragraph ($a$)(i) that either—
\begin{enumerate}\item[]
(i) the information or evidence required of him does not exist; or

(ii) that it is not possible for him to obtain it.
\end{enumerate}
\end{enumerate}

(5) The Secretary of State may suspend the payment of a relevant benefit, in whole or in part, to any person to whom paragraph (2)($b$) to ($e$) applies who fails to satisfy the requirements of paragraph (4).

(6) In this regulation, “evidence” includes evidence which a person is required to provide in accordance with regulation 2 of the Social Security (Medical Evidence) Regulations 1976\footnote{\frenchspacing S.I. 1976/615; relevant amending instruments are S.I. 1982/699, 1992/247 and 1994/2975.}.

\subsubsection[18. Termination in cases of failure to furnish information or evidence]{Termination in cases of failure to furnish information or evidence}

18.—(1) Subject to paragraphs (2), (3) and (4), the Secretary of State shall decide that where a person—
\begin{enumerate}\item[]
($a$) whose benefit has been suspended in accordance with regulation 16 and who subsequently fails to comply with an information requirement made in pursuance of regulation 17; or

($b$) whose benefit has been suspended in accordance with regulation 17(5),
\end{enumerate}
that person shall cease to be entitled to that benefit from the date on which payment was suspended except where entitlement to benefit ceases on an earlier date other than under this regulation.

(2) Paragraph (1)($a$) shall not apply where not more than one month has elapsed since the information requirement was made in pursuance of regulation 17.

(3) Paragraph (1)($b$) shall not apply where not more than one month has elapsed since the first payment was suspended in accordance with regulation 17.

(4) Paragraph (1) shall not apply where benefit has been suspended in part under regulation 16 or, as the case may be, regulation 17.

\subsubsection[19. Suspension and termination for failure to submit to medical examination]{Suspension and termination for failure to submit to medical examination}

19.—(1) Except where regulation 8 of the Social Security (Incapacity for Work) (General) Regulations 1995\footnote{\frenchspacing S.I. 1995/311.} applies (where a question arises as to whether a person is capable of work), the Secretary of State may require a person to submit to a medical examination by a medical practitioner where that person is in receipt of a relevant benefit, and either—
\begin{enumerate}\item[]
($a$) the Secretary of State considers it necessary to satisfy himself as to the correctness of the award of the benefit, or of the rate at which it was awarded; or

($b$) that person applies for a revision or supersession of the award and the Secretary of State considers that the examination is necessary for the purpose of making his decision.
\end{enumerate}

(2) The Secretary of State may suspend payment of a relevant benefit in whole or in part, to a person who fails, without good cause, on two consecutive occasions to submit to a medical examination in accordance with requirements under paragraph (1) except where entitlement to benefit is suspended on an earlier date other than under this regulation.

(3) Subject to paragraph (4), the Secretary of State may determine that the entitlement to a relevant benefit of a person, in respect of whom payment of such a benefit has been suspended under paragraph (2), shall cease from a date not earlier than the date on which payment was suspended except where entitlement to benefit ceases on an earlier date other than under this regulation.

(4) Paragraph (3) shall not apply where not more than one month has elapsed since the first payment was suspended under paragraph (2).

\subsubsection[20. Making of payments which have been suspended]{Making of payments which have been suspended}

20.—(1) Subject to paragraphs (2) and (3), payment of a benefit suspended in accordance with regulation 16 
or 17  % Words inserted (5.7.99) by SI 1999/1623 reg 6(a)
shall be made where—
\begin{enumerate}\item[]
($a$) in a case to which regulation 16(2) or (3)($a$)(i) to (iii) applies, the Secretary of State is satisfied that the benefit suspended is properly payable and no outstanding issues remain to be resolved;

($b$) in a case to which regulation 16(3)($a$)(iv) applies, the Secretary of State is satisfied that he has been notified of the address at which the person is residing;

($c$) in a case to which regulation 16(3)($b$) applies, an appeal is no longer pending and the benefit suspended remains payable following the determination of the appeal;

% Reg 20(1)(d) inserted (5.7.99) by SI 1999/1623 reg 6(b)
($d$) in a case to which regulation 17(5) applies, the Secretary of State is satisfied that the benefit suspended is properly payable and the requirements of regulation 17(4) have been satisfied.
\end{enumerate}

(2) Where regulation 16(4)($a$) applies, payment of a benefit suspended shall be made if, within one month of the date on which he received a copy of the tribunal’s decision, the Secretary of State has not notified the claimant in writing that he has requested, pursuant to regulation 53(4), a statement of the reasons for the decision.

(3) Where regulation 16(4)($b$) or ($c$) applies, payment of a benefit suspended shall be made if the Secretary of State fails to notify the claimant in writing, within one month of the date on which the Secretary of State receives the reasons in writing for the decision on appeal which was pending for the purposes of regulation 16(3)($b$), that an appeal or, as the case may be, an application for leave to appeal has been made against the decision.

(4) Payment of benefit which has been suspended in accordance with regulation 19 for failure to submit to a medical examination shall be made where the Secretary of State is satisfied that it is no longer necessary for the person referred to in that regulation to submit to a medical examination.

\amendment{
Words inserted in reg. 20(1) and reg. 20(1)(d) inserted (5.7.99) by the Social Security and Child Support (Decisions and Appeals) Amendment (No. 2) Regulations 1999 reg. 6.
}

\subsection[Chapter II --- Other matters]{Chapter II\\*Other matters}

\subsubsection[21. Decisions involving issues that arise on appeal in other cases]{Decisions involving issues that arise on appeal in other cases}

\renewcommand\parthead{--- Part III Chapter II}

21.—(1) For the purposes of section 25(3)($b$) (prescribed cases and circumstances in which a decision may be made on a prescribed basis) a case which satisfies the condition in paragraph (2) is a prescribed case.

(2) The condition is that the claimant would be entitled to the benefit to which the decision which falls to be made relates, even if the appeal in the other case referred to in section 25(1)($b$) were decided in a way which is the most unfavourable to him.

(3) For the purposes of section 25(3)($b$), the prescribed basis on which the Secretary of State may make the decision is as if—
\begin{enumerate}\item[]
($a$) the appeal in the other case which is referred to in section 25(1)($b$) had already been determined; and

($b$) that appeal had been decided in a way which is the most unfavourable to the claimant.
\end{enumerate}

(4) The circumstance prescribed under section 25(5)($c$), where an appeal is pending against a decision for the purposes of that section, even though an appeal against the decision has not been brought (or, as the case may be, an application for leave to appeal against the decision has not been made) but the time for doing so has not yet expired, is where the Secretary of State—
\begin{enumerate}\item[]
($a$) certifies in writing that he is considering appealing against that decision; and

($b$) considers that, if such an appeal were to be determined in a particular way—
\begin{enumerate}\item[]
(i) there would be no entitlement to benefit in a case to which section 25(1)($a$) refers; or

(ii) the appeal would affect the decision in that case in some other way.
\end{enumerate}
\end{enumerate}

\subsubsection[22. Appeals involving issues that arise in other cases]{Appeals involving issues that arise in other cases}

22.  The circumstance prescribed under section 26(6)($c$), where an appeal is pending against a decision in the case described in section 26(1)($b$) even though an appeal against the decision has not been brought (or, as the case may be, an application for leave to appeal against the decision has not been made) but the time for doing so has not yet expired, is where the Secretary of State—
\begin{enumerate}\item[]
($a$) certifies in writing that he is considering appealing against that decision; and

($b$) considers that, if such an appeal were already determined, it would affect the determination of the appeal described in section 26(1)($a$).
\end{enumerate}

\subsubsection[23. Child support decisions involving issues that arise on appeal in other cases]{Child support decisions involving issues that arise on appeal in other cases}

23.—(1) For the purposes of section 28ZA(2)($b$) of the Child Support Act\footnote{\frenchspacing Section 28ZA was inserted by section 43 of the Social Security Act 1998.} (prescribed cases and circumstances in which a decision may be made on a prescribed basis), a case which satisfies either of the conditions in paragraph (2) is a prescribed case.

(2) The conditions referred to in paragraph (1) are that—
\begin{enumerate}\item[]
($a$) if a decision were not made on the basis prescribed in paragraph (3), the parent with care would become entitled to income support if a claim were made, or to an increased amount of that benefit;

($b$) the absent parent is an employed earner or a self-employed earner.
\end{enumerate}

(3) For the purposes of section 28ZA(2)($b$) of the Child Support Act, the prescribed basis on which the Secretary of State may make the decision is as if—
\begin{enumerate}\item[]
($a$) the appeal in relation to the different maintenance assessment, which is referred to in section 28ZA(1)($b$) of that Act had already been determined; and

($b$) that appeal had been decided in a way that was the most unfavourable to the applicant for the decision mentioned in section 28ZA(1)($a$) of that Act.
\end{enumerate}

(4) The circumstances prescribed under section 28ZA(4)($c$) of the Child Support Act (where an appeal is pending against a decision for the purposes of that section, even though an appeal against the decision has not been brought or, as the case may be, an application for leave to appeal against the decision has not been made but the time for doing so has not expired), are that the Secretary of State—
\begin{enumerate}\item[]
($a$) certifies in writing that he is considering appealing against that decision; and

($b$) he considers that, if such an appeal were to be determined in a particular way—
\begin{enumerate}\item[]
(i) there would be no liability for child support maintenance, or

(ii) such liability would be less than would be the case were an appeal not made.
\end{enumerate}
\end{enumerate}

(5) In this regulation—
\begin{enumerate}\item[]
“absent parent” and “parent with care” have the same meaning as in section 54 of the Child Support Act;

“employed earner” and “self-employed earner” have the same meaning as in section 2(1) of the Contributions and Benefits Act.
\end{enumerate}

\subsubsection[24. Child support appeals involving issues that arise in other cases]{Child support appeals involving issues that arise in other cases}

24.  The circumstances prescribed under section 28ZB(6)($c$) of the Child Support Act\footnote{\frenchspacing Section 28ZB was inserted by section 43 of the Social Security Act 1998.}, where an appeal is pending against a decision in the case described in section 28ZB(1)($b$) even though an appeal against the decision has not been brought (or, as the case may be, an application for leave to appeal against the decision has not been made), is where the Secretary of State—
\begin{enumerate}\item[]
($a$) certifies in writing that he is considering appealing against that decision, and

($b$) considers that, if such an appeal were already determined, it would affect the determination of the appeal described in section 28ZB(1)($a$).
\end{enumerate}

\section[Part IV --- Rights of appeal and procedure for bringing appeals]{Part IV\\*Rights of appeal and procedure for bringing appeals}

\subsection[Chapter I --- General appeals matters not including child support appeals]{Chapter I\\*General appeals matters not including child support appeals}

\renewcommand\parthead{--- Part IV Chapter I}

\subsubsection[25. Other persons with a right of appeal]{Other persons with a right of appeal}

25.  For the purposes of section 12(2)($b$), the following other persons have a right to appeal to an appeal tribunal—
\begin{enumerate}\item[]
($a$) any person appointed by the Secretary of State under regulation 33(1) of the Claims and Payments Regulations (persons unable to act) to act on behalf of another;

($b$) any person claiming attendance allowance or disability living allowance on behalf of another under section 66(2)($b$) of the Contriburions and Benefits Act or, as the case may be, section 76(3) of that Act (claims on behalf of terminally ill persons);

($c$) in relation to a pension scheme, any person who, for the purposes of Part X of the Pension Schemes Act 1993\footnote{\frenchspacing 1993 c. 48.}, is an employer, member, trustee or manager by virtue of section 146(8) of that Act.
\end{enumerate}

\subsubsection[26. Decisions against which an appeal lies]{Decisions against which an appeal lies}

26.  An appeal shall lie to an appeal tribunal against a decision made by the Secretary of State—
\begin{enumerate}\item[]
($a$) as to whether a person is entitled to a relevant benefit for which no claim is required by virtue of regulation 3 of the Claims and Payments Regulations\footnote{\frenchspacing The relevant amending instruments are S.I. 1989/136, S.I. 1994/2943 and S.I. 1996/1460.}; or

($b$) as to whether a payment be made out of the social fund to a person to meet expenses for heating by virtue of regulations made under section 138(2) of the Contributions and Benefits Act (payments out of the social fund).
\end{enumerate}

\subsubsection[27. Decisions against which no appeal lies]{Decisions against which no appeal lies}

27.—(1) No appeal lies to an appeal tribunal against a decision set out in Schedule 2.

(2) In paragraph (1) and Schedule 2, “decision” includes determinations embodied in or necessary to a decision.

(3) An appeal made against a decision specified in paragraph (1) may be struck out in accordance with regulation 46.

\subsubsection[28. Notice of decision against which appeal lies]{Notice of decision against which appeal lies}

28.—(1) A person with a right of appeal under the Act or these Regulations against any decision of the Secretary of State shall—
\begin{enumerate}\item[]
($a$) be given written notice of the decision against which the appeal lies;

($b$) be informed that, in a case where that written notice does not include a statement of the reasons for that decision, he may, within one month of the date of notification of that decision, request that the Secretary of State provide him with a written statement of the reasons for that decision; and

($c$) be given written notice of his right of appeal against that decision.
\end{enumerate}

(2) Where a written statement of the reasons for the decision is not included in the written notice of the decision and is requested under paragraph (1)($b$), the Secretary of State shall provide that statement within 14 days of receipt of the request.

\subsubsection[29. Further particulars required relating to certificate of recoverable benefits appeals or applications]{Further particulars required relating to certificate of recoverable benefits appeals or applications}

29.—(1) An appeal or application under the 1997 Act relating to a certificate of recoverable benefits shall, in addition to any requirements imposed by regulations, include also the following particulars—
\begin{enumerate}\item[]
($a$) in the case of an appeal, the date of the certificate of recoverable benefits or the decision by the Secretary of State on review against which the appeal is brought, the question under section 11 of the 1997 Act to which the appeal relates and a summary of the arguments relied upon by the appellant to support his contention that the certificate is wrong;

($b$) in the case of an application for an extension of time under regulation 32, in relation to the appeal which it is proposed to bring, the particulars required under sub-paragraph ($a$) together with particulars of the special circumstances on which the application is based.
\end{enumerate}

(2) Where the appeal or the application for an extension of time is made by a person to whom a compensation payment has been made, a copy of the statement given to that person under section 9 of the 1997 Act or if that statement was not in writing, a written summary of it, shall be sent with that appeal or application.

(3) Where it appears to the Secretary of State that an appeal or application does not contain the further particulars required under paragraph (1) or is not accompanied by a written statement or summary as required under paragraph (2) he may direct the appellant or applicant to provide such particulars or such a statement or summary.

(4) Where paragraph (3) applies, the time specified for making the appeal or application may be extended by such period, not exceeding 14 days from the date of the Secretary of State’s direction under paragraph (3), as the Secretary of State may determine.

(5) Where further particulars or a written statement or summary are required under paragraph (3) they shall be sent to or delivered to the Compensation Recovery Unit of the Department of Social Security at Reyrolle Building, Hebburn, Tyne and Wear, \textsc{\lowercase{NE31 1XB}} within such period as the Secretary of State may direct.

(6) The Secretary of State may treat any appeal relating to the certificate of recoverable benefits as an application for review under section 10 of the 1997 Act.

\subsection[Chapter II --- General appeals matters including child support appeals]{Chapter II\\*General appeals matters including child support appeals}

\renewcommand\parthead{--- Part IV Chapter II}

\subsubsection[30. Appeal against a decision which has been revised]{Appeal against a decision which has been revised}

30.—(1) An appeal against a decision of the Secretary of State shall not lapse where the decision is revised under section 16 of the Child Support Act or section 9 before the appeal is determined and the decision as revised is not more advantageous to the appellant than the decision before it was revised.

(2) Decisions which are more advantageous for the purposes of this regulation include decisions where—
\begin{enumerate}\item[]
($a$) any relevant benefit paid to the appellant is greater or is awarded for a longer period in consequence of the decision made under section 9;

($b$) it would have resulted in the amount of relevant benefit in payment being greater but for the operation of any provision of the Administration Act or the Contributions and Benefits Act restricting or suspending the payment of, or disqualifying a claimant from receiving, some or all of the benefit;

($c$) as a result of the decision, a denial or disqualification for the receiving of any relevant benefit, is lifted, wholly or in part;

($d$) it reverses a decision to pay benefit to a third party;

($e$) in consequence of the revised decision, benefit paid is not recoverable under section 71, 71A or 74 of the Administration Act\footnote{\frenchspacing Section 71A was inserted by section 18 of the Jobseekers Act 1995 (c. 18).} or regulations made under any of those sections, or the amount so recoverable is reduced; or

($f$) a financial gain accrued or will accrue to the appellant in consequence of the decision.
\end{enumerate}

(3) Where a decision as revised under section 16 of the Child Support Act or under section 9 is not more advantageous to the appellant than the decision before it was revised, the appeal shall be treated as though it had been brought against the decision as revised.

(4) The appellant shall have a period of one month from the date of notification of the decision as revised to make further representations as to the appeal.

(5) After the expiration of the period specified in paragraph (4), or within that period if the appellant consents in writing, the appeal to the appeal tribunal shall proceed except where, in the light of the further representations from the appellant, the Secretary of State further revises his decision and that decision is more advantageous to the appellant than the decision before it was revised.

\subsubsection[31. Time within which an appeal is to be brought]{Time within which an appeal is to be brought}

31.—(1) Where an appeal lies from a decision of the Secretary of State to an appeal tribunal, except in the case of a decision of the Secretary of State under section 3 or 3A of the Vaccine Damage Payments Act, the time within which that appeal must be brought is, subject to the following provisions of this Part—
\begin{enumerate}\item[]
($a$) within one month of the date of notification of the decision against which the appeal is brought; or

($b$) where a written statement of reasons for that decision is requested, within 14 days of the expiry of the period specified in sub-paragraph ($a$).
\end{enumerate}

(2) Where the Secretary of State—
\begin{enumerate}\item[]
($a$) revises, or following an application for a revision under regulation 3(1) or (3) does not revise, a decision under section 16 of the Child Support Act or under section 9, or

($b$) supersedes a decision under section 17 of the Child Support Act or under section 10,
\end{enumerate}
the period of one month specified in paragraph (1) shall begin to run from the date of notification of the revision or supersession of the decision, or following an application for a revision under regulation 3(1) or (3), the date the Secretary of State issues a notice that he is not revising the decision.

(3) An appeal against a certificate of recoverable benefits must be brought—
\begin{enumerate}\item[]
($a$) not later than one month after the date a person making a compensation payment discharges his liability under section 6 of the 1997 Act;

($b$) where the certificate is reviewed by the Secretary of State in accordance with regulations made under section 11(5)($c$) of the 1997 Act, not later than one month after the date the certificate is confirmed, or, as the case may be, a fresh certificate is issued; or

($c$) where an agreement is made under which an earlier compensation payment is treated as having been made in final discharge of a claim made by or in respect of an injured person and arising out of the accident, injury or disease, not later than one month after the date of that agreement.
\end{enumerate}

(4) Where a dispute arises as to whether an appeal was brought within the time limit specified in this regulation, the dispute shall be referred to, and be determined by, a legally qualified panel member.

(5) The time limit specified in this regulation for bringing an appeal may be extended in accordance with regulation 32.

\subsubsection[32. Late appeals]{Late appeals}

32.—(1) The time within which an appeal must be brought may be extended where the conditions specified in paragraphs (2) to (8) are satisfied, but no appeal shall in any event be brought more than one year after the expiration of the last day for appealing under regulation 31.

(2) An application for an extension of time under this regulation shall be made in accordance with regulation 33 and shall be determined by a legally qualified panel member.

(3) An application under this regulation shall contain particulars of the grounds on which the extension of time is sought, including details of any relevant special circumstances for the purposes of paragraph (4).

(4) An application for an extension of time shall not be granted unless the panel member is satisfied that—
\begin{enumerate}\item[]
($a$) if the application is granted there are reasonable prospects that the appeal will be successful;

($b$) it is in the interests of justice for the application to be granted.
\end{enumerate}

(5) For the purposes of paragraph (4) it is not in the interests of justice to grant an application unless the panel member is satisfied that—
\begin{enumerate}\item[]
($a$) the special circumstances specified in paragraph (6) are relevant to the application; or

($b$) some other special circumstances exist which are wholly exceptional and relevant to the application,
\end{enumerate}
and as a result of those special circumstances, it was not practicable for the application to be made within the time limit specified in regulation 31.

(6) For the purposes of paragraph (5)($a$), the special circumstances are that—
\begin{enumerate}\item[]
($a$) the applicant or a spouse or dependant of the applicant has died or suffered serious illness;

($b$) the applicant is not resident in the United Kingdom; or

($c$) normal postal services were disrupted.
\end{enumerate}

(7) In determining whether it is in the interests of justice to grant the application, the panel member shall have regard to the principle that the greater the amount of time that has elapsed between the expiration of the time within which the appeal is to be brought under regulation 31 and the making of the application for an extension of time, the more compelling should be the special circumstances on which the application is based.

(8) In determining whether it is in the interests of justice to grant an application, no account shall be taken of the following—
\begin{enumerate}\item[]
($a$) that the applicant or any person acting for him was unaware of or misunderstood the law applicable to his case (including ignorance or misunderstanding of the time limits imposed by these Regulations); or

($b$) that a Commissioner or a court has taken a different view of the law from that previously understood and applied.
\end{enumerate}

(9) An application under this regulation for an extension of time which has been refused may not be renewed.

(10) The panel member who determines an application under this regulation shall record a summary of his decision in such written form as has been approved by the President.

(11) As soon as practicable after the decision is made a copy of the decision shall be sent or given to every party to the proceedings.

\subsubsection[33. Making of appeals and applications]{Making of appeals and applications}

33.—(1) An appeal, or an application for an extension of time for making an appeal to an appeal tribunal shall be in writing either on a form approved for the purpose by the Secretary of State or in such other format as the Secretary of State accepts as sufficient for the purpose and shall—
\begin{enumerate}\item[]
($a$) be signed by—
\begin{enumerate}\item[]
(i) the person who, under 
section 4(1) of the Vaccine Damage Payments Act,  % Words inserted (18.10.99) by SI 1999/2677 reg 9
section 20 of the Child Support Act as extended by paragraph 3 of Schedule 4C to that Act, section 11(2) of the 1997 Act or section 12(2), has a right of appeal; or

(ii) where the person in head (i) has provided written authority to a representative to act on his behalf, by that representative;
\end{enumerate}

($b$) be sent or delivered to an appropriate office;

($c$) contain particulars of the grounds on which it is made; and

($d$) contain sufficient particulars of the decision, the certificate of recoverable benefits or the subject of the application, as the case may be, to enable that decision, certificate or subject of the application to be identified.
\end{enumerate}

(2) In this regulation, “an appropriate office” means—
\begin{enumerate}\item[]
($a$) in the case of an appeal under the 1997 Act against a certificate of recoverable benefits, the Compensation Recovery Unit of the Department of Social Security at Reyrolle Building, Hebburn, Tyne and Wear, \textsc{\lowercase{NE31 1XB}};

($b$) in the case of an appeal against a decision relating to a jobseeker’s allowance, an office of the Department of Social Security or of the Department for Education and Employment;

($c$) in the case of a contributions decision which falls within Part II of Schedule 3 to the Act, any National Insurance Contributions office;

($d$) in the case of an appeal under section 20 of the Child Support Act as extended by paragraph 3 of Schedule 4C to that Act, an office of the Child Support Agency; and

($e$) in any other case, an office of the Department of Social Security.
\end{enumerate}

(3) A form which is not completed in accordance with the instructions on the form—
\begin{enumerate}\item[]
($a$) except where paragraph (4) applies, does not satisfy the requirements of paragraph (1), and

($b$) may be returned by the Secretary of State to the sender for completion in accordance with those instructions.
\end{enumerate}

(4) Where the Secretary of State is satisfied that the form, although not completed in accordance with the instructions on it, includes sufficient information to enable the appeal or application to proceed, he may treat the form as satisfying the requirements of paragraph (1).

(5) Where an appeal or application is made in writing otherwise than on the approved form (“the letter”), and the letter includes sufficient information to enable the appeal or application to proceed, the Secretary of State may treat the letter as satisfying the requirements of paragraph (1).

(6) Where the letter does not include sufficient information to enable the appeal or application to proceed, the Secretary of State may request further information in writing (“further particulars”) from the person who wrote the letter.

(7) Where a person to whom a form is returned or from whom further particulars are requested duly completes and returns the form or sends the further particulars and the form or particulars (as the case may be) are received by the Secretary of State within—
\begin{enumerate}\item[]
($a$) 14 days of the date on which the form was returned to him by the Secretary of State,

($b$) 14 days of the date on which the Secretary of State’s request was made (“the date of request”), or

($c$) such longer period as the Secretary of State may direct,
\end{enumerate}
the time for making the appeal shall be extended by 14 days from the date the form was returned, the date of request or the date of the Secretary of State’s direction, as the case may be.

(8) Where a person to whom a form is returned or from whom further particulars are requested does not complete and return the form or send further particulars within the period of time specified in paragraph (7)—
\begin{enumerate}\item[]
($a$) the Secretary of State shall forward a copy of the form, or as the case may be, the letter, together with any other relevant documents or evidence to a legally qualified panel member, and

($b$) the panel member shall determine whether the form or the letter satisfies the requirement of paragraph (1), and shall inform the appellant or applicant and the Secretary of State of his determination.
\end{enumerate}

(9) Where—
\begin{enumerate}\item[]
($a$) a form is duly completed and returned or further particulars are sent after the expiry of the period of time allowed in accordance with paragraph (7), and

($b$) no decision has been made under paragraph (8) at the time the form or the further particulars are received by the Secretary of State,
\end{enumerate}
that form or further particulars shall also be forwarded to the legally qualified panel member who shall take into account any further information or evidence set out in the form or further particulars.

\amendment{
Words inserted in reg. 33(1)(a)(i) (18.10.99) by the Social Security and Child Support (Decisions and Appeals), Vaccine Damage Payments and Jobseeker's Allowance (Amendment) Regulations 1999 reg. 9.
}

\subsubsection[34. Death of a party to an appeal]{Death of a party to an appeal}

34.—(1) In any proceedings, on the death of a party to those proceedings (other than the Secretary of State), the Secretary of State may appoint such person as he thinks fit to proceed with the appeal in the place of such deceased party.

(2) A grant of probate, confirmation or letters of administration to the estate of the deceased party, whenever taken out, shall have no effect on an appointment made under paragraph (1).

(3) Where a person appointed under paragraph (1) has, prior to the date of such appointment, taken any action in relation to the appeal on behalf of the deceased party, the effective date of appointment by the Secretary of State shall be the day immediately prior to the first day on which such action was taken.

\section[Part V --- Appeal tribunals for social security, contracting out of pensions, vaccine damage and child support]{Part V\\*Appeal tribunals for social security, contracting out of pensions, vaccine damage and child support}

\subsection[Chapter I --- The panel and appeal tribunals]{Chapter I\\*The panel and appeal tribunals}

\subsubsection[35. Persons appointed to the panel]{Persons appointed to the panel}

\renewcommand\parthead{--- Part V Chapter I}

35.  For the purposes of section 6(3), the panel shall include persons with the qualifications specified in Schedule 3.

\subsubsection[36. Composition of appeal tribunals]{Composition of appeal tribunals}

36.—(1) Subject to the following provisions of this regulation, an appeal tribunal, including an appeal tribunal determining a misconceived appeal as a preliminary issue in accordance with regulation 48, shall consist of a legally qualified panel member.

%(2) Subject to paragraphs (3), (4) and (5), an appeal tribunal shall consist of a medically qualified panel member and a legally qualified panel member where—
%\begin{enumerate}\item[]
%($a$) the issue, or one of the issues raised on the appeal relates to—
%\begin{enumerate}\item[]
%(i) incapacity benefit under section 30A of the Contributions and Benefits Act\footnote{\frenchspacing Section 30A was inserted by section 1 of the Social Security (Incapacity for Work) Act 1994 (c. 18).};
%
%(ii) industrial injuries benefit under Part V of that Act; or
%
%(iii) severe disablement allowance under section 68 of that Act;
%\end{enumerate}
%
%($b$) the appeal is made under section 11(1)($b$) of the 1997 Act; or
%
%($c$) the appeal is made under section 4 of the Vaccine Damage Payments Act.
%\end{enumerate}

% Reg 36(2) substituted (1.6.99) by SI 1999/1466 reg 2(a)
(2) Subject to paragraphs (3) to (5), an appeal tribunal shall consist of a legally qualified panel member and—
\begin{enumerate}\item[]
($a$) a medically qualified panel member where—
\begin{enumerate}\item[]
(i) the issue, or one of the issues, raised on the appeal is whether the all work test is satisfied; or

(ii) the appeal is made under section 11(1)($b$) of the 1997 Act; or
\end{enumerate}

($b$) one medically qualified panel member or two such members or one medically qualified panel member and an additional member drawn from the panel for the purposes described in paragraph (5) below where—
\begin{enumerate}\item[]
(i) the issue, or one of the issues, raised on the appeal relates to either industrial injuries benefit under Part V of the Contributions and Benefits Act or severe disablement allowance under section 68 of that Act; or

(ii) the appeal is made under section 4 of the Vaccine Damage Payments Act.
\end{enumerate}
\end{enumerate}

(3) An appeal tribunal shall consist of a financially qualified panel member and a legally qualified panel member where—
\begin{enumerate}\item[]
($a$) the issue raised, or one of the issues raised on appeal or referral, relates to child support or a relevant benefit; and

($b$) the appeal or referral may require consideration by members of the appeal tribunal of issues which are, in the opinion of the President, difficult and which relate to—
\begin{enumerate}\item[]
(i) profit and loss accounts, revenue accounts or balance sheets relating to any enterprise;

(ii) an income and expenditure account in the case of an enterprise not trading for profit; or

(iii) the accounts of any trust fund.
\end{enumerate}
\end{enumerate}

(4) Where the composition of an appeal tribunal would fall to be prescribed under both paragraphs (2) and (3), it shall consist of a medically qualified panel member, a financially qualified panel member and a legally qualified panel member.

(5) Where the composition of an appeal tribunal is prescribed under 
%paragraphs (1), (2) or 
paragraph (1), (2)($a$) or  % Words substituted (1.6.99) by SI 1999/1466 reg 2(b)
(3), the President may determine that the appeal tribunal shall include such an additional member drawn from the panel constituted under section 6 as he considers appropriate for the purposes of providing further experience for that additional member or for assisting the President in the monitoring of standards of decision making by panel members.

(6) An appeal tribunal shall consist of a legally qualified panel member, a medically qualified panel member and a panel member with a disability qualification in any appeal which relates to an attendance allowance or a disability living allowance under Part III of the Contributions and Benefits Act or a disability working allowance under section 129 of that Act.

% Reg 36(7) added (1.6.99) by SI 1999/1466 reg 2(b)
(7) In paragraph (2)($a$)(i) above, “all work test” has the meaning it bears in regulation 2(1) of the Social Security (Incapacity for Work) (General) Regulations 1995\footnote{\frenchspacing S.I. 1995/311.}.

\amendment{
Words substituted in reg. 36(5), reg. 36(7) added and reg. 36(2) substituted (1.6.99) by the Social Security and Child Support (Decisions and Appeals) (Amendment) Regulations 1999 reg. 2.
}

\subsubsection[37. Assignment of clerks to appeal tribunals: function of clerks]{Assignment of clerks to appeal tribunals: function of clerks}

37.  The Secretary of State shall assign a clerk to service each appeal tribunal and the clerk so assigned shall be responsible for summoning members of the panel constituted under section 6 to serve on the tribunal.

\subsection[Chapter II --- Procedure in connection with determination of appeals and referrals]{Chapter II\\*Procedure in connection with determination of appeals and referrals}

\renewcommand\parthead{--- Part V Chapter II}

\subsubsection[38. Consideration and determination of appeals and referrals]{Consideration and determination of appeals and referrals}

38.—(1) The procedure in connection with the consideration and determination of an appeal or a referral shall, subject to the following provisions of these Regulations, be such as a legally qualified panel member shall determine.

(2) A legally qualified panel member may give directions requiring a party to the proceedings to comply with any provision of these Regulations and may at any stage of the proceedings, either of his own motion or on a written application made to the clerk to the appeal tribunal by any party to the proceedings, give such directions as he may consider necessary or desirable for the just, effective and efficient conduct of the proceedings and may direct any party to the proceedings to provide such particulars or to produce such documents as may be reasonably required.

(3) Where a clerk to the appeal tribunal is authorised to take steps in relation to the procedure of the tribunal he may give directions requiring any party to the proceedings to comply with any provision of these Regulations.

% Reg 38A inserted (5.7.99) by SI 1999/1670 reg 2(4)
\subsubsection[38A. Appeals raising issues for decision by officers of Inland Revenue]{Appeals raising issues for decision by officers of Inland Revenue}

38A.---(1)  Where, on consideration of any appeal, it appears to an appeal tribunal that an issue arises which, by virtue of section 8 of the Transfer Act, falls to be decided by an officer of the Board, that tribunal shall—
\begin{enumerate}\item[]
($a$) refer the appeal to the Secretary of State pending the decision of that issue by an officer of the Board; and

($b$) require the Secretary of State to refer that issue to the Board;
\end{enumerate}
and the Secretary of State shall refer that issue accordingly.

(2) Pending the final decision of any issue which has been referred to the Board in accordance with paragraph (1) above, the Secretary of State may revise the decision under appeal, or make a further decision superseding that decision, in accordance with his determination of any issue other than one which has been so referred.

(3) On receipt by the Secretary of State of the final decision of an issue which has been referred in accordance with paragraph (1) above, he shall consider whether the decision under appeal ought to be revised under section 9 or superseded under section 10, and—
\begin{enumerate}\item[]
($a$) if so, revise it or, as the case may be, make a further decision which supersedes it; or

($b$) if not, forward the appeal to the appeal tribunal which shall determine the appeal in accordance with the final decision of the issue so referred.
\end{enumerate}

(4) In paragraphs (2) and (3) above, “final decision” has the same meaning as in regulation 11A(3) and (4).

\amendment{
Reg. 38A inserted (5.7.99) by the Social Security and Child Support (Decisions and Appeals) Amendment (No. 3) Regulations 1999 reg. 2(4).
}

\subsubsection[39. Directions concerning oral hearings]{Directions concerning oral hearings}

39.—(1) Where an appeal or a referral is made to an appeal tribunal, the clerk to the appeal tribunal shall direct the appellant and any other party to the proceedings to notify the clerk to the appeal tribunal in writing whether he wishes to have an oral hearing of the appeal or whether he is content for the appeal or referral to proceed without an oral hearing.

(2) Except in the case of a referral, a direction under paragraph (1) shall include a statement informing the appellant that, if he does not respond in writing to the direction within the period specified in paragraph (3), the appeal may be struck out in accordance with regulation 46.

(3) A notification given in accordance with paragraph (1) must be received by the clerk to the appeal tribunal within 14 days of the date of issue of the direction of the clerk to the appeal tribunal under paragraph (1) or within such longer period as the clerk to the appeal tribunal may direct.

(4) Where a party to the proceedings notifies the clerk to the appeal tribunal in accordance with paragraph (3) that he wishes to have an oral hearing of the appeal or referral, the appeal tribunal shall hold an oral hearing.

(5) The chairman, or in the case of an appeal tribunal which has only one member, that member, may of his own motion direct that an oral hearing of the appeal or referral be held if he is satisfied that such a hearing is necessary to enable the appeal tribunal to reach a decision.

\subsubsection[40. Withdrawal of appeal or referral]{Withdrawal of appeal or referral}

40.—(1) An appeal may be withdrawn by the appellant or an authorised representative of the appellant and a referral may be withdrawn by the Secretary of State, as the case may be, either—
\begin{enumerate}\item[]
($a$) at an oral hearing; or

($b$) at any other time before the appeal or referral is determined, by giving notice in writing of withdrawal to the clerk to the appeal tribunal.
\end{enumerate}

(2) If an appeal or a referral is withdrawn (as the case may be) in accordance with paragraph (1)($a$), the clerk to the appeal tribunal shall send a notice in writing to any party to the proceedings who is not present when the appeal or referral is withdrawn, informing him that the appeal or referral (as the case may be) has been withdrawn.

(3) If an appeal or a referral is withdrawn (as the case may be) in accordance with paragraph (1)($b$), the clerk to the appeal tribunal shall send a notice in writing to every party to the proceedings informing them that the appeal or referral (as the case may be) has been withdrawn.

\subsubsection[41. Medical examination required by appeal tribunal]
{Medical examination required by appeal tribunal}

41.  For the purposes of section 20(2) (medical examination required by appeal tribunal) the prescribed condition which must be satisfied is that the issue, or one of the issues, raised on the appeal—
\begin{enumerate}\item[]
($a$) is whether the claimant satisfies the conditions for entitlement to—
\begin{enumerate}\item[]
(i) the care component of a disability living allowance specified in section 72(1) and (2) of the Contributions and Benefits Act;

(ii) the mobility component of a disability living allowance specified in section 73(1), (8) and (9) of that Act;

(iii) an attendance allowance specified in section 64 and 65(1) of that Act;

(iv) a disability working allowance specified in section 129(1)($b$) of that Act;

% Reg 41(a)(v) omitted (5.7.99) by SI 1999/1670 reg 2(5)(a)
%(v) incapacity benefit under section 30A of that Act; or

(vi) severe disablement allowance under section 68 of that Act;
\end{enumerate}

($b$) relates to the period throughout which the claimant is likely to satisfy the conditions for entitlement to an attendance allowance or a disability living allowance;

($c$) is the rate at which an attendance allowance is payable;

($d$) is the rate at which the care component or the mobility component of a disability living allowance is payable;

% Reg 41(dd) inserted (5.7.99) by SI 1999/1670 reg 2(5)(b)
($dd$) is whether a person is incapable of work for the purposes of the Contributions and Benefits Act;

% Reg 41(e) omitted (5.7.99) by SI 1999/1670 reg 2(5)(c)
%($e$) relates to either statutory sick pay or statutory maternity pay and the appeal is made by the employer concerned;

($f$) relates to the extent of a person’s disablement and its assessment in accordance with Schedule 6 to the Contributions and Benefits Act;

($g$) is whether the claimant suffers a loss of physical or mental faculty as a result of the relevant accident for the purposes of section 103 of the Contributions and Benefits Act;

($h$) relates to any disease or injury prescribed for the purposes of section 108 of the Contributions and Benefits Act; or

($i$) relates to any payment arising under, or by virtue of a scheme having effect under, section 111 of, and Schedule 8 to, the Contributions and Benefits Act (workmen’s compensation).
\end{enumerate}

\amendment{
Reg. 41(dd) inserted and reg. 41(a)(v), (e) omitted (5.7.99) by the Social Security and Child Support (Decisions and Appeals) Amendment (No. 3) Regulations 1999 reg. 2(5).
}

\subsubsection[42. Non-disclosure of medical advice or evidence]{Non-disclosure of medical advice or evidence}

42.—(1) Where, in connection with the consideration and determination of an appeal or referral there is before an appeal tribunal medical advice or medical evidence relating to a person which has not been disclosed to him and in the opinion of the chairman, or in the case of an appeal tribunal which has only one member, in the opinion of that member, the disclosure to that person of that advice or evidence would be harmful to his health, such advice or evidence shall not be required to be disclosed to that person.

(2) Advice or evidence such as is mentioned in paragraph (1) shall not be disclosed to any person acting for or representing the person to whom it relates or, in a case where a claim for benefit is made by reference to the disability of a person other than the claimant and the advice or evidence relates to that other person, shall not be disclosed to the claimant or any person acting for or representing him, unless the chairman, or in the case of an appeal tribunal which has only one member, that member, is satisfied that it is in the interests of the person to whom the advice or evidence relates to do so.

(3) A tribunal shall not be precluded from taking into account for the purposes of the determination advice or evidence which has not been disclosed to a person under the provisions of paragraph (1) or (2).

\subsubsection[43. Summoning of witnesses and administration of oaths]{Summoning of witnesses and administration of oaths}

43.—(1) A chairman, or in the case of an appeal tribunal which has only one member, that member, may by summons, or in Scotland, by citation, require any person in Great Britain to attend as a witness at a hearing of an appeal, application or referral at such time and place as shall be specified in the summons or citation and, subject to paragraph (2), at the hearing to answer any question or produce any documents in his custody or under his control which relate to any matter in question in the appeal, application or referral but—
\begin{enumerate}\item[]
($a$) no person shall be required to attend in obedience to such summons or citation unless he has been given at least 14 days' notice of the hearing or, if less than 14 days' notice is given, he has informed the tribunal that the notice given is sufficient; and

($b$) no person shall be required to attend and give evidence or to produce any document in obedience to such summons or citation unless the necessary expenses of attendance are paid or tendered to him.
\end{enumerate}

(2) No person shall be compelled to give any evidence or produce any document or other material that he could not be compelled to give or produce on a trial of an action in a court of law in that part of Great Britain where the hearing takes place.

(3) In exercising the powers conferred by this regulation, the chairman, or in the case of an appeal tribunal which has only one member, that member, shall take into account the need to protect any matter that relates to intimate personal or financial circumstances, is commercially sensitive, consists of information communicated or obtained in confidence or concerns national security.

(4) Every summons or citation issued under this regulation shall contain a statement to the effect that the person in question may apply in writing to a chairman to vary or set aside the summons or citation.

(5) A chairman, or in the case of an appeal tribunal which has only one member, that member, may require any witness, including a witness summoned under the powers conferred by this regulation, to give evidence on oath or affirmation and for that purpose there may be administered an oath or affirmation in due form.

\subsubsection[44. Confidentiality in child support appeals or referrals]{Confidentiality in child support appeals or referrals}

44.—(1) In the circumstances specified in paragraph (2), for the purposes of paragraph 7 of Schedule 1 to the Act (President to secure confidentiality), in a child support appeal or referral, the prescribed material is—
\begin{enumerate}\item[]
($a$) the address of the absent parent; the parent with care; the child; a parent of the child or any other person with care of the child; or

($b$) any information the use of which could reasonably be expected to lead to the location of any person specified in paragraph ($a$).
\end{enumerate}

(2) Except where the appeal is brought against a reduced benefit direction within the meaning of section 46(11) of the Child Support Act\footnote{\frenchspacing Section 46(11) is amended by paragraph 43 of Schedule 7 to the Social Security Act 1998.}, paragraph (1) applies where in response to an enquiry from the Secretary of State, the absent parent or, as the case may be, the parent with care, has within 14 days of issue of that enquiry notified the Secretary of State that he would like the information specified in paragraph (1) which relates to him to remain confidential.

(3) In this regulation, the expressions “absent parent” and “parent with care” have the meanings those expressions bear in section 54 of the Child Support Act.

\subsubsection[45. Consideration of more than one appeal under section 20 of the Child Support Act]{Consideration of more than one appeal under section 20 of the Child Support Act}

45.  An appeal tribunal which is considering an appeal under section 20 of the Child Support Act in respect of a departure direction\footnote{\frenchspacing Section 20 of the Child Support Act 1991 as extended by Schedule 4C to that Act applies to an appeal against a departure direction by virtue of section 28H of the Act as substituted by paragraph 39 of Schedule 7 to the Social Security Act 1998.} which relates to a maintenance assessment may, if it considers it appropriate to do so, consider at the same time any appeal under that section in respect of another departure direction which relates to the same maintenance assessment.

\subsection[Chapter III --- Striking out appeals]{Chapter III\\*Striking out appeals}

\subsubsection[46. Appeals which may be struck out]{Appeals which may be struck out}

\renewcommand\parthead{--- Part V Chapter III}

46.—(1) Subject to paragraphs (2) and (3), an appeal may be struck out by the clerk to the appeal tribunal—
\begin{enumerate}\item[]
($a$) where it is an out of jurisdiction appeal and the appellant has been notified by the Secretary of State that an appeal brought against such a decision may be struck out;

($b$) for want of prosecution including an appeal not made within the time specified in these Regulations; or

($c$) subject to regulation 39(4), for failure of the appellant to comply with a direction given under these Regulations where the appellant has been notified that failure to comply with the direction could result in the appeal being struck out.
\end{enumerate}

(2) Where the clerk to the appeal tribunal determines to strike out the appeal, he shall notify the appellant that his appeal has been struck out and of the procedure for reinstatement of the appeal as specified in regulation 47.

(3) The clerk to the appeal tribunal may refer any matter for determination under this regulation to a legally qualified panel member for decision by the panel member rather than the clerk to the appeal tribunal.

(4) Subject to regulation 48, a misconceived appeal may be struck out by a legally qualified panel member but such an appeal shall not be struck out unless the appellant has been given notice of—
\begin{enumerate}\item[]
($a$) the intention to strike out the appeal,

($b$) the ground on which the intention to strike out is based, and

($c$) the requirement to notify the clerk to the appeal tribunal in writing of the matters specified in regulation 48(1)($a$) or ($b$) and that failure to comply with this requirement may result in the appeal being struck out.
\end{enumerate}

\subsubsection[47. Reinstatement of struck out appeals]{Reinstatement of struck out appeals}

47.  A legally qualified panel member may reinstate an appeal which has been struck out in accordance with regulation 46 or regulation 48(2) where—
\begin{enumerate}\item[]
($a$) the appellant has made representations, or as the case may be, further representations in support of his appeal with reasons why he considers that his appeal should not have been struck out, to the clerk to the appeal tribunal, in writing within one month of the order to strike out the appeal being issued, and the panel member is satisfied in the light of those representations that there are reasonable grounds for reinstating the appeal;

($b$) the panel member is satisfied that the appellant did not receive the notification required under regulation 46(4);

($c$) the panel member is satisfied that the appeal is not an appeal which may be struck out under regulation 46; or

($d$) the panel member is satisfied that notwithstanding that the appeal is one which may be struck out under regulation 46, it is not in the interests of justice for the appeal to be struck out.
\end{enumerate}

\subsubsection[48. Misconceived appeals]{Misconceived appeals}

48.—(1) Where the appellant has been given notice under regulation 46(4) of intention to strike out an appeal on the ground that it is a misconceived appeal that person must within 14 days of the issue of such notice notify the clerk to the appeal tribunal in writing that—
\begin{enumerate}\item[]
($a$) he wishes the question of whether his appeal is misconceived to be determined by an appeal tribunal as a preliminary issue at an oral hearing, or

($b$) he is content for an appeal tribunal to consider the question of whether his appeal is misconceived as a preliminary issue without an oral hearing and make representations in writing to the clerk to the appeal tribunal as to why he considers that the appeal is not misconceived.
\end{enumerate}

(2) Where the appellant fails to notify or to make representations to the clerk to the appeal tribunal in writing as required in paragraph (1) within the period specified in that paragraph, a legally qualified panel member may strike out the appeal.

(3) Where the appellant notifies the clerk to the appeal tribunal under paragraph (1) within the period specified in that paragraph that he wishes an appeal tribunal to determine the question of whether his appeal is misconceived as a preliminary issue at an oral hearing, the appeal tribunal shall hold an oral hearing for that preliminary issue.

(4) Where the appeal tribunal determine as a preliminary issue that the appeal is a misconceived appeal, the appeal shall be struck out and the clerk to the appeal tribunal shall notify the appellant that the appeal is struck out.

(5) Where the appeal tribunal determine as a preliminary issue that the appeal is not a misconceived appeal—
\begin{enumerate}\item[]
($a$) the appeal tribunal shall refer the appeal and all the supporting documentation to the Secretary of State together with a statement of the reasons why the appeal tribunal considers that the appeal is not misconceived;

($b$) the clerk to the appeal tribunal shall notify the appellant of the referral of the appeal to the Secretary of State and send the appellant a copy of the reasons why the appeal tribunal considers that the appeal is not misconceived;

($c$) the Secretary of State may revise or supersede the decision against which the appeal is brought; and

($d$) if the Secretary of State does not revise or supersede the decision against the appeal is brought in the appellant’s favour, the Secretary of State shall refer the appeal for determination by an appeal tribunal.
\end{enumerate}

(6) Chapter IV of this Part shall apply to an oral hearing held under this regulation.

\subsection[Chapter IV --- Oral hearings]{Chapter IV\\*Oral hearings}

\renewcommand\parthead{--- Part V Chapter IV}

\subsubsection[49. Procedure at oral hearings]{Procedure at oral hearings}

49.—(1) Subject to the following provisions of this Part, the procedure for an oral hearing shall be such as the chairman, or in the case of an appeal tribunal which has only one member, such as that member, shall determine.

(2) Except where paragraph (3) applies, not less than 14 days notice (beginning with the day on which the notice is given and ending on the day before the hearing of the appeal is to take place) of the time and place of any oral hearing of an appeal shall be given to every party to the proceedings, and if such notice has not been given to a person to whom it should have been given under the provisions of this paragraph the hearing may proceed only with the consent of that person.

(3) Any party to the proceedings may waive his right to receive not less than 14 days notice of the time and place of any oral hearing by giving notice to the clerk to the appeal tribunal.

(4) If a party to the proceedings to whom notice has been given under paragraph (2) fails to appear at the hearing the chairman, or in the case of an appeal tribunal which has only one member, that member, may, having regard to all the circumstances including any explanation offered for the absence, proceed with the hearing notwithstanding his absence, or give such directions with a view to the determination of the appeal as he may think proper.

(5) If a party to the proceedings has waived his right to be given notice under paragraph (2) the chairman, or in the case of an appeal tribunal which has only one member, that member, may proceed with the hearing notwithstanding his absence.

(6) Any oral hearing shall be in public except—
\begin{enumerate}\item[]
($a$) where the appellant requests a private hearing, or

($b$) where the chairman, or in the case of an appeal tribunal which has only one member, that member, is satisfied that intimate personal or financial circumstances may have to be disclosed or that considerations of national security are involved, in which case the hearing shall be in private.
\end{enumerate}

(7) Any party to the proceedings shall be entitled to be present and be heard at an oral hearing.

(8) A person who has the right to be heard at a hearing may be accompanied and may be represented by another person whether having professional qualifications or not and, for the purposes of the proceedings at the hearing, any such representative shall have all the rights and powers to which the person whom he represents is entitled.

(9) The following persons shall also be entitled to be present at an oral hearing (whether or not it is otherwise in private) but shall take no part in the proceedings—
\begin{enumerate}\item[]
($a$) the President;

($b$) any person undergoing training as a chairman or panel member of an appeal tribunal or as a clerk to an appeal tribunal;

($c$) any person acting on behalf of the President in the training or supervision of panel members or in the monitoring of standards of decision-making by panel members;

($d$) with the leave of the chairman, or in the case of an appeal tribunal which has only one member, with the leave of that member, and the consent of every party to the proceedings actually present, any other person; and

($e$) a member of the Council on Tribunals or of the Scottish Committee of the Council on Tribunals.
\end{enumerate}

(10) Nothing in paragraph (9) affects the rights of any person mentioned in sub-paragraphs ($a$) and ($b$) of that paragraph at any oral hearing where he is sitting as a member of the tribunal or acting as its clerk, and nothing in this regulation prevents the presence at an oral hearing of any witness.

(11) Any person entitled to be heard at an oral hearing may address the tribunal, may give evidence, may call witnesses and may put questions directly to any other person called as a witness.

(12) For the purpose of arriving at its decision an appeal tribunal shall, and for the purpose of discussing any question of procedure may, notwithstanding anything contained in these Regulations, order all persons not being members of the tribunal, other than the person acting as clerk to the appeal tribunal, to withdraw from the hearing except that—
\begin{enumerate}\item[]
($a$) a member of the Council on Tribunals or of the Scottish Committee of the Council on Tribunals, the President or any person mentioned in paragraph (9)($c$); and

($b$) with the leave of the chairman, or in the case of an appeal tribunal which has only one member, with the leave of that member, any person mentioned in paragraph (9)($b$) or ($d$),
\end{enumerate}
may remain present at any such sitting.

\subsubsection[50. Manner of providing expert assistance]{Manner of providing expert assistance}

50.—(1) Where an appeal tribunal require one or more experts to provide assistance to it in dealing with a question of fact of special difficulty under section 7(4), such an expert shall, if the chairman, or in the case of a tribunal with only one member, that member, so requests, attend at the hearing and give evidence and if the chairman or member sitting alone considers it appropriate, the expert shall enquire into and provide a written report on the question.

(2) A copy of any written report received from an expert in accordance with paragraph (1) shall be supplied to every party to the proceedings.

\subsubsection[51. Postponement and adjournment]{Postponement and adjournment}

51.—(1) Where a person to whom notice of an oral hearing is given wishes to request a postponement of that hearing he shall do so in writing to the clerk to the appeal tribunal stating his reasons for the request, and the clerk to the appeal tribunal may grant or refuse the request as he thinks fit or may pass the request to a legally qualified panel member who may grant or refuse the request as he thinks fit.

(2) Where the clerk to the appeal tribunal or the panel member, as the case may be, refuses a request to postpone the hearing he shall—
\begin{enumerate}\item[]
($a$) notify in writing the person making the request of the refusal; and

($b$) place before the appeal tribunal at the hearing both the request for the postponement and notification of its refusal.
\end{enumerate}

(3) A panel member or the clerk to the appeal tribunal may of his own motion at any time before the beginning of the hearing postpone the hearing.

(4) An oral hearing may be adjourned by the appeal tribunal at any time on the application of any party to the proceedings or of its own motion.

(5) Where a hearing has been adjourned and it is not practicable, or would cause undue delay, for it to be resumed before a tribunal consisting of the same member or members, the appeal or referral shall be heard by a differently constituted tribunal and the proceedings shall be by way of a complete rehearing.

\subsubsection[52. Physical examinations at oral hearings]{Physical examinations at oral hearings}

52.  For the purposes of section 20(3) an appeal tribunal may not carry out a physical examination except in a case which relates to—
\begin{enumerate}\item[]
($a$) the extent of a person’s disablement and its assessment in accordance with section 68(6) of, and Schedule 6 to, the Contributions and Benefits Act;

($b$) the extent of a person’s disablement and its assessment in accordance with section 103 of that Act;

($c$) diseases or injuries prescribed for the purposes of section 108 of that Act.
\end{enumerate}

\subsection[Chapter V --- Decisions of appeal tribunals and related matters]{Chapter V\\*Decisions of appeal tribunals and related matters}

\subsection{\itshape Appeal tribunal decisions}

\subsubsection[53. Decisions of appeal tribunals]{Decisions of appeal tribunals}

53.—(1) Every decision of an appeal tribunal shall be recorded in summary by the chairman, or in the case of an appeal tribunal which has only one member, by that member.

(2) The decision notice specified in paragraph (1) shall be in such written form as shall have been approved by the President and shall be signed by the chairman, or in the case of an appeal tribunal which has only one member, by that member.

(3) As soon as may be practicable after an appeal or referral has been decided by an appeal tribunal, a copy of the decision notice prepared in accordance with paragraph (1) and (2) shall be sent or given to every party to the proceedings who shall also be informed of—
\begin{enumerate}\item[]
($a$) his right under paragraph (4); and

%($b$) the conditions governing appeals to a Commissioner.

% Reg 53(3)(b) substituted (18.10.99) by SI 1999/2677 reg 10
($b$) except in the case of an appeal under the Vaccine Damage Payments Act, the conditions governing appeals to a Commissioner.
\end{enumerate}

(4) A party to the proceedings may apply in writing to the chairman, or in the case of a tribunal with only one member, to that member, for a copy of a statement of the reasons for the tribunal’s decision within one month of the sending or giving of the decision notice to every party to the proceedings or within such longer period as may be allowed in accordance with regulation 54.

(5) If the decision is not unanimous, the decision notice specified in paragraph (1) shall record that one of the members dissented and the statement of reasons referred to in paragraph (4) shall include the reasons given by the dissenting member for dissenting.

\amendment{
Reg. 53(3)(b) substituted (18.10.99) by the Social Security and Child Support (Decisions and Appeals), Vaccine Damage Payments and Jobseeker's Allowance (Amendment) Regulations 1999 reg. 10.
}

\subsubsection[54. Late applications for a statement of reasons of tribunal decision]{Late applications for a statement of reasons of tribunal decision}

54.—(1) The time for making an application for a copy of the statement of the reasons for a tribunal’s decision may be extended where the conditions specified in paragraphs (2) to (8) are satisfied, but no application shall in any event be brought more than three months after the date of the sending or giving of the notice of the decision of the appeal tribunal.

(2) An application for an extension of time under this regulation shall be made in writing and shall be determined by a legally qualified panel member.

(3) An application under this regulation shall contain particulars of the grounds on which the extension of time is sought, including details of any relevant special circumstances for the purposes of paragraph (4).

(4) The application for an extension of time shall not be granted unless the panel member is satisfied that it is in the interests of justice for the application to be granted.

(5) For the purposes of paragraph (4) it is not in the interests of justice to grant the application unless the panel member is satisfied that—
\begin{enumerate}\item[]
($a$) the special circumstances specified in paragraph (6) are relevant to the application; or

($b$) some other special circumstances are relevant to the application,
\end{enumerate}
and as a result of those special circumstances it was not practicable for the application to be made within the time limit specified in regulation 53(4).

(6) For the purposes of paragraph (5)($a$), the special circumstances are that—
\begin{enumerate}\item[]
($a$) the applicant or a spouse or dependant of the applicant has died or suffered serious illness;

($b$) the applicant is not resident in the United Kingdom; or

($c$) normal postal services were adversely disrupted.
\end{enumerate}

(7) In determining whether it is in the interests of justice to grant the application, the panel member shall have regard to the principle that the greater the amount of time that has elapsed between the expiration of the time within which the application for a copy of the statement of reasons for a tribunal’s decision is to be made and the making of the application for an extension of time, the more compelling should be the special circumstances on which the application is based.

(8) In determining whether it is in the interests of justice to grant the application, no account shall be taken of the following—
\begin{enumerate}\item[]
($a$) that the person making the application or any person acting for him was unaware of, or misunderstood, the law applicable to his case (including ignorance or misunderstanding of the time limits imposed by these Regulations); or

($b$) that a Commissioner or a court has taken a different view of the law from that previously understood and applied.
\end{enumerate}

(9) An application under this regulation for an extension of time which has been refused may not be renewed.

(10) The panel member who determines the application shall record a summary of his decision in such written form as has been approved by the President.

(11) As soon as practicable after the decision is made a copy of the decision shall be sent or given to every party to the proceedings.

(12) Any person who under paragraph (11) receives a copy of the decision may, within one month of the decision being sent to him, apply in writing for a copy of the reasons for that decision and a copy shall be supplied to him.

\subsubsection[55. Record of tribunal proceedings]{Record of tribunal proceedings}

55.—(1) A record of the proceedings at an oral hearing, which is sufficient to indicate the evidence taken, shall be made by the chairman, or in the case of an appeal tribunal which has only one member, by that member, in such medium as he may direct.

(2) Such record shall be preserved by the clerk to the appeal tribunal for six months from the date of the decision made by the appeal tribunal to which the record relates and any party to the proceedings may within that period apply in writing for a copy of that record and a copy shall be supplied to him.

\subsubsection[56. Correction of accidental errors]{Correction of accidental errors}

56.—(1) The clerk to the appeal tribunal, or where the clerk refers the matter to a legally qualified panel member, that member, may at any time correct accidental errors in any decision, or the record of any such decision, of an appeal tribunal made under a relevant enactment, the Child Support Act or the Vaccine Damage Payments Act.

(2) A correction made to, or to the record of, a decision shall be deemed to be part of the decision or record of that decision and written notice of it shall be given as soon as practicable to every party to the proceedings.

(3) In this regulation and regulation 57, “relevant enactment” has the same meaning as in section 28(3).

\subsubsection[57. Setting aside decisions on certain grounds]{Setting aside decisions on certain grounds}

57.—(1) On an application made by a party to the proceedings, a decision of an appeal tribunal made under a relevant enactment, the Child Support Act or the Vaccine Damage Payments Act, may be set aside by a legally qualified panel member in a case where it appears just to set the decision aside on the ground that—
\begin{enumerate}\item[]
($a$) a document relating to the proceedings in which the decision was made was not sent to, or was not received at an appropriate time by, a party to the proceedings or the party’s representative or was not received at an appropriate time by the person who made the decision;

($b$) a party to the proceedings in which the decision was made or the party’s representative was not present at a hearing relating to the proceedings.
\end{enumerate}

(2) In determining whether it is just to set aside a decision on the ground set out in paragraph (1)($b$), the panel member shall determine whether the party making the application gave notice that he wished to have an oral hearing, and if that party did not give such notice the decision shall not be set aside unless the chairman, or in the case of an appeal tribunal which has only one member, unless that member is satisfied that the interests of justice manifestly so require.

(3) An application under this regulation shall be made in accordance with regulations 31 to 33.

(4) Where an application to set aside a decision is entertained under paragraph (1), every party to the proceedings shall be sent a copy of the application and shall be afforded a reasonable opportunity of making representations on it before the application is determined.

(5) Notice in writing of a determination on an application to set aside a decision shall be sent or given to every party to the proceedings as soon as may be practicable and the notice shall contain a statement giving the reasons for the determination.

\subsection{\itshape Applications for leave to appeal to a Commissioner (not including child support)}

\subsubsection[58. Application for leave to appeal to a Commissioner from an appeal tribunal]{Application for leave to appeal to a Commissioner from an appeal tribunal}

58.—(1) An application for leave to appeal to a Commissioner from a decision of an appeal tribunal under section 12 or 13 shall—
\begin{enumerate}\item[]
($a$) be made within the period of one month commencing on the date the applicant is sent a written statement of the reasons for the decision against which leave to appeal is sought; and

($b$) have annexed to it a copy of that written statement of the reasons for the decision.
\end{enumerate}

(2) Where an application for leave to appeal to a Commissioner is made by the Secretary of State, the clerk to an appeal tribunal shall, as soon as may be practicable, send a copy of the application to every other party to the proceedings.

(3) Any party to the proceedings who is sent a copy of an application for leave to appeal in accordance with paragraph (2) may make representations in writing within one month of the date the application is sent.

(4) A person determining an application for leave to appeal to a Commissioner, shall take into account any further representations received from the applicant before the determination is made, and shall record his decision in writing and send a copy to every party to the proceedings.

(5) Where there has been a failure to apply for leave to appeal within the period of time specified in paragraph (1)($a$) but an application is made within one year of the last date for making an application within that period, a legally qualified panel member may, if for special reasons he thinks fit, accept and proceed to consider and determine the application.

(6) Where in any case it is impracticable, or it would be likely to cause undue delay for an application for leave to appeal against a decision of an appeal tribunal to be determined by the person who was the chairman, or in the case of an appeal tribunal which has only one member, the member, of that tribunal, the application shall be determined by a legally qualified panel member.

\section[Part VI --- Revocations]{Part VI\\*Revocations}

\subsection[59. Revocations]{Revocations}

\renewcommand\parthead{--- Part VI}

59.—(1) The Regulations listed in column (2) of Schedule 4 are hereby revoked to the extent specified in column (3) of that Schedule.

(2) Notwithstanding their revocation for particular purposes, the Regulations listed in column (2) of Schedule 4 shall continue to have full effect up to and including 28th November 1999 in relation to any benefit to which these Regulations do not apply for the time being by virtue of regulation 1(2).

(3) So much of any document as refers expressly or by implication to any regulation revoked by paragraph (1) shall, in so far as the context permits, for the purposes of these Regulations be treated as referring to the corresponding provision of these Regulations.

\bigskip

Signed 
by authority of the Secretary of State for Social Security.

{\raggedleft
\emph{Angela Eagle}\\*Parliamentary Under-Secretary of State,\\*Department of Social Security

}

26th March 1999

\bigskip

I concur

{\raggedleft
\emph{Irvine of Lairg}\\*Lord Chancellor

}

26th March 1999

\small

\part*{S C H E D U L E S}

\part[Schedule 1 --- Provisions conferring powers exercised in making these Regulations]{Schedule 1\\*Provisions conferring powers exercised in making these Regulations}

\renewcommand\parthead{--- Schedule 1}

{\footnotesize
%\begin{tabulary}{\linewidth}{JJJ}
\begin{longtable}{p{150pt}p{102pt}p{102pt}}
\hline
\itshape Column (1) & & \itshape Column (2)\\
\itshape Provision & & \itshape Relevant Amendments\\
\hline
\endhead
\hline
\endlastfoot
Vaccine Damage Payments Act 1979\footnote{\frenchspacing 1979 c. 17.}&Section 4(2) and (3)&The Act, Section 46.\\
&Section 7A(1)&The Act, Section 47.\\
Child Support Act 1991\footnote{\frenchspacing 1991 c. 48.}&Section 16(6)&The Act, Section 40.\\
&Section 20(5) and (6)&The Act, Section 42.\\
&Section 28ZA(2)($b$) and (4)($c$)&The Act, Section 43.\\
&Section 28ZB(6)($c$)&The Act, Section 43.\\
&Section 28ZC(7)&The Act, Section 44.\\
&Section 28ZD(1) and (2)&The Act, Section 44.\\
&Section 46B&The Act, Schedule 7, paragraph 44.\\
&Section 51(2)&The Act, Schedule 7, paragraph 46.\\
&Schedule 4A, paragraph 8&The Act, Schedule 7, paragraph 53.\\
Social Security Administration Act 1992\footnote{\frenchspacing 1992 c. 5.}&Section 5(1)($hh$)&The Act, Section 74.\\
&Section 159&The Act, Schedule 7, paragraph 95.\\
&Section 159A&The Act, Schedule 7, paragraph 96.\\
Pension Schemes Act 1993\footnote{\frenchspacing 1993 c. 48.}&Section 170(3)&The Act, Schedule 7, paragraph 131.\\
Social Security (Recovery of Benefits) Act 1997\footnote{\frenchspacing 1997 c. 27.}&Section 10&The Act, Schedule 7, paragraph 149.\\
&Section 11(5)\\
Social Security Act 1998\footnote{\frenchspacing 1998 c. 14.}&Section 6(3)\\
&Section 7(6)\\
&Section 9(1), (4) and (6)\\
&Section 10(3) and (6)\\
&Section 11(1)\\
&Section 12(2) and (3), (6) and (7)\\
&Section 14(10)($a$) and (11)\\
&Section 16(1) and Schedule 5\\
&Section 17\\
&Section 18(1)\\
&Section 20\\
&Section 21(1) to (3)\\
&Section 22\\
&Section 23\\
&Section 24\\
&Section 25(3)($b$) and (5)($c$)\\
&Section 26(6)($c$)\\
&Section 28(1)\\
&Section 31(2)\\
&Section 79(1) and (3) to (7)\\
&Section 84\\
&Schedule 1, paragraphs 7, 11 and 12\\
&Schedule 2, paragraph 9\\
&Schedule 3, paragraphs 1, 4 and 9\\
%\end{tabulary}
\end{longtable}

}

\part[Schedule 2 --- Decisions against which no appeal lies]{Schedule 2\\*Decisions against which no appeal lies}

\renewcommand\parthead{--- Schedule 2}

\subsection*{Child Benefit}

1.  A decision of the Secretary of State as to whether an educational establishment be recognised for the purposes of Part IX of the Contributions and Benefits Act.

\medskip

2.  A decision of the Secretary of State to recognise education provided otherwise than at a recognised educational establishment.

\medskip

3.  A decision of the Secretary of State made in accordance with the discretion conferred upon him by the following provisions of the Child Benefit (Residence and Persons Abroad) Regulations 1976\footnote{\frenchspacing S.I. 1976/963; the relevant amending instrument is S.I. 1976/1758.}—
\begin{enumerate}\item[]
($a$) regulation 2(2)($c$)(iii) (decision relating to a child’s temporary absence abroad);

($b$) regulation 7(3) (certain days of absence abroad disregarded).
\end{enumerate}

\medskip

4.  A decision of the Secretary of State made in accordance with the discretion conferred upon him by regulation 2(1) or (3) of the Child Benefit (General) Regulations 1976\footnote{\frenchspacing S.I. 1976/965; the relevant amending instrument is S.I. 1976/1758.} (provisions relating to contributions and expenses in respect of a child).

\subsection*{Claims and Payments}

5.  A decision of the Secretary of State under the Claims and Payments Regulations except a decision under—
\begin{enumerate}\item[]
($a$) regulation 19 as to the time for claiming benefit;

($b$) regulation 37AB\footnote{\frenchspacing Regulation 37AB was inserted by S.I. 1994/2319.} as to the payment of withheld benefit;

($c$) regulation 38 as to the extinguishment of the right to payment of sums by way of benefit where payment is not obtained within the prescribed period; and

($d$) the following provisions of Schedule 9 and regulation 35(1) in so far as it relates to them—
\begin{enumerate}\item[]
(i) paragraph 3 relating to the amount deductible by way of housing costs;

(ii) paragraph 4 relating to the amount of miscellaneous housing costs payable direct to a third party;

(iii) paragraph 4A\footnote{\frenchspacing Paragraph (4A) was inserted by S.I. 1991/2284 and amended by S.I. 1992/2595.} relating to the direct payment to a third party of benefit payable to or in respect of persons resident in hostels;

(iv) paragraph 5 relating to payments of benefit direct to the claimant’s or his partner’s landlord;

(v) paragraph 6 relating to the payment out of benefit of fuel costs;

(vi) paragraph 7 relating to the payment out of benefit of water charges;

(vii) paragraph 7A\footnote{\frenchspacing Paragraph 7A was inserted by S.I. 1993/478, substituted by S.I. 1993/2113 and then amended by S.I. 1996/481.} in connection with amounts payable in place of child support maintenance;

(viii) paragraph 7B\footnote{\frenchspacing Paragraph 7B was inserted by S.I. 1996/2344.} as to whether an amount in respect of arrears of child support maintenance is to be deducted from a person’s jobseeker’s allowance; and

(ix) paragraph 9(3) as to the priority between liabilities for items of gas and electricity.
\end{enumerate}
\end{enumerate}

\subsection*{Contracted Out Pension Schemes}

6.  A decision of the Secretary of State under section 109 of the Pension Schemes Act 1993\footnote{\frenchspacing 1993 c. 48.} or any Order made under it (annual increase of guaranteed minimum pensions).

\subsection*{Decisions depending on other cases}

7.  A decision of the Secretary of State under section 25 or 26 (decisions and appeals depending on other cases).

\subsection*{Deductions}

8.  A decision which falls to be made by the Secretary of State under the Fines (Deductions from Income Support) Regulations 1992\footnote{\frenchspacing S.I. 1992/2182.}, other than 
%one falling within regulation 4 of those Regulations.
a decision whether benefit is sufficient for a deduction to be made.  % Words substituted (29.11.99) by SI 1999/3178 Sch 19 para 2

\amendment{
Words substituted in para. 8 (29.11.99) by the Social Security Act 1998 (Commencement No. 12 and Consequential and Transitional Provisions) Order 1999 Sch. 19 para. 2.
}

\medskip

9.—(1) Except in relation to a decision to which sub-paragraph (2) applies, any decision of the Secretary of State under the Community Charges (Deductions from Income Support) (No.\ 2) Regulations 1990\footnote{\frenchspacing S.I. 1990/545.}, the Community Charges (Deductions from Income Support) (Scotland) Regulations 1989\footnote{\frenchspacing S.I. 1989/507.} or the Council Tax (Deductions from Income Support) Regulations 1993\footnote{\frenchspacing S.I. 1993/494.}.

(2) This sub-paragraph applies to a decision—
\begin{enumerate}\item[]
($a$) whether there is an outstanding sum due of the amount sought to be deducted;

($b$) whether benefit is sufficient for a deduction to be made; and

($c$) on the priority to be given to any deduction.
\end{enumerate}

\subsection*{European Community Regulations}

10.  An authorization given by the Secretary of State in accordance with article 22(1) or 55(1) of Council Regulation (EEC) No. 1408/71\footnote{\frenchspacing See Council Regulation (EEC) No. 1408/71, O.J. No. L149/2, 5.7.71 (O.J./S.E. 1971(II) page 416).} on the application of social security schemes to employed persons, to self-employed persons and to members of their families moving within the Community.

\subsection*{Expenses}

11.  A decision of the Secretary of State whether to pay expenses to any person under section 180 of the Administration Act.

\subsection*{Guardian’s Allowance}

12.  A decision of the Secretary of State relating to the giving of a notice under regulation 5(8) of the Social Security (Guardian’s Allowance) Regulations 1975\footnote{\frenchspacing S.I. 1975/515.} (children whose surviving parents are in prison or legal custody).

\subsection*{Income Support}

13.  A decision of the Secretary of State which embodies a determination made in accordance with paragraph (1) or (2) of regulation 13 (income support and social fund determinations on incomplete evidence).

\subsection*{Industrial Injuries Benefit}

14.  A decision of the Secretary of State relating to the question whether—
\begin{enumerate}\item[]
($a$) disablement pension be increased under section 104 of the Contributions and Benefits Act (constant attendance); or

($b$) disablement pension be further increased under section 105 of the Contributions and Benefits Act (exceptionally severe disablement);
\end{enumerate}
and if an increase is to be granted or renewed, the period for which and the amount at which it is payable.

\medskip

15.  A decision of the Secretary of State under regulation 2(2) of the Social Security (Industrial Injuries and Diseases) Miscellaneous Provisions Regulations 1986\footnote{\frenchspacing S.I. 1986/1561.} as to the length of any period of interruption of education which is to be disregarded.

\medskip

16.  A decision of the Secretary of State to approve or not to approve a person undertaking work for the purposes of regulation 17 of the Social Security (General Benefit) Regulations 1982\footnote{\frenchspacing S.I. 1982/1408; the relevant amending instruments are S.I. 1983/186 and S.I. 1986/1561.}.

\medskip

17.  A decision of the Secretary of State as to how the limitations under Part VI of Schedule 7 to the Contributions and Benefits Act on the benefit payable in respect of any death are to be applied in the circumstances of any case.

\subsection*{Invalid Vehicle Scheme}

18.  A decision of the Secretary of State relating to the issue of certificates under regulation 13 of, and Schedule 2 to, the Social Security (Disability Living Allowance) Regulations 1991\footnote{\frenchspacing S.I. 1991/2890.}.

\subsection*{Jobseeker’s Allowance}

19.—(1) A decision of the Secretary of State under Chapter IV of Part II of the Jobseeker’s Allowance Regulations as to the day and the time a claimant is to attend at a job centre.

(2) A decision of the Secretary of State as to the day of the week on which a claimant is required to provide a signed declaration under regulation 24(10) of the Jobseeker’s Allowance Regulations.

(3) A decision of the Secretary of State which embodies a determination made in accordance with regulation 15 (Jobseeker’s allowance determinations on incomplete evidence).

\subsection*{Payments on Account, Overpayments and Recovery}

20.  A decision of the Secretary of State under the Social Security (Payments on account, Overpayments and Recovery) Regulations 1988\footnote{\frenchspacing S.I. 1988/664; the relevant amending instruments are S.I. 1988/668, 1991/2742, 1993/650 and 1996/1345.}, except a decision of the Secretary of State under the following provisions of those Regulations—
\begin{enumerate}\item[]
($a$) regulation 3(1)($a$) to offset any interim payment made in anticipation of an award of benefit;

($b$) regulation 4(1) as to the overpayment of an interim payment;

($c$) regulation 5 as to the offsetting of a prior payment against a subsequent award;

($d$) regulation 11(1) as to whether a payment in excess of entitlement has been credited to a bank or other account;

($e$) regulation 13 as to the sums to be deducted in calculating recoverable amounts;

($f$) regulation 14(1) as to the treatment of capital to be reduced;

($g$) regulation 19 determining a claimant’s protected earnings; and

($h$) regulation 24 whether a determination as to a claimant’s protected earnings is revised or superseded.
\end{enumerate}

\subsection*{Persons Abroad}

21.  A decision of the Secretary of State made under—
\begin{enumerate}\item[]
($a$) regulation 2(1)($a$) of the Social Security Benefit (Persons Abroad) Regulations 1975\footnote{\frenchspacing S.I. 1975/563; the relevant amending instruments are S.I. 1976/409, 1977/342 and 1679, 1979/463 and 1432, 1981/1157, 1982/388 and 1738, 1983/186, 1984/1303, 1986/1545 and 1561, 1988/435, 1989/1642, 1990/40 and 621, 1991/2742, 1992/1700 and 2595, 1994/268 and 1832, 1995/829 and 1996/207 and 1345.} whether to certify that it is consistent with the proper administration of the Contributions and Benefits Act that a disqualification under section 113(1)($a$) of that Act should not apply;

($b$) regulation 9(4) or (5) of those Regulations whether to allow a person to avoid disqualification for receiving benefit during a period of temporary absence from Great Britain longer than that specified in the regulation.
\end{enumerate}

\subsection*{Reciprocal Agreements}

22.  A decision of the Secretary of State made in accordance with an Order made under section 179 of the Administration Act (reciprocal agreements with countries outside the United Kingdom).

\subsection*{Social Fund Awards}

23.  A decision of the Secretary of State under section 78 of the Administration Act relating to the recovery of social fund awards.

\subsection*{Suspension}

24.  A decision of the Secretary of State relating to the suspension of a relevant benefit or to the payment of such a benefit which has been suspended under Part III.

\subsection*{Up-rating}

25.  A decision of the Secretary of State relating to the up-rating of benefits under Part X of the Administration Act.

\part[Schedule 3 --- Qualifications of persons appointed to the panel]{Schedule 3\\*Qualifications of persons appointed to the panel}

\subsection*{Legal Qualifications}

\renewcommand\parthead{--- Schedule 3}

1.  Persons who—
\begin{enumerate}\item[]
($a$) have a general qualification (construed in accordance with section 71 of the Courts and Legal Services Act 1990\footnote{\frenchspacing 1990 c. 41.}); or

($b$) are advocates or solicitors in Scotland.
\end{enumerate}

\subsection*{Medical Qualifications}

2.  Fully registered medical practitioners, where—
\begin{enumerate}\item[]
($a$) the practitioner’s name appears on a medical specialist register maintained in any EEA State in accordance with the Medical Directive; or

($b$) the practitioner holds a vocational training certificate or a certificate of acquired rights in an EEA State other than the United Kingdom which must in his case be recognised in the United Kingdom by virtue of the Medical Directive (whether or not as read with the EEA Agreement) or by virtue of an enforceable community right; or

($c$) the practitioner does not satisfy the requirements of sub-paragraphs ($a$) or ($b$) above, but has not less than 10 years experience in clinical practice, or as a medical analyst or research worker in disciplines which are the same or similar to those undertaken by practitioners to whom those sub-paragraphs apply.
\end{enumerate}

\medskip

3.  In paragraph 2 above and in this paragraph—
\begin{enumerate}\item[]
“EEA Agreement” means the Agreement of the European Economic Area signed at Oporto on 2nd May 1992 as adjusted by the Protocol signed at Brussels on 17th March 1993\footnote{\frenchspacing Cm. 2183 and OJ No.L1, 3.1. 1994, p.572.};

“EEA State” means a state which is a contracting party to the EEA Agreement;

“Medical Directive” means Council Directive 93/16/EEC of 5th April 1993 to facilitate the free movement of doctors and the mutual recognitions of their diplomas, certificates and other evidence of formal qualifications\footnote{\frenchspacing OJ. No. L165, 7.7. 1993 page 1.}, as amended by Council Directive 97/50/EC of 6th October 1997\footnote{\frenchspacing OJ. No. L921, 24.10. 1997, page 35.};

“Vocational training certificate” means a diploma, certificate or other evidence of formal qualifications awarded on completion of a course of specific training in general medical practice and referred to in article 30 of the Medical Directive.
\end{enumerate}

\subsection*{Financial Qualifications}

4.  Accountants who are members of—
\begin{enumerate}\item[]
($a$) the Institute of Chartered Accountants in England and Wales;

($b$) the Institute of Chartered Accountants in Scotland;

($c$) the Institute of Chartered Accountants in Ireland;

($d$) the Association of Chartered Certified Accountants;

($e$) the Chartered Institute of Management Accountants; or

($f$) the Chartered Institute of Public Finance and Accountancy.
\end{enumerate}

\subsection*{Disability Qualifications}

5.  Persons, other than registered medical practitioners, who are experienced in dealing with the needs of disabled persons—
\begin{enumerate}\item[]
($a$) in a professional or voluntary capacity; or

($b$) because they are themselves disabled.
\end{enumerate}

\part[Schedule 4 --- Revocations]{Schedule 4\\*Revocations}

\renewcommand\parthead{--- Schedule 4}

{\footnotesize 
%\begin{tabulary}{\linewidth}{JJJ}
\begin{longtable}{p{50pt}p{188pt}p{116pt}}
\hline
\itshape Column 1 & \itshape Column 2 & \itshape Column 3\\
\itshape Statutory Instrument Number & \itshape Statutory Instrument & \itshape Provision Revoked\\
\hline
\endhead
\hline
\endlastfoot
1979/432&The Vaccine Damage Payments Regulations 1979&Part III\\
1992/2641&The Child Support Appeal Tribunals (Procedure) Regulations 1992&The whole Regulations\\
1995/311&The Social Security (Incapacity for Work) (General) Regulations 1995&Regulations 19 and 20 to 22\\
1995/1801&The Social Security (Adjudication) Regulations 1995&The whole Regulations\\
1996/182&The Social Security (Adjudication) and Child Support Amendment Regulations 1996&Regulation 2\\
1996/425&The Social Security (Industrial Injuries and Diseases) (Miscellaneous Amendments) Regulations 1996&Regulation 2\\
1996/1518&The Social Security (Adjudication) Amendment Regulations 1996&The whole Regulations\\
1996/2306&The Social Security (Claims and Payments and Adjudication) Amendment Regulations 1996&Regulations 8 and 9\\
1996/2450&The Social Security (Adjudication) and Child Support Amendment (No.\ 2) Regulations 1996&Regulations 2 to 13\\
1996/2659&The Social Security (Adjudication) Amendment (No.\ 2) Regulations 1996&The whole Regulations\\
1997/65&The Income-Related Benefits and Jobseeker’s Allowance (Miscellaneous Amendments) Regulations 1997&Regulation 16\\
1997/793&The Social Security (Miscellaneous Amendments) (No. 2) Regulations 1997&Regulations 1(2)($a$) and 8 to 17\\
1997/810&The Social Security (Industrial Injuries) (Miscellaneous Amendments) Regulations 1997&Regulations 2, 3 and 4\\
%1997/995
1997/955  % Entry amended (5.7.99) by SI 1999/1623 reg 6
&The Social Security (Adjudication) and Commissioners Procedure and Child Support Commissioners (Procedure) Amendment Regulations 1997&In regulation 1(2), the definition of “the Adjudication Regulations” and regulations 2 to 6\\
1997/1839&The Social Security (Attendance Allowance and Disability Living Allowance) (Miscellaneous Amendments) Regulations 1997&In regulation 1(2) the definition of “the Adjudication Regulations” and regulation 4\\
1997/2237&The Social Security (Recovery of Benefits) (Appeals) Regulations 1997&The whole Regulations\\
1997/2305&The Social Security (Miscellaneous Amendments) (No.\ 4) Regulations 1997&Regulation 4\\
%\end{tabulary}
\end{longtable}

}

\amendment{
Entry ``1997/995'' in col. 1 of Sch. 4 substituted (5.7.99) by the Social Security and Child Support (Decisions and Appeals) Amendment (No. 2) Regulations 1999 reg. 7.
}

\part{Explanatory Note}

\renewcommand\parthead{--- Explanatory Note}

\subsection*{(This note is not part of the Regulations)}

 These Regulations are made by virtue of, or in consequence of, provisions in the Social Security Act 1998 (c.\ 14) (“the Act”) and supplement changes introduced by that Act to the decision-making process for social security and related matters. The Regulations also provide for the procedural rules and other requirements of a new unified appeals system introduced by the Act for social security, contracting out of pensions, child support and vaccine damage.

  The Regulations are made before the end of the period of six months beginning with the coming into force of the relevant provisions in the Act and are therefore exempted from the requirement in section 172(1) of the Social Security Administration Act 1992 (c.\ 5) to refer proposals to make these Regulations to the Social Security Advisory Committee and are made without reference to that Committee. The Regulations are made after consultation with the Council on Tribunals in accordance with section 8 of the Tribunals and Inquiries Act 1992 (c.\ 53).

  Part I of the Regulations contains provisions relating to commencement, citation and interpretation as well as service of notices or documents.

  Part II of the Regulations makes provision for decisions by the Secretary of State in social security and related matters. Chapters I and II provide for the circumstances in which the Secretary of State may revise or supersede decisions, when such decisions take effect and related procedural rules. Chapter III contains miscellaneous provisions relating to decisions of the Secretary of State in social security, including certain relevant requirements formerly contained in the Social Security (Adjudication) Regulations 1995 and other related regulations which are now revoked. It also includes provision in relation to industrial injuries benefits for the Secretary of State to seek advice from a medical practitioner.

  Part III of the Regulations makes provision for the suspension and termination of benefit and for dealing with decisions where there are related appeals or decisions.

  Part IV concerns rights of appeal and the procedure for bringing appeals. Chapter I makes provision for additional persons to have a right of appeal, for decisions (other than those in the Act) against which there is no right of appeal and decisions where there is a right of appeal. It also provides procedural rules for bringing appeals.

  Part V makes provision for appeal tribunals for social security, contracting out of pensions, vaccine damage and child support. Chapter I provides for the panel from which appeal tribunal members are drawn to include people with specified qualifications, for requirements relating to the composition of appeal tribunals and the assignment of clerks to tribunals. Chapters II to V of that Part provide for further matters relating to appeals and referrals. Chapter II makes provision for procedural requirements in the determination of appeals and referrals, including withdrawal of appeals or referrals, medical examinations and witnesses. Chapter III provides for the striking out of certain appeals and related procedures. Chapter IV provides for procedures at oral hearings and Chapter V makes provision relating to the decisions and reports of appeal tribunals and applications for leave to a Commissioner.

  Part VI and Schedule 4 provide for revocations.

  These Regulations do not impose a charge on business.

\end{document}