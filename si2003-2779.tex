\documentclass[12pt,a4paper]{article}

\newcommand\regstitle{The Child Support (Miscellaneous Amendments) (No.\ 2) Regulations 2003}

\newcommand\regsnumber{2003/2779}

%\opt{newrules}{
\title{\regstitle}
%}

%\opt{2012rules}{
%\title{Child Maintenance and Other Payments Act 2008\\(2012 scheme version)}
%}

\author{S.I.\ 2003 No.\ 2779}

\date{Made
4th November 2003\\
%Laid before Parliament
%15th December 2003\\
Coming into force
in accordance with regulation 1
}

%\opt{oldrules}{\newcommand\versionyear{1993}}
%\opt{newrules}{\newcommand\versionyear{2003}}
%\opt{2012rules}{\newcommand\versionyear{2012}}

\usepackage{csa-regs}

\setlength\headheight{27.57402pt}

\begin{document}

\maketitle

\noindent
Whereas a draft of this instrument was laid before Parliament in accordance with section 52(2) of the Child Support Act 1991\footnote{1991 c.\ 48. Section 52(2) was amended by paragraph 15 of Schedule 3 to the Child Support Act 1995 (c.\ 34) and is substituted by section 25 of the Child Support, Pensions and Social Security Act 2000 (c.\ 19) (“the 2000 Act”).} and approved by resolution of each House of Parliament:

Now, therefore, the Secretary of State for Work and Pensions, in exercise of the powers conferred upon him by sections 12(2), 16(1), 17(3), 28E(5), 42, 46, 51, 52 and 54 of, and paragraphs 4, 5, 6, 10(1) and 11 of Schedule 1 and paragraphs 2, 3(1)($b$), 4(1)($b$)  and 5 of Schedule 4B to, the Child Support Act 1991\footnote{Section 12 was amended by section 11 of the Child Support Act 1995 and paragraph 25 of Schedule 7 to the Social Security Act 1998 (c.\ 14). Section 16 was substituted by section 40 of the Social Security Act 1998 and is amended by section 8 of the 2000 Act. Section 17 was substituted by section 41 of the Social Security Act 1998 and is amended by section 9 of the 2000 Act. Section 28E(5) was inserted by section 5 of the Child Support Act 1995. Section 42 is amended by paragraph 11(2) of Schedule 3 to the 2000 Act. Section 46 was amended by paragraph 12 of Schedule 3 to the Child Support Act 1995, paragraph 20(4) of Schedule 2 to the Jobseekers Act 1995 (c.\ 18) and paragraph 43 of Schedule 7 to the Social Security Act 1998 and is substituted by section 19 of the 2000 Act. Section 51 was amended by paragraph 46 of Schedule 7 to the Social Security Act 1998 and is amended by paragraph 11(19) of Schedule 3 to the 2000 Act. Paragraphs 4 and 10(1) of Schedule 1 are substituted by Schedule 1 to the 2000 Act. Paragraph 5 of Schedule 1 was amended by paragraph 20(7) of Schedule 2 to the Jobseekers Act 1995 and is substituted by Schedule 1 to the 2000 Act. Schedule 4B was inserted by Schedule 2 to the Child Support Act 1995 and substituted by Part II of Schedule 2 to the 2000 Act. Section 54 is cited because of the meaning ascribed to “prescribed”. \emph{See} also S.I.\ 2003/192 (C.11).} and section 29 of the Child Support, Pensions and Social Security Act 2000\footnote{2000 c.\ 19.} and of all other powers enabling him in that behalf, hereby makes the following Regulations: 

{\sloppy

\tableofcontents

}

\bigskip

\setcounter{secnumdepth}{-2}

\subsection[1. Citation and commencement]{Citation and commencement}

1.  These Regulations may be cited as the Child Support (Miscellaneous Amendments) (No.\ 2) Regulations 2003 and shall come into force on the day after the day that they are made.

\subsection[2. Amendment of the Child Support Departure Direction and Consequential Amendments Regulations 1996]{Amendment of the Child Support Departure Direction and Consequential Amendments Regulations 1996}

2.  In the Child Support Departure Direction and Consequential Amendments Regulations 1996\footnote{S.I.\ 1996/2907, which is revoked, with savings, by S.I.\ 2001/156.}—
\begin{enumerate}\item[]
($a$) in the heading to, and in paragraphs (1)($a$), ($b$)  and ($c$), (2)($a$)  and ($b$)  and (3)($a$), ($b$)  and ($c$)  of, regulation 9 (departure directions and persons in receipt of income support, income-based jobseeker’s allowance or working tax credit)\footnote{Relevant amending instruments are S.I.\ 1998/58 and 2003/328.}; and

($b$) in regulation 12 (meaning of “benefit” for the purposes of section 28E of the Child Support Act 1991)\footnote{Relevant amending instrument is S.I.\ 2003/328.};
\end{enumerate}
after “income support” insert “, state pension credit”.

\subsection[3. Amendment of the Child Support (Maintenance Assessment Procedure) Regulations 1992]{Amendment of the Child Support (Maintenance Assessment Procedure) Regulations 1992}

3.---(1)  The Child Support (Maintenance Assessment Procedure) Regulations 1992\footnote{S.I.\ 1992/1813, which is revoked, with savings, by S.I.\ 2001/157.} shall be amended in accordance with the following paragraphs.

(2) In regulation 1(3) (citation, commencement and interpretation)\footnote{Relevant amending instruments are S.I.\ 1995/3261 and 1996/1345.}—
\begin{enumerate}\item[]
($a$) in sub-paragraph ($a$), for the words after “payable, or” substitute “the circumstances in regulation 40(3) or regulation 40ZA(4), as the case may be, apply;”; and

($b$) in sub-paragraph ($b$)  for the words after “given,” substitute “, the circumstances in regulation 40(3) or regulation 40ZA(4), as the case may be, apply; or”.
\end{enumerate}

(3) In regulation 8D(8) (miscellaneous provisions in relation to interim maintenance assessments)\footnote{Regulation 8D was inserted by S.I.\ 1995/3261. Relevant amending instruments are S.I.\ 1996/3196, 1998/58 and 1999/1047.} and in regulation 30A(5) (effective dates of new maintenance assessments in particular cases)\footnote{Regulation 30A was inserted by S.I.\ 1995/3261. Relevant amending instruments are S.I.\ 1996/3196, 1998/58 and 1999/1047.}, after “income support” in each place where those words occur insert “, state pension credit”.

(4) In regulation 40 (suspension of a reduced benefit direction in specified circumstances)\footnote{Relevant amending instruments are S.I.\ 1993/913 and 1995/1045.}—
\begin{enumerate}\item[]
($a$) in the heading, for the words after “direction” substitute “(income support)”;

($b$) in paragraph (1)—
\begin{enumerate}\item[]
(i) for the words after “concerned” to “paragraph (3)”, substitute “but the circumstances in paragraph (3) apply to her”; and

(ii) for the words after “so long as” to “that paragraph”, substitute “those circumstances apply”;
\end{enumerate}

($c$) omit paragraph (1A);

($d$) in paragraph (2) omit “or (1A)”; and

($e$) for paragraph (3) substitute—
\begin{quotation}
“(3) The circumstances referred to in paragraph (1) are that—
\begin{enumerate}\item[]
($a$) she is resident in a care home or an independent hospital;

($b$) she is being provided with a care home service or an independent health care service; or

($c$) her applicable amount falls to be calculated under regulation 21 of and any of paragraphs 1 to 3 of Schedule 7 to the Income Support Regulations (patients)\footnote{S.I.\ 1987/1967. Relevant amending instruments are S.I.\ 1990/547, 1996/1803, 1998/563 and 2003/526 and 1195.}.
\end{enumerate}

(4) In paragraph (3)—
\begin{enumerate}\item[]
“care home” has the meaning assigned to it by section 3 of the Care Standards Act 2000\footnote{2000 c.\ 14.};

“care home service” has the meaning assigned to it by section 2(3) of the Regulation of Care (Scotland) Act 2001\footnote{2001 asp 8.};

“independent health care service” has the meaning assigned to it by section 2(5)($a$)  and ($b$)  of the Regulation of Care (Scotland) Act 2001; and

“independent hospital” has the meaning assigned to it by section 2 of the Care Standards Act 2000.”.
\end{enumerate}
\end{quotation}
\end{enumerate}

(5) In regulation 40ZA (suspension of a reduced benefit direction in specified circumstances)\footnote{Regulation 40ZA was inserted by S.I.\ 1996/1345.}—
\begin{enumerate}\item[]
($a$) in the heading, for the words after “direction” substitute “(income-based jobseeker’s allowance)”;

($b$) in paragraph (1)—
\begin{enumerate}\item[]
(i) for the words after “concerned” to “paragraph (4)”, substitute “but the circumstances in paragraph (4) apply to her”; and

(ii) for the words after “so long as” to “those provisions”, substitute “those circumstances apply”;
\end{enumerate}

($c$) omit paragraph (2);

($d$) in paragraph (3), omit “or (2)”; and

($e$) for paragraph (4), substitute—
\begin{quotation}
“(4) The circumstances referred to in paragraph (1) are that—
\begin{enumerate}\item[]
($a$) she is resident in a care home or an independent hospital;

($b$) she is being provided with a care home service or an independent health care service; or

($c$) her applicable amount falls to be calculated under regulation 85 of and paragraph 1 or 2 of Schedule 5 to the Jobseeker’s Allowance Regulations (patients)\footnote{S.I.\ 1996/207. Relevant amending instruments are S.I.\ 1996/1516 and 2003/526 and 1195.}.
\end{enumerate}

(5) In paragraph (4)—
\begin{enumerate}\item[]
“care home” has the meaning assigned to it by section 3 of the Care Standards Act 2000;

“care home service” has the meaning assigned to it by section 2(3) of the Regulation of Care (Scotland) Act 2001;

“independent health care service” has the meaning assigned to it by section 2(5)($a$)  and ($b$)  of the Regulation of Care (Scotland) Act 2001; and

“independent hospital” has the meaning assigned to it by section 2 of the Care Standards Act 2000.”.
\end{enumerate}
\end{quotation}
\end{enumerate}

\subsection[4. Amendment of the Child Support (Maintenance Assessments and Special Cases) Regulations 1992]{Amendment of the Child Support (Maintenance Assessments and Special Cases) Regulations 1992}

4.---(1)  The Child Support (Maintenance Assessments and Special Cases) Regulations 1992\footnote{S.I.\ 1992/1815, which is revoked, with savings, by S.I.\ 2001/155.} shall be amended in accordance with the following paragraphs.

(2) In regulation 1(2) (citation, commencement and interpretation)\footnote{Relevant amending instrument is S.I.\ 1999/1510 (C.43).}—
\begin{enumerate}\item[]
($a$) in the definition of “home”, for “a residential care home or a nursing home” substitute “a care home or an independent hospital or to the provision of a care home service or an independent health care service”;

($b$) omit the definitions of “nursing home” and “residential care home”;

($c$) after the definition of “the Act” insert—
\begin{quotation}
““care home” has the meaning assigned to it by section 3 of the Care Standards Act 2000;

“care home service” has the meaning assigned to it by section 2(3) of the Regulation of Care (Scotland) Act 2001;”;
\end{quotation}

($d$) after the definition of “Income Support Regulations” insert—
\begin{quotation}
““independent health care service” has the meaning assigned to it by section 2(5)($a$)  and ($b$)  of the Regulation of Care (Scotland) Act 2001;

“independent hospital” has the meaning assigned to it by section 2 of the Care Standards Act 2000;”; and
\end{quotation}

($e$) after the definition of “self-employed earner” insert—
\begin{quotation}
““state pension credit” means the social security benefit of that name payable under the State Pension Credit Act 2002\footnote{2002 c.\ 16.};”.
\end{quotation}
\end{enumerate}

(3) In regulation 9(1) (exempt income: calculation or estimation of E)\footnote{Relevant amending instruments are S.I.\ 1995/1045 and 3261, 1996/1803 and 1945, 1998/58, 2002/1204 and 2003/328.}, for paragraph ($h$)  substitute—
\begin{quotation}
“($h$) where the absent parent or his partner is resident in a care home or an independent hospital or is being provided with a care home service or an independent health care service, the amount of fees paid in respect of that home, hospital or service, as the case may be, but where it has been determined that the absent parent in question or his partner is entitled to housing benefit in respect of fees for that home, hospital or service, as the case may be, the net amount of such fees after deduction of housing benefit;”.
\end{quotation}

(4) After regulation 10A (assessable income: working tax credit paid to or in respect of a parent with care or an absent parent)\footnote{Regulation 10A was inserted by S.I.\ 1996/3196. Relevant amending instruments are S.I.\ 1999/1510 (C.43) and 2003/328.} insert—
\begin{quotation}
\subsection*{“Assessable income: state pension credit paid to or in respect of a parent with care or an absent parent}

10B.  Where state pension credit is paid to or in respect of a parent with care or an absent parent, that parent shall, for the purposes of Schedule 1 to the Act, be taken to have no assessable income.”.
\end{quotation}

(5) In regulation 11(1) (protected income)\footnote{Relevant amending instruments are S.I.\ 1994/227, 1995/1045 and 3261, 1996/1803 and 1945, 1998/58 and 2003/328.}, for paragraph ($i$)  substitute—
\begin{quotation}
“($i$) where the absent parent or his partner is resident in a care home or an independent hospital or is being provided with a care home service or an independent health care service, the amount of fees paid in respect of that home, hospital or service, as the case may be, but where it has been determined that the absent parent in question or his partner is entitled to housing benefit in respect of fees for that home, hospital or service, as the case may be, the net amount of such fees after deduction of housing benefit;”.
\end{quotation}

(6) In Schedule 1 (calculation of N and M)\footnote{Relevant amending instruments are S.I.\ 1993/913, 1995/1045, 1996/1345, 1803, 1945 and 3196, 1998/58, 1999/977 and 1510 (C.43) and 2003/328.}—
\begin{enumerate}\item[]
($a$) in paragraph 9A(2) after “war widow’s pension” insert “and a war widower’s pension”; and

($b$) in paragraph 22(1B)(2) after “war widow’s pension” insert “and a war widower’s pension”.
\end{enumerate}

(7) In Schedule 2 (amounts to be disregarded when calculating or estimating N and M)\footnote{Relevant amending instruments are S.I.\ 1993/913, 1995/1045 and 3261, 1996/481, 1345 and 3196, 1998/58, 1999/977 and 2003/328.}—
\begin{enumerate}\item[]
($a$) in paragraph 15 after “income support” insert “, state pension credit”;

($b$) in paragraph 18($a$)  and ($b$)(ii)  after “war widow’s pension” insert “or war widower’s pension”;

($c$) for paragraph 48E substitute—
\begin{quotation}
“48E.  Any payment made by a local authority, or by the National Assembly for Wales, to a person relating to a service which is provided to develop or sustain the capacity of that person to live independently in his accommodation.”; and
\end{quotation}

($d$) after paragraph 48E add—
\begin{quotation}
“48F.  Any supplementary pension under article 29(1A) of the Naval, Military and Air Forces etc.\ (Disablement and Death) Service Pensions Order 1983 (pensions to widows and widowers)\footnote{S.I.\ 1983/883. Article 29(1A) was inserted by S.I.\ 1994/1906. Relevant amending instruments are S.I.\ 2002/792 and 2003/434.} or under article 27(3) of the Personal Injuries (Civilians) Scheme 1983 (pensions to widows and widowers)\footnote{S.I.\ 1983/686. Article 27(3) was added by S.I.\ 1994/2021. Relevant amending instrument is S.I.\ 2002/672.}.”.
\end{quotation}
\end{enumerate}

\subsection[5. Amendment of the Child Support (Maintenance Calculation Procedure) Regulations 2000]{Amendment of the Child Support (Maintenance Calculation Procedure) Regulations 2000}

5.---(1)  The Child Support (Maintenance Calculation Procedure) Regulations 2000\footnote{S.I.\ 2001/157.} shall be amended in accordance with the following paragraphs.

(2) In regulation 8(2)($b$)  (interpretation of Part IV)—
\begin{enumerate}\item[]
($a$) in head (i)  for the words after “payable, or” substitute “the circumstances in regulation 14(4) or 15(4), as the case may be, apply;”; and

($b$) in head (ii)  for the words after “given,” substitute “the circumstances in regulation 14(4) or 15(4), as the case may be, apply,”.
\end{enumerate}

(3) In regulation 14 (suspension of a reduced benefit decision in specified circumstances (income support))—
\begin{enumerate}\item[]
($a$) in the heading, omit “when a modified applicable amount is payable”;

($b$) in paragraph (1)—
\begin{enumerate}\item[]
(i) for the words after “concerned” to “paragraph (4)”, substitute “but the circumstances in paragraph (4) apply to her”; and

(ii) for the words after “so long as” to “that paragraph”, substitute “those circumstances apply”;
\end{enumerate}

($c$) omit paragraph (2);

($d$) in paragraph (3) omit “or (2)”; and

($e$) for paragraph (4) substitute—
\begin{quotation}
“(4) The circumstances referred to in paragraph (1) are that—
\begin{enumerate}\item[]
($a$) she is resident in a care home or an independent hospital;

($b$) she is being provided with a care home service or an independent health care service; or

($c$) her applicable amount falls to be calculated under regulation 21 of and any of paragraphs 1 to 3 of Schedule 7 to the Income Support Regulations (patients).
\end{enumerate}

(5) In paragraph (4)—
\begin{enumerate}\item[]
“care home” has the meaning assigned to it by section 3 of the Care Standards Act 2000;

“care home service” has the meaning assigned to it by section 2(3) of the Regulation of Care (Scotland) Act 2001;

“independent health care service” has the meaning assigned to it by section 2(5)($a$)  and ($b$)  of the Regulation of Care (Scotland) Act 2001; and

“independent hospital” has the meaning assigned to it by section 2 of the Care Standards Act 2000.”.
\end{enumerate}
\end{quotation}
\end{enumerate}

(4) In regulation 15 (suspension of a reduced benefit decision in specified circumstances (income-based jobseeker’s allowance))—
\begin{enumerate}\item[]
($a$) in the heading, omit “when a modified applicable amount is payable”;

($b$) in paragraph (1)—
\begin{enumerate}\item[]
(i) for the words after “concerned” to “paragraph (4)”, substitute “but the circumstances in paragraph (4) apply to her”; and

(ii) for the words after “so long as” to “those provisions”, substitute “those circumstances apply”;
\end{enumerate}

($c$) omit paragraph (2);

($d$) in paragraph (3) omit “or (2)”; and

($e$) for paragraph (4) substitute—
\begin{quotation}
“(4) The circumstances referred to in paragraph (1) are that—
\begin{enumerate}\item[]
($a$) she is resident in a care home or an independent hospital;

($b$) she is being provided with a care home service or an independent health care service; or

($c$) her applicable amount falls to be calculated under regulation 85 of and paragraph 1 or 2 of Schedule 5 to the Jobseeker’s Allowance Regulations (patients).
\end{enumerate}

(5) In paragraph (4)—
\begin{enumerate}\item[]
“care home” has the meaning assigned to it by section 3 of the Care Standards Act 2000;

“care home service” has the meaning assigned to it by section 2(3) of the Regulation of Care (Scotland) Act 2001;

“independent health care service” has the meaning assigned to it by section 2(5)($a$)  and ($b$)  of the Regulation of Care (Scotland) Act 2001; and

“independent hospital” has the meaning assigned to it by section 2 of the Care Standards Act 2000.”.
\end{enumerate}
\end{quotation}
\end{enumerate}

\subsection[6. Amendment of the Child Support (Maintenance Calculations and Special Cases) Regulations 2000]{Amendment of the Child Support (Maintenance Calculations and Special Cases) Regulations 2000}

6.---(1)  The Child Support (Maintenance Calculations and Special Cases) Regulations 2000\footnote{S.I.\ 2001/155.} shall be amended in accordance with the following paragraphs.

(2) In regulation 1(2) (citation, commencement and interpretation)\footnote{Relevant amending instruments are S.I.\ 2002/3019 and 2003/328.}—
\begin{enumerate}\item[]
($a$) in the definition of “home”, for “a residential care home or a nursing home” substitute “a care home or an independent hospital or the provision of a care home service or an independent health care service”;

($b$) omit the definitions of “nursing home” and “residential care home”;

($c$) after the definition of “the Act” insert—
\begin{quotation}
““care home” has the meaning assigned to it by section 3 of the Care Standards Act 2000;

“care home service” has the meaning assigned to it by section 2(3) of the Regulation of Care (Scotland) Act 2001;”;
\end{quotation}

($d$) after the definition of “Income Support Regulations” insert—
\begin{quotation}
““independent health care service” has the meaning assigned to it by section 2(5)($a$)  and ($b$)  of the Regulation of Care (Scotland) Act 2001;

“independent hospital” has the meaning assigned to it by section 2 of the Care Standards Act 2000;”;
\end{quotation}

($e$) in the definition of “occupational pension scheme” after “1988” add “or is a statutory scheme to which section 594 of that Act applies”;

($f$) for the definition of “training allowance” substitute—
\begin{quotation}
““training allowance” means a payment under section 2 of the Employment and Training Act 1973 (“the 1973 Act”)\footnote{1973 c. 50. Section 2 was substituted by section 25(1) of the Employment Act 1988 (c.\ 19).}, or section 2 of the Enterprise and New Towns (Scotland) Act 1990 (“the 1990 Act”)\footnote{1990 c.\ 35.}, which is paid—
\begin{enumerate}\item[]
($a$) 
to a person for his maintenance; and

($b$) 
in respect of a period during which that person—
\begin{enumerate}\item[]
(i) 
is undergoing training pursuant to arrangements made under section 2 of the 1973 Act or section 2 of the 1990 Act; and

(ii) 
has no net weekly income of a type referred to in Part II or Part III of the Schedule;”; and
\end{enumerate}
\end{enumerate}
\end{quotation}

($g$) after the definition of “training allowance” insert—
\begin{quotation}
““war widow’s pension” means any pension or allowance payable for a widow which is—
\begin{enumerate}\item[]
($a$) 
granted in respect of a death due to service or war injury and payable by virtue of the Air Force (Constitution) Act 1917\footnote{7 \& 8 Geo. 5 c.\ 51. Section 3 was amended by and section 13 was repealed by S.I.\ 1964/488. Section 4 was amended by the Armed Forces Act 1981 (c.\ 55). Sections 5 and 11 were repealed by the Statute Law Revision Act 1927 (17 \& 18 Geo.\ 5 c.\ 42). Section 6 was repealed by the Statute Law (Repeals) Act 1976 (c.\ 16). Section 7 and Schedule 1 were repealed by the Naval Discipline Act 1957 (5 \& 6 Eliz.\ 2 c.\ 53). Sections 8 to 10 were repealed by the Defence (Transfer of Functions) Act 1964 (c.\ 15). Section 12 and Schedule 2 were repealed by the Revision of the Army and Air Forces Acts (Transitional Provisions) Act 1955 (3 \& 4 Eliz.\ 2 c.\ 20).}, the Personal Injuries (Emergency Provisions) Act 1939\footnote{2 \& 3 Geo.\ 6 c.\ 82. Section 2 was amended, and sections 3, 4 and 5 were repealed, by the Statute Law Revision Act 1953 (1 \& 2 Eliz.\ 2 c.\ 5). Section 6 was repealed by the Theft Act 1968 (c.\ 60). Section 8 was modified by the Northern Ireland Act 1998 (c.\ 47). Section 9 was amended by the Statute Law Revision Act 1950 (14 Geo.\ 6 c.\ 6).}, the Pensions (Navy, Army, Air Force and Mercantile Marine) Act 1939\footnote{2 \& 3 Geo.\ 6 c.\ 83. Section 1 was repealed by S.I.\ 1964/488. Section 2 was repealed by the War Orphans Act 1942 (5 \& 6 Geo.\ 6 c.\ 8). Sections 3, 4, 5, 6, 7 and 10 were amended by the Pensions (Mercantile Marine) Act 1942 (5 \& 6 Geo.\ 6 c.\ 26). Section 4 was amended by the Pilotage Act 1983 (c.\ 21). Section 5 was amended by the Armed Forces Act 1981 (c.\ 55). Section 6 was amended by the Merchant Shipping Act 1970 (c.\ 36). Section 8 was repealed by the Theft Act 1968. Section 9 was repealed by S.I.\ 1965/145.}, the Polish Resettlement Act 1947\footnote{10 \& 11 Geo.\ 6 c.\ 19. Section 2 and the Schedule were amended by the National Assistance Act 1948 (11 \& 12 Geo.\ 6 c.\ 29). Section 3 was amended by S.I.\ 1951/174 and 1968/1699, the Supplementary Benefits Act 1976 (c.\ 71) and the Social Security Act 1980 (c.\ 30). Section 4 was amended by S.I.\ 1968/1699, the National Health Service Act 1977 (c.\ 49), the Social Security Act 1980, the Mental Health Act 1983 (c.\ 20) and the Health Authorities Act 1995 (c.\ 17). Sections 6, 7 and 12 and the Schedule were amended by the Social Security Act 1980. Sections 8 and 9 were repealed, and sections 10 and 12 were amended, by the Statute Law Revision Act 1953. Section 11 was amended by the Mental Health Act 1983 and the Mental Health (Scotland) Act 1984 (c.\ 36).} or Part VII or section 151 of the Reserve Forces Act 1980\footnote{1980 c.\ 9. Part VII was amended by the Armed Forces Act 1981, the Army Act 1992 (c.\ 39), the Statute Law (Repeals) Act 1993 (c.\ 50) and the Reserve Forces Act 1996 (c.\ 14).};

($b$) 
payable under so much of any Order in Council, Royal Warrant, order or scheme as relates to death due to service in the armed forces of the Crown, wartime service in the merchant navy or war injuries;

($c$) 
payable in respect of death due to peacetime service in the armed forces of the Crown before 3rd September 1939, and payable at rates, and subject to conditions, similar to those of a pension within sub-paragraph ($b$); or

($d$) 
payable under the law of a country other than the United Kingdom and of a character substantially similar to a pension within sub-paragraph ($a$), ($b$)  or ($c$),
\end{enumerate}
and “war widower’s pension” shall be construed accordingly;”.
\end{quotation}
\end{enumerate}

(3) In regulation 4(1) (flat rate)—
\begin{enumerate}\item[]
($a$) in sub-paragraph ($e$), omit “or war widow’s pension”; and

($b$) after sub-paragraph ($e$)  add—
\begin{quotation}
    “and

    ($f$) 
    a war widow’s pension or a war widower’s pension.”. 
\end{quotation}
\end{enumerate}

(4) In regulation 5 (nil rate)\footnote{Relevant amending instruments are S.I.\ 2002/3019 and 2003/1195.}, in paragraph ($f$), for “in a residential care home or nursing home” substitute “who is resident in a care home or an independent hospital or is being provided with a care home service or an independent health care service”.

(5) In regulation 11 (non-resident parent liable to pay maintenance under a maintenance order)—
\begin{enumerate}\item[]
($a$) in paragraph (3) after “applies,” insert “subject to paragraph (5),”; and

($b$) after paragraph (4) add—
\begin{quotation}
“(5) If the application of paragraph (3) would decrease the weekly amount of child support maintenance (or the aggregate of all such amounts) payable by the non-resident parent to the person with care (or all of them) to an amount which is less than a figure equivalent to the flat rate of child support maintenance payable under paragraph 4(1) of Schedule 1 to the Act, the non-resident parent shall instead be liable to pay child support maintenance at a rate equivalent to that flat rate apportioned (if appropriate) as provided in paragraph 6 of Schedule 1 to the Act.”.
\end{quotation}
\end{enumerate}

\subsection[7. Amendment of the Child Support (Transitional Provisions) Regulations 2000]{Amendment of the Child Support (Transitional Provisions) Regulations 2000}

7.---(1)  The Child Support (Transitional Provisions) Regulations 2000\footnote{S.I.\ 2000/3186.} shall be amended in accordance with the following paragraphs.

(2) In regulation 2(1) (interpretation)\footnote{Relevant amending instrument is S.I.\ 2003/328.}, in the definition of “maximum transitional amount” for the words after “amount”” substitute “has the meaning given in regulation 25(5), (6) or (7), whichever is applicable;”.

(3) In regulation 7($g$)  (grounds on which a conversion decision may not be revised, superseded or altered on appeal) in sub-paragraph (ii)  omit—
\begin{enumerate}\item[]
($a$) “where” where it first appears in that sub-paragraph; and

($b$) “the relevant property transfer to be replaced with”.
\end{enumerate}

(4) In regulation 17 (relevant departure direction and relevant property transfer)\footnote{Relevant amending instrument is S.I.\ 2002/1204.}, after paragraph (9) add—
\begin{quotation}
“(10) Where—
\begin{enumerate}\item[]
($a$) a relevant property transfer is taken into account for the purposes of a conversion decision;

($b$) an application is made for a variation of a type referred to in paragraph 3 of Schedule 4B to the Act and Part IV of the Variations Regulations (property or capital transfers)\footnote{Relevant amending instrument is S.I.\ 2002/1204.} which relates to the same property or capital transfer as the relevant property transfer referred to in sub-paragraph ($a$); and

($c$) the variation is agreed to,
\end{enumerate}
the relevant property transfer shall cease to have effect on the effective date of the subsequent decision which resulted from the application for a variation.”.
\end{quotation}

(5) In regulation 24 (phasing amount)\footnote{Relevant amending instruments are S.I.\ 2002/1204 and 2003/328.}—
\begin{enumerate}\item[]
($a$) in paragraph (3) for “and (5)” substitute “, (5) and (6)”; and

($b$) after paragraph (5) add—
\begin{quotation}
“(6) Where a subsequent decision is made the effective date of which is the case conversion date—
\begin{enumerate}\item[]
($a$) the reference in paragraph (3) to the conversion decision shall apply as if it were a reference to the subsequent decision; and

($b$) the reference in paragraph (5) to the new amount shall apply as if it were a reference to the subsequent decision amount.”.
\end{enumerate}
\end{quotation}
\end{enumerate}

(6) In regulation 25 (maximum transitional amount)\footnote{Relevant amending instrument is S.I.\ 2003/328.} after paragraph (4) add—
\begin{quotation}
“(5) Subject to paragraphs (6) and (7), “maximum transitional amount” means 30\% of the non-resident parent’s net weekly income taken into account in the conversion decision, or the subsequent decision, as the case may be.

(6) Where the new amount is calculated under regulation 22(1)\footnote{Relevant amending instruments are S.I.\ 2002/1204 and 2003/328.}, “maximum transitional amount” means 30\% of the aggregate of the income calculated under regulation 22(1)($b$) .

(7) Where the new amount or the subsequent decision amount, as the case may be, is calculated under regulation 26(1) of the Variations Regulations “maximum transitional amount” means 30\% of the additional income arising under the variation.”.
\end{quotation}

(7) In regulation 27 (subsequent decision with effect in transitional period—amount payable)\footnote{Relevant amending instruments are S.I.\ 2002/1204 and 2003/328.}—
\begin{enumerate}\item[]
($a$) in paragraph (9) for “Where” substitute “Subject to paragraph (10), where”; and

($b$) after paragraph (9) add—
\begin{quotation}
“(10) Where a subsequent decision (“decision B”) is made in respect of a decision which is itself a subsequent decision (“decision A”) and—
\begin{enumerate}\item[]
($a$) decision B has the same effective date as decision A; or

($b$) decision B—
\begin{enumerate}\item[]
(i) is a revision or alteration on appeal of decision A; and

(ii) includes within it a determination that the effective date of decision A was incorrect,
\end{enumerate}
\end{enumerate}
paragraphs (2) to (5) shall apply so that the subsequent decision amount of decision B is compared with the new amount or the subsequent decision amount, as the case may be, which was in place immediately before decision A was made.”.
\end{quotation}
\end{enumerate}

\subsection[8. Amendment of the Child Support (Variations) Regulations 2000]{Amendment of the Child Support (Variations) Regulations 2000}

8.  In regulation 28 (transitional provisions—conversion decisions) of the Child Support (Variations) Regulations 2000\footnote{S.I.\ 2001/156.} for “Where” substitute “Subject to regulation 17(10) of the Transitional Regulations, where”.

\subsection[9. Savings]{Savings}

9.  Regulations 1(3), 40 and 40ZA of the Child Support (Maintenance Assessment Procedure) Regulations 1992 and regulations 8(2)($b$), 14 and 15 of the Child Support (Maintenance Calculation Procedure) Regulations 2000 shall continue to have effect in relation to a person to whom any of those provisions applied before the date these Regulations come into force as if regulations 3(2), (4) and (5) and 5 of these Regulations had not come into force. 

\bigskip

Signed 
by authority of the Secretary of State for Work and Pensions.

{\raggedleft
\emph{P.~Hollis}\\*Parliamentary Under-Secretary of State,\\*Department of Work and Pensions

}

%St Andrew's House, Edinburgh

%Dated
4th November 2003

\small

\part{Explanatory Note}

\renewcommand\parthead{— Explanatory Note}

\subsection*{(This note is not part of the Regulations)}

These Regulations provide for the amendment of regulations relating to child support.

The powers exercised to make these Regulations are those contained in the Child Support Act 1991 (“the 1991 Act”). Some of those powers are conferred by provisions of the 1991 Act prior to the amendments made to the 1991 Act by the Child Support, Pensions and Social Security Act 2000 (“the 2000 Act”), which amendments are not yet fully in force, and relate to the child support scheme which was in force prior to 3rd March 2003 and which remains in force for the purposes of certain cases (“the old scheme”). Other powers are conferred by provisions of the 1991 Act as amended by the 2000 Act, which relate to the child support scheme provided for by those amendments, which came into force for the purposes of specified categories of cases on 3rd March 2003 (\emph{see} the Child Support, Pensions and Social Security Act 2000 (Commencement No.\ 12) Order 2003) (“the new scheme”).

Regulations 2, 3(3), 4(2)($e$), (4) and (7)($a$)  amend, respectively, the Child Support Departure Direction and Consequential Amendments Regulations 1996, the Child Support (Maintenance Assessment Procedure) Regulations 1992 (“the Assessment Procedure Regulations”) and the Child Support (Maintenance Assessments and Special Cases) Regulations 1992 (“the Assessments and Special Cases Regulations”), all of which relate to the old scheme, in consequence of the introduction of state pension credit.

Regulations 3(2), (4) and (5) and 5 respectively amend the Assessment Procedure Regulations, which relate to the old scheme, and the Child Support (Maintenance Calculation Procedure) Regulations 2000, which relate to the new scheme. The amendments are made in consequence of the abolition of the residential allowance in income support and income-based jobseeker’s allowance and other changes resulting from the coming into force of the Care Standards Act 2000 and the Regulation of Care (Scotland) Act 2001. Regulation 9 makes savings provisions in respect of the amendments made by regulations 3(2), (4) and (5) and 5.

Regulations 4(2)($a$)  to ($d$), (3) and (5) and 6(2)($a$)  to ($d$)  and (4) respectively amend the Assessments and Special Cases Regulations and the Calculations and Special Cases Regulations to reflect changes in terminology introduced in the Care Standards Act 2000 and the Regulation of Care (Scotland) Act 2001 and make provision for how persons who are in a care home or an independent hospital or who are being provided with a care home service or an independent health care service should be treated for child support purposes.

Regulations 4(6) and (7)($b$)  and ($d$)  and 6(2)($g$)  and (3) respectively amend the Assessments and Special Cases Regulations, which relate to the old scheme, and the Child Support (Maintenance Calculations and Special Cases) Regulations 2000 (“the Calculations and Special Cases Regulations”), which relate to the new scheme. The amendments provide for a war widower’s pension to be treated for child support purposes in the same way as a war widow’s pension is to be treated under provisions already in force and for the treatment for child support purposes of specific payments for war widows and widowers.

Regulation 4(7)($c$)  amends the provision in the Assessments and Special Cases Regulations which provides that specified payments from local authorities or the National Assembly for Wales shall not be counted as income for child support purposes.

In the Calculations and Special Cases Regulations, regulation 6(2)($e$)  amends the definition of “occupational pension scheme”, regulation 6(2)($f$)  substitutes the definition of “training allowance” and regulation 6(5) amends the provision for cases where a non-resident parent is liable to pay child support maintenance as well as child maintenance under a court order in respect of a different child.

Regulation 7 amends the Child Support (Transitional Provisions) Regulations 2000 (“the Transitional Regulations”), which make provision for the conversion of cases from the old scheme to the new scheme. Regulation 7(2) and (6) amends provisions in respect of the “maximum transitional amount”, which is the most a non-resident parent can be required to pay in child support maintenance, to make provision for cases where regulation 22 of the Transitional Regulations or regulation 26 of the Child Support (Variations) Regulations 2000 (“the Variations Regulations”) applies. Regulation 7(3) amends regulation 7($g$)(ii)  of the Transitional Regulations to link the ground set out in that provision to the making of an application for a variation in relation to the same transfer of capital or property as has been taken into account as a “relevant property transfer” in the conversion decision. Regulation 7(4) amends the Transitional Regulations to ensure that a relevant property transfer and a variation cannot be in force at the same time in relation to the same property or capital transfer. Regulation 7(5) and (7) amends regulations 24 and 27 of the Transitional Regulations respectively to make provision for specific cases where a subsequent decision is made.

Regulation 8 makes an amendment to the Variations Regulations, which relate to the new scheme, in consequence of the amendment made by regulation 7(4).

These Regulations do not impose any costs on business. 

\end{document}