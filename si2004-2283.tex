\documentclass[12pt,a4paper]{article}

\newcommand\regstitle{The Social Security (Retirement Pensions) Amendment Regulations 2004}

\newcommand\regsnumber{2004/2283}

%\opt{newrules}{
\title{\regstitle}
%}

%\opt{2012rules}{
%\title{Child Maintenance and~Other Payments Act 2008\\(2012 scheme version)}
%}

\author{S.I.\ 2004 No.\ 2283}

\date{Made
2nd September 2004\\
Laid before Parliament
6th September 2004\\
Coming into~force
27th September 2004
}

%\opt{oldrules}{\newcommand\versionyear{1993}}
%\opt{newrules}{\newcommand\versionyear{2003}}
%\opt{2012rules}{\newcommand\versionyear{2012}}

\usepackage{csa-regs}

\setlength\headheight{27.61603pt}

%\hbadness=10000

\begin{document}

\maketitle

\noindent
The Secretary of State for Work and Pensions, in exercise of the powers conferred upon him by sections~5(1)($b$), 189(1) and (4) and 191 of the Social Security Administration Act 1992\footnote{1992 c.~5; section~191 is cited because of the meaning it gives to “prescribe”.} and sections~9(4), 79(1) and (4) and 84 of the Social Security Act 1998\footnote{1998 c.~14; section~84 is cited because of the meaning it gives to “prescribe”.} and of all other powers enabling him in that behalf, after agreement by the Social Security Advisory Committee that the proposals in respect of these Regulations should not be referred to it\footnote{\emph{See} sections~170 and 173(1)($b$)  of the Social Security Administration Act 1992.}, hereby makes the following Regulations: 

{\sloppy

\tableofcontents

}

\bigskip

\setcounter{secnumdepth}{-2}

\subsection[1. Citation and commencement]{Citation and commencement}

1.  These Regulations may be cited as the Social Security (Retirement Pensions) Amendment Regulations 2004 and shall come into force on 27th September 2004.

\subsection[2. Amendment of the Social Security (Claims and Payments) Regulations 1987]{Amendment of the Social Security (Claims and Payments) Regulations 1987}

2.  In the Social Security (Claims and Payments) Regulations 1987\footnote{S.I.~1987/1968.}, in regulations 6 (date of claim), after paragraph (30)\footnote{Paragraph (30) was added by S.I.~2002/428.} add—
\begin{quotation}
“(31) Subject to paragraph (32), where—
\begin{enumerate}\item[]
($a$) a person—
\begin{enumerate}\item[]
(i) has attained pensionable age, but for the time being makes no claim for a Category A retirement pension; or

(ii) has attained pensionable age and has a spouse who has attained pensionable age, but for the time being makes no claim for a Category B retirement pension;
\end{enumerate}

($b$) in accordance with regulation 50A of the Social Security (Contributions) Regulations 2001\footnote{S.I.~2001/1004; regulation 50A was inserted by S.I.~2004/1362.}, (Class 3 contributions: tax years 1996--97 to 2001--02) the Commissioners of Inland Revenue subsequently accept Class 3 contributions paid after the due date by the person or, in the case of a Category B retirement pension, the spouse;

($c$) in accordance with regulation 6A of the Social Security (Crediting and Treatment of Contributions, and National Insurance Numbers) Regulations 2001\footnote{S.I.~2001/769; regulation 6A was inserted by S.I.~2004/1361.} the contributions are treated as paid on a date earlier than the date on which they were paid; and

($d$) the person claims a Category A or, as the case may be, a Category B retirement pension,
\end{enumerate}
the claim shall be treated as made on—
\begin{enumerate}\item[]
(i) 1st October 1998; or

(ii) the date on which the person attained pensionable age in the case of a Category A retirement pension, or, in the case of a Category B retirement pension, the date on which the person’s spouse attained pensionable age,
\end{enumerate}
whichever is later.

(32) Paragraph (31) shall not apply where—
\begin{enumerate}\item[]
($a$) the person’s entitlement to a Category A or B retirement pension has been deferred by virtue of section~55(2)($a$)  of the Contributions and Benefits Act\footnote{Section 55 was substituted by the Pensions Act 1995 (c.~26), section~134(3).} (increase of retirement pension where entitlement is deferred); or

($b$) the person’s nominal entitlement to a Category A or B retirement pension is deferred in pursuance of section~36(4) and (7) of the National Insurance Act 1965\footnote{1965 c.~51; section~36(4) was continued in force and modified by S.I.~1978/393, regulation 3(3).} (increase of graduated retirement benefit where entitlement is deferred),
\end{enumerate}
nor where sub-paragraph ($a$)  and ($b$)  both apply.”.
\end{quotation}

\subsection[3. Amendment of the Social Security and Child Support (Decisions and Appeals) Regulations 1999]{Amendment of the Social Security and Child Support (Decisions and Appeals) Regulations 1999}

3.  In the Social Security and Child Support (Decisions and Appeals) Regulations 1999\footnote{S.I.~1999/991.}, regulation 5 (date from which a decision revised under section~9 takes effect) shall be renumbered paragraph (1) of regulation 5 and after paragraph (1) add—
\begin{quotation}
“(2) Where—
\begin{enumerate}\item[]
($a$) a person attains pensionable age, claims a retirement pension after the prescribed time for claiming and the Secretary of State decides (“the original decision”) that he is not entitled because—
\begin{enumerate}\item[]
(i) in the case of a Category A retirement pension, the person has not satisfied the contribution conditions; or

(ii) in the case of a Category B retirement pension, the person’s spouse has not satisfied the contribution conditions;
\end{enumerate}

($b$) in accordance with regulation 50A of the Social Security (Contributions) Regulations 2001\footnote{S.I.~2001/1004; regulation 50A was inserted by S.I.~2004/1362.} (Class 3 contributions: tax years 1996--97 to 2001--02) the Board subsequently accepts Class 3 contributions paid after the due date by the claimant or, as the case may be, the spouse;

($c$) in accordance with regulation 6A of the Social Security (Crediting and Treatment of Contributions, and National Insurance Numbers) Regulations 2001\footnote{S.I.~2001/769; regulation 6A was inserted by S.I.~2004/1361.} the contributions are treated as paid on a date earlier than the date on which they were paid; and

($d$) the Secretary of State revises the original decision in accordance with regulation 11A(4)($a$),
\end{enumerate}
the revised decision shall take effect from—
\begin{enumerate}\item[]
(i) 1st October 1998; or

(ii) the date on which the claimant attained pensionable age in the case of a Category A pension, or, in the case of a Category B pension, the date on which the claimant’s spouse attained pensionable age,
\end{enumerate}
whichever is later.”.
\end{quotation}

\bigskip

%Signed 
%by authority of the 
%Secretary of State for~Work and~Pensions.
%I concur
%By authority of the Lord Chancellor

{\raggedleft
\emph{Andrew Smith}\\*
Secretary
%Minister
%Parliamentary Under-Secretary 
of State,\\*Department 
for~Work and~Pensions

}

2nd September 2004

\small

\part{Explanatory Note}

\renewcommand\parthead{— Explanatory Note}

\subsection*{(This note is not part of the Regulations)}

These Regulations amend the Social Security (Claims and Payments) Regulations 1987 and the Social Security and Child Support (Decisions and Appeals) Regulations 1999 in respect of late claims for retirement pension where the claimant or his spouse has paid insufficient contributions for the tax years 1996--97 to 2001--02.

Regulation 2 provides for a late claim for a Category A or Category B retirement pension to be treated as made at pensionable age when those deficient contributions are paid. Regulation 3 provides for a decision disallowing a late claim for a Category A or B retirement pension to be revised as from pensionable age when the deficient contributions are paid.

A regulatory impact assessment has not been produced for this instrument as it has no impact on the costs of business. 

\end{document}