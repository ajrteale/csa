\documentclass[12pt,a4paper]{article}

\newcommand\regstitle{The Child Support Collection and Enforcement (Deduction Orders) Amendment Regulations 2009}

\newcommand\regsnumber{2009/1815}

%\opt{newrules}{
\title{\regstitle}
%}

%\opt{2012rules}{
%\title{Child Maintenance and~Other Payments Act 2008\\(2012 scheme version)}
%}

\author{S.I.\ 2009 No.\ 1815}

\date{Made
7th July 2009\\
%Laid before Parliament
%4th March 2009\\
Coming into~force
3rd August 2009
}

%\opt{oldrules}{\newcommand\versionyear{1993}}
%\opt{newrules}{\newcommand\versionyear{2003}}
%\opt{2012rules}{\newcommand\versionyear{2012}}

\usepackage{csa-regs}

\setlength\headheight{27.61603pt}

%\hbadness=10000

\begin{document}

\maketitle

\noindent
The Secretary of State for~Work and~Pensions makes the following Regulations in exercise of the powers conferred by sections~32A(4)($a$), 32C(1) to~(4), 32D(1)($b$), 32E(2)($a$)  and~(8)($a$), 32F(1)($b$), 32H(6), 32I(1) to~(4), 32J(1) to~(5), 32K(1)($b$), 51(1) and~(2)($h$)  and~($i$), 52(4) and~54 of the Child Support Act 1991\footnote{1991 c.~48. Sections 32A to~32K were inserted into~the Child Support Act 1991 by sections~22 and~23 of the Child Maintenance and~Other Payments Act 2008 (c.~6); the power in section~32H(6) is contained in paragraph~($c$)  of the definition of “relevant time”. Section 54 is cited for~the meaning given to~the word “prescribed”.}.

A draft of this instrument was laid before and~approved by a resolution of each House of Parliament in accordance with section~52(2)($a$)  of that Act. 

{\sloppy

\tableofcontents

}

\bigskip

\setcounter{secnumdepth}{-2}

\subsection[1. Commencement, citation and~interpretation]{Commencement, citation and~interpretation}

1.---(1)  These Regulations may be cited as the Child Support Collection and~Enforcement (Deduction Orders) Amendment Regulations 2009 and~shall come into~force on 3rd August 2009.

(2) In these Regulations “the Collection and~Enforcement Regulations” means the Child Support (Collection and~Enforcement) Regulations 1992\footnote{S.I.~1992/1989.}.

\subsection[2. Amendment of the Collection and~Enforcement Regulations]{Amendment of the Collection and~Enforcement Regulations}

2.  After Part III of the Collection and~Enforcement Regulations insert—
\begin{quotation}
\section*{“Part IIIA\\*Deduction Orders}

\subsection*{Chapter I\\*Interpretation}

\subsubsection*{Interpretation of this Part}

25A.---(1)  In this Part—
\begin{enumerate}\item[]
“assessable income” means the amount calculated in accordance with paragraph~5 of Schedule 1 to~the Act as it applies to~a 1993 scheme case and~regulations made for~the purposes of that paragraph;

“deduction period” means the period of a week, a month or~other period at which deductions are to~be made from the amount (if any) standing to~the credit of the account specified in a regular deduction order;

“garnishee order” means an order made in accordance with the provisions of order 30 of the County Court Rules 1981\footnote{S.I.~1981/1687, these Rules are replaced by the Civil Procedure Rules 1998, except to~the extent that Rule 2.1(2) of the Civil Procedure Rules also provides that those Rules do not apply to~family proceedings and~specifies enactments under which rules may be made for~the purposes of such proceedings.} or~order 49 of the Rules of the Supreme Court 1965\footnote{S.I.~1965/1976, these Rules are replaced by the Civil Procedure Rules 1998, except to~the extent that Rule 2.1(2) of the Civil Procedure Rules also provides that those Rules do not apply to~family proceedings and~specifies enactments under which rules may be made for~the purposes of such proceedings.};

“net weekly income” has the meaning given in the Schedule to~the Child Support (Maintenance Calculations and~Special Cases) Regulations 2000\footnote{S.I.~2001/155, relevant amending instruments are S.I.~2002/1204, 2003/328, 2004/2415 and~3168, 2005/785, 2060 and~2929, 2007/1979 and~2008/2544.};

“lump sum deduction order” means an order under section~32E(1) or, as the case may be, 32F(1) of the Act;

“regular deduction order” means an order under section 32A(1) of the Act;

“third party debt order” means an order made in accordance with the provisions of Part LXXII of the Civil Procedure Rules 1998\footnote{S.I.~1998/3132, relevant amending instruments are S.I.~2001/2792 and~2005/2292.};

“working day” means any day other than a Saturday, a Sunday, Christmas Day, Good Friday or~a day which is a bank holiday within the meaning of the Banking and~Financial Dealings Act 1971\footnote{1971 c. 80.} in the part of the United Kingdom where a copy of a regular deduction order or~a lump sum deduction order is served or~a notification sent by the Commission is received.
\end{enumerate}

(2) Any person against whom an order under section~32A(1) of the Act may be made by the Commission is referred to~in this Chapter and~Chapters II and~IV as “the liable person”.

(3) Where a copy of a regular deduction order or~a lump sum deduction order is served by the Commission in accordance with section~32A(7), 32E(6) or~32F(6) of the Act—
\begin{enumerate}\item[]
($a$) on a deposit-taker—
\begin{enumerate}\item[]
(i) where that copy of the order is sent by electronic communication or~fax to~the deposit-taker’s last notified address for~electronic communication or, as the case may be, fax number, it is to~be treated as having been served at the end of the first working day after the day it was sent by the Commission, or

(ii) where that copy of the order is sent by post to~the deposit-taker’s last notified address, it is to~be treated as having been served at the end of the second working day after the day it was posted by the Commission; or
\end{enumerate}

($b$) on a liable person, where that copy of the order is sent by post to~that person’s last known or~notified address, it is to~be treated as having been served at the end of the day on which the copy of the order is posted.
\end{enumerate}

(4) Any notification sent by the Commission in accordance with this Part to~a deposit-taker or~a liable person is to~be treated as having been received at the same time as an order is treated as having been served in accordance with the provisions of paragraph~(3).

(5) Where a copy of a regular deduction order or~a lump sum deduction order or~any notification has been sent by electronic communication in accordance with paragraph~(3)($a$)(i)  the record held on an official computer system is conclusive (or~in Scotland, sufficient) evidence—
\begin{enumerate}\item[]
($a$) that a copy of that order has been sent; and

($b$) of the content of that order.
\end{enumerate}

(6) This Part applies to~a 1993 scheme case in the same way as it applies to~a 2003 scheme case and—
\begin{enumerate}\item[]
($a$) any references to~expressions in the Act (including “maintenance calculation”) or~to~regulations made under the Act are to~be read, in relation to~a 1993 scheme case, with the necessary modifications; and

($b$) any reference in this Part to~“net weekly income” is to~be read as if it were a reference to~“assessable income” where these Regulations apply to~a 1993 scheme case.
\end{enumerate}

(7) In this regulation—
\begin{enumerate}\item[]
($a$) “electronic communication” has the meaning given in section~15(1) of the Electronic Communications Act 2000\footnote{2000 c.~7.};

($b$) “an official computer system” means a computer system maintained by or~on behalf of the Commission for~sending an order or~any notification;

($c$) “1993 scheme case” means a case in respect of which the provisions of the Child Support, Pensions and~Social Security Act 2000\footnote{2000 c.~19.} have not been brought into~force in accordance with article 3 of the Child Support, Pensions and~Social Security Act 2000 (Commencement No.~12) Order 2003\footnote{S.I.~2003/192 (C.~11).}; and

($d$) “2003 scheme case” means a case in respect of which those provisions have been brought into~force.
\end{enumerate}

\subsection*{Chapter II\\*Regular deduction orders}

\subsubsection*{Regular deduction orders}

25B.---(1)  A regular deduction order must specify—
\begin{enumerate}\item[]
($a$) the amount of the regular deduction; and

($b$) the dates on which regular deductions (referred to~in this Chapter as “deduction dates”) are due to~be made.
\end{enumerate}

(2) Where the date on which the regular deduction is due to~be made is not a working day, the deduction must be made on the first working day after the date specified in the order.

\subsubsection*{Maximum deduction rate}

25C.---(1)  The deduction rate under a regular deduction order in respect of any deduction period—
\begin{enumerate}\item[]
($a$) is not to~exceed 40\% of the liable person’s net weekly income in respect of that period as calculated—
\begin{enumerate}\item[]
(i) at the date of the current maintenance calculation, or

(ii) where a maintenance calculation has been in force and~there are arrears of child support maintenance, at the date of the most recent previous maintenance calculation; or
\end{enumerate}

($b$) where a default maintenance decision has been made, is not to~exceed £80 per week.
\end{enumerate}

(2) In this Chapter “previous maintenance calculation” means a maintenance calculation which is no longer in force.

\subsubsection*{Minimum amount}

25D.---(1)  A deduction must not be made where the amount standing to~the credit of the account specified in the regular deduction order is below the minimum amount on the date a deduction is due to~be made.

(2) The minimum amount (for~the purposes of this Chapter) is, where the deduction period is—
\begin{enumerate}\item[]
($a$) monthly, £40;

($b$) weekly, £10; or

($c$) for~any other period, £10 for~each whole week in that period plus £1 for~each additional day in that period,
\end{enumerate}
plus the amount of administrative costs authorised by regulation~25Z($a$)  (administrative costs).

\subsubsection*{Notification by the deposit-taker to~the Commission}

25E.---(1)  A deposit-taker at which a regular deduction order is directed must notify the Commission in writing, within 7 days—
\begin{enumerate}\item[]
($a$) of a copy of the order or~the order as varied being served; or

($b$) of notification being received by the deposit-taker that an order has been revived,
\end{enumerate}
of the matters set out in paragraph~(2).

(2) The matters are—
\begin{enumerate}\item[]
($a$) if the account specified in the order does not exist; and

($b$) where the name of the liable person specified in the order is different to~the name in which the account specified in the order is held—
\begin{enumerate}\item[]
(i) whether the account was previously held in the name of the liable person specified in the order, and

(ii) if so, the new name in which the account is held,
\end{enumerate}
only where the liable person named in the order is the same person as the person in whose name the account specified in the order is held.
\end{enumerate}

(3) A deposit-taker at which a regular deduction order is directed must notify the Commission within 7 days of notification being received that an order has lapsed or~has been discharged—
\begin{enumerate}\item[]
($a$) if the account specified in the order does not exist; and

($b$) where the name of the liable person specified in the order is different to~the name in which the account specified in the order is held—
\begin{enumerate}\item[]
(i) whether the account was previously held in the name of the liable person specified in the order, and

(ii) if so, the new name in which the account is held,
\end{enumerate}
only where the liable person named in the order is the same person as the person in whose name the account specified in the order is held.
\end{enumerate}

(4) The deposit-taker at which a regular deduction order is directed must notify the Commission within 7 days starting on the date on which a deduction is due to~be made—
\begin{enumerate}\item[]
($a$) if the account specified in the order has been closed;

($b$) if the amount standing to~the credit of the account specified in the order is less than the minimum amount; and

($c$) where the name of the liable person specified in the order is different to~the name in which the account specified in the order is held—
\begin{enumerate}\item[]
(i) whether the account was previously held in the name of the liable person specified in the order, and

(ii) if so, the new name in which the account is held,
\end{enumerate}
only where the liable person named in the order is the same person as the person in whose name the account specified in the order is held.
\end{enumerate}

(5) The deposit-taker at which a regular deduction order is directed must notify the Commission within 7 days of receipt of a request made by the Commission of the details of any other account held by the liable person with that deposit-taker and~the details of that account, including—
\begin{enumerate}\item[]
($a$) the number and~sort code of that account; and

($b$) the type of account.
\end{enumerate}

(6) The requirements of this regulation~apply only in so far as the deposit-taker has the information or~can reasonably be expected to~acquire it.

\subsubsection*{Notification by the Commission to~the deposit-taker}

25F.  The Commission must notify the deposit-taker within 7 days of making a decision that a regular deduction order has—
\begin{enumerate}\item[]
($a$) been varied by virtue of regulation~25I (variation of a regular deduction order);

($b$) lapsed under regulation~25J (lapse of a regular deduction order);

($c$) been revived under regulation~25K (revival of a regular deduction order); or

($d$) ceased to~have effect by virtue of regulation~25L (discharge of a regular deduction order).
\end{enumerate}

\subsubsection*{Review of a regular deduction order}

25G.---(1)  A deposit-taker at which a regular deduction order is directed or~the liable person against whom the order is made may apply to~the Commission for~a review of the order.

(2) The circumstances in which an application may be made under paragraph~(1) are that—
\begin{enumerate}\item[]
($a$) the liable person or~the deposit-taker satisfies the Commission that some or~all of the amount standing to~the credit of the account specified in the order is not an amount in which the liable person has a beneficial interest;

($b$) there has been a change in the amount of the maintenance calculation in question;

($c$) any amounts payable under the order have been paid;

($d$) the maximum deduction rate has been calculated in accordance with regulation~25C(1)($a$)(ii)  (maximum deduction rate) and~there has been a change in the liable persons net weekly income since the date of the most recent previous maintenance calculation;

($e$) due to~an official error, an incorrect amount has been specified in the order; or

($f$) the order does not comply with the requirements of section~32A(5) of the Act or~regulation~25B(1) or~25C.
\end{enumerate}

(3) Following a review of an order under this regulation—
\begin{enumerate}\item[]
($a$) where the Commission changes the amount to~be deducted by the deposit-taker under the order, it may vary the order; or\looseness=-1

($b$) where the Commission extinguishes the amount to~be deducted by the deposit-taker under the order, it must discharge the order.
\end{enumerate}

(4) In paragraph~(2)($e$)  “official error” has the same meaning as in regulation~1(3) of the Social Security and~Child Support (Decisions and~Appeals) Regulations 1999 (interpretation)\footnote{S.I.~1999/991, relevant amending instruments are S.I.~2002/1379, 2008/2656 and 2008/2683.}.

\subsubsection*{Priority as between orders---regular deduction orders}

25H.---(1)  Paragraphs (2) to~(5) apply where one or~more third party debt orders or~garnishee orders provide for~deductions to~be made from the same account as that specified in a regular deduction order.

(2) Where—
\begin{enumerate}\item[]
($a$) one or~more third party debt orders or~garnishee orders are served on a deposit-taker before or~on the day a payment is due to~be made under a regular deduction order; and

($b$) the regular deduction order was served on the same deposit-taker before those orders,
\end{enumerate}
the deposit-taker must make that payment except where the deposit-taker has taken action to~comply with the obligations under any third party debt order or~garnishee order.

(3) Where a regular deduction order is served after an interim third party debt order or~a garnishee order nisi the deposit-taker must take action to~comply with any of those orders before making a deduction under the regular deduction order.

(4) Where paragraph~(2) or~(3) applies, the deposit-taker must take action to~comply with any third party debt orders or~garnishee orders before making further deductions under the regular deduction order.

(5) Where a decision to~revive a regular deduction order takes effect on the same day as or~any day after a third party debt order or~garnishee order has been served, the deposit-taker must take action to~comply with any of those orders before making a deduction under the regular deduction order.

(6) Paragraphs (1) to~(5) do not apply to~Scotland.

(7) In Scotland, paragraphs (8) to~(10) apply where a deposit-taker receives one or~more arrestment schedules (“arrestments”) and~a regular deduction order which apply to~the same account.

(8) Where—
\begin{enumerate}\item[]
($a$) one or~more arrestments are served on a deposit-taker before or~on the day a payment is due to~be made under a regular deduction order; and

($b$) the regular deduction order was served on the same deposit-taker before any of those arrestments,
\end{enumerate}
the deposit-taker must make that payment except where the deposit-taker has taken action to~comply with the obligations under any of the arrestments.

(9) Where paragraph~(8) applies, the deposit-taker must take action to~comply with any of those arrestments before making further deductions under the regular deduction order.

(10) Where a decision to~revive a regular deduction order takes effect on the same day as or~any day after any arrestments have been served, the deposit-taker must take action to~comply with any of those arrestments before making a deduction under the regular deduction order.

\subsubsection*{Variation of a regular deduction order}

25I.---(1)  The Commission may vary a regular deduction order by changing the amount to~be deducted in the circumstances set out in paragraph~(2).

(2) The circumstances are that—
\begin{enumerate}\item[]
($a$) the Commission has accepted—
\begin{enumerate}\item[]
(i) that a payment of arrears has been made by the liable person, and

(ii) no alternative method of payment of child support maintenance has been arranged;
\end{enumerate}

($b$) a decision has been made under section~11, 12, 16 or~17 of the Act or~there has been an appeal against a maintenance calculation;

($c$) the Commission has reviewed the order under regulation~25G (review of a regular deduction order); or

($d$) there has been an appeal under regulation~25AB(1)($a$)  or~($b$)  (appeals).
\end{enumerate}

(3) The Commission may from time to~time vary the deduction period.

(4) Where—
\begin{enumerate}\item[]
($a$) a regular deduction order has been varied under this regulation; and

($b$) a copy of the order as varied has been served on the deposit-taker at which it is directed,
\end{enumerate}
that deposit-taker must comply with the order; but the deposit-taker is not to~be under any liability for~non-compliance before the end of the period of 7 days beginning on the day on which the copy of the order as varied is served on the deposit-taker.

\subsubsection*{Lapse of a regular deduction order}

25J.---(1)  A regular deduction order is to~lapse in the circumstances set out in  paragraph~(2).

(2) The circumstances are where—
\begin{enumerate}\item[]
($a$) the Commission has agreed with the liable person an alternative method of payment of the child support maintenance due under the maintenance calculation; or

($b$) there is an insufficient amount standing to~the credit of the account specified in the order to~enable a deduction to~be made on two consecutive deduction dates, unless the Commission has decided that the order is to~continue for~a greater number of deduction dates,
\end{enumerate}
and~the Commission considers it is reasonable in all the circumstances that the order is to~lapse.

(3) A regular deduction order lapses on the day on which the deposit-taker receives notification that the order has lapsed from the Commission.

(4) A regular deduction order which has lapsed under this regulation~is to~be treated as remaining in force for~the purposes of regulations 25E (notification by the deposit-taker to~the Commission), 25G (review of a regular deduction order) and~25AB (appeals).

\subsubsection*{Revival of a regular deduction order}

25K.---(1)  Where a regular deduction order has lapsed it may be revived by the Commission where—
\begin{enumerate}\item[]
($a$) the liable person has failed to~comply with any agreement reached under regulation~25J(2)($a$)  (lapse of a regular deduction order); or

($b$) the Commission has reason to~believe that following the lapse of an order under regulation~25J(2)($b$)  there is sufficient amount standing to~the credit of the account specified in the order to~enable a deduction to~be made.
\end{enumerate}

(2) Where the Commission decides to~revive a regular deduction order that decision is to~take effect on the day notification that the order has been revived is received by the deposit-taker.

\subsubsection*{Discharge of a regular deduction order}

25L.---(1)  A regular deduction order must be discharged by the Commission where—
\begin{enumerate}\item[]
($a$) the account specified in the order has been closed;

($b$) the maintenance calculation in question is no longer in force and~the amount of child support maintenance due under that calculation has been paid in full in accordance with regulation~2 (payment of child support maintenance);

($c$) the liable person has complied with any agreement reached under regulation~25J(2)($a$)  for~such period as the Commission considers appropriate in the circumstances of the case;

($d$) the Commission has reviewed the order under regulation~25G and~it has extinguished the amount to~be deducted by the deposit-taker under the order;

($e$) on an appeal under regulation~25AB(1)($a$)  (appeals) the court has set aside the order;

($f$) unless sub-paragraph~($g$)  applies, a regular deduction order has lapsed under regulation~25J(2) and~6 months have passed beginning on the day the lapse took effect;

($g$) an appeal is brought by virtue of regulation~25AB(1)($a$)  or~($b$), against a regular deduction order which has lapsed under regulation~25J(2) and~1 month has passed beginning on—
\begin{enumerate}\item[]
(i) the day proceedings on the appeal (including any further appeal) concluded, or

(ii) the end of any period during which a further appeal may ordinarily be brought,
\end{enumerate}
whichever is the later; or

($h$) the liable person has died.
\end{enumerate}

(2) A regular deduction order may be discharged where the Commission considers it is appropriate to~do so in the circumstances of the case.

(3) Where a regular deduction order is discharged that discharge takes effect immediately after the payment of the last regular deduction prior~to~discharge.

\subsection*{Chapter III\\*Lump sum deduction orders}

\subsubsection*{Period in which representations may be made}

25M.  Where a lump sum deduction order has been made under section~32E(1) of the Act the period for~making representations to~the Commission in respect of the proposal specified in that order is 14 days beginning on the day a copy of the order was served.

\subsubsection*{Disapplication of sections~32G(1) and~32H(2)($b$)  of the Act}

25N.---(1)  Something that would otherwise be in breach of sections~32G(1) and~32H(2)($b$)  of the Act may, with the consent of the Commission, be done in the following circumstances—
\begin{enumerate}\item[]
($a$) the liable person, the liable person’s partner or~any relevant other child is suffering hardship in meeting ordinary living expenses;

($b$) the liable person is under a written contractual obligation, agreed before the lump sum deduction order was made, to~make a payment;

($c$) the deposit-taker has a right of set off and~satisfies the Commission that an intention to~exercise that right was formed within 30 days before the date the lump sum deduction order under section~32E of the Act was served;

($d$) the deposit-taker and~the liable person have made a written agreement in which the availability of an amount standing to~the credit of the account specified in the lump sum deduction order was required as security for~that agreement; or

($e$) any other circumstances the Commission considers appropriate in the particular case.
\end{enumerate}

(2) The liable person or~the deposit-taker at which a lump sum deduction order is directed may apply to~the Commission for~consent.

(3) When deciding whether to~give consent, the Commission must take into~account—
\begin{enumerate}\item[]
($a$) any adverse impact the decision may have on the liable person or~any other person; and

($b$) any alternative arrangements which may be made by the liable person or~the deposit-taker.
\end{enumerate}

(4) Where the Commission gives consent it is to~take effect on the day on which the deposit-taker receives notification from the Commission to~disapply section~32G(1) or~32H(2)($b$)  of the Act.

(5) Something that would otherwise be in breach of section~32G(1) and~32H(2)($b$)  of the Act may be done where—
\begin{enumerate}\item[]
($a$) the amount standing to~the credit of the account specified in the lump sum deduction order is less than the amount specified in that order, except in respect of any amount dealt with in compliance with section~32G(1) of the Act; or

($b$) the deposit-taker has made a payment in accordance with section~32H(1)($a$)  of the Act.
\end{enumerate}

(6) Paragraph (5) has effect until the Commission gives notice to~the deposit-taker that paragraph~(5) has ceased to~have effect in a particular case and~that notification is to~take effect on the day on which the deposit-taker receives notification from the Commission.

(7) In this regulation—
\begin{enumerate}\item[]
\begin{sloppypar}
“partner” has the same meaning as in regulation~3(9) (method of payment) and the definition of “couple” in that regulation is to apply accordingly; and
\end{sloppypar}

“relevant other child” is to~be interpreted in accordance with paragraph~10C(2) of Schedule 1 to~the Act\footnote{Paragraph 10C was inserted by section~1(3) of, and~Schedule 1 to, the Child Support, Pensions and~Social Security Act 2000 (c. 34).} and~regulations made for~the purposes of that paragraph.
\end{enumerate}

\subsubsection*{Information}

25O.---(1)  A deposit-taker at which a lump sum deduction order is directed must supply to~the Commission in writing, within 7 days—
\begin{enumerate}\item[]
($a$) of a copy of the order or~order as varied being served; or

($b$) of notification being received by the deposit-taker that an order has been revived,
\end{enumerate}
the information set out in paragraph~(2).

(2) The information is—
\begin{enumerate}\item[]
($a$) if the account specified in the order—
\begin{enumerate}\item[]
(i) does not exist,

(ii) cannot be traced, or

(iii) has been closed;
\end{enumerate}

($b$) whether the amount standing to~the credit of the account specified in the order—
\begin{enumerate}\item[]
(i) on the day the order is served, or

(ii) where an order is revived, on the day the decision to~revive the order takes effect,
\end{enumerate}
is at least the same or~less than the amount specified in the order and~where it is less, that amount; and

($c$) where the name of the liable person specified in the order is different to~the name in which the account specified in the order is held—
\begin{enumerate}\item[]
(i) whether the account was previously held in the name of the liable person specified in the order, and

(ii) if so, the new name in which the account is held,
\end{enumerate}
only where the liable person named in the order is the same person as the person in whose name the account specified in the order is held.
\end{enumerate}

(3) A deposit-taker at which a lump sum deduction order is directed must notify the Commission within 7 days of notification being received that an order has lapsed or~has been discharged—
\begin{enumerate}\item[]
($a$) if the account specified in the order cannot be traced; or

($b$) where the name of the liable person specified in the order is different to~the name in which the account specified in the order is held—
\begin{enumerate}\item[]
(i) whether the account was previously held in the name of the liable person specified in the order, and

(ii) if so, the new name in which the account is held,
\end{enumerate}
only where the liable person named in the order is the same person as the person in whose name the account specified in the order is held.
\end{enumerate}

(4) A deposit-taker at which a lump sum deduction order is directed, must supply to~the Commission within 7 days of receipt of a request being made by the Commission, the following information—
\begin{enumerate}\item[]
($a$) whether the liable person holds another account or~has opened an account with that deposit-taker or~with another deposit-taker and, if so, the details of that account, including—
\begin{enumerate}\item[]
(i) the number and~sort code of that account, and

(ii) the type of account; and
\end{enumerate}

($b$) whether the amount standing to~the credit of the account specified in the order on the day the request is received is at least the same or~less than the amount specified in the order or~the remaining amount and~where it is less, that amount.
\end{enumerate}

(5) In so far as a deposit-taker at which a lump sum deduction order is directed (“$A$”) has the information, the details of an account held with another deposit-taker (“$B$”) must be supplied to~the Commission in accordance with paragraph~(4) only if—
\begin{enumerate}\item[]
($a$) the liable person has—
\begin{enumerate}\item[]
(i) closed the account specified in the order and~held with $A$,

(ii) opened an account with $B$, and

(iii) transferred the amount standing to~the credit of the account held with $A$ to~the account held with $B$;
\end{enumerate}

($b$) either—
\begin{enumerate}\item[]
(i) a lump sum deduction order has lapsed, or

(ii) $A$ has notified the Commission in accordance with paragraph~(2)($a$)(iii), that the account specified in the order has been closed; and
\end{enumerate}

($c$) the Commission has made a request for~the information within 1 month of the order lapsing or, as the case may be, notification being received by the Commission that the account has been closed.
\end{enumerate}

(6) The requirements of paragraphs (1) to~(3) and~paragraph~(4) as it applies to~a deposit-taker at which a lump sum deduction order is directed, apply only in so far as the deposit-taker has the information or~can reasonably be expected to~acquire it.

(7) In paragraph~(4)($b$)  and~regulation~25T(1)($b$)  and~($c$)  “remaining amount” has the same meaning as in section~32H(6) of the Act.

\subsubsection*{Priority as between orders---lump sum deduction orders}

25P.---(1)  Where a deposit-taker would, but for~this paragraph, be obliged to~comply with an order under section~32F of the Act, and~one or~more interim third party debt orders or~garnishee orders nisi, it must take action to~comply with the orders according to~the order in which they were served on the deposit-taker.

(2) Paragraph (1) does not apply where an order under section~32E of the Act was served after an interim third party debt order or~a garnishee order nisi except where there remains an amount standing to~the credit of the account specified in the order under section~32F of the Act after any third party debt orders or~garnishee orders have been complied with by the deposit-taker (referred to~in this regulation~as “an outstanding amount”).

(3) Where there is an outstanding amount section~32G(1) of the Act applies in respect of that amount.

(4) Where a decision to~revive a lump sum deduction order takes effect on the same day as or~any day after a third party debt order or~garnishee order has been served, the deposit-taker must take action to~comply with any of those orders before making a deduction under the lump sum deduction order.

(5) Paragraphs (1) to~(4) do not apply to~Scotland.

(6) In Scotland, where a deposit-taker would, but for~this paragraph, be obliged to~comply with an order under section~32F of the Act, and~one or~more arrestment schedules (“arrestments”) it must give preference to~that order and~those arrestments according to~the order in which they were served on the deposit-taker.

(7) Where there remains an amount standing to~the credit of the account specified in the order under section~32F of the Act after any arrestments have been complied with by the deposit-taker, section~32G(1) of the Act applies in respect of that amount.

(8) Where a decision to~revive a lump sum deduction order takes effect on the same day as or~any day after any arrestments have been served, the deposit-taker must take action to~comply with any of those arrestments before making a deduction under the lump sum deduction order.

\subsubsection*{Minimum amount}

25Q.---(1)  A deduction must not be made where the amount standing to~the credit of the account specified in the lump sum deduction order is below the minimum amount on the date the deduction is due to~be made.

(2) The minimum amount is £55 plus the amount of administrative costs authorised by regulation~25Z($b$)  (administrative costs).

\subsubsection*{Variation of a lump sum deduction order}

25R.---(1)  The Commission may, in the circumstances set out in paragraph~(2), vary a lump sum deduction order by reducing the amount specified in that order.

(2) The circumstances are that—
\begin{enumerate}\item[]
($a$) the Commission accepts the liable person’s agreement to~make a payment;

($b$) a decision has been made under section~11, 12, 16 or~17 of the Act or~there has been an appeal against a maintenance calculation;

\begin{sloppypar}
($c$) the Commission has consented to~the doing of things that would otherwise be in breach of sections~32G(1) and~32H(2)($b$)  of the Act;
\end{sloppypar}

\begin{sloppypar}
($d$) there has been an appeal made under regulation 25AB(1)($c$)  or ($d$)  (appeals); or
\end{sloppypar}

($e$) representations made in respect of the proposals specified in the order made under section~32E of the Act have been accepted by the Commission.
\end{enumerate}

(3) Where—
\begin{enumerate}\item[]
($a$) a lump sum deduction order has been varied under this regulation; and

($b$) a copy of the order as varied has been served on the deposit-taker at which it is directed,
\end{enumerate}
that deposit-taker must comply with the order when that order is served.

\subsubsection*{Lapse of a lump sum deduction order}

25S.---(1)  A lump sum deduction order is to~lapse in the circumstances set out in paragraph~(2).

(2) The circumstances are where—
\begin{enumerate}\item[]
($a$) the amount in the account specified in the order under section~32E of the Act is nil;

($b$) in consequence of the consent given by the Commission under regulation~25N(1) (disapplication of section~32G(1) and~32H(2)($b$)  of the Act) the amount in the account specified in the lump sum deduction order is reduced to~nil; or

($c$) the Commission has agreed with the liable person an alternative method of payment of the child support maintenance due under the maintenance calculation,
\end{enumerate}
and~the Commission considers it is reasonable in all the circumstances that the order is to~lapse.

(3) A lump sum deduction order lapses on the day on which the deposit-taker receives notification that the order has lapsed from the Commission.

(4) A lump sum deduction order which has lapsed under this regulation~is to~be treated as remaining in force for~the purposes of regulations 25M (period in which representations may be made),~25O (information) and~25AB (appeals).

\subsubsection*{Revival of a lump sum deduction order}

25T.---(1)  Where a lump sum deduction order has lapsed it may be revived by the Commission where—
\begin{enumerate}\item[]
($a$) in the case of an order under section~32E of the Act, the amount standing to~the credit of the account specified in that order was nil and~the Commission is informed in accordance with the requirement in regulation~25O(4)($b$)  that there is an amount at least the same as or~less than the amount specified in the order standing to~the credit of the account specified in the order;

($b$) a lump sum deduction order has lapsed under regulation~25S(2)($b$)  (lapse of a lump sum deduction order) and~the Commission is informed in accordance with the requirement in regulation~25O(4)($b$)  that there is an amount at least the same as or~less than the amount specified in the order, or~the remaining amount, standing to~the credit of the account specified in the order; or

($c$) in the case of an order under section~32F of the Act, there is a remaining amount and~the liable person has failed to~comply with the agreement referred to~in regulation~25S(2)($c$).
\end{enumerate}

(2) Where the Commission decides to~revive a lump sum deduction order that decision is to~take effect on the day notification that the order has been revived is received by the deposit-taker.

\subsubsection*{Discharge of a lump sum deduction order}

25U.---(1)  A lump sum deduction order must be discharged where—
\begin{enumerate}\item[]
($a$) the account specified in the order has been closed;

($b$) the amount of arrears of child support maintenance specified in the order has been paid in full in accordance with regulation~2 (payment of child support maintenance);

($c$) the liable person has paid the total amount of arrears of child support maintenance specified in the order by an alternative method agreed between the Commission and~the liable person;

($d$) the Commission has considered representations made in respect of an order under section~32E of the Act and~it has decided not to~make an order under section~32F of the Act;

($e$) unless sub-paragraph~($f$)  applies—
\begin{enumerate}\item[]
(i) an order under section~32F of the Act has lapsed under regulation~25S(2) and~6 months have passed beginning on the day on which the deposit-taker received notification that the order had lapsed from the Commission, or

(ii) regulation~25N(5) applies and~6 months have passed beginning on the day on which payment was made under section~32H(1)($a$)  of the Act;
\end{enumerate}

($f$) an appeal is brought by virtue of regulation~25AB(1)($d$)  and~1 month has passed beginning on—
\begin{enumerate}\item[]
(i) the day proceedings on the appeal (including any further appeal) concluded, or

(ii) the end of any period during which a further appeal may ordinarily be brought,
\end{enumerate}
whichever is the later; or

($g$) the liable person has died.
\end{enumerate}

(2) A lump sum deduction order may be discharged where the Commission considers it is appropriate to~do so in the circumstances of the case.

(3) A lump sum deduction order is discharged on the day notification that the order has been discharged is received by the deposit-taker.

\subsubsection*{Time at which a lump sum deduction order under section~32E of the Act ceases to~be in force}

25V.  For~the purposes of section~32E(8)($a$)  of the Act the prescribed period is—
\begin{enumerate}\item[]
($a$) unless paragraph~($b$)  applies, 6 months beginning on—
\begin{enumerate}\item[]
(i) the day the order under section~32E of the Act was served on the deposit-taker, or

(ii) where that order has lapsed under regulation~25S, the day on which the deposit-taker received notification that the order had lapsed from the Commission; or
\end{enumerate}

\begin{sloppypar}
($b$) where an appeal is brought by virtue of regulation 25AB(1)($c$)  (appeal against the withholding of consent), 1 month beginning on—
\end{sloppypar}
\begin{enumerate}\item[]
(i) the day proceedings on the appeal (including any further appeal) concluded, or

(ii) the end of any period during which a further appeal may ordinarily be brought,
\end{enumerate}
whichever is the later.
\end{enumerate}

\subsubsection*{Meaning of “the relevant time”}

25W.  For~the purposes of the meaning of “the relevant time” in section~32H(6) of the Act the prescribed circumstances are that—
\begin{enumerate}\item[]
($a$) unless to~paragraph~($b$)  applies, 6 months have passed beginning on the day the order under section~32F of the Act was served on the deposit-taker; or

\begin{sloppypar}
($b$) where an appeal is brought by virtue of regulation 25AB(1)($d$), 1 month has passed beginning on—
\end{sloppypar}
\begin{enumerate}\item[]
(i) the day proceedings on the appeal (including any further appeal) concluded, or

(ii) the end of any period during which a further appeal may ordinarily be brought,
\end{enumerate}
whichever is the later.
\end{enumerate}

\subsection*{Chapter IV\\*General matters for deduction orders}

\subsubsection*{Accounts of a prescribed description}

25X.---(1)  A regular deduction order or~a lump sum deduction order may not be made in respect of an account which—
\begin{enumerate}\item[]
($a$) the liable person operates solely for~the purposes of exercising the function of a trustee or~office holder and~the account is one in which all the funds are held on behalf of other persons or~for~the purposes of that office; or

($b$) is used wholly or~in part for~business purposes.
\end{enumerate}

(2) For~the purposes of paragraph~(1)($b$), whether an account is used wholly or~in part for~business purposes is to~be decided by the Commission.

(3) Paragraph (1)($b$)  does not apply where a regular deduction order is made in respect of an account which is used by the liable person as a sole trader.

\subsubsection*{Circumstances in which amounts standing to~the credit of an account are to~be disregarded}

25Y.  The circumstances in which amounts standing to~the credit of an account are to~be disregarded for~the purposes of sections~32A,~32E, 32G and~32H of the Act are where the liable person has no beneficial interest in the amount.

\subsubsection*{Administrative costs}

25Z.  A deposit-taker at which an order under section~32A or~32F of the Act is directed may deduct from the amount standing to~the credit of the account specified in the order an amount towards its administrative costs for~each deduction made, not exceeding—
\begin{enumerate}\item[]
($a$) in the case of a regular deduction order, £10; or

($b$) in the case of a lump sum deduction order under section~32F of the Act, £55,
\end{enumerate}
before making any payment to~the Commission required by section~32A or, as the case may be, section~32H of the Act.

\subsubsection*{Payment by deposit-taker to~the Commission}

25AA.---(1)  Amounts deducted by a deposit-taker at which a regular deduction order or~a lump sum deduction order under section~32F of the Act is directed must be paid to~the Commission within—
\begin{enumerate}\item[]
($a$) in the case of a regular deduction order, 10 days of the date the regular deduction is due to~be made; or

($b$) in the case of a lump sum deduction order under section~32F of the Act, 10 days of the end of the relevant period.
\end{enumerate}

(2) The payment to~the Commission of amounts deducted under that order may be made by—
\begin{enumerate}\item[]
($a$) cheque;

($b$) automated credit transfer; or

($c$) such other method as the Commission may specify.
\end{enumerate}

(3) In this regulation~“the relevant period” has the same meaning as in section~32G(5) and~(6) of the Act.

\subsubsection*{Appeals}

25AB.---(1)  A qualifying person has a right of appeal to~a county court or~in Scotland~the sheriff of the sheriffdom in which that person resides, against—
\begin{enumerate}\item[]
($a$) the making of a regular deduction order;

($b$) any decision made by the Commission on an application made under regulation~25G (review of a regular deduction order);

($c$) the withholding of the consent to~be obtained in accordance with regulation~25N (disapplication of sections~32G(1) and~32H(2)($b$)  of the Act);

($d$) the making of an order under section~32F of the Act.
\end{enumerate}

(2) In this regulation~a “qualifying person” means—
\begin{enumerate}\item[]
($a$) in relation to~paragraph~(1)($a$)  and~($b$), any person affected by—
\begin{enumerate}\item[]
(i) a regular deduction order, or, as the case may be,

(ii) the decision referred to~in paragraph~(1)($b$);
\end{enumerate}

($b$) in relation to~paragraph~(1)($c$), the persons prescribed in regulation~25N(2); and

($c$) in relation to~paragraph~(1)($d$), any person affected by an order under section~32F of the Act.
\end{enumerate}

\subsubsection*{Offences}

25AC. The following regulations are designated for~the purposes of sections~32D(1)($b$)  and~32K(1)($b$)  of the Act—
\begin{enumerate}\item[]
($a$) regulation~25E(1) to~(5) (notification by the deposit-taker to~the Commission);

($b$) regulation~25I(4) (variation of a regular deduction order);

($c$) regulation~25O(1) to~(5) (information);

($d$) regulation~25R(3) (variation of a lump sum deduction order); and

($e$) regulation~25AA(1) (payment by deposit-taker to~the Commission).
\end{enumerate}

\subsubsection*{Commission to~warn of consequences of failing to~comply with an order or~to~provide information}

25AD.  Where information is required by virtue of regulation~25E or~25O, the Commission must set out in writing the possible consequences of failure to—
\begin{enumerate}\item[]
($a$) comply with a regular deduction order or~lump sum deduction order; and

($b$) provide the information required under the regulations designated by regulation~25AC($a$)  and~($b$)  (offences),
\end{enumerate}
including details of the offences provided for~by virtue of sections~32D and~32K of the   Act, as the case may be.”.
\end{quotation}

\bigskip

\pagebreak[3]

Signed 
by authority of the 
Secretary of State for~Work and~Pensions.
%I concur
%By authority of the Lord Chancellor

{\raggedleft
\emph{Helen Goodman}\\*
%Secretary
%Minister
Parliamentary Under-Secretary 
of State\\*Department 
for~Work and~Pensions

}

7th July 2009

\small

\part{Explanatory Note}

\renewcommand\parthead{— Explanatory Note}

\subsection*{(This note is not part of the Regulations)}

These Regulations amend the Child Support (Collection and~Enforcement) Regulations 1992 (S.I.~1992/1989) (“the Collection and~Enforcement Regulations”) inserting a new Part IIIA into~those Regulations. Part IIIA provides for~regular deduction orders under section~32A of the Child Support Act 1991 (c.~48) (“the Act”) and~lump sum deduction orders under sections~32E and~32F of that Act which were inserted into~that Act by sections~22 and~23 of the Child Maintenance and~Other Payments Act 2008 (c.~6) (“the 2008 Act”).

These Regulations have effect in cases to~which the Act applies prior~to~the amendments made to~that Act by the Child Support, Pensions and~Social Security Act 2000 (c.~19) (“the 2000 Act”), some of which amendments are not fully in force, and~relate to~the child support scheme which was in force prior~to~3rd March 2003 and~which remains in force for~the purposes of certain cases (a “1993 scheme case”). The regulations also have effect in cases to~which the Act as amended by the 2000 Act, which relate to~the child support scheme provided for~by those amendments, which came into~force for~the purposes of specified categories of cases on 3rd March 2003 (\emph{see} the Child Support, Pensions and~Social Security Act 2000 (Commencement No.~12) Order 2003) (a “2003 scheme case”).

These Regulations are made in respect of functions of the Child Maintenance and~Enforcement Commission (“the Commission”). Section 13 of the 2008 Act transfers functions from the Secretary of State to~the Commission and~that section~was brought into~force by the Child Maintenance and~Other Payments Act 2008 (Commencement No.~4 and~Transitional Provision) Order 2008 on 1st November 2008.

These Regulations make provision for~regular deduction orders and~lump sum deduction orders, by inserting regulations 25A to~25AD into~the Collection and~Enforcement Regulations.

Regulation 25A sets out definitions and~makes provision for: the service of orders and~the sending of notifications, evidence and~the application of these Regulations to~a 1993 scheme case and~a 2003 scheme case.

Regulation 25B provides the additional matters which must be specified in a regular deduction order.

Regulation 25C provides the maximum deduction rate under a regular deduction order.

Regulation 25D specifies the minimum amount which may be deducted under a regular deduction order.

Regulation 25E makes provision for~the matters which have to~be notified by a deposit-taker to~the Commission giving the time limits within which the deposit-taker is required to~comply. The information which is required to~be notified includes whether the account specified in the order does not exist, whether the account contains less than the minimum amount and~where requested information relating to~other accounts held by the liable person with the deposit-taker.

Regulation 25F provides that the Commission must notify the deposit-taker where a regular deduction order is varied, lapsed, revived or~ceases to~have effect.

Regulation 25G makes provision for~the circumstances in which a regular deduction order may be reviewed and~where it is reviewed the decision which may result from that review.

Regulation 25H makes provision for~the priority of orders. This regulation~makes separate provision for~Scotland.

Regulations 25I to~25K provide for~the circumstances in which a regular deduction order may be varied, is to~lapse or~may be revived. An order may be varied in circumstances which may lead to~a change in the amount to~be deducted under the order. The Commission may also vary the deduction period. An order is to~lapse either where the Commission has agreed an alternative method of payment or~where there is an insufficient amount standing to~the credit of the account specified in the regular deduction order on a specified number of deduction dates. An order may be revived where the liable person has failed to~comply with the alternative method of payment agreed by the Commission or~it has reason to~believe that there is a sufficient amount standing to~the credit of the account specified in the order to~make a regular deduction.

Regulation 25L makes provision setting out the circumstances in which the Commission must discharge a regular deduction order, which includes where the account specified in the order has been closed, the amount due under the maintenance calculation has been paid and~no further payment is due and~the amount to~be deducted has been extinguished on review. It also includes provision for~a lapsed order to~be discharged after the expiry of specified periods of time.

Regulation 25M makes provision for~representations to~be made in respect of the proposal specified in the interim lump sum deduction order made under section~32E(1) of the Act within 14 days for~the deposit-taker and~the liable person from the date a copy of that order was served.

Regulation 25N provides the circumstances in which something that would otherwise be a breach of sections~32G(1) and~32H(2)($b$)  of the Act, with the consent of the Commission can be done. It sets out the matters which must be taken into~account when the Commission is deciding whether to~give consent. Further provision is made for~the circumstances in which those sections~are to~be disapplied until the deposit-taker receives notification from the Commission.

Regulation 25O makes provision for~the information which has to~be provided by a deposit-taker to~the Commission giving the time limits within which the deposit-taker is required to~comply. The information which is required to~be notified includes whether the account specified in the order does not exist, cannot be traced or~has been closed and~where requested, information relating to~other accounts held by the liable person is required to~be provided by the deposit-taker to~the Commission in specified circumstances.

Regulation 25P makes provision for~the priority of orders. This regulation~makes separate provision for~Scotland.

Regulation 25Q specifies the minimum amount which may be deducted under a lump sum deduction order.

Regulations 25R to~25T provide for~the circumstances in which a lump sum deduction order may be varied, is to~lapse or~may be revived. An order may be varied in circumstances which lead to~a reduction in the amount to~be deducted under the order. An order is to~lapse either where the amount in the account specified in the order is nil, the amount in the account specified in the order is reduced to~nil where the Commission has consented to~the disapplication of sections~32G(1) and~32H(2)($b$)  of the Act or~the Commission has agreed an alternative method of payment. The circumstances in which an order may be revived include where an amount has been paid into~the account specified in the order, or~where the liable person has failed to~comply with the alternative method of payment agreed by the Commission.

Regulation 25U makes provision setting out the circumstances in which the Commission must discharge a lump sum deduction order, including where the account specified in the order has been closed, the amount of arrears specified in the order has been paid in the required manner and~the Commission has decided not to~make an order under section~32F of the Act having considered representations.

Regulation 25V prescribes the time at which an order made under section~32E of the Act ceases to~be in force.

Regulation 25W prescribes the circumstances which arise for~the purposes of the meaning of “the relevant time” as defined in section~32H(6) of the Act (the time at which a lump sum deduction order will cease to~operate).

Regulations 25X and~25Y prescribe accounts in respect of which a regular or~lump sum deduction order may not be made and~the circumstances in which amounts standing to~the credit of an account are to~be disregarded, respectively.

Regulation 25Z specifies the amount which a deposit-taker may deduct towards its administrative costs before making any payment to~the Commission.

Regulation 25AA specifies the time at which and~the method by which payment is to~be made by the deposit-taker to~the Commission.

Regulation 25AB makes provision for~appeals against the making of a regular deduction order, any decision on the review of a regular deduction order, the withholding of consent to~the disapplication of sections~32G(1) and~32H(2)($b$)  of the Act by the Commission and~the making of an order under section~32F of the Act.

\begin{sloppypar}
Regulation 25AC designates regulations 25E(1) to~(5), 25I(4), 25O(1) to~(5),~25R(3) and~25AA(1) for~the purposes of sections~32D(1)($b$)  and~32K(1)($b$)  of the Act. Failure to~comply with the requirements of those regulations is an offence in accordance with those sections~of the Act.
\end{sloppypar}

Regulation 25AD provides that the Commission must set out the consequences of a failure to~comply with a regular or~lump sum deduction order or~a requirement to~provide information in accordance with regulation~25E or~25O including details of the offences provided for~by sections~32D and~32K of the Act.

An assessment of the impact of these Regulations on the private and~voluntary sectors has been made. Copies of this impact assessment are available in the libraries of both Houses of Parliament, and~may also be obtained from the Better Regulation Unit of the Department for~Work and~Pensions, 7F Caxton House, Tothill Street, London \textsc{\lowercase{SW1H~9NA}}, or~from the DWP website: \url{http://www.dwp.gov.uk/resourcecentre/ria.asp}

\end{document}