\documentclass[12pt,a4paper]{article}

\newcommand\regstitle{The Social Security (Miscellaneous Amendments) (No.~3) Regulations 2006}

\newcommand\regsnumber{2006/2377}

%\opt{newrules}{
\title{\regstitle}
%}

%\opt{2012rules}{
%\title{Child Maintenance and~Other Payments Act 2008\\(2012 scheme version)}
%}

\author{S.I.\ 2006 No.\ 2377}

\date{Made
4th September 2006\\
Laid before Parliament
8th September 2006\\
Coming into~force
2nd October 2006
}

%\opt{oldrules}{\newcommand\versionyear{1993}}
%\opt{newrules}{\newcommand\versionyear{2003}}
%\opt{2012rules}{\newcommand\versionyear{2012}}

\usepackage{csa-regs}

\setlength\headheight{42.11603pt}

%\hbadness=10000

\begin{document}

\maketitle

\enlargethispage{1.65405pt}

\noindent
The Secretary of State for Work and~Pensions makes the following Regulations in exercise of the powers conferred by sections~5(1)($a$)  to ($c$)  and~($p$), 189(1) and~(4) and~191 of the Social Security Administration Act 1992\footnote{1992 c.~5. Section 191 is cited for the meaning given to “prescribe”.} and~sections~10(3) and~(6), 79(1) and~(4) and~84 of the Social Security Act 1998\footnote{1998 c.~14. Section 84 is cited for the meaning given to “prescribe”.}.

In accordance with section~173(1)($b$)  of the Social Security Administration Act 1992, the Secretary of State has obtained the agreement of the Social Security Advisory Committee that proposals in respect of these Regulations need not be referred to it. 

{\sloppy

\tableofcontents

}

\bigskip

\setcounter{secnumdepth}{-2}

\subsection[1. Citation and~commencement]{Citation and~commencement}

1.  These Regulations may be cited as the Social Security (Miscellaneous Amendments) (No.~3) Regulations 2006 and~shall come into force on 2nd October 2006.

\subsection[2. Amendment of the Social Security (Claims and~Payments) Regulations 1987]{Amendment of the Social Security (Claims and~Payments) Regulations 1987}

2.---(1)  The Social Security (Claims and~Payments) Regulations 1987\footnote{S.I.~1987/1968.} are amended as follows.

(2) In regulation~4D(11)\footnote{Regulation 4D was inserted by S.I.~2002/3019 and paragraph~(11) was amended by S.I.~2003/1632.} (making a claim for state pension credit) after “month” insert “,~or such longer period as the Secretary of State considers reasonable,”.

(3) In regulation~6\footnote{The relevant amending instruments are S.I.~2002/428 and 2005/337.} (date of claim)—
\begin{enumerate}\item[]
($a$) in sub-paragraph~(19)($b$)  before “re-awarded” insert “awarded or”;

($b$) for paragraph~(20) substitute—
\begin{quotation}
“(20) The circumstances referred to in paragraph~(19) are—
\begin{enumerate}\item[]
($a$) that the award of the qualifying benefit has itself been terminated or reduced by means of a revision, supersession, appeal or termination of an award for a fixed period in such a way as to affect the original award; or

($b$) at the date the original award was terminated the claimant’s claim for a qualifying benefit had not been decided.”; and
\end{enumerate}
\end{quotation}

($c$) in paragraph~(21) for sub-paragraph~($b$)  substitute—
\begin{quotation}
“($b$) the further claim is made within three months of the date on which the qualifying benefit is awarded following a claim, whether initially, on revision or on appeal, or re-awarded following revision, supersession, appeal or further claim when an award for a fixed period expires, whether benefit is re-awarded when the further claim is decided or following a revision of, or an appeal against, such a decision.”.
\end{quotation}
\end{enumerate}

(4) In regulation~19(7)\footnote{Regulation 19 was substituted by S.I.~1997/793.} (time for claiming benefit) after sub-paragraph~($i$) add—
\begin{quotation}
“($j$) the claimant was unable to make telephone contact with the appropriate office where he would be expected to notify his intention of making a claim because the telephone lines to that office were busy or inoperative.”.
\end{quotation}

(5) In Schedule 9 (deductions from benefit and~direct payment to third parties)—
\begin{enumerate}\item[]
($a$) in paragraph~5 (service charges for fuel, and~rent not falling within paragraph~2(1)($a$))—
\begin{enumerate}\item[]
(i) in sub-paragraph~(5) omit from “a sum equal to” to the end and~substitute “a sum calculated in accordance with paragraph~8(4);”; and

(ii) omit sub-paragraph~(5A);
\end{enumerate}

($b$) in paragraph~6 (fuel costs)—
\begin{enumerate}\item[]
(i) in sub-paragraph~(6) omit from “a sum equal to” to the end and~substitute “a sum calculated in accordance with paragraph~8(4);”; and

(ii) omit sub-paragraph~(6A);
\end{enumerate}

($c$) in paragraph~7 (water charges)—
\begin{enumerate}\item[]
(i) in sub-paragraph~(8) omit from “a sum equal to” to the end and~substitute “a sum calculated in accordance with paragraph~8(4);”; and

(ii) omit sub-paragraph~(9);
\end{enumerate}

($d$) in paragraph~8\footnote{The relevant amending instruments are S.I.~1990/2208, 1991/2284, 1992/1026, 1993/495, 1996/481 and 1460, 1997/827, 2002/3019 and 2005/777.} (maximum amount of payment to third parties)—
\begin{enumerate}\item[]
(i) in sub-paragraph~(2) omit “a sum equal to” to the end and~substitute “a sum calculated in accordance with sub-paragraph~(4);”;

(ii) omit sub-paragraph~(2A); and

(iii) after sub-paragraph~(3) add—
\begin{quotation}
“(4) The sum referred to in sub-paragraph~(2) is—
\begin{enumerate}\item[]
($a$) where the claimant or partner does not receive child tax credit, 25 per cent of—
\begin{enumerate}\item[]
(i) in the case of income support, the applicable amount for the family as is awarded under sub-paragraphs ($a$)  to ($d$)  of regulation~17(1) (applicable amounts) or sub-paragraphs ($a$)  to ($e$)  of regulation~18(1) (polygamous marriages) of the Income Support Regulations;

(ii) in the case of jobseeker’s allowance, the applicable amount for the family as is awarded under paragraphs ($a$)  to ($e$)  of regulation~83 (applicable amounts) or sub-paragraphs ($a$)  to ($f$)  of regulation~84(1) (polygamous marriages) of the Jobseeker’s Allowance Regulations; or

(iii) in the case of state pension credit, the appropriate minimum guarantee less any housing costs under Schedule 2 to the State Pension Credit Regulations 2002\footnote{S.I.~2002/1792.} which may be applicable in the particular case; or
\end{enumerate}

($b$) where the claimant or his partner receives child tax credit, 25 per cent of the sum of—
\begin{enumerate}\item[]
(i) the amount mentioned in sub-paragraphs ($a$)(i)  to (iii), which applies to the claimant;

(ii) the amount of child benefit awarded to him or his partner by the Board under Part II of the Tax Credits Act 2002\footnote{2002 c.~21.}; and

(iii) the amount of child tax credit awarded to him or his partner by the Board under section~8 of that Act\footnote{Section 8 was amended as it applies to polygamous units by S.I.~2003/742.}.”.
\end{enumerate}
\end{enumerate}
\end{quotation}
\end{enumerate}
\end{enumerate}

\subsection[3. Amendment of the Social Security and~Child Support (Decisions and~Appeals) Regulations 1999]{Amendment of the Social Security and~Child Support (Decisions and~Appeals) Regulations 1999}

3.---(1)  The Social Security and~Child Support (Decisions and~Appeals) Regulations 1999\footnote{S.I.~1999/991.} are amended as follows.

(2) In regulation~7(2) (date from which a decision superseded under section~10 takes effect) after sub-paragraph~($bb$)\footnote{Sub-paragraph ($bb$) was inserted by S.I.~2000/1596.}  insert—
\begin{quotation}
“($bc$) where the decision is advantageous to the claimant and~is made in connection with the cessation of payment of a carer’s allowance, the day after the last day for which that allowance was paid;”.
\end{quotation}

(3) In Schedule 3A\footnote{Schedule 3A was inserted by S.I.~2000/1596 and amended by S.I.~2001/518, 2002/398, 2005/337 and 2878.} (date from which superseding decision takes effect where a claimant is in receipt of income support or jobseeker’s allowance) in paragraph~3, after sub-paragraph~($g$)  add—
\begin{quotation}
“($h$) regulation~9 of the Social Security (Disability Living Allowance) Regulations 1991\footnote{S.I.~1991/2890.} (persons in certain accommodation other than hospitals) applies, or ceases to apply, to the claimant for a period of less than one week.”.
\end{quotation}

\bigskip

Signed 
by authority of the 
Secretary of State for~Work and~Pensions.
%I concur
%By authority of the Lord Chancellor

{\raggedleft
\emph{Philip Hunt}\\*
%Secretary
%Minister
Parliamentary Under-Secretary 
of State,\\*Department 
for~Work and~Pensions

}

4th September 2006

\small

\part{Explanatory Note}

\renewcommand\parthead{— Explanatory Note}

\subsection*{(This note is not part of the Regulations)}

These Regulations amend the Social Security (Claims and~Payments) Regulations 1987 (“the 1987 Regulations”) and~the Social Security and~Child Support (Decisions and~Appeals) Regulations 1999 (“the 1999 Regulations”).

Regulation 2 amends the 1987 Regulations. Paragraph~(2) allows the Secretary of State to extend the time in which defective claims for state pension credit may be corrected.

Paragraph~(3) provides that the date of a claim for a relevant benefit made following the termination of an earlier award of that benefit is the date of entitlement to that qualifying benefit where the qualifying benefit is re-awarded or awarded on a claim made before termination of the award of the relevant benefit, provided the claim for the relevant benefit is made within 3 months of the date of the award of the qualifying benefit.

Paragraph~(4) provides that the time for claiming income support or jobseeker’s allowance may be extended by a period not exceeding 1 month where the claimant was unable to notify the appropriate office of his intention of making a claim because the telephone lines were busy or inoperative.

Paragraph~(5) provides that where a claimant receives child tax credit, the amount that may be deducted from his income support or jobseeker’s allowance may not exceed, without his consent, 25 per cent of the sum of his applicable amount, child benefit and~the child tax credit he or his partner is entitled to. Where the claimant receives state pension credit, deductions may not, without his consent, exceed 25 per cent of the sum of the appropriate minimum guarantee less any housing costs under Schedule 2 to the State Pension Credit Regulations and~the amount of child benefit and~child tax credit he or his partner is entitled to.

Regulation 3 amends the 1999 Regulations. Paragraph~(2) provides that where a decision is superseded in connection with the cessation of payment of a carer’s allowance, the decision takes effect from the day after the day in respect of which the carer’s allowance was paid.

Paragraph~(3) amends Schedule 3A to provide that a decision superseding a decision where regulation~9 of the Social Security (Disability Living Allowance) Regulations 1991 applies to him or ceases to apply to him, for a period of less than 1 week, takes effect from the date of the change.

A full regulatory impact assessment has not been carried out in respect of these Regulations as they do not impose a cost on business, charities or the voluntary sector. 

\end{document}