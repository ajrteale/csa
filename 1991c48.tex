\documentclass[12pt,a4paper]{article}

\usepackage[welsh,english]{babel}

\usepackage[oldrules]{optional}

\newcommand\regstitle{Child Support Act 1991}

\newcommand\regsnumber{c.~48}

\opt{oldrules}{
\title{\regstitle}
}

\opt{newrules}{
\title{Child Support Act 1991\\ (2003 Scheme version)}
}

\opt{2012rules}{
\title{Child Support Act 1991\\ (2012 Scheme version)}
}

\opt{oldrules}{\newcommand\versionyear{1993}}
\opt{newrules}{\newcommand\versionyear{2003}}
\opt{2012rules}{\newcommand\versionyear{2012}}

\author{1991 Chapter 48}

\date{Royal Assent 25th July 1991}

\usepackage{csa-regs}

\setlength\headheight{27.57402pt}

%\hbadness=10000

\renewcommand\siprefix{\relax}

\begin{document}

\maketitle

\noindent
{\large An Act to make provision for the calculation, collection and enforcement of periodical maintenance payable by certain parents with respect to children of theirs who are not in their care; for the collection and enforcement of certain other kinds of maintenance; and for connected purposes.}

\amendment{The long title was amended (3.3.03 for 2003 and 2012 scheme cases) by section 1(2)(b) of the Child Support, Pensions and Social Security Act 2000.

For 1993 scheme cases, for ``calculation'' read ``assessment''.
}

\bigskip

\lettrine{B}{e it enacted} by the Queen’s most Excellent Majesty, by and with the advice and consent of the Lords Spiritual and Temporal, and Commons, in this present Parliament assembled, and by the authority of the same, as follows:—


{\sloppy

\tableofcontents

}

\setcounter{secnumdepth}{-2}

\section{The basic principles}

\subsection{1. The duty to mantain}

(1) For the purposes of this Act, each parent of a qualifying child is
responsible for maintaining him.

(2) For the purposes of this Act, \emph{a non-resident parent} shall be
taken to have met his responsibility to maintain any qualifying child of his by making
periodical payments of maintenance with respect to the child of such amount, and at
such intervals, as may be determined in accordance with the provisions of this Act.

(3) Where a \emph{maintenance calculation} made under this
Act requires the making of periodical payments, it shall be the duty of the \emph{non-resident parent} with respect to whom the \emph{calculation} was
made to make those payments.

\amendment{
S.~1 amended (3.3.03 for 2003 and 2012 scheme cases) by section 1(2)(a) of, and paragraph 11 of Schedule 3 to, the Child Support, Pensions and Social Security Act 2000.

For 1993 scheme cases, for ``(a) non-resident parent'' substitute ``(an) absent parent'' and for ``maintenance assessment'' or ``assessment'' substitute ``maintenance calculation'' or ``calculation''.
}

\subsection{2. Welfare of children: the general principle}

Where, in any case which falls to be dealt with under this Act, the Secretary of State is considering the exercise of any discretionary power conferred by this
Act, the Secretary of State shall have regard to the welfare of any child likely to be
affected by the decision.

\amendment{Section 2 has been amended (1.6.99) by Schedule 8 to the Social Security Act 1998 and (1.8.12) by paragraph 2 of the Schedule to the Public Bodies (Child Maintenance and Enforcement
Commission: Abolition and Transfer of Functions) Order 2012.}

\subsection{3. Meaning of certain terms used in this Act}

(1) A child is a “qualifying child” if---
\begin{enumerate}\item[]
($a$) one of his parents is, in relation to him, a non-resident parent; or

($b$) both of his parents are, in relation to him, non-resident parents.
\end{enumerate}

(2) The parent of any child is a “non-resident parent”, in relation to him, if---

\begin{enumerate}\item[]
($a$)
that parent is not living in the same household with the child; and

($b$)
the child has his home with a person who is, in relation to him, a person
with care.
\end{enumerate}

(3)
A person is a “person with care”, in relation to any child, if he is a person---
\begin{enumerate}\item[]
($a$)
with whom the child has his home;

($b$)
who usually provides day to day care for the child (whether exclusively
or in conjunction with any other person); and

($c$)
who does not fall within a prescribed category of person.
\end{enumerate}

(4)
The Secretary of State shall not, under subsection (3)($c$), prescribe as a
category---
\begin{enumerate}\item[]
($a$)
parents;

($b$)
guardians;

($c$)
persons in whose favour residence orders under section 8 of the Children
Act 1989
 are in force;

($d$)
in Scotland, persons 
%having the right of custody of a child. %orig
with whom a child is to live by virtue of a residence order under section 11 of the Children (Scotland) Act 1995.
\end{enumerate}

(5)
For the purposes of this Act there may be more than one person with care
in relation to the same qualifying child.

(6)
Periodical payments which are required to be paid in accordance with a maintenance calculation
are referred to in this Act as “child
support maintenance”.

(7)
Expressions are defined in this section only for the purposes of this Act.

\amendment{
Section 3 has been amended (31.1.01 for certain purposes, see S.I.~2000/3354; 3.3.03 for 2003 and 2012 scheme cases) by section 1(2)(a) of, and paragraph 11(2) of Schedule 3 to, the Child Support, Pensions and Social Security Act 2000.  Subsection (4)(d) has been amended (1.11.96) by paragraph 52 of Schedule 7 to the Children (Scotland) Act 1995.

For 1993 scheme cases for ``(an) absent parent(s)'' substitute ``(a) non-resident parent(s)'' and for ``maintenance assesment'' substitute ``maintenance calculation''.
}

\subsection{4. Child support maintenance}

(1) A
person who is, in relation to any qualifying child or any qualifying
children, either the person with care or the non-resident parent
may
apply to the Secretary of State for a maintenance calculation
to be made under this Act with respect to that child, or any of those
children.

(2)
Where a maintenance calculation
has been made in
response to an application under this section the Secretary of State may, if the
person with care or non-resident parent
with respect to whom the
\opt{oldrules}{assessment}\opt{newrules,2012rules}{calculation}
was made applies to the Secretary of State under this
subsection, arrange for---
\begin{enumerate}\item[]
($a$)
the collection of the child support maintenance payable in accordance
with the calculation;

($b$)
the enforcement of the obligation to pay child support maintenance in
accordance with the calculation.

\end{enumerate}

(3)
Where an application under subsection (2) for the enforcement of the
obligation mentioned in subsection (2)($b$) authorises the Secretary of State to take
steps to enforce that obligation whenever the Secretary of State considers it
necessary to do so, the Secretary of State may act accordingly.

(4)
A person who applies to the Secretary of State under this section shall, so
far as that person reasonably can, comply with such regulations as may be made
by the Secretary of State with a view to the Secretary of State being provided
with the information which is required to enable---

\begin{enumerate}\item[]
($a$)
the non-resident parent
to be identified or 
traced (where
that is necessary);

($b$)
the amount of child support maintenance payable by the non-resident parent
to be assessed; and

($c$)
that amount to be recovered from the non-resident parent.
\end{enumerate}

(5)
Any person who has applied to the Secretary of State under this section may
at any time request the Secretary of State to cease acting under this section.

(6)
It shall be the duty of the Secretary of State to comply with any request
made under subsection (5) (but subject to any regulations made under subsection~(8)).

(7)
The obligation to provide information which is imposed by subsection~(4)---
\begin{enumerate}\item[]
($a$)
shall not apply in such circumstances as may be prescribed; and

($b$)
may, in such circumstances as may be prescribed, be waived by the Secretary of State.
\end{enumerate}

(8)
The Secretary of State may by regulations make such incidental, supplemental or transitional provision as he thinks appropriate with respect to cases
in which he is requested to cease to act under this section.

(10) No
application may be made at any time under this section with respect
to a qualifying child or any qualifying children if---
\begin{enumerate}\item[]
($a$)
there is in force a written maintenance agreement made before 5th April
1993, or a maintenance order 
made before a prescribed date, in respect
of that child, or those children and the person who is, at that time, the non-resident parent; or

($aa$) a
maintenance order made on or after the date prescribed for the purposes
of paragraph ($a$) is in force in respect of them, but has been so for less than
the period of one year beginning with the date on which it was made;

($ab$) a
maintenance agreement---
\begin{enumerate}\item[]
(i)
made on or after the date prescribed for the purposes of paragraph ($a$);
and

(ii)
registered for execution in the Books of Council and Session or the
sheriff court books,
\end{enumerate}
is in force in respect of them, but has been so for less than the period of one year
beginning with the date on which it was made.
\end{enumerate}

\amendment{
Section 4 amended (3.3.03 for 2003 and 2012 scheme cases only) by section 1(2) of, and paragraph 11 of Schedule 3 to, the Child Support, Pensions and Social Security Act 2000, and (1.8.12) by paragraph 3 of the Schedule to the Public Bodies (Child Maintenance
and Enforcement Commission: Abolition and Transfer of Functions) Order 2012.

Subsection (4) amended (1.6.99) by Schedule 8 to the Social Security Act 1998.

Subsection (6) should be read in conjunction with regulation 52(8) of the Child Support (Maintenance Assessment Procedure) Regulations 1992.

Subsection (9) repealed (27.10.08) by Schedule 8 to the Child Maintenance and Other Payments Act 2008.

Subsections (10) and (11) inserted (4.9.95) by section 18(1) of the Child Support Act 1995, but under section 18(6) of that Act subsection (10) does not apply to maintenance orders under section 8(7) or (8) below.  Subsection (10)(a) amended (4.2.03 for regulation-making purposes, 3.3.03 for 2003 and 2012 scheme cases) by section 2(2) of the Child Support, Pensions and Social Security Act 2000.  Subsection (10)(aa) inserted (3.3.03 for 2003 and 2012 scheme cases only) by section 2(3) of the Child Support, Pensions and Social Security Act 2000.  Subsection (10)(ab) inserted (6.6.08) by section 35(1) of the Child Maintenance and Other Payments Act 2008.  Subsection (10)(b) repealed (14.7.08 for certain cases, 27.10.08 otherwise) by Schedule 8 to the Child Maintenance and Other Payments Act 2008.  Subsection (11) repealed (27.10.08) by Schedule 8 to the Child Maintenance and Other Payments Act 2008.

For 1993 scheme cases, references to maintenance calculations should be read as references to maintenance assessments; references to calculations should be read as references to assessments; references to non-resident parents should be read as references to absent parents; subsection (4)(a) should be read as:
\begin{quotation}
($a$)
the absent parent
to be 
traced (where
that is necessary);
\end{quotation}
and subsection (10) should be read as:
\begin{quotation}
(10) No
application may be made at any time under this section with respect
to a qualifying child or any qualifying children if---
\begin{enumerate}\item[]
($a$)
there is in force a written maintenance agreement made before 5th April
1993, or a maintenance order, in respect
of that child, or those children and the person who is, at that time, the absent
parent; or

($ab$) a
maintenance agreement---
\begin{enumerate}\item[]
(i)
made on or after the date prescribed for the purposes of paragraph ($a$);
and

(ii)
registered for execution in the Books of Council and Session or the
sheriff court books,
\end{enumerate}
is in force in respect of them, but has been so for less than the period of one year
beginning with the date on which it was made.
\end{enumerate}
\end{quotation}

}

\subsection{5. Child support
maintenance:
supplemental
provisions}

(1)
Where---
\begin{enumerate}\item[]
($a$)
there is more than one person with care of a qualifying child; and

($b$)
one or more, but not all, of them have parental responsibility for 
%(or, in Scotland, parental rights over) 
the child;
\end{enumerate}
no application may be made for a maintenance calculation
with respect to the child by any of those persons who do not have parental responsibility
for 
%(or, in Scotland, parental rights over) 
the child.

(2)
Where more than one application for a maintenance calculation
is made with respect to the child concerned, only one of them may be
proceeded with.

(3)
The Secretary of State may by regulations make provision as to which of
two or more applications for a maintenance calculation
with respect to the same child is to be proceeded with. 

\amendment{
Section 5 is amended (3.3.03 for 2003 and 2012 scheme cases) by section 1(2)(a) of the Child Support, Pensions and Social Security Act 2000.  Subsection (1) is amended (1.11.96) by paragraph 52 of Schedule 4 to the Children (Scotland) Act 1995.

For 1993 scheme cases, references to maintenance calculations should be read as references to maintenance assessments.

\medskip

Section 6 (applications by those receiving benefit) was repealed (14.7.08 for certain cases, 27.10.08 and 1.6.09 for all other purposes, subject to transitional provisions in article 4 of S.I.~2008/2548) by Schedule 8 to the Child Maintenance and Other Payments Act 2008.
}


\subsection{7. Right of child in Scotland to apply for assessment}

(1) A qualifying child who has attained the age of 12 years and who is habitually
resident in Scotland may apply to the Secretary of State for a maintenance calculation to be made with respect to him if
no such application has been made by a person who is, with respect to that
child, a person with care or a non-resident parent.

(2)
An application made under subsection (1) shall authorise the 
Secretary of
State to make a maintenance calculation
with respect to
any other children of the non-resident parent
who are qualifying
children in the care of the same person as the child making the application.

(3)
Where a maintenance calculation
has been made in
response to an application under this section the Secretary of State may, if the
person with care, the non-resident parent
with respect to whom the
calculation
was made or the child concerned applies to 
the Secretary
of State under this subsection, arrange for---
\begin{enumerate}\item[]
($a$)
the collection of the child support maintenance payable in accordance
with the calculation;

($b$)
the enforcement of the obligation to pay child support maintenance in
accordance with the calculation.
\end{enumerate}
	
(4)
Where an application under subsection (3) for the enforcement of the
obligation mentioned in subsection (3)($b$) authorises the Secretary of State to take
steps to enforce that obligation whenever the Secretary of State considers it
necessary to do so, the Secretary of State may act accordingly.

(5)
Where a child has asked the Secretary of State to proceed under this section,
the person with care of the child, the non-resident parent
and the
child concerned shall, so far as they reasonably can, comply with such regulations as
may be made by the Secretary of State with a view to the Secretary of State being provided with the information which is required to enable---
\begin{enumerate}\item[]
($a$)
the non-resident parent
to be traced (where that is necessary);

($b$)
the amount of child support maintenance payable by the non-resident parent
to be assessed; and

($c$)
that amount to be recovered from the non-resident parent.
\end{enumerate}

(6)
The child who has made the application (but not the person having care of him)
may at any time request the Secretary of State to cease acting under this section.

(7)
It shall be the duty of the Secretary of State to comply with any request made
under subsection (6) (but subject to any regulations made under subsection~(9)).

(8)
The obligation to provide information which is imposed by subsection~(5)---
\begin{enumerate}\item[]
($a$)
shall not apply in such circumstances as may be prescribed by the Secretary
of State; and

($b$)
may, in such circumstances as may be so prescribed, be waived by the
Secretary of State.
\end{enumerate}

(9)
The Secretary of State may by regulations make such incidental, supplemental
or transitional provision as he thinks appropriate with respect to cases
in which he is requested to cease to act under this section.

(10) No application may be made at any time under this section by a qualifying
child if---
\begin{enumerate}\item[]
($a$)
there is in force a written maintenance agreement made before 5th
April 1993, or a maintenance order made before a prescribed date, in
respect of that child and the person who is, at that time, the non-resident
parent; or

($b$)
a maintenance order made on or after the date prescribed for the purposes
of paragraph ($a$) is in force in respect of them, but has been so for less than
the period of one year beginning with the date on which it was made; or

($c$)
a maintenance agreement---
\begin{enumerate}\item[]
(i)
made on or after the date prescribed for the purposes of paragraph~($a$); and

(ii)
registered for execution in the Books of Council and Session or the
sheriff court books,
\end{enumerate}
is in force in respect of them, but has been so for less than the period of one year
beginning with the date on which it was made.
\end{enumerate}

\amendment{
Section 7 was amended (3.3.03 for 2003 and 2012 scheme cases only) by section 1(2)(a) of, and paragraph 11(2) of Schedule 3 to, the Child Support, Pensions and Social Security Act 2000, and (1.8.12) by paragraph 4 of the Schedule to the Public Bodies (Child Maintenance and Enforcement Commission: Abolition and Transfer of Functions) Order 2012.

Subsection (1)(b) was repealed (14.7.08 for certain cases only as provided in S.I.~2008/1476, 1.6.09 for all other purposes) by Schedule 8 to the Child Maintenance and Other Payments Act 2008.

Subsection (5) was amended (1.6.99) by Schedule 8 to the Social Security Act 1998.

Subsection (10) was inserted (4.9.95) by section 18(2) of the Child Support Act 1995, but under section 18(6) of that Act does not apply to maintenance orders under section 8(7) or (8) below.  Subsection (10) was substituted (4.2.03 for regulation-making purposes, 3.3.03 for 2003 and 2012 scheme cases) by paragraph 11(4) of Schedule 3 to the Child Support, Pensions and Social Security Act 2000.  Subsection (10)(c) was inserted (6.6.08 for 2003 and 2012 scheme cases) by section 35(2) of the Child Maintenance and Other Payments Act 2008.

For 1993 scheme cases, references to maintenance calculations should be read as references to maintenance assessments, references to absent parents should be read as references to non-resident parents, and subsection (10) should read as:
\begin{quotation}
(10) No
application may be made at any time under this section by a qualifying
child if there is in force a written maintenance agreement made before 5th
April 1993, or a maintenance order, in respect of that child and the person who is,
at that time, the absent parent.\end{quotation}

}


\subsection{8. Role of the courts with respect to maintenance for children}

(1) This
subsection applies in any case where the Secretary of State would
have jurisdiction to make a maintenance calculation
with
respect to a qualifying child and a non-resident parent
of his on an
application duly made by a person entitled to apply for such a calculation
with respect to that child.

(2)
Subsection (1) applies even though the circumstances of the case are such
that the Secretary of State would not make a calculation if it were
applied for.

(3)
Except as provided in subsection (3A), in any case where subsection (1)
applies, no court shall exercise any power which it would otherwise have to make,
vary or revive any maintenance order in relation to the child and absent parent
concerned.

(3A) Unless a maintenance calculation has been made with respect to the child
concerned, subsection (3) does not prevent a court from varying a maintenance
order in relation to that child and the non-resident parent concerned---
\begin{enumerate}\item[]
($a$)
if the maintenance order was made on or after the date prescribed for the
purposes of section 4(10)($a$) or 7(10)($a$); or

($b$)
where the order was made before then, in any case in which section~4(10)
or 7(10) prevents the making of an application for a maintenance
calculation with respect to or by that child.
\end{enumerate}

(4)
Subsection (3) does not prevent a court from revoking a maintenance order.

(5)
The Lord Chancellor or in relation to Scotland the Lord Advocate may by order
provide that, in such circumstances as may be specified by the order, this section shall
not prevent a court from exercising any power which it has to make a maintenance
order in relation to a child if---
\begin{enumerate}\item[]
($a$)
a written agreement (whether or not enforceable) provides for the
making, or securing, by a non-resident parent
of the
child of periodical payments to or for the benefit of the child; and

($b$)
the maintenance order which the court makes is, in all material respects,
in the same terms as that agreement.
\end{enumerate}


(5A)
The Lord Chancellor may make an order under subsection (5) only with the
concurrence of the Lord Chief Justice.

(6)
This section shall not prevent a court from exercising any power which it
has to make a maintenance order in relation to a child if---
\begin{enumerate}\item[]
($a$)
a maintenance calculation
is in force with respect
to the child;

($b$) the non-resident parent’s gross weekly income exceeds the figure
referred to in paragraph 10(3) of Schedule 1 (as it has effect from time to
time pursuant to regulations made under paragraph 10A(1)($b$)); and

($c$)
the court is satisfied that the circumstances of the case make it appropriate for the non-resident parent
to make or secure the making
of periodical payments under a maintenance order in addition to the child
support maintenance payable by him in accordance with the maintenance calculation.
\end{enumerate}

(7)
This section shall not prevent a court from exercising any power which it
has to make a maintenance order in relation to a child if---
\begin{enumerate}\item[]
($a$)
the child is, will be or (if the order were to be made) would be receiving
instruction at an educational establishment or undergoing training for a
trade, profession or vocation (whether or not while in gainful employment); and

($b$)
the order is made solely for the purposes of requiring the person making
or securing the making of periodical payments fixed by the order to meet
some or all of the expenses incurred in connection with the provision of
the instruction or training.
\end{enumerate}

(8)
This section shall not prevent a court from exercising any power which it
has to make a maintenance order in relation to a child if---
\begin{enumerate}\item[]
($a$) an allowance under Part IV of the Welfare Reform Act 2012 (personal independence payment) or
a disability living allowance is paid to or in respect of him; or

($b$)
no such allowance is paid but he is disabled,
\end{enumerate}
and the order is made solely for the purpose of requiring the person making or
securing the making of periodical payments fixed by the order to meet some or all
of any expenses attributable to the child’s disability.

(9) For the purposes of subsection (8), a child is disabled if he is blind, deaf or
dumb or is substantially and permanently handicapped by illness, injury, mental
disorder or congenital deformity or such other disability as may be prescribed.

(10) This section shall not prevent a court from exercising any power which it
has to make a maintenance order in relation to a child if the order is made against
a person with care of the child.

(11) In this Act “maintenance order”, in relation to any child, means an order
which requires the making or securing of periodical payments to or for the benefit
of the child and which is made under---
\begin{enumerate}\item[]
($a$)
Part II of the Matrimonial Causes Act 1973;

($b$)
the Domestic Proceedings and Magistrates’ Courts Act 1978;

($c$)
Part III of the Matrimonial and Family Proceedings Act 1984;

($d$)
the Family Law (Scotland) Act 1985;

($e$)
Schedule 1 to the Children Act 1989;

($ea$)
Schedule 5, 6 or 7 to the Civil Partnership Act 2004; or

($f$)
any other prescribed enactment.
\end{enumerate}
and includes any order varying or reviving such an order.

(12)
The Lord Chief Justice may nominate a judicial office holder (as defined in
section 109(4) of the Constitutional Reform Act 2005) to exercise his functions under
this section.

\amendment{
Section 8 has been amended (1.6.99) by paragraph 22 of Schedule 7 to the Social Security Act 1998, (3.3.03 for 2003 and 2012 scheme cases only) by section 1(2) of, and paragraph 11(2) and (5) of Schedule 3 to, the Child Support, Pensions and Social Security Act 2000, and (1.8.12) by paragraph 5 of the Schedule to the Public Bodies (Child Maintenance and
Enforcement Commission: Abolition and Transfer of Functions) Order 2012.

Subsection (1) was amended (27.10.08) by Schedule 8 to the Child Maintenance and Other Payments Act 2008.

Subsection (3A) was inserted (4.9.95) by section 18(3) of the Child Support Act 1995 and substituted (3.3.03 for new-rules cases only) by paragraph 11(5) of Schedule 3 to the Child Support, Pensions and Social Security Act 2000.

Subsections (5A) and (12) were inserted (3.4.06) by paragraph 219 of Schedule 4 to the Constitutional Reform Act 2005.

Subsection (6) was amended (10.12.12 for 2012 scheme cases only) by paragraph 1(2) of Schedule 7 to the Child Maintenance and Other Payments Act 2008.

Subsection (8)(a) was amended (8.4.13 for certain cases only as provided in article 7 of S.I.~2013/358) by paragraph 2 of Schedule 9 to the Welfare Reform Act 2012.

Subsection (11)(e) was amended and subsection (11)(ea) inserted (5.12.05) by Part I of Schedule 24 to the Civil Partnership Act 2004.

For 1993 scheme cases, references to maintenance calculations should be read as references to maintenance assessments, references to calculations should be read as references to assessments, references to absent parents should be read as references to non-resident parents, subsections (3) and (3A) should read as:
\begin{quotation}
(3)
In any case where subsection (1)
applies, no court shall exercise any power which it would otherwise have to make,
vary or revive any maintenance order in relation to the child and absent parent
concerned.

(3A) In
any case in which section 4(10) or 7(10) prevents the making of an
application for a maintenance assessment, and---
\begin{enumerate}\item[]
($a$)
no application has been made for a maintenance assessment under section
6, or

($b$)
such an application has been made but no maintenance assessment has been
made in response to it,
\end{enumerate}
subsection (3) shall have effect with the omission of the word “vary”.
\end{quotation}
and subsection (6)(b) should read as:
\begin{quotation}
($b$)
the amount of the child support maintenance payable in accordance with
the assessment was determined by reference to the alternative formula
mentioned in paragraph 4(3) of Schedule 1; and.
\end{quotation}

For 2003 scheme cases, the reference to gross weekly income in subsection (6)(b) should be read as a reference to net weekly income.
}

\subsection{9. Agreements about maintenance}

(1) In
this section “maintenance agreement” means any agreement for the
making, or for securing the making, of periodical payments by way of maintenance, or in Scotland aliment, to or for the benefit of any child.

(2) Nothing in this Act shall be taken to prevent any person from entering into
a maintenance agreement.

(3) Subject to section 4(10)($a$) and ($ab$) and section 7(10), the existence of a
maintenance agreement shall not prevent any party to the agreement, or any other
person, from applying for a \opt{oldrules}{maintenance assessment}\opt{newrules,2012rules}{maintenance calculation}
with
respect to any child to or for whose benefit periodical payments are to be made or
secured under the agreement.

(4) Where any agreement contains a provision which purports to restrict the
right of any person to apply for a \opt{oldrules}{maintenance assessment}\opt{newrules,2012rules}{maintenance calculation},
that provision shall be void.

(5) Where section 8 would prevent any court from making a maintenance order
in relation to a child and \opt{oldrules}{an absent parent}\opt{newrules,2012rules}{a non-resident parent}
of his, no court
shall exercise any power that it has to vary any agreement so as---
\begin{enumerate}\item[]
($a$) to insert a provision requiring that \opt{oldrules}{absent parent}\opt{newrules,2012rules}{non-resident parent}
to
make or secure the making of periodical payments by way of maintenance,
or in Scotland aliment, to or for the benefit of that child; or

($b$)
to increase the amount payable under such a provision.
\end{enumerate}

(6) In any case in which section 4(10) or 7(10) prevents the making of an
application for a maintenance assessment, subsection (5) shall have effect with
the omission of paragraph ($b$).

\amendment{
Words inserted in s. 9(3) (4.9.95) by the Child Support Act 1995 (c. 34) s. 18(4).

S. 9(6) inserted (4.9.95) by the Child Support Act 1995 (c. 34) s. 18(4).

\opt{newrules,2012rules}{Words “maintenance assessment(s)” substituted by “maintenance calculation(s)” (3.3.03) for the
purposes of certain cases only (see S.I. 2003/192) by the Child Support, Pensions
and Social Security Act 2000 (c. 19) s. 1(2)($a$).

Words “(an) absent parent” substituted by “($a$) non-resident parent(s)” (3.3.03) for the purposes of
certain cases only (see S.I. 2003/192) by the Child Support, Pensions and Social
Security Act 2000 (c. 19) Sch. 3 para. 11(2).}

Words inserted in s. 9(3) (6.6.08) by the Child Maintenance and Other Payments Act 2008 (c.
6) s. 35(3).

S.
9(6)($a$), ($b$) repealed (27.10.08) by the Child Maintenance and Other Payments Act 2008
(c.
6) Sch. 8.
}

\subsection{10. Relationship between
maintenance assessments
and certain court orders
and related matters}

(1) Where an order of a kind prescribed for the purpose of this subsection is in
force with respect to any qualifying child with respect to whom a \opt{oldrules}{maintenance
assessment}\opt{newrules,2012rules}{maintenance calculation}
is made, the order---
\begin{enumerate}\item[]
($a$)
shall, so far as it relates to the making or securing of periodical payments,
cease to have effect to such extent as may be determined in accordance
with regulations made by the Secretary of State; or

($b$)
where the regulations so provide, shall, so far as it so relates, have effect
subject to such modifications as may be so determined.
\end{enumerate}

(2)
Where an agreement of a kind prescribed for the purpose of this subsection
is in force with respect to any qualifying child with respect to whom a \opt{oldrules}{maintenance
assessment}\opt{newrules,2012rules}{maintenance calculation}
is made, the agreement---
\begin{enumerate}\item[]
($a$)
shall, so far as it relates to the making or securing of periodical payments,
be unenforceable to such extent as may be determined in accordance with
regulations made by the Secretary of State; or

($b$)
where the regulations so provide, shall, so far as it so relates, have effect
subject to such modifications as may be so determined.
\end{enumerate}

(3)
Any regulations under this section may, in particular, make such provision
with respect to---
\begin{enumerate}\item[]
($a$)
any case where any person with respect to whom an order or agreement
of a kind prescribed for the purposes of subsection (1) or (2) has effect
applies to the prescribed court, before the end of the prescribed period,
for the order or agreement to be varied in the light of the \opt{oldrules}{maintenance
assessment}\opt{newrules,2012rules}{maintenance calculation}
and of the provisions of this Act;

($b$)
the recovery of any arrears under the order or agreement which fell due
before the coming into force of the \opt{oldrules}{maintenance assessment}\opt{newrules,2012rules}{maintenance
calculation},
\end{enumerate}
as the Secretary of State considers appropriate and may provide that, in prescribed
circumstances, any application to any court which is made with respect to an order
of a prescribed kind relating to the making or securing of periodical payments to
or for the benefit of a child shall be treated by the court as an application for the
order to be revoked.

(4)
The Secretary of State may by regulations make provision for---
\begin{enumerate}\item[]
($a$)
notification to be given by 
the 
Secretary of State to the prescribed person
in any case where the Secretary of State considers that the making of a
\opt{oldrules}{maintenance assessment}\opt{newrules,2012rules}{maintenance calculation}
has affected, or is likely
to affect, any order of a kind prescribed for the purposes of this subsection;

($b$)
notification to be given by the prescribed person to the Secretary of State
in any case where a court makes an order which it considers has affected,
or is likely to affect, a \opt{oldrules}{maintenance assessment}\opt{newrules,2012rules}{maintenance calculation}.
\end{enumerate}

(5) Rules 
%may be made under section 144 of the Magistrates’ Courts Act 1980
%(rules of procedure) requiring 
of court may require   % Words substituted (22.4.14) by 2013 c 22 Sch 11 para 123(a)
any person who, in prescribed circumstances, makes
an application to 
%a magistrates’ court 
the family court     % Words substituted (22.4.14) by 2013 c 22 Sch 11 para 123(b)
for a maintenance order to furnish the court
with a statement in a prescribed form, and signed by an officer of the Secretary of
State, as to whether or not, at the time when the statement is made, there is a
\opt{oldrules}{maintenance assessment}\opt{newrules,2012rules}{maintenance calculation}
in force with respect to that person
or the child concerned.

(6) In this subsection---
\begin{enumerate}\item[]
“maintenance order” means an order of a prescribed kind for the making or
securing of periodical payments to or for the benefit of a child; and

“prescribed” means prescribed by the rules.
\end{enumerate}

\amendment{
Words substituted in s. 10 (1.6.99) by the Social Security Act 1998 (c. 14) Sch. 7 para. 23.

\opt{newrules,2012rules}{
Words “maintenance assessment(s)” substituted by “maintenance calculation(s)” (3.3.03) for
the purposes of certain cases only (see S.I. 2003/192) by the Child Support,
Pensions and Social Security Act 2000 (c. 19) s. 1(2)($a$).
}

Words substituted in s. 10 (1.8.12) by the Public Bodies (Child Maintenance and Enforcement
Commission: Abolition and Transfer of Functions) Order 2012 Sch. para. 6.

Words substituted in s. 10(5) (24.4.14) by the Crime and Courts Act 2013 Sch. 11. para. 123.
}

\opt{oldrules}{
\section{Maintenance assessments}

\subsection{11. Maintenance assessments}

(1) Any application
for a maintenance assessment made to the Secretary of State
shall be dealt with by the Secretary of State in accordance with the provision made by or under this
Act.

%(1A)
%Where---
%\begin{enumerate}\item[]
%($a$)
%an application for a maintenance assessment is made under section 6, but
%
%($b$)
%the Secretary of State becomes aware before determining the application that
%the claim mentioned in subsection (1) of that section has been disallowed or
%withdrawn,
%\end{enumerate}
%he shall, subject to subsection (1B), treat the application as if it had not been made.
%
%(1B) If it
%appears to the Secretary of State that subsection (10) of section 4
%would not have prevented the parent with care concerned from making an application for a maintenance assessment under that section he shall---
%\begin{enumerate}\item[]
%($a$)
%notify her of the effect of this subsection, and
%
%($b$)
%if, before the end of the period of 28 days beginning with the day on
%which notice was sent to her, she asks him to do so, treat the application
%as having been made not under section 6 but under section 4.
%\end{enumerate}
%
%(1C)
%Where the application is not preserved under subsection (1B) (and so is
%treated as not having been made) the Secretary of State shall notify---
%\begin{enumerate}\item[]
%($a$)
%the parent with care concerned; and
%
%($b$)
%the absent parent (or alleged absent parent), where it appears to him that
%that person is aware of the application.
%\end{enumerate}

(2)
The amount of child support maintenance to be fixed by any maintenance
assessment shall be determined in accordance with the provisions of Part I of
Schedule 1.

(3)
Part II of Schedule 1 makes further provision with respect to maintenance
assessments.

\amendment{
%S. 11(1A)--(1C) inserted (4.9.95) by Child Support Act 1995 (c. 34), s. 19.

Words substituted in s. 11(1), (1A)($b$) (5.7.99) by the Social Security Act 1998 (c. 14)
Sch. 7 para. 24.

S. 11(1A)--(1C) repealed (14.7.08 for certain cases only (see S.I. 2008/1476), 1.6.09 for all other purposes)
by the Child Maintenance and Other Payments Act 2008 (c. 6) Sch. 7 para. 1(34).

Words substituted in s. 11 (1.11.08) by the Child Maintenance and Other Payments Act 2008
(c.
6) Sch. 3 para. 8.

Words substituted in s. 11 (1.8.12) by the Public Bodies (Child Maintenance and Enforcement
Commission: Abolition and Transfer of Functions) Order 2012 Sch. para. 7.
}
}

\opt{newrules,2012rules}{
\section{Maintenance calculations}

\subsection{11. Maintenance calculations}

(1) An application for a maintenance calculation made to the Secretary
of State shall be dealt with by the Secretary of State in accordance with the
provision made by or under this Act.

(2)
The Secretary of State shall (unless the Secretary of State decides not to
make a maintenance calculation in response to the application, or makes a decision
under section 12) determine the application by making a decision under this section
about whether any child support maintenance is payable and, if so, how much.

(6)
The amount of child support maintenance to be fixed by a maintenance
calculation shall be determined in accordance with Part I of Schedule 1 unless an
application for a variation has been made and agreed.

(7)
If the Secretary of State has agreed to a variation, the amount of child
support maintenance to be fixed shall be determined on the basis determined
under section 28F(4).

(8)
Part II of Schedule 1 makes further provision with respect to maintenance
calculations.

\amendment{
S. 11 substituted (3.3.03) for the purposes of certain cases only (see S.I. 2003/192) by the Child Support, Pensions and Social Security Act 2000 (c. 19) s. 1(1).

S. 11(3)--(5) repealed (14.7.08 for certain cases only, 27.10.08 for all other purposes) by the Child Maintenance and Other Payments Act 2008 (c. 6) Sch. 8.

Words substituted in s. 11 (1.8.12) by the Public Bodies (Child Maintenance and Enforcement
Commission: Abolition and Transfer of Functions) Order 2012 Sch. para. 7.
}

}

\opt{oldrules}{\subsection{12. Interim maintenance assessments}

(1) Where the Secretary of State---
\begin{enumerate}\item[]
($a$)
is required to make a maintenance assessment, or

($b$)
is proposing to make a decision under 16 or 17,
\end{enumerate}
and (in either case) it appears to him that he does not have sufficient
information to enable him to do so, he may make an interim assessment.

(2)
The Secretary of State may by regulations make provision as to the interim
maintenance assessments.

(3) The regulations may, in particular, make provision as to---
\begin{enumerate}\item[]
($a$) the procedure to be followed in making an interim maintenance assessment; and

($b$) the basis on which the amount of child support maintenance fixed by an interim assessment is to be calculated.
\end{enumerate}

(4) Before making any interim assessment the Secretary of State shall, if it is reasonably practicable to do so, give written notice of his intention to make such an assessment to---
\begin{enumerate}\item[]
($a$) the absent parent concerned;

($b$) the person with care concerned; and

($c$) where the application for a maintenance assessment was made under section 7, the child concerned.
\end{enumerate}

(5) Where the Secretary of State serves notice under subsection (4), he shall not make the proposed interim assessment before the end of such period as may be prescribed.

\amendment{

S. 12(1), (1A) substituted for s. 12(1) (1.6.99) by the Social Security Act 1998 (c. 14) Sch. 7 para. 25(1).

Words in s. 12(4), (5) substituted (1.6.99) by the Social Security Act 1998 (c. 14) Sch. 7 para. 25(2).

Words substituted in s. 12 (1.8.12) by the Public Bodies (Child Maintenance and Enforcement
Commission: Abolition and Transfer of Functions) Order 2012, Sch. para. 8.

\medskip

S. 13 repealed (1.6.99) by the Social Security Act 1998 (c. 14) Sch. 8.
}
}

\opt{newrules,2012rules}{

\subsection{12. Default and interim maintenance decisions}

(1) Where the Secretary of State---
\begin{enumerate}\item[]
($a$) is required to make a maintenance calculation; or

($b$)
is proposing to make a decision under section 16 or 17, 
\end{enumerate}
and it appears to the Secretary of State that the Secretary of State does not
have sufficient information to enable such a decision to be made, the Secretary of State may make a default maintenance decision.

(2)
Where an application for a variation has been made under section 28A(1) in connection with an application for a maintenance calculation, the Secretary of State may make an interim maintenance decision.

(3)
The amount of child support maintenance fixed by an interim maintenance decision shall be determined in accordance with Part I of Schedule 1.

(4)
The Secretary of State may by regulations make provision as to default and interim maintenance decisions.

(5)
The regulations may, in particular, make provision as to---
\begin{enumerate}\item[]
($a$) the procedure to be followed in making a default or an interim maintenance decision; and

($b$) a default rate of child support maintenance to apply where a default maintenance decision is made.
\end{enumerate}

\amendment{
S. 12 substituted (10.11.00 for the purpose of making regulations and Acts of Sederunt only,
3.3.03 for the purposes of certain cases only, see S.I. 2003/192) by the Child Support, Pensions and Social Security Act 2000 (c. 19), s. 4.

Words repealed in s. 12(2) (27.10.08) by the Child Maintenance and Other Payments Act 2008
(c. 6), Sch. 8. 

Words substituted in s. 12 (1.8.12) by the Public Bodies (Child Maintenance and Enforcement
Commission: Abolition and Transfer of Functions) Order 2012, Sch. para. 8.

\medskip

S. 13 repealed (1.6.99) by the Social Security Act 1998 (c. 14) Sch. 8.
}
}

\section{Information}

\subsection{14. Information required by Secretary of State}

(1) The Secretary of State may make regulations requiring any information or
evidence needed for the determination of any application made under this Act,
or any question arising in connection with such an application, or needed for
the making of any decision or in connection with the imposition of any condition or
requirement under this Act, or needed in connection with the collection or
enforcement of child support or other maintenance under this Act, to be furnished---
\begin{enumerate}\item[]
($a$)
by such persons as may be determined in accordance with regulations
made by the Secretary of State; and

($b$)
in accordance with the regulations.
\end{enumerate}

(1A)
Regulations under subsection (1) may make provision for notifying any
person who is required to furnish any information or evidence under the regulations of
the possible consequences of failing to do so.

(3)
The Secretary of State may by regulations make provision authorising the
disclosure by the Secretary of State, in such circumstances as may be prescribed,
of such information held by the Secretary of State for purposes of this Act as may be
prescribed.

(4)
The provisions of Schedule 2 (which relate to information which is held for
purposes other than those of this Act but which is required by the Secretary of State)
shall have effect. 

\amendment{

S. 14(1A) inserted (1.10.95) by Child Support Act 1995 (c. 34) Sch. 3 para. 3(1).

S. 14(2), (2A) repealed (8.9.98) by the Social Security Act 1998 (c. 14) Sch. 7 para. 27($a$).

Words in s. 14(3) repealed (1.6.99) by the Social Security Act 1998 (c. 14) Sch. 8.

\opt{newrules,2012rules}{Words in s. 14(1) inserted (3.3.03) for the purposes of certain cases only (see S.I. 2003/192) by the Child Support, Pensions and Social Security Act 2000 (c. 19) s. 12 and Sch. 3 para. 11(7).}

Words repealed in s. 14(1) (27.10.08) by the Child Maintenance and Other Payments Act (c. 6) Sch. 8.

Words substituted in s. 14 (1.8.12) by the Public Bodies (Child Maintenance and Enforcement Commission: Abolition and Transfer of Functions) Order 2012 Sch. para. 9.

}

\subsection{14A. Information offences}

(1) This section applies to---
\begin{enumerate}\item[]
($a$)
persons who are required to comply with regulations under section 4(4) or
7(5); and

($b$)
persons specified in regulations under section 14(1)($a$).
\end{enumerate}

(2)
Such a person is guilty of an offence if, pursuant to a request for information
under or by virtue of those regulations---
\begin{enumerate}\item[]
($a$)
he makes a statement or representation which he knows to be false; or

($b$)
he provides, or knowingly causes or knowingly allows to be provided, a
document or other information which he knows to be false in a material
particular.
\end{enumerate}

(3)
Such a person is guilty of an offence if, following such a request, he fails to
comply with it.

(3A) In
the case of regulations under section 14 which require a person liable to
make payments of child support maintenance to notify---
\begin{enumerate}\item[]
($a$) a change of address, or

($b$)
any other change of circumstances,
\end{enumerate}
a person who fails to comply with the requirement is guilty of an offence.

(4)
It is an defence for a person charged with an offence under subsection (3) or
(3A) to prove that he had a reasonable excuse for failing to comply.

(5)
A person guilty of an offence under this section is liable on summary conviction
to a fine not exceeding level 3 on the standard scale.

(6) In England and Wales, an information relating to an offence under subsection
(2)
may be tried by a magistrates’ court if it is laid within the period of 12 months
beginning with the commission of the offence.

(7)
In Scotland, summary proceedings for an offence under subsection (2) may be
commenced within the period of 12 months beginning with the commission of the
offence.

(8)
Section 136(3) of the Criminal Procedure (Scotland) Act 1995 (date
when proceedings deemed to be commenced) applies for the purposes of subsection
(7) as it applies for the purposes of that section.

\amendment{

S. 14A inserted (31.1.01) by the Child Support, Pensions and Social Security Act
2000 (c. 19) s. 13($b$).

Words inserted in s. 14A(4) (26.9.08 for reg.
making purposes and 27.10.08 for all other
purposes) by the Child Maintenance and Other Payments Act 2008 (c. 6) s. 36.

S. 14A(6)--(8) inserted (14.1.10) by the Welfare Reform Act 2009 (c 24) s. 55(3).

S. 14A(3A) substituted (8.10.12) by the Welfare Reform Act 2009 (c. 24) s. 55(2). 


}

\subsection{15. Powers of inspectors}

(1) The Secretary of State may appoint, on such terms as the Secretary of State thinks fit, persons to act as inspectors under this section.

(2)
The function of inspectors is to acquire information which the Secretary of State needs for any of the purposes of this Act.

(3)
Every inspector is to be given a certificate of his appointment.

(4)
An inspector has power, at any reasonable time and either alone or accompanied by such other persons as he thinks fit, to enter any premises which---
\begin{enumerate}\item[]
($a$) are liable to inspection under this section; and

($b$) are premises to which it is reasonable for him to require entry in order that he may exercise his functions under this section,
\end{enumerate}
and may there make such examination and inquiry as he considers appropriate.

(4A) Premises liable to inspection under this section are those which are not used wholly as a dwelling house and which the inspector has reasonable grounds for suspecting are---
\begin{enumerate}\item[]
($a$) premises at which a non-resident parent is or has been employed;

($b$) premises at which a non-resident parent carries out, or has carried out, a trade, profession, vocation or business;

($c$) premises at which there is information held by a person (“A”) whom the inspector has reasonable grounds for suspecting has information about a non-resident parent acquired in the course of A’s own trade, profession, vocation or business.
\end{enumerate}

(5)
An inspector exercising his powers may question any person aged 18 or over whom he finds on the premises.

(6)
If required to do so by an inspector exercising his powers, any such person shall furnish to the inspector all such information and documents as the inspector may reasonably require.

(7)
No person shall be required under this section to answer any question or to give any evidence tending to incriminate himself or, in the case of a person who is married or is a civil partner, his or her spouse or civil partner.

(8)
On applying for admission to any premises in the exercise of his powers, an inspector shall, if so required, produce his certificate.

(9)
If any person---
\begin{enumerate}\item[]
($a$)
intentionally delays or obstructs any inspector exercising his powers; or

($b$)
without reasonable excuse, refuses or neglects to answer any question or furnish any information or to produce any document when required to do so under this section,
\end{enumerate}
he shall be guilty of an offence and liable on summary conviction to a fine not exceeding level 3 on the standard scale.

(10) In this section---
\begin{enumerate}\item[]
 “certificate” means a certificate of appointment issued under this section; 

“inspector” means an inspector appointed under this section; 

“powers” means powers conferred by this section.
\end{enumerate}

(11) In this section “premises” includes---
\begin{enumerate}\item[]
($a$)
moveable structures and vehicles, vessels, aircraft and hovercraft;

($b$)
installations that are offshore installations for the purposes of the Mineral Workings (Offshore Installations) Act 1971; and

($c$)
places of all other descriptions whether or not occupied as land or otherwise,
\end{enumerate}
and references in this section to the occupier of premises are to construed, in relation to premises that are not occupied as land, as references to any person for the time being present at the place in question.

\amendment{

S. 15(1)--(4A) substituted for s. 15(1)--(4), words in s. 15(6) substituted and s. 15(11) inserted (31.1.01) by the Child Support,
Pensions and Social Security Act 2000 (c. 19) s. 14.

Defn. of “specified” in s. 15(10) deleted (2.4.01) by the Child Support, Pensions and Social
Security Act 2000 (c. 19) Sch. 9 Pt. I.

Words inserted in s. 15(7) (5.12.05) by the Civil Partnership Act 2004 (c. 33) Sch. 24 para. 2.

Words substituted in s. 15(1), (2) (1.8.12) by the Public Bodies (Child Maintenance and
Enforcement Commission: Abolition and Transfer of Functions) Order 2012 Sch. para. 10.
	

}

\section{Reviews and appeals}

\subsection{16. Revision of decisions}

(1) Any decision \opt{oldrules}{of the Secretary of State under section 11, 12 or 17 }\opt{newrules,2012rules}{to which subsection (1A) applies }may be revised by the Secretary of State---
\begin{enumerate}\item[]
($a$)
either within the prescribed period or in prescribed cases or circumstances; and

($b$)
either on an application made for the purpose or on the Secretary of State’s own initiative;
\end{enumerate}
and regulations may prescribe the procedure by which a decision of the Secretary of State may be so revised.

\opt{newrules,2012rules}{
(1A) This subsection applies to---
\begin{enumerate}\item[]
($a$) a decision of the Secretary of State under section 11, 12 or 17;

($c$) a decision of the First-tier tribunal on a referral under section 28D(1)($b$).
\end{enumerate}

(1B) Where the Secretary of State revises a decision under section 12(1)---
\begin{enumerate}\item[]
($a$) the Secretary of State may (if appropriate) do so as if revising a decision under section 11; and

($b$) if the Secretary of State does that, the decision as revised is to be treated as one under section 11 instead of section 12(1) (and, in particular, is to be so treated for the purposes of an appeal against it under section 20).
\end{enumerate}
}

(2)
In making a decision under subsection (1), the Secretary of State need not consider any issue that is not raised by the application or, as the case may be, did not cause the Secretary of State to act on the Secretary of State’s own initiative.

(3)
Subject to subsections (4) and (5) and section 28ZC, a revision under this section shall take effect as from the date on which the original decision took (or was to take) effect.

(4)
Regulations may provide that, in prescribed cases or circumstances, a revision under this section shall take effect as from such other date as may be prescribed.

(5)
Where a decision is revised under this section, for the purpose of any rule as to the time allowed for bringing an appeal, the decision shall be regarded as made on the date on which it is so revised.

(6)
Except in prescribed circumstances, an appeal against a decision of the Secretary of State shall lapse if the decision is revised under this section before the appeal is determined.

\amendment{

S. 16 substituted (16.11.98 for the purposes of authorising the making of regulations, 7.12.98 for all other purposes, subject to transitional provisions in S.I. 1998/2780 reg. 3) by the Social Security 1998 (c. 14) s. 40.  

%\opt{oldrules}{Under S.I. 1998/2780 reg. 3(4) any maintenance assessment which takes place on or before 8.12.96 under the previous version of s. 16:
%
%\begin{quotation}
%\subsection*{\itshape 16. Periodical reviews}
%
%(1) The Secretary of State shall make such arrangements as he considers necessary to secure that, where any maintenance assessment has been in force for a prescribed period, the amount of child support maintenance fixed by that assessment (“the original assessment”) is reviewed by a child support officer under this section as soon as is reasonably practicable after the end of that prescribed period.
%
%(2)
%Before conducting any review under this section, the child support officer concerned shall give, to such persons as may be prescribed, such notice of the proposed review as may be prescribed.
%
%(3)
%A review shall be conducted under this section as if a fresh application for a maintenance assessment had been made by the person in whose favour the original assessment was made.
%
%(4)
%On completing any review under this section, the child support officer concerned shall make a fresh maintenance assessment, unless he is satisfied that the original assessment has ceased to have effect or should be brought to an end.
%
%(5)
%Where a fresh maintenance assessment is made under subsection (4), it shall take effect---
%\begin{enumerate}\item[]
%($a$) on the day immediately after the end of the prescribed period mentioned in subsection (1); or
%
%($b$) in such circumstances as may be prescribed, on such later date as may be determined in accordance with regulations made by the Secretary of State.
%\end{enumerate}
%
%(6) The Secretary of State may by regulations prescribe circumstances (for example, where the maintenance assessment is about to terminate) in which a child support officer may decide not to conduct a review under this section.
%\end{quotation}}

\opt{newrules,2012rules}{Words in s. 16(1) substituted and s. 16(1A), (1B) inserted (3.3.03) for the purposes of certain cases only (see S.I. 2003/192) by the Child Support, Pensions and Social Security Act 2000 (c. 19) s. 8.

S. 16(1A)($b$) repealed (14.7.08) by the Child Maintenance and Other Payments Act 2008 (c.
6) Sch. 8.

Words substituted in s. 16(1A)($c$) (3.11.08) by the Transfer of Tribunal Functions Order 2008 Sch. 3 para. 78.
}

Words substituted and omitted in s. 16 (1.8.12) by the Public Bodies (Child Maintenance and Enforcement
Commission: Abolition and Transfer of Functions) Order 2012 Sch. para. 11.

}

\subsection{17. Decisions superseding earlier decisions}

(1) Subject to subsection (2), the following, namely---
\begin{enumerate}\item[]
($a$) any decision of the Secretary of State under section 11 or 12 or this section,
whether as originally made or as revised under section 16;

($b$) any decision of 
an appeal tribunal or  % Words inserted (3.11.08) by 2012 c 6 Sch 12 para 2(2)(a)
the First-tier Tribunal under section 20;

($d$) any decision of 
an appeal tribunal or  % Words inserted (3.11.08) by 2012 c 6 Sch 12 para 2(2)(a)
the First-tier Tribunal on a referral under section~28D(1)($b$);

($e$) any decision of 
a Child Support Commissioner or  % Words inserted (3.11.08) by 2012 c 6 Sch 12 para 2(2)(b)
the Upper Tribunal on an appeal from such a decision as
is mentioned in paragraph ($b$) or ($d$),
\end{enumerate}
may be superseded by a decision made by the Secretary of State, either on an application made for the purpose or on the Secretary of State’s own initiative.

\opt{oldrules,newrules}{
(2) In making a decision under subsection (1), the Secretary of State need not consider any issue that is not raised by the application or, as the case may be, did not cause him to act on his own initiative.

(3) Regulations may prescribe the cases and circumstances in which, and the procedure by which, a decision may be made under this section.
}

\opt{2012rules}{
(2) The Secretary of State may by regulations make provision with respect to the exercise of the power under subsection (1).

(3) Regulations under subsection (2) may, in particular—
\begin{enumerate}\item[]
($a$) make provision about the cases and circumstances in which the power under subsection (1) is exercisable, including provision restricting the exercise of that power by virtue of change of circumstance;

($b$) make provision with respect to the consideration by the Secretary of State, when acting under subsection (1), of any issue which has not led to the Secretary of State's so acting;

($c$) make provision with respect to procedure in relation to the exercise of the power under subsection (1).
\end{enumerate}
}

\opt{oldrules}{(4) Subject to subsection (5) and section 28ZC, a decision under this section shall take effect as from the date on which it is made or, where applicable, the date on which the application was made.}

\opt{newrules,2012rules}{(4) Subject to subsection (5) and section 28ZC, a decision under this section shall take effect as from the beginning of the maintenance period in which it is made or, where applicable, the beginning of the maintenance period in which the application was made.

(4A) In subsection (4), a “maintenance period” is (except where a different meaning is prescribed for prescribed cases) a period of seven days, the first one beginning on the effective date of the first decision made by the Secretary of State under section 11 or (if earlier) the Secretary of State’s first default or interim
maintenance decision (under section 12) in relation to the non-resident parent in question, and each subsequent one beginning on the day after the last day of the previous one.}

(5) Regulations may provide that, in prescribed cases or circumstances, a decision under this section shall take effect as from such other date as may be prescribed.

% S 17(6) inserted (3.11.08) by 2012 c 6 Sch 12 para 2(3)
(6) In this section—
\begin{enumerate}\item[]
    “appeal tribunal” means an appeal tribunal constituted under Chapter I of Part I of the Social Security Act 1998 (the functions of which have been transferred to the First-tier Tribunal);

    “Child Support Commissioner” means a person appointed as such under section 22 (the functions of whom have been transferred to the Upper Tribunal).
\end{enumerate}

\amendment{
S. 17 substituted for ss. 17--19 (4.3.99 for certain purposes, 1.6.99 for all other purposes) by the Social Security Act 1998 (c. 14) s. 41 (subject to transitional provisions in S.I. 1999/1510 arts. 48, 49).

\opt{newrules,2012rules}{
S. 17(1)($c$)--($e$), (4), (4A) substituted (10.11.00 for the purpose of making regulations and Acts of Sederunt only, 3.3.03) for the purposes of certain cases only (see S.I. 2003/192) by the Child Support, Pensions and Social Security Act 2000 (c. 19) s. 9.

Word “and” repealed in s. 17(1)($b$) (3.3.03) for the purposes of certain cases only (see S.I. 2003/192) by the Child Support, Pensions and Social Security Act 2000 (c. 19) Sch. 9 Pt. I.
}

Words substituted in s. 17(1) (3.11.08) by the Transfer of Tribunal Functions Order 2008 Sch. 3 para. 79.

Words inserted in s. 17(1) and s. 17(6) inserted (3.11.08) by the Welfare Reform Act 2012 Sch. 12 para. 2.

S. 17(1)($c$) repealed (14.7.08) by the Child Maintenance and Other Payments Act 2008
(c. 6) Sch. 8.

Words substituted in s. 17 (1.8.12) by the Public Bodies (Child Maintenance and Enforcement Commission: Abolition and Transfer of Functions) Order 2012 Sch. para. 12.

\opt{2012rules}{
Words substituted in s. 17(2) and (3) (10.12.12) for the purposes of certain cases only (see S.I. 2012/3042) by the Child Maintenance and Other Payments Act 2008 (c. 6) s. 17 as amended by the Public Bodies (Child Maintenance and Enforcement Commission: Abolition and Transfer of Functions) Order 2012 Sch. para. 77.
}

}

\opt{oldrules}{

\subsection{20. Appeals to First-tier Tribunal}

(1) Where an application for a maintenance assessment is refused, the person who made that application shall have a right of appeal to the First-tier Tribunal against the refusal.

(2) Where a maintenance assessment is in force---
\begin{enumerate}\item[]
 ($a$) the absent parent or person with care with respect to whom it was made; or

 ($b$) where the application for the assessment was made under section 7, either of them or the child concerned, 
\end{enumerate}
shall have a right of appeal to the First-tier Tribunal against the amount of the assessment or the date from which the assessment takes effect.

(3) Where a maintenance assessment is cancelled, or an application for the cancellation of a maintenance assessment is refused---
\begin{enumerate}\item[]
($a$) the absent parent or person with care with respect to whom the maintenance assessment in question was, or remains, in force; or 

($b$) where the application for that assessment was made under section 7, either of them or the child concerned, 
\end{enumerate}
shall have a right of appeal to the First-tier Tribunal against the cancellation or refusal.

(3A) Regulations may provide that, in such cases or circumstances as may be prescribed, there is a right of appeal against a decision only if the Secretary of State has considered whether to revise the decision under section 16.

(3B) The regulations may in particular provide that that condition is met only where—
\begin{enumerate}\item[]
($a$) the consideration by the Secretary of State was on an application,

($b$) the Secretary of State considered issues of a specified description, or

($c$) the consideration by the Secretary of State satisfied any other condition specified in the regulations.
\end{enumerate}

(4) A person with a right of appeal under this section shall be given such notice of that right and, in the case of a right conferred by subsection (1) or (3), such notice of the decision as may be prescribed.

(5) Regulations may make---
\begin{enumerate}\item[]
($a$) provision as to the manner in which, and the time within which, appeals are to be brought; and

($c$) provision that, where in accordance with regulations under subsection (3A) there is no right of appeal against a decision, any purported appeal may be treated as an application for revision under section 16.
\end{enumerate}

(6)
The regulations may in particular make any provision of a kind mentioned in Schedule 5 to the Social Security Act 1998.

(7)
In deciding an appeal under this section, the First-tier Tribunal---
\begin{enumerate}\item[]
($a$) need not consider any issue that is not raised by the appeal; and

($b$) shall not take into account any circumstances not obtaining at the time when the decision or assessment appealed against was made.
\end{enumerate}


\amendment{

S. 20 substituted for ss. 20--21 (1.6.99) by the Social Security Act 1998 (c. 14) s. 42.

Words substituted in s. 20(1)--(3), (7) and heading and s. 20(5)($b$) omitted (3.11.08) by the Transfer of Tribunal Functions Order 2008 Sch. 3 para. 80. 

See art. 3 of S.I 2002/1915 for details of when an appeal under s. 21 is made to a court instead of an appeal tribunal.

S. 20(3A), (3B), (5)($c$) inserted (25.2.13 for the purposes of making regulations, 29.4.13 for all other purposes) by the Welfare Reform Act 2012 (c. 5) Sch. 11 para. 6 as amended by the Public Bodies (Child Maintenance and Enforcement Commission: Abolition and Transfer of Functions) Order 2012 Sch. para. 108(3).

The insertion of s. 20(2)($aa$) by the Child Maintenance and Other Payments Act 2008 (c. 6) Sch. 7 para. 1(4) is not yet in force.

The insertion of s. 20(5A) by the Child Maintenance and Other Payments Act 2008 (c. 6) Sch. 7 para. 1(5) is not yet in force.

The insertion of s. 20(7A) by the Child Maintenance and Other Payments Act 2008 (c. 6) Sch. 7 para. 1(6) is not yet in force.

\medskip

S. 22 omitted (3.11.08) by the Transfer of Tribunal Functions Order 2008 Sch. 3 para. 82.

}
}

\opt{newrules,2012rules}{

\subsection{20. Appeals to First-tier Tribunal}

(1) A qualifying person has a right of appeal to the First-tier Tribunal against---
\begin{enumerate}\item[]
($a$) a decision of the Secretary of State under section 11, 12 or 17 (whether as originally made or as revised under section 16);

($b$) a decision of the Secretary of State not to make a maintenance calculation under section 11 or not to supersede a decision under section 17;

($d$) the imposition (by virtue of section 41A) of a requirement to make penalty payments, or their amount.
\end{enumerate}

(2)
In subsection (1), “qualifying person” means---
\begin{enumerate}\item[]
($a$)
in relation to paragraphs ($a$) and ($b$)---
\begin{enumerate}\item[]
(i)
the person with care, or non-resident parent, with respect to whom the Secretary of State made the decision, or

(ii)
in a case relating to a maintenance calculation which was applied for under section 7, either of those persons or the child concerned;
\end{enumerate}

($c$) in relation to paragraph ($d$), the parent who has been required to make penalty payments; and

($d$) in relation to paragraph ($e$), the person required to pay fees.
\end{enumerate}

(2A) Regulations may provide that, in such cases or circumstances as may be prescribed, there is a right of appeal against a decision mentioned in subsection (1)($a$) or ($b$) only if the Secretary of State has considered whether to revise the decision under section 16.

(2B) The regulations may in particular provide that that condition is met only where—
\begin{enumerate}\item[]
($a$) the consideration by the Secretary of State was on an application,

($b$) the Secretary of State considered issues of a specified description, or

($c$) the consideration by the Secretary of State satisfied any other condition specified in the regulations.
\end{enumerate}

(3) A person with a right of appeal under this section shall be given such notice as may be prescribed of---
\begin{enumerate}\item[]
($a$) that right; and

($b$) the relevant decision, or the imposition of the requirement.
\end{enumerate}

(4) Regulations may make---
\begin{enumerate}\item[]
($a$) provision as to the manner in which, and the time within which, appeals are to be brought; and

($c$) provision that, where in accordance with regulations under subsection (2A) there is no right of appeal against a decision, any purported appeal may be treated as an application for revision under section 16.
\end{enumerate}

(5)
The regulations may in particular make any provision of a kind mentioned in Schedule 5 to the Social Security Act 1998.

(7) In deciding an appeal under this section, the First-tier Tribunal---
\begin{enumerate}\item[]
($a$) need not consider any issue that is not raised by the appeal; and

($b$) shall not take into account any circumstances not obtaining at the time when the Secretary of State made the decision or imposed the requirement.
\end{enumerate}

(8) If an appeal under this section is allowed, the First-tier Tribunal may---
\begin{enumerate}\item[]
($a$) itself make such decision as it considers appropriate; or

($b$) remit the case to the Secretary of State, together with such directions (if any) as it considers appropriate.
\end{enumerate}


\amendment{

%S. 20 substituted for ss. 20--21 (1.6.99) by the Social Security Act 1998 (c. 14) s. 42.

S. 20 substituted (10.11.00 for the purpose of making regulations and Acts of Sederunt only, 3.3.03) for the purposes of certain cases only (see S.I. 2003/192) by the Child Support, Pensions and Social Security Act 2000 (c. 19) s. 10. 

Words substituted in s. 20(1), (7), (8) and heading and s. 20(4)($b$) omitted (3.11.08) by the Transfer of Tribunal Functions Order 2008 Sch. 3 para. 81. 

Words substituted in s. 20 (1.8.12) by the Public Bodies (Child Maintenance and Enforcement Commission: Abolition and Transfer of Functions) Order 2012 Sch. para.
13.

S. 20(1)($c$), ($e$), (2)($b$), (6) repealed (14.7.08) by the Child Maintenance and Other Payments
Act 2008 (c. 6) Sch. 8. 

S. 20(2A), (2B), (4)($c$) inserted (25.2.13 for the purposes of making regulations, 29.4.13 for all other purposes) by the Welfare Reform Act 2012 (c. 5) Sch. 11 para. 5 as amended by the Public Bodies (Child Maintenance and Enforcement Commission: Abolition and Transfer of Functions) Order 2012 Sch. para. 108(2).

The insertion of s. 20(2)($aa$) by the Child Maintenance and Other Payments Act 2008 (c. 6) Sch. 7 para. 1(4) is not yet in force.

The insertion of s. 20(5A) by the Child Maintenance and Other Payments Act 2008 (c. 6) Sch. 7 para. 1(5) is not yet in force.

The insertion of s. 20(7A) by the Child Maintenance and Other Payments Act 2008 (c. 6) Sch. 7 para. 1(6) is not yet in force.

\medskip

S. 22 omitted (3.11.08) by the Transfer of Tribunal Functions Order 2008 Sch. 3 para. 82.

}
}

\subsection{23. Child Support Commissioners for Northern Ireland}

(1) Her Majesty may from time to time appoint a Chief Child Support Commissioner for Northern Ireland and other Child Support Commissioners for Northern Ireland.

(2)
The Chief Child Support Commissioner for Northern Ireland and the other Child Support Commissioners for Northern Ireland shall be appointed from among persons who are barristers or solicitors of not less than 7 years’ standing.

(3)
Schedule 4 shall have effect with respect to Child Support Commissioners for Northern Ireland.

%(4)
%Subject to any Order made after the passing of this Act by virtue of subsection (1)($a$) of section 3 of the Northern Ireland Constitution Act 1973, the matters to which this subsection applies shall not be transferred matters for the purposes of that Act but shall for the purposes of subsection (2) of that section be treated as specified in Schedule 3 to that Act.
%
%(5)
%Subsection (4) applies to all matters relating to Child Support Commissioners, including procedure and appeals, other than those specified in paragraph 9 of Schedule 2 to the Northern Ireland Constitution Act 1973.

\amendment{

S. 23(4), (5) repealed (2.12.99) by the Northern Ireland Act 1998 (c. 47) Sch. 15.

‘7’ substituted for ‘10’ in s. 23(2) (21.7.08) by the Tribunals, Courts and Enforcement Act
2007 (c. 5) Sch. 10 para. 22(3).

Words in s. 23(3) omitted (3.11.08) by the Transfer of Tribunal Functions Order 2008 Sch. 3 para. 83.

Words substituted in s. 23(1) (12.3.09) by the Northern Ireland Act 2009 (c. 3) Sch. 4 para. 22. 

The amendment of s. 23(1) by the Justice (Northern Ireland) Act 2002 (c. 26) is not yet in force.

}

\subsection{23A. Redetermination of appeals}

(1) This section applies where an application is made to the First-tier Tribunal for permission to appeal to the Upper Tribunal from any decision of the First-tier Tribunal under section 20.

(3) If each of the principal parties to the case expresses the view that the decision was erroneous in the point of law, the First-tier Tribunal shall set aside the decision and refer the case for determination by a differently constituted First-tier Tribunal.

(4) The “principal parties” are---
\begin{enumerate}\item[]
($a$) the Secretary of State; and

($b$) those who are qualifying persons for the purpose of section 20(2) in relation
to the decision in question.
\end{enumerate}

\amendment{

S. 23A inserted (15.2.01) by the Child Support, Pensions and Social Security Act
2000 (c. 19) s. 11.

Words substituted in s. 23A(1), (3) and s. 23A(2) omitted (3.11.08) by the Transfer of Tribunal Functions Order 2008 Sch. 3 para. 84.

S. 23A(4)($za$) omitted (1.8.12) by the Public Bodies (Child Maintenance and Enforcement Commission: Abolition and Transfer of Functions) Order 2012 Sch. para. 14.

}

\subsection{24. Appeals to Upper Tribunal}

(1) Each of the following may appeal to the Upper Tribunal under section
11 of the Tribunals, Courts and Enforcement Act 2007 from any decision of the First-tier Tribunal under section 20 of this Act---
\begin{enumerate}\item[]
($b$) the Secretary of State, and

($c$) any person who is aggrieved by the decision of an appeal tribunal.
\end{enumerate}

(2) Where a question which would otherwise fall to be determined by the Secretary of State under this Act first arises in the course of an
appeal to the Upper Tribunal, that tribunal may, if it thinks fit, determine the question even though it has not been considered by the Secretary of State.

\amendment{

S. 24(1A) repealed (1.6.99) by the Social Security Act 1998 (c. 14) Sch. 8. 

S. 24(1) substituted (1.11.08) by the Child Maintenance and Other
Payments Act 2008 (c. 6) Sch. 3 para. 16.

Words in s. 24(1) substituted (3.11.08) by the Transfer of Tribunal Functions Order 2008 Sch. 3 para. 85($c$). 

S. 24(2) substituted for s. 24(2)--(9) (3.11.08) by the Transfer of Tribunal Functions Order 2008 Sch. 3 para. 85($d$). 

S. 24(1)($a$) and words in s. 24(2) omitted (1.8.12) by the Public Bodies (Child Maintenance and Enforcement
Commission: Abolition and Transfer of Functions) Order 2012, Sch. para. 15. 

\medskip

S. 25 omitted (3.11.08) by the Transfer of Tribunal Functions Order 2008 Sch. 3 para. 86.

}

\subsection{26. Disputes about parentage}

(1) Where a person who is alleged to be a parent of the child with respect to whom an application for a \opt{oldrules}{maintenance assessment}\opt{newrules,2012rules}{maintenance calculation} has
been made (“the alleged parent”) denies that he is one of the child’s parents,
the Secretary of State shall not make a \opt{oldrules}{maintenance assessment}\opt{newrules,2012rules}{maintenance calculation} on the assumption that the alleged parent is one of the
child’s parents unless the case falls within one of those set out in subsection (2).

(2) The Cases are---

\subsubsection*{Case A1}

Where---
\begin{enumerate}\item[]
($a$) the child is habitually resident in England and Wales;

($b$) the Secretary of State is satisfied that the alleged parent was married to
the child's mother at some time in the period beginning with the conception
and ending with the birth of the child; and

($c$) the child has not been adopted.
\end{enumerate}

\subsubsection*{Case A2}

Where---
\begin{enumerate}\item[]
($a$) the child is habitually resident in England and Wales;

($b$)
the alleged parent has been registered as father of the child under section 10 or 10A of the Births and Deaths Registration Act 1953, or in any register kept under section 13 (register of births and stillbirths) or section 44 (register of corrections etc) of the Registration of Births, Deaths and Marriages (Scotland) Act 1965, or under Article 14 or 18(1)($b$)(ii) of the Births and Deaths Registration (Northern Ireland) Order 1976; and

($c$)
the child has not subsequently been adopted.
\end{enumerate}

\subsubsection*{Case A3}

Where the result of a scientific test (within the meaning of section 27A) taken by the alleged parent would be relevant to determining the child’s parentage, and the alleged parent---
\begin{enumerate}\item[]
($a$)
refuses to take such a test; or

($b$)
has submitted to such a test, and it shows that there is no reasonable doubt that the alleged parent is a parent of the child.
\end{enumerate}

\subsubsection*{Case A}

Where the alleged parent is a parent of the child in question by virtue of having adopted him.

\subsubsection*{Case B}

Where the alleged parent is a parent of the child in question by virtue of an order under section 30 of the Human Fertilisation and Embryology Act 1990 or section 54 of the Human Fertilisation and Embryology Act 2008 (parental orders).

\subsubsection*{Case B1}

Where the Secretary of State is satisfied that the alleged parent is a parent of a child in question by virtue of section 27 or 28 of the Human Fertilisation and Embryology Act 1990 or any of sections 33 to 46 of the Human Fertilisation and Embryology Act 2008 (which relate to children resulting from assisted reproduction).

\subsubsection*{Case C}

Where---
\begin{enumerate}\item[]
($a$)
either---
\begin{enumerate}\item[]
 (i) a declaration that the alleged parent is a parent of the child in question (or a declaration which has that effect) is in force under section 55A or 56 of the Family Law Act 1986 or Article 32 of the Matrimonial and Family Proceedings (Northern Ireland) Order 1989 (declarations of parentage); or 

(ii) a declarator by a court in Scotland that the alleged parent is a parent of the child in question (or a declarator which has that effect) is in force; and
\end{enumerate}
($b$)
the child has not subsequently been adopted.
\end{enumerate}

%\subsubsection*{Case D}
%
%Where---
%\begin{enumerate}\item[]
%($a$) 
%a declaration to the effect that the alleged parent is one of the parents of the child in question has been made under section 27; and 
%
%($b$) the child has not subsequently been adopted.
%\end{enumerate}

\subsubsection*{Case E}

Where---
\begin{enumerate}\item[]
($a$) 
the child is habitually resident in Scotland; 

($b$) the Secretary of State is satisfied that one or other of the presumptions set out in section 5(1) of the Law Reform (Parent and Child) (Scotland) Act 1986 applies; and

($c$) the child has not subsequently been adopted.
\end{enumerate}

\subsubsection*{Case F}

Where---
\begin{enumerate}\item[]
($a$) the alleged parent has been found, or adjudged, to be the father of the child in question---
\begin{enumerate}\item[]
(i)
in proceedings before any court in England and Wales which are relevant proceedings for the purposes of section 12 of the Civil Evidence Act 1968 or in proceedings before any court in Northern Ireland which are relevant proceedings for the purposes of section 8 of the Civil Evidence Act (Northern Ireland) 1971; or

(ii)
in affiliation proceedings before any court in the United Kingdom (whether or not he offered any defence to the allegation of paternity) and that finding or adjudication still subsists; and
\end{enumerate}

($b$) the child has not subsequently been adopted.
\end{enumerate}

(3) In this section---
\begin{enumerate}\item[]
 “adopted” means adopted within the meaning of Part IV of the Adoption Act 1976 or Chapter IV of Part I of the Adoption and Children Act 2002 or, in relation to Scotland, Part IV of the Adoption (Scotland) Act 1978 or Chapter III of Part I of the Adoption and Children (Scotland) Act 2007; and 

“affiliation proceedings”, in relation to Scotland, means any action of affiliation and aliment.
\end{enumerate}


\amendment{

Words inserted in Cases C and F (4.11.96) by the Children (Northern Ireland Consequential Amendments) Order 1995 art. 13.

Words in s. 26 substituted (1.6.99) by the Social Security Act 1998 (c. 14) Sch. 7 para. 31. 

Cases A1--A3 inserted (31.1.01) by the Child Support, Pensions and Social Security Act 2000
(c. 19) s. 15.

Words in Case C inserted (1.4.01) by the Child Support, Pensions and Social Security Act 2000 (c. 19) Sch. 8 para. 12.

Case D repealed (1.4.01) by the Child Support, Pensions and Social Security Act 2000 (c. 19) Sch. 9 Pt. IX.

\opt{newrules,2012rules}{

Words “maintenance assessment(s)” substituted by “maintenance calculation(s)” (3.3.03) for
the purposes of certain cases only (see S.I. 2003/192) by the Child Support, Pensions and Social Security Act 2000 (c. 19) s. 1(2)($a$).

}

Words inserted in s. 26(3) (30.12.05) by the Adoption and Children Act 2002 (c. 38) Sch. 3 para. 81.

Words repealed in s. 26(1) (27.10.08) by the Child Maintenance and Other Payments Act 2008
(c. 6) Sch. 8. 

Cases B and B1 substituted (6.4.09) by the Human Fertilisation and Embryology Act 2008 (c. 22) Sch. 6 para. 36.

Words inserted in s. 26(3) (28.9.09) by the Adoption and Children (Scotland) Act 2007 (asp 4) Sch. 2 para. 7.

Words substituted in s. 26 (1.8.12) by the Public Bodies (Child Maintenance and Enforcement Commission: Abolition and Transfer of Functions) order 2012 Sch. para. 16.

The amendment of s. 26(2) by the Welfare Reform Act 2009 (c. 24) Sch. 6 para. 23 is not yet in force.

}

\subsection{27. Applications for declaration of parentage under Family Law Act 1986}

(1) This section applies where---
\begin{enumerate}\item[]
($a$) an application for a maintenance calculation has been made, or a maintenance calculation is in force, with respect to a person (“the alleged parent”) who denies that he is a parent of a child with respect to whom the application or calculation was made;

($b$) the Secretary of State is not satisfied that the case falls within one of those set out in section 26(2); and

($c$) the Secretary of State or the person with care makes an application for the declaration under section 55A of the Family Law Act 1986 as to whether or not the alleged parent is one of the child’s parents.
\end{enumerate}

(2) Where this section applies---
\begin{enumerate}\item[]
($a$) if it is the person with care who makes the application, she shall be treated as having a sufficient personal interest for the purposes of subsection (3) of that section; and

($b$) if it is the Secretary of State who makes the application, that subsection shall not apply.
\end{enumerate}

(3) This section does not apply to Scotland.

\amendment{

S. 27 substituted (1.4.01) by the Child Support, Pensions and Social Security Act 2000 (c. 19) Sch. 8 para. 13.

Words repealed in s. 27(1)($a$) (27.10.08) by the Child Maintenance and Other Payments Act 2008 (c. 6) Sch. 8.

Words substituted in s. 27 (1.8.12) by the Public Bodies (Child Maintenance and Enforcement Commission: Abolition and Transfer of Functions) Order 2012 Sch. para. 17.
}

\subsection{27A. Recovery of fees for scientific tests}

(1) This section applies in any case where---
\begin{enumerate}\item[]
($a$) an application for a \opt{oldrules}{maintenance assessment}\opt{newrules,2012rules}{maintenance calculation} has been made or a \opt{oldrules}{maintenance assessment}\opt{newrules,2012rules}{maintenance calculation} is in force;

($b$)
scientific tests have been carried out (otherwise than under a direction or in response to a request) in relation to bodily samples obtained from a person who is alleged to be a parent of a child with respect to whom the application or \opt{oldrules}{assessment}\opt{newrules,2012rules}{calculation} is made;

($c$)
the results of the tests do not exclude the alleged parent from being one of the child’s parents; and

($d$)
one of the conditions set out in subsection (2) is satisfied.
\end{enumerate}

(2) The conditions are that---
\begin{enumerate}\item[]
($a$)
the alleged parent does not deny that he is one of the child’s parents;

($b$)
in proceedings under section 55A of the Family Law Act 1986, a court has made a declaration that the alleged parent is a parent of the child in question;
or

($c$)
in an action under section 7 of the Law Reform (Parent and Child) (Scotland) Act 1986, brought by the Secretary of State by virtue of section 28, a court has granted a decree of declarator of parentage to the effect that the alleged parent is a parent of the child in question.
\end{enumerate}

(3) In any case to which this section applies, any fee paid by the Secretary of State in connection with scientific tests may be recovered by the Secretary of State from the alleged parent as a debt due to the Crown.

(4) In this section---
\begin{enumerate}\item[]
 “bodily sample” means a sample of bodily fluid or bodily tissue taken for the purpose of scientific tests; 

“direction” means a direction given by a court under section 20 of the Family Law Reform Act 1969 (tests to determine paternity); 

“request” means a request made by a court under section 70 of the Law Reform (Miscellaneous Provisions) (Scotland) Act 1990 (blood and other samples in civil proceedings); and 

“scientific tests” means scientific tests made with the object of ascertaining the inheritable characteristics of bodily fluids or bodily tissue.
\end{enumerate}

(5) Any sum recovered by the Secretary of State under this section shall be paid by the Secretary of State into the Consolidated Fund.

\amendment{

S. 27A inserted (4.9.95) by the Child Support Act 1995 (c. 34) s. 21.

Words substituted (1.4.01) in s. 27A(2)($b$) by the Child Support, Pensions and Social Security Act 2000 (c. 19) Sch. 8 para. 14.

\opt{newrules,2012rules}{Words “maintenance assessment(s)” substituted by “maintenance calculation(s)” (3.3.03) for the purposes of certain cases only (see S.I. 2003/192) by the Child Support, Pensions and Social Security Act 2000 (c. 19) s. 1(2)($a$).

Word “assessment” (or any variant of that term) substituted by “calculation” (or other variants)
(3.3.03) for the purposes of certain cases only (see S.I. 2003/192) by the Child Support, Pensions and Social Security Act 2000 (c. 19) s. 1(2)($b$).}

Words repealed in s. 27A(1)($a$), ($b$) (27.10.08) by the Child Maintenance and Other Payments Act 2008 (c. 6) Sch. 8.

Words substituted in s. 27A (1.8.12) by the Public Bodies (Child Maintenance and Enforcement Commission: Abolition and Transfer of Functions) Order 2012 Sch. para. 18.

}

\subsection{28. Power of Secretary of State to initiate or defend actions of declarator: Scotland}

(1) Subsection (1A) applies in any case where---
\begin{enumerate}\item[]
($a$) an application for a \opt{oldrules}{maintenance assessment}\opt{newrules,2012rules}{maintenance calculation} has been made, or a \opt{oldrules}{maintenance assessment}\opt{newrules,2012rules}{maintenance calculation} is in force, with respect to a person (“the alleged parent”) who denies that he is a parent of a child with respect to whom the application \opt{oldrules}{or assessment was made}\opt{newrules,2012rules}{was made or the calculation was made}; and

($b$) the Secretary of State is not satisfied that the case falls within one of those set out in section 26(2).
\end{enumerate}

(1A) In any case where this subsection applies, the Secretary of State may bring an action for declarator of parentage under section 7 of the Law Reform (Parent and Child) (Scotland) Act 1986.

(2) The Secretary of State may defend an action for declarator of non-parentage or illegitimacy brought by a person named as the alleged parent in an application for a \opt{oldrules}{maintenance assessment}\opt{newrules,2012rules}{maintenance calculation} or in a \opt{oldrules}{maintenance assessment}\opt{newrules,2012rules}{maintenance calculation} which is in force.

(3) This section applies to Scotland only.

\amendment{
S. 28(1), (1A) substituted (4.9.95) for s. 28(1) by the Child Support Act 1995 (c. 34) s. 20(6).

Words substituted in s. 28(1)($b$) (5.7.99) by the Social Security Act 1998 (c. 14) Sch. 7, para. 33.

\opt{newrules,2012rules}{
Words “maintenance assessment(s)” substituted by “maintenance calculation(s)” (3.3.03) for the purposes of certain cases only (see S.I. 2003/192) by the Child Support, Pensions and Social Security Act 2000 (c. 19) s. 1(2)($a$).

Words in s. 28(1)($a$) substituted for words “or assessment was made” (3.3.03) for the purposes of certain cases only (see S.I. 2003/192) by the Child Support, Pensions and Social Security Act 2000 (c. 19) Sch. 3 para. 11(10)($b$).

}

Words repealed in s. 28(1)($a$) (27.10.08) by the Child Maintenance and Other Payments Act 2008
(c. 6) Sch. 8.

Words substituted in s. 28 and the title (1.8.12) by the Public Bodies (Child Maintenance and Enforcement Commission: Abolition and Transfer of Functions) Order 2012, para. 19.

}

\section{Decisions and appeals dependent on other cases}

\subsection{28ZA. Decisions involving issues that arise on appeal in other cases}

(1) This section applies where---
\begin{enumerate}\item[]
($a$) a decision by the Secretary of State falls to be made under section 11, 12, 16 or 17\opt{oldrules}{ in relation to a maintenance assessment}; and

\opt{oldrules}{($b$) an appeal is pending against a decision given in relation to a different maintenance assessment by the Upper Tribunal or a court.}

\opt{newrules,2012rules}{($b$) an appeal is pending against a decision given in relation to a different matter by the Upper Tribunal or a court.}
\end{enumerate}

(2) If the Secretary of State considers it possible that the result of the appeal will be such that, if it were already determined, it would affect the decision in some way---
\begin{enumerate}\item[]
($a$) the Secretary of State need not, except in such cases or circumstances as may be prescribed, make the decision while the appeal is pending;

($b$) the Secretary of State may, in such cases or circumstances as may be prescribed, make the decision on such basis as may be prescribed.
\end{enumerate}

(3) Where the Secretary of State acts in accordance with subsection (2)($b$), following the determination of the appeal the Secretary of State shall if appropriate revise the decision (under section 16) in accordance with that determination.

(4) For the purposes of this section, an appeal against a decision is pending if---
\begin{enumerate}\item[]
($a$) an appeal against the decision has been brought but not determined;

($b$) an application for leave to appeal against the decision has been made but not determined; or

($c$) in such circumstances as may be prescribed, an appeal against the decision has not been brought (or, as the case may be, an application for leave to appeal against the decision has not been made) but the time for doing so has not yet expired.
\end{enumerate}

(5) In paragraphs ($a$), ($b$), and ($c$) of subsection (4), any reference to an appeal, or an application for leave to appeal, against a decision includes a reference to---
\begin{enumerate}\item[]
($a$) an application for, or for leave to apply for, judicial review of the decision under section 31 of the Senior Courts Act 1981; or

($b$) an application to the supervisory jurisdiction of the Court of Session in respect of the decision.
\end{enumerate}

\amendment{
S. 28ZA inserted (1.6.99) by the Social Security Act 1998 (c. 14) s. 43.

\opt{newrules,2012rules}{S. 28ZA(1)($b$) substituted (3.3.03) for the purpose of certain cases only (see S.I. 2003/192) by the Child Support, Pensions and Social Security Act 2000 (c. 19) Sch. 3 para. 11(11).

Words repealed in s. 28ZA(1)($a$) (14.7.08) by the Child Maintenance and Other Payments Act 2008 (c. 6) Sch. 8.}

Words in s. 28ZA(1)($b$) substituted (3.11.08) by the Transfer of Tribunal Functions Order 2008 Sch. 3 para. 87.

Reference to Supreme Court Act 1981 in s. 28ZA(5)($a$) substituted (1.10.09) by Constitutional Reform
Act 2005 (c. 4) Sch. 11 para 1(2).

Words substituted in s. 28ZA (1.8.12) by the Public Bodies (Child Maintenance and Enforcement Commission: Abolition and Transfer of Functions) Order 2012 Sch. para. 20.

}

\subsection{28ZB. Appeals involving issues that arise on appeal in other cases}

(1) This section applies where---
\begin{enumerate}\item[]
\opt{oldrules}{
($a$) an appeal (“appeal A”) in relation to a decision falling within section 20(1)
or (3), or an assessment falling within section 20(2), is made to the First-tier
Tribunal, or from the First-tier Tribunal to the Upper Tribunal; and
}

\opt{newrules,2012rules}{
($a$) an appeal (“appeal A”) in relation to a decision or the imposition of a requirement falling within section 20(1) is made to the First-tier Tribunal, or from the First-tier Tribunal to the Upper Tribunal;
}

($b$) an appeal (“appeal B”) is pending against a decision given in a different case by the Upper Tribunal or a court.
\end{enumerate}

(2) If the Secretary of State considers it possible that the result of appeal B will be such that, if it were already determined, it would affect the determination of appeal A, the Secretary of State may serve notice requiring the First-tier Tribunal or Upper Tribunal---
\begin{enumerate}\item[]
($a$) not to be determine appeal A but to refer it to the Secretary of State; or

($b$) to deal with the appeal in accordance with subsection (4).
\end{enumerate}

(3) Where appeal A is referred to the Secretary of State under subsection (2)($a$), following the determination of appeal B and in accordance with that determination, the Secretary of State shall if appropriate---
\begin{enumerate}\item[]
($a$) in a case where appeal A has not been determined by the First-tier Tribunal, revise (under section 16) the decision which gave rise to that appeal; or

($b$) in a case where appeal A has been determined by the First-tier Tribunal, make a decision (under section 17) superseding the tribunal’s decision.
\end{enumerate}

(4) Where appeal A is to be dealt with in accordance with this section, the First-tier Tribunal or Upper Tribunal shall either---
\begin{enumerate}\item[]
($a$) stay appeal A until appeal B is determined; or

($b$) if the First-tier Tribunal or Upper Tribunal considers it to be in the interests of the appellant to do so, determine appeal A as if---
\begin{enumerate}\item[]
(i)
appeal B had already been determined; and

(ii)
the issues arising on appeal B had been decided in the way that was most unfavourable to the appellant.
\end{enumerate}
\end{enumerate}

In this subsection “the appellant” means the person who appealed or, as the case may
be, first appealed against the decision \opt{oldrules}{or assessment}\opt{newrules,2012rules}{or the imposition of the
requirement} mentioned in subsection (1)($a$).

(5) Where the First-tier Tribunal or Upper Tribunal acts in accordance with
subsection (4)($b$), following the determination of appeal B the Secretary of State
shall, if appropriate, make a decision (under section 17) superseding the decision of
the First-tier Tribunal or Upper Tribunal in accordance with that determination.

(6) For the purposes of this section, an appeal against a decision is pending if---
\begin{enumerate}\item[]
($a$) an appeal against the decision has been brought but not determined;

($b$) an application for leave to appeal against the decision has been made but
not determined; or

($c$) in such circumstances as may be prescribed, an appeal against the decision
has not been brought (or, as the case may be, an application for leave to
appeal against the decision has not been made) but the time for doing so has
not yet expired.
\end{enumerate}

(7) In this section---
\begin{enumerate}\item[]
($a$) the reference in subsection (1)($a$) to an appeal to the Upper Tribunal
includes a reference to an application for leave to appeal to the Upper
Tribunal; and

($b$) any reference in paragraph ($a$), ($b$) or ($c$) of subsection (6) to an appeal, or to
an application for leave to appeal, against a decision includes a reference
to---
\begin{enumerate}\item[]
(i) an application for, or for leave to apply for, judicial review of the decision
under section 31 of the Senior Courts Act 1981; or

(ii) an application to the supervisory jurisdiction of the Court of Session in
respect of the decision.
\end{enumerate}
\end{enumerate}

(8) Regulations may make provision supplementing that made by this section.

\amendment{
S. 28ZB inserted (1.6.99) by the Social Security Act 1998 (c. 14) s. 43.

\opt{newrules,2012rules}{S. 28ZB(1)($a$) substituted (3.3.03) for the purposes of certain cases only (see S.I. 2003/192) by the Child Support, Pensions and Social Security Act 2000 (c. 19) Sch. 3 para. 11(12)($a$).

Words in s. 28ZB(4) substituted (3.3.03) for the purposes of certain cases only (see S.I. 2003/192) by the Child Support, Pensions and Social Security Act 2000 (c. 19) Sch. 3 para. 11(12)($b$).}

Words substituted in s. 28ZB (3.11.08) by the Transfer of Tribunal Functions Order 2008 Sch. 3 para. 88.

Reference to Supreme Court Act 1981 in s. 28ZB(7)($b$)(i) substituted (1.10.09) by Constitutional Reform Act 2005 (c. 4) Sch. 11 para 1(2).

Words substituted in s. 28ZB (1.8.12) by the Public Bodies (Child Maintenance and Enforcement Commission: Abolition and Transfer of Functions) Order 2012 Sch. para. 21.

}

\section{Cases of error}

\subsection{28ZC. Restrictions on liability in certain cases of error}

(1) Subject to subsection (2), this section applies where---
\begin{enumerate}\item[]
($a$) the effect of the determination, whenever made, of an appeal to the Upper
Tribunal or the court (“the relevant determination”) is that the adjudicating
authority’s decision out of which the appeal arose was erroneous in point of
law; and

($b$) after the date of the relevant determination a decision falls to be made by the
the Secretary of State in accordance with that determination (or would,
apart from this section, fall to be so made)---
\begin{enumerate}\item[]
(i) with respect to an application for a \opt{oldrules}{maintenance assessment}\opt{newrules,2012rules}{maintenance calculation} (made after the commencement date);

(ii) as to whether to revise, under section 16, \opt{oldrules}{a decision (made after the commencement date) with respect to such an assessment}\opt{newrules,2012rules}{any decision (made after the commencement date) referred to in section 16(1A)}; or

(iii) on an application under section 17 (made after the commencement date) for \opt{oldrules}{a decision with respect to such an assessment to be superseded}\opt{newrules,2012rules}{any decision (made after the commencement date) referred to in section 17(1)}.
\end{enumerate}
\end{enumerate}

(2) This section does not apply where the decision of the Secretary of State mentioned in subsection (1)($b$)---
\begin{enumerate}\item[]
($a$) is one which, but for section 28ZA(2)($a$), would have been made before the date of the relevant determination; or

($b$) is one made in pursuance of section 28ZB(3) or (5).
\end{enumerate}

(3)
In so far as the decision relates to a person’s liability in respect of a period before the date of the relevant determination, it shall be made as if the adjudicating authority’s decision had been found by the Upper Tribunal or court not to have been erroneous in point of law.

(4)
Subsection (1)($a$) shall be read as including a case where---
\begin{enumerate}\item[]
($a$) the effect of the relevant determination is that part or all of a purported regulation or order is invalid; and

($b$) the error of law made by the adjudicating authority was to act on the basis that the purported regulation or order (or the part held to be invalid) was valid.
\end{enumerate}

(5) It is immaterial for the purposes of subsection (1)---
\begin{enumerate}\item[]
($a$) where such a decision as is mentioned in paragraph ($b$)(i) falls to be made; or

($b$) where such a decision as is mentioned in paragraph ($b$)(ii) or (iii) falls to be made on an application under section 16 or (as the case may be) section 17,
\end{enumerate}
whether the application was made before or after the date of the relevant determination.

(6) In this section---
\begin{enumerate}\item[]
 “adjudicating authority” means the Secretary of State, or a child support officer\opt{newrules,2012rules}{
or, in the case of a decision made or a referral under section 28D(1)($b$), the First-tier Tribunal}; 

“the commencement date” means the date of coming into force of section 44 of the Social Security Act 1998; and 

“the court” means the High Court, the Court of Appeal, the Court of Session, the High Court or Court of Appeal in Northern Ireland, the Supreme Court or the Court of Justice of the European Community.
\end{enumerate}

(7) The date of the relevant determination shall, in prescribed cases, be determined for the purposes of this section in accordance with any regulations made for that purpose.

(8) Regulations made under section (7) may include provision---
\begin{enumerate}\item[]
($a$) for a determination of a higher court to be treated as if it had been made on the date of a determination of a lower court or the Upper Tribunal; or

($b$) for a determination of a lower court or the Upper Tribunal to be treated as if it had been made on the date of a determination of a higher court.
\end{enumerate}

\amendment{
S. 28ZC inserted (1.6.99) by Social Security Act 1998 (c. 14) s. 44. 

\opt{newrules,2012rules}{Words “maintenance assessment(s)” substituted by “maintenance calculation(s)” (3.3.03) for
the purposes of certain cases only (see S.I. 2003/192) by the Child Support, Pensions and Social Security Act 2000 (c. 19) s. 1(2)($a$). 

Words in s. 28ZC(1)($b$)(ii) and (iii) substituted for the words from “a decision” to the end
(3.3.03) for the purposes of certain cases only (see S.I. 2003/192) by the Child Support, Pensions and Social Security Act 2000 (c. 19) Sch. 3 para. 11(13)($b$), ($c$). 

Words in s. 28ZC(6) inserted (3.3.03) for the purposes of certain cases only (see S.I. 2003/192) by the Child Support, Pensions and Social Security Act 2000 (c. 19) Sch. 3 para. 11(13)($e$).

Words repealed in s. 28ZC(1)($b$)(i), (3) (14.7.08) by the Child Maintenance and Other Payments Act 2008 (c. 6) Sch. 8.}

Words substituted in s. 28ZC (3.11.08) by the Transfer of Tribunal Functions Order 2008 Sch. 3. para. 89.

Words substituted in defn. of ``the court'' in s. 28ZC(6) (1.10.09) by the Constitutional Reform Act 2005 (c. 4) Sch. 9 para. 54.

Words substituted in s. 28ZC (1.8.12) by the Public Bodies (Child Maintenance and Enforcement Commission: Abolition and Transfer of Functions) Order 2012 Sch. para. 22. 

}

\subsection{28ZD.  Correction of errors and setting aside of decisions}

(1) Regulations may make provision with respect to the correction of accidental errors in any decision of the Secretary of State or record of a decision of the Secretary of State given under this Act.

(2) Nothing in subsection (1) shall be construed as derogating from any power to correct errors which is exercisable apart from regulations made by virtue of that subsection.

\amendment{
S. 28ZD inserted (4.3.99) by Social Security Act 1998 (c. 14) s. 44.

Words inserted in s. 28ZD(1) and s. 28ZD(1)($b$) omitted (3.11.08) by the Transfer of Tribunal Functions Order 2008 Sch. 3 para. 90($a$).

Words omitted in s. 28ZD(2) (3.11.08) by the Transfer of Tribunal Functions Order 2008 Sch. 3 para. 90($b$).
}

\opt{oldrules}{

\section{Departure from usual rules for determining maintenance assessments}

\subsection{28A. Application for a departure direction}

(1) Where a maintenance assessment (“the current assessment”) is in force---
\begin{enumerate}\item[]
($a$) the person with care, or absent parent, with respect to whom it was made, or

($b$) where the application for the current assessment was made under section 7, either of those persons or the child concerned,
\end{enumerate}
may apply to the Secretary of State for a direction under section 28F (a “departure direction”).

(2)
An application for a departure direction shall state in writing the grounds on which it is made and shall, in particular, state whether it is based on---
\begin{enumerate}\item[]
($a$) the effect of the current assessment; or

($b$) a material change in the circumstances of the case since the current assessment was made.
\end{enumerate}

(3)
In other respects, an application for a departure direction shall be made in such manner as may be prescribed.

(4) An application may be made under this section even though an application has been made under section 16(1) or 17(1) with respect to the current assessment.

(5)
If the Secretary of State considers it appropriate to do so, he may by regulations provide for the question whether a change of circumstances is material to be determined in accordance with the regulations.

(6) Schedule 4A has effect in relation to departure directions.

\amendment{
S. 28A inserted (2.12.96) by the Child Support Act 1995 (c. 34) s. 1(1).

S. 28A(4) substituted (1.6.99) by the Social Security Act 1998 (c. 14) Sch. 7 para. 34.

%Words substituted in s. 28A (1.11.08) by the Child Maintenance and Other Payments Act 2008 (c. 6) Sch. 3 para. 25.

%For the reference to the Commission in s. 28A(1) substitute (1.8.12) a reference to the Secretary of State (see the Public Bodies (Child Maintenance and Enforcement Commission: Abolition and Transfer of Functions) Order 2012 art. 5(3)).
}

\subsection{28B. Preliminary consideration of applications}

(1) Where an application for a departure direction has been duly made to the Secretary of State, he may give the application a preliminary consideration.

(2) Where the Secretary of State does so he may, on completing the preliminary consideration, reject the application if it appears to him---
\begin{enumerate}\item[]
($a$) that there are no grounds on which a departure direction could be given in response to the application; or

($b$) that the difference between the current amount and the revised amount is less than an amount to be calculated in accordance with regulations made by the Secretary of State for the purposes of this subsection and section 28F(4).
\end{enumerate}

(3) In subsection (2)---
\begin{enumerate}\item[]
 “the current amount” means the amount of the child support maintenance fixed by the current assessment; and 

“the revised amount” means the amount of child support maintenance which, but for subsection (2)($b$) would be fixed if a fresh maintenance assessment were to be made as a result of a departure direction allowing the departure applied for.
\end{enumerate}

(6) Where a decision as to a maintenance assessment is revised or superseded under section 16 or 17, the Secretary of State---
\begin{enumerate}\item[]
($a$) shall notify the applicant and such other persons as may be prescribed that the decision has been revised or superseded; and

($b$) may direct that the application is to lapse unless, before the end of such period as may be prescribed, the applicant notifies the Secretary of State that he wishes it to stand.
\end{enumerate}

\amendment{
S. 28B inserted by the Child Support Act 1995 (c. 34) s. 2.

S. 28B(6) substituted (4.3.99) by the Social Security Act 1998 (c. 14) Sch. 7 para 35(2).

S. 28B(4), (5) repealed (1.6.99) by the Social Security Act 1998 (c. 14) Sch. 8.

%Words substituted in s. 28B (1.11.08) by the Child Maintenance and Other Payments Act 2008 (c. 6) Sch. 3 para. 26.

%For the references to the Commission in s. 28B(1), (2), (6) substitute (1.8.12) references to the Secretary of State (see the Public Bodies (Child Maintenance and Enforcement Commission: Abolition and Transfer of Functions) Order 2012 art. 5(3)).

}

\subsection{28C. Imposition of a regular payments condition}

(1) Where an application for a departure direction is made by an absent parent, the Secretary of State may impose on him one of the conditions mentioned in subsection (2) (“a regular payments condition”).

(2) The conditions are that---
\begin{enumerate}\item[]
($a$) the applicant must make the payments of child support maintenance fixed by the current assessment;

($b$) the applicant must make such reduced payments of child support maintenance as may be determined in accordance with regulations made by the Secretary of State.
\end{enumerate}

(3)
Where the Secretary of State imposes a regular payments condition, he shall give written notice to the absent parent and person with care concerned of the imposition of the condition and of the effect of failure to comply with it.

(4)
A regular payments condition shall cease to have effect on the failure or determination of the application.

(5)
For the purposes of subsection (4), an application for a departure direction fails if---
\begin{enumerate}\item[]
($a$) it lapses or is withdrawn; or

($b$) the Secretary of State rejects it on completing a preliminary consideration under section 28B.
\end{enumerate}

(6) Where an absent parent has failed to comply with a regular payments condition---
\begin{enumerate}\item[]
($a$) the Secretary of State may refuse to consider the application; and

($b$) in prescribed circumstances the application shall lapse.
\end{enumerate}

(7)
The question whether an absent parent has failed to comply with a regular payments condition shall be determined by the Secretary of State.

(8)
Where the Secretary of State determines that an absent parent has failed to comply with a regular payments condition he shall give that parent, and the person with care concerned, written notice of his decision.

\amendment{
S. 28C inserted by the Child Support Act 1995 (c. 34) s. 3.

%Words substituted in s. 28C (1.11.08) by the Child Maintenance and Other Payments Act 2008 (c. 6) Sch. 3 para. 27.

%For the references to the Commission in s. 28C substitute (1.8.12) references to the Secretary of State (see the Public Bodies (Child Maintenance and Enforcement Commission: Abolition and Transfer of Functions) Order 2012 art. 5(3)).
}

}

\opt{newrules,2012rules}{
\section{Variations}

\subsection[28A. Application for variation of usual rules for calculating maintenance]{\sloppy 28A. Application for variation of usual rules for calculating maintenance}

(1) Where an application for a maintenance calculation is made under section 4 or 7,  the person with care or the non-resident parent or (in the case of an application under section 7) either of them or the child concerned may apply to the Secretary of State for the rules by which the calculation is made to be varied in accordance with this Act.

(2) Such an application is referred to in this Act as an “application for a variation”.

(3) An application for a variation may be made at any time before the Secretary of State has reached a decision (under section 11 or 12(1)) on the application for a maintenance calculation.

(4) A person who applies for a variation---
\begin{enumerate}\item[]
($a$) need not make the application in writing unless the Secretary of State directs in any case that he must; and

($b$) must say upon what grounds the application is made.
\end{enumerate}

(5)
In other respects an application for a variation is to be made in such manner as may be prescribed.

(6)
Schedule 4A has effect in relation to applications for a variation.

\amendment{
S. 28A substituted (10.11.00 for the purposes of making regulations and Acts of Sederunt only, 3.3.03) for the purposes of certain cases only (see S.I. 2003/192) by the Child Support, Pensions and Social Security Act 2000 (c. 19) s. 5(2).

S. 28A(1) modified (31.1.01) where an application for a variation is made under s. 28G by S.I. 2000/3173 reg. 3 to read:
\begin{quotation}
(1) Where a maintenance calculation other than an interim maintenance decision is in force, the person with care or the non-resident parent or (where the maintenance calculation was made following an application under section 7) either of them or the child concerned may apply to the Secretary of State for the rules by which the calculation is made to be varied in accordance with this Act.
\end{quotation}

S. 28A(3) omitted (31.1.01) where an application for a variation is made under s. 28G by S.I. 2000/3173 reg. 3.

Words repealed in s. 28A(1), (3) (27.10.08) by the Child Maintenance and Other Payments Act 2008 (c. 6) Sch. 8.

Words substituted in s. 28A (1.8.12) by the Public Bodies (Child Maintenance and Enforcement Commission: Abolition and Transfer of Functions) Order 2012 Sch. para. 23.
}

\subsection{28B. Preliminary consideration of applications}

(1) Where an application for a variation has been duly made to the Secretary of State, the Secretary of State may give it a preliminary consideration.

(2) The Secretary of State may on completing such a preliminary consideration, reject the application (and proceed to make a decision on the application for a maintenance calculation without any variation) if it appears to the Secretary of State---
\begin{enumerate}\item[]
($a$) that there are no grounds on which a variation could be agreed to;

($b$) that the Secretary of State has insufficient information to make a decision on the application for the maintenance calculation under section 11 (apart from any information needed in relation to the application for a variation), and therefore that the Secretary of State's decision would be made under section 12(1); or

($c$) that other prescribed circumstances apply.
\end{enumerate}

\amendment{
S. 28B substituted (10.11.00 for the purposes of making regulations and Acts of Sederunt only, 3.3.03) for the purposes of certain cases only (see S.I. 2003/192) by the Child Support, Pensions and Social Security Act 2000 (c. 19) s. 5(2).

S. 28B(2) modified (31.1.01) where an application for a variation is made under s. 28G by S.I. 2000/3173 reg. 4 to read:
\begin{quotation}
(2) The Secretary of State may on completing such a preliminary consideration, reject the application (and proceed to revise or supersede a decision under section 16 or 17 respectively without taking the variation into account, or not revise or supersede a decision under section 16 or 17) if it appears to the Secretary of State---
\begin{enumerate}\item[]
($a$) that there are no grounds on which a variation could be agreed to;

%($b$) that the Secretary of State has insufficient information to make a decision on the application for the maintenance calculation under section 11 (apart from any information needed in relation to the application for a variation), and therefore that the Secretary of State's decision would be made under section 12(1); or

($c$) that other prescribed circumstances apply.
\end{enumerate}
\end{quotation}


Words substituted in s. 28B (1.8.12) by the Public Bodies (Child Maintenance and Enforcement Commission: Abolition and Transfer of Functions) Order 2012 Sch. para. 24.
}

\subsection{28C. Imposition of regular payments condition}

(1) Where---
\begin{enumerate}\item[]
($a$) an application for a variation is made by the non-resident parent; and

($b$) the Secretary of State makes an interim maintenance decision,
\end{enumerate}
the Secretary of State may also, if the Secretary of State has completed a
preliminary consideration (under section 28B) of the application for a variation
and has not rejected it under that section, impose on the non-resident parent one of
the conditions mentioned in subsection (2) (a “regular payments condition”).

(2) The conditions are that---
\begin{enumerate}\item[]
($a$) the non-resident parent must make the payments of child support
maintenance specified in the interim maintenance decision;

($b$) the non-resident parent must make such lesser payments of child support
maintenance as may be determined in accordance with regulations made
by the Secretary of State.
\end{enumerate}

(3) Where the Secretary of State imposes a regular payments condition, the
Secretary of State shall give written notice of the imposition of the condition and of
the effect of failure to comply with it to---
\begin{enumerate}\item[]
($a$) the non-resident parent;

($b$) all the persons with care concerned; and

($c$) if the application for the maintenance calculation was made under section
7, the child who made the application.
\end{enumerate}

(4) A regular payments condition shall cease to have effect---
\begin{enumerate}\item[]
($a$) when the Secretary of State has made a decision on the application for a
maintenance calculation under section 11 (whether he agrees to a variation
or not);

($b$) on the withdrawal of the application for a variation.
\end{enumerate}

(5) Where a non-resident parent has failed to comply with a regular payments condition, the Secretary of State may in prescribed circumstances refuse to consider the application for a variation, and instead reach a decision under section 11 as if no such application had been made.

(6)
The question whether a non-resident parent has failed to comply with a regular payments condition is to be determined by the Secretary of State.

(7)
Where the Secretary of State determines that a non-resident parent has failed to comply with a regular payments condition the Secretary of State shall give written notice of the determination to---
\begin{enumerate}\item[]
($a$) that parent;

($b$) all the persons with care concerned; and

($c$) if the application for the maintenance calculation was made under section 7, the child who made the application.
\end{enumerate}

\amendment{
S. 28C substituted (10.11.00 for the purposes of making regulations and Acts of Sederunt only, 3.3.03) for the purposes of certain cases only (see S.I. 2003/192) by the Child Support, Pensions and Social Security Act 2000 (c. 19) s. 5(2).

S. 28C(1) modified (31.1.01) where an application for a variation is made under s. 28G by S.I. 2000/3173 reg. 5(2) by the omission of s. 28C(1)($b$) and the word ``also'' in the following text.

S. 28C(2)($a$) modified (31.1.01) where an application for a variation is made under s. 28G by S.I. 2000/3173 reg. 5(3) to read:

\begin{quotation}
($a$) the non-resident parent must make the payment of
child support maintenance specified in the
maintenance calculation in force;
\end{quotation}

S. 28C(3)($c$), (7)($c$) modified (31.1.01) where an application for a variation is made under s. 28G by S.I. 2000/3173 reg. 5(4) to read:

\begin{quotation}
($c$) if the maintenance calculation in force was made
in response to an application made under section
7, the child who made the application.
\end{quotation}

S. 28C(4)($a$) modified (31.1.01) where an application for a variation is made under s. 28G by S.I. 2000/3173 reg. 5(5) to read:

\begin{quotation}
($a$) when in response to the application for a variation the Secretary of State has revised or superseded a decision under section 16 and 17 respectively (whether he agrees to a variation or not) or not revised or superseded a decision under section 16 or 17;
\end{quotation}

S. 28C(5) modified (31.1.01) where an application for a variation is made under s. 28G by S.I. 2000/3173 reg. 5(6) to read:

\begin{quotation}
(5) Where a non-resident parent has failed to comply with a regular payments condition, the Secretary of State may in prescribed circumstances refuse to consider the application for a variation, and instead revise or supersede a decision under section 16 or 17 respectively, or not revise or supersede a decision under section 16 or 17, as if the application had
failed.
\end{quotation}

Words substituted in s. 28C (1.8.12) by the Public Bodies (Child Maintenance and Enforcement Commission: Abolition and Transfer of Functions) Order 2012 Sch. para. 25.
}
}

\subsection{28D. Determination of applications}

\opt{oldrules}{(1) Where an application for a departure direction has not failed, the Secretary of State shall---
\begin{enumerate}\item[]
($a$) determine the application in accordance with the relevant provisions of, or made under, this Act; or

($b$) refer the application to the First-tier Tribunal for the tribunal to determine it in accordance with those provisions.
\end{enumerate}}

\opt{newrules,2012rules}{(1) Where an application for a variation has not failed, the Secretary of State shall, in accordance with the relevant provisions of, or made under, this Act—
\begin{enumerate}\item[]
($a$) either agree or not to a variation, and make a decision under section 11 or 12(1); or

($b$) refer the application to the First-tier Tribunal for the tribunal to determine what variation, if any, is to be made.
\end{enumerate}}

(2) For the purposes of subsection (1), \opt{oldrules}{an application for a departure direction}\opt{newrules,2012rules}{an application for a variation} has failed if---
\begin{enumerate}\item[]
($a$) it has 
\opt{oldrules}{lapsed or}%
been withdrawn; or

($b$) the Secretary of State has rejected it on completing a preliminary consideration under section 28B\opt{newrules,2012rules}{; or

($c$) the Secretary of State has refused to consider it under section 28C(5)}.
\end{enumerate}

\opt{2012rules}{
(2A) Subsection (2B) applies if—
\begin{enumerate}\item[]
($a$) the application for a variation is made by the person with care or (in the case of an application for a maintenance calculation under section 7) the person with care or the child concerned, and

($b$) it appears to the Secretary of State that consideration of further information or evidence may affect the decision under subsection (1)($a$) whether or not to agree to a variation.
\end{enumerate}

(2B) Before making the decision under subsection (1)($a$) the Secretary of State must—
\begin{enumerate}\item[]
($a$) consider any such further information or evidence that is available to the Secretary of State, and

($b$) where necessary, take such steps as the Secretary of State considers appropriate to obtain any such further information or evidence.
\end{enumerate}
}

(3)
In dealing with \opt{oldrules}{an application for a departure direction}\opt{newrules,2012rules}{an application for a variation} which has been referred to it under subsection (1)($b$), the First-tier Tribunal shall have the same powers, and be subject to the same duties\opt{2012rules}{, apart from the duty under subsection (2B),} 
as would the Secretary of State in dealing with the application.

\amendment{
S. 28D inserted by the Child Support Act 1995 (c. 34) s. 4.

\opt{newrules,2012rules}{S. 28D(1) substituted (10.11.00 for the purposes of making regulations and Acts of Sederunt only, 3.3.03) for the purposes of certain cases only (see S.I. 2003/192) by the Child Support, Pensions and Social Security Act 2000 (c. 19) s. 5(3)($a$).

Words substituted in s. 28D(2), (3) (10.11.00 for the purposes of making regulations and Acts of Sederunt only, 3.3.03) for the purposes of certain cases only (see S.I. 2003/192) by the Child Support, Pensions and Social Security Act 2000 (c. 19) s. 5(3)($b$).

Words omitted in s. 28D(2)($a$) and s. 28D(2)($c$) inserted (10.11.00 for the purposes of making regulations and Acts of Sederunt only, 3.3.03) for the purposes of certain cases only (see S.I. 2003/192) by the Child Support, Pensions and Social Security Act 2000 (c. 19) s. 5(3)($c$).

S. 28D(1)($a$) modified (31.1.01) where an application for a variation is made under s. 28G by S.I. 2000/3173 reg. 6 to read:

\begin{quotation}
($a$) either agree or not to a variation, and make a decision under section 16 or 17; or
\end{quotation}

}

Words in s. 28D(1)($b$), (3) substituted (3.11.08) by the Transfer of Tribunal Functions Order 2008 Sch. 3 para. 91.

Words substituted in s. 28D (1.8.12) by the Public Bodies (Child Maintenance and Enforcement Commission: Abolition and Transfer of Functions) Order 2012 Sch. para. 26.

\opt{2012rules}{
S. 28D(2A), (2B) and words in s. 28D(3) inserted (10.12.12) for the purposes of certain cases only (see S.I. 2012/3042) by the Child Maintenance and Other Payments Act 2008 (c. 6) s. 18 as amended by the Public Bodies (Child Maintenance and Enforcement Commission: Abolition and Transfer of Functions) Order 2012 Sch. para. 78.
}
}

\subsection{28E. Matters to be taken into account}

(1) In determining \opt{olrdules}{any application for a departure direction}\opt{newrules,2012rules}{whether to agree to a variation}, the Secretary of State shall have regard both to the general principles set out in subsection (2) and to such other considerations as may be prescribed.

(2) The general principles are that---
\begin{enumerate}\item[]
($a$) parents should be responsible for maintaining their children whenever they can afford to do so;

($b$) where a parent has more than one child, his obligation to maintain any one of them should be no less of an obligation than his obligation to maintain any other of them.
\end{enumerate}

(3) In determining \opt{oldrules}{any application for a departure direction}\opt{newrules,2012rules}{whether to agree to a variation}, the Secretary of State shall take into account any representations made to \opt{oldrules}{him}\opt{newrules,2012rules}{the Secretary of State}—
\begin{enumerate}\item[]
($a$) by the person with care or absent parent concerned; or

($b$) where the application for the current assessment was made under section 7, by either of them or the child concerned.
\end{enumerate}

(4) In determining \opt{oldrules}{any application for a departure direction}\opt{newrules,2012rules}{whether to agree to a variation}, no account shall be taken of the fact that—
\begin{enumerate}\item[]
($a$) any part of the income of the person with care concerned is, or would be if \opt{oldrules}{a departure direction were made}\opt{newrules,2012rules}{the Secretary of State agreed to a variation}, derived from any benefit; or

($b$) some or all of any child support maintenance might be taken into account in any manner in relation to any entitlement to benefit.
\end{enumerate}

(5) In this section “benefit” has such meaning as may be prescribed.

\amendment{
S. 28E inserted by the Child Support Act 1995 (c. 34) s. 5.

For the definition of ``benefit'' in s. 28E(5) see S.I. 1996/635 reg. 12.

\opt{newrules,2012rules}{Words substituted in s. 28E (10.11.00 for the purposes of making regulations and Acts of Sederunt only, 3.3.03) for the purposes of certain cases only (see S.I. 2003/192) by the Child Support, Pensions and Social Security Act 2000 (c. 19) s. 5(4).

S. 28E(3)($b$) modified (31.1.01) where an application for a variation is made under s. 28G by S.I. 2000/3173 reg. 6 to read:

\begin{quotation}
($b$) where the maintenance calculation in force was made in response to an application under section 7, by either of them or the child concerned.
\end{quotation}}

Words substituted in s. 28E (1.8.12) by the Public Bodies (Child Maintenance and Enforcement Commission: Abolition and Transfer of Functions) Order 2012 Sch. para. 27.
}

\opt{oldrules}{
\subsection{28F. Departure directions}

(1) The Secretary of State may give a departure direction if—
\begin{enumerate}\item[]
($a$) he is satisfied that the case is one which falls within one or more of the cases set out in Part I of Schedule 4B or in regulations made under that Part; and

($b$) it is his opinion that, in all the circumstances of the case, it would be just and equitable to give a departure direction.
\end{enumerate}

(2) In considering whether it would be just and equitable in any case to give a departure direction, the Secretary of State shall have regard, in particular, to—
\begin{enumerate}\item[]
($a$) the financial circumstances of the absent parent concerned,

($b$) the financial circumstances of the person with care concerned, and

($c$) the welfare of any child likely to be affected by the direction.
\end{enumerate}

(3) The Secretary of State may by regulations make provision—
\begin{enumerate}\item[]
($a$) for factors which are to be taken into account in determining whether it would be just and equitable to give a departure direction in any case;

($b$) for factors which are not to be taken into account in determining such a question.
\end{enumerate}

(4) The Secretary of State shall not give a departure direction if he is satisfied that the difference between the current amount and the revised amount is less than an amount to be calculated in accordance with regulations made by the Secretary of State for the purposes of this subsection and section 28B(2).

(5) In subsection (4)—
\begin{enumerate}\item[]
“the current amount” means the amount of the child support maintenance fixed by the current assessment, and

“the revised amount” means the amount of child support maintenance which would be fixed if a fresh maintenance assessment were to be made as a result of the departure direction which the Secretary of State would give in response to the application but for subsection (4).
\end{enumerate}

(6) A departure direction shall—
\begin{enumerate}\item[]
($a$) require 
%a child support officer to make 
the making of
one or more fresh maintenance assessments; and

($b$) specify the basis on which the amount of child support maintenance is to be fixed by any assessment made in consequence of the direction.
\end{enumerate}

(7) In giving a departure direction, the Secretary of State shall comply with the provisions of regulations made under Part II of Schedule 4B.

(8) Before the end of such period as may be prescribed, the Secretary of State shall notify the applicant for a departure direction, and such other persons as may be prescribed—
\begin{enumerate}\item[]
($a$) of his decision in relation to the application, and

($b$) of the reasons for his decision.
\end{enumerate}

\amendment{
S. 28F inserted (2.12.96) by the Child Support Act 1995 (c. 34) s. 6.

Words in s. 28F(6) substituted (1.6.99) by the Social Security Act 1998 (c. 14) Sch. 7 para. 37.
}
}

\opt{newrules,2012rules}{

\subsection{28F. Agreement to a variation}

(1) The Secretary of State may agree to a variation if---
\begin{enumerate}\item[]
($a$) the Secretary of State is satisfied that the case is one which falls within one or more of the cases set out in Part I of Schedule 4B or in regulations made under that Part; and

($b$) it is the Secretary of State’s opinion that, in all the circumstances of the case, it would be just and equitable to agree to a variation.
\end{enumerate}

(2) In considering whether it would be just and equitable in any case to agree to a variation, the Secretary of State---
\begin{enumerate}\item[]
($a$) must have regard, in particular, to the welfare of any child likely to be affected if the Secretary of State did agree to a variation; and

($b$) must, or as the case may be must not, take any prescribed factors into account, or must take them into account (or not) in prescribed circumstances.
\end{enumerate}

(3) The Secretary of State shall not agree to a variation (and shall proceed
to make a decision on the application for a maintenance calculation without any
variation) if satisfied that---
\begin{enumerate}\item[]
($a$) the Secretary of State has insufficient information to make a decision on the application for the maintenance calculation under section 11, and therefore that the decision would be made under section 12(1); or

($b$) other prescribed circumstances apply.
\end{enumerate}

(4) Where the Secretary of State agrees to a variation, the Secretary of State shall---
\begin{enumerate}\item[]
($a$) determine the basis on which the amount of child support maintenance is to be calculated in response to the application for a maintenance calculation; and

($b$) make a decision under section 11 on that basis.
\end{enumerate}

(5) If the Secretary of State has made an interim maintenance decision, it is to be treated as having been replaced by the Secretary of State’s decision under section 11, and except in prescribed circumstances any appeal connected with it (under section 20) shall lapse.

(6) In determining whether or not to agree to a variation, the Secretary of State shall comply with regulations made under Part II of Schedule 4B.

\amendment{
S. 28F substituted (10.11.00 for the purposes of making regulations and Acts of Sederunt only, 3.3.03 for the purposes of certain cases only, see S.I. 2003/192) by the Child Support, Pensions and Social Security Act 2000 (c. 19) s. 5(5).

S. 28F(3) modified (31.1.01) where an application for a variation is made under s. 28G by S.I. 2000/3173 reg. 7($a$) to read:

\begin{quotation}
(3) The Secretary of State shall not agree to a variation (and shall proceed to revise or supersede a decision under section 16 or 17 respectively without taking the variation into account, or not revise or supersede a decision under section 16 or 17) if he is satisfied that prescribed circumstances apply.
\end{quotation}

S. 28F(4) modified (31.1.01) where an application for a variation is made under s. 28G by S.I. 2000/3173 reg. 7($b$), ($c$) to read:

\begin{quotation}
(4) Where the Secretary of State agrees to a variation, the Secretary of State shall---
\begin{enumerate}\item[]
($a$) determine the basis on which the amount of child support maintenance is to be calculated in response to the application% 
%for a maintenance calculation
; and

%($b$) make a decision under section 11 on that basis.
($b$) revise or supersede a decision under section 16 or 17 respectively on that basis.
\end{enumerate}
\end{quotation}

S. 28F(5) omitted (31.1.01) where an application for a variation is made under s. 28G by S.I. 2000/3173 reg. 7($d$).

Words repealed in s. 28F(4)($a$) (27.10.08) by the Child Maintenance and Other Payments Act 2008 (c. 6) Sch. 8.

Words substituted in s. 28F (1.11.08) by the Child Maintenance and Other Payments Act 2008 (c. 6) Sch. 3 para. 30.

Words substituted in s. 28F (1.8.12) by the Public Bodies (Child Maintenance and Enforcement Commission: Abolition and Transfer of Functions) Order 2012 Sch. para. 28.
}

}

\opt{oldrules}{
\subsection{28G. Effect and duration of departure directions}

%(1) Where a departure direction is given, it shall be the duty of the child support officer to whom the case is referred to comply with the direction as soon as is reasonably practicable.

(2) A departure direction may be given so as to have effect—
\begin{enumerate}\item[]
($a$) for a specified period; or

($b$) until the occurrence of a specified event.
\end{enumerate}

(3) The Secretary of State may by regulations make provision for the cancellation of a departure direction in prescribed circumstances.

(4) The Secretary of State may by regulations make provision as to when a departure direction is to take effect.

(5) Regulations under subsection (4) may provide for a departure direction to have effect from a date earlier than that on which the direction is given.

\amendment{
S. 28G inserted (2.12.96) by the Child Support Act 1995 (c. 34) s. 7.

S. 28G(1) repealed (1.6.99) by the Social Security Act 1998 (c. 14) Sch. 8.

%S. 28G(2) substituted (1.1.01) by the Child Support, Pensions and Social Security Act 2000 (c. 19) s. 7.
}
}

\opt{newrules,2012rules}{

\subsection{28G. Variations: revision and supersession}

(1) An application for a variation may also be made when a maintenance calculation is in force.

(2) The Secretary of State may by regulations provide for---
\begin{enumerate}\item[]
($a$) sections 16, 17 and 20; and

($b$) sections 28A to 28F and Schedules 4A and 4B,
\end{enumerate}
to apply with prescribed modifications in relation to such applications.

(3) The Secretary of State may by regulations provide that, in prescribed cases (or except in prescribed cases), a decision under section 17 made otherwise than pursuant to an application for a variation may be made on the basis of a variation agreed to for the purposes of an earlier decision without a new application for a variation having to be made.

\amendment{
S. 28G substituted (1.1.01) by the Child Support, Pensions and Social Security Act 2000 (c. 19) s. 7.

\medskip

Ss. 28H, 28I repealed (3.3.03) for the purposes of certain cases only (see S.I. 2003/192) by the Child Support, Pensions and Social Security Act 2000 (c. 19) Sch. 3 para. 11(14).
}

}

\opt{oldrules}{

\subsection{28H. Departure directions: decisions and appeals}

Schedule 4C shall have effect for applying sections 16, 17,
20 and 28ZA to 28ZC to decisions with respect to departure
directions.

\amendment{
S. 28H substituted (4.3.99) by the Social Security Act 1998 (c. 14) Sch. 7 para. 39.
}

\subsection{28I. Transitional provisions}

(4) The Secretary of State may by regulations make provision---
\begin{enumerate}\item[]
($a$) enabling applications for departure directions made before the coming into force of section 28A to be considered even though that section is not in force;

($b$) for the determination of any such application as if section 28A and the other provisions of this Act relating to departure directions were in force; and

($c$) as to the effect of any departure direction given before the coming into force of section 28A.
\end{enumerate}

(5) Regulations under section 28G(4) may not provide for a departure direction to have effect from a date earlier than that on which that section came into force.

\amendment{
S. 28I(5) inserted (22.1.96) by the Child Support Act 1995 (c. 34) s. 9.

S. 28I(4) inserted (14.10.96) by the Child Support Act 1995 (c. 34) s. 9.

S. 28I(1)--(3) not yet in force.
}

}

\opt{newrules,2012rules}{

\section{Voluntary payments}

\subsection{28J. Voluntary payments}

(1) This section applies where---
\begin{enumerate}\item[]
($a$) a person has applied for a maintenance calculation under section 4(1) or 7(1);

($b$) the Secretary of State has neither made a decision under section 11 or 12 on the application, nor decided not to make a maintenance calculation; and

($c$) the non-resident parent makes a voluntary payment.
\end{enumerate}

(2) A “voluntary payment” is a payment---
\begin{enumerate}\item[]
($a$) on account of child support maintenance which the non-resident parent expects to become liable to pay following the determination of the application (whether or not the amount of the payment is based on any estimate of his potential liability which the Secretary of State has agreed to give); and

($b$) made before the maintenance calculation has been notified to the non-resident parent or (as the case may be) before the Secretary of State has notified the non-resident parent that the Secretary of State has decided not to make a maintenance calculation.
\end{enumerate}

(3) In such circumstances and to such extent as may be prescribed---
\begin{enumerate}\item[]
($a$) the voluntary payment may be set off against arrears of child support maintenance which accrued by virtue of the maintenance calculation taking effect on a date earlier than that on which it was notified to the non-resident parent;

($b$) the amount payable under a maintenance calculation may be adjusted to take account of the voluntary payment.
\end{enumerate}

(4)
A voluntary payment shall be made to the Secretary of State unless the Secretary of State agrees, on such conditions as he may specify, that it may be made to the person with care, or to or through another person.

(5)
The Secretary of State may by regulations make provision as to voluntary payments, and the regulations may in particular---
\begin{enumerate}\item[]
($a$) prescribe what payments or descriptions of payment are, or are not, to count as “voluntary payments”;

($b$) prescribe the extent to which and circumstances in which a payment, or a payment of a prescribed description, counts.
\end{enumerate}

\amendment{
S. 28J inserted (10.11.00 for the purposes of making regulations and Acts of Sederunt only, 3.3.03 for the purposes of certain cases only, see S.I. 2003/192) by the Child Support, Pensions and Social Security Act 2000 (c. 19) s. 20(1).

Words repealed in s. 28J(1)($a$) (27.10.08) by the Child Maintenance and Other Payments Act 2008 (c. 6) Sch. 8.

Words substituted in s. 28J (1.8.12) by the Public Bodies (Child Maintenance and Enforcement Commission: Abolition and Transfer of Functions) Order 2012 Sch. para. 29.
}
}

\section{Collection and enforcement}

\subsection{29. Collection of child support maintenance}

(1) The Secretary of State may arrange for the collection of any child support maintenance payable in accordance with a \opt{oldrules}{maintenance assessment}\opt{newrules,2012rules}{maintenance calculation} where
an application has been made to the Secretary of State under section 4(2)
or 7(3) for the Secretary of State to arrange for its collection.

(2) Where a \opt{oldrules}{maintenance assessment}\opt{newrules,2012rules}{maintenance calculation} is made under this Act, payments of child support maintenance under the \opt{oldrules}{assessment}\opt{newrules,2012rules}{calculation} shall be made in accordance with regulations made by the Secretary of State.

(3) The regulations may, in particular, make provision---
\begin{enumerate}\item[]
($a$) for payments of child support maintenance to be made---
\begin{enumerate}\item[]
(i) to the person caring for the child or children in question;

(ii) to, or through, the Secretary of State; or

(iii) to, or through, such other person as the Secretary of State may, from
time to time, specify;
\end{enumerate}

($b$) as to the method by which payments of child support maintenance are to
be made;

%($c$) as to the intervals at which such payments are to be made;
($c$) for determining, on the basis of prescribed assumptions, the total amount of the payments of child support maintenance payable in a reference period (including provision for adjustments to such an amount);

($ca$) requiring payments of child support maintenance to be made—
\begin{enumerate}\item[]
(i) by reference to such an amount and a reference period; and

(ii) at prescribed intervals falling in a reference period;
\end{enumerate}

($d$) as to the method and timing of the transmission of payments which are
made, to or through the Secretary of State or any other person, in accordance
with the regulations;

($e$) empowering the Secretary of State to direct any person liable to make
payments in accordance with the \opt{oldrules}{assessment}\opt{newrules,2012rules}{calculation}---
\begin{enumerate}\item[]
(i) to make them by standing order or by any other method which
requires one person to give his authority for payments to be made
from an account of his to an account of another’s on specific dates
during the period for which the authority is in force and without the
need for any further authority from him;

(ii) to open an account from which payments under the \opt{oldrules}{assessment}\opt{newrules,2012rules}{calculation} may be made in accordance with the method of payment
which that person is obliged to adopt;
\end{enumerate}

($f$) providing for the making of representations with respect to matters with
which the regulations are concerned.
\end{enumerate}

(3A) In subsection (3)($c$) and ($ca$) “a reference period” means—
\begin{enumerate}\item[]
($a$) a period of 52 weeks beginning with a prescribed date; or

($b$) in prescribed circumstances, a prescribed period.
\end{enumerate}

(4) If the regulations include provision for payment by means of deduction in accordance with an order under section 31, they must make provision---
\begin{enumerate}\item[]
($a$) for that method of payment not to be used in any case where there is good
reason not to use it; and

($b$) for the person against whom the order under section 31 would be made to
have a right of appeal to a magistrates’ court (or, in Scotland, to the sheriff)
against a decision that the exclusion required by paragraph ($a$) does not
apply.
\end{enumerate}

(5) On an appeal under regulations made under subsection (4)($b$) the court or (as the case may be) the sheriff shall not question the maintenance calculation by reference to which the order under section 31 would be made.

(6) Regulations under subsection (4)($b$) may include---
\begin{enumerate}\item[]
($a$) provision with respect to the period within which a right of appeal under the
regulations may be exercised;

($b$) provision with respect to the powers of a magistrates’ court (or, in Scotland, of 
the sheriff) in relation to an appeal under the regulations.
\end{enumerate}

(7) If the regulations include provision for payment by means of deduction in accordance with an order under section 31, they may make provision---
\begin{enumerate}\item[]
($a$) prescribing matters which are, or are not, to be taken into account in determining whether there is good reason not to use that method of payment;

($b$) prescribing circumstances in which good reason not to use that method of payment is, or is not, to be regarded as existing.
\end{enumerate}


\amendment{
\opt{newrules,2012rules}{
Words ``maintenance assessment(s)'' substituted by ``maintenance calculation(s)'' (3.3.03) for the purposes of certain cases only (see S.I. 2003/192) by the Child Support, Pensions and Social Security Act 2000 (c. 19) s. 1(2)($a$).

Word ``assessment'' (or any variant of that term) substituted by ``calculation'' (or other variants) (3.3.03) for the purposes of certain cases only (see S.I. 2003/192) by the Child Support, Pensions and Social Security Act 2000 (c. 19) s. 1(2)($b$).

}

S. 29(1)($a$) repealed (27.10.08) by the Child Maintenance and Other Payments Act 2008 (c. 6) Sch. 8.

S. 29(4)--(7) inserted (29.9.08 for the purpose of making regulations, 27.10.08 for all other purposes) by the Child Maintenance and Other Payments Act 2008 (c. 6) s. 20.

Words substituted in s. 29 (1.8.12) by the Public Bodies (Child Maintenance and Enforcement Commission: Abolition and Transfer of Functions) Order 2012 Sch. para. 30.

S. 29(3)($c$), ($ca$) substituted for s. 29(3)($c$) and s. 29(3A) inserted (8.10.12) by the Welfare Reform Act 2009 (c. 24) s. 54.

The insertion in s. 29(1) by the Welfare Reform Act 2012 (c. 5) s. 137(4) is not yet in force.

}

\subsection{30. Collection and enforcement of other forms of maintenance}

(1) Where the Secretary of State is arranging for the collection of any 
payments under section 29 or subsection (2), the Secretary of State may also arrange for the collection of any periodical payments, or secured periodical payments, of a prescribed kind which are payable to or for the benefit of any person who falls within a prescribed category.

(2) The Secretary of State may, except in prescribed cases, arrange for the collection of any periodical payments, or secured periodical payments, of a prescribed kind which are payable for the benefit of a child even though the Secretary of State is not arranging for the collection of child support maintenance with respect to that child.

(3) Where---
\begin{enumerate}\item[]
($a$) the Secretary of State is arranging, under this Act, for the collection of different payments (“the payments”) from the same \opt{oldrules}{absent parent}\opt{newrules,2012rules}{non-resident parent};

($b$) an amount is collected by the Secretary of State from the \opt{oldrules}{absent parent}\opt{newrules,2012rules}{non-resident parent} which is less than the total amount due in respect of the payments; and

($c$) the \opt{oldrules}{absent parent}\opt{newrules,2012rules}{non-resident parent} has not stipulated how that amount
is to be allocated by the Secretary of State as between the payments, 
\end{enumerate}
the Secretary of State may allocate that amount as he sees fit.

(4)
In relation to England and Wales, the Secretary of State may by regulations make provision for sections 29 and 31 to 40 to apply, with such modifications (if any) as he considers necessary or expedient, for the purpose of enabling the Secretary of State to enforce any obligation to pay any amount for the collection of which the Secretary of State is authorised under this section to make arrangements.

(5)
In relation to Scotland, the Secretary of State may by regulations make provision for the purpose of enabling the Secretary of State to enforce any obligation to pay any amount for the collection of which the Secretary of State is authorised under this section to make arrangements---
\begin{enumerate}\item[]
($a$) empowering the Secretary of State to bring any proceedings or take any other steps (other than diligence against earnings) which could have been brought or taken by or on behalf of the person to whom the periodical payments are payable;

($b$) applying sections 29, 31 and 32 with such modifications (if any) as he considers necessary or expedient.
\end{enumerate}

\amendment{
\opt{newrules,2012rules}{ Words “(an) absent parent” substituted by “($a$) non-resident parent(s)” (3.3.03) for the purposes of certain cases only (see S.I. 2003/192) by the Child Support, Pensions and Social Security Act 2000 (c. 19) Sch. 3 para. 11(2).}

 S. 30(2) substituted (3.3.03) by the Child Support, Pensions and Social Security Act 2000 (c. 19) Sch. 3 para. 11(15).

Words substituted in s. 30(4), (5) (1.6.09) by the Child Maintenance and Other Payments Act 2008 (c. 6) Sch. 7 para. 1(7).

Words substituted in s. 30 (1.8.12) by the Public Bodies (Child Maintenance and Enforcement Commission: Abolition and Transfer of Functions) Order 2012 Sch. para. 31.

The insertion of s. 30(5A) by the Child Support Act 1995 (c. 34) Sch. 3 para. 9 is not yet in force.
}

\subsection{31. Deduction from earnings orders}

(1) This section applies where any person (“the liable person”) is liable
to make payments of child support maintenance.

(2)
The Secretary of State may make an order (“a deduction from earnings order”) against the liable person to secure the payment of any amount due under the \opt{oldrules}{maintenance assessment}\opt{newrules,2012rules}{maintenance calculation} in question.

(3)
A deduction from earnings order may be made so as to secure the payment of---
\begin{enumerate}\item[]
($a$) arrears of child support maintenance payable under the \opt{oldrules}{assessment}\opt{newrules,2012rules}{calculation};

($b$) amounts of child support maintenance which will become due under the
\opt{oldrules}{assessment}\opt{newrules,2012rules}{calculation}; or

($c$) both such arrears and such future amounts.
\end{enumerate}

(4) A deduction from earnings order---
\begin{enumerate}\item[]
($a$) shall be expressed to be directed at a person (“the employer”) who has
the liable person in his employment; and

($b$) shall have effect from such date as may be specified in the order.
\end{enumerate}

(5)
A deduction from earnings order shall operate as an instruction to the employer to---
\begin{enumerate}\item[]
($a$) make deductions from the liable person’s earnings; and

($b$) pay the amounts deducted to the Secretary of State.
\end{enumerate}

(6)
The Secretary of State shall serve a copy of any deduction from earnings order made under this section on---
\begin{enumerate}\item[]
($a$) the person who appears to the Secretary of State to have the liable person
in question in his employment; and

($b$) the liable person.
\end{enumerate}

(7) Where---
\begin{enumerate}\item[]
($a$) the deduction from earnings order has been made; and

($b$)
a copy of the order has been served on the liable person’s employer, 
\end{enumerate}
it shall be the duty of that employer to comply with the order; but he shall not be
under any liability for non-compliance before the end of the period of 7 days beginning with the date on which the copy was served on him.

(8)
In this section and in section 32 “earnings” has such meaning as may be prescribed.

\amendment{
\opt{newrules,2012rules}{Words “maintenance assessment(s)” substituted by “maintenance calculation(s)” (3.3.03) for the purposes of certain cases only (see S.I. 2003/192) by the Child Support, Pensions and Social Security Act 2000 (c. 19) s. 1(2)($a$).

Word “assessment” (or any variant of that term) substituted by “calculation” (or other variants)
(3.3.03) for the purposes of certain cases only (see S.I. 2003/192) by the Child Support, Pensions and Social Security Act 2000 (c. 19) s. 1(2)($b$).}

Words substituted in s. 31 (1.8.12) by the Public Bodies (Child Maintenance and Enforcement Commission: Abolition and Transfer of Functions) Order 2012 Sch. para. 32.

The substitution of s. 31(8), (9) for s. 31(8) by the Child Maintenance and Other Payments Act 2008 (c. 6) s. 21 is not yet in force.

}

\subsection{32. Regulations about deduction from earnings orders}

(1) The Secretary of State may by regulations make provision with 
respect to deduction from earnings orders.

(2) The regulations may, in particular, make provision---
\begin{enumerate}\item[]
($a$) as to the circumstances in which one person is to be treated as employed
by another;

($b$) requiring any deduction from earnings under an order to be made in the
prescribed manner;

\opt{newrules,2012rules}{($bb$) for the amount or amounts which are to be deducted from the liable person’s
earnings not to exceed a prescribed proportion of his earnings (as
determined by the employer);}

($c$) requiring an order to specify the amount or amounts to which the order relates and the amount or amounts which are to be deducted from the liable person’s earnings in order to meet his liabilities under the \opt{oldrules}{maintenance assessment}\opt{newrules,2012rules}{maintenance calculation} in question;

($d$) requiring the intervals between deductions to be made under an order to be specified in the order;

($e$) as to the payment of sums deducted under an order to the Secretary of State;

($f$) allowing the person who deducts and pays any amount under an order to deduct from the liable person’s earnings a prescribed sum towards his administrative costs;

($g$) with respect to the notification to be given to the liable person of amounts deducted, and amounts paid, under the order;

($h$) requiring any person on whom a copy of an order is served to notify the Secretary of State in the prescribed manner and within a prescribed period if he does not have the liable person in his employment or if the liable person ceases to be in his employment;

($i$) as to the operation of an order where the liable person is in the employment of the Crown;

($j$) for the variation of orders;

($k$) similar to that made by section 31(7), in relation to any variation of an order;

($l$) for an order to lapse when the employer concerned ceases to have the liable person in his employment;

($m$) as to the revival of an order in such circumstances as may be prescribed;

($n$) allowing or requiring an order to be discharged;

($o$) as to the giving of notice by the Secretary of State to the employer concerned that an order has lapsed or has ceased to have effect.
\end{enumerate}

(3) The regulations may include provision that while a deduction from earnings order is in force---
\begin{enumerate}\item[]
($a$) the liable person shall from time to time notify the Secretary of State, in the prescribed manner and within a prescribed period, of each occasion on which he leaves any employment or becomes employed, or re-employed, and shall include in such a notification a statement of his earnings and expected earnings from the employment concerned and of such other matters as may be prescribed;

($b$) any person who becomes the liable person’s employer and knows that the order is in force shall notify the Secretary of State, in the prescribed manner and within a prescribed period, that he is the liable person’s employer, and shall include in such a notification a statement of the liable person’s earnings and expected earnings from the employment concerned and of such other matters as may be prescribed.
\end{enumerate}

(4) The regulations may include provision with respect to the priority as between a deduction from earnings order and---
\begin{enumerate}\item[]
($a$) any other deduction from earnings order;

($b$) any order under any other enactment relating to England and Wales which provides for deductions from the liable person’s earnings;

($c$) any diligence against earnings.
\end{enumerate}

(5) The regulations may include a provision that a liable person may appeal to a magistrates’ court (or in Scotland to the sheriff) if he is aggrieved by the making of a deduction from earnings order against him, or by the terms of any such order,
or there is a dispute as to whether payments constitute earnings or as to any other prescribed matter relating to the order.

(6) On an appeal under subsection (5) the court or (as the case may be) the sheriff shall not question the \opt{oldrules}{maintenance assessment}\opt{newrules,2012rules}{maintenance calculation} by reference to which the deduction from earnings order was made.

(7) Regulations made by virtue of subsection (5) may include---
\begin{enumerate}\item[]
($a$) provision with respect to the period within which a right of appeal under the
regulations may be exercised;

($b$) provision as to the powers of a magistrates’ court, or in Scotland of the
sheriff, in relation to an appeal (which may include provision as to the
quashing of a deduction from earnings order or the variation of the terms of
such an order).
\end{enumerate}

(8)
If any person fails to comply with the requirements of a deduction from earnings order, or with any regulation under this section which is designated for the purposes of this subsection, he shall be guilty of an offence.

(9) In subsection (8) “designated” means designated by the regulations.

(10)
It shall be a defence for a person charged with an offence under subsection (8) to prove that he took all reasonable steps to comply with the requirements in question.

(11)
Any person guilty of an offence under subsection (8) shall be liable on summary conviction to a fine not exceeding level two on the standard scale.

\amendment{
\opt{newrules,2012rules}{Words “maintenance assessment(s)” substituted by “maintenance calculation(s)” (3.3.03) for the purposes of certain cases only (see S.I. 2003/192) by the Child Support, Pensions and Social Security Act 2000 (c. 19) s. 1(2)($a$).

S. 32(2)($bb$) inserted (3.3.03) for the purposes of certain cases only (see S.I. 2003/192) by the Child Support, Pensions and Social Security Act 2000 (c. 19) Sch. 3 para. 11(16).}

S. 32(7)($a$) inserted (1.6.09) by the Child Maintenance and Other Payments Act 2008 (c. 6) Sch. 7 para. 1(9).

Words substituted in s. 32 (1.8.12) by the Public Bodies (Child Maintenance and Enforcement Commission: Abolition and Transfer of Functions) Order 2012 Sch. para. 33.

The substitution of words in s. 32(2)(i) by the Child Maintenance and Other Payments Act 2008 (c. 6) Sch. 7 para. 1(8) is not yet in force.

The repeal of s. 32(2)($a$) by the Child Maintenance and Other Payments Act 2008 (c. 6) Sch. 8 is not yet in force.

}

\subsection{32A. Orders for regular deductions from accounts}

(1) If in relation to any person it appears to the Secretary of State---
\begin{enumerate}\item[]
($a$) that the person has failed to pay an amount of child support maintenance;
and

($b$) that the person holds an account with a deposit-taker;
\end{enumerate}
the Secretary of State may make an order against that person to secure the payment of any amount due under the maintenance calculation in question by means of regular deductions from the account.

(2) An order under this section may be made so as to secure the payment of---
\begin{enumerate}\item[]
($a$) arrears of child support maintenance payable under the calculation;

($b$) amounts of child support maintenance which will become payable under the
calculation; or

($c$) both such arrears and such future amounts.
\end{enumerate}

(3) An order under this section may be made in respect of amounts due under a maintenance calculation which is the subject of an appeal only if it appears to the Secretary of State---
\begin{enumerate}\item[]
($a$) that liability for the amounts would not be affected were the appeal to succeed;
or

($b$) where paragraph ($a$) does not apply, that the making of an order under this
section in respect of the amounts would nonetheless be fair in all the
circumstances.
\end{enumerate}

(4) An order under this section---
\begin{enumerate}\item[]
($a$) may not be made in respect of an account of a prescribed description; and

($b$) may be made in respect of a joint account which is held by the person against whom the order is made and one or more other persons, and which is not of a description prescribed under paragraph ($a$), if (but only if) regulations made by the Secretary of State so provide.
\end{enumerate}

(5) An order under this section---
\begin{enumerate}\item[]
($a$) shall specify the account in respect of which it is made;

($b$) shall be expressed to be directed at the deposit-taker with which the account is held; and

($c$) shall have effect from such date as may be specified in the order.
\end{enumerate}

(6) An order under this section shall operate as an instruction to the deposit-taker at which it is directed to---
\begin{enumerate}\item[]
($a$) make deductions from the amount (if any) standing to the credit of the account specified in the order; and

($b$) pay the amount deducted to the Secretary of State.
\end{enumerate}

(7) The Secretary of State shall serve a copy of any order made under this section on---
\begin{enumerate}\item[]
($a$) the deposit-taker at which it is directed;

($b$) the person against whom it is made; and

($c$) if the order is made in respect of a joint account, the other account-holders.
\end{enumerate}

(8)
Where---
\begin{enumerate}\item[]
($a$)
an order under this section has been made; and

($b$)
a copy of the order has been served on the deposit-taker at which it is directed, 
\end{enumerate}
it shall be the duty of that deposit-taker to comply with the order; but the deposit-taker shall not be under any liability for non-compliance before the end of the period of 7 days beginning with the day on which the copy was served on the deposit-taker.

(9)
Where regulations have been made under section 29(3)($a$), a person liable to pay an amount of child support maintenance is to be taken for the purposes of this section to have failed to pay an amount of child support maintenance unless it is paid to or through the person specified in, or by virtue of, the regulations for the case in question.

\amendment{
S. 32A inserted (1.6.09 for the purpose of making regulations, 3.8.09 for all other purposes) by the Child Maintenance and Other Payments Act 2008 (c. 6)
s. 22.

\opt{oldrules}{Under the Child Maintenance and Other Payments Act 2008 (c. 6) s. 59, in this section “child support maintenance” is to be read as a reference to periodical payments required to be paid in accordance with a maintenance assessment under the Act, and ``maintenance calculation'' is to be read as a reference to a maintenance assessment.
}

Words substituted in s. 32A (1.8.12) by the Public Bodies (Child Maintenance and Enforcement Commission: Abolition and Transfer of Functions) Order 2012 Sch. para. 34.
}

\subsection{32B. Orders under section 32A: joint accounts}

(1) Before making an order under section 32A in respect of a joint account 
the Secretary of State shall offer each of the account-holders an opportunity to make representations about---
\begin{enumerate}\item[]
($a$) the proposal to make the order; and

($b$) the amounts to be deducted under the order, if it is made.
\end{enumerate}

(2)
The amounts to be deducted from a joint account under such an order shall not exceed the amounts that appear to the Secretary of State to be fair in all the circumstances.

(3)
In determining those amounts the Secretary of State shall have particular regard to---
\begin{enumerate}\item[]
($a$) any representations made in accordance with subsection (1)($b$);

($b$) the amount contributed to the account by each of the account-holders; and

($c$) such other matters as may be prescribed.
\end{enumerate}

\amendment{
S. 32B inserted (1.6.09 for the purpose of making regulations, 3.8.09 for all other purposes) by the Child Maintenance and Other Payments Act 2008 (c. 6)
s. 22.

Words substituted in s. 32B (1.8.12) by the Public Bodies (Child Maintenance and Enforcement Commission: Abolition and Transfer of Functions) Order 2012 Sch. para. 35.
}

\subsection{32C. Regulations about orders under section 32A}

(1) The Secretary of State may by regulations make provision with respect
to orders under section 32A.

(2) Regulations under subsection (1) may, in particular, make provision---
\begin{enumerate}\item[]
($a$) requiring an order to specify the amount or amounts in respect of which it is made;

($b$) requiring an order to specify the amounts which are to be deducted under it
in order to meet liabilities under the maintenance calculation in question;

($c$) requiring an order to specify the dates on which deductions are to be made
under it;

($d$) for the rate of deduction under an order not to exceed such rate as may be
specified in, or determined in accordance with, the regulations;

($e$) as to circumstances in which amounts standing to the credit of an account
are to be disregarded for the purposes of section 32A;

($f$) as to the payment of sums deducted under an order to the Secretary of
State;

($g$) allowing the deposit-taker at which an order is directed to deduct from the
amount standing to the credit of the account specified in the order a prescribed
amount towards its administrative costs before making any deduction required
by section 32A(6)($a$);

($h$) with respect to notifications to be given to the person against whom an order
is made (and, in the case of an order made in respect of a joint account, to the
other account-holders) of amounts deducted, and amounts paid, under the
order;

($i$) requiring the deposit-taker at which an order is directed to notify the
Secretary of State in the prescribed manner and within a prescribed period---
\begin{enumerate}\item[]
(i)
if the account specified in the order does not exist at the time at which
the order is served on the deposit-taker;

(ii)
of any other accounts held with the deposit-taker at that time by the
person against whom the order is made;
\end{enumerate}

($j$) requiring the deposit-taker at which an order is directed to notify the
Secretary of State in the prescribed manner and within a prescribed period
if, after the time at which the order is served on the deposit-taker---
\begin{enumerate}\item[]
(i) the account specified in the order is closed;

(ii) a new account of any description is opened with the deposit-taker by
the person against whom the order is made;
\end{enumerate}

($k$) as to circumstances in which the deposit-taker at which an order is directed,
the person against whom the order is made and (in the case of an order made
in respect of a joint account) the other account-holders may apply to the
Secretary of State for the Secretary of State to review the order and as to
such a review;

($l$) for the variation of orders;

($m$) similar to that made by section 32A(8), in relation to any variation of an
order;

($n$) for an order to lapse in such circumstances as may be prescribed;

($o$) as to the revival of an order in such circumstances as may be prescribed;

($p$) allowing or requiring an order to be discharged;

($q$) as to the giving of notice by the Secretary of State to the deposit-taker that an
order has lapsed or ceased to have effect.
\end{enumerate}

(3)
The Secretary of State may by regulations make provision with respect to priority as between an order under section 32A and---
\begin{enumerate}\item[]
($a$) any other order under that section;

($b$) any order under any other enactment relating to England and Wales which
provides for deductions from the same account;

($c$) any diligence done in Scotland against the same account.
\end{enumerate}

(4)
The Secretary of State shall by regulations make provision for any person affected to have a right to appeal to a court---
\begin{enumerate}\item[]
($a$) against the making of an order under section 32A;

($b$) against any decision made by the Secretary of State on an application
under regulations made under subsection (2)($k$).
\end{enumerate}

(5)
On an appeal under regulations made under subsection (4)($a$), the court shall not question the maintenance calculation by reference to which the order was made.

(6)
Regulations under subsection (4) may include---
\begin{enumerate}\item[]
($a$) provision with respect to the period within which a right of appeal under the regulations may be exercised;

($b$) provision with respect to the powers of the court to which the appeal under the regulations lies.
\end{enumerate}

\amendment{
S. 32C inserted (1.6.09 for the purpose of making regulations, 3.8.09 for all other purposes) by the Child Maintenance and Other Payments Act 2008 (c. 6)
s. 22.

\opt{oldrules}{Under the Child Maintenance and Other Payments Act 2008 (c. 6) s. 59(5), in this section 
%“child support maintenance” is to be read as a reference to periodical payments required to be paid in accordance with a maintenance assessment under the Act, and 
``maintenance calculation'' is to be read as a reference to a maintenance assessment.
}

Words substituted in s. 32C (1.8.12) by the Public Bodies (Child Maintenance and Enforcement Commission: Abolition and Transfer of Functions) Order 2012 Sch. para. 36.

}

\subsection{32D. Orders under section 32A: offences}

(1) A person who fails to comply with the requirements of---
\begin{enumerate}\item[]
($a$) an order under section 32A, or

($b$) any regulation under section 32C which is designated by the regulations for
the purposes of this paragraph, 
\end{enumerate}
commits an offence.

(2)
It shall be a defence for a person charged with an offence under subsection (1) to prove that the person took all reasonable steps to comply with the requirements in question.

(3)
A person guilty of an offence under subsection (1) shall be liable on summary conviction to a fine not exceeding level two on the standard scale.

\amendment{
S. 32D inserted (1.6.09 for the purpose of making regulations, 3.8.09 for all other purposes) by the Child Maintenance and Other Payments Act 2008 (c. 6)
s. 22.
}

\subsection{32E. Lump sum deductions: interim orders}

(1) The Secretary of State may make an order under this section if it appears to the Secretary of State that a person (referred to in this section and sections 32F to 32J as “the liable person”) has failed to pay an amount of child support maintenance and---
\begin{enumerate}\item[]
($a$) an amount stands to the credit of an account held by the liable person with a deposit-taker; or

($b$) an amount not within paragraph ($a$) that is of a prescribed description is due or accruing to the liable person from another person (referred to in this section and sections 32F to 32J as the “third party”).
\end{enumerate}

(2) An order under this section---
\begin{enumerate}\item[]
($a$) may not be made by virtue of subsection (1)($a$) in respect of an account of a prescribed description; and

($b$) may be made by virtue of subsection (1)($a$) in respect of a joint account which is held by the liable person and one or more other persons, and which is not of a description prescribed under paragraph ($a$) of this subsection, if (but only if) regulations made by the Secretary of State so provide.
\end{enumerate}

(3)
The Secretary of State may by regulations make provision as to conditions that are to be disregarded in determining whether an amount is due or accruing to the liable person for the purposes of subsection (1)($b$).

(4)
An order under this section---
\begin{enumerate}\item[]
($a$) shall be expressed to be directed at the deposit-taker or third party in question;

($b$) if made by virtue of subsection (1)($a$), shall specify the account in respect of which it is made; and

($c$) shall specify the amount of arrears of child support maintenance in respect of which the Secretary of State proposes to make an order under section 32F.
\end{enumerate}

(5) An order under this section may specify an amount of arrears due under a maintenance calculation which is the subject of an appeal only if it appears to the Secretary of State---
\begin{enumerate}\item[]
($a$) that liability for the amount would not be affected were the appeal to succeed; or

($b$) where paragraph ($a$) does not apply, that the making of an order under section 32F in respect of the amount would nonetheless be fair in all the circumstances.
\end{enumerate}

(6) The Secretary of State shall serve a copy of any order made under this section on---
\begin{enumerate}\item[]
($a$)
the deposit-taker or third party at which it is directed;

($b$)
the liable person; and

($c$)
if the order is made in respect of a joint account, the other account-holders.
\end{enumerate}

(7) An order under this section shall come into force at the time at which it is served on the deposit-taker or third party at which it is directed.

(8) An order under this section shall cease to be in force at the earliest of the following---
\begin{enumerate}\item[]
($a$)
the time at which the prescribed period ends;

($b$)
the time at which the order under this section lapses or is discharged; and

($c$)
the time at which an order under section 32F made in pursuance of the proposal specified in the order under this section is served on the deposit-taker or third party at which that order is directed.
\end{enumerate}

(9) Where regulations have been made under section 29(3)($a$), a person liable to pay an amount of child support maintenance is to be taken for the purposes of this section to have failed to pay the amount unless it is paid to or through the person specified in, or by virtue of, the regulations for the case in question.


\amendment{
S. 32E inserted (1.6.09 for the purpose of making regulations, 3.8.09 for all other purposes) by the Child Maintenance and Other Payments Act 2008 (c. 6)
s. 23.

\opt{oldrules}{Under the Child Maintenance and Other Payments Act 2008 (c. 6) s. 59, in this section 
“child support maintenance” is to be read as a reference to periodical payments required to be paid in accordance with a maintenance assessment under the Act, and 
``maintenance calculation'' is to be read as a reference to a maintenance assessment.
}

Words substituted in s. 32E (1.8.12) by the Public Bodies (Child Maintenance and Enforcement Commission: Abolition and Transfer of Functions) Order 2012 Sch. para. 37.

}

\subsection{32F. Lump sum deductions: final orders}

(1) The Secretary of State may make an order under this section in pursuance of a proposal specified in an order under section 32E if---
\begin{enumerate}\item[]
($a$)
the order in which the proposal was specified (“the interim order”) is in force;

($b$)
the period prescribed for the making of representations to the Secretary of State in respect of the proposal specified in the interim order has expired; and

($c$)
the Secretary of State has considered any representations made to the Secretary of State during that period.
\end{enumerate}

(2) An order under this section---
\begin{enumerate}\item[]
($a$)
shall be expressed to be directed at the deposit-taker or third party at which the interim order was directed;

($b$)
if the interim order was made by virtue of section 32E(1)($a$), shall specify the account specified in the interim order; and

($c$)
shall specify the amount of arrears of child support maintenance in respect of which it is made.
\end{enumerate}

(3) The amount so specified---
\begin{enumerate}\item[]
($a$) shall not exceed the amount of arrears specified in the interim order which remain unpaid at the time at which the order under this section is made; and

($b$) if the order is made in respect of a joint account, shall not exceed the amount that appears to the Commission to be fair in all the circumstances.
\end{enumerate}

(4)
In determining the amount to be specified in an order made in respect of a joint account the Secretary of State shall have particular regard---
\begin{enumerate}\item[]
($a$) to the amount contributed to the account by each of the account-holders; and

($b$) to such other matters as may be prescribed.
\end{enumerate}

(5)
An order under this section may specify an amount of arrears due under a maintenance calculation which is the subject of an appeal only if it appears to the Secretary of State---
\begin{enumerate}\item[]
($a$) that liability for the amount would not be affected were the appeal to succeed; or

($b$) where paragraph ($a$) does not apply, that the making of an order under this section in respect of the amount would nonetheless be fair in all the circumstances.
\end{enumerate}

(6) The Secretary of State shall serve a copy of any order made under this section on---
\begin{enumerate}\item[]
($a$) the deposit-taker or third party at which it is directed;

($b$) the liable person; and

($c$) if the order is made in respect of a joint account, the other account-holders.
\end{enumerate}

\amendment{
S. 32F inserted (1.6.09 for the purpose of making regulations, 3.8.09 for all other purposes) by the Child Maintenance and Other Payments Act 2008 (c. 6)
s. 23.

\opt{oldrules}{Under the Child Maintenance and Other Payments Act 2008 (c. 6) s. 59, in this section 
“child support maintenance” is to be read as a reference to periodical payments required to be paid in accordance with a maintenance assessment under the Act, and 
``maintenance calculation'' is to be read as a reference to a maintenance assessment.
}

Words substituted in s. 32F (1.8.12) by the Public Bodies (Child Maintenance and Enforcement Commission: Abolition and Transfer of Functions) Order 2012 Sch. para. 38.
}

\subsection{32G. Orders under sections 32E and 32F: freezing of accounts etc.}

(1) During the relevant period, an order under section 32E or 32F which
specifies an account held with a deposit-taker shall operate as an instruction to the deposit-taker not to do anything that would reduce the amount standing to the credit of the account below the amount specified in the order (or, if already below that amount, that would further reduce it).

(2)
During the relevant period, any other order under section 32E or 32F shall operate as an instruction to the third party at which it is directed not to do anything that would reduce the amount due to the liable person below the amount specified in the order (or, if already below that amount, that would further reduce it).

(3)
Subsections (1) and (2) have effect subject to regulations made under section 32I(1).

(4)
In this section “the relevant period”, in relation to an order under section 32E, means the period during which the order is in force.

(5)
In this section and section 32H “the relevant period”, in relation to an order under section 32F, means the period which---
\begin{enumerate}\item[]
($a$) begins with the service of the order on the deposit-taker or third party at which it is directed; and

($b$) (subject to subsection (6)) ends with the end of the period during which an appeal can be brought against the order by virtue of regulations under section 32J(5).
\end{enumerate}

(6) If an appeal is brought by virtue of the regulations, the relevant period ends at the time at which---
\begin{enumerate}\item[]
($a$) proceedings on the appeal (including any proceedings on a further appeal) have been concluded; and

($b$) any period during which a further appeal may ordinarily be brought has ended.
\end{enumerate}

(7) References in this section and sections 32H and 32J to the amount due to the liable person are to be read as references to the total of any amounts within section 32E(1)($b$) that are due or accruing to the liable person from the third party in question.

\amendment{
S. 32G inserted (1.6.09 for the purpose of making regulations, 3.8.09 for all other purposes) by the Child Maintenance and Other Payments Act 2008 (c. 6)
s. 23.
}

\subsection{32H. Orders under section 32F: deductions and payments}

(1) Once the relevant period has ended, an order under section 32F which specifies an account held with a deposit-taker shall operate as an instruction to the deposit-taker---
\begin{enumerate}\item[]
($a$) if the amount standing to the credit of the account is less than the remaining amount, to pay to the Secretary of State the amount standing to the credit of the account; and

($b$) otherwise, to deduct from the account and pay to the Secretary of State the remaining amount.
\end{enumerate}

(2) If an amount of arrears specified in the order remains unpaid after any payment required by subsection (1) has been made, the order shall operate until the relevant time as an instruction to the deposit-taker---
\begin{enumerate}\item[]
($a$) to pay to the Secretary of State any amount (not exceeding the remaining
amount) standing to the credit of the account specified in the order; and

($b$) not to do anything else that would reduce the amount standing to the credit
of the account.
\end{enumerate}

(3) Once the relevant period has ended, any other order under section 32F shall
operate as an instruction to the third party at which it is directed---
\begin{enumerate}\item[]
($a$) if the amount due to the liable person is less than the remaining amount, to
pay to the Secretary of State the amount due to the liable person; and

($b$) otherwise, to deduct from the amount due to the liable person and pay to the
Secretary of State the remaining amount.
\end{enumerate}

(4) If an amount of arrears specified in the order remains unpaid after any payment
required by subsection (3) has been made, the order shall operate until the relevant
time as an instruction to the third party---
\begin{enumerate}\item[]
($a$) to pay to the Secretary of State any amount (not exceeding the remaining
amount) due to the liable person; and

($b$) not to do anything else that would reduce any amount due to the liable
person.
\end{enumerate}

(5) This section has effect subject to regulations made under sections 32I(1) and
32J(2)($c$).

(6) In this section---
\begin{enumerate}\item[]
“the relevant time” means the earliest of the following---
\begin{enumerate}\item[]
($a$) the time at which the remaining amount is paid;

($b$) the time at which the order lapses or is discharged; and

($c$) the time at which a prescribed event occurs or prescribed circumstances
arise;
\end{enumerate}

“the remaining amount”, in relation to any time, means the amount of arrears
specified in the order under section 32F which remains unpaid at that time.
\end{enumerate}

\amendment{
S. 32H inserted (1.6.09 for the purpose of making regulations, 3.8.09 for all other purposes) by the Child Maintenance and Other Payments Act 2008 (c. 6)
s. 23.

Words substituted in s. 32H (1.8.12) by the Public Bodies (Child Maintenance and Enforcement Commission: Abolition and Transfer of Functions) Order 2012 Sch. para. 39.
}

\subsection{\sloppy 32I. Power to disapply sections 32G(1) and (2) and 32H(2)($b$) and (4)($b$)}

(1) The Secretary of State may by regulations make provision as to
circumstances in which things that would otherwise be in breach of sections 32G(1) and (2) and 32H(2)($b$) and (4)($b$) may be done.

(2) Regulations under subsection (1) may require the Secretary of State’s consent
to be obtained in prescribed circumstances.

(3) Regulations under subsection (1) which require the Secretary of State’s consent
to be obtained may provide for an application for that consent to be made---
\begin{enumerate}\item[]
($a$) by the deposit-taker or third party at which the order under section 32E or
32F is directed;

($b$) by the liable person; and

($c$) if the order is made in respect of a joint account, by any of the other account-holders.
\end{enumerate}

(4) If regulations under subsection (1) require the Secretary of State’s consent to
be obtained, the Secretary of State shall by regulations provide for a person of a
prescribed description to have a right of appeal to a court against the withholding of
that consent.

(5) Regulations under subsection (4) may include---
\begin{enumerate}\item[]
($a$) provision with respect to the period within which a right of appeal under the
regulations may be exercised;

($b$) provision with respect to the powers of the court to which the appeal under
the regulations lies.
\end{enumerate}

\amendment{
S. 32I inserted (1.6.09 for the purpose of making regulations, 3.8.09 for all other purposes) by the Child Maintenance and Other Payments Act 2008 (c. 6)
s. 23.

Words substituted in s. 32I (1.8.12) by the Public Bodies (Child Maintenance and Enforcement Commission: Abolition and Transfer of Functions) Order 2012 Sch. para. 40.
}

\subsection{32J. Regulations about orders under section 32E or 32F}

(1) The Secretary of State may by regulations make provision with respect to orders under section 32E or 32F.

(2) The regulations may, in particular, make provision---
\begin{enumerate}\item[]
($a$) as to circumstances in which amounts standing to the credit of an account are to be disregarded for the purposes of sections 32E, 32G and 32H;

($b$) as to the payment to the Secretary of State of sums deducted under an order under section 32F;

($c$) allowing a deposit-taker or third party at which an order under section 32F is directed to deduct from the amount standing to the credit of the account specified in the order, or due to the liable person, a prescribed amount towards its administrative costs before making any payment to the Secretary of State required by section 32H;

($d$) with respect to notifications to be given to the liable person (and, in the case of an order made in respect of a joint account, to the other account-holders) as to amounts deducted, and amounts paid, under an order under section 32F;

($e$) requiring a deposit-taker or third party at which an order under section 32E or 32F is directed to supply information of a prescribed description to the Secretary of State, or to notify the Secretary of State if a prescribed event occurs or prescribed circumstances arise;

($f$) for the variation of an order under section 32E or 32F;

($g$) for an order under section 32E or 32F to lapse in such circumstances as may be prescribed;

($h$) as to the revival of an order under section 32E or 32F in such circumstances as may be prescribed;

($i$) allowing or requiring an order under section 32E or 32F to be discharged.
\end{enumerate}

(3)
Where regulations under subsection (1) make provision for the variation of an order under section 32E or 32F, the power to vary the order shall not be exercised so as to increase the amount of arrears of child support maintenance specified in the order.

(4)
The Secretary of State may by regulations make provision with respect to priority as between an order under section 32F and---
\begin{enumerate}\item[]
($a$) any other order under that section;

($b$) any order under any other enactment relating to England and Wales which provides for payments to be made from amounts to which the order under section 32F relates;

($c$) any diligence done in Scotland against amounts to which the order under section 32F relates.
\end{enumerate}

(5)
The Secretary of State shall by regulations make provision for any person affected by an order under section 32F to have a right to appeal to a court against the making of the order.

(6)
On an appeal under regulations under subsection (5), the court shall not question the maintenance calculation by reference to which the order under section 32F was made.

(7)
Regulations under subsection (5) may include---
\begin{enumerate}\item[]
($a$) provision with respect to the period within which a right of appeal under the regulations may be exercised;

($b$) provision with respect to the powers of the court to which the appeal under the regulations lies.
\end{enumerate}

\amendment{
S. 32J inserted (1.6.09 for the purpose of making regulations, 3.8.09 for all other purposes) by the Child Maintenance and Other Payments Act 2008 (c. 6)
s. 23.

\opt{oldrules}{Under the Child Maintenance and Other Payments Act 2008 (c. 6) s. 59, in this section 
“child support maintenance” is to be read as a reference to periodical payments required to be paid in accordance with a maintenance assessment under the Act, and 
``maintenance calculation'' is to be read as a reference to a maintenance assessment.
}

Words substituted in s. 32J(2) (1.8.12) by the Public Bodies (Child Maintenance and Enforcement Commission: Abolition and Transfer of Functions) Order 2012 Sch. para. 41.
}

\subsection{32K. Lump sum deduction orders: offences}

(1) A person who fails to comply with the requirements of---
\begin{enumerate}\item[]
($a$) an order under section 32E or 32F; or

($b$) any regulation under section 32J which is designated by the regulations for the purposes of this paragraph,
\end{enumerate}
commits an offence.

(2)
It shall be a defence for a person charged with an offence under subsection (1) to prove that the person took all reasonable steps to comply with the requirements in question.

(3)
A person guilty of an offence under subsection (1) shall be liable on summary conviction to a fine not exceeding level two on the standard scale.

\amendment{
S. 32K inserted (1.6.09 for the purpose of making regulations, 3.8.09 for all other purposes) by the Child Maintenance and Other Payments Act 2008 (c. 6)
s. 23.
}

\subsection{32L. Orders preventing avoidance}

(1) The Secretary of State may apply to the court, on the grounds that a person---
\begin{enumerate}\item[]
($a$) has failed to pay an amount of child support maintenance, and

($b$) with the intention of avoiding payment of child support maintenance, is about to make a disposition or to transfer out of the jurisdiction or otherwise deal with any property,
\end{enumerate}
for an order restraining or, in Scotland, interdicting the person from doing so.

(2) The Secretary of State may apply to the court, on the grounds that a person---
\begin{enumerate}\item[]
($a$) has failed to pay an amount of child support maintenance, and

($b$) with the intention of avoiding payment of child support maintenance, has at any time made a reviewable disposition,
\end{enumerate}
for an order setting aside or, in Scotland, reducing the disposition.

(3)
If the court is satisfied of the grounds mentioned in subsection (1) or (2) it may make an order under that subsection.

(4)
Where the court makes an order under subsection (1) or (2) it may make such consequential provision by order or directions as it thinks fit for giving effect to the order (including provision requiring the making of any payments or the disposal of any property).

(5)
Any disposition is a reviewable disposition for the purposes of subsection (2), unless it was made for valuable or, in Scotland, adequate consideration (other than marriage) to a person who, at the time of the disposition, acted in relation to it in good faith and without notice of an intention to avoid payment of child support maintenance.

(6)
Subsection (7) applies where an application is made under this section with respect to---
\begin{enumerate}\item[]
($a$) a disposition or other dealing with property which is about to take place, or

($b$) a disposition which took place after the making of the application on which the maintenance calculation concerned was made.
\end{enumerate}

(7) If the court is satisfied---
\begin{enumerate}\item[]
($a$) in a case falling within subsection (1), that the disposition or other dealing would (apart from this section) have the consequence of making ineffective a step that has been or may be taken to recover the amount outstanding, or

($b$) in a case falling within subsection (2), that the disposition has had that consequence,
\end{enumerate}
it is to be presumed, unless the contrary is shown, that the person who disposed of or is about to dispose of or deal with the property did so or, as the case may be, is about to do so, with the intention of avoiding payment of child support maintenance.

(8)
In this section "disposition" does not include any provision contained in a will or codicil but, with that exception, includes any conveyance, assurance or gift of property of any description, whether made by an instrument or otherwise.

(9)
This section does not apply to a disposition made before the coming into force of section 24 of the Child Maintenance and Other Payments Act 2008.

(10)
In this section "the court" means---
\begin{enumerate}\item[]
($a$) in relation to England and Wales, the High Court
or the family court%  % Words inserted (22.4.14) by 2013 c 22 Sch 11 para 124
;

($b$) in relation to Scotland, the Court of Session or the sheriff.
\end{enumerate}

(11) An order under this section interdicting a person---
\begin{enumerate}\item[]
($a$) is effective for such period (including an indefinite period) as the order may specify;

($b$) may, on application to the court, be varied or recalled.
\end{enumerate}

\amendment{
S. 32L inserted (6.4.10) by the Child Maintenance and Other Payments Act 2008 (c. 6)
s. 24.

\opt{oldrules}{Under the Child Maintenance and Other Payments Act 2008 (c. 6) s. 59, in this section 
“child support maintenance” is to be read as a reference to periodical payments required to be paid in accordance with a maintenance assessment under the Act, and 
``maintenance calculation'' is to be read as a reference to a maintenance assessment.
}

Words substituted in s. 32L (1.8.12) by the Public Bodies (Child Maintenance and Enforcement Commission: Abolition and Transfer of Functions) Order 2012 Sch. para. 42.

Words substituted in s. 32L(10) (24.4.14) by the Crime and Courts Act 2013 Sch. 11. para. 124.

\medskip

The insertions of ss. 32M, 32N by the Child Maintenance and Other Payments Act 2008 (c. 6) s. 25 are not yet in force.
}

\subsection{33. Liability orders}

(1) This section applies where---
\begin{enumerate}\item[]
($a$) the person who is liable to make payments of child support maintenance (“the liable person”) fails to make one or more of those payments; and

($b$) it appears to the Secretary of State that---
\begin{enumerate}\item[]
(i)
it is inappropriate to make a deduction from earnings order against him (because, for example, he is not employed); or

(ii)
although such an order has been made against him, it has proved ineffective as a means of securing that payments are made in accordance with the \opt{oldrules}{maintenance assessment}\opt{newrules,2012rules}{maintenance calculation} in question.
\end{enumerate}
\end{enumerate}

(2)
The Secretary of State may apply to a magistrates’ court or, in Scotland, to the sheriff for an order (“the liability order”) against the liable person.

(3)
Where the Secretary of State applies for a liability order, the magistrates’ court or (as the case may be) sheriff shall make the order if satisfied that the payments in question have become payable by the liable person and have not been paid.

(4)
On an application under subsection (2), the court or (as the case may be) the sheriff shall not question the \opt{oldrules}{maintenance assessment}\opt{newrules,2012rules}{maintenance calculation} under which the payments of child support maintenance fell to be made.

(5) If the Secretary of State designates a liability order for the purposes of this subsection it shall be treated as a judgment entered in a county court for the purposes of section 98 of the Courts Act 2003 (register of judgments and orders etc).

(6) Where regulations have been made under section 29(3)($a$)---
\begin{enumerate}\item[]
($a$) the liable person fails to make a payment (for the purposes of subsection (1)($a$) of this section); and

($b$)
a payment is not paid (for the purposes of subsection (3)),
\end{enumerate}
unless the payment is made to, or though, the person specified in or by virtue of those regulations for the case of the liable person in question.

\amendment{
S. 33(5) inserted (4.9.95) by the Child Support Act 1995 (c. 34) Sch. 3 para. 10.

S. 33(6) inserted (1.1.01) by the Child Support, Pensions and Social Security Act 2000 (c. 19) Sch. 3 para. 11(17).

\opt{newrules,2012rules}{Words ``maintenance assessment(s)'' substituted by ``maintenance calculation(s)'' (3.3.03) for the purposes of certain cases only (see S.I. 2003/192) by the Child Support, Pensions and Social Security Act 2000 (c. 19) s. 1(2)($a$).}

Words substituted in s. 33(5) (6.4.06) by the Courts Act 2003 (Consequential Amendment) Order 2006 art. 2.

Words substituted in s. 33 (1.8.12) by the Public Bodies (Child Maintenance and Enforcement Commission: Abolition and Transfer of Functions) Order 2012 Sch. para. 43.

The repeal of s. 33 by the Child Maintenance and Other Payments Act 2008 (c. 6) Sch. 8 is not yet in force.
}

\subsection{34. Regulations about liability orders}

(1) The Secretary of State may make regulations in relation to England and Wales---
\begin{enumerate}\item[]
($a$)
prescribing the procedure to be followed in dealing with an application by the Secretary of State for a liability order;

($b$)
prescribing the form and contents of a liability order; and

($c$)
providing that where a magistrates’ court has made a liability order, the person against whom it is made shall, during such time as the amount in respect of which the order was made remains wholly or partly unpaid, be under a duty to supply relevant information to the Secretary of State.
\end{enumerate}

\amendment{
Words substituted in s. 34 (1.8.12) by the Public Bodies (Child Maintenance and Enforcement Commission: Abolition and Transfer of Functions) Order 2012 Sch. para. 44.

S. 34(2) is not yet in force.

The repeal of s. 34 by the Child Maintenance and Other Payments Act 2008 (c. 6) Sch. 8 is not yet in force.
}

\subsection{35. Enforcement of liability orders by distress}

(1) Where a liability order has been made against a person (“the liable person”), the Secretary of State may levy the appropriate amount by distress and sale of the liable person’s goods.

(2) In subsection (1), “the appropriate amount” means the aggregate of---
\begin{enumerate}\item[]
($a$)
the amount in respect of which the order was made, to the extent that it remains unpaid; and

($b$)
an amount, determined in such manner as may be prescribed, in respect of the charges connected with the distress.
\end{enumerate}

(3) The Secretary of State may, in exercising the powers under subsection (1) against the liable person’s goods, seize---
\begin{enumerate}\item[]
($a$)
any of the liable person’s goods except---
\begin{enumerate}\item[]
(i) such tools, books, vehicles and other items of equipment as are necessary to him for use personally by him in his employment, business or vocation;

(ii) such clothing, bedding, furniture, household equipment and provisions as are necessary for satisfying his basic domestic needs; and
\end{enumerate}
($b$)
any money, banknotes, bills of exchange, promissory notes, bonds, specialities or securities for money belonging to the liable person.
\end{enumerate}

(4) For the purposes of subsection (3), the liable person’s domestic needs shall be 
taken to include those of any member of his family with whom he resides

(5) No person levying a distress under this section shall be taken to be a trespasser---
\begin{enumerate}\item[]
 ($a$) on that account; or 

($b$) from the beginning, on account of any subsequent irregularity in levying the distress.
\end{enumerate}

(6) A person sustaining special damage by reason of any irregularity in levying a distress under this section may recover full satisfaction for the damage (and no more) by proceedings in trespass or otherwise.

(7) The Secretary of State may make regulations supplementing the provisions of this section.

(8) The regulations may, in particular---
\begin{enumerate}\item[]
 ($a$) provide that a distress under this section may be levied anywhere in England and Wales;

($b$) provide that such a distress shall not be deemed unlawful on account of any defect or want of form in the liability order;

($c$) provide for an appeal to a magistrates’ court by any person aggrieved by the levying of, or an attempt to levy, a distress under this section;

($d$) make provision as to the powers of the court on an appeal (which may include provision as to the discharge of goods distrained or the payment of compensation in respect of goods distrained and sold).
\end{enumerate}

\amendment{

Words substituted in s. 35 (1.8.12) by the Public Bodies (Child Maintenance and Enforcement Commission: Abolition and Transfer of Functions) Order 2012 Sch. para. 45.

The substitutions in s. 35(1) and in the heading and the repeal of s. 35(2)--(8) by the Tribunals, Courts and Enforcement Act 2007 (c. 15) Sch. 13 para. 94 are not yet in force.

}

\subsection{36. Enforcement in county courts}

(1) Where a liability order has been made against a person, the amount in respect of which the order was made, to the extent that it remains unpaid, shall, if a county court so orders, be recoverable by means of a third party debt order or a charging order, as if it were payable under a county court order.

\amendment{
Words substituted in s. 36(1) and s. 36(2) repealed (1.6.09) by the Child Maintenance and Other Payments Act 2008 (c. 6) Sch. 7 para. 1(10).

The repeal of words in s. 36(1) by the Child Maintenance and Other Payments Act 2008 (c. 6) s. 26 is not yet in force.
}

\subsection{37. Regulations about liability orders: Scotland}

(1) Section 34(1) does not apply to Scotland.

\amendment{
S. 37(2), (3) are not yet in force.

The repeal of s. 37 by the Child Maintenance and Other Payments Act 2008 (c. 6) Sch. 8 is not yet in force.
}

\subsection{38. Enforcement of liability orders by diligence: Scotland}

(1) In Scotland, where a liability order has been made against a person, the order shall be warrant anywhere in Scotland---
\begin{enumerate}\item[]
($a$) for the Secretary of State to charge the person to pay the appropriate amount and to recover that amount by an attachment and, in connection therewith, for the opening of shut and lockfast places;

($b$) for an arrestment (other than an arrestment of the person’s earnings in the hands of his employers) and action of furthcoming or sale;

($c$) for an inhibition.
\end{enumerate}
%and shall be apt to found a Bill of Inhibition or an action of adjudication at the instance of the Secretary of State.

(2) In subsection (1) the “appropriate amount” means the amount in respect of which the order was made, to the extent that it remains unpaid.

\amendment{
Words substituted in s. 38(1)($a$) (30.12.02) by the Debt Arrangement and Attachment (Scotland) Act 2002 (asp 17) Sch. 3 para. 20.

S. 38(1)($c$) substituted (22.4.09) for certain purposes (see S.S.I. 2009/67) by the Bankruptcy and Diligence etc. (Scotland) Act 2007 (asp 3) Sch. 5 para. 18($a$)(ii).

Words substituted in s. 38 (1.8.12) by the Public Bodies (Child Maintenance and Enforcement Commission: Abolition and Transfer of Functions) Order 2012 Sch. para. 47.

The insertion of s. 38(1)($aa$) by the Bankruptcy and Diligence etc. (Scotland) Act 2007 (asp 3) Sch. 5 para. 18($a$)(i) is not yet in force.

The substitution of s. 38(2) by the Bankruptcy and Diligence etc. (Scotland) Act 2007 (asp 3) Sch. 5 para. 18($b$) is not yet in force.

}

\subsection{39. Liability orders: enforcement throughout United Kingdom}

(1) The Secretary of State may by regulations provide for---
\begin{enumerate}\item[]
($a$) any liability order made by a court in England and Wales; or

($b$) any corresponding order made by a court in Northern Ireland,
\end{enumerate}
to be enforced in Scotland as if it had been made by the sheriff.

(2)
The power conferred on the Court of Session by section 32 of the Sheriff Courts (Scotland) Act 1971 (power of Court of Session to regulate civil procedure in the sheriff court) shall extend to making provision for the registration in the sheriff court for enforcement of any such order as is referred to in subsection (1).

(3)
The Secretary of State may by regulations make provision for, or in connection with, the enforcement in England and Wales of---
\begin{enumerate}\item[]
($a$) any liability order made by the sheriff in Scotland; or

($b$) any corresponding order made by a court in Northern Ireland,
\end{enumerate}
as if it had been made by a magistrates’ court in England and Wales.

(4) Regulations under subsection (3) may, in particular, make provision for the registration of any such order as is referred to in that subsection in connection with its enforcement in England and Wales.

\amendment{
The substitution of s. 39 by the Child Maintenance and Other Payments Act 2008 (c. 6) Sch. 7 para. 1(11) is not yet in force.
}

\subsection{39A. Commitment to prison and disqualification from driving}

(1) Where the Secretary of State has sought---
\begin{enumerate}\item[]
($a$) in England and Wales to levy an amount by distress under this Act; or

($b$) to recover an amount by virtue of section 36 or 38, 
\end{enumerate}
and that amount, or any
portion of it, remains unpaid the Secretary of State may apply to the court
under this section.

(2)
An application under this section is for whichever the court considers appropriate in all the circumstances of---
\begin{enumerate}\item[]
($a$) the issue of a warrant committing the liable person to prison; or

($b$) an order for him to be disqualified from holding or obtaining a driving
licence.
\end{enumerate}

(3)
On any such application the court shall (in the presence of the liable person) inquire as to---
\begin{enumerate}\item[]
($a$) whether he needs a driving licence to earn his living;

($b$) his means; and

($c$) whether there has been wilful refusal or culpable neglect on his part.
\end{enumerate}

(4)
The Secretary of State may make representations to the court as to whether the Secretary of State thinks it more appropriate to commit the liable person to prison or to disqualify him from holding or obtaining a driving licence; and the liable person may reply to those representations.

(5)
In this section and section 40B, “driving licence” means a licence to drive a motor vehicle granted under Part III of the Road Traffic Act 1988.

(6) In this section “the court” means---
\begin{enumerate}\item[]
($a$) in England and Wales, a magistrates’ court;

($b$) in Scotland, the sheriff.
\end{enumerate}

\amendment{
S. 39A inserted (10.11.00 for the purposes of making regulations and Acts of Sederunt, 2.4.01 for all other purposes) by the Child Support, Pensions and Social Security Act 2000 (c. 19) s. 16(1).

Words substituted in s. 39A (1.8.12) by the Public Bodies (Child Maintenance and Enforcement Commission: Abolition and Transfer of Functions) Order 2012 Sch. para. 48.

The substitution of words in s. 39A(1)($a$) by the Tribunals, Courts and Enforcement Act 2007 (c. 15) Sch. 13 para. 95 is not yet in force.

The repeal of s. 39A by the Child Maintenance and Other Payments Act 2008 (c. 6) Sch. 8 is not yet in force.

\medskip

The insertion of ss. 39B--39G by the Child Maintenance and Other Payments Act 2008 (c. 6) s. 27 is not yet in force.

The insertion of ss. 39H--39Q by the Child Maintenance and Other Payments Act 2008 (c. 6) s. 28 is not yet in force.

The insertion of ss. 39CA, 39CB by the Welfare Reform Act 2009 (c. 24) s. 51(4) is not yet in force.

The insertion of s. 39DA by the Welfare Reform Act 2009 (c. 24) s. 51(5) is not yet in force.

The repeal of s. 39G by the Welfare Reform Act 2009 (c. 24) Sch. 5 para. 7 is not yet in force.

}

\subsection{40. Commitment to prison}

(3)
If, but only if, the court is of the opinion that there has been wilful refusal or culpable neglect on the part of the liable person it may---
\begin{enumerate}\item[]
($a$) issue a warrant of commitment against him; or

($b$) fix a term of imprisonment and postpone the issue of the warrant until such
time and on such conditions (if any) as its thinks just.
\end{enumerate}

(4) Any such warrant---
\begin{enumerate}\item[]
($a$) shall be made in respect of an amount equal to the aggregate of---
\begin{enumerate}\item[]
(i) the amount mentioned in section 35(1) or so much of it as remains
outstanding; and

(ii) an amount (determined in accordance with regulations made by the
Secretary of State) in respect of the costs of commitment; and
\end{enumerate}

($b$) shall state that amount.
\end{enumerate}

(5)
No warrant may be issued under this section against a person who is under the age of 18.

(6) A warrant issued under this section shall order the liable person---
\begin{enumerate}\item[]
($a$) to be imprisoned for a specified period; but

($b$) to be released (unless he is in custody for some other reason) on payment of the amount stated in the warrant.
\end{enumerate}

(7)
The maximum period of imprisonment which may be imposed by virtue of subsection (6) shall be calculated in accordance with Schedule 4 to the Magistrates’ Courts Act 1980 (maximum periods of imprisonment in default of payment) but shall not exceed six weeks.

(8)
The Secretary of State may by regulations make provision for the period of imprisonment specified in any warrant issued under this section to be reduced where there is part payment of the amount in respect of which the warrant was issued.

(9)
A warrant issued under this section may be directed to such person or persons as the court issuing it thinks fit.

(10)
Section 80 of the Magistrates’ Courts Act 1980 (application of money found on defaulter) shall apply in relation to a warrant issued under this section against a liable person as it applies in relation to the enforcement of a sum mentioned in subsection (1) of that section.

(11)
The Secretary of State may by regulations make provision---
\begin{enumerate}\item[]
($a$) as to the form of any warrant issued under this section;

($b$) allowing an application under this section to be renewed where no warrant is issued or term of imprisonment is fixed;

($c$) that a statement in writing to the effect that wages of any amount have been paid to the liable person during any period, purporting to be signed by or on behalf of his employer, shall be evidence of the facts stated;

($d$) that, for the purposes of enabling an inquiry to be made as to the liable person’s conduct and means, a justice of the peace may issue a summons to him to appear before a magistrates’ court and (if he does not obey) may issue a warrant for his arrest;

($e$) that for the purpose of enabling such an inquiry, a justice of the peace may issue a warrant for the liable person’s arrest without issuing a summons;

($f$) as to the execution of a warrant for arrest.
\end{enumerate}

(12) This section does not apply to Scotland.


\amendment{
%\opt{oldrules}{Under the Child Maintenance and Other Payments Act 2008 (c. 6) s. 59(5), in this section 
%``maintenance calculation'' is to be read as a reference to a maintenance assessment.
%}

S. 40(1), (2) omitted (10.11.00 for the purposes of making regulations and Acts of Sederunt, 2.4.01 for all other purposes) by the Child Support, Pensions and Social Security Act 2000 (c. 19) s. 16(2).

S. 40(12) substituted for s. 40(12)--(14)  (10.11.00 for the purposes of making regulations and Acts of Sederunt, 2.4.01 for all other purposes) by the Child Support, Pensions and Social Security Act 2000 (c. 19) s. 17(1).

The substitution of s. 40(4)($a$)(i) by the Tribunals, Courts and Enforcement Act 2007 (c. 15) Sch. 13 para. 96 is not yet in force.

The insertion of s. 40(2A)--(2D) by the Child Maintenance and Other Payments Act 2008 (c. 6) s. 29(1) is not yet in force.

The substitution of s. 40(10)--(10C) for s. 40(10) by the Child Maintenance and Other Payments Act 2008 (c. 6) s. 29(2) is not yet in force.

The substitution of words in s. 40(4)($a$)(i) by the Child Maintenance and Other Payments Act 2008 (c. 6) s. 29(1) Sch. 7 para. 1(12) is not yet in force.



}

\subsection{40A. Commitment to prison: Scotland}

(1) If, but only if, the sheriff is satisfied that there has been wilful refusal or culpable neglect on the part of the liable person he may---
\begin{enumerate}\item[]
($a$) issue a warrant for his committal to prison; or

($b$) fix a term of imprisonment and postpone the issue of the warrant until such time and on such conditions (if any) as he thinks just.
\end{enumerate}

(2) A warrant under this section---
\begin{enumerate}\item[]
($a$) shall be made in respect of an amount equal to the aggregate of---
\begin{enumerate}\item[]
(i) the appropriate amount under section 38; and

(ii) an amount (determined in accordance with regulations made by the
Secretary of State) in respect of the expenses of commitment; and
\end{enumerate}

($b$) shall state that amount.
\end{enumerate}

(3) No warrant may be issued under this section against a person who is under the
age of 18.

(4) A warrant issued under this section shall order the liable person---
\begin{enumerate}\item[]
($a$) to be imprisoned for a specified period; but

($b$) to be released (unless he is in custody for some other reason) on payment of
the amount stated in the warrant.
\end{enumerate}

(5) The maximum period of imprisonment which may be imposed by virtue of
subsection (4) is six weeks.

(6) The Secretary of State may by regulations make provision for the period of
imprisonment specified in any warrant issued under this section to be reduced where
there is part payment of the amount in respect of which the warrant was issued.

(7) A warrant issued under this section may be directed to such person as the sheriff
thinks fit.

(8) The power of the Court of Session by Act of Sederunt to regulate the procedure
and practice in civil proceedings in the sheriff court shall include power to make
provision---
\begin{enumerate}\item[]
($a$) as to the form of any warrant issued under this section;

($b$) allowing an application under this section to be renewed where no warrant is
issued or term of imprisonment is fixed;

($c$) that a statement in writing to the effect that wages of any amount have been
paid to the liable person during any period, purporting to be signed by or on
behalf of his employer, shall be sufficient evidence of the facts stated;

($d$) that, for the purposes of enabling an inquiry to be made as to the liable
person’s conduct and means, the sheriff may issue a citation to him to appear
before the sheriff and (if he does not obey) may issue a warrant for his arrest;

($e$) that for the purpose of enabling such an inquiry, the sheriff may issue a
warrant for the liable person’s arrest without issuing a citation;

($f$) as to the execution of a warrant of arrest.
\end{enumerate}

\amendment{
S. 40A inserted (10.11.00 for the purposes of making regulations and Acts of Sederunt, 2.4.01 for all other purposes) by the Child Support, Pensions and Social Security Act 2000 (c. 19) s. 16(3).

%\opt{oldrules}{Under the Child Maintenance and Other Payments Act 2008 (c. 6) s. 59(5), in this section 
%``maintenance calculation'' is to be read as a reference to a maintenance assessment.
%}

The insertion of s. 40A(A1)--(A4) by the Child Maintenance and Other Payments Act 2008 (c. 6) s. 29(3) (as amended by the Public Bodies (Child Maintenance and Enforcement Commission: Abolition and Transfer of Functions) Order 2012 Sch. para. 82) is not yet in force.

The insertion of s. 40A(7A)--(7D) by the Child Maintenance and Other Payments Act 2008 (c. 6) s. 29(4) is not yet in force.

The substitution of words in s. 40A(1) by the Child Maintenance and Other Payments Act 2008 (c. 6) Sch. 7 para. 1(13)($a$) is not yet in force.

The substitution of s. 40A(2)($a$)(i) by the Child Maintenance and Other Payments Act 2008 (c. 6) Sch. 7 para. 1(13)($b$) is not yet in force.

The insertion of s. 40A(6)($b$) by the Child Maintenance and Other Payments Act 2008 (c. 6) Sch. 7 para. 1(14) is not yet in force.

The repeal of s. 40A(8)($c$) by the Child Maintenance and Other Payments Act 2008 (c. 6) Sch. 8 is not yet in force.

}

\subsection{40B. Disqualification from driving: further provision}

(1) If, but only if, the court is of the opinion that there has been wilful
refusal or culpable neglect on the part of the liable person, it may---
\begin{enumerate}\item[]
 ($a$) order him to be disqualified, for such period specified in the order but not
exceeding two years as it thinks fit, from holding or obtaining a driving
licence (a “disqualification order”); or

($b$) make a disqualification order but suspend its operation until such time and
on such conditions (if any) as it thinks just.
\end{enumerate}

(2) The court may not take action under both section 40 and this section.

(3) A disqualification order must state the amount in respect of which it is made,
which is to be the aggregate of---
\begin{enumerate}\item[]
($a$) the amount mentioned in section 35(1), or so much of it as remains
outstanding; and

($b$) an amount (determined in accordance with regulations made by the Secretary
of State) in respect of the costs of the application under section 39A.
\end{enumerate}

(4) A court which makes a disqualification order shall require the person to whom
it relates to produce any driving licence held by him, and its counterpart (within the
meaning of section 108(1) of the Road Traffic Act 1988).

(5) On an application by the Secretary of State or the liable person, the court---
\begin{enumerate}\item[]
($a$) may make an order substituting a shorter period of disqualification, or make an order revoking the disqualification order, if part of the amount referred to in subsection (3) (the “amount due”) is paid to any person authorised to receive it; and

($b$) must make an order revoking the disqualification order if all of the amount due is so paid.
\end{enumerate}

(6)
The Secretary of State may make representations to the court as to the amount which should be paid before it would be appropriate to make an order revoking the disqualification order under subsection (5)($a$), and the person liable may reply to those representations.

(7)
The Secretary of State may make a further application under section 39A if the amount due has not been paid in full when the period of disqualification specified in the disqualification order expires.

(8)
Where a court---
\begin{enumerate}\item[]
($a$) makes a disqualification order;

($b$) makes an order under subsection (5); or

($c$) allows an appeal against a disqualification order,
\end{enumerate}
it shall send notice of that fact to the Secretary of State; and the notice shall contain such particulars and be sent in such manner and to such address as the Secretary of State may determine.

(9)
Where a court makes a disqualification order, it shall also send the driving licence and its counterpart, on their being produced to the court, to the Secretary of State at such address as the Secretary of State may determine.

(10)
Section 80 of the Magistrates’ Courts Act 1980 (application of money found on defaulter) shall apply in relation to a disqualification order under this section in relation to a liable person as it applies in relation to the enforcement of a sum mentioned in subsection (1) of that section.

(11)
The Secretary of State may by regulations make provision in relation to disqualification orders corresponding to the provision he may make under section 40(11).

(12)
In the application to Scotland of this section---
\begin{enumerate}\item[]
($a$) in subsection (2) for “section 40” substitute “section 40A”;

($b$) in subsection (3) for paragraph ($a$) substitute– 
\begin{quotation}
“($a$) the appropriate amount under section 38;”;
\end{quotation}

($c$) subsection (10) is omitted; and

($d$) for subsection (11) substitute---
\begin{quotation}
 “(11) The power of the Court of Session by Act of Sederunt to regulate the procedure and practice in civil proceedings in the sheriff court shall include power to make, in relation to disqualification orders, provision corresponding to that which may be made by virtue of section 40A(8).”
\end{quotation}
\end{enumerate}

\amendment{
S. 40B inserted (10.11.00 for the purposes of making regulations and Acts of Sederunt, 2.4.01 for all other purposes) by the Child Support, Pensions and Social Security Act 2000 (c. 19) s. 17(2).

%\opt{oldrules}{Under the Child Maintenance and Other Payments Act 2008 (c. 6) s. 59(5), in this section 
%``maintenance calculation'' is to be read as a reference to a maintenance assessment.
%}

Words substituted in s. 40B (1.8.12) by the Public Bodies (Child Maintenance and Enforcement Commission: Abolition and Transfer of Functions) Order 2012 Sch. para. 49.

The repeal of words in s. 40B(4), (9) and substitution of words in s. 40B(9) by the Road Safety Act 2006 (c. 49) Sch. 3 para. 65 is not yet in force.

The substitution of s. 40B(3)($a$) by the Tribunals, Courts and Enforcement Act 2007 (c. 15) Sch. 13 para. 97 is not yet in force.

The substitution of s. 40B(A1)--(1) for s. 40B(1) by the Child Maintenance and Other Payments Act 2008 (c. 6) s. 30(1) (as amended by the Public Bodies (Child Maintenance and Enforcement Commission: Abolition and Transfer of Functions) Order 2012 Sch. para. 83) is not yet in force.

The substitution of s. 40B(10)--(10C) for s. 40B(10) by the Child Maintenance and Other Payments Act 2008 (c. 6) s. 30(2) is not yet in force.

The substitution of words in s. 40B(3) by the Child Maintenance and Other Payments Act 2008 (c. 6) Sch. 7 para. 1(15) is not yet in force.

The substitution of words in s. 40B(7) by the Child Maintenance and Other Payments Act 2008 (c. 6) Sch. 7 para. 1(16) is not yet in force.

The substitution of words in s. 40B(12) by the Child Maintenance and Other Payments Act 2008 (c. 6) Sch. 7 para. 1(17) is not yet in force.

The insertion of s. 40B(13) by the Child Maintenance and Other Payments Act 2008 (c. 6) Sch. 7 para. 1(18) is not yet in force.

The repeal of s. 40B(12)($b$), ($c$) by the Child Maintenance and Other Payments Act 2008 (c. 6) Sch. 8 is not yet in force.

The repeal of s. 40B by the Welfare Reform Act 2009 (c. 24) Sch. 5 para. 8 is not yet in force.

}

\subsection{41. Arrears of child support maintenance}

(1) This section applies where---
\begin{enumerate}\item[]
($a$) the Secretary of State is authorised under section 4 or 7 to recover
child support maintenance payable by \opt{oldrules}{an absent parent}\opt{newrules,2012rules}{a non-resident
parent} in accordance with a \opt{oldrules}{maintenance assessment}\opt{newrules,2012rules}{maintenance
calculation}; and

($b$) the \opt{oldrules}{absent parent}\opt{newrules,2012rules}{non-resident parent} has failed to make one or more
payments of child support maintenance due from him in accordance with
that \opt{oldrules}{assessment}\opt{newrules,2012rules}{calculation}.
\end{enumerate}

(2) Where the Secretary of State recovers any such arrears the Secretary of State may, in such circumstances as may be prescribed and to such extent as may be prescribed, retain them if the Secretary of State is satisfied that the amount of any benefit paid to or in respect of the person with care of the child or children in question would have been less had the \opt{oldrules}{absent parent}\opt{newrules,2012rules}{non-resident parent} made the payment or payments of child support maintenance in question.

(2A) In determining for the purposes of subsection (2) whether the amount of any benefit paid would have been less at any time than the amount which was paid at that time, in a case where the \opt{oldrules}{maintenance assessment}\opt{newrules,2012rules}{maintenance calculation} had effect from a date earlier than that on which it was made, the \opt{oldrules}{assessment}\opt{newrules,2012rules}{calculation} shall be taken to have been in force at that time.

\opt{oldrules}{
(3) In such circumstances as may be prescribed, the absent parent shall be liable to make such payments of interest with respect to the arrears of child support maintenance as may be prescribed.

(4) The Secretary of State may by regulations make provision---
\begin{enumerate}\item[]
($a$) as to the rate of interest payable by virtue of subsection (3);

($b$) as to the time at which, and person to whom, any such interest shall be
payable%;
%
%[7($c$) as to the circumstances in which, in a case where the Secretary of State
%has been acting under section 6, any such interest may be retained by
%him;
%
%($d$) for the Secretary of State, in a case where he has been acting under
%section 6 and in such circumstances as may be prescribed, to waive any
%such interest (or part of any such interest)
.
\end{enumerate}

(5) The provisions of this Act with respect to---
\begin{enumerate}\item[]
($a$) the collection of child support maintenance;

($b$) the enforcement of any obligation to pay child support maintenance,
\end{enumerate}
shall apply equally to interest payable by virtue of this section.
}

(6) Any sums retained by the Secretary of State by virtue of this section shall be paid by the Secretary of State into the Consolidated Fund.

\amendment{

S. 41(2), (2A) substituted for s. 41(2) by the Child Support Act 1995 (c. 34) Sch. 3 para. 11.

\opt{newrules,2012rules}{Words ``maintenance assessment(s)'' substituted by ``maintenance calculation(s)'' (3.3.03) for the purposes of certain cases only (see S.I. 2003/192) by the Child Support, Pensions and Social Security Act 2000 (c. 19) s. 1(2)($a$).

Word ``assessment'' (or any variant of that term) substituted by ``calculation'' (or other variants) (3.3.03) for the purposes of certain cases only (see S.I. 2003/192) by the Child Support, Pensions and Social Security Act 2000 (c. 19) s. 1(2)($b$).

S. 41(3)--(5) ceased to have effect (10.11.00 for the purposes of making regulations and Acts of Sederunt only, 3.3.03) for the purposes of certain cases only (see S.I. 2003/192) by the Child Support, Pensions and Social Security Act 2000 (c. 19) s. 18(1).

Words ``(an) absent parent(s)'' substituted by ``($a$) non-resident parent(s)'' (3.3.03) for the purposes of certain cases only (see S.I. 2003/192) by the Child Support, Pensions and Social Security Act 2000 (c. 19) Sch. 3 para. 11(2).

}

\opt{oldrules}{
S. 41(4)($c$), ($d$) repealed (14.7.08) for certain cases only (see S.I. 2008/1476) by the Child Maintenance and Other Payments Act 2008 (c. 6) Sch. 7 para. 1(34)($a$).

}

Word repealed in s. 41(1)($a$) (27.10.08) by the Child Maintenance and Other Payments Act 2008 (c. 6) Sch. 8.

Words substituted in s. 41 (1.8.12) by the Public Bodies (Child Maintenance and Enforcement Commission: Abolition and Transfer of Functions) Order 2012 Sch. para. 50.

\opt{oldrules}{

\medskip

S. 41A repealed (3.3.03) for the purposes of certain cases only (see S.I. 2003/192) by the Child Support, Pensions and Social Security Act 2000 (c. 19) Sch. 9 Pt. I (which repealed the inserting provision, the Child Support Act 1995 (c. 34) s. 22).


}

}

\opt{newrules,2012rules}{

\subsection{41A. Penalty payments}

(1) The Secretary of State may by regulations make provision for the payment to the Secretary of State by non-resident parents who are in arrears with payments of child support maintenance of penalty payments determined in accordance with the regulations.

(2)
The amount of a penalty payment in respect of any week may not exceed 25\% of the amount of child support maintenance payable for that week, but otherwise is to be determined by the Secretary of State.

(3)
The liability of a non-resident parent to make a penalty payment does not affect his liability to pay the arrears of child support maintenance concerned.

(4)
Regulations under subsection (1) may, in particular, make provision---
\begin{enumerate}\item[]
($a$) as to the time at which a penalty payment is to be payable;

($b$) for the Secretary of State to waive a penalty payment, or part of it.
\end{enumerate}

(5) The provisions of this Act with respect to---
\begin{enumerate}\item[]
($a$) the collection of child support maintenance;

($b$) the enforcement of an obligation to pay child support maintenance,
\end{enumerate}
apply equally (with any necessary modifications) to penalty payments payable by virtue of regulations under this section.

(6) The Secretary of State shall pay penalty payments received by the Secretary of State into the Consolidated Fund.

\amendment{

S. 41A substituted (3.3.03) for the purposes of certain cases only (see S.I. 2003/192) by the Child Support, Pensions and Social Security Act 2000 (c. 19) s. 18(2).

Words substituted in s. 41A (1.8.12) by the Public Bodies (Child Maintenance and Enforcement Commission: Abolition and Transfer of Functions) Order 2012 Sch. para. 51.

}
}

\subsection{41B. Repayment of overpaid child support maintenance}

(1) This section applies where it appears to the Secretary of State that \opt{oldrules}{an absent parent}\opt{newrules,2012rules}{a non-resident parent} has made a payment by way of child support maintenance which amounts to an overpayment by him of that maintenance and that---
\begin{enumerate}\item[]
($a$) it would not be possible for the \opt{oldrules}{absent parent}\opt{newrules,2012rules}{non-resident parent} to recover the amount of the overpayment by way of an adjustment of the amount payable under a \opt{oldrules}{maintenance assessment}\opt{newrules,2012rules}{maintenance calculation}; or

($b$) it would be inappropriate to rely on an adjustment of the amount payable under a \opt{oldrules}{maintenance assessment}\opt{newrules,2012rules}{maintenance calculation} as the means of enabling the \opt{oldrules}{absent parent}\opt{newrules,2012rules}{non-resident parent} to recover the amount of the overpayment.
\end{enumerate}

\opt{newrules,2012rules}{
(1A) This section also applies where the non-resident parent has made a voluntary payment and it appears to the Secretary of State---
\begin{enumerate}\item[]
($a$) that he is not liable to pay child support maintenance; or

($b$) that he is liable, but some or all of the payment amounts to an overpayment,
\end{enumerate}
and, in a case falling within paragraph ($b$), it also appears to the Secretary of State that subsection (1)($a$) or ($b$) applies.}

(2)
The Secretary of State may make such payment to the \opt{oldrules}{absent parent}\opt{newrules,2012rules}{non-resident parent} by way of reimbursement, or partial reimbursement, of the overpayment as the Secretary of State considers appropriate.

(3)
Where the Secretary of State has made a payment under this section the Secretary of State may, in such circumstances as may be prescribed, require the relevant person to pay to the Secretary of State the whole, or a specified proportion, of the amount of that payment.

(4)
Any such requirement shall be imposed by giving the relevant person a written demand for the amount which the Secretary of State wishes to recover from him.

(5)
Any sum which a person is required to pay to the Secretary of State under this section shall be recoverable from him by the Secretary of State as a debt due to the Crown.

(6)
The Secretary of State may by regulations make provision in relation to any case in which---
\begin{enumerate}\item[]
($a$) one or more overpayments of child support maintenance are being
reimbursed to the Secretary of State by the relevant person; and

($b$) child support maintenance has continued to be payable by the \opt{oldrules}{absent
parent}\opt{newrules,2012rules}{non-resident parent} concerned to the person with care concerned,
or again becomes so payable.
\end{enumerate}

\opt{oldrules}{
(7)
For the purposes of this section any payments made by a person under a maintenance assessment which was not validly made shall be treated as overpayments of child support maintenance made by an absent parent.
}

\opt{newrules,2012rules}{
(7) For the purposes of this section---
\begin{enumerate}\item[]
($a$) a payment made by a person under a maintenance calculation which was
not validly made; and

($b$) a voluntary payment made the circumstances set out in subsection (1A)($a$).
\end{enumerate}
shall be treated as an overpayment of child support maintenance made by a non-resident parent.}

(8)
In this section “relevant person”, in relation to an overpayment, means the person with care to whom the overpayment was made.

(9)
Any sum recovered by the Secretary of State under this section shall be paid by the Secretary of State into the Consolidated Fund.

\amendment{
S. 41B inserted (4.9.95 for sub-ss. (1), (2), (7), 1.10.95 otherwise) by the Child Support Act 1995 (c. 34) s. 23.

\opt{newrules,2012rules}{Words ``maintenance assessment(s)'' substituted by ``maintenance calculation(s)'' (3.3.03) for the purposes of certain cases only (see S.I. 2003/192) by the Child Support, Pensions and Social Security Act 2000 (c. 19) s. 1(2)($a$).

S. 41B(1A) inserted (10.11.00 for the purposes of making regulations and Acts of Sederunt only, 3.3.03) for the purposes of certain cases only (see S.I. 2003/192) by the Child Support, Pensions and Social Security Act 2000 (c. 19) s. 20(3).

S. 41B(7) substituted (10.11.00 for the purposes of making regulations and Acts of Sederunt only, 3.3.03) for the purposes of certain cases only (see S.I. 2003/192) by the Child Support, Pensions and Social Security Act 2000 (c. 19) s. 20(4).

Words ``(an) absent parent(s)'' substituted by ``($a$) non-resident parent(s)'' (3.3.03) for the purposes of certain cases only (see S.I. 2003/192) by the Child Support, Pensions and Social Security Act 2000 (c. 19) Sch. 3 para. 11(2).

}

Words substituted in s. 41B (1.8.12) by the Public Bodies (Child Maintenance and Enforcement Commission: Abolition and Transfer of Functions) Order 2012 Sch. para. 52.

}

\subsection{41C. Power to treat liability as satisfied}

(1) The Secretary of State may by regulations---
\begin{enumerate}\item[]
($a$) make provision enabling the Secretary of State in prescribed circumstances to set off liabilities to pay child support maintenance to which this section applies;

($b$) make provision enabling the Secretary of State in prescribed circumstances to set off against a person’s liability to pay child support maintenance to which this section applies a payment made by the person which is of a prescribed description.
\end{enumerate}

(2)
Liability to pay child support maintenance shall be treated as satisfied to the extent that it is the subject of setting off under regulations under subsection (1).

(3)
In subsection (1), the references to child support maintenance to which this section applies are to child support maintenance for the collection of which the Secretary of State is authorised to make arrangements.

\amendment{
S. 41C inserted (26.11.09 for regulation-making purposes, 25.1.10 for all other purposes) by the Child Maintenance and Other Payments Act 2008 (c. 6) s. 31.

\opt{oldrules}{Under the Child Maintenance and Other Payments Act 2008 (c. 6) s. 59(4), in this section 
``child support maintenance'' includes periodical payments required to be paid in accordance with a maintenance assessment under the Act.
}

Words substituted in s. 41C (1.8.12) by the Public Bodies (Child Maintenance and Enforcement Commission: Abolition and Transfer of Functions) Order 2012 Sch. para. 53.

}

\subsection{41D. Power to accept part payment of arrears in full and final satisfaction}

(1) The 
%Commission 
Secretary of State
may, in relation to any arrears of child support  maintenance, accept payment of part in satisfaction of liability for the whole.

(2)
The Secretary of State must by regulations make provision with respect to the exercise of the power under subsection (1).

(3)
The regulations must provide that unless one of the conditions in subsection
(4)
is satisfied the 
%Commission 
Secretary of State %amendment SI 2012/2007 Sch para 84
may not exercise the power under subsection (1) without the appropriate consent.

(4)
The conditions are---
\begin{enumerate}\item[]
($a$) that the 
%Commission 
Secretary of State %amendment SI 2012/2007 Sch para 84
would be entitled to retain the whole of the arrears under section 41(2) if 
%it 
the Secretary of State %amendment SI 2012/2007 Sch para 84
recovered them;

($b$) that the 
%Commission 
Secretary of State %amendment SI 2012/2007 Sch para 84
would be entitled to retain part of the arrears under section 41(2) if 
%it 
the Secretary of State %amendment SI 2012/2007 Sch para 84
recovered them, and the part of the arrears that the 
%Commission 
Secretary of State %amendment SI 2012/2007 Sch para 84
would not be entitled to retain is equal to or less than the payment accepted under subsection (1).
\end{enumerate}

(5)
Unless the maintenance calculation was made under section 7, the appropriate consent is the written consent of the person with care with respect to whom the maintenance calculation was made.

(6)
If the maintenance calculation was made under section 7, the appropriate consent is---
\begin{enumerate}\item[]
($a$) the written consent of the child who made the application under section 7(1), and

($b$) if subsection (7) applies, the written consent of the person with care of that child.
\end{enumerate}

(7) This subsection applies if---
\begin{enumerate}\item[]
($a$) the maintenance calculation was made under section 7(2), or

($b$) the Secretary of State has made arrangements under section 7(3) on the application of the person with care.
\end{enumerate}

\amendment{
S. 41D inserted (10.12.12) by the Child Maintenance and Other Payments Act 2008 (c. 6) s. 32 as amended by the Public Bodies (Child Maintenance and Enforcement Commission: Abolition and Transfer of Functions) Order 2012 Sch. para. 84.

\opt{oldrules}{Under the Child Maintenance and Other Payments Act 2008 (c. 6) s. 59(4), in this section 
``child support maintenance'' includes periodical payments required to be paid in accordance with a maintenance assessment under the Act.
}

}

\subsection{41E. Power to write off arrears}

(1) The 
%Commission %substituted by SI 2012/2007 Sch para 85
Secretary of State %end substitution
may extinguish liability in respect of arrears of child support maintenance if it appears to 
%it %substituted by SI 2012/2007 Sch para 85
the Secretary of State%end substitution
---
\begin{enumerate}\item[]
($a$) that the circumstances of the case are of a description specified in regulations made by the Secretary of State, and

($b$) that it would be unfair or otherwise inappropriate to enforce liability in
respect of the arrears.
\end{enumerate}

(2)
The Secretary of State may by regulations make provision with respect to the exercise of the power under subsection (1).

\amendment{
S. 41E inserted (10.12.12) by the Child Maintenance and Other Payments Act 2008 (c. 6) s. 33 as amended by the Public Bodies (Child Maintenance and Enforcement Commission: Abolition and Transfer of Functions) Order 2012 Sch. para. 85.

\opt{oldrules}{Under the Child Maintenance and Other Payments Act 2008 (c. 6) s. 59(4), in this section 
``child support maintenance'' includes periodical payments required to be paid in accordance with a maintenance assessment under the Act.
}

}

\subsection{42. Special cases}

(1) The Secretary of State may by regulations provide that in prescribed 
circumstances a case is to be treated as a special case for the purposes of this Act.

(2)
Those regulations may, for example, provide for the following to be special cases---
\begin{enumerate}\item[]
($a$) each parent of a child is \opt{oldrules}{an absent parent}\opt{newrules,2012rules}{a non-resident parent} in relation
to the child;

($b$) there is more than one person who is a person with care in relation to the
same child;

($c$) there is more than one qualifying child in relation to the same \opt{oldrules}{absent
parent}\opt{newrules,2012rules}{non-resident parent} but the person who is the person with care in
relation to one of those children is not the person who is the person with
care in relation to all of them;

($d$) a person is \opt{oldrules}{an absent parent}\opt{newrules,2012rules}{a non-resident parent} in relation to more than
one child and the other parent of each of those children is not the same
person;

($e$) the person with care has care of more than one qualifying child and there
is more than one \opt{oldrules}{absent parent}\opt{newrules,2012rules}{non-resident parent} in relation to those
children;

($f$) a qualifying child has his home in two or more separate households;

($g$) the same persons are the parents of two or more children and each parent is---
\begin{enumerate}\item[]
(i) a non-resident parent in relation to one or more of the children, and

(ii) a person with care in relation to one or more of the children.
\end{enumerate}
\end{enumerate}

(3)
The Secretary of State may by regulations make provision with respect to special cases.

(4) Regulations made under subsection (3) may, in particular---
\begin{enumerate}\item[]
($a$) modify any provision made by or under this Act, in its application to any
special case or any special case falling within a prescribed category;

($b$) make new provision for any such case; or

($c$) provide for any prescribed provision made by or under this Act not to
apply to any such case.
\end{enumerate}

\amendment{
\opt{newrules,2012rules}{
Words ``(an) absent parent(s)'' substituted by ``($a$) non-resident parent(s)'' (3.3.03) for the purposes of certain cases only (see S.I. 2003/192) by the Child Support, Pensions and Social Security Act 2000 (c. 19) Sch. 3 para. 11(2).

}

S. 42(2)($g$) inserted (8.10.12) by the Child Maintenance and Other Payments Act 2008 (c. 6) s. 37.
}

\opt{oldrules}{

\subsection{43. Contribution to maintenance by deduction from benefit}

(1) This section applies where– 
\begin{enumerate}\item[]
($a$) by virtue of paragraph 5(4) of Schedule 1, an absent parent is taken for the
purposes of that Schedule to have no assessable income; and

($b$) such conditions as may be prescribed for the purposes of this section are
satisfied.
\end{enumerate}

(2) The power of the Secretary of State to make regulations under section 5 of the Social Security Administration Act 1992 by virtue of subsection (1)($p$) (deductions from benefits) may be exercised in relation to cases to which this section applies with a view to securing that---
\begin{enumerate}\item[]
($a$) payments of prescribed amounts are made with respect to qualifying
children in place of payments of child support maintenance; and

($b$) arrears of child support maintenance are recovered.
\end{enumerate}

(3) Schedule 4C shall have effect for applying sections 16, 17, 20 and 28ZA to 28ZC to any decision with respect to a person’s liability under this section, that is to say, his liability to make payments under regulations made by virtue of this section.

\amendment{
Words in s. 43(2) substituted (1.7.92) by the Social Security (Consequential Provisions) Act 1992 (c. 6) Sch. 2 para. 113.

S. 43(3) inserted (4.3.99) by the Social Security Act 1998 (c. 14) Sch. 7 para. 40.
}
}

\opt{newrules,2012rules}{

\subsection{43. Recovery of child support maintenance by deduction from benefit}

(1) This section applies where---
\begin{enumerate}\item[]
($a$) a non-resident parent is liable to pay a flat rate of child support maintenance (or would be so liable but for a variation having been agreed to), and that rate applies (or would have applied) because he falls within paragraph 4(1)($b$) or ($c$) or 4(2) of Schedule 1; and

($b$) such conditions as may be prescribed for the purposes of this section are satisfied.
\end{enumerate}

(2)
The power of the Secretary of State to make regulations under section 5 of the Social Security Administration Act 1992 by virtue of subsection (1)($p$) (deductions from benefits) may be exercised in relation to cases to which this section applies with a view to securing that payments in respect of child support maintenance are made or that arrears of child support maintenance are recovered.

(3)
For the purposes of this section, the benefits to which section 5 of the 1992 Act applies are to be taken as including war disablement pensions and war widows’ pensions (within the meaning of section 150 of the Social Security Contributions and Benefits Act 1992 (interpretation)).

\amendment{
S. 43 substituted (10.11.00 for the purposes of making regulations and Acts of Sederunt only, 3.3.03) for the purposes of certain case only (see S.I. 2003/192) by the Child Support, Pensions and Social Security Act 2000 (c. 19) s. 21.

The substitution of s. 43(1), (2) by the Welfare Reform Act 2012 (c. 5) s. 139 is not yet in force.
}
}

\subsection{43A. Recovery of arrears from deceased's estate}

(1) The Secretary of State may by regulations make provision for the recovery from the estate of a deceased person of arrears of child support maintenance for which the deceased person was liable immediately before death.

(2) Regulations under subsection (1) may, in particular---
\begin{enumerate}\item[]
($a$) make provision for arrears of child support maintenance for which a deceased person was so liable to be a debt payable by the deceased’s executor or administrator out of the deceased’s estate to the Secretary of State;

($b$) make provision for establishing the amount of any such arrears;

($c$) make provision about procedure in relation to claims under the regulations.
\end{enumerate}

(3) Regulations under subsection (1) may include provision for proceedings (whether by appeal or otherwise) to be instituted, continued or withdrawn by the deceased’s executor or administrator.

\amendment{
S. 43A inserted (26.11.09) by the Child Maintenance and Other Payments Act 2008 (c. 6) s. 37.

\opt{oldrules}{Under the Child Maintenance and Other Payments Act 2008 (c. 6) s. 59(4), in this section 
``child support maintenance'' includes periodical payments required to be paid in accordance with a maintenance assessment under the Act.
}

Words substituted in s. 43A (1.8.12) by the Public Bodies (Child Maintenance and Enforcement Commission: Abolition and Transfer of Functions) Order 2012 Sch. para. 54.

}

\section{Jurisdiction}

\subsection{44. Jurisdiction}

(1) The Secretary of State shall have jurisdiction to make a \opt{oldrules}{maintenance assessment}\opt{newrules,2012rules}{maintenance calculation} with respect to a person who is---
\begin{enumerate}\item[]
($a$) a person with care;

($b$) a non-resident parent; or

($c$) a qualifying child,
\end{enumerate}
only if that person is habitually resident in the United Kingdom, except in the case of a non-resident parent who falls within subsection (2A).

(2) Where the person with care is not an individual, subsection (1) shall have effect as if paragraph ($a$) were omitted.

(2A) A non-resident parent falls within this subsection if he is not habitually resident in the United Kingdom, but is---
\begin{enumerate}\item[]
($a$) employed in the civil service of the Crown, including Her Majesty’s
Diplomatic Service and Her Majesty’s Overseas Civil Service;

($b$) a member of the naval, military or air forces of the Crown, including any
person employed by an association established for the purposes of Part XI of
the Reserve Forces Act 1996;

($c$) employed by a company of a prescribed description registered under the
Companies Act 2006; or

($d$) employed by a body of a prescribed description.
\end{enumerate}

\opt{oldrules}{

(3) The Secretary of State may by regulations make provision for the cancellation of any maintenance assessment where---
\begin{enumerate}\item[]
($a$) the person with care, absent parent or qualifying child with respect to whom it was made ceases to be habitually resident in the United Kingdom;

($b$) in a case falling within subsection (2), the absent parent, or qualifying child with respect to whom it was made ceases to be habitually resident in the United Kingdom; or

($c$) in such circumstances as may be prescribed, a maintenance order of a prescribed kind is made with respect to any qualifying child with respect to whom the maintenance assessment was made.
\end{enumerate}

}

(4) The Commission does not have jurisdiction under this section if the exercise of jurisdiction would be contrary to the jurisdictional requirements of the Maintenance Regulation.

(5) In subsection (4) “the Maintenance Regulation” means Council Regulation (EC) No 4/2009 including as applied in relation to Denmark by virtue of the Agreement made on 19th October 2005 between the European Community and the Kingdom of Denmark.

\amendment{

Words substituted in s. 44 (1.6.99) by the Social Security Act 1998 (c. 14) Sch. 7 para. 41.

S. 44(2A) inserted (10.11.00 for the purposes of making regulations and Acts of Sederunt, 31.1.01 for all other purposes) by the Child Support, Pensions and Social Security Act 2000 (c. 19) s. 22(3).

Words substituted and inserted in s. 44(1) (31.1.01) by the Child Support, Pensions and Social Security Act 2000 (c. 19) s. 22(2).

\opt{newrules,2012rules}{Words ``maintenance assessment(s)'' substituted by ``maintenance calculation(s)'' (3.3.03) for the purposes of certain cases only (see S.I. 2003/192) by the Child Support, Pensions and Social Security Act 2000 (c. 19) s. 1(2)($a$).

S. 44(3) ceased to have effect (3.3.03) for the purpose of certain cases only (see S.I. 2003/192) by the Child Support, Pensions and Social Security Act 2000 (c. 19) s. 22(4).

}

Words in s. 44(2A)($c$) substituted (1.10.09) by the Companies Act 2006 (Consequential Amendments, Transitional Provisions and Savings) Order 2009 Sch. 1 para. 123.

S. 44(4), (5) inserted (18.6.11) by the Civil Jurisdiction and Judgments (Maintenance) Regulations 2011 Sch. 7 para. 13.

Words substituted in s. 44 (1.8.12) by the Public Bodies (Child Maintenance and Enforcement Commission: Abolition and Transfer of Functions) Order 2012 Sch. para. 55.

}

\subsection{45. Jurisdiction of courts in certain proceedings under this Act}

(1) The Lord Chancellor or, in relation to Scotland, the Lord Advocate may by order make such provision as he considers necessary to secure that appeals, or such certain class of appeals as may be specified in the order---
\begin{enumerate}\item[]
($a$) shall be made to a court instead of being made to the First-tier Tribunal; or

($b$) shall be so made in such circumstances as may be so specified.
\end{enumerate}

(2) In subsection (1), “court” means---
\begin{enumerate}\item[]
($a$) in relation to England and Wales% 
%and subject to any provision made under Schedule 11 to the Children Act 1989 (jurisdiction of courts with respect to certain proceedings relating to children) the High Court, a county court or a magistrates’ court; and
, the High Court or the family court; and  % Words substituted (22.4.14) by 2013 c 22 Sch 11 para 125

($b$) in relation to Scotland, the Court of Session or the sheriff.
\end{enumerate}

%(3)
%Schedule 11 to the Act of 1989 shall be amended in accordance with subsections (4) and (5).
%
%(4)
%The following sub-paragraph shall be inserted in paragraph 1, after sub-paragraph (2)---
%\begin{quotation}
% “(2A) Sub-paragraphs (1) and (2) shall also apply in relation to proceedings---
%\begin{enumerate}\item[]
%($a$)
%under section 27 of the Child Support Act 1991 (reference to court for declaration of parentage); or
%
%($b$)
%which are to be dealt with in accordance with an order made under section 45 of that Act (jurisdiction of courts in certain proceedings under that Act).”
%\end{enumerate}
%\end{quotation}
%
%(5)
%In paragraphs 1(3) and 2(3), the following shall be inserted after “Act 1976”---
%\begin{quotation}
%“($bb$) section 20 (appeals) or 27 (reference to court for declaration of parentage) of the Child Support Act 1991;”.
%\end{quotation}

(7)
Any order under subsection (1) may make---
\begin{enumerate}\item[]
($a$) such modifications of any provision of this Act or of any other enactment; and

($b$) such transitional provision,
\end{enumerate}
as the Minister making the order considers appropriate in consequence of any provision made by the order.

(8) The functions of the Lord Chancellor under this section may be exercised only after consultation with the Lord Chief Justice.

(9) The Lord Chief Justice may nominate a judicial office holder (as defined in section 109(4) of the Constitutional Reform Act 2005) to exercise his functions under this section.

\amendment{
S. 45(8), (9) inserted (3.4.06) by the Constitutional Reform Act 2005 (c. 4) Sch. 4 para. 220.

Words substituted in s. 45(1)($a$) (3.11.08) by the Transfer of Tribunal Functions Order 2008 Sch. 3 para. 92($a$).

S. 45(6) omitted (3.11.08) by the Transfer of Tribunal Functions Order 2008 Sch. 3 para. 92($b$).

Words omitted in s. 45(7) (3.11.08) by the Transfer of Tribunal Functions Order 2008 Sch. 3 para. 92($c$).

Words substituted in s. 45(2)(a) and s. 45(3)--(5) omitted (24.4.14) by the Crime and Courts Act 2013 Sch. 11. para. 125, 210.
}

\section{Miscellaneous and supplemental}

\amendment{
S. 46 repealed (14.7.08) by the Child Maintenance and Other Payments Act 2008 (c. 6) s. 15($b$).
}

\subsection{46A. Finality of decisions}

(1) Subject to the provisions of this Act and to any provision made by or under Chapter II of Part I of the Tribunals, Courts and Enforcement Act 2007, any decision of the Secretary of State or the First-tier Tribunal made in accordance with the foregoing provisions of this Act shall be final.

(2) If and to the extent that regulations so provide, any finding of fact or other determination embodied in or necessary to such a decision, or on which such a decision is based, shall be conclusive for the purposes of---
\begin{enumerate}\item[]
($a$) further such decisions;

($b$) decisions made in accordance with sections 8 to 16 of the Social Security
Act 1998, or with regulations under section 11 of that Act; and

($c$) decisions made under the Vaccine Damage Payments Act 1979.
\end{enumerate}

\amendment{
S. 46A inserted (1.6.99) by the Social Security Act 1998 (c. 14) Sch. 7 para. 44.

Words inserted and substituted in s. 46A(1) (3.11.08) by the Transfer of Tribunal Functions Order 2008 Sch. 3 para. 93.

Words omitted in s. 46A (1.8.12) by the Public Bodies (Child Maintenance and Enforcement Commission: Abolition and Transfer of Functions) Order 2012 Sch. para. 56.

}

\subsection{46B. Matters arising as respects decisions}

(1) Regulations may make provision as respects matters arising pending---
\begin{enumerate}\item[]
($a$) any decision of the Secretary of State under section 11, 12 or 17;

($b$) any decision of the First-tier Tribunal under section 20; or

($c$) any decision of the Upper Tribunal in relation to a decision of the First-tier
Tribunal under this Act under section 24.
\end{enumerate}

(2) Regulations may also make provision as respects matters arising out of the
revision under section 16, or on appeal, of any such decision as is mentioned in
subsection (1).

\opt{oldrules}{
(3) Any reference in this section to section 16, 17 or 20 includes
a reference to that section as extended by Schedule 4C.
}

\amendment{
S. 46B inserted (1.6.99) by the Social Security Act 1998 (c. 14) Sch. 7 para. 44.

\opt{newrules,2012rules}{S. 46B(3) repealed (3.3.03) for the purposes of certain cases only (see S.I. 2003/192) by the Child Support, Pensions and Social Security Act 2000 (c. 19) Sch. 9 Pt. I.}

Words substituted in s. 46B(1) (3.11.08) by the Transfer of Tribunal Functions Order 2008 Sch. 3 para. 94.

Words substituted in s. 46B (1.8.12) by the Public Bodies (Child Maintenance and Enforcement Commission: Abolition and Transfer of Functions) Order 2012 Sch. para. 57.

}

\subsection{47. Fees}

(1) The Secretary of State may by regulations provide for the payment, by
the \opt{oldrules}{absent parent}\opt{newrules,2012rules}{non-resident parent} or the person with care (or by both), of such
fees as may be prescribed in cases where the Secretary of State takes%, [4or proposes to take] 
 any action under section 4 or 6.

(2) The Secretary of State may by regulations provide for the payment, by the
\opt{oldrules}{absent parent}\opt{newrules,2012rules}{non-resident parent}, the person with care or the child concerned (or
by any or all of them), of such fees as may be prescribed in cases where the Secretary
of State takes any action under section 7.

(3) Regulations made under this section---
\begin{enumerate}\item[]
($a$) may require any information which is needed for the purpose of determining
the amount of any such fee to be furnished, in accordance with
the regulations, by such person as may be prescribed;

($b$) shall provide that no such fees shall be payable by any person to or in
respect of whom income support, an income-based jobseeker’s allowance,
any element of child tax credit other than the family element, working tax
credit or any other benefit of a prescribed kind is paid; and

($c$) may, in particular, make provision with respect to the recovery by the Secretary
of State of any fees payable under the regulations.
\end{enumerate}

\opt{newrules,2012rules}{
(4) The provisions of this Act with respect to---
\begin{enumerate}\item[]
($a$) the collection of child support maintenance;

($b$) the enforcement of any obligation to pay child support maintenance,
\end{enumerate}
shall apply equally (with any necessary modifications) to fees payable by virtue of
regulations made under this section.
}

\amendment{
Words inserted in s. 47(3)($b$) (7.10.96) by the Jobseekers Act 1995 (c. 18) Sch. 2 para. 20(5).

\opt{newrules,2012rules}{
S. 47(4) inserted (10.11.00 for the purposes of making regulations and Acts of Sederunt only, 3.3.03) for the purposes of certain cases only (see S.I. 2003/192) by the Child Support, Pensions and Social Security Act 2000 (c. 19) Sch. 3 para. 11(18).

Words ``(an) absent parent(s)'' substituted by ``($a$) non-resident parent(s)'' (3.3.03) for the purposes of certain cases only (see S.I. 2003/192) by the Child Support, Pensions and Social Security Act 2000 (c. 19) Sch. 3 para. 11(2).

}

Words substituted in s. 47(3)($b$) (6.4.03) by the Tax Credits Act 2002 (c. 21) Sch. 3 para. 22.

The insertion of words in s. 47(1), (2) by the Child Support Act 1995 (c. 34) Sch. 3 para. 13 is not yet in force.

The insertion of words in s. 47(3)($b$) by the Welfare Reform Act 2007 (c. 5) Sch. 3 para. 7(6) is not yet in force.

The repeal of s. 47 by the Child Maintenance and Other Payments Act 2008 (c. 6) Sch. 8 is not yet in force.

}

\subsection{48. Right of audience}

(1) Any officer of the Secretary of State who is authorised by the Secretary of State for the purposes of this section shall have, in relation to any proceedings under this Act before 
the family court or  % Words inserted (22.4.14) by 2013 c 22 Sch 11 para 126
a magistrates’ court, a right of audience and the right to conduct litigation.

(2) In this section “right of audience” and “right to conduct litigation” have the same meaning as in section 119 of the Courts and Legal Services Act 1990.

\amendment{

Words substituted in s. 48(1) (4.9.95) by the Child Support Act 1995 (c. 34) Sch. 3 para. 14.

Words substituted in s. 48(1) (1.8.12) by the Public Bodies (Child Maintenance and Enforcement Commission: Abolition and Transfer of Functions) Order 2012 Sch. para. 58.

Words inserted in s. 48(1) (24.4.14) by the Crime and Courts Act 2013 Sch. 11. para. 126.
}

\subsection{49. Right of audience: Scotland}

In relation to any proceedings before the sheriff under any provision of this Act, the power conferred on the Court of Session by section 32 of the Sheriff Courts (Scotland) Act 1971 (power of Court of Session to regulate civil procedure in sheriff court) shall extend to the making of rules permitting a party to such proceedings, in such circumstances as may be specified in the rules, to be represented by a person who is neither an advocate nor a solicitor.

\amendment{
The insertion of s. 49A by the Child Maintenance and Other Payments Act 2008 (c. 6) s. 34 (as amended by the Public Bodies (Child Maintenance and Enforcement Commission: Abolition and Transfer of Functions) Order 2012 Sch. para. 86) is not yet in force.

\medskip

The insertion of ss. 49B, 49C by the Child Maintenance and Other Payments Act 2008 (c. 6) s. 39 (as amended by the Public Bodies (Child Maintenance and Enforcement Commission: Abolition and Transfer of Functions) Order 2012 Sch. para. 87) is not yet in force.

\medskip

The insertion of s. 49D by the Child Maintenance and Other Payments Act 2008 (c. 6) s. 40 (as amended by the Public Bodies (Child Maintenance and Enforcement Commission: Abolition and Transfer of Functions) Order 2012 Sch. para. 88) is not yet in force.
}

\subsection{50. Unauthorised disclosure of information}

(1) Any person who is, or has been, employed in employment to which this subsection applies is guilty of an offence if, without lawful authority, he discloses any information which---
\begin{enumerate}\item[]
($a$)
was acquired by him in the course of that employment; and

($b$)
relates to a particular person.
\end{enumerate}

(1A) Subsection (1) applies to employment as---
\begin{enumerate}\item[]
 ($za$) any member of staff appointed under section 40(1) of the Tribunals, Courts and Enforcement Act 2007 in connection with the carrying out of any functions in relation to appeals from decisions made under this Act;

($a$) any clerk to, or other officer of, an appeal tribunal constituted under Chapter I of Part I of the Social Security Act 1998; 

($b$) any member of the staff of any such appeal tribunal;

($c$)
a civil servant in connection with the carrying out of any functions under this Act;

($e$)
any person who provides, or is employed in the provision of, services to the Secretary of State,
\end{enumerate}
and to employment of any other kind which is prescribed for the purposes of this subsection.

(1B) Any person who is, or has been, employed in employment to which this subsection applies is guilty of an offence if, without lawful authority, he discloses any information which---
\begin{enumerate}\item[]
($a$)
was acquired by him in the course of that employment;

($b$)
is information which is, or is derived from, information acquired or held for the purposes of this Act; and

($c$)
relates to a particular person.
\end{enumerate}

(1C)
Subsection (1B) applies to any employment which---
\begin{enumerate}\item[]
($a$)
is not employment to which subsection (1) applies, and

($b$)
is of a kind prescribed for the purposes of this subsection.
\end{enumerate}

(2) It is not an offence under this section---
\begin{enumerate}\item[]
($a$)
to disclose information in the form of a summary or collection of information so framed as not to enable information relating to any particular person to be ascertained from it; or

($b$) to disclose information which has previously been disclosed to the public
with lawful authority.
\end{enumerate}

(3)
It is a defence for a person charged with an offence under this section to prove that at the time of the alleged offence---
\begin{enumerate}\item[]
($a$) he believed that he was making the disclosure in question with lawful
authority and had no reasonable cause to believe otherwise; or

($b$) he believed that the information in question had previously been disclosed
to the public with lawful authority and had no reasonable cause to
believe otherwise.
\end{enumerate}

(4) A person guilty of an offence under this section shall be liable---
\begin{enumerate}\item[]
($a$) on conviction on indictment, to imprisonment for a term not exceeding
two years or a fine or both; or

($b$) on summary conviction, to imprisonment for a term not exceeding six
months or a fine not exceeding the statutory maximum or both.
\end{enumerate}

(6)
For the purposes of this section a disclosure is to be regarded as made with lawful authority if, and only if, it is made---
\begin{enumerate}\item[]
($a$) by a civil servant in accordance with his official duty; or

($b$) by any other person either---
\begin{enumerate}\item[]
(i)
for the purposes of the function in the exercise of which he holds the
information and without contravening any restriction duly imposed
by the responsible person; or

(ii)
to, or in accordance with an authorisation duly given by, the responsible
person;
\end{enumerate}

($c$) in accordance with any enactment or order of a court;

($d$) for the purpose of instituting, or otherwise for the purposes of, any
proceedings before a court or before any tribunal or other body or person
mentioned in this Act; or

($e$) with the consent of the appropriate person.
\end{enumerate}

(7) “The responsible person” means---
\begin{enumerate}\item[]
($a$) the Lord Chancellor;

($b$) the Secretary of State;

($c$) any person authorised for the purposes of this subsection by the Lord
Chancellor or the Secretary of State, or

($d$) any other prescribed person, or person falling within a prescribed
category.
\end{enumerate}

(8) “The appropriate person” means the person to whom the information in question relates, except that if the affairs of that person are being dealt with---
\begin{enumerate}\item[]
($a$) under a power of attorney; or

($c$) by a Scottish mental health custodian, that is to say a guardian or other person entitled to act on behalf of the person under the Adults with Incapacity (Scotland) Act 2000;
\end{enumerate}
the appropriate person is the attorney or custodian (as the case may be) or, in a case falling within paragraph ($a$), the person to whom the information relates.

(9) Where the person to whom the information relates lacks capacity (within the
meaning of the Mental Capacity Act 2005) to consent to its disclosure, the appropriate
person is---
\begin{enumerate}\item[]
($a$) a donee of an enduring power of attorney or lasting power of attorney (within the meaning of that act), or

($b$) a deputy appointed for him, or any other person authorised, by the Court of
Protection,
\end{enumerate}
with power in that respect.

\amendment{
Words substituted for s. 50(8)($c$)(i), (ii) by the Adults with Incapacity (Scotland) Act 2000 (Consequential Modifications) (England, Wales and Northern Ireland) Order 2005 para. 2.

S. 50(8)($b$), ($d$) omitted, words substituted in s. 50(8) and s. 50(9) inserted (1.10.07) by the Mental Capacity Act 2005 (c. 9) Sch. 6 para. 36.

S. 50(5) repealed (14.7.08) by the Child Maintenance and Other Payments Act 2008 (c. 6) Sch. 8.

Words substituted in s. 50(1) (1.11.08) by the Child Maintenance and Other Payments Act 2008 (c. 6) Sch. 7 para. 1(19).

S. 50(1A)--(1C) inserted (1.11.08) by the Child Maintenance and Other Payments Act 2008 (c. 6) Sch. 7 para. 1(20).

S. 50(7)($c$) substituted  (1.11.08) by the Child Maintenance and Other Payments Act 2008 (c. 6) Sch. 7 para. 1(21).

S. 50(1A)($za$) inserted, words inserted in s. 50(1A)($a$) and words substituted in s. 50(1A)($b$) (3.11.08) by the Transfer of Tribunal Functions Order 2008 Sch. 3 para. 95.

S. 50(1A)($d$), (7)(ba) omitted and words substituted in s. 50(1A)($e$), (7)($c$) (1.8.12) by the Public Bodies (Child Maintenance and Enforcement Commission: Abolition and Transfer of Functions) Order 2012 Sch. para. 59.
}

\subsection{50A. Use of computers}

Any decision falling to be made under or by virtue of this Act by the Secretary of State may be made, not only by a person authorised to exercise the Secretary of State’s decision-making function, but also by a computer for whose
operation such a person is responsible.

\amendment{
S. 50A inserted (1.11.08) by the Child Maintenance and Other Payments Act 2008 (1.11.08) Sch. 3 para. 51.

Words inserted in s. 50A (1.8.12) by the Public Bodies (Child Maintenance and Enforcement Commission: Abolition and Transfer of Functions) Order 2012 Sch. para. 60.
}

\subsection{51. Supplementary powers to make regulations}

(1) The Secretary of State may by regulations make such incidental, supplemental and transitional provision as he considers appropriate in connection with any provision made by or under this Act.

(2)
The regulations may, in particular, make provision---
\begin{enumerate}\item[]
($a$)
as to the procedure to be followed with respect to---
\begin{enumerate}\item[]
(i)
the making of applications for \opt{oldrules}{maintenance assessments}\opt{newrules,2012rules}{maintenance calculations};

\opt{oldrules}{
(ii)
the making, cancellation or refusal to make maintenance assessments;

(iii) the making of decisions under section 16 or 17;}

\opt{newrules,2012rules}{
(ii) the making of decisions under section 11;

(iii) the making of decisions under section 16 or 17;}
%The version of section 51(2)($a$)(iii) below continues to apply to:—
%($a$)
%reviews under S. 16 which commence on or before 8 December 1996
%($b$)
%reviews under sections 17 or 19 as per S.S. Act 98 (Commencement No. 2) Order 1998/2780 reg. 3(5)($a$) & ($b$).
%(iii) reviews under sections 16 to 19;
\end{enumerate}

\opt{oldrules}{($b$) extending the categories of case to which Schedule 4C applies;}

\opt{newrules,2012rules}{($b$) extending the categories of case to which section 16, 17 or 20 applies;}

($c$) as to the date on which an application for a \opt{oldrules}{maintenance assessment}\opt{newrules,2012rules}{maintenance calculation} is to be treated as having been made;

($d$) for attributing payments made under \opt{oldrules}{maintenance assessments}\opt{newrules,2012rules}{maintenance
calculations} to the payment of arrears;

($e$) for the adjustment, for the purpose of taking account of the retrospective
effect of a \opt{oldrules}{maintenance assessment}\opt{newrules,2012rules}{maintenance calculation}, of amounts
payable under the \opt{oldrules}{assessment}\opt{newrules,2012rules}{calculation};

($f$) for the adjustment, for the purpose of taking account of over-payments or
under-payments of child support maintenance, of amounts payable under
a \opt{oldrules}{maintenance assessment}\opt{newrules,2012rules}{maintenance calculation};

($g$) as to the evidence which is to be required in connection with such matters
as may be prescribed;

($h$) as to the circumstances in which any official record or certificate is to be
conclusive (or in Scotland, sufficient) evidence;

($i$) with respect to the giving of notices or other documents;

($j$) for the rounding up or down of any amounts calculated, estimated or
otherwise arrived at in applying any provision made by or under this Act.
\end{enumerate}

(3)
No power to make regulations conferred by any other provision of this Act shall be taken to limit the powers given to the Secretary of State by this section.

\amendment{
\opt{oldrules}{
S. 51(2)($a$)(iii) substituted and words substituted in s. 52(2)($b$) (16.11.98) by Social Security Act 1998 Sch. 7 para. 46 (subject to transitional provisions in S.I. 1998/2780 reg. 3(5)).
}

\opt{newrules,2012rules}{
Words ``maintenance assessment(s)'' substituted by ``maintenance calculation(s)'' (3.3.03) for the purposes of certain cases only (see S.I. 2003/192) by the Child Support, Pensions and Social Security Act 2000 (c. 19) s. 1(2)($a$).

Word ``assessment'' (or any variant of that term) substituted by ``calculation'' (or other variants) (3.3.03) for the purposes of certain cases only (see S.I. 2003/192) by the Child Support, Pensions and Social Security Act 2000 (c. 19) s. 1(2)($b$).

S. 51(2)($a$)(ii), (iii), ($b$) substituted (3.3.03) for the purposes of certain cases only (see S.I. 2003/192) by the Child Support, Pensions and Social Security Act 2000 (c. 19) Sch. 3 para. 11(19).
}
}

\subsection{51A. Pilot schemes}

(1) Any regulations made under this Act may be made so as to have effect for a specified period not exceeding 24 months.

(2)
Regulations which, by virtue of subsection (1), are to have effect for a limited period are referred to in this section as a “pilot scheme”.

(3) A pilot scheme may provide that its provisions are to apply only in relation to---
\begin{enumerate}\item[]
($a$) one or more specified areas or localities;

($b$) one or more specified classes of person;

($c$) persons selected by reference to prescribed criteria, or on a sampling basis.
\end{enumerate}

(4)
A pilot scheme may make consequential or transitional provision with respect to the cessation of the scheme on the expiry of the specified period.

(5)
A pilot scheme may be replaced by a further pilot scheme making the same or similar provision.

\amendment{
S. 51A inserted (8.10.12) by the Child Maintenance and Other Payments Act 2008 (c. 6) s. 41.

The insertion of s. 51A(6) by the Welfare Reform Act 2012 (c. 5) Sch. 11 para. 7 is not yet in force.
}

\subsection{52. Regulations and orders}

(1) Any power conferred on the Lord Advocate or the Secretary of State by this Act to make regulations or orders (other than a deduction from earnings order) shall be exercisable by statutory instrument.

%\opt{oldrules}{(2)
%No statutory instrument containing (whether alone or with other provisions) regulations made under section 4(7), 5(3), (9) or (10), 7(8), 12(2), 28C(2)($b$), 28F(3), 30(5A), 32A to 32C, 32E to 32J, 41(2), (3) or (4), 41A, 41B(6), 41E(1)($a$), 42, 43(1) or 47 or under Part I of Schedule 1 or under Schedule 4B, or an order made under section 45(1) or (6), shall be made unless a draft of the instrument has been laid before Parliament and approved by a resolution of each House of Parliament.}

(2) No statutory instrument containing (whether alone or with other provisions) regulations made under---
\begin{enumerate}\item[]
($a$) section 12(4) (so far as the regulations make provision for the default
rate of child support maintenance mentioned in section 12(5)($b$)), 28C(2)($b$),
28F(2)($b$), 30(5A), 32A to 32C, 32E to 32J, 41(2), 41A, 41B(6), 41E(1)($a$), 43(1), 44(2A)($d$) or
47;

($b$) paragraph 3(2) or 10A(1) of Part I of Schedule 1; or

($c$) Schedule 4B, 
\end{enumerate}
or an order made under section 45(1) or (6), shall be made unless a draft of the
instrument has been laid before Parliament and approved by a resolution of each House of Parliament.

(2A) No statutory instrument containing (whether alone or with other provisions)---
\begin{enumerate}\item[]
($a$) the first regulations under section 17(2) to make provision of the kind
mentioned in section 17(3)($a$) or ($b$),

($b$) the first regulations under section 39F, 39M(4), 39P, 39Q, 41D(2), 41E(2) or
49A,

($c$) the first regulations under paragraph 5A(6)($b$) of Schedule 1,

($d$) the first regulations under paragraph 9(1)(ba) of Schedule 1 to make provision
of the kind mentioned in sub-paragraph (2) of that paragraph, or

($e$) the first regulations under paragraph 10(1) of Schedule 1 to make provision
of the kind mentioned in sub-paragraph (2)($a$) or ($b$) of that paragraph,
\end{enumerate}
shall be made unless a draft of the instrument has been laid before Parliament and approved by a resolution of each House of Parliament.

(2B) No statutory instrument containing (whether alone or with other provisions) regulations which by virtue of section 51A are to have effect for a limited period shall be made unless a draft of the instrument has been laid before Parliament and approved by a resolution of each House of Parliament.

(3) Any other statutory instrument made under this Act (except an order made under section 58(2)) shall be subject to annulment in pursuance of a resolution of either House of Parliament.

(4) Any power of a kind mentioned in subsection (1) may be exercised---
\begin{enumerate}\item[]
($a$) in relation to all cases to which it extends, in relation to those cases but
subject to specified exceptions or in relation to any specified cases or
classes of case;

($b$) so as to make, as respects the cases in relation to which it is exercised---
\begin{enumerate}\item[]
(i)
the full provision to which it extends or any lesser provision
(whether by way of exception or otherwise);

(ii)
the same provision for all cases, different provision for different
cases or classes of case or different provision as respects the same
case or class of case but for different purposes of this Act;

(iii) provision which is either unconditional or is subject to any specified
condition;
\end{enumerate}

($c$) so to provide for a person to exercise a discretion in dealing with any
matter.
\end{enumerate}


\amendment{
\opt{oldrules}{Words inserted in s. 52(2) (4.9.95) by the Child Support Act 1995 (c. 34) Sch. 3 para. 15.
}

S. 52(2), (2A) substituted for s. 52(2) (10.11.00 for the purposes of making regulations and Acts of Sederunt only, 3.3.03 for 2003 and 2012 scheme cases, 16.5.14 for 1993 scheme cases) for the purposes of certain cases only (see S.I. 2003/192) by the Child Support, Pensions and Social Security Act 2000 (c. 19) s. 25.

Words in s. 52(1) omitted (3.4.06) by the Constitutional Reform Act 2005 (c. 4) Sch. 18 Pt. II.

Words repealed in s. 52(2) (27.10.08) by the Child Maintenance and Other Payments Act 2008 (c. 6) Sch. 8.

Words inserted in s. 52(2) (27.6.12) by the Child Maintenance and Other Payments Act 2008 (c. 6) Sch. 7 para. 1(22).

\opt{newrules,2012rules}{
S. 52(2A) substituted (27.6.12) by the Child Maintenance and Other Payments Act 2008 (c. 6) Sch. 7 para. 1(23).
}

S. 52(2B) inserted (27.6.12) by the Child Maintenance and Other Payments Act 2008 (c. 6) Sch. 7 para. 1(24).

The insertion of words in s. 52(2A)($b$) by the Welfare Reform Act 2009 (c. 24) Sch. 5 para. 9 is not yet in force.

The insertion of words in s. 52(2)($a$) by the Welfare Reform Act 2012 (c. 5) Sch. 11 para. 8 is not yet in force.

}

\subsection{53. Financial provisions}

Any expenses of the Lord Chancellor or the Secretary of State under this Act shall be payable out of money provided by Parliament.

\subsection{54. Interpretation}

(1) In this Act---
\begin{enumerate}\item[]
\opt{oldrules}{“absent parent” has the meaning given in section 3(2);}

“application for a \opt{oldrules}{departure direction}\opt{newrules,2012rules}{variation}” means an application under section 28A\opt{newrules,2012rules}{ or 28G};

\opt{oldrules}{“assessable income” has the meaning given in paragraph 5 of Schedule 1;}

“benefits Acts” means the Social Security Contributions and Benefits Act
1992 and the Social Security Administration Act 1992;

“charging order” has the same meaning as in section 1 of the Charging Orders Act 1979;

“child benefit” has the same meaning as in the Child Benefit Act 1975;

“child support maintenance” has the meaning given in section 3(6);

\opt{oldrules}{“current assessment”, in relation to an application for a departure direction, means (subject to any regulations made under paragraph 10 of Schedule 4A) the maintenance assessment with respect to which the application is made;}

“deduction from earnings order” has the meaning given in section 31(2);

\opt{newrules,2012rules}{“default maintenance decision” has the meaning given in section 12;}

\opt{oldrules}{“departure direction” has the meaning given in section 28A;}

“deposit-taker” means a person who, in the course of a business, may lawfully accept deposits in the United Kingdom;

“disability living allowance” has the same meaning as in the benefit Acts;

%“working families’ tax credit” has the same meaning as in the benefit Acts;

“income support” has the same meaning as in the benefit Acts;

“income-based jobseeker’s allowance” has the same meaning as in the
Jobseekers Act 1995;

“income-related employment and support allowance” means an income-related
allowance under Part I of the Welfare Reform Act 2007 (employment and support
allowance);

“interim maintenance \opt{oldrules}{assessment}\opt{newrules,2012rules}{decision}” has the meaning given in section
12;

“liability order” has the meaning given in section 33(2);

“maintenance agreement” has the meaning given in section 9(1);

\opt{oldrules}{“maintenance assessment” means an assessment of maintenance made under
this Act and, except in prescribed circumstances, includes an interim maintenance
assessment;}

\opt{newrules,2012rules}{“maintenance calculation” means a calculation of maintenance made under this Act and, except in prescribed circumstances, includes a default maintenance decision and an interim maintenance decision;}

“maintenance order” has the meaning given in section 8(11);

\opt{oldrules}{“maintenance requirement” means the amount calculated in accordance with paragraph 1 of Schedule 1;}

“parent”, in relation to any child, means any person who is in law the mother
or father of the child;

“parent with care” means a person who is, in relation to a child, both a parent and
a person with care;

%“parental responsibility” has the same meaning as in the Children Act 1989;
“parental responsibility”, in the application of this Act---
\begin{enumerate}\item[]
($a$) to England and Wales, has the same meaning as in the Children Act 1989; and

($b$) to Scotland, shall be construed as a reference to “parental responsibilities” within the meaning given by section 1(3) of the Children (Scotland) Act 1995;
\end{enumerate}

%“parental rights” has the same meaning as in the Law Reform (Parent and Child) (Scotland) Act 1986;

“person with care” has the meaning given in section 3(3);

“prescribed” means prescribed by regulations made by the Secretary of State;

“qualifying child” has the meaning given in section 3(1)\opt{newrules,2012rules}{;

“voluntary payment” has the meaning given in section 28J}.
\end{enumerate}

(2) The definition of “deposit-taker” in subsection (1) is to be read with—
\begin{enumerate}\item[]
($a$) section 22 of the Financial Services and Markets Act 2000;

($b$) any relevant order under that section; and

($c$) Schedule 2 to that Act.
\end{enumerate}

\amendment{
Words in s. 54 substituted (1.7.92) by the Social Security (Consequential Provisions) Act 1992 (c. 6) Sch. 2 para. 114.

Definitions in s. 54 inserted (4.9.95) by the Child Support Act 1995 (c. 34) Sch. 3 para. 16.

Definition in s. 54 inserted (7.10.96) by the Jobseekers Act 1995 (c. 18) Sch. 2 para. 20(6).

Definition of ``parental responsibility'' in s. 54 substituted (1.11.96) by the Children (Scotland) Act 1995 (c. 48) Sch. 4 para. 52(4)($a$).

Definition of ``parental rights'' in s. 54 repealed (1.11.96) by the Children (Scotland) Act 1995 (c. 48) Sch. 4 para. 52(4)($b$).

Definitions in s. 54 repealed (1.6.99) by the Social Security Act 1998 (c. 14) Sch. 8.

Words substituted for ``family credit'' (5.10.99) by the Tax Credits Act 1999 (c. 10) Sch. 1 para. 6($f$)(ii).

\opt{newrules,2012rules}{
Words in s. 54 inserted, substituted and omitted (3.3.03) for the purpose of certain cases only (see S.I. 2003/192) by the Child Support, Pensions and Social Security Act 2000 (c. 19) Sch. 3 para. 11(20).
}

Definition in s. 54 repealed (6.4.03) by the Tax Credits Act 2002 (c. 21) Sch. 6.

Definition in s. 54 inserted (27.10.08) by the Welfare Reform Act 2007 (c. 5) Sch. 3 para. 7(7).

Definition in s. 54 omitted (3.11.08) by the Tribunal, Courts and Enforcement Act 2007 (c. 15) Sch. 23. Pt. II.

Definition in s. 54 omitted (3.11.08) by the Transfer of Tribunal Functions Order 2008  Sch. 3 para. 96.

Definitions in s. 54(1) inserted (1.6.09) by the Child Maintenance and Other Payments Act 2008 (c. 6) Sch. 7 para. 1(25).

S. 54(2) inserted (1.6.09) by the Child Maintenance and Other Payments Act 2008 (c. 6) Sch. 7 para. 1(26).

Definition in s. 54(1) omitted (1.8.12) by the Public Bodies (Child Maintenance and Enforcement Commission: Abolition and Transfer of Functions) Order 2012 Sch. para. 61.

The repeal of the definition of ``income support'' by the Welfare Reform Act 2009 (c. 24) Sch. 7 Pt. I is not yet in force.

The repeal of definitions by the Welfare Reform Act 2012 (c. 5) Sch. 14 Pt. I is not yet in force.
}

\subsection{55. Meaning of ``child''}

(1) In this Act, “child” means (subject to subsection (2)) a person who---
\begin{enumerate}\item[]
($a$) has not attained the age of 16, or

($b$) has not attained the age of 20 and satisfies such conditions as may be prescribed.
\end{enumerate}

(2)
A person who is or has been party to a marriage or civil partnership is not a child for the purposes of this Act.

(3)
For the purposes of subsection (2), “marriage” and “civil partnership” include a void marriage and a void civil partnership respectively.

\amendment{
S. 55 substituted (10.12.12) by the Child Maintenance and Other Payments Act 2008 (c. 6) s. 42.
}

\subsection{56. Corresponding provision for and co-ordination with Northern Ireland}

(1) An Order in Council made under paragraph 1(1)($b$) of Schedule 1 to the Northern Ireland Act 1974 which contains a statement that it is made only for purposes corresponding to those of the provisions of this Act, other than provisions which relate to the appointment of Child Support Commissioners for Northern Ireland---
\begin{enumerate}\item[]
($a$) shall not be subject to sub-paragraphs (4) and (5) of paragraph 1 of that Schedule (affirmative resolution of both Houses of Parliament); but

($b$) shall be subject to annulment in pursuance of a resolution of either House of Parliament.
\end{enumerate}

%(2) The Secretary of State may make arrangements with the Department of Health and Social Services for Northern Ireland with a view to securing, to the extent allowed for in the arrangements, that---
%\begin{enumerate}\item[]
%($a$) the provision made by or under this Act (“the provision made for Great Britain”); and
%
%($b$) the provision made by or under any corresponding enactment having effect
%with respect to Northern Ireland (“the provision made for Northern Ireland”), 
%\end{enumerate}
%provide for a single system within the United Kingdom.
%
%(3) The Secretary of State may make regulations for giving effect to any such arrangements.
%
%(4) The regulations may, in particular---
%\begin{enumerate}\item[]
%($a$) adapt legislation (including subordinate legislation) for the time being in
%force in Great Britain so as to secure its reciprocal operation with the provision
%made for Northern Ireland; and
%
%($b$) make provision to secure that acts, omissions and events which have any
%effect for the purposes of the provision made for Northern Ireland have
%a corresponding effect for the purposes of the provision made for Great
%Britain.
%\end{enumerate}

\amendment{
S. 56(2)--(4) repealed (2.12.99) by the Northern Ireland Act 1998 (c. 47) s. 87(8)($c$).
}

\subsection{57. Application to Crown}

(1) The power of the Secretary of State to make regulations under section 14 requiring prescribed persons to furnish information may be exercised so as to require information to be furnished by persons employed in the service of the Crown or otherwise in the discharge of Crown functions.

(2)
In such circumstances, and subject to such conditions, as may be prescribed, an inspector appointed under section 15 may enter any Crown premises for the purpose of exercising any powers conferred on him by that section.

(3)
Where such an inspector duly enters any Crown premises for those purposes, section 15 shall apply in relation to persons employed in the service of the Crown or otherwise in the discharge of Crown functions as it applies in relation to other persons.

(4)
Where a liable person is in the employment of the Crown, a deduction from earnings order may be made under section 31 in relation to that person; but in such a case subsection (8) of section 32 shall apply only in relation to the failure of that person to comply with any requirement imposed on him by regulations made under section 32.

\amendment{
The substitution of words in s. 57(4) by the Child Maintenance and Other Payments Act 2008 (c. 6) Sch. 7 para. 1(27) is not yet in force.
}

\subsection{58. Short title, commencement and extent etc.}

(1) This Act may be cited as the Child Support Act 1991.

(2)
Section 56(1) and subsections (1) to (11) and (14) of this section shall come into force on the passing of this Act but otherwise this Act shall come into force on such date as may be appointed by order made by the Lord Chancellor, the Secretary of State or Lord Advocate, or by any of them acting jointly.

(3)
Different dates may be appointed for different provisions of this Act and for different purposes (including, in particular, for different cases or categories of case).

(4)
An order under subsection (2) may make such supplemental, incidental or transitional provision as appears to the person making the order to be necessary or expedient in connection with the provisions brought into force by the order, including such adaptations or modifications of---
\begin{enumerate}\item[]
($a$) the provisions so brought into force;

($b$) any provisions of this Act then in force; or

($c$)
any provision of any other enactment, 
\end{enumerate}
as appear to him to be necessary or expedient.

(5)
Different provision may be made by virtue of subsection (4) with respect to different periods.

(6)
Any provision made by virtue of subsection (4) may, in particular, include provision for---
\begin{enumerate}\item[]
($a$) the enforcement of a \opt{oldrules}{maintenance assessment}\opt{newrules,2012rules}{maintenance calculation} (including the collection of sums payable under the \opt{oldrules}{assessment}\opt{newrules,2012rules}{calculation}) as if the \opt{oldrules}{assessment}\opt{newrules,2012rules}{calculation} were a court order of a prescribed kind;

($b$) the registration of \opt{oldrules}{maintenance assessments}\opt{newrules,2012rules}{maintenance calculations} with the appropriate court in connection with any provision of a kind mentioned in paragraph ($a$);

($c$) the variation, on application made to a court, of the provisions of a \opt{oldrules}{maintenance assessment}\opt{newrules,2012rules}{maintenance calculation} relating to the method of making payments fixed by the \opt{oldrules}{assessment}\opt{newrules,2012rules}{calculation} or the intervals at which such payments are to be made;

($d$) a \opt{oldrules}{maintenance assessment}\opt{newrules,2012rules}{maintenance calculation}, or an order of a prescribed kind relating to one or more children, to be deemed, in prescribed circumstances, to have been validly made for all purposes or for such purpose as may be prescribed.
\end{enumerate}

In paragraph ($c$) “court” includes a single justice.

(7)
The Lord Chancellor, the Secretary of State of the Lord Advocate may by order make such amendments or repeals in, or such modifications of, such enactments as may be specified in the order, as appear to him to be necessary or expedient in consequence of any provision made by or under this Act (including any provision made by virtue of subsection (4)).

(8)
This Act shall, in its application to the Isles of Scilly, have effect subject to such exceptions, adaptations and modifications as the Secretary of State may by order prescribe.

(9)
Sections 27, 35, 40 and 48 and paragraph 7 of Schedule 5 do not extend to Scotland.

(10) Sections 7, 28, 40A and 49 extend only to Scotland.

(11)
With the exception of sections 23 and 56(1), subsections (1) to (3) of this section and Schedules 2 and 4, and (in so far as it amends any enactment extending to Northern Ireland) Schedule 5, this Act does not extend to Northern Ireland.

%[4(12) Until Schedule 1 to the Disability Living Allowance and Disability Working Allowance Act 1991 comes into force*, paragraph 1(1) of Schedule 3 shall have effect with the omission of the words “and disability appeal tribunals” and the insertion, after “social security appeal tribunals”, of the word “and”.]
%*All of Sch. 1 to the D.L.A. & D.W.A. Act 1991 was in force by 6.4.92.

(13) The consequential amendments set out in Schedule 5 shall have effect.

(14) In Schedule 1 to the Children Act 1989 (financial provision for children), paragraph 2(6)($b$) (which is spent) is hereby repealed.


\amendment{
S. 58(12) is not yet in force.

\opt{newrules,2012rules}{
Words ``maintenance assessment(s)'' substituted by ``maintenance calculation(s)'' (3.3.03) for the purposes of certain cases only (see S.I. 2003/192) by the Child Support, Pensions and Social Security Act 2000 (c. 19) s. 1(2)($a$).

Word ``assessment'' (or any variant of that term) substituted by ``calculation'' (or other variants) (3.3.03) for the purposes of certain cases only (see S.I. 2003/192) by the Child Support, Pensions and Social Security Act 2000 (c. 19) s. 1(2)($b$).

Words inserted in s. 58(9), (10) (3.3.03) for the purposes of certain cases only (see S.I. 2003/192) by the Child Support, Pensions and Social Security Act 2000 (c. 19) Sch. 3 para. 11(21).
}

}

\clearpage

\part*{S C H E D U L E S}

\opt{oldrules}{

\part[Schedule 1 --- Maintenance assessments]{Schedule 1\\*Maintenance assessments}

\section[Part I --- Calculation of child support maintenance]{Part I\\*Calculation of child support maintenance}

\subsection*{The maintenance requirement}

\renewcommand\parthead{--- Schedule 1 Part I}

1.---(1) In this Schedule “the maintenance requirement” means the amount, calculated in accordance with the formula set out in sub-paragraph (2), which is to be taken as the minimum amount necessary for the maintenance of the qualifying child or, where there is more than one qualifying child, all of them.

(2) The formula is---
\[
\mathrm{MR} = \mathrm{AG} - \mathrm{CB}
\]
where---
\begin{enumerate}\item[]
 MR is the amount of the maintenance requirement; 

AG is the aggregate of the amounts to be taken into account under sub-paragraph (3); and 

CB is the amount payable by way of child benefit (or which would be so payable if the person with care of the qualifying child were an individual) or, where there is more than one qualifying child, the aggregate of the amounts so payable with respect of each of them.
\end{enumerate}

(3) The amounts to be taken into account for the purpose of calculating AG are---
\begin{enumerate}\item[]
($a$) such amount or amounts (if any), with respect to each qualifying child, as may be prescribed;

($b$) such amount or amounts (if any), with respect to the person with care of the qualifying child or qualifying children, as may be prescribed; and

($c$) such further amount or amounts (if any) as may be prescribed.
\end{enumerate}

(4)
For the purposes of calculating CB it shall be assumed that child benefit is payable with respect to any qualifying child at the basic rate.

(5)
In sub-paragraph (4) “basic rate” has the meaning for the time being prescribed.

\subsection*{The general rule}

2.---(1) In order to determine the amount of any maintenance assessment, first calculate---
\[ (\textrm{A} + \textrm{C}) \times \textrm{P}\]
where---
\begin{enumerate}\item[]
A is the absent parent’s assessable income;

C is the assessable income of the other parent, where that parent is the person with care, and otherwise has such value (if any) as may be prescribed; and

P is such number greater than zero but less than 1 as may be prescribed.
\end{enumerate}

(2) Where the result of the calculation made under sub-paragraph (1) is an amount which is equal to, or less than, the amount of the maintenance requirement for the qualifying child or qualifying children, the amount of maintenance payable by the absent parent for that child or those children shall be an amount equal to—
\[\textrm{A} \times \textrm{P}\]
where A and P have the same values as in the calculation made under sub-paragraph (1).

(3) Where the result of the calculation made under sub-paragraph (1) is an amount which exceeds the amount of the maintenance requirement for the qualifying child or qualifying children, the amount of maintenance payable by the absent parent for that child or those children shall consist of---
\begin{enumerate}\item[]
($a$) a basic element calculated in accordance with the provisions of paragraph 3; and

($b$) an additional element calculated in accordance with the provisions of paragraph 4.
\end{enumerate}

\subsection*{The basic element}

3.---(1) The basic element shall be calculated by applying the formula---
\[\textrm{BE} = \textrm{A} \times \textrm{G} \times \textrm{P}
\]
where---
\begin{enumerate}\item[]
BE is the amount of the basic element;

A and P have the same values as in the calculation made under paragraph 2(1); and

G has the value determined under sub-paragraph (2).
\end{enumerate}

(2) The value of G shall be determined by applying the formula---
\[\textrm{G} = \frac { \textrm{MR} } { (\textrm{A} + \textrm{C}) \times \textrm{P}}\]
where---
\begin{enumerate}\item[]
MR is the amount of the maintenance requirement for the qualifying child or qualifying children; and

A, C and P have the same values as in the calculation made under paragraph 2(1).
\end{enumerate}

\subsection*{The additional element}

4.---(1) Subject to sub-paragraph (2), the additional element shall be calculated by applying the formula---
\[
\textrm{AE} = (1 - \textrm{G}) \times \textrm{A} \times \textrm{R}
\]
where---
\begin{enumerate}\item[]
AE is the amount of the additional element;

A has the same value as in the calculation made under paragraph 2(1);


G has the value determined under paragraph 3(2); and

R is such number greater than zero but less than 1 as may be prescribed.
\end{enumerate}

(2) Where applying the alternative formula set out in sub-paragraph (3) would result in a lower amount for the additional element, that formula shall be applied in place of the formula set out in sub-paragraph (1).

(3) The alternative formula is---
\[
\textrm{AE} = \textrm{Z} \times \textrm{Q} \times \frac{\textrm{A}}{\textrm{A} + \textrm{C}}
\]
where---
\begin{enumerate}\item[]
A and C have the same values as in the calculation made under paragraph 2(1);

Z is such number as may be prescribed; and

Q is the aggregate of---
\begin{enumerate}\item[]
($a$) any amount taken into account by virtue of paragraph 1(3)($a$) in calculating the maintenance requirement; and

($b$) any amount which is both taken into account by virtue of paragraph 1(3)($c$) in making that calculation and is an amount prescribed for the purposes of this paragraph.
\end{enumerate}
\end{enumerate}

\subsection*{Assessable income}

5.---(1) The assessable income of an absent parent shall be calculated by applying the formula---
\[\textrm{A} = \textrm{N} - \textrm{E}\]
where---
\begin{enumerate}\item[]
A is the amount of that parent’s assessable income;

N is the amount of that parent’s net income, calculated or estimated in accordance with regulations made by the Secretary of State for the purposes of this sub-paragraph; and

E is the amount of that parent’s exempt income, calculated or estimated in accordance with regulations made by the Secretary of State for those purposes.
\end{enumerate}


(2) The assessable income of a parent who is a person with care of the qualifying child or children shall be calculated by applying the formula---
\[
\textrm{C} = \textrm{M} - \textrm{F}
\]
where---
\begin{enumerate}\item[]
C is the amount of that parent’s assessable income;

M is the amount of that parent’s net income, calculated or estimated in accordance with regulations made by the Secretary of State for the purposes of this sub-paragraph; and

F is the amount of that parent’s exempt income, calculated or estimated in accordance with regulations made by the Secretary of State for those purposes.
\end{enumerate}

(3) Where the preceding provisions of this paragraph would otherwise result in a person’s assessable income being taken to be a negative amount his assessable income shall be taken to be nil.

(4) Where income support%
, an income-based jobseeker’s allowance%
, an income-related employment and support allowance
 or any other benefit of a prescribed kind is paid to or in respect of a parent who is an absent parent or a person with care that parent shall, for the purposes of this Schedule, be taken to have no assessable income.

\amendment{
Words inserted in para. 5(4) (7.10.96) by the Jobseeker's Act 1995 (c. 18) Sch. 2 para. 20(7).

Words inserted in para. 5(4) (27.10.08) by the Welfare Reform Act 2007 (c. 5) Sch. 3 para. 8.

The repeal of words in para. 5(4) by the Welfare Reform Act 2009 (c. 24) Sch. 7 Pt. I is not yet in force.

The insertion of words in para. 5(4) by the Welfare Reform Act 2012 (c. 5) Sch. 2 para. 2 is not yet in force.

The repeal of words in para. 5(4) by the Welfare Reform Act 2012 (c. 5) Sch. 14 Pt. I is not yet in force.
}

\subsection*{Protected income}

6.---(1) This paragraph applies where---
\begin{enumerate}\item[]
($a$) one or more maintenance assessments have been made with respect to an absent parent; and

($b$) payment by him of the amount, or the aggregate of the amounts, so assessed would otherwise reduce his disposable income below his protected income level.
\end{enumerate}

(2) The amount of the assessment, or (as the case may be) of each assessment, shall be adjusted in accordance with such provisions as may be prescribed with a view to securing so far as is reasonably practicable that payment by the absent parent of the amount, or (as the case may be) aggregate of the amounts, so assessed will not reduce his disposable income below his protected income level.

(3) Regulations made under sub-paragraph (2) shall secure that, where the prescribed minimum amount fixed by regulations made under paragraph 7 applies, no maintenance assessment is adjusted so as to provide for the amount payable by an absent parent in accordance with that assessment to be less than that amount.

(4) The amount which is to be taken for the purposes of this paragraph as an absent parent’s disposable income shall be calculated, or estimated, in accordance with regulations made by the Secretary of State.

(5) Regulations made under sub-paragraph (4) may, in particular, provide that, in such circumstances and to such extent as may be prescribed---
\begin{enumerate}\item[]
($a$) income of any child who is living in the same household with the absent parent; and

($b$) where the absent parent---
\begin{enumerate}\item[]
(i)
is living together in the same household with another adult of the
opposite sex (regardless of whether or not they are married),

(ii)
is living together in the same household with another adult of the same
sex who is his civil partner, or

(iii) is living together in the same household with another adult of the same
sex as if they were civil partners,
\end{enumerate}
income of that other adult,
\end{enumerate}
is to be treated as the absent parent’s income for the purposes of calculating his disposable income.

(5A) For the purposes of this paragraph, two adults of the same sex are to be regarded as living together in the same household as if they were civil partners if, but only if, they would be regarded as living together as husband and wife were they instead two adults of the opposite sex.

(6) In this paragraph the “protected income level” of a particular absent parent means an amount of income calculated, by reference to the circumstances of that parent, in accordance with regulations made by the Secretary of State.

\amendment{
Para. 6(5)($b$) substituted (5.12.05) by the Civil Partnership Act 2004 (c. 33) Sch. 24 para. 4.

Para. 6(5A) inserted (5.12.05) by the Civil Partnership Act 2004 (c. 33) Sch. 24 para. 5.
}

\subsection*{The minimum amount of child support maintenance}

7.---(1) The Secretary of State may prescribe a minimum amount for the purposes of this paragraph.

(2) Where the amount of child support maintenance which would be fixed by a maintenance assessment but for this paragraph is nil, or less than the prescribed minimum amount, the amount to be fixed by the assessment shall be the prescribed minimum amount.

(3) In any case to which section 43 applies, and in such other cases (if any) as may be prescribed, sub-paragraph (2) shall not apply.

\subsection*{Housing costs}

8. Where regulations under this Schedule require the Secretary of State to take account of the housing costs of any person in calculating, or estimating, his assessable income or disposable income, those regulations may make provision---
\begin{enumerate}\item[]
($a$) as to the costs which are to be treated as housing costs for the purpose of the regulations;

($b$) for the apportionment of housing costs; and

($c$) for the amount of housing costs to be taken into account for prescribed purposes not to exceed such amount (if any) as may be prescribed by, or determined in accordance with, the regulations.
\end{enumerate}

\amendment{
Words in para. 8 substituted (1.6.99) by the Social Security Act 1998 Sch. 7 para. 48(1).
}

\subsection*{Regulations about income and capital}

9. The Secretary of State may by regulations provide that, in such circumstances and to such extent as may be prescribed---
\begin{enumerate}\item[]
($a$) income of a child shall be treated as income of a parent of his;

($b$) where the Secretary of State is satisfied that a person has intentionally deprived himself of a source of income with a view to reducing the amount of his assessable income, his net income shall be taken to include income from that source of an amount estimated by the Secretary of State;

($c$) a person is to be treated as possessing capital or income which he does not possess;

($d$) capital or income which a person does possess is to be disregarded;

($e$) income is to be treated as capital;

($f$) capital is to be treated as income.
\end{enumerate}

\amendment{
Words in para. 9 substituted (1.6.99) by the Social Security Act 1998 Sch. 7 para. 48(2).
}

\subsection*{References to qualifying children}

10. References in this Part of this Schedule to “qualifying children” are to those qualifying children with respect to whom the maintenance assessment falls to be made.

}

\opt{newrules,2012rules}{

\part[Schedule 1 --- Maintenance calculations]{Schedule 1\\*Maintenance calculations}

\section[Part I --- Calculation of weekly amount of child support maintenance]{Part I\\*Calculation of weekly amount of child support maintenance}

\renewcommand\parthead{--- Schedule 1 Part I}

\amendment{
Sch. 1 Pt. I substituted (10.11.00 for the purposes of making regulations and Acts of Sederunt only, 3.3.03) for the purposes of certain cases only (see S.I. 2003/192) by the Child Support, Pensions and Social Security Act 2000 (c. 19) Sch. 1.
}

\subsection*{General rule}

1.---(1) \opt{newrules}{T}\opt{2012rules}{Subject to paragraph 5A, t}he weekly rate of child support maintenance is the basic rate unless a reduced rate, a flat rate or the nil rate applies.

(2) Unless the nil rate applies, the amount payable weekly to a person with care is---
\begin{enumerate}\item[]
($a$) the applicable rate, if paragraph 6 does not apply; or

($b$) if paragraph 6 does apply, that rate as apportioned between the persons with care in accordance with paragraph 6,
\end{enumerate}
as adjusted, in either case, by applying the rules about shared care in paragraph 7 or 8.

\opt{2012rules}{
\amendment{
Words inserted in para. 1(1) (10.12.12) for the purposes of certain cases only (see S.I. 2012/3042) by the Child Maintenance and Other Payments Act 2008 (c. 6) Sch. 4 para. 5(1).
}
}

\subsection*{Basic rate}

\opt{newrules}{

2.---(1) The basic rate is the following percentage of the non-resident parent’s net weekly income---
\begin{enumerate}\item[]
15 percent; where he has one qualifying child;

20 percent; where he has two qualifying children;

25 percent; where he has three or more qualifying children.
\end{enumerate}

(2) If the non-resident parent also has one or more relevant other children, the appropriate percentage referred to in sub-paragraph (1) is to be applied instead to his net weekly income less---
\begin{enumerate}\item[]
15 percent; where he has one relevant other child;

20 percent; where he has two relevant other children;

25 percent; where he has three or more relevant other children.
\end{enumerate}

}

\opt{2012rules}{

2.---(1) Subject to sub-paragraph (2), the basic rate is the following percentage of the non-resident parent's gross weekly income—
\begin{enumerate}\item[]
12\% where the non-resident parent has one qualifying child;

16\% where the non-resident parent has two qualifying children;

19\% where the non-resident parent has three or more qualifying children.
\end{enumerate}

(2) If the gross weekly income of the non-resident parent exceeds £800, the basic rate is the aggregate of the amount found by applying sub-paragraph (1) in relation to the first £800 of that income and the following percentage of the remainder—
\begin{enumerate}\item[]
9\% where the non-resident parent has one qualifying child;

12\% where the non-resident parent has two qualifying children;

15\% where the non-resident parent has three or more qualifying children.
\end{enumerate}

(3) If the non-resident parent also has one or more relevant other children, gross weekly income shall be treated for the purposes of sub-paragraphs (1) and (2) as reduced by the following percentage—
\begin{enumerate}\item[]
11\% where the non-resident parent has one relevant other child;

14\% where the non-resident parent has two relevant other children;

16\% where the non-resident parent has three or more relevant other children.
\end{enumerate}

\amendment{
Para. 2 substituted (10.12.12) for the purposes of certain cases only (see S.I. 2012/3042) by the Child Maintenance and Other Payments Act 2008 (c. 6) Sch. 4 para. 3.

Figures in para. 2(3) substituted (10.12.12) by S.I. 2012/2678 reg. 2.
}

}

\subsection*{Reduced rate}

3.---(1) A reduced rate is payable if---
\begin{enumerate}\item[]
($a$) neither a flat rate nor the nil rate applies; and

($b$) the non-resident parent’s \opt{newrules}{net}\opt{2012rules}{gross} weekly income is less than £200 but more than £100.
\end{enumerate}

(2) The reduced rate payable shall be prescribed in, or determined in accordance with, regulations.

(3) The regulations may not prescribe, or result in, a rate of less than £5.

\amendment{
\opt{2012rules}{
Word ``gross'' substituted for ``net'' in para. 3(1)($b$) (10.12.12) for the purposes of certain cases only (see S.I. 2012/3042) by the Child Maintenance and Other Payments Act 2008 (c. 6) Sch. 4 para. 2.
}

The substitution in para. 3(3) by the Child Maintenance and Other Payments Act 2008 (c. 6) Sch. 4 para. 4($a$) is not yet in force.
}

\subsection*{Flat rate}

4.---(1) Except in a case falling within sub-paragraph (2), a flat rate of £5 is payable if the nil rate does not apply and---
\begin{enumerate}\item[]
($a$) the non-resident parent’s \opt{newrules}{net}\opt{2012rules}{gross} weekly income is £100 or less; or

($b$) he receives any benefit, pension or allowance prescribed for the purposes of this paragraph of this sub-paragraph; or

($c$) he or his partner (if any) receives any benefit prescribed for the purposes of this paragraph of this sub-paragraph.
\end{enumerate}

(2)
A flat rate of a prescribed amount is payable if the nil rate does not apply and---
\begin{enumerate}\item[]
($a$) the non-resident parent has a partner who is also a non-resident parent;

($b$) the partner is a person with respect to whom a maintenance calculation is in force; and

($c$) the non-resident parent or his partner receives any benefit prescribed under sub-paragraph (1)($c$).
\end{enumerate}

(3)
The benefits, pensions and allowances which may be prescribed for the purposes of sub-paragraph (1)($b$) include ones paid to the non-resident parent under the law of a place outside the United Kingdom.

\amendment{
\opt{2012rules}{
Word ``gross'' substituted for ``net'' in para. 4(1)($a$) (10.12.12) for the purposes of certain cases only (see S.I. 2012/3042) by the Child Maintenance and Other Payments Act 2008 (c. 6) Sch. 4 para. 2.
}

The substitution in para. 4(1) by the Child Maintenance and Other Payments Act 2008 (c. 6) Sch. 4 para. 4($b$) is not yet in force.
}

\subsection*{Nil rate}

5. The rate payable is nil if the non-resident parent---
\begin{enumerate}\item[]
($a$) is of a prescribed description; or

($b$) has a \opt{newrules}{net}\opt{2012rules}{gross}  weekly income of below £5.
\end{enumerate}

\amendment{
\opt{2012rules}{
Word ``gross'' substituted for ``net'' in para. 5($b$) (10.12.12) for the purposes of certain cases only (see S.I. 2012/3042) by the Child Maintenance and Other Payments Act 2008 (c. 6) Sch. 4 para. 2.
}

The substitution in para. 5($b$) by the Child Maintenance and Other Payments Act 2008 (c. 6) Sch. 7 para. 1(28) is not yet in force.
}

%\opt{newrules}{
%\amendment{
%Para. 5A is in force only for 2012 scheme cases.
%}
%}

\opt{2012rules}{

\subsection*{Non-resident parent party to other maintenance agreement}

5A.---(1) This paragraph applies where—
\begin{enumerate}\item[]
($a$) the non-resident parent is a party to a qualifying maintenance arrangement with respect to a child of his who is not a qualifying child, and

($b$) the weekly rate of child support maintenance apart from this paragraph would be the basic rate or a reduced rate or calculated following agreement to a variation where the rate would otherwise be a flat rate or the nil rate.
\end{enumerate}

(2) The weekly rate of child support maintenance is the greater of £5 and the amount found as follows.

(3) First, calculate the amount which would be payable if the non-resident parent's qualifying children also included every child with respect to whom the non-resident parent is a party to a qualifying maintenance arrangement.

(4) Second, divide the amount so calculated by the number of children taken into account for the purposes of the calculation.

(5) Third, multiply the amount so found by the number of children who, for purposes other than the calculation under sub-\hspace{0pt}paragraph (3), are qualifying children of the non-resident parent.

(6) For the purposes of this paragraph, the non-resident parent is a party to a qualifying maintenance arrangement with respect to a child if the non-resident parent is—
\begin{enumerate}\item[]
($a$) liable to pay maintenance or aliment for the child under a maintenance order, or

($b$) a party to an agreement of a prescribed description which provides for the non-resident parent to make payments for the benefit of the child,
\end{enumerate}
and the child is habitually resident in the United Kingdom.

\amendment{
Para. 5A inserted (8.10.12 for the purpose of making regulations only, 10.12.12) for the purposes of certain cases only (see S.I. 2012/3042) by the Child Maintenance and Other Payments Act 2008 (c. 6) Sch. 4 para. 5(2).

Figure in para. 5A(2) substituted (10.12.12) by S.I. 2012/2678 reg. 3.
}

}

\subsection*{Apportionment}

6.---(1) If the non-resident parent has more than one qualifying child and in relation to them there is more than one person with care, the amount of child support maintenance payable is (subject to paragraph 7 or 8) to be determined by apportioning the rate between the persons with care.

(2) The rate of maintenance liability is to be divided by the number of qualifying children, and shared among the persons with care according to the number of qualifying children in relation to whom each is a person with care.

\subsection*{Shared care---basic and reduced rate}

\opt{newrules}{
7.---(1) This paragraph applies only if the rate of child support maintenance payable is the basic rate or a reduced rate.
}

\opt{2012rules}{
7.---(1) This paragraph applies where the rate of child support maintenance payable is the basic rate or a reduced rate or is determined under paragraph 5A.
}

(2) \opt{newrules}{If the care of a qualifying child is shared}\opt{2012rules}{If the care of a qualifying child is, or is to be, shared} between the non-resident parent and the person with care, so that the non-resident parent from time to time has care of the child overnight, the amount of child support maintenance which he would otherwise have been liable to pay the person with care, as calculated in accordance with the preceding paragraphs of this Part of this Schedule, is to be decreased in accordance with this paragraph.

(3) First, there is to be a decrease according to the number of such nights which the Secretary of State determines there to have been, or expects there to be, or both during a prescribed twelve-month period.

(4) The amount of that decrease for one child is set out in the following Table---

\begin{center}
\begin{tabular}{ll}
\hline
\itshape Number of nights & \itshape Fraction to subtract\\
\hline
52 to 103 &One-seventh\\
104 to 155 &Two-sevenths\\
156 to 174 &Three-sevenths\\
175 or more &One-half\\
\hline
\end{tabular}
\end{center}

(5)
If the person with care is caring for more than one qualifying child of the non-resident parent, the applicable decrease is the sum of the appropriate fractions in the Table divided by the number of such qualifying children.

(6)
If the applicable fraction is one-half in relation to any qualifying child in the care of the person with care, the total amount payable to the person with care is then to be further decreased by £7 for each such child.

(7)
If the application of the preceding provisions of this paragraph would decrease the weekly amount of child support maintenance (or the aggregate of all such amounts) payable by the non-resident parent to the person with care (or all of them) to less than £5, he is instead liable to pay child support maintenance at the rate of £5 a week, apportioned (if appropriate) in accordance with paragraph 6.

\amendment{
Words substituted in para. 7(3) (1.8.12) by the Public Bodies (Child Maintenance and Enforcement Commission: Abolition and Transfer of Functions) Order 2012 Sch. para. 62(2)($a$).

\opt{2012rules}{
Words substituted in para. 7(2) (10.12.12) for the purposes of certain cases only (see S.I. 2012/3042) by the Child Maintenance and Other Payments Act 2008 (c. 6) Sch. 4 para. 6.

Para. 7(1) substituted (10.12.12) for the purposes of certain cases only (see S.I. 2012/3042) by the Child Maintenance and Other Payments Act 2008 (c. 6) Sch. 7 para. 1(29).
}

The substitution in para. 7(7) by the Child Maintenance and Other Payments Act 2008 (c. 6) Sch. 4 para. 4($c$) is not yet in force.
}

\subsection*{Shared care---flat rate}

8.—(1) This paragraph applies only if---
\begin{enumerate}\item[]
($a$) the rate of child support maintenance payable is a flat rate; and

($b$) that rate applies because the non-resident parent falls within paragraph
4(1)($b$) or ($c$) or 4(2).
\end{enumerate}

(2) \opt{newrules}{If the care of a qualifying child is shared}\opt{2012rules}{If the care of a qualifying child is, or is to be, shared}
 as mentioned in paragraph 7(2) for at least 52 nights during a prescribed 12-month period, the amount of child support maintenance payable by the non-resident parent to the person with care of that child is nil.

\opt{2012rules}{
\amendment{
Words substituted in paragraph 8(2) (8.10.12 for the purpose of making regulations only, 10.12.12) for the purposes of certain cases only (see S.I. 2012/3042) by the Child Maintenance and Other Payments Act 2008 (c. 6) Sch. 4 para. 7.
}
}

\subsection*{Regulations about shared care}

9.\opt{2012rules}{—(1)} The Secretary of State may by regulations provide---
\begin{enumerate}\item[]
\opt{2012rules}{
($za$) for how it is to be determined whether the care of a qualifying child is to be
shared as mentioned in paragraph 7(2);
}

($a$) for which nights are to count for the purposes of shared care under
paragraphs 7 and 8\opt{newrules}{, or for how it is to be determined whether a night
counts};

($b$) for what counts, or does not count, as “care” for those purposes; 

\opt{2012rules}{
(ba) for how it is to be determined how many nights count for those purposes; and
}

($c$) for paragraph 7(3) or 8(2) to have effect, in prescribed circumstances, as
if the period mentioned there were other than 12 months, and in such
circumstances for the Table in paragraph 7(4) (or that Table as modified pursuant to regulations made under paragraph 10A(2)($a$)), or the period mentioned in paragraph 8(2), to have effect with prescribed adjustments.
\end{enumerate}

\opt{2012rules}{
(2) Regulations under sub-paragraph (1)(ba) may include provision enabling the 
%Commission 
Secretary of State
to proceed for a prescribed period on the basis of a prescribed assumption.
}

\opt{2012rules}{
\amendment{
Para. 9 renumbered as 9(1) and para. 9(1)($za$), (ba), (2) inserted (8.10.12 for the purpose of making regulations only, 10.12.12) for the purposes of certain cases only (see S.I. 2012/3042) by the Child Maintenance and Other Payments Act 2008 (c. 6) Sch. 4 para. 8 as amended by the Public Bodies (Child Maintenance and Enforcement Commission: Abolition and Transfer of Functions) Order 2012 Sch. para. 95(2).

Words in para. 9(1)($a$) repealed (10.12.12) for the purposes of certain cases only (see S.I. 2012/3042) by the Child Maintenance and Other Payments Act 2008 (c. 6) Sch. 8.
}


}

\subsection*{\opt{newrules}{Net}\opt{2012rules}{Gross} weekly income}

10.—(1) For the purposes of this Schedule, \opt{newrules}{net}\opt{2012rules}{gross} weekly income is to be determined in such manner as is provided for in regulations.

\opt{newrules}{
(2) The regulations may, in particular, provide for the Secretary of State to estimate any income or make an assumption as to any fact where, in the Secretary of State's view, the information at the Secretary of State's disposal is unreliable, insufficient, or relates to an atypical period in the life of the non-resident parent.
}

\opt{2012rules}{
(2) The regulations may, in particular---
\begin{enumerate}\item[]
($a$) provide for determination in prescribed circumstances by reference to income of a prescribed description in a prescribed past period;

($b$) provide for the Secretary of State to estimate any income or make an assumption as to any fact where, in the Secretary of State's view, the information at the Secretary of State's disposal is unreliable or insufficient, or relates to an atypical period in the life of the non-resident parent.
\end{enumerate}
}

(3) Any amount of \opt{newrules}{net}\opt{2012rules}{gross} weekly income (calculated as above) over \opt{newrules}{£2,000}\opt{2012rules}{£3,000} is to be ignored for the purposes of this Schedule.

\amendment{
\opt{2012rules}{
Para. 10(2) substituted (8.10.12 for the purpose of making regulations only, 10.12.12) for the purposes of certain cases only (see S.I. 2012/3042) by the Child Maintenance and Other Payments Act 2008 (c. 6) Sch. 4 para. 9 as amended by the Public Bodies (Child Maintenance and Enforcement Commission: Abolition and Transfer of Functions) Order 2012 Sch. para. 95(3).

Word ``gross'' substituted for ``net'' in para. 10(1), (3) and in the heading to para. 10 (10.12.12) for the purposes of certain cases only (see S.I. 2012/3042) by the Child Maintenance and Other Payments Act 2008 (c. 6) Sch. 4 para. 2.

Figure £3,000 substituted for £2,000 (10.12.12) for the purposes of certain cases only (see S.I. 2012/3042) by the Child Maintenance and Other Payments Act 2008 (c. 6) Sch. 4 para. 10.
}

Words substituted in para. 10(2) (1.8.12) by the Public Bodies (Child Maintenance and Enforcement Commission: Abolition and Transfer of Functions) Order 2012 Sch. para. 62(2)($b$).
}

\subsection*{Regulations about rates, figures, etc.}

10A.—(1) The Secretary of State may by regulations provide that---
\begin{enumerate}\item[]
($a$) paragraph 2 is to have effect as if different percentages were substituted for those set out there;

($b$) paragraph 2(2), 3(1) or (3), 4(1), 5, 5A(2), 7(7) or 10(3) is to have effect as if different amounts were substituted for those set out there.
\end{enumerate}

(2) The Secretary of State may by regulations provide that---
\begin{enumerate}\item[]
($a$) the Table in paragraph 7(4) is to have effect as if different numbers of nights were set out in the first column and different fractions were substituted for those set out in the second column;

($b$) paragraph 7(6) is to have effect as if a different amount were substituted for that mentioned there, or as if the amount were an aggregate amount and not an amount for each qualifying child, or both.
\end{enumerate}

\amendment{
Words inserted in para. 10A(1)($b$) (8.10.12) by the Child Maintenance and Other Payments Act 2008 (c. 6) Sch. 7 para. 1(30).
}

\subsection*{Regulations about income}

10B. The Secretary of State may by regulations provide that, in such circumstances and to such extent as may be prescribed---
\begin{enumerate}\item[]
($a$) where the Secretary of State is satisfied that a person has intentionally deprived himself of a source of income with a view to reducing the amount of his \opt{newrules}{net}\opt{2012rules}{gross} weekly income, his \opt{newrules}{net}\opt{2012rules}{gross} weekly income shall be taken to include income from that source of an amount estimated by the Secretary of State;

($b$) a person is to be treated as possessing income which he does not possess;

($c$) income which a person does possess is to be disregarded.
\end{enumerate}

\amendment{
Words substituted in para. 10B($a$) (1.8.12) by the Public Bodies (Child Maintenance and Enforcement Commission: Abolition and Transfer of Functions) Order 2012 Sch. para. 62(2)($c$).

\opt{2012rules}{
Word ``gross'' substituted for ``net'' in para. 10B($a$) (10.12.12) for the purposes of certain cases only (see S.I. 2012/3042) by the Child Maintenance and Other Payments Act 2008 (c. 6) Sch. 4 para. 2.
}

}

\subsection*{References to various terms}

10C.—(1) References in this Part of this Schedule to “qualifying children” are to those qualifying children with respect to whom the maintenance calculation falls to be made or with respect to whom a maintenance calculation in respect of the non-resident parent has effect.

(2) References in this Part of this Schedule to “relevant other children” are to---
\begin{enumerate}\item[]
($a$) children other than qualifying children in respect of whom the non-resident
parent or his partner receives child benefit under Part IX of the Social
Security Contributions and Benefits Act 1992; and

($b$) such other description of children as may be prescribed.
\end{enumerate}

(3) In this Part of this Schedule, a person “receives” a benefit, pension, or allowance for any week if it is paid or due to be paid to him in respect of that week.

(4) In this Part of this Schedule, a person’s “partner” is---
\begin{enumerate}\item[]
($a$) if they are a couple, the other member of that couple;

($b$) if the person is a husband or wife by virtue of a marriage entered into
under a law which permits polygamy, another party to the marriage who
is of the opposite sex and is a member of the same household.
\end{enumerate}

(5) In sub-paragraph (4)($a$), “couple” means---
\begin{enumerate}\item[]
($a$) a man and a woman who are married to each other and are members of
the same household,

($b$) a man and a woman who are not married to each other but are living
together as husband and wife,

($c$) two people of the same sex who are civil partners of each other and are
members of the same household, or

($d$) two people of the same sex who are not civil partners of each other but are
living together as if they were civil partners.
\end{enumerate}

(6) For the purposes of this paragraph, two people of the same sex are to be regarded as living together as if they were civil partners if, but only if, they would be regarded as living together as husband and wife were they instead two people of the opposite sex.

\amendment{
Para. 10C(5) substituted and 10C(6) inserted (5.12.05) by the Civil Partnership Act 2004 (c. 33) Sch. 24 para. 6.

Words inserted in para. 10C(1) (8.10.12) by the Child Maintenance and Other Payments Act 2008 (c. 6) Sch. 7 para. 1(31).
}

}

\opt{oldrules}{
\section[Part II --- General provisions about maintenance assessments]{Part II\\*General provisions about maintenance assessments}
}

\opt{newrules,2012rules}{
\section[Part II --- General provisions about maintenance calculations]{Part II\\*General provisions about maintenance calculations}

\amendment{
Words ``maintenance assessment(s)'' substituted by ``maintenance calculation(s)'' (3.3.03) for the purposes of certain cases only (see S.I. 2003/192) by the Child Support, Pensions and Social Security Act 2000 (c. 19) s. 1(2)($a$).
}
}

\renewcommand\parthead{--- Schedule 1 Part II}

\subsection*{Effective date of \opt{oldrules}{assessment}\opt{newrules,2012rules}{calculation}}

11.---(1) A \opt{oldrules}{maintenance assessment}\opt{newrules,2012rules}{maintenance calculation} shall take effect on such date as may be determined in accordance with regulations made by the Secretary of State.

(2) That date may be earlier than the date on which the \opt{oldrules}{assessment}\opt{newrules,2012rules}{calculation} is made.

\opt{newrules,2012rules}{
\amendment{
Words ``maintenance assessment(s)'' substituted by ``maintenance calculation(s)'' (3.3.03) for the purposes of certain cases only (see S.I. 2003/192) by the Child Support, Pensions and Social Security Act 2000 (c. 19) s. 1(2)($a$).

Word ``assessment'' (or any variant of that term) substituted by ``calculation'' (or other variants) (3.3.03) for the purposes of certain cases only (see S.I. 2003/192) by the Child Support, Pensions and Social Security Act 2000 (c. 19) s. 1(2)($a$).
}
}

\subsection*{Form of \opt{oldrules}{assessment}\opt{newrules,2012rules}{calculation}}

12. Every \opt{oldrules}{maintenance assessment}\opt{newrules,2012rules}{maintenance calculation} shall be made in such form and contain such information as the Secretary of State may direct.

\amendment{
\opt{newrules,2012rules}{
Words ``maintenance assessment(s)'' substituted by ``maintenance calculation(s)'' (3.3.03) for the purposes of certain cases only (see S.I. 2003/192) by the Child Support, Pensions and Social Security Act 2000 (c. 19) s. 1(2)($a$).

Word ``assessment'' (or any variant of that term) substituted by ``calculation'' (or other variants) (3.3.03) for the purposes of certain cases only (see S.I. 2003/192) by the Child Support, Pensions and Social Security Act 2000 (c. 19) s. 1(2)($a$).
}

Words substituted in para. 12 (1.8.12) by the Public Bodies (Child Maintenance and Enforcement Commission: Abolition and Transfer of Functions) Order 2012 Sch. para. 62(3)($a$).

\opt{newrules,2012rules}{
\medskip

Para. 13 ceases to have effect (3.3.03) for the purposes of certain cases only (see S.I. 2003/192) by the Child Support, Pensions and Social Security Act 2000 (c. 19) Sch. 3 para. 11(22)($a$).

}
}

\opt{oldrules}{

\subsection*{Assessments where amount of child support is nil}

13. The Secretary of State shall not decline to make a
maintenance assessment only on the ground that the amount of
the assessment is nil.

\amendment{
Words substituted in para. 13 (1.8.12) by the Public Bodies (Child Maintenance and Enforcement Commission: Abolition and Transfer of Functions) Order 2012 Sch. para. 62(3)($a$).
}
}

\subsection*{Consolidated applications and \opt{oldrules}{assessments}\opt{newrules,2012rules}{calculations}}

14. The Secretary of State may by regulations provide---
\begin{enumerate}\item[]
($a$) for two or more applications for \opt{oldrules}{maintenance assessments}\opt{newrules,2012rules}{maintenance calculations} to be treated, in prescribed circumstances, as a single application; and

($b$) for the replacement, in prescribed circumstances, of a \opt{oldrules}{maintenance assessment}\opt{newrules,2012rules}{maintenance calculation} made on the application of one person by a later \opt{oldrules}{maintenance assessment}\opt{newrules,2012rules}{maintenance calculation} made on the application of that or any other person.
\end{enumerate}

\opt{newrules,2012rules}{
\amendment{
Words ``maintenance assessment(s)'' substituted by ``maintenance calculation(s)'' (3.3.03) for the purposes of certain cases only (see S.I. 2003/192) by the Child Support, Pensions and Social Security Act 2000 (c. 19) s. 1(2)($a$).

Word ``assessment'' (or any variant of that term) substituted by ``calculation'' (or other variants) (3.3.03) for the purposes of certain cases only (see S.I. 2003/192) by the Child Support, Pensions and Social Security Act 2000 (c. 19) s. 1(2)($a$).

Para. 14 substituted (1.6.09) by the Child Maintenance and Other Payments Act 2008 (c. 6) Sch. 7 para. 1(32). 
}
}

\subsection*{Separate \opt{oldrules}{assessments}\opt{newrules,2012rules}{calculations} for different periods}

15. Where the Secretary of State is satisfied that the circumstances of a case require different amounts of child support maintenance to be assessed in respect of different periods, the Secretary of State may make separate \opt{oldrules}{maintenance assessments}\opt{newrules,2012rules}{maintenance calculations} each expressed to have effect in relation to a different specified period.

\amendment{
\opt{newrules,2012rules}{
Words ``maintenance assessment(s)'' substituted by ``maintenance calculation(s)'' (3.3.03) for the purposes of certain cases only (see S.I. 2003/192) by the Child Support, Pensions and Social Security Act 2000 (c. 19) s. 1(2)($a$).

Word ``assessment'' (or any variant of that term) substituted by ``calculation'' (or other variants) (3.3.03) for the purposes of certain cases only (see S.I. 2003/192) by the Child Support, Pensions and Social Security Act 2000 (c. 19) s. 1(2)($a$).
}

Words substituted in para. 15 (1.8.12) by the Public Bodies (Child Maintenance and Enforcement Commission: Abolition and Transfer of Functions) Order 2012 Sch. para. 62(3)($b$).
}

\subsection*{Termination of \opt{oldrules}{assessments}\opt{newrules,2012rules}{calculations}}

16.—(1) A \opt{oldrules}{maintenance assessment}\opt{newrules,2012rules}{maintenance calculation} shall cease to have effect---
\begin{enumerate}\item[]
($a$) on the death of the \opt{oldrules}{absent parent}\opt{newrules,2012rules}{non-resident parent}, or of the person with care, with respect to whom it was made;

($b$) on there no longer being any qualifying child with respect to whom it would have effect;

($c$) on the \opt{oldrules}{absent parent}\opt{newrules,2012rules}{non-resident parent} with respect to whom it was made ceasing to be a parent of---
\begin{enumerate}\item[]
(i)
the qualifying child with respect to whom it was made; or

(ii)
where it was made with respect to more than one qualifying child, all of the qualifying children with respect to whom it was made\opt{oldrules}{;}\opt{newrules,2012rules}{.}
\end{enumerate}

\opt{oldrules}{
($d$) where the absent parent and the person with care with
respect to whom it was made have been living together
for a continuous period of six months;

($e$) where a new maintenance assessment is made with
respect to any qualifying child with respect to whom the
assessment in question was in force immediately before
the making of the new assessment.
}
\end{enumerate}

\opt{oldrules}{

(2) A maintenance assessment made in response to an application under section 4 or 7 shall be cancelled by the Secretary of State if the person on whose application the assessment was made asks him to do so.

%[2(3) A maintenance assessment made in response to an application under section 6 shall be cancelled by [1the Secretary of State] if–
%($a$) the person on whose application the assessment was made (“the applicant”) asks him to do so; and
%($b$) he is satisfied that the applicant has ceased to fall within subsection (1) of that section.]
(4) Where the Secretary of State is satisfied that the person with care with respect to whom a maintenance assessment was made has ceased to be a person with care in relation to the qualifying child, or any of the qualifying children, with respect to whom the assessment was made, he may cancel the assessment with effect from the date on which, in his opinion, the change of circumstances took place.

(4A) A maintenance assessment may be cancelled by the Secretary of State if he is proposing to make a decision under section 16 or 17 and it appears to him–
\begin{enumerate}\item[]
($a$) that the person with care with respect to whom the
maintenance assessment in question was made has failed
to provide him with sufficient information to enable him
to make the decision%; and
%
%[4($b$) where the maintenance assessment in question was made
%in response to an application under section 6, that the
%person with care with respect to whom the assessment
%was made has ceased to fall within subsection (1) of that
%section.]]
.
\end{enumerate}

(5) Where---
\begin{enumerate}\item[]
($a$) at any time a maintenance assessment is in force but
the Secretary of State would no longer have jurisdiction
to make it if it were to be applied for at that time; and

($b$) the assessment has not been cancelled, or has not ceased
to have effect, under or by virtue of any other provision
made by or under this Act,
\end{enumerate}
it shall be taken to have continuing effect unless cancelled by the Secretary of State in accordance with such prescribed provision (including provision as to the effective date of cancellation) as the Secretary of State considers it appropriate to make.

(6)
Where both the absent parent and the person with care with respect to whom a maintenance assessment was made request the Secretary of State to cancel the assessment, he may do so if he is satisfied that they are living together.

(7)
Any cancellation of a maintenance assessment under sub-paragraph (4A), (5) or (6) shall have effect from such date as may be determined by the Secretary of State.

(8)
Where the Secretary of State cancels a maintenance assessment, he shall immediately notify the absent parent and person with care, so far as that is reasonably practicable.

(9)
Any notice under sub-paragraph (8) shall specify the date with effect from which the cancellation took effect.

}

(10) A person with care with respect to whom a \opt{oldrules}{maintenance assessment}\opt{newrules,2012rules}{maintenance calculation} is in force shall provide the Secretary of State with such information, in such circumstances, as may be prescribed, with a view to assisting the Secretary of State in determining whether the \opt{oldrules}{assessment}\opt{newrules,2012rules}{calculation} has ceased to have effect\opt{oldrules}{,
%Words in sub-para. (10) of para. 16 shall cease to have effect (3.3.03) for the purposes of certain cases only (see S.I. 2003/192 at page 4141) by the Child Support, Pensions & Social Security Act 2000 (c. 19), Sch. 3, para. 11(22)($c$)(iii).
or should be cancelled}.

(11) The Secretary of State may by regulations make such supplemental, incidental or transitional provision as he thinks necessary or expedient in consequence of the provisions of this paragraph.

\amendment{
\opt{oldrules}{

Para. 16(4A) inserted (22.1.96) by the Child Support Act 1995 (c. 34) s. 14(2). %OR

Words inserted in para 16(7) (22.1.96) by the Child Support Act 1995 (c. 34) s. 14(3). %OR

Words in para. 16 substituted (1.6.99) by the Social Security Act 1998 (c. 14) Sch. 7 para. 48(5). %OR

}

Words repealed in para. 16(10) (1.6.99) by the Social Security Act 1998 (c. 14) Sch. 8. %OR


\opt{newrules,2012rules}{

Words ``maintenance assessment(s)'' substituted by ``maintenance calculation(s)'' (3.3.03) for the purposes of certain cases only (see S.I. 2003/192) by the Child Support, Pensions and Social Security Act 2000 (c. 19) s. 1(2)($a$). %NR

Word ``assessment'' (or any variant of that term) substituted by ``calculation'' (or other variants) (3.3.03) for the purposes of certain cases only (see S.I. 2003/192) by the Child Support, Pensions and Social Security Act 2000 (c. 19) s. 1(2)($a$). %NR

Words ``(an) absent parent'' substituted by ``($a$) non-resident parent'' (3.3.03) for the purposes of certain cases only (see S.I. 2003/192) by the Child Support, Pensions and Social Security Act 2000 (c. 19) Sch. 3 para. 11(2). %NR

Para. 16(1)($d$), ($e$), (2)--(9) and words in para. 16(10) cease to have effect (3.3.03) for the purposes of certain cases only (see S.I. 2003/192) by the Child Support, Pensions and Social Security Act 2000 (c. 19) Sch. 3 para. 11(22)($c$). %NR

}

\opt{oldrules}{

Para. 16(3), (4A)($b$) repealed (14.7.08 for certain cases, 1.6.09 for all other purposes) by the Child Maintenance and Other Payments Act 2008 (c. 6) Sch. 7 para. 1(34). %OR

}

Words substituted in para. 16(10) (1.8.12) by the Public Bodies (Child Maintenance and Enforcement Commission: Abolition and Transfer of Functions) Order 2012 Sch. para. 62(3)($c$).

\medskip

Sch. 2 ceased to have effect (1.11.08) by the Child Maintenance and Other Payments Act 2008 (c. 6) Sch. 7 para. 1(33).

\medskip

Sch. 3 repealed (1.6.99) by the Social Security Act 1998 (c. 14) Sch. 8.

}

\part[Schedule 4 --- Child Support Commissioners for Northern Ireland]{Schedule 4\\* Child Support Commissioners for Northern Ireland}

\renewcommand\parthead{--- Schedule 4}

\amendment{
Words inserted in title of Sch. 4 (3.11.08) by the Transfer of Tribunal Functions Order 2008  Sch. 3 para. 97(2).
}

\subsection*{Tenure of office}

1.—(1) Every Child Support Commissioner for Northern Ireland shall vacate his office on the date on which he reaches the age of 70; but this sub-paragraph is subject to section 26(4) to (6) of the Judicial Pensions and Retirement Act 1993 (power to authorise continuance in office up to the age of 75).

\amendment{
Words substituted in para. 1(1) (31.3.95) by the Judicial Pensions and Retirement Act 1993 (c. 8) Sch. 6 para. 23(2)($a$).

Para. 1(2) repealed (31.3.95) by the Judicial Pensions and Retirement Act 1993 (c. 8) Sch. 6 para. 23(2)($b$) and Sch. 9.

Words inserted in para. 1(1) and para. 1(3)--(3B) omitted (3.11.08) by the Transfer of Tribunal Functions Order 2008  Sch. 3 para. 97(3).

}

\subsection*{Commissioners' remuneration and their pensions}

2.—(1) The Department of Justice shall pay, or make such payments towards the provision of such remuneration, allowances or gratuities to or in respect of persons appointed as Child Support Commissioners for Northern Ireland as, with the consent of the Treasury, the Lord Chancellor may determine.

(2) The Department of Justice shall pay to a Child Support Commissioner for Northern Ireland such expenses incurred in connection with his work as such a Commissioner as may be determined by the Treasury.

(3) Sub-paragraph (1), so far as relating to pensions, allowances or gratuities, shall not have effect in relation to any person to whom Part I of the Judicial Pensions and Retirement Act 1993 applies, except to the extent provided by or under that Act.

\amendment{
Para. 2(3) inserted (31.3.95) by the Judicial Pensions and Retirement Act 1993 (c. 8) Sch. 8 para. 21(2).

Words omitted in para. 2(1) and inserted in para. 2(1), (2) (3.11.08) by the Transfer of Tribunal Functions Order 2008  Sch. 3 para. 97(4).

Words substituted in para. 2 (12.4.10) by the Northern Ireland Act 1998 (Devolution of Policing and Justice Functions) Order 2010 Sch. 18 para. 42.

\medskip

Para. 2A omitted (3.11.08) by the Transfer of Tribunal Functions Order 2008  Sch. 3 para. 97(5).

}

\subsection*{Commissioners barred from legal practice}

3. A Child Support Commissioner for Northern Ireland, so long as he holds office as such, shall not practise as a barrister or act for any remuneration to himself as arbitrator or referee or be directly or indirectly concerned in any matter as a conveyancer, notary public or solicitor.

\amendment{
Para. 3 substituted (3.11.08) by the Transfer of Tribunal Functions Order 2008  Sch. 3 para. 97(6).
}

\subsection*{Deputy Child Support Commissioners}

4.—(1) The Northern Ireland Judicial Appointments Commission may appoint persons to act as Child Support Commissioners for Northern Ireland (but to be known as deputy Child Support Commissioners for Northern Ireland) in order to facilitate the disposal of the business of Child Support Commissioners for Northern Ireland.

(2) A deputy Child Support Commissioner for Northern Ireland shall be appointed---
\begin{enumerate}\item[]
($a$) from among persons who are barristers or solicitors of not less than the number
of years’ standing specified in section 23(2), and

($b$) subject to sub-paragraph (2A), for such period or on such occasions as
the Commission determines with the agreement of the Department of Justice.
\end{enumerate}

(2A) No appointment of a person to be a deputy Child Support Commissioner for Northern Ireland shall be such as to extend beyond the date on which he reaches the age of 70; but this sub-paragraph is subject to section 26(4) to (6) of the Judicial Pensions and Retirement Act 1993 (power to authorise continuance in office up to the age of 75).

(3) Paragraph 2 applies to Deputy Child Support Commissioners for Northern Ireland, but paragraph 3 does not apply to them.

\amendment{
Para. 4(2A) inserted (31.3.95) by the Judicial Pensions and Retirement Act 1993 (c. 8) Sch. 6 para. 23(3).

Words inserted in para. 4(2A) and para. 4(3) substituted (3.11.08) by the Transfer of Tribunal Functions Order 2008  Sch. 3 para. 97(7).

Para. 4(1), (2) substituted (12.3.09) by the Northern Ireland Act 2009 (c. 48) Sch. 4 para. 23.

Words substituted in para. 4(2)($b$) (12.4.10) by the Department of Justice Act (Northern Ireland) 2010 (c. 3 (N.I.)) Sch. para. 7.

\medskip

Paras. 4A--8 omitted (3.11.08) by the Transfer of Tribunal Functions Order 2008  Sch. 3 para. 97(8).

}

\opt{oldrules}{

\part[Schedule 4A --- Departure directions]{Schedule 4A\\*Departure directions}

\renewcommand\parthead{--- Schedule 4A}

\amendment{
Sch. 4A inserted by the Child Support Act 1995 (c. 34) Sch. 1.
}

\subsection*{Interpretation}

1. In this Schedule---
\begin{enumerate}\item[]
“departure application” means an application for a departure direction;

“regulations” means regulations made by the Secretary of State.
\end{enumerate}

\amendment{
Words repealed in para. 1 (1.6.99) by the Social Security Act 1998 (c. 14) Sch. 8.
}

\subsection*{Applications for departure directions}

2. Regulations may make provision---
\begin{enumerate}\item[]
($a$) as to the procedure to be followed in considering a departure application;

($c$) for the giving of a direction by the Secretary of State as to the order in which,
in a particular case, a decision on a departure application and a decision
under section 16 or 17 are to be made;

($d$) for the reconsideration of a departure application in a case where further
information becomes available to the Secretary of State after the application
has been determined.
\end{enumerate}

\amendment{
Words in para. 2($c$) substituted (1.6.99) by the Social Security Act 1998 (c. 14) Sch. 7 para. 53(2)($b$).

Para. 2($b$) omitted (3.11.08) by the Transfer of Tribunal Functions Order 2008  Sch. 3 para. 98($a$).
}

\subsection*{Completion of preliminary consideration}

3. Regulations may provide for determining when the preliminary consideration of a departure application is to be taken to have been completed.

\subsection*{Information}

4.—(1) Regulations may make provision for the use for any purpose of this Act of---
\begin{enumerate}\item[]
($a$) information acquired by the Secretary of State in connection with an
application for, or the making of, a departure direction;

($b$) information acquired by the Secretary of State in connection with
an application for, or the making of, a maintenance assessment.
\end{enumerate}

(2) If any information which is required (by regulations under this Act) to be furnished to the Secretary of State in connection with a departure application has not been furnished within such period as may be prescribed, the Secretary of State may nevertheless proceed to determine the application.

\amendment{
Words in para. 4(1)($b$) repealed (1.6.99) by the Social Security Act 1998 (c. 14) Sch. 8.

Words substituted in para. 4 (1.8.12) by the Public Bodies (Child Maintenance and Enforcement Commission: Abolition and Transfer of Functions) Order 2012 Sch. para. 63.
}

\subsection*{Anticipation of change of circumstances}

5.—(1) A departure direction may be given so as to provide that if the circumstances of the case change in such manner as may be specified in the direction a fresh maintenance assessment is to be made.

\amendment{
Para. 5(2) repealed (27.10.08) by the Child Maintenance and Other Payments Act 2008 (c. 6) Sch. 8.

\medskip

Para. 6 repealed (1.6.99) by the Social Security Act 1998 (c. 14) Sch. 8.

}

\subsection*{Subsequent departure directions}

7.—(1) Regulations may make provision with respect to any departure application
made with respect to a maintenance assessment which was made as a result of a departure
direction.

(2) The regulations may, in particular, provide for the application to be considered by reference to the maintenance assessment which would have been made had the departure direction not been given.

\subsection*{Joint consideration of departure applications and appeals}

8.—(1) Regulations may provide for two or more departure applications with respect to the same current assessment to be considered together.

\amendment{
Para. 8(2) omitted (3.11.08) by the Transfer of Tribunal Functions Order 2008  Sch. 3 para. 98($a$).

\medskip

Para. 9 omitted (3.11.08) by the Transfer of Tribunal Functions Order 2008  Sch. 3 para. 98($b$).
}

\subsection*{Current assessments which are replaced by fresh assessments}

10. Regulations may make provision as to the circumstances in which prescribed references in this Act to a current assessment are to have effect as if they were references to any later maintenance assessment made with respect to the same persons as the current assessment.

\part[Schedule 4B --- Departure directions: the cases and controls]{Schedule 4B\\*Departure directions: the cases and controls}

\amendment{
Sch. 4B inserted by the Child Support Act 1995 (c. 34) Sch. 2.
}

\section[Part I --- The cases]{Part I\\*The cases}

\renewcommand\parthead{--- Schedule 4B Part I}

\subsection*{General}

1.—(1) The cases in which a departure direction may be given are those set out in this Part of this Schedule or in regulations made under this Part.

(2) In this Schedule “applicant” means the person whose application for a departure direction is being considered.

\subsection*{Special expenses}

2.—(1) A departure direction may be given with respect to special expenses of the applicant which were not, and could not have been, taken into account in determining the current assessment in accordance with the provisions of, or made under, Part I of Schedule 1.

(2)
In this paragraph “special expenses” means the whole, or any prescribed part, of expenses which fall within a prescribed description of expenses.

(3)
In prescribing descriptions of expenses for the purposes of this paragraph, the Secretary of State may, in particular, make provision with respect to---
\begin{enumerate}\item[]
($a$) costs incurred in travelling to work;

($b$) costs incurred by an absent parent in maintaining contact with the child, or
with any of the children, with respect to whom he is liable to pay child
support maintenance under the current assessment;

($c$) costs attributable to a long-term illness or disability of the applicant or of a
dependant of the applicant;

($d$) debts incurred, before the absent parent became an absent parent in relation
to a child with respect to whom the current assessment was made---
\begin{enumerate}\item[]
(i)
for the joint benefit of both parents;

(ii)
for the benefit of any child with respect to whom the current assessment
was made; or

(iii) for the benefit of any other child falling within a prescribed category;
\end{enumerate}

($e$) pre-1993 financial commitments from which it is impossible for the parent
concerned to withdraw or from which it would be unreasonable to expect
that parent to have to withdraw;

($f$) costs incurred by a parent in supporting a child who is not his child but who
is part of his family.
\end{enumerate}

(4) For the purposes of sub-paragraph (3)($c$)---
\begin{enumerate}\item[]
($a$) the question whether one person is a dependent of another shall be determined
in accordance with regulations made by the Secretary of State;

($b$) “disability” and “illness” have such meaning as may be prescribed; and

($c$) the question whether an illness or disability is long-term shall be determined
in accordance with regulations made by the Secretary of State.
\end{enumerate}

(5)
For the purposes of sub-paragraph (3)($e$), “pre-1993 financial commitments” means financial commitments of a prescribed kind entered into before 5th April 1993 in any case where---
\begin{enumerate}\item[]
($a$) a court order of a prescribed kind was in force with respect to the absent
parent and the person with care concerned at the time when they were entered
into; or

($b$) an agreement between them of a prescribed kind was in force at that time.
\end{enumerate}

(6)
For the purposes of sub-paragraph (3)($f$), a child who is not the child of a particular person is a part of that person’s family in such circumstances as may be prescribed.

\subsection*{Property or capital transfers}

3.—(1) A departure direction may be given if----
\begin{enumerate}\item[]
($a$) before 5th April 1993---
\begin{enumerate}\item[]
(i) a court order of a prescribed kind was in force with respect to the absent
parent and either the person with care with respect to whom the current
assessment was made or the child, or any of the children, with respect to
whom that assessment was made; or

(ii) an agreement of a prescribed kind between the absent parent and any of
those persons was in force;
\end{enumerate}

($b$) in consequence of one or more transfers of property of a prescribed kind---
\begin{enumerate}\item[]
(i) the amount payable by the absent parent by way of maintenance was
less than would have been the case had that transfer or those transfers
not been made; or

(ii) no amount was payable by the absent parent by way of maintenance;
and
\end{enumerate}

($c$) the effect of that transfer, or those transfers, is not properly reflected in the
current assessment.
\end{enumerate}

(2)
For the purposes of sub-paragraph (1)($b$), “maintenance” means periodical payments of maintenance made (otherwise than under this Act) with respect to the child, or any of the children, with respect to whom the current assessment was made.

(3)
For the purposes of sub-paragraph (1)($c$), the question whether the effect of one or more transfers of property is properly reflected in the current assessment shall be determined in accordance with regulations made by the Secretary of State.

\medskip

4.—(1) A departure direction may be given if---
\begin{enumerate}\item[]
($a$) before 5th April 1993---
\begin{enumerate}\item[]
(i) a court order of a prescribed kind was in force with respect to the absent parent and either the person with care with respect to whom the current assessment was made or the child, or any of the children, with respect to whom that assessment was made, or

(ii) an agreement of a prescribed kind between the absent parent and any of those persons was in force;
\end{enumerate}

($b$) in pursuance of the court order or agreement, the absent parent has made one or more transfers of property of a prescribed kind;

($c$) the amount payable by the absent parent by way of maintenance was not reduced as a result of that transfer or those transfers;

($d$) the amount payable by the absent parent by way of child support maintenance under the current assessment has been reduced as a result of that transfer or those transfers, in accordance with provisions of or made under this Act; and

($e$) it is nevertheless inappropriate, having regard to the purposes for which the transfer or transfers was or were made, for that reduction to have been made.
\end{enumerate}

(2) For the purposes of sub-paragraph (1)($c$), “maintenance” means periodical payments of maintenance made (otherwise than under this Act) with respect to the child, or any of the children, with respect to whom the current assessment was made.

\subsection*{Additional cases}

5.—(1) The Secretary of State may by regulations prescribe other cases in which a departure direction may be given.

(2) Regulations under this paragraph may, for example, make provision with respect to cases where---
\begin{enumerate}\item[]
($a$) assets which do not produce income are capable of producing income;

($b$) a person’s life-style is inconsistent with the level of his income;

($c$) housing costs are unreasonably high;

($d$) housing costs are in part attributable to housing persons whose circumstances are such as to justify disregarding a part of those costs;

($e$) travel costs are unreasonably high; or

($f$) travel costs should be disregarded.
\end{enumerate}

\section[Part II --- Regulatory controls]{Part II\\*Regulatory controls}

\renewcommand\parthead{--- Schedule 4B Part II}

6.—(1) The Secretary of State may by regulations make provision with respect to the directions which may be given in a departure direction.

(2) No directions may be given other than those which are permitted by the regulations.

(3)
Regulations under this paragraph may, in particular, make provision for a departure direction to require---
\begin{enumerate}\item[]
($a$) the substitution, for any formula set out in Part I of Schedule 1, of such other
formula as may be prescribed;

($b$) any prescribed amount by reference to which any calculation is to be made
in fixing the amount of child support maintenance to be increased or reduced
in accordance with the regulations;

($c$) the substitution, for any provision in accordance with which any such
calculation is to be made, of such other provision as may be prescribed.
\end{enumerate}

(4)
Regulations may limit the extent to which the amount of the child support maintenance fixed by a maintenance assessment made as a result of a departure direction may differ from the amount of the child support maintenance which would be fixed by a maintenance assessment made otherwise than as a result of the direction.

(5)
Regulations may provide for the amount of any special expenses to be taken into account in a case falling within paragraph 2, for the purposes of a departure direction, not to exceed such amount as may be prescribed or as may be determined in accordance with the regulations.

(6)
No departure direction may be given so as to have the effect of denying to an absent parent the protection of paragraph 6 of Schedule 1.

(7)
Sub-paragraph (6) does not prevent the modification of the provisions of, or made under, paragraph 6 of Schedule 1 to the extent permitted by regulations under this paragraph.

(8)
Any regulations under this paragraph may make different provision with respect to different levels of income.

}

\opt{newrules,2012rules}{

\part[Schedule 4A --- Applications for a variation]{Schedule 4A\\*Applications for a variation}

\renewcommand\parthead{--- Schedule 4A}

\amendment{
Sch. 4A substituted (10.11.00 for the purposes of making regulations and Acts of Sederunt only, 3.3.03) for the purposes of certain cases only (see S.I. 2003/192) by the Child Support, Pensions and Social Security Act 2000 (c. 19) Sch. 2.
}

\subsection*{Interpretation}

1. In this Schedule, “regulations” means regulations made by the Secretary of State.

\subsection*{Applications for a variation}

2. Regulations may make provision as to the procedure to be followed in considering an application for a
variation.

\amendment{
Para. 2($b$) omitted (3.11.08) by the Transfer of Tribunal Functions Order 2008 Sch. 3 para. 99.
}

\subsection*{Completion of preliminary consideration}

3. Regulations may provide for determining when the preliminary consideration of an application for a variation is to be taken to have been completed.

\subsection*{Information}

4. If any information which is required (by regulations under this Act) to be furnished to the Secretary of State in connection with an application for a variation has not been furnished within such period as may be prescribed, the Secretary of
State may nevertheless proceed to consider the application.

\amendment{
Words substituted in para. 4 (1.8.12) by the Public Bodies (Child Maintenance and Enforcement Commission: Abolition and Transfer of Functions) Order 2012 Sch. para. 63.
}

\subsection*{Joint consideration of applications for a variation and appeals}

5.---(1) Regulations may provide for two or more applications for a variation with respect to the same application for a maintenance calculation to be considered together.

%(2)
%In sub-paragraph (1), the reference to an application for a maintenance calculation includes an application treated as having been made under section 6.

\amendment{
Sch. 4A, para. 5(1) modified (31.1.01) where an application for a variation is made under s. 28G by reg. 8 of S.I. 2000/3173 to read:
\begin{quotation}
5.—(1) Regulations may provide for two or more applications for a variation with respect to the same maintenance calculation to be considered together.
\end{quotation}

Para. 5(2) repealed (27.10.08) by the Child Maintenance and Other Payments Act 2008 (c. 14) Sch. 8.

Para. 5(3) omitted (3.11.08) by the Transfer of Tribunal Functions Order 2008 Sch. 3 para. 99.

}

\part[Schedule 4B --- Applications for a variation: the cases and controls]{Schedule 4B\\*Applications for a variation: the cases and controls}

\section[Part I --- The cases]{Part I\\*The cases}

\renewcommand\parthead{--- Schedule 4B Part I}

\amendment{
Sch. 4B substituted (10.11.00 for the purposes of making regulations and Acts of Sederunt only, 3.3.03) for the purposes of certain cases only (see S.I. 2003/192) by the Child Support, Pensions and Social Security Act 2000 (c. 19) Sch. 2.
}

\subsection*{General}

1.—(1) The cases in which a variation may be agreed are those set out in this Part of this Schedule or in regulations made under this Part.

(2) In this Schedule “applicant” means the person whose application for a
variation is being considered.

\subsection*{Special expenses}

2.—(1) A variation applied for by a non-resident parent may be agreed with respect to his special expenses.

(2) In this paragraph “special expenses” means the whole, or any amount above a prescribed amount, or any prescribed part, of expenses which fall within a prescribed description of expenses.

(3) In prescribing descriptions of expenses for the purposes of this paragraph, the Secretary of State may, in particular, make provision with respect to---
\begin{enumerate}\item[]
($a$) costs incurred by a non-resident parent in maintaining contact with the child, or with any of the children, with respect to whom the application for a maintenance calculation has been made;

($b$) costs attributable to a long-term illness or disability of a relevant other child (within the meaning of paragraph 10C(2) of Schedule 1);

($c$) debts of a prescribed description incurred, before the non-resident parent became a non-resident parent in relation to a child with respect to whom the maintenance calculation has been applied for---
\begin{enumerate}\item[]
(i)
for the joint benefit of both parents;

(ii)
for the benefit of any such child; or

(iii) for the benefit of any other child falling within a prescribed category;
\end{enumerate}

($d$) boarding school fees for a child in relation to whom the application for a maintenance calculation has been made;

($e$) the cost to the non-resident parent of making payments in relation to a mortgage on the home he and the person with care shared, if he no longer has an interest in it, and she and a child in relation to whom the application for a maintenance calculation has been made still live there.
\end{enumerate}

(4) For the purposes of sub-paragraph (3)($b$)---
\begin{enumerate}\item[]
($a$) “disability” and “illness” have such meaning as may be prescribed; and

($b$) the question whether an illness or disability is long-term shall be determined in accordance with regulations made by the Secretary of State.
\end{enumerate}

(5) For the purposes of sub-paragraph (3)($d$), the Secretary of State may prescribe---
\begin{enumerate}\item[]
($a$) the meaning of “boarding school fees”; and

($b$) components of such fees (whether or not itemised as such) which are, or are not, to be taken into account,
\end{enumerate}
and may provide for estimating any such component.

\amendment{
Para. 2(3) modified (31.1.01), where an application for a variation is made under s. 28G, by reg. 8 of S.I. 2000/3173 to read:
\begin{quotation}
(3) In prescribing descriptions of expenses for the purposes of this paragraph the Secretary of State may, in particular,make provision with respect to---
\begin{enumerate}\item[]
($a$) costs incurred by a non-resident parent in maintaining contact with the child or with any of the children, with respect to whom there is a maintenance calculation in force;

($b$) costs attributable to a long-term illness or disability of a relevant other child (within the meaning of paragraph 10C(2) of Schedule 1);

($c$) debts of a prescribed description incurred, before the non-resident parent became a non-resident parent in relation to a child with respect to whom the maintenance calculation is in force---
\begin{enumerate}\item[]
(i)
for the joint benefit of both parents;

(ii)
for the benefit of any such child; or

(iii)
for the benefit of any other child falling within a
prescribed category;
\end{enumerate}

($d$) boarding school fees for a child in relation to whom there is a maintenance calculation in force;

($e$) the cost to the non-resident parent of making payments in relation to a mortgage on the home he and the person with care shared, if he no longer has an interest in it, and she and a child in relation to whom there is a maintenance calculation in force still live there.
\end{enumerate}
\end{quotation}

Words in para. 2(3) repealed (27.10.08) by the Child Maintenance and Other Payments Act 2008 (c. 6) Sch. 8.
}

\subsection*{Property or capital transfers}

3.—(1) A variation may be agreed in the circumstances set out in sub-paragraph
(2) if before 5th April 1993---
\begin{enumerate}\item[]
 ($a$) a court order of a prescribed kind was in force with respect to the non-resident parent and either the person with care with respect to the
application for the maintenance calculation or the child, or any of the children, with respect to whom that application was made; or

($b$) an agreement of a prescribed kind between the non-resident parent and any of those persons was in force.
\end{enumerate}

(2) The circumstances are that in consequence of one or more transfers of property of a prescribed kind and exceeding (singly or in aggregate) a prescribed minimum value---
\begin{enumerate}\item[]
($a$) the amount payable by the non-resident parent by way of maintenance was less than would have been the case had that transfer or those transfers not been made; or

($b$) no amount was payable by the non-resident parent by way of maintenance.
\end{enumerate}

(3) For the purposes of sub-paragraph (2), “maintenance” means periodical payments of maintenance made (otherwise than under this Act) with respect to the child, or any of the children, with respect to whom the application for a maintenance calculation has been made.

\amendment{
Para. 3(1)($a$) modified (31.1.01), where an application for a variation is made under s. 28G, by reg. 8 of S.I. 2000/3173 to read:
\begin{quotation}
($a$) a court order of a prescribed kind was in force with respect to the non-resident parent and either the person with care with respect to the maintenance calculation in force or the child, or any of the children, with respect to whom that maintenance calculation is in force; or
\end{quotation}

Para. 3(3) modified (31.1.01), where an application for a variation is made under s. 28G, by reg. 8 of S.I. 2000/3173 to read:
\begin{quotation}
(3) For the purposes of sub-paragraph (2), “maintenance” means periodical payments of maintenance made (otherwise than under this Act) with respect to the child, or any of the children, with respect to whom the maintenance calculation in force has been made.
\end{quotation}

}

\subsection*{Additional cases}

4.—(1) The Secretary of State may by regulations prescribe other cases in which a variation may be agreed.

(2) Regulations under this paragraph may, for example, make provision with respect to cases where---
\begin{enumerate}\item[]
($a$) the non-resident parent has assets which exceed a prescribed value;

($b$) a person’s lifestyle is inconsistent with his income for the purposes of a calculation made under Part I of Schedule 1;

($c$) a person has income which is not taken into account in such a calculation;

($d$) a person has unreasonably reduced the income which is taken into account in such a calculation.
\end{enumerate}

\section[Part II --- Regulatory controls]{Part II\\*Regulatory controls}

5.—(1) The Secretary of State may by regulations make provision with respect
to the variations from the usual rules for calculating maintenance which may be
allowed when a variation is agreed.

(2)
No variations may be made other than those which are permitted by the regulations.

(3)
Regulations under this paragraph may, in particular, make provision for a variation to result in---
\begin{enumerate}\item[]
($a$) a person’s being treated as having more, or less, income than would
be taken into account without the variation in a calculation under
Part I of Schedule 1;

($b$) a person’s being treated as liable to pay a higher, or a lower, amount of
child support maintenance than would result without the variation from a
calculation under that Part.
\end{enumerate}

(4)
Regulations may provide for the amount of any special expenses to be taken into account in a case falling within paragraph 2, for the purposes of a variation, not to exceed such amount as may be prescribed or as may be determined in accordance with the regulations.

(5)
Any regulations under this paragraph may in particular make different provision with respect to different levels of income.

(6)
The Secretary of State may by regulations provide for the application, in connection with child support maintenance payable following a variation, of paragraph 7(2) to (7) of Schedule 1 (subject to any prescribed modifications).

\amendment{
Sch. 4C repealed (3.3.03) for the purposes of certain cases only (see S.I. 2003/192) by the Child Support, Pensions and Social Security Act 2000 (c. 19) Sch. 9 Pt. I.
}

}

\opt{oldrules}{

\part[Schedule 4C --- Decisions and appeals: departure directions and reduced benefit directions etc.]{Schedule 4C\\*Decisions and appeals: departure directions and reduced benefit directions etc.}

\renewcommand\parthead{--- Schedule 4C}

\amendment{
Schedule 4C inserted (4.3.99) by the Social Security Act 1998 (c. 14) Sch. 7 para. 54.
}

\subsection*{Revision of decisions}

1. Section 16 shall apply in relation to---
\begin{enumerate}\item[]
($a$) any decision of the Secretary of State with respect to a departure direction or a person’s liability under section 43;

($b$) any decision of the Secretary of State under section 17 as extended by paragraph 2; and

($c$) any decision of the First-tier Tribunal on a referral under section 28D(1)($b$),
\end{enumerate}
as it applies in relation to any decision of the Secretary of State under section 11, 12 or 17.

\amendment{
Words in para. 1($a$) repealed (14.7.08 for certain cases, 1.6.09 for all other purposes) by the Child Maintenance and Other Payments Act 2008 (c. 6) Sch. 7 para. 1(34).

Words substituted in para. 1($c$) (3.11.08) by the Transfer of Tribunal Functions Order 2008  Sch. 3 para. 100(2).

}

\subsection*{Decisions superseding earlier decisions}

2.---(1) Section 17 shall apply in relation to---
\begin{enumerate}\item[]
($a$) any decision of the Secretary of State with respect to a departure direction or a person’s liability under section 43;

($b$) any decision of the Secretary of State under section 17 as extended by this sub-paragraph; and

($c$) any decision of 
an appeal tribunal or  % Words inserted (3.11.08) by 2012 c 5 Sch 12 para 3(2)
the First-tier Tribunal on a referral under section 28D(1)($b$),
\end{enumerate}
whether as originally made or as revised under section 16 as extended by paragraph 1, as it applies in relation to any decision of the Secretary of State under section 11, 12 or 17, whether as originally made or as revised under section 16.

(2) Section 17 shall apply in relation to any decision of 
an appeal tribunal or  % Words inserted (3.11.08) by 2012 c 5 Sch 12 para 3(3)
the First-tier Tribunal under section 20 as extended by paragraph 3 as it applies in relation to any decision of
an appeal tribunal or  % Words inserted (3.11.08) by 2012 c 5 Sch 12 para 3(3)
the First-tier Tribunal under section 20.

% Para 2(3) inserted (3.11.08) by 2012 c 5 Sch 12 para 3(4)
(3) In this paragraph “appeal tribunal“ means an appeal tribunal constituted under Chapter I of Part I of the Social Security Act 1998 (the functions of which have been transferred to the First-tier Tribunal).

\amendment{
Words in para. 2(1)($a$) repealed (14.7.08 for certain cases, 1.6.09 for all other purposes) by the Child Maintenance and Other Payments Act 2008 (c. 6) Sch. 7 para. 1(34).

Words substituted in para. 2 (3.11.08) by the Transfer of Tribunal Functions Order 2008  Sch. 3 para. 100(3).

Words inserted in para. 2(1), (2) and para. 2(3) inserted (3.11.08) by the Welfare Reform Act 2012 (c. 5) Sch. 12 para. 3.

}

\subsection*{Appeals to appeal tribunals}

3.---(1) Subject to sub-paragraphs (2) and (3), section 20 shall apply---
\begin{enumerate}\item[]
($a$) in relation to a qualifying person who is aggrieved by any decision of the Secretary of State with respect to a departure direction; and

($b$) in relation to any person who is aggrieved by a decision of the Secretary of State%---
%\begin{enumerate}\item[]
%(i) with respect to a reduced benefit direction; or
%
%(ii) 
{} with respect to a person’s liability under section 43,
\end{enumerate}
as it applies in relation to a person whose application for a maintenance assessment is refused or to such a person as is mentioned in subsection (2) of section 20.

(2) On an appeal under section 20 as extended by sub-paragraph (1)($a$), the First-tier Tribunal shall---
\begin{enumerate}\item[]
($a$) consider the matter---
\begin{enumerate}\item[]
(i) as if it were exercising the powers of the Secretary of State in relation to the application in question; and

(ii) as if it were subject to the duties imposed on him in relation to that application;
\end{enumerate}

($b$) have regard to any representations made to it by the Secretary of State; and

($c$) confirm the decision or replace it with such decision as the tribunal considers appropriate.
\end{enumerate}

%(3) No appeal shall lie under section 20 as extended by sub-paragraph (1)($b$)(i) unless the amount of the person’s benefit is reduced in accordance with the reduced benefit direction; and the time within which such an appeal may be brought shall run from the date of the notification of the reduction.

(4) In sub-paragraph (1) “qualifying person” means---
\begin{enumerate}\item[]
($a$) the person with care, or absent parent, with respect to whom the current assessment was made; or

($b$) where the application for the current assessment was made under section 7, either of those persons or the child concerned.
\end{enumerate}

\amendment{
Paras. 3(1)($b$)(i), (3) repealed (14.7.08 for certain cases, 1.6.09 for all other purposes) by the Child Maintenance and Other Payments Act 2008 (c. 6) Sch. 7 para. 1(34).

Words substituted in para. 3(2) (3.11.08) by the Transfer of Tribunal Functions Order 2008  Sch. 3 para. 100(4).
}

\subsection*{Decisions and appeals dependent on other cases}

4.---(1) Section 28ZA shall also apply where---
\begin{enumerate}\item[]
($a$) a decision falls to be made—
\begin{enumerate}\item[]
(i) with respect to a departure direction
%, a reduced benefit direction
 or a person’s liability under section 43, by the Secretary of State; or

(ii) with respect to a departure direction, by the First-tier Tribunal on a referral under section 28D(1)($b$); and
\end{enumerate}

($b$) an appeal is pending against a decision given with respect to a different direction by the Upper Tribunal or a court.
\end{enumerate}

(2) Section 28ZA as it applies by virtue of sub-paragraph (1) shall have effect as if the reference in subsection (3) to section 16 were a reference to that section as extended by paragraph 1.

(3) Section 28ZA as it applies by virtue of sub-paragraph (1)($a$)(ii) shall have effect as if---
\begin{enumerate}\item[]
($a$) in subsection (2)---
\begin{enumerate}\item[]
(i) for the words “the Secretary of State” there were substituted the words “the First-tier Tribunal”; and

(ii) for the word “he”, in both places where it occurs, there were substituted the word “it”; and
\end{enumerate}

($b$) in subsection (3)---
\begin{enumerate}\item[]
(i) for the words “the Secretary of State” there were substituted the words “the First-tier Tribunal”;

(ii) for the word “he” there were substituted the words “the Secretary of State”; and

(iii) for the word “his” there were substituted the words “the tribunal's”.
\end{enumerate}
\end{enumerate}

\amendment{
Words in paras. 4(1)($a$)(i) repealed (14.7.08 for certain cases, 1.6.09 for all other purposes) by the Child Maintenance and Other Payments Act 2008 (c. 6) Sch. 7 para. 1(34).

Words substituted in para. 4 (3.11.08) by the Transfer of Tribunal Functions Order 2008  Sch. 3 para. 100(5).
}

\medskip

5.---(1) Section 28ZB shall also apply where---
\begin{enumerate}\item[]
($a$) an appeal is made to the First-tier Tribunal under section 20 as extended by paragraph 3; and

($b$) an appeal is pending against a decision given in a different case by the Upper Tribunal or a court.
\end{enumerate}

(2) Section 28ZB as it applies by virtue of sub-paragraph (1) shall have effect as if any reference to section 16 or section 17 were a reference to that section as extended by paragraph 1 or, as the case may be, paragraph 2.

\amendment{
Words substituted in para. 5(1) (3.11.08) by the Transfer of Tribunal Functions Order 2008  Sch. 3 para. 100(6).
}

\subsection*{Cases of error}

\begin{sloppypar}
6.---(1) Subject to sub-paragraph (2) below, section 28ZC shall also apply where---
\end{sloppypar}
\begin{enumerate}\item[]
($a$) the effect of the determination, whenever made, of an appeal to the Upper Tribunal or the court (“the relevant determination”) is that the adjudicating authority’s decision out of which the appeal arose was erroneous in point of law; and

($b$) after the date of the relevant determination a decision falls to be made by the Secretary of State in accordance with that determination (or would, apart from this paragraph, fall to be so made)---
\begin{enumerate}\item[]
(i) in relation to an application for a departure direction (made after the commencement date);

(ii) as to whether to revise, under section 16 as extended by paragraph 1, a decision (made after the commencement date) in relation to a departure direction
%, a reduced benefit direction 
or a person’s liability under section 43; or

(iii) on an application made under section 17 as extended by paragraph 2 before the date of the relevant determination (but after the commencement date) for a decision in relation to a departure direction
%, a reduced benefit direction 
or a person’s liability under section 43 to be superseded.
\end{enumerate}
\end{enumerate}

(2) Section 28ZC shall not apply where the decision of the Secretary of State mentioned in sub-paragraph (1)($b$) above---
\begin{enumerate}\item[]
($a$) is one which, but for section 28ZA(2)($a$) as it applies by virtue of paragraph 4(1), would have been made before the date of the relevant determination; or

($b$) is one made in pursuance of section 28ZB(3) or (5) as it applies by virtue of paragraph 5(1).
\end{enumerate}

(3) Section 28ZC as it applies by virtue of sub-paragraph (1) shall have effect as if in subsection (4), in the definition of “adjudicating authority”, at the end there were inserted the words “or, in the case of a decision made on a referral under section 28D(1)($b$), the First-tier Tribunal”.

(4) In this paragraph “adjudicating authority”, “the commencement date” and “the court” have the same meanings as in section 28ZC.

\amendment{
Words in paras. 6(1)($b$)(ii), (iii) repealed (14.7.08 for certain cases, 1.6.09 for all other purposes) by the Child Maintenance and Other Payments Act 2008 (c. 6) Sch. 7 para. 1(34).

Words substituted in para. 6 (3.11.08) by the Transfer of Tribunal Functions Order 2008  Sch. 3 para. 100(7).
}

}

\part[Schedule 5 --- Consequential amendments]{Schedule 5\\*Consequential amendments}

\renewcommand\parthead{--- Schedule 5}

%\subsection*{The Tribunals and Inquiries Act 1971}

%1.---(1) In section 7(3) of the Tribunals and Inquiries Act 1971 (chairmen of certain tribunals to be drawn from panels) after “paragraph” there shall be inserted “4A”.
%
%(2) In Schedule 1 to that Act (tribunals under the general supervision of the Council on Tribunals) the following entry shall be inserted at the appropriate place---
%\begin{quotation}
%\subsection*{“Child support maintenance}
%
%4A.---($a$) The child support appeal tribunals established under section 21 of the Child Support Act 1991.
%
%($b$) A Child Support Commissioner appointed under section 22 of the Child Support Act 1991 and any tribunal presided over by such a Commissioner.”
%\end{quotation}

%\subsection*{The Northern Ireland Constitution Act 1973}
%
%2. In paragraph 9 of Schedule 2 to the Northern Ireland Constitution Act 1973 (certain judicial appointments to be an excepted matter), after the words “for Northern Ireland”, where they first occur, there shall be inserted “the Chief and other Child Support Commissioners for Northern Ireland”.

\amendment{
Para. 1 repealed by the Tribunals and Inquiries Act 1992 (c. 53) Sch. 4 Pt. I.

\medskip

Para. 2 repealed (2.12.99) by the Northern Ireland Act 1998 (c. 47) Sch. 15.}

\subsection*{The House of Commons Disqualification Act 1975}

3.---(1) The House of Commons Disqualification Act 1975 shall be amended as follows.

(2) In Part I of Schedule 1 (disqualifying judicial offices), the following entries shall be inserted at the appropriate places— 
\begin{quotation}
“Chief or other Child Support Commissioner (excluding a person appointed under paragraph 4 of Schedule 4 to the Child Support Act 1991).''
\end{quotation}

%(3) In Part III (other disqualifying offices), the following entry shall be inserted at the appropriate place— 
%\begin{quotation}
%“Regional or other full-time chairman of a child support appeal tribunal established under section 21 of the Child Support Act 1991.''
%\end{quotation}

\amendment{
Words inserted in para. 3(2) (4.9.95) by the Child Support Act 1995 (c. 34) s. 19(1).

Para. 3(3) repealed (1.6.99) by the Social Security Act 1998 (c. 14) Sch. 8.
}

\subsection*{The Northern Ireland Assembly Disqualification Act 1975}

4.---(1) In Part I of Schedule 1 to the Northern Ireland Assembly Disqualification Act 1975 (disqualifying judicial offices), the following entries shall be inserted at the appropriate places---
\begin{quotation}
 “Chief or other Child Support Commissioner (excluding a person appointed under paragraph 4 of Schedule 4 to the Child Support Act 1991).''
\end{quotation}

\amendment{
Words inserted in para. 4(1) (4.9.95) by the Child Support Act 1995 (c. 34) s. 19(3).
}

\subsection*{The Family Law (Scotland) Act 1985}

5. In section 4 (amount of aliment) of the Family Law (Scotland) Act 1985, at the end there shall be added—
\begin{quotation}
“(4) Where a court makes an award of aliment in an action brought by or on behalf of a child under the age of 16 years, it may include in that award such provision as it considers to be in all the circumstances reasonable in respect of the expenses incurred wholly or partly by the person having care of the child for the purpose of caring for the child.”
\end{quotation}

\subsection*{Bankruptcy (Scotland) Act 1985}

6.---(1) The Bankruptcy (Scotland) Act 1985 shall be amended as follows.

\begin{sloppypar}
(2) In section 32 (vesting of estate and dealings of debtor after se\-ques\-tra\-tion)---
\end{sloppypar}
\begin{enumerate}\item[]
($a$) in subsection (3)---
\begin{enumerate}\item[]
(i) after paragraph ($b$) there shall be inserted—
\begin{quotation}
“($c$) any obligation of his to pay child support maintenance under the Child Support Act 1991,”;
\end{quotation}

(ii) after “relevant obligations” where second occurring there shall be inserted “referred to in paragraphs ($a$) and ($b$) above”;
\end{enumerate}

($b$) in subsection (5) after “Diligence” there shall be inserted “(which, for the purposes of this section, includes the making of a deduction from earnings order under the Child Support Act 1991)”.
\end{enumerate}

(3) In section 37 (effect of sequestration on diligence), in subsection (5A) for “or a conjoined arrestment order” there is substituted “, a conjoined arrestment order or a deduction from earnings order under the Child Support Act 1991”.

(4) In section 55 (effect of discharge under section 54), in subsection (2)($d$)---
\begin{enumerate}\item[]
($a$) after “being” there shall be inserted “(i)”;

($b$) at the end there shall be inserted---
\begin{quotation}
 “or

(ii) child support maintenance within the meaning of the Child Support Act 1991 which was unpaid in respect of any period before the date of sequestration of---
\begin{enumerate}\item[]
($aa$) any person by whom it was due to be paid; or

($bb$) any employer by whom it was, or was due to be, deducted under section 31(5) of that Act.”.
\end{enumerate}
\end{quotation}
\end{enumerate}

\subsection*{The Insolvency Act 1986}

7. In section 281(5)($b$) of the Insolvency Act 1986 (effect of discharge of bankrupt), after “family proceedings” there shall be inserted “or under a maintenance assessment made under the Child Support Act 1991”.

\subsection*{The Debtors (Scotland) Act 1987}

8.---(1) The Debtors (Scotland) Act 1987 shall be amended as follows.

(2) In section 1(5) (time to pay directions not competent in certain cases) after paragraph ($c$) there shall be inserted---
\begin{quotation}
“($cc$) in connection with a liability order within the meaning of the Child Support Act 1991;”.
\end{quotation}

(3) In section 15(3) (interpretation of Part I), in the definition of “decree or other document”, after “maintenance order” there shall be inserted “, a liability order within the meaning of the Child Support Act 1991”.

(4) In section 54(1) (maintenance arrestment to be preceded by default) in paragraph ($c$) for “the aggregate of 3 instalments” there shall be substituted “one instalment”.

(5) In section 72 (effect of sequestration on diligence against earnings)---
\begin{enumerate}\item[]
($a$) in subsection (2) after “order” there shall be inserted “or deduction from earnings order under the Child Support Act 1991”;

($b$) after subsection (3) there shall be inserted---
\begin{quotation}
“(3A) Any sum deducted by the employer under such a deduction from earnings order made before the date of sequestration shall be paid to the Secretary of State, notwithstanding that the date of payment will be after the date of sequestration.”;
\end{quotation}

($c$) after subsection (4) there shall be inserted—
\begin{quotation}
“(4A) A deduction from earnings order under the said Act shall not be competent after the date of sequestration to secure the payment of any amount due by the debtor under a maintenance assessment within the meaning of that Act in respect of which a claim could be made in the sequestration.”.
\end{quotation}
\end{enumerate}

(6) In section 73(1) (interpretation of Part III), in the definition of “net earnings”---
\begin{enumerate}\item[]
($a$) in paragraph ($c$) for “within the meaning of the Wages Councils Act 1979” there shall be substituted----
\begin{quotation}  “, namely any enactment, rules, deed or other instrument providing for the payment of annuities or lump sums---
\begin{enumerate}\item[]
(i) to the persons with respect to whom the instrument has effect on their retirement at a specified age or on becoming incapacitated at some earlier age, or

(ii) to the personal representatives or the widows, relatives or dependants of such persons on their death or otherwise,
\end{enumerate}
whether with or without any further or other benefit;”; and
\end{quotation}

($b$) at the end there shall be added—
\begin{quotation}
“($d$) any amount deductible by virtue of a deduction from earnings order which, in terms of regulations made under section 32(4)($c$) of the Child Support Act 1991, is to have priority over diligences against earnings.”
\end{quotation}
\end{enumerate}

(7) In section 106 (interpretation) in the definition of “maintenance order”—
\begin{enumerate}\item[]
($a$) the word “or” where it appears after paragraph ($g$), shall be omitted; and

($b$) at the end there shall be inserted---
\begin{quotation}  “or

($j$) a maintenance assessment within the meaning of the Child Support Act 1991.”.
\end{quotation}
\end{enumerate}

\amendment{
The repeal of para. 8(2) by the Child Maintenance and Other Payments Act 2008 (c. 6) Sch. 8 is not yet in force.
}


\begin{center}
\textcopyright\ Crown copyright 2013
\end{center}

\end{document}