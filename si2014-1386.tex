\documentclass[12pt,a4paper]{article}

\newcommand\regstitle{The Child Support (Consequential and Miscellaneous Amendments) Regulations 2014}

\newcommand\regsnumber{2014/1386}

\title{\regstitle}

\author{S.I.\ 2014 No.\ 1386}

\date{Made
31st May 2014\\
Laid before Parliament
6th June 2014\\
Coming into force
in accordance with regulation 1
}

%\opt{oldrules}{\newcommand\versionyear{1993}}
%\opt{newrules}{\newcommand\versionyear{2003}}
%\opt{2012rules}{\newcommand\versionyear{2012}}

\usepackage{csa-regs}

\setlength\headheight{27.61603pt}

%\hbadness=10000

\begin{document}

\maketitle

\enlargethispage{\baselineskip}

\noindent
The Secretary of State for Work and Pensions, in exercise of the powers conferred by sections 16(1), 28E(5), 29(2) and (3), 32(1) and (2)($n$), 34(1), 42, 43A, 51(1) and (2)($a$)  and ($i$), 52(4) and 54(1) of, and paragraph 11 of Schedule 1 to, the Child Support Act 1991\footnote{1991 c.~48. Section 16 was substituted by section 40 of the Social Security Act 1998 (c.~14) and sub-section (1) was amended by section 8 of the Child Support, Pensions and Social Security Act 2000 (c.~19) (“the 2000 Act”). Section 28E was inserted by section 5 of the Child Support Act 1995 (c.~34). Section 29(2) and (3) was amended by section 1(2) of the 2000 Act and subsection (3) was amended by the Welfare Reform Act 2009 (c.~24). Section 42 was amended by section 26 of, and Schedule 3 to, the 2000 Act and by section 37 of the Child Maintenance and Other Payments Act 2008 (c.~6) (“the 2008 Act”). Section 43A was inserted by section 38 of the 2008 Act. Section 51(2)($a$)  was amended by sections 1(2)($a$)  and 26 of, and Schedule 3 to, the 2000 Act. Section 54(1) is cited for the meaning of “prescribed”. Subsection (1) of section 54 was inserted by Schedule 7 to the 2008 Act. Paragraph 11 of Schedule 1 was amended by section 1(2) of the 2000 Act. References in the 1991 Act to “the Commission” were replaced by references to “the Secretary of State” by S.I.~2012/2007. Amendments made by the 2000 Act were made only in relation to cases other than 1993 scheme cases (a “1993 scheme case” means a case in respect of which the provisions of the 2000 Act have not been brought into force in accordance with article 3 of S.I.~2003/192), with the exception of the amendment made to section 42 by section 26 of, and Schedule 3 to, the 2000 Act which was made for all purposes.} and section 55(3) and (4) of, and paragraphs 2 and 5 of Schedule~5 to, the Child Maintenance and Other Payments Act 2008\footnote{2008 c.~6. Schedule~5 was amended by section 136 of the Welfare Reform Act 2012 (c.~5) and by S.I.~2012/2007.}, makes the following Regulations: 

{\sloppy

\tableofcontents

}

\bigskip

\setcounter{secnumdepth}{-2}

\subsection[1. Citation and commencement]{Citation and commencement}

1.—(1) These Regulations may be cited as the Child Support (Consequential and Miscellaneous Amendments) Regulations 2014.

%(2) This regulation and regulations 3, 4, 5, 7(1), (2) and (4) and 8 come into force on 30th June 2014.
%
%(3) Regulations 2 and 6(1) (in so far as it relates to paragraph (4)) and (4) come into force in relation to a case to which the new calculation rules apply on the day on which section 137 (collection of child support maintenance) of the Welfare Reform Act 2012\footnote{2012 c.~5. Section 137 was amended by S.I.~2012/2007.} comes into force.
%
%(4) Regulation 6(1) (in so far as it relates to paragraphs (2) and (3)), (2) and (3) comes into force in relation to a case to which the new calculation rules apply on the day on which section 19 (transfer of cases to new rules) of the Child Maintenance and Other Payments Act 2008 comes into force for all purposes.
%
%(5) Regulation 7(3) comes into force on the date on which section 137 of the Welfare Reform Act 2012 comes into force.
%
%(6) In this regulation, “a case to which the new calculation rules apply” means a case in which liability to pay child support maintenance is calculated in accordance with Part~I of Schedule 1 to the Child Support Act 1991 as amended by paragraph 2 of Schedule 4 to the Child Maintenance and Other Payments Act 2008.

% Reg 1(2) substituted for reg 1(2)--(6) by SI 2014/1621 reg 3(3)
(2) These Regulations come into force on 30th June 2014.

\amendment{
Reg. 1(2) substituted for reg. 1(2)--(6) (24.6.14) by the Child Support (Consequential and Miscellaneous Amendments) (No. 2) Regulations 2014 reg. 3(3).
}

\subsection[2. Amendments to the Child Support (Collection and Enforcement) Regulations 1992]{Amendments to the Child Support (Collection and Enforcement) Regulations 1992}

2.—(1) The Child Support (Collection and Enforcement) Regulations 1992\footnote{S.I.~1992/1989; relevant amending instruments are S.I.~1995/1045, 2001/162, 2008/2544, 2009/1815, 2012/2007, 2012/2785.} 
%are amended 
are modified, in relation to a case in which liability to pay child support maintenance is calculated in accordance with Part I of Schedule 1 to the Child Support Act 1991 as amended by paragraph 2 of Schedule 4 to the Child Maintenance and Other Payments Act 2008, as if they had been amended  % Words substituted by SI 2014/1621 reg 3(4)
as follows.

(2) In regulation 1 (citation, commencement and interpretation)—
\begin{enumerate}\item[]
($a$) in paragraph (2)\footnote{Paragraph (2) was substituted by S.I.~2001/162.}, after the definition of “the 2000 Act” insert—
\begin{quotation}
““collection fee” means a fee payable by a non-resident parent under regulation 7 (the collection fee) of the Child Support Fees Regulations 2014\footnote{S.I.~2014/612.};

“enforcement fee” means a fee payable under regulation 10 (the enforcement fee) of the Child Support Fees Regulations 2014;”;
\end{quotation}

($b$) for paragraph (2A)\footnote{Paragraph (2A) was inserted by S.I.~2001/162.} substitute—
\begin{quotation}
“(2A) Except in relation to regulation 8(3)($a$)  and Schedule 2, in these Regulations “fee” means a collection fee or an enforcement fee (or both).”.
\end{quotation}
\end{enumerate}

(3) In regulation 2(2) (payment of child support maintenance), after “maintenance” insert “or liable to make payment of a fee (or both)”.

(4) In regulation 3 (method of payment)—
\begin{enumerate}\item[]
($a$) in paragraph (2)\footnote{Paragraph (2) was amended by S.I.~2001/162.}, after “paragraph (1)” insert “and from which payments of collection fees (where payable) may be made”;

($b$) in paragraph (6)\footnote{Paragraph (6) was inserted by S.I.~2008/2544.}, after “liability” in sub-paragraphs ($b$)  and ($c$)  insert “, that parent’s liability to pay a fee or the amount of a fee payable by that parent”.
\end{enumerate}

(5) In regulation 4(1)\footnote{Regulation 4 was substituted by S.I.~2012/2785.} (payments to be scheduled over reference period), after “payments of child support maintenance” insert “and any collection fees”.

(6) In regulation 7 (notice to liable person as to requirements about payment)\footnote{Paragraphs (1) and (2) were amended, and paragraphs (1A) and (3) inserted, by S.I.~2001/162.}—
\begin{enumerate}\item[]
($a$) in paragraph (1)—
\begin{enumerate}\item[]
(i) in sub-paragraph ($a$), after “child support maintenance” insert “and any collection fees”,

(ii) in sub-paragraph ($b$), for “it is” substitute “child support maintenance and any collection fees are”,

(iii) in sub-paragraph ($c$), after “payment” insert “of child support maintenance and any collection fees”,

(iv) in sub-paragraph ($d$), after “payments” insert “of child support maintenance and any collection fees”,

(v) in sub-paragraph ($e$), after “child support maintenance” insert “and any payment of a collection fee”;
\end{enumerate}

($b$) in paragraph (1A), in both places where it appears, for “, interest or fees” substitute “or interest”;

($c$) after paragraph (1A) insert—
\begin{quotation}
“(1B) In the case of an enforcement fee, the Secretary of State shall send the liable person a notice stating—
\begin{enumerate}\item[]
($a$) the amount of the enforcement fee payable; and

($b$) the method of enforcement action in respect of which that fee is payable.”;
\end{enumerate}
\end{quotation}

($d$) in paragraph (3), for “, interest or fees” substitute “or interest”;

($e$) after paragraph (3) insert—
\begin{quotation}
“(4) A notice under paragraph (1B) shall be sent to the liable person as soon as is reasonably practicable after an enforcement fee becomes payable.”.
\end{quotation}
\end{enumerate}

(7) In regulation 17(1)($b$)  (requirement to review deduction from earnings orders)\footnote{Regulation 17 was substituted by S.I.~1995/1045. Paragraph (1)($b$)  was amended by S.I.~2001/162.}—
\begin{enumerate}\item[]
($a$) after “any arrears” insert “of child support maintenance, arrears of collection fees”;

($b$) for “fees” substitute “enforcement fee”.
\end{enumerate}

(8) In regulation 20 (discharge of deduction from earnings orders)\footnote{Regulation 20 was substituted by S.I.~1995/1045. Paragraph (1)($f$)  was amended by S.I.~2001/162.}—
\begin{enumerate}\item[]
($a$) in paragraph (1), after “where” insert “paragraph (1A) applies or”;

($b$) in paragraph (1)($f$), after “maintenance calculation” insert “and any requirement to pay collection fees”;

($c$) after paragraph (1), insert—
\begin{quotation}
“(1A) This paragraph applies where—
\begin{enumerate}\item[]
($a$) the Secretary of State has agreed with the liable person an alternative method of payment of the child support maintenance due under the maintenance calculation and an alternative method of payment of fees (where payable); and

($b$) the Secretary of State considers it is reasonable to discharge the order in the circumstances of the case.”.
\end{enumerate}
\end{quotation}
\end{enumerate}

(9) In regulation 27(2) (notice of intention to apply for a liability order)\footnote{Regulation 27(2) was amended by S.I.~2001/162.}, for “or fees” substitute “, collection fees or enforcement fees”.

(10) In Schedule 1 (liability order prescribed form)\footnote{Schedule 1 was amended by S.I.~2001/162.}, after “Part IV of the Child Support (Collection and Enforcement) Regulations 1992” insert “and Parts III and IV of the Child Support Fees Regulations 2014”.

\amendment{
Words substituted in reg. 2(1) (24.6.14) by the Child Support (Consequential and Miscellaneous Amendments) (No. 2) Regulations 2014 reg. 3(4).
}

\subsection[3. Amendments to the Child Support (Maintenance Assessments and Special Cases) Regulations 1992]{Amendments to the Child Support (Maintenance Assessments and Special Cases) Regulations 1992}

3.  Regulation 27A (child who is allowed to live with his parent under section 23(5) of the Children Act 1989) of the Child Support (Maintenance Assessments and Special Cases) Regulations 1992\footnote{S.I.~1992/1815, which was revoked with savings by S.I.~2001/155 and 2012/2785. Regulation 27A was inserted by S.I.~1993/913.} is amended as follows—
\begin{enumerate}\item[]
($a$) for the title substitute “Child in care who is allowed to live with their parent”;

($b$) in paragraph (1), after “section” insert “22C(2) or”;

($c$) in paragraph (2), insert “with” after “parent of a child” and from “allow the child to live” to the end substitute “has allowed the child to live.”.
\end{enumerate}

\subsection[4. Amendment to the Child Support Departure Direction and Consequential Amendments Regulations 1996]{Amendment to the Child Support Departure Direction and Consequential Amendments Regulations 1996}

4.  In regulation 12 (meaning of “benefit” for the purposes of section 28E of the Child Support Act 1991) of the Child Support Departure Direction and Consequential Amendments Regulations 1996\footnote{S.I.~1996/2907. Regulation 12 was amended by S.I.~2001/156, 2003/328, 2003/2779, 2008/1554, 2012/2785, 2013/458 and 2013/630.}, for the words from “working tax credit” to the end substitute “working tax credit, housing benefit and relevant universal credit.”.

\subsection[5. Amendments to the Child Support (Maintenance Calculations and Special Cases) Regulations 2000]{Amendments to the Child Support (Maintenance Calculations and Special Cases) Regulations 2000}

5.  Regulation 13 (child who is allowed to live with his parent under section 23(5) of the Children Act 1989) of the Child Support (Maintenance Calculations and Special Cases) Regulations 2000\footnote{S.I.~2001/155, which was revoked with savings by S.I.~2012/2785.} is amended as follows—
\begin{enumerate}\item[]
($a$) for the title substitute “Child in care who is allowed to live with their parent”;

($b$) in paragraph (1), after “section” insert “22C(2) or”;

($c$) in paragraph (2), from “allow the child to live” to the end substitute “has allowed the child to live.”.
\end{enumerate}

\subsection[6. Amendments to the Child Support (Management of Payments and Arrears) Regulations 2009]{\sloppy Amendments to the Child Support (Management of Payments and Arrears) Regulations 2009}

6.—(1) The Child Support (Management of Payments and Arrears) Regulations 2009\footnote{S.I.~2009/3151, which was amended by S.I.~2012/2007; there are other amending instruments but none is relevant.} 
%are amended 
are modified, in relation to a case in which liability to pay child support maintenance is calculated in accordance with Part I of Schedule 1 to the Child Support Act 1991 as amended by paragraph 2 of Schedule 4 to the Child Maintenance and Other Payments Act 2008, as if they had been amended  % Words substituted by SI 2014/1621 reg 3(5)
as follows.

(2) In regulation 2 (interpretation), after paragraph (2) insert—
\begin{quotation}
“(3) For the purposes of regulations 3 and 3A, there are “arrangements for direct pay” where the Secretary of State has specified that payments of child support maintenance shall be made by the non-resident parent to the person caring for the child or children in question or to a child who made an application under section 7(1) of the 1991 Act\footnote{The Secretary of State may so specify under regulation 2(1)($a$)  of the Child Support (Collection and Enforcement) Regulations 1992 (S.I.~1992/1989).}.”.
\end{quotation}

(3) In regulation 3(1) (arrears notices)\footnote{Regulation 3(1) was amended by S.I.~2012/2007.}—
\begin{enumerate}\item[]
($a$) in paragraph ($a$), for “; and” substitute “or there are arrangements for direct pay;”;

($b$) at the end of paragraph ($b$)  insert—
\begin{quotation}
“; and

($c$) regulation 3A(1) does not apply or regulation 3A(1) does apply but the notice referred to in regulation 3A(2) has not been given.”.
\end{quotation}
\end{enumerate}

(4) After regulation 3 insert—
\begin{quotation}
\subsection*{“Notice of consequences of failure to pay child support maintenance due}

3A.—(1) This paragraph applies to a case where—
\begin{enumerate}\item[]
($a$) either—
\begin{enumerate}\item[]
(i) there are arrangements for direct pay, or

(ii) the Secretary of State is arranging for the collection of child support maintenance under section 29 of the 1991 Act but there are no arrangements for enforcement under the 1991 Act; and
\end{enumerate}

($b$) the non-resident parent has failed to make one or more payments of child support maintenance due.
\end{enumerate}

(2) Where paragraph (1) applies to a case, the Secretary of State may only start making arrangements for collection under section 29 of the 1991 Act or arrangements for enforcement under the 1991 Act (or both) where the non-resident parent has been given a notice, within the preceding 12 month period, setting out that the Secretary of State will consider making such arrangements where there is a failure to make one or more payments of child support maintenance due.”.
\end{quotation}

% Reg 6(5) revoked by SI 2014/1621 reg 3(6)
%(5) In regulation 11 (recovery of arrears from a deceased person’s estate), after “maintenance” insert “and collection fees (payable under regulation 7 (the collection fee) of the Child Support Fees Regulations 2014\footnote{S.I.~2014/612.})”.

\amendment{
Words substituted in reg. 6(1) and reg. 6(5) revoked (24.6.14) by the Child Support (Consequential and Miscellaneous Amendments) (No. 2) Regulations 2014 reg. 3(5), (6).
}

\subsection[7. Amendments to the Child Support Maintenance Calculation Regulations 2012]{Amendments to the Child Support Maintenance Calculation Regulations 2012}

7.—(1) The Child Support Maintenance Calculation Regulations 2012\footnote{S.I.~2012/2677, to which there are amendments not relevant to these Regulations.} are amended as follows.

(2) In regulation 11 (notice of application)—
\begin{enumerate}\item[]
($a$) in paragraph (1)—
\begin{enumerate}\item[]
(i) after “Act” insert “, and the requirements in paragraph (3) are satisfied,”,

(ii) omit “, as soon as reasonably practicable,”;
\end{enumerate}

($b$) in paragraph (2), after “parent” insert “(as ascertained and verified in accordance with paragraph (3)($a$))”;

($c$) after paragraph (2) insert—
\begin{quotation}
“(3) The requirements referred to in paragraph (1) are—
\begin{enumerate}\item[]
($a$) the address of the non-resident parent in relation to the application has been ascertained and verified; and

($b$) any application fee payable under regulation 3(1) (the application fee) of the Child Support Fees Regulations 2014 has been paid or waived in accordance with those Regulations.
\end{enumerate}

(4) Except where paragraph (5) or (6) applies to an application, notice must be given as soon as is reasonably practicable.

(5) Where—
\begin{enumerate}\item[]
($a$) there is an existing case related to the application; or

($b$) the applicant—
\begin{enumerate}\item[]
(i) has been required to choose in an existing case whether or not to stay in the statutory scheme (under Schedule~5 (maintenance calculations: transfer of cases to new rules) to the 2008 Act\footnote{Schedule~5 was amended by section 6 of the Welfare Reform Act 2012 (c.5).}), as a result of that applicant’s existing case being related to an application made under section 4(1) or 7(1) of the 1991 Act, and

(ii) has chosen, by way of the application, to remain in the statutory scheme,
\end{enumerate}
\end{enumerate}
notice must be given as soon as is reasonable.

(6) Subject to paragraph (8), where the applicant—
\begin{enumerate}\item[]
($a$) has been required to choose in an existing case whether or not to stay in the statutory scheme (under Schedule~5 to the 2008 Act), in circumstances where the existing case is not related to an application made under section~4(1) or 7(1) of the 1991 Act; and

($b$) has chosen, by way of the application, to remain in the statutory scheme,
\end{enumerate}
notice must be given in accordance with paragraph (7).

(7) Where paragraph (6) applies, notice must be given—
\begin{enumerate}\item[]
($a$) where the application is made and the requirements in paragraph (3) are satisfied before the day 39 days before the liability end date (which means the date determined in accordance with regulation 6 (liability end date) of the Ending Liability Regulations) in relation to the existing case has passed, as soon as is reasonable once that day has passed; or

($b$) where the application is made and the requirements in paragraph (3) are satisfied after the day 39 days before the liability end date has passed, as soon as is reasonable.
\end{enumerate}

(8) Where an application to which paragraph (6) applies becomes an application to which paragraph (5) applies (because it becomes an existing case related to an application), paragraph (6) ceases to apply to that application.

(9) For the purposes of paragraphs (5) to (8) and this paragraph—
\begin{enumerate}\item[]
($a$) “the 2008 Act” means the Child Maintenance and Other Payments Act 2008\footnote{2008 c.~6.};

\begin{sloppypar}
($b$) “existing case” has the meaning given in paragraph~1(2) of Schedule~5 to the 2008 Act;
\end{sloppypar}

($c$) “the Ending Liability Regulations” means the Child Support (Ending Liability in Existing Cases and Transition to New Calculation Rules) Regulations 2014\footnote{S.I.~2014/614.};

($d$) an existing case is related to an application if—
\begin{enumerate}\item[]
(i) the non-resident parent in relation to that application is also the non-resident parent in relation to the existing case and the person with care in relation to that application is not the person with care in relation to the existing case, or

(ii) the non-resident parent in relation to that application is a partner of a non-resident parent in relation to the existing case and either or both are in receipt of a benefit prescribed by regulations made under paragraph 4(1)($c$)  (flat rate) of Schedule 1 to the 1991 Act\footnote{The substitution of Part~I of Schedule 1 to the Child Support Act 1991 (c.~48) by section 1(3) of, and Schedule 1 to, the Child Support, Pensions and Social Security Act 2000 (c.~19) was partially commenced for the types of cases specified in article 3 of S.I.~2003/192.}.”.
\end{enumerate}
\end{enumerate}
\end{quotation}
\end{enumerate}

(3) In regulation 12 (initial effective date)—
\begin{enumerate}\item[]
($a$) regulation 12 becomes paragraph (1) of that regulation;

($b$) in paragraph (1), for “on which notice is” substitute “provided as the initial effective date in the notice” and for “in accordance with” substitute “under”;

($c$) after paragraph (1) insert—
\begin{quotation}
“(2) The non-resident parent must be notified of the initial effective date—
\begin{enumerate}\item[]
($a$) by written notice posted to the last known address of the non-resident parent at least two days prior to the initial effective date; or

($b$) by telephone on or before the initial effective date and by written notice sent by post to the last known address of the non-resident parent.”.
\end{enumerate}
\end{quotation}
\end{enumerate}

(4) In regulation 14 (grounds for revision), after paragraph (3) insert—
\begin{quotation}
“(3A) Where—
\begin{enumerate}\item[]
($a$) the Secretary of State makes a decision and there is an appeal;

($b$) there is a further decision in relation to the appellant (“decision $\mathcal{B}$”) after the appeal but before the appeal results in a decision by the First-tier Tribunal (“decision $\mathcal{C}$”); and

($c$) the Secretary of State would have made decision $\mathcal{B}$ differently if aware of decision $\mathcal{C}$ at the time of making decision~$\mathcal{B}$,
\end{enumerate}
decision $\mathcal{B}$ may be revised at any time.”.
\end{quotation}

\subsection[8. Amendments to the Child Support (Ending Liability in Existing Cases and Transition to New Calculation Rules) Regulations 2014]{Amendments to the Child Support (Ending Liability in Existing Cases and Transition to New Calculation Rules) Regulations 2014}

8.—(1) The Child Support (Ending Liability in Existing Cases and Transition to New Calculation Rules) Regulations 2014 are amended as follows.

(2) In regulation 1(2) (citation, commencement and interpretation), in the definition of “transition period”, after “regulation 3(2)” insert “and (3)”.

(3) In regulation 6 (liability end date)—
\begin{enumerate}\item[]
($a$) in paragraph (1)($a$), for “the day falling” substitute “a date specified by the Secretary of State which shall be no less than”;

($b$) in paragraph (2), after “falling” insert “at least”.
\end{enumerate}

\bigskip

\pagebreak[3]

Signed 
by authority of the 
Secretary of State for~Work and~Pensions.
%I concur
%By authority of the Lord Chancellor

{\raggedleft
\emph{Steve Webb}\\*
%Secretary
Minister
%Parliamentary Under Secretary 
of State\\*Department 
for~Work and~Pensions

}

31st May 2014

\small

\part{Explanatory Note}

\renewcommand\parthead{— Explanatory Note}

\subsection*{(This note is not part of the Regulations)}

These Regulations make miscellaneous amendments and amendments consequential on the Child Support Fees Regulations 2014 (S.I.~2014/612) (“the Fees Regulations”) to various Child Support Regulations.

Regulation 2 amends the Child Support (Collection and Enforcement) Regulations 1992 (S.I.~1992/1989) (“the 1992 Regulations”) for the purposes of cases administered under the 2012 scheme of child support. The amendments are consequential on the introduction of charging of collection fees (a fee payable in a case where the Secretary of State arranges for collection of child maintenance) and enforcement fees (a fee payable where the Secretary of State takes enforcement action) under the Fees Regulations.

A new definition of “fee” is inserted into the 1992 Regulations, which means a collection fee or an enforcement fee payable under the Fees Regulations. Regulation 3 of the 1992 Regulations is amended so that the Secretary of State may require a person liable to pay child maintenance and fees (“the liable person”) to take reasonable steps to open an account from which payments of child maintenance and collection fees can be made. Regulation 3 is further amended so that, when the Secretary of State is considering specifying a deduction from earnings order as the method of payment, it is not relevant that a third party would become aware, or that the liable person prefers that their employer would not be informed, of that person’s liability to pay a fee or the amount of the fee payable. Regulation 4 is amended so that the Secretary of State may schedule payments of collection fees in the same way that child maintenance is scheduled.

Regulation 7 is amended so that when a maintenance calculation is made the Secretary of State must send the liable person a notice stating, in relation to child maintenance and any collection fees: the amount payable, to whom payments are to be made, the method of payment, the day and interval by reference to which payments are to be made and the amount of any overdue child maintenance or collection fees. Provision is also made so that when an enforcement fee becomes payable the liable person must be sent a notice stating the amount of the enforcement fee payable and the enforcement action in respect of which it is payable. In both scenarios, the notice must be sent as soon as is reasonably practicable.

The Secretary of State will, when necessary, use existing methods of enforcement to recover fees. Paragraphs (7), (8)($b$), (9) and (10) of regulation 2 make amendments to provisions relating to enforcement so that, where relevant, reference is also made to fees. Paragraph (8)($a$)  and ($c$)  amends regulation 20 so that a deduction from earnings order may be discharged where the Secretary of State agrees an alternative method of payment of fees and child maintenance with the liable person and considers that discharging the order is reasonable.

Regulation 3 makes technical amendments to regulation 27A of the Child Support (Maintenance Assessments and Special Cases) Regulations 1992 (S.I.~1992\slash 1815) which are consequential on an amendment made to the Children Act 1989 (c.~41) by section 8 of the Children and Young Persons Act 2008 (c.~23). Regulation 5 makes analogous amendments to the Child Support (Maintenance Calculations and Special Cases) Regulations 2000 (S.I.~2001/155).

Regulation 4 makes a clarificatory amendment to the Child Support Departure Direction and Consequential Amendments Regulations 1996 (S.I.~1996/2907).

Regulation 6 amends the Child Support (Management of Payment and Arrears) Regulations 2009 (S.I.~2009/3151) for the purposes of 2012 scheme cases. A new regulation 3A is inserted which provides that, in a case where there are arrangements for direct pay or collection (but not enforcement) and the non-resident parent fails to make a payment of child maintenance, the Secretary of State may make arrangements for collection or enforcement if the non-resident parent has been given a notice during the previous 12 months setting out that the Secretary of State would consider taking such action. If that notice has not been given, notice must be given to the non-resident parent in accordance with regulation 3 before the Secretary of State may make arrangements for collection or enforcement. Regulation 6 also amends regulation 11 to enable the Secretary of State to recover collection fees from a deceased person’s estate.

Regulation 7 amends the Child Support Maintenance Calculation Regulations 2012 (S.I.~2012/2677). Regulation 11 is amended so that the Secretary of State must send a notice under regulation 11 where: an application has been made; the address of the non-resident parent has been ascertained and verified; and any application fee payable under the Fees Regulations has been paid or waived. The notice must normally be sent as soon as is reasonably practicable, unless the provisions for cases affected by the Child Support (Ending Liability in Existing Cases and Transition to New Calculation Rules) Regulations 2014 (S.I.~2014/614) apply. Regulation 12 is amended so that the initial effective date is the date provided in the notice given under regulation 11. Notice of that date must be given to the non-resident parent either by written notice posted at least two days before the initial effective date or by phone on or before that date and by post. Regulation 14 is amended so that a new paragraph (3A) is inserted to provide for certain decisions to be revised at any time.

Regulation 8 amends the Child Support (Ending Liability in Existing Cases and Transition to New Calculation Rules) Regulations 2014. Regulation 8(2) makes a technical amendment to the definition of “transition period” in those Regulations. Regulation 8(3) amends regulation 6 of those Regulations, which provides for the date on which liability under a maintenance calculation or assessment ceases to accrue for the purposes of Schedule~5 to the Child Maintenance and Other Payments Act 2008 (c.~6) (“the liability end date”). Regulation 6(1)($a$)  is amended so that, in a case that is linked to an application made under section 4(1) or 7(1) of the Child Support Act 1991 (c.~48), the liability end date will be the date specified by the Secretary of State in the notice which must be at least 30 days after notice is given. Regulation 8(3)($b$)  makes a similar amendment in relation to cases that become related to an application made under section 4(1) or 7(1).

A full impact assessment has not been produced for this instrument as it has no impact on the private sector and civil society organisations. 

\end{document}