\documentclass[a4paper]{article}

\usepackage[welsh,english]{babel}

\usepackage[utf8]{inputenc}
\usepackage[T1]{fontenc}

\usepackage{textcomp}

%\usepackage[2012rules]{optional}

\usepackage[osf]{mathpazo}

\usepackage{perpage} %the perpage package
\MakePerPage{footnote} %the perpage package command
\renewcommand{\thefootnote}{\fnsymbol{footnote}}

\usepackage[perpage,para,symbol]{footmisc}

%\opt{newrules}{
\title{The Child Support Departure Direction (Anticipatory Application) Regulations 1996}
%}

%\opt{2012rules}{
%\title{Child Maintenance and Other Payments Act 2008\\(2012 scheme version)}
%}

\author{S.I. 1996 No. 635}

\date{Made 7th March 1996\\Laid before Parliament 11th March 1996\\Coming into force 9th April 1996
}

%\opt{oldrules}{\newcommand\versionyear{1993}}
%\opt{newrules}{\newcommand\versionyear{2003}}
%\opt{2012rules}{\newcommand\versionyear{2012}}

\usepackage{fancyhdr}
\pagestyle{fancy}
\fancyhead[L]{}
\fancyhead[C]{\itshape The Child Support Departure Direction (Anticipatory Application) Regulations 1996 (S.I.~1996/635) \parthead%\phantom{...}% (\versionyear{} scheme version)
}
\fancyhead[R]{}
\fancyfoot[C]{\thepage}
\newcommand{\parthead}{}

\usepackage{array}
\usepackage{multirow}
\usepackage[debugshow]{tabulary}
\usepackage{longtable}
\usepackage{multicol}
\usepackage{lettrine}

\usepackage[colorlinks=true]{hyperref}
\usepackage{microtype}

\hyphenation{Aw-dur-dod}
\hyphenation{bank-rupt-cy}
\hyphenation{Ec-cles-ton}
\hyphenation{Eux-ton}
\hyphenation{Hogh-ton}
\hyphenation{Pres-ton}
\hyphenation{Pru-den-tial}
\hyphenation{Riv-ing-ton}

\newcolumntype{x}[1]
	{>{\raggedright}p{#1}}
\newcommand{\tn}{\tabularnewline}
\setlength\tymin{50pt}

\newcommand\amendment[1]{\subsubsection*{Notes}{\itshape\frenchspacing\footnotesize #1 \par}}

\setlength\headheight{22.87003pt}

\begin{document}

\maketitle

\amendment{
Regs. revoked (2.12.96) by the Child Support Departure Direction and Consequential Amendments Regulations 1996 reg. 51.
}

%\noindent
%The Secretary of State for Social Security, in exercise of the powers conferred by sections 12, 14, 28I(4)\footnote{\frenchspacing These Regulations make provision for enabling applications for departure directions to be made and for the determination of such applications even though section 28A of the Child Support Act 1991 is not in force, and for such applications to be determined as if sections 28A to 28H and section 28I(1) to (3) of, and Schedules 4A and 4B to, that Act were in force. Sections 28A to 28I and Schedules 4A and 4B were inserted into the Child Support Act 1991 by the Child Support Act 1995 (1995 c. 34).}, 51, 52(4) and 54\footnote{\frenchspacing Section 54 is cited because of the meaning ascribed to the word “prescribed”.} of the Child Support Act 1991\footnote{\frenchspacing 1991 c. 48.} and of all other powers enabling him in that behalf, hereby makes the following Regulations—
%
%{\sloppy
%
%\tableofcontents
%
%}
%
%\setcounter{secnumdepth}{-2}
%
%\section[Part I --- General]{Part I\\*General}
%
%\renewcommand\parthead{--- Part I}
%
%\subsection[1. Citation, commencement and interpretation]{Citation, commencement and interpretation}
%
%1.—(1) These Regulations may be cited as the Child Support Departure Direction (Anticipatory Application) Regulations 1996 and shall come into force on 9th April 1996.
%
%(2) In these Regulations, unless the context otherwise requires—
%\begin{enumerate}\item[]
%“the Act” means the Child Support Act 1991;
%
%“applicant” has the same meaning as in Schedule 4B to the Act;
%
%“application” means an application for a departure direction;
%
%“Contributions and Benefits Act” means the Social Security Contributions and Benefits Act 1992\footnote{\frenchspacing 1992 c. 4.};
%
%“departure direction application form” means the form provided by the Secretary of State in accordance with regulation 4(1);
%
%“effective application” has the meaning given in regulation 4(4);
%
%“effective date” in relation to a departure direction means the date on which that direction takes effect;
%
%“Maintenance Assessment Procedure Regulations” means the Child Support (Maintenance Assessment Procedure) Regulations 1992\footnote{\frenchspacing S.I. 1992/1813; the relevant amending instruments are S.I. 1994/227, 1995/123, 1995/1045 and 3261.};
%
%“Maintenance Assessments and Special Cases Regulations” means the Child Support (Maintenance Assessments and Special Cases) Regulations 1992\footnote{\frenchspacing S.I. 1992/1815; the relevant amending instruments are S.I. 1993/913, 1994/227, 1995/1045 and 3261.};
%
%“maintenance period” has the same meaning as in regulation 33 of the Maintenance Assessment Procedure Regulations;
%
%“non-applicant” means—
%\begin{enumerate}\item[]
%($a$) where the application has been made by a person with care, the absent parent;
%
%($b$) where the application has been made by an absent parent, the person with care;
%\end{enumerate}
%
%“partner” has the same meaning as in paragraph (2) of regulation 1 of the Maintenance Assessments and Special Cases Regulations\footnote{\frenchspacing Paragraph (2) of regulation 1 has been amended by S.I. 1993/913, 1995/1045 and 3261.};
%
%“relevant person” means—
%\begin{enumerate}\item[]
%($a$) an absent parent whose liability under a maintenance assessment may be affected by any departure direction given following an application;
%
%($b$) a person with care, or a child to whom section 7 of the Act applies, where the amount of child support maintenance payable under a maintenance assessment relevant to that person with care or that child may be affected by any departure direction given following an application.
%\end{enumerate}
%\end{enumerate}
%
%(3) In these Regulations, a maintenance assessment calculated in accordance with the provisions of Part I of Schedule 1 to the Act includes an assessment calculated in accordance with provision made under section 12 of the Act.
%
%(4) Except where express provision is made to the contrary, where, by any provision of these Regulations—
%\begin{enumerate}\item[]
%($a$) any document is given or sent to the Secretary of State, that document shall, subject to paragraph (5), be treated as having been so given or sent on the date it is received by the Secretary of State; and
%
%($b$) any document is given or sent to any person, that document shall, if sent by post to that person’s last known or notified address, and subject to paragraph (6), be treated as having been given or sent on the second day after the day of posting, excluding any Sunday or any day which is a Bank Holiday in England, Wales or Northern Ireland under the Banking and Financial Dealings Act 1971\footnote{\frenchspacing 1971 c. 80.}.
%\end{enumerate}
%
%(5) The Secretary of State may treat any document given or sent to him as given or sent on such day, earlier than the day it was received by him, as he may determine, if he is satisfied that there was unavoidable delay in his receiving the document in question.
%
%(6) Where, by any provision of these Regulations, and in relation to a particular 
%application, notice or notification—
%\begin{enumerate}\item[]
%($a$) more than one document is required to be given or sent to a person, and more than one such document is sent by post to that person but not all the documents are posted on the same day; or
%
%($b$) documents are required to be given or sent to more than one person, and not all such documents are posted on the same day,
%\end{enumerate}
%all those documents shall be treated as having been posted on the later or, as the case may be, the latest day of posting.
%
%(7) In these Regulations, unless the context otherwise requires, a reference—
%\begin{enumerate}\item[]
%($a$) to the Schedule, is to the Schedule to these Regulations;
%
%($b$) to a numbered regulation is to the regulation in these Regulations bearing that number;
%
%($c$) in a regulation or the Schedule to a numbered paragraph is to the paragraph in that regulation or the Schedule bearing that number;
%
%($d$) in a paragraph to a lettered or numbered sub-paragraph is to the sub-paragraph in that paragraph bearing that letter or number.
%\end{enumerate}
%
%\subsection[2. Applications for a departure direction and the determination of applications]{Applications for a departure direction and the determination of applications}
%
%2.—(1) Where a maintenance assessment is in force, a person mentioned in section 28A(1) of the Act may make an application if—
%\begin{enumerate}\item[]
%($a$) the person with care resides in the county of Essex, Kent, East Sussex, West Sussex, Surrey, Hertfordshire or Greater London; or
%
%($b$) it appears to the Secretary of State that, if an application were to be made and he were to give it a preliminary consideration under section 28B of the Act on the basis of information in his possession prior to 9th April 1996, such a preliminary consideration would not lead him to reject the application.
%\end{enumerate}
%
%(2) Applications for a departure direction made before the coming into force of section 28A of the Act shall be determined as if that section and the other provisions of the Act relating to departure directions were in force.
%
%\subsection[3. Determination of amounts]{Determination of amounts}
%
%3.—(1) Where any amount is required to be determined for the purposes of these Regulations, it shall be determined as a weekly amount and, except where the context otherwise requires, any reference to such an amount shall be construed accordingly.
%
%(2) Where any calculation made under these Regulations results in a fraction of a penny that fraction shall be treated as a penny if it is either one half or exceeds one half and shall be otherwise disregarded.
%
%\section[Part II --- Procedure on an application for a departure direction and preliminary consideration]{Part II\\*Procedure on an application for a departure direction and preliminary consideration}
%
%\renewcommand\parthead{--- Part II}
%
%\subsection[4. Application for a departure direction]{Application for a departure direction}
%
%4.—(1) Any person who applies for a departure direction shall do so on a form (a “departure direction application form”) provided by the Secretary of State, or in such other manner, being in writing, as the Secretary of State may accept as sufficient in the circumstances of any particular case.
%
%(2) Departure direction application forms shall be supplied without charge by such persons as the Secretary of State authorises for that purpose.
%
%(3) A completed departure direction application form shall be given or sent to the Secretary of State or to such persons as he may authorise for that purpose.
%
%(4) An application shall be an effective application if it has been made in accordance with the Secretary of State’s instructions.
%
%(5) Where an application is not effective, the Secretary of State may—
%\begin{enumerate}\item[]
%($a$) give or send a departure direction application form or, as appropriate, a fresh departure direction application form to the person who made the application, and request that the application be re-submitted so as to comply with the provisions of paragraph (4); or
%
%($b$) request the person who made the application to provide such additional information or evidence as the Secretary of State may specify.
%\end{enumerate}
%
%(6) If a completed departure direction application form, or, as the case may be, the additional information or evidence requested, is received by the Secretary of State—
%\begin{enumerate}\item[]
%($a$) within the specified period, he shall treat the application as made on the date on which the earlier or earliest application would have been treated as made had it been an effective application;
%
%($b$) outside the specified period, unless he is satisfied that the delay has been unavoidable, he shall treat the application as a fresh application, made on the date upon which the departure direction application form, or the additional information or evidence, was received.
%\end{enumerate}
%
%(7) Where a completed departure direction application form or the additional information or evidence requested by the Secretary of State in accordance with paragraph (5) is not provided by the applicant within the specified period, his application shall be deemed to have been withdrawn.
%
%(8) For the purposes of paragraphs (6) and (7), the specified period shall be the period of 14 days commencing with the date upon which notice of the Secretary of State’s request is given or sent to the applicant.
%
%(9) A person applying for a departure direction may authorise a representative, whether or not legally qualified, to receive notices and other documents on his behalf, and to act on his behalf in relation to an application.
%
%(10) Where a person has, under paragraph (9), authorised a representative who is not legally qualified, he shall confirm that authorisation in writing, or as otherwise required, to the Secretary of State, unless such authorisation has already been approved by the Secretary of State under regulation 53 of the Maintenance Assessment Procedure Regulations (authorisation of representative).
%
%\subsection[5. Amendment or withdrawal of application]{Amendment or withdrawal of application}
%
%5.  A person who has made an effective application may amend or withdraw his application by notice in writing to the Secretary of State at any time prior to a determination being made in relation to that application.
%
%\subsection[6. Provision of information]{Provision of information}
%
%6.—(1) Where an application has been made, the Secretary of State may request further information or evidence from the applicant to enable that application to be determined.
%
%(2) Any information or evidence requested by the Secretary of State in accordance with paragraph (1) shall be given within 14 days of the request for such information or evidence having been given or sent.
%
%(3) Where the time limit specified in paragraph (2) is not complied with, the Secretary of State may determine that application, in the absence of that information or evidence.
%
%\subsection[7. Rejection of application on completion of a preliminary consideration]{Rejection of application on completion of a preliminary consideration}
%
%7.  The Secretary of State may, on completing a preliminary consideration of an application, reject that application on the ground set out in section 28B(2)($b$) of the Act if it appears to him that the difference between the current amount and the revised amount is less than £1.00.
%
%\subsection[8. Procedure in relation to the determination of an application]{Procedure in relation to the determination of an application}
%
%8.—(1) Where an application has not failed within the meaning of section 28D of the Act, the Secretary of State shall—
%\begin{enumerate}\item[]
%($a$) give notice of that application to the relevant persons other than the applicant;
%
%($b$) send to them a copy of the application and any relevant information the applicant has given except where the Secretary of State considers that information to be harmful information;
%
%($c$) invite representations in writing from the relevant persons on any matter relating to that application; and
%
%($d$) set out the provisions of paragraphs (4) and (5) in relation to such representations.
%\end{enumerate}
%
%(2) The notice referred to in paragraph (1) shall be given as soon as reasonably practicable after—
%\begin{enumerate}\item[]
%($a$) completion of the preliminary consideration of that application under section 28B of the Act; or
%
%($b$) where the Secretary of State has requested information or evidence under regulation 6, receipt of that information or evidence or the expiry of the period of 14 days referred to in regulation 6(2).
%\end{enumerate}
%
%(3) For the purposes of this regulation “harmful information” means medical evidence or medical advice that has not been disclosed to the applicant or a relevant person and which the Secretary of State considers would be harmful to the health of the applicant or that relevant person if disclosed to him.
%
%(4) Where the Secretary of State does not receive written representations from a relevant person within 14 days of the date on which representations were invited under paragraph (1) or (6), he may, in the absence of written representations from that person, proceed to determine the application.
%
%(5) The Secretary of State may, if he considers it reasonable to do so, send a copy of any written representations made following an invitation under paragraph (1)($c$), whether or not they were received within the time specified in paragraph (4), to the applicant and invite him to submit representations in writing on any matters contained in those representations and the provisions of paragraph (4) shall apply to any representations so made.
%
%(6) Where any information or evidence requested by the Secretary of State under regulation 6 is received after notification has been given under paragraph (1), the Secretary of State may, if he considers it reasonable to do so and except where he considers that information to be harmful information, send a copy of such information or evidence to the relevant persons and invite them to submit representations in writing on that information or evidence.
%
%(7) Except where a person gives written permission to the Secretary of State that the information in relation to him mentioned in sub-paragraphs ($a$) and ($b$) may be conveyed to other persons, any document given or sent under the provisions of paragraph (1), (5), (6) or (9) shall not contain—
%\begin{enumerate}\item[]
%($a$) the address of any person other than the recipient of the document in question (other than the address of the office of the Secretary of State) or any other information the use of which could reasonably be expected to lead to any such person being located;
%
%($b$) any other information, the use of which could reasonably be expected to lead to any person other than a qualifying child or relevant person being identified.
%\end{enumerate}
%
%(8) In deciding whether to make a departure direction under section 28F of the Act, the Secretary of State shall take into account—
%\begin{enumerate}\item[]
%($a$) any information given by the applicant for that direction; and
%
%($b$) any written representations made by the applicant or by a relevant person and received by him at the date upon which he determines the application,
%\end{enumerate}
%and may in addition take into account—
%\begin{enumerate}\item[]
%(i) any relevant information received by him or by a child support officer, in relation to any application for a maintenance assessment or for a review of a maintenance assessment made in respect of the absent parent, person with care and any child in respect of whom the current assessment was made;
%
%(ii) any relevant information acquired by him in connection with any of his functions under any of the benefit Acts or the Jobseekers Act 1995\footnote{\frenchspacing 1995 c. 18.}.
%\end{enumerate}
%
%(9) Where the Secretary of State has determined an application he shall, as soon as is reasonably practicable—
%\begin{enumerate}\item[]
%($a$) notify the relevant persons of that determination;
%
%($b$) where a departure direction has been given, refer the case to a child support officer.
%\end{enumerate}
%
%(10) A notification under paragraph (9)($a$) shall set out—
%\begin{enumerate}\item[]
%($a$) the reasons for that determination;
%
%($b$) where a departure direction has been given, the basis on which the amount of child support maintenance is to be fixed by any assessment made in consequence of that direction.
%\end{enumerate}
%
%\subsection[9. Disclosure of information by a child support officer]{Disclosure of information by a child support officer}
%
%9.  A child support officer may disclose to the Secretary of State, for the purposes of the determination of an application, information held by him for the purposes of the Act which has been provided by or in relation to a person in connection with an application for a maintenance assessment, a review of a maintenance assessment, or otherwise in connection with an assessment which is or has been in force.
%
%\subsection[10. Departure directions and interim maintenance assessments]{Departure directions and interim maintenance assessments}
%
%10.—(1) No application may be made where a maintenance assessment in force is a Category A interim maintenance assessment, as defined in paragraph (3)($a$) of regulation 8 of the Maintenance Assessment Procedure Regulations\footnote{\frenchspacing Regulation 8(3) was substituted by regulation 16 of S.I. 1995/3261.} (categories of interim maintenance assessment) or a Category C interim maintenance assessments, as defined in paragraph (3)($c$) of that regulation.
%
%(2) No application may be made in reliance on regulation 18 (costs of supporting certain children) where the maintenance assessment in force at the time of that application is a Category B interim maintenance assessment, as defined in paragraph (3)($b$) of regulation 8 of the Maintenance Assessment Procedure Regulations and that Category B interim maintenance assessment was made because the applicant fell within that paragraph.
%
%(3) No application may be made by an absent parent against whom there is in force a Category D interim maintenance assessment, as defined in paragraph (3)($d$) of regulation 8 of the Maintenance Assessment Procedure Regulations.
%
%\subsection[11. Lapse of an application]{Lapse of an application}
%
%11.  Where a case falls within subsection (6) of section 28B of the Act, the prescribed period for the purposes of paragraph ($b$) of that subsection shall be the period of 14 days commencing on the date on which the Secretary of State gives or sends the notification mentioned in paragraph ($a$) of that subsection.
%
%\subsection[12. Meaning of “benefit” for the purposes of section 28E of the Act]{Meaning of “benefit” for the purposes of section 28E of the Act}
%
%12.  For the purposes of section 28E of the Act, “benefit” means income support, income-based jobseeker’s allowance, family credit, disability working allowance, housing benefit, and council tax benefit.
%
%\section[Part III --- Special expenses]{Part III\\*Special expenses}
%
%\subsection[13. Costs incurred in travelling to work]{Costs incurred in travelling to work}
%
%\renewcommand\parthead{--- Part III}
%
%13.—(1) Subject to paragraphs (2) and (3), the following costs shall constitute expenses for the purposes of paragraph 2(2) of Schedule 4B to the Act where they are incurred by the applicant for the purposes of travel between his home and his work place—
%\begin{enumerate}\item[]
%($a$) the cost of purchasing a ticket for such travel;
%
%($b$) the cost of purchasing fuel, where such travel is by a vehicle which is not carrying fare-paying passengers; or
%
%($c$) in exceptional circumstances, the taxi fare for a journey which must unavoidably be undertaken during hours when no other reasonable mode of travel is available,
%\end{enumerate}
%and any minor incidental costs such as tolls or fees for the use of a particular road or bridge incurred in connection with such travel.
%
%(2) Where the Secretary of State considers any costs referred to in paragraph (1) to be unreasonably high or to have been unreasonably incurred he may substitute such lower amount as he considers reasonable, including a nil amount.
%
%(3) Costs which can be set off against the income of the applicant under the Income and Corporation Taxes Act 1988\footnote{\frenchspacing 1988 c. 1.} shall not constitute expenses for the purposes of paragraph (1).
%
%(4) For the purposes of paragraph (1) “work place” means the normal place or places—
%\begin{enumerate}\item[]
%($a$) in which an applicant is employed in employed earner’s employment; or
%
%($b$) in which an applicant carries out his business if he is a self-employed earner.
%\end{enumerate}
%
%(5) For the purposes of paragraph (4), “employed earner” and “self-\hspace{0pt}employed earner” have the same meaning as in the Contributions and Benefits Act.
%
%\subsection[14. Contact costs]{Contact costs}
%
%14.—(1) Where at the time a departure direction is applied for a set pattern has been established as to frequency of contact between the absent parent and a child in respect of whom the current assessment was made, the following costs, based upon that pattern and incurred by that absent parent for the purpose of maintaining contact with that child, shall, subject to paragraphs (2) to (4), constitute expenses for the purposes of paragraph 2(2) of Schedule 4B to the Act—
%\begin{enumerate}\item[]
%($a$) the cost of purchasing a ticket for such travel; or
%
%($b$) the cost of purchasing fuel, where such travel is by a vehicle which is not carrying fare-paying passengers,
%\end{enumerate}
%and any minor incidental costs such as tolls or fees for the use of a particular road or bridge, incurred in connection with such travel.
%
%(2) Subject to paragraph (3), where the Secretary of State considers any costs referred to in paragraph (1) to be unreasonably high or to have been unreasonably incurred he may substitute such lower amount as he considers reasonable, including a nil amount.
%
%(3) Any lower amount substituted by the Secretary of State under paragraph (2) shall not be so low as to make it impossible, in the Secretary of State’s opinion, for contact to be maintained at the frequency specified in any court order made in respect of the absent parent and the child mentioned in paragraph (1).
%
%(4) Paragraph (1) shall not apply where regulation 20 of the Maintenance Assessments and Special Cases Regulations (persons treated as absent parents) applies to the applicant.
%
%(5) Where, at the time a departure direction is applied for, no set pattern has been established as to frequency of contact between the absent parent and a child in respect of whom the current assessment was made, but the Secretary of State is satisfied that that absent parent and the person with care of that child have agreed upon a pattern of contact for the future, the costs mentioned in paragraph (1) and which are based upon that intended pattern of contact shall constitute expenses for the purposes of paragraph 2(2) of Schedule 4B to the Act, and paragraphs (2) to (4) shall apply to that application.
%
%\subsection[15. Illness or disability]{Illness or disability}
%
%15.—(1) Subject to paragraphs (2) and (3), the costs incurred in respect of the items listed in sub-paragraphs ($a$) to ($m$), which arise from long-term illness or disability of the applicant or a dependant of that applicant and which are in excess of the costs which would be incurred if that illness or disability did not exist shall constitute special expenses for the purposes of paragraph 2(2) of Schedule 4B to the Act—
%\begin{enumerate}\item[]
%($a$) personal care and attendance;
%
%($b$) personal communication needs;
%
%($c$) mobility;
%
%($d$) domestic help;
%
%($e$) medical aids where these cannot be provided under the health service;
%
%($f$) heating;
%
%($g$) clothing;
%
%($h$) laundry requirements;
%
%($i$) payments for food essential to comply with a diet recommended by a medical practitioner;
%
%($j$) adaptations required to the applicant’s home;
%
%($k$) day care;
%
%($l$) rehabilitation; or
%
%($m$) respite care.
%\end{enumerate}
%
%(2) Where the Secretary of State considers any costs referred to in paragraph (1) to be unreasonably high or to have been unreasonably incurred, he may substitute such lower amount as he considers reasonable, including a nil amount.
%
%(3) Where—
%\begin{enumerate}\item[]
%($a$) an applicant or his dependant has, at the date an application is made, received, or at that date is in receipt of, financial assistance from any source in respect of his long-term illness or disability or that of his dependent; or
%
%($b$) that applicant or his dependant is adjudged eligible for either of the allowances referred to in paragraph (4),
%\end{enumerate}
%only the net amount of the costs incurred in respect of the items listed in paragraph (1), after the deduction of the financial assistance referred to in sub-paragraph ($a$) and, where applicable, the allowance referred to in sub-paragraph ($b$) shall constitute special expenses for the purposes of paragraph 2(2) of Schedule 4B to the Act.
%
%(4) Where an application is made with respect to special expenses falling within paragraph (1), and the Secretary of State considers that the applicant or his dependant may be entitled to disability living allowance under section 71 of the Contributions and Benefits Act or attendance allowance under section 64 of that Act, that application shall not be determined until a decision has been made by the adjudicating authority on the eligibility for that allowance of that applicant or that dependant.
%
%(5) For the purposes of this regulation, a dependant of an applicant shall be—
%\begin{enumerate}\item[]
%($a$) where the applicant is an absent parent—
%\begin{enumerate}\item[]
%(i) the partner of that absent parent;
%
%(ii) any child of whom that absent parent or his partner is a parent and who lives with them; or
%\end{enumerate}
%
%($b$) where the applicant is a parent with care—
%\begin{enumerate}\item[]
%(i) the partner of that parent with care;
%
%(ii) any child of whom that parent with care or her partner is a parent and who lives with them, except any child in respect of whom the absent parent against whom the current assessment is made is the parent.
%\end{enumerate}
%\end{enumerate}
%
%(6) For the purposes of this regulation—
%\begin{enumerate}\item[]
%($a$) a person is disabled if he is blind, deaf or dumb or is substantially or permanently handicapped by illness, injury, mental disorder or congenital deformity;
%
%($b$) “long-term illness” means an illness from which the applicant or his dependant is suffering at the date of the application and which is likely to last for at least 52 weeks in total from that date or if likely to be shorter than 52 weeks, for the rest of the life of that applicant or his dependant;
%
%($c$) “the health service” has the same meaning as in section 128 of the National Health Service Act 1977\footnote{\frenchspacing 1977 c. 49.}.
%\end{enumerate}
%
%\subsection[16. Debts incurred before the absent parent became an absent parent]{Debts incurred before the absent parent became an absent parent}
%
%16.—(1) Subject to paragraphs (2) to (6), repayment of debts incurred—
%\begin{enumerate}\item[]
%($a$) for the joint benefit of the applicant and the non-applicant parent;
%
%($b$) for the benefit of the non-applicant parent where the applicant remains wholly responsible for that repayment;
%
%($c$) for the benefit of any person who at the time the debt was incurred—
%\begin{enumerate}\item[]
%(i) was a child;
%
%(ii) lived with the applicant and non-applicant parent; and
%
%(iii) of whom the applicant or the non-applicant parent is the parent, or both are the parents; or
%\end{enumerate}
%
%($d$) for the benefit of any child with respect to whom the current assessment was made,
%\end{enumerate}
%shall constitute expenses for the purposes of paragraph 2(2) of Schedule 4B to the Act where those debts were incurred before the absent parent became an absent parent in relation to a child with respect to whom the current assessment was made and at a time when the applicant and the non-applicant parent were a married or unmarried couple who were living together.
%
%(2) Paragraph (1) shall not apply to repayment of—
%\begin{enumerate}\item[]
%($a$) a debt which would otherwise fall within paragraph (1) where the applicant has retained for his own use and benefit the asset the purchase of which incurred the debt;
%
%($b$) a loan taken out for the purposes of any trade or business carried on by the applicant;
%
%($c$) a gambling debt of the applicant;
%
%($d$) a fine imposed on the applicant;
%
%($e$) unpaid legal costs of the applicant in respect of separation or divorce from the non-applicant parent;
%
%($f$) amounts due after use of a credit card by the applicant;
%
%($g$) a loan taken out by the applicant to pay any of the items listed in sub-paragraphs ($c$) to ($f$);
%
%($h$) amounts payable by the applicant under a mortgage or loan taken out on the security of any property except where that mortgage or loan was taken out to facilitate the purchase of any property which at the time the application is made is the home of the parent with care and any child in respect of whom the current assessment was made;
%
%($i$) amounts payable by the applicant in respect of a policy of insurance of a kind referred to in paragraph 3(4) or (5) of Schedule 3 to the Maintenance Assessments and Special Cases Regulations\footnote{\frenchspacing Paragraph 3(4) was amended by S.I. 1995/1045 and paragraph 3(5) by S.I. 1994/227.} (eligible housing costs) except where that policy of insurance was obtained or retained to discharge a mortgage or charge taken out to facilitate the purchase of any property which, at the time the application is made, is the home of the parent with care and any child in respect of whom the current assessment was made;
%
%($j$) bank overdrafts except where the overdraft was, at the time it was taken out, agreed to be for a specified amount repayable over a specified period;
%
%($k$) a loan obtained by the applicant other than a loan obtained from a qualifying lender;
%
%($l$) a debt in respect of which a departure direction has already been given and which has not been repaid during the period that direction was in force except where the maintenance assessment in respect of which that direction was given was cancelled or ceased to have effect and, during the period for which that direction was given, a further maintenance assessment was made in respect of the same applicant, non-applicant and qualifying child with respect to whom the earlier assessment was made; or
%
%($m$) any other debt which the Secretary of State is satisfied it is reasonable to exclude.
%\end{enumerate}
%
%(3) Repayment of a debt shall not constitute expenses for the purposes of paragraph (1) where the Secretary of State is satisfied that the applicant agreed with the non-applicant parent to take responsibility for repayment of that debt, as, or as part of, a financial settlement between them.
%
%(4) Where a debt was incurred prior to the date upon which the absent parent became an absent parent in relation to a child with respect to whom the current assessment was made, partly to repay a debt or debts repayment of which would have fallen within paragraph (1), the repayment of that part of the debt incurred which is referrable to the debts repayment of which would have fallen within that paragraph shall constitute expenses for the purposes of paragraph 2(2) of Schedule 4B to the Act.
%
%(5) For the purposes of this regulation—
%\begin{enumerate}\item[]
%($a$) “married or unmarried couple” has the meaning set out in regulation 1 of the Maintenance Assessments and Special Cases Regulations;
%
%($b$) “non-applicant parent” means—
%\begin{enumerate}\item[]
%(i) where the applicant is the person with care, the absent parent;
%
%(ii) where the applicant is the absent parent, the partner of that absent parent at the time the debt in respect of which the application is made was entered into;
%\end{enumerate}
%
%($c$) “qualifying lender” has the meaning given to it in section 376(4) of the Income and Corporation Taxes Act 1988\footnote{\frenchspacing 1988 c. 1.}.
%\end{enumerate}
%
%\subsection[17. Pre-1993 financial commitments]{Pre-1993 financial commitments}
%
%17.—(1) A financial commitment entered into by an absent parent before 5th April 1993, except any commitment falling within regulation 16, shall constitute expenses for the purposes of paragraph 2(2) of Schedule 4B to the Act where—
%\begin{enumerate}\item[]
%($a$) there was in force on 5th April 1993 and at the date that commitment was entered into, a court order or maintenance agreement made before 5th April 1993 in respect of that absent parent and a child in relation to whom the current assessment was made; and
%
%($b$) the Secretary of State is satisfied that it is impossible for the absent parent to withdraw from that commitment or unreasonable to expect him to do so.
%\end{enumerate}
%
%(2) For the purposes of paragraph (1)—
%\begin{enumerate}\item[]
%($a$) “court order” means an order made under an enactment listed in or prescribed under section 8(11) of the Act, for the making or securing the making of financial provision for the benefit of a child in respect of whom the current assessment was made;
%
%($b$) “maintenance agreement” means an agreement in writing for the making or securing the making of financial provision for the benefit of a child in respect of whom the current assessment was made.
%\end{enumerate}
%
%\subsection[18. Costs incurred in supporting certain children]{Costs incurred in supporting certain children}
%
%18.—(1) The costs incurred by a parent in supporting a child who is not his child but who is part of his family (a “relevant child”) shall constitute special expenses for the purposes of paragraph 2(2) of Schedule 4B to the Act if the conditions set out in paragraph (2) are satisfied and shall, if those conditions are satisfied, equal the amount specified in paragraph (3).
%
%(2) The conditions referred to in paragraph (1) are—
%\begin{enumerate}\item[]
%($a$) such costs were first incurred prior to 5th April 1993;
%
%($b$) subject to paragraph (7), the liability of the absent parent of a relevant child to pay maintenance to or for the benefit of that child under a court order, a written maintenance agreement or a maintenance assessment is less than the amount specified in paragraph (4), or there is no such liability; and
%
%($c$) the net income of the parent’s current partner where the relevant child is the child of that partner, calculated in accordance with paragraph (5), is less than the amount calculated in accordance with paragraph (6) (“the partner’s outgoings”).
%\end{enumerate}
%
%(3) The amount referred to in paragraph (1) constituting special expenses for a case falling within this regulation is the difference between the amount specified in paragraph (4) and, subject to paragraph (7), the liability of the absent parent of a relevant child to pay maintenance of a kind mentioned in paragraph (2)($b$), and if there is no such liability is the amount specified in paragraph (4).
%
%(4) The amount referred to in paragraphs (2)($b$) and (3) is the aggregate of—
%\begin{enumerate}\item[]
%($a$) an amount in respect of each relevant child equal to the personal allowance for that child specified in column (2) of paragraph 2 of the relevant Schedule (income support personal allowance);
%
%($b$) if the conditions set out in paragraph 14($b$) and ($c$) of the relevant Schedule (income support disabled child premium) are satisfied in respect of a relevant child, an amount equal to the amount specified in column (2) of paragraph 15(6) of that Schedule in respect of each such child;
%
%($c$) an amount equal to the income support family premium specified in paragraph 3 of that Schedule that would be payable if the parent were a claimant, except where the family includes other children of the parent; and
%
%($d$) an amount equal to the income support lone parent premium specified in column (2) of paragraph 15(1) of that Schedule that would be payable, if the parent were a claimant, except where the family includes children of the parent.
%\end{enumerate}
%
%(5) For the purposes of paragraph (2)($c$), the net income of the parent’s partner shall be the aggregate of—
%\begin{enumerate}\item[]
%($a$) the income of that partner, calculated in accordance with regulation 7(1) of the Maintenance Assessments and Special Cases Regulations (but excluding the amount mentioned in sub-paragraph ($d$) of that regulation) as if that partner were an absent parent to whom that regulation applied;
%
%($b$) the child benefit payable in respect of each relevant child; and
%
%($c$) any income, other than earnings, in excess of £10.00 per week in respect of each relevant child.
%\end{enumerate}
%
%(6) For the purposes of paragraph (2)($c$), a current partner’s outgoings shall be the aggregate of—
%\begin{enumerate}\item[]
%($a$) an amount equal to the amount specified in column (2) of paragraph 1(1)($e$) of the relevant Schedule (income support personal allowance for a single claimant aged not less than 25);
%
%($b$) where a departure direction has already been given in a case falling within regulation 27 in respect of the housing costs attributable to the partner, the amount determined in accordance with that regulation as the housing costs the partner is able to contribute;
%
%($c$) the amount of any reduction in the parent’s exempt income, calculated under paragraph (1) of regulation 9 of the Maintenance Assessments and Special Cases Regulations\footnote{\frenchspacing Paragraph (1) of regulation 9 was amended by regulation 44(2) of S.I. 1995/1045. Paragraph (2) was amended by regulation 9(2)($c$) of S.I. 1993/913 and regulation 44(3) of S.I. 1995/1045.}, in consequence of the application of paragraph (2) of that regulation; and
%
%($d$) the amount specified in paragraph (3).
%\end{enumerate}
%
%(7) The Secretary of State may, if he is satisfied that it is appropriate in the particular circumstances of the case, treat a liability of a kind mentioned in paragraph (2)($b$) as not constituting a liability for the purposes of that paragraph and of paragraph (3).
%
%(8) For the purposes of this regulation—
%\begin{enumerate}\item[]
%($a$) a child who is not the child of a particular person is a part of that person’s family where that child is the child of a current or former partner of that person;
%
%($b$) “relevant Schedule” means Schedule 2 to the Income Support (General) Regulations 1987\footnote{\frenchspacing S.I. 1987/1967. Paragraphs 1 and 2 of Schedule 2 were substituted by Schedule 4 to S.I. 1995/559; paragraph 15 was substituted by Schedule 5 to that instrument.}.
%\end{enumerate}
%
%\subsection[19. Special expenses for a case falling within regulation 13, 14, 16 or 17]{Special expenses for a case falling within regulation 13, 14, 16 or 17}
%
%19.—(1) This regulation applies where the expenses of an applicant fall within one or more of the descriptions of expenses falling within regulation 13 (travel to work costs), 14 (contact costs), 16 (debts incurred before the absent parent became an absent parent) or 17 (pre-1993 financial commitments).
%
%(2) Special expenses for the purposes of paragraph 2(2) of Schedule 4B to the Act in respect of the expenses mentioned in paragraph (1) shall be—
%\begin{enumerate}\item[]
%($a$) where the expenses fall within only one description of expenses, those expenses in excess of £15.00;
%
%($b$) where the expenses fall within more than one description of expenses, the aggregate of those expenses in excess of £15.00.
%\end{enumerate}
%
%\subsection[20. Application for a departure direction in respect of special expenses other than those with respect to which a direction has already been given]{Application for a departure direction in respect of special expenses other than those with respect to which a direction has already been given}
%
%20.  Where a departure direction with respect to special expenses falling within one or more of the descriptions of expenses falling within regulation 13, 14, 16 or 17 has already been given and an application with respect to special expenses falling within one or more of those descriptions of expenses is made where none of those expenses are ones with respect to which the earlier direction has been given, the special expenses with respect to which any later direction is given shall be the expenses, determined in accordance with regulation 13, 14, 16 or 17, as the case may be, with respect to which the later application is made, and the provisions of regulation 19 shall not apply.
%
%\section[Part IV --- Property or capital transfers]{Part IV\\*Property or capital transfers}
%
%\subsection[21. Prescription of certain terms for the purposes of paragraphs 3 and 4 of Schedule 4B to the Act]{Prescription of certain terms for the purposes of paragraphs 3 and 4 of Schedule 4B to the Act}
%
%\renewcommand\parthead{--- Part IV}
%
%21.—(1) For the purposes of paragraphs 3(1)($a$) and 4(1)($a$) of Schedule 4B to the Act—
%\begin{enumerate}\item[]
%($a$) a court order means an order made—
%\begin{enumerate}\item[]
%(i) under one or more of the enactments listed in or prescribed under section 8(11) of the Act; and
%
%(ii) in connection with the transfer of property of a kind defined in paragraph (2);
%\end{enumerate}
%
%($b$) an agreement means a written agreement made in connection with the transfer of property of a kind defined in paragraph (2).
%\end{enumerate}
%
%(2) Subject to paragraphs (3) to (5), for the purposes of paragraph 3(1)($b$) and 4(1)($b$) of Schedule 4B to the Act, a transfer of property is a transfer by the absent parent of his beneficial interest in any asset to the person with care, to a child in respect of whom the current assessment was made or to trustees where the object or one of the objects of the trust is the provision of basic day to day necessities for that child.
%
%(3) Where the provision of maintenance for the child referred to in paragraph (2) is only one of the objects of a trust, the transfer of property for the purposes of paragraph 3(1)($b$) and 4(1)($b$) of Schedule 4B to the Act shall be the transfer of that proportion only of the asset transferred to the trustees which in the opinion of the Secretary of State relates to the maintenance of that child.
%
%(4) Where a transfer of property would not originally have fallen within paragraph (2) but the Secretary of State is satisfied that some or all of the amount of that property transferred was subsequently transferred to the person currently with care of a child in respect of whom the current assessment was made, the transfer of that property to the person currently with care shall count as a transfer of property for the purposes of paragraph 3(1)($b$) and 4(1)($b$) of Schedule 4B to the Act.
%
%(5) Where, if the Act had been in force at the time a transfer of property falling within paragraph (2) was made, the person who, at the time the application is made is the person with care would have been the absent parent and the person who, at the time the application is made is the absent parent would have been the person with care, that transfer shall not count as a transfer of property for the purposes of this regulation.
%
%(6) For the purposes of paragraph 3(3) of Schedule 4B to the Act, the effect of a transfer of property is properly reflected in the current assessment if the amount of child support maintenance payable under any fresh maintenance assessment which would be made in consequence of a departure direction differs from the amount of child support maintenance payable under that current assessment by less than the amount specified in regulation 31.
%
%\subsection[22. Value of a transfer of property and its equivalent weekly value for a case falling within paragraph 3 of Schedule 4B to the Act]{Value of a transfer of property and its equivalent weekly value for a case falling within paragraph 3 of Schedule 4B to the Act}
%
%22.—(1) Where the conditions specified in paragraph 3(1) of Schedule 4B to the Act are satisfied, the value of a transfer of property for the purposes of that paragraph shall be that part of the transfer made by the absent parent (making allowance for any transfer by the person with care to the absent parent) which the Secretary of State is satisfied is in lieu of maintenance.
%
%(2) The equivalent weekly value of a transfer of property shall be determined in accordance with the provisions of the Schedule.
%
%\section[Part V --- Additional cases]{Part V\\*Additional cases}
%
%\subsection[23. Assets capable of producing income or higher income]{Assets capable of producing income or higher income}
%
%\renewcommand\parthead{--- Part V}
%
%23.—(1) Subject to paragraphs (2) to (4), a case shall constitute a case for the purposes of paragraph 5(1) of Schedule 4B to the Act where—
%\begin{enumerate}\item[]
%($a$) the Secretary of State is satisfied that any asset in which the non-applicant has a legal estate or beneficial interest, or which he has the ability to control—
%\begin{enumerate}\item[]
%(i) is capable of being utilised to produce income but has not been so utilised;
%
%(ii) has been invested in such a way that the income obtained from it is less than might reasonably be expected;
%
%(iii) is a chose in action which has not been enforced where the Secretary of State is satisfied that such enforcement would be reasonable; or
%
%(iv) has not been sold where the Secretary of State is satisfied that the sale of the asset would be reasonable;
%\end{enumerate}
%
%($b$) any asset has been transferred by the non-applicant to trustees and the non-applicant is a beneficiary of the trust so created; or
%
%($c$) any asset has become subject to an implied resulting or contructive trust of which the non-applicant is a beneficiary.
%\end{enumerate}
%
%(2) Paragraph (1) shall not apply where—
%\begin{enumerate}\item[]
%($a$) the total value of the asset or assets referred to in that paragraph does not exceed £10,000.00 after deduction of the amount owing under any mortgage or charge on that asset; or
%
%($b$) the Secretary of State is satisfied that any asset referred to in that paragraph is being retained by the non-applicant to be used for a purpose which the Secretary of State considers reasonable in all the circumstances of the case.
%\end{enumerate}
%
%(3) No application may be made under this regulation where income support is paid to or in respect of the non-applicant.
%
%(4) For the purposes of this regulation the term “asset” means—
%\begin{enumerate}\item[]
%($a$) money, whether in cash or on deposit;
%
%($b$) any asset used in the course of a trade or business carried on—
%\begin{enumerate}\item[]
%(i) by the non-applicant as a sole trader or in partnership; or
%
%(ii) by a close company within the meaning of section 414 and 415 of the Income and Corporation Taxes Act 1988 in which the non-applicant is a participator;
%\end{enumerate}
%
%($c$) a legal estate or equitable interest in land and rights in or over land;
%
%($d$) shares as defined in section 455 of the Companies Act 1948\footnote{\frenchspacing 11 \& 12 Geo. 6 c. 38.}, stock and unit trusts as defined in section 6 of the Charging Orders Act 1979\footnote{\frenchspacing 1979 c. 53.}, gilt edged securities as defined in paragraph 1 of Schedule 9 to the Taxation of Chargeable Gains Act 1979\footnote{\frenchspacing 1979 c. 12.}, and other similar financial instruments.
%\end{enumerate}
%
%(5) For the purposes of paragraph (4) the term “asset” includes any asset falling within that paragraph which is located outside Great Britain.
%
%\subsection[24. Diversion of income]{Diversion of income}
%
%24.  A case shall constitute a case for the purposes of paragraph 5(1) of Schedule 4B to the Act where—
%\begin{enumerate}\item[]
%($a$) the non-applicant has the ability to control the amount of income he receives, including earnings from employment or self-employment and dividends from shares, whether or not the whole of that income is derived from the company or business from which his earnings are derived; and
%
%($b$) the Secretary of State is satisfied that the non-applicant has reduced the amount of income he would otherwise receive by diverting it to other persons or for purposes other than the provision of income for himself.
%\end{enumerate}
%
%\subsection[25. Life-style inconsistent with declared income]{Life-style inconsistent with declared income}
%
%25.—(1) Subject to paragraph (2), a case shall constitute a case for the purposes of paragraph 5(1) of Schedule 4B to the Act where the Secretary of State is satisfied that the current maintenance assessment is based upon a level of income of the non-applicant which is substantially lower than the level of income required to support the overall life-style of that non-applicant.
%
%(2) Paragraph (1) shall not apply where—
%\begin{enumerate}\item[]
%($a$) income support is paid to or in respect of the non-applicant;
%
%($b$) the Secretary of State is satisfied that the life-style of the non-applicant is paid for—
%\begin{enumerate}\item[]
%(i) out of capital belonging to him; or
%
%(ii) by his partner unless the non-applicant is able to influence or control the amount of income received by that partner.
%\end{enumerate}
%\end{enumerate}
%
%(3) Where the Secretary of State is satisfied that the life-style of the non-applicant is paid for by his partner, the Secretary of State shall, whether or not any application on that ground has been made, consider whether the case falls within regulation 27.
%
%\subsection[26. Unreasonably high housing costs]{Unreasonably high housing costs}
%
%26.  A case shall constitute a case for the purposes of paragraph 5(1) of Schedule 4B to the Act where—
%\begin{enumerate}\item[]
%($a$) the housing costs of the non-applicant exceed the limits set out in paragraph (1) of regulation 18 of the Maintenance Assessments and Special Cases Regulations (excessive housing costs);
%
%($b$) the non-applicant falls within paragraph (2) of that regulation or would fall within that paragraph if it applied to parents with care; and
%
%($c$) the Secretary of State is satisfied that the housing costs of the non-applicant are substantially higher than is necessary taking into account any special circumstances applicable to that parent.
%\end{enumerate}
%
%\subsection[27. Partner’s contribution to housing costs]{Partner’s contribution to housing costs}
%
%27.  A case shall constitute a case for the purposes of paragraph 5(1) of Schedule 4B to the Act where a partner of the non-applicant occupies the home with him and the Secretary of State considers that it is reasonable for that partner to contribute to the payment of the housing costs of the non-applicant.
%
%\subsection[28. Unreasonably high travel costs]{Unreasonably high travel costs}
%
%28.  A case shall constitute a case for the purposes of paragraph 5(1) of Schedule 4B to the Act where an amount in respect of travel to work costs has been included in the calculation of the exempt income of the non-applicant under regulation 9(1)($i$) of the Maintenance Assessments and Special Cases Regulations\footnote{\frenchspacing Sub-paragraph ($i$) was added to regulation 9(1) by regulation 44(2)($b$) of S.I. 1995/1045.} (exempt income: calculation or estimation of E) or, as the case may be, under regulation 10 of those Regulations (exempt income: calculation or estimation of F)\footnote{\frenchspacing Regulation 10 was amended by regulation 45 of S.I. 1995/1045.} applying regulation 9(1)($i$), and the Secretary of State is satisfied that, in all the circumstances of the case, that amount is unreasonably high.
%
%\subsection[29. Travel costs to be disregarded]{Travel costs to be disregarded}
%
%29.  A case shall constitute a case for the purposes of paragraph 5(1) of Schedule 4B to the Act where—
%\begin{enumerate}\item[]
%($a$) an amount in respect of travel to work costs has, in the calculation of a maintenance assessment, been included in the calculation of the exempt income of the non-applicant under regulation 9(1)($i$) of the Maintenance Assessments and Special Cases Regulations or, as the case may be, under regulation 10 of those Regulations applying regulation 9(1)($i$); and
%
%($b$) the Secretary of State is satisfied that the non-applicant has sufficient income remaining after the deduction of the amount that would be payable under that assessment, had the amount referred to in paragraph ($a$) not been included in its calculation, for it to be inappropriate for all or part of that amount to be included in the exempt income of the non-applicant.
%\end{enumerate}
%
%\section[Part VI --- Factors to be taken into account for the purposes of section 28F of the Act]{Part VI\\*Factors to be taken into account for the purposes of section 28F of the Act}
%
%\subsection[30. Factors to be taken into account and not to be taken into account in determining whether it would be just and equitable to give a departure direction]{Factors to be taken into account and not to be taken into account in determining whether it would be just and equitable to give a departure direction}
%
%\renewcommand\parthead{--- Part VI}
%
%30.—(1) The factors to be taken into account in determining whether it would be just and equitable to give a departure direction in any case shall include—
%\begin{enumerate}\item[]
%($a$) where the application is made on any ground—
%\begin{enumerate}\item[]
%(i) whether, in the opinion of the Secretary of State, the giving of a departure direction would be likely to result in a relevant person ceasing paid employment;
%
%(ii) if the applicant is the absent parent, the extent, if any, of his liability to pay child maintenance under a court order or other agreement in the period prior to the effective date of the maintenance assessment;
%\end{enumerate}
%
%($b$) where an application is made on the ground that the case falls within regulations 13 to 20 (special expenses), whether, in the opinion of the Secretary of State—
%\begin{enumerate}\item[]
%(i) the financial arrangements made by the applicant could have been such as to enable the expenses cited to be paid without a departure direction being given;
%
%(ii) the applicant has at his disposal financial resources which are currently utilised for 
%the payment of expenses other than those arising from essential everyday requirements and which could be used to pay the expenses cited.
%\end{enumerate}
%\end{enumerate}
%
%(2) The following factors are not to be taken into account in determining whether it would be just and equitable to give a departure direction in any case—
%\begin{enumerate}\item[]
%($a$) the fact that the conception of a child in respect of whom the current assessment was made was not planned by one or both of the parents;
%
%($b$) whether the parent with care or the absent parent was responsible for the breakdown of the relationship between them;
%
%($c$) the fact that the parent with care or the absent parent has formed a new relationship with a person who is not a parent of the child in respect of whom the current assessment was made;
%
%($d$) the existence of particular arrangements for contact with the child in respect of whom the current assessment was made, including whether any arrangements made are being adhered to by the parents;
%
%($e$) the failure by an absent parent to make payments under a maintenance order, a written maintenance agreement, or a maintenance assessment;
%
%($f$) representatons made by persons other than the relevant persons.
%\end{enumerate}
%
%\section[Part VII --- Effective date and duration of a departure direction]{Part VII\\*Effective date and duration of a departure direction}
%
%\renewcommand\parthead{--- Part VII}
%
%\subsection[31. Refusal to give a departure direction under section 28F(4) of the Act]{Refusal to give a departure direction under section 28F(4) of the Act}
%
%31.  The Secretary of State shall not give a departure direction in accordance with section 28F of the Act if he is satisfied that the difference between the current amount and the revised amount is less than £1.00.
%
%\subsection[32. Effective date of a departure direction]{Effective date of a departure direction}
%
%32.—(1) Where an application is made on the grounds set out in section 28A(2)($a$) of the Act (the effect of the current assessment) and that application is received within 28 days of the date of notification of the current assessment (whether or not that assessment has been made following an interim maintenance assessment), a departure direction given in response to that application shall take effect—
%\begin{enumerate}\item[]
%($a$) where it is given on grounds that relate to the whole of the period between the effective date of the current assessment and the date on which that assessment is made, on the effective date of that assessment;
%
%($b$) in a case not falling within sub-paragraph ($a$), on the first day of the maintenance period following the date upon which the circumstances giving rise to that application first arose.
%\end{enumerate}
%
%(2) Where an application is made on the grounds set out in section 28A(2)($a$) of the Act (the effect of the current assessment) and that application is given or sent later than 28 days after the date of notification of the current assessment (whether or not that assessment has been made following an interim maintenance assessment)—
%\begin{enumerate}\item[]
%($a$) subject to sub-paragraph ($b$), a departure direction given in response to that application shall take effect on the first day of the maintenance period during which that application is received or on that day if it is the first day of a maintenance period;
%
%($b$) where the Secretary of State is satisfied that there was unavoidable delay, he may, for the purposes of determining the date on which a departure direction takes effect, treat the application as if it were given or sent within 28 days of the date of notification of the current assessment.
%\end{enumerate}
%
%(3) Where an application for a departure direction is made on the grounds set out in section 28A(2)($b$) of the Act (a material change in the circumstances of the case since the current assessment was made), any departure direction given shall take effect on the first day of the maintenance period during which the application was received or on that day if it is the first day of the maintenance period.
%
%(4) An application may be made on the grounds set out in section 28A(2)($b$) of the Act only if the material change in the circumstances on which it is based has already occurred.
%
%\subsection[33. Cancellation of a departure direction following a review under section 16, 17, 18 or 19 of the Act or on a change of circumstances]{Cancellation of a departure direction following a review under section 16, 17, 18 or 19 of the Act or on a change of circumstances}
%
%33.—(1) Where the Secretary of State is satisfied that, following a review under section 16, 17, 18 or 19 of the Act or a change in the circumstances of the case, it is no longer appropriate for a departure direction to continue to have effect, he shall cancel that direction.
%
%(2) A departure direction that is cancelled under the provisions of paragraph (1) shall cease to have effect on the last day of the maintenance period during which the Secretary of State is given or sent, or becomes aware of, the information which leads him to become satisfied that it is no longer appropriate for the departure direction to continue to have effect or on that day if it is the last day of a maintenance period.
%
%(3) Where a departure direction has effect and the applicant in respect of whom the direction has been given applies for a further departure direction on grounds to which the same regulation is or the same regulations are applicable as that or those applicable to the departure direction that has effect, any departure direction given in response to that application shall take effect on the first day of the maintenance period during which the application was given or sent or on that day if it is the first day of a maintenance period, and the earlier direction shall cease to have effect immediately prior to the coming into effect of the later direction.
%
%\subsection[34. Cancellation of a departure direction on recognition of an error]{Cancellation of a departure direction on recognition of an error}
%
%34.—(1) Where the Secretary of State is satisfied that a departure direction was given in error, he shall cancel that direction.
%
%(2) The cancellation of a departure direction under paragraph (1) shall take effect from the date on which that direction took effect.
%
%\subsection[35. Termination and suspension of departure directions]{Termination and suspension of departure directions}
%
%35.—(1) Subject to paragraph (2), where a maintenance assessment made in consequence of a departure direction is cancelled or ceases to have effect, the departure direction shall cease to have effect and shall not subsequently take effect.
%
%(2) Where a child support officer ceases to have jurisdiction to make a maintenance assessment and subsequently acquires jurisdiction to make a maintenance assessment in respect of the same absent parent, person with care and any child with respect to whom the earlier assessment was made, a departure direction for a case falling within paragraph 3 or 4 of Schedule 4B to the Act shall again take effect from the effective date of a fresh maintenance assessment.
%
%\section[Part VIII --- Maintenance assessment following a departure direction]{Part VIII\\*Maintenance assessment following a departure direction}
%
%\renewcommand\parthead{--- Part VIII}
%
%\subsection[36. Effect of a departure direction—general]{Effect of a departure direction—general}
%
%36.—(1) Except where a case falls within regulation 22, 41 or 42, a departure direction shall specify, as the basis on which the amount of child support maintenance is to be fixed by any fresh assessment made in consequence of the direction, that the amount of net income or exempt income of the parent with care or absent parent or the amount of protected income of the absent parent be increased or, as the case may be, decreased in accordance with those provisions of regulations 37, 38 and 40 which are applicable to the particular case.
%
%(2) Where the provisions of paragraph (1) apply to a departure direction, the amount of child support maintenance fixed by a fresh maintenance assessment shall be determined in accordance with the provisions of Part I of Schedule 1 to the Act, but with the substitution of the amounts changed in consequence of the direction for the amounts determined in accordance with those provisions.
%
%\subsection[37. Effect of a departure direction in respect of special expenses—exempt income]{Effect of a departure direction in respect of special expenses—\hspace{0pt}exempt income}
%
%37.—(1) Subject to paragraph (2), where a departure direction is given in respect of special expenses, the exempt income of the absent parent or, as the case may be, the parent with care shall be increased by the amount constituting the special expenses or the aggregate of the special expenses determined in accordance with regulations 13 to 20.
%
%(2) Where a departure direction is given with respect to costs incurred in travelling to work or expenses which include such costs, and a component of exempt income has been determined in accordance with regulation 9(1)($i$) of the Maintenance Assessments and Special Cases Regulations or regulation 10 of those Regulations applying regulation 9(1)($i$), the increase in exempt income determined in accordance with paragraph (1) shall be reduced by that component of exempt income.
%
%(3) A departure direction with respect to special expenses for a case falling within regulation 16 shall be given only for the repayment period remaining applicable to that debt at the date on which that direction takes effect except—
%\begin{enumerate}\item[]
%($a$) where, in consequence of the applicant’s unemployment or incapacity for work, the repayment period of that debt has been extended by agreement with the creditor, a departure direction may be given to cover the additional weeks allowed for repayment; or
%
%($b$) where the Secretary of State is satisfied that, as a consequence of the income of the applicant having been substantially reduced the repayment period of that debt has been extended by agreement with the creditor, a departure direction may be given for such repayment period as the Secretary of State considers is reasonable.
%\end{enumerate}
%
%(4) Where paragraph (4) of regulation 16 applies, a departure direction may be given in respect only of repayment of that part of the debt incurred which is referrable to the debt, repayment of which would have fallen within paragraph (1) of that regulation, based upon the amount, rate of repayment and repayment period agreed in respect of that part at the time it was taken out.
%
%\subsection[38. Effect of a departure direction in respect of special expenses—protected income]{Effect of a departure direction in respect of special expenses—\hspace{0pt}protected income}
%
%38.—(1) Subject to paragraphs (2) and (3), where a departure direction is given with respect to special expenses in response to an absent parent’s application, his protected income shall be determined in accordance with regulation 11(1) of the Maintenance Assessments and Special Cases Regulations\footnote{\frenchspacing Sub-paragraphs ($a$) to ($k$) of paragraph (1) have been amended by regulation 4(4) of S.I. 1994/227, by regulation 46(2)($a$), ($b$) and ($c$) of S.I. 1995/1045, and by regulation 43(1), (2) and (3) of S.I. 1995/3261. Sub-paragraph ($kk$) was added to paragraph (1) of regulation 11 by regulation 46(2)($d$) of S.I. 1995/1045.} with the modification that the increase of exempt income as determined in accordance with regulation 37 shall be added to the aggregate of the amounts mentioned in sub-paragraphs ($a$) to ($kk$) of that regulation.
%
%(2) Protected income shall not be increased in accordance with paragraph (1) on account of special expenses constituted by costs falling within regulation 18 (costs incurred in supporting certain children).
%
%(3) Where a departure direction is given with respect to costs which include costs incurred in travelling to work, the absent parent’s protected income shall be determined in accordance with paragraph (1), but without inclusion of the amount determined in accordance with sub-paragraph ($kk$) of regulation 11(1) of the Maintenance Assessments and Special Cases Regulations within the aggregate of the amounts mentioned in that regulation.
%
%\subsection[39. Effect of a departure direction in respect of a transfer of property]{Effect of a departure direction in respect of a transfer of property}
%
%39.—(1) Where a departure direction is given in respect of a transfer of property for a case falling within paragraph 3 of Schedule 4B to the Act—
%\begin{enumerate}\item[]
%($a$) where the exempt income of an absent parent includes a component of exempt income determined in accordance with regulation 9(1)($bb$) of the Maintenance Assessments and Special Cases Regulations\footnote{\frenchspacing Sub-paragraph ($bb$) was added to paragraph (1) of regulation 9 by regulation 44(2)($a$) of 1995/1045.}, the exempt income of the absent parent shall be reduced by that component of exempt income;
%
%($b$) subject to sub-paragraph ($c$) and paragraphs (2) and (3), the fresh maintenance assessment made in consequence of the direction shall be the maintenance assessment calculated in accordance with the provisions of Part I of Schedule 1 to the Act, as modified by sub-paragraph ($a$) where that sub-paragraph is applicable to the case in question, reduced by the equivalent weekly value of the property transferred as determined in accordance with regulation 22;
%
%($c$) where the equivalent weekly value is nil, the fresh maintenance assessment made in consequence of the direction shall be the maintenance assessment calculated in accordance with the provisions of Part I of Schedule 1 to the Act, as modified by sub-paragraph ($a$), where that paragraph is applicable to the case in question.
%\end{enumerate}
%
%(2) The amount of child support maintenance fixed by an assessment made in consequence of a direction falling within paragraph (1) shall not be less than the amount prescribed by regulation 13 of the Maintenance Assessments and Special Cases Regulations.
%
%(3) Where there has been a transfer by the applicant of property to trustees as set out in regulation 21(2) and the equivalent weekly value is greater than nil, any monies paid to the parent with care out of that trust fund for maintenance of a child with respect to whom the current assessment was made shall be disregarded in calculating the assessable income of that parent with care in accordance with the provisions of Part I of Schedule 1 to the Act.
%
%(4) A departure direction falling within paragraph (1) shall cease to have effect at the end of the number of years of liability, as defined in paragraph 1 of the Schedule, for the case in question.
%
%(5) Where a departure direction has ceased to have effect under the provisions of paragraph (4), the exempt income of an absent parent shall be determined as if regulation 9(1)($bb$) of the Maintenance Assessments and Special Cases Regulations were omitted.
%
%(6) Where a departure direction is given in respect of a transfer of property for a case falling within paragraph 4 of Schedule 4B to the Act, the exempt income of the absent parent shall be reduced by the component of exempt income determined in accordance with regulation 9(1)($bb$) of the Maintenance Assessments and Special Cases Regulations.
%
%\subsection[40. Effect of a departure direction in respect of additional cases]{Effect of a departure direction in respect of additional cases}
%
%40.—(1) This regulation applies where a departure direction is given for an additional case falling within paragraph 5 of Schedule 4B to the Act.
%
%(2) In a case falling within paragraph (1)($a$) of regulation 23 (assets capable of producing income or higher income, subject to paragraph (4) the net income of the non-applicant shall be increased by an amount calculated by applying interest at the statutory rate prescribed for a judgment debt\footnote{\frenchspacing See Order 42, rule 1 of the Rules of the Supreme Court, S.I. 1965/1776.} at the date on which the departure direction is given to—
%\begin{enumerate}\item[]
%($a$) any monies falling within that paragraph;
%
%($b$) the net value of any asset, other than monies, falling within that paragraph, after deduction of the amount owing on any mortgage or charge on that asset,
%\end{enumerate}
%less any income received in respect of that asset which has been taken into account in the calculation of the current assessment.
%
%(3) In a case falling within paragraph (1)($b$) or ($c$) of regulation 23, subject to paragraph (4), the net income of the non-applicant shall be increased by an amount calculated by applying interest at the statutory rate prescribed for a judgment debt at the date of the application to the value of the asset subject to the trust less any income received from the trust which has been taken into account in the calculation of the current assessment.
%
%(4) In a case to which regulation 24 (diversion of income) applies, the net income of the non-applicant who is a parent of a child in respect of whom the current assessment is made shall be increased by the amount by which the Secretary of State is satisfied that parent has reduced his income.
%
%(5) In a case which regulation 25 (life-style inconsistent with declared income) applies, the net income of the non-applicant who is a parent of a child in respect of whom the current assessment is made shall be increased by the difference between the two levels of income referred to in paragraph (1) of that regulation.
%
%(6) In a case to which regulation 26 applies (unreasonably high housing costs) the amount of housing costs included in exempt income and the amount referred to in regulation 11(1)($b$) of the Maintenance Assessments and Special Cases Regulations shall not exceed the amounts set out in regulation 18(1)($a$) or ($b$), as the case may be, of the Maintenance Assessments and Special Cases Regulations (excessive housing costs) and the provisions of regulation 18(2) of those Regulations shall not apply.
%
%(7) In a case to which regulation 27 applies (partner’s contribution to housing costs) that part of the exempt income constituted by the eligible housing costs determined in accordance with regulation 14 of the Maintenance Assessments and Special Cases Regulations (eligible housing costs) shall, subject to paragraphs (8) and (9), be reduced by the percentage of the housing costs which the Secretary of State considers appropriate, taking into account the income of that parent and that partner.
%
%(8) Where paragraph (7) applies, the housing costs determined in accordance with regulation 11(1)($b$) of the Maintenance Assessments and Special Cases Regulations (protected income) shall remain unchanged.
%
%(9) Where a Category B interim maintenance assessment is in force in respect of a non-applicant, the whole of the eligible housing costs may be deducted from the exempt income of that non-applicant.
%
%(10) In a case to which regulation 28 (unreasonably high travel costs) or regulation 29 (travel costs to be disregarded) applies, for the component of exempt income determined in accordance with regulation 9(1)($i$) of the Maintenance Assessments and Special Cases Regulations or in accordance with that regulation as applied by regulation 10 of those Regulations and, in the case of an absent parent, for the amount determined in accordance with regulation 11(1)($kk$) of those Regulations, there shall be substituted such amount, including a nil amount, as the Secretary of State considers to be appropriate in all the circumstances of the case.
%
%\section[Part IX --- Maintenance assessment following a departure direction: particular cases]{Part IX\\*Maintenance assessment following a departure direction: particular cases}
%
%\subsection[41. Child support maintenance payable following a departure direction which results in a decrease in an absent parent’s assessable income]{Child support maintenance payable following a departure direction which results in a decrease in an absent parent’s assessable income}
%
%\renewcommand\parthead{--- Part IX}
%
%41.—(1) Subject to paragraph (6), where the effect of a departure direction is to reduce an absent parent’s assessable income, and his assessable income following the direction is such that the case falls within paragraph 2(3) of Schedule 1 to the Act (additional element of maintenance payable), the child support maintenance payable following that direction shall be determined by a child support officer in accordance with paragraphs (2) to (4).
%
%(2) The child support officer shall first calculate the amount equal to \(A\times P\), where A is the absent parent’s assessable income following the departure direction and P has the value prescribed in regulation 5 of the Maintenance Assessments and Special Cases Regulations.
%
%(3) Subject to paragraph (5), the amount of child support maintenance payable in consequence of the direction shall be the lower of—
%\begin{enumerate}\item[]
%($a$) the amount calculated in accordance with paragraph (2);
%
%($b$) the amount that would be payable under a maintenance assessment calculated by reference to the circumstances at the time the application is made, in accordance with the provisions of Part I of Schedule 1 to the Act.
%\end{enumerate}
%
%(4) Where the assessable income of an absent parent changes following a review under section 16, 17, 18 or 19 of the Act, the provisions of paragraph (3) shall be applied to—
%\begin{enumerate}\item[]
%($a$) the amount calculated under paragraph (2) which takes account of the change in assessable income; and
%
%($b$) the amount payable under the maintenance assessment calculated in accordance with the provisions of Part I of Schedule 1 to the Act which takes account of that change in assessable income.
%\end{enumerate}
%
%(5) Where the provisions of paragraph 6 of Schedule 1 to the Act (protected income) as modified by the provisions of regulation 38 apply following a departure direction, and the amount of child support maintenance payable under those provisions is lower than that payable under paragraph (3), the amount of child support maintenance payable shall be that payable under those provisions.
%
%(6) Where the effect of a departure direction (“the later direction”) is to change an absent parent’s assessable income and a departure direction which resulted in a change in that absent parent’s assessable income already has effect, the provisions of paragraphs (1) to (5) of this regulation shall, subject to the modification set out in paragraph (7), apply where—
%\begin{enumerate}\item[]
%($a$) the absent parent’s assessable income following the later direction is such that the case falls within paragraph 2(3) of Schedule 1 to the Act; and
%
%($b$) the assessable income following the later direction is less than the assessable income would be if it were calculated in accordance with the provisions of Part I of Schedule 1 to the Act by reference to the circumstances at the time the application for the later direction is made.
%\end{enumerate}
%
%(7) The modification referred to in paragraph (6) is that in paragraph (2), A is the absent parent’s assessable income following the later direction.
%
%\subsection[42. Maintenance assessment following a departure direction for certain cases falling within regulation 22 of the Maintenance Assessments and Special Cases Regulations]{Maintenance assessment following a departure direction for certain cases falling within regulation 22 of the Maintenance Assessments and Special Cases Regulations}
%
%42.—(1) Where the provisions of regulation 41 are applicable to a case falling within regulation 22 of the Maintenance Assessments and Special Cases Regulations\footnote{\frenchspacing Regulation 22 was amended by regulation 23 of S.I. 1993/913, regulation 51 of S.I. 1995/1045 and regulation 45 of S.I. 1995/3261.} (multiple applications relating to an absent parent), provisions of shall apply for the purposes of determining the total maintenance payable in consequence of a departure direction.
%
%(2) In a case falling within paragraph (1), the amount of child support maintenance payable in respect of each application for child support maintenance following the direction shall be the total maintenance payable in consequence of the direction divided in the ratio that the maintenance assessments calculated in accordance with the provisions of Part I of Schedule 1 to the Act in respect of each application bear to each other.
%
%(3) Where, in a case falling within regulation 22 of the Maintenance Assessments and Special Cases Regulations, a departure direction has been given in respect of an absent parent in a case falling within paragraph 3 of Schedule 4B to the Act (property or capital transfers), the equivalent weekly value of the transfer of property as calculated in accordance with regulation 22 shall be deducted from the amount of any maintenance assessment calculated in accordance with the provisions of Part I of Schedule 1 to the Act in respect of the person with care or child to or in respect of whom the property transfer was made.
%
%\subsection[43. Maintenance assessment following a departure direction where there is a phased maintenance assessment]{Maintenance assessment following a departure direction where there is a phased maintenance assessment}
%
%43.—(1) Where a departure direction is given in a case falling within a relevant enactment, the assessment made in consequence of that direction shall be the assessment that fixes the amount of child support maintenance that would be payable but for the provisions of that enactment (“the unadjusted departure amount”).
%
%(2) Where a departure direction takes effect on the effective date of a maintenance assessment to which the provisions of a relevant enactment become applicable, those provisions shall remain applicable to that case following the departure direction.
%
%(3) Where a departure direction takes effect on a date later than the date on which the provisions of a relevant enactment become applicable to a maintenance assessment, the amount of child support maintenance payable in consequence of that direction shall be—
%\begin{enumerate}\item[]
%($a$) where the unadjusted departure amount is more than the formula amount, the phased amount plus the difference between the unadjusted departure amount and the formula amount;
%
%($b$) where the unadjusted departure amount is more than the phased amount but less than the formula amount, the phased amount;
%
%($c$) where the unadjusted departure amount is less than the phased amount, the unadjusted departure amount.
%\end{enumerate}
%
%(4) Regulation 31 shall have effect for cases falling within this regulation as if “current amount” referred to the amount payable under the maintenance assessment that would be in force when the departure direction is given but for the provisions of the relevant enactment and “revised amount” referred to the unadjusted departure amount.
%
%(5) In this regulation—
%\begin{enumerate}\item[]
%“the 1992 enactment” means Part II of the Schedule to the Child Support Act 1991 (Commencement No.\ 3 and Transitional Provisions) Order 1992\footnote{\frenchspacing S.I. 1992/2644. The relevant amending instrument is S.I. 1993/966.} (modification of maintenance assessment in certain cases);
%
%“the 1994 enactment” means Part III of the Child Support (Miscellaneous Amendments and Transitional Provisions) Regulations 1994\footnote{\frenchspacing S.I. 1994/227. The relevant amending instrument is S.I. 1995/1045.} (transitional provisions);
%
%“formula amount” has the same meaning as in the relevant enactment;
%
%“phased amount” means—
%\begin{enumerate}\item[]
%($a$) where the 1992 enactment is applicable to the particular case, the modified amount as defined in paragraph 6 of that enactment;
%
%($b$) where the 1994 enactment is applicable to the particular case, the transitional amount as defined in regulation 6(1) of that enactment;
%\end{enumerate}
%
%“relevant enactment” means—
%\begin{enumerate}\item[]
%($a$) the 1992 enactment where that enactment is applicable to the particular case;
%
%($b$) the 1994 enactment where that enactment is applicable to the particular case.
%\end{enumerate}
%\end{enumerate}
%
%\bigskip
%
%Signed by authority of the Secretary of State for Social Security.
%
%{\raggedleft
%\emph{A. J. B. Mitchell}\\*Parliamentary Under-Secretary of State,\\*Department of Social Security
%
%}
%
%7th March 1996
%
%\clearpage
%
%\part[Schedule --- Equivalent weekly value of a transfer of property]{S C H E D U L E\\*Equivalent weekly value of a transfer of property}
%
%\renewcommand\parthead{--- Schedule}
%
%1.—(1) Subject to paragraphs 3 and 4, the equivalent weekly value of a transfer of property shall be calculated by multiplying the value of a transfer of property determined in accordance with regulation 22(1) by the relevant factor specified in the Table set out in paragraph 2 (“the Table”).
%
%(2) For the purposes of sub-paragraph (1), the relevant factor is the number in the Table at the intersection of the column for the relevant statutory rate prescribed for a judgment debt and of the row for the number of years of liability.
%
%(3) In sub-paragraph (2)—
%\begin{enumerate}\item[]
%($a$) “the relevant statutory rate prescribed for a judgment debt” means the interest at the statutory rate prescribed for a judgment debt\footnote{\frenchspacing See Order 42, rule 1 of the Rules of the Supreme Court, S.I. 1965/1776.} applying at the date of the court order or written agreement relating to the transfer of the property;
%
%($b$) “the number of years of liability” means the number of years, beginning on the date of the court order or written agreement relating to the transfer of the property and ending on—
%\begin{enumerate}\item[]
%(i) the date specified in that order or agreement as the date on which maintenance for the youngest child in respect of whom that order or agreement was made shall cease, or the occurrence of the event specified in that order or agreement as the event on the occurrence of which such maintenance shall cease; or
%
%(ii) if no such date or event is specified, the date on which the youngest child specified in the order or agreement reaches the age of 18,
%\end{enumerate}
%and where that period includes a fraction of a year, that fraction shall be treated as a full year if it is either one half or exceeds one half of a year, and shall otherwise be disregarded.
%\end{enumerate}
%
%\medskip
%
%2.  The Table referred to in paragraph 1(1) is set out below—
%
%\noindent
%\begin{longtable}{p{96.00522pt} lll lll}
%\hline
%Number of years of liability&\multicolumn{6}{l}{Statutory rate prescribed for a judgment debt}\\
%& 8.0\%&10.0\%&12.0\%&12.5\%&14.0\%&15.0\%\\
%\hline
%\endhead
%\hline
%\endlastfoot
%1&.02077&.02115&.02154&.02163&.02192&.02212\\
%2&.01078&.01108&.01138&.01145&.01168&.01183\\
%3&.00746&.00773&.00801&.00808&.00828&.00842\\
%4&.00581&.00607&.00633&.00640&.00660&.00674\\
%5&.00482&.00507&.00533&.00540&.00560&.00574\\
%6&.00416&.00442&.00468&.00474&.00495&.00508\\
%7&.00369&.00395&.00421&.00428&.00448&.00462\\
%8&.00335&.00360&.00387&.00394&.00415&.00429\\
%9&.00308&.00334&.00361&.00368&.00389&.00403\\
%10&.00287&.00313&.00340&.00347&.00369&.00383\\
%11&.00269&.00296&.00324&.00331&.00353&.00367\\
%12&.00255&.00282&.00310&.00318&.00340&.00355\\
%13&.00243&.00271&.00299&.00307&.00329&.00344\\
%14&.00233&.00261&.00290&.00298&.00320&.00336\\
%15&.00225&.00253&.00282&.00290&.00313&.00329\\
%16&.00217&.00246&.00276&.00283&.00307&.00323\\
%17&.00211&.00240&.00270&.00278&.00302&.00318\\
%18&.00205&.00234&.00265&.00273&.00297&.00314\\ 
%\end{longtable}
%
%\medskip
%
%3. The equivalent weekly value of the property transferred shall be nil if the value of the transfer of the property is less than £5,000.
%
%\medskip
%
%4.  The Secretary of State may determine a lower equivalent weekly value than that determined in accordance with paragraphs 1 and 2 where the amount of child support maintenance that would be payable in consequence of a departure direction specifying that value is lower than the amount of maintenance that was payable under the court order or written agreement referred to in regulation 21.
%
%\medskip
%
%5.  In this Schedule, “maintenance” has the same meaning as in paragraph 3(2) of Schedule 4B to the Act.
%
%\part{Explanatory Note}
%
%\renewcommand\parthead{--- Explanatory Note}
%
%\subsection*{(This note is not part of the Regulations)}
%
%These Regulations provide for applications for departure directions to be made even though section 28A of the Child Support Act 1991 (“the 1991 Act”) is not in force, and for such applications to be determined as if sections 28A to 28H and section 28I(1) to (3) of, and Schedules 4A and 4B to, the 1991 Act were in force. Sections 28A to 28I and Schedules 4A and 4B were inserted into the 1991 Act by the Child Support Act 1995.
%
%  Regulations 1 to 3 contain interpretation provisions, provisions as to applications for departure directions and their determination and rounding provisions.
%
%  Regulations 4 to 12 contain provisions relating to the manner in which an application is to be made and to the Secretary of State’s preliminary consideration of an application.
%
%  Regulations 13 to 29 and the Schedule make provision in relation to cases in which a departure direction may be given: regulations 13 to 20 relate to special expenses, regulations 21 and 22 and the Schedule to property or capital transfers and regulations 23 to 29 to additional cases where a departure direction may be given.
%
%  The Schedule contains a table for calculating the equivalent weekly value of a transfer of property. The factors in the table are derived from the standard formula used in annuity calculations.
%
%  Regulation 30 prescribes factors to be taken into account and not to be taken into account in determining whether it would be just and equitable to give a departure direction.
%
%  Regulations 31 to 35 contain provisions as to the effective date and the duration of a departure direction.
%
%  Regulations 36 to 44 contain provisions as to the maintenance assessment which is to be made in consequence of a departure direction.
%
%  These Regulations impose no costs on business.
%

\end{document}