\documentclass[12pt,a4paper]{article}

\newcommand\regstitle{The Child Support (Written Agreements) (Scotland) Order 1997}

\newcommand\regsnumber{1997/2943}

%\opt{newrules}{
\title{\regstitle}
%}

%\opt{2012rules}{
%\title{Child Maintenance and Other Payments Act 2008\\(2012 scheme version)}
%}

\author{S.I. 1997 No. 2943 (S.188)}

\date{Made 3rd December 1997\\Laid before Parliament 12th December 1997\\Coming into force 2nd January 1998}

%\opt{oldrules}{\newcommand\versionyear{1993}}
%\opt{newrules}{\newcommand\versionyear{2003}}
%\opt{2012rules}{\newcommand\versionyear{2012}}

\usepackage{csa-regs}

%\setlength\headheight{42.07402pt}

\begin{document}

\maketitle

\noindent
The Lord Advocate, in exercise of the powers conferred on him by section 8(5) of the Child Support Act 1991\footnote{\frenchspacing 1991 c. 48.} and of all other powers enabling him in that behalf, hereby makes the following Order:

%{\sloppy
%
%\tableofcontents
%
%}

\bigskip

\setcounter{secnumdepth}{-2}

1.  This Order may be cited as the Child Support (Written Agreements) (Scotland) Order 1997 and shall come into force on 2nd January 1998.

\medskip

2.  Section 8 of the Child Support Act 1991 shall not prevent a court from exercising any power which it has to make a maintenance order in relation to a child in any circumstances in which paragraphs (a) and (b) of section 8(5) apply.

%\bigskip
%
%Signed 
%by authority of the Secretary of State for Social Security.
%
%{\raggedleft
%\emph{Roger Evans}\\*Parliamentary Under-Secretary of State,\\*Department of Social Security
%
%}
%
%18th March 1997

\bigskip

%Signed in connection with regulations 2 and 3 of these Regulations by authority of the Secretary of State for Education and Employment.

{\raggedleft
\emph{Hardie}\\*Lord Advocate%,\\*Department for Education and Employment

}

Lord Advocate’s Chambers Edinburgh

3rd December 1997

\small

\part{Explanatory Note}

\renewcommand\parthead{--- Explanatory Note}

\subsection*{(This note is not part of the Order)}

This Order specifies the circumstances in which a Scottish court will continue to have jurisdiction to make a maintenance order in relation to a child even though a child support officer has jurisdiction to make a maintenance assessment for that child under the Child Support Act 1991. It enables persons who have entered into a maintenance agreement (whether or not registered for execution in the books of Council and Session) to which a court has not interponed authority to have a maintenance order made by the court in the same terms as their agreement. Such an order if made under the Family Law (Scotland) Act 1985 or any other prescribed enactment can be taken into account in the calculation of “income” under the Maintenance Assessment and Special Cases Regulations 1992 (S.I.\ No 1815).

\end{document}