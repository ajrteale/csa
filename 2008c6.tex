\documentclass[12pt,a4paper]{article}

\usepackage[2012rules]{optional}

\newcommand\regstitle{Child Maintenance and Other Payments Act 2008}

\newcommand\regsnumber{c.~6}

\opt{newrules}{
\title{\regstitle\\(2003 scheme version)}
}

\opt{2012rules}{
\title{Child Maintenance and Other Payments Act 2008%
%\\(2012 scheme version)
}
}

\author{2008 Chapter 6}

\date{Royal Assent
5th June 2008\\
%Laid before Parliament
%27th January 2000\\
%Coming into force
%19th June 2000
}

%\opt{oldrules}{\newcommand\versionyear{1993}}
%\opt{newrules}{\newcommand\versionyear{2003}}
%\opt{2012rules}{\newcommand\versionyear{2012}}

\usepackage{csa-regs}

\setlength\headheight{27.57402pt}

%\hbadness=10000

\renewcommand\siprefix{\relax}

\begin{document}

\maketitle

\noindent
{\large An Act to establish the Child Maintenance and Enforcement Commission; to amend the law relating to child support; to make provision about lump sum payments to or in respect of persons with diffuse mesothelioma; and for connected purposes.}

\bigskip

\lettrine{B}{e it enacted} by the Queen’s most Excellent Majesty, by and with the advice and consent of the Lords Spiritual and Temporal, and Commons, in this present Parliament assembled, and by the authority of the same, as follows:—


{\sloppy

\tableofcontents

}

\setcounter{secnumdepth}{-2}

\part[Part I --- The Child Maintenance and Enforcement Commission]{Part I\\*The Child Maintenance and Enforcement Commission}

\renewcommand\parthead{--- Part I}

\amendment{
Ss. 1--5 omitted (1.8.12) by the Public Bodies (Child Maintenance and Enforcement Commission: Abolition and Transfer of Functions) Order 2012 Sch. para.~71.
}

%\subsection{1. The Child Maintenance and Enforcement Commission}
%
%(1) There shall be a body corporate to be known as the Child Maintenance and Enforcement Commission (referred to in this Act as “the Commission”).
%
%(2) Schedule 1 (which makes further provision about the Commission) has effect.
%
%\subsection{2. Objectives of the Commission}
%
%(1) The Commission's main objective is to maximise the number of those children who live apart from one or both of their parents for whom effective maintenance arrangements are in place.
%
%(2) The Commission's main objective is supported by the following subsidiary objectives—
%\begin{enumerate}\item[]
%($a$) to encourage and support the making and keeping by parents of appropriate voluntary maintenance arrangements for their children;
%
%($b$) to support the making of applications for child support maintenance under the Child Support Act 1991 and to secure compliance when appropriate with parental obligations under that Act.
%\end{enumerate}
%
%(3) The Commission shall aim to pursue, and to have regard to, its objectives when exercising a function that is relevant to them.
%
%\subsection{3. Functions of the Commission: general}
%
%(1) The Commission has—
%\begin{enumerate}\item[]
%($a$) the functions relating to child support transferred to it from the Secretary of State by virtue of this Act, and
%
%($b$) such other functions as are conferred by, or by virtue of, this or any other enactment.
%\end{enumerate}
%
%(2) The Secretary of State may by regulations provide for the Commission to have an additional function if it appears to the Secretary of State that it is necessary or expedient for the Commission to have the function in relation to any of its objectives.
%
%(3) The Commission must exercise its functions effectively and efficiently.
%
%\subsection{4. Promotion of child maintenance}
%
%The Commission must take such steps as it thinks appropriate for the purpose of raising awareness among parents of the importance of—
%\begin{enumerate}\item[]
%($a$) taking responsibility for the maintenance of their children, and
%
%($b$) making appropriate arrangements for the maintenance of children of theirs who live apart from them.
%\end{enumerate}
%
%\subsection{5. Provision of information and guidance}
%
%(1) The Commission must provide to parents such information and guidance as it thinks appropriate for the purpose of helping to secure the existence of effective maintenance arrangements for children who live apart from one or both of their parents.
%
%(2) The Commission may provide information for other purposes in the course of exercising its function under subsection~(1).

\subsection{6. Fees}

(1) The Secretary of State may by regulations make provision about the charging of fees by the Secretary of State in connection with the exercise of its functions.

(2) Regulations under subsection~(1) may, in particular, make provision—
\begin{enumerate}\item[]
($a$) about when a fee may be charged;

($b$) about the amount which may be charged;

($c$) for the supply of information needed for the purpose of determining the amount which may be charged;

($d$) about who is liable to pay any fee charged
(including provision for the apportionment of fees and the matters to be taken into account in determining an apportionment)%  % Words inserted by 2012 c 5 s 140(a)
;

($e$) about when any fee charged is payable;

($f$) about the recovery of fees charged;

($g$) about 
%waiver,  % Word repealed by 2012 c 5 s 140(b)
reduction or repayment of fees;

% S 6(2)(h) inserted by 2012 c 5 s 140(c)
($h$) about waiver of fees (including the matters to be taken into account in determining a waiver).
\end{enumerate}

(3) The power conferred by subsection~(1) includes power to make provision for the charging of fees which are not related to costs.

% S 6(3A)--(3D) inserted by 2012 c 5 s 141
(3A) The Secretary of State must review the effect of the first regulations made under subsection (1).

(3B) The review must take place before the end of the period of 30 months beginning with the day on which those regulations come into force.

(3C) After the review, the Secretary of State must make and publish a report containing—
\begin{enumerate}\item[]
($a$) the conclusions of the review, and

($b$) a statement as to what the Secretary of State proposes to do in view of those conclusions.
\end{enumerate}

(3D) The report must be laid before Parliament by the Secretary of State.

(4) The Secretary of State may by regulations provide that the provisions of the Child Support Act 1991 with respect to—
\begin{enumerate}\item[]
($a$) the collection of child support maintenance,

($b$) the enforcement of any obligation to pay child support maintenance,
\end{enumerate}
shall apply equally (with any necessary modifications) to fees payable by virtue of regulations under subsection~(1).

(5) The Secretary of State may by regulations make provision for a person affected by a decision of the Secretary of State under regulations under subsection~(1) to have a right of appeal against the decision to the First-tier Tribunal.

(6) Subsections (3) to (5), (7) and~(8) of section 20 of the Child Support Act 1991 (appeals to First-tier Tribunal) apply to appeals under regulations under subsection~(5) as they apply to appeals under that section.

(7) The Secretary of State shall pay into the Consolidated Fund any amount which the Secretary of State receives in respect of fees charged by the Secretary of State under regulations under this section.

\amendment{
Words substituted in s. 6(5), (6) (3.11.08) by the Transfer of Tribunal Functions Order 2008 Sch. 3 para.~225.

Words substituted in s. 6(1), (5), (7) (1.8.12) by the Public Bodies (Child Maintenance and Enforcement Commission: Abolition and Transfer of Functions) Order 2012 Sch. para.~72.

Words inserted in s. 6(2)(d), word in s. 6(2)(g) repealed and s. 6(2)(h) inserted (25.11.13) by the Welfare Reform Act 2012 s. 140.

S. 6(3A)--(3D) inserted (25.11.13) by the Welfare Reform Act 2012 s. 141.

\medskip

S. 7 omitted (1.8.12) by the Public Bodies (Child Maintenance and Enforcement Commission: Abolition and Transfer of Functions) Order 2012 Sch. para.~73.
}

%\subsection{7. Agency arrangements and provision of services}
%
%(1) Arrangements may be made between the Commission and any relevant authority for—
%\begin{enumerate}\item[]
%($a$) any functions of one of them to be exercised on their behalf by, or by members of staff of, the other;
%
%($b$) the provision of administrative, professional or technical services by one of them for the other.
%\end{enumerate}
%
%(2) The reference in subsection~(1)($a$) to functions does not include functions of making, confirming or approving subordinate legislation.
%
%(3) The Commission may make arrangements under this section on such terms and conditions as it thinks fit.
%
%(4) In this section “relevant authority” means—
%\begin{enumerate}\item[]
%($a$) any Minister of the Crown or department of the Government of the United Kingdom;
%
%($b$) a public body specified in regulations made by the Secretary of State for the purposes of this section.
%\end{enumerate}

\subsection{8. Contracting out}

(1) Any function of the Secretary of State relating to child support may be exercised by, or by employees of, such person (if any) as the Secretary of State may authorise for the purpose.

(2) An authorisation given by virtue of subsection~(1) may authorise the exercise of the function concerned—
\begin{enumerate}\item[]
($a$) either wholly or to such extent as may be specified in the authorisation,

($b$) either generally or in such cases or areas as may be so specified, and

($c$) either unconditionally or subject to the fulfilment of such conditions as may be so specified.
\end{enumerate}

(3) An authorisation given by virtue of subsection~(1)—
\begin{enumerate}\item[]
($a$) may specify its duration,

($b$) may be revoked at any time by the Secretary of State, and

($c$) shall not prevent the Secretary of State or any other person from exercising the function to which the authorisation relates.
\end{enumerate}

(4) Where a person is authorised to exercise any function by virtue of subsection~(1), anything done or omitted to be done by or in relation to that person (or an employee of that person) in, or in connection with, the exercise or purported exercise of the function shall be treated for all purposes as done or omitted to be done by or in relation to the Secretary of State.

(5) Subsection (4) shall not apply—
\begin{enumerate}\item[]
($a$) for the purposes of so much of any contract made between the authorised person and the Secretary of State as relates to the exercise of the function, or

($b$) for the purposes of any criminal proceedings brought in respect of anything done or omitted to be done by the authorised person (or an employee of that person).
\end{enumerate}

(6) Where—
\begin{enumerate}\item[]
($a$) a person is authorised to exercise any function by virtue of subsection~(1), and

($b$) the authorisation is revoked at a time when a relevant contract is subsisting,
\end{enumerate}
the authorised person shall be entitled to treat the relevant contract as repudiated by the Secretary of State (and not as frustrated by reason of the revocation).

(7) In subsection~(6), the reference to a relevant contract is to so much of any contract made between the authorised person and the Secretary of State as relates to the exercise of the function.

\amendment{
Words substituted in s. 8 (1.8.12) by the Public Bodies (Child Maintenance and Enforcement Commission: Abolition and Transfer of Functions) Order 2012 Sch. para.~74.

\medskip

Ss. 9--12 omitted (1.8.12) by the Public Bodies (Child Maintenance and Enforcement Commission: Abolition and Transfer of Functions) Order 2012 Sch. para.~75.

\medskip

Pt. II (ss. 13, 14) omitted (1.8.12) by the Public Bodies (Child Maintenance and Enforcement Commission: Abolition and Transfer of Functions) Order 2012 Sch. para.~76.
}

%\subsection{9. Annual report to Secretary of State}
%
%(1) The Commission must prepare a report for each financial year.
%
%(2) Each report under this section must—
%\begin{enumerate}\item[]
%($a$) deal with the activities of the Commission in the financial year for which it is prepared, including the matters mentioned in subsection~(3),
%
%($b$) include the report prepared under paragraph 20(5) of Schedule 1 by the committee established under that paragraph.
%\end{enumerate}
%
%(3) The matters referred to in subsection~(2)($a$) are—
%\begin{enumerate}\item[]
%($a$) the strategic direction of the Commission and the manner in which it has been kept under review;
%
%($b$) the Commission's objectives and targets, the steps taken to meet them and the extent to which they have been met;
%
%($c$) the steps taken to monitor the performance of the Commission in ensuring that its functions are exercised effectively and efficiently;
%
%($d$) the extent to which the Commission has relied on sections 7(1) and 8(1).
%\end{enumerate}
%
%(4) The Commission must—
%\begin{enumerate}\item[]
%($a$) send each report to the Secretary of State as soon as practicable after the end of the financial year for which it is prepared, and
%
%($b$) publish the report in such manner as the Commission considers appropriate.
%\end{enumerate}
%
%(5) The Secretary of State must lay before Parliament a copy of every report received under this section.
%
%(6) In this section, “financial year” means—
%\begin{enumerate}\item[]
%($a$) the period beginning with the date on which the Commission is established and ending with the next following 31st March, and
%
%($b$) each successive period of 12 months.
%\end{enumerate}
%
%\subsection{10. Directions and guidance}
%
%(1) The Secretary of State may give the Commission—
%\begin{enumerate}\item[]
%($a$) guidance as to the exercise of its functions;
%
%($b$) general or specific directions as to the exercise of its functions.
%\end{enumerate}
%
%(2) In exercising its functions, the Commission must—
%\begin{enumerate}\item[]
%($a$) have regard to any guidance under subsection~(1)($a$), and
%
%($b$) comply with any directions under subsection~(1)($b$).
%\end{enumerate}
%
%(3) Guidance or directions under this section must be in writing.
%
%(4) Power under this section to give guidance or directions includes power to vary or revoke guidance or directions given in previous exercise of the power.
%
%(5) The Secretary of State must lay before Parliament a copy of any direction given under subsection~(1)($b$).
%
%(6) The Secretary of State may exclude from what is laid before Parliament—
%\begin{enumerate}\item[]
%($a$) any information which the Secretary of State considers to be against the commercial interests of any person;
%
%($b$) any information which relates to an individual who can be identified from that information.
%\end{enumerate}
%
%\subsection{11. Review of the status of the Commission}
%
%(1) The Secretary of State must review the status of the Commission as a Crown body.
%
%(2) The review under subsection~(1) must be conducted as soon as reasonably practicable after the end of the initial period.
%
%(3) The Secretary of State may review the status of the Commission as a Crown body at any other time after the end of the initial period, if the Secretary of State considers it appropriate to do so.
%
%(4) The Secretary of State must prepare a report of any review under subsection~(1) or~(3).
%
%(5) The Secretary of State must lay before Parliament a copy of the report.
%
%(6) If, on a review under this section, it appears to the Secretary of State appropriate to do so, the Secretary of State may by order made by statutory instrument provide that the Commission is to cease to be a Crown body.
%
%(7) An order under subsection~(6) may—
%\begin{enumerate}\item[]
%($a$) make any amendment to Schedule 1 that appears to the Secretary of State to be necessary or expedient in consequence of the Commission ceasing to be a Crown body;
%
%($b$) provide for the Transfer of Undertakings (Protection of Employment) Regulations 2006 to apply, subject to such modifications and exceptions as may be prescribed, as if, on the Commission ceasing to be a Crown body, there were a transfer of an undertaking or business which is a relevant transfer.
%\end{enumerate}
%
%(8) In this section—
%\begin{enumerate}\item[]
%“Crown body” means a body whose functions are to be exercised on behalf of the Crown;
%
%“initial period” means the period of 3 years beginning with the day on which section 13 comes into force.
%\end{enumerate}
%
%\subsection{12. Supplementary provisions}
%
%(1) In this Part, “child” has the same meaning as in the Child Support Act 1991.
%
%(2) The Secretary of State may by regulations make provision about when a child is, or is not, to be regarded for the purposes of this Part as living apart from a parent.

%\part[Part II --- Transfer of child support functions etc to the Commission]{Part II\\*Transfer of child support functions etc to the Commission}
%
%\subsection{13. Transfer of child support functions}
%
%(1) Any function under the Child Support Act 1991 which—
%\begin{enumerate}\item[]
%($a$) is a function of the Secretary of State, and
%
%($b$) is not an excepted function,
%\end{enumerate}
%is by virtue of this subsection transferred to the Commission.
%
%(2) The following functions of the Secretary of State under the Child Support Act 1991 are excepted functions for the purposes of subsection~(1)—
%\begin{enumerate}\item[]
%($a$) functions under sections 23A, 24 or 25 (appeals),
%
%($b$) functions under section~46 (reduced benefit decisions) or any other provision of the Act, so far as relating to such decisions,
%
%($c$) the function under section 50(7)($c$) (authorisation of a person as a “responsible person” for the purposes of section 50),
%
%($d$) functions under section 58 (commencement power and power to make consequential amendments),
%
%($e$) the function under paragraph 2A of Schedule 4 (payment of expenses), and
%
%($f$) power to make regulations under any other provision of the Act.
%\end{enumerate}
%
%(3) The functions of the Secretary of State under the provisions of subordinate legislation specified in Schedule 2, except so far as relating to reduced benefit decisions under section~46 of the Child Support Act 1991, are by virtue of this subsection transferred to the Commission.
%
%(4) Schedule 3 (which makes consequential amendments and transitional provision and savings) has effect.
%
%\subsection{14. Transfer of property, rights and liabilities}
%
%(1) The Secretary of State may make one or more schemes for the transfer to the Commission of any of the following—
%\begin{enumerate}\item[]
%($a$) property, rights and liabilities which the Secretary of State is entitled or subject to in connection with the transferred functions;
%
%($b$) property, rights and liabilities which the Secretary of State is entitled or subject to and which the Secretary of State considers it appropriate to transfer to the Commission in consequence of any function conferred on it by or under Part I of this Act.
%\end{enumerate}
%
%(2) A scheme under subsection~(1) (“a transfer scheme”)—
%\begin{enumerate}\item[]
%($a$) may provide for the transfer of property, rights and liabilities whether or not they would otherwise be capable of being transferred or assigned;
%
%($b$) may create for the Secretary of State interests in or rights over property transferred by virtue of the scheme;
%
%($c$) may create for the Commission interests in or rights over property retained by the Secretary of State;
%
%($d$) may create rights or liabilities between the Secretary of State and the Commission;
%
%($e$) may make such supplementary, incidental, consequential or transitional provision or savings as the Secretary of State considers appropriate.
%\end{enumerate}
%
%(3) A transfer scheme shall come into force in accordance with its terms.
%
%(4) A certificate given by the Secretary of State that any property, rights or liabilities have been transferred by virtue of a transfer scheme is conclusive evidence of the transfer.
%
%(5) In this section, “transferred functions” means functions transferred to the Commission by virtue of section 13.

\part[Part III --- Child support etc]{Part III\\*Child support etc}

\renewcommand\parthead{--- Part III}

\section{\itshape Removal of compulsion for benefit claimants}

\subsection{15. Repeal of sections 6 and 46}

The following provisions of the Child Support Act 1991  cease to have effect—
\begin{enumerate}\item[]
($a$) section 6 (under which the claim of benefit by or in respect of a parent with care, or the payment of benefit to or in respect of such a person, triggers an application by her or him for child support maintenance), and

($b$) section~46 (which enables the Secretary of State in certain circumstances to reduce the benefit of a person in relation to whom section 6 triggers the making of an application for child support maintenance).
\end{enumerate}

\section{\itshape Maintenance calculations}

\subsection[16. Changes to the calculation of maintenance]{16. Changes to the calculation of maintenance\\*\emph{2012 scheme only}}

Schedule 4 (which makes various changes to the provisions about the calculation of maintenance) has effect.

\amendment{
S. 16 is only in force for 2012 scheme cases --- see S.I. 2012/3042.
}

\subsection[17. Power to regulate supersession]{17. Power to regulate supersession\\*\emph{2012 scheme only}}

In section 17 of the Child Support Act 1991 (decisions superseding earlier decisions), 
for subsections (2) and~(3) substitute—
\begin{quotation}
“(2) The Secretary of State may by regulations make provision with respect to the exercise of the power under subsection~(1).

(3) Regulations under subsection~(2) may, in particular—
\begin{enumerate}\item[]
($a$) make provision about the cases and circumstances in which the power under subsection~(1) is exercisable, including provision restricting the exercise of that power by virtue of change of circumstance;

($b$) make provision with respect to the consideration by the Secretary of State, when acting under subsection~(1), of any issue which has not led to the Secretary of State's so acting;

($c$) make provision with respect to procedure in relation to the exercise of the power under subsection~(1).”
\end{enumerate}
\end{quotation}

\amendment{
S. 17 is only in force for 2012 scheme cases --- see S.I. 2012/3042.

Words substituted in s. 17 (1.8.12) by the Public Bodies (Child Maintenance and Enforcement Commission: Abolition and Transfer of Functions) Order 2012 Sch. para.~77.

}

\subsection[18. Determination of applications for a variation]{18. Determination of applications for a variation\\*\emph{2012 scheme only}}

(1) Section 28D of the Child Support Act 1991 is amended as follows.

(2) After subsection~(2) insert—
\begin{quotation}
“(2A) Subsection (2B) applies if—
\begin{enumerate}\item[]
($a$) the application for a variation is made by the person with care or~(in the case of an application for a maintenance calculation under section 7) the person with care or the child concerned, and

($b$) it appears to the Secretary of State that consideration of further information or evidence may affect the decision under subsection~(1)($a$) whether or not to agree to a variation.
\end{enumerate}

(2B) Before making the decision under subsection~(1)($a$) the Secretary of State must—
\begin{enumerate}\item[]
($a$) consider any such further information or evidence that is available to the Secretary of State, and

($b$) where necessary, take such steps as the Secretary of State considers appropriate to obtain any such further information or evidence.”
\end{enumerate}
\end{quotation}

(3) In subsection~(3), after “duties” insert “, apart from the duty under subsection~(2B)”.

\amendment{
S. 18 is only in force for 2012 scheme cases --- see S.I. 2012/3042.

Words substituted in s. 18(2) (1.8.12) by the Public Bodies (Child Maintenance and Enforcement Commission: Abolition and Transfer of Functions) Order 2012 Sch. para.~78.
}

\subsection{19. Transfer of cases to new rules}

Schedule 5 (which makes provision for, and in connection with, enabling the Commission to require existing cases to transfer to the new maintenance calculation rules or to leave the statutory scheme, so far as future accrual of liability is concerned) has effect. 

\section{\itshape Collection and enforcement}

\subsection{20. Use of deduction from earnings orders as basic method of payment}

In section 29 of the Child Support Act 1991 (under which payments of child support maintenance are to be made in accordance with regulations) at the end insert—
\begin{quotation}
“(4) If the regulations include provision for payment by means of deduction in accordance with an order under section~31, they must make provision—
\begin{enumerate}\item[]
($a$) for that method of payment not to be used in any case where there is good reason not to use it; and

($b$) for the person against whom the order under section~31 would be made to have a right of appeal to a magistrates' court (or, in Scotland, to the sheriff) against a decision that the exclusion required by paragraph ($a$) does not apply.
\end{enumerate}

\begin{sloppypar}
(5) On an appeal under regulations made under subsection~(4)($b$) the court or~(as the case may be) the sheriff shall not question the maintenance calculation by reference to which the order under section~31 would be made.
\end{sloppypar}

(6) Regulations under subsection~(4)($b$) may include—
\begin{enumerate}\item[]
($a$) provision with respect to the period within which a right of appeal under the regulations may be exercised;

($b$) provision with respect to the powers of a magistrates' court (or, in Scotland, of the sheriff) in relation to an appeal under the regulations.
\end{enumerate}

(7) If the regulations include provision for payment by means of deduction in accordance with an order under section~31, they may make provision—
\begin{enumerate}\item[]
($a$) prescribing matters which are, or are not, to be taken into account in determining whether there is good reason not to use that method of payment;

($b$) prescribing circumstances in which good reason not to use that method of payment is, or is not, to be regarded as existing.”
\end{enumerate}
\end{quotation}

\amendment{S. 21 is not yet in force.}

\subsection{22. Orders for regular deductions from accounts}

After section~32 of the Child Support Act 1991 insert---
\begin{quotation}
\subsection*{``32A. Orders for regular deductions from accounts}

(1) If in relation to any person it appears to the Commission---
\begin{enumerate}\item[]
($a$) that the person has failed to pay an amount of child support maintenance;
and

($b$) that the person holds an account with a deposit-taker;
\end{enumerate}
it may make an order against that person to secure the payment of any amount due under the maintenance calculation in question by means of regular deductions from the account.

(2) An order under this section may be made so as to secure the payment of---
\begin{enumerate}\item[]
($a$) arrears of child support maintenance payable under the calculation;

($b$) amounts of child support maintenance which will become payable under the
calculation; or

($c$) both such arrears and such future amounts.
\end{enumerate}

\begin{sloppypar}
(3) An order under this section may be made in respect of amounts due under a maintenance calculation which is the subject of an appeal only if it appears to the Commission---
\end{sloppypar}
\begin{enumerate}\item[]
($a$) that liability for the amounts would not be affected were the appeal to succeed;
or

($b$) where paragraph ($a$) does not apply, that the making of an order under this
section in respect of the amounts would nonetheless be fair in all the
circumstances.
\end{enumerate}

(4) An order under this section---
\begin{enumerate}\item[]
($a$) may not be made in respect of an account of a prescribed description; and

($b$) may be made in respect of a joint account which is held by the person against whom the order is made and one or more other persons, and which is not of a description prescribed under paragraph ($a$), if (but only if) regulations made by the Secretary of State so provide.
\end{enumerate}

(5) An order under this section---
\begin{enumerate}\item[]
($a$) shall specify the account in respect of which it is made;

($b$) shall be expressed to be directed at the deposit-taker with which the account is held; and

($c$) shall have effect from such date as may be specified in the order.
\end{enumerate}

(6) An order under this section shall operate as an instruction to the deposit-taker at which it is directed to---
\begin{enumerate}\item[]
($a$) make deductions from the amount (if any) standing to the credit of the account specified in the order; and

($b$) pay the amount deducted to the Commission.
\end{enumerate}

(7) The Commission shall serve a copy of any order made under this section on---
\begin{enumerate}\item[]
($a$) the deposit-taker at which it is directed;

($b$) the person against whom it is made; and

($c$) if the order is made in respect of a joint account, the other account-holders.
\end{enumerate}

(8)
Where---
\begin{enumerate}\item[]
($a$)
an order under this section has been made; and

($b$)
a copy of the order has been served on the deposit-taker at which it is directed, 
\end{enumerate}
it shall be the duty of that deposit-taker to comply with the order; but the deposit-taker shall not be under any liability for non-compliance before the end of the period of 7 days beginning with the day on which the copy was served on the deposit-taker.

\begin{sloppypar}
(9)
Where regulations have been made under section 29(3)($a$), a person liable to pay an amount of child support maintenance is to be taken for the purposes of this section to have failed to pay an amount of child support maintenance unless it is paid to or through the person specified in, or by virtue of, the regulations for the case in question.
\end{sloppypar}

\subsection*{32B. Orders under section~32A: joint accounts}

(1) Before making an order under section~32A in respect of a joint account 
the Commission shall offer each of the account-holders an opportunity to make representations about---
\begin{enumerate}\item[]
($a$) the proposal to make the order; and

($b$) the amounts to be deducted under the order, if it is made.
\end{enumerate}

(2)
The amounts to be deducted from a joint account under such an order shall not exceed the amounts that appear to the Commission to be fair in all the circumstances.

(3)
In determining those amounts the Commission shall have particular regard to---
\begin{enumerate}\item[]
($a$) any representations made in accordance with subsection~(1)($b$);

\begin{sloppypar}
($b$) the amount contributed to the account by each of the account-holders; and
\end{sloppypar}

($c$) such other matters as may be prescribed.
\end{enumerate}

\subsection*{\sloppy 32C. Regulations about orders under section 32A}

(1) The Secretary of State may by regulations make provision with respect
to orders under section~32A.

(2) Regulations under subsection~(1) may, in particular, make provision---
\begin{enumerate}\item[]
($a$) requiring an order to specify the amount or amounts in respect of which it is made;

($b$) requiring an order to specify the amounts which are to be deducted under it
in order to meet liabilities under the maintenance calculation in question;

($c$) requiring an order to specify the dates on which deductions are to be made
under it;

($d$) for the rate of deduction under an order not to exceed such rate as may be
specified in, or determined in accordance with, the regulations;

($e$) as to circumstances in which amounts standing to the credit of an account
are to be disregarded for the purposes of section~32A;

($f$) as to the payment of sums deducted under an order to the Secretary of
State;

($g$) allowing the deposit-taker at which an order is directed to deduct from the
amount standing to the credit of the account specified in the order a prescribed
amount towards its administrative costs before making any deduction required
by section~32A(6)($a$);

($h$) with respect to notifications to be given to the person against whom an order
is made (and, in the case of an order made in respect of a joint account, to the
other account-holders) of amounts deducted, and amounts paid, under the
order;

($i$) requiring the deposit-taker at which an order is directed to notify the
Commission in the prescribed manner and within a prescribed period---
\begin{enumerate}\item[]
(i)
if the account specified in the order does not exist at the time at which
the order is served on the deposit-taker;

(ii)
of any other accounts held with the deposit-taker at that time by the
person against whom the order is made;
\end{enumerate}

($j$) requiring the deposit-taker at which an order is directed to notify the
Commission in the prescribed manner and within a prescribed period
if, after the time at which the order is served on the deposit-taker---
\begin{enumerate}\item[]
(i) the account specified in the order is closed;

(ii) a new account of any description is opened with the deposit-taker by
the person against whom the order is made;
\end{enumerate}

($k$) as to circumstances in which the deposit-taker at which an order is directed,
the person against whom the order is made and~(in the case of an order made
in respect of a joint account) the other account-holders may apply to the
Commission for it to review the order and as to
such a review;

($l$) for the variation of orders;

($m$) similar to that made by section~32A(8), in relation to any variation of an
order;

($n$) for an order to lapse in such circumstances as may be prescribed;

($o$) as to the revival of an order in such circumstances as may be prescribed;

($p$) allowing or requiring an order to be discharged;

($q$) as to the giving of notice by the Commission to the deposit-taker that an
order has lapsed or ceased to have effect.
\end{enumerate}

(3)
The Secretary of State may by regulations make provision with respect to priority as between an order under section~32A and---
\begin{enumerate}\item[]
($a$) any other order under that section;

($b$) any order under any other enactment relating to England and Wales which
provides for deductions from the same account;

($c$) any diligence done in Scotland against the same account.
\end{enumerate}

(4)
The Secretary of State shall by regulations make provision for any person affected to have a right to appeal to a court---
\begin{enumerate}\item[]
($a$) against the making of an order under section~32A;

($b$) against any decision made by the Commission on an application
under regulations made under subsection~(2)($k$).
\end{enumerate}

\begin{sloppypar}
(5)
On an appeal under regulations made under subsection~(4)($a$), the court shall not question the maintenance calculation by reference to which the order was made.
\end{sloppypar}

(6)
Regulations under subsection~(4) may include---
\begin{enumerate}\item[]
($a$) provision with respect to the period within which a right of appeal under the regulations may be exercised;

($b$) provision with respect to the powers of the court to which the appeal under the regulations lies.
\end{enumerate}

\subsection*{32D. Orders under section~32A: offences}

(1) A person who fails to comply with the requirements of---
\begin{enumerate}\item[]
($a$) an order under section~32A, or

($b$) any regulation under section~32C which is designated by the regulations for
the purposes of this paragraph, 
\end{enumerate}
commits an offence.

(2)
It shall be a defence for a person charged with an offence under subsection~(1) to prove that the person took all reasonable steps to comply with the requirements in question.

(3)
A person guilty of an offence under subsection~(1) shall be liable on summary conviction to a fine not exceeding level two on the standard scale.''
\end{quotation}

\subsection{23. Lump sum deduction orders}

After section~32D of the Child Support Act 1991 (inserted by section 22 of this Act) insert---
\begin{quotation}
\subsection*{``32E. Lump sum deductions: interim orders}

(1) The Secretary of State may make an order under this section if it appears to the Commission that a person (referred to in this section and sections 32F to 32J as “the liable person”) has failed to pay an amount of child support maintenance and---
\begin{enumerate}\item[]
($a$) an amount stands to the credit of an account held by the liable person with a deposit-taker; or

($b$) an amount not within paragraph ($a$) that is of a prescribed description is due or accruing to the liable person from another person (referred to in this section and sections 32F to 32J as the “third party”).
\end{enumerate}

(2) An order under this section---
\begin{enumerate}\item[]
($a$) may not be made by virtue of subsection~(1)($a$) in respect of an account of a prescribed description; and

($b$) may be made by virtue of subsection~(1)($a$) in respect of a joint account which is held by the liable person and one or more other persons, and which is not of a description prescribed under paragraph ($a$) of this subsection, if (but only if) regulations made by the Secretary of State so provide.
\end{enumerate}

(3)
The Secretary of State may by regulations make provision as to conditions that are to be disregarded in determining whether an amount is due or accruing to the liable person for the purposes of subsection~(1)($b$).

(4)
An order under this section---
\begin{enumerate}\item[]
($a$) shall be expressed to be directed at the deposit-taker or third party in question;

($b$) if made by virtue of subsection~(1)($a$), shall specify the account in respect of which it is made; and

($c$) shall specify the amount of arrears of child support maintenance in respect of which the Commission proposes to make an order under section~32F.
\end{enumerate}

(5) An order under this section may specify an amount of arrears due under a maintenance calculation which is the subject of an appeal only if it appears to the Commission---
\begin{enumerate}\item[]
($a$) that liability for the amount would not be affected were the appeal to succeed; or

($b$) where paragraph ($a$) does not apply, that the making of an order under section~32F in respect of the amount would nonetheless be fair in all the circumstances.
\end{enumerate}

(6) The Commission shall serve a copy of any order made under this section on---
\begin{enumerate}\item[]
($a$)
the deposit-taker or third party at which it is directed;

($b$)
the liable person; and

($c$)
if the order is made in respect of a joint account, the other account-holders.
\end{enumerate}

(7) An order under this section shall come into force at the time at which it is served on the deposit-taker or third party at which it is directed.

(8) An order under this section shall cease to be in force at the earliest of the following---
\begin{enumerate}\item[]
($a$)
the time at which the prescribed period ends;

($b$)
the time at which the order under this section lapses or is discharged; and

($c$)
the time at which an order under section~32F made in pursuance of the proposal specified in the order under this section is served on the deposit-taker or third party at which that order is directed.
\end{enumerate}

\begin{sloppypar}
(9) Where regulations have been made under section 29(3)($a$), a person liable to pay an amount of child support maintenance is to be taken for the purposes of this section to have failed to pay the amount unless it is paid to or through the person specified in, or by virtue of, the regulations for the case in question.
\end{sloppypar}

\subsection*{32F. Lump sum deductions: final orders}

(1) The Commission may make an order under this section in pursuance of a proposal specified in an order under section~32E if---
\begin{enumerate}\item[]
($a$)
the order in which the proposal was specified (“the interim order”) is in force;

($b$)
the period prescribed for the making of representations to the Commission in respect of the proposal specified in the interim order has expired; and

($c$)
the Commmission has considered any representations made to it during that period.
\end{enumerate}

(2) An order under this section---
\begin{enumerate}\item[]
($a$)
shall be expressed to be directed at the deposit-taker or third party at which the interim order was directed;

\begin{sloppypar}
($b$)
if the interim order was made by virtue of section 32E(1)($a$), shall specify the account specified in the interim order; and
\end{sloppypar}

($c$)
shall specify the amount of arrears of child support maintenance in respect of which it is made.
\end{enumerate}

(3) The amount so specified---
\begin{enumerate}\item[]
($a$) shall not exceed the amount of arrears specified in the interim order which remain unpaid at the time at which the order under this section is made; and

($b$) if the order is made in respect of a joint account, shall not exceed the amount that appears to the Commission to be fair in all the circumstances.
\end{enumerate}

(4)
In determining the amount to be specified in an order made in respect of a joint account the Commission shall have particular regard---
\begin{enumerate}\item[]
($a$) to the amount contributed to the account by each of the account-holders; and

($b$) to such other matters as may be prescribed.
\end{enumerate}

(5)
An order under this section may specify an amount of arrears due under a maintenance calculation which is the subject of an appeal only if it appears to the Commission---
\begin{enumerate}\item[]
($a$) that liability for the amount would not be affected were the appeal to succeed; or

($b$) where paragraph ($a$) does not apply, that the making of an order under this section in respect of the amount would nonetheless be fair in all the circumstances.
\end{enumerate}

(6) The Commission shall serve a copy of any order made under this section on---
\begin{enumerate}\item[]
($a$) the deposit-taker or third party at which it is directed;

($b$) the liable person; and

($c$) if the order is made in respect of a joint account, the other account-holders.
\end{enumerate}

\subsection*{32G. Orders under sections 32E and 32F: freezing of accounts etc.}

(1) During the relevant period, an order under section~32E or 32F which
specifies an account held with a deposit-taker shall operate as an instruction to the deposit-taker not to do anything that would reduce the amount standing to the credit of the account below the amount specified in the order (or, if already below that amount, that would further reduce it).

(2)
During the relevant period, any other order under section~32E or 32F shall operate as an instruction to the third party at which it is directed not to do anything that would reduce the amount due to the liable person below the amount specified in the order (or, if already below that amount, that would further reduce it).

(3)
Subsections (1) and~(2) have effect subject to regulations made under section~32I(1).

(4)
In this section “the relevant period”, in relation to an order under section~32E, means the period during which the order is in force.

(5)
In this section and section~32H “the relevant period”, in relation to an order under section~32F, means the period which---
\begin{enumerate}\item[]
($a$) begins with the service of the order on the deposit-taker or third party at which it is directed; and

($b$) (subject to subsection~(6)) ends with the end of the period during which an appeal can be brought against the order by virtue of regulations under section~32J(5).
\end{enumerate}

(6) If an appeal is brought by virtue of the regulations, the relevant period ends at the time at which---
\begin{enumerate}\item[]
($a$) proceedings on the appeal (including any proceedings on a further appeal) have been concluded; and

($b$) any period during which a further appeal may ordinarily be brought has ended.
\end{enumerate}

(7) References in this section and sections 32H and 32J to the amount due to the liable person are to be read as references to the total of any amounts within section~32E(1)($b$) that are due or accruing to the liable person from the third party in question.

\subsection*{\sloppy 32H. Orders under section~32F: deductions and payments}

(1) Once the relevant period has ended, an order under section~32F which specifies an account held with a deposit-taker shall operate as an instruction to the deposit-taker---
\begin{enumerate}\item[]
($a$) if the amount standing to the credit of the account is less than the remaining amount, to pay to the Commission the amount standing to the credit of the account; and

($b$) otherwise, to deduct from the account and pay to the Commission the remaining amount.
\end{enumerate}

(2) If an amount of arrears specified in the order remains unpaid after any payment required by subsection~(1) has been made, the order shall operate until the relevant time as an instruction to the deposit-taker---
\begin{enumerate}\item[]
($a$) to pay to the Commission any amount (not exceeding the remaining
amount) standing to the credit of the account specified in the order; and

($b$) not to do anything else that would reduce the amount standing to the credit
of the account.
\end{enumerate}

(3) Once the relevant period has ended, any other order under section~32F shall
operate as an instruction to the third party at which it is directed---
\begin{enumerate}\item[]
($a$) if the amount due to the liable person is less than the remaining amount, to
pay to the Commission the amount due to the liable person; and

($b$) otherwise, to deduct from the amount due to the liable person and pay to the
Commission the remaining amount.
\end{enumerate}

(4) If an amount of arrears specified in the order remains unpaid after any payment
required by subsection~(3) has been made, the order shall operate until the relevant
time as an instruction to the third party---
\begin{enumerate}\item[]
($a$) to pay to the Commission any amount (not exceeding the remaining
amount) due to the liable person; and

($b$) not to do anything else that would reduce any amount due to the liable
person.
\end{enumerate}

(5) This section has effect subject to regulations made under sections 32I(1) and
32J(2)($c$).

(6) In this section---
\begin{enumerate}\item[]
“the relevant time” means the earliest of the following---
\begin{enumerate}\item[]
($a$) the time at which the remaining amount is paid;

($b$) the time at which the order lapses or is discharged; and

($c$) the time at which a prescribed event occurs or prescribed circumstances
arise;
\end{enumerate}

“the remaining amount”, in relation to any time, means the amount of arrears
specified in the order under section~32F which remains unpaid at that time.
\end{enumerate}

\begin{sloppypar}
\subsection*{32I. Power to disapply sections 32G(1) and~(2) and 32H(2)($b$) and~(4)($b$)}
\end{sloppypar}

(1) The Secretary of State may by regulations make provision as to
circumstances in which things that would otherwise be in breach of sections 32G(1) and~(2) and 32H(2)($b$) and~(4)($b$) may be done.

(2) Regulations under subsection~(1) may require the Commission’s consent
to be obtained in prescribed circumstances.

(3) Regulations under subsection~(1) which require the Commission’s consent
to be obtained may provide for an application for that consent to be made---
\begin{enumerate}\item[]
($a$) by the deposit-taker or third party at which the order under section~32E or
32F is directed;

($b$) by the liable person; and

($c$) if the order is made in respect of a joint account, by any of the other account-holders.
\end{enumerate}

(4) If regulations under subsection~(1) require the Commission’s consent to
be obtained, the Secretary of State shall by regulations provide for a person of a
prescribed description to have a right of appeal to a court against the withholding of
that consent.

(5) Regulations under subsection~(4) may include---
\begin{enumerate}\item[]
($a$) provision with respect to the period within which a right of appeal under the
regulations may be exercised;

($b$) provision with respect to the powers of the court to which the appeal under
the regulations lies.
\end{enumerate}

\subsection*{\sloppy 32J. Regulations about orders under section 32E or~32F}

(1) The Secretary of State may by regulations make provision with respect to orders under section~32E or 32F.

(2) The regulations may, in particular, make provision---
\begin{enumerate}\item[]
($a$) as to circumstances in which amounts standing to the credit of an account are to be disregarded for the purposes of sections 32E, 32G and 32H;

($b$) as to the payment to the Commission of sums deducted under an order under section~32F;

($c$) allowing a deposit-taker or third party at which an order under section~32F is directed to deduct from the amount standing to the credit of the account specified in the order, or due to the liable person, a prescribed amount towards its administrative costs before making any payment to the Commission required by section~32H;

($d$) with respect to notifications to be given to the liable person (and, in the case of an order made in respect of a joint account, to the other account-holders) as to amounts deducted, and amounts paid, under an order under section~32F;

($e$) requiring a deposit-taker or third party at which an order under section~32E or 32F is directed to supply information of a prescribed description to the Commission, or to notify the Commission if a prescribed event occurs or prescribed circumstances arise;

($f$) for the variation of an order under section~32E or 32F;

($g$) for an order under section~32E or 32F to lapse in such circumstances as may be prescribed;

($h$) as to the revival of an order under section~32E or 32F in such circumstances as may be prescribed;

($i$) allowing or requiring an order under section~32E or 32F to be discharged.
\end{enumerate}

(3)
Where regulations under subsection~(1) make provision for the variation of an order under section~32E or 32F, the power to vary the order shall not be exercised so as to increase the amount of arrears of child support maintenance specified in the order.

(4)
The Secretary of State may by regulations make provision with respect to priority as between an order under section~32F and---
\begin{enumerate}\item[]
($a$) any other order under that section;

($b$) any order under any other enactment relating to England and Wales which provides for payments to be made from amounts to which the order under section~32F relates;

($c$) any diligence done in Scotland against amounts to which the order under section~32F relates.
\end{enumerate}

(5)
The Secretary of State shall by regulations make provision for any person affected by an order under section~32F to have a right to appeal to a court against the making of the order.

(6)
On an appeal under regulations under subsection~(5), the court shall not question the maintenance calculation by reference to which the order under section~32F was made.

(7)
Regulations under subsection~(5) may include---
\begin{enumerate}\item[]
($a$) provision with respect to the period within which a right of appeal under the regulations may be exercised;

($b$) provision with respect to the powers of the court to which the appeal under the regulations lies.
\end{enumerate}

\subsection*{32K. Lump sum deduction orders: offences}

(1) A person who fails to comply with the requirements of---
\begin{enumerate}\item[]
($a$) an order under section~32E or 32F; or

($b$) any regulation under section~32J which is designated by the regulations for the purposes of this paragraph,
\end{enumerate}
commits an offence.

(2)
It shall be a defence for a person charged with an offence under subsection~(1) to prove that the person took all reasonable steps to comply with the requirements in question.

(3)
A person guilty of an offence under subsection~(1) shall be liable on summary 
conviction to a fine not exceeding level two on the standard scale.''
\end{quotation}

\subsection{24. Orders preventing avoidance}

After section~32K of the Child Support Act 1991 (inserted by section 23 of this Act) insert---
\begin{quotation}
\subsection*{``32L. Orders preventing avoidance}

(1) The Commission may apply to the court, on the grounds that a person---
\begin{enumerate}\item[]
($a$) has failed to pay an amount of child support maintenance, and

($b$) with the intention of avoiding payment of child support maintenance, is about to make a disposition or to transfer out of the jurisdiction or otherwise deal with any property,
\end{enumerate}
for an order restraining or, in Scotland, interdicting the person from doing so.

(2) The Commission may apply to the court, on the grounds that a person---
\begin{enumerate}\item[]
($a$) has failed to pay an amount of child support maintenance, and

($b$) with the intention of avoiding payment of child support maintenance, has at any time made a reviewable disposition,
\end{enumerate}
for an order setting aside or, in Scotland, reducing the disposition.

(3)
If the court is satisfied of the grounds mentioned in subsection~(1) or~(2) it may make an order under that subsection.

(4)
Where the court makes an order under subsection~(1) or~(2) it may make such consequential provision by order or directions as it thinks fit for giving effect to the order (including provision requiring the making of any payments or the disposal of any property).

(5)
Any disposition is a reviewable disposition for the purposes of subsection~(2), unless it was made for valuable or, in Scotland, adequate consideration (other than marriage) to a person who, at the time of the disposition, acted in relation to it in good faith and without notice of an intention to avoid payment of child support maintenance.

(6)
Subsection (7) applies where an application is made under this section with respect to---
\begin{enumerate}\item[]
($a$) a disposition or other dealing with property which is about to take place, or

($b$) a disposition which took place after the making of the application on which the maintenance calculation concerned was made.
\end{enumerate}

(7) If the court is satisfied---
\begin{enumerate}\item[]
($a$) in a case falling within subsection~(1), that the disposition or other dealing would (apart from this section) have the consequence of making ineffective a step that has been or may be taken to recover the amount outstanding, or

($b$) in a case falling within subsection~(2), that the disposition has had that consequence,
\end{enumerate}
it is to be presumed, unless the contrary is shown, that the person who disposed of or is about to dispose of or deal with the property did so or, as the case may be, is about to do so, with the intention of avoiding payment of child support maintenance.

(8)
In this section "disposition" does not include any provision contained in a will or codicil but, with that exception, includes any conveyance, assurance or gift of property of any description, whether made by an instrument or otherwise.

(9)
This section does not apply to a disposition made before the coming into force of section 24 of the Child Maintenance and Other Payments Act 2008.

(10)
In this section "the court" means---
\begin{enumerate}\item[]
($a$) in relation to England and Wales, the High Court;

($b$) in relation to Scotland, the Court of Session or the sheriff.
\end{enumerate}

(11) An order under this section interdicting a person---
\begin{enumerate}\item[]
($a$) is effective for such period (including an indefinite period) as the order may specify;

($b$) may, on application to the court, be varied or recalled.''
\end{enumerate}
\end{quotation}

\amendment{Ss. 25--30 are not yet in force.}

\section{\itshape Debt management powers}

\subsection{31. Power to treat liability as satisfied}

After section~41B of the Child Support Act 1991 insert---
\begin{quotation}
\subsection*{``41C. Power to treat liability as satisfied}

(1) The Secretary of State may by regulations---
\begin{enumerate}\item[]
($a$) make provision enabling the Commission in prescribed circumstances to set off liabilities to pay child support maintenance to which this section applies;

($b$) make provision enabling the Commission in prescribed circumstances to set off against a person’s liability to pay child support maintenance to which this section applies a payment made by the person which is of a prescribed description.
\end{enumerate}

(2)
Liability to pay child support maintenance shall be treated as satisfied to the extent that it is the subject of setting off under regulations under subsection~(1).

(3)
In subsection~(1), the references to child support maintenance to which this section applies are to child support maintenance for the collection of which the Commission is authorised to make arrangements.''
\end{quotation}

\subsection{32. Power to accept part payment of arrears in full and final satisfaction}

After section~41C of the Child Support Act 1991 (inserted by section~31 of this Act) insert---

\begin{quotation}
\subsection*{``41D. Power to accept part payment of arrears in full and final satisfaction}

(1) The 
%Commission 
Secretary of State
may, in relation to any arrears of child support  maintenance, accept payment of part in satisfaction of liability for the whole.

(2)
The Secretary of State must by regulations make provision with respect to the exercise of the power under subsection~(1).

(3)
The regulations must provide that unless one of the conditions in subsection
(4)
is satisfied the 
%Commission 
Secretary of State %amendment SI 2012/2007 Sch para 84
may not exercise the power under subsection~(1) without the appropriate consent.

(4)
The conditions are---
\begin{enumerate}\item[]
($a$) that the 
%Commission 
Secretary of State %amendment SI 2012/2007 Sch para 84
would be entitled to retain the whole of the arrears under section~41(2) if 
%it 
the Secretary of State %amendment SI 2012/2007 Sch para 84
recovered them;

($b$) that the 
%Commission 
Secretary of State %amendment SI 2012/2007 Sch para 84
would be entitled to retain part of the arrears under section~41(2) if 
%it 
the Secretary of State %amendment SI 2012/2007 Sch para 84
recovered them, and the part of the arrears that the 
%Commission 
Secretary of State %amendment SI 2012/2007 Sch para 84
would not be entitled to retain is equal to or less than the payment accepted under subsection~(1).
\end{enumerate}

(5)
Unless the maintenance calculation was made under section 7, the appropriate consent is the written consent of the person with care with respect to whom the maintenance calculation was made.

(6)
If the maintenance calculation was made under section 7, the appropriate consent is---
\begin{enumerate}\item[]
($a$) the written consent of the child who made the application under section 7(1), and

($b$) if subsection~(7) applies, the written consent of the person with care of that child.
\end{enumerate}

(7) This subsection applies if---
\begin{enumerate}\item[]
($a$) the maintenance calculation was made under section 7(2), or

($b$) the Secretary of State has made arrangements under section 7(3) on the application of the person with care.''
\end{enumerate}
\end{quotation}

\amendment{
Words substituted in s. 32 (1.8.12) by the Public Bodies (Child Maintenance and Enforcement Commission: Abolition and Transfer of Functions) Order 2012 Sch. para.~84.
}

\subsection{33. Power to write off arrears}

After section~41D of the Child Support Act 1991 (inserted by section~32 of this Act) insert---

\begin{quotation}
\subsection*{``41E. Power to write off arrears}

(1) The 
%Commission %substituted by SI 2012/2007 Sch para 85
Secretary of State %end substitution
may extinguish liability in respect of arrears of child support maintenance if it appears to 
%it %substituted by SI 2012/2007 Sch para 85
the Secretary of State%end substitution
---
\begin{enumerate}\item[]
($a$) that the circumstances of the case are of a description specified in regulations made by the Secretary of State, and

($b$) that it would be unfair or otherwise inappropriate to enforce liability in
respect of the arrears.
\end{enumerate}

(2)
The Secretary of State may by regulations make provision with respect to the exercise of the power under subsection~(1).''
\end{quotation}

\amendment{
Words substituted in s. 33 (1.8.12) by the Public Bodies (Child Maintenance and Enforcement Commission: Abolition and Transfer of Functions) Order 2012 Sch. para.~85.

\medskip

S. 34 is not yet in force.}

\section{\itshape Miscellaneous}

\subsection{35. Registered maintenance agreements: Scotland}

(1) In section~4(10) of the Child Support Act 1991 (exclusion of application for maintenance calculation), after paragraph ($aa$) insert---
\begin{quotation}
“($ab$) a maintenance agreement—
\begin{enumerate}\item[]
(i) made on or after the date prescribed for the purposes of paragraph ($a$); and

(ii) registered for execution in the Books of Council and Session or the sheriff court books,
\end{enumerate}
is in force in respect of them, but has been so for less than the period of one year beginning with the date on which it was made; or”.
\end{quotation}

(2) In section 7(10) of that Act (exclusion of application by child in Scotland for maintenance calculation), at the end of paragraph ($b$) insert “;~or 
\begin{quotation}
($c$) a maintenance agreement—
\begin{enumerate}\item[]
(i) made on or after the date prescribed for the purposes of paragraph ($a$); and

(ii) registered for execution in the Books of Council and Session or the sheriff court books,
\end{enumerate}
is in force in respect of them, but has been so for less than the period of one year beginning with the date on which it was made.”
\end{quotation}

(3) In section 9(3) of that Act (agreements about maintenance), after “4(10)($a$)” insert “and~($ab$)”.

\subsection{36. Offence of failing to notify change of address}

(1) In section 14A of the Child Support Act 1991 (information offences), after subsection~(3) insert—
\begin{quotation}
“(3A) In the case of regulations under section 14 which require a person liable to make payments of child support maintenance to notify a change of address, a person who fails to comply with the requirement is guilty of an offence.”
\end{quotation}

(2) In that section, in subsection~(4), after “subsection~(3)” insert “or~(3A)”.

\subsection{37. Additional special case}

In section~42(2) of the Child Support Act 1991 (examples of cases in relation to which the power under subsection~(1) to prescribe circumstances in which a case is to be treated as a special case for the purposes of the Act may be exercised), at the end insert---
\begin{quotation}
``($g$) the same persons are the parents of two or more children and each parent is---
\begin{enumerate}\item[]
(i) a non-resident parent in relation to one or more of the children, and

(ii) a person with care in relation to one or more of the children.''
\end{enumerate}
\end{quotation}

\subsection{38. Recovery of arrears from deceased's estate}

After section~43 of the Child Support Act 1991 insert---

\begin{quotation}
\subsection*{``43A. Recovery of arrears from deceased's estate}

(1) The Secretary of State may by regulations make provision for the recovery from the estate of a deceased person of arrears of child support maintenance for which the deceased person was liable immediately before death.

(2) Regulations under subsection~(1) may, in particular---
\begin{enumerate}\item[]
($a$) make provision for arrears of child support maintenance for which a deceased person was so liable to be a debt payable by the deceased’s executor or administrator out of the deceased’s estate to the Commission;

($b$) make provision for establishing the amount of any such arrears;

($c$) make provision about procedure in relation to claims under the regulations.
\end{enumerate}

(3) Regulations under subsection~(1) may include provision for proceedings (whether by appeal or otherwise) to be instituted, continued or withdrawn by the deceased’s executor or administrator.''
\end{quotation}

\amendment{Ss. 39--40 are not yet in force.}

\subsection{41. Pilot schemes}

After section 51 of the Child Support Act 1991 insert---

\begin{quotation}
\subsection*{``51A. Pilot schemes}

(1) Any regulations made under this Act may be made so as to have effect for a specified period not exceeding 24 months.

(2)
Regulations which, by virtue of subsection~(1), are to have effect for a limited period are referred to in this section as a “pilot scheme”.

(3) A pilot scheme may provide that its provisions are to apply only in relation to---
\begin{enumerate}\item[]
($a$) one or more specified areas or localities;

($b$) one or more specified classes of person;

($c$) persons selected by reference to prescribed criteria, or on a sampling basis.
\end{enumerate}

(4)
A pilot scheme may make consequential or transitional provision with respect to the cessation of the scheme on the expiry of the specified period.

(5)
A pilot scheme may be replaced by a further pilot scheme making the same or similar provision.''
\end{quotation}

\subsection{42. Meaning of ``child''}

For section 55 of the Child Support Act 1991 substitute---

\begin{quotation}
\subsection*{``55. Meaning of ``child''}

(1) In this Act, “child” means (subject to subsection~(2)) a person who---
\begin{enumerate}\item[]
($a$) has not attained the age of 16, or

($b$) has not attained the age of 20 and satisfies such conditions as may be prescribed.
\end{enumerate}

(2)
A person who is or has been party to a marriage or civil partnership is not a child for the purposes of this Act.

(3)
For the purposes of subsection~(2), “marriage” and “civil partnership” include a void marriage and a void civil partnership respectively.''
\end{quotation}

\subsection{43. Extinction of liability in respect of interest and fees}

Any outstanding liability in respect of the following is extinguished—
\begin{enumerate}\item[]
($a$) interest under the Child Support (Arrears, Interest and Adjustment of Maintenance Assessments) Regulations 1992;

($b$) fees under the Child Support Fees Regulations 1992.
\end{enumerate}

%\subsection{44. Use of information}
%
%Schedule 6 (which makes provision about the use of information for purposes of public administration) has effect.

\amendment{
S. 44 omitted (1.8.12) by the Public Bodies (Child Maintenance and Enforcement Commission: Abolition and Transfer of Functions) Order 2012 Sch. para.~89.
}

\subsection{45. Liable relative provisions: exclusion of parental duty to maintain}

(1) In section 105 of the Social Security Administration Act 1992 (failure to maintain), for subsection~(3) substitute—
\begin{quotation}
“(3) Subject to subsection~(4), for the purposes of this Part, a person shall be liable to maintain another person if that other person is—
\begin{enumerate}\item[]
($a$) his or her spouse or civil partner, or

($b$) a person whom he or she would be liable to maintain if sections 78(6)($c$) and~(9) had effect for the purposes of this Part.”
\end{enumerate}
\end{quotation}

(2) In that section, in subsection~(4), for the words from “a person” to the end, substitute “subsection~(3)($b$) shall not apply”.

\amendment{Part IV is not relevant to Child Support and is not reproduced here.}


\part[Part V --- General]{Part V\\*General}

\renewcommand\parthead{--- Part V}

\subsection{55. Regulations and orders: general}

(1) This section has effect in relation to regulations under this Act, except Part IV.

(2) Power to make regulations is exercisable by statutory instrument.

(3) Power to make regulations includes power to make incidental, supplementary, consequential or transitional provision or savings.

(4) Power to make regulations may be exercised---
\begin{enumerate}\item[]
($a$) in relation to all cases to which it extends, in relation to those cases but subject to specified exceptions or in relation to any specified cases or classes of case;

($b$) so as to make, as respects the cases in relation to which it is exercised---
\begin{enumerate}\item[]
(i) the full provision to which it extends or any lesser provision (whether by way of exception or otherwise);

(ii) the same provision for all cases, different provision for different cases or classes of case or different provision as respects the same case or class of case but for different purposes of this Act;

(iii) provision which is either unconditional or is subject to any specified condition;
\end{enumerate}

($c$) so as to provide for a person to exercise a discretion in dealing with any matter.
\end{enumerate}

(5) A statutory instrument containing---
\begin{enumerate}\item[]
($a$) regulations under section 6(1) or~(4), or

($b$) the first regulations under paragraphs 2(1), 3(1), 5(1) or~(2), 6(1) or~(3) or 7 of Schedule 5,% or
%
%($c$) an order under section 11(6),
\end{enumerate}
shall not be made unless a draft of the statutory instrument containing the regulations or order has been laid before, and approved by a resolution of, each House of Parliament.

(6) A statutory instrument that---
\begin{enumerate}\item[]
($a$) contains regulations, and

($b$) is not subject to a requirement that a draft of the instrument be laid before, and approved by a resolution of, each House of Parliament,
\end{enumerate}
shall be subject to annulment in pursuance of a resolution of either House of Parliament.

\amendment{
S. 55(5)($c$) omitted (1.8.12) by the Public Bodies (Child Maintenance and Enforcement Commission: Abolition and Transfer of Functions) Order 2012 Sch. para.~90.
}

\subsection{56. General interpretation}

%(1) In this Act, “the Commission” has the meaning given by section 1(1).

(2) Where—
\begin{enumerate}\item[]
($a$) this Act amends or repeals an enactment contained in the Child Support Act 1991 which has been amended by the Child Support, Pensions and Social Security Act 2000, and

($b$) the amendment by the 2000 Act has been brought into force for limited purposes only,
\end{enumerate}
the reference to the enactment shall, unless the contrary intention appears, be read as a reference to the enactment as it has effect apart from the 2000 Act, as well as to the enactment as amended by that Act.

\amendment{
S. 56(1) omitted (1.8.12) by the Public Bodies (Child Maintenance and Enforcement Commission: Abolition and Transfer of Functions) Order 2012 Sch. para.~91.
}

\subsection{57. Minor and consequential amendments}

(1) Schedule 7 (which makes minor and consequential amendments) has effect.

(2) The Secretary of State may by regulations make provision consequential on this Act amending, repealing or revoking any provision of—
\begin{enumerate}\item[]
($a$) an Act passed on or before the last day of the Session in which this Act is passed, or

($b$) an instrument made under an Act before the passing of this Act.
\end{enumerate}

\subsection{58. Repeals}

The enactments specified in Schedule 8 are repealed to the extent specified.

\subsection{59. Transition}

%(1) Until the coming into force of section 13, the Child Support Act 1991 shall have effect as if references to the Commission were to the Secretary of State.

(2) The Secretary of State may by regulations make provision for the Child Support Act 1991, as amended by Schedule 3, to have effect, until the coming into force of section 15, with such modifications as the Secretary of State considers necessary in consequence of the retention of functions under section~46 of that Act.

(3) The Secretary of State may, in relation to section 6 or 46 of the Child Support Act 1991, by regulations make provision for the section to have effect with such modifications as the Secretary of State considers expedient in anticipation of the coming into force of section 15.

(4) Sections 20(5A), 32A, 32E, 32F, 32J, 32L, 32M, 41C to 41E, 43A, 49A, 49B and 49D of the Child Support Act 1991 shall have effect as if “child support maintenance” included periodical payments required to be paid in accordance with a maintenance assessment under the Act.

(5) Sections 20(7A), 32A, 32C, 32E, 32F, 32J, 32L, 32M, 39B, 39H, 40, 40A, 40B and 49B of the Child Support Act 1991 shall have effect as if “maintenance calculation” included a maintenance assessment under the Act.

(6) Sections 35, 36, 38, 39B, 39H, 39K, 40, 40B and 49D of the Child Support Act 1991 shall have effect as if orders made under section~33 of that Act had been made under section~32M of that Act.

(7) An order may be made under section~32M of the Child Support Act 1991 in respect of an amount even though the time within which an application could have been instituted under section~33 of that Act for an order in respect of that amount has expired.

(8) The Secretary of State may by regulations make in connection with the coming into force of any provision of this Act such transitional provision or savings as the Secretary of State considers necessary or expedient.

\amendment{
S. 59(1) omitted (1.8.12) by the Public Bodies (Child Maintenance and Enforcement Commission: Abolition and Transfer of Functions) Order 2012 Sch. para.~92.
}

\subsection{60. Financial provisions}

(1) There shall be paid out of money provided by Parliament—
\begin{enumerate}\item[]
($a$) any expenditure incurred by the Secretary of State or a government department in consequence of this Act, and

($b$) any increase attributable to this Act in the sums payable out of money so provided under any other enactment.
\end{enumerate}

\amendment{
S. 60(2) is not yet in force.
}

\subsection{61. Extent}

(1) Subject to the following provisions, this Act extends to England and Wales and Scotland only.

(2) The following provisions also extend to Northern Ireland---
\begin{enumerate}\item[]
($a$) this section and sections 55, 57(2), 62 and 63%;
%
%($b$) paragraphs 4 to 6 of Schedule 6, and section~44 so far as relating to those paragraphs%
.
\end{enumerate}

(3) Any amendment or repeal made by this Act has the same extent as the enactment to which it relates.

\amendment{
S. 61(2)($b$) omitted (1.8.12) by the Public Bodies (Child Maintenance and Enforcement Commission: Abolition and Transfer of Functions) Order 2012 Sch. para.~93.
}

\subsection{62. Commencement}

(1) This section and sections 55, 59(8), 61 and 63 shall come into force on the day on which this Act is passed.

(2) Section 35 shall come into force on the day after the day on which this Act is passed.

(3) The remaining provisions of this Act shall come into force on such day as the Secretary of State may by order made by statutory instrument appoint, and different days may be so appointed for different purposes.

(4) An order under subsection~(3) may include such transitional provision or savings as the Secretary of State considers necessary or expedient in connection with bringing any provision of this Act into force.

(5) An order under subsection~(3) appointing the day on which section~39 is to come into force in England and Wales may be made only with the consent of the Lord Chancellor.

\subsection{63. Citation}

This Act may be cited as the Child Maintenance and Other Payments Act 2008.

\small

\amendment{
Sch. 1 omitted  (1.8.12) by the Public Bodies (Child Maintenance and Enforcement Commission: Abolition and Transfer of Functions) Order 2012 Sch. para.~71.

\medskip

Sch. 2 omitted  (1.8.12) by the Public Bodies (Child Maintenance and Enforcement Commission: Abolition and Transfer of Functions) Order 2012 Sch. para.~76.

}

%\part[Schedule 1 --- The Commission]{Schedule 1\\*The Commission}
%
%\subsection*{Constitution of the Commission}
%
%1. The Commission shall consist of the following members—
%\begin{enumerate}\item[]
%($a$) a person to chair the Commission,
%
%($b$) the chief executive of the Commission (who is to be known as the Commissioner for Child Maintenance),
%
%($c$) one or more directors appointed from the staff of the Commission (“executive directors”), and
%
%($d$) two or more directors appointed otherwise than from the staff of the Commission (“non-executive directors”).
%\end{enumerate}
%
%\subsection*{Appointment of a person to chair the Commission}
%
%2. Appointments for the purposes of paragraph 1($a$) are to be made by the Secretary of State otherwise than from the staff of the Commission.
%
%\subsection*{Appointment of directors}
%
%3.---(1) Appointments for the purposes of paragraph 1($c$) are to be made by the Commission, with the approval of the Secretary of State.
%
%(2) Appointments for the purposes of paragraph 1($d$) are to be made by the person appointed to chair the Commission, with the approval of the Secretary of State.
%
%(3) The power under sub-paragraph (1) may not be exercised if the result of exercising it would be to make the number of executive members of the Commission equal to or greater than the number of non-executive members of the Commission.
%
%(4) The power under sub-paragraph (2) must be exercised so as to secure, so far as practicable, that the Commission always has more non-executive members than executive members.
%
%\subsection*{Terms of appointment and tenure of members}
%
%4.---(1) The terms and conditions of a person's appointment to chair the Commission are to be such as the Secretary of State may determine.
%
%(2) The terms and conditions of a person's appointment as a non-executive director are to be such as the person appointed to chair the Commission may determine with the approval of the Secretary of State.
%
%(3) The matters with which the terms and conditions of a member's appointment may deal include, in particular—
%\begin{enumerate}\item[]
%($a$) the period for which the member is to hold office;
%
%($b$) the member's eligibility for re-appointment;
%
%($c$) circumstances in which membership may be suspended or terminated.
%\end{enumerate}
%
%\medskip
%
%5.---(1) Subject to sub-paragraphs (2) and~(3), a person appointed to be a member of the Commission—
%\begin{enumerate}\item[]
%($a$) is to hold and vacate office in accordance with the terms and conditions of his or her appointment, and
%
%($b$) may resign or be removed from office in accordance with those terms and conditions.
%\end{enumerate}
%
%(2) A person appointed as an executive director ceases to be a member of the Commission if he or she ceases to be a member of its staff.
%
%(3) A person appointed to chair the Commission or as a non-executive director ceases to be a member of the Commission if he or she becomes a member of its staff.
%
%\subsection*{Remuneration etc.\ of non-executive members}
%
%6.---(1) The Commission may pay, or make provision for paying, the person appointed to chair the Commission such remuneration as the Secretary of State may determine.
%
%(2) The Commission may—
%\begin{enumerate}\item[]
%($a$) pay to or in respect of any person who is or has been appointed to chair the Commission such pension, allowances or gratuities as the Secretary of State may determine, or
%
%($b$) make such payments as the Secretary of State may determine towards provision for the payment of a pension, allowance or gratuity to or in respect of such a person.
%\end{enumerate}
%
%(3) Where—
%\begin{enumerate}\item[]
%($a$) the person appointed to chair the Commission ceases to be a member of the Commission otherwise than on the expiry of his term of office, and
%
%($b$) it appears to the Secretary of State that there are circumstances which make it right for that person to receive compensation,
%\end{enumerate}
%the Commission may make a payment to that person of such amount as the Secretary of State may determine.
%
%\medskip
%
%7.---(1) The Commission may pay, or make provision for paying, non-executive directors of the Commission such remuneration as the person appointed to chair the Commission may determine with the approval of the Secretary of State.
%
%(2) The Commission may—
%\begin{enumerate}\item[]
%($a$) pay to or in respect of any person who is or has been a non-executive director such pension, allowances or gratuities as the person appointed to chair the Commission may determine with the approval of the Secretary of State, or
%
%($b$) make such payments as the person appointed to chair the Commission may determine with the approval of the Secretary of State towards provision for the payment of a pension, allowance or gratuity to or in respect of any person who is or has been a non-executive director.
%\end{enumerate}
%
%(3) Where—
%\begin{enumerate}\item[]
%($a$) a non-executive director ceases to be a member of the Commission otherwise than on the expiry of his term of office, and
%
%($b$) it appears to the person appointed to chair the Commission that there are circumstances which make it right for that person to receive compensation,
%\end{enumerate}
%the Commission may make a payment to that person of such amount as the person appointed to chair the Commission may determine with the approval of the Secretary of State.
%
%\subsection*{Appointment of deputy chair}
%
%8. The person appointed to chair the Commission must appoint one of the non-executive directors as his or her deputy for such period (not exceeding the remainder of the non-executive director's period of office as director) as he or she may specify on making the appointment.
%
%\subsection*{Staff}
%
%9.---(1) The Commission is to have a chief executive.
%
%(2) The chief executive is employed in the civil service of the State.
%
%(3) The first appointment of a chief executive—
%\begin{enumerate}\item[]
%($a$) is to be made by the Secretary of State, and
%
%($b$) is to be on such terms and conditions as to remuneration and other matters as the Secretary of State may, with the approval of the Minister for the Civil Service, determine.
%\end{enumerate}
%
%(4) Subsequent appointments of a chief executive—
%\begin{enumerate}\item[]
%($a$) are to be made by the Commission with the approval of the Secretary of State, and
%
%($b$) are to be on such terms and conditions as to remuneration and other matters as the Commission may determine with the approval of the Secretary of State and the Minister for the Civil Service.
%\end{enumerate}
%
%\medskip
%
%10.---(1) The Commission may appoint such other staff as it considers appropriate.
%
%(2) Any such appointments are to be on such terms and conditions as to remuneration and other matters as the Commission may, with the approval of the Minister for the Civil Service, determine.
%
%\subsection*{Committees}
%
%11.---(1) The Commission may establish committees for any purpose.
%
%(2) Any committee established under sub-paragraph (1) may establish sub-committees.
%
%(3) Any committee or sub-committee established under this paragraph may consist of or include persons who are not members of the Commission.
%
%(4) Any sub-committee established under sub-paragraph (2) may consist of or include persons who are not members of the committee by which it is established.
%
%(5) Sub-paragraphs (2) to (4) do not apply to the committee established under paragraph 20(1) or to any of its sub-committees.
%
%\medskip
%
%12. Appointment as a member of a committee or sub-committee of the Commission of a person who is not a member of the Commission or its staff is to be on such terms and conditions as to remuneration and other matters as the Commission may determine.
%
%\subsection*{Procedure}
%
%13. The Commission may determine—
%\begin{enumerate}\item[]
%($a$) its own procedure (including quorum), and
%
%($b$) the procedure (including quorum) of any of its committees.
%\end{enumerate}
%
%\subsection*{Delegation}
%
%14.---(1) The Commission may authorise—
%\begin{enumerate}\item[]
%($a$) any member of the Commission,
%
%($b$) any member of its staff, or
%
%($c$) any of its committees,
%\end{enumerate}
%to exercise on its behalf such of its functions, in such circumstances, as it may determine.
%
%(2) This paragraph does not apply to the functions listed in paragraph 20(1).
%
%\medskip
%
%15.---(1) The person appointed to chair the Commission may authorise—
%\begin{enumerate}\item[]
%($a$) any executive member of the Commission,
%
%($b$) any member of its staff, or
%
%($c$) subject to sub-paragraph (2), any of its committees,
%\end{enumerate}
%to exercise on his or her behalf the functions under paragraph 4(2) or 7.
%
%(2) Authority may not be given under sub-paragraph (1)($c$) to a committee that includes a non-executive director; and authority given under that provision ceases to have effect if a non-executive director becomes a member of the committee concerned.
%
%\subsection*{Instruments}
%
%16.---(1) The fixing of the common seal of the Commission must be authenticated by the signature of a person authorised for that purpose by the Commission (whether generally or specifically).
%
%(2) A document purporting—
%\begin{enumerate}\item[]
%($a$) to be duly executed under the seal of the Commission, or
%
%($b$) to be signed on its behalf,
%\end{enumerate}
%is to be received in evidence and taken, without further proof, to be so executed or signed unless the contrary is proved.
%
%(3) This paragraph does not apply in relation to Scotland.
%
%\subsection*{Finance}
%
%17.---(1) The Secretary of State may out of money provided by Parliament make such payments to the Commission as the Secretary of State considers appropriate for the purpose of enabling the Commission to meet its expenses.
%
%(2) Payments under this paragraph may be made at such times and subject to such conditions (if any) as the Secretary of State considers appropriate.
%
%\subsection*{Accounts and audit}
%
%18.---(1) The Commission must—
%\begin{enumerate}\item[]
%($a$) keep proper accounts and proper records in relation to its accounts, and
%
%($b$) prepare in respect of each financial year a statement of accounts.
%\end{enumerate}
%
%(2) Each statement of accounts must comply with any directions given by the Secretary of State with the approval of the Treasury as to—
%\begin{enumerate}\item[]
%($a$) the information to be contained in it and the manner in which it is to be presented;
%
%($b$) the methods and principles according to which the statement is to be prepared;
%
%($c$) the additional information (if any) which is to be provided for the information of Parliament.
%\end{enumerate}
%
%(3) The Commission must send a copy of each statement of accounts—
%\begin{enumerate}\item[]
%($a$) to the Secretary of State, and
%
%($b$) to the Comptroller and Auditor General,
%\end{enumerate}
%before the end of the month of August next following the financial year to which the statement relates.
%
%(4)The Comptroller and Auditor General must—
%\begin{enumerate}\item[]
%($a$) examine, certify and report on each statement of accounts received under sub-paragraph (3), and
%
%($b$) send a copy of each report and certified statement to the Secretary of State.
%\end{enumerate}
%
%(5) The Secretary of State must lay before Parliament a copy of each report and statement sent under sub-paragraph (4)($b$).
%
%(6) In this paragraph, “financial year” means—
%\begin{enumerate}\item[]
%($a$) the period beginning with the date on which the Commission is established and ending with the next following 31st March, and
%
%($b$) each successive period of 12 months.\end{enumerate}
%
%\medskip
%
%19. The Commission must keep under review the question whether its internal financial controls secure the proper conduct of its financial affairs.
%
%\subsection*{Non-executive functions committee}
%
%20.---(1) The Commission must establish a committee to discharge the following functions on its behalf—
%\begin{enumerate}\item[]
%($a$) the function under paragraph 9(4)($b$);
%
%($b$) the function under paragraph 10(2), so far as relating to executive directors;
%
%($c$) the function under paragraph 12;
%
%($d$) the function under paragraph 19.
%\end{enumerate}
%
%(2) The committee under sub-paragraph (1) is to consist of at least three members.
%
%(3) Only non-executive members of the Commission may be members of the committee under sub-paragraph (1).
%
%(4) The committee under sub-paragraph (1) is to be chaired by a person other than the person appointed to chair the Commission.
%
%(5) The committee under sub-paragraph (1) must prepare a report on the discharge of the functions mentioned in that sub-paragraph for inclusion in the annual report of the Commission to the Secretary of State under section 9.
%
%(6) The report under sub-paragraph (5) must relate to the same period as the Commission's report.
%
%(7) The committee under sub-paragraph (1) may establish sub-committees.
%
%(8) A sub-committee of the committee under sub-paragraph (1) may consist of or include persons who are not members of that committee or the Commission.
%
%(9) The members of any sub-committee of the committee under sub-\hspace{0pt}paragraph (1) must not include persons who are executive members or other staff of the Commission.
%
%(10) The committee under sub-paragraph (1) may authorise any of its members or any of its sub-committees to discharge on its behalf—
%\begin{enumerate}\item[]
%($a$) the function mentioned in sub-paragraph (1)($d$);
%
%($b$) the duty to prepare a report under sub-paragraph (5).
%\end{enumerate}
%
%\subsection*{Supplementary powers}
%
%21. The Commission may do anything (except borrow money) which is calculated to facilitate, or is conducive or incidental to, the carrying out of its functions.
%
%\subsection*{Status of the Commission}
%
%22.---(1) The functions of the Commission, and of its members, are to be exercised on behalf of the Crown.
%
%(2) For the purposes of any civil proceedings arising out of those functions—
%\begin{enumerate}\item[]
%($a$) the Crown Proceedings Act 1947 applies to the Commission as if it were a government department, and
%
%($b$) the Crown Suits (Scotland) Act 1857 applies to it as if it were a public department.
%\end{enumerate}
%
%\subsection*{Validity}
%
%23. The validity of any proceedings of the Commission (including proceedings of any of its committees) is not to be affected by—
%\begin{enumerate}\item[]
%($a$) any vacancy among the members of the Commission or any of its committees,
%
%($b$) any defect in the appointment of any member of the Commission or any of its committees,
%
%($c$) any defect in the appointment of the Commissioner for Child Maintenance, or
%
%($d$) the composition for the time being of the membership of the Commission.
%\end{enumerate}
%
%\subsection*{Public records}
%
%24. In Schedule 1 to the Public Records Act 1958 (definition of public records), in Part II of the Table at the end of paragraph 3, at the appropriate place insert— 
%\begin{quotation}
%“Child Maintenance and Enforcement Commission.”
%\end{quotation}
%
%\subsection*{Investigation by Parliamentary Commissioner}
%
%25. In Schedule 2 to the Parliamentary Commissioner Act 1967 (departments and authorities subject to investigation), at the appropriate place insert— 
%\begin{quotation}
%“Child Maintenance and Enforcement Commission.”
%\end{quotation}
%
%\subsection*{Civil service pensions}
%
%26.---(1) The Commission must pay to the Minister for the Civil Service, at such times as he or she may direct, such sums as he or she may determine in respect of the increase in the sums payable out of money provided by Parliament that is attributable to the provision of relevant pensions.
%
%(2) In sub-paragraph (1), “relevant pensions” means pensions, allowances or gratuities under section 1 of the Superannuation Act 1972 payable to or in respect of persons who are or have been in the service of the Commission.
%
%\subsection*{House of Commons disqualification}
%
%27. In Part II of Schedule 1 to the House of Commons Disqualification Act 1975 (bodies of which all members are disqualified), at the appropriate place insert—
%\begin{quotation}
%“The Child Maintenance and Enforcement Commission.”
%\end{quotation}
%
%\subsection*{Northern Ireland Assembly disqualification}
%
%28. In Part II of Schedule 1 to the Northern Ireland Assembly Disqualification Act 1975 (bodies of which all members are disqualified), at the appropriate place insert— 
%\begin{quotation}
%“The Child Maintenance and Enforcement Commission.”
%\end{quotation}
%
%\subsection*{Freedom of information}
%
%29. In Part VI of Schedule 1 to the Freedom of Information Act 2000 (public authorities), at the appropriate place insert— 
%\begin{quotation}
%“The Child Maintenance and Enforcement Commission.”
%\end{quotation}
%
%\subsection*{Interpretation}
%
%30. In this Schedule—
%\begin{enumerate}\item[]
%($a$) references to executive members of the Commission are to the Commissioner for Child Maintenance and the executive directors;
%
%($b$) references to non-executive members of the Commission are to those members of the Commission who are not executive members of it.
%\end{enumerate}
%
%\medskip
%
%31. In this Schedule, references to the staff of the Commission are to the Commissioner for Child Maintenance and the other staff appointed under paragraph 10.
%
%\medskip
%
%32. In this Schedule, references to the committees of the Commission are to—
%\begin{enumerate}\item[]
%($a$) the committee established under paragraph 20 and any of its sub-committees, and
%
%($b$) any committees or sub-committees established under paragraph 11.
%\end{enumerate}

%\part[Schedule 2 --- Transfer of functions under subordinate legislation]{Schedule 2\\*Transfer of functions under subordinate legislation}
%
%\noindent
%\begin{longtable}{x{50pt}p{134.46266pt}p{124.5426pt}}
%\hline
%\itshape Number & \itshape Title & \itshape Provisions conferring functions transferred\\
%\hline
%\endhead
%\hline
%\endlastfoot
%SI 1992/1812&The Child Support (Information, Evidence and Disclosure) Regulations 1992&All the regulations.\\
%SI 1992/1813&\textls[25]{The Child Support (Mainte\-}nance Assessment Procedure) Regulations 1992&Regulation 1 so far as relating to other functions transferred to the Commission by virtue of section 13.\\
%&&All other regulations except regulations 35A, 36, 38, 47 and 49.\\
%SI 1992/1815&\textls[25]{The Child Support (Mainte\-}nance Assessments and Special Cases) Regulations 1992&All the regulations.\\
%SI 1992/1816&The Child Support (Arrears, Interest and Adjustment of Maintenance Assessments) Regulations 1992&All the regulations.\\
%SI 1992/1989&The Child Support (Collection and Enforcement) Regulations 1992&All the regulations.\\
%SI 1992/2643&\textls[25]{The Child Support (Collec\-}tion and Enforcement of Other Forms of Maintenance) Regulations 1992&All the regulations.\\
%SI 1992/2645&\textls[25]{The Child Support (Mainte\-}nance Arrangements and Jurisdiction) Regulations 1992&All the regulations.\\
%SI 1993/627&\textls[100]{The Family Proceedings} \textls[25]{Courts (Child Support Act} 1991) Rules 1993&All the rules.\\
%SI 1994/227&The Child Support (Miscellaneous Amendments and Tran\-\textls[25]{sitional Provisions) Regula\-}tions 1994&All the regulations.\\
%SI 1995/1045&The Child Support and Income Support (Amendment) Regulations 1995&All the regulations.\\
%SI 1996/2907&The Child Support Departure Direction and Consequential Amendments Regulations 1996&All the regulations except regulation 47.\\
%SI 1999/991&The Social Security and Child \textls[25]{Support (Decisions and Ap\-}peals) Regulations 1999&Regulations 3A, 6A, 6B, 7B, 7C, 15A to 15D and 24.\\
%&& Regulations 4, 32 to 34, 39 and 40 so far as relating to child support.\\
%&&Regulation 23 so far as relating to other functions transferred to the Commission by virtue of section 13.\\
%SI 1999/1305&The Child Support Commis\-\textls[25]{sioners (Procedure) Regula\-}tions 1999&Regulation 20.\\
%SI 1999/1510&The Social Security Act 1998 \textls[25]{(Commencement No.\ 7 and} Consequential and Transitional Provisions) Order 1999&All the articles.\\
%SI 2000/3173&The Child Support (Variations) (Modification of Statutory Provisions) Regulations 2000&All the regulations.\\
%SI 2000/3177&The Child Support (Voluntary Payments) Regulations 2000&All the regulations.\\
%SI 2000/3186&\textls[25]{The Child Support (Transi\-}tional Provisions) Regulations 2000&All the regulations.\\
%SI 2001/155&\textls[25]{The Child Support (Mainte\-}nance Calculations and Special Cases) Regulations 2000&All the regulations.\\
%SI 2001/156&The Child Support (Variations) Regulations 2000&All the regulations.\\
%SI 2001/157&\textls[25]{The Child Support (Mainte\-}nance Calculation Procedure) Regulations 2000&All the regulations except regulations 10 to 19.\\
%\hline
%\end{longtable}

\part[Schedule 3 --- Transfer of child support functions]{Schedule 3\\*Transfer of child support functions}

\renewcommand\parthead{--- Schedule 3}

\amendment{
Many of the amendments in this Schedule have been reversed by the Public Bodies (Child Maintenance and Enforcement Commission: Abolition and Transfer of Functions) Order 2012.
}

\section[Part I --- Consequential amendments]{Part I\\*Consequential amendments}

\subsection*{Child Support Act 1991}

1. The Child Support Act 1991 is amended as follows.

\medskip

2. In section 2 (welfare of children: the general principle)—
\begin{enumerate}\item[]
($a$) for “Secretary of State” substitute “Commission”;

($b$) for “he” substitute “it”;

($c$) for “his” substitute “its”.
\end{enumerate}

\medskip

3.---(1) Section 4 (child support maintenance) is amended as follows.

(2) In subsection~(1), for “Secretary of State” substitute “Commission”.

(3) In subsection~(2)—
\begin{enumerate}\item[]
($a$) for “Secretary of State” substitute “Commission”;

($b$) for “him” substitute “it”.
\end{enumerate}

(4) In subsection~(3)—
\begin{enumerate}\item[]
($a$) for “Secretary of State” (in both places where it occurs) substitute “Commission”;

($b$) for “he” substitute “it”.
\end{enumerate}

(5) In subsection~(4), for “Secretary of State” (in the first and third places where it occurs) substitute “Commission”.

(6) In subsection~(5)—
\begin{enumerate}\item[]
($a$) for “Secretary of State” substitute “Commission”;

($b$) for “him” substitute “it”.
\end{enumerate}

(7) In subsections (6) and~(7), for “Secretary of State” substitute “Commission”.

\medskip

4.---(1) Section 6 (applications by those claiming or receiving benefit) is amended as follows.

(2) After subsection~(2) insert—
\begin{quotation}
“(2A) The Secretary of State must notify the Commission of circumstances giving rise to the application of this section.”
\end{quotation}

(3) In subsections (3) and~(4), for “Secretary of State” substitute “Commission”.

(4) In subsection~(5)—
\begin{enumerate}\item[]
($a$) for “Secretary of State” substitute “Commission”;

($b$) for “him” substitute “it”.
\end{enumerate}

(5) In subsection~(7), for “Secretary of State's” substitute “Commission's”.

(6) In subsection~(8), for “Secretary of State” substitute “Commission”.

(7) In subsection~(9)—
\begin{enumerate}\item[]
($a$) for “Secretary of State” substitute “Commission”;

($b$) for “he” substitute “it”.
\end{enumerate}

(8) In subsection~(10), for “Secretary of State” substitute “Commission”.

(9) In subsection~(11), for “he” (in the second place where it occurs) substitute “the Commission”.

(10) In subsection~(12), for “Secretary of State's” substitute “Commission's”.

\medskip

5.---(1) Section 7 (right of child in Scotland to apply for calculation) is amended as follows.

(2) In subsections (1) and~(2), for “Secretary of State” substitute “Commission”.

(3) In subsection~(3)—
\begin{enumerate}\item[]
($a$) for “Secretary of State” substitute “Commission”;

($b$) for “him” substitute “it”.
\end{enumerate}

(4) In subsection~(4)—
\begin{enumerate}\item[]
($a$) for “Secretary of State” (in both places where it occurs) substitute “Commission”;

($b$) for “he” substitute “it”.
\end{enumerate}

(5) In subsection~(5), for “Secretary of State” (in the first and third places where it occurs) substitute “Commission”.

(6) In subsections (6) and~(7), for “Secretary of State” substitute “Commission”.

(7) In subsection~(8)($b$), for “Secretary of State” substitute “Commission”.

\medskip

6. In section 8 (role of the courts with respect to maintenance for children), in subsections (1) and~(2), for “Secretary of State” substitute “Commission”.

\medskip

7.---(1) Section 10 (relationship between maintenance calculations and certain court orders etc.)\ is amended as follows.

(2) In subsection~(4)—
\begin{enumerate}\item[]
($a$) for “Secretary of State” (in the second and third places where it occurs) substitute “Commission”;

($b$) for “he” substitute “it”.
\end{enumerate}

(3) In subsection~(5), for “Secretary of State” substitute “Commission”.

\medskip

8.---(1) Section 11 (maintenance calculations) is amended as follows.

(2) In subsection~(1)—
\begin{enumerate}\item[]
($a$) for “Secretary of State”substitute “Commission”;

($b$) for “him” substitute “it”.
\end{enumerate}

(3) In subsection~(2)—
\begin{enumerate}\item[]
($a$) for “Secretary of State” substitute “Commission”;

($b$) for “he” substitute “it”.
\end{enumerate}

(4) In subsection~(3)—
\begin{enumerate}\item[]
($a$) for “Secretary of State” substitute “Commission”;

($b$) for “he” substitute “the Commission”.
\end{enumerate}

(5) In subsection~(4)—
\begin{enumerate}\item[]
($a$) for “Secretary of State” substitute “Commission”;

($b$) for “he” substitute “the Commission”;

($c$) for “him” substitute “the Commission”.
\end{enumerate}

(6) In subsection~(5)—
\begin{enumerate}\item[]
($a$) for “Secretary of State” substitute “Commission”;

($b$) for “him” substitute “the Commission”.
\end{enumerate}

(7) In subsection~(7)—
\begin{enumerate}\item[]
($a$) for “Secretary of State” substitute “Commission”;

($b$) for “he” substitute “it”.
\end{enumerate}

\medskip

9.---(1) Section 12 (default and interim maintenance decisions) is amended as follows.

(2) In subsection~(1)—
\begin{enumerate}\item[]
($a$) for “Secretary of State” substitute “Commission”;

($b$) for “him”, in the first place where it occurs, substitute “the Commission” and, in the second place where it occurs, substitute “it”;

($c$) for “he” (in both places where it occurs) substitute “it”.
\end{enumerate}

(3) In subsection~(2), for “Secretary of State” substitute “Commission”.

\medskip

10.---(1) Section 14 (information required by Secretary of State) is amended as follows.

(2) In the title, for “Secretary of State” substitute “Commission”.

(3) In subsection~(3), for “him” (in both places where it occurs) substitute “the Commission”.

(4) In subsection~(4), for “Secretary of State” substitute “Commission”.

\medskip

11.---(1) Section 15 (powers of inspectors) is amended as follows.

(2) In subsection~(1)—
\begin{enumerate}\item[]
($a$) for “Secretary of State” substitute “Commission”;

($b$) for “he” substitute “it”.
\end{enumerate}

(3) In subsection~(2), for “Secretary of State” substitute “Commission”.

\medskip

12. In section 16 (revision of decisions)—
\begin{enumerate}\item[]
($a$) for “Secretary of State” (in each place where it occurs) substitute “Commission”;

($b$) for “his” (in each place where it occurs) substitute “its”;

($c$) for “he” (in each place where it occurs) substitute “it”;

($d$) for “him” substitute “it”.
\end{enumerate}

\medskip

13. In section 17 (decisions superseding earlier decisions)—
\begin{enumerate}\item[]
($a$) for “Secretary of State” (in each place where it occurs) substitute “Commission”;

($b$) for “his” (in each place where it occurs) substitute “its”;

($c$) for “him” substitute “it”.
\end{enumerate}

\medskip

14.---(1) Section 20 (appeals to appeal tribunals) is amended as follows.

(2) In subsections (1)($a$) and~($b$), (2)($a$)(i), (7)($b$) and~(8)($b$), for “Secretary of State” substitute “Commission”.

\medskip

15. In section 23A (redetermination of appeals), in subsection~(4), before paragraph ($a$) insert—
\begin{quotation}
“($za$) the Commission;”.
\end{quotation}

\medskip

16.---(1) Section 24 (appeal to Child Support Commissioner) is amended as follows.

(2) For subsection~(1) substitute—
\begin{quotation}
“(1) Each of the following may appeal to a Child Support Commissioner on a question of law—
\begin{enumerate}\item[]
($a$) the Commission,

($b$) the Secretary of State, and

($c$) any person who is aggrieved by the decision of an appeal tribunal.”
\end{enumerate}
\end{quotation}

%(3) In subsection~(3)($c$) and~($d$), before “the Secretary of State” insert “the Commission or”.
%
%(4) In subsection~(4)—
%\begin{enumerate}\item[]
%($a$) before “the Secretary of State” insert “the Commission or”;
%
%($b$) for “to an officer of his, or a person providing him with services,” substitute “to an officer of, or a person providing services to, the Commission or the Secretary of State,”.
%\end{enumerate}
%
%(5) In subsection~(8), before “the Secretary of State” (in both places where it occurs) insert “the Commission or”.

\amendment{
Para. 16(3)--(5), 17 omitted (3.11.08) by the Transfer of Tribunal Functions Order 2008 Sch. 3 para.~228(s).
}
%\medskip
%
%17. In section 25 (appeal from Child Support Commissioner on question of law), in subsection~(3), after paragraph ($a$) insert—
%\begin{quotation}
%“($aa$) the Commission;”.
%\end{quotation}

\medskip

18. In section 26 (disputes about parentage), for “Secretary of State” (in each place where it occurs) substitute “Commission”.

\medskip

19. In section 27 (applications for declaration of parentage) for “Secretary of State” (in each place where it occurs) substitute “Commission”.

\medskip

20. In section 27A (recovery of fees for scientific tests)—
\begin{enumerate}\item[]
($a$) for “Secretary of State” (in each place where it occurs) substitute “Commission”;

($b$) for “him” (in each place where it occurs) substitute “it”.
\end{enumerate}

\medskip

21. In section 28 (power to initiate or defend actions of declarator), for “Secretary of State” (in each place where it occurs, including the title) substitute “Commission”.

\medskip

22. In section 28ZA (decisions involving issues that arise on appeal in other cases)—
\begin{enumerate}\item[]
($a$) for “Secretary of State” (in each place where it occurs) substitute “Commission”;

($b$) for “he” (in each place where it occurs) substitute “it”;

($c$) for “his” substitute “its”.
\end{enumerate}

\medskip

23. In section 28ZB (appeals involving issues that arise on appeal in other cases)—
\begin{enumerate}\item[]
($a$) for “Secretary of State” (in each place where it occurs) substitute “Commission”;

($b$) for “he” (in both places where it occurs) substitute “the Commission”;

($c$) for “him” substitute “the Commission”;

($d$) for “his” substitute “its”.
\end{enumerate}

\medskip

24. In section 28ZC (restrictions on liability in certain cases of error), for “Secretary of State” (in each place where it occurs) substitute “Commission”.

\medskip

25. In section 28A (application for variation of usual rules for calculating maintenance), for “Secretary of State” (in each place where it occurs) substitute “Commission”.

\medskip

26.---(1) Section 28B (preliminary consideration of applications) is amended as follows.

(2) In subsection~(1)—
\begin{enumerate}\item[]
($a$) for “Secretary of State” substitute “Commission”;

($b$) for “he” substitute “it”.
\end{enumerate}

(3) In subsection~(2)—
\begin{enumerate}\item[]
($a$) for “he” (in the first place where it occurs) substitute “the Commission”;

($b$) for “he” (in each other place where it occurs) substitute “it”;

($c$) for “his” (in both places where it occurs) substitute “its”;

($d$) for “him” substitute “the Commission”.
\end{enumerate}

\medskip

27. In section 28C (imposition of regular payments condition), in subsections (1) and~(3) to (7)—
\begin{enumerate}\item[]
($a$) for “Secretary of State” (in each place where it occurs) substitute “Commission”;

($b$) for “he” (in each place where it occurs) substitute “it”;

($c$) for “his” (in each place where it occurs) substitute “its”.
\end{enumerate}

\medskip

28. In section 28D (determination of applications)—
\begin{enumerate}\item[]
($a$) for “Secretary of State” (in each place where it occurs) substitute “Commission”;

($b$) for “he” substitute “it”.
\end{enumerate}

\medskip

29. In section 28E (matters to be taken into account)—
\begin{enumerate}\item[]
($a$) for “Secretary of State” (in each place where it occurs) substitute “Commission”;

($b$) for “him” substitute “it”.
\end{enumerate}

\medskip

30. In section 28F (agreement to variation)—
\begin{enumerate}\item[]
($a$) for “Secretary of State” (in each place where it occurs) substitute “Commission”;

($b$) for “he” (in each place where it occurs) substitute “it”;

($c$) for “his” (in each place where it occurs) substitute “its”.
\end{enumerate}

\medskip

31. In section 28J (voluntary payments), in subsections (1), (2) and~(4)—
\begin{enumerate}\item[]
($a$) for “Secretary of State” (in each place where it occurs) substitute “Commission”;

($b$) for “he” (in each place where it occurs) substitute “it”.
\end{enumerate}

\medskip

32.---(1) Section 29 (collection of child support maintenance) is amended as follows.

(2) In subsection~(1)—
\begin{enumerate}\item[]
($a$) for “Secretary of State” (in both places where it occurs) substitute “Commission”;

($b$) in paragraph ($b$), for “him” substitute “it”.
\end{enumerate}

(3) In subsection~(3), for “Secretary of State” (in each place where it occurs) substitute “Commission”.

\medskip

33.---(1) Section 30 (collection and enforcement of other forms of maintenance) is amended as follows.

(2) In subsections (1), (2) and~(3)—
\begin{enumerate}\item[]
($a$) for “Secretary of State” (in each place where it occurs) substitute “Commission”;

($b$) for “he” (in each place where it occurs) substitute “it”.
\end{enumerate}

(3) In subsection~(4)—
\begin{enumerate}\item[]
($a$) for “him” substitute “the Commission”;

($b$) for “he” (in the second place where it occurs) substitute “it”.
\end{enumerate}

(4) In subsection~(5)—
\begin{enumerate}\item[]
($a$) for “him” (in both places where it occurs) substitute “the Commission”;

($b$) for “he” (in the first place where it occurs) substitute “it”.
\end{enumerate}

\medskip

34. In section~31 (deduction from earnings orders)—
\begin{enumerate}\item[]
($a$) for “Secretary of State” (in each place where it occurs) substitute “Commission”;

($b$) in subsection~(6), for “he” substitute “it”.
\end{enumerate}

\medskip

35. In section~32 (regulations about deduction from earnings orders), in subsections (2) and~(3), for “Secretary of State” (in each place where it occurs) substitute “Commission”.

\medskip

36. In section~33 (liability orders), for “Secretary of State” (in each place where it occurs) substitute “Commission”.

\medskip

37. In section~34 (regulations about liability orders), in subsections (1)($a$) and~($c$) and~(2), for “Secretary of State” substitute “Commission”.

\medskip

38.---(1) Section 35 (enforcement of liability orders by distress) is amended as follows.

(2) In subsection~(1), for “Secretary of State” substitute “Commission”.

(3) In subsection~(3)—
\begin{enumerate}\item[]
($a$) for “Secretary of State” substitute “Commission”;

($b$) for “his” substitute “its”.
\end{enumerate}

\medskip

39. In section~37 (regulations about liability orders), in subsection~(2), for “Secretary of State” (in the second place where it occurs) substitute “Commission”.

\medskip

40. In section~38 (enforcement of liability orders by diligence), subsection~(1) is amended as follows—
\begin{enumerate}\item[]
($a$) in paragraph ($a$), for “Secretary of State” substitute “Commission”%;
%
%($b$) in paragraph ($aa$) (inserted by paragraph 18($a$)(i) of schedule 5 to the Bankruptcy and Diligence etc. (Scotland) Act 2007 (asp 3)), for “Secretary of State” substitute “Commission”;
%
%($c$) in the words at the end, the reference to the Secretary of State (in effect repealed by paragraph 18($a$)(ii) of that schedule) has effect until the coming into force of that paragraph as a reference to the Commission%
.
\end{enumerate}

\amendment{
Para. 40($b$), ($c$) omitted (1.8.12) by the Public Bodies (Child Maintenance and Enforcement Commission: Abolition and Transfer of Functions) Order 2012 Sch. para.~94($a$).

%The reference to ``that schedule'' in para.~40($c$) refers to Schedule 5 to the Bankruptcy and Diligence etc (Scotland) Act 2007 (asp 3).
}

\medskip

41. In section~39A (commitment to prison and disqualification from driving)—
\begin{enumerate}\item[]
($a$) for “Secretary of State” (in both places where it occurs) substitute “Commission”;

($b$) in subsections (1) and~(4), for “he” substitute “it”.
\end{enumerate}

\medskip

42.---(1) Section 40B (disqualification from driving) is amended as follows.

(2) In subsections (5) to (8), for “Secretary of State” (in each place where it occurs) substitute “Commission”.

(3) In subsection~(9)—
\begin{enumerate}\item[]
($a$) for “Secretary of State” substitute “Commission”;

($b$) for “he” substitute “it”.
\end{enumerate}

\medskip

43.---(1) Section 41 (arrears of child support maintenance) is amended as follows.

(2) In subsection~(1), for “Secretary of State” substitute “Commission”.

(3) In subsection~(2)—
\begin{enumerate}\item[]
($a$) for “Secretary of State” substitute “Commission”;

($b$) for “he” (in both places where it occurs) substitute “it”.
\end{enumerate}

(4) In subsection~(6)—
\begin{enumerate}\item[]
($a$) for “Secretary of State” substitute “Commission”;

($b$) for “him” substitute “it”.
\end{enumerate}

44.---(1) Section 41A (penalty payments) is amended as follows.

(2) In subsection~(1), for “him” substitute “the Commission”.

(3) In subsections (2) and~(4), for “Secretary of State” substitute “Commission”.

(4) In subsection~(6)—
\begin{enumerate}\item[]
($a$) for “Secretary of State” substitute “Commission”;

($b$) for “he” substitute “it”.
\end{enumerate}

\medskip

45.---(1) Section 41B (repayment of overpaid child support maintenance) is amended as follows.

(2) In subsection~(1), for “Secretary of State” substitute “Commission”.

(3) In subsection~(1A)—
\begin{enumerate}\item[]
($a$) for “Secretary of State” substitute “Commission”;

($b$) for “him” substitute “it”.
\end{enumerate}

(4) In subsection~(2), for “Secretary of State” (in both places where it occurs) substitute “Commission”.

(5) In subsection~(3)—
\begin{enumerate}\item[]
($a$) for “Secretary of State” substitute “Commission”;

($b$) for “he” substitute “it”;

($c$) for “him” substitute “it”.
\end{enumerate}

(6) In subsections (4), (5) and~(6)($a$), for “Secretary of State” substitute “Commission”.

(7) In subsection~(9)—
\begin{enumerate}\item[]
($a$) for “Secretary of State” substitute “Commission”;

($b$) for “him” substitute “it”.
\end{enumerate}

\medskip

46. In section~44 (jurisdiction), in subsection~(1), for “Secretary of State” substitute “Commission”.

\medskip

47. In section~46 (reduced benefit decisions), in subsection~(3)($a$), for “Secretary of State” substitute “Commission”.

\medskip

48. In section~46A (finality of decisions), in subsection~(1), after “decision of” insert “the Commission,”.

\medskip

49. In section~46B (matters arising as respects decisions), in subsection~(1)($a$), for “Secretary of State” substitute “Commission”.

\medskip

50. In section~48 (right of audience), in subsection~(1), for “Secretary of State” (in both places where it occurs) substitute “Commission”.

\medskip

51. After section 50 insert—
\begin{quotation}
\subsection*{“50A. Use of computers} Any decision falling to be made under or by virtue of this Act by the Commission may be made, not only by a person authorised to exercise the Commission's decision-making function, but also by a computer for whose operation such a person is responsible.”
\end{quotation}

\medskip

52.---(1) Schedule 1 (maintenance calculations) is amended as follows.

(2) In paragraph 7(3), for “Secretary of State” substitute “Commission”.

(3) In paragraph 10(2)—
\begin{enumerate}\item[]
($a$) for “Secretary of State” substitute “Commission”;

($b$) for “his” substitute “its”.
\end{enumerate}

(4) In paragraph 10B($a$), for “Secretary of State” (in both places where it occurs) substitute “Commission”.

(5) In paragraphs 12 and 13, for “Secretary of State” substitute “Commission”.

(6) In paragraph 15—
\begin{enumerate}\item[]
($a$) for “Secretary of State” substitute “Commission”;

($b$) for “he” substitute “it”.
\end{enumerate}

(7) In paragraph 16(10), for “Secretary of State” (in both places where it occurs) substitute “Commission”.

\medskip

53. In paragraph 4 of Schedule 4A, for “Secretary of State” (in both places where it occurs) substitute “Commission”.


%\subsection*{Social Security Act 1998}
%
%54. In paragraph 10 of Schedule 1 to the Social Security Act 1998 (report on the standards achieved in the making of decisions against which an appeal lies to an appeal tribunal), after “Secretary of State” (in the first and second places where it occurs) insert “and the Child Maintenance and Enforcement Commission”. 

%\section[Part II --- Transitional provision and savings]{Part II\\*Transitional provision and savings}
%
%55.---(1) Anything which—
%\begin{enumerate}\item[]
%($a$) relates to any function transferred to the Commission by virtue of section 13, and
%
%($b$) immediately before commencement, is in the process of being done by or in relation to the Secretary of State,
%\end{enumerate}
%may be continued by or in relation to the Commission.
%
%(2) Anything done (or having effect as if done) by or in relation to the Secretary of State before commencement for the purpose of, or in connection with, any function transferred by virtue of section 13 shall, so far as is required for continuing its effect after that time, have effect as if done by or in relation to the Commission.
%
%(3) Any enactment, instrument or other document has effect, so far as necessary for the purposes of or in consequence of the transfer effected by section 13, as if any reference to the Secretary of State were a reference to the Commission.
%
%(4) Nothing in section 13, this Schedule or Schedule 2 shall—
%\begin{enumerate}\item[]
%($a$) affect the validity of anything done by or in relation to the Secretary of State before commencement;
%
%($b$) affect the responsibility of the Secretary of State for anything done or omitted before commencement;
%
%($c$) enable legal proceedings relating to anything done or omitted before commencement to be brought, or continued, against the Commission.
%\end{enumerate}
%
%(5) In this paragraph, “commencement” means the coming into force of section 13.

\amendment{
Para. 54 omitted (3.11.08) by the Transfer of Tribunal Functions Order 2008 Sch. 3 para.~228(s).

\medskip

Para. 55 omitted (1.8.12) by the Public Bodies (Child Maintenance and Enforcement Commission: Abolition and Transfer of Functions) Order 2012 Sch. para.~94($b$).
}

\part[Schedule 4 --- Changes to the calculation of maintenance]{Schedule 4\\*Changes to the calculation of maintenance\\*\emph{2012 scheme only}}

\renewcommand\parthead{--- Schedule 4}

\amendment{
Sch. 4 is in force only for 2012 scheme cases --- see S.I. 2012/3042.
}

\subsection*{\itshape Introductory}

1. Part I of Schedule 1 to the Child Support Act 1991 (calculation of weekly amount of child support maintenance) is amended as follows.

\subsection*{\itshape Calculation by reference to gross weekly income}

2. In Part I (under which the weekly amount of child support maintenance payable is calculated by reference to the non-resident parent's net weekly income), for “net”, in each place where it occurs, substitute “gross”.

\subsection*{\itshape Change to basic rate}

3. For paragraph 2 (basic rate) substitute—
\begin{quotation}
“2.---(1) Subject to sub-paragraph (2), the basic rate is the following percentage of the non-resident parent's gross weekly income—
\begin{enumerate}\item[]
12\% where the non-resident parent has one qualifying child;

16\% where the non-resident parent has two qualifying children;

19\% where the non-resident parent has three or more qualifying children.
\end{enumerate}

(2) If the gross weekly income of the non-resident parent exceeds £800, the basic rate is the aggregate of the amount found by applying sub-paragraph (1) in relation to the first £800 of that income and the following percentage of the remainder—
\begin{enumerate}\item[]
9\% where the non-resident parent has one qualifying child;

12\% where the non-resident parent has two qualifying children;

15\% where the non-resident parent has three or more qualifying children.
\end{enumerate}

(3) If the non-resident parent also has one or more relevant other children, gross weekly income shall be treated for the purposes of sub-paragraphs (1) and~(2) as reduced by the following percentage—
\begin{enumerate}\item[]
12\% where the non-resident parent has one relevant other child;

16\% where the non-resident parent has two relevant other children;

19\% where the non-resident parent has three or more relevant other children.
\end{enumerate}
\end{quotation}

\subsection*{Increase in flat rate and minimum amounts of liability}

4. In the following provisions, for “£5” substitute “£7”---
\begin{enumerate}\item[]
($a$) paragraph 3(3) (minimum amount of liability in the case of reduced rate);

($b$) paragraph 4(1) (amount of flat rate of liability);

($c$) paragraph 7(7) (minimum amount of liability in the case of basic and reduced rates where reduction because of shared care applies).
\end{enumerate}

\subsection*{\itshape\sloppy Applicable rate where non-resident parent party to other maintenance arrangement}

5.---(1) In paragraph 1(1) (under which the weekly rate of child support maintenance is the basic rate unless a reduced rate, a flat rate or a nil rate applies), at the beginning insert “Subject to paragraph 5A,”.

(2) After paragraph 5 insert—
\begin{quotation}
\subsection*{\itshape “Non-resident parent party to other maintenance arrangement}

5A.---(1) This paragraph applies where—
\begin{enumerate}\item[]
($a$) the non-resident parent is a party to a qualifying maintenance arrangement with respect to a child of his who is not a qualifying child, and

($b$) the weekly rate of child support maintenance apart from this paragraph would be the basic rate or a reduced rate or calculated following agreement to a variation where the rate would otherwise be a flat rate or the nil rate.
\end{enumerate}

(2) The weekly rate of child support maintenance is the greater of £7 and the amount found as follows.

(3) First, calculate the amount which would be payable if the non-resident parent's qualifying children also included every child with respect to whom the non-resident parent is a party to a qualifying maintenance arrangement.

(4) Second, divide the amount so calculated by the number of children taken into account for the purposes of the calculation.

(5) Third, multiply the amount so found by the number of children who, for purposes other than the calculation under sub-\hspace{0pt}paragraph (3), are qualifying children of the non-resident parent.

(6) For the purposes of this paragraph, the non-resident parent is a party to a qualifying maintenance arrangement with respect to a child if the non-resident parent is—
\begin{enumerate}\item[]
($a$) liable to pay maintenance or aliment for the child under a maintenance order, or

($b$) a party to an agreement of a prescribed description which provides for the non-resident parent to make payments for the benefit of the child,
\end{enumerate}
and the child is habitually resident in the United Kingdom.”
\end{quotation}

\subsection*{\itshape Shared care}

6. In paragraph 7(2) (circumstances in which decrease for shared care applies in cases where child support maintenance is payable at the basic rate or a reduced rate), for “If the care of a qualifying child is shared” substitute “If the care of a qualifying child is, or is to be, shared”.

\medskip

7. In paragraph 8(2) (circumstances in which decrease for shared care applies in cases where child support maintenance payable at a flat rate), for “If the care of a qualifying child is shared” substitute “If the care of a qualifying child is, or is to be, shared”.

\medskip

8.---(1) In paragraph 9 (regulations about shared care), the existing provision becomes sub-paragraph (1).

(2) In that sub-paragraph, before paragraph ($a$) insert—
\begin{quotation}
“($za$) for how it is to be determined whether the care of a qualifying child is to be shared as mentioned in paragraph 7(2);”.
\end{quotation}

(3) In that sub-paragraph, after paragraph ($b$) insert—
\begin{quotation}
“($ba$) for how it is to be determined how many nights count for those purposes;”.
\end{quotation}

(4) After that sub-paragraph insert—
\begin{quotation}
“(2) Regulations under sub-paragraph (1)($ba$) may include provision enabling the 
%Commission 
Secretary of State
to proceed for a prescribed period on the basis of a prescribed assumption.”
\end{quotation}

\amendment{
Words substituted in para.~8(4) (1.8.12) by the Public Bodies (Child Maintenance and Enforcement Commission: Abolition and Transfer of Functions) Order 2012 Sch. para.~95(2).


}


\subsection*{\itshape Weekly income}

9. In paragraph 10 (which confers power to make regulations about the manner in which weekly income is to be determined), for sub-paragraph (2) substitute—
\begin{quotation}
“(2) The regulations may, in particular—
\begin{enumerate}\item[]
($a$) provide for determination in prescribed circumstances by reference to income of a prescribed description in a prescribed past period;

($b$) provide for the Secretary of State to estimate any income or make an assumption as to any fact where, in the Secretary of State's view, the information at the Secretary of State's disposal is unreliable or insufficient, or relates to an atypical period in the life of the non-resident parent.”
\end{enumerate}
\end{quotation}

\amendment{
Words substituted in para.~9 (1.8.12) by the Public Bodies (Child Maintenance and Enforcement Commission: Abolition and Transfer of Functions) Order 2012 Sch. para.~95(3).


}

\medskip

10. In paragraph 10(3) (under which weekly income over £2,000 is to be ignored for the purposes of Schedule 1), for “£2,000” substitute “£3,000”.

\part[Schedule 5 --- Maintenance calculations: transfer of cases to new rules]{Schedule 5\\*Maintenance calculations: transfer of cases to new rules}

\renewcommand\parthead{--- Schedule 5}

\section*{\itshape Power to require a decision about whether to stay in the statutory scheme}

%1(1)The Commission may require the interested parties in relation to an existing case to choose whether or not to stay in the statutory scheme, so far as future accrual of liability is concerned.
%
%(2)The reference in sub-paragraph (1) to an existing case is to any of the following—
%
%($a$)a maintenance assessment,
%
%($b$)an application for a maintenance assessment,
%
%($c$)a maintenance calculation made under existing rules, and
%
%($d$)an application for a maintenance calculation which will fall to be made under existing rules.
%
%(3)For the purposes of this paragraph, a maintenance calculation is made (or will fall to be made) under existing rules if the amount of the periodical payments required to be paid in accordance with it is (or will be) determined otherwise than in accordance with Part 1 of Schedule 1 to the Child Support Act 1991 (c. 48) as amended by this Act.

\amendment{
Para. 1 is not yet in force.
}

\medskip

2.---(1) The Secretary of State may by regulations make provision about the exercise of the power under paragraph 1(1).

(2) Regulations under sub-paragraph (1) may, in particular—
\begin{enumerate}\item[]
($a$) make provision about timing in relation to exercise of the power;

($b$) make provision for exercise of the power in stages;

($c$) specify principles for determining the order in which particular cases are to be dealt with under the power;

($d$) make provision about procedure in relation to exercise of the power;

($e$) make provision for exercise of the power in accordance with a scheme prepared by the Commission and approved by the Secretary of State.
\end{enumerate}

\amendment{
Para. 2 is in force only for the purpose of making regulations.
}

\medskip

3.---(1) The Secretary of State shall by regulations make such provision as he thinks fit about exercise of the right to make a choice required under paragraph 1(1).

(2) Regulations under sub-paragraph (1) shall, in particular—
\begin{enumerate}\item[]
($a$) make provision about the time within which the choice must be made;

($b$) make provision for a choice to stay in the statutory scheme to be made by means of an application to the Commission for a maintenance calculation;

($c$) make provision about the form and content of any application required by provision under paragraph ($b$).
\end{enumerate}

% Para 3(3) inserted (25.11.13) by 2012 c 5 s 136(2)
(3) The 
%Commission 
Secretary of State  % Words substituted (1.8.12) by SI 2012/2007 Sch para 105(3)
may before accepting an application required by provision under sub-paragraph (2)($b$) invite the applicant to consider with the 
%Commission 
Secretary of State  % Words substituted (1.8.12) by SI 2012/2007 Sch para 105(3)
whether it is possible to make a maintenance agreement (within the meaning of section 9 of the Child Support Act 1991).

\amendment{
Para. 3 is in force only for the purpose of making regulations.

Para. 3(3) inserted (25.11.13) by the Welfare Reform Act 2012 s. 136(2) as amended by the Public Bodies (Child Maintenance and Enforcement Commission: Abolition and Transfer of Functions) Order 2012 Sch. para.~105(3).

\medskip

Para. 4 is not yet in force.
}

%4If, in a particular case, any of the interested parties chooses not to stay in the statutory scheme, that person's choice shall be disregarded if any of the other interested parties chooses to stay in the statutory scheme.

\section*{\itshape Effect on accrual of liability of exercise of power under paragraph 1}

5.---(1) Where the power under paragraph 1(1) is exercised in relation to a maintenance assessment or maintenance calculation, liability under the assessment or calculation shall cease to accrue with effect from such date as may be determined in accordance with regulations made by the Secretary of State.

(2) Where the power under paragraph 1(1) is exercised in relation to an application for a maintenance assessment or maintenance calculation, liability under any assessment or calculation made in response to the application shall accrue only in respect of the period ending with such date as may be determined in accordance with regulations made by the Secretary of State.

\amendment{
Para. 5 is in force only for the purpose of making regulations.
}

\section*{\itshape Additional powers}

6.---(1) The Secretary of State may by regulations make such provision as appears to the Secretary of State to be necessary or expedient for the purposes of, or in connection with, giving effect to a decision not to leave the statutory scheme.

(2) Regulations under sub-paragraph (1) may, in particular—
\begin{enumerate}\item[]
($a$) make provision about procedure in relation to determination of an application made in pursuance of regulations under paragraph 3;

($b$) make provision about the application of the Child Support Act 1991 in relation to a maintenance calculation made in response to such an application;

($c$) prescribe circumstances in which liability under such a maintenance calculation is to be subject to a prescribed adjustment.
\end{enumerate}

(3) The Secretary of State may by regulations make provision enabling the Commission to treat an application of the kind mentioned in paragraph 1(2)($b$) or~($d$) as withdrawn if none of the interested parties chooses to stay in the statutory scheme.

\amendment{
Para. 6 is in force only for the purpose of making regulations.
}

\section*{\itshape Interpretation}

7. In this Schedule—
\begin{enumerate}\item[]
    “interested parties” has such meaning as may be prescribed;

    “maintenance assessment” means an assessment of maintenance made under the Child Support Act 1991;

    “maintenance calculation” means a calculation of maintenance made under that Act;

    “prescribed” means prescribed by regulations made by the Secretary of State;

    “statutory scheme” means the scheme for child support maintenance under that Act. 
\end{enumerate}

\amendment{
Para. 7 is in force only for the purpose of making regulations.

\medskip

Sch. 6 omitted (1.8.12) by the Public Bodies (Child Maintenance and Enforcement Commission: Abolition and Transfer of Functions) Order 2012 Sch. para.~89.

}

%\part[Schedule 6 --- Use of information]{Schedule 6\\*Use of information}
%
%\subsection*{Powers in relation to use of information}
%
%1. Information which is held for the purposes of any functions relating to child support—
%\begin{enumerate}\item[]
%($a$) by the Commission, or
%
%($b$) by a person providing services to the Commission, in connection with the provision of those services,
%\end{enumerate}
%may be used, or supplied to any person providing services to the Commission, for the purposes of, or for any purposes connected with, the exercise of any such functions.
%
%\medskip
%
%2.---(1) This paragraph applies to information which is held for the purposes of functions relating to income tax, contributions, tax credits, child benefit or guardian's allowance—
%\begin{enumerate}\item[]
%($a$) by the Commissioners for Her Majesty's Revenue and Customs, or
%
%($b$) by a person providing services to them, in connection with the provision of those services.
%\end{enumerate}
%
%(2) Information to which this paragraph applies may be supplied—
%\begin{enumerate}\item[]
%($a$) to the Commission, or
%
%($b$) to a person providing services to the Commission,
%\end{enumerate}
%for use for the purposes of functions relating to child support.
%
%(3) In this paragraph, “contributions” means contributions under Part I of the Social Security Contributions and Benefits Act 1992.
%
%\medskip
%
%3.---(1) This paragraph applies to information which is held for the purposes of functions relating to social security or employment or training—
%\begin{enumerate}\item[]
%($a$) by the Secretary of State, or
%
%($b$) by a person providing services to the Secretary of State, in connection with the provision of those services.
%\end{enumerate}
%
%(2) Information to which this paragraph applies may be supplied—
%\begin{enumerate}\item[]
%($a$) to the Commission, or
%
%($b$) to a person providing services to the Commission,
%\end{enumerate}
%for use for the purposes of functions relating to child support.
%
%\medskip
%
%4.---(1) This paragraph applies to information which is held for the purposes of functions relating to social security, child support or employment or training—
%\begin{enumerate}\item[]
%($a$) by the Northern Ireland Department, or
%
%($b$) by a person providing services to that Department, in connection with the provision of those services.
%\end{enumerate}
%
%(2) Information to which this paragraph applies may be supplied—
%\begin{enumerate}\item[]
%($a$) to the Commission, or
%
%($b$) to a person providing services to the Commission,
%\end{enumerate}
%for use for the purposes of functions relating to child support.
%
%\medskip
%
%5.---(1) This paragraph applies to information which is held for the purposes of functions relating to child support—
%\begin{enumerate}\item[]
%($a$) by the Commission, or
%
%($b$) by a person providing services to the Commission, in connection with the provision of those services.
%\end{enumerate}
%
%(2) Information to which this paragraph applies may be supplied—
%\begin{enumerate}\item[]
%($a$) to the Secretary of State, or
%
%($b$) to a person providing services to the Secretary of State,
%\end{enumerate}
%for use for the purposes of functions relating to social security, war pensions, employment or training, private pensions policy or retirement planning.
%
%(3) Information to which this paragraph applies may be supplied—
%\begin{enumerate}\item[]
%($a$) to the Commissioners for Her Majesty's Revenue and Customs, or
%
%($b$) to a person providing services to them,
%\end{enumerate}
%for use for the purposes of any of their functions.
%
%(4) Information to which this paragraph applies may be supplied—
%\begin{enumerate}\item[]
%($a$) to the Northern Ireland Department, or
%
%($b$) to a person providing services to that Department,
%\end{enumerate}
%for use for the purposes of functions relating to social security, child support, employment or training, private pensions policy or retirement planning.
%
%(5) In this paragraph—
%\begin{enumerate}\item[]
%“private pensions policy” means policy relating to occupational pension schemes or personal pension schemes (within the meaning given by section 1 of the Pension Schemes Act 1993);
%
%“retirement planning” means promoting financial planning for retirement;
%
%“war pension” has the meaning given by section 25(4) of the Social Security Act 1989.
%\end{enumerate}
%
%\subsection*{Interpretation}
%
%6. In this Schedule, “Northern Ireland Department” means the Department for Social Development in Northern Ireland or the Department for Employment and Learning in Northern Ireland.

\part[Schedule 7 --- Minor and consequential amendments]{Schedule 7\\*Minor and consequential amendments}

\renewcommand\parthead{--- Schedule 7}

\subsection*{\itshape Child Support Act 1991}

1.---(1) The Child Support Act 1991 is amended as follows.

\emph{
(2) In section 8 (role of the courts with respect to maintenance for children), in subsection~(6)($b$) (which refers to the non-resident parent's net weekly income), for “net” substitute “gross”.}

% see amendments to para.~1(3), (6) (1.8.12) by the Public Bodies (Child Maintenance and Enforcement Commission: Abolition and Transfer of Functions) Order 2012 Sch. para.~97.

(7) In section~30 (collection and enforcement of other forms of maintenance), in subsections (4) and~(5) for “which he is authorised to collect under this section” substitute “for the collection of which he is authorised under this section to make arrangements”.

(9) In section~32(7) (regulations about appeals), after “include” insert “—
\begin{enumerate}\item[]
 ($a$) provision with respect to the period within which a right of appeal under the regulations may be exercised;

($b$)”.
\end{enumerate}

(10) In section~36(1) (enforcement in county courts), for “garnishee proceedings” substitute “a third party debt order”.

(19) In section 50 (unauthorised disclosure of information), in subsection~(1) (offence of unauthorised disclosure), for “this section” substitute “this subsection”.

(20) In that section, after subsection~(1) insert—
\begin{quotation}
“(1A) Subsection (1) applies to employment as—
\begin{enumerate}\item[]
($a$) any clerk to, or other officer of, an appeal tribunal;

($b$) any member of the staff of an appeal tribunal;

($c$) a civil servant in connection with the carrying out of any functions under this Act;

($d$) any member of, or of the staff of, the Commission;

($e$) any person who provides, or is employed in the provision of, services to the Commission,
\end{enumerate}
and to employment of any other kind which is prescribed for the purposes of this subsection.

(1B) Any person who is, or has been, employed in employment to which this subsection applies is guilty of an offence if, without lawful authority, he discloses any information which—
\begin{enumerate}\item[]
($a$) was acquired by him in the course of that employment;

($b$) is information which is, or is derived from, information acquired or held for the purposes of this Act; and

($c$) relates to a particular person.
\end{enumerate}

(1C) Subsection (1B) applies to any employment which—
\begin{enumerate}\item[]
($a$) is not employment to which subsection~(1) applies, and

($b$) is of a kind prescribed for the purposes of this subsection.”
\end{enumerate}
\end{quotation}

(21) In that section, in subsection~(7) (definition of “responsible person”)—
\begin{enumerate}\item[]
($a$) at the end of paragraph ($b$) insert—
\begin{quotation}
“($ba$) the person appointed to chair the Commission;”;
\end{quotation}

($b$) for paragraph ($c$) substitute—
\begin{quotation}
“($c$) any person authorised for the purposes of this subsection by the Lord Chancellor, the Secretary of State or the person appointed to chair the Commission;”.
\end{quotation}
\end{enumerate}

(22) In section 52 (regulations and orders), in subsection~(2)($a$) (regulations subject to affirmative resolution procedure)—
\begin{enumerate}\item[]
($a$) after “30(5A),” insert  “32A to 32C, 32E to 32J,”;

($b$) after “41B(6),” insert  “41E(1)($a$),”.
\end{enumerate}

(23) In that section, for subsection~(2A) substitute—
\begin{quotation}
“(2A) No statutory instrument containing (whether alone or with other provisions)—
\begin{enumerate}\item[]
($a$) the first regulations under section 17(2) to make provision of the kind mentioned in section 17(3)($a$) or~($b$),

($b$) the first regulations under section~39F, 39M(4), 39P, 39Q, 41D(2), 41E(2) or 49A,

($c$) the first regulations under paragraph 5A(6)($b$) of Schedule 1,

($d$) the first regulations under paragraph 9(1)($ba$) of Schedule 1 to make provision of the kind mentioned in sub-paragraph (2) of that paragraph, or

($e$) the first regulations under paragraph 10(1) of Schedule 1 to make provision of the kind mentioned in sub-paragraph (2)($a$) or~($b$) of that paragraph,
\end{enumerate}
shall be made unless a draft of the instrument has been laid before Parliament and approved by a resolution of each House of Parliament.”
\end{quotation}

(24) After that subsection insert—
\begin{quotation}
“(2B) No statutory instrument containing (whether alone or with other provisions) regulations which by virtue of section 51A are to have effect for a limited period shall be made unless a draft of the instrument has been laid before Parliament and approved by a resolution of each House of Parliament.”
\end{quotation}

(25) In section 54 (interpretation), the existing provision becomes subsection~(1), and in that subsection—
\begin{enumerate}\item[]
($a$) at the appropriate places insert—
\begin{quotation}
““charging order” has the same meaning as in section 1 of the Charging Orders Act 1979;”;

““Commission” means the Child Maintenance and Enforcement Commission;”;

““deposit-taker” means a person who, in the course of a business, may lawfully accept deposits in the United Kingdom;”.
\end{quotation}
\end{enumerate}

(26) In that section, after subsection~(1) insert—
\begin{quotation}
“(2) The definition of “deposit-taker” in subsection~(1) is to be read with—
\begin{enumerate}\item[]
($a$) section 22 of the Financial Services and Markets Act 2000;

($b$) any relevant order under that section; and

($c$) Schedule 2 to that Act.”
\end{enumerate}
\end{quotation}

\emph{
(28) In Schedule 1 (maintenance calculations), in paragraph 5($b$) (weekly rate of child support maintenance to be nil if the non-resident parent has a weekly income below £5), for “£5” substitute “£7”.
}

\emph{
(29) In that Schedule, in paragraph 7 (reduction of basic and reduced rates for shared care), for sub-paragraph (1) substitute—
\begin{quotation}
“(1) This paragraph applies where the rate of child support maintenance payable is the basic rate or a reduced rate or is determined under paragraph 5A.”
\end{quotation}
}

(30) In that Schedule, in paragraph 10A(1)($b$) (regulations about amounts set out in Schedule 1)—
\begin{enumerate}\item[]
($a$) after “paragraph” insert  “2(2),”;

($b$) after “5,” insert  “5A(2),”.
\end{enumerate}

(31) In that Schedule, in paragraph 10C(1) (references to “qualifying children” to be read as references to those qualifying children with respect to whom the maintenance calculation falls to be made), at end insert  “or with respect to whom a maintenance calculation in respect of the non-resident parent has effect”.

(32) In that Schedule, for paragraph 14 substitute—
\begin{quotation}
“14. The Secretary of State may by regulations provide—
\begin{enumerate}\item[]
($a$) for two or more applications for maintenance calculations to be treated, in prescribed circumstances, as a single application; and

($b$) for the replacement, in prescribed circumstances, of a maintenance calculation made on the application of one person by a later maintenance calculation made on the application of that or any other person.”
\end{enumerate}
\end{quotation}

(33) Schedule 2 (provision of information to Secretary of State) ceases to have effect.

(34) In the Act as it has effect apart from the Child Support, Pensions and Social Security Act 2000, the following (which relate to section 6 or 46) are repealed—
\begin{enumerate}\item[]
($a$) sections 11(1A) to (1C), 41(4)($c$) and~($d$) and 41A(5)($c$) and~($d$);

($b$) paragraph 16(3) and~(4A)($b$) of Schedule 1;

($c$) in Schedule 4C—
\begin{enumerate}\item[]
(i) in paragraphs 1($a$) and 2(1)($a$), the words “, a reduced benefit direction”;

(ii) in paragraph 3, in sub-paragraph (1)($b$), sub-paragraph (i) and the word “or” at the end of it, and sub-paragraph (3);

(iii) in paragraphs 4(1)($a$)(i) and 6(1)($b$)(ii) and~(iii), the words “, a reduced benefit direction”.
\end{enumerate}
\end{enumerate}

\amendment{Para. 1(3)--(6), (8), (11)--(18), (22)($b$), (23), (24), (25)($b$), (27) is not yet in force.  Para. 1(2), (28), (29) are in force for 2012 scheme cases only.

The definition of ``curfew order'' in para.~1(25)($a$) is not yet in force.

}

\subsection*{\itshape Social Security Administration Act 1992}

2.---(1) The Social Security Administration Act 1992 is amended as follows.

(2) In section 108 (certain maintenance orders to be enforceable by the Secretary of State), for subsection~(8) substitute—
\begin{quotation}
“(8) In this section “maintenance order”—
\begin{enumerate}\item[]
($a$) in England and Wales, means—
\begin{enumerate}\item[]
(i) any order for the making of periodical payments which is, or has at any time been, a maintenance order within the meaning of the Attachment of Earnings Act 1971;

(ii) any order under Part III of the Matrimonial and Family Proceedings Act 1984 (overseas divorce) for the making of periodical payments;

(iii) any order under Schedule 7 to the Civil Partnership Act 2004 for the making of periodical payments;
\end{enumerate}

($b$) in Scotland, means any order, except an order for the payment of a lump sum, falling within the definition of “maintenance order” in section 106 of the Debtors (Scotland) Act 1987, but disregarding paragraph ($h$) (alimentary bond or agreement).”
\end{enumerate}
\end{quotation}

(3) In section 121E (supply of information held by Revenue and Customs to the Secretary of State or the Northern Ireland Department for use for the purposes of functions relating, inter alia, to child support), for subsection~(2) substitute—
\begin{quotation}
“(2) Information to which this section applies may, and subject to subsection~(2A), must if an authorised officer so requires, be supplied—
\begin{enumerate}\item[]
($a$) to the Secretary of State, or

($b$) to a person providing services to the Secretary of State,
\end{enumerate}
for use for the purposes of functions relating to social security, war pensions or employment or training.

(2ZA) Information to which this section applies may, and subject to subsection~(2A), must if an authorised officer so requires, be supplied—
\begin{enumerate}\item[]
($a$) to the Northern Ireland Department, or

($b$) to a person providing services to that Department,
\end{enumerate}
for use for the purposes of functions relating to social security, child support, war pensions or employment or training.”
\end{quotation}

(4) In that section, in subsection~(2A) (exclusion of power to require supply in case of information for use for the purposes of functions relating to employment or training), after “subsection~(2)” insert  “or~(2ZA)”.

(5) In section 121F (supply to Revenue and Customs of information held by Secretary of State or Northern Ireland Department, including information held for the purposes of functions relating to child support), for subsection~(1) substitute—
\begin{quotation}
“(1) This section applies to information which is held for the purposes of functions relating to social security, war pensions or employment or training—
\begin{enumerate}\item[]
($a$) by the Secretary of State, or

($b$) by a person providing services to the Secretary of State, in connection with the provision of those services.
\end{enumerate}

(1A) This section also applies to information which is held for the purposes of functions relating to social security, child support, war pensions or employment or training—
\begin{enumerate}\item[]
($a$) by the Northern Ireland Department, or

($b$) by a person providing services to that Department, in connection with the provision of 
those services.”
\end{enumerate}
\end{quotation}

(6) In section 122 (supply of information held by tax authorities for fraud prevention and verification), in subsection~(3) (prohibition of onward supply by recipient, except in specified circumstances), at the end of paragraph ($c$) insert---
\begin{quotation}
 “or

 ($d$) it is supplied under paragraph 2 of Schedule 6 to the Child Maintenance and Other Payments Act 2008;”.
\end{quotation}

\subsection*{\itshape Social Security Act 1998}

3.---(1) The Social Security Act 1998 is amended as follows.

(2) In section~3 (use of information held by the Secretary of State or the Northern Ireland Department which relates to certain matters), in subsection~(1A) (which lists the matters concerned)—
\begin{enumerate}\item[]
($a$) in paragraph ($a$), the words “, child support” are omitted;

($b$) after that paragraph insert—
\begin{quotation}
“($aa$) child support in Northern Ireland;”.
\end{quotation}
\end{enumerate}

%(3) In section 81 (duty of Secretary of State to report on the standards achieved in the making of decisions from which an appeal lies to an appeal tribunal), after subsection~(1) insert—
%\begin{quotation}
%“(1A) In its application to decisions against which an appeal lies under the Child Support Act 1991 or regulations made under section 6(5) of the Child Maintenance and Other Payments Act 2008, subsection~(1) shall have effect as if the references to the Secretary of State were references to the Child Maintenance and Enforcement Commission.”
%\end{quotation}

\amendment{
Para. 3(3) repealed (8.5.12) by the Welfare Reform Act 2012 (c. 5) Sch. 14 Pt. XIV.
}

\subsection*{\itshape Tax Credits Act 2002}

4.---(1) Schedule 5 to the Tax Credits Act 2002 (use and disclosure of information) is amended as follows.

(2) In paragraph 4 (supply of information held by Revenue and Customs to the Secretary of State or the Northern Ireland Department for use for the purposes of functions relating, inter alia, to child support), for sub-paragraphs (2) and~(3) substitute—
\begin{quotation}
“(2) Information to which this paragraph applies may be supplied—
\begin{enumerate}\item[]
($a$) to the Secretary of State, or

($b$) to a person providing services to the Secretary of State,
\end{enumerate}
for use for the purposes of functions relating to social security or war pensions or for such purposes relating to evaluation or statistical studies as may be prescribed.

(3) An authorised officer may require information to which this paragraph applies to be supplied—
\begin{enumerate}\item[]
($a$) to the Secretary of State, or

($b$) to a person providing services to the Secretary of State,
\end{enumerate}
for use for the purposes of functions relating to social security.

(3A) Information to which this paragraph applies may be supplied—
\begin{enumerate}\item[]
($a$) to the Northern Ireland Department, or

($b$) to a person providing services to the Northern Ireland Department,
\end{enumerate}
for use for the purposes of functions relating to social security, child support or war pensions or for such purposes relating to evaluation or statistical studies as may be prescribed.

(3B) An authorised officer may require information to which this paragraph applies to be supplied—
\begin{enumerate}\item[]
($a$) to the Northern Ireland Department, or

($b$) to a person providing services to the Northern Ireland Department,
\end{enumerate}
for use for the purposes of functions relating to social security or child support.”
\end{quotation}

(3) In that paragraph, in sub-paragraph (4) (definition of “authorised officer”), for “sub-paragraph (3)” substitute “sub-paragraphs (3) and~(3B)”.

(4) In paragraph 6 (supply to Revenue and Customs of information held by Secretary of State or Northern Ireland Department for the purposes of functions including child support), for sub-paragraph (1) substitute—
\begin{quotation}
“(1) This paragraph applies to information which is held for the purposes of functions relating to social security, war pensions or employment or training—
\begin{enumerate}\item[]
($a$) by the Secretary of State, or

($b$) by a person providing services to the Secretary of State, in connection with the provision of those services.
\end{enumerate}

(1A) This paragraph also applies to information which is held for the purposes of functions relating to social security, child support, war pensions or employment or training—
\begin{enumerate}\item[]
($a$) by the Northern Ireland Department or the Department for Employment and Learning in Northern Ireland, or

($b$) by a person providing services to either of those Departments, in connection with the provision of those services.”
\end{enumerate}
\end{quotation}

\amendment{
Paras. 5, 6 are not yet in force.
}

\part[Schedule 8 --- Repeals]{Schedule 8\\*Repeals}

\renewcommand\parthead{--- Schedule 8}

\begin{longtable}{p{137.53691pt}p{228.4536pt}}
%\begin{tabulary}{\linewidth}{JJ}
\hline
\itshape Short title and chapter & \itshape Extent of repeal\\
\hline
\endhead
\hline
\endlastfoot
Child Support Act 1991 (c. 48) & In section~4—\newline
\hspace*{1em}($a$) subsections (9) and~(11);\\
&\hspace*{1em}($b$)
in subsection~(10), paragraph ($b$) and the \hspace*{1em}word “or” immediately before it.\\ 
&Section 6.\\
&In section 7(1), paragraph ($b$) and the word “or” immediately before it. \\
&In section 8(1), the words “(or treated as made)”. \\
&In section 9(6), paragraphs ($a$) and~($b$) and the word “and” immediately preceding them. \\
&Section 11(3) to (5). \\
&In section 12(2), the words from “(or” to “made)”. \\
&\textls[25]{In section 14(1), the words “or treated} as made” and “(or application treated as made)”. \\
&Sections 16(1A)($b$), 17(1)($c$) and 20(1)($c$), (2)($b$) and~(6). \\
&Section 20(1)($e$). \\
&In section 26(1), the words “or treated as made”. \\
&In section 27(1)($a$), the words “(or is treated as having been made)” and “or treated as made”. \\
&In section 27A(1)—\\
&\hspace*{1em}($a$)
in paragraph ($a$), the words “or treated as \hspace*{1em}made”;\\
&\hspace*{1em}($b$)
in paragraph ($b$), the words “or, as the case \hspace*{1em}may be, treated as made”. \\
&In section 28(1)($a$), the words “or treated as made”, in both places. \\
&In section 28ZA(1)($a$), the words from “or with” to “section~46”.\\
&In section 28ZC—\\
&\hspace*{1em}($a$) in subsection~(1)($b$)(i), the words from “or \hspace*{1em}one” to “benefit”;\\
&\hspace*{1em}($b$) in subsection~(3), the words “or the reduc\-\hspace*{1em}tion of a person's benefit”. \\
&In section 28A—\\
&\hspace*{1em}($a$)
in subsection~(1), the words “, or treated \hspace*{1em}as made under section 6,”;\\
&\hspace*{1em}($b$)
in subsection~(3), the words from “(or” to \hspace*{1em}“section 6)”. \\
&In section 28F(4)($a$), the words from “(including” to “made)”\\
&In section 28J(1)($a$), the words from “, or is” to “section 6”. \\
&In section 29(1), paragraph ($a$) and the word “or” at the end of it. \\
&In section~36—\\
%&($a$) in subsection~(1), the words “, if a county court so orders,”;\\
&\hspace*{1em}($b$) subsection~(2). \\
&In section~41(1)($a$), the word “, 6”. \\
&Section 46 and 50(5).\\
&In section 52(2), the words “6(1),” and “, 46”. \\
&\emph{In Schedule 1, in paragraph 9(1)($a$), the words from “or” to the end.}\\
&Schedule 2.\\
&In Schedule 4A, paragraph 5(2).\\
&In Schedule 4B, in paragraph 2(3)—\\
&\hspace*{1em}($a$)
in paragraphs ($a$), ($d$) and~($e$), the words \hspace*{1em}“(or treated as made)”;\\
&\hspace*{1em}($b$)
in paragraph ($c$), the words “(or treated as \hspace*{1em}having been applied for)”. \\
Social Security Administration Act 1992 (c.\ 5) & Section 106(7) and 107.\newline
In section 122(3), the word “or” at the end of paragraph ($b$). \\
Child Support Act 1995 (c. 34)	&In Schedule 3, paragraph 9.\\
Welfare Reform and Pensions Act 1999 (c.\ 30)	&Section 80.\\
Child Support, Pensions and Social Security Act 2000 (c. 19) & Sections 3 and 19.\newline
In Schedule 3, paragraph 11(3)($b$), (4)($a$), (5)($a$), (6), (8), (9), (10)($a$), (11)($a$), (13)($a$) and~($d$) and~(22)($b$).\\
\emph{Civil Partnership Act 2004 (c.\ 33)}&\emph{In Schedule 24, paragraph 3.}\\
\end{longtable}
%\end{tabulary}

\amendment{
Sch. 8 is only partially in force.  

The entries relating to Schedule 1 to the Child Support Act 1991 and to Schedule 24 to the Civil Partnership Act 2004 are in force for 2012 scheme cases only.
}

\begin{center}
\textcopyright\ Crown copyright 2013
\end{center}

\end{document}