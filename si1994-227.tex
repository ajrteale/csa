\documentclass[a4paper]{article}

\usepackage[welsh,english]{babel}

\usepackage[utf8]{inputenc}
\usepackage[T1]{fontenc}

\usepackage{textcomp}

%\usepackage[2012rules]{optional}

\usepackage[osf]{mathpazo}

%\opt{newrules}{
\title{The Child Support (Miscellaneous Amendments and Transitional Provisions) Regulations 1994}
%}

%\opt{2012rules}{
%\title{Child Maintenance and Other Payments Act 2008\\(2012 scheme version)}
%}

\author{S.I. 1994 No. 227}

\date{Made 3rd February 1994\\Coming into force 7th February 1994}

%\opt{oldrules}{\newcommand\versionyear{1993}}
%\opt{newrules}{\newcommand\versionyear{2003}}
%\opt{2012rules}{\newcommand\versionyear{2012}}

\usepackage{fancyhdr}
\pagestyle{fancy}
\fancyhead[L]{}
\fancyhead[C]{\itshape The Child Support (Miscellaneous Amendments and Transitional Provisions) Regulations 1994 (S.I.~1994/227) \parthead%\phantom{...}% (\versionyear{} scheme version)
}
\fancyhead[R]{}
\fancyfoot[C]{\thepage}
\newcommand{\parthead}{}

\usepackage{array}
\usepackage{multirow}
\usepackage[debugshow]{tabulary}
\usepackage{longtable}
\usepackage{multicol}
\usepackage{lettrine}

\usepackage[colorlinks=true]{hyperref}
\usepackage{microtype}

\hyphenation{Aw-dur-dod}
\hyphenation{bank-rupt-cy}
\hyphenation{Ec-cles-ton}
\hyphenation{Eux-ton}
\hyphenation{Hogh-ton}
\hyphenation{Pres-ton}
\hyphenation{Pru-den-tial}
\hyphenation{Riv-ing-ton}

\newcolumntype{x}[1]
	{>{\raggedright}p{#1}}
\newcommand{\tn}{\tabularnewline}
\setlength\tymin{50pt}

\newcommand\amendment[1]{\subsubsection*{Notes}{\itshape\frenchspacing\footnotesize #1 \par}}

\setlength\headheight{22.82501pt}

\usepackage{perpage} %the perpage package
\MakePerPage{footnote} %the perpage package command
\renewcommand{\thefootnote}{\fnsymbol{footnote}}

\usepackage[perpage,para,symbol]{footmisc}

\begin{document}

\maketitle

\noindent
Whereas a draft of this instrument was laid before Parliament in accordance with section 52(2) of the Child Support Act 1991\footnote{\frenchspacing 1991 c. 48.} and approved by a resolution of each House of Parliament:

Now, therefore, the Secretary of State for Social Security, in exercise of the powers conferred by sections 16, 17(6)($b$), 32(2)($c$), 35(2)($b$), 47, 51 and 52(4) of, and paragraphs 1(3), 4(1), 6(6) and 8 of Schedule 1 to, the Child Support Act 1991, and of all other powers enabling him in that behalf, hereby makes the following Regulations:

{\sloppy

\tableofcontents

}

\setcounter{secnumdepth}{-2}

\section[Part I --- General]{Part I\\*General}

\renewcommand\parthead{--- Part I}

\subsection[1. Citation and commencement]{Citation and commencement}

1.  These Regulations may be cited as the Child Support (Miscellaneous Amendments and Transitional Provisions) Regulations 1994 and shall come into force on 7th February 1994.

\section[Part II --- Amendment of regulations]{Part II\\*Amendment of regulations}

\renewcommand\parthead{--- Part II}

\subsection[2. Amendment of the Child Support (Maintenance Assessment Procedure) Regulations 1992]{\sloppy Amendment of the Child Support (Maintenance Assessment Procedure) Regulations 1992}

2.—(1) The Child Support (Maintenance Assessment Procedure) Regulations 1992\footnote{\frenchspacing S.I. 1992/1813. The relevant amending instrument is S.I. 1993/913.} shall be amended in accordance with the following provisions of this regulation.

(2) For sub-paragraph ($c$) of paragraph (4) of regulation 10 of those Regulations there shall be substituted the following sub-paragraphs—
\begin{quotation}
“($c$) where a fresh maintenance assessment is made following a review under section 18 of the Act, sections 16, 17 and 20 of the Act;

($d$) where a fresh maintenance assessment is made following a review under section 19 of the Act, sections 16, 17 and 18 of the Act.”.
\end{quotation}

(3) In paragraph (2) of regulation 20 of those Regulations, for the words after “the provisions of paragraph 6 of Schedule 1 to the Act would apply to that assessment,” there shall be substituted the words—
\begin{quotation}
“he shall not make a fresh assessment if—
\begin{enumerate}\item[]
($a$) where the amount fixed by the original assessment is less than the amount that would be fixed by the fresh assessment, the difference between the two amounts is less than £5.00 a week; and

($b$) where the amount fixed by the original assessment is more than the amount that would be fixed by the fresh assessment, the difference between the two amounts is less than £1.00 a week.”.
\end{enumerate}
\end{quotation}

(4) In paragraph (2) of regulation 21 of those Regulations for the words “that difference is less than £1.00 per week” there shall be substituted the words
\begin{quotation}
“that difference is less than—
\begin{enumerate}\item[]
($a$) where the aggregate amount fixed by the original assessments is less than the aggregate amount that would be fixed by the fresh assessments, £5.00 a week; and

($b$) where the aggregate amount fixed by the original assessments is more than the aggregate amount that would be fixed by the fresh assessments, £1.00 a week.”.
\end{enumerate}
\end{quotation}

(5) At the end of paragraph (1) of regulation 31 of those Regulations there shall be added the words “disregarding any previous assessment made following a review made under section 18 or 19 of the Act”.

\subsection[3. Amendment of the Child Support (Collection and Enforcement) Regulations 1992]{Amendment of the Child Support (Collection and Enforcement) Regulations 1992}

3.—(1) At the beginning of paragraph ($e$) of regulation 9 of the Child Support (Collection and Enforcement) Regulations 1992\footnote{\frenchspacing S.I. 1992/1989. The relevant amending instrument is S.I. 1993/913.} there shall be inserted the words “except in the case of a Category A or Category B interim maintenance assessment within the meaning of regulation 8(1A) and (1B) of the Child Support (Maintenance Assessment Procedure) Regulations 1992,”.

(2) In Schedule 2 to those Regulations—
\begin{enumerate}\item[]
($a$) after head B of the table to paragraph 1 there shall be inserted the following head
\begin{enumerate}\item[]
In column 1—
\begin{quotation}
“BB For preparing and sending a letter advising the liable person that the written authorisation of the Secretary of State is with the person levying the distress and requesting the total sum due:”; and
\end{quotation}

In column 2—
\begin{quotation}
“£10.00.”; and
\end{quotation}
\end{enumerate}

($b$) in head D(ii) of that table, in column 2 for the words “45p per day” there shall be substituted the words “10p per day.”.
\end{enumerate}

\subsection[4. Amendment of the Child Support (Maintenance Assessments and Special Cases) Regulations 1992]{Amendment of the Child Support (Maintenance Assessments and Special Cases) Regulations 1992}

4.—(1) The Child Support (Maintenance Assessments and Special Cases) Regulations 1992\footnote{\frenchspacing S.I. 1992/1815, to which there are amendments not relevant to these Regulations.} shall be amended in accordance with the following provisions of this regulation.

(2) For sub-paragraph ($b$) of paragraph (1) of regulation 3 of those Regulations there shall be substituted the following sub-paragraph—
\begin{quotation}
“($b$) with respect to a person with care of one or more qualifying children—
\begin{enumerate}\item[]
(i) where one or more of those children is aged less than 11, an amount equal to the amount specified in column (2) of paragraph 1(1)($e$) of the relevant Schedule (income support personal allowance for a single claimant aged not less than 25);

(ii) where none of those children are aged less than 11 but one or more of them is aged less than 14, an amount equal to 75 per centum of the amount specified in head (i) above; and

(iii) where none of those children are aged less than 14 but one or more of them is aged less than 16, an amount equal to 50 per centum of the amount specified in head (i) above;”.
\end{enumerate}
\end{quotation}

(3) For paragraph (1) of regulation 6 of those Regulations there shall be substituted the following paragraph—
\begin{quotation}
“(1) For the purposes of the formula in paragraph 4(1) of Schedule 1 to the Act, the value of R is—
\begin{enumerate}\item[]
($a$) where the maintenance assessment in question relates to one qualifying child, 0.15;

($b$) where the maintenance assessment in question relates to two qualifying children, 0.20; and

($c$) where the maintenance assessment in question relates to three or more qualifying children, 0.25.”.
\end{enumerate}
\end{quotation}

(4) In regulation 11(1)($k$) of those Regulations for “£8.00” there shall be substituted “£30.00”.

(5) In regulation 11(1)($l$) of those Regulations for the words “10 per centum” there shall be substituted the words “15 per centum”.

(6) At the beginning of paragraph (2) of regulation 23 of those Regulations there shall be inserted the words “Subject to paragraph (2A)” and after that paragraph there shall be inserted the following paragraph—
\begin{quotation}
“(2A) In applying the provisions of paragraph (2) to the amount which is to be included in the maintenance requirements under regulation 3(1)($b$)—
\begin{enumerate}\item[]
($a$) first take the amount specified in head (i) of regulation 3(1)($b$) and divide it by the relevant number;

($b$) then apply the provisions of regulation 3(1)($b$) as if the references to the amount specified in column (2) of paragraph 1(1)($e$) of the relevant Schedule were references to the amount which is the product of the calculation required by head ($a$) above, and as if, in relation to an absent parent, the only qualifying children to be included in the assessment were those qualifying children in relation to whom he is the absent parent.”.
\end{enumerate}
\end{quotation}

(7) In paragraph (3) of regulation 23 of those Regulations for the words “In paragraph (2)” there shall be substituted the words “In paragraph (2) and (2A)”.

(8) For sub-paragraph (5) of paragraph 3 of Schedule 3 to those Regulations there shall be substituted the following sub-paragraph—
\begin{quotation}
“(5) Where a policy of insurance has been obtained and retained for the purpose of discharging a mortgage or charge on the home of the parent in question and also for the purpose of accruing profits on the maturity of the policy, there shall be eligible to be taken into account as a housing cost—
\begin{enumerate}\item[]
($a$) where the sum secured by the mortgage or charge does not exceed £60,000, the whole of the premiums paid under that policy; and

($b$) where the sum secured by the mortgage or charge exceeds £60,000, the part of the premiums paid under that policy which are necessarily incurred for the purpose of discharging the mortgage or charge or, where that part cannot be ascertained, 0.0277 per centum of the amount secured by the mortgage or charge.”.
\end{enumerate}
\end{quotation}

\subsection[5. Amendment of the Child Support Fees Regulations 1992]{Amendment of the Child Support Fees Regulations 1992}

5.—(1) The Child Support Fees Regulations 1992\footnote{\frenchspacing S.I. 1992/3094.} shall be amended in accordance with the following provisions of this regulation.

(2) In regulation 1(2) of those Regulations, for the definition of “collection fee” there shall be substituted the following definition—
\begin{quotation}
““collection fee” means a fee in respect of services provided by the Secretary of State for the collection of child support maintenance or for enforcing payment of such maintenance or both such collection and such enforcement;”.
\end{quotation}

(3) For paragraph (3) of regulation 3 of those Regulations there shall be substituted the following paragraph—
\begin{quotation}
“(3) In a case falling within paragraph (1)($b$) the fee payable shall be the assessment fee and if, but only if, collection or enforcement services (or both) are provided by the Secretary of State, the collection fee.”.
\end{quotation}

(4) For paragraph (2) of regulation 4 of those Regulations, there shall be substituted the following paragraph—
\begin{quotation}
“(2) Where a collection fee is payable under regulation 3(2) or 3(3) the first such fee shall become payable on the date the Secretary of State first takes action to collect or enforce payment of child support maintenance, and any subsequent fee which becomes so payable shall be payable on the date the assessment fee becomes payable.”.
\end{quotation}

\section[Part III --- Transitional provisions]{Part III\\*Transitional provisions}

\renewcommand\parthead{--- Part III}

6.—(1) In this Part and Part IV of these Regulations—
\begin{enumerate}\item[]
“the Act” means the Child Support Act 1991\footnote{\frenchspacing 1991 c. 48.};

“excess” means the amount by which the formula amount exceeds the old amount;

“existing case” means a case in which before the date when these Regulations come into force, a maintenance assessment has been made which has an effective date which also falls before that date;

“formula amount” means the amount of child support maintenance that would, but for the provisions of this Part of these Regulations, be payable under the maintenance assessment in force on the date these Regulations come into force or, if there is no such assessment, under the first assessment to come into force on or after that date;

“new case” means a case in which the effective date of the maintenance assessment falls on or after the date when these Regulations come into force;

“old amount” means subject to paragraph (2) below, the aggregate weekly amount which was payable under the orders, agreements or arrangements mentioned in regulation 7(1)($a$) below;

“pending case” means a case in which an application for a maintenance assessment has been made before the date when these Regulations come into force but no maintenance assessment has been made before that date;

“Procedure Regulations” means the Child Support (Maintenance Assessment Procedure) Regulations 1992\footnote{\frenchspacing S.I. 1992/1813. The relevant amending instrument is S.I. 1993/913.};

“transitional amount” means an amount determined in accordance with regulation 8 below; and

“transitional period” means a period of, where the formula amount does not exceed £60, 52 weeks, and in any other case 78 weeks, beginning—
\begin{enumerate}\item[]
($a$) in relation to an existing case, with the day that the maintenance assessment in that case is reviewed following an application under regulation 10(1) to (3) below;

($b$) in relation to a new case, the effective date of the maintenance assessment in that case;

($c$) in relation to a pending case, the effective date of the maintenance assessment in that case or the date when these Regulations come into force, whichever is the later.
\end{enumerate}
\end{enumerate}

(2) In determining the old amount the child support officer shall disregard any payments in kind and any payments made to a third party on behalf of or for the benefit of the qualifying child or qualifying children or the person with care.

\subsection[7. Scope of this Part]{Scope of this Part}

7.—(1) Subject to paragraph (2) below, this Part of these Regulations applies to cases where—
\begin{enumerate}\item[]
($a$) on 4th April 1993, and at all times thereafter until the date when a maintenance assessment was or is made under the Act, there was in force, in respect of one or more of the qualifying children in respect of whom an application for a maintenance assessment was or is made under the Act and the absent parent concerned, one or more—
\begin{enumerate}\item[]
(i) maintenance orders;

(ii) orders under section 151 of the Army Act 1955\footnote{\frenchspacing 3 \& 4 Eliz. 2 c. 18.} (deductions from pay for maintenance of wife or child) or section 151 of the Air Force Act 1955\footnote{\frenchspacing 3 \& 4 Eliz. 2 c. 19.} (deduction from pay for maintenance of wife or child) or arrangements corresponding to such an order and made under Article 1($b$) or 3 of the Naval and Marine Pay and Pensions (Deductions for Maintenance) Order 1959\footnote{\frenchspacing This Order in Council is not a statutory instrument but copies may be obtained from the Ministry of Defence, Naval Pay (Pensions and Conditions of Service) Branch, Old Admiralty Building, Spring Gardens, London, SW1A 2BE.}; or

(iii) maintenance agreements (being agreements which are made or evidenced in writing); and
\end{enumerate}

($b$) the absent parent was on the relevant date and continues to be a member of a family, as defined in regulation 1(2) of the Child Support (Maintenance Assessments and Special Cases) Regulations 1992\footnote{\frenchspacing S.I. 1992/1815. The relevant amending instrument is S.I. 1993/913.}, which includes one or more children;

($c$) the formula amount exceeds the old amount.
\end{enumerate}

(2) Nothing in this Part of these Regulations applies to—
\begin{enumerate}\item[]
($a$) a Category A 
or Category D  % Words inserted (18.4.95) by SI 1995/1045 reg 60
interim maintenance assessment within the meaning of 
%regulation 8(1B) 
regulation 8(3)  % Words substituted (22.1.96) by SI 1995/3261 reg 51(2)
of the Procedure Regulations
%\footnote{\frenchspacing Regulation 8(1B) was inserted by regulation 3(2) of S.I. 1993/913.} 
and made under section 12 of the Act; 
%or % Word omitted (22.1.96) by SI 1995/3261 reg 51(2)

($b$) a case falling within the provisions of Part II of the Schedule to the Child Support Act 1991 (Commencement No.\ 3 and Transitional Provisions) Order 1992\footnote{\frenchspacing S.I. 1992/2644. The relevant amending instrument is S.I. 1993/966.} (modification of maintenance assessment in certain cases);
or % Word inserted (22.1.96) by SI 1995/3261 reg 51(3)

% Reg 7(2)($c$) inserted (22.1.96) by SI 1995/3261 reg 51(4)
($c$) a maintenance assessment calculated in accordance with Part I of Schedule 1 to the Act which is made following a Category A or Category D interim maintenance assessment within the meaning of regulation 8 of the Procedure Regulations where that Category A or Category D interim maintenance assessment is made after 22nd January 1996.
\end{enumerate}

(3) In sub-paragraph (1)($b$) above “the relevant date” means—
\begin{enumerate}\item[]
($a$) in an existing case, the date these Regulations come into force;

($b$) in a new case, the effective date of the maintenance assessment in that case; and

($c$) in a pending case, the effective date of the maintenance assessment in that case or the date on which these Regulations come into force, whichever is the later.
\end{enumerate}

\amendment{
Words inserted in reg. 7(2)($a$) (18.4.95) by the Child Support and Income Support (Amendment) Regulations 1995 reg. 60.

Words substituted in reg. 7(2)($a$) and reg. 7(2)($c$) inserted (22.1.96) by the Child Support (Miscellaneous Amendments) (No. 2) Regulations 1995 reg. 51.
}

\subsection[8. Transitional amount of child support maintenance]{Transitional amount of child support maintenance}

8.—(1) In a case to which this Part of these Regulations applies the amount of child support maintenance payable under a maintenance assessment during the transitional period shall, instead of being the formula amount, be the transitional amount.

(2) The transitional amount is—
\begin{enumerate}\item[]
($a$) where the formula amount is not more than £60, an amount which is £20 greater than the old amount;

($b$) where the formula amount is more than £60—
\begin{enumerate}\item[]
(i) during the first 26 weeks of the transitional period, the old amount plus either 25 per centum of the excess or £20.00, whichever is the greater;

(ii) during the next 26 weeks of the transitional period, the old amount plus either 50 per centum of the excess or £40.00, whichever is the greater; and

(iii) during the last 26 weeks of the transitional period, the old amount plus either 75 per centum of the excess or £60.00, whichever is the greater.
\end{enumerate}
\end{enumerate}

(3) If in any case the application of the provisions of this Part of these Regulations would result in an amount of child support maintenance becoming payable which is greater than the formula amount, then those provisions shall not apply or, as the case may be, shall cease to apply to that case and the amount of child support maintenance payable in that case shall be the formula amount.

\section[Part IV --- Procedure etc.]{Part IV\\*Procedure etc.}

\renewcommand\parthead{--- Part IV}

\subsection[9. Interpretation]{Interpretation}

9.  In this Part of these Regulations “the Procedure Regulations” means the Child Support (Maintenance Assessment Procedure) Regulations 1992\footnote{\frenchspacing S.I. 1992/1813. The relevant amending instrument is S.I. 1993/913.}.

\subsection[10. Procedure]{Procedure}

10.—(1) The provisions of Part III of these Regulations shall not apply to a case in which there is a maintenance assessment in force on the date they come into force unless the absent parent in relation to whom that assessment was made makes an application for a review of that assessment under section 17 of the Act.

(2) Such an application must be made not later than 3 months after the date when these Regulations come into force, but if an application is made after that period it may be accepted if the Secretary of State is satisfied that there is good reason for its being made late.

(3) Where a maintenance assessment is reviewed solely because of the coming into force of Part III of these Regulations the provisions of regulations 10(2) and 19 of the Procedure Regulations shall not apply in relation to that review but instead the child support officer shall notify to the relevant persons (as defined in regulation 1(2) of those Regulations) details of how the provisions of Part III of these Regulations have been applied in that case.

%\subsection[11. Reviews on change of circumstances]{Reviews on change of circumstances}

%Heading substituted (18.4.95) by SI 1995/1045 reg 61(2)
\subsection[11. Reviews]{Reviews}

11.—%(1) The provisions of the following paragraphs shall apply where there is a review of a previous assessment under section 17 of the Act (reviews on change of circumstances) at any time when the amount payable under that assessment is the transitional amount.
%
% Reg 11(1) substituted (18.4.95) by SI 1995/1045 reg 61(3)
(1) The provisions of the following paragraphs shall apply where there is a review of a previous assessment under section 17, 18 or 19 of the Act (reviews) at any time when the amount payable under that assessment is 
or was  % Words inserted (22.1.96) by SI 1995/3261 reg 52(2)
the transitional amount.

(2) Where the child support officer determines that, were a fresh assessment to be made as a result of the review, the amount payable under it (disregarding the provisions of Part III of these Regulations) (in this regulation called “the reviewed formula amount”) would be—
\begin{enumerate}\item[]
($a$) more than the formula amount, the amount of child support maintenance payable shall be the transitional amount plus the difference between the formula amount and the reviewed formula amount;

($b$) less than the formula amount but more than the transitional amount, the amount of child support maintenance payable shall be the transitional amount;

($c$) less than the transitional amount, the amount of child support maintenance payable shall be the reviewed formula amount.
\end{enumerate}

(3) The child support officer shall, in determining the reviewed formula amount
on a review under section 17 of the Act%Words inserted (18.4.95) by SI 1995/1045 reg 61(4)
, apply the provisions of regulations 20 to 22 of the Procedure Regulations.

% Reg 11(4) inserted (18.4.95) by SI 1995/1045 reg 61(5)
(4) Where a child support officer makes a fresh maintenance assessment following a review under section 18 or 19 of the Act, the effective date of that fresh maintenance assessment shall be the date prescribed under 
%regulation 31 
regulations 31 to 31C  % Words substituted (22.1.96) by SI 1995/3261 reg 52(3)
of the Maintenance Assessment Procedure Regulations or the first day of the maintenance period following 18th April 1995, whichever is the later.

\amendment{
Words inserted in para. 11(3), para. 11(4) inserted and para. 11(1) and heading substituted (18.4.95) by the Child Support and Income Support (Amendment) Regulations 1995 reg. 61.

Words inserted in para. 11(1) and words substituted in para. 11(4) (22.1.96) by the Child Support (Miscellaneous Amendments) (No. 2) Regulations 1995 reg. 52.
}

\subsection[12. Reviews consequent on the amendments made by Part II]{Reviews consequent on the amendments made by Part II}

12.—(1) Where a child support officer reviews a maintenance assessment in consequence only of the amendments made by Part II of these Regulations he shall not make a fresh assessment if the difference between the amount of child support maintenance fixed by the assessment currently in force and the amount that would be fixed if a fresh assessment were to be made as a result of the review is less than £1.00 a week.

(2) For the purposes of regulation 17(2) (intervals between periodical reviews and notice of a periodical review) and 31 (effective date of maintenance assessments following a review under sections 16 to 19 of the Act) of the Procedure Regulations, a review such as is mentioned in paragraph (1) above shall be disregarded.

(3) Except in relation to the amendment made by regulation 4(8) above, notwithstanding anything in regulation 31 of the Procedure Regulations the effective date of a maintenance assessment such as is mentioned in paragraph (1) above shall be the date when these Regulations come into force.

\subsection[13. Reviews consequent on the provisions of Part III]{Reviews consequent on the provisions of Part III}

13.  For the purposes of regulations 17(1) and 31 of the Procedure Regulations, a review made following an application under regulation 10 above shall be disregarded.

\subsection[14. Notification]{Notification}

14.  Regulations 17(4) to (7) and 19(1) and (2) of the Procedure Regulations shall not apply to a review such as is mentioned in regulations 12(1) and 13 above.

\bigskip

Signed by authority of the Secretary of State for Social Security.

{\raggedleft
\emph{Alistair Burt}\\*Parliamentary Under-Secretary of State,\\*Department of Social Security

}

3rd February 1994

\part{Explanatory Note}

\renewcommand\parthead{--- Explanatory Note}

\subsection*{(This note is not part of the Regulations)}

Part II of these Regulations makes amendments to various regulations concerned with child support maintenance under the Child Support Act 1991. Part III makes transitional provisions and Part IV makes provision for the procedure to be followed in consequence of the other provisions of the Regulations.

  In Part II amendments are made to the Child Support (Maintenance Assessment Procedure) Regulations 1992 to alter the amount by which a fresh assessment must differ from the original assessment before it has effect and to make drafting changes (regulation 2). The Child Support (Collection and Enforcement) Regulations 1992 are amended to exclude interim maintenance assessments from the scope of regulation 9($e$) which requires a deduction from earnings order to state the level of protected earnings and to make fresh provision for the determination of the amounts which may be charged when levying distress (regulation 3). The Child Support (Maintenance Assessments and Special Cases) Regulations 1992 are amended so as to alter the amounts which are to be taken into account in assessing child support maintenance (regulation 4). The Child Support Fees Regulations 1992 are amended to provide that a collection fee is payable in relation to an assessment made under section 6 of the Child Support Act only where the Secretary of State is providing services for the collection or enforcement of payment of child support maintenance (regulation 5).

  Part III of the Regulations makes further provision for those cases where on the coming into force of the Child Support Act there was already in force a maintenance order or agreement. For cases fulfilling specified conditions transitional relief is provided for a period of up to 78 weeks.

  Part IV makes provision about reviews of maintenance assessments to give effect to the other provisions of the Regulations and for notification of such reviews.

  The changes made by these Regulations will involve some cost to businesses in those cases where a deduction from earnings order is in force. It is not possible to quantify this but it is expected to be negligible.


\end{document}