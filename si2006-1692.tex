\documentclass[12pt,a4paper]{article}

\newcommand\regstitle{The Contracting Out (Functions Relating to Child Support) Order 2006}

\newcommand\regsnumber{2006/1692}

%\opt{newrules}{
\title{\regstitle}
%}

%\opt{2012rules}{
%\title{Child Maintenance and Other Payments Act 2008\\(2012 scheme version)}
%}

\author{S.I.\ 2006 No.\ 1692}

\date{Made
27th June 2006\\
%Laid before Parliament
%16th June 2006\\
Coming into force
3rd July 2006
}

%\opt{oldrules}{\newcommand\versionyear{1993}}
%\opt{newrules}{\newcommand\versionyear{2003}}
%\opt{2012rules}{\newcommand\versionyear{2012}}

\usepackage{csa-regs}

\setlength\headheight{42.11603pt}

\hbadness=10000

\begin{document}

\maketitle

\noindent
This Order is made in exercise of the powers conferred by section 69 of the Deregulation and Contracting Out Act 1994\footnote{1994 c.\ 40.}.

A draft of this Order has been laid before Parliament in accordance with section 77(2) of the Deregulation and Contracting Out Act 1994 and approved by a resolution of each House of Parliament.

Accordingly, the Secretary of State for Work and Pensions makes the following Order: 

{\sloppy

\tableofcontents

}

\bigskip

\setcounter{secnumdepth}{-2}

\subsection[1. Citation, commencement and interpretation]{Citation, commencement and interpretation}

1.---(1)  This Order may be cited as the Contracting Out (Functions Relating to Child Support) Order 2006 and shall come into force on 3rd July 2006.

(2) In this Order “the 1991 Act” means the Child Support Act 1991\footnote{1991 c.\ 48. The Child Support Act 1991 has been amended by the Child Support Act 1995 (c.\ 34), Part I of the Social Security Act 1998 (c.\ 14) and Part I of the Child Support, Pensions and Social Security Act 2000 (c.\ 19) (“the 2000 Act”). Much of Part I of the 2000 Act is in force only in relation to certain categories of cases: \emph{see} the Child Support, Pensions and Social Security Act 2000 (Commencement No.\ 12) Order 2003 (S.I.\ 2003/192).}.

\subsection[2. Contracting out of functions relating to child support]{Contracting out of functions relating to child support}

2.---(1)  Subject to paragraph (2), any function of the Secretary of State relating to child support under or by virtue of the provisions of the 1991 Act, may be exercised by, or by employees of, such person (if any) as may be authorised by the Secretary of State.

(2) Paragraph (1) shall not include functions under the following provisions—
\begin{enumerate}\item[]
($a$) section 15 of the 1991 Act (powers of inspectors)\footnote{Section 15(1)--(4) and (4A) were substituted by section 14(1) and (2), and section 15(6) was amended by section 14(1) and (3), of the 2000 Act. Section 15(7) was amended by section 254(1) of, and paragraph 2 of Schedule 24 to, the Civil Partnership Act 2004 (c.\ 33). Section 15(10) was amended by Schedule 9 to, and section 15(11) was inserted by section 14(1) and (4) of, the 2000 Act.};

($b$) section 35 of the 1991 Act (enforcement of liability orders by distress);

($c$) section 39A of the 1991 Act (commitment to prison and disqualification from driving)\footnote{Section 39A was inserted by section 16(1) of the 2000 Act.}; and

($d$) section 40B of the 1991 Act (disqualification from driving: further provision)\footnote{Section 40B was inserted by section 16(3) of the 2000 Act.},
\end{enumerate}
or any other functions excluded from section 69 of the Deregulation and Contracting Out Act 1994 by section 71 of that Act. 

\bigskip

Signed 
by authority of the 
Secretary of State for Work and Pensions.

{\raggedleft
\emph{Philip Hunt}\\*Parliamentary Under-Secretary of State,\\*Department for Work and Pensions

}

27th June 2006

\small

\part{Explanatory Note}

\renewcommand\parthead{— Explanatory Note}

\subsection*{(This note is not part of the Order)}

This Order makes provision to enable the Secretary of State to authorise another person to exercise functions relating to child support other than certain functions which are excluded from contracting out by section 71 of the Deregulation and Contracting Out Act 1994 such as rights of entry and seizure, and the power to make subordinate legislation.

A full regulatory impact assessment has not been produced for this instrument as it has no impact on the costs of business, charities or voluntary bodies. 

\end{document}