\documentclass[a4paper]{article}

\usepackage[welsh,english]{babel}

\usepackage[utf8]{inputenc}
\usepackage[T1]{fontenc}

\usepackage{textcomp}

%\usepackage[2012rules]{optional}

\usepackage[osf]{mathpazo}
\usepackage{cfr-lm}

\usepackage{perpage} %the perpage package
\MakePerPage{footnote} %the perpage package command
\renewcommand{\thefootnote}{\fnsymbol{footnote}}

\usepackage[perpage,para,symbol]{footmisc}
%\opt{newrules}{
\title{The Child Support (Maintenance Arrangements and Jurisdiction) Regulations 1992}
%}

%\opt{2012rules}{
%\title{Child Maintenance and Other Payments Act 2008\\(2012 scheme version)}
%}

\author{S.I. 1992 No. 2645}

\date{Made 26th October 1992\\Laid before Parliament 29th October 1992\\Coming into force 5th April 1993}

%\opt{oldrules}{\newcommand\versionyear{1993}}
%\opt{newrules}{\newcommand\versionyear{2003}}
%\opt{2012rules}{\newcommand\versionyear{2012}}

\usepackage{fancyhdr}
\pagestyle{fancy}
\fancyhead[L]{}
\fancyhead[C]{\itshape The Child Support (Maintenance Arrangements and Jurisdiction) Regulations 1992 (S.I.~1992/2645) \parthead%\phantom{...}% (\versionyear{} scheme version)
}
\fancyhead[R]{}
\fancyfoot[C]{\thepage}
\newcommand{\parthead}{}

\usepackage{array}
\usepackage{multirow}
\usepackage[debugshow]{tabulary}
\usepackage{longtable}
\usepackage{multicol}
\usepackage{lettrine}

\usepackage[colorlinks=true]{hyperref}
\usepackage{microtype}

\hyphenation{Aw-dur-dod}
\hyphenation{bank-rupt-cy}
\hyphenation{Ec-cles-ton}
\hyphenation{Eux-ton}
\hyphenation{Hogh-ton}
\hyphenation{Pres-ton}
\hyphenation{Pru-den-tial}
\hyphenation{Riv-ing-ton}

\newcolumntype{x}[1]
	{>{\raggedright}p{#1}}
\newcommand{\tn}{\tabularnewline}
\setlength\tymin{50pt}

\newcommand\amendment[1]{\subsubsection*{Notes}{\itshape\frenchspacing\footnotesize #1 \par}}

\setlength\headheight{22.91502pt}

\begin{document}

\maketitle

\noindent
The Secretary of State for Social Security, in exercise of the powers conferred upon him by sections 8(11), 10(1), (2) and (4), 44(3), 51, 52(4) and 54 of, and paragraph 11 of Schedule 1 to, the Child Support Act 1991\footnote{\frenchspacing 1991 c. 48. Section 54 is cited because of the meaning ascribed to the word “prescribed”.} and of all other powers enabling him in that behalf hereby makes the following Regulations:

{\sloppy

\tableofcontents

}

\setcounter{secnumdepth}{-2}

\subsection[1. Citation, commencement and interpretation]{Citation, commencement and interpretation}

1.—(1) These Regulations may be cited as the Child Support (Maintenance Arrangements and Jurisdiction) Regulations 1992 and shall come into force on 5th April 1993.

(2) In these Regulations—
\begin{enumerate}\item[]
“the Act” means the Child Support Act 1991;

%Definition of ``Maintenance Assessment Procedure Regulations'' inserted (18.4.95) by SI 1995/1045 reg 25
“Maintenance Assessment Procedure Regulations” means the Child Support (Maintenance Assessment Procedure) Regulations 1992\footnote{\frenchspacing S.I. 1992/1813. Regulation 5 was amended by S.I. 1993/913.};

“Maintenance Assessments and Special Cases Regulations” means the Child Support (Maintenance Assessments and Special Cases) Regulations 1992\footnote{\frenchspacing S.I. 1992/1815.};

“effective date” means the date on which a maintenance assessment takes effect for the purposes of the Act;

“maintenance order” has the meaning given in section 8(11) of the Act.
\end{enumerate}

(3) In these Regulations, unless the context otherwise requires, a reference—
\begin{enumerate}\item[]
($a$) to a numbered regulation is to the regulation in these Regulations bearing that number;

($b$) in a regulation to a numbered paragraph is to the paragraph in that regulation bearing that number;

($c$) in a paragraph to a lettered or numbered sub-paragraph is to the sub-paragraph in that paragraph bearing that letter or number.
\end{enumerate}

\amendment{
Definition of ``Maintenance Assessment Procedure Regulations'' inserted in reg. 1(2) (18.4.95) by the Child Support and Income Support (Amendment) Regulations 1995 reg. 25.
}

\subsection[2. Prescription of enactment for the purposes of section 8(11) of the Act]{Prescription of enactment for the purposes of section 8(11) of the Act}

%2.  The Affiliation Proceedings Act 1957\footnote{\frenchspacing 5 \& 6 Eliz. 2 c.55.} is prescribed for the purposes of section 8(11) of the Act.

%Reg 2 substituted (18.4.95) by SI 1995/1045 reg 26
2.  The following enactments are prescribed for the purposes of section 8(11)($f$) of the Act—
\begin{enumerate}\item[]
($a$) the Conjugal Rights (Scotland) Amendment Act 1861\footnote{\frenchspacing 24 \& 25 Vict. c. 86.};

($b$) the Court of Session Act 1868\footnote{\frenchspacing 31 \& 32 Vict. c. 100.};

($c$) the Sheriff Courts (Scotland) Act 1907\footnote{\frenchspacing 7 Edw. 7 c. 51.};

($d$) the Guardianship of Infants Act 1925\footnote{\frenchspacing 15 \& 16 Geo. 5 c. 45.};

($e$) the Illegitimate Children (Scotland) Act 1930\footnote{\frenchspacing 20 \& 21 Geo. 5 c. 33.};

($f$) the Children and Young Persons (Scotland) Act 1932\footnote{\frenchspacing 22 \& 23 Geo. 5 c. 47.};

($g$) the Children and Young Persons (Scotland) Act 1937\footnote{\frenchspacing 1 Edw. 8 \& 1 Geo. 6 c. 37.};

($h$) the Custody of Children (Scotland) Act 1939\footnote{\frenchspacing 2 \& 3 Geo. 6 c. 4.};

($i$) the National Assistance Act 1948\footnote{\frenchspacing 11 \& 12 Geo. 6 c. 29.};

($j$) the Affiliation Orders Act 1952\footnote{\frenchspacing 15 \& 16 Geo. 6 \& 1 Eliz. 2 c. 41.};

($k$) the Affiliation Proceedings Act 1957\footnote{\frenchspacing 5 \& 6 Eliz. 2 c. 55.};

($l$) the Matrimonial Proceedings (Children) Act 1958\footnote{\frenchspacing 6 \& 7 Eliz. 2 c. 40.};

($m$) the Guardianship of Minors Act 1971\footnote{\frenchspacing 1971 c. 3.};

($n$) the Guardianship Act 1973\footnote{\frenchspacing 1973 c. 29.};

($o$) the Children Act 1975\footnote{\frenchspacing 1975 c. 72.};

($p$) the Supplementary Benefits Act 1976\footnote{\frenchspacing 1976 c. 71.};

($q$) the Social Security Act 1986\footnote{\frenchspacing 1986 c. 50.};

($r$) the Social Security Administration Act 1992\footnote{\frenchspacing 1992 c. 5.}.
\end{enumerate}

\amendment{
Reg. 2 substituted (18.4.95) by the Child Support and Income Support (Amendment) Regulations 1995 reg. 26.
}

\subsection[3. Relationship between maintenance assessments and certain court orders]{\sloppy Relationship between maintenance assessments and certain court orders}

3.—%(1) Orders made under the following enactments are of a kind prescribed for the purposes of section 10(1) of the Act—
%\begin{enumerate}\item[]
%($a$) the Affiliation Proceedings Act 1957;
%
%($b$) Part II of the Matrimonial Causes Act 1973\footnote{\frenchspacing 1973 c. 18.};
%
%($c$) the Domestic Proceedings and Magistrates' Courts Act 1978\footnote{\frenchspacing 1978 c. 22.};
%
%($d$) Part III of the Matrimonial and Family Proceedings Act 1984\footnote{\frenchspacing 1984 c. 42.};
%
%($e$) the Family Law (Scotland) Act 1985\footnote{\frenchspacing 1985 c. 37.};
%
%($f$) Schedule 1 to the Children Act 1989\footnote{\frenchspacing 1989 c. 41.}.
%\end{enumerate}
%
% Reg 3(1) substituted (18.4.95) by SI 1995/1045 reg 27(2)
(1) Orders made under the following enactments are of a kind prescribed for the purposes of section 10(1) of the Act—
\begin{enumerate}\item[]
($a$) the Conjugal Rights (Scotland) Amendment Act 1861;

($b$) the Court of Session Act 1868;

($c$) the Sheriff Courts (Scotland) Act 1907;

($d$) the Guardianship of Infants Act 1925;

($e$) the Illegitimate Children (Scotland) Act 1930;

($f$) the Children and Young Persons (Scotland) Act 1932;

($g$) the Children and Young Persons (Scotland) Act 1937;

($h$) the Custody of Children (Scotland) Act 1939;

($i$) the National Assistance Act 1948;

($j$) the Affiliation Orders Act 1952;

($k$) the Affiliation Proceedings Act 1957;

($l$) the Matrimonial Proceedings (Children) Act 1958;

($m$) the Guardianship of Minors Act 1971;

($n$) the Guardianship Act 1973;

($o$) Part II of the Matrimonial Causes Act 1973\footnote{\frenchspacing 1973 c. 18.};

($p$) the Children Act 1975;

($q$) the Supplementary Benefits Act 1976;

\begin{sloppypar}
($r$) the Domestic Proceedings and Magistrates Courts Act 1978\footnote{\frenchspacing 1978 c. 22.};
\end{sloppypar}

($s$) Part III of the Matrimonial and Family Proceedings Act 1984\footnote{\frenchspacing 1984 c. 42.};

($t$) the Family Law (Scotland) Act 1985\footnote{\frenchspacing 1985 c. 37.};

($u$) the Social Security Act 1986;

($v$) Schedule 1 to the Children Act 1989\footnote{\frenchspacing 1989 c. 41.};

($w$) the Social Security Administration Act 1992.
\end{enumerate}

(2) Subject to paragraphs (3) and (4), where a maintenance assessment is made with respect to—
\begin{enumerate}\item[]
($a$) all of the children with respect to whom an order falling within paragraph (1) is in force; or

($b$) one or more but not all of the children with respect to whom an order falling within paragraph (1) is in force and where the amount payable under the order to or for the benefit of each child is separately specified,
\end{enumerate}
that order shall, so far as it relates to the making or securing of periodical payments to or for the benefit of the children with respect to whom the maintenance assessment has been made, cease to have effect.

(3) The provisions of paragraph (2) shall not apply where a maintenance order has been made in accordance with section 8(7) or (8) of the Act.

(4) In Scotland, where—
\begin{enumerate}\item[]
($a$) an order has ceased to have effect by virtue of the provisions of paragraph (2) to the extent specified in that paragraph; and

($b$) a child support officer no longer has jurisdiction to make a maintenance assessment with respect to a child with respect to whom the order ceased to have effect,
\end{enumerate}
that order shall, so far as it relates to that child, again have effect from the date a child support officer no longer has jurisdiction to make a maintenance assessment with respect to that child.

(5) 
Subject to regulation 33(7) of the Maintenance Assessment Procedure Regulations,  % Words inserted (22.1.96) by SI 1995/3261 reg 13
where a maintenance assessment is made with respect to children with respect to whom an order falling within paragraph (1) is in force, the effective date of that assessment shall be two days after the assessment is made.

(6) Where the provisions of paragraph (2) apply to an order, that part of the order to which those provisions apply shall cease to have effect from the effective date of the maintenance assessment.

%Reg 3(7) inserted (16.2.95) by SI 1995/123 reg 3
(7) Where at the time an interim maintenance assessment was made there was in force with respect to children in respect of whom that interim maintenance assessment was made an order falling within paragraph (1), the effective date of a maintenance assessment subsequently made in accordance with Part I of Schedule 1 to the Act in respect of those children shall be the effective date of that interim maintenance assessment as determined under paragraph (5).

%Reg 3(8) inserted (18.4.95) by SI 1995/1045 reg 27(3)
(8) 
Subject to regulation 33(7) of the Maintenance Assessment Procedure Regulations,  % Words inserted (22.1.96) by SI 1995/3261 reg 13
where—
\begin{enumerate}\item[]
($a$) a maintenance assessment is made in accordance with Part I of Schedule 1 to the Act in respect of children with respect to whom an order falling within paragraph (1) was in force; and

($b$) that order ceases to have effect on or after 18th April 1995, for reasons other than the making of an interim maintenance assessment, but prior to the date on which the maintenance assessment is made and after—
\begin{enumerate}\item[]
(i) the date on which a maintenance enquiry form referred to in regulation 5(2) of the Maintenance Assessment Procedure Regulations was given or sent to the absent parent, where the application for a maintenance assessment was made by a person with care or a child under section 7 of the Act; or

(ii) the date on which a maintenance application which complies with the provisions of regulation 2 of the Maintenance Assessment Procedure Regulations was received by the Secretary of State from an absent parent,
\end{enumerate}
\end{enumerate}
the effective date of that maintenance assessment shall be the day following that on which the court order ceased to have effect.

\amendment{
Reg. 3(7) inserted (16.2.95) by the Child Support (Miscellaneous Amendments) Regulations 1995 reg. 3.

Reg. 3(1) substituted and reg. 3(8) inserted (18.4.95) by the Child Support and Income Support (Amendment) Regulations 1995 reg. 27.

Words inserted in reg. 3(5), (8) (22.1.96) by the Child Support (Miscellaneous Amendments) (No. 2) Regulations 1995 reg. 13.
}

\subsection[4. Relationship between maintenance assessments and certain agreements]{\sloppy Relationship between maintenance assessments and certain agreements}

4.—(1) Maintenance agreements within the meaning of section 9(1) of the Act are agreements of a kind prescribed for the purposes of section 10(2) of the Act.

(2) Where a maintenance assessment is made with respect to—
\begin{enumerate}\item[]
($a$) all of the children with respect to whom an agreement falling within paragraph (1) is in force; or

($b$) one or more but not all of the children with respect to whom an agreement falling within paragraph (1) is in force and where the amount payable under the agreement to or for the benefit of each child is separately specified,
\end{enumerate}
that agreement shall, so far as it relates to the making or securing of periodical payments to or for the benefit of the children with respect to whom the maintenance assessment has been made, become unenforceable from the effective date of the assessment.

(3) Where an agreement becomes unenforceable under the provisions of paragraph (2) to the extent specified in that paragraph, it shall remain unenforceable in relation to a particular child until such date as a child support officer no longer has jurisdiction to make a maintenance assessment with respect to that child.

\subsection[5. Notifications by child support officers]{Notifications by child support officers}

5.—(1) Where a child support officer is aware that an order of a kind prescribed in paragraph (2) is in force and considers that the making of a maintenance assessment has affected, or is likely to affect, that order, he shall notify the persons prescribed in paragraph (3) in respect of whom that maintenance assessment is in force, and the persons prescribed in paragraph (4) holding office in the court where the order in question was made or subsequently registered, of the assessment and its effective date.

(2) The prescribed orders are those made under an enactment mentioned in regulation 3(1).

(3) The prescribed persons in respect of whom the maintenance assessment is in force are—
\begin{enumerate}\item[]
($a$) a person with care;

($b$) an absent parent;

($c$) a person who is treated as an absent parent under regulation 20 of the Maintenance Assessments and Special Cases Regulations;

($d$) a child who has made an application for a maintenance assessment under section 7 of the Act.
\end{enumerate}

(4) The prescribed person holding office in the court where the order in question was made or subsequently registered is—
\begin{enumerate}\item[]
($a$) in England and Wales—
\begin{enumerate}\item[]
(i) in relation to the High Court, the senior district judge of the principal registry of the Family Division or, where proceedings were instituted in a district registry, the district judge;

(ii) in relation to a county court, the proper officer of that court within the meaning of Order 1, Rule 3 of the County Court Rules 1981\footnote{\frenchspacing S.I. 1981/1687, to which there are amendments not relevant to these Regulations.};

(iii) in relation to a magistrates' court, the clerk to the justices of that court;
\end{enumerate}

($b$) in Scotland—
\begin{enumerate}\item[]
(i) in relation to the Court of Session, the Deputy Principal Clerk of Session;

(ii) in relation to a sheriff court, the sheriff clerk.
\end{enumerate}
\end{enumerate}

\subsection[6. Notification by the court]{Notification by the court}

6.—(1) Where a court is aware that a maintenance assessment is in force and makes an order mentioned in regulation 3(1) which it considers has affected, or is likely to affect, that assessment, the person prescribed in paragraph (2) shall notify the Secretary of State to that effect.

(2) The prescribed person is the person holding the office specified below in the court where the order in question was made or subsequently registered—
\begin{enumerate}\item[]
($a$) in England and Wales—
\begin{enumerate}\item[]
(i) in relation to the High Court, the senior district judge of the principal registry of the Family Division or, where proceedings were instituted in a district registry, the district judge;

(ii) in relation to a county court, the proper officer of that court within the meaning of Order 1, Rule 3 of the County Court Rules 1981;

(iii) in relation to a magistrates' court, the clerk to the justices of that court;
\end{enumerate}

($b$) in Scotland—
\begin{enumerate}\item[]
(i) in relation to the Court of Session, the Deputy Principal Clerk of Session;

(ii) in relation to a sheriff court, the sheriff clerk.
\end{enumerate}
\end{enumerate}

\subsection[7. Cancellation of a maintenance assessment on grounds of lack of jurisdiction]{Cancellation of a maintenance assessment on grounds of lack of jurisdiction}

7.—(1) Where—
\begin{enumerate}\item[]
($a$) a person with care;

($b$) an absent parent; or

($c$) a qualifying child,
\end{enumerate}
with respect to whom a maintenance assessment is in force ceases to be habitually resident in the United Kingdom, a child support officer shall cancel that assessment.

(2) Where the person with care is not an individual, paragraph (1) shall apply as if sub-paragraph ($a$) were omitted.

(3) Where a child support officer cancels a maintenance assessment under paragraph (1) or by virtue of paragraph (2), the assessment shall cease to have effect from the date that the child support officer determines is the date on which—
\begin{enumerate}\item[]
($a$) where paragraph (1) applies, the person with care, absent parent or qualifying child; or

($b$) where paragraph (2) applies, the absent parent or qualifying child
with respect to whom the assessment was made ceases to be habitually resident in the United Kingdom.
\end{enumerate}

%Reg 7(4) inserted (5.4.93) by SI 1993/913 reg 45
(4) Where a parent is treated as an absent parent for the purposes of the Act and of the Maintenance Assessments and Special Cases Regulations by virtue of regulation 20 of those Regulations, he shall be treated as an absent parent for the purposes of paragraphs (1) to (3).

\amendment{
Reg. 7(4) inserted (5.4.93) by the Child Support (Miscellaneous Amendments) Regulations 1993 reg. 45.
}

\subsection[8. Maintenance assessments and maintenance orders made in error]{Maintenance assessments and maintenance orders made in error}

8.—(1) Where—
\begin{enumerate}\item[]
($a$) at the time that a mainenance assessment with respect to a qualifying child was made a maintenance order was in force with respect to that child;

%Reg 8(1)(aa) inserted (5.4.93) by SI 1993/913 reg 46($a$)
(aa) the maintenance order has ceased to have effect by virtue of the provisions of regulation 3;

($b$) the absent parent has made payments of child support maintenance due under that assessment; and

($c$) the child support officer cancels that assessment on the grounds that it was made in error,
\end{enumerate}
the payments of child support maintenance shall be treated as payments under the maintenance order and that order shall be treated as having continued in force.

(2) Where—
\begin{enumerate}\item[]
($a$) at the time that a maintenance order with respect to a qualifying child was made a maintenance assessment was in force with respect to that child;

%Reg 8(2)(aa) inserted (5.4.93) by SI 1993/913 reg 46($b$)(i)
(aa) the maintenance assessment is cancelled or ceases to have effect;

($b$) the absent parent has made payments of maintenance due under that order; and

($c$) the maintenance order is revoked by the court on the grounds that it was made in error,
\end{enumerate}
the payments under the maintenance order shall be treated as payments of child support maintenance and the maintenance assessment shall be treated as not having been cancelled
or, as the case may be, as not having ceased to have effect. % Words inserted (5.4.93) by SI 1993/913 reg 46($b$)(ii)

\amendment{
Words inserted in reg. 8(2) and reg. 8(1)(aa), (2)(aa) inserted (5.4.93) by the Child Support (Miscellaneous Amendments) Regulations 1993 reg. 46.
}

% Reg 9 inserted (22.1.96) by SI 1995/3261 reg 14
\subsection[9. Cases in which application may be made under section 4 or 7 of the Act]{Cases in which application may be made under section 4 or 7 of the Act}

9.  The provisions of section 4(10) or 7(10) of the Act\footnote{\frenchspacing Sections 4(10) and 7(10) were inserted by section 18(7) of the Child Support Act 1995.} shall not apply to prevent an application being made under those sections after 22nd January 1996 where a decision has been made by the relevant court either that it has no power to vary or that it has no power to enforce a maintenance order in a particular case.

\amendment{
Reg. 9 inserted (22.1.96) by the Child Support (Miscellaneous Amendments) (No. 2) Regulations 1995 reg. 14.
}

\bigskip

Signed by authority of the Secretary of State for Social Security.

{\raggedleft
\emph{Alistair Burt}\\*Parliamentary Under-Secretary of State,\\*Department of Social Security

}

26th October 1992

\part{Explanatory Note}

\renewcommand\parthead{--- Explanatory Note}

\subsection*{(This note is not part of the Regulations)}

 These Regulations make provision as to the effect that making a maintenance assessment under the Child Support Act 1991 (“the Act”) has on a maintenance order or a maintenance agreement, the cancellation of a maintenance assessment where a child support officer no longer has jurisdiction to make an assessment under the provisions of section 44 of the Act (which imposes conditions as to habitual residence in the United Kingdom), and related matters.

  Regulation 1 contains interpretation provisions.

  Regulation 2 prescribes the repealed Affiliation Proceedings Act 1957 for the purposes of section 8(11) of the Act (definition of “maintenance order”) to cover orders which continue to have effect under the 1957 Act.

  Regulations 3 and 4 provide for some prescribed orders and agreements ceasing to have effect where a maintenance assessment is made.

  Regulation 5 provides, in certain cases where a maintenance assessment has been made, for a child support officer to notify the court. Regulation 6 provides, in certain cases where a maintenance order has been made, for the court to notify the Secretary of State.

  Regulation 7 provides for the cancellation of a maintenance assessment where a child support officer no longer has jurisdiction to make an assessment by virtue of the provisions of section 44 of the Act.

  Regulation 8 provides for treating payments of child support maintenance as payments under a maintenance order, and vice versa, where an error has been made.

\end{document}