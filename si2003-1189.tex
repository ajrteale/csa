\documentclass[12pt,a4paper]{article}

\newcommand\regstitle{The Social Security and Child Support (Miscellaneous Amendments) (No.\ 2) Regulations 2003}

\newcommand\regsnumber{2003/1189}

%\opt{newrules}{
\title{\regstitle}
%}

%\opt{2012rules}{
%\title{Child Maintenance and Other Payments Act 2008\\(2012 scheme version)}
%}

\author{S.I.\ 2003 No.\ 1189}

\date{Made
29th April 2003\\
Laid before Parliament
2nd May 2003\\
Coming into force
4th May 2003
}

%\opt{oldrules}{\newcommand\versionyear{1993}}
%\opt{newrules}{\newcommand\versionyear{2003}}
%\opt{2012rules}{\newcommand\versionyear{2012}}

\usepackage{csa-regs}

\setlength\headheight{27.57402pt}

\begin{document}

\maketitle

\begin{center}
\itshape 
This statutory instrument has been made in consequence of a defect in S.I.\ 2003/1050 and is being issued free of charge to all known recipients of that Statutory Instrument. 
\end{center}

\noindent
The Secretary of State for Work and Pensions, in exercise of the powers conferred upon him by sections 5(1)($hh$), ($i$) and ($j$), 189 and 191 of the Social Security Administration Act 1992\footnote{1992 c.\ 5. Sub-paragraph ($hh$)  of section 5(1) was inserted by the Social Security Act 1998 (c.\ 14), section 191 is cited because of the meaning ascribed to the word “prescribe”.} and sections 9(1), 10(3) and (6), 79 and 84 of the Social Security Act 1998\footnote{1998 c.\ 14. Section 84 is cited because of the meaning ascribed to the word “prescribe”.} and all other powers enabling him in that behalf, without having referred these Regulations to the Social Security Advisory Committee as it appears to the Secretary of State that by reason of urgency it is inexpedient to so refer them hereby makes the following Regulations: 

{\sloppy

\tableofcontents

}

\bigskip

\setcounter{secnumdepth}{-2}

\subsection[1. Citation and commencement]{Citation and commencement}

1.  These Regulations may be cited as the Social Security and Child Support (Miscellaneous Amendments) (No.\ 2) Regulations 2003 and shall come into force on 4th May 2003.

\subsection[2. Revocation]{Revocation}

2.  Regulation 6 of the Social Security and Child Support (Miscellaneous Amendments) Regulations 2003\footnote{S.I.\ 2003/1050.} is hereby revoked.

\subsection[3. Amendment of the Social Security and Child Support (Miscellaneous Amendments) Regulations 2003]{Amendment of the Social Security and Child Support (Miscellaneous Amendments) Regulations 2003}

3.  In regulation 1(1)($a$)  of the Social Security and Child Support (Miscellaneous Amendments) Regulations 2003 for the words “4 and 5” there shall be substituted “and 4 to 6”.

\medskip

4.  After regulation 5 there shall be inserted—
\begin{quotation}
\subsection*{“Tax Credits}

6.  Nothing in these Regulations shall affect the application of the Claims and Payments Regulations and the Decisions and Appeals Regulations to working families' tax credit and disabled person’s tax credit.”.
\end{quotation}

\bigskip

Signed 
by authority of the Secretary of State for Work and Pensions.

{\raggedleft
\emph{P.~Hollis}\\*Parliamentary Under-Secretary of State,\\*Department of Work and Pensions

}

%St Andrew's House, Edinburgh

%Dated
29th April 2003

\small

\part{Explanatory Note}

\renewcommand\parthead{— Explanatory Note}

\subsection*{(This note is not part of the Regulations)}

These Regulations amend the Social Security and Child Support (Miscellaneous Amendments) Regulations 2003 (“the Miscellaneous Amendments Regulations”) which contained no express commencement date for regulation 6.

Regulation 2 revokes regulation 6 of the Miscellaneous Amendments Regulations.

Regulation 3 amends regulation 1(1)($a$)  of the Miscellaneous Amendments Regulations so as to bring regulation 6 into force on 5th May 2003.

Regulation 4 re-makes regulation 6 of the Miscellaneous Amendments Regulations.

These Regulations do not impose a cost on business. 

\end{document}