% Outstanding effects from
% Social Security Act 1998 c 14 Sch 6 para 9
% Northern Ireland Act 1998 c 47 Sch 15
% Justice (Northern Ireland) Act 2002 c 26 Sch 3 paras 20--22
% Constitutional Reform Act 2005 c 4 Sch 9 para 54
% Mental Capacity Act 2005 c 9 Sch 6 para 36
% Road Safety Act 2006 c 49 Sch 3 para 65 Sch 7(4)
% Bankruptcy and Diligence (Scotland) Act 2007 asp 3 Sch 5 para 18
% Adoption and Children (Scotland) Act 2007 asp 4 Sch 2 para 7
% Welfare Reform Act 2007 c 5 Sch 3 para 7
% Tribunals, Courts and Enforcement Act 2007 c 15 Sch 10 para 22, Sch 13 paras 94--97, Sch 23 Pt II

\documentclass[12pt,a4paper]{article}

\usepackage[welsh,english]{babel}

\newcommand\regstitle{Child Support Act 1991}

\newcommand\regsnumber{c.~48}

\title{\regstitle}

\author{1991 Chapter 48}

\date{Royal Assent 25th July 1991}

\usepackage{csa-regs}

\setlength\headheight{27.57402pt}

%\hbadness=10000

\renewcommand\siprefix{\relax}

\begin{document}

\maketitle

\noindent
{\large An Act to make provision for the 
%assessment
\emph{calculation}%  % Words substituted by 2000 c 19 s 1(2)(b)
, collection and enforcement of periodical maintenance payable by certain parents with respect to children of theirs who are not in their care; for the collection and enforcement of certain other kinds of maintenance; and for connected purposes.}

\amendment{
Words ``maintenance assessment'' and ``assessment'' substituted throughout this Act by ``maintenance calculation'' and ``calculation'' (3.3.03 for 2003 scheme cases) by the Child Support, Pensions and Social Security Act 2000 s. 1(2).

Words ``absent parent'' substituted throughout this Act by ``non-resident parent'' (3.3.03 for 2003 scheme cases) by the Child Support, Pensions and Social Security Act 2000 Sch. 3 para. 11(2).

Throughout this Act where the terms ``maintenance calculation'', ``calculation'' and ``non-resident parent'' appear in italics they are to be read in relation to 1993 scheme cases as ``maintenance assessment'', ``assessment'' and ``absent parent'' respectively.
}

\bigskip

\lettrine{B}{e it enacted} by the Queen’s most Excellent Majesty, by and with the advice and consent of the Lords Spiritual and Temporal, and Commons, in this present Parliament assembled, and by the authority of the same, as follows:—


{\sloppy

\tableofcontents

}

\setcounter{secnumdepth}{-2}

\section{\itshape The basic principles}

\amendment{
Ss. 1, 2 came into force 5.4.93.
}

\subsection{1. The duty to maintain}

(1) For the purposes of this Act, each parent of a qualifying child is responsible for maintaining him.

(2) For the purposes of this Act, 
%an absent parent 
\emph{a non-resident parent}  % Words substituted by 2000 c 19 Sch 3 para 11(2)
shall be taken to have met his responsibility to maintain any qualifying child of his by making periodical payments of maintenance with respect to the child of such amount, and at such intervals, as may be determined in accordance with the provisions of this Act.

(3) Where a 
%maintenance assessment 
\emph{maintenance calculation}  % Words substituted by 2000 c 19 s 1(2)(a)
made under this Act requires the making of periodical payments, it shall be the duty of the 
%absent parent 
\emph{non-resident parent}  % Words substituted by 2000 c 19 Sch 3 para 11(2)
with respect to whom the 
%assessment 
\emph{calculation}  % Words substituted by 2000 c 19 s 1(2)(b)
was made to make those payments.

\subsection{2. Welfare of children: the general principle}

Where, in any case which falls to be dealt with under this Act, the Secretary of State 
%or any child support officer  % Words repealed (prosp) by 1998 c 14 Sch 7 para 18
is considering the exercise of any discretionary power conferred by this Act, he shall have regard to the welfare of any child likely to be affected by his decision.

\amendment{
Words repealed in s. 2 (1.6.99) by the Social Security Act 1998 Sch. 7 para. 18.
}

\subsection{3. Meaning of certain terms used in this Act}

(1) A child is a “qualifying child” if—
\begin{enumerate}\item[]
($a$) one of his parents is, in relation to him, 
%an absent parent% 
\emph{a non-resident parent}%  % Words substituted by 2000 c 19 Sch 3 para 11(2)
; or

($b$) both of his parents are, in relation to him, 
%absent parents% 
\emph{non-resident parents}%  % Words substituted by 2000 c 19 Sch 3 para 11(2)
.
\end{enumerate}

(2) The parent of any child is 
%an ``absent parent''% 
\emph{a ``non-resident parent''}%  % Words substituted by 2000 c 19 Sch 3 para 11(2)
, in relation to him, if—
\begin{enumerate}\item[]
($a$) that parent is not living in the same household with the child; and

($b$) the child has his home with a person who is, in relation to him, a person with care.
\end{enumerate}

(3) A person is a “person with care”, in relation to any child, if he is a person—
\begin{enumerate}\item[]
($a$) with whom the child has his home;

($b$) who usually provides day to day care for the child (whether exclusively or in conjunction with any other person); and

($c$) who does not fall within a prescribed category of person.
\end{enumerate}

(4) The Secretary of State shall not, under subsection (3)($c$), prescribe as a category—
\begin{enumerate}\item[]
($a$) parents;

($b$) guardians;

($c$) persons in whose favour residence orders under section 8 of the Children Act 1989 are in force;

($d$) in Scotland, persons 
%having the right to custody of a child.
with whom a child is to live by virtue of a residence order under section 11 of the Children (Scotland) Act 1995.  % Words substituted (1.11.96) by 1995 c 36 Sch 4 para 52(2)
\end{enumerate}

(5) For the purposes of this Act there may be more than one person with care in relation to the same qualifying child.

(6) Periodical payments which are required to be paid in accordance with a 
%maintenance assessment 
\emph{maintenance calculation}  % Words substituted by 2000 c 19 s 1(2)(a)
are referred to in this Act as “child support maintenance”.

(7) Expressions are defined in this section only for the purposes of this Act.

\amendment{
S. 3(3)(c) came into force 27.6.92, rest of section 5.4.93.

Words substituted in s. 3(4)(d) (1.11.96) by the Children (Scotland) Act 1995 Sch. 4 para. 52(2).
}

\subsection{4. Child support maintenance}

(1) A person who is, in relation to any qualifying child or any qualifying children, either the person with care or the 
%absent parent 
\emph{non-resident parent}  % Words substituted by 2000 c 19 Sch 3 para 11(2)
may apply to the Secretary of State for a 
%maintenance assessment 
\emph{maintenance calculation}  % Words substituted by 2000 c 19 s 1(2)(a)
to be made under this Act with respect to that child, or any of those children.

(2) Where a 
%maintenance assessment 
\emph{maintenance calculation}  % Words substituted by 2000 c 19 s 1(2)(a)
has been made in response to an application under this section the Secretary of State may, if the person with care or 
%absent parent 
\emph{non-resident parent}  % Words substituted by 2000 c 19 Sch 3 para 11(2)
with respect to whom the 
%assessment 
\emph{calculation}  % Words substituted by 2000 c 19 s 1(2)(b)
was made applies to him under this subsection, arrange for—
\begin{enumerate}\item[]
($a$) the collection of the child support maintenance payable in accordance with the 
%assessment 
\emph{calculation}%  % Words substituted by 2000 c 19 s 1(2)(b)
;

($b$) the enforcement of the obligation to pay child support maintenance in accordance with the 
%assessment 
\emph{calculation}%  % Words substituted by 2000 c 19 s 1(2)(b)
.
\end{enumerate}

(3) Where an application under subsection (2)  for the enforcement of the obligation mentioned in subsection (2)($b$)  authorises the Secretary of State to take steps to enforce that obligation whenever he considers it necessary to do so, the Secretary of State may act accordingly.

(4) A person who applies to the Secretary of State under this section shall, so far as that person reasonably can, comply with such regulations as may be made by the Secretary of State with a view to the Secretary of State 
%or the child support officer  % Words repealed (prosp) by 1998 c 14 Sch 7 para 19
being provided with the information which is required to enable—
\begin{enumerate}\item[]
($a$) [\emph{1993 scheme version}] the absent parent to be traced (where that is necessary);

($a$) [\emph{2003 scheme version}] the 
%absent parent 
non-resident parent  % Words substituted by 2000 c 19 Sch 3 para 11(2)
to be 
identified or  % Words inserted by 2000 c 19 Sch 3 para 11(3)(a)
traced (where that is necessary);

($b$) the amount of child support maintenance payable by the 
%absent parent 
\emph{non-resident parent}  % Words substituted by 2000 c 19 Sch 3 para 11(2)
to be assessed; and

($c$) that amount to be recovered from the 
%absent parent% 
\emph{non-resident parent}%  % Words substituted by 2000 c 19 Sch 3 para 11(2)
.
\end{enumerate}

(5) Any person who has applied to the Secretary of State under this section may at any time request him to cease acting under this section.

(6) It shall be the duty of the Secretary of State to comply with any request made under subsection (5)  (but subject to any regulations made under subsection (8)).

(7) The obligation to provide information which is imposed by subsection~(4)—
\begin{enumerate}\item[]
($a$) shall not apply in such circumstances as may be prescribed; and

($b$) may, in such circumstances as may be prescribed, be waived by the Secretary of State.
\end{enumerate}

(8) The Secretary of State may by regulations make such incidental, supplemental or transitional provision as he thinks appropriate with respect to cases in which he is requested to cease to act under this section.

(9) [\emph{1993 scheme version}] No application may be made under this section if there is in force with respect to the person with care and absent parent in question a maintenance assessment made in response to an application under section 6.

(9) [\emph{2003 scheme version}] No application may be made under this section if there is in force with respect to the person with care and 
%absent parent 
non-resident parent  % Words substituted by 2000 c 19 Sch 3 para 11(2)
in question a 
%maintenance assessment 
maintenance calculation  % Words substituted by 2000 c 19 s 1(2)(a)
made in response to an application 
treated as made  % Words inserted by 2000 c 19 Sch 3 para 11(3)(b)
under section 6.

% S 4(10), (11) inserted (4.9.95) by 1995 c 34 s 18(1)
(10) [\emph{1993 scheme version}] No application may be made at any time under this section with respect to a qualifying child or any qualifying children if—
\begin{enumerate}\item[]
($a$) there is in force a written maintenance agreement made before 5th April 1993, or a maintenance order, in respect of that child or those children and the person who is, at that time, the absent parent; or

($b$) benefit is being paid to, or in respect of, a parent with care of that child or 
those children.
\end{enumerate}

(10) [\emph{2003 scheme version}] No application may be made at any time under this section with respect to a qualifying child or any qualifying children if—
\begin{enumerate}\item[]
($a$) there is in force a written maintenance agreement made before 5th April 1993, or a maintenance order
made before a prescribed date%  % Words inserted by 2000 c 19 s 2(2)
, in respect of that child or those children and the person who is, at that time, the 
%absent parent% 
\emph{non-resident parent}%  % Words substituted by 2000 c 19 Sch 3 para 11(2)
; or

% S 4(10)(aa) inserted by 2000 c 19 s 2(3)
($aa$) a maintenance order made on or after the date prescribed for the purposes of paragraph~($a$)  is in force in respect of them, but has been so for less than the period of one year beginning with the date on which it was made; or

($b$) benefit is being paid to, or in respect of, a parent with care of that child or 
those children.
\end{enumerate}

(11) In subsection (10) “benefit” means any benefit which is mentioned in, or prescribed by regulations under, section 6(1).

\amendment{
S. 4(4), (7), (8) came into force 27.6.92, rest of section 5.4.93.

S. 4(10), (11) inserted (4.9.95) by the Child Support Act 1995 s.~18(1) subject to a restriction in s.~18(6).

Words repealed in s. 4(4) (1.6.99) by the Social Security Act 1998 Sch. 7 para. 19.

Words inserted in s. 4(10)(a) (4.2.03 for regulation-making purposes, 3.3.03 for 2003 scheme cases) by the Child Support, Pensions and Social Security Act 2000 s. 2(1), (2).

Words inserted in s. 4(4)(a), (9) (3.3.03 for 2003 scheme cases) by the Child Support, Pensions and Social Security Act 2000 Sch. 3 para. 11(3).
}


\subsection{5. Child support maintenance: supplemental provisions}

(1) Where—
\begin{enumerate}\item[]
($a$) there is more than one person with care of a qualifying child; and

($b$) one or more, but not all, of them have parental responsibility for 
%(or, in Scotland, parental rights over)  % Words omitted (prosp) by 1995 c 36 Sch 4 para 52(3)
the child;
\end{enumerate}
no application may be made for a 
%maintenance assessment 
\emph{maintenance calculation}  % Words substituted by 2000 c 19 s 1(2)(a)
with respect to the child by any of those persons who do not have parental responsibility for 
%(or, in Scotland, parental rights over)  % Words omitted (prosp) by 1995 c 36 Sch 4 para 52(3)
the child.

(2) Where more than one application for a 
%maintenance assessment 
\emph{maintenance calculation}  % Words substituted by 2000 c 19 s 1(2)(a)
is made with respect to the child concerned, only one of them may be proceeded with.

(3) The Secretary of State may by regulations make provision as to which of two or more applications for a 
%maintenance assessment 
\emph{maintenance calculation}  % Words substituted by 2000 c 19 s 1(2)(a)
with respect to the same child is to be proceeded with.

\amendment{
S. 5(3) came into force 27.6.92, rest of section 5.4.93.

Words repealed in s. 5(1) (1.11.96) by the Children (Scotland) Act 1995 Sch. 4 para. 52(3).
}

\subsection[6. Applications by those receiving benefit --- \emph{1993 scheme version}]{6. Applications by those receiving benefit\\*\emph{1993 scheme version}}

(1) Where income support, 
an income-based jobseeker’s allowance  % Words inserted (7.10.96) by 1995 c 18 Sch 2 para 20(2)
%, family credit  % Words repealed (5.10.99) by 1999 c 10 Sch 2 para 17(a)
%, an income-related employment and support allowance  % Words inserted by 2007 c 5 Sch 3 para 7(3)
or any other benefit of a prescribed kind is claimed by or in respect of, or paid to or in respect of, the parent of a qualifying child she shall, if—
\begin{enumerate}\item[]
($a$) she is a person with care of the child; and

($b$) she is required to do so by the Secretary of State,
\end{enumerate}
authorise the Secretary of State to take action under this Act to recover child support maintenance from the 
%absent parent% 
\emph{non-resident parent}%  % Words substituted by 2000 c 19 Sch 3 para 11(2)
.

(2) The Secretary of State shall not require a person (“the parent”) to give him the authorisation mentioned in subsection (1)  if he considers that there are reasonable grounds for believing that—
\begin{enumerate}\item[]
($a$) if the parent were to be required to give that authorisation; or

($b$) if she were to give it,
\end{enumerate}
there would be a risk of her, or of any child living with her, suffering harm or undue distress as a result.

(3) Subsection (2)  shall not apply if the parent requests the Secretary of State to disregard it.

(4) The authorisation mentioned in subsection (1)  shall extend to all children of the 
%absent parent 
\emph{non-resident parent}  % Words substituted by 2000 c 19 Sch 3 para 11(2)
in relation to whom the parent first mentioned in subsection (1)  is a person with care.

(5) That authorisation shall be given, without unreasonable delay, by completing and returning to the Secretary of State an application—
\begin{enumerate}\item[]
($a$) for the making of a 
%maintenance assessment 
\emph{maintenance calculation}  % Words substituted by 2000 c 19 s 1(2)(a)
with respect to the qualifying child or qualifying children; and

($b$) for the Secretary of State to take action under this Act to recover, on her behalf, the amount of child support maintenance so assessed.
\end{enumerate}

(6) Such an application shall be made on a form (“a maintenance application form”) provided by the Secretary of State.

(7) A maintenance application form shall indicate in general terms the effect of completing and returning it.

(8) Subsection (1)  has effect regardless of whether any of the benefits mentioned there is payable with respect to any qualifying child.

(9) A person who is under the duty imposed by subsection (1)  shall, so far as she reasonably can, comply with such regulations as may be made by the Secretary of State with a view to the Secretary of State 
%or the child support officer  % Words repealed (prosp) by 1998 c 14 Sch 7 para 20
being provided with the information which is required to enable—
\begin{enumerate}\item[]
($a$) the 
%absent parent 
\emph{non-resident parent}  % Words substituted by 2000 c 19 Sch 3 para 11(2)
to be traced;

($b$) the amount of child support maintenance payable by the 
%absent parent 
\emph{non-resident parent}  % Words substituted by 2000 c 19 Sch 3 para 11(2)
to be assessed; and

($c$) that amount to be recovered from the 
%absent parent% 
\emph{non-resident parent}%  % Words substituted by 2000 c 19 Sch 3 para 11(2)
.
\end{enumerate}

(10) The obligation to provide information which is imposed by subsection~(9)—
\begin{enumerate}\item[]
($a$) shall not apply in such circumstances as may be prescribed; and

($b$) may, in such circumstances as may be prescribed, be waived by the Secretary of State.
\end{enumerate}

(11) A person with care who has authorised the Secretary of State under subsection (1)  but who subsequently ceases to fall within that subsection may request the Secretary of State to cease acting under this section.

(12) It shall be the duty of the Secretary of State to comply with any request made under subsection (11)  (but subject to any regulations made under subsection (13)).

(13) The Secretary of State may by regulations make such incidental or transitional provision as he thinks appropriate with respect to cases in which he is requested under subsection (11)  to cease to act under this section.

(14) The fact that a 
%maintenance assessment 
\emph{maintenance calculation}  % Words substituted by 2000 c 19 s 1(2)(a)
is in force with respect to a person with care shall not prevent the making of a new 
%maintenance assessment 
\emph{maintenance calculation}  % Words substituted by 2000 c 19 s 1(2)(a)
with respect to her in response to an application under this section.

\amendment{
S. 6(1) (in so far as it confers power to prescribe kinds of benefit for the purposes of that subsection), (9), (10), (13) came into force 27.6.92.  S. came fully into force 5.4.93.

Words inserted in s. 6(1) (7.10.96) by the Jobseekers Act 1995 Sch.~2 para.~20(2).

Words repealed in s. 6(9) (1.6.99) by the Social Security Act 1998 Sch. 7 para. 20.

Words repealed in s. 6(1) (5.10.99) by the Tax Credits Act 1999 Sch. 2 para. 17(a).

Words inserted in s. 6(1) (prosp) by the Welfare Reform Act 2007 Sch. 3 para. 7(3).
}

\subsection[6. Applications by those claiming or receiving benefit --- \emph{2003 scheme version}]{6. Applications by those claiming or receiving benefit\\*\emph{2003 scheme version}}

(1) This section applies where income support, an income-based jobseeker’s allowance
%, an income-related employment and support allowance  % Words inserted by 2007 c 5 Sch 3 para 7(2)
or any other benefit of a prescribed kind is claimed by or in respect of, or paid to or in respect of, the parent of a qualifying child who is also a person with care of the child.

(2) In this section, that person is referred to as “the parent”.

(3) The Secretary of State may—
\begin{enumerate}\item[]
($a$) treat the parent as having applied for a maintenance calculation with respect to the qualifying child and all other children of the non-resident parent in relation to whom the parent is also a person with care; and

($b$) take action under this Act to recover from the non-resident parent, on the parent’s behalf, the child support maintenance so determined.
\end{enumerate}

(4) Before doing what is mentioned in subsection~(3), the Secretary of State must notify the parent in writing of the effect of subsections (3)  and (5)  and section~46. 

(5) The Secretary of State may not act under subsection~(3)  if the parent asks him not to (a request which need not be in writing).

(6) Subsection~(1)  has effect regardless of whether any of the benefits mentioned there is payable with respect to any qualifying child.

(7) Unless she has made a request under subsection~(5), the parent shall, so far as she reasonably can, comply with such regulations as may be made by the Secretary of State with a view to the Secretary of State’s being provided with the information which is required to enable—
\begin{enumerate}\item[]
($a$) the non-resident parent to be identified or traced;

($b$) the amount of child support maintenance payable by him to be calculated; and

($c$) that amount to be recovered from him.
\end{enumerate}

(8) The obligation to provide information which is imposed by subsection~(7)—
\begin{enumerate}\item[]
($a$) does not apply in such circumstances as may be prescribed; and

($b$) may, in such circumstances as may be prescribed, be waived by the Secretary of State.
\end{enumerate}

(9) If the parent ceases to fall within subsection~(1), she may ask the Secretary of State to cease acting under this section, but until then he may continue to do so.

(10) The Secretary of State must comply with any request under subsection~(9)  (but subject to any regulations made under subsection~(11)).

(11) The Secretary of State may by regulations make such incidental or transitional provision as he thinks appropriate with respect to cases in which he is asked under subsection~(9)  to cease to act under this section.

(12) The fact that a maintenance calculation is in force with respect to a person with care does not prevent the making of a new maintenance calculation with respect to her as a result of the Secretary of State’s acting under subsection~(3).

\amendment{
S. 6 substituted (10.11.00 for regulation-making purposes, 3.3.03 for 2003 scheme cases) by the Child Support, Pensions and Social Security Act 2000 s. 3.

Words inserted in s. 6(1) (prosp) by the Welfare Reform Act 2007 Sch. 3 para. 7(2).
}

\subsection{7. Right of child in Scotland to apply for 
%assessment 
\emph{calculation}  % Words substituted by 2000 c 19 s 1(2)(b)
}

(1) A qualifying child who has attained the age of 12 years and who is habitually resident in Scotland may apply to the Secretary of State for a 
%maintenance assessment 
\emph{maintenance calculation}  % Words substituted by 2000 c 19 s 1(2)(a)
to be made with respect to him if—
\begin{enumerate}\item[]
($a$) no such application has been made by a person who is, with respect to that child, a person with care or 
%an absent parent% 
\emph{a non-resident parent}%  % Words substituted by 2000 c 19 Sch 3 para 11(2)
; or

($b$) [\emph{1993 scheme version}] the Secretary of State has not been authorised under section 6 to take action under this Act to recover child support maintenance from the absent parent (other than in a case where he has waived any requirement that he should be so authorised).

% S 7(1)(b) substituted by 2000 c 19 Sch 3 para 11(4)(a)
($b$) [\emph{2003 scheme version}] no parent has been treated under section~6(3)  as having applied for a maintenance calculation with respect to the child.
\end{enumerate}

(2) An application made under subsection (1)  shall authorise the Secretary of State to make a 
%maintenance assessment 
\emph{maintenance calculation}  % Words substituted by 2000 c 19 s 1(2)(a)
with respect to any other children of the 
%absent parent 
\emph{non-resident parent}  % Words substituted by 2000 c 19 Sch 3 para 11(2)
who are qualifying children in the care of the same person as the child making the application.

(3) Where a 
%maintenance assessment 
\emph{maintenance calculation}  % Words substituted by 2000 c 19 s 1(2)(a)
has been made in response to an application under this section the Secretary of State may, if the person with care, the 
%absent parent 
\emph{non-resident parent}  % Words substituted by 2000 c 19 Sch 3 para 11(2)
with respect to whom the 
%assessment 
\emph{calculation}  % Words substituted by 2000 c 19 s 1(2)(b)
was made or the child concerned applies to him under this subsection, arrange for—
\begin{enumerate}\item[]
($a$) the collection of the child support maintenance payable in accordance with the 
%assessment 
\emph{calculation}%  % Words substituted by 2000 c 19 s 1(2)(b)
;

($b$) the enforcement of the obligation to pay child support maintenance in accordance with the 
%assessment 
\emph{calculation}%  % Words substituted by 2000 c 19 s 1(2)(b)
.
\end{enumerate}

(4) Where an application under subsection (3)  for the enforcement of the obligation mentioned in subsection (3)($b$)  authorises the Secretary of State to take steps to enforce that obligation whenever he considers it necessary to do so, the Secretary of State may act accordingly.

(5) Where a child has asked the Secretary of State to proceed under this section, the person with care of the child, the 
%absent parent 
\emph{non-resident parent}  % Words substituted by 2000 c 19 Sch 3 para 11(2)
and the child concerned shall, so far as they reasonably can, comply with such regulations as may be made by the Secretary of State with a view to the Secretary of State 
%or the child support officer  % Words repealed (prosp) by 1998 c 14 Sch 7 para 21
being provided with the information which is required to enable—
\begin{enumerate}\item[]
($a$) the 
%absent parent 
\emph{non-resident parent}  % Words substituted by 2000 c 19 Sch 3 para 11(2)
to be traced (where that is necessary);

($b$) the amount of child support maintenance payable by the 
%absent parent 
\emph{non-resident parent}  % Words substituted by 2000 c 19 Sch 3 para 11(2)
to be assessed; and

($c$) that amount to be recovered from the 
%absent parent% 
\emph{non-resident parent}%  % Words substituted by 2000 c 19 Sch 3 para 11(2)
.
\end{enumerate}

(6) The child who has made the application (but not the person having care of him) may at any time request the Secretary of State to cease acting under this section.

(7) It shall be the duty of the Secretary of State to comply with any request made under subsection (6)  (but subject to any regulations made under subsection (9)).

(8) The obligation to provide information which is imposed by subsection~(5)—
\begin{enumerate}\item[]
($a$) shall not apply in such circumstances as may be prescribed by the Secretary of State; and

($b$) may, in such circumstances as may be so prescribed, be waived by the Secretary of State.
\end{enumerate}

(9) The Secretary of State may by regulations make such incidental, supplemental or transitional provision as he thinks appropriate with respect to cases in which he is requested to cease to act under this section.

% S 7(10) inserted (4.9.95) by 1995 c 34 s 18(2)
(10) [\emph{1993 scheme version}] No application may be made at any time under this section by a qualifying child if there is in force a written maintenance agreement made before 5th April 1993, or a maintenance order, in respect of that child and the person who is, at that time, the absent parent.

(10) [\emph{2003 scheme version}] No application may be made at any time under this section by a qualifying child if%
---
\begin{enumerate}\item[]
($a$) % Word inserted by 2000 c 19 Sch 3 para 11(4)(b)(i)
 there is in force a written maintenance agreement made before 5th April 1993, or a maintenance order
made before a prescribed date%  % Words inserted by 2000 c 19 Sch 3 para 11(4)(b)(ii)
, in respect of that child and the person who is, at that time, the 
%absent parent% 
non-resident parent%  % Words substituted by 2000 c 19 Sch 3 para 11(2)
; or

($b$) a maintenance order made on or after the date prescribed for the purposes of paragraph~($a$)  is in force in respect of them, but has been so for less than the period of one year beginning with the date on which it was made.
\end{enumerate}

\amendment{
S. 7(5), (8), (9) came into force 27.6.92.  S. came fully into force 5.4.93.

S. 7(10) inserted (4.9.95) by the Child Support Act 1995 s.~18(2) subject to a restriction in s.~18(6).

Words repealed in s. 7(5) (1.6.99) by the Social Security Act 1998 Sch. 7 para. 21.

Words inserted in s. 7(10) (4.2.03 for regulation-making purposes, 3.3.03 for 2003 scheme cases) by the Child Support, Pensions and Social Security Act 2000 Sch. 3 para. 11(4)(b)(i), (ii).

S. 7(10)(b) inserted (3.3.03 for 2003 scheme cases) by the Child Support, Pensions and Social Security Act 2000 Sch. 3 para. 11(4)(b)(iii).
}

\subsection{8. Role of the courts with respect to maintenance for children}

(1) [\emph{1993 scheme version}] This subsection applies in any case where 
%a child support officer 
the Secretary of State  % Words substituted (1.6.99) by 1998 c 14 Sch 7 para 22
would have jurisdiction to make a maintenance assessment with respect to a qualifying child and an absent parent of his on an application duly made by a person entitled to apply for such an assessment with respect to that child.

(1) [\emph{2003 scheme version}] This subsection applies in any case where 
%a child support officer 
the Secretary of State  % Words substituted (1.6.99) by 1998 c 14 Sch 7 para 22
would have jurisdiction to make a 
%maintenance assessment 
maintenance calculation  % Words substituted by 2000 c 19 s 1(2)(a)
with respect to a qualifying child and 
%an absent parent 
a non-resident parent  % Words substituted by 2000 c 19 Sch 3 para 11(2)
of his on an application duly made 
(or treated as made)  % Words inserted by 2000 c 19 Sch 3 para 11(5)(a)
by a person entitled to apply for such 
%an assessment 
a calculation  % Words substituted by 2000 c 19 s 1(2)(b)
with respect to that child.

(2) Subsection (1)  applies even though the circumstances of the case are such that
%a child support officer 
the Secretary of State  % Words substituted (1.6.99) by 1998 c 14 Sch 7 para 22
would not make 
%an assessment 
\emph{a calculation}  % Words substituted by 2000 c 19 s 1(2)(b)
if it were applied for.

(3) [\emph{1993 scheme version}] In any case where subsection (1)  applies, no court shall exercise any power which it would otherwise have to make, vary or revive any maintenance order in relation to the child and absent parent concerned.

(3) [\emph{2003 scheme version}] 
Except as provided in subsection~(3A),  % Words inserted by 2000 c 19 Sch 3 para 11(5)(b)
in any case where subsection (1)  applies, no court shall exercise any power which it would otherwise have to make, vary or revive any maintenance order in relation to the child and 
%absent parent 
\emph{non-resident parent}  % Words substituted by 2000 c 19 Sch 3 para 11(2)
concerned.

% S 8(3A) inserted (4.9.95) by 1995 c 34 s 18(3)
(3A) [\emph{1993 scheme version}] In any case in which section 4(10) or 7(10) prevents the making of an application for a maintenance assessment, and—
\begin{enumerate}\item[]
($a$) no application has been made for a maintenance assessment under section 6, or

($b$) such an application has been made but no maintenance assessment has been made in response to it,
\end{enumerate}
subsection (3) shall have effect with the omission of the word “vary”.

% S 8(3A) substituted by 2000 c 19 Sch 3 para 11(5)(c)
(3A) [\emph{2003 scheme version}] Unless a maintenance calculation has been made with respect to the child concerned, subsection~(3)  does not prevent a court from varying a maintenance order in relation to that child and the non-resident parent concerned—
\begin{enumerate}\item[]
($a$) if the maintenance order was made on or after the date prescribed for the purposes of section~4(10)($a$)  or~7(10)($a$); or

($b$) where the order was made before then, in any case in which section~4(10)  or~7(10)  prevents the making of an application for a maintenance calculation with respect to or by that child.
\end{enumerate}

(4) Subsection (3)  does not prevent a court from revoking a maintenance order.

(5) The Lord Chancellor or in relation to Scotland the Lord Advocate may by order provide that, in such circumstances as may be specified by the order, this section shall not prevent a court from exercising any power which it has to make a maintenance order in relation to a child if—
\begin{enumerate}\item[]
($a$) a written agreement (whether or not enforceable) provides for the making, or securing, by 
%an absent parent 
\emph{a non-resident parent}  % Words substituted by 2000 c 19 Sch 3 para 11(2)
of the child of periodical payments to or for the benefit of the child; and

($b$) the maintenance order which the court makes is, in all material respects, in the same terms as that agreement.
\end{enumerate}

% S 8(5A) inserted by 2005 c 4 Sch 4 para 219(2)
(5A) The Lord Chancellor may make an order under subsection (5) only with the concurrence of the Lord Chief Justice.

(6) This section shall not prevent a court from exercising any power which it has to make a maintenance order in relation to a child if—
\begin{enumerate}\item[]
($a$) a 
%maintenance assessment 
\emph{maintenance calculation}  % Words substituted by 2000 c 19 s 1(2)(a)
is in force with respect to the child;

($b$) [\emph{1993 scheme version}] the amount of the child support maintenance payable in accordance with the assessment was determined by reference to the alternative formula mentioned in paragraph 4(3)  of Schedule 1; and

% S 8(6)(b) substituted by 2000 c 19 Sch 3 para 11(5)(d)
($b$) [\emph{2003 scheme version}] the non-resident parent’s net weekly income exceeds the figure referred to in paragraph~10(3)  of Schedule 1 (as it has effect from time to time pursuant to regulations made under paragraph~10A(1)($b$)); and

($c$) the court is satisfied that the circumstances of the case make it appropriate for the 
%absent parent 
\emph{non-resident parent}  % Words substituted by 2000 c 19 Sch 3 para 11(2)
to make or secure the making of periodical payments under a maintenance order in addition to the child support maintenance payable by him in accordance with the 
%maintenance assessment% 
\emph{maintenance calculation}%  % Words substituted by 2000 c 19 s 1(2)(a)
.
\end{enumerate}

(7) This section shall not prevent a court from exercising any power which it has to make a maintenance order in relation to a child if—
\begin{enumerate}\item[]
($a$) the child is, will be or (if the order were to be made) would be receiving instruction at an educational establishment or undergoing training for a trade, profession or vocation (whether or not while in gainful employment); and

($b$) the order is made solely for the purposes of requiring the person making or securing the making of periodical payments fixed by the order to meet some or all of the expenses incurred in connection with the provision of the instruction or training.
\end{enumerate}

(8) This section shall not prevent a court from exercising any power which it has to make a maintenance order in relation to a child if—
\begin{enumerate}\item[]
($a$) a disability living allowance is paid to or in respect of him; or

($b$) no such allowance is paid but he is disabled,
\end{enumerate}
and the order is made solely for the purpose of requiring the person making or securing the making of periodical payments fixed by the order to meet some or all of any expenses attributable to the child’s disability.

(9) For the purposes of subsection (8), a child is disabled if he is blind, deaf or dumb or is substantially and permanently handicapped by illness, injury, mental disorder or congenital deformity or such other disability as may be prescribed.

(10) This section shall not prevent a court from exercising any power which it has to make a maintenance order in relation to a child if the order is made against a person with care of the child.

(11) In this Act “maintenance order”, in relation to any child, means an order which requires the making or securing of periodical payments to or for the benefit of the child and which is made under—
\begin{enumerate}\item[]
($a$) Part II of the Matrimonial Causes Act 1973;

($b$) the Domestic Proceedings and Magistrates' Courts Act 1978;

($c$) Part III of the Matrimonial and Family Proceedings Act 1984;

($d$) the Family Law (Scotland) Act 1985;

($e$) Schedule 1 to the Children Act 1989; 
%or  % Word repealed by 2004 c 33 Sch 30

% S 8(11)(ea) inserted by 2004 c 33 Sch 24 para 1
($ea$) Schedule 5, 6 or 7 to the Civil Partnership Act 2004; or

($f$) any other prescribed enactment,
\end{enumerate}
and includes any order varying or reviving such an order.

% S 8(12) inserted by 2005 c 4 Sch 4 para 219(2)
(12) The Lord Chief Justice may nominate a judicial office holder (as defined in section 109(4) of the Constitutional Reform Act 2005) to exercise his functions under this section.

\amendment{
S. 8(5), (9), (11)(f) came into force 27.6.92.  S. came fully into force 5.4.93.

S. 8(3A) inserted (4.9.95) by the Child Support Act 1995 s.~18(3).

Words substituted in s. 8(1), (2) (1.6.99) by the Social Security Act 1998 Sch. 7 para. 22.

Words inserted in s. 8(1), (3) and s. 8(3A), (6) substituted (3.3.03 for 2003 scheme cases) by the Child Support, Pensions and Social Security Act 2000 Sch. 3 para. 11(5).

S. 8(11)(ea) inserted (5.12.05) by the Civil Partnership Act 2004 Sch. 24 para. 1.

Word ``or'' at end of s. 8(11)(e) repealed (5.12.05) by the Civil Partnership Act 2004 Sch. 30.

S. 8(5A), (12) inserted (3.4.06) by the Constitutional Reform Act 2005 Sch. 4 para. 219.
}

\subsection{9. Agreements about maintenance}

(1) In this section “maintenance agreement” means any agreement for the making, or for securing the making, of periodical payments by way of maintenance, or in Scotland aliment, to or for the benefit of any child.

(2) Nothing in this Act shall be taken to prevent any person from entering into a maintenance agreement.

(3) 
Subject to section 4(10)($a$) and section 7(10),  % Words inserted (4.9.95) by 1995 c 34 s 18(4)
the existence of a maintenance agreement shall not prevent any party to the agreement, or any other person, from applying for a 
%maintenance assessment 
\emph{maintenance calculation}  % Words substituted by 2000 c 19 s 1(2)(a)
with respect to any child to or for whose benefit periodical payments are to be made or secured under the agreement.

(4) Where any agreement contains a provision which purports to restrict the right of any person to apply for a 
%maintenance assessment% 
\emph{maintenance calculation}%  % Words substituted by 2000 c 19 s 1(2)(a)
, that provision shall be void.

(5) Where section 8 would prevent any court from making a maintenance order in relation to a child and 
%an absent parent 
\emph{a non-resident parent}  % Words substituted by 2000 c 19 Sch 3 para 11(2)
of his, no court shall exercise any power that it has to vary any agreement so as—
\begin{enumerate}\item[]
($a$) to insert a provision requiring that 
%absent parent 
\emph{non-resident parent}  % Words substituted by 2000 c 19 Sch 3 para 11(2)
to make or secure the making of periodical payments by way of maintenance, or in Scotland aliment, to or for the benefit of that child; or

($b$) to increase the amount payable under such a provision.
\end{enumerate}

% S 9(6) inserted (4.9.95) by 1995 c 34 s 18(4)
(6) In any case in which section 4(10) or 7(10) prevents the making of an application for a 
%maintenance assessment% 
\emph{maintenance calculation}%  % Words substituted by 2000 c 19 s 1(2)(a)
, and—
\begin{enumerate}\item[]
% S 9(6)(a), (b) substituted by 2000 c 19 Sch 3 para 11(6)
($a$) [\emph{1993 scheme version}] no application has been made for a maintenance assessment under section 6, or

($a$) [\emph{2003 scheme version}] no parent has been treated under section~6(3)  as having applied for~a maintenance calculation with respect to the child; or

($b$) [\emph{1993 scheme version}] such an application has been made but no maintenance assessment has been made in response to it,

($b$) [\emph{2003 scheme version}] a parent has been so treated but no maintenance calculation has been made,
\end{enumerate}
subsection (5) shall have effect with the omission of paragraph ($b$).

\amendment{
S. 9 came into force 5.4.93.

Words inserted in s. 9(3) and s. 9(6) inserted (4.9.95) by the Child Support Act 1995 s.~18(4).

S. 9(6)(a), (b) substituted (3.3.03 for 2003 scheme cases) by the Child Support, Pensions and Social Security Act 2000 Sch. 3 para. 11(6).
}

\subsection[10. Relationship between 
%maintenance assessments 
\emph{maintenance calculations}  % Words substituted by 2000 c 19 s 1(2)(a)
and certain court orders and related matters]{\sloppy 10. Relationship between 
%maintenance assessments 
\emph{maintenance calculations}  % Words substituted by 2000 c 19 s 1(2)(a)
and certain court orders and related matters}

(1) Where an order of a kind prescribed for the purposes of this subsection is in force with respect to any qualifying child with respect to whom a 
%maintenance assessment 
\emph{maintenance calculation}  % Words substituted by 2000 c 19 s 1(2)(a)
is made, the order—
\begin{enumerate}\item[]
($a$) shall, so far as it relates to the making or securing of periodical payments, cease to have effect to such extent as may be determined in accordance with regulations made by the Secretary of State; or

($b$) where the regulations so provide, shall, so far as it so relates, have effect subject to such modifications as may be so determined.
\end{enumerate}

(2) Where an agreement of a kind prescribed for the purposes of this subsection is in force with respect to any qualifying child with respect to whom a 
%maintenance assessment 
\emph{maintenance calculation}  % Words substituted by 2000 c 19 s 1(2)(a)
is made, the agreement—
\begin{enumerate}\item[]
($a$) shall, so far as it relates to the making or securing of periodical payments, be unenforceable to such extent as may be determined in accordance with regulations made by the Secretary of State; or

($b$) where the regulations so provide, shall, so far as it so relates, have effect subject to such modifications as may be so determined.
\end{enumerate}

(3) Any regulations under this section may, in particular, make such provision with respect to—
\begin{enumerate}\item[]
($a$) any case where any person with respect to whom an order or agreement of a kind prescribed for the purposes of subsection (1)  or~(2)  has effect applies to the prescribed court, before the end of the prescribed period, for the order or agreement to be varied in the light of the 
%maintenance assessment 
\emph{maintenance calculation}  % Words substituted by 2000 c 19 s 1(2)(a)
and of the provisions of this Act;

($b$) the recovery of any arrears under the order or agreement which fell due before the coming into force of the 
%maintenance assessment% 
\emph{maintenance calculation}%  % Words substituted by 2000 c 19 s 1(2)(a)
,
\end{enumerate}
as the Secretary of State considers appropriate and may provide that, in prescribed circumstances, an application to any court which is made with respect to an order of a prescribed kind relating to the making or securing of periodical payments to or for the benefit of a child shall be treated by the court as an application for the order to be revoked.

(4) The Secretary of State may by regulations make provision for—
\begin{enumerate}\item[]
($a$) notification to be given by 
%the child support officer concerned 
the Secretary of State  % Words substituted (1.6.99) by 1998 c 14 Sch 4 para 23(1)(a)
to the prescribed person in any case where 
%that officer 
he  % Words substituted (1.6.99) by 1998 c 14 Sch 4 para 23(1)(b)
considers that the making of a 
%maintenance assessment 
\emph{maintenance calculation}  % Words substituted by 2000 c 19 s 1(2)(a)
has affected, or is likely to affect, any order of a kind prescribed for the purposes of this subsection;

($b$) notification to be given by the prescribed person to the Secretary of State in any case where a court makes an order which it considers has affected, or is likely to affect, a 
%maintenance assessment% 
\emph{maintenance calculation}%  % Words substituted by 2000 c 19 s 1(2)(a)
.
\end{enumerate}

(5) Rules may be made under section 144 of the Magistrates' Courts Act 1980 (rules of procedure) requiring any person who, in prescribed circumstances, makes an application to a magistrates' court for a maintenance order to furnish the court with a statement in a prescribed form, and signed by 
%a child support officer%
an officer of the Secretary of State%  % Words substituted (1.6.99) by 1998 c 14 Sch 4 para 23(2)
, as to whether or not, at the time when the statement is made, there is a 
%maintenance assessment 
\emph{maintenance calculation}  % Words substituted by 2000 c 19 s 1(2)(a)
in force with respect to that person or the child concerned.

In this subsection—
\begin{enumerate}\item[]
    “maintenance order” means an order of a prescribed kind for the making or securing of periodical payments to or for the benefit of a child; and

    “prescribed” means prescribed by the rules. 
\end{enumerate}

\amendment{
S. 10 came into force 27.6.92.

Words substituted in s. 10(4), (5) (1.6.99) by the Social Security Act 1998 Sch. 7 para. 23.
}

\section{\itshape 
%Maintenance assessments 
\emph{Maintenance calculations}  % Words substituted by 2000 c 19 s 1(2)(a)
}

\subsection[11. Maintenance assessments --- \emph{1993 scheme version}]{11. Maintenance assessments\\*\emph{1993 scheme version}}

(1) Any application for a maintenance assessment made to the Secretary of State shall be 
%referred by him to a child support officer whose duty it shall be to deal with the application 
dealt with by him  % Words substituted (1.6.99) by 1998 c 14 Sch 7 para 24(1)
in accordance with the provision made by or under this Act.

% S 11(1A)--(1C) inserted (4.9.95) by 1995 c 34 s 19
(1A) Where—
\begin{enumerate}\item[]
($a$) an application for a maintenance assessment is made under section~6, but

($b$) the Secretary of State becomes aware, 
%before referring the application to a child support officer%
before determining the application%  % Words substituted (1.6.99) by 1998 c 14 Sch 7 para 24(2)
, that the claim mentioned in subsection (1) of that section has been disallowed or withdrawn,
\end{enumerate}
he shall, subject to subsection (1B), treat the application as if it had not been made.

(1B) If it appears to the Secretary of State that subsection (10) of section~4 would not have prevented the parent with care concerned from making an application for a maintenance assessment under that section he shall—
\begin{enumerate}\item[]
($a$) notify her of the effect of this subsection, and

($b$) if, before the end of the period of 28 days beginning with the day on which notice was sent to her, she asks him to do so, treat the application as having been made not under section 6 but under section 4.
\end{enumerate}

(1C) Where the application is not preserved under subsection (1B) (and so is treated as not having been made) the Secretary of State shall notify—
\begin{enumerate}\item[]
($a$) the parent with care concerned; and

($b$) the absent parent (or alleged absent parent), where it appears to him that that person is aware of the application.
\end{enumerate}

(2) The amount of child support maintenance to be fixed by any maintenance assessment shall be determined in accordance with the provisions of Part I of Schedule 1.

(3) Part II of Schedule 1 makes further provision with respect to maintenance assessments.

\amendment{
S. 11 came into force 27.6.92.

S. 11(1A)--(1C) inserted (4.9.95) by the Child Support Act 1995 s.~19.

Words substituted in s. 11(1), (1A) (1.6.99) by the Social Security Act 1998 Sch. 7 para. 24.
}

\subsection[11. Maintenance calculations --- \emph{2003 scheme version}]{11. Maintenance calculations\\*\emph{2003 scheme version}}

(1) An application for a maintenance calculation made to the Secretary of State shall be dealt with by him in accordance with the provision made by or under this Act.

(2) The Secretary of State shall (unless he decides not to make a maintenance calculation in response to the application, or makes a decision under section 12) determine the application by making a decision under this section about whether any child support maintenance is payable and, if so, how much.

(3) Where—
\begin{enumerate}\item[]
($a$) a parent is treated under section 6(3)  as having applied for a maintenance calculation; but

($b$) the Secretary of State becomes aware before determining the application that the parent has ceased to fall within section 6(1),
\end{enumerate}
he shall, subject to subsection (4), cease to treat that parent as having applied for a maintenance calculation.

(4) If it appears to the Secretary of State that subsection (10)  of section 4 would not have prevented the parent with care concerned from making an application for a maintenance calculation under that section he shall—
\begin{enumerate}\item[]
($a$) notify her of the effect of this subsection; and

($b$) if, before the end of the period of one month beginning with the day on which notice was sent to her, she asks him to do so, treat her as having applied not under section~6 but under section~4. 
\end{enumerate}

(5) Where subsection~(3)  applies but subsection~(4)  does not, the Secretary of State shall notify—
\begin{enumerate}\item[]
($a$) the parent with care concerned; and

($b$) the non-resident parent (or alleged non-resident parent), where it appears to him that that person is aware that the parent with care has been treated as having applied for a maintenance calculation.
\end{enumerate}

(6) The amount of child support maintenance to be fixed by a maintenance calculation shall be determined in accordance with Part I of Schedule 1 unless an application for a variation has been made and agreed.

(7) If the Secretary of State has agreed to a variation, the amount of child support maintenance to be fixed shall be determined on the basis he determines under section~28F(4).

(8) Part II of Schedule 1 makes further provision with respect to maintenance calculations.

\amendment{
S. 11 substituted (3.3.03 for 2003 scheme cases) by the Child Support, Pensions and Social Security Act 2000 s. 1(1).
}

\subsection[12. Interim maintenance assessments --- \emph{1993 scheme version}]{12. Interim maintenance assessments\\*\emph{1993 scheme version}}

%(1) Where it appears to a child support officer who is required to make a maintenance assessment that he does not have sufficient information to enable him to make an assessment in accordance with the provision made by or under this Act, he may make an interim maintenance assessment.

% S 12(1), (1A) substituted for s 12(1) (22.1.96) by 1995 c 36 s 11
%(1) This section applies where a child support officer—
%\begin{enumerate}\item[]
%($a$) is required to make a maintenance assessment;
%
%($b$) is proposing to conduct a review under section 16, 17, 18 or 19; or
%
%($c$) is conducting such a review.
%\end{enumerate}
%
%(1A) If it appears to the child support officer that he does not have sufficient information to enable him—
%\begin{enumerate}\item[]
%($a$) in a case falling within subsection (1)($a$), to make the assessment,
%
%($b$) in a case falling within subsection (1)($b$), to conduct the proposed review, or
%
%($c$) in a case falling within subsection (1)($c$), to complete the review,
%\end{enumerate}
%he may make an interim maintenance assessment.

% S 12(1) substituted for s 12(1), (1A) (1.6.99) by 1998 c 14 Sch 7 para 25(1)
(1) Where the Secretary of State—
\begin{enumerate}\item[]
($a$) is required to make a maintenance assessment; or

($b$) is proposing to make a decision under section 16 or 17,
\end{enumerate}
and (in either case) it appears to him that he does not have sufficient information to enable him to do so, he may make an interim maintenance assessment.

(2) The Secretary of State may by regulations make provision as to interim maintenance assessments.

(3) The regulations may, in particular, make provision as to—
\begin{enumerate}\item[]
($a$) the procedure to be followed in making an interim maintenance assessment; and

($b$) the basis on which the amount of child support maintenance fixed by an interim assessment is to be calculated.
\end{enumerate}

(4) Before making any interim assessment 
%a child support officer 
the Secretary of State  % Words substituted (1.6.99) by 1998 c 14 Sch 7 para 25(2)
shall, if it is reasonably practicable to do so, give written notice of his intention to make such an assessment to—
\begin{enumerate}\item[]
($a$) the absent parent concerned;

($b$) the person with care concerned; and

($c$) where the application for a maintenance assessment was made under section 7, the child concerned.
\end{enumerate}

(5) Where 
%a child support officer 
the Secretary of State  % Words substituted (1.6.99) by 1998 c 14 Sch 7 para 25(2)
serves notice under subsection (4), he shall not make the proposed interim assessment before the end of such period as may be prescribed.

\amendment{
S. 12(2), (3), (5) came into force 27.6.92.  S. 12 came fully into force 5.4.93.

S. 12(1), (1A) substituted for s. 12(1) (22.1.96) by the Child Support Act 1995 s. 11.

S. 12(1) substituted for s. 12(1), (1A) and words substituted in s. 12(4), (5) (1.6.99) by the Social Security Act 1998 Sch. 7 para. 25.
}

\subsection[12. Default and interim maintenance decisions --- \emph{2003 scheme version}]{12. Default and interim maintenance decisions\\*\emph{2003 scheme version}}

(1) Where the Secretary of State—
\begin{enumerate}\item[]
($a$) is required to make a maintenance calculation; or

($b$) is proposing to make a decision under section~16 or~17,
\end{enumerate}
and it appears to him that he does not have sufficient information to enable him to do so, he may make a default maintenance decision.

(2) Where an application for a variation has been made under section~28A(1)  in connection with an application for a maintenance calculation (or in connection with such an application which is treated as having been made), the Secretary of State may make an interim maintenance decision.

(3) The amount of child support maintenance fixed by an interim maintenance decision shall be determined in accordance with Part I of Schedule~1. 

(4) The Secretary of State may by regulations make provision as to default and interim maintenance decisions.

(5) The regulations may, in particular, make provision as to—
\begin{enumerate}\item[]
($a$) the procedure to be followed in making a default or an interim maintenance decision; and

($b$) a default rate of child support maintenance to apply where a default maintenance decision is made.
\end{enumerate}

\amendment{
S. 12 substituted (10.11.00 for regulation-making purposes, 3.3.03 for 2003 scheme cases) by the Child Support, Pensions and Social Security Act 2000 s. 4.

\medskip

S. 13 repealed (1.6.99) by the Social Security Act 1998 Sch. 7 para. 26.
}

% S 13 repealed (prosp) by 1998 c 14 Sch 7 para 26
%\section{\itshape Child support officers}
%
%\subsection{13. Child support officers}
%
%(1) The Secretary of State shall appoint persons (to be known as child support officers) for the purpose of exercising functions—
%\begin{enumerate}\item[]
%($a$) conferred on them by this Act, or by any other enactment; or
%
%($b$) assigned to them by the Secretary of State.
%\end{enumerate}
%
%(2) A child support officer may be appointed to perform only such functions as may be specified in his instrument of appointment.
%
%(3) The Secretary of State shall appoint a Chief Child Support Officer.
%
%(4) It shall be the duty of the Chief Child Support Officer to—
%\begin{enumerate}\item[]
%($a$) advise child support officers on the discharge of their functions in relation to making, reviewing or cancelling maintenance assessments;
%
%($b$) keep under review the operation of the provision made by or under this Act with respect to making, reviewing or cancelling maintenance assessments; and
%
%($c$) report to the Secretary of State annually, in writing, on the matters with which the Chief Child Support Officer is concerned.
%\end{enumerate}
%
%(5) The Secretary of State shall publish, in such manner as he considers appropriate, any report which he receives under subsection (4)($c$).
%
%(6) Any proceedings (other than for an offence) in respect of any act or omission of a child support officer which, apart from this subsection, would fall to be brought against a child support officer resident in Northern Ireland may instead be brought against the Chief Child Support Officer.
%
%(7) For the purposes of any proceedings brought by virtue of subsection~(6), the acts or omissions of the child support officer shall be treated as the acts or omissions of the Chief Child Support Officer.
%
%\amendment{
%S. 13 came into force 1.9.92.
%
%}

\section{\itshape Information}

\subsection{14. Information required by Secretary of State}

(1) [\emph{1993 scheme version}] The Secretary of State may make regulations requiring any information or evidence needed for the determination of any application under this Act, or any question arising in connection with such an application
(or application treated as made), or needed for the making of any decision or in connection with the imposition of any condition or requirement under this Act%  % Words inserted by 2000 c 19 s 12
, or needed in connection with the collection or enforcement of child support or other maintenance under this Act, to be furnished—
\begin{enumerate}\item[]
($a$) by such persons as may be determined in accordance with regulations made by the Secretary of State; and

($b$) in accordance with the regulations.
\end{enumerate}

(1) [\emph{2003 scheme version}] The Secretary of State may make regulations requiring any information or evidence needed for the determination of any application 
made or treated as made  % Words inserted by 2000 c 19 Sch 3 para 11(7)
under this Act, or any question arising in connection with such an application
(or application treated as made), or needed for the making of any decision or in connection with the imposition of any condition or requirement under this Act%  % Words inserted by 2000 c 19 s 12
, or needed in connection with the collection or enforcement of child support or other maintenance under this Act, to be furnished—
\begin{enumerate}\item[]
($a$) by such persons as may be determined in accordance with regulations made by the Secretary of State; and

($b$) in accordance with the regulations.
\end{enumerate}

% S 14(1A) inserted (1.10.95) by 1995 c 34 Sch 3 para 3(1)
(1A) Regulations under subsection (1) may make provision for notifying any person who is required to furnish any information or evidence under the regulations of the possible consequences of failing to do so.

% S 14(2), (2A) repealed (8.9.98) by 1998 c 14 Sch 7 para 27(a)
%(2) Where the Secretary of State has in his possession any information acquired by him in connection with his functions under any of the benefit Acts
%or the Jobseekers Act 1995%  % Words inserted (7.10.96) by 1995 c 18 Sch 2 para 20(3)
%, he may—
%\begin{enumerate}\item[]
%($a$) make use of that information for purposes of this Act; or
%
%($b$) disclose it to the Department of Health and Social Services for Northern Ireland for purposes of any enactment corresponding to this Act and having effect with respect to Northern Ireland.
%\end{enumerate}
%
%% S 14(2A) inserted (4.9.95) by 1995 c 34 Sch 3 para 3(2)
%(2A) Where the Secretary of State has in his possession any information acquired by him in connection with his functions under this Act, he may—
%\begin{enumerate}\item[]
%($a$) make use of that information for purposes of any of the benefit Acts or of the Jobseekers Act 1995; or
%
%($b$) disclose it to the Department of Health and Social Services for Northern Ireland for purposes of any enactment corresponding to any of those Acts and having effect with respect to Northern Ireland.
%\end{enumerate}

(3) The Secretary of State may by regulations make provision authorising the disclosure by him% 
%or by child support officers%  % Words repealed (1.6.99) by 1998 c 14 Sch 7 para 27(b)
, in such circumstances as may be prescribed, of such information held by 
%them 
him  % Words substituted (1.6.99) by 1998 c 14 Sch 7 para 27(b)
for purposes of this Act as may be prescribed.

(4) The provisions of Schedule 2 (which relate to information which is held for purposes other than those of this Act but which is required by the Secretary of State) shall have effect.

\amendment{
S. 14(1), (3), (4) came into force 27.6.92.  S. 14 came fully into force 5.4.93.

S. 14(2A) inserted (4.9.95) by the Child Support Act 1995 Sch.~3 para.~3(2).

S. 14(1A) inserted (1.10.95) by the Child Support Act 1995 Sch.~3 para.~3(1).

Words inserted in s. 14(2) (7.10.96) by the Jobseekers Act 1995 Sch.~2 para.~20(3).

S. 14(2), (2A) repealed (8.9.98) by the Social Security Act 1998 Sch. 7 para. 27(a).

Words in s. 14(3) repealed and word in s. 14(3) substituted (1.6.99) by the Social Security Act 1998 Sch. 7 para. 27(b).

Words inserted in s. 14(1) (3.3.03 for 2003 scheme cases only, 26.9.08 for all cases) by the Child Support, Pensions and Social Security Act 2000 s. 12.

Words inserted in s. 14(1) (3.3.03 for 2003 scheme cases only) by the Child Support, Pensions and Social Security Act 2000 Sch. 3 para. 11(7).
}

% S 14A inserted by 2000 c 19 s 13
\subsection{14A. Information—offences}

(1) This section applies to—
\begin{enumerate}\item[]
($a$) persons who are required to comply with regulations under section~4(4)  or~7(5); and

($b$) persons specified in regulations under section~14(1)($a$).
\end{enumerate}

(2) Such a person is guilty of an offence if, pursuant to a request for information under or by virtue of those regulations—
\begin{enumerate}\item[]
($a$) he makes a statement or representation which he knows to be false; or

($b$) he provides, or knowingly causes or knowingly allows to be provided, a document or other information which he knows to be false in a material particular.
\end{enumerate}

(3) Such a person is guilty of an offence if, following such a request, he fails to comply with it.

(4) It is a defence for a person charged with an offence under subsection~(3)  to prove that he had a reasonable excuse for failing to comply.

(5) A person guilty of an offence under this section is liable on summary conviction to a fine not exceeding level 3 on the standard scale.

\amendment{
S. 14A inserted (31.1.01) by the Child Support, Pensions and Social Security Act 2000 s. 13.
}

\subsection{15. Powers of inspectors}

%(1) Where, in a particular case, the Secretary of State considers it appropriate to do so for the purpose of acquiring information which he 
%%or any child support officer  % Words repealed (1.6.99) by 1998 c 14 Sch 7 para 27
%requires for purposes of this Act, he may appoint a person to act as an inspector under this section.
%
%(2) Every inspector shall be furnished with a certificate of his appointment.
%
%(3) Without prejudice to his being appointed to act in relation to any other case, or being appointed to act for a further period in relation to the case in question, an inspector’s appointment shall cease at the end of such period as may be specified.
%
%(4) An inspector shall have power—
%\begin{enumerate}\item[]
%($a$) to enter at all reasonable times—
%\begin{enumerate}\item[]
%(i) any specified premises, other than premises used solely as a dwelling-house; and
%
%(ii) any premises which are not specified but which are used by any specified person for the purpose of carrying on any trade, profession, vocation or business; and
%\end{enumerate}
%
%($b$) to make such examination and enquiry there as he considers appropriate.
%\end{enumerate}

% S 15(1)--(4A) substituted for s 15(1)--(4) by 2000 c 19 s 14(2)
(1) The Secretary of State may appoint, on such terms as he thinks fit, persons to act as inspectors under this section.

(2) The function of inspectors is to acquire information which the Secretary of State needs for any of the purposes of this Act.

(3) Every inspector is to be given a certificate of his appointment.

(4) An inspector has power, at any reasonable time and either alone or accompanied by such other persons as he thinks fit, to enter any premises which—
\begin{enumerate}\item[]
($a$) are liable to inspection under this section; and

($b$) are premises to which it is reasonable for him to require entry in order that he may exercise his functions under this section,
\end{enumerate}
and may there make such examination and inquiry as he considers appropriate.

(4A) Premises liable to inspection under this section are those which are not used wholly as a dwelling house and which the inspector has reasonable grounds for suspecting are—
\begin{enumerate}\item[]
($a$) premises at which a non-resident parent is or has been employed;

($b$) premises at which a non-resident parent carries out, or has carried out, a trade, profession, vocation or business;

($c$) premises at which there is information held by a person (“$\mathcal{A}$”) whom the inspector has reasonable grounds for suspecting has information about a non-resident parent acquired in the course of $\mathcal{A}$’s own trade, profession, vocation or business.
\end{enumerate}

(5) An inspector exercising his powers may question any person aged 18 or over whom he finds on the premises.

(6) If required to do so by an inspector exercising his powers, 
%any person who is or has been—
%\begin{enumerate}\item[]
%($a$) an occupier of the premises in question;
%
%($b$) an employer or an employee working at or from those premises;
%
%($c$) carrying on at or from those premises any trade, profession, vocation or business;
%
%($d$) an employee or agent of any person mentioned in paragraphs ($a$)  to~($c$),
%\end{enumerate}
any such person  % Words substituted (31.1.01) by 2000 c 19 s 15(3)
shall furnish to the inspector all such information and documents as the inspector may 
reasonably require.

(7) No person shall be required under this section to answer any question or to give any evidence tending to incriminate himself or, in the case of a person who is married
or is a civil partner%  % Words inserted by 2004 c 33 Sch 24 para 2(a)
, his or her spouse
or civil partner%  % Words inserted by 2004 c 33 Sch 24 para 2(b)
.

(8) On applying for admission to any premises in the exercise of his powers, an inspector shall, if so required, produce his certificate.

(9) If any person—
\begin{enumerate}\item[]
($a$) intentionally delays or obstructs any inspector exercising his powers; or

($b$) without reasonable excuse, refuses or neglects to answer any question or furnish any information or to produce any document when required to do so under this section,
\end{enumerate}
he shall be guilty of an offence and liable on summary conviction to a fine not exceeding level 3 on the standard scale.

(10) In this section—
\begin{enumerate}\item[]
    “certificate” means a certificate of appointment issued under this section;

    “inspector” means an inspector appointed under this section;

    “powers” means powers conferred by this section%; and
%
% Definition repealed by 2000 c 19 Sch 9 Pt I
%   “specified” means specified in the certificate in question
. 
\end{enumerate}

% S 15(11) inserted by 2000 c 19 s 15(4)
(11) In this section, “premises” includes—
\begin{enumerate}\item[]
($a$) moveable structures and vehicles, vessels, aircraft and hovercraft;

($b$) installations that are offshore installations for the purposes of the Mineral Workings (Offshore Installations) Act 1971; and

($c$) places of all other descriptions whether or not occupied as land or otherwise,
\end{enumerate}
and references in this section to the occupier of premises are to be construed, in relation to premises that are not occupied as land, as references to any person for the time being present at the place in question.

\amendment{
S. 15 came into force 5.4.93.

Words in s. 15(1) repealed (1.6.99) by the Social Security Act 1998 Sch. 7 para. 28.

S. 15(1)--(4A) substituted for s. 15(1)--(4), words substituted in s. 15(6) and s. 15(11) inserted (31.1.01) by the Child Support, Pensions and Social Security Act 2000 s. 14.

Definition of ``specified'' in s. 15(10) repealed (2.4.01) by the Child Support, Pensions and Social Security Act 2000 Sch. 9 Pt. I.

Words inserted in s. 15(7) (5.12.05) by the Civil Partnership Act 2004 Sch. 24 para. 2.
}

\section{\itshape Reviews and appeals}

%\subsection{16. Periodical reviews}
%
%(1) The Secretary of State shall make such arrangements as he considers necessary to secure that, where any maintenance assessment has been in force for a prescribed period, the amount of child support maintenance fixed by that assessment (“the original assessment”) is reviewed by a child support officer under this section as soon as is reasonably practicable after the end of that prescribed period.
%
%(2) Before conducting any review under this section, the child support officer concerned shall give, to such persons as may be prescribed, such notice of the proposed review as may be prescribed.
%
%(3) A review shall be conducted under this section as if a fresh application for a maintenance assessment had been made by the person in whose favour the original assessment was made.
%
%(4) On completing any review under this section, the child support officer concerned shall make a fresh maintenance assessment, unless he is satisfied that the original assessment has ceased to have effect or should be brought to an end.
%
%(5) Where a fresh maintenance assessment is made under subsection (4), it shall take effect—
%\begin{enumerate}\item[]
%($a$) on the day immediately after the end of the prescribed period mentioned in subsection (1); or
%
%($b$) in such circumstances as may be prescribed, on such later date as may be determined in accordance with regulations made by the Secretary of State.
%\end{enumerate}
%
%(6) The Secretary of State may by regulations prescribe circumstances (for example, where the maintenance assessment is about to terminate) in which a child support officer may decide not to conduct a review under this section.
%
%\amendment{
%S. 16(1), (2), (5), (6) came into force 27.6.92.  
%S. 16 came fully into force 5.4.93.
%}

% S 16 substituted (16.11.98 for regulation-making purposes, 7.12.98 for all purposes) by 1998 c 14 s 40
\subsection{16. Revision of decisions}

(1) [\emph{1993 scheme version}] Any decision of the Secretary of State under section~11, 12 or 17 may be revised by the Secretary of State—
\begin{enumerate}\item[]
($a$) either within the prescribed period or in prescribed cases or circumstances; and

($b$) either on an application made for the purpose or on his own initiative;
\end{enumerate}
and regulations may prescribe the procedure by which a decision of the Secretary of State may be so revised.

(1) [\emph{2003 scheme version}] Any decision 
%of the Secretary of State under section 11, 12 or 17 
to which subsection~(1A)  applies  % Words substituted by 2000 c 19 s 8(2)
may be revised by the Secretary of State—
\begin{enumerate}\item[]
($a$) either within the prescribed period or in prescribed cases or circumstances; and

($b$) either on an application made for the purpose or on his own initiative;
\end{enumerate}
and regulations may prescribe the procedure by which a decision of the Secretary of State may be so revised.

% S 16(1A), (1B) inserted by 2000 c 19 s 8(3)
(1A) [\emph{2003 scheme only}] This subsection applies to—
\begin{enumerate}\item[]
($a$) a decision of the Secretary of State under section~11, 12 or~17;

($b$) a reduced benefit decision under section~46;

($c$) a decision of an appeal tribunal on a referral under section~28D(1)($b$).
\end{enumerate}

(1B) [\emph{2003 scheme only}] Where the Secretary of State revises a decision under section~12(1)—
\begin{enumerate}\item[]
($a$) he may (if appropriate) do so as if he were revising a decision under section~11; and

($b$) if he does that, his decision as revised is to be treated as one under section~11 instead of section~12(1)  (and, in particular, is to be so treated for the purposes of an appeal against it under section~20).
\end{enumerate}

(2) In making a decision under subsection (1), the Secretary of State need not consider any issue that is not raised by the application or, as the case may be, did not cause him to act on his own initiative.

(3) Subject to subsections (4) and (5) and section 28ZC, a revision under this section shall take effect as from the date on which the original decision took (or was to take) effect.

(4) Regulations may provide that, in prescribed cases or circumstances, a revision under this section shall take effect as from such other date as may be prescribed.

(5) Where a decision is revised under this section, for the purpose of any rule as to the time allowed for bringing an appeal, the decision shall be regarded as made on the date on which it is so revised.

(6) Except in prescribed circumstances, an appeal against a decision of the Secretary of State shall lapse if the decision is revised under this section before the appeal is determined.

\amendment{
S. 16 substituted (16.11.98 for regulation-making purposes, 7.12.98 for all other purposes subject to a saving, 28.7.00 for all purposes) by the Social Security Act 1998 s. 40.

Words substituted in s. 16(1) and s. 16(1A), (1B) inserted (3.3.03 for 2003 scheme cases only) by the Child Support, Pensions and Social Security Act 2000 s. 8.
}

%\subsection{17. Reviews on change of circumstances}
%
%(1) Where a maintenance assessment is in force—
%\begin{enumerate}\item[]
%($a$) the absent parent or person with care with respect to whom it was made; or
%
%($b$) where the application for the assessment was made under section 7, either of them or the child concerned,
%\end{enumerate}
%may apply to the Secretary of State for the amount of child support maintenance fixed by that assessment (“the original assessment”) to be reviewed under this section.
%
%(2) An application under this section may be made only on the ground that, by reason of a change of circumstance since the original assessment was made, the amount of child support maintenance payable by the absent parent would be significantly different if it were to be fixed by a maintenance assessment made by reference to the circumstances of the case as at the date of the application.
%
%% S 17(2A) inserted (22.1.96) by 1995 c 34 s 12(2)
%(2A) The Secretary of State shall refer to a child support officer any application under this section which is duly made.
%
%(3) The child support officer to whom an application under this section has been referred shall not proceed unless, on the information before him, he considers that it is likely that he will be required by subsection (6)%  
%, or by virtue of subsection (7),  % Words inserted (22.1.96) by 1995 c 34 s 12(3)(a)
%to make a fresh maintenance assessment if he conducts 
%%the review applied for
%a review%  % Words substituted (22.1.96) by 1995 c 34 s 12(3)(b)
%.
%
%(4) Before conducting any review under this section, the child support officer concerned shall give to such persons as may be prescribed, such notice of the proposed review as may be prescribed.
%
%% S 17(4A) inserted (22.1.96) by 1996 c 34 s 12(4)
%(4A) Where a child support officer is conducting a review under this section, and the original assessment has ceased to have effect, he may continue the review as if the application for a review related to the original assessment and any subsequent assessment.
%
%%(5) A review shall be conducted under this section as if a fresh application for a maintenance assessment had been made by the person in whose favour the original assessment was made.
%
%% S 17(5) substituted (1.10.95) by 1995 c 34 s 12(5)
%(5) In conducting a review under this section, the child support officer shall take into account a change of circumstance only if—
%\begin{enumerate}\item[]
%($a$) he has been notified of it in such manner, and by such person, as may be prescribed; or
%
%($b$) it is one which he knows has taken place.
%\end{enumerate}
%
%(6) On completing 
%%any review 
%a review of the original assessment  % Words substituted (22.1.96) by 1995 c 34 s 12(6)(a)
%under this section, the child support officer concerned shall make a fresh maintenance assessment
%by reference to the circumstances of the case as at the date of the application under this section%  % Words inserted (22.1.96) by 1995 c 34 s 12(6)(b)
%, unless—
%\begin{enumerate}\item[]
%($a$) he is satisfied that the original assessment has ceased to have effect or should be brought to an end; or
%
%($b$) the difference between the amount of child support maintenance fixed by the original assessment and the amount that would be fixed if a fresh assessment were to be made as a result of the review is less than such amount as may be prescribed.
%\end{enumerate}
%
%% S 17(7), (8) added (1.10.95) by 1995 c 34 s 12(7)
%(7) On completing a review of any subsequent assessment under this section, the child support officer concerned shall make a fresh maintenance assessment except in such circumstances as may be prescribed.
%
%(8) In this section “subsequent assessment” means a maintenance assessment made after the original assessment with respect to the same persons as the original assessment.
%
%\amendment{
%S. 17(4), (6)(b) came into force 27.6.92.
%S. 17 came fully into force 5.4.93.
%
%S. 17(5) substituted and s. 17(7), (8) added (1.10.95 for the purpose of making regulations) by the Child Support Act 1995 s. 12(5), (7).
%
%S. 17(2A) inserted, words inserted and substituted in s.~17(3), s. 17(4A) inserted and words substituted and inserted in s.~17(6) (22.1.96) by the Child Support Act 1995 s. 12(2)--(4), (6).
%}
%
%\subsection{18. Reviews of decisions of child support officers}
%
%(1) Where—
%\begin{enumerate}\item[]
%($a$) an application for a maintenance assessment is refused; or
%
%($b$) an application, under section 17, for the review of a maintenance assessment which is in force is refused,
%\end{enumerate}
%the person who made that application may apply to the Secretary of State for the refusal to be reviewed.
%
%(2) Where a maintenance assessment is in force—
%\begin{enumerate}\item[]
%($a$) the absent parent or person with care with respect to whom it was made; or
%
%($b$) where the application for the assessment was made under section 7, either of them or the child concerned,
%\end{enumerate}
%may apply to the Secretary of State for the assessment to be reviewed.
%
%(3) Where a maintenance assessment is cancelled the appropriate person may apply to the Secretary of State for the cancellation to be reviewed.
%
%(4) Where an application for the cancellation of a maintenance assessment is refused, the appropriate person may apply to the Secretary of State for the refusal to be reviewed.
%
%(5) An application under this section shall give the applicant’s reasons (in writing) for making it.
%
%(6) The Secretary of State shall refer to a child support officer any application under this section which is duly made; and the child support officer shall conduct the review applied for unless in his opinion there are no reasonable grounds for supposing that the refusal, assessment or cancellation in question—
%\begin{enumerate}\item[]
%($a$) was made in ignorance of a material fact;
%
%($b$) was based on a mistake as to a material fact;
%or  % Word inserted (4.9.95) by 1995 c 34 Sch 3 para 4
%
%($c$) was wrong in law.
%\end{enumerate}
%
%% S 13(6A) inserted (22.1.96) by 1995 c 34 s 13
%(6A) Where a child support officer is conducting a review under this section and the maintenance assessment in question (“the original assessment”) is no longer in force, he may continue the review as if the application for a review related to the original assessment and any maintenance assessment made after the original assessment with respect to the same persons as the original assessment.
%
%(7) The Secretary of State shall arrange for a review under this section to be conducted by a child support officer who played no part in taking the decision which is to be reviewed.
%
%(8) Before conducting any review under this section, the child support officer concerned shall give to such persons as may be prescribed, such notice of the proposed review as may be prescribed.
%
%(9) If a child support officer conducting a review under this section is satisfied that a maintenance assessment or (as the case may be) a fresh maintenance assessment should be made, he shall proceed accordingly.
%
%(10) In making a maintenance assessment by virtue of subsection (9), a child support officer shall, if he is aware of any material change of circumstance since the decision being reviewed was taken, take account of that change of circumstance in making the assessment.
%
%% S 18(10A) inserted (22.1.96) by 1995 c 34 s 14(1)
%(10A) If a child support officer conducting a review under this section is satisfied that the maintenance assessment in question was not validly made he may cancel it with effect from the date on which it took effect.
%
%(11) The Secretary of State may make regulations—
%\begin{enumerate}\item[]
%($a$) as to the manner in which applications under this section are to be made;
%
%($b$) as to the procedure to be followed with respect to such applications; and
%
%($c$) with respect to reviews conducted under this section.
%\end{enumerate}
%
%(12) In this section “appropriate person” means—
%\begin{enumerate}\item[]
%($a$) the absent parent or person with care with respect to whom the maintenance assessment in question was, or remains, in force; or
%
%($b$) where the application for that assessment was made under section 7, either of those persons or the child concerned.
%\end{enumerate}
%
%\amendment{
%S. 18(8), (11) came into force 27.6.92.
%S. 18 came fully into force 5.4.93.
%
%Word inserted after s. 18(6)(b) (4.9.95) by the Child Support Act 1995 Sch.~3 para.~4.
%
%S. 18(6A) inserted (22.1.96) by the Child Support Act 1995 s. 13.
%
%S. 18(10A) inserted (22.1.96) by the Child Support Act 1995 s. 14(1).
%}
%
%%\subsection{19. Reviews at instigation of child support officers}
%%
%%(1) Where a child support officer is not conducting a review under section 16, 17 or 18 but is nevertheless satisfied that a maintenance assessment which is in force is defective by reason of—
%%\begin{enumerate}\item[]
%%($a$) having been made in ignorance of a material fact;
%%
%%($b$) having been based on a mistake as to a material fact; or
%%
%%($c$) being wrong in law,
%%\end{enumerate}
%%he may make a fresh maintenance assessment on the assumption that the person in whose favour the original assessment was made has made a fresh application for a maintenance assessment.
%%
%%(2) Where a child support officer is not conducting such a review but is nevertheless satisfied that if an application were to be made under section 17 or 18 it would be appropriate to make a fresh maintenance assessment, he may do so.
%%
%%%(3) Before making a fresh maintenance assessment under this section, a child support officer shall give to such persons as may be prescribed such notice of his proposal to make a fresh assessment as may be prescribed.
%%
%%\amendment{
%%S. 19(1), (2) came into force 5.4.93.  S. 19(3) is not yet in force.
%%}
%
%% S 19 substituted (22.1.96) by 1995 c 34 s 15
%\subsection{19. Reviews at instigation of child support officers}
%
%(1) Where a child support officer is not conducting a review under section 16, 17 or 18, he may nevertheless review—
%\begin{enumerate}\item[]
%($a$) a refusal to make a maintenance assessment,
%
%($b$) a refusal to review a maintenance assessment under section 17,
%
%($c$) a maintenance assessment (whether or not in force),
%
%($d$) a cancellation of a maintenance assessment, or
%
%($e$) a refusal to cancel a maintenance assessment,
%\end{enumerate}
%if he suspects that it may be defective for one or more of the reasons set out in subsection (2).
%
%(2) The reasons are that the refusal, assessment or cancellation—
%\begin{enumerate}\item[]
%($a$) was made in ignorance of a material fact;
%
%($b$) was based on a mistake as to a material fact; or
%
%($c$) was wrong in law.
%\end{enumerate}
%
%(3) If, on completing such a review, the child support officer is satisfied that the refusal, assessment or cancellation is defective for one or more of those reasons, he may—
%\begin{enumerate}\item[]
%($a$) take no further action;
%
%($b$) in the case of a maintenance assessment which has been cancelled, set aside the cancellation;
%
%($c$) make a maintenance assessment;
%
%($d$) make a fresh maintenance assessment;
%
%($e$) cancel the maintenance assessment in question.
%\end{enumerate}
%
%(4) Where a child support officer sets a cancellation aside under subsection~(3), the maintenance assessment in question shall have effect as if it had never been cancelled.
%
%(5) Any cancellation of a maintenance assessment under this section shall have effect from such date as may be determined by the child support officer.
%
%(6) Where a child support officer suspects that if an application for a review of a maintenance assessment were to be made under section 17 it would be appropriate to make one or more fresh maintenance assessments, he may review the maintenance assessment even though no application for its review has been made under that section.
%
%(7) If, on completing a review by virtue of subsection (6), the child support officer is satisfied that it would be appropriate to make one or more fresh maintenance assessments, he may do so.
%
%\amendment{
%S. 19 substituted (22.1.96) by the Child Support Act 1995 s. 15.
%}

% S 17 substituted for ss 17--19 (1.6.99) by 1998 c 41
\subsection[17. Decisions superseding earlier decisions --- \emph{1993 scheme version}]{17. Decisions superseding earlier decisions\\*\emph{1993 scheme version}}

(1) Subject to subsection (2), the following, namely—
\begin{enumerate}\item[]
($a$) any decision of the Secretary of State under section 11 or 12 or this section, whether as originally made or as revised under section 16;

($b$) any decision of an appeal tribunal under section 20; and

($c$) any decision of a Child Support Commissioner on an appeal from such a decision as is mentioned in paragraph ($b$),
\end{enumerate}
may be superseded by a decision made by the Secretary of State, either on an application made for the purpose or on his own initiative.

(2) In making a decision under subsection (1), the Secretary of State need not consider any issue that is not raised by the application or, as the case may be, did not cause him to act on his own initiative.

(3) Regulations may prescribe the cases and circumstances in which, and the procedure by which, a decision may be made under this section.

(4) Subject to subsection (5) and section 28ZC, a decision under this section shall take effect as from the date on which it is made or, where applicable, the date on which the application was made.

(5) Regulations may provide that, in prescribed cases or circumstances, a decision under this section shall take effect as from such other date as may be prescribed.

\amendment{
S. 17 substituted for ss. 17--19 (1.6.99) by the Social Security Act 1998 s. 41.
}

\subsection[17. Decisions superseding earlier decisions --- \emph{2003 scheme version}]{17. Decisions superseding earlier decisions\\*\emph{2003 scheme version}}

(1) Subject to subsection (2), the following, namely—
\begin{enumerate}\item[]
($a$) any decision of the Secretary of State under section 11 or 12 or this section, whether as originally made or as revised under section 16;

($b$) any decision of an appeal tribunal under section 20; 
%and  % Word repealed by 2000 c 19 Sch 9 Pt I

%($c$) any decision of a Child Support Commissioner on an appeal from such a decision as is mentioned in paragraph ($b$),

% S 17(1)(c)--(e) substituted for s 17(1)(c) by 2000 c 19 s 9(2)
($c$) any reduced benefit decision under section~46;

\begin{sloppypar}
($d$) any decision of an appeal tribunal on a referral under section~28D(1)($b$);
\end{sloppypar}

($e$) any decision of a Child Support Commissioner on an appeal from such a decision as is mentioned in paragraph~($b$)  or~($d$),
\end{enumerate}
may be superseded by a decision made by the Secretary of State, either on an application made for the purpose or on his own initiative.

(2) In making a decision under subsection (1), the Secretary of State need not consider any issue that is not raised by the application or, as the case may be, did not cause him to act on his own initiative.

(3) Regulations may prescribe the cases and circumstances in which, and the procedure by which, a decision may be made under this section.

%(4) Subject to subsection (5) and section 28ZC, a decision under this section shall take effect as from the date on which it is made or, where applicable, the date on which the application was made.

% S 17(4), (4A) substituted for s 17(4) by 2000 c 19 s 9(3)
(4) Subject to subsection~(5)  and section~28ZC, a decision under this section shall take effect as from the beginning of the maintenance period in which it is made or, where applicable, the beginning of the maintenance period in which the application was made.

(4A) In subsection~(4), a “maintenance period” is (except where a different meaning is prescribed for prescribed cases) a period of seven days, the first one beginning on the effective date of the first decision made by the Secretary of State under section~11 or (if earlier) his first default or interim maintenance decision (under section~12) in relation to the non-resident parent in question, and each subsequent one beginning on the day after the last day of the previous one.

(5) Regulations may provide that, in prescribed cases or circumstances, a decision under this section shall take effect as from such other date as may be prescribed.

\amendment{
S. 17(1)(c)--(e) substituted for s. 17(1)(c) and s. 17(4), (4A) substituted for s. 17(4) (10.11.00 for regulation-making purposes, 3.3.03 for 2003 scheme cases) by the Child Support, Pensions and Social Security Act 2000 s. 9.

Word ``and'' after s. 17(1)(b) repealed (3.3.03 for 2003 scheme cases) by the Child Support, Pensions and Social Security Act 2000 Sch. 9 Pt. I.
}

%\subsection{20. Appeals}
%
%(1) Any person who is aggrieved by the decision of a child support officer—
%\begin{enumerate}\item[]
%($a$) on a review under section 18;
%
%($b$) to refuse an application for such a review,
%\end{enumerate}
%may appeal to a child support appeal tribunal against that decision.
%
%(2) Except with leave of the chairman of a child support appeal tribunal, no appeal under this section shall be brought after the end of the period of 28 days beginning with the date on which notification was given of the decision in question.
%
%(3) Where an appeal under this section is allowed, the tribunal shall remit the case to the Secretary of State, who shall arrange for it to be dealt with by a child support officer.
%
%(4) The tribunal may, in remitting any case under this section, give such directions as it considers appropriate.
%
%% S 20(5) inserted temp (prosp) by 1998 c 14 Sch 6 para 9
%\emph{(5) In deciding an appeal under this section, the tribunal shall not take into account any circumstances not obtaining at the time when the decision appealed against was made.}
%
%\amendment{
%S. 20 came into force 5.4.93.
%
%S. 20(5) inserted (prosp. in relation to appeals brought between the passing of the Social Security Act 1998 and the commencement of s. 42 of that Act) by the Social Security Act 1998 Sch. 6 para. 9.
%}
%
%\subsection{20A. Lapse of appeals}
%
%(1) This section applies where—
%\begin{enumerate}\item[]
%($a$) a person has brought an appeal under section 20; and
%
%($b$) before the appeal is heard, the decision appealed against is reviewed under section 19.
%\end{enumerate}
%
%(2) If the child support officer conducting the review considers that the decision which he has made on the review is the same as that which would have been made on the appeal had every ground of the appeal succeeded, the appeal shall lapse.
%
%(3) In any other case, the review shall be of no effect and the appeal shall proceed accordingly.
%
%\amendment{
%S. 20A inserted (18.12.95) by the Child Support Act 1995 s. 16.
%}
%
%\subsection{21. Child support appeal tribunals}
%
%(1) There shall be tribunals to be known as child support appeal tribunals which shall, subject to any order made under section 45, hear and determine appeals under section 20
%and have such other functions as are conferred by this Act%  % Words added (2.12.96) by 1995 c 34 Sch 3 para 6
%.
%
%(2) The Secretary of State may make such regulations with respect to proceedings before child support appeal tribunals as he considers appropriate.
%
%(3) The regulations may in particular make provision—
%\begin{enumerate}\item[]
%($a$) as to procedure;
%
%($b$) for the striking out of appeals for want of prosecution;
%
%($c$) as to the persons entitled to appear and be heard on behalf of any of the parties;
%
%($d$) requiring persons to attend and give evidence or to produce documents;
%
%($e$) about evidence;
%
%($f$) for authorising the administration of oaths;
%
%($g$) as to confidentiality;
%
%($h$) for notification of the result of an appeal to be given to such persons as may be prescribed.
%\end{enumerate}
%
%(4) Schedule 3 shall have effect with respect to child support appeal tribunals.
%
%\amendment{
%S. 21(2)--(4) came into force 27.6.92, rest of section came into force 1.9.92.
%
%Words added to s. 21(1) (2.12.96) by the Child Support Act 1995 Sch. 3 para. 6.
%}

% S 20 substituted for ss 20--21 (1.6.99) by 1998 c 14 s 42
\subsection[20. Appeals to appeal tribunals --- \emph{1993 scheme version}]{20. Appeals to appeal tribunals\\*\emph{1993 scheme version}}

(1) Where an application for a maintenance assessment is refused, the person who made that application shall have a right of appeal to an appeal tribunal against the refusal.

(2) Where a maintenance assessment is in force—
\begin{enumerate}\item[]
($a$) the absent parent or person with care with respect to whom it was made; or

($b$) where the application for the assessment was made under section 7, either of them or the child concerned,
\end{enumerate}
shall have a right of appeal to an appeal tribunal against the amount of the assessment or the date from which the assessment takes effect.

(3) Where a maintenance assessment is cancelled, or an application for the cancellation of a maintenance assessment is refused—
\begin{enumerate}\item[]
($a$) the absent parent or person with care with respect to whom the maintenance assessment in question was, or remains, in force; or

($b$) where the application for that assessment was made under section 7, either of them or the child concerned,
\end{enumerate}
shall have a right of appeal to an appeal tribunal against the cancellation or refusal.

(4) A person with a right of appeal under this section shall be given such notice of that right and, in the case of a right conferred by subsection (1) or~(3), such notice of the decision as may be prescribed.

(5) Regulations may make—
\begin{enumerate}\item[]
($a$) provision as to the manner in which, and the time within which, appeals are to be brought; and

($b$) such provision with respect to proceedings before appeal tribunals as the Secretary of State considers appropriate.
\end{enumerate}

(6) The regulations may in particular make any provision of a kind mentioned in Schedule 5 to the Social Security Act 1998.

(7) In deciding an appeal under this section, an appeal tribunal—
\begin{enumerate}\item[]
($a$) need not consider any issue that is not raised by the appeal; and

($b$) shall not take into account any circumstances not obtaining at the time when the decision or assessment appealed against was made.
\end{enumerate}

\amendment{
S. 20 substituted for ss. 20--21 (1.6.99) by the Social Security Act 1998 s. 42.
}

% S 20 substituted by 2000 c 19 s 10
\subsection[20. Appeals to appeal tribunals --- \emph{2003 scheme version}]{20. Appeals to appeal tribunals\\*\emph{2003 scheme version}}

(1) A qualifying person has a right of appeal to an appeal tribunal against—
\begin{enumerate}\item[]
($a$) a decision of the Secretary of State under section~11, 12 or~17 (whether as originally made or as revised under section~16);

($b$) a decision of the Secretary of State not to make a maintenance calculation under section~11 or not to supersede a decision under section~17;

($c$) a reduced benefit decision under section~46;

($d$) the imposition (by virtue of section~41A) of a requirement to make penalty payments, or their amount;

($e$) the imposition (by virtue of section~47) of a requirement to pay fees.
\end{enumerate}

(2) In subsection~(1), “qualifying person” means—
\begin{enumerate}\item[]
($a$) in relation to paragraphs~($a$)  and ($b$)—
\begin{enumerate}\item[]
(i) the person with care, or~non-resident parent, with respect to whom the Secretary of State made the decision, or

(ii) in a case relating to a maintenance calculation which was applied for under section~7, either of those persons or the child concerned;
\end{enumerate}

($b$) in relation to paragraph~($c$), the person in respect of whom the benefits are payable;

($c$) in relation to paragraph~($d$), the parent who has been required to make penalty payments; and

($d$) in relation to paragraph~($e$), the person required to pay fees.
\end{enumerate}

(3) A person with a right of appeal under this section shall be given such notice as may be prescribed of—
\begin{enumerate}\item[]
($a$) that right; and

($b$) the relevant decision, or the imposition of the requirement.
\end{enumerate}

(4) Regulations may make—
\begin{enumerate}\item[]
($a$) provision as to the manner in which, and the time within which, appeals are to be brought; and

($b$) such provision with respect to proceedings before appeal tribunals as the Secretary of State considers appropriate.
\end{enumerate}

(5) The regulations may in particular make any provision of a kind mentioned in Schedule 5 to the Social Security Act 1998. 

(6) No appeal lies by virtue of subsection~(1)($c$)  unless the amount of the person’s benefit is reduced in accordance with the reduced benefit decision; and the time within which such an appeal may be brought runs from the date of notification of the reduction.

(7) In deciding an appeal under this section, an appeal tribunal—
\begin{enumerate}\item[]
($a$) need not consider any issue that is not raised by the appeal; and

($b$) shall not take into account any circumstances not obtaining at the time when the Secretary of State made the decision or imposed the requirement.
\end{enumerate}

(8) If an appeal under this section is allowed, the appeal tribunal may—
\begin{enumerate}\item[]
($a$) itself make such decision as it considers appropriate; or

($b$) remit the case to the Secretary of State, together with such directions (if any) as it considers appropriate.
\end{enumerate}

\amendment{
S. 20 substituted (10.11.00 for regulation-making purposes, 3.3.03 for 2003 scheme cases) by the Child Support, Pensions and Social Security Act 2000 s. 10.
}

\subsection{22. Child Support Commissioners}

(1) Her Majesty may from time to time appoint a Chief Child Support Commissioner and such number of other Child Support Commissioners as she may think fit.

(2) The Chief Child Support Commissioner and the other Child Support Commissioners shall be appointed from among persons who—
\begin{enumerate}\item[]
($a$) have a 10 year general qualification; or

% S 22(2)(a) substituted by 2007 c 15 Sch 10 para 22(2)(a)
%($a$) satisfy the judicial-appointment eligibility condition on a 7-year basis; or

($b$) are advocates or solicitors in Scotland of 
10 
%7  % Figure substituted by 2007 c 15 Sch 10 para 22(2)(b)
years' standing.
\end{enumerate}

(3) The Lord Chancellor, after consulting the Lord Advocate, may make such regulations with respect to proceedings before Child Support Commissioners as he considers appropriate.

(4) The regulations—
\begin{enumerate}\item[]
($a$) may, in particular, make any provision of a kind mentioned in 
%section~21(3)%
Schedule 5 to the Social Security Act 1998%  % Words substituted (prosp) by 1998 c 14 Sch 7 para 29
; and

($b$) shall provide that any hearing before a Child Support Commissioner shall be in public except in so far as the Commissioner for special reasons directs otherwise.
\end{enumerate}

(5) Schedule 4 shall have effect with respect to Child Support Commissioners.

\amendment{
S. 22(3), (4) came into force 27.6.92, rest of section came into force 1.9.92.

Words substituted in s. 22(4)(a) (1.6.99) by the Social Security Act 1998 Sch. 7 para. 29.

S. 22(2)(a) substituted and figure substituted in s. 22(2)(b) (prosp) by the Tribunals, Courts and Enforcement Act 2007 Sch. 10 para. 22(2).
}

\subsection{23. Child Support Commissioners for Northern Ireland}

(1) Her Majesty may from time to time
%, on the recommendation of the First Minister and deputy First Minister, acting jointly,  % Words inserted by 2002 c 26 Sch 3 para 21(a)
 appoint a Chief Child Support Commissioner for Northern Ireland and 
such number of other Child Support Commissioners for Northern Ireland as she may think fit.
%other Child Support Commissioners.   % Words substituted by 2002 c 26 Sch 3 para 21(b)

(2) The Chief Child Support Commissioner for Northern Ireland and the other Child Support Commissioners for Northern Ireland shall be appointed from among persons who are barristers or solicitors of not less than 
10 
%7  % Figure substituted by 2007 c 15 Sch 10 para 22(3)
years' standing.

(3) Schedule 4 shall have effect with respect to Child Support Commissioners for Northern Ireland, subject to the modifications set out in paragraph~8.

% S 23(4), (5) repealed (2.12.99) by 1998 c 47 Sch 15
%(4) Subject to any Order made after the passing of this Act by virtue of subsection (1)($a$)  of section 3 of the Northern Ireland Constitution Act 1973, the matters to which this subsection applies shall not be transferred matters for the purposes of that Act but shall for the purposes of subsection (2)  of that section be treated as specified in Schedule 3 to that Act.
%
%(5) Subsection (4)  applies to all matters relating to Child Support Commissioners, including procedure and appeals, other than those specified in paragraph 9 of Schedule 2 to the Northern Ireland Constitution Act 1973.

\amendment{
S. 23 came into force 1.9.92.

S. 23(4), (5) repealed (2.12.99) by the Northern Ireland Act 1998 Sch. 15.

Words inserted and substituted in s. 23(1) (prosp) by the Justice (Northern Ireland) Act 2002 Sch. 3 para. 21.

Figure substituted in s. 23(2) (prosp) by the Tribunals, Courts and Enforcement Act 2007 Sch. 10 para. 22(3).
}

\subsection{23A. Redetermination of appeals}

(1) This section applies where an application is made to a person under section~24(6)($a$)  for leave to appeal from a decision of an appeal tribunal.

(2) If the person who constituted, or was the chairman of, the appeal tribunal considers that the decision was erroneous in law, he may set aside the decision and refer the case either for redetermination by the tribunal or for determination by a differently constituted tribunal.

(3) If each of the principal parties to the case expresses the view that the decision was erroneous in point of law, the person shall set aside the decision and refer the case for determination by a differently constituted tribunal.

(4) The “principal parties” are—
\begin{enumerate}\item[]
($a$) the Secretary of State; and

($b$) those who are qualifying persons for the purposes of section~20(2)  in relation to the decision in question.
\end{enumerate}

\amendment{
S. 23A inserted (5.2.01) by the Child support, Pensions and Social Security Act 2000 s. 11.
}

\subsection{24. Appeal to Child Support Commissioner}

(1) Any person who is aggrieved by a decision of 
%a child support appeal tribunal, and any child support officer%
an appeal tribunal, and the Secretary of State%  % Words substituted (1.6.99) by 1998 c 14 Sch 7 para 30(1)
, may appeal to a Child Support Commissioner on a question of law.

% S 24(1A) inserted (2.12.96) by 1995 c 34 Sch 3 para 7(2), repealed (1.6.99) by 1998 c 14 Sch 7 para 30(2)
%(1A) The Secretary of State may appeal to a Child Support Commissioner on a question of law in relation to any decision of a child support appeal tribunal made in connection with an application for a departure direction.

(2) Where, on an appeal under this section, a Child Support Commissioner holds that the decision appealed against was wrong in law he shall set it aside.

(3) Where a decision is set aside under subsection (2), the Child Support Commissioner may—
\begin{enumerate}\item[]
($a$) if he can do so without making fresh or further findings of fact, give the decision which he considers should have been given by 
%the child support appeal tribunal%
the appeal tribunal%  % Words substituted (1.6.99) by 1998 c 14 Sch 7 para 30(3)(a)
;

($b$) if he considers it expedient, make such findings and give such decision as he considers appropriate in the light of those findings; or

%($c$) refer the case, with directions for its determination, to a child support officer or, if he considers it appropriate, to a child support appeal tribunal.

% S 24(3)(c), (d) substituted for s 24(3)(c) (2.12.96) by 1995 c 34 Sch 3 para 7(3)
($c$) on an appeal by the Secretary of State, refer the case to 
%a child support appeal tribunal 
an appeal tribunal  % Words substituted (1.6.99) by 1998 c 14 Sch 7 para 30(3)(b)
with directions for its determination; or

($d$) on any other appeal, refer the case to 
%a child support officer 
the Secretary of State  % Words substituted (1.6.99) by 1998 c 14 Sch 7 para 30(3)(c)
or, if he considers it appropriate, to 
%a child support appeal tribunal 
an appeal tribunal  % Words substituted (1.6.99) by 1998 c 14 Sch 7 para 30(3)(b)
with directions for its determination.
\end{enumerate}

%(4) Any reference under subsection (3)  to a child support officer shall, subject to any direction of the Child Support Commissioner, be to a child support officer who has taken no part in the decision originally appealed against.

% S 24(4) substituted (1.6.99) by 1998 c 14 Sch 7 para 30(4)
(4) The reference under subsection (3) to the Secretary of State shall, subject to any direction of the Child Support Commissioner, be to an officer of his, or a person providing him with services, who has taken no part in the decision originally appealed against.

(5) On a reference under subsection (3)  to 
%a child support appeal tribunal% 
an appeal tribunal%  % Words substituted (1.6.99) by 1998 c 14 Sch 7 para 30(5)
, the tribunal shall, subject to any direction of the Child Support Commissioner, consist of persons who were not members of the tribunal which gave the decision which has been appealed against.

(6) No appeal lies under this section without the leave—
\begin{enumerate}\item[]
($a$) of the person 
%who was the chairman of the child support appeal tribunal 
who constituted, or was the chairman of, the appeal tribunal  % Words substituted (1.6.99) by 1998 c 14 Sch 7 para 30(6)(a)
when the decision appealed against was given or of 
%such other chairman of a child support appeal tribunal 
such other person  % Words substituted (1.6.99) by 1998 c 14 Sch 7 para 30(6)(b)
as may be determined in accordance with regulations made by the Lord Chancellor; or

($b$) subject to and in accordance with regulations so made, of a Child Support Commissioner.
\end{enumerate}

(7) The Lord Chancellor may by regulations make provision as to the manner in which, and the time within which, appeals under this section are to be brought and applications for leave under this section are to be made.

(8) Where a question which would otherwise fall to be determined by 
%a child support officer 
the Secretary of State  % Words substituted (1.6.99) by 1998 c 14 Sch 7 para 30(3)(c)
first arises in the course of an appeal to a Child Support Commissioner, he may, if he thinks fit, determine it even though it has not been considered by 
%a child support officer% 
the Secretary of State%  % Words substituted (1.6.99) by 1998 c 14 Sch 7 para 30(3)(c)
.

(9) Before making any regulations under subsection (6)  or (7), the Lord Chancellor shall consult the Lord Advocate.

\amendment{
S. 24(6), (7) came into force 27.6.92; s. 24(9) came into force 1.9.92; s. 24 came fully into force 5.4.93.

S. 24(1A) inserted and s. 24(3)(c), (d) substituted for s. 24(3)(c) (2.12.96) by the Child Support Act 1995 Sch. 3 para. 17.

Words substituted in s. 24(1), s. 24(1A) repealed, words substituted in s. 24(3), s. 24(4) substituted and words substituted in s. 24(5), (6), (8) (1.6.99) by the Social Security Act 1998 Sch. 7 para. 30.
}

\subsection{25. Appeal from Child Support Commissioner on question of law}

(1) An appeal on a question of law shall lie to the appropriate court from any decision of a Child Support Commissioner.

(2) No such appeal may be brought except—
\begin{enumerate}\item[]
($a$) with leave of the Child Support Commissioner who gave the decision or, where regulations made by the Lord Chancellor so provide, of a Child Support Commissioner selected in accordance with the regulations; or

($b$) if the Child Support Commissioner refuses leave, with the leave of the appropriate court.
\end{enumerate}

(3) An application for leave to appeal under this section against a decision of a Child Support Commissioner (“the appeal decision”) may only be made by—
\begin{enumerate}\item[]
($a$) a person who was a party to the proceedings in which the original decision, or appeal decision, was given;

($b$) the Secretary of State; or

($c$) any other person who is authorised to do so by regulations made by the Lord Chancellor.
\end{enumerate}

% S 25(3A), (3B) inserted (4.9.95) by 1995 c 34 Sch 3 para 8(1)
(3A) The Child Support Commissioner to whom an application for leave to appeal under this section is made shall specify as the appropriate court either the Court of Appeal or the Court of Session.

(3B) In determining the appropriate court, the Child Support Commissioner shall have regard to the circumstances of the case, and in particular the convenience of the persons who may be parties to the appeal.

(4) In this section—
\begin{enumerate}\item[]
    “appropriate court”% 
%means the Court of Appeal unless in a particular case the Child Support Commissioner to whom the application for leave is made directs that, having regard to the circumstances of the case, and in particular the convenience of the persons who may be parties to the appeal, the appropriate court is the Court of Session
, except in subsections (3A) and (3B), means the court specified in accordance with those subsections%  % Words substituted (4.9.95) by 1995 c 34 Sch 3 para 8(2)
; and

    “original decision” means the decision to which the appeal decision in question relates. 
\end{enumerate}

(5) The Lord Chancellor may by regulations make provision with respect to—
\begin{enumerate}\item[]
($a$) the manner in which and the time within which applications must be made to a Child Support Commissioner for leave under this section; and

($b$) the procedure for dealing with such applications.
\end{enumerate}

(6) Before making any regulations under subsection (2), (3)  or (5), the Lord Chancellor shall consult the Lord Advocate.

\amendment{
S. 25(2)(a), (3)(c), (5), (6) came into force 27.6.92; s. 25 came fully into force 5.4.93.

S. 25(3A), (3B) inserted and words substituted in definition of ``appropriate court'' in s.~25(4) (4.9.95) by the Child Support Act 1995 Sch.~3 para.~8.
}

\subsection{26. Disputes about parentage}

(1) [\emph{1993 scheme version}] Where a person who is alleged to be a parent of the child with respect to whom an application for a maintenance assessment has been made (“the alleged parent”) denies that he is one of the child’s parents, 
%the child support officer concerned 
the Secretary of State  % Words substituted (1.6.99) by 1998 c 14 Sch 7 para 31(1)
shall not make a maintenance assessment on the assumption that the alleged parent is one of the child’s parents unless the case falls within one of those set out in subsection (2).

(1) [\emph{2003 scheme version}] Where a person who is alleged to be a parent of the child with respect to whom an application for a 
%maintenance assessment 
maintenance calculation  % Words substituted by 2000 c 19 s 1(2)(a)
has been made 
or treated as made  % Words inserted by 2000 c 19 Sch 3 para 11(8)
(“the alleged parent”) denies that he is one of the child’s parents, 
%the child support officer concerned 
the Secretary of State  % Words substituted (1.6.99) by 1998 c 14 Sch 7 para 31(1)
shall not make a 
%maintenance assessment 
maintenance calculation  % Words substituted by 2000 c 19 s 1(2)(a)
on the assumption that the alleged parent is one of the child’s parents unless the case falls within one of those set out in subsection (2).

(2) The Cases are—
\begin{enumerate}\item[]
% Cases A1--A3 inserted by 2000 c 19 s 15(1)
\subsubsection*{Case A1}

Where—
\begin{enumerate}\item[]
($a$) the child is habitually resident in England and Wales;

($b$) the Secretary of State is satisfied that the alleged parent was married to the child’s mother at some time in the period beginning with the conception and ending with the birth of the child; and

($c$) the child has not been adopted.
\end{enumerate}

\subsubsection*{Case A2}

Where—
\begin{enumerate}\item[]
($a$) the child is habitually resident in England and Wales;

($b$) the alleged parent has been registered as father of the child under section~10 or~10A of the Births and Deaths Registration Act 1953, or in any register kept under section~13 (register of births and still-births) or section~44 (Register of Corrections Etc) of the Registration of Births, Deaths and Marriages (Scotland) Act 1965, or under Article 14 or~18(1)($b$)(ii)  of the Births and Deaths Registration (Northern Ireland) Order 1976; and

($c$) the child has not subsequently been adopted.
\end{enumerate}

\subsubsection*{Case A3}

Where the result of a scientific test (within the meaning of section~27A) taken by the alleged parent would be relevant to determining the child’s parentage, and the alleged parent—
\begin{enumerate}\item[]
($a$) refuses to take such a test; or

($b$) has submitted to such a test, and it shows that there is no reasonable doubt that the alleged parent is a parent of the child.
\end{enumerate}

    \subsubsection*{Case A}

    Where the alleged parent is a parent of the child in question by virtue of having adopted him.

    \subsubsection*{Case B}

    Where the alleged parent is a parent of the child in question by virtue of an order under section 30 of the Human Fertilisation and Embryology Act 1990 (parental orders in favour of gamete donors).

% Case B1 inserted by 2000 c 19 s 15(2)
\subsubsection*{Case B1}

Where the Secretary of State is satisfied that the alleged parent is a parent of the child in question by virtue of section~27 or~28 of that Act (meaning of “mother” and of “father” respectively).

    \subsubsection*{Case C}

    Where—
\begin{enumerate}\item[]
    ($a$) 
    either—
\begin{enumerate}\item[]
    (i) 
    a declaration that the alleged parent is a parent of the child in question (or a declaration which has that effect) is in force under section 
55A or  % Words inserted by 2000 c 19 Sch 8 para 12
56 of the Family Law Act 1986 
or Article 32 of the Matrimonial and Family Proceedings (Northern Ireland) Order 1989  % Words inserted (4.11.96) by SI 1995/756 art 13(a)
(declarations of parentage); or

    (ii) 
    a declarator by a court in Scotland that the alleged parent is a parent of the child in question (or a declarator which has that effect) is in force; and
\end{enumerate}

    ($b$) 
    the child has not subsequently been adopted.
\end{enumerate}

% Case D repealed by 2000 c 19 Sch 9 Pt IX
%    \subsubsection*{Case D}
%
%    Where—
%\begin{enumerate}\item[]
%    ($a$) 
%    a declaration to the effect that the alleged parent is one of the parents of the child in question has been made under section 27; and
%
%    ($b$) 
%    the child has not subsequently been adopted.
%\end{enumerate}

    \subsubsection*{Case E}

    Where—
\begin{enumerate}\item[]
    ($a$) 
    the child is habitually resident in Scotland;

    ($b$) 
%    the child support officer 
the Secretary of State  % Words substituted (1.6.99) by 1998 c 14 Sch 7 para 31(2)
is satisfied that one or other of the presumptions set out in section 5(1)  of the Law Reform (Parent and Child) (Scotland) Act 1986 applies; and

    ($c$) 
    the child has not subsequently been adopted.
\end{enumerate}

    \subsubsection*{Case F}

    Where—
\begin{enumerate}\item[]
    ($a$) 
    the alleged parent has been found, or adjudged, to be the father of the child in question—
\begin{enumerate}\item[]
    (i) 
    in proceedings before any court in England and Wales which are relevant proceedings for the purposes of section 12 of the Civil Evidence Act 1968
or in proceedings before any court in Northern Ireland which are relevant proceedings for the purposes of section 8 of the Civil Evidence Act (Northern Ireland) 1971%  % Words inserted (4.11.96) by SI 1995/756 art 13(b)
; or

    (ii) 
    in affiliation proceedings before any court in the United Kingdom,
\end{enumerate}

    (whether or not he offered any defence to the allegation of paternity) and that finding or adjudication still subsists; and

    ($b$) 
    the child has not subsequently been adopted. 
\end{enumerate}
\end{enumerate}

(3) In this section—
\begin{enumerate}\item[]
    “adopted” means adopted within the meaning of Part IV of the Adoption Act 1976 
or Chapter IV of Part I of the Adoption and Children Act 2002  % Words inserted by 2002 c 38 Sch 3 para 81
or, in relation to Scotland, Part IV of the Adoption (Scotland) Act 1978%
%or Chapter III of Part I of the Adoption and Children (Scotland) Act 2007%  % Words inserted by 2007 asp 4 Sch 2 para 7
; and

    “affiliation proceedings”, in relation to Scotland, means any action of affiliation and aliment. 
\end{enumerate}

\amendment{
S. 26 came into force 5.4.93.

Words inserted in Cases C, F in s. 26(2) (4.11.96) by the Children (Northern Ireland Consequential Amendments) Order 1995 art. 13.

Words substituted in s. 26(1) and in Case E in s. 26(2) (1.6.99) by the Social Security Act 1998 Sch. 7 para. 31.

Cases A1--A3, B1 inserted in s. 26(2) (31.1.01) by the Child Support, Pensions and Social Security Act 2000 s. 15.

Words inserted in Case C in s. 26(2) (1.4.01) by the Child Support, Pensions and Social Security Act 2000 Sch. 8 para. 12.

Case D in s. 26(2) repealed (1.4.01) by the Child Support, Pensions and Social Security Act 2000 Sch. 9 Pt. IX.

Words inserted in s. 26(1) (3.3.03 for 2003 scheme cases) by the Child Support, Pensions and Social Security Act 2000 Sch. 3 para. 11(8).

Words inserted in s. 26(3) (30.12.05) by the Adoption and Children Act 2002 Sch. 3 para. 81.

Words inserted in s. 26(3) (prosp) by the Adoption and Children (Scotland) Act 2007 Sch. 2 para. 7.
}

%\subsection{27. Reference to court for declaration of parentage}
%
%%(1) Where—
%%\begin{enumerate}\item[]
%%($a$) a child support officer is considering whether to make a maintenance assessment with respect to a person who is alleged to be a parent of the child, or one of the children, in question (“the alleged parent”);
%%
%%($b$) the alleged parent denies that he is one of the child’s parents; and
%%
%%($c$) the child support officer is not satisfied that the case falls within one of those set out in section 26(2),
%%\end{enumerate}
%%the Secretary of State or the person with care may apply to the court for a declaration as to whether or not the alleged parent is one of the child’s parents.
%
%% S 27(1), (1A) substituted for s 27(1) (4.9.95) by 1995 c 34 s 20(2)
%(1) Subsection (1A) applies in any case where—
%\begin{enumerate}\item[]
%($a$) an application for a maintenance assessment has been made, or a maintenance assessment is in force, with respect to a person (“the alleged parent”) who denies that he is a parent of a child with respect to whom the application or assessment was made; and
%
%($b$) 
%%a child support officer to whom the case is referred 
%the Secretary of State  % Words substituted (1.6.99) by 1998 c 14 Sch 7 para 32
%is not satisfied that the case falls within one of those set out in section 26(2).
%\end{enumerate}
%
%(1A) In any case where this subsection applies, the Secretary of State or the person with care may apply to the court for a declaration as to whether or not the alleged parent is one of the child’s parents.
%
%(2) If, on hearing any application under subsection 
%%(1)
%(1A)%  % Word substituted (4.9.95) by 1995 c 34 s 20(3)
%, the court is satisfied that the alleged parent is, or is not, a parent of the child in question it shall make a declaration to that effect.
%
%%(3) A declaration under this section shall have effect only for the purposes of this Act.
%
%% S 27(3) substituted (4.9.95) by 1995 c 34 s 20(4)
%(3) A declaration under this section shall have effect only for the purposes of—
%\begin{enumerate}\item[]
%($a$) this Act; and
%
%($b$) proceedings in which a court is considering whether to make a maintenance order in the circumstances mentioned in subsection (6),~(7) or (8) of section 8.
%\end{enumerate}
%
%(4) In this section “court” means, subject to any provision made under Schedule 11 to the Children Act 1989 (jurisdiction of courts with respect to certain proceedings relating to children) the High Court, a county court or a magistrates' court.
%
%(5) In the definition of “relevant proceedings” in section 12(5)  of the Civil Evidence Act 1968 (findings of paternity etc.\ as evidence in civil proceedings) the following paragraph shall be added at the end—
%\begin{quotation}
%“($d$) section 27 of the Child Support Act 1991.”
%\end{quotation}
%
%(6) This section does not apply to Scotland.
%
%\amendment{
%S. 27 came into force 5.4.93.
%
%S. 27(1), (1A) substituted for s. 27(1), word substituted in s. 27(2) and s.27(3) substituted (4.9.95) by the Child Support Act 1995 s.~20(1)--(4).
%
%Words substituted in s. 27(1) (1.6.99) by the Social Security Act 1998 Sch. 7 para. 32.
%}

% S 27 substituted by 2000 c 19 Sch 8 para 13
\subsection{27. Applications for declaration of parentage under Family Law Act 1986}

(1) This section applies where—
\begin{enumerate}\item[]
($a$) an application for a maintenance calculation has been made (or is treated as having been made), or a maintenance calculation is in force, with respect to a person (“the alleged parent”) who denies that he is a parent of a child with respect to whom the application or calculation was made or treated as made;

($b$) the Secretary of State is not satisfied that the case falls within one of those set out in section~26(2); and

($c$) the Secretary of State or the person with care makes an application for a declaration under section~55A of the Family Law Act 1986 as to whether or not the alleged parent is one of the child’s parents.
\end{enumerate}

(2) Where this section applies—
\begin{enumerate}\item[]
($a$) if it is the person with care who makes the application, she shall be treated as having a sufficient personal interest for the purposes of subsection~(3)  of that section; and

($b$) if it is the Secretary of State who makes the application, that subsection shall not apply.
\end{enumerate}

(3) This section does not apply to Scotland.

\amendment{
S. 27 substituted (1.4.01) by the Child Support, Pensions and Social Security Act 2000 Sch. 8 para. 13.
}

% S 27A inserted (4.9.95) by 1995 c 34 s 21
\subsection{27A. Recovery of fees for scientific tests}

(1) This section applies in any case where—
\begin{enumerate}\item[]
($a$) [\emph{1993 scheme version}] an application for a maintenance assessment has been made or a maintenance assessment is in force;

($a$) [\emph{2003 scheme version}] an application for a 
%maintenance assessment 
maintenance calculation  % Words substituted by 2000 c 19 s 1(2)(a)
has been made 
or treated as made  % Words inserted by 2000 c 19 Sch 3 para 11(9)(a)
or a 
%maintenance assessment 
maintenance calculation  % Words substituted by 2000 c 19 s 1(2)(a)
is in force;

($b$) [\emph{1993 scheme version}] scientific tests have been carried out (otherwise than under a direction or in response to a request) in relation to bodily samples obtained from a person who is alleged to be a parent of a child with respect to whom the application or assessment is made;

($b$) [\emph{2003 scheme version}] scientific tests have been carried out (otherwise than under a direction or in response to a request) in relation to bodily samples obtained from a person who is alleged to be a parent of a child with respect to whom the application or 
%assessment 
calculation  % Words substituted by 2000 c 19 s 1(2)(b)
is made
or, as the case may be, treated as made%  % Words inserted by 2000 c 19 Sch 3 para 11(9)(b)
;

($c$) the results of the tests do not exclude the alleged parent from being one of the child’s parents; and

($d$) one of the conditions set out in subsection (2) is satisfied.
\end{enumerate}

(2) The conditions are that—
\begin{enumerate}\item[]
($a$) the alleged parent does not deny that he is one of the child’s parents;

($b$) in proceedings under 
%section 27%
section~55A of the Family Law Act 1986%  % Words substituted by 2000 c 19 Sch 8 para 14
, a court has made a declaration that the alleged parent is a parent of the child in question; or

($c$) in an action under section 7 of the Law Reform (Parent and Child) (Scotland) Act 1986, brought by the Secretary of State by virtue of section 28, a court has granted a decree of declarator of parentage to the effect that the alleged parent is a parent of the child in question.
\end{enumerate}

(3) In any case to which this section applies, any fee paid by the Secretary of State in connection with scientific tests may be recovered by him from the alleged parent as a debt due to the Crown.

(4) In this section—
\begin{enumerate}\item[]
“bodily sample” means a sample of bodily fluid or bodily tissue taken for the purpose of scientific tests;

“direction” means a direction given by a court under section 20 of the Family Law Reform Act 1969 (tests to determine paternity);

“request” means a request made by a court under section 70 of the Law Reform (Miscellaneous Provisions) (Scotland) Act 1990 (blood and other samples in civil proceedings); and

“scientific tests” means scientific tests made with the object of ascertaining the inheritable characteristics of bodily fluids or bodily tissue.
\end{enumerate}

(5) Any sum recovered by the Secretary of State under this section shall be paid by him into the Consolidated Fund.

\amendment{
S. 27A inserted (4.9.95) by the Child Support Act 1995 s.~21.

Words substituted in s. 27A (1.4.01) by the Child Support, Pensions and Social Security Act 2000 Sch. 8 para. 14.

Words inserted in s. 27A(1)(a), (b) (3.3.03 for 2003 scheme cases) by the Child Support, Pensions and Social Security Act 2000 Sch. 3 para. 11(9).

}

\subsection{28. Power of Secretary of State to initiate or defend actions of declarator: Scotland}

%(1) Where—
%\begin{enumerate}\item[]
%($a$) a child support officer is considering whether to make a maintenance assessment with respect to a person who is alleged to be a parent of the child, or one of the children, in question (“the alleged parent”);
%
%($b$) the alleged parent denies that he is a parent of the child in question; and
%
%($c$) the child support officer is not satisfied that the case falls within one of those set out in section 26(2),
%\end{enumerate}
%the Secretary of State may bring an action for declarator of parentage under section 7 of the Law Reform (Parent and Child) (Scotland) Act 1986.

%S 28(1), (1A) substituted for s 28(1) (4.9.95) by 1995 c 34 s 20(6)
(1) Subsection (1A) applies in any case where—
\begin{enumerate}\item[]
($a$) [\emph{1993 scheme version}] an application for a maintenance assessment has been made, or a maintenance assessment is in force, with respect to a person (“the alleged parent”) who denies that he is a parent of a child with respect to whom the application or assessment was made; and

($a$) [\emph{2003 scheme version}] an application for a 
%maintenance assessment 
maintenance calculation  % Words substituted by 2000 c 19 s 1(2)(a)
has been made
or treated as made%  % Words inserted by 2000 c 19 Sch 3 para 11(10)(a)
, or a 
%maintenance assessment 
maintenance calculation  % Words substituted by 2000 c 19 s 1(2)(a)
is in force, with respect to a person (“the alleged parent”) who denies that he is a parent of a child with respect to whom the application 
%or assessment was made
was made or treated as made or the calculation was made%  % Words substituted by 2000 c 19 Sch 3 para 11(10)(b)
; and

($b$) 
%a child support officer to whom the case is referred 
the Secretary of State  % Words substituted (1.6.99) by 1998 c 14 Sch 7 para 33
is not satisfied that the case falls within one of those set out in section 26(2).
\end{enumerate}

(1A) In any case where this subsection applies, the Secretary of State may bring an action for declarator of parentage under section 7 of the Law Reform (Parent and Child) (Scotland) Act 1986.

(2) The Secretary of State may defend an action for declarator of non-parentage or illegitimacy brought by a person named as the alleged parent in an application for a 
%maintenance assessment 
\emph{maintenance calculation}  % Words substituted by 2000 c 19 s 1(2)(a)
or in a 
%maintenance assessment 
\emph{maintenance calculation}  % Words substituted by 2000 c 19 s 1(2)(a)
which is in force.  % Words inserted (4.9.95) by 1995 c 34 s 20(7)

(3) This section applies to Scotland only.

\amendment{
S. 28 came into force 5.4.93.

S. 28(1), (1A) substituted for s. 28(1) and words inserted in s. 28(2) (4.9.95) by the Child Support Act 1995 s.~20(5)--(7).

Words substituted in s. 28(1) (1.6.99) by the Social Security Act 1998 Sch. 7 para. 33.

Words inserted and substituted in s. 28(1)(a) (3.3.03 for 2003 scheme cases) by the Child Support, Pensions and Social Security Act 2000 Sch. 3 para. 11(10).
}

% Ss 28ZA, 28ZB inserted (1.6.99) by 1998 c 14 s 43
\section[\itshape Decisions and appeals dependent on other cases]{\itshape\sloppy Decisions and appeals dependent on other cases}

\amendment{
Ss. 28ZA, 28ZB inserted (1.6.99) by the Social Security Act 1998 s. 43.
}

\subsection{28ZA. Decisions involving issues that arise on appeal in other cases}

(1) This section applies where—
\begin{enumerate}\item[]
($a$) [\emph{1993 scheme version}] a decision by the Secretary of State falls to be made under section~11,~12, 16 or 17 in relation to a maintenance assessment; and

($a$) [\emph{2003 scheme version}] a decision by the Secretary of State falls to be made under section~11,~12, 16 or 17 
%in relation to a maintenance assessment
or with respect to a reduced benefit decision under section~46%  % Words substituted by 2000 c 19 Sch 3 para 11(11)(a)
; and

($b$) [\emph{1993 scheme version}] an appeal is pending against a decision given in relation to a different maintenance assessment by a Child Support Commissioner or a court.

($b$) [\emph{2003 scheme version}] an appeal is pending against a decision given in relation to a different matter by a Child Support Commissioner or~a court.
\end{enumerate}

(2) If the Secretary of State considers it possible that the result of the appeal will be such that, if it were already determined, it would affect the decision in some way—
\begin{enumerate}\item[]
($a$) he need not, except in such cases or circumstances as may be prescribed, make the decision while the appeal is pending;

($b$) he may, in such cases or circumstances as may be prescribed, make the decision on such basis as may be prescribed.
\end{enumerate}

(3) Where the Secretary of State acts in accordance with subsection (2)($b$), following the determination of the appeal he shall if appropriate revise his decision (under section 16) in accordance with that determination.

(4) For the purposes of this section, an appeal against a decision is pending if—
\begin{enumerate}\item[]
($a$) an appeal against the decision has been brought but not determined;

($b$) an application for leave to appeal against the decision has been made but not determined; or

($c$) in such circumstances as may be prescribed, an appeal against the decision has not been brought (or, as the case may be, an application for leave to appeal against the decision has not been made) but the time for doing so has not yet expired.
\end{enumerate}

(5) In paragraphs ($a$), ($b$) and ($c$) of subsection (4), any reference to an appeal, or an application for leave to appeal, against a decision includes a reference to—
\begin{enumerate}\item[]
($a$) an application for, or for leave to apply for, judicial review of the decision under section 31 of the Supreme Court Act 1981; or

($b$) an application to the supervisory jurisdiction of the Court of Session in respect of the decision.
\end{enumerate}

\amendment{
Words substituted in s. 28ZA(1)(a) and s. 28ZA(1)(b) substituted (3.3.03 for 2003 scheme cases) by the Child Support, Pensions and Social Security Act 2000 Sch. 3 para. 11(11).
}

\subsection{28ZB. Appeals involving issues that arise on appeal in other cases}

(1) This section applies where—
\begin{enumerate}\item[]
($a$) [\emph{1993 scheme version}] an appeal (“appeal $\mathcal{A}$”) in relation to a decision falling within section~20(1) or (3), or an assessment falling within section 20(2), is made to an appeal tribunal, or from an appeal tribunal to a Child Support Commissioner; and

% S 28ZB(1)(a) substituted by 2000 c 19 Sch 3 para 11(12)(a)
($a$) [\emph{2003 scheme version}] an appeal (“appeal $\mathcal{A}$”) in relation to a decision or the imposition of a requirement falling within section~20(1)  is made to an appeal tribunal, or from an appeal tribunal to a Child Support Commissioner;
and

($b$) an appeal (“appeal $\mathcal{B}$”) is pending against a decision given in a different case by a Child Support Commissioner or a court.
\end{enumerate}

(2) If the Secretary of State considers it possible that the result of appeal $\mathcal{B}$ will be such that, if it were already determined, it would affect the determination of appeal $\mathcal{A}$, he may serve notice requiring the tribunal or Child Support Commissioner—
\begin{enumerate}\item[]
($a$) not to determine appeal $\mathcal{A}$ but to refer it to him; or

($b$) to deal with the appeal in accordance with subsection (4).
\end{enumerate}

(3) Where appeal $\mathcal{A}$ is referred to the Secretary of State under subsection~(2)($a$), following the determination of appeal $\mathcal{B}$ and in accordance with that determination, he shall if appropriate—
\begin{enumerate}\item[]
($a$) in a case where appeal $\mathcal{A}$ has not been determined by the tribunal, revise (under section 16) his decision which gave rise to that appeal; or

($b$) in a case where appeal $\mathcal{A}$ has been determined by the tribunal, make a decision (under section 17) superseding the tribunal’s decision.
\end{enumerate}

(4) Where appeal $\mathcal{A}$ is to be dealt with in accordance with this subsection, the appeal tribunal or Child Support Commissioner shall either—
\begin{enumerate}\item[]
($a$) stay appeal $\mathcal{A}$ until appeal $\mathcal{B}$ is determined; or

($b$) if the tribunal or Child Support Commissioner considers it to be in the interests of the appellant to do so, determine appeal $\mathcal{A}$ as if—
\begin{enumerate}\item[]
(i) appeal $\mathcal{B}$ had already been determined; and

(ii) the issues arising on appeal $\mathcal{B}$ had been decided in the way that was most unfavourable to the appellant.
\end{enumerate}
\end{enumerate}

[\emph{1993 scheme version}] In this subsection “the appellant” means the person who appealed or, as the case may be, first appealed against the decision or assessment mentioned in subsection (1)($a$).

[\emph{2003 scheme version}] In this subsection “the appellant” means the person who appealed or, as the case may be, first appealed against the decision 
%or assessment 
or the imposition of the requirement  % Words substituted by 2000 c 19 Sch 3 para 11(12)(b)
mentioned in subsection (1)($a$).

(5) Where the appeal tribunal or Child Support Commissioner acts in accordance with subsection (4)($b$), following the determination of appeal $\mathcal{B}$ the Secretary of State shall, if appropriate, make a decision (under section~17) superseding the decision of the tribunal or Child Support Commissioner in accordance with that determination.

(6) For the purposes of this section, an appeal against a decision is pending if—
\begin{enumerate}\item[]
($a$) an appeal against the decision has been brought but not determined;

($b$) an application for leave to appeal against the decision has been made but not determined; or

($c$) in such circumstances as may be prescribed, an appeal against the decision has not been brought (or, as the case may be, an application for leave to appeal against the decision has not been made) but the time for doing so has not yet expired.
\end{enumerate}

(7) In this section—
\begin{enumerate}\item[]
($a$) the reference in subsection (1)($a$) to an appeal to a Child Support Commissioner includes a reference to an application for leave to appeal to a Child Support Commissioner; and

($b$) any reference in paragraph ($a$), ($b$) or ($c$) of subsection (6) to an appeal, or to an application for leave to appeal, against a decision includes a reference to—
\begin{enumerate}\item[]
(i) an application for, or for leave to apply for, judicial review of the decision under section 31 of the Supreme Court Act 1981; or

(ii) an application to the supervisory jurisdiction of the Court of Session in respect of the decision.
\end{enumerate}
\end{enumerate}

(8) Regulations may make provision supplementing that made by this section.

\amendment{
S. 28ZB(1)(a) substituted and words substituted in s. 28ZB(4)  (3.3.03 for 2003 scheme cases) by the Child Support, Pensions and Social Security Act 2000 Sch. 3 para. 11(12).

\medskip

Ss. 28ZC, 28ZD inserted (1.6.99) by the Social Security Act 1998 s. 44.
}

% Ss 28ZC, 28ZD inserted (1.6.99) by 1998 c 14 s 44
\section{\itshape Cases of error}

\subsection{28ZC. Restrictions on liability in certain cases of error}

(1) Subject to subsection (2), this section applies where—
\begin{enumerate}\item[]
($a$) the effect of the determination, whenever made, of an appeal to a Child Support Commissioner or the court (“the relevant determination”) is that the adjudicating authority’s decision out of which the appeal arose was erroneous in point of law; and

($b$) after the date of the relevant determination a decision falls to be made by the Secretary of State in accordance with that determination (or would, apart from this section, fall to be so made)—
\begin{enumerate}\item[]
(i) [\emph{1993 scheme version}] with respect to an application for a maintenance assessment (made after the commencement date);

(i) [\emph{2003 scheme version}] with respect to an application for a 
%maintenance assessment 
maintenance calculation  % Words substituted by 2000 c 19 s 1(2)(a)
(made after the commencement date)
or one treated as having been so made, or under section~46 as to the reduction of benefit%  % Words inserted by 2000 c 19 Sch 3 para 11(13)(a)
;

(ii) [\emph{1993 scheme version}] as to whether to revise, under section~16, a decision (made after the commencement date) with respect to such an assessment; or

(ii) [\emph{2003 scheme version}] as to whether to revise, under section~16, 
%a decision (made after the commencement date) with respect to such an assessment%
any decision (made after the commencement date) referred to in section~16(1A)%  % Words substituted by 2000 c 19 Sch 3 para 11(13)(b)
; or

(iii) [\emph{1993 scheme version}] on an application under section 17 (made after the commencement date) for a decision with respect to such an assessment to be superseded.

(iii) [\emph{2003 scheme version}] on an application under section 17 (made after the commencement date) for 
%a decision with respect to such an assessment to be superseded%
any decision (made after the commencement date) referred to in section~17(1)%  % Words substituted by 2000 c 19 Sch 3 para 11(13)(c)
.
\end{enumerate}
\end{enumerate}

(2) This section does not apply where the decision of the Secretary of State mentioned in subsection (1)($b$)—
\begin{enumerate}\item[]
($a$) is one which, but for section 28ZA(2)($a$), would have been made before the date of the relevant determination; or

($b$) is one made in pursuance of section 28ZB(3) or (5).
\end{enumerate}

(3) [\emph{1993 scheme version}] In so far as the decision relates to a person’s liability in respect of a period before the date of the relevant determination, it shall be made as if the adjudicating authority’s decision had been found by the Commissioner or court not to have been erroneous in point of law.

(3) [\emph{2003 scheme version}] In so far as the decision relates to a person’s liability 
or the reduction of a person’s benefit  % Words inserted by 2000 c 19 Sch 3 para 11(13)(d)
in respect of a period before the date of the relevant determination, it shall be made as if the adjudicating authority’s decision had been found by the Commissioner or court not to have been erroneous in point of law.

(4) Subsection (1)($a$) shall be read as including a case where—
\begin{enumerate}\item[]
($a$) the effect of the relevant determination is that part or all of a purported regulation or order is invalid; and

($b$) the error of law made by the adjudicating authority was to act on the basis that the purported regulation or order (or the part held to be invalid) was valid.
\end{enumerate}

(5) It is immaterial for the purposes of subsection (1)—
\begin{enumerate}\item[]
($a$) where such a decision as is mentioned in paragraph ($b$)(i) falls to be made; or

($b$) where such a decision as is mentioned in paragraph ($b$)(ii) or (iii) falls to be made on an application under section 16 or (as the case may be) section 17,
\end{enumerate}
whether the application was made before or after the date of the relevant determination.

(6) In this section—
\begin{enumerate}\item[]
    [\emph{1993 scheme version}] “adjudicating authority” means the Secretary of State, or a child support officer;

    [\emph{2003 scheme version}] “adjudicating authority” means the Secretary of State, or a child support officer
or, in the case of a decision made on a referral under section~28D(1)($b$), an appeal tribunal%  % Words inserted by 2000 c 19 Sch 3 para 11(13)(e)
;

    “the commencement date” means the date of the coming into force of section 44 of the Social Security Act 1998; and

    “the court” means the High Court, the Court of Appeal, the Court of Session, the High Court or Court of Appeal in Northern Ireland, the 
House of Lords 
%Supreme Court  % Words substituted by 2005 c 4 Sch 9 para 54
or the Court of Justice of the European Community. 
\end{enumerate}

(7) The date of the relevant determination shall, in prescribed cases, be determined for the purposes of this section in accordance with any regulations made for that purpose.

(8) Regulations made under subsection (7) may include provision—
\begin{enumerate}\item[]
($a$) for a determination of a higher court to be treated as if it had been made on the date of a determination of a lower court or a Child Support Commissioner; or

($b$) for a determination of a lower court or a Child Support Commissioner to be treated as if it had been made on the date of a determination of a higher court.
\end{enumerate}

\amendment{
Words inserted in s. 28ZC(1)(b)(i), words substituted in s. 28ZC(1)(b)(ii), (iii) and words inserted in s. 28ZC(3), (6)  (3.3.03 for 2003 scheme cases) by the Child Support, Pensions and Social Security Act 2000 Sch. 3 para. 11(13).

Words substituted in definition of ``the court'' in s. 28ZC(6) (prosp) by the Constitutional Reform Act 2005 Sch. 9 para. 54.
}

\subsection{28ZD. Correction of errors and setting aside of decisions}

(1) Regulations may make provision with respect to—
\begin{enumerate}\item[]
($a$) the correction of accidental errors in any decision or record of a decision given under this Act; and

($b$) the setting aside of any such decision in a case where it appears just to set the decision aside on the ground that—
\begin{enumerate}\item[]
(i) a document relating to the proceedings in which the decision was given was not sent to, or was not received at an appropriate time by, a party to the proceedings or a party’s representative or was not received at an appropriate time by the body or person who gave the decision; or

(ii) a party to the proceedings or a party’s representative was not present at a hearing related to the proceedings.
\end{enumerate}
\end{enumerate}

(2) Nothing in subsection (1) shall be construed as derogating from any power to correct errors or set aside decisions which is exercisable apart from regulations made by virtue of that subsection.

% S 28A inserted (2.12.96) by 1995 c 34 s 1(1)
\section[\itshape Departure from usual rules for determining maintenance assessments --- \emph{1993 scheme version}]{\itshape Departure from usual rules for determining maintenance assessments\\*\emph{1993 scheme version}}

\subsection{28A. Application for a departure direction}

(1) Where a maintenance assessment (“the current assessment”) is in force—
\begin{enumerate}\item[]
($a$) the person with care, or absent parent, with respect to whom it was made, or

($b$) where the application for the current assessment was made under section 7, either of those persons or the child concerned,
\end{enumerate}
may apply to the Secretary of State for a direction under section 28F (a “departure direction”).

(2) An application for a departure direction shall state in writing the grounds on which it is made and shall, in particular, state whether it is based on—
\begin{enumerate}\item[]
($a$) the effect of the current assessment; or

($b$) a material change in the circumstances of the case since the current assessment was made.
\end{enumerate}

(3) In other respects, an application for a departure direction shall be made in such manner as may be prescribed.

%(4) An application may be made under this section even though—
%\begin{enumerate}\item[]
%($a$) an application for a review has been made under section 17 or 18 with respect to the current assessment; or
%
%($b$) a child support officer is conducting a review of the current assessment under section 16 or 19.
%\end{enumerate}

% S 28A(4) substituted (1.6.99) by 1998 c 14 Sch 7 para 34
(4) An application may be made under this section even though an application has been made under section 16(1) or 17(1) with respect to the current assessment.

(5) If the Secretary of State considers it appropriate to do so, he may by regulations provide for the question whether a change of circumstances is material to be determined in accordance with the regulations.

(6) Schedule 4A has effect in relation to departure directions.

\amendment{
S. 28A inserted (2.12.96) by the Child Support Act 1995 s.~1(1).

S. 28A(4) substituted (1.6.99) by the Social Security Act 1998 Sch. 7 para. 34.
}

% S 28B inserted (2.12.96) by 1995 c 34 s 2
\subsection{28B. Preliminary consideration of applications}

(1) Where an application for a departure direction has been duly made to the Secretary of State, he may give the application a preliminary consideration.

(2) Where the Secretary of State does so he may, on completing the preliminary consideration, reject the application if it appears to him—
\begin{enumerate}\item[]
($a$) that there are no grounds on which a departure direction could be given in response to the application; or

($b$) that the difference between the current amount and the revised amount is less than an amount to be calculated in accordance with regulations made by the Secretary of State for the purposes of this subsection and section 28F(4).
\end{enumerate}

(3) In subsection (2)—
\begin{enumerate}\item[]
    “the current amount” means the amount of the child support maintenance fixed by the current assessment; and

    “the revised amount” means the amount of child support maintenance which, but for subsection (2)($b$), would be fixed if a fresh maintenance assessment were to be made as a result of a departure direction allowing the departure applied for. 
\end{enumerate}

% S 28B(4), (5) repealed (1.6.99) by 1998 c 14 Sch 7 para 35(1)
%(4) Before completing any preliminary consideration, the Secretary of State may refer the current assessment to a child support officer for it to be reviewed as if an application for a review had been made under section 17 or~18.
%
%(5) A review initiated by a reference under subsection (4) shall be conducted as if subsection (4) of section 17, or (as the case may be) subsection~(8) of section 18, were omitted.
%
%(6) Where, as a result of a review of the current assessment under section 16, 17, 18 or 19 (including a review initiated by a reference under subsection (4)), a fresh maintenance assessment is made, the Secretary of State—
%\begin{enumerate}\item[]
%($a$) shall notify the applicant and such other persons as may be prescribed that the fresh maintenance assessment has been made; and
%
%($b$) may direct that the application is to lapse unless, before the end of such period as may be prescribed, the applicant notifies the Secretary of State that he wishes it to stand.
%\end{enumerate}

% S 28B(6) substituted (1.6.99) by 1998 c 14 Sch 7 para 35(2)
(6) Where a decision as to a maintenance assessment is revised or superseded under section 16 or 17, the Secretary of State—
\begin{enumerate}\item[]
($a$) shall notify the applicant and such other persons as may be prescribed that the decision has been revised or superseded; and

($b$) may direct that the application is to lapse unless, before the end of such period as may be prescribed, the applicant notifies the Secretary of State that he wishes it to stand.
\end{enumerate}

\amendment{
S. 28B(1)--(5) inserted (2.12.96) by the Child Support Act 1995 s.~2.  

S. 28B(6) inserted (prosp.) by the Child Support Act 1995 s.~2.

S. 28B(4), (5) repealed and s. 28B(6) substituted (1.6.99) by the Social Security Act 1998 Sch. 7 para. 35.
}

% S 28C inserted (2.12.96) by 1995 c 34 s 3
\subsection{28C. Imposition of a regular payments condition}

(1) Where an application for a departure direction is made by an absent parent, the Secretary of State may impose on him one of the conditions mentioned in subsection (2) (“a regular payments condition”).

(2) The conditions are that—
\begin{enumerate}\item[]
($a$) the applicant must make the payments of child support maintenance fixed by the current assessment;

($b$) the applicant must make such reduced payments of child support maintenance as may be determined in accordance with regulations made by the Secretary of State.
\end{enumerate}

(3) Where the Secretary of State imposes a regular payments condition, he shall give written notice to the absent parent and person with care concerned of the imposition of the condition and of the effect of failure to comply with it.

(4) A regular payments condition shall cease to have effect on the failure or determination of the application.

(5) For the purposes of subsection (4), an application for a departure direction fails if—
\begin{enumerate}\item[]
($a$) it lapses or is withdrawn; or

($b$) the Secretary of State rejects it on completing a preliminary consideration under section 28B.
\end{enumerate}

(6) Where an absent parent has failed to comply with a regular payments condition—
\begin{enumerate}\item[]
($a$) the Secretary of State may refuse to consider the application; and

($b$) in prescribed circumstances the application shall lapse.
\end{enumerate}

(7) The question whether an absent parent has failed to comply with a regular payments condition shall be determined by the Secretary of State.

(8) Where the Secretary of State determines that an absent parent has failed to comply with a regular payments condition he shall give that parent, and the person with care concerned, written notice of his decision.”

\amendment{
S. 28C inserted (2.12.96) by the Child Support Act 1995 s.~3.  

Ss. 28A--28C substituted (10.11.00 for regulation-making purposes) by the Child Support, Pensions and Social Security Act 2000 s. 5(2).
}

% Ss 28A--28C substituted by 2000 c 19 s 5(2)
\section[\itshape Variations --- \emph{2003 scheme version}]{\itshape Variations\\*\emph{2003 scheme version}}

\subsection{28A. Application for variation of usual rules for calculating maintenance}

(1) Where an application for a maintenance calculation is made under section~4 or~7, or treated as made under section~6, the person with care or the non-resident parent or (in the case of an application under section~7) either of them or the child concerned may apply to the Secretary of State for the rules by which the calculation is made to be varied in accordance with this Act.

(2) Such an application is referred to in this Act as an “application for a variation”.

(3) An application for a variation may be made at any time before the Secretary of State has reached a decision (under section~11 or~12(1)) on the application for a maintenance calculation (or the application treated as having been made under section~6).

(4) A person who applies for a variation—
\begin{enumerate}\item[]
($a$) need not make the application in writing unless the Secretary of State directs in any case that he must; and

($b$) must say upon what grounds the application is made.
\end{enumerate}

(5) In other respects an application for a variation is to be made in such manner as may be prescribed.

(6) Schedule 4A has effect in relation to applications for a variation.

\amendment{
See the Child Support (Variations) (Modification of Statutory Provisions) Regulations 2000 reg. 3 for modifications to this section where an application for a variation is made under s. 28G.
}

\subsection{28B. Preliminary consideration of applications}

(1) Where an application for a variation has been duly made to the Secretary of State, he may give it a preliminary consideration.

(2) Where he does so he may, on completing the preliminary consideration, reject the application (and proceed to make his decision on the application for a maintenance calculation without any variation) if it appears to him—
\begin{enumerate}\item[]
($a$) that there are no grounds on which he could agree to a variation;

($b$) that he has insufficient information to make a decision on the application for the maintenance calculation under section~11 (apart from any information needed in relation to the application for a variation), and therefore that his decision would be made under section~12(1); or

($c$) that other prescribed circumstances apply.
\end{enumerate}

\amendment{
See the Child Support (Variations) (Modification of Statutory Provisions) Regulations 2000 reg. 4 for modifications to this section where an application for a variation is made under s. 28G.
}

\subsection{28C. Imposition of regular payments condition}

(1) Where—
\begin{enumerate}\item[]
($a$) an application for a variation is made by the non-resident parent; and

($b$) the Secretary of State makes an interim maintenance decision,
\end{enumerate}
the Secretary of State may also, if he has completed his preliminary consideration (under section~28B) of the application for a variation and has not rejected it under that section, impose on the non-resident parent one of the conditions mentioned in subsection~(2)  (a “regular payments condition”).

(2) The conditions are that—
\begin{enumerate}\item[]
($a$) the non-resident parent must make the payments of child support maintenance specified in the interim maintenance decision;

($b$) the non-resident parent must make such lesser payments of child support maintenance as may be determined in accordance with regulations made by the Secretary of State.
\end{enumerate}

(3) Where the Secretary of State imposes a regular payments condition, he shall give written notice of the imposition of the condition and of the effect of failure to comply with it to—
\begin{enumerate}\item[]
($a$) the non-resident parent;

($b$) all the persons with care concerned; and

($c$) if the application for the maintenance calculation was made under section~7, the child who made the application.
\end{enumerate}

(4) A regular payments condition shall cease to have effect—
\begin{enumerate}\item[]
($a$) when the Secretary of State has made a decision on the application for a maintenance calculation under section~11 (whether he agrees to a variation or not);

($b$) on the withdrawal of the application for a variation.
\end{enumerate}

(5) Where a non-resident parent has failed to comply with a regular payments condition, the Secretary of State may in prescribed circumstances refuse to consider the application for a variation, and instead reach his decision under section~11 as if no such application had been made.

(6) The question whether a non-resident parent has failed to comply with a regular payments condition is to be determined by the Secretary of State.

(7) Where the Secretary of State determines that a non-resident parent has failed to comply with a regular payments condition he shall give written notice of his determination to—
\begin{enumerate}\item[]
($a$) that parent;

($b$) all the persons with care concerned; and

($c$) if the application for~the maintenance calculation was made under section~7, the child who made the application.
\end{enumerate}

\amendment{
See the Child Support (Variations) (Modification of Statutory Provisions) Regulations 2000 reg. 5 for modifications to this section where an application for a variation is made under s. 28G.
}

% S 28D inserted (2.12.96) by 1995 c 34 s 4
\subsection[28D. Determination of applications --- \emph{1993 scheme version}]{28D. Determination of applications\\*\emph{1993 scheme version}}

(1) Where an application for a departure direction has not failed, the Secretary of State shall—
\begin{enumerate}\item[]
($a$) determine the application in accordance with the relevant provisions of, or made under, this Act; or

($b$) refer the application to 
%a child support appeal tribunal 
an appeal tribunal  % Words substituted (1.6.99) by 1998 c 14 Sch 7 para 36
for the tribunal to determine it in accordance with those provisions.
\end{enumerate}

(2) For the purposes of subsection (1), an application for a departure direction has failed if—
\begin{enumerate}\item[]
($a$) it has lapsed or been withdrawn; or

($b$) the Secretary of State has rejected it on completing a preliminary consideration under section 28B.
\end{enumerate}

(3) In dealing with an application for a departure direction which has been referred to it under subsection (1)($b$), 
%a child support appeal tribunal 
an appeal tribunal  % Words substituted (1.6.99) by 1998 c 14 Sch 7 para 36
shall have the same powers, and be subject to the same duties, as would the Secretary of State if he were dealing with the application.

\amendment{
S. 28D inserted (2.12.96) by the Child Support Act 1995 s.~4.  

Words substituted in s. 28D(1), (3) (1.6.99) by the Social Security Act 1998 Sch. 7 para. 36.
}

\subsection[28D. Determination of applications --- \emph{2003 scheme version}]{28D. Determination of applications\\*\emph{2003 scheme version}}

%(1) Where an application for a departure direction has not failed, the Secretary of State shall—
%\begin{enumerate}\item[]
%($a$) determine the application in accordance with the relevant provisions of, or made under, this Act; or
%
%($b$) refer the application to 
%%a child support appeal tribunal 
%an appeal tribunal  % Words substituted (1.6.99) by 1998 c 14 Sch 7 para 36
%for the tribunal to determine it in accordance with those provisions.
%\end{enumerate}

% S 28D(1) substituted by 2000 c 19 s 5(3)(a)
(1) Where an application for a variation has not failed, the Secretary of State shall, in accordance with the relevant provisions of, or made under, this Act—
\begin{enumerate}\item[]
($a$) either agree or not to a variation, and make a decision under section~11 or~12(1); or

($b$) refer the application to an appeal tribunal for the tribunal to determine what variation, if any, is to be made.
\end{enumerate}

(2) For the purposes of subsection (1), 
%an application for a departure direction 
an application for a variation  % Words substituted by 2000 c 19 s 5(3)(b)
has failed if—
\begin{enumerate}\item[]
($a$) it has 
%lapsed or  % Words omitted by 2000 c 19 s 5(3)(c)
been withdrawn; or

($b$) the Secretary of State has rejected it on completing a preliminary consideration under section 28B;
or  % Word inserted by 2000 c 19 s 5(3)(c)

% S 28D(2)(c) inserted by 2000 c 19 s 5(3)(c)
($c$) the Secretary of State has refused to consider it under section~28C(5).
\end{enumerate}

(3) In dealing with 
%an application for a departure direction 
an application for a variation  % Words substituted by 2000 c 19 s 5(3)(b)
which has been referred to it under subsection (1)($b$), 
%a child support appeal tribunal 
an appeal tribunal  % Words substituted (1.6.99) by 1998 c 14 Sch 7 para 36
shall have the same powers, and be subject to the same duties, as would the Secretary of State if he were dealing with the application.

\amendment{
S. 28D(1) substituted, words substituted in s. 28D(2), (3), words omitted in s. 28D(2)(b) and s. 28D(2)(c) inserted (10.11.00 for regulation-making purposes only, 3.3.03 for new-rules cases only) by the Child Support, Pensions and Social Security Act 2000 s. 5(3).

See the Child Support (Variations) (Modification of Statutory Provisions) Regulations 2000 reg. 6 for modifications to this section where an application for a variation is made under s. 28G.
}

% S 28E inserted (2.12.96) by 1995 c 34 s 5
\subsection[28E. Matters to be taken into account --- \emph{1993 scheme version}]{28E. Matters to be taken into account\\*\emph{1993 scheme version}}

(1) In determining any application for a departure direction, the Secretary of State shall have regard both to the general principles set out in subsection~(2) and to such other considerations as may be prescribed.

(2) The general principles are that—
\begin{enumerate}\item[]
($a$) parents should be responsible for maintaining their children whenever they can afford to do so;

($b$) where a parent has more than one child, his obligation to maintain any one of them should be no less of an obligation than his obligation to maintain any other of them.
\end{enumerate}

(3) In determining any application for a departure direction, the Secretary of State shall take into account any representations made to him—
\begin{enumerate}\item[]
($a$) by the person with care or absent parent concerned; or

($b$) where the application for the current assessment was made under section 7, by either of them or the child concerned.
\end{enumerate}

(4) In determining any application for a departure direction, no account shall be taken of the fact that—
\begin{enumerate}\item[]
($a$) any part of the income of the person with care concerned is, or would be if a departure direction were made, derived from any benefit; or

($b$) some or all of any child support maintenance might be taken into account in any manner in relation to any entitlement to benefit.
\end{enumerate}

(5) In this section “benefit” has such meaning as may be prescribed.

\amendment{
S. 28E inserted (2.12.96) by the Child Support Act 1995 s.~5.  
}

\subsection[28E. Matters to be taken into account --- \emph{2003 scheme version}]{28E. Matters to be taken into account\\*\emph{2003 scheme version}}

(1) In determining 
%any application for a departure direction%
whether to agree to a variation%  % Words substituted by 2000 c 19 s 5(4)(a)
, the Secretary of State shall have regard both to the general principles set out in subsection~(2) and to such other considerations as may be prescribed.

(2) The general principles are that—
\begin{enumerate}\item[]
($a$) parents should be responsible for maintaining their children whenever they can afford to do so;

($b$) where a parent has more than one child, his obligation to maintain any one of them should be no less of an obligation than his obligation to maintain any other of them.
\end{enumerate}

(3) In determining 
%any application for a departure direction%
whether to agree to a variation%  % Words substituted by 2000 c 19 s 5(4)(a)
, the Secretary of State shall take into account any representations made to him—
\begin{enumerate}\item[]
($a$) by the person with care or absent parent concerned; or

($b$) where the application for the current 
%assessment 
calculation  % Words substituted by 2000 c 19 s 1(2)(b)
was made under section 7, by either of them or the child concerned.
\end{enumerate}

(4) In determining 
%any application for a departure direction%
whether to agree to a variation%  % Words substituted by 2000 c 19 s 5(4)(a)
, no account shall be taken of the fact that—
\begin{enumerate}\item[]
($a$) any part of the income of the person with care concerned is, or would be if 
%a departure direction were made%
the Secretary of State agreed to a variation%  % Words substituted by 2000 c 19 s 5(4)(b)
, derived from any benefit; or

($b$) some or all of any child support maintenance might be taken into account in any manner in relation to any entitlement to benefit.
\end{enumerate}

(5) In this section “benefit” has such meaning as may be prescribed.

\amendment{
Words substituted in s. 28E(1), (3), (4) (10.11.00 for regulation-making purposes, 3.3.03 for 2003 scheme cases) by the Child Support, Pensions and Social Security Act 2000 s. 5(4).

See the Child Support (Variations) (Modification of Statutory Provisions) Regulations 2000 reg. 6 for modifications to this section where an application for a variation is made under s. 28G.
}


% S 28F inserted (2.12.96) by 1995 c 34 s 6(1)
\subsection[28F. Departure directions --- \emph{1993 scheme version}]{28F. Departure directions\\*\emph{1993 scheme version}}

(1) The Secretary of State may give a departure direction if—
\begin{enumerate}\item[]
($a$) he is satisfied that the case is one which falls within one or more of the cases set out in Part I of Schedule 4B or in regulations made under that Part; and

($b$) it is his opinion that, in all the circumstances of the case, it would be just and equitable to give a departure direction.
\end{enumerate}

(2) In considering whether it would be just and equitable in any case to give a departure direction, the Secretary of State shall have regard, in particular, to—
\begin{enumerate}\item[]
($a$) the financial circumstances of the absent parent concerned,

($b$) the financial circumstances of the person with care concerned, and

($c$) the welfare of any child likely to be affected by the direction.
\end{enumerate}

(3) The Secretary of State may by regulations make provision—
\begin{enumerate}\item[]
($a$) for factors which are to be taken into account in determining whether it would be just and equitable to give a departure direction in any case;

($b$) for factors which are not to be taken into account in determining such a question.
\end{enumerate}

(4) The Secretary of State shall not give a departure direction if he is satisfied that the difference between the current amount and the revised amount is less than an amount to be calculated in accordance with regulations made by the Secretary of State for the purposes of this subsection and section~28B(2).

(5) In subsection (4)—
\begin{enumerate}\item[]
    “the current amount” means the amount of the child support maintenance fixed by the current assessment, and

    “the revised amount” means the amount of child support maintenance which would be fixed if a fresh maintenance assessment were to be made as a result of the departure direction which the Secretary of State would give in response to the application but for subsection (4). 
\end{enumerate}

(6) A departure direction shall—
\begin{enumerate}\item[]
($a$) require 
%a child support officer to make 
the making of  % Words substituted (1.6.99) by 1998 c 14 Sch 7 para 37
one or more fresh maintenance assessments; and

($b$) specify the basis on which the amount of child support maintenance is to be fixed by any assessment made in consequence of the direction.
\end{enumerate}

(7) In giving a departure direction, the Secretary of State shall comply with the provisions of regulations made under Part II of Schedule 4B.

(8) Before the end of such period as may be prescribed, the Secretary of State shall notify the applicant for a departure direction, and such other persons as may be prescribed—
\begin{enumerate}\item[]
($a$) of his decision in relation to the application, and

($b$) of the reasons for his decision.
\end{enumerate}

\amendment{
S. 28F inserted (2.12.96) by the Child Support Act 1995 s.~6(1).  

Words substituted in s. 28F(6) (1.6.99) by the Social Security Act 1998 Sch. 7 para. 37.
}

% S 28F substituted by 2000 c 19 s 5(5)
\subsection[28F. Agreement to a variation --- \emph{2003 scheme version}]{28F. Agreement to a variation\\*\emph{2003 scheme version}}

(1) The Secretary of State may agree to a variation if—
\begin{enumerate}\item[]
($a$) he is satisfied that the case is one which falls within one or more of the cases set out in Part I of Schedule 4B or in regulations made under that Part; and

($b$) it is his opinion that, in all the circumstances of the case, it would be just and equitable to agree to a variation.
\end{enumerate}

(2) In considering whether it would be just and equitable in any case to agree to a variation, the Secretary of State—
\begin{enumerate}\item[]
($a$) must have regard, in particular, to the welfare of any child likely to be affected if he did agree to a variation; and

($b$) must, or as the case may be must not, take any prescribed factors into account, or must take them into account (or not) in prescribed circumstances.
\end{enumerate}

(3) The Secretary of State shall not agree to a variation (and shall proceed to make his decision on the application for a maintenance calculation without any variation) if he is satisfied that—
\begin{enumerate}\item[]
($a$) he has insufficient information to make a decision on the application for the maintenance calculation under section~11, and therefore that his decision would be made under section~12(1); or

($b$) other prescribed circumstances apply.
\end{enumerate}

(4) Where the Secretary of State agrees to a variation, he shall—
\begin{enumerate}\item[]
($a$) determine the basis on which the amount of child support maintenance is to be calculated in response to the application for a maintenance calculation (including an application treated as having been made); and

($b$) make a decision under section~11 on that basis.
\end{enumerate}

(5) If the Secretary of State has made an interim maintenance decision, it is to be treated as having been replaced by his decision under section~11, and except in prescribed circumstances any appeal connected with it (under section~20) shall lapse.

(6) In determining whether or not to agree to a variation, the Secretary of State shall comply with regulations made under Part II of Schedule 4B.

\amendment{
S. 28F substituted (10.11.00 for regulation-making purposes, 3.3.03 for 2003 scheme cases) by the Child Support, Pensions and Social Security Act 2000 s. 5(5).

See the Child Support (Variations) (Modification of Statutory Provisions) Regulations 2000 reg. 7 for modifications to this section where an application for a variation is made under s. 28G.
}

% S 28G inserted (2.12.96) by 1995 c 34 s 7
\subsection[28G. Effect and duration of departure directions --- \emph{1993 scheme version}]{28G. Effect and duration of departure directions\\*\emph{1993 scheme version}}

% S 28G(1) repealed (1.6.99) by 1998 c 14 Sch 7 para 38
%(1) Where a departure direction is given, it shall be the duty of the child support officer to whom the case is referred to comply with the direction as soon as is reasonably practicable.

(2) A departure direction may be given so as to have effect—
\begin{enumerate}\item[]
($a$) for a specified period; or

($b$) until the occurrence of a specified event.
\end{enumerate}

(3) The Secretary of State may by regulations make provision for the cancellation of a departure direction in prescribed circumstances.

(4) The Secretary of State may by regulations make provision as to when a departure direction is to take effect.

(5) Regulations under subsection (4) may provide for a departure direction to have effect from a date earlier than that on which the direction is given.

\amendment{
S. 28G inserted (2.12.96) by the Child Support Act 1995 s.~7.  

S. 28G(1) repealed (1.6.99) by the Social Security Act 1998 Sch. 7 para. 38.
}

% S 28G substituted by 2000 c 19 s 7
\subsection[28G. Variations: revision and supersession --- \emph{2003 scheme version}]{28G. Variations: revision and supersession\\*\emph{2003 scheme version}}

(1) An application for a variation may also be made when a maintenance calculation is in force.

(2) The Secretary of State may by regulations provide for—
\begin{enumerate}\item[]
($a$) sections 16, 17 and 20; and

($b$) sections 28A to 28F and Schedules 4A and 4B,
\end{enumerate}
to apply with prescribed modifications in relation to such applications.

(3) The Secretary of State may by regulations provide that, in prescribed cases (or except in prescribed cases), a decision under section~17 made otherwise than pursuant to an application for a variation may be made on the basis of a variation agreed to for the purposes of an earlier decision without a new application for a variation having to be made.

\amendment{
S. 28G substituted (10.11.00 for regulation-making purposes, 1.1.01 for s. 28G(2) for all purposes, 3.3.03 for 2003 scheme cases) by the Child Support, Pensions and Social Security Act 2000 s. 7.
}

% S 28H inserted (2.12.96) by 1995 c 34 s 8
%\subsection{28H. Appeals in relation to applications for departure directions}
%
%(1) Any qualifying person who is aggrieved by any decision of the Secretary of State on an application for a departure direction may appeal to a child support appeal tribunal against that decision.
%
%(2) In subsection (1), “qualifying person” means—
%\begin{enumerate}\item[]
%($a$) the person with care, or absent parent, with respect to whom the current assessment was made, or
%
%($b$) where the application for the current assessment was made under section 7, either of those persons or the child concerned.
%\end{enumerate}
%
%(3) Except with leave of the chairman of a child support appeal tribunal, no appeal under this section shall be brought after the end of the period of 28 days beginning with the date on which notification was given of the decision in question.
%
%(4) On an appeal under this section, the tribunal shall—
%\begin{enumerate}\item[]
%($a$) consider the matter—
%\begin{enumerate}\item[]
%(i) as if it were exercising the powers of the Secretary of State in relation to the application in question; and
%
%(ii) as if it were subject to the duties imposed on him in relation to that application;
%\end{enumerate}
%
%($b$) have regard to any representations made to it by the Secretary of State; and
%
%($c$) confirm the decision or replace it with such decision as the tribunal considers appropriate.
%\end{enumerate}
%
%\amendment{
%S. 28H inserted (2.12.96) by the Child Support Act 1995 s.~8.  
%}

% S 28H substituted (1.6.99) by 1998 c 14 Sch 7 para 39, repealed by 2000 c 19 Sch 3 para 11(14)
\subsection[28H. Departure directions: decisions and appeals --- \emph{1993 scheme only}]{28H. Departure directions: decisions and appeals\\*\emph{1993 scheme only}}

Schedule 4C shall have effect for applying sections 16, 17, 20 and 28ZA to~28ZC to decisions with respect to departure directions.

\amendment{
S. 28H substituted (1.6.99) by the Social Security Act 1998 Sch. 7 para. 39.

S. 28H repealed (3.3.03 for 2003 scheme cases) by the Child Support, Pensions and Social Security Act 2000 Sch. 3 para. 11(14).
}

% S 28I repealed by 2000 c 19 Sch 3 para 11(14)
\subsection[28I. Transitional provisions --- \emph{1993 scheme only}]{28I. Transitional provisions\\*\emph{1993 scheme only}}

%(1) In the case of an application for a departure direction relating to a maintenance assessment which was made before the coming into force of section 28A, the period within which the application must be made shall be such period as may be prescribed.
%
%(2) The Secretary of State may by regulations make provision for applications for departure directions to be dealt with according to an order determined in accordance with the regulations.
%
%(3) The regulations may, for example, provide for—
%\begin{enumerate}\item[]
%($a$) applications relating to prescribed descriptions of maintenance assessment, or
%
%($b$) prescribed descriptions of application,
%\end{enumerate}
%to be dealt with before applications relating to other prescribed descriptions of assessment or (as the case may be) other prescribed descriptions of application.

(4) The Secretary of State may by regulations make provision—
\begin{enumerate}\item[]
($a$) enabling applications for departure directions made before the coming into force of section 28A to be considered even though that section is not in force;

($b$) for the determination of any such application as if section 28A and the other provisions of this Act relating to departure directions were in force; and

($c$) as to the effect of any departure direction given before the coming into force of section 28A.
\end{enumerate}

%(5) Regulations under section 28G(4) may not provide for a departure direction to have effect from a date earlier than that on which that section came into force.

\amendment{
S. 28I(4) inserted (22.1.96) by the Child Support Act 1995 s. 9.

S. 28I repealed (3.3.03 for 2003 scheme cases) by the Child Support, Pensions and Social Security Act 2000 Sch. 3 para. 11(14).
}

% S 28J inserted by 2000 c 19 s 20(1)
\section{\itshape Voluntary payments}

\subsection[28J. Voluntary payments --- \emph{2003 scheme only}]{28J. Voluntary payments\\*\emph{2003 scheme only}}

(1) This section~applies where—
\begin{enumerate}\item[]
($a$) a person has applied for a maintenance calculation under section~4(1)  or~7(1), or is treated as having applied for one by virtue of section~6;

($b$) the Secretary of State has neither made a decision under section~11 or~12 on the application, nor decided not to make a maintenance calculation; and

($c$) the non-resident parent makes a voluntary payment.
\end{enumerate}

(2) A “voluntary payment” is a payment—
\begin{enumerate}\item[]
($a$) on account of child support maintenance which the non-resident parent expects to become liable to pay following the determination of the application (whether or not the amount of the payment is based on any estimate of his potential liability which the Secretary of State has agreed to give); and

($b$) made before the maintenance calculation has been notified to the non-resident parent or (as the case may be) before the Secretary of State has notified the non-resident parent that he has decided not to make a maintenance calculation.
\end{enumerate}

(3) In such circumstances and to such extent as may be prescribed—
\begin{enumerate}\item[]
($a$) the voluntary payment may be set off against arrears of child support maintenance which accrued by virtue of the maintenance calculation taking effect on a date earlier than that on which it was notified to the non-resident parent;

($b$) the amount payable under a maintenance calculation may be adjusted to take account of the voluntary payment.
\end{enumerate}

(4) A voluntary payment shall be made to the Secretary of State unless he agrees, on such conditions as he may specify, that it may be made to the person with care, or to or through another person.

(5) The Secretary of State may by regulations make provision as to voluntary payments, and the regulations may in particular—
\begin{enumerate}\item[]
($a$) prescribe what payments or descriptions of payment are, or are not, to count as “voluntary payments”;

($b$) prescribe the extent to which and circumstances in which a payment, or a payment of a prescribed description, counts.
\end{enumerate}

\amendment{
S. 28J inserted (10.11.00 for regulation-making purposes only, 3.3.03 for 2003 scheme cases) by the Child Support, Pensions and Social Security Act 2000 s. 20(1).
}

\section{\itshape Collection and enforcement}

\subsection{29. Collection of child support maintenance}

(1) The Secretary of State may arrange for the collection of any child support maintenance payable in accordance with a 
%maintenance assessment 
\emph{maintenance calculation}  % Words substituted by 2000 c 19 s 1(2)(a)
where—
\begin{enumerate}\item[]
($a$) the 
%assessment 
\emph{calculation}  % Words substituted by 2000 c 19 s 1(2)(b)
is made by virtue of section 6; or

($b$) an application has been made to the Secretary of State under section~4(2)  or 7(3)  for him to arrange for its collection.
\end{enumerate}

(2) Where a 
%maintenance assessment 
\emph{maintenance calculation}  % Words substituted by 2000 c 19 s 1(2)(a)
is made under this Act, payments of child support maintenance under the 
%assessment 
\emph{calculation}  % Words substituted by 2000 c 19 s 1(2)(b)
shall be made in accordance with regulations made by the Secretary of State.

(3) The regulations may, in particular, make provision—
\begin{enumerate}\item[]
($a$) for payments of child support maintenance to be made—
\begin{enumerate}\item[]
(i) to the person caring for the child or children in question;

(ii) to, or through, the Secretary of State; or

(iii) to, or through, such other person as the Secretary of State may, from time to time, specify;
\end{enumerate}

($b$) as to the method by which payments of child support maintenance are to be made;

($c$) as to the intervals at which such payments are to be made;

($d$) as to the method and timing of the transmission of payments which are made, to or through the Secretary of State or any other person, in accordance with the regulations;

($e$) empowering the Secretary of State to direct any person liable to make payments in accordance with the 
%assessment% 
\emph{calculation}%  % Words substituted by 2000 c 19 s 1(2)(b)
—
\begin{enumerate}\item[]
(i) to make them by standing order or by any other method which requires one person to give his authority for payments to be made from an account of his to an account of another’s on specific dates during the period for which the authority is in force and without the need for any further authority from him;

(ii) to open an account from which payments under the 
%assessment 
\emph{calculation}  % Words substituted by 2000 c 19 s 1(2)(b)
may be made in accordance with the method of payment which that person is obliged to adopt;
\end{enumerate}

($f$) providing for the making of representations with respect to matters with which the regulations are concerned.
\end{enumerate}

\amendment{
S. 29(2), (3) came into force 27.6.92; s. 29 came fully into force 5.4.93.
}

\subsection{30. Collection and enforcement of other forms of maintenance}

(1) Where the Secretary of State is arranging for the collection of any payments under section 29 or subsection (2), he may also arrange for the collection of any periodical payments, or secured periodical payments, of a prescribed kind which are payable to or for the benefit of any person who falls within a prescribed category.

%(2) The Secretary of State may arrange for the collection of any periodical payments or secured periodical payments of a prescribed kind which are payable for the benefit of a child even though he is not arranging for the collection of child support maintenance with respect to that child.

% S 30(2) substituted by 2000 c 19 Sch 3 para 11(15)
(2) The Secretary of State may, except in prescribed cases, arrange for the collection of any periodical payments, or secured periodical payments, of a prescribed kind which are payable for the benefit of a child even though he is not arranging for the collection of child support maintenance with respect to that child.

(3) Where—
\begin{enumerate}\item[]
($a$) the Secretary of State is arranging, under this Act, for the collection of different payments (“the payments”) from the same 
%absent parent% 
\emph{non-resident parent}%  % Words substituted by 2000 c 19 Sch 3 para 11(2)
;

($b$) an amount is collected by the Secretary of State from the 
%absent parent 
\emph{non-resident parent}  % Words substituted by 2000 c 19 Sch 3 para 11(2)
which is less than the total amount due in respect of the payments; and

($c$) the 
%absent parent 
\emph{non-resident parent}  % Words substituted by 2000 c 19 Sch 3 para 11(2)
has not stipulated how that amount is to be allocated by the Secretary of State as between the payments,
\end{enumerate}
the Secretary of State may allocate that amount as he sees fit.

(4) In relation to England and Wales, the Secretary of State may by regulations make provision for sections 29 and 31 to 40 to apply, with such modifications (if any) as he considers necessary or expedient, for the purpose of enabling him to enforce any obligation to pay any amount which he is authorised to collect under this section.

(5) In relation to Scotland, the Secretary of State may by regulations make provision for the purpose of enabling him to enforce any obligation to pay any amount which he is authorised to collect under this section—
\begin{enumerate}\item[]
($a$) empowering him to bring any proceedings or take any other steps (other than diligence against earnings) which could have been brought or taken by or on behalf of the person to whom the periodical payments are payable;

($b$) applying sections 29, 31 and 32 with such modifications (if any) as he considers necessary or expedient.
\end{enumerate}

\amendment{
S. 30(1), (4), (5) came into force 27.6.92; s. 30 (3) came into force 5.4.93.
S. 30(2) is not yet in force.

S. 30(2) substituted (3.3.03) by the Child Support, Pensions and Social Security Act 2000 Sch. 3 para. 11(15).
}

\subsection{31. Deduction from earnings orders}

(1) This section applies where any person (“the liable person”) is liable to make payments of child support maintenance.

(2) The Secretary of State may make an order (“a deduction from earnings order”) against a liable person to secure the payment of any amount due under the 
%maintenance assessment 
\emph{maintenance calculation}  % Words substituted by 2000 c 19 s 1(2)(a)
in question.

(3) A deduction from earnings order may be made so as to secure the payment of—
\begin{enumerate}\item[]
($a$) arrears of child support maintenance payable under the 
%assessment% 
\emph{calculation}%  % Words substituted by 2000 c 19 s 1(2)(b)
;

($b$) amounts of child support maintenance which will become due under the 
%assessment% 
\emph{calculation}%  % Words substituted by 2000 c 19 s 1(2)(b)
; or

($c$) both such arrears and such future amounts.
\end{enumerate}

(4) A deduction from earnings order—
\begin{enumerate}\item[]
($a$) shall be expressed to be directed at a person (“the employer”) who has the liable person in his employment; and

($b$) shall have effect from such date as may be specified in the order.
\end{enumerate}

(5) A deduction from earnings order shall operate as an instruction to the employer to—
\begin{enumerate}\item[]
($a$) make deductions from the liable person’s earnings; and

($b$) pay the amounts deducted to the Secretary of State.
\end{enumerate}

(6) The Secretary of State shall serve a copy of any deduction from earnings order which he makes under this section on—
\begin{enumerate}\item[]
($a$) the person who appears to the Secretary of State to have the liable person in question in his employment; and

($b$) the liable person.
\end{enumerate}

(7) Where—
\begin{enumerate}\item[]
($a$) a deduction from earnings order has been made; and

($b$) a copy of the order has been served on the liable person’s employer,
\end{enumerate}
it shall be the duty of that employer to comply with the order; but he shall not be under any liability for non-compliance before the end of the period of 7 days beginning with the date on which the copy was served on him.

(8) In this section and in section 32 “earnings” has such meaning as may be prescribed.

\amendment{
S. 31(8) came into force 27.6.92; s. 31 came fully into force 5.4.93.
}

\subsection{32. Regulations about deduction from earnings orders}

(1) The Secretary of State may by regulations make provision with respect to deduction from earnings orders.

(2) The regulations may, in particular, make provision—
\begin{enumerate}\item[]
($a$) as to the circumstances in which one person is to be treated as employed by another;

($b$) requiring any deduction from earnings under an order to be made in the prescribed manner;

% S 32(bb) inserted by 2000 c 19 Sch 3 para 11(16)
($bb$) for the amount or amounts which are to be deducted from the liable person’s earnings not to exceed a prescribed proportion of his earnings (as determined by the employer);

($c$) requiring an order to specify the amount or amounts to which the order relates and the amount or amounts which are to be deducted from the liable person’s earnings in order to meet his liabilities under the 
%maintenance assessment 
\emph{maintenance calculation}  % Words substituted by 2000 c 19 s 1(2)(a)
in question;

($d$) requiring the intervals between deductions to be made under an order to be specified in the order;

($e$) as to the payment of sums deducted under an order to the Secretary of State;

($f$) allowing the person who deducts and pays any amount under an order to deduct from the liable person’s earnings a prescribed sum towards his administrative costs;

($g$) with respect to the notification to be given to the liable person of amounts deducted, and amounts paid, under the order;

($h$) requiring any person on whom a copy of an order is served to notify the Secretary of State in the prescribed manner and within a prescribed period if he does not have the liable person in his employment or if the liable person ceases to be in his employment;

($i$) as to the operation of an order where the liable person is in the employment of the Crown;

($j$) for the variation of orders;

($k$) similar to that made by section 31(7), in relation to any variation of an order;

($l$) for an order to lapse when the employer concerned ceases to have the liable person in his employment;

($m$) as to the revival of an order in such circumstances as may be prescribed;

($n$) allowing or requiring an order to be discharged;

($o$) as to the giving of notice by the Secretary of State to the employer concerned that an order has lapsed or has ceased to have effect.
\end{enumerate}

(3) The regulations may include provision that while a deduction from earnings order is in force—
\begin{enumerate}\item[]
($a$) the liable person shall from time to time notify the Secretary of State, in the prescribed manner and within a prescribed period, of each occasion on which he leaves any employment or becomes employed, or re-employed, and shall include in such a notification a statement of his earnings and expected earnings from the employment concerned and of such other matters as may be prescribed;

($b$) any person who becomes the liable person’s employer and knows that the order is in force shall notify the Secretary of State, in the prescribed manner and within a prescribed period, that he is the liable person’s employer, and shall include in such a notification a statement of the liable person’s earnings and expected earnings from the employment concerned and of such other matters as may be prescribed.
\end{enumerate}

(4) The regulations may include provision with respect to the priority as between a deduction from earnings order and—
\begin{enumerate}\item[]
($a$) any other deduction from earnings order;

($b$) any order under any other enactment relating to England and Wales which provides for deductions from the liable person’s earnings;

($c$) any diligence against earnings.
\end{enumerate}

(5) The regulations may include a provision that a liable person may appeal to a magistrates' court (or in Scotland to the sheriff) if he is aggrieved by the making of a deduction from earnings order against him, or by the terms of any such order, or there is a dispute as to whether payments constitute earnings or as to any other prescribed matter relating to the order.

(6) On an appeal under subsection (5)  the court or (as the case may be) the sheriff shall not question the 
%maintenance assessment 
\emph{maintenance calculation}  % Words substituted by 2000 c 19 s 1(2)(a)
by reference to which the deduction from earnings order was made.

(7) Regulations made by virtue of subsection (5)  may include provision as to the powers of a magistrates' court, or in Scotland of the sheriff, in relation to an appeal (which may include provision as to the quashing of a deduction from earnings order or the variation of the terms of such an order).

(8) If any person fails to comply with the requirements of a deduction from earnings order, or with any regulation under this section which is designated for the purposes of this subsection, he shall be guilty of an offence.

(9) In subsection (8)  “designated” means designated by the regulations.

(10) It shall be a defence for a person charged with an offence under subsection (8)  to prove that he took all reasonable steps to comply with the requirements in question.

(11) Any person guilty of an offence under subsection (8)  shall be liable on summary conviction to a fine not exceeding level two on the standard scale.

\amendment{
S. 32(1)--(5), (7)--(9) came into force 27.6.92; s. 32 came fully into force 5.4.93.

S. 32(bb) inserted (3.3.03 for 2003 scheme cases, 26.9.08 for regulation-making purposes, 27.10.08 for all other purposes) by the Child Support, Pensions and Social Security Act 2000 Sch. 3 para. 11(16).
}

\subsection{33. Liability orders}

(1) This section applies where—
\begin{enumerate}\item[]
($a$) a person who is liable to make payments of child support maintenance (“the liable person”) fails to make one or more of those payments; and

($b$) it appears to the Secretary of State that—
\begin{enumerate}\item[]
(i) it is inappropriate to make a deduction from earnings order against him (because, for example, he is not employed); or

(ii) although such an order has been made against him, it has proved ineffective as a means of securing that payments are made in accordance with the 
%maintenance assessment 
\emph{maintenance calculation}  % Words substituted by 2000 c 19 s 1(2)(a)
in question.
\end{enumerate}
\end{enumerate}

(2) The Secretary of State may apply to a magistrates' court or, in Scotland, to the sheriff for an order (“a liability order”) against the liable person.

(3) Where the Secretary of State applies for a liability order, the magistrates' court or (as the case may be) sheriff shall make the order if satisfied that the payments in question have become payable by the liable person and have not been paid.

(4) On an application under subsection (2), the court or (as the case may be) the sheriff shall not question the 
%maintenance assessment 
\emph{maintenance calculation}  % Words substituted by 2000 c 19 s 1(2)(a)
under which the payments of child support maintenance fell to be made.

% S 33(5) inserted (4.9.95) by 1995 c 34 Sch 3 para 10
(5) If the Secretary of State designates a liability order for the purposes of this subsection it shall be treated as a judgment entered in a county court for the purposes of %section 73 of the County Courts Act 1984 (register of judgments and orders)
section 98 of the Courts Act 2003 (register of judgments and orders etc)%  % Words substituted by SI 2006/1001 art 2
.

% S 33(6) inserted by 2000 c 19 Sch 3 para 11(17)
(6) [\emph{2003 scheme only}] Where regulations have been made under section~29(3)($a$)—
\begin{enumerate}\item[]
($a$) the liable person fails to make a payment (for~the purposes of subsection~(1)($a$)  of this section); and

($b$) a payment is not paid (for~the purposes of subsection~(3)),
\end{enumerate}
unless the payment is made to, or~through, the person specified in or~by virtue of those regulations for~the case of the liable person in question.

\amendment{
S. 33 came into force 5.4.93.

S. 33(5) inserted (4.9.95) by the Child Support Act 1995 Sch.~3 para.~10.

S. 33(6) inserted (3.3.03 for 2003 scheme cases) by the Child Support, Pensions and Social Security Act 2000 Sch. 3 para. 11(17).

Words substituted in s. 33(5) (6.4.06) by the Courts Act 2003 (Consequential Amendment) Order 2006 art. 2.
}

\subsection{34. Regulations about liability orders}

(1) The Secretary of State may make regulations in relation to England and Wales—
\begin{enumerate}\item[]
($a$) prescribing the procedure to be followed in dealing with an application by the Secretary of State for a liability order;

($b$) prescribing the form and contents of a liability order; and

($c$) providing that where a magistrates' court has made a liability order, the person against whom it is made shall, during such time as the amount in respect of which the order was made remains wholly or partly unpaid, be under a duty to supply relevant information to the Secretary of State.
\end{enumerate}

%(2) In subsection (1)  “relevant information” means any information of a prescribed description which is in the possession of the liable person and which the Secretary of State has asked him to supply.

\amendment{
S. 34(1) came into force 27.6.92.

S. 34(2) is not yet in force.
}

\subsection{35. Enforcement of liability orders by 
distress
%taking control of goods  % Words substituted by 2007 c 15 Sch 13 para 94(2)
}

(1) Where a liability order has been made against a person (“the liable person”), the Secretary of State may 
levy the appropriate amount by distress and sale of the liable person’s goods.
%use the procedure in Schedule 12 to the Tribunals, Courts and Enforcement Act 2007 (taking control of goods) to recover the amount in respect of which the order was made, to the extent that it remains unpaid.  % Words substituted by 2007 c 15 Sch 13 para 94(3)

% S 35(2)--(8) repealed by 2007 c 15 Sch 13 para 94(4)
(2) In subsection (1), “the appropriate amount” means the aggregate of—
\begin{enumerate}\item[]
($a$) the amount in respect of which the order was made, to the extent that it remains unpaid; and

($b$) an amount, determined in such manner as may be prescribed, in respect of the charges connected with the distress.
\end{enumerate}

(3) The Secretary of State may, in exercising his powers under subsection~(1)  against the liable person’s goods, seize—
\begin{enumerate}\item[]
($a$) any of the liable person’s goods except—
\begin{enumerate}\item[]
(i) such tools, books, vehicles and other items of equipment as are necessary to him for use personally by him in his employment, business or vocation;

(ii) such clothing, bedding, furniture, household equipment and provisions as are necessary for satisfying his basic domestic needs; and
\end{enumerate}

($b$) any money, banknotes, bills of exchange, promissory notes, bonds, specialties or securities for money belonging to the liable person.
\end{enumerate}

(4) For the purposes of subsection (3), the liable person’s domestic needs shall be taken to include those of any member of his family with whom he resides.

(5) No person levying a distress under this section shall be taken to be a trespasser—
\begin{enumerate}\item[]
($a$) on that account; or

($b$) from the beginning, on account of any subsequent irregularity in levying the distress.
\end{enumerate}

(6) A person sustaining special damage by reason of any irregularity in levying a distress under this section may recover full satisfaction for the damage (and no more) by proceedings in trespass or otherwise.

(7) The Secretary of State may make regulations supplementing the provisions of this section.

(8) The regulations may, in particular—
\begin{enumerate}\item[]
($a$) provide that a distress under this section may be levied anywhere in England and Wales;

($b$) provide that such a distress shall not be deemed unlawful on account of any defect or want of form in the liability order;

($c$) provide for an appeal to a magistrates' court by any person aggrieved by the levying of, or an attempt to levy, a distress under this section;

($d$) make provision as to the powers of the court on an appeal (which may include provision as to the discharge of goods distrained or the payment of compensation in respect of goods distrained and sold).
\end{enumerate}

\amendment{
S. 35(2)(b), (7), (8) came into force 27.6.92; s. 35 came fully into force 5.4.93.

Words substituted in heading to s. 35 and in s. 35(1) and s. 35(2)--(8) repealed (prosp) by the Tribunals, Courts and Enforcement Act 2007 Sch. 13 para. 94.
}

\subsection{36. Enforcement in county courts}

(1) Where a liability order has been made against a person, the amount in respect of which the order was made, to the extent that it remains unpaid, shall, if a county court so orders, be recoverable by means of garnishee proceedings or a charging order, as if it were payable under a county court order.

(2) In subsection (1)  “charging order” has the same meaning as in section~1 of the Charging Orders Act 1979.

\amendment{
S. 36 came into force 5.4.93.
}

\subsection{37. Regulations about liability orders: Scotland}

(1) Section 34(1)  does not apply to Scotland.

%(2) In Scotland, the Secretary of State may make regulations providing that where the sheriff has made a liability order, the person against whom it is made shall, during such time as the amount in respect of which the order was made remains wholly or partly unpaid, be under a duty to supply relevant information to the Secretary of State.
%
%(3) In this section “relevant information” has the same meaning as in section 34(2).

\amendment{
S. 37(1) came into force 5.4.93.  S. 37(2), (3) are not yet in force.
}

\subsection{38. Enforcement of liability orders by diligence: Scotland}

(1) In Scotland, where a liability order has been made against a person, the order shall be warrant anywhere in Scotland—
\begin{enumerate}\item[]
($a$) for the Secretary of State to charge the person to pay the appropriate amount and to recover that amount by 
%a poinding and sale under Part~II of the Debtors (Scotland) Act 1987 
an attachment  % Words substituted by 2002 asp 17 Sch 3 para 20
and, in connection therewith, for the opening of shut and lockfast places;

% S 38(1)(aa) inserted by 2007 asp 3 Sch 5 para 18(a)(i)
%($aa$) for the Secretary of State—
%\begin{enumerate}\item[]
%(i) to charge the person to pay the appropriate amount; and
%
%(ii) to execute, in respect of the person’s land, a land attachment;
%\end{enumerate}

($b$) for an arrestment (other than an arrestment of the person’s earnings in the hands of his employers) and action of furthcoming or sale,
\end{enumerate}
and shall be apt to found a Bill of Inhibition or an action of adjudication at the instance of the Secretary of State.

% S 38(1)(c) substituted for words by 2007 asp 3 Sch 5 para 18(a)(ii)
%($c$) for an inhibition.
%\end{enumerate}

(2) In subsection (1)  the “appropriate amount” means the amount in respect of which the order was made, to the extent that it remains unpaid.

% S 38(2) substituted by 2007 asp 3 Sch 5 para 18(b)
%(2) In subsection (1)—
%\begin{enumerate}\item[]
%($a$) the “appropriate amount” means the amount in respect of which the order was made, to the extent that it remains unpaid; and
%
%($b$) in paragraph ($aa$), “land” has the same meaning as in section 82 of the Bankruptcy and Diligence etc. (Scotland) Act 2007.
%\end{enumerate}

\amendment{
S. 38 came into force 5.4.93.

Words substituted in s. 38(1) (30.12.02) by the Debt Arrangement and Attachment (Scotland) Act 2002 Sch. 3 para. 20.

S. 38(1)(aa) inserted and s. 38(1)(c), (2) substituted (prosp) by the Bankruptcy and Diligence etc. (Scotland) Act 2007 Sch. 5 para. 18.
}

\subsection{39. Liability orders: enforcement throughout United Kingdom}

(1) The Secretary of State may by regulations provide for—
\begin{enumerate}\item[]
($a$) any liability order made by a court in England and Wales; or

($b$) any corresponding order made by a court in Northern Ireland,
\end{enumerate}
to be enforced in Scotland as if it had been made by the sheriff.

(2) The power conferred on the Court of Session by section 32 of the Sheriff Courts (Scotland) Act 1971 (power of Court of Session to regulate civil procedure in the sheriff court) shall extend to making provision for the registration in the sheriff court for enforcement of any such order as is referred to in subsection (1).

(3) The Secretary of State may by regulations make provision for, or in connection with, the enforcement in England and Wales of—
\begin{enumerate}\item[]
($a$) any liability order made by the sheriff in Scotland; or

($b$) any corresponding order made by a court in Northern Ireland,
\end{enumerate}
as if it had been made by a magistrates' court in England and Wales.

(4) Regulations under subsection (3)  may, in particular, make provision for the registration of any such order as is referred to in that subsection in connection with its enforcement in England and Wales.

\amendment{
S. 39 came into force 27.6.92.
}

% S 39A inserted by 2000 c 19 s 16(1)
\subsection[39A. Commitment to prison and disqualification from driving]{\sloppy 39A. Commitment to prison and disqualification from driving}

(1) Where the Secretary of State has sought—
\begin{enumerate}\item[]
($a$) in England and Wales to 
levy an amount by distress under this Act%
%recover an amount by virtue of section 35(1)%  % Words substituted by 2007 c 15 Sch 13 para 95
; or

($b$) to recover an amount by virtue of section~36 or~38,
\end{enumerate}
and that amount, or any portion of it, remains unpaid he may apply to the court under this section.

(2) An application under this section is for whichever the court considers appropriate in all the circumstances of—
\begin{enumerate}\item[]
($a$) the issue of a warrant committing the liable person to prison; or

($b$) an order for him to be disqualified from holding or obtaining a driving licence.
\end{enumerate}

(3) On any such application the court shall (in the presence of the liable person) inquire as to—
\begin{enumerate}\item[]
($a$) whether he needs a driving licence to earn his living;

($b$) his means; and

($c$) whether there has been wilful refusal or culpable neglect on his part.
\end{enumerate}

(4) The Secretary of State may make representations to the court as to whether he thinks it more appropriate to commit the liable person to prison or to disqualify him from holding or obtaining a driving licence; and the liable person may reply to those representations.

(5) In this section and section~40B, “driving licence” means a licence to drive a motor vehicle granted under Part III of the Road Traffic Act 1988. 

(6) In this section “the court” means—
\begin{enumerate}\item[]
($a$) in England and Wales, a magistrates' court;

($b$) in Scotland, the sheriff.
\end{enumerate}

\amendment{
S. 39A inserted (10.11.00 for regulation-making purposes only, 2.4.01 for all other purposes) by the Child Support, Pensions and Social Security Act 2000 s. 16(1).

Words substituted in s. 39A(1)(a) (prosp) by the Tribunals, Courts and Enforcement Act 2007 Sch. 13 para. 95.
}

\subsection{40. Commitment to prison}

% S 40(1), (2) omitted by 2000 c 19 s 16(2)
%(1) Where the Secretary of State has sought—
%\begin{enumerate}\item[]
%($a$) to levy an amount by distress under this Act; or
%
%($b$) to recover an amount by virtue of section 36,
%\end{enumerate}
%and that amount, or any portion of it, remains unpaid he may apply to a magistrates' court for the issue of a warrant committing the liable person to prison.
%
%(2) On any such application the court shall (in the presence of the liable person) inquire as to—
%\begin{enumerate}\item[]
%($a$) the liable person’s means; and
%
%($b$) whether there has been wilful refusal or culpable neglect on his part.
%\end{enumerate}

(3) If, but only if, the court is of the opinion that there has been wilful refusal or culpable neglect on the part of the liable person it may—
\begin{enumerate}\item[]
($a$) issue a warrant of commitment against him; or

($b$) fix a term of imprisonment and postpone the issue of the warrant until such time and on such conditions (if any) as it thinks just.
\end{enumerate}

(4) Any such warrant—
\begin{enumerate}\item[]
($a$) shall be made in respect of an amount equal to the aggregate of—
\begin{enumerate}\item[]
(i) the amount mentioned in section 35(1)  or so much of it as remains outstanding; and

% S 40(4)(a)(i) substituted by 2007 c 15 Sch 13 para 96
%(i) the amount outstanding, as defined by paragraph 50(3) of Schedule 12 to the Tribunals, Courts and Enforcement Act 2007 (taking control of goods); and

(ii) an amount (determined in accordance with regulations made by the Secretary of State) in respect of the costs of commitment; and
\end{enumerate}

($b$) shall state that amount.
\end{enumerate}

(5) No warrant may be issued under this section against a person who is under the age of 18.

(6) A warrant issued under this section shall order the liable person—
\begin{enumerate}\item[]
($a$) to be imprisoned for a specified period; but

($b$) to be released (unless he is in custody for some other reason) on payment of the amount stated in the warrant.
\end{enumerate}

(7) The maximum period of imprisonment which may be imposed by virtue of subsection (6)  shall be calculated in accordance with Schedule 4 to the Magistrates' Courts Act 1980 (maximum periods of imprisonment in default of payment) but shall not exceed six weeks.

(8) The Secretary of State may by regulations make provision for the period of imprisonment specified in any warrant issued under this section to be reduced where there is part payment of the amount in respect of which the warrant was issued.

(9) A warrant issued under this section may be directed to such person or persons as the court issuing it thinks fit.

(10) Section 80 of the Magistrates' Courts Act 1980 (application of money found on defaulter) shall apply in relation to a warrant issued under this section against a liable person as it applies in relation to the enforcement of a sum mentioned in subsection (1)  of that section.

(11) The Secretary of State may by regulations make provision—
\begin{enumerate}\item[]
($a$) as to the form of any warrant issued under this section;

($b$) allowing an application under this section to be renewed where no warrant is issued or term of imprisonment is fixed;

($c$) that a statement in writing to the effect that wages of any amount have been paid to the liable person during any period, purporting to be signed by or on behalf of his employer, shall be evidence of the facts stated;

($d$) that, for the purposes of enabling an inquiry to be made as to the liable person’s conduct and means, a justice of the peace may issue a summons to him to appear before a magistrates' court and (if he does not obey) may issue a warrant for his arrest;

($e$) that for the purpose of enabling such an inquiry, a justice of the peace may issue a warrant for the liable person’s arrest without issuing a summons;

($f$) as to the execution of a warrant for arrest.
\end{enumerate}

%(12) Subsections (1)  to (11)  do not apply to Scotland.
%
%(13) For the avoidance of doubt, it is declared that a sum payable under a liability order is a sum decerned for aliment for the purposes of the Debtors (Scotland) Act 1880 and the Civil Imprisonment (Scotland) Act 1882.
%
%(14) Where a liability order has been made, the Secretary of State (and he alone) shall be regarded as, and may exercise all the powers of, the creditor for the purposes of section 4 (imprisonment for failure to obey decree for alimentary debt) of the Civil Imprisonment (Scotland) Act 1882.

% S 40(12) substituted for s 40(12)--(14) by 2000 c 19 s 17(1)
(12) This section does not apply to Scotland.

\amendment{
S. 40(4)(a)(ii), (8), (11) came into force 27.6.92; s. 40 came fully into force 5.4.93.

S. 40(1), (2) omitted (10.11.00 for regulation-making purposes, 2.4.01 otherwise) by the Child Support, Pensions and Social Security Act 2000 s. 16(2).

S. 40(12) substituted for s. 40(12)--(14) (10.11.00 for regulation-making purposes only, 2.4.01 otherwise) by the Child Support, Pensions and Social Security Act 2000 s. 17(1).

S. 40(4)(a)(i) substituted (prosp) by the Tribunals, Courts and Enforcement Act 2007 Sch. 13 para. 96.
}

% S 40A inserted by 2000 c 19 s 17(2)
\subsection{40A. Commitment to prison: Scotland}

(1) If, but only if, the sheriff is satisfied that there has been wilful refusal or culpable neglect on the part of the liable person he may—
\begin{enumerate}\item[]
($a$) issue a warrant for his committal to prison; or

($b$) fix a term of imprisonment and postpone the issue of the warrant until such time and on such conditions (if any) as he thinks just.
\end{enumerate}

(2) A warrant under this section—
\begin{enumerate}\item[]
($a$) shall be made in respect of an amount equal to the aggregate of—
\begin{enumerate}\item[]
(i) the appropriate amount under section~38; and

(ii) an amount (determined in accordance with regulations made by the Secretary of State) in respect of the expenses of commitment; and
\end{enumerate}

($b$) shall state that amount.
\end{enumerate}

(3) No warrant may be issued under this section against a person who is under the age of 18. 

(4) A warrant issued under this section shall order the liable person—
\begin{enumerate}\item[]
($a$) to be imprisoned for a specified period; but

($b$) to be released (unless he is in custody for some other reason) on payment of the amount stated in the warrant.
\end{enumerate}

(5) The maximum period of imprisonment which may be imposed by virtue of subsection~(4)  is six weeks.

(6) The Secretary of State may by regulations make provision for the period of imprisonment specified in any warrant issued under this section to be reduced where there is part payment of the amount in respect of which the warrant was issued.

(7) A warrant issued under this section may be directed to such person as the sheriff thinks fit.

(8) The power of the Court of Session by Act of Sederunt to regulate the procedure and practice in civil proceedings in the sheriff court shall include power to make provision—
\begin{enumerate}\item[]
($a$) as to the form of any warrant issued under this section;

($b$) allowing an application under this section to be renewed where no warrant is issued or term of imprisonment is fixed;

($c$) that a statement in writing to the effect that wages of any amount have been paid to the liable person during any period, purporting to be signed by or on behalf of his employer, shall be sufficient evidence of the facts stated;

($d$) that, for the purposes of enabling an inquiry to be made as to the liable person’s conduct and means, the sheriff may issue a citation to him to appear before the sheriff and (if he does not obey) may issue a warrant for his arrest;

($e$) that for the purpose of enabling such an inquiry, the sheriff may issue a warrant for~the liable person’s arrest without issuing a citation;

($f$) as to the execution of a warrant of arrest.
\end{enumerate}

\amendment{
S. 40A inserted (10.11.00 for regulation-making purposes only, 2.4.01 otherwise) by the Child Support, Pensions and Social Security Act 2000 s. 17(2).
}

% S 40B inserted by 2000 c 19 s 16(3)
\subsection{40B. Disqualification from driving: further provision}
 
(1) If, but only if, the court is of the opinion that there has been wilful refusal or culpable neglect on the part of the liable person, it may—
\begin{enumerate}\item[]
($a$) order him to be disqualified, for such period specified in the order but not exceeding two years as it thinks fit, from holding or obtaining a driving licence (a “disqualification order”); or

($b$) make a disqualification order but suspend its operation until such time and on such conditions (if any) as it thinks just.
\end{enumerate}

(2) The court may not take action under both section~40 and this section.

(3) A disqualification order must state the amount in respect of which it is made, which is to be the aggregate of—
\begin{enumerate}\item[]
($a$) the amount mentioned in section~35(1), or so much of it as remains outstanding; and

% S 40B(3)(a) substituted by 2007 c 15 Sch 13 para 97
%($a$) the amount outstanding, as defined by paragraph 50(3) of Schedule 12 to the Tribunals, Courts and Enforcement Act 2007 (taking control of goods); and

($b$) an amount (determined in accordance with regulations made by the Secretary of State) in respect of the costs of the application under section~39A.
\end{enumerate}

(4) A court which makes a disqualification order shall require the person to whom it relates to produce any driving licence held by him%
, and its counterpart (within the meaning of section~108(1)  of the Road Traffic Act 1988)%  % Words repealed by 2006 c 49 Sch 3 para 65(2)
.

(5) On an application by the Secretary of State or the liable person, the court—
\begin{enumerate}\item[]
($a$) may make an order substituting a shorter period of disqualification, or make an order revoking the disqualification order, if part of the amount referred to in subsection~(3)  (the “amount due”) is paid to any person authorised to receive it; and

($b$) must make an order revoking the disqualification order if all of the amount due is so paid.
\end{enumerate}

(6) The Secretary of State may make representations to the court as to the amount which should be paid before it would be appropriate to make an order revoking the disqualification order under subsection~(5)($a$), and the person liable may reply to those representations.

(7) The Secretary of State may make a further application under section~39A if the amount due has not been paid in full when the period of disqualification specified in the disqualification order expires.

(8) Where a court—
\begin{enumerate}\item[]
($a$) makes a disqualification order;

($b$) makes an order under subsection~(5); or

($c$) allows an appeal against a disqualification order,
\end{enumerate}
it shall send notice of that fact to the Secretary of State; and the notice shall contain such particulars and be sent in such manner and to such address as the Secretary of State may determine.

(9) Where a court makes a disqualification order, it shall also send the driving licence 
and its counterpart%  % Words repealed by 2006 c 49 Sch 3 para 65(3)(a)
, on 
their 
%its  % Word substituted by 2006 c 49 Sch 3 para 65(3)(b)
being produced to the court, to the Secretary of State at such address as he may determine.

(10) Section 80 of the Magistrates' Courts Act 1980 (application of money found on defaulter) shall apply in relation to a disqualification order under this section in relation to a liable person as it applies in relation to the enforcement of a sum mentioned in subsection~(1)  of that section.

(11) The Secretary of State may by regulations make provision in relation to disqualification orders corresponding to the provision he may make under section~40(11).

(12) In the application to Scotland of this section—
\begin{enumerate}\item[]
\begin{sloppypar}
($a$) in subsection~(2)  for “section~40” substitute “section~40A”;
\end{sloppypar}

($b$) in subsection~(3)  for paragraph~($a$)  substitute—
\begin{quotation}
“($a$) the appropriate amount under section~38;”;
\end{quotation}

($c$) subsection~(10)  is omitted; and

($d$) for subsection~(11)  substitute—
\begin{quotation}
“(11) The power of the Court of Session by Act of Sederunt to regulate the procedure and practice in civil proceedings in the sheriff court shall include power to make, in relation to disqualification orders, provision corresponding to that which may be made by virtue of section~40A(8).”
\end{quotation}
\end{enumerate}

\amendment{
S. 40B inserted (10.11.00 for regulation-making purposes only, 2.4.01 otherwise) by the Child Support, Pensions and Social Security Act 2000 s. 16(3).

Words in s. 40B(4), (9) repealed and word in s. 40B(9) substituted (prosp) by the Road Safety Act 2006 Sch. 3 para. 65.

S. 40B(3)(a) substituted (prosp) by the Tribunals, Courts and Enforcement Act 2007 Sch. 13 para. 97.
}

\subsection{41. Arrears of child support maintenance}

(1) This section applies where—
\begin{enumerate}\item[]
($a$) the Secretary of State is authorised under section 4, 6 or 7 to recover child support maintenance payable by 
%an absent parent 
\emph{a non-resident parent}  % Words substituted by 2000 c 19 Sch 3 para 11(2)
in accordance with a 
%maintenance assessment% 
\emph{maintenance calculation}%  % Words substituted by 2000 c 19 s 1(2)(a)
; and

($b$) the 
%absent parent 
\emph{non-resident parent}  % Words substituted by 2000 c 19 Sch 3 para 11(2)
has failed to make one or more payments of child support maintenance due from him in accordance with that 
%assessment% 
\emph{calculation}%  % Words substituted by 2000 c 19 s 1(2)(b)
.
\end{enumerate}

%(2) Where the Secretary of State recovers any such arrears he may, in such circumstances as may be prescribed and to such extent as may be prescribed, retain them if he is satisfied that the amount of any benefit paid to the person with care of the child or children in question would have been less had the absent parent not been in arrears with his payments of child support maintenance.

% S 41(2), (2A) substituted for s 41(2) (1.10.95) by 1995 c 34 Sch 3 para 11
(2) Where the Secretary of State recovers any such arrears he may, in such circumstances as may be prescribed and to such extent as may be prescribed, retain them if he is satisfied that the amount of any benefit paid to or in respect of the person with care of the child or children in question would have been less had the 
%absent parent 
\emph{non-resident parent}  % Words substituted by 2000 c 19 Sch 3 para 11(2)
made the payment or payments of child support maintenance in question.

(2A) In determining for the purposes of subsection (2) whether the amount of any benefit paid would have been less at any time than the amount which was paid at that time, in a case where the 
%maintenance assessment 
\emph{maintenance calculation}  % Words substituted by 2000 c 19 s 1(2)(a)
had effect from a date earlier than that on which it was made, the 
%assessment 
\emph{calculation}  % Words substituted by 2000 c 19 s 1(2)(b)
shall be taken to have been in force at that time.

% S 41(3)--(5) ceased to have effect by 2000 c 19 s 18(1)
(3) [\emph{1993 scheme only}] In such circumstances as may be prescribed, the absent parent shall be liable to make such payments of interest with respect to the arrears of child support maintenance as may be prescribed.

(4) [\emph{1993 scheme only}] The Secretary of State may by regulations make provision—
\begin{enumerate}\item[]
($a$) as to the rate of interest payable by virtue of subsection (3);

($b$) as to the time at which, and person to whom, any such interest shall be payable;

($c$) as to the circumstances in which, in a case where the Secretary of State has been acting under section 6, any such interest may be retained by him;

($d$) for the Secretary of State, in a case where he has been acting under section 6 and in such circumstances as may be prescribed, to waive any such interest (or part of any such interest).
\end{enumerate}

(5) [\emph{1993 scheme only}] The provisions of this Act with respect to—
\begin{enumerate}\item[]
($a$) the collection of child support maintenance;

($b$) the enforcement of any obligation to pay child support maintenance,
\end{enumerate}
shall apply equally to interest payable by virtue of this section.

(6) Any sums retained by the Secretary of State by virtue of this section shall be paid by him into the Consolidated Fund.

\amendment{
S. 41(2)--(4) came into force 27.6.92; s. 41 came fully into force 5.4.93.

S. 41(2), (2A) substituted for s. 41(2) (1.10.95) by the Child Support Act 1995 Sch.~3 para.~11.

S. 41(3)--(5) ceased to have effect (10.11.00 for regulation-making purposes, 3.3.03 for 2003 scheme cases) by the Child Support, Pensions and Social Security Act 2000 s. 18(1).
}

% S 41A inserted by 2000 c 19 s 18(2)
\subsection[41A. Penalty payments --- \emph{2003 scheme only}]{41A. Penalty payments\\*\emph{2003 scheme only}}

(1) The Secretary of State may by regulations make provision for the payment to him by non-resident parents who are in arrears with payments of child support maintenance of penalty payments determined in accordance with the regulations.

(2) The amount of a penalty payment in respect of any week may not exceed 25\% of the amount of child support maintenance payable for that week, but otherwise is to be determined by the Secretary of State.

(3) The liability of a non-resident parent to make a penalty payment does not affect his liability to pay the arrears of child support maintenance concerned.

(4) Regulations under subsection~(1)  may, in particular, make provision—
\begin{enumerate}\item[]
($a$) as to the time at which a penalty payment is to be payable;

($b$) for the Secretary of State to waive a penalty payment, or part of it.
\end{enumerate}

(5) The provisions of this Act with respect to—
\begin{enumerate}\item[]
($a$) the collection of child support maintenance;

($b$) the enforcement of an obligation to pay child support maintenance,
\end{enumerate}
apply equally (with any necessary modifications) to penalty payments payable by virtue of regulations under this section.

(6) The Secretary of State shall pay penalty payments received by him into the Consolidated Fund.

\amendment{
S. 41A inserted (10.11.00 for regulation-making purposes, 3.3.03 for 2003 scheme cases) by the Child Support, Pensions and Social Security Act 2000 s. 18(2).
}

\subsection{41B. Repayment of overpaid child support maintenance}

(1) This section applies where it appears to the Secretary of State that 
%an absent parent 
\emph{a non-resident parent}  % Words substituted by 2000 c 19 Sch 3 para 11(2)
has made a payment by way of child support maintenance which amounts to an overpayment by him of that maintenance and that—
\begin{enumerate}\item[]
($a$) it would not be possible for the 
%absent parent 
\emph{non-resident parent}  % Words substituted by 2000 c 19 Sch 3 para 11(2)
to recover the amount of the overpayment by way of an adjustment of the amount payable under a 
%maintenance assessment% 
\emph{maintenance calculation}%  % Words substituted by 2000 c 19 s 1(2)(a)
; or

($b$) it would be inappropriate to rely on an adjustment of the amount payable under a 
%maintenance assessment 
\emph{maintenance calculation}  % Words substituted by 2000 c 19 s 1(2)(a)
as the means of enabling the 
%absent parent 
\emph{non-resident parent}  % Words substituted by 2000 c 19 Sch 3 para 11(2)
to recover the amount of the overpayment.
\end{enumerate}

% S 41B(1A) inserted by 2000 c 19 s 20(3)
(1A) [\emph{2003 scheme only}] This section also applies where the non-resident parent has made a voluntary payment and it appears to the Secretary of State—
\begin{enumerate}\item[]
($a$) that he is not liable to pay child support maintenance; or

($b$) that he is liable, but some or all of the payment amounts to an overpayment,
\end{enumerate}
and, in a case falling within paragraph~($b$), it also appears to him that subsection~(1)($a$)  or~($b$)  applies.

(2) The Secretary of State may make such payment to the 
%absent parent 
\emph{non-resident parent}  % Words substituted by 2000 c 19 Sch 3 para 11(2)
by way of reimbursement, or partial reimbursement, of the overpayment as the Secretary of State considers appropriate.

(3) Where the Secretary of State has made a payment under this section he may, in such circumstances as may be prescribed, require the relevant person to pay to him the whole, or a specified proportion, of the amount of that payment.

(4) Any such requirement shall be imposed by giving the relevant person a written demand for the amount which the Secretary of State wishes to recover from him.

(5) Any sum which a person is required to pay to the Secretary of State under this section shall be recoverable from him by the Secretary of State as a debt due to the Crown.

(6) The Secretary of State may by regulations make provision in relation to any case in which—
\begin{enumerate}\item[]
($a$) one or more overpayments of child support maintenance are being reimbursed to the Secretary of State by the relevant person; and

($b$) child support maintenance has continued to be payable by the 
%absent parent 
\emph{non-resident parent}  % Words substituted by 2000 c 19 Sch 3 para 11(2)
concerned to the person with care concerned, or again becomes so payable.
\end{enumerate}

(7) [\emph{1993 scheme version}] For the purposes of this section any payments made by a person under a maintenance assessment which was not validly made shall be treated as overpayments of child support maintenance made by an absent parent.

% S 41B(7) substituted by 2000 c 19 s 20(4)
(7) [\emph{2003 scheme version}] For the purposes of this section—
\begin{enumerate}\item[]
($a$) a payment made by a person under a maintenance calculation which was not validly made; and

($b$) a voluntary payment made in the circumstances set out in subsection~(1A)($a$),
\end{enumerate}
shall be treated as an overpayment of child support maintenance made by a non-resident parent.

(8) In this section “relevant person”, in relation to an overpayment, means the person with care to whom the overpayment was made.

(9) Any sum recovered by the Secretary of State under this section shall be paid by him into the Consolidated Fund.

\amendment{
S. 41B(1), (2), (7) inserted (4.9.95) by the Child Support Act 1995 s.~23.

S. 41B(3)--(6), (8), (9) inserted (1.10.95) by the Child Support Act 1995 s.~23.

S. 41B(1A) inserted and s. 41B(7) substituted (10.11.00 for regulation-making purposes, 3.3.03 for 2003 scheme cases) by the Child Support, Pensions and Social Security Act 2000 s. 20(3), (4).
}

\section{\itshape Special cases}

\subsection{42. Special cases}

(1) The Secretary of State may by regulations provide that in prescribed circumstances a case is to be treated as a special case for the purposes of this Act.

(2) Those regulations may, for example, provide for the following to be special cases—
\begin{enumerate}\item[]
($a$) each parent of a child is 
%an absent parent 
\emph{a non-resident parent}  % Words substituted by 2000 c 19 Sch 3 para 11(2)
in relation to the child;

($b$) there is more than one person who is a person with care in relation to the same child;

($c$) there is more than one qualifying child in relation to the same 
%absent parent 
\emph{non-resident parent}  % Words substituted by 2000 c 19 Sch 3 para 11(2)
but the person who is the person with care in relation to one of those children is not the person who is the person with care in relation to all of them;

($d$) a person is 
%an absent parent 
\emph{a non-resident parent}  % Words substituted by 2000 c 19 Sch 3 para 11(2)
in relation to more than one child and the other parent of each of those children is not the same person;

($e$) the person with care has care of more than one qualifying child and there is more than one 
%absent parent 
\emph{non-resident parent}  % Words substituted by 2000 c 19 Sch 3 para 11(2)
in relation to those children;

($f$) a qualifying child has his home in two or more separate households.
\end{enumerate}

(3) The Secretary of State may by regulations make provision with respect to special cases.

(4) Regulations made under subsection (3)  may, in particular—
\begin{enumerate}\item[]
($a$) modify any provision made by or under this Act, in its application to any special case or any special case falling within a prescribed category;

($b$) make new provision for any such case; or

($c$) provide for any prescribed provision made by or under this Act not to apply to any such case.
\end{enumerate}

\amendment{
S. 42 came into force 27.6.92.
}

\subsection[43. Contribution to maintenance by deduction from benefit --- \emph{1993 scheme version}]{43. Contribution to maintenance by deduction from benefit\\*\emph{1993 scheme version}}

(1) This section applies where—
\begin{enumerate}\item[]
($a$) by virtue of paragraph 5(4)  of Schedule 1, an absent parent is taken for the purposes of that Schedule to have no assessable income; and

($b$) such conditions as may be prescribed for the purposes of this section are satisfied.
\end{enumerate}

(2) The power of the Secretary of State to make regulations under 
%section~51 of the Social Security Act 1986 by virtue of subsection (1)($r$), 
section~5 of the Social Security Administration Act 1992 by virtue of subsection (1)($t$),  % Words substituted (1.7.92) by 1992 c 6 Sch 2 para 113
(deductions from benefits) may be exercised in relation to cases to which this section applies with a view to securing that—
\begin{enumerate}\item[]
($a$) payments of prescribed amounts are made with respect to qualifying children in place of payments of child support maintenance; and

($b$) arrears of child support maintenance are recovered.
\end{enumerate}

% S 43(3) inserted (1.6.99) by 1998 c 14 Sch 7 para 40
(3) Schedule 4C shall have effect for applying sections 16, 17, 20 and~28ZA to 28ZC to any decision with respect to a person’s liability under this section, that is to say, his liability to make payments under regulations made by virtue of this section.

\amendment{
S. 43(1)(b), (2)(a) came into force 27.6.92; s. 43 came fully into force 5.4.93.

Words substituted in s.~43(2) (1.7.92) by the Social Security (Consequential Provisions) Act 1992 Sch.~2 para.~113.

S. 43(3) inserted (1.6.99) by the Social Security Act 1998 Sch. 7 para. 40.
}

% S 43 substituted by 2000 c 19 s 21
\subsection[43. Recovery of child support maintenance by deduction from benefit --- \emph{2003 scheme version}]{43. Recovery of child support maintenance by deduction from benefit\\*\emph{2003 scheme version}}

(1) This section~applies where—
\begin{enumerate}\item[]
($a$) a non-resident parent is liable to pay a flat rate of child support maintenance (or would be so liable but for a variation having been agreed to), and that rate applies (or would have applied) because he falls within paragraph~4(1)($b$)  or~($c$)  or~4(2)  of Schedule 1; and

($b$) such conditions as may be prescribed for the purposes of this section are satisfied.
\end{enumerate}

(2) The power of the Secretary of State to make regulations under section~5 of the Social Security Administration Act 1992 by virtue of subsection~(1)($p$)  (deductions from benefits) may be exercised in relation to cases to which this section applies with a view to securing that payments in respect of child support maintenance are made or that arrears of child support maintenance are recovered.

(3) For the purposes of this section, the benefits to which section~5 of the 1992 Act applies are to be taken as including war disablement pensions and war widows' pensions (within the meaning of section~150 of the Social Security Contributions and Benefits Act 1992 (interpretation)).

\amendment{
S. 43 substituted (10.11.00 for regulation-making purposes only) by the Child Support, Pensions and Social Security Act 2000 s. 21.
}

\section{\itshape Jurisdiction}

\subsection{44. Jurisdiction}

(1) 
%A child support officer 
The Secretary of State  % Words substituted (1.6.99) by 1998 c 14 Sch 7 para 41
shall have jurisdiction to make a 
%maintenance assessment 
\emph{maintenance calculation}  % Words substituted by 2000 c 19 s 1(2)(a)
with respect to a person who is—
\begin{enumerate}\item[]
($a$) a person with care;

($b$) 
%an absent parent% 
\emph{a non-resident parent}%  % Words substituted by 2000 c 19 Sch 3 para 11(2)
; or

($c$) a qualifying child,
\end{enumerate}
only if that person is habitually resident in the United Kingdom%
, except in the case of a non-resident parent who falls within subsection~(2A)%  % Words inserted by 2000 c 19 s 22(2)
.

(2) Where the person with care is not an individual, subsection (1)  shall have effect as if paragraph ($a$)  were omitted.

% S 44(2A) inserted by 2000 c 19 s 22(3)
(2A) A non-resident parent falls within this subsection if he is not habitually resident in the United Kingdom, but is—
\begin{enumerate}\item[]
($a$) employed in the civil service of the Crown, including Her Majesty’s Diplomatic Service and Her Majesty’s Overseas Civil Service;

($b$) a member of the naval, military or air forces of the Crown, including any person employed by an association established for the purposes of Part XI of the Reserve Forces Act 1996;

($c$) employed by a company of a prescribed description registered under the Companies Act 1985 in England and Wales or in Scotland, or under the Companies (Northern Ireland) Order 1986; or

($d$) employed by a body of a prescribed description.
\end{enumerate}

(3) [\emph{1993 scheme only}] The Secretary of State may by regulations make provision for the cancellation of any maintenance assessment where—
\begin{enumerate}\item[]
($a$) the person with care, absent parent or qualifying child with respect to whom it was made ceases to be habitually resident in the United Kingdom;

($b$) in a case falling within subsection (2), the absent parent or qualifying child with respect to whom it was made ceases to be habitually resident in the United Kingdom; or

($c$) in such circumstances as may be prescribed, a maintenance order of a prescribed kind is made with respect to any qualifying child with respect to whom the maintenance assessment was made.
\end{enumerate}

\amendment{
S. 44(3) came into force 27.6.92; s. 44 came fully into force 5.4.93.

Words substituted in s. 44(1) (1.6.99) by the Social Security Act 1998 Sch. 7 para. 41.

S. 44(2A) inserted (10.11.00 for regulation-making purposes only, 31.1.01 for all other purposes) by the Child Support, Pensions and Social Security Act 2000 s. 22(3).

Words inserted in s. 44(1) (31.1.01) by the Child Support, Pensions and Social Security Act 2000 s. 22(2).

S. 44(3) repealed (3.3.03 for 2003 scheme cases only) by the Child Support, Pensions and Social Security Act 2000 s. 22(4).
}

\subsection{45. Jurisdiction of courts in certain proceedings under this Act}

(1) The Lord Chancellor or, in relation to Scotland, the Lord Advocate may by order make such provision as he considers necessary to secure that appeals, or such class of appeals as may be specified in the order—
\begin{enumerate}\item[]
($a$) shall be made to a court instead of being made to 
%a child support appeal tribunal%
an appeal tribunal%  % Words substituted (1.6.99) by 1998 c 14 Sch 7 para 42(1)
; or

($b$) shall be so made in such circumstances as may be so specified.
\end{enumerate}

(2) In subsection (1), “court” means—
\begin{enumerate}\item[]
($a$) in relation to England and Wales and subject to any provision made under Schedule 11 to the Children Act 1989 (jurisdiction of courts with respect to certain proceedings relating to children) the High Court, a county court or a magistrates' court; and

($b$) in relation to Scotland, the Court of Session or the sheriff.
\end{enumerate}

(3) Schedule 11 to the Act of 1989 shall be amended in accordance with subsections (4)  and (5).

(4) The following sub-paragraph shall be inserted in paragraph 1, after sub-paragraph (2)—
\begin{quotation}
“(2A) Sub-paragraphs (1)  and (2)  shall also apply in relation to proceedings—
\begin{enumerate}\item[]
($a$) under section 27 of the Child Support Act 1991 (reference to court for declaration of parentage); or

($b$) which are to be dealt with in accordance with an order made under section 45 of that Act (jurisdiction of courts in certain proceedings under that Act)”.
\end{enumerate}
\end{quotation}

(5) In paragraphs 1(3)  and 2(3), the following shall be inserted after “Act 1976”—
\begin{quotation}
“($bb$) section 20 (appeals) or 27 (reference to court for declaration of parentage) of the Child Support Act 1991;”.
\end{quotation}

(6) Where the effect of any order under subsection (1)  is that there are no longer any appeals which fall to be dealt with by 
%child support appeal tribunals%
appeal tribunals%  % Words substituted (1.6.99) by 1998 c 14 Sch 7 para 42(2)
, the Lord Chancellor after consultation with the Lord Advocate may by order provide for the abolition of those tribunals.

(7) Any order under subsection (1)  or (6)  may make—
\begin{enumerate}\item[]
($a$) such modifications of any provision of this Act or of any other enactment; and

($b$) such transitional provision,
\end{enumerate}
as the Minister making the order considers appropriate in consequence of any provision made by the order.

% S 45(8), (9) inserted by 2005 c 4 Sch 4 para 220
(8) The functions of the Lord Chancellor under this section may be exercised only after consultation with the Lord Chief Justice.

(9) The Lord Chief Justice may nominate a judicial office holder (as defined in section 109(4) of the Constitutional Reform Act 2005) to exercise his functions under this section.

\amendment{
S. 45 came into force 27.6.92.

Words substituted in s. 45(1), (6) (1.6.99) by the Social Security Act 1998 Sch. 7 para. 42.

S. 45(8), (9) inserted (3.4.06) by the Constitutional Reform Act 2005 Sch. 4 para. 220.
}

\section{\itshape Miscellaneous and supplemental}

\subsection[46. Failure to comply with obligations imposed by section 6 --- \emph{1993 scheme version}]{46. Failure to comply with obligations imposed by section 6\\*\emph{1993 scheme version}}

(1) This section applies where any person (“the parent”)—
\begin{enumerate}\item[]
($a$) fails to comply with a requirement imposed on her by the Secretary of State under section 6(1); or

($b$) fails to comply with any regulation made under section 6(9).
\end{enumerate}

(2) 
%A child support officer 
The Secretary of State  % Words substituted (1.6.99) by 1998 c 14 Sch 7 para 43(1)
may serve written notice on the parent requiring her, before the end of the specified period, either to comply or to give him her reasons for failing to do so.

(3) When the specified period has expired, 
%the child support officer 
the Secretary of State  % Words substituted (1.6.99) by 1998 c 14 Sch 7 para 43(2)
shall consider whether, having regard to any reasons given by the parent, there are reasonable grounds for believing that, if she were to be required to comply, there would be a risk of her or of any children living with her suffering harm or undue distress as a result of complying.

(4) If 
%the child support officer 
the Secretary of State  % Words substituted (1.6.99) by 1998 c 14 Sch 7 para 43(2)
considers that there are such reasonable grounds, he shall—
\begin{enumerate}\item[]
($a$) take no further action under this section in relation to the failure in question; and

($b$) notify the parent, in writing, accordingly.
\end{enumerate}

(5) If 
%the child support officer 
the Secretary of State  % Words substituted (1.6.99) by 1998 c 14 Sch 7 para 43(2)
considers that there are no such reasonable grounds, he may% 
, except in prescribed circumstances,  % Words inserted (1.10.95) by 1995 c 34 Sch 3 para 12
give a reduced benefit direction with respect to the parent.

(6) Where 
%the child support officer 
the Secretary of State  % Words substituted (1.6.99) by 1998 c 14 Sch 7 para 43(2)
gives a reduced benefit direction he shall send a copy of it to the parent.

%(7) Any person who is aggrieved by a decision of a child support officer to give a reduced benefit direction may appeal to a child support appeal tribunal against that decision.
%
%(8) Sections 20(2)  to (4)  and 21 shall apply in relation to appeals under subsection (7)  as they apply in relation to appeals under section 20.

% S 46(7) substituted for s 46(7), (8) (1.6.99) by 1998 c 14 Sch 7 para 43(3)
(7) Schedule 4C shall have effect for applying sections 16, 17, 20 and 28ZA to 28ZC to decisions with respect to reduced benefit directions.

(9) A reduced benefit direction shall take effect on such date as may be specified in the direction.

(10) Reasons given in response to a notice under subsection (2)  may be given either in writing or orally.

(11) In this section—
\begin{enumerate}\item[]
    “comply” means to comply with the requirement or with the regulation in question; and “complied” and “complying” shall be construed accordingly;

    “reduced benefit direction” means a direction%
%, binding on the adjudication officer,  % Words repealed (1.6.99) by 1998 c 14 Sch 7 para 43(4)
that the amount payable by way of any relevant benefit to, or in respect of, the parent concerned be reduced by such amount, and for such period, as may be prescribed;

    “relevant benefit” means income support, 
an income-based jobseeker’s allowance  % Words inserted by 1995 c 18 Sch 2 para 20(4)
%, family credit  % Words repealed by 1999 c 10 Sch 6
%, an income-related employment and support allowance  % Words inserted by 2007 c 5 Sch 3 para 7(5)
or any other benefit of a kind prescribed for the purposes of section 6; and

    “specified”, in relation to any notice served under this section, means specified in the notice; and the period to be specified shall be determined in accordance with regulations made by the Secretary of State. 
\end{enumerate}

\amendment{
S. 46(11) came into force 27.6.92; s. 46 came fully into force 5.4.93.

Words inserted in s.~46(5) (1.10.95) by the Child Support Act 1995 Sch.~3 para.~12.

Words inserted in definition of ``relevant benefit'' in s. 46(11) (7.10.96) by the Jobseekers Act 1995 Sch.~2 para.~20(4).

Words substituted in s. 46(2)--(6), s. 46(7) substituted for s. 46(7), (8) and words repealed in s. 46(11) (1.6.99) by the Social Security Act 1998 Sch. 7 para. 43.

Words repealed in s. 46(11) (5.10.99) by the Tax Credits Act 1999 Sch. 6.

Words inserted in definition of ``relevant benefit'' in s. 46(11) (prosp) by the Welfare Reform Act 2007 Sch. 3 para. 7(5).
}

\subsection[46. Reduced benefit decisions --- \emph{2003 scheme version}]{46. Reduced benefit decisions\\*\emph{2003 scheme version}}

(1) This section applies where any person (“the parent”)—
\begin{enumerate}\item[]
($a$) has made a request under section~6(5);

($b$) fails to comply with any regulation made under section~6(7); or

($c$) having been treated as having applied for a maintenance calculation under section~6, refuses to take a scientific test (within the meaning of section~27A).
\end{enumerate}

(2) The Secretary of State may serve written notice on the parent requiring her, before the end of a specified period—
\begin{enumerate}\item[]
($a$) in a subsection~(1)($a$)  case, to give him her reasons for~making the request;

($b$) in a subsection~(1)($b$)  case, to give him her reasons for~failing to do so; or

($c$) in a subsection~(1)($c$)  case, to give him her reasons for~her refusal.
\end{enumerate}

(3) When the specified period has expired, the Secretary of State shall consider whether, having regard to any reasons given by the parent, there are reasonable grounds for~believing that—
\begin{enumerate}\item[]
($a$) in a subsection~(1)($a$)  case, if the Secretary of State were to do what is mentioned in section~6(3);

($b$) in a subsection~(1)($b$)  case, if she were to be required to comply; or

($c$) in a subsection~(1)($c$)  case, if she took the scientific test,
\end{enumerate}
there would be a risk of her, or of any children living with her, suffering harm or undue distress as a result of his taking such action, or her complying or taking the test.

(4) If the Secretary of State considers that there are such reasonable grounds, he shall—
\begin{enumerate}\item[]
($a$) take no further action under this section in relation to the request, the failure or the refusal in question; and

($b$) notify the parent, in writing, accordingly.
\end{enumerate}

(5) If the Secretary of State considers that there are no such reasonable grounds, he may, except in prescribed circumstances, make a reduced benefit decision with respect to the parent.

(6) In a subsection~(1)($a$)  case, the Secretary of State may from time to time serve written notice on the parent requiring her, before the end of a specified period—
\begin{enumerate}\item[]
($a$) to state whether her request under section~6(5)  still stands; and

($b$) if so, to give him her reasons for maintaining her request,
\end{enumerate}
and subsections (3)  to (5)  have effect in relation to such a notice and any response to it as they have effect in relation to a notice under subsection~(2)($a$)  and any response to it.

(7) Where the Secretary of State makes a reduced benefit decision he must send a copy of it to the parent.

(8) A reduced benefit decision is to take effect on such date as may be specified in the decision.

(9) Reasons given in response to a notice under subsection~(2)  or~(6)  need not be given in writing unless the Secretary of State directs in any case that they must.

(10) In this section—
\begin{enumerate}\item[]
($a$) “comply” means to comply with the requirement or with the regulation in question; and “complied” and “complying” are to be construed accordingly;

($b$) “reduced benefit decision” means a decision that the amount payable by way of any relevant benefit to, or in respect of, the parent concerned be reduced by such amount, and for such period, as may be prescribed;

($c$) “relevant benefit” means income support
or an income-based jobseeker’s allowance 
%, an income-based jobseeker's allowance, an income-related employment and support allowance  % Words substituted by 2007 c 5 Sch 3 para 7(4)
or any other benefit of a kind prescribed for the purposes of section~6; and

($d$) “specified”, in relation to a notice served under this section, means specified in the notice; and the period to be specified is to be determined in accordance with regulations made by the Secretary of State.
\end{enumerate}

\amendment{
S. 46 substituted (10.11.00 for regulation-making purposes, 3.3.03 for 2003 scheme cases) by the Child Support, Pensions and Social Security Act 2000 s. 19.

Words substituted in s. 46(10)(c) (prosp) by the Welfare Reform Act 2007 Sch. 3 para. 7(4).

\medskip

Ss. 46A, 46B inserted (1.6.99) by the Social Security Act 1998 Sch. 7 para. 44.
}

% Ss 46A, 46B inserted (1.6.99) by 1998 c 14 Sch 7 para 44
\subsection{46A. Finality of decisions}

(1) Subject to the provisions of this Act, any decision of the Secretary of State or an appeal tribunal made in accordance with the foregoing provisions of this Act shall be final.

(2) If and to the extent that regulations so provide, any finding of fact or other determination embodied in or necessary to such a decision, or on which such a decision is based, shall be conclusive for the purposes of—
\begin{enumerate}\item[]
($a$) further such decisions;

($b$) decisions made in accordance with sections 8 to 16 of the Social Security Act 1998, or with regulations under section 11 of that Act; and

($c$) decisions made under the Vaccine Damage Payments Act 1979.
\end{enumerate}

\subsection{46B. Matters arising as respects decisions}

(1) Regulations may make provision as respects matters arising pending—
\begin{enumerate}\item[]
($a$) any decision of the Secretary of State under section 11, 12 or 17;

($b$) any decision of an appeal tribunal under section 20; or

($c$) any decision of a Child Support Commissioner under section 24.
\end{enumerate}

(2) Regulations may also make provision as respects matters arising out of the revision under section 16, or on appeal, of any such decision as is mentioned in subsection (1).

% S 46B(3) repealed by 2000 c 19 Sch 9 Pt I
(3) [\emph{1993 scheme only}] Any reference in this section to section 16, 17 or~20 includes a reference to that section as extended by Schedule 4C.

\amendment{
S. 46B(3) repealed (3.3.03 for 2003 scheme cases) by the Child Support, Pensions and Social Security Act 2000 Sch. 9 Pt. I.
}

\subsection{47. Fees}

(1) The Secretary of State may by regulations provide for the payment, by the 
%absent parent 
\emph{non-resident parent}  % Words substituted by 2000 c 19 Sch 3 para 11(2)
or the person with care (or by both), of such fees as may be prescribed in cases where the Secretary of State takes any action under section 4 or 6.

(2) The Secretary of State may by regulations provide for the payment, by the 
%absent parent% 
\emph{non-resident parent}%  % Words substituted by 2000 c 19 Sch 3 para 11(2)
, the person with care or the child concerned (or by any or all of them), of such fees as may be prescribed in cases where the Secretary of State takes any action under section 7.

(3) Regulations made under this section—
\begin{enumerate}\item[]
($a$) may require any information which is needed for the purpose of determining the amount of any such fee to be furnished, in accordance with the regulations, by such person as may be prescribed;

($b$) shall provide that no such fees shall be payable by any person to or in respect of whom income support, 
an income-based jobseeker’s allowance,  % Words inserted (7.10.96) by 1995 c 18 Sch 2 para 20(5)
%family credit 
%working families' tax credit  % Words substituted (5.10.99) by 1999 c 10 Sch 1 para 1(a), 6(f)(i)
%an income-related employment and support allowance,  % Words inserted by 2007 c 5 Sch 3 para 7(6)
any element of child tax credit other than the family element, working tax credit  % Words substituted by 2002 c 21 Sch 3 para 22
or any other benefit of a prescribed kind is paid; and

($c$) may, in particular, make provision with respect to the recovery by the Secretary of State of any fees payable under the regulations.
\end{enumerate}

% S 47(4) inserted by 2000 c 19 Sch 3 para 11(18)
(4) [\emph{2003 scheme only}] The provisions of this Act with respect to—
\begin{enumerate}\item[]
($a$) the collection of child support maintenance;

($b$) the enforcement of any obligation to pay child support maintenance,
\end{enumerate}
shall apply equally (with any necessary modifications) to fees payable by virtue of regulations made under this section.

\amendment{
S. 47 came into force 27.6.92.

Words inserted in s. 47(3)(b) (7.10.96) by the Jobseekers Act 1995 Sch.~2 para.~20(5).

Words substituted in s. 47(3)(b) (5.10.99) by the Tax Credits Act 1999 Sch. 1. paras. 1(a), 6(f)(i).

S. 47(4) inserted (10.11.00 for regulation-making purposes, 3.3.03 for 2003 scheme cases) by the Child Support, Pensions and Social Security Act 2000 Sch. 3 para. 11(18).

Words substituted in s. 47(3)(b) (6.4.03) by the Tax Credits Act 2002 Sch. 3 para. 22.

Words inserted in s. 47(3)(b) (prosp) by the Welfare Reform Act 2007 Sch. 3 para. 7(6).
}

\subsection{48. Right of audience}

(1) Any 
%person authorised 
officer of the Secretary of State who is authorised  % Words substituted (4.9.95) by 1995 c 34 Sch 3 para 14
by the Secretary of State for the purposes of this section shall have, in relation to any proceedings under this Act before a magistrates' court, a right of audience and the right to conduct litigation.

(2) In this section “right of audience” and “right to conduct litigation” have the same meaning as in section 119 of the Courts and Legal Services Act 1990.

\amendment{
S. 48 came into force 5.4.93.

Words substituted in s.~48(1) (4.9.95) by the Child Support Act 1995 Sch.~3 para.~14.
}

\subsection{49. Right of audience: Scotland}

In relation to any proceedings before the sheriff under any provision of this Act, the power conferred on the Court of Session by section 32 of the Sheriff Courts (Scotland) Act 1971 (power of Court of Session to regulate civil procedure in sheriff court) shall extend to the making of rules permitting a party to such proceedings, in such circumstances as may be specified in the rules, to be represented by a person who is neither an advocate nor a solicitor.

\amendment{
S. 49 came into force 27.6.92.
}

\subsection{50. Unauthorised disclosure of information}

(1) Any person who is, or has been, employed in employment to which this section applies is guilty of an offence if, without lawful authority, he discloses any information which—
\begin{enumerate}\item[]
($a$) was acquired by him in the course of that employment; and

($b$) relates to a particular person.
\end{enumerate}

(2) It is not an offence under this section—
\begin{enumerate}\item[]
($a$) to disclose information in the form of a summary or collection of information so framed as not to enable information relating to any particular person to be ascertained from it; or

($b$) to disclose information which has previously been disclosed to the public with lawful authority.
\end{enumerate}

(3) It is a defence for a person charged with an offence under this section to prove that at the time of the alleged offence—
\begin{enumerate}\item[]
($a$) he believed that he was making the disclosure in question with lawful authority and had no reasonable cause to believe otherwise; or

($b$) he believed that the information in question had previously been disclosed to the public with lawful authority and had no reasonable cause to believe otherwise.
\end{enumerate}

(4) A person guilty of an offence under this section shall be liable—
\begin{enumerate}\item[]
($a$) on conviction on indictment, to imprisonment for a term not exceeding two years or a fine or both; or

($b$) on summary conviction, to imprisonment for a term not exceeding six months or a fine not exceeding the statutory maximum or both.
\end{enumerate}

(5) This section applies to employment as—
\begin{enumerate}\item[]
($a$) the Chief Child Support Officer;

($b$) any other child support officer;

($c$) any clerk to, or other officer of, 
an appeal tribunal or  % Words inserted (1.6.99) by 1998 c 14 Sch 7 para 45
a child support appeal tribunal;

($d$) any member of the staff of such a tribunal;

($e$) a civil servant in connection with the carrying out of any functions under this Act,
\end{enumerate}
and to employment of any other kind which is prescribed for the purposes of this section.

(6) For the purposes of this section a disclosure is to be regarded as made with lawful authority if, and only if, it is made—
\begin{enumerate}\item[]
($a$) by a civil servant in accordance with his official duty; or

($b$) by any other person either—
\begin{enumerate}\item[]
(i) for the purposes of the function in the exercise of which he holds the information and without contravening any restriction duly imposed by the responsible person; or

(ii) to, or in accordance with an authorisation duly given by, the responsible person;
\end{enumerate}

($c$) in accordance with any enactment or order of a court;

($d$) for the purpose of instituting, or otherwise for the purposes of, any proceedings before a court or before any tribunal or other body or person mentioned in this Act; or

($e$) with the consent of the appropriate person.
\end{enumerate}

(7) “The responsible person” means—
\begin{enumerate}\item[]
($a$) the Lord Chancellor;

($b$) the Secretary of State;

($c$) any person authorised by the Lord Chancellor, or Secretary of State, for the purposes of this subsection; or

($d$) any other prescribed person, or person falling within a prescribed category.
\end{enumerate}

(8) “The appropriate person” means the person to whom the information in question relates, except that if the affairs of that person are being dealt with—
\begin{enumerate}\item[]
($a$) under a power of attorney;
%or  % Word inserted by 2005 c 9 Sch 6 para 36(a)(i)

% S 50(8)(b) repealed by 2005 c 9 Sch 6 para 36(a)(ii)
($b$) by a receiver appointed under section 99 of the Mental Health Act 1983;

($c$) by a Scottish mental health custodian, that is to say
%—
%\begin{enumerate}\item[]
%(i) a curator bonis, tutor or judicial factor; or
%
%(ii) the managers of a hospital acting on behalf of that person under section 94 of the Mental Health (Scotland) Act 1984; 
%or  % Word repealed by 2005 c 9 Sch 6 para 36(a)(ii)
%\end{enumerate}
a guardian or other person entitled to act on behalf of the person under the Adults with Incapacity (Scotland) Act 2000;  % Words substituted by 2000 asp 4 Sch 5 para 22 (Scotland), SI 2005/1790 art 2 (England and Wales)

% S 50(8)(d) repealed by 2005 c 9 Sch 6 para 36(a)(ii)
($d$) by a mental health appointee, that is to say—
\begin{enumerate}\item[]
(i) a person directed or authorised as mentioned in sub-\hspace{0pt}paragraph~($a$)  of rule 41(1)  of the Court of Protection Rules 1984; or

(ii) a receiver ad interim appointed under sub-paragraph ($b$)  of that rule;
\end{enumerate}
\end{enumerate}
the appropriate person is the attorney%
, receiver, custodian or appointee 
%or custodian  % Words substituted by 2005 c 9 Sch 6 para 36(a)(iii)
(as the case may be) or, in a case falling within paragraph ($a$), the person to whom the information relates.

% S 50(9) inserted by 2005 c 9 Sch 6 para 36(b)
%(9) Where the person to whom the information relates lacks capacity (within the meaning of the Mental Capacity Act 2005) to consent to its disclosure, the appropriate person is—
%\begin{enumerate}\item[]
%($a$) a donee of an enduring power of attorney or lasting power of attorney (within the meaning of that Act), or
%
%($b$) a deputy appointed for him, or any other person authorised, by the Court of Protection,
%\end{enumerate}
%with power in that respect.

\amendment{
S. 50(5), (7)(d) came into force 27.6.92; s. 50 came fully into force 5.4.93.

Words inserted in s. 50(5)(c) (1.6.99) by the Social Security Act 1998 Sch. 7 para. 45.

Words substituted for s. 50(8)(c)(i), (ii) (2.4.01 for certain purposes, 1.4.02 for all purposes in Scotland) by the Adults with Incapacity (Scotland) Act 2000 Sch. 5 para. 22.

Words substituted for s. 50(8)(c)(i), (ii) (30.6.05 for England and Wales) by the Adults with Incapacity (Scotland) Act 2000 (Consequential Modifications) (England, Wales and Northern Ireland) Order 2005 art. 2.

S. 50(8)(b), (d) repealed, words substituted in s. 50(8) and s. 50(9) inserted (prosp) by the Mental Capacity Act 2005 Sch. 6 para. 36.
}

\subsection{51. Supplementary powers to make regulations}

(1) The Secretary of State may by regulations make such incidental, supplemental and transitional provision as he considers appropriate in connection with any provision made by or under this Act.

(2) The regulations may, in particular, make provision—
\begin{enumerate}\item[]
($a$) as to the procedure to be followed with respect to—
\begin{enumerate}\item[]
(i) the making of applications for 
%maintenance assessments% 
\emph{maintenance calculations}%  % Words substituted by 2000 c 19 s 1(2)(a)
;

(ii) [\emph{1993 scheme version}] the making, cancellation or refusal to make maintenance assessments;

% S 51(2)(a)(ii) substituted by 2000 c 19 Sch 3 para 11(19)(a)
(ii) [\emph{2003 scheme version}] the making of decisions under section~11;

%(iii) reviews under sections 16 to 19;

% S 51(2)(a)(iii) substituted (16.11.98) by 1998 c 14 Sch 7 para 46(a) and by 2000 c 19 Sch 3 para 11(19)(a)
(iii) the making of decisions under section 16 or 17;
\end{enumerate}

($b$) [\emph{1993 scheme version}] extending the categories of case to which 
%section 18 or 19 
Schedule 4C  % Words substituted (1.6.99) by 1998 c 14 Sch 7 para 46(b)
applies;

% S 51(2)(b) substituted by 2000 c 19 Sch 3 para 11(19)(b)
($b$) [\emph{2003 scheme version}] extending the categories of case to which section~16, 17 or~20 applies;

($c$) as to the date on which an application for a 
%maintenance assessment 
\emph{maintenance calculation}  % Words substituted by 2000 c 19 s 1(2)(a)
is to be treated as having been made;

($d$) for attributing payments made under 
%maintenance assessments 
\emph{maintenance calculations}  % Words substituted by 2000 c 19 s 1(2)(a)
to the payment of arrears;

($e$) for the adjustment, for the purpose of taking account of the retrospective effect of a 
%maintenance assessment% 
\emph{maintenance calculation}%  % Words substituted by 2000 c 19 s 1(2)(a)
, of amounts payable under the 
%assessment% 
\emph{calculation}%  % Words substituted by 2000 c 19 s 1(2)(b)
;

($f$) for the adjustment, for the purpose of taking account of over-payments or under-payments of child support maintenance, of amounts payable under a 
%maintenance assessment% 
\emph{maintenance calculation}%  % Words substituted by 2000 c 19 s 1(2)(a)
;

($g$) as to the evidence which is to be required in connection with such matters as may be prescribed;

($h$) as to the circumstances in which any official record or certificate is to be conclusive (or in Scotland, sufficient) evidence;

($i$) with respect to the giving of notices or other documents;

($j$) for the rounding up or down of any amounts calculated, estimated or otherwise arrived at in applying any provision made by or under this Act.
\end{enumerate}

(3) No power to make regulations conferred by any other provision of this Act shall be taken to limit the powers given to the Secretary of State by this section.

\amendment{
S. 51 came into force 27.6.92.

S. 51(2)(a)(iii) substituted (16.11.98) by the Social Security Act 1998 Sch. 7 para. 46(a), subject to transitional provisions in the Social Security Act 1998 (Commencement No. 2) Order 1998 art. 3(5).

Words substituted in s. 51(2)(b) (1.6.99) by the Social Security Act 1998 Sch. 7 para. 46(b).

S. 51(2)(a)(ii), (iii), (b) substituted (3.3.03 for 2003 scheme cases) by the Child Support, Pensions and Social Security Act 2000 Sch. 3 para. 11(19).
}

\subsection{52. Regulations and orders}

(1) Any power conferred on 
%the Lord Chancellor,  % Words repealed by 2005 c 4 Sch 18 Pt II
the Lord Advocate or the Secretary of State by this Act to make regulations or orders (other than a deduction from earnings order) shall be exercisable by statutory instrument.

% S 52(2), (2A) substituted for s 52(2) by 2000 c 19 s 25
(2) No statutory instrument containing (whether alone or with other provisions) regulations made under—
\begin{enumerate}\item[]
($a$) section~6(1), 12(4)  (so far as the regulations make provision for~the default rate of child support maintenance mentioned in section~12(5)($b$)), 28C(2)($b$), 28F(2)($b$), 30(5A), 41(2), 41A, 41B(6), 43(1), 44(2A)($d$), 46 or~47;

($b$) paragraph~3(2)  or~10A(1)  of Part I of Schedule 1; or

($c$) Schedule 4B,
\end{enumerate}
or~an order made under section~45(1)  or~(6), shall be made unless a draft of the instrument has been laid before Parliament and approved by a resolution of each House of Parliament.

(2A) No statutory instrument containing (whether alone or with other provisions) the first set of regulations made under paragraph~10(1)  of Part I of Schedule 1 as substituted by section~1(3)  of the Child Support, Pensions and Social Security Act 2000 shall be made unless a draft of the instrument has been laid before Parliament and approved by a resolution of each House of Parliament.

(3) Any other statutory instrument made under this Act (except an order made under section 58(2)) shall be subject to annulment in pursuance of a resolution of either House of Parliament.

(4) Any power of a kind mentioned in subsection (1)  may be exercised—
\begin{enumerate}\item[]
($a$) in relation to all cases to which it extends, in relation to those cases but subject to specified exceptions or in relation to any specified cases or classes of case;

($b$) so as to make, as respects the cases in relation to which it is exercised—
\begin{enumerate}\item[]
(i) the full provision to which it extends or any lesser provision (whether by way of exception or otherwise);

(ii) the same provision for all cases, different provision for different cases or classes of case or different provision as respects the same case or class of case but for different purposes of this Act;

(iii) provision which is either unconditional or is subject to any specified condition;
\end{enumerate}

($c$) so to provide for a person to exercise a discretion in dealing with any matter.
\end{enumerate}

\amendment{
S. 52 came into force 27.6.92.

Words inserted in s.~52(2) (4.9.95) by the Child Support Act 1995 Sch.~3 para.~15.

S. 52(2), (2A) substituted for s. 52(2) (10.11.00 for regulation-making purposes, 3.3.03 for 2003 scheme cases, 16.5.14 for all other purposes) by the Child Support, Pensions and Social Security Act 2000 s. 25.

Words repealed in s. 52(1) (3.4.06) by the Constitutional Reform Act 2005 Sch. 18 Pt. II.
}

\subsection{53. Financial provisions}

Any expenses of the Lord Chancellor or the Secretary of State under this Act shall be payable out of money provided by Parliament.

\amendment{
S. 53 came into force 5.4.93.
}

\subsection{54. Interpretation}

In this Act—
\begin{enumerate}\item[]
    “%
%absent parent 
\emph{non-resident parent}%  % Words substituted by 2000 c 19 Sch 3 para 11(2)
”, has the meaning given in section 3(2);

% Definition of ``adjudication officer'' repealed (29.11.99) by 1998 c 14 Sch 7 para 47(b)
%    “adjudication officer” has the same meaning as in the benefit Acts;

% Definition of ``appeal tribunal'' inserted (1.6.99) by 1998 c 14 Sch 7 para 47(a)
“appeal tribunal” means an appeal tribunal constituted under Chapter~I of Part I of the Social Security Act 1998;

% Definition of ``application for a departure direction'' inserted (4.9.95) by 1995 c 34 Sch 3 para 16
[\emph{1993 scheme version}] “application for a departure direction” means an application under section 28A;

[\emph{2003 scheme version}] “application for a 
%departure direction%
variation%  % Words substituted by 2000 c 19 Sch 3 para 11(20)(a)
” means an application under section 28A
or~28G%  % Words inserted by 2000 c 19 Sch 3 para 11(20)(a)
;

% Definition of ``assessable income'' omitted by 2000 c 19 Sch 3 para 11(20)(e)
[\emph{1993 scheme only}]     “assessable income” has the meaning given in paragraph 5 of Schedule~1;

    “benefit Acts” means the 
%Social Security Acts 1975 to 1991
Social Security Contributions and Benefits Act 1992 and the Social Security Administration Act 1992%  % Words substituted (1.7.92) by 1992 c 6 Sch 2 para 114(a)
;

% Definition of ``Chief Adjudication Officer'' repealed (29.11.99) by 1998 c 14 Sch 7 para 47(b)
%    “Chief Adjudication Officer” has the same meaning as in the benefit Acts;

% Definition of ``Chief Child Support Officer'' repealed (1.6.99) by 1998 c 14 Sch 7 para 47(b)
%    “Chief Child Support Officer” has the meaning given in section 13;

    “child benefit” has the same meaning as in the Child Benefit Act 1975;

% Definition of ``child support appeal tribunal'' repealed (1.6.99) by 1998 c 14 Sch 7 para 47(b)
%    “child support appeal tribunal” means a tribunal appointed under section~21;

    “child support maintenance” has the meaning given in section 3(6);

% Definition of ``child support officer'' repealed (1.6.99) by 1998 c 14 Sch 7 para 47(b)
%    “child support officer” has the meaning given in section 13;

% Definition of ``current assessment'' inserted (4.9.95) by 1995 c 34 Sch 3 para 16, omitted by 2000 c 19 Sch 3 para 11(20)(e)
[\emph{1993 scheme only}]     “current assessment”, in relation to an application for a departure direction, means (subject to any regulations made under paragraph 10 of Schedule 4A) the maintenance assessment with respect to which the application is made;

    “deduction from earnings order” has the meaning given in section 31(2);

% Definition of ``default maintenance decision'' inserted by 2000 c 19 Sch 3 para 11(20)(b)
[\emph{2003 scheme only}] “default maintenance decision” has the meaning given in section~12;

% Definition of ``departure direction'' inserted (4.9.95) by 1995 c 34 Sch 3 para 16, omitted by 2000 c 19 Sch 3 para 11(20)(e)
[\emph{1993 scheme only}]     “departure direction” has the meaning given in section 28A;

    “disability living allowance” has the same meaning as in the 
%Social Security Act 1975
benefit Acts%  % Words substituted (1.7.92) by 1992 c 6 Sch 2 para 114(b)
;

% Definition of ``working families' tax credit'' repealed by 2002 c 21 Sch 6
%    “%
%%family credit 
%working families' tax credit%  % Words substituted (5.10.99) by 1999 c 10 Sch 1 para 1(a), 6(f)(ii)
%” has the same meaning as in the benefit Acts;

    “general qualification” shall be construed in accordance with section 71 of the Courts and Legal Services Act 1990 (qualification for judicial appointments);

    “income support” has the same meaning as in the benefit Acts;

% Definition of ``income-based jobseeker's allowance'' inserted (7.10.96) by 1995 c 18 Sch 2 para 20(6)
“income-based jobseeker’s allowance” has the same meaning as in the Jobseekers Act 1995;

% Definition of ``income-related employment and support allowance'' inserted by 2007 c 5 Sch 3 para 7(7)
%“income-related employment and support allowance” means an income-related allowance under Part I of the Welfare Reform Act 2007 (employment and support allowance);

    [\emph{1993 scheme version}] “interim maintenance assessment” has the meaning given in section 12;

    [\emph{2003 scheme version}] “interim maintenance 
%assessment%
decision%  % Word substituted by 2000 c 19 Sch 3 para 11(20)(c)
” has the meaning given in section 12;

    “liability order” has the meaning given in section 33(2);

    “maintenance agreement” has the meaning given in section 9(1);

    [\emph{1993 scheme version}] “maintenance assessment” means an assessment of maintenance made under this Act and, except in prescribed circumstances, includes an interim maintenance assessment;

% Definition of ``maintenance assessment'' substituted by 2000 c 19 Sch 3 para 11(20)(d)
[\emph{2003 scheme version}] “maintenance calculation” means a calculation of maintenance made under this Act and, except in prescribed circumstances, includes a default maintenance decision and an interim maintenance decision;

    “maintenance order” has the meaning given in section 8(11);

% Definition of ``maintenance requirement'' omitted by 2000 c 19 Sch 3 para 11(20)(e)
[\emph{1993 scheme only}]         “maintenance requirement” means the amount calculated in accordance with paragraph 1 of Schedule 1;

    “parent”, in relation to any child, means any person who is in law the mother or father of the child;

% Definition of ``parent with care'' inserted (4.9.95) by 1995 c 34 Sch 3 para 16
“parent with care” means a person who is, in relation to a child, both a parent and a person with care;

%    “parental responsibility” has the same meaning as in the Children Act 1989;

% Definition of ``parental responsibility'' substituted (1.11.96) by 1995 c 36 Sch 4 para 52(4)(a)
“parental responsibility”, in the application of this Act—
\begin{enumerate}\item[]
($a$) to England and Wales, has the same meaning as in the Children Act 1989; and

($b$) to Scotland, shall be construed as a reference to “parental responsibilities” within the meaning given by section 1(3) of the Children (Scotland) Act 1995;
\end{enumerate}

% Definition of ``parental rights'' repealed (1.11.96) by 1995 c 36 Sch 4 para 52(4)(b)
%    “parental rights” has the same meaning as in the Law Reform (Parent and Child) (Scotland) Act 1986;

    “person with care” has the meaning given in section 3(3);

    “prescribed” means prescribed by regulations made by the Secretary of State;

    “qualifying child” has the meaning given in section 3(1);

% Definition of ``voluntary payment'' inserted by 2000 c 19 Sch 3 para 11(20)(f)
[\emph{2003 scheme only}] “voluntary payment” has the meaning given in section~28J.
\end{enumerate}

\amendment{
S. 54 came into force 27.6.92.

Words substituted in definitions of ``benefit Acts'' and ``disability living allowance'' (1.7.92) by the Social Security (Consequential Provisions) Act 1992 Sch.~2 para.~114.

Definitions of ``application for a departure direction'', ``current assessment'', ``departure direction'', ``parent with care'' inserted in s.~54 (4.9.95) by the Child Support Act 1995 Sch.~3 para.~16.

Definition of ``income-based jobseeker's allowance'' inserted in s. 54 (7.10.96) by the Jobseekers Act 1995 Sch.~2 para.~20(6).

Definition of ``parental responsibility'' in s. 54 substituted and definition of ``parental rights'' in s. 54 repealed (1.11.96) by the Children (Scotland) Act 1995 Sch. 4 para. 52(4).

Definition of ``appeal tribunal'' inserted in s. 54 and definitions of ``Chief Child Support Officer'', ``child support appeal tribunal'' and ``child support officer'' in s. 54 repealed (1.6.99) by the Social Security Act 1998 Sch. 7 para. 47.

Definitions of ``adjudication officer'', ``Chief Adjudication Officer'' in s. 54 repealed (29.11.99) by the Social Security Act 1998 Sch. 7 para. 47(b).

Words substituted in s. 47(3)(b) (5.10.99) by the Tax Credits Act 1999 Sch. 1. paras. 1(a), 6(f)(ii).

Words substituted in definition of ``application for a departure direction'', definition of ``default maintenance decision'' inserted, word substituted in definition of ``interim maintenance assessment'', definition of ``maintenance calculation'' substituted for definition of ``maintenance assessment'', definitions of ``assessable income'', ``current assessment'', ``departure direction'' and ``maintenance requirement'' omitted and definition of ``voluntary payment'' inserted in s. 54 (3.3.03 for 2003 scheme cases) by the Child Support, Pensions and Social Security Act 2000 Sch. 3 para. 11(20).

Definition of ``working families' tax credit'' in s. 54 repealed (6.4.03) by the Tax Credits Act 2002 Sch. 6.

Definition of ``income-related employment and support allowance'' inserted in s. 54 (prosp) by the Welfare Reform Act 2007 Sch. 3 para. 7(7).

Definition of ``general qualification'' in s. 54 repealed (prosp) by the Tribunals, Courts and Enforcement Act 2007 Sch. 23 Pt. II.
}

\subsection{55. Meaning of “child”}

(1) For the purposes of this Act a person is a child if—
\begin{enumerate}\item[]
($a$) he is under the age of 16;

($b$) he is under the age of 19 and receiving full-time education (which is not advanced education)—
\begin{enumerate}\item[]
(i) by attendance at a recognised educational establishment; or

(ii) elsewhere, if the education is recognised by the Secretary of State; or
\end{enumerate}

($c$) he does not fall within paragraph ($a$)  or ($b$)  but—
\begin{enumerate}\item[]
(i) he is under the age of 18, and

(ii) prescribed conditions are satisfied with respect to him.
\end{enumerate}
\end{enumerate}

(2) A person is not a child for the purposes of this Act if he—
\begin{enumerate}\item[]
($a$) is or has been married
or a civil partner%  % Words inserted by 2004 c 33 Sch 24 para 3(a)
;

($b$) has celebrated a marriage%
, or been a party to a civil partnership,  % Words inserted by 2004 c 33 Sch 24 para 3(b)
which is void; or

($c$) has celebrated a marriage in respect of which a decree of nullity has been granted
or has been a party to a civil partnership in respect of which a nullity order has been made%  % Words inserted by 2004 c 33 Sch 24 para 3(c)
.
\end{enumerate}

(3) In this section—
\begin{enumerate}\item[]
    “advanced education” means education of a prescribed description; and

    “recognised educational establishment” means an establishment recognised by the Secretary of State for the purposes of this section as being, or as comparable to, a university, college or school. 
\end{enumerate}

(4) Where a person has reached the age of 16, the Secretary of State may recognise education provided for him otherwise than at a recognised educational establishment only if the Secretary of State is satisfied that education was being so provided for him immediately before he reached the age of 16.

(5) The Secretary of State may provide that in prescribed circumstances education is or is not to be treated for the purposes of this section as being full-time.

(6) In determining whether a person falls within subsection (1)($b$), no account shall be taken of such interruptions in his education as may be prescribed.

(7) The Secretary of State may by regulations provide that a person who ceases to fall within subsection (1)  shall be treated as continuing to fall within that subsection for a prescribed period.

(8) No person shall be treated as continuing to fall within subsection (1)  by virtue of regulations made under subsection (7)  after the end of the week in which he reaches the age of 19.

\amendment{
S. 55 came into force 27.6.92.

Words inserted in s. 55(2)(a), (b), (c) (5.12.05) by the Civil Partnership Act 2004 Sch. 24 para. 3.
}

\subsection{56. Corresponding provision for and co-ordination with Northern Ireland}

(1) An Order in Council made under paragraph 1(1)($b$)  of Schedule 1 to the Northern Ireland Act 1974 which contains a statement that it is made only for purposes corresponding to those of the provisions of this Act, other than provisions which relate to the appointment of Child Support Commissioners for Northern Ireland—
\begin{enumerate}\item[]
($a$) shall not be subject to sub-paragraphs (4)  and (5)  of paragraph 1 of that Schedule (affirmative resolution of both Houses of Parliament); but

($b$) shall be subject to annulment in pursuance of a resolution of either House of Parliament.
\end{enumerate}

% S 56(2)--(4) repealed (2.12.99) by 1998 c 47 s 87(8)(c)
%(2) The Secretary of State may make arrangements with the Department of Health and Social Services for Northern Ireland with a view to securing, to the extent allowed for in the arrangements, that—
%\begin{enumerate}\item[]
%($a$) the provision made by or under this Act (“the provision made for Great Britain”); and
%
%($b$) the provision made by or under any corresponding enactment having effect with respect to Northern Ireland (“the provision made for Northern Ireland”),
%\end{enumerate}
%provide for a single system within the United Kingdom.
%
%(3) The Secretary of State may make regulations for giving effect to any such arrangements.
%
%(4) The regulations may, in particular—
%\begin{enumerate}\item[]
%($a$) adapt legislation (including subordinate legislation) for the time being in force in Great Britain so as to secure its reciprocal operation with the provision made for Northern Ireland; and
%
%($b$) make provision to secure that acts, omissions and events which have any effect for the purposes of the provision made for Northern Ireland have a corresponding effect for the purposes of the provision made for Great Britain.
%\end{enumerate}

\amendment{
S. 56(1) came into force 25.7.91; s. 56(2)--(4) came into force 27.6.92.

S. 56(2)--(4) repealed (2.12.99) by the Northern Ireland Act 1998 s. 87(8)(c).
}

\subsection{57. Application to Crown}

(1) The power of the Secretary of State to make regulations under section 14 requiring prescribed persons to furnish information may be exercised so as to require information to be furnished by persons employed in the service of the Crown or otherwise in the discharge of Crown functions.

(2) In such circumstances, and subject to such conditions, as may be prescribed, an inspector appointed under section 15 may enter any Crown premises for the purpose of exercising any powers conferred on him by that section.

(3) Where such an inspector duly enters any Crown premises for those purposes, section 15 shall apply in relation to persons employed in the service of the Crown or otherwise in the discharge of Crown functions as it applies in relation to other persons.

(4) Where a liable person is in the employment of the Crown, a deduction from earnings order may be made under section 31 in relation to that person; but in such a case subsection (8)  of section 32 shall apply only in relation to the failure of that person to comply with any requirement imposed on him by regulations made under section 32.

\amendment{
S. 57 came into force 27.6.92.
}

\subsection{58. Short title, commencement and extent, etc}

(1) This Act may be cited as the Child Support Act 1991.

(2) Section 56(1)  and subsections (1)  to (11)  and (14)  of this section shall come into force on the passing of this Act but otherwise this Act shall come into force on such date as may be appointed by order made by the Lord Chancellor, the Secretary of State or Lord Advocate, or by any of them acting jointly.

(3) Different dates may be appointed for different provisions of this Act and for different purposes (including, in particular, for different cases or categories of case).

(4) An order under subsection (2)  may make such supplemental, incidental or transitional provision as appears to the person making the order to be necessary or expedient in connection with the provisions brought into force by the order, including such adaptations or modifications of—
\begin{enumerate}\item[]
($a$) the provisions so brought into force;

($b$) any provisions of this Act then in force; or

($c$) any provision of any other enactment,
\end{enumerate}
as appear to him to be necessary or expedient.

(5) Different provision may be made by virtue of subsection (4)  with respect to different periods.

(6) Any provision made by virtue of subsection (4)  may, in particular, include provision for—
\begin{enumerate}\item[]
($a$) the enforcement of a 
%maintenance assessment 
\emph{maintenance calculation}  % Words substituted by 2000 c 19 s 1(2)(a)
(including the collection of sums payable under the 
%assessment% 
\emph{calculation}%  % Words substituted by 2000 c 19 s 1(2)(b)
) as if the 
%assessment 
\emph{calculation}  % Words substituted by 2000 c 19 s 1(2)(b)
were a court order of a prescribed kind;

($b$) the registration of 
%maintenance assessments 
\emph{maintenance calculations}  % Words substituted by 2000 c 19 s 1(2)(a)
with the appropriate court in connection with any provision of a kind mentioned in paragraph~($a$);

($c$) the variation, on application made to a court, of the provisions of a 
%maintenance assessment 
\emph{maintenance calculation}  % Words substituted by 2000 c 19 s 1(2)(a)
relating to the method of making payments fixed by the 
%assessment 
\emph{calculation}  % Words substituted by 2000 c 19 s 1(2)(b)
or the intervals at which such payments are to be made;

($d$) a 
%maintenance assessment% 
\emph{maintenance calculation}%  % Words substituted by 2000 c 19 s 1(2)(a)
, or an order of a prescribed kind relating to one or more children, to be deemed, in prescribed circumstances, to have been validly made for all purposes or for such purposes as may be prescribed.
\end{enumerate}

In paragraph ($c$)  “court” includes a single justice.

(7) The Lord Chancellor, the Secretary of State or the Lord Advocate may by order make such amendments or repeals in, or such modifications of, such enactments as may be specified in the order, as appear to him to be necessary or expedient in consequence of any provision made by or under this Act (including any provision made by virtue of subsection (4)).

(8) This Act shall, in its application to the Isles of Scilly, have effect subject to such exceptions, adaptations and modifications as the Secretary of State may by order prescribe.

(9) [\emph{1993 scheme version}] Sections 27, 35 and 48 and paragraph 7 of Schedule 5 do not extend to Scotland.

(9) [\emph{2003 scheme version}] Sections 27, 35%
, 40  % Words inserted by 2000 c 19 Sch 3 para 11(21)(a)
 and 48 and paragraph 7 of Schedule 5 do not extend to Scotland.

(10) [\emph{1993 scheme version}] Sections 7, 28 and 49 extend only to Scotland.

(10) [\emph{2003 scheme version}] Sections 7, 28%
, 40A  % Words inserted by 2000 c 19 Sch 3 para 11(21)(b)
 and 49 extend only to Scotland.

(11) With the exception of sections 23 and 56(1), subsections (1)  to (3)  of this section and Schedules 2 and 4, and (in so far as it amends any enactment extending to Northern Ireland) Schedule 5, this Act does not extend to Northern Ireland.

%(12) Until Schedule 1 to the Disability Living Allowance and Disability Working Allowance Act 1991 comes into force, paragraph 1(1)  of Schedule~3 shall have effect with the omission of the words “and disability appeal tribunals” and the insertion, after “social security appeal tribunals”, of the word “and”.

(13) The consequential amendments set out in Schedule 5 shall have effect.

(14) In Schedule 1 to the Children Act 1989 (financial provision for children), paragraph 2(6)($b$)  (which is spent) is hereby repealed.

\amendment{
S. 56(1)--(11), (14) came into force 25.7.91.  S. 56(13) came into force 1.9.92.  S. 56(12) is not yet in force.

Words inserted in s. 58(9), (10) (3.3.03 for new-rules cases) by the Child Support, Pensions and Social Security Act 2000 Sch. 3 para. 11(21).
}

\bigskip

\small

\part[Schedule 1 --- Maintenance 
%assessments 
\emph{calculations}  % Words substituted by 2000 c 19 s 1(2)(a)
]{Schedule 1\\*Maintenance 
%maintenance assessment 
\emph{calculations}  % Words substituted by 2000 c 19 s 1(2)(a)
}

\section[Part I --- Calculation of child support maintenance --- \emph{1993 scheme version}]{Part I\\*Calculation of child support maintenance\\*\emph{1993 scheme version}}

\renewcommand\parthead{ --- Schedule 1 Part I}

\subsection*{\itshape The maintenance requirement}

1.---(1) In this Schedule “the maintenance requirement” means the amount, calculated in accordance with the formula set out in sub-paragraph (2), which is to be taken as the minimum amount necessary for the maintenance of the qualifying child or, where there is more than one qualifying child, all of them.

(2) The formula is—
\[
MR = AG - CB
\]
where—
\begin{enumerate}\item[]
    $MR$ is the amount of the maintenance requirement;

    $AG$ is the aggregate of the amounts to be taken into account under sub-paragraph (3); and

    $CB$ is the amount payable by way of child benefit (or which would be so payable if the person with care of the qualifying child were an individual) or, where there is more than one qualifying child, the aggregate of the amounts so payable with respect to each of them. 
\end{enumerate}

(3) The amounts to be taken into account for the purpose of calculating $AG$ are—
\begin{enumerate}\item[]
($a$) such amount or amounts (if any), with respect to each qualifying child, as may be prescribed;

($b$) such amount or amounts (if any), with respect to the person with care of the qualifying child or qualifying children, as may be prescribed; and

($c$) such further amount or amounts (if any) as may be prescribed.
\end{enumerate}

(4) For the purposes of calculating $CB$ it shall be assumed that child benefit is payable with respect to any qualifying child at the basic rate.

(5) In sub-paragraph (4)  “basic rate” has the meaning for the time being prescribed.

\amendment{
Para. 1(3), (5) came into force 27.6.92; para. 1 came fully into force 5.4.93.
}

\subsection*{\itshape The general rule}

2.---(1) In order to determine the amount of any maintenance assessment, first calculate—
\[
(A + C) \times P
\]
where—
\begin{enumerate}\item[]
    $A$ is the absent parent’s assessable income;

    $C$ is the assessable income of the other parent, where that parent is the person with care, and otherwise has such value (if any) as may be prescribed; and

    $P$ is such number greater than zero but less than 1 as may be prescribed. 
\end{enumerate}

(2) Where the result of the calculation made under sub-paragraph (1)  is an amount which is equal to, or less than, the amount of the maintenance requirement for the qualifying child or qualifying children, the amount of maintenance payable by the absent parent for that child or those children shall be an amount equal to—
\[
A \times P
\]
where $A$ and $P$ have the same values as in the calculation made under sub-paragraph~(1).

(3) Where the result of the calculation made under sub-paragraph (1)  is an amount which exceeds the amount of the maintenance requirement for the qualifying child or qualifying children, the amount of maintenance payable by the absent parent for that child or those children shall consist of—
\begin{enumerate}\item[]
($a$) a basic element calculated in accordance with the provisions of paragraph~3; and

($b$) an additional element calculated in accordance with the provisions of paragraph 4.
\end{enumerate}

\amendment{
Para. 2(1) came into force 27.6.92; para. 2 came fully into force 5.4.93.
}

\subsection*{\itshape The basic element}

3.---(1) The basic element shall be calculated by applying the formula—
\[
BE = A \times G \times P
\]
where—
\begin{enumerate}\item[]
    $BE$ is the amount of the basic element;

    $A$ and $P$ have the same values as in the calculation made under paragraph~2(1); and

    $G$ has the value determined under sub-paragraph (2). 
\end{enumerate}

(2) The value of $G$ shall be determined by applying the formula—
\[
G = \frac{MR}{(A + C) \times P}
\]
where—
\begin{enumerate}\item[]
    $MR$ is the amount of the maintenance requirement for the qualifying child or qualifying children; and

    $A$, $C$ and $P$ have the same values as in the calculation made under paragraph~2(1). 
\end{enumerate}

\amendment{
Para. 3 came into force 5.4.93.
}

\subsection*{\itshape The additional element}

4.---(1) Subject to sub-paragraph (2), the additional element shall be calculated by applying the formula—
\[
AE = (1 - G) \times A \times R
\]
where—
\begin{enumerate}\item[]
    $AE$ is the amount of the additional element;

    $A$ has the same value as in the calculation made under paragraph 2(1);

    $G$ has the value determined under paragraph 3(2); and

    $R$ is such number greater than zero but less than 1 as may be prescribed. 
\end{enumerate}

(2) Where applying the alternative formula set out in sub-paragraph (3)  would result in a lower amount for the additional element, that formula shall be applied in place of the formula set out in sub-paragraph (1).

(3) The alternative formula is—
\[
AE = \frac{Z \times Q \times A}{A + C}
\]
where—
\begin{enumerate}\item[]
    $A$ and $C$ have the same values as in the calculation made under paragraph~2(1);

    $Z$ is such number as may be prescribed; and

    $Q$ is the aggregate of— 
\begin{enumerate}\item[]
($a$) any amount taken into account by virtue of paragraph 1(3)($a$)  in calculating the maintenance requirement; and

($b$) any amount which is both taken into account by virtue of paragraph~1(3)($c$)  in making that calculation and is an amount prescribed for the purposes of this paragraph.
\end{enumerate}
\end{enumerate}

\amendment{
Para. 4(1), (3) came into force 27.6.92; para. 4 came fully into force 5.4.93.
}

\subsection*{\itshape Assessable income}

5.---(1) The assessable income of an absent parent shall be calculated by applying the formula—
\[
A = N - E
\]
where—
\begin{enumerate}\item[]
    $A$ is the amount of that parent’s assessable income;

    $N$ is the amount of that parent’s net income, calculated or estimated in accordance with regulations made by the Secretary of State for the purposes of this sub-paragraph; and

    $E$ is the amount of that parent’s exempt income, calculated or estimated in accordance with regulations made by the Secretary of State for those purposes. 
\end{enumerate}

(2) The assessable income of a parent who is a person with care of the qualifying child or children shall be calculated by applying the formula—
\[
C = M - F
\]
where—
\begin{enumerate}\item[]
    $C$ is the amount of that parent’s assessable income;

    $M$ is the amount of that parent’s net income, calculated or estimated in accordance with regulations made by the Secretary of State for the purposes of this sub-paragraph; and

    $F$ is the amount of that parent’s exempt income, calculated or estimated in accordance with regulations made by the Secretary of State for those purposes. 
\end{enumerate}

(3) Where the preceding provisions of this paragraph would otherwise result in a person’s assessable income being taken to be a negative amount his assessable income shall be taken to be nil.

(4) Where income support%
, an income-based jobseeker’s allowance  % Words inserted (7.10.96) by 1995 c 18 Sch 2 para 20(7)
% , an income-related employment and support allowance  % Words inserted by 2007 c 5 Sch 3 para 7(8)
 or any other benefit of a prescribed kind is paid to or in respect of a parent who is an absent parent or a person with care that parent shall, for the purposes of this Schedule, be taken to have no assessable income.

\amendment{
Para. 5(1), (2), (4) came into force 27.6.92; para. 5 came fully into force 5.4.93.

Words inserted in para. 5(4) (7.10.96) by the Jobseekers Act 1995 Sch.~2 para.~20(7).

Words inserted in para. 5(4) (prosp) by the Welfare Reform Act 2007 Sch. 3 para. 7(8).
}

\subsection*{\itshape Protected income}

6.---(1) This paragraph applies where—
\begin{enumerate}\item[]
($a$) one or more maintenance assessments have been made with respect to an absent parent; and

($b$) payment by him of the amount, or the aggregate of the amounts, so assessed would otherwise reduce his disposable income below his protected income level.
\end{enumerate}

(2) The amount of the assessment, or (as the case may be) of each assessment, shall be adjusted in accordance with such provisions as may be prescribed with a view to securing so far as is reasonably practicable that payment by the absent parent of the amount, or (as the case may be) aggregate of the amounts, so assessed will not reduce his disposable income below his protected income level.

(3) Regulations made under sub-paragraph (2)  shall secure that, where the prescribed minimum amount fixed by regulations made under paragraph 7 applies, no maintenance assessment is adjusted so as to provide for the amount payable by an absent parent in accordance with that assessment to be less than that amount.

(4) The amount which is to be taken for the purposes of this paragraph as an absent parent’s disposable income shall be calculated, or estimated, in accordance with regulations made by the Secretary of State.

(5) Regulations made under sub-paragraph (4)  may, in particular, provide that, in such circumstances and to such extent as may be prescribed—
\begin{enumerate}\item[]
($a$) income of any child who is living in the same household with the absent parent; and

%($b$) where the absent parent is living together in the same household with another adult of the opposite sex (regardless of whether or not they are married), income of that other adult,

% Para 6(5)(b) substituted by 2004 c 33 Sch 24 para 4
($b$) where the absent parent—
\begin{enumerate}\item[]
(i) is living together in the same household with another adult of the opposite sex (regardless of whether or not they are married),

(ii) is living together in the same household with another adult of the same sex who is his civil partner, or

(iii) is living together in the same household with another adult of the same sex as if they were civil partners,
\end{enumerate}
income of that other adult,
\end{enumerate}
is to be treated as the absent parent’s income for the purposes of calculating his disposable income.

% Para 6(5A) inserted by 2004 c 33 Sch 24 para 4
(5A) For the purposes of this paragraph, two adults of the same sex are to be regarded as living together in the same household as if they were civil partners if, but only if, they would be regarded as living together as husband and wife were they instead two adults of the opposite sex.

(6) In this paragraph the “protected income level” of a particular absent parent means an amount of income calculated, by reference to the circumstances of that parent, in accordance with regulations made by the Secretary of State.

\amendment{
Para. 6(2)--(6) came into force 27.6.92; para. 6 came fully into force 5.4.93.

Para. 6(5)(b) substituted and para. 6(5A) inserted (5.12.05) by the Civil Partnership Act 2004 Sch. 24 paras. 4, 5.


\medskip

Paras. 7--9 came into force 27.6.92.
}

\subsection*{\itshape The minimum amount of child support maintenance}

7.---(1) The Secretary of State may prescribe a minimum amount for the purposes of this paragraph.

(2) Where the amount of child support maintenance which would be fixed by a maintenance assessment but for this paragraph is nil, or less than the prescribed minimum amount, the amount to be fixed by the assessment shall be the prescribed minimum amount.

(3) In any case to which section 43 applies, and in such other cases (if any) as may be prescribed, sub-paragraph (2)  shall not apply.

\subsection*{\itshape Housing costs}

8. Where regulations under this Schedule require 
%a child support officer 
the Secretary of State  % Words substituted (1.6.99) by 1998 c 14 Sch 7 para 48(1)
to take account of the housing costs of any person in calculating, or estimating, his assessable income or disposable income, those regulations may make provision—
\begin{enumerate}\item[]
($a$) as to the costs which are to be treated as housing costs for the purpose of the regulations;

($b$) for the apportionment of housing costs; and

($c$) for the amount of housing costs to be taken into account for prescribed purposes not to exceed such amount (if any) as may be prescribed by, or determined in accordance with, the regulations.
\end{enumerate}

\amendment{
Words substituted in para. 8 (1.6.88) by the Social Security Act 1998 Sch. 7 para. 48(1).
}

\subsection*{\itshape Regulations about income and capital}

9. The Secretary of State may by regulations provide that, in such circumstances and to such extent as may be prescribed—
\begin{enumerate}\item[]
($a$) income of a child shall be treated as income of a parent of his;

($b$) where 
%the child support officer concerned 
the Secretary of State  % Words substituted (1.6.99) by 1998 c 14 Sch 7 para 48(2)
is satisfied that a person has intentionally deprived himself of a source of income with a view to reducing the amount of his assessable income, his net income shall be taken to include income from that source of an amount estimated by 
%the child support officer%
the Secretary of State%  % Words substituted (1.6.99) by 1998 c 14 Sch 7 para 48(2)
;

($c$) a person is to be treated as possessing capital or income which he does not possess;

($d$) capital or income which a person does possess is to be disregarded;

($e$) income is to be treated as capital;

($f$) capital is to be treated as income.
\end{enumerate}

\amendment{
Words substituted in para. 9 (1.6.99) by the Social Security Act 1998 Sch. 7 para. 48(2).
}

\subsection*{\itshape References to qualifying children}

10. References in this Part of this Schedule to “qualifying children” are to those qualifying children with respect to whom the maintenance assessment falls to be made.

\amendment{
Para. 10 came into force 5.4.93.

\medskip

Sch. 1 Pt. I substituted (10.11.00 for regulation-making purposes only, 3.3.03 for new-rules cases only) by the Child Support, Pensions and Social Security Act 2000 Sch. 1.
}

\section[Part I --- Calculation of weekly amount of child support maintenance --- \emph{2003 scheme version}]{Part I\\*Calculation of weekly amount of child support maintenance\\*\emph{2003 scheme version}}

\subsection*{\itshape General rule}

1.---(1) The weekly rate of child support maintenance is the basic rate unless a reduced rate, a flat rate or the nil rate applies.

(2) Unless the nil rate applies, the amount payable weekly to a person with care is—
\begin{enumerate}\item[]
($a$) the applicable rate, if paragraph~6 does not apply; or

($b$) if paragraph~6 does apply, that rate as apportioned between the persons with care in accordance with paragraph~6,
\end{enumerate}
as adjusted, in either case, by applying the rules about shared care in paragraph~7 or~8. 

\subsection*{\itshape Basic rate}

2.---(1) The basic rate is the following percentage of the non-resident parent’s net weekly income—
\begin{enumerate}\item[]
    15\% where he has one qualifying child;

    20\% where he has two qualifying children;

    25\% where he has three or more qualifying children. 
\end{enumerate}

(2) If the non-resident parent also has one or more relevant other children, the appropriate percentage referred to in sub-paragraph~(1)  is to be applied instead to his net weekly income less—
\begin{enumerate}\item[]
    15\% where he has one relevant other child;

    20\% where he has two relevant other children;

    25\% where he has three or more relevant other children. 
\end{enumerate}

\subsection*{\itshape Reduced rate}

3.---(1) A reduced rate is payable if—
\begin{enumerate}\item[]
($a$) neither a flat rate nor the nil rate applies; and

($b$) the non-resident parent’s net weekly income is less than £200 but more than £100. 
\end{enumerate}

(2) The reduced rate payable shall be prescribed in, or determined in accordance with, regulations.

(3) The regulations may not prescribe, or result in, a rate of less than £5. 

\subsection*{\itshape Flat rate}

4.---(1) Except in a case falling within sub-paragraph~(2), a flat rate of £5 is payable if the nil rate does not apply and—
\begin{enumerate}\item[]
($a$) the non-resident parent’s net weekly income is £100 or less; or

($b$) he receives any benefit, pension or allowance prescribed for the purposes of this paragraph of this sub-paragraph; or

($c$) he or his partner (if any) receives any benefit prescribed for the purposes of this paragraph of this sub-paragraph.
\end{enumerate}

(2) A flat rate of a prescribed amount is payable if the nil rate does not apply and—
\begin{enumerate}\item[]
($a$) the non-resident parent has a partner who is also a non-resident parent;

($b$) the partner is a person with respect to whom a maintenance calculation is in force; and

($c$) the non-resident parent or his partner receives any benefit prescribed under sub-paragraph~(1)($c$).
\end{enumerate}

(3) The benefits, pensions and allowances which may be prescribed for the purposes of sub-paragraph~(1)($b$)  include ones paid to the non-resident parent under the law of a place outside the United Kingdom.

\subsection*{\itshape Nil rate}

5. The rate payable is nil if the non-resident parent—
\begin{enumerate}\item[]
($a$) is of a prescribed description; or

($b$) has a net weekly income of below £5. 
\end{enumerate}

\subsection*{\itshape Apportionment}

6.---(1) If the non-resident parent has more than one qualifying child and in relation to them there is more than one person with care, the amount of child support maintenance payable is (subject to paragraph~7 or~8) to be determined by apportioning the rate between the persons with care.

(2) The rate of maintenance liability is to be divided by the number of qualifying children, and shared among the persons with care according to the number of qualifying children in relation to whom each is a person with care.

\subsection*{\itshape Shared care—basic and reduced rate}

7.---(1) This paragraph applies only if the rate of child support maintenance payable is the basic rate or a reduced rate.

(2) If the care of a qualifying child is shared between the non-resident parent and the person with care, so that the non-resident parent from time to time has care of the child overnight, the amount of child support maintenance which he would otherwise have been liable to pay the person with care, as calculated in accordance with the preceding paragraphs of this Part of this Schedule, is to be decreased in accordance with this paragraph.

(3) First, there is to be a decrease according to the number of such nights which the Secretary of State determines there to have been, or expects there to be, or both during a prescribed twelve-month period.

(4) The amount of that decrease for one child is set out in the following Table—

\medskip

{\footnotesize\noindent
%\begin{tabulary}{\linewidth}{JJ}
\begin{longtable}{ll}
\hline
\itshape Number of nights	& \itshape Fraction to subtract\\
\hline
\endhead
\hline
\endlastfoot
52 to 103	&One-seventh\\
104 to 155	&Two-sevenths\\
156 to 174	&Three-sevenths\\
175 or~more	&One-half\\
%\hline
%\end{tabulary}
\end{longtable}

}

\medskip

(5) If the person with care is caring for more than one qualifying child of the non-resident parent, the applicable decrease is the sum of the appropriate fractions in the Table divided by the number of such qualifying children.

(6) If the applicable fraction is one-half in relation to any qualifying child in the care of the person with care, the total amount payable to the person with care is then to be further decreased by £7 for each such child.

(7) If the application of the preceding provisions of this paragraph would decrease the weekly amount of child support maintenance (or the aggregate of all such amounts) payable by the non-resident parent to the person with care (or all of them) to less than £5, he is instead liable to pay child support maintenance at the rate of £5 a week, apportioned (if appropriate) in accordance with paragraph~6. 

\subsection*{\itshape Shared care—flat rate}

8.---(1) This paragraph applies only if—
\begin{enumerate}\item[]
($a$) the rate of child support maintenance payable is a flat rate; and

($b$) that rate applies because the non-resident parent falls within paragraph~4(1)($b$)  or~($c$)  or~4(2).
\end{enumerate}

(2) If the care of a qualifying child is shared as mentioned in paragraph~7(2)  for at least 52 nights during a prescribed 12-month period, the amount of child support maintenance payable by the non-resident parent to the person with care of that child is nil.

\subsection*{\itshape Regulations about shared care}

9. The Secretary of State may by regulations provide—
\begin{enumerate}\item[]
($a$) for which nights are to count for the purposes of shared care under paragraphs 7 and 8, or for how it is to be determined whether a night counts;

($b$) for what counts, or does not count, as “care” for those purposes; and

($c$) for paragraph~7(3)  or~8(2)  to have effect, in prescribed circumstances, as if the period mentioned there were other than 12 months, and in such circumstances for the Table in paragraph~7(4)  (or that Table as modified pursuant to regulations made under paragraph~10A(2)($a$)), or the period mentioned in paragraph~8(2), to have effect with prescribed adjustments.
\end{enumerate}

\subsection*{\itshape Net weekly income}

10.---(1) For the purposes of this Schedule, net weekly income is to be determined in such manner as is provided for in regulations.

(2) The regulations may, in particular, provide for the Secretary of State to estimate any income or make an assumption as to any fact where, in his view, the information at his disposal is unreliable, insufficient, or relates to an atypical period in the life of the non-resident parent.

(3) Any amount of net weekly income (calculated as above) over £2,000 is to be ignored for the purposes of this Schedule.

\subsection*{\itshape Regulations about rates, figures, etc.}

10A.---(1) The Secretary of State may by regulations provide that—
\begin{enumerate}\item[]
($a$) paragraph~2 is to have effect as if different percentages were substituted for those set out there;

($b$) paragraph~3(1)  or~(3), 4(1), 5, 7(7)  or~10(3)  is to have effect as if different amounts were substituted for~those set out there.
\end{enumerate}

(2) The Secretary of State may by regulations provide that—
\begin{enumerate}\item[]
($a$) the Table in paragraph~7(4)  is to have effect as if different numbers of nights were set out in the first column and different fractions were substituted for those set out in the second column;

($b$) paragraph~7(6)  is to have effect as if a different amount were substituted for that mentioned there, or as if the amount were an aggregate amount and not an amount for each qualifying child, or both.
\end{enumerate}

\subsection*{\itshape Regulations about income}

10B. The Secretary of State may by regulations provide that, in such circumstances and to such extent as may be prescribed—
\begin{enumerate}\item[]
($a$) where the Secretary of State is satisfied that a person has intentionally deprived himself of a source of income with a view to reducing the amount of his net weekly income, his net weekly income shall be taken to include income from that source of an amount estimated by the Secretary of State;

($b$) a person is to be treated as possessing income which he does not possess;

($c$) income which a person does possess is to be disregarded.
\end{enumerate}

\subsection*{\itshape References to various terms}

10C.---(1) References in this Part of this Schedule to “qualifying children” are to those qualifying children with respect to whom the maintenance calculation falls to be made.

(2) References in this Part of this Schedule to “relevant other children” are to—
\begin{enumerate}\item[]
($a$) children other than qualifying children in respect of whom the non-resident parent or his partner receives child benefit under Part IX of the Social Security Contributions and Benefits Act 1992; and

($b$) such other description of children as may be prescribed.
\end{enumerate}

(3) In this Part of this Schedule, a person “receives” a benefit, pension, or allowance for any week if it is paid or due to be paid to him in respect of that week.

(4) In this Part of this Schedule, a person’s “partner” is—
\begin{enumerate}\item[]
($a$) if they are a couple, the other member of that couple;

($b$) if the person is a husband or wife by virtue of a marriage entered into under a law which permits polygamy, another party to the marriage who is of the opposite sex and is a member of the same household.
\end{enumerate}

%(5) In sub-paragraph~(4)($a$), “couple” means a man and a woman who are—
%\begin{enumerate}\item[]
%($a$) married to each other and are members of the same household; or
%
%($b$) not married to each other but are living together as husband and wife.
%\end{enumerate}

% Para 10C(5), (6) substituted for para 10C(5) by 2004 c 33 Sch 24 para 6
(5) In sub-paragraph (4)($a$), “couple” means—
\begin{enumerate}\item[]
($a$) a man and a woman who are married to each other and are members of the same household,

($b$) a man and a woman who are not married to each other but are living together as husband and wife,

($c$) two people of the same sex who are civil partners of each other and are members of the same household, or

($d$) two people of the same sex who are not civil partners of each other but are living together as if they were civil partners.
\end{enumerate}

(6) For the purposes of this paragraph, two people of the same sex are to be regarded as living together as if they were civil partners if, but only if, they would be regarded as living together as husband and wife were they instead two people of the opposite sex.

\amendment{
Para. 10C(5), (6) substituted for para. 10C(5) (5.12.05) by the Civil Partnership Act 2004 Sch. 24 para. 6.
}

\section[Part II --- General provisions about 
%maintenance assessments 
\emph{maintenance calculations}  % Words substituted by 2000 c 19 s 1(2)(a)
]{Part II\\*General provisions about 
%maintenance assessments 
\emph{maintenance calculations}  % Words substituted by 2000 c 19 s 1(2)(a)
}

\renewcommand\parthead{--- Schedule 1 Part II}

\subsection*{\itshape Effective date of 
%assessment 
\emph{calculation}  % Words substituted by 2000 c 19 s 1(2)(b)
}

11.---(1) A 
%maintenance assessment 
\emph{maintenance calculation}  % Words substituted by 2000 c 19 s 1(2)(a)
shall take effect on such date as may be determined in accordance with regulations made by the Secretary of State.

(2) That date may be earlier than the date on which the 
%assessment 
\emph{calculation}  % Words substituted by 2000 c 19 s 1(2)(b)
is made.

\amendment{
Para. 11 came into force 27.6.92.

\medskip

Paras. 12, 13 came into force 5.4.93.
}

\subsection*{\itshape Form of 
%assessment 
\emph{calculation}  % Words substituted by 2000 c 19 s 1(2)(b)
}

12. Every 
%maintenance assessment 
\emph{maintenance calculation}  % Words substituted by 2000 c 19 s 1(2)(a)
shall be made in such form and contain such information as the Secretary of State may direct.

% Para 13 repealed by 2000 c 19 Sch 3 para 11(21)(a)
\subsection*{\itshape Assessments where amount of child support is nil\\*\emph{1993 scheme only}}

13.
%A child support officer 
The Secretary of State%  % Words substituted (1.6.99) by 1998 c 14 Sch 7 para 48(3)
shall not decline to make a maintenance assessment only on the ground that the amount of the assessment is nil.

\amendment{
Words substituted in para. 13 (1.6.99) by the Social Security Act 1998 Sch. 7 para. 48(3).

Para. 13 repealed (3.3.03 for 2003 scheme cases) by the Child Support, Pensions and Social Security Act 2000 Sch. 3 para. 11(22)(a).
}

\subsection*{\itshape Consolidated applications and 
%assessments 
\emph{calculations}  % Words substituted by 2000 c 19 s 1(2)(b)
}

14.%
---(1)   % Paragraph renumbered as para 14(1) by 2000 c 19 Sch 3 para 11(21)(b)
The Secretary of State may by regulations provide—
\begin{enumerate}\item[]
($a$) for two or more applications for 
%maintenance assessments 
\emph{maintenance calculations}  % Words substituted by 2000 c 19 s 1(2)(a)
to be treated, in prescribed circumstances, as a single application; and

($b$) for the replacement, in prescribed circumstances, of a 
%maintenance assessment 
\emph{maintenance calculation}  % Words substituted by 2000 c 19 s 1(2)(a)
made on the application of one person by a later 
%maintenance assessment 
\emph{maintenance calculation}  % Words substituted by 2000 c 19 s 1(2)(a)
made on the application of that or any other person.
\end{enumerate}

% Para 14(2) inserted by 2000 c 19 Sch 3 para 11(21)(b)
(2) [\emph{2003 scheme only}] In sub-paragraph~(1), the references (however expressed) to applications for maintenance calculations include references to applications treated as made.

\amendment{
Para. 14 came into force 27.6.92.

Para. 14(2) inserted (3.3.03 for 2003 scheme cases) by the Child Support, Pensions and Social Security Act 2000 Sch. 3 para. 11(22)(b).

}

\subsection*{\itshape Separate 
%assessments 
\emph{calculations}  % Words substituted by 2000 c 19 s 1(2)(b)
for different periods}

15. Where 
%a child support officer 
the Secretary of State  % Words substituted (prosp) by 1998 c 14 Sch 7 para 48(4)
is satisfied that the circumstances of a case require different amounts of child support maintenance to be assessed in respect of different periods, he may make separate 
%maintenance assessments 
\emph{maintenance calculations}  % Words substituted by 2000 c 19 s 1(2)(a)
each expressed to have effect in relation to a different specified period.

\amendment{
Para. 15 came into force 5.4.93.

Words substituted in para. 15 (1.6.99) by the Social Security Act 1998 Sch. 7 para. 48(4).
}

\subsection*{\itshape Termination of 
%assessments
\emph{calculations}  % Words substituted by 2000 c 19 s 1(2)(b)
}

16.---(1) A 
%maintenance assessment 
\emph{maintenance calculation}  % Words substituted by 2000 c 19 s 1(2)(a)
shall cease to have effect—
\begin{enumerate}\item[]
($a$) on the death of the 
%absent parent% 
\emph{non-resident parent}%  % Words substituted by 2000 c 19 Sch 3 para 11(2)
, or of the person with care, with respect to whom it was made;

($b$) on there no longer being any qualifying child with respect to whom it would have effect;

($c$) on the 
%absent parent 
\emph{non-resident parent}  % Words substituted by 2000 c 19 Sch 3 para 11(2)
with respect to whom it was made ceasing to be a parent of—
\begin{enumerate}\item[]
(i) the qualifying child with respect to whom it was made; or

(ii) where it was made with respect to more than one qualifying child, all of the qualifying children with respect to whom it was made;
\end{enumerate}

% Para 16(1)(d), (e) repealed by 2000 c 19 Sch 3 para 11(22)(c)(i)
($d$) [\emph{1993 scheme only}] where the absent parent and the person with care with respect to whom it was made have been living together for a continuous period of six months;

($e$) [\emph{1993 scheme only}] where a new maintenance assessment is made with respect to any qualifying child with respect to whom the assessment in question was in force immediately before the making of the new assessment.
\end{enumerate}

% Para 16(2)--(9) repealed by 2000 c 19 Sch 3 para 11(22)(c)(ii)
(2) [\emph{1993 scheme only}] A maintenance assessment made in response to an application under section~4 or 7 shall be cancelled by 
%a child support officer 
the Secretary of State  % Words substituted (1.6.99) by 1998 c 14 Sch 7 para 48(5)(a)
if the person on whose application the assessment was made asks him to do so.

(3) [\emph{1993 scheme only}] A maintenance assessment made in response to an application under section~6 shall be cancelled by 
%a child support officer 
the Secretary of State  % Words substituted (1.6.99) by 1998 c 14 Sch 7 para 48(5)(a)
if—
\begin{enumerate}\item[]
($a$) the person on whose application the assessment was made (“the applicant”) asks him to do so; and

($b$) he is satisfied that the applicant has ceased to fall within subsection (1)  of that section.
\end{enumerate}

(4) [\emph{1993 scheme only}] Where 
%a child support officer 
the Secretary of State  % Words substituted (1.6.99) by 1998 c 14 Sch 7 para 48(5)(a)
is satisfied that the person with care with respect to whom a maintenance assessment was made has ceased to be a person with care in relation to the qualifying child, or any of the qualifying children, with respect to whom the assessment was made, he may cancel the assessment with effect from the date on which, in his opinion, the change of circumstances took place.

% Para 16(4A) inserted (22.1.96) by 1995 c 34 s 14(2)
(4A) [\emph{1993 scheme only}] A maintenance assessment may be cancelled by 
%a child support officer 
the Secretary of State  % Words substituted (1.6.99) by 1998 c 14 Sch 7 para 48(5)(a)
if he is 
%conducting a review under section 16, 17, 18 or 19 
proposing to make a decision under section 16 or 17  % Words substituted (1.6.99) by 1998 c 14 Sch 7 para 48(5)(b)
and it appears to him—
\begin{enumerate}\item[]
($a$) that the person with care with respect to whom the maintenance assessment in question was made has failed to provide him with sufficient information to enable him to 
%complete the review%
make the decision%  % Words substituted (1.6.99) by 1998 c 14 Sch 7 para 48(5)(b)
; and

($b$) where the maintenance assessment in question was made in response to an application under section 6, that the person with care with respect to whom the assessment was made has ceased to fall within subsection (1) of that section.
\end{enumerate}

(5) [\emph{1993 scheme only}] Where—
\begin{enumerate}\item[]
($a$) at any time a maintenance assessment is in force but a child support officer would no longer have jurisdiction to make it if it were to be applied for at that time; and

($b$) the assessment has not been cancelled, or has not ceased to have effect, under or by virtue of any other provision made by or under this Act,
\end{enumerate}
it shall be taken to have continuing effect unless cancelled by 
%a child support officer 
the Secretary of State  % Words substituted (1.6.99) by 1998 c 14 Sch 7 para 48(5)(a)
in accordance with such prescribed provision (including provision as to the effective date of cancellation) as the Secretary of State considers it appropriate to make.\looseness=1

(6) [\emph{1993 scheme only}] Where both the absent parent and the person with care with respect to whom a maintenance assessment was made request 
%a child support officer 
the Secretary of State  % Words substituted (1.6.99) by 1998 c 14 Sch 7 para 48(5)(a)
to cancel the assessment, he may do so if he is satisfied that they are living together.\looseness=1

(7) [\emph{1993 scheme only}] Any cancellation of a maintenance assessment under sub-paragraph 
(4A),~% % Word inserted (22.1.96) by 1995 c 34 s 14(3)
(5)  or~(6)  shall have effect from such date as may be determined by 
%the child support officer%
the Secretary of State%  % Words substituted (1.6.99) by 1998 c 14 Sch 7 para 48(5)(c)
.

(8) [\emph{1993 scheme only}] Where 
%a child support officer 
the Secretary of State  % Words substituted (1.6.99) by 1998 c 14 Sch 7 para 48(5)(a)
cancels a maintenance assessment, he shall immediately notify the absent parent and person with care, so far as that is reasonably practicable.

(9) [\emph{1993 scheme only}] Any notice under sub-paragraph (8)  shall specify the date with effect from which the cancellation took effect.

(10) [\emph{1993 scheme version}] A person with care with respect to whom a maintenance assessment is in force shall provide the Secretary of State with such information, in such circumstances, as may be prescribed, with a view to assisting the Secretary of State 
%or a child support officer  % Words repealed (1.6.99) by 1998 c 14 Sch 7 para 48(5)(d)
in determining whether the assessment has ceased to have effect, or should be cancelled.

(10) [\emph{2003 scheme version}] A person with care with respect to whom a 
%maintenance assessment 
maintenance calculation  % Words substituted by 2000 c 19 s 1(2)(a)
is in force shall provide the Secretary of State with such information, in such circumstances, as may be prescribed, with a view to assisting the Secretary of State 
%or a child support officer  % Words repealed (1.6.99) by 1998 c 14 Sch 7 para 48(5)(d)
in determining whether the 
%assessment 
calculation  % Words substituted by 2000 c 19 s 1(2)(b)
has ceased to have effect%
%, or should be cancelled  % Words repealed by 2000 c 19 Sch 3 para 11(22)(c)(iii)
.

(11) The Secretary of State may by regulations make such supplemental, incidental or transitional provision as he thinks necessary or expedient in consequence of the provisions of this paragraph.

\amendment{
Para. 16(5), (10), (11) came into force 27.6.92; para. 16 came fully into force 5.4.93.

Para. 16(4A) inserted and words inserted in para.~16(7) (22.1.96) by the Child Support Act 1995 s. 14(2), (3).

Words substituted in para. 16(2)--(8) and words repealed in para. 16(10) (1.6.99) by the Social Security Act 1998 Sch. 7 para. 48(5).

Para. 16(1)(d), (e), (2)--(9) and words in para. 16(10) repealed (3.3.03 for 2003 scheme cases) by the Child Support, Pensions and Social Security Act 2000 Sch. 3 para. 11(22)(c).
}

\part[Schedule 2 --- Provision of information to Secretary of State]{Schedule 2\\*Provision of information to Secretary of State}

\renewcommand\parthead{--- Schedule 2}

\amendment{
Words substituted in Sch. 2 (18.4.05) by the Commissioners for Revenue and Customs Act 2005 Sch. 4 para. 42(a), (b).
}

\subsection*{\itshape Inland Revenue records}

1.---(1) This paragraph applies where the Secretary of State or the Department of Health and Social Services for Northern Ireland requires information for the purpose of tracing—
\begin{enumerate}\item[]
($a$) the current address of 
%an absent parent% 
\emph{a non-resident parent}%  % Words substituted by 2000 c 19 Sch 3 para 11(2)
; or

($b$) the current employer of 
%an absent parent% 
\emph{a non-resident parent}%  % Words substituted by 2000 c 19 Sch 3 para 11(2)
.
\end{enumerate}

(2) In such a case, no obligation as to secrecy imposed by statute or otherwise on 
%a person employed in relation to the Inland Revenue 
a Revenue and Customs official  % Words substituted by 2005 c 11 Sch 4 para 42(a)
shall prevent any information obtained or held in connection with the assessment or collection of income tax from being disclosed to—
\begin{enumerate}\item[]
($a$) the Secretary of State;

($b$) the Department of Health and Social Services for Northern Ireland; or

($c$) an officer of either of them authorised to receive such information in connection with the operation of this Act or of any corresponding Northern Ireland legislation.
\end{enumerate}

(3) This paragraph extends only to disclosure by or under the authority of 
%the Commissioners of Inland Revenue%
the Commissioners for Her Majesty’s Revenue and Customs%  % Words substituted by 2005 c 11 Sch 4 para 42(a)
.

(4) Information which is the subject of disclosure to any person by virtue of this paragraph shall not be further disclosed to any person except where the further disclosure is made—
\begin{enumerate}\item[]
($a$) to a person to whom disclosure could be made by virtue of sub-\hspace{0pt}paragraph~(2); or

($b$) for the purposes of any proceedings (civil or criminal) in connection with the operation of this Act or of any corresponding Northern Ireland legislation.
\end{enumerate}

\amendment{
Para. 1 came into force 5.4.93.
}

\medskip

% Para 1A inserted by 1999 c 30 s 80
1A.---(1) This paragraph applies to any information which—
\begin{enumerate}\item[]
($a$) relates to any earnings or other income of 
%an absent parent 
\emph{a non-resident parent}  % Words substituted by 2000 c 19 Sch 3 para 11(2)
in respect of a tax year in which he is or was a self-employed earner, and

($b$) is required by the Secretary of State or the Department of Health and Social Services for Northern Ireland for any purposes of this Act
or any corresponding Northern Ireland legislation%  % Words inserted (1.12.99) by SI 1999/3147 art 71(2)
.
\end{enumerate}

(2) No obligation as to secrecy imposed by statute or otherwise on 
%a person employed in relation to the Inland Revenue 
a Revenue and Customs official  % Words substituted by 2005 c 11 Sch 4 para 42(a)
shall prevent any such information obtained or held in connection with the assessment or collection of income tax from being disclosed to—
\begin{enumerate}\item[]
($a$) the Secretary of State;

($b$) the Department of Health and Social Services for Northern Ireland; or

($c$) an officer of either of them authorised to receive such information in connection with the operation of this Act
or any corresponding Northern Ireland legislation%  % Words inserted (1.12.99) by SI 1999/3147 art 71(2)
.
\end{enumerate}

(3) This paragraph extends only to disclosure by or under the authority of 
%the Commissioners of Inland Revenue%
the Commissioners for Her Majesty’s Revenue and Customs%  % Words substituted by 2005 c 11 Sch 4 para 42(a)
.

(4) Information which is the subject of disclosure to any person by virtue of this paragraph shall not be further disclosed to any person except where the further disclosure is made—
\begin{enumerate}\item[]
($a$) to a person to whom disclosure could be made by virtue of sub-\hspace{0pt}paragraph~(2); or

($b$) for the purposes of any proceedings (civil or criminal) in connection with the operation of this Act
or any corresponding Northern Ireland legislation%  % Words inserted (1.12.99) by SI 1999/3147 art 71(2)
.
\end{enumerate}

(5) For the purposes of this paragraph “self-employed earner” and “tax year” have the same meaning as in Parts I to VI of the Social Security Contributions and Benefits Act 1992
or, in relation to Northern Ireland, the Social Security Contributions and Benefits (Northern Ireland) Act 1992%  % Words inserted (1.12.99) by SI 1999/3147 art 71(3)
.

\amendment{
Para. 1A inserted (11.11.99) by the Welfare Reform and Pensions Act 1999 s. 80.

Words inserted in para. 1A(1), (2), (4), (5) (1.12.99) by the Welfare Reform and Pensions (Northern Ireland) Order 1999 art. 71.

\medskip

Para. 2 repealed (8.9.98) by the Social Security Act 1998 Sch. 7 para. 49.
}

\medskip

% Para 3 added by 2005 c 11 Sch 4 para 42(c)
3. In this Schedule “Revenue and Customs official” has the same meaning as in section 18 of the Commissioners for Revenue and Customs Act 2005 (confidentiality).


\amendment{
Para. 3 added (18.4.05) by the Commissioners for Revenue and Customs Act 2005 Sch. 4 para. 42(c).

\medskip

Sch. 3 repealed (1.6.99) by the Social Security Act 1998 Sch. 7 para. 50.
}

% Para 2 repealed (8.9.98) by 1998 c 14 Sch 7 para 49
%\subsection*{\itshape Local authority records}
%
%2.---(1) This paragraph applies where—
%\begin{enumerate}\item[]
%($a$) the Secretary of State requires relevant information in connection with the discharge by him, or by any child support officer, of functions under this Act; or
%
%($b$) the Department of Health and Social Services for Northern Ireland requires relevant information in connection with the discharge of any functions under any corresponding Northern Ireland legislation.
%\end{enumerate}
%
%(2) The Secretary of State may give a direction to the appropriate authority requiring them to give him such relevant information in connection with any housing benefit or 
%%community charge benefit 
%council tax benefit  % Words substituted (1.4.93) by 1992 c 14 Sch 13 para 94(a)
%to which an absent parent or person with care is entitled as the Secretary of State considers necessary in connection with his determination of—
%\begin{enumerate}\item[]
%($a$) that person’s income of any kind;
%
%($b$) the amount of housing costs to be taken into account in determining that person’s income of any kind; or
%
%($c$) the amount of that person’s protected income.
%\end{enumerate}
%
%(3) The Secretary of State may give a similar direction for the purposes of enabling the Department of Health and Social Services for Northern Ireland to obtain such information for the purposes of any corresponding Northern Ireland legislation.
%
%(4) In this paragraph—
%\begin{enumerate}\item[]
%    “appropriate authority” means—
%\begin{enumerate}\item[]
%    ($a$) 
%    in relation to housing benefit, the housing or local authority concerned; and
%
%    ($b$) 
%    in relation to 
%%community charge benefit, the charging authority 
%council tax benefit, the billing authority  % Words substituted (1.4.93) by 1992 c 14 Sch 13 para 94(b)
%or, in Scotland, the levying authority; and
%\end{enumerate}
%
%    “relevant information” means information of such a description as may be prescribed. 
%\end{enumerate}
%
%\amendment{
%Para. 2(4) came into force 27.6.92; para. 2 came fully into force 5.4.93.
%
%Words substituted in para.~2(2), (4) (1.4.93) by the Local Government Finance Act 1992 Sch.~13 para.~94.
%}

% Sch 3 repealed (1.6.99) by 1998 c 14 Sch 7 para 50
%\part[Schedule 3 --- Child support appeal tribunals]{Schedule 3\\*Child support appeal tribunals}
%
%\renewcommand\parthead{--- Schedule 3}
%
%\amendment{
%Paras. 1, 2 came into force 5.4.93.
%}
%
%\subsection*{\itshape The President}
%
%1.---(1) The person appointed 
%%under Schedule 10 to the Social Security Act 1975  % Words repealed (1.7.92) by 1992 c 6 Sch 1
%as President of the social security appeal tribunals, medical appeal tribunals and disability appeal tribunals shall, by virtue of that appointment, also be President of the child support appeal tribunals.
%
%(2) It shall be the duty of the President to arrange such meetings of the chairmen and members of child support appeal tribunals, and such training for them, as he considers appropriate.
%
%(3) The President may, with the consent of the Secretary of State as to numbers, remuneration and other terms and conditions of service, appoint such officers and staff as he thinks fit for the child support appeal tribunals and their full-time chairmen.
%
%\amendment{
%Words in para. 1(1) repealed (1.7.92) by the Social Security (Consequential Provisions) Act 1992 Sch.~1.
%}
%
%\subsection*{\itshape Membership of child support appeal tribunals}
%
%2.---(1) A child support appeal tribunal shall consist of a chairman and two other persons.
%
%(2) The chairman and the other members of the tribunal must not all be of the same sex.
%
%(3) Sub-paragraph (2)  shall not apply to any proceedings before a child support appeal tribunal if the chairman of the tribunal rules that it is not reasonably practicable to comply with that sub-paragraph in those proceedings.
%
%% Para 2(4) added (2.12.96) by 1995 c 34 Sch 3 para 17
%(4) This paragraph is subject to the provisions of any regulations made under paragraph 9 of Schedule 4A.
%
%\amendment{
%Para. 2(4) added (2.12.96) by the Child Support Act 1995 Sch. 3 para. 17.
%}
%
%\subsection*{\itshape The chairmen}
%
%3.---(1) The chairman of a child support appeal tribunal shall be nominated by the President.
%
%(2) The President may nominate himself or a person drawn—
%\begin{enumerate}\item[]
%($a$) from the appropriate panel appointed by the Lord Chancellor, or (as the case may be) the Lord President of the Court of Session, under section 7 of the Tribunals and Inquiries Act 1971;
%
%($b$) from among those appointed under paragraph 4; or
%
%($c$) from among those appointed 
%%under paragraph 1A of Schedule 10 to the Social Security Act 1975  % Words repealed (1.7.92) by 1992 c 6 Sch 1
%to act as full-time chairmen of social security appeal tribunals.
%\end{enumerate}
%
%(3) Subject to any regulations made by the Lord Chancellor, no person shall be nominated as a chairman of a child support appeal tribunal by virtue of sub-paragraph (2)($a$)  unless he has a 5 year general qualification or is an advocate or solicitor in Scotland of 5 years' standing.
%
%\amendment{
%Para. 3(3) came into force 27.6.92; para. 3 came fully into force 5.4.93.
%
%Words in para. 3(2)(c) repealed (1.7.92) by the Social Security (Consequential Provisions) Act 1992 Sch.~1.
%
%\medskip
%
%Paras. 4--8 came into force 5.4.93.
%}
%
%\medskip
%
%4.---(1) The Lord Chancellor may appoint regional and other full-time chairmen for child support appeal tribunals.
%
%(2) A person is qualified to be appointed as a full-time chairman if he has a 7 year general qualification or is an advocate or solicitor in Scotland of 7 years' standing.
%
%(3) A person appointed to act as a full-time chairman shall hold and vacate office in accordance with the terms of his appointment, except that he must vacate his office 
%%at the end of the completed year of service in which he reaches the age of 72 unless his appointment is continued under sub-paragraph (4)%
%on the date on which he reaches the age of 70; but this sub-paragraph is subject to section 26(4) to (6) of the Judicial Pensions and Retirement Act 1993 (power to authorise continuance in office up to the age of 75)%  % Words substituted (31.3.95) by 1993 c 8 Sch 6 para 23(1)(a)
%.
%
%% Para 4(4) omitted (31.3.95) by 1993 c 8 Sch 6 para 23(1)(b)
%%(4) Where the Lord Chancellor considers it desirable in the public interest to retain a full-time chairman in office after the end of the completed year of service in which he reaches the age of 72, he may from time to time authorise the continuance of that person in office until any date not later than that on which that person reaches the age of 75.
%
%(5) A person appointed as a full-time chairman may be removed from office by the Lord Chancellor, on the ground of misbehaviour or incapacity.
%
%(6) Section 75 of the Courts and Legal Services Act 1990 (judges etc.\ barred from legal practice) shall apply to any person appointed as a full-time chairman under this Schedule as it applies to any person holding as a full-time appointment any of the offices listed in Schedule 11 to that Act.
%
%(7) The Secretary of State may pay, or make such payments towards the provision of, such remuneration, pensions, allowances or gratuities to or in respect of persons appointed as full-time chairmen under this paragraph as, with the consent of the Treasury, he may determine.
%
%% Para 4(8) added (31.3.95) by 1993 c 8 Sch 8 para 21(1)
%(8) Sub-paragraph (7), so far as relating to pensions, allowances or gratuities, shall not have effect in relation to any person to whom Part I of the Judicial Pensions and Retirement Act 1993 applies, except to the extent provided by or under that Act.
%
%\amendment{
%Words substituted in para.~4(3) and para.~4(4) omitted (31.3.95) by the Judicial Pensions and Retirement Act 1991 Sch.~6 para.~23(1).
%
%Para.~4(8) added (31.3.95) by the Judicial Pensions and Retirement Act 1991 Sch.~8 para.~21(1).
%}
%
%\subsection*{\itshape Other members of child support appeal tribunals}
%
%5.---(1) The members of a child support appeal tribunal other than the chairman shall be drawn from the appropriate panel constituted under this paragraph.
%
%(2) The panels shall be constituted by the President for the whole of Great Britain, and shall—
%\begin{enumerate}\item[]
%($a$) act for such areas; and
%
%($b$) be composed of such persons,
%\end{enumerate}
%as the President thinks fit.
%
%(3) The panel for an area shall be composed of persons appearing to the President to have knowledge or experience of conditions in the area and to be representative of persons living or working in the area.
%
%(4) Before appointing members of a panel, the President shall take into consideration any recommendations from such organisations or persons as he considers appropriate.
%
%(5) The members of the panels shall hold office for such period as the President may direct.
%
%(6) The President may at any time terminate the appointment of any member of a panel.
%
%\subsection*{\itshape Clerks of tribunals}
%
%6.---(1) Each child support appeal tribunal shall be serviced by a clerk appointed by the President.
%
%(2) The duty of summoning members of a panel to serve on a child support appeal tribunal shall be performed by the clerk to the tribunal.
%
%\subsection*{\itshape Expenses of tribunal members and others}
%
%7.---(1) The Secretary of State may pay—
%\begin{enumerate}\item[]
%($a$) to any member of a child support appeal tribunal, such remuneration and travelling and other allowances as the Secretary of State may determine with the consent of the Treasury;
%
%($b$) to any person required to attend at any proceedings before a child support appeal tribunal, such travelling and other allowances as may be so determined; and
%
%($c$) such other expenses in connection with the work of any child support appeal tribunal as may be so determined.
%\end{enumerate}
%
%(2) In sub-paragraph (1), references to travelling and other allowances include references to compensation for loss of remunerative time.
%
%(3) No compensation for loss of remunerative time shall be paid to any person under this paragraph in respect of any time during which he is in receipt of other remuneration so paid.
%
%\subsection*{\itshape Consultation with Lord Advocate}
%
%8. Before exercising any of his powers under paragraph 3(3)  or 4(1)
%%, (4)   % Word repealed (31.3.95) by 1993 c 8 Sch 9
%or (5), the Lord Chancellor shall consult the Lord Advocate.
%
%\amendment{
%Word repealed in para.~8 (31.3.95) by the Judicial Pensions and Retirement Act 1991 Sch.~9.
%}

\part[Schedule 4 --- Child Support Commissioners]{Schedule 4\\*Child Support Commissioner}

\renewcommand\parthead{--- Schedule 4}

\amendment{
Sch. 4 came into force 1.9.92.
}

\subsection*{\itshape Tenure of office}

1.---(1) Every Child Support Commissioner shall vacate his office 
%at the end of the completed year of service in which he reaches the age of 72.
on the date on which he reaches the age of 70; but this sub-paragraph is subject to section 26(4) to (6) of the Judicial Pensions and Retirement Act 1993 (power to authorise continuance in office up to the age of 75).  % Words substituted (31.3.95) by 1993 c 8 Sch 6 para 23(2)(a)

% Para 1(2) omitted (31.3.95) by 1993 c 8 Sch 6 para 23(2)(b)
%(2) Where the Lord Chancellor considers it desirable in the public interest to retain a Child Support Commissioner in office after the end of the completed year of service in which he reaches the age of 72, he may from time to time authorise the continuance of that Commissioner in office until any date not later than that on which he reaches the age of 75.

(3) A Child Support Commissioner may be removed from office by the Lord Chancellor on the ground of misbehaviour or incapacity.

% Para 1(3A), (3B) inserted by 2005 c 4 Sch 4 para 221(2)
(3A) The Lord Chancellor may remove a Child Support Commissioner under sub-paragraph (3) only with the concurrence of the appropriate senior judge.

(3B) The appropriate senior judge is the Lord Chief Justice of England and Wales, unless the Commissioner exercises functions wholly or mainly in Scotland, in which case it is the Lord President of the Court of Session.

\amendment{
Words substituted in para.~1(1) and para.~1(2) omitted (31.3.95) by the Judicial Pensions and Retirement Act 1991 Sch.~6 para.~23(2).

Para. 1(3A), (3B) inserted (3.4.06) by the Constitutional Reform Act 2005 Sch. 4 para. 221(2).
}

\subsection*{\itshape Commissioners' remuneration and their pensions}

2.---(1) The Lord Chancellor may pay, or make such payments towards the provision of such remuneration, pensions, allowances or gratuities to or in respect of persons appointed as Child Support Commissioners as, with the consent of the Treasury, he may determine.

(2) The Lord Chancellor shall pay to a Child Support Commissioner such expenses incurred in connection with his work as such a Commissioner as may be determined by the Treasury.

% Para 2(3) added (31.3.95) by 1993 c 8 Sch 8 para 21(2)
(3) Sub-paragraph (1), so far as relating to pensions, allowances or gratuities, shall not have effect in relation to any person to whom Part I of the Judicial Pensions and Retirement Act 1993 applies, except to the extent provided by or under that Act.

\amendment{
Para.~2(3) added (31.3.95) by the Judicial Pensions and Retirement Act 1991 Sch.~8 para.~21(2).
}

% Para 2A inserted (18.12.95) by 1995 c 34 Sch 3 para 18(1)
\subsection*{\itshape Expenses of other persons}

2A.---%(1) The Secretary of State may pay to any person required to attend at any proceedings before a Child Support Commissioner such travelling and other allowances as, with the consent of the Treasury, the Secretary of State may determine.
%
% Para 2A(1) substituted (prosp) by 1998 c 14 Sch 7 para 51
(1) The Lord Chancellor or, in Scotland, the Secretary of State may pay to any person who attends any proceedings before a Child Support Commissioner such travelling and other allowances as he may determine.

(2) In sub-paragraph (1), references to travelling and other allowances include references to compensation for loss of remunerative time.

(3) No compensation for loss of remunerative time shall be paid to any person under this paragraph in respect of any time during which he is in receipt of other remuneration so paid.

\amendment{
Para. 2A inserted (18.12.95) by the Child Support Act 1995 Sch. 3 para. 18(1).

Para. 2A(1) substituted (1.6.99) by the Social Security Act 1998 Sch. 7 para. 51.
}

\subsection*{\itshape Commissioners barred from legal practice}

3. Section 75 of the Courts and Legal Services Act 1990 (judges etc.\ barred from legal practice) shall apply to any person appointed as a Child Support Commissioner as it applies to any person holding as a full-time appointment any of the offices listed in Schedule 11 to that Act.

\subsection*{\itshape Deputy Child Support Commissioners}

4.---(1) The Lord Chancellor may appoint persons to act as Child Support Commissioners (but to be known as deputy Child Support Commissioners) in order to facilitate the disposal of the business of Child Support Commissioners.

(2) 
Subject to sub-paragraph (2A)  % Words inserted (31.3.95) by 1993 c 8 Sch 6 para 23(3)
a deputy Child Support Commissioner shall be appointed—
\begin{enumerate}\item[]
($a$) from among persons who 
have a 10 year general qualification 
%satisfy the judicial-appointment eligibility condition on a 7-year basis  % Words substituted by 2007 c 15 Sch 10 para 22(4)(a)
or are advocates or solicitors in Scotland of 
10 
%7  % Figure substituted by 2007 c 15 Sch 10 para 22(4)(b)
years' standing; and

($b$) for such period or on such occasions as the Lord Chancellor thinks fit.
\end{enumerate}

% Para 4(2A) inserted (31.3.95) by 1993 c 8 Sch 6 para 23(3)
(2A) No appointment of a person to be a deputy Child Support Commissioner shall be such as to extend beyond the date on which he reaches the age of 70; but this sub-paragraph is subject to section 26(4) to (6) of the Judicial Pensions and Retirement Act 1993 (power to authorise continuance in office up to the age of 75).

(3) Paragraph 2 applies to deputy Child Support Commissioners as if the reference to pensions were omitted and paragraph 3 does not apply to them.

\amendment{
Words inserted in para.~4(2) and para.~4(2A) inserted (31.3.95) by the Judicial Pensions and Retirement Act 1991 Sch.~6 para.~23(3).

Words substituted in para. 4(2)(a) (prosp) by the Tribunals, Courts and Enforcement Act 2007 Sch. 10 para. 22(4).
}

% Para 4A inserted (18.12.95) by 1995 c 34 s 17
\subsection*{\itshape Determination of questions by other officers}

4A.---(1) The Lord Chancellor may by regulations provide—
\begin{enumerate}\item[]
($a$) for officers authorised—
\begin{enumerate}\item[]
(i) by the Lord Chancellor; or

(ii) in Scotland, by the Secretary of State,
\end{enumerate}
to determine any question which is determinable by a Child Support Commissioner and which does not involve the determination of any appeal, application for leave to appeal or reference;

($b$) for the procedure to be followed by any such officer in determining any such question;

($c$) for the manner in which determinations of such questions by such officers may be called in question.
\end{enumerate}

(2) A determination which would have the effect of preventing an appeal, application for leave to appeal or reference being determined by a Child Support Commissioner is not a determination of the appeal, application or reference for the purposes of sub-paragraph (1).

\amendment{
Para. 4A inserted (18.12.95) by the Child Support Act 1995 s. 17(1).
}

\subsection*{\itshape Tribunals of Commissioners}

5.---(1) If it appears to the Chief Child Support Commissioner (or, in the case of his inability to act, to such other of the Child Support Commissioners as he may have nominated to act for the purpose) 
%that an appeal 
that—
\begin{enumerate}\item[]
($a$) an application for leave under section 24(6)($b$); or

($b$) an appeal,
\end{enumerate}  % Words substituted (1.6.99) by 1998 c 14 Sch 7 para 52(1)(a)
falling to be heard by one of the Child Support Commissioners involves a question of law of special difficulty, he may direct 
%that the appeal 
that the application or appeal  % Words substituted (1.6.99) by 1998 c 14 Sch 7 para 52(1)(b)
be dealt with by a tribunal consisting of any three 
or more  % Words inserted (1.6.99) by 1998 c 14 Sch 7 para 52(1)(c)
of the Child Support Commissioners.

(2) If the decision of such a tribunal is not unanimous, the decision of the majority shall be the decision of the tribunal%
; and the presiding Child Support Commissioner shall have a casting vote if the votes are equally divided  % Words inserted (1.6.99) by 1998 c 14 Sch 7 para 52(2)
.

% Para 5(3) inserted (1.6.99) by 1998 c 14 Sch 7 para 52(3)
(3) Where a direction is given under sub-paragraph (1)($a$), section 24(6)($b$) shall have effect as if the reference to a Child Support Commissioner were a reference to such a tribunal as is mentioned in sub-paragraph (1).

\amendment{
Words inserted and substituted in para. 5(1), words inserted in para. 5(2) and para. 5(3) inserted (1.6.99) by the Social Security Act 1998 Sch. 7 para. 52(1)--(3).
}

\subsection*{\itshape Finality of decisions}

6.---(1) Subject to section 25, the decision of any Child Support Commissioner shall be final.

%(2) Sub-paragraph (1)  shall not be taken to make any finding of fact or other determination embodied in, or necessary to, a decision, or on which it is based, conclusive for the purposes of any further decision.

% Para 6(2) substituted (prosp) by 1998 c 14 Sch 7 para 52(4)
(2) If and to the extent that regulations so provide, any finding of fact or other determination which is embodied in or necessary to a decision, or on which a decision is based, shall be conclusive for the purposes of any further decision.

\amendment{
Para. 6(2) substituted (1.6.99) by the Social Security Act 1998 Sch. 7 para. 52(4).
}

\subsection*{\itshape Consultation with Lord Advocate}

7. Before exercising any of his powers under 
%paragraph 1(2)  or (3)%
paragraph 1(3)%  % Words substituted (31.3.95) by 1993 c 8 Sch 6 para 23(4)
, 
%or 4(1)  or (2)($b$)
4(1) or (2)($b$) or 4A(1)%  % Words substituted (18.12.95) by 1995 c 34 s 17(2)
, the Lord Chancellor shall consult the Lord Advocate.

\amendment{
Words substituted in para.~7 (31.3.95) by the Judicial Pensions and Retirement Act 1991 Sch.~6 para.~23(4).

Words substituted in para.~7 (18.12.95) by the Child Support Act 1995 s. 17(2).
}

\subsection*{\itshape Northern Ireland}

8. In its application to Northern Ireland this Schedule shall have effect as if—
\begin{enumerate}\item[]
($a$) for any reference to a Child Support Commissioner (however expressed) there were substituted a corresponding reference to a Child Support Commissioner for Northern Ireland;

% Para 8(aa) inserted by 2002 c 26 Sch 12 para 47
($aa$) paragraph 1(3) were omitted;

% Para 8(ab) inserted by 2005 c 4 Sch 4 para 221(3)
($ab$) paragraph 1(3A) and (3B) were omitted;

($b$) in paragraph 2(1), the word “pensions” were omitted;

% Para 8(bb) inserted (18.12.95) by 1995 c 34 Sch 3 para 18(2)
($bb$) paragraph 2A were omitted;

($c$) for paragraph 3, there were substituted—
\begin{quotation}
“3. A Child Support Commissioner for Northern Ireland, so long as he holds office as such, shall not practise as a barrister or act for any remuneration to himself as arbitrator or referee or be directly or indirectly concerned in any matter as a conveyancer, notary public or solicitor.”;
\end{quotation}

($d$) in paragraph 4—
\begin{enumerate}\item[]
% Para 8(d)(ai) inserted by 2002 c 26 Sch 3 para 22(a)
%(ai) in sub-paragraph (1), for “Lord Chancellor” there were substituted “First Minister and deputy First Minister, acting jointly,'';

(i) for paragraph ($a$)  of sub-paragraph (2)  there were substituted—
\begin{quotation}
“($a$) from among persons who are barristers or solicitors of not less than 
10 
%7  % Words substituted by 2007 c 15 Sch 10 para 22(5)
years' standing; and”;
\end{quotation}

% Para 8(d)(ia) inserted by 2002 c 26 Sch 3 para 22(b)
%(ia) in paragraph ($b$) of sub-paragraph (2), for “Lord Chancellor thinks” there were substituted “First Minister and deputy First Minister think”;

(ii) for sub-paragraph (3)  there were substituted—
\begin{quotation}
“(3) Paragraph 2 applies to deputy Child Support Commissioners for Northern Ireland, but paragraph 3 does not apply to them.”; and
\end{quotation}
\end{enumerate}

($e$) 
%paragraphs 5 
paragraphs 4A  % Words substituted (18.12.95) by 1995 c 34 s 17(3)
to 7 were omitted.
\end{enumerate}

\amendment{
Para. 8(bb) inserted (18.12.95) by the Child Support Act 1995 Sch. 3 para. 18(2).

Words substituted in para.~8(e) (18.12.95) by the Child Support Act 1995 s. 17(3).

Para. 8(aa) inserted (3.4.06) by the Justice (Northern Ireland) Act 2002 Sch. 12 para. 47.

Para. 8(ab) inserted (3.4.06) by the Constitutional Reform Act 2005 Sch. 4 para. 221(3).

Para. 8(d)(ai), (ia) inserted (prosp) by the Justice (Northern Ireland) Act 2002 Sch. 3 para. 22.

Figure substituted in para. 8(d)(i) (prosp) by the Tribunals, Courts and Enforcement Act 2007 Sch. 10 para. 22(5).
}

\part[Schedule 4A --- Departure directions --- \emph{1993 scheme version}]{Schedule 4A\\*Departure directions\\*\emph{1993 scheme version}}

\renewcommand\parthead{--- Schedule 4A}

\amendment{
Sch. 4A inserted (2.12.96) by the Child Support Act 1995 Sch. 1.
}

\subsection*{\itshape Interpretation}

1. In this Schedule—
\begin{enumerate}\item[]
    “departure application” means an application for a departure direction;

    “regulations” means regulations made by the Secretary of State.

% Definition of ``review'' repealed (prosp) by 1998 c 14 Sch 7 para 53(1)
%    “review” means a review under section 16, 17, 18 or 19. 
\end{enumerate}

\amendment{
Definition of ``review'' para. 1 repealed (1.6.99) by the Social Security Act 1998 Sch. 7 para. 53(1).
}

\subsection*{\itshape Applications for departure directions}

2. Regulations may make provision—
\begin{enumerate}\item[]
($a$) as to the procedure to be followed in considering a departure application;

($b$) as to the procedure to be followed when a departure application is referred to 
%a child support appeal tribunal 
an appeal tribunal  % Words substituted (1.6.99) by 1998 c 14 Sch 7 para 53(2)(a)
under section 28D(1)($b$);

($c$) for the giving of a direction by the Secretary of State as to the order in which, in a particular case, 
%a departure application and a review are to be dealt with
a decision on a departure application and a decision under section 16 or 17 are to be made%  % Words substituted (1.6.99) by 1998 c 14 Sch 7 para 53(2)(b)
;

($d$) for the reconsideration of a departure application in a case where further information becomes available to the Secretary of State after the application has been determined.
\end{enumerate}

\amendment{
Words substituted in para. 2(b), (c) (1.6.99) by the Social Security Act 1998 Sch. 7 para. 53(2).
}

\subsection*{\itshape Completion of preliminary consideration}

3. Regulations may provide for determining when the preliminary consideration of a departure application is to be taken to have been completed.

\subsection*{\itshape Information}

4.---(1) Regulations may make provision for the use for any purpose of this Act of—
\begin{enumerate}\item[]
($a$) information acquired by the Secretary of State in connection with an application for, or the making of, a departure direction;

($b$) information acquired by 
%a child support officer or  % Words repealed (prosp) by 1998 c 14 Sch 7 para 53(3)
the Secretary of State in connection with an application for, or the making of, a maintenance assessment.
\end{enumerate}

(2) If any information which is required (by regulations under this Act) to be furnished to the Secretary of State in connection with a departure application has not been furnished within such period as may be prescribed, the Secretary of State may nevertheless proceed to determine the application.

\amendment{
Words repealed in para. 4(1) (1.6.99) by the Social Security Act 1998 Sch. 7 para. 53(3).
}

\subsection*{\itshape Anticipation of change of circumstances}

5.---(1) A departure direction may be given so as to provide that if the circumstances of the case change in such manner as may be specified in the direction a fresh maintenance assessment is to be made.

(2) Where any such provision is made, the departure direction may provide for the basis on which the amount of child support maintenance is to be fixed by the fresh maintenance assessment to differ from the basis on which the amount of child support maintenance was fixed by any earlier maintenance assessment made as a result of the direction.

\amendment{
Para. 6 repealed (1.6.99) by the Social Security Act 1998 Sch. 7 para. 53(4).
}

% Para 6 repealed (prosp) by 1998 c 14 Sch 7 para 53(4)
%\subsection*{\itshape Reviews and departure directions}
%
%6. Regulations may make provision—
%\begin{enumerate}\item[]
%($a$) with respect to cases in which a child support officer is conducting a review of a maintenance assessment which was made as a result of a departure direction;
%
%($b$) with respect to cases in which a departure direction is made at a time when a child support officer is conducting a review.
%\end{enumerate}

\subsection*{\itshape Subsequent departure directions}

7.---(1) Regulations may make provision with respect to any departure application made with respect to a maintenance assessment which was made as a result of a departure direction.

(2) The regulations may, in particular, provide for the application to be considered by reference to the maintenance assessment which would have been made had the departure direction not been given.

\subsection*{\itshape Joint consideration of departure applications and appeals}

8.---(1) Regulations may provide for two or more departure applications with respect to the same current assessment to be considered together.

(2) 
%A child support appeal tribunal 
An appeal tribunal  % Words substituted (1.6.99) by 1998 c 14 Sch 7 para 53(5)
considering—
\begin{enumerate}\item[]
($a$) a departure application referred to it under section 28D(1)($b$), or

($b$) an appeal under section 28H,
\end{enumerate}
may consider it at the same time as hearing an appeal under section 20 in respect of the current assessment, if it considers that to be appropriate.

\amendment{
Words substituted in para. 8 (1.6.99) by the Social Security Act 1998 Sch. 7 para. 53(5).
}

\subsection*{\itshape 
%Child support appeal tribunals
Appeal tribunals  % Heading substituted (1.6.99) by 1998 c 14 Sch 7 para 53(6)(a)
}

9.---(1) Regulations may provide that, in prescribed circumstances, where—
\begin{enumerate}\item[]
($a$) a departure application is referred to 
%a child support appeal tribunal 
an appeal tribunal  % Words substituted (1.6.99) by 1998 c 14 Sch 7 para 53(6)(b)
under section~28D(1)($b$), or

($b$) an appeal is brought under section 28H,
\end{enumerate}
the application or appeal may be dealt with by a tribunal constituted by the chairman sitting alone.

(2) Sub-paragraph (1) does not apply in relation to any appeal which is being heard together with an appeal under section 20.

\amendment{
Heading substituted and words substituted in para. 9(1) (1.6.99) by the Social Security Act 1998 Sch. 7 para. 53(6).
}

\subsection*{\itshape Current assessments which are replaced by fresh assessments}

10. Regulations may make provision as to the circumstances in which prescribed references in this Act to a current assessment are to have effect as if they were references to any later maintenance assessment made with respect to the same persons as the current assessment.

% Sch 4A substituted by 2000 c 19 Sch 2 Pt I
\part[Schedule 4A --- Applications for a variation --- \emph{2003 scheme version}]{\noindent S\lowercase{CHEDULE} 4A\\*Applications for a variation\\*\emph{2003 scheme version}}

\amendment{
Sch. 4A substituted (10.11.00 for regulation-making purposes, 3.3.03 for 2003 scheme cases) by the Child Support, Pensions and Social Security Act 2000 Sch. 2 Pt. I.

See the Child Support (Variations) (Modification of Statutory Provisions) Regulations 2000 reg. 8 for modifications to this Schedule where an application for a variation is made under s. 28G.
}

\subsection*{\itshape Interpretation}

1. In this Schedule, “regulations” means regulations made by the Secretary of State.

\subsection*{\itshape Applications for a variation}

2. Regulations may make provision—
\begin{enumerate}\item[]
($a$) as to the procedure to be followed in considering an application for a variation;

($b$) as to the procedure to be followed when an application for a variation is referred to an appeal tribunal under section~28D(1)($b$).
\end{enumerate}

\subsection*{\itshape Completion of preliminary consideration}

3. Regulations may provide for determining when the preliminary consideration of an application for a variation is to be taken to have been completed.

\subsection*{\itshape Information}

4. If any information which is required (by regulations under this Act) to be furnished to the Secretary of State in connection with an application for a variation has not been furnished within such period as may be prescribed, the Secretary of State may nevertheless proceed to consider the application.

\subsection*{\itshape Joint consideration of applications for a variation and appeals}

5.---(1) Regulations may provide for two or more applications for a variation with respect to the same application for a maintenance calculation to be considered together.

(2) In sub-paragraph~(1), the reference to an application for a maintenance calculation includes an application treated as having been made under section~6. 

(3) An appeal tribunal considering an application for a variation under section~28D(1)($b$)  may consider it at the same time as an appeal under section~20 in connection with an interim maintenance decision, if it considers that to be appropriate.

% Sch 4B inserted (2.12.96) by 1995 c 34 Sch 2
\part[Schedule 4B --- Departure directions: the cases and controls --- \emph{1993 scheme version}]{Schedule 4B\\*Departure directions: the cases and controls\\*\emph{1993 scheme version}}

\amendment{
Sch. 4B inserted (2.12.96) by the Child Support Act 1995 Sch. 2.
}

\section[Part I --- The cases]{Part I\\*The cases}

\renewcommand\parthead{--- Schedule 4B Part I}

\subsection*{\itshape General}

1.---(1) The cases in which a departure direction may be given are those set out in this Part of this Schedule or in regulations made under this Part.

(2) In this Schedule “applicant” means the person whose application for a departure direction is being considered.

\subsection*{\itshape Special expenses}

2.---(1) A departure direction may be given with respect to special expenses of the applicant which were not, and could not have been, taken into account in determining the current assessment in accordance with the provisions of, or made under, Part I of Schedule 1.

(2) In this paragraph “special expenses” means the whole, or any prescribed part, of expenses which fall within a prescribed description of expenses.

(3) In prescribing descriptions of expenses for the purposes of this paragraph, the Secretary of State may, in particular, make provision with respect to—
\begin{enumerate}\item[]
($a$) costs incurred in travelling to work;

($b$) costs incurred by an absent parent in maintaining contact with the child, or with any of the children, with respect to whom he is liable to pay child support maintenance under the current assessment;

($c$) costs attributable to a long-term illness or disability of the applicant or of a dependant of the applicant;

($d$) debts incurred, before the absent parent became an absent parent in relation to a child with respect to whom the current assessment was made—
\begin{enumerate}\item[]
(i) for the joint benefit of both parents;

(ii) for the benefit of any child with respect to whom the current assessment was made; or

(iii) for the benefit of any other child falling within a prescribed category;
\end{enumerate}

($e$) pre-1993 financial commitments from which it is impossible for the parent concerned to withdraw or from which it would be unreasonable to expect that parent to have to withdraw;

($f$) costs incurred by a parent in supporting a child who is not his child but who is part of his family.
\end{enumerate}

(4) For the purposes of sub-paragraph (3)($c$)—
\begin{enumerate}\item[]
($a$) the question whether one person is a dependant of another shall be determined in accordance with regulations made by the Secretary of State;

($b$) “disability” and “illness” have such meaning as may be prescribed; and

($c$) the question whether an illness or disability is long-term shall be determined in accordance with regulations made by the Secretary of State.
\end{enumerate}

(5) For the purposes of sub-paragraph (3)($e$), “pre-1993 financial commitments” means financial commitments of a prescribed kind entered into before 5th April 1993 in any case where—
\begin{enumerate}\item[]
($a$) a court order of a prescribed kind was in force with respect to the absent parent and the person with care concerned at the time when they were entered into; or

($b$) an agreement between them of a prescribed kind was in force at that time.
\end{enumerate}

(6) For the purposes of sub-paragraph (3)($f$), a child who is not the child of a particular person is a part of that person’s family in such circumstances as may be prescribed.

\subsection*{\itshape Property or capital transfers}

3.---(1) A departure direction may be given if—
\begin{enumerate}\item[]
($a$) before 5th April 1993—
\begin{enumerate}\item[]
(i) a court order of a prescribed kind was in force with respect to the absent parent and either the person with care with respect to whom the current assessment was made or the child, or any of the children, with respect to whom that assessment was made, or

(ii) an agreement of a prescribed kind between the absent parent and any of those persons was in force;
\end{enumerate}

($b$) in consequence of one or more transfers of property of a prescribed kind—
\begin{enumerate}\item[]
(i) the amount payable by the absent parent by way of maintenance was less than would have been the case had that transfer or those transfers not been made; or

(ii) no amount was payable by the absent parent by way of maintenance; and
\end{enumerate}

($c$) the effect of that transfer, or those transfers, is not properly reflected in the current assessment.
\end{enumerate}

(2) For the purposes of sub-paragraph (1)($b$), “maintenance” means periodical payments of maintenance made (otherwise than under this Act) with respect to the child, or any of the children, with respect to whom the current assessment was made.

(3) For the purposes of sub-paragraph (1)($c$), the question whether the effect of one or more transfers of property is properly reflected in the current assessment shall be determined in accordance with regulations made by the Secretary of State.

\medskip

4.---(1) A departure direction may be given if—
\begin{enumerate}\item[]
($a$) before 5th April 1993—
\begin{enumerate}\item[]
(i) a court order of a prescribed kind was in force with respect to the absent parent and either the person with care with respect to whom the current assessment was made or the child, or any of the children, with respect to whom that assessment was made, or

(ii) an agreement of a prescribed kind between the absent parent and any of those persons was in force;
\end{enumerate}

($b$) in pursuance of the court order or agreement, the absent parent has made one or more transfers of property of a prescribed kind;

($c$) the amount payable by the absent parent by way of maintenance was not reduced as a result of that transfer or those transfers;

($d$) the amount payable by the absent parent by way of child support maintenance under the current assessment has been reduced as a result of that transfer or those transfers, in accordance with provisions of or made under this Act; and

($e$) it is nevertheless inappropriate, having regard to the purposes for which the transfer or transfers was or were made, for that reduction to have been made.
\end{enumerate}

(2) For the purposes of sub-paragraph (1)($c$), “maintenance” means periodical payments of maintenance made (otherwise than under this Act) with respect to the child, or any of the children, with respect to whom the current assessment was made.

\subsection*{\itshape Additional cases}

5.---(1) The Secretary of State may by regulations prescribe other cases in which a departure direction may be given.

(2) Regulations under this paragraph may, for example, make provision with respect to cases where—
\begin{enumerate}\item[]
($a$) assets which do not produce income are capable of producing income;

($b$) a person’s life-style is inconsistent with the level of his income;

($c$) housing costs are unreasonably high;

($d$) housing costs are in part attributable to housing persons whose circumstances are such as to justify disregarding a part of those costs;

($e$) travel costs are unreasonably high; or

($f$) travel costs should be disregarded.
\end{enumerate}

\section[Part II --- Regulatory Controls]{Part II\\*Regulatory Controls}

\renewcommand\parthead{--- Schedule 4B Part II}

6.---(1) The Secretary of State may by regulations make provision with respect to the directions which may be given in a departure direction.

(2) No directions may be given other than those which are permitted by the regulations.

(3) Regulations under this paragraph may, in particular, make provision for a departure direction to require—
\begin{enumerate}\item[]
($a$) the substitution, for any formula set out in Part I of Schedule 1, of such other formula as may be prescribed;

($b$) any prescribed amount by reference to which any calculation is to be made in fixing the amount of child support maintenance to be increased or reduced in accordance with the regulations;

($c$) the substitution, for any provision in accordance with which any such calculation is to be made, of such other provision as may be prescribed.
\end{enumerate}

(4) Regulations may limit the extent to which the amount of the child support maintenance fixed by a maintenance assessment made as a result of a departure direction may differ from the amount of the child support maintenance which would be fixed by a maintenance assessment made otherwise than as a result of the direction.

(5) Regulations may provide for the amount of any special expenses to be taken into account in a case falling within paragraph 2, for the purposes of a departure direction, not to exceed such amount as may be prescribed or as may be determined in accordance with the regulations.

(6) No departure direction may be given so as to have the effect of denying to an absent parent the protection of paragraph 6 of Schedule 1.

(7) Sub-paragraph (6) does not prevent the modification of the provisions of, or made under, paragraph 6 of Schedule 1 to the extent permitted by regulations under this paragraph.

(8) Any regulations under this paragraph may make different provision with respect to different levels of income.

\part[Schedule 4B --- Applications for a variation: the cases and controls --- \emph{2003 scheme version}]{\noindent S\lowercase{CHEDULE} 4B\\*Applications for a variation: the cases and controls\\*\emph{2003 scheme version}}

\amendment{
Sch. 4B substituted (10.11.00 for regulation-making purposes, 3.3.03 for 2003 scheme cases) by the Child Support, Pensions and Social Security Act 2000 Sch. 2 Pt. II.

See the Child Support (Variations) (Modification of Statutory Provisions) Regulations 2000 reg. 8 for modifications to this Schedule where an application for a variation is made under s. 28G.
}

\section[Part I --- The cases]{Part I\\*The cases}

\subsection*{\itshape General}

1.---(1) The cases in which a variation may be agreed are those set out in this Part of this Schedule or in regulations made under this Part.

(2) In this Schedule “applicant” means the person whose application for~a variation is being considered.

\subsection*{\itshape Special expenses}

2.---(1) A variation applied for by a non-resident parent may be agreed with respect to his special expenses.

(2) In this paragraph “special expenses” means the whole, or any amount above a prescribed amount, or any prescribed part, of expenses which fall within a prescribed description of expenses.

(3) In prescribing descriptions of expenses for the purposes of this paragraph, the Secretary of State may, in particular, make provision with respect to—
\begin{enumerate}\item[]
($a$) costs incurred by a non-resident parent in maintaining contact with the child, or with any of the children, with respect to whom the application for a maintenance calculation has been made (or treated as made);

($b$) costs attributable to a long-term illness or disability of a relevant other child (within the meaning of paragraph~10C(2)  of Schedule 1);

($c$) debts of a prescribed description incurred, before the non-resident parent became a non-resident parent in relation to a child with respect to whom the maintenance calculation has been applied for (or treated as having been applied for)—
\begin{enumerate}\item[]
(i) for the joint benefit of both parents;

(ii) for the benefit of any such child; or

(iii) for the benefit of any other child falling within a prescribed category;
\end{enumerate}

($d$) boarding school fees for a child in relation to whom the application for a maintenance calculation has been made (or~treated as made);

($e$) the cost to the non-resident parent of making payments in relation to a mortgage on the home he and the person with care shared, if he no longer has an interest in it, and she and a child in relation to whom the application for a maintenance calculation has been made (or treated as made) still live there.
\end{enumerate}

(4) For the purposes of sub-paragraph~(3)($b$)—
\begin{enumerate}\item[]
($a$) “disability” and “illness” have such meaning as may be prescribed; and

($b$) the question whether an illness or disability is long-term shall be determined in accordance with regulations made by the Secretary of State.
\end{enumerate}

(5) For the purposes of sub-paragraph~(3)($d$), the Secretary of State may prescribe—
\begin{enumerate}\item[]
($a$) the meaning of “boarding school fees”; and

($b$) components of such fees (whether or not itemised as such) which are, or are not, to be taken into account,
\end{enumerate}
and may provide for estimating any such component.

\subsection*{\itshape Property or capital transfers}

3.---(1) A variation may be agreed in the circumstances set out in sub-paragraph~(2)  if before 5th April 1993—
\begin{enumerate}\item[]
($a$) a court order of a prescribed kind was in force with respect to the non-resident parent and either the person with care with respect to the application for the maintenance calculation or the child, or any of the children, with respect to whom that application was made; or

($b$) an agreement of a prescribed kind between the non-resident parent and any of those persons was in force.
\end{enumerate}

(2) The circumstances are that in consequence of one or more transfers of property of a prescribed kind and exceeding (singly or in aggregate) a prescribed minimum value—
\begin{enumerate}\item[]
($a$) the amount payable by the non-resident parent by way of maintenance was less than would have been the case had that transfer or those transfers not been made; or

($b$) no amount was payable by the non-resident parent by way of maintenance.
\end{enumerate}

(3) For the purposes of sub-paragraph~(2), “maintenance” means periodical payments of maintenance made (otherwise than under this Act) with respect to the child, or any of the children, with respect to whom the application for a maintenance calculation has been made.

\subsection*{\itshape Additional cases}

4.---(1) The Secretary of State may by regulations prescribe other cases in which a variation may be agreed.

(2) Regulations under this paragraph may, for example, make provision with respect to cases where—
\begin{enumerate}\item[]
($a$) the non-resident parent has assets which exceed a prescribed value;

($b$) a person’s lifestyle is inconsistent with his income for the purposes of a calculation made under Part I of Schedule 1;

($c$) a person has income which is not taken into account in such a calculation;

($d$) a person has unreasonably reduced the income which is taken into account in such a calculation.
\end{enumerate}

\section[Part II --- Regulatory Controls]{Part II\\*Regulatory Controls}

5.---(1) The Secretary of State may by regulations make provision with respect to the variations from the usual rules for calculating maintenance which may be allowed when a variation is agreed.

(2) No variations may be made other than those which are permitted by the regulations.

(3) Regulations under this paragraph may, in particular, make provision for a variation to result in—
\begin{enumerate}\item[]
($a$) a person’s being treated as having more, or less, income than would be taken into account without the variation in a calculation under Part~I of Schedule 1;

($b$) a person’s being treated as liable to pay a higher, or a lower, amount of child support maintenance than would result without the variation from a calculation under that Part.
\end{enumerate}

(4) Regulations may provide for the amount of any special expenses to be taken into account in a case falling within paragraph~2, for the purposes of a variation, not to exceed such amount as may be prescribed or as may be determined in accordance with the regulations.

(5) Any regulations under this paragraph may in particular make different provision with respect to different levels of income.

\medskip

6. The Secretary of State may by regulations provide for the application, in connection with child support maintenance payable following a variation, of paragraph~7(2)  to (7)  of Schedule 1 (subject to any prescribed modifications).

% Sch 4C inserted (prosp) by 1998 c 14 Sch 7 para 54
\part[Schedule 4C --- Decisions and appeals: departure directions and reduced benefit directions etc. --- \emph{1993 scheme only}]{Schedule 4C\\*Decisions and appeals: departure directions and reduced benefit directions etc.\\*\emph{1993 scheme only}}

\renewcommand\parthead{--- Schedule 4C}

\amendment{
Sch. 4C inserted (1.6.99) by the Social Security Act 1998 Sch. 7 para. 54.

Sch. 4C repealed (3.3.03 for 2003 scheme cases) by the Child Support, Pensions and Social Security Act 2000 Sch. 9 Pt. I.
}

\subsection*{\itshape Revision of decisions}

1. Section 16 shall apply in relation to—
\begin{enumerate}\item[]
($a$) any decision of the Secretary of State with respect to a departure direction, a reduced benefit direction or a person’s liability under section 43;

($b$) any decision of the Secretary of State under section 17 as extended by paragraph 2; and

($c$) any decision of an appeal tribunal on a referral under section 28D(1)($b$),
\end{enumerate}
as it applies in relation to any decision of the Secretary of State under section 11,~12 or 17.

\subsection*{\itshape Decisions superseding earlier decisions}

2.---(1) Section 17 shall apply in relation to—
\begin{enumerate}\item[]
($a$) any decision of the Secretary of State with respect to a departure direction, a reduced benefit direction or a person’s liability under section 43;

($b$) any decision of the Secretary of State under section 17 as extended by this sub-paragraph; and

($c$) any decision of an appeal tribunal on a referral under section 28D(1)($b$),
\end{enumerate}
whether as originally made or as revised under section 16 as extended by paragraph~1, as it applies in relation to any decision of the Secretary of State under section 11,~12 or 17, whether as originally made or as revised under section 16.

(2) Section 17 shall apply in relation to any decision of an appeal tribunal under section 20 as extended by paragraph 3 as it applies in relation to any decision of an appeal tribunal under section 20.

\subsection*{\itshape Appeals to appeal tribunals}

3.---(1) Subject to sub-paragraphs (2) and (3), section 20 shall apply—
\begin{enumerate}\item[]
($a$) in relation to a qualifying person who is aggrieved by any decision of the Secretary of State with respect to a departure direction; and

($b$) in relation to any person who is aggrieved by a decision of the Secretary of State—
\begin{enumerate}\item[]
(i) with respect to a reduced benefit direction; or

(ii) with respect to a person’s liability under section 43,
\end{enumerate}
\end{enumerate}
as it applies in relation to a person whose application for a maintenance assessment is refused or to such a person as is mentioned in subsection (2) of section 20.

(2) On an appeal under section 20 as extended by sub-paragraph (1)($a$), the appeal tribunal shall—
\begin{enumerate}\item[]
($a$) consider the matter—
\begin{enumerate}\item[]
(i) as if it were exercising the powers of the Secretary of State in relation to the application in question; and

(ii) as if it were subject to the duties imposed on him in relation to that application;
\end{enumerate}

($b$) have regard to any representations made to it by the Secretary of State; and

($c$) confirm the decision or replace it with such decision as the tribunal considers appropriate.
\end{enumerate}

(3) No appeal shall lie under section 20 as extended by sub-paragraph (1)($b$)(i) unless the amount of the person’s benefit is reduced in accordance with the reduced benefit direction; and the time within which such an appeal may be brought shall run from the date of the notification of the reduction.

(4) In sub-paragraph (1) “qualifying person” means—
\begin{enumerate}\item[]
($a$) the person with care, or absent parent, with respect to whom the current assessment was made; or

($b$) where the application for the current assessment was made under section~7, either of those persons or the child concerned.
\end{enumerate}

\subsection*{\itshape Decisions and appeals dependent on other cases}

4.---(1) Section 28ZA shall also apply where—
\begin{enumerate}\item[]
($a$) a decision falls to be made—
\begin{enumerate}\item[]
(i) with respect to a departure direction, a reduced benefit direction or a person’s liability under section 43, by the Secretary of State; or

(ii) with respect to a departure direction, by an appeal tribunal on a referral under section 28D(1)($b$); and
\end{enumerate}

($b$) an appeal is pending against a decision given with respect to a different direction by a Child Support Commissioner or a court.
\end{enumerate}

(2) Section 28ZA as it applies by virtue of sub-paragraph (1) shall have effect as if the reference in subsection (3) to section 16 were a reference to that section as extended by paragraph 1.

(3) Section 28ZA as it applies by virtue of sub-paragraph (1)($a$)(ii) shall have effect as if—
\begin{enumerate}\item[]
($a$) in subsection (2)—
\begin{enumerate}\item[]
(i) for the words “the Secretary of State” there were substituted the words “the appeal tribunal”; and

(ii) for the word “he”, in both places where it occurs, there were substituted the word “it”; and
\end{enumerate}

($b$) in subsection (3)—
\begin{enumerate}\item[]
(i) for the words “the Secretary of State” there were substituted the words “the appeal tribunal”;

(ii) for the word “he” there were substituted the words “the Secretary of State”; and

(iii) for the word “his” there were substituted the words “the tribunal's”.
\end{enumerate}
\end{enumerate}

\medskip

5.---(1) Section 28ZB shall also apply where—
\begin{enumerate}\item[]
($a$) an appeal is made to an appeal tribunal under section 20 as extended by paragraph 3; and

($b$) an appeal is pending against a decision given in a different case by a Child Support Commissioner or a court.
\end{enumerate}

(2) Section 28ZB as it applies by virtue of sub-paragraph (1) shall have effect as if any reference to section 16 or section 17 were a reference to that section as extended by paragraph 1 or, as the case may be, paragraph 2.

\subsection*{\itshape Cases of error}

6.---(1) Subject to sub-paragraph (2) below, section 28ZC shall also apply where—
\begin{enumerate}\item[]
($a$) the effect of the determination, whenever made, of an appeal to a Child Support Commissioner or the court (“the relevant determination”) is that the adjudicating authority’s decision out of which the appeal arose was erroneous in point of law; and

($b$) after the date of the relevant determination a decision falls to be made by the Secretary of State in accordance with that determination (or would, apart from this paragraph, fall to be so made)—
\begin{enumerate}\item[]
(i) in relation to an application for a departure direction (made after the commencement date);

(ii) as to whether to revise, under section 16 as extended by paragraph~1, a decision (made after the commencement date) in relation to a departure direction, a reduced benefit direction or a person’s liability under section 43; or

(iii) on an application made under section 17 as extended by paragraph~2 before the date of the relevant determination (but after the commencement date) for a decision in relation to a departure direction, a reduced benefit direction or a person’s liability under section 43 to be superseded.
\end{enumerate}
\end{enumerate}

(2) Section 28ZC shall not apply where the decision of the Secretary of State mentioned in sub-paragraph (1)($b$) above—
\begin{enumerate}\item[]
($a$) is one which, but for section 28ZA(2)($a$) as it applies by virtue of paragraph 4(1), would have been made before the date of the relevant determination; or

($b$) is one made in pursuance of section 28ZB(3) or (5) as it applies by virtue of paragraph 5(1).
\end{enumerate}

(3) Section 28ZC as it applies by virtue of sub-paragraph (1) shall have effect as if in subsection (4), in the definition of “adjudicating authority”, at the end there were inserted the words “or, in the case of a decision made on a referral under section 28D(1)($b$), an appeal tribunal”.

(4) In this paragraph “adjudicating authority”, “the commencement date” and “the court” have the same meanings as in section 28ZC.

\part[Schedule 5 --- Consequential amendments]{Schedule 5\\*Consequential amendments}

\renewcommand\parthead{--- Schedule 5}

\amendment{
Paras. 1--4 came into force 1.9.92.

\medskip

Para. 1 repealed (1.10.92) by the Tribunals and Inquiries Act 1992 Sch.~4 Pt.~I.

\medskip

Para. 2 repealed (2.12.99) by the Northern Ireland Act 1998 Sch. 15.
}

% Para 1 repealed (1.10.92) by 1992 c 53 Sch 4 Pt I
%\subsection*{\itshape The Tribunals and Inquiries Act 1971}
%
%1.---(1) In section 7(3)  of the Tribunals and Inquiries Act 1971 (chairmen of certain tribunals to be drawn from panels) after “paragraph” there shall be inserted “4A”.
%
%(2) In Schedule 1 to that Act (tribunals under the general supervision of the Council on Tribunals) the following entry shall be inserted at the appropriate place—
%\begin{quotation}
%\subsection*{\itshape “Child support maintenance}
%
%4A.---($a$) The child support appeal tribunals established under section~21 of the Child Support Act 1991.
%
%($b$) A Child Support Commissioner appointed under section 22 of the Child Support Act 1991 and any tribunal presided over by such a Commissioner.”
%\end{quotation}

% Para 2 repealed (2.12.99) by 1998 c 47 Sch 15
%\subsection*{\itshape The Northern Ireland Constitution Act 1973}
%
%2. In paragraph 9 of Schedule 2 to the Northern Ireland Constitution Act 1973 (certain judicial appointments to be an excepted matter), after the words “for Northern Ireland”, where they first occur, there shall be inserted “the Chief and other Child Support Commissioners for Northern Ireland”.

\subsection*{\itshape The House of Commons Disqualification Act 1975}

3.---(1) The House of Commons Disqualification Act 1975 shall be amended as follows.

(2) In Part I 
of Schedule 1  % Words inserted (4.9.95) by 1995 c 34 Sch 3 para 19(1)
(disqualifying judicial offices), the following entries shall be inserted at the appropriate places— 
\begin{quotation}
“Chief or other Child Support Commissioner (excluding a person appointed under paragraph 4 of Schedule 4 to the Child Support Act 1991).''
\end{quotation}

% Para 3(3) repealed (1.6.99) by 1998 c 14 Sch 8
%(3) In Part III 
%of Schedule 1  % Words inserted (4.9.95) by 1995 c 34 Sch 3 para 19(2)
%(other disqualifying offices), the following entry shall be inserted at the appropriate place— 
%\begin{quotation}
%“Regional or other full-time chairman of a child support appeal tribunal established under section 21 of the Child Support Act 1991.''
%\end{quotation}

\amendment{
Words inserted in para.~3(2), (3) (4.9.95) by the Child Support Act 1995 Sch.~3 para.~19(1), (2).

Para. 3(3) repealed (1.6.99) by the Social Security Act 1998 Sch. 8.
}

\subsection*{\itshape The Northern Ireland Assembly Disqualification Act 1975}

4.---(1) In Part I of 
Schedule 1 to  % Words inserted (4.9.95) by 1995 c 34 Sch 3 para 19(3)
the Northern Ireland Assembly Disqualification Act 1975 (disqualifying judicial offices), the following entries shall be inserted at the appropriate places— 
\begin{quotation}
“Chief or other Child Support Commissioner (excluding a person appointed under paragraph 4 of Schedule 4 to the Child Support Act 1991).''
\end{quotation}

\amendment{
Words inserted in para.~4(1) (4.9.95) by the Child Support Act 1995 Sch.~3 para.~19(3).

\medskip

Paras. 5--8 came into force 5.4.93.
}

\subsection*{\itshape The Family Law (Scotland) Act 1985}

5. In section 4 (amount of aliment) of the Family Law (Scotland) Act 1985, at the end there shall be added—
\begin{quotation}
“(4) Where a court makes an award of aliment in an action brought by or on behalf of a child under the age of 16 years, it may include in that award such provision as it considers to be in all the circumstances reasonable in respect of the expenses incurred wholly or partly by the person having care of the child for the purpose of caring for the child.”
\end{quotation}

\subsection*{\itshape Bankruptcy (Scotland) Act 1985}

6.---(1) The Bankruptcy (Scotland) Act 1985 shall be amended as follows.

(2) In section 32 (vesting of estate and dealings of debtor after sequestration)—
\begin{enumerate}\item[]
($a$) in subsection (3)—
\begin{enumerate}\item[]
(i) after paragraph ($b$)  there shall be inserted—
\begin{quotation}
“($c$) any obligation of his to pay child support maintenance under the Child Support Act 1991,”;
\end{quotation}

(ii) after “relevant obligations” where second occurring there shall be inserted “referred to in paragraphs ($a$)  and ($b$)  above”;
\end{enumerate}

($b$) in subsection (5)  after “Diligence” there shall be inserted “(which, for the purposes of this section, includes the making of a deduction from earnings order under the Child Support Act 1991)”.
\end{enumerate}

(3) In section 37 (effect of sequestration on diligence), in subsection (5A)  for “or a conjoined arrestment order” there is substituted “, a conjoined arrestment order or a deduction from earnings order under the Child Support Act 1991”.

(4) In section 55 (effect of discharge under section 54), in subsection (2)($d$)—
\begin{enumerate}\item[]
($a$) after “being” there shall be inserted “(i)”;

($b$) at the end there shall be inserted— “or
\begin{quotation}
(ii) child support maintenance within the meaning of the Child Support Act 1991 which was unpaid in respect of any period before the date of sequestration of—
\begin{enumerate}\item[]
($aa$) any person by whom it was due to be paid; or

($bb$) any employer by whom it was, or was due to be, deducted under section 31(5)  of that Act.”.
\end{enumerate}
\end{quotation}
\end{enumerate}

\subsection*{\itshape The Insolvency Act 1986}

7. In section 281(5)($b$)  of the Insolvency Act 1986 (effect of discharge of bankrupt), after “family proceedings” there shall be inserted “or under a 
%maintenance assessment 
\emph{maintenance calculation}  % Words substituted by 2000 c 19 s 1(2)(a)
made under the Child Support Act 1991”.

\subsection*{\itshape The Debtors (Scotland) Act 1987}

8.---(1) The Debtors (Scotland) Act 1987 shall be amended as follows.

(2) In section 1(5)  (time to pay directions not competent in certain cases) after paragraph ($c$)  there shall be inserted—
\begin{quotation}
“($cc$) in connection with a liability order within the meaning of the Child Support Act 1991;”.
\end{quotation}

(3) In section 15(3)  (interpretation of Part I), in the definition of “decree or other document”, after “maintenance order” there shall be inserted “, a liability order within the meaning of the Child Support Act 1991”.

(4) In section 54(1)  (maintenance arrestment to be preceded by default) in paragraph ($c$)  for “the aggregate of 3 instalments” there shall be substituted “one instalment”.

(5) In section 72 (effect of sequestration on diligence against earnings)—
\begin{enumerate}\item[]
($a$) in subsection (2)  after “order” there shall be inserted “or deduction from earnings order under the Child Support Act 1991”;

($b$) after subsection (3)  there shall be inserted—
\begin{quotation}
“(3A) Any sum deducted by the employer under such a deduction from earnings order made before the date of sequestration shall be paid to the Secretary of State, notwithstanding that the date of payment will be after the date of sequestration.”;
\end{quotation}

($c$) after subsection (4)  there shall be inserted—
\begin{quotation}
“(4A) A deduction from earnings order under the said Act shall not be competent after the date of sequestration to secure the payment of any amount due by the debtor under a 
%maintenance assessment 
\emph{maintenance calculation}  % Words substituted by 2000 c 19 s 1(2)(a)
within the meaning of that Act in respect of which a claim could be made in the sequestration.”.
\end{quotation}
\end{enumerate}

(6) In section 73(1)  (interpretation of Part III), in the definition of “net earnings”,
\begin{enumerate}\item[]
($a$) in paragraph ($c$)  for “within the meaning of the Wages Councils Act 1979” there shall be substituted “, namely any enactment, rules, deed or other instrument providing for the payment of annuities or lump sums—
\begin{enumerate}\item[]
(i) to the persons with respect to whom the instrument has effect on their retirement at a specified age or on becoming incapacitated at some earlier age, or

(ii) to the personal representatives or the widows, relatives or dependants of such persons on their death or otherwise,
\end{enumerate}
whether with or without any further or other benefit;”; and

($b$) at the end there shall be added—
\begin{quotation}
“($d$) any amount deductible by virtue of a deduction from earnings order which, in terms of regulations made under section~32(4)($c$)  of the Child Support Act 1991, is to have priority over diligences against earnings.”
\end{quotation}
\end{enumerate}

(7) In section 106 (interpretation) in the definition of “maintenance order”—
\begin{enumerate}\item[]
($a$) the word “or” where it appears after paragraph ($g$), shall be omitted; and

($b$) at the end there shall be inserted “or
\begin{quotation}
($j$) a 
%maintenance assessment 
\emph{maintenance calculation}  % Words substituted by 2000 c 19 s 1(2)(a)
within the meaning of the Child Support Act 1991.”.
\end{quotation}
\end{enumerate}

\end{document}