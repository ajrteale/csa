\documentclass[12pt,a4paper]{article}

\newcommand\regstitle{The Social Security (Miscellaneous Amendments) (No.~3) Regulations 2008}

\newcommand\regsnumber{2008/2365}

\title{\regstitle}

\author{S.I.\ 2008 No.\ 2365}

\date{Made
4th September 2008\\
Laid before Parliament
10th September 2008\\
Coming into force
1st October 2008
}

%\opt{oldrules}{\newcommand\versionyear{1993}}
%\opt{newrules}{\newcommand\versionyear{2003}}
%\opt{2012rules}{\newcommand\versionyear{2012}}

\usepackage{csa-regs}

\setlength\headheight{27.61603pt}

%\hbadness=10000

\begin{document}

\maketitle

\enlargethispage{\baselineskip}

\noindent
The Secretary of State for Work and Pensions, in exercise of the powers conferred by sections 30C(3), 30DD(4), 30E(1), 171D(2), 171G(2) and 175(1) of, and paragraph 2(3) of Schedule 7 to, the Social Security Contributions and Benefits Act 1992\footnote{1992 c.~4. Sections 30C and 30E were inserted by section 3(1) of the Social Security (Incapacity for Work) Act 1994 (c.~18) and sections 171D and 171G were inserted by section 6(1) of the same Act. Section 171G(2) is cited because of the meaning given to the word “prescribed”. Section 175(1) was amended by paragraph 29 of Schedule 3 to the Social Security Contributions (Transfer of Functions, etc.)\ Act 1999 (c.~2).}, section 189(4) of Social Security Administration Act 1992\footnote{1992 c.~5.}, sections 1A(3), 30(4) and 57(2) of the Social Security (Recovery of Benefits) Act 1997\footnote{1997 c.~27; section 1A was inserted by section 54 of the Child Maintenance and Other Payments Act 2008.} and sections 47(1)($a$), (2)($a$)  and (3)($e$)  and 53 of the Child Maintenance and Other Payments Act 2008\footnote{2008 c.~6.} makes the following Regulations:

The Social Security Advisory Committee and the Industrial Injuries Advisory Council have agreed that the proposals in respect of regulation 2, 3, 4 and 6 of these Regulations should not be referred to them\footnote{\emph{See} sections 170, 172(1) and 173(1)($b$)  of the Social Security Administration Act 1992.}. 

{\sloppy

\tableofcontents

}

\bigskip

\setcounter{secnumdepth}{-2}

\subsection[1. Citation and commencement]{Citation and commencement}

1.  These Regulations may be cited as the Social Security (Miscellaneous Amendments) (No.~3) Regulations 2008 and shall come into force on 1st October 2008.

\subsection[2. Amendment of the Social Security (General Benefit) Regulations 1982]{Amendment of the Social Security (General Benefit) Regulations 1982}

2.  In regulation 16 of the Social Security (General Benefit) Regulations 1982\footnote{S.I.~1982/1408. The amount in regulation 16 was most recently amended by S.I.~2007/2618.} (earnings level for the purpose of unemployability supplement) for “£4,602” substitute “£4,784”.

\subsection[3. Amendment of the Social Security (Incapacity Benefit) Regulations 1994]{Amendment of the Social Security (Incapacity Benefit) Regulations 1994}

\begin{sloppypar}
3.---(1)  The Social Security (Incapacity Benefit) Regulations 1994\footnote{S.I.~1994/2946. The amount in regulation 8 was most recently amended by S.I.~2007/2618.} are amended as follows.
\end{sloppypar}

(2) In regulation 8 (limit of earnings from councillor’s allowance) for “£88$.$50” substitute “£92$.$00”.

(3) In regulation 21 (disregard of certain pension payments) after sub-paragraph ($b$)  omit “or” and after sub-paragraph ($c$)  add—
\begin{quotation}
“($d$) any payment made under article 14(1)($b$)  or 21(1)($a$)  of the Armed Forces and Reserve Forces (Compensation Scheme) Order 2005\footnote{S.I.~2005/439.}; or

($e$) any pension payment made under an instrument specified in section 639(2) of the Income Tax (Earnings and Pensions) Act 2003\footnote{2003 c.~1.}.”.
\end{quotation}

\subsection[4. Amendment of the Social Security (Incapacity for Work) (General) Regulations 1995]{\sloppy\hbadness=1142 Amendment of the Social Security (Incapacity for Work) (General) Regulations 1995}

4.---(1)  The Social Security (Incapacity for Work) (General) Regulations 1995\footnote{S.I.~1995/311. Regulation 17 was substituted by S.I.~2006/757 and the amount was most recently amended by S.I.~2007/2618.} are amended as follows.

(2) In regulation 17 (exempt work)—
\begin{enumerate}\item[]
($a$) in paragraphs (3) and (4) for “£88$.$50” substitute “£92$.$00”; and

($b$) in paragraph (7) for “Duties undertaken on not more than one day a week” substitute “Duties undertaken on either one full day or two half days a week”.
\end{enumerate}

\subsection[5. Amendment of the Mesothelioma Lump Sum Payments (Conditions and Amounts) Regulations 2008]{Amendment of the Mesothelioma Lump Sum Payments (Conditions and Amounts) Regulations 2008}

5.  In regulation 2(1) of the Mesothelioma Lump Sum Payments (Conditions and Amounts) Regulations 2008\footnote{S.I.~2008/1963.} for sub-paragraph ($c$)  substitute—
\begin{quotation}
“($c$) payment from any government department, authority, body corporate or employer exempted from insurance by or under section 3 of the Employers’ Liability (Compulsory Insurance) Act 1969\footnote{1969 c.~57.} or Article 7 of the Employers’ Liability (Defective Equipment and Compulsory Insurance) (Northern Ireland) Order 1972\footnote{S.I.~1972/963 (N.I.~6); Article 7 was amended by paragraph 1 of Schedule 2 to the Health and Personal Social Services (Northern Ireland) Order 1991 (S.I.~1991/194 (N.I.~1)) and Part~V of Schedule 7 to the Criminal Justice and Police Act 2001 (c.~16).}.”.
\end{quotation}

\subsection[6. Amendment of the Social Security (Recovery of Benefits) (Lump Sum Payments) Regulations 2008]{Amendment of the Social Security (Recovery of Benefits) (Lump Sum Payments) Regulations 2008}

6.---(1)  The Social Security (Recovery of Benefits) (Lump Sum Payments) Regulations 2008\footnote{S.I.~2008/1596.} are amended as follows.

(2) In regulation 9(5) (information contained in certificates) for “paragraph~(3)” substitute “paragraph (4)”.

(3) For paragraph 6 of Schedule 1 (modification of the Social Security (Recovery of Benefits) Act 1997) substitute—
\begin{quotation}
“6.  Where this Schedule applies, section 13 (appeal to Social Security Commissioner) is to apply as if in—
\begin{enumerate}\item[]
($a$) subsection (2)($b$)  there were omitted “of recoverable benefits”;

($b$) subsection (2)($bb$)  for “section 7(2)($a$)” there were substituted “regulation 11(2)($a$)  of the Lump Sum Payment Regulations”; and

($c$) subsection (2)($c$)  for “section 8) the injured person” there were substituted “regulation 12 of the Lump Sum Payments Regulations).”.
\end{enumerate}
\end{quotation}

(4) In paragraph 1($b$)  of Schedule 2 (amendment of the Social Security and Child Support (Decisions and Appeals) Regulations 1999), in the inserted regulation 9ZA (Review of certificates) for paragraph (2)($b$)  substitute—
\begin{quotation}
““certificate” means a certificate of recoverable lump sum payments, including where any of the amounts is nil;”.
\end{quotation}

(5) In paragraph 1($c$)(ii)  of Schedule 2 for “regulation 14” substitute “regulation 13”.

(6) For paragraph 1($e$)  of Schedule 2 substitute—
\begin{quotation}
“($e$) in regulation 33 (making of appeals and applications)—
\begin{enumerate}\item[]
(i) in paragraph (1)($d$)  after “recoverable benefits” insert “,~recoverable lump sum payments”;

(ii) in paragraph (2)($a$)  after “recoverable benefits” insert “or, as the case maybe, recoverable lump sum payments”.”.
\end{enumerate}
\end{quotation}

\bigskip

\pagebreak[3]

Signed 
by authority of the 
Secretary of State for~Work and~Pensions.
%I concur
%By authority of the Lord Chancellor

{\raggedleft
\emph{William D.\ McKenzie}\\*
%Secretary
%Minister
Parliamentary Under-Secretary 
of State\\*Department 
for~Work and~Pensions

}

4th September 2008

\small

\part{Explanatory Note}

\renewcommand\parthead{— Explanatory Note}

\subsection*{(This note is not part of the Regulations)}

These Regulations further amend—
\begin{enumerate}\item[]
(i) the Social Security (General Benefit) Regulations 1982,

(ii) the Social Security (Incapacity Benefit) Regulations 1994,

(iii) the Social Security (Incapacity for Work) (General) Regulations 1995,

(iv) and amend—

(v) the Mesothelioma Lump Sum Payments (Conditions and Amounts) Regulations 2008,and

(vi) the Social Security (Recovery of Benefits) (Lump Sum Payments) Regulations 2008.
\end{enumerate}

Regulation 2 amends regulation 16 of the Social Security (General Benefit) Regulations 1982 to increase the prescribed amount of earnings in a year, for the purposes of Part~I of Schedule 7 to the Social Security Contributions and Benefits Act 1992 (earnings level for unemployability supplement), from £4,602 to £4,784.

Regulation 3 amends regulation 8 of the Social Security (Incapacity Benefit) Regulations 1994 to increase the prescribed amount for the purposes of section 30E(1) of the 1992 Act (net amount of councillor’s allowance in excess of prescribed amount to be deducted from incapacity benefit) from £88$.$50 to £92$.$00. It further amends regulation 21 so that payments under article 14(1)($b$)  or 21(1)($a$)  of the Armed Forces and Reserve Forces (Compensation Scheme) Order 2005 and payments under section 639(2) of the Income Tax (earnings and Pensions) Act 2003 are disregarded when calculating the amount of incapacity benefit due.

Regulation 4 amends regulation 17 of the Social Security (Incapacity for Work) (General) Regulations 1995 so that members of the Disability Living Allowance Advisory Board and members of an appeal tribunal who have a disability qualification are able to work for one day or two half days a week while receiving benefits because of incapacity for work.

It further amends regulation 17 to increase the exempt work limits in paragraphs (3) and (4) from £88$.$50 to £92$.$00.

Regulation 5 amends regulation 2(1)($c$)  of the Mesothelioma Lump Sum Payments (Conditions and Amounts) Regulations 2008 so that a person is disqualified from receiving a lump sum payment where a payment is made by certain bodies or employers exempted from insurance under the specified Northern Ireland Order.

Regulation 6 makes minor amendments to Schedules 1 and 2 to the Social Security (Recovery of Benefits) (Lump Sum Payments) Regulations 2008, modifying, for lump sum payments, the way the Social Security (Recovery of Benefits) Act 1997 applies to appeals to a Social Security Commissioner and the Social Security and the Child Support (Decisions and Appeals) Regulations 1999 apply to reviews of certificates and appeals.

A full impact assessment has not been published for this instrument as it has no impact on the costs of business, charities, private or voluntary sectors. 

\end{document}