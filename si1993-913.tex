\documentclass[a4paper]{article}

\usepackage[welsh,english]{babel}

\usepackage[utf8]{inputenc}
\usepackage[T1]{fontenc}

\usepackage{textcomp}

%\usepackage[2012rules]{optional}

\usepackage[osf]{mathpazo}

%\opt{newrules}{
\title{The Child Support (Miscellaneous Amendments) Regulations 1993}
%}

%\opt{2012rules}{
%\title{Child Maintenance and Other Payments Act 2008\\(2012 scheme version)}
%}

\author{S.I. 1993 No. 913}

\date{Made 29th March 1993\\Coming into force 5th April 1993}

%\opt{oldrules}{\newcommand\versionyear{1993}}
%\opt{newrules}{\newcommand\versionyear{2003}}
%\opt{2012rules}{\newcommand\versionyear{2012}}

\usepackage{fancyhdr}
\pagestyle{fancy}
\fancyhead[L]{}
\fancyhead[C]{\itshape The Child Support (Miscellaneous Amendments) Regulations 1993 (S.I.~1993/913) \parthead%\phantom{...}% (\versionyear{} scheme version)
}
\fancyhead[R]{}
\fancyfoot[C]{\thepage}
\newcommand{\parthead}{}

\usepackage{array}
\usepackage{multirow}
\usepackage[debugshow]{tabulary}
\usepackage{longtable}
\usepackage{multicol}
\usepackage{lettrine}

\usepackage[colorlinks=true]{hyperref}
\usepackage{microtype}

\hyphenation{Aw-dur-dod}
\hyphenation{bank-rupt-cy}
\hyphenation{Ec-cles-ton}
\hyphenation{Eux-ton}
\hyphenation{Hogh-ton}
\hyphenation{Pres-ton}
\hyphenation{Pru-den-tial}
\hyphenation{Riv-ing-ton}

\newcolumntype{x}[1]
	{>{\raggedright}p{#1}}
\newcommand{\tn}{\tabularnewline}
\setlength\tymin{50pt}

\newcommand\amendment[1]{\subsubsection*{Notes}{\itshape\frenchspacing\footnotesize #1 \par}}

\usepackage{perpage} %the perpage package
\MakePerPage{footnote} %the perpage package command
\renewcommand{\thefootnote}{\fnsymbol{footnote}}

\usepackage[perpage,para,symbol]{footmisc}

\begin{document}

\maketitle

\noindent
Whereas a draft of this instrument was laid before Parliament in accordance with section 52(2) of the Child Support Act 1991\footnote{\frenchspacing 1991 c. 48.} and approved by a resolution of each House of Parliament: 

Now, therefore, the Secretary of State for Social Security, in exercise of the powers conferred by sections 3(3)($c$), 12(2) and (3), 14(3), 16(1), 17(6)($b$), 18(11), 30, 32, 34, 41(3), 42, 43, 51, 52(4), 54 and 55(3) of, and paragraphs 5(1) and (2), 14($b$) and 16(5) and (11) of Schedule 1 to, the Child Support Act 1991\footnote{\frenchspacing Section 54 is cited because of the meaning ascribed to the word “prescribed”.} and of all other powers enabling him in that behalf, hereby makes the following Regulations:


{\sloppy

\tableofcontents

}

\setcounter{secnumdepth}{-2}

\subsection[1. Citation, commencement and interpretation]{Citation, commencement and interpretation}

1.—(1) These Regulations, which may be cited as the Child Support (Miscellaneous Amendments) Regulations 1993, shall come into force on 5th April 1993 immediately after the regulations which they amend come into force.

(2) In these Regulations---
\begin{enumerate}\item[]
“Arrears, Interest and Adjustment of Maintenance Assessments Regulations” means the Child Support (Arrears, Interest and Adjustment of Maintenance Assessments) Regulations 1992\footnote{\frenchspacing S.I. 1992/1816.};

“Collection and Enforcement of Other Forms of Maintenance Regulations” means the Child Support (Collection and Enforcement of Other Forms of Maintenance) Regulations 1992\footnote{\frenchspacing S.I. 1992/2643.};

“Collection and Enforcement Regulations” means the Child Support (Collection and Enforcement) Regulations 1992\footnote{\frenchspacing S.I. 1992/1989.};

“Maintenance Arrangements and Jurisdiction Regulations” means the Child Support (Maintenance Arrangements and Jurisdiction) Regulations 1992\footnote{\frenchspacing S.I. 1992/2645.};

“Maintenance Assessments and Special Cases Regulations” means the Child Support (Maintenance Assessments and Special Cases) Regulations 1992\footnote{\frenchspacing S.I. 1992/1815.};

“Maintenance Assessment Procedure Regulations” means the Child Support (Maintenance Assessment Procedure) Regulations 1992\footnote{\frenchspacing S.I. 1992/1813.}.
\end{enumerate}

\subsection[2. Amendment of regulation 5 of the Maintenance Assessment Procedure Regulations]{Amendment of regulation 5 of the Maintenance Assessment Procedure Regulations}

2.—(1) Regulation 5 of the Maintenance Assessment Procedure Regulations shall be amended in accordance with the following provisions of this regulation.

(2) In paragraph (1) for the word “Where” there shall be substituted the words “Subject to paragraph (2A), where”.

(3) In paragraph (2) after the word “shall” there shall be inserted the words “,~subject to paragraph (2A),”.

(4) After paragraph (2) there shall be inserted the following paragraph---
\begin{quotation}
“(2A) The provisions of paragraphs (1) and (2) shall not apply where the Secretary of State is satisfied that an application for a maintenance assessment can be dealt with in the absence of a completed and returned maintenance enquiry form.”.
\end{quotation}

\subsection[3. Amendment of regulation 8 of the Maintenance Assessment Procedure Regulations]{Amendment of regulation 8 of the Maintenance Assessment Procedure Regulations}

3.—(1) Regulation 8 of the Maintenance Assessment Procedure Regulations shall be amended in accordance with the following provisions of this regulation.

(2) After paragraph (1) there shall be inserted the following paragraphs---
\begin{quotation}
“(1A) There shall be two categories of interim maintenance assessment, Category A interim maintenance assessments and Category B interim maintenance assessments. 

(1B) An interim maintenance assessment made by a child support officer shall be---
\begin{enumerate}\item[]
($a$) a Category A interim maintenance assessment, where the information that is required by him as to the income of the absent parent to enable him to make an assessment in accordance with the provisions of Part I of Schedule 1 to the Act has not been provided by that absent parent, and that parent has that information in his possession or can reasonably be expected to acquire it;

($b$) a Category B interim maintenance assessment, where the information that is required by him as to the income of the partner or other member of the family of the absent parent or parent with care to enable him to make an assessment in accordance with the provisions of Part I of Schedule 1 to the Act has not been provided by that partner or other member of the family, and that partner or other member of the family has that information in his possession or can reasonably be expected to acquire it.”.
\end{enumerate}
\end{quotation}

(3) In paragraph (2)---
\begin{enumerate}\item[]
($a$) for the word “an” there shall be substituted the words “a Category A”; and

($b$) after the words “not apply to” there shall be inserted the words “Category A”.
\end{enumerate}

(4) After paragraph (2) there shall be inserted the following paragraphs---
\begin{quotation}
“(2A) The amount of child support maintenance fixed by a Category B interim maintenance assessment shall be determined in accordance with paragraphs (2B) and (2C).

(2B) Where a child support officer is unable to determine the exempt income---
\begin{enumerate}\item[]
($a$) of an absent parent under regulation 9 of the Maintenance Assessments and Special Cases Regulations because he is unable to determine whether regulation 9(2) of those Regulations applies;

($b$) of a parent with care under regulation 10 of those Regulations because he is unable to determine whether regulation 9(2) of those Regulations, as modified by and applied by regulation 10 of those Regulations applies,
\end{enumerate}
the amount of the Category B interim maintenance assessment shall be the maintenance assessment calculated in accordance with Part I of Schedule 1 to the Act on the assumption that---
\begin{enumerate}\item[]
(i) in a case falling within sub-paragraph ($a$), regulation 9(2) of those Regulations does apply;

(ii) in a case falling within sub-paragraph ($b$), regulation 9(2) of those Regulations as modified by and applied by regulation 10 of those Regulations does apply.
\end{enumerate}

(2C) Where the disposable income of an absent parent would, without taking account of the income of any member of his family, bring him within the provisions of paragraph 6 of Schedule 1 to the Act (protected income), and a child support officer is unable to ascertain the disposable income of the other members of his family, the amount of the Category B interim maintenance assessment shall be the maintenance assessment calculated in accordance with Part I of Schedule 1 to the Act on the assumption that the provisions of paragraph 6 of Schedule 1 to the Act do not apply to the absent parent.”.
\end{quotation}

(5) In paragraph (11) for the word “an” there shall be substituted the words “a Category A”.

(6) In paragraph (12) after the words “not apply to” there shall be inserted the words “Category A”.

(7) After paragraph (12) there shall be added the following paragraph---
\begin{quotation}
“(13) In this regulation “family” and “partner” have the same meanings as in the Maintenance Assessments and Special Cases Regulations.”.
\end{quotation}

\subsection[4. Amendment of regulation 9 of the Maintenance Assessment Procedure Regulations]{Amendment of regulation 9 of the Maintenance Assessment Procedure Regulations}

4.—(1) Regulation 9 of the Maintenance Assessment Procedure Regulations shall be amended in accordance with the following provisions of this regulation.

(2) In paragraph (1), for the words “an interim maintenance assessment” there shall be substituted the words “a Category A interim maintenance assessment”.

(3) In paragraph (8) after the words “Regulations 10, 11” there shall be inserted the words “, 24”.

\subsection[5. Amendment of regulation 12 of the Maintenance Assessment Procedure Regulations]{Amendment of regulation 12 of the Maintenance Assessment Procedure Regulations}

5.—(1) Regulation 12 of the Maintenance Assessment Procedure Regulations shall be amended in accordance with the following provisions of this regulation.

(2) In paragraph (1)---
\begin{enumerate}\item[]
($a$) for the words “section 17 or” where they first occur there shall be substituted the words “section 17 of the Act or to make an assessment or a fresh assessment following a review under section”;

($b$) for the comma at the end of sub-paragraph ($c$) there shall be substituted a semi-colon, and after sub-paragraph ($c$) there shall be inserted the following sub-paragraph---
\begin{quotation}
“($d$) where there is a refusal to make an assessment following a review under section 18 of the Act, the applicant,”.
\end{quotation}
\end{enumerate}

(3) In paragraph (2), for the full stop at the end of sub-paragraph ($c$) there shall be substituted a semi-colon, and after sub-paragraph ($c$) there shall be added the following sub-paragraph---
\begin{quotation}
“($d$) where there is a refusal to make an assessment following a review under section 18 of the Act, section 20 of the Act.”.
\end{quotation}

\subsection[6. Amendment of regulation 16 of the Maintenance Assessment Procedure Regulations]{Amendment of regulation 16 of the Maintenance Assessment Procedure Regulations}

6.  In paragraph ($a$) of regulation 16 of the Maintenance Assessment Procedure Regulations, for the words “over the age of 12” there shall be substituted the words “who have attained the age of 12 years”.

\subsection[7. Amendment of regulation 17 of the Maintenance Assessment Procedure Regulations]{Amendment of regulation 17 of the Maintenance Assessment Procedure Regulations}

7.—(1) Regulation 17 of the Maintenance Assessment Procedure Regulations shall be amended in accordance with the following provisions of this regulation.

(2) For paragraphs (1) and (2) there shall be substituted the following paragraphs---
\begin{quotation}
“(1) Subject to regulation 18(1), where a maintenance assessment in force is---
\begin{enumerate}\item[]
($a$) an assessment that has not been previously reviewed;

($b$) a fresh assessment following an earlier review under section 16 of the Act; or

($c$) a fresh assessment following a review under section 17 of the Act,
\end{enumerate}
that assessment shall be reviewed by a child support officer under section 16 of the Act after it has been in force for a period of 52 weeks.

(2) Where a maintenance assessment in force is a fresh assessment following a review under section 18 or 19 of the Act, that assessment shall be reviewed by a child support officer under section 16 of the Act after it has been in force for a period of 52 weeks less the period between the effective date of the previous assessment falling within paragraph (1) above and the effective date of the fresh assessment following the review under section 18 or 19 of the Act.”.
\end{quotation}

(3) In paragraph (7)($a$) for the words “a case is” there shall be substituted the words “the case is one”.

\subsection[8. Amendment of regulation 19 of the Maintenance Assessment Procedure Regulations]{Amendment of regulation 19 of the Maintenance Assessment Procedure Regulations}

8.—(1) Regulation 19 of the Maintenance Assessment Procedure Regulations shall be amended in accordance with the following provisions of this regulation.

(2) In paragraph (2), for the words “paragraphs (3) and (4), and except where the circumstances set out in regulation 17(7) apply” there shall be substituted the words “paragraphs (3), (4) and (4A)”.

(3) After paragraph (4) there shall be inserted the following paragraph---
\begin{quotation}
“(4A) The provisions of paragraph (2) shall not apply in relation to a relevant person where---
\begin{enumerate}\item[]
($a$) the case is one prescribed in regulation 22 or 23 of the Maintenance Assessments and Special Cases Regulations as a case to be treated as a special case for the purposes of the Act;

($b$) there has been a review under section 16 or 17 of the Act in relation to another maintenance assessment in force relating to that person;

($c$) the child support officer concerned has notified that person of the assessments following that review not earlier than 13 weeks prior to the date the child support officer gives notice under paragraph (1); and

($d$) the child support officer has no reason to believe that there has been a change in that person’s circumstances.”.
\end{enumerate}
\end{quotation}

\subsection[9. Amendment of regulation 20 of the Maintenance Assessment Procedure Regulations]{Amendment of regulation 20 of the Maintenance Assessment Procedure Regulations}

9.—(1) Regulation 20 of the Maintenance Assessment Procedure Regulations shall be amended in accordance with the following provisions of this regulation.

(2) In paragraph (1), for the words “paragraphs (2) and (3)” there shall be substituted the words “paragraphs (2) to (4)”.

(3) After paragraph (3) there shall be added the following paragraph---
\begin{quotation}
“(4) Where a child support officer on completing a review under section 17 of the Act determines that---
\begin{enumerate}\item[]
($a$) the absent parent is, by virtue of paragraph 5(4) of Schedule 1 to the Act, to be taken for the purposes of that Schedule to have no assessable income; or

($b$) the case falls within paragraph 7(2) of Schedule 1 to the Act,
\end{enumerate}
he shall make a fresh maintenance assessment.”.
\end{quotation}

\subsection[10. Amendment of regulation 24 of the Maintenance Assessment Procedure Regulations]{Amendment of regulation 24 of the Maintenance Assessment Procedure Regulations}

10.—(1) Regulation 24 of the Maintenance Assessment Procedure Regulations shall be amended in accordance with the following provisions of this regulation.

(2) After paragraph (2) there shall be added the following paragraph---
\begin{quotation}
“(3) Where---
\begin{enumerate}\item[]
($a$) a child support officer refuses an application for a maintenance assessment on the grounds of lack of jurisdiction;

($b$) the applicant makes no application at that stage for that refusal to be reviewed under section 18(1)($a$) of the Act but applies to a court for a maintenance order in relation to the children concerned;

($c$) the court refuses to make a maintenance order on the grounds of lack of jurisdiction; and

($d$) the applicant then makes an application for the refusal mentioned in sub-paragraph ($a$) to be reviewed under section 18(1)($a$) of the Act,
\end{enumerate}
the date the applicant is notified of the court’s decision shall, for the purposes of paragraphs (1) and (2), be treated as the date of notification to the applicant of the decision whose review he seeks.”.
\end{quotation}

\subsection[11. Insertion of regulation 26A into the Maintenance Assessment Procedure Regulations]{Insertion of regulation 26A into the Maintenance Assessment Procedure Regulations}

11.  After regulation 26 of the Maintenance Assessment Procedure Regulations there shall be inserted the following regulation---
\begin{quotation}
\subsection*{“Review under section 18 of the Act where parentage is an issue}

26A.  Where an applicant for a review under section 18 of the Act gives as one, but not the only, reason for making the application that---
\begin{enumerate}\item[]
($a$) the decision of which he seeks the review has been made on the basis that a particular person (whether the applicant or some other person) either is, or is not, a parent of a child in question; and

($b$) the decision should not have been made on that basis,
\end{enumerate}
the Secretary of State shall treat the application as two applications, one relating solely to the issue of parentage and the other relating to all other matters giving rise to the application, and shall proceed accordingly.”.
\end{quotation}

\subsection[12. Insertion of regulation 32A into the Maintenance Assessment Procedure Regulations]{Insertion of regulation 32A into the Maintenance Assessment Procedure Regulations}

12.  After regulation 32 of the Maintenance Assessment Procedure Regulations there shall be inserted the following regulation---
\begin{quotation}
\subsection*{“Cancellation of maintenance assessments made under section 7 of the Act where the child is no longer habitually resident in Scotland}

32A.—(1) Where a maintenance assessment made in response to an application by a child under section 7 of the Act is in force and that child ceases to be habitually resident in Scotland, a child support officer shall cancel that assessment.

(2) In any case where paragraph (1) applies, the assessment shall cease to have effect from the date that the child support officer determines is the date on which the child concerned ceased to be habitually resident in Scotland.”.
\end{quotation}

\subsection[13. Amendment of regulation 40 of the Maintenance Assessment Procedure Regulations]{Amendment of regulation 40 of the Maintenance Assessment Procedure Regulations}

13.  In paragraph (3)($c$) of regulation 40 of the Maintenance Assessment Procedure Regulations the words “, 10D” shall be omitted.

\subsection[14. Amendment of regulation 42 of the Maintenance Assessment Procedure Regulations]{Amendment of regulation 42 of the Maintenance Assessment Procedure Regulations}

14.—(1) Regulation 42 of the Maintenance Assessment Procedure Regulations shall be amended in accordance with the following provisions of this regulation.

(2) In paragraph (1)---
\begin{enumerate}\item[]
($a$) after the words “a direction is in force” there shall be inserted the words “or some other person”;

($b$) in sub-paragraph ($a$), for the word “her” where it first occurs there shall be substituted the words “the parent with care”; and

($c$) in sub-paragraph ($b$), for the word “she” there shall be substituted the words “the parent with care”.
\end{enumerate}

(3) In paragraph (2), after the words “a direction is in force” there shall be inserted the words “or some other person”.

(4) After paragraph (2), there shall be inserted the following paragraphs---
\begin{quotation}
“(2A) Where a direction is in force and the Secretary of State becomes aware that a question arises as to whether the welfare of a child is likely to be affected by the direction continuing to be in force, he shall refer the matter to a child support officer who shall conduct a review to determine whether the direction is to continue or is to cease to be in force.

(2B) Where a direction is in force and a child support officer becomes aware that a question arises as to whether the welfare of a child is likely to be affected by the direction continuing to be in force, a child support officer shall conduct a review to determine whether the direction is to continue or is to cease to be in force.”.
\end{quotation}

(5) In paragraphs (5) and (6), for the words “the parent concerned gave the reasons specified in paragraph (1)” there shall be substituted the words “the reasons specified in paragraph (1) were given”.

(6) Paragraph (7) shall be omitted.

(7) For paragraph (9) there shall be substituted the following paragraphs---
\begin{quotation}
“(9) A parent with care who is aggrieved by a decision of a child support officer following a review may appeal to a child support appeal tribunal against that decision.

(10) Sections 20(2) to (4) and 21 of the Act shall apply in relation to appeals under paragraph (9) as they apply in relation to appeals under section 20 of the Act.

(11) A notification under paragraph (8) shall include information as to the provisions of paragraphs (9) and (10).”.
\end{quotation}

\subsection[15. Amendment of regulation 51 of the Maintenance Assessment Procedure Regulations]{Amendment of regulation 51 of the Maintenance Assessment Procedure Regulations}

15.  In paragraph (1)($b$) of regulation 51 of the Maintenance Assessment Procedure Regulations, after the words “Children Act 1989,” there shall be added the words “except where that person is a parent of such a child and the local authority allow the child to live with that parent under section 23(5) of that Act;”.

\subsection[16. Addition of regulation 57 to the Maintenance Assessment Procedure Regulations]{Addition of regulation 57 to the Maintenance Assessment Procedure Regulations}

16.  After regulation 56 of the Maintenance Assessment Procedure Regulations there shall be added the following regulation–
\begin{quotation}
\subsection*{“Action by the Secretary of State on receipt of an application under section 17 or 18 of the Act where a question as to the entitlement to benefit arises}

57.—(1) Where an application for a review under section 17 or 18 of the Act has been made to the Secretary of State and he is of the opinion that the application gives rise to a question as to the entitlement to benefit of any person, he may disclose the information contained in that application to an adjudication officer or, in the case of housing benefit or council tax benefit, to an appropriate authority.

(2) Where the Secretary of State discloses information under paragraph (1), he need not refer the application to a child support officer earlier than the expiration of a period of 28 days beginning with the date prescribed in paragraph (3).

(3) The date prescribed for the purposes of paragraph (2) is the second day after the date the Secretary of State receives the application for a review under section 17 or 18 of the Act, excluding any Saturday, Sunday, or any day which is a bank holiday in England, Wales, Scotland or Northern Ireland under the Banking and Financial Dealings Act 1971\footnote{\frenchspacing 1971 c. 80.}.

(4) In this regulation---
\begin{enumerate}\item[]
($a$) “benefit” is to be construed in accordance with the benefit Acts;

($b$) “appropriate authority” means–
\begin{enumerate}\item[]
(i) in relation to housing benefit, the housing or local authority concerned; and

(ii) in relation to council tax benefit the billing authority or, in Scotland, the levying authority.”.
\end{enumerate}
\end{enumerate}
\end{quotation}

\subsection[17. Amendment of paragraph 2 of Schedule 1 to the Maintenance Assessment Procedure Regulations]{Amendment of paragraph 2 of Schedule 1 to the Maintenance Assessment Procedure Regulations}

17.  In paragraph 2 of Schedule 1 to the Maintenance Assessment Procedure Regulations, for the word “Technician” in both places where it occurs there shall be substituted the word “Technology”.

\subsection[18. Amendment of paragraph 6 of Schedule 2 to the Maintenance Assessment Procedure Regulations]{Amendment of paragraph 6 of Schedule 2 to the Maintenance Assessment Procedure Regulations}

18.  In sub-paragraph (3) of paragraph 6 of Schedule 2 to the Maintenance Assessment Procedure Regulations, for the words “that child and all other children of the absent parent who are in her care” there shall be substituted the words “one or more children of the absent parent who are in her care, that application shall be treated as an application for a maintenance assessment with respect to all the children of the absent parent who are in her care, and”.

\subsection[19. Amendment of regulation 1 of the Maintenance Assessments and Special Cases Regulations]{Amendment of regulation 1 of the Maintenance Assessments and Special Cases Regulations}

19.—(1) Regulation 1 of the Maintenance Assessments and Special Cases Regulations shall be amended in accordance with the following provisions of this regulation.

(2) In regulation 1(2)---
\begin{enumerate}\item[]
($a$) in the definition of “course of advanced education” for the word “Technician” there shall be substituted the word “Technology”;

($b$) after the definition of “council tax benefit” there shall be inserted the following definition---
\begin{quotation}
““couple” means a married or unmarried couple;”;
\end{quotation}

($c$) after the definition of “Income Support Regulations” there shall be inserted the following definitions---
\begin{enumerate}\item[]
(i) ““Independent Living (1993) Fund” means the charitable trust of that name established by a deed made between the Secretary of State for Social Security of the one part and Robin Glover Wendt and John Fletcher Shepherd of the other part;”; and

(ii) ““Independent Living (Extension) Fund” means the charitable trust of that name established by a deed made between the Secretary of State for Social Security of the one part and Robin Glover Wendt and John Fletcher Shepherd of the other part;”;
\end{enumerate}

($d$) for the definition of “relevant week” there shall be substituted the following definition–
\begin{quotation}
““relevant week” means---
\begin{enumerate}\item[]
($a$) in relation to an application for child support maintenance\hspace{0pt}---
\begin{enumerate}\item[]
(i) in the case of the person making the application, the period of 7 days immediately preceding the date on which the appropriate maintenance assessment application form (being an effective application within the meaning of regulation 2(4) of the Maintenance Assessment Procedure Regulations) is submitted to the Secretary of State;

(ii) in the case of a person to whom a maintenance assessment enquiry form is given or sent as the result of such an application, the period of 7 days immediately preceding the date on which that form is given to him or, as the case may be, the date on which it is treated as having been sent to him under regulation 1(6)($b$) of the Maintenance Assessment Procedure Regulations;
\end{enumerate}

($b$) in relation to a review of a maintenance assessment under section 16 or 17 of the Act, the period of 7 days immediately preceding the date on which a request is made for information or evidence under regulation 17(5) or, as the case may be, regulation 19(2) of the Maintenance Assessment Procedure Regulations;”; and
\end{enumerate}
\end{quotation}

(e) the definition of “the Independent Living Fund” shall be omitted.
\end{enumerate}

(3) After paragraph (2) of that regulation there shall be inserted the following paragraph---
\begin{quotation}
“(2A) Where any provision of these Regulations requires the income of a person to be estimated and that or any other provision of these Regulations requires that the amount of such estimated income is to be taken into account for any purpose after deducting from it a sum in respect of income tax or of primary Class 1 contributions under the Contributions and Benefits Act or of contributions paid by that person towards an occupational or personal pension scheme, then---
\begin{enumerate}\item[]
($a$) the amount to be deducted in respect of income tax shall be calculated by applying to that income the rates of income tax applicable at the effective date less only the personal relief to which that person is entitled under Chapter I of Part VII of the Income and Corporation Taxes Act 1988\footnote{\frenchspacing 1988 c. 50; the definition of “lower rate” was added by the Finance Act 1992 (c. 20), s.9(9).} (personal relief); but if the period in respect of which that income is to be estimated is less than a year, the amount of the personal relief deductible under this sub-paragraph shall be calculated on a pro rata basis;

($b$) the amount to be deducted in respect of Class 1 contributions under the Contributions and Benefits Act shall be calculated by applying to that income the appropriate primary percentage applicable in the relevant week; and

($c$) the amount to be deducted in respect of contributions paid by that person towards an occupational or personal pension scheme shall be one-half of the sums so paid.”.
\end{enumerate}
\end{quotation}

\subsection[20. Amendment of regulation 9 of the Maintenance Assessments and Special Cases Regulations]{Amendment of regulation 9 of the Maintenance Assessments and Special Cases Regulations}

20.  In regulation 9(2)($c$) of the Maintenance Assessments and Special Cases Regulations, after the words “calculated under regulation 7(1)” there shall be inserted the words “(but excluding the amount mentioned in sub-paragraph ($d$) of that regulation)”.

\subsection[21. Amendment of regulation 10 of the Maintenance Assessments and Special Cases Regulations]{Amendment of regulation 10 of the Maintenance Assessments and Special Cases Regulations}

21.  At the end of regulation 10 of the Maintenance Assessments and Special Cases Regulations there shall be added the words “except that paragraphs (3) and (4) of that regulation shall apply only in a case where the parent with care shares day to day care of the child mentioned in those paragraphs with one or more other persons.”.

\subsection[22. Amendment of regulation 15 of the Maintenance Assessments and Special Cases Regulations]{Amendment of regulation 15 of the Maintenance Assessments and Special Cases Regulations}

22.  In regulation 15(5) of the Maintenance Assessments and Special Cases Regulations, for the words “paragraph 63” there shall be substituted the words “paragraphs (1), (2) and (9) of regulation 63”, and for the words “that regulation” there shall be substituted the words “those paragraphs (disregarding any other provision of that regulation)”.

\subsection[23. Amendment of regulation 22 of the Maintenance Assessments and Special Cases Regulations]{Amendment of regulation 22 of the Maintenance Assessments and Special Cases Regulations}

23.  For paragraph (4) of regulation 22 of the Maintenance Assessments and Special Cases Regulations there shall be substituted the following paragraph---
\begin{quotation}
\begin{sloppypar}
“(4) Where the aggregate of the child support maintenance payable by the absent parent is less than the minimum amount prescribed in regulation 13(1), the child support maintenance payable shall be---
\end{sloppypar}
\begin{enumerate}\item[]
($a$) that prescribed minimum amount apportioned between the two or more applications in the same ratio as the maintenance requirements in question bear to each other; or

($b$) where, because of the application of regulation 2(2), such an apportionment produces an aggregate amount which is different from that prescribed minimum amount, that different amount.”.
\end{enumerate}
\end{quotation}

\subsection[24. Amendment of regulation 27 of the Maintenance Assessments and Special Cases Regulations]{Amendment of regulation 27 of the Maintenance Assessments and Special Cases Regulations}

24.  In regulation 27(2) of the Maintenance Assessments and Special Cases Regulations, after the words “modified so” there shall be inserted the word “that”.

\subsection[25. Insertion of regulation 27A into the Maintenance Assessments and Special Cases Regulations]{Insertion of regulation 27A into the Maintenance Assessments and Special Cases Regulations}

25.  After regulation 27 of the Maintenance Assessments and Special Cases Regulations there shall be inserted the following regulation---
\begin{quotation}
\subsection*{“Child who is allowed to live with his parent under section 23(5) of the Children Act 1989}

27A.—(1) Where the circumstances of a case are that a qualifying child who is in the care of a local authority in England and Wales is allowed by the authority to live with a parent of his under section 23(5) of the Children Act 1989\footnote{\frenchspacing 1989 c. 41.}, that case shall be treated as a special case for the purposes of the Act.

(2) For the purposes of this case, section 3(3)($b$) of the Act shall be modified so that for the reference to the person who usually provides day to day care for the child there shall be substituted a reference to the parent of a child whom the local authority allow the child to live with under section 23(5) of the Children Act 1989.”.
\end{quotation}

\subsection[26. Amendment of regulation 28 of the Maintenance Assessments and Special Cases Regulations and insertion of Schedule 5]{Amendment of regulation 28 of the Maintenance Assessments and Special Cases Regulations and insertion of Schedule 5}

26.—(1) In paragraph (1) of regulation 28 of the Maintenance Assessments and Special Cases Regulations---
\begin{enumerate}\item[]
($a$) in sub-paragraph ($b$), after the words “(income support family premium)” there shall be inserted the words “and does not have day to day care of any child (whether or not a relevant child)”; and

($b$) in sub-paragraph ($c$), for the words “he does not satisfy the conditions for entitlement to” there shall be substituted the words “his income does not include”.
\end{enumerate}

(2) After paragraph (2) of that regulation there shall be inserted the following paragraphs---
\begin{quotation}
“(3) Subject to paragraph (4), where an absent parent is liable under section 43 of the Act and this regulation to make payments in place of payments of child support maintenance with respect to two or more qualifying children in relation to whom there is more than one person with care, the prescribed amount mentioned in paragraph (2) shall be apportioned between the persons with care in the same ratio as the maintenance requirements of the qualifying child or children in relation to each of those persons with care bear to each other.

(4) If, in making the apportionment required by paragraph (3), the effect of the application of regulation 2(2) would be such that the aggregate amount payable would be different from the amount prescribed in paragraph (2) the Secretary of State shall adjust that apportionment so as to eliminate that difference; and that adjustment shall be varied from time to time so as to secure that, taking one week with another and so far as is practicable, each person with care receives the amount which she would have received if no adjustment had been made under this paragraph.

(5) The provisions of Schedule 5 shall have effect in relation to cases to which section 43 of the Act and this regulation apply.”.
\end{quotation}

(3) After Schedule 4 to the Maintenance Assessments and Special Cases Regulations there shall be inserted the Schedule set out in the Schedule to these Regulations.

\subsection[27. Amendment of paragraph 3 of Chapter II of Part I of Schedule 1 to the Maintenance Assessments and Special Cases Regulations]{Amendment of paragraph 3 of Chapter II of Part I of Schedule 1 to the Maintenance Assessments and Special Cases Regulations}

27.—(1) In paragraph 3(3) of Chapter II of Part I of Schedule 1 to the Maintenance Assessments and Special Cases Regulations---
\begin{enumerate}\item[]
($a$) at the beginning there shall be inserted the words “Subject to sub-paragraph (7),”; and

($b$) at the beginning of each of heads ($a$) and ($b$) of sub-paragraph (3) there shall be inserted the words “except in a case to which paragraph 4 applies,”.
\end{enumerate}

(2) In sub-paragraph (5) of that paragraph, after the word “calculated” there shall be inserted the words “on the basis of chargeable earnings and”.

(3) In sub-paragraph (6) of that paragraph, after the word “applicable” there shall be inserted the words “to the chargeable earnings”.

(4) After sub-paragraph (6) of that paragraph there shall be inserted the following sub-paragraphs---
\begin{quotation}
“(7) In the case of a self-employed earner whose employment is carried on in partnership or is that of a share fisherman within the meaning of the Social Security (Mariners' Benefits) Regulations 1975\footnote{\frenchspacing  S.I. 1975/470.}, sub-paragraph (3) shall have effect as though it requires a deduction from the earner’s gross receipts of an amount calculated by---
\begin{enumerate}\item[]
($a$) deducting from the gross receipts of the partnership or fishing boat the sums mentioned in heads ($a$) and ($b$) of that sub-paragraph; and

($b$) deducting from the earner’s share of the balance after such deductions the sums mentioned in heads ($c$) to (e) of that sub-paragraph;
\end{enumerate}

(8) In sub-paragraphs (5) and (6) “chargeable earnings” means the gross receipts of the employment less any deductions mentioned in sub-paragraph (3)($a$) and ($b$).”.
\end{quotation}

\subsection[28. Amendment of paragraph 23 of Schedule 2 to the Maintenance Assessments and Special Cases Regulations]{Amendment of paragraph 23 of Schedule 2 to the Maintenance Assessments and Special Cases Regulations}

28.  For sub-paragraph ($a$) of paragraph 23 of Schedule 2 to the Maintenance Assessments and Special Cases Regulations there shall be substituted the following sub-paragraph---
\begin{quotation}
“($a$) payments which are to be taken into account as eligible housing costs under sub-paragraphs ($b$), ($c$), ($d$) and (t) of paragraph 1 of Schedule 3 (eligible housing costs for the purposes of determining exempt income and protected income) and paragraph 3 of that Schedule (exempt income: additional provisions relating to eligible housing costs);”.
\end{quotation}

\subsection[29. Amendment of paragraph 25 of Schedule 2 to the Maintenance Assessments and Special Cases Regulations]{Amendment of paragraph 25 of Schedule 2 to the Maintenance Assessments and Special Cases Regulations}

29.  In sub-paragraph ($a$) of paragraph 25 of Schedule 2 to the Maintenance Assessments and Special Cases Regulations, for the words “the amount referred to in regulation 9(1)($g$)(i)” there shall be substituted the words “the aggregate of the amounts to be taken into account in the calculation of E under regulation 9(1)($g$)”.

\subsection[30. Amendment of paragraph 26 of Schedule 2 to the Maintenance Assessments and Special Cases Regulations]{Amendment of paragraph 26 of Schedule 2 to the Maintenance Assessments and Special Cases Regulations}

30.  In paragraph 26 of Schedule 2 to the Maintenance Assessments and Special Cases Regulations, for the words “regulation 9(1)($g$)(i)” there shall be substituted the words “regulation 9(1)($g$)”.

\subsection[31. Substitution of paragraph 46 of Schedule 2 to the Maintenance Assessments and Special Cases Regulations]{Substitution of paragraph 46 of Schedule 2 to the Maintenance Assessments and Special Cases Regulations}

31.  For paragraph 46 of Schedule 2 to the Maintenance Assessments and Special Cases Regulations there shall be substituted the following paragraph---
\begin{quotation}
“46.  Except in the case of a self-employed earner, payments in kind.”.
\end{quotation}

\subsection[32. Insertion of paragraphs 48A and 48B into Schedule 2 to the Maintenance Assessments and Special Cases Regulations]{Insertion of paragraphs 48A and 48B into Schedule 2 to the Maintenance Assessments and Special Cases Regulations}

32.  After paragraph 48 of Schedule 2 to the Maintenance Assessments and Special Cases Regulations there shall be inserted the following paragraphs---
\begin{quotation}
“48A.  Any guardian’s allowance under Part III of the Contributions and Benefits Act. 

\medskip

48B. Any payment in respect of duties mentioned in paragraph 1(1)(i) of Chapter I of Part I of Schedule 1 relating to a period of one year or more.”.
\end{quotation}

\subsection[33. Amendment of paragraph 1 of Schedule 3 to the Maintenance Assessments and Special Cases Regulations]{Amendment of paragraph 1 of Schedule 3 to the Maintenance Assessments and Special Cases Regulations}

33.  In paragraph 1 of Schedule 3 to the Maintenance Assessments and Special Cases Regulations---
\begin{enumerate}\item[]
($a$) in sub-paragraph ($d$), after the word “home” there shall be inserted the words “, including interest on a loan for any service charge imposed to meet the cost of such repairs and improvements;”; and

($b$) sub-paragraph ($s$) shall be omitted.
\end{enumerate}

\subsection[34. Amendment of Schedule 4 to the Maintenance Assessments and Special Cases Regulations]{Amendment of Schedule 4 to the Maintenance Assessments and Special Cases Regulations}

34.  In head ($c$) of Schedule 4 to the Maintenance Assessments and Special Cases Regulations, for the words “the Independent Living Fund” there shall be substituted the words “the Independent Living (1993) Fund or the Independent Living (Extension) Fund”.

\subsection[35. Amendment of regulation 3 of the Arrears, Interest and Adjustment of Maintenance Assessments Regulations]{Amendment of regulation 3 of the Arrears, Interest and Adjustment of Maintenance Assessments Regulations}

35.  In paragraphs (4), (5) and (6) of regulation 3 of the Arrears, Interest and Adjustment of Maintenance Assessments Regulations after the words “review under section” there shall be inserted the words “16, 17,”.

\subsection[36. Amendment of regulation 4 of the Arrears, Interest and Adjustment of Maintenance Assessments Regulations]{Amendment of regulation 4 of the Arrears, Interest and Adjustment of Maintenance Assessments Regulations}

36.  After paragraph (2) of regulation 4 of the Arrears, Interest and Adjustment of Maintenance Assessments Regulations there shall be added the following paragraph---
\begin{quotation}
“(3) An absent parent who pays all outstanding arrears of interest within 28 days of the due date shall not be liable to make payments of interest with respect to those arrears.”.
\end{quotation}

\subsection[37. Amendment of regulation 5 of the Arrears, Interest and Adjustment of Maintenance Assessments Regulations]{Amendment of regulation 5 of the Arrears, Interest and Adjustment of Maintenance Assessments Regulations}

37.  In regulation 5 of the Arrears, Interest and Adjustment of Maintenance Assessments Regulations for paragraphs (1) and (2) there shall be substituted the following paragraphs---
\begin{quotation}
“(1) The Secretary of State may at any time enter into an agreement with an absent parent (an “arrears agreement”) for the absent parent to pay all outstanding arrears by making payments on agreed dates of agreed amounts.

(2) Where an arrears agreement has been entered into, the Secretary of State shall prepare a schedule of the dates on which payments of arrrears shall be made and the amount to be paid on each such date, and shall send a copy of the schedule to such persons as he thinks fit.”.
\end{quotation}

\subsection[38. Amendment of regulation 6 of the Arrears, Interest and Adjustment of Maintenance Assessments Regulations]{Amendment of regulation 6 of the Arrears, Interest and Adjustment of Maintenance Assessments Regulations}

38.  After paragraph (4) of regulation 6 of the Arrears, Interest and Adjustment of Maintenance Assessments Regulations there shall be added the following paragraph---
\begin{quotation}
“(5) Where any calculation of interest payable under this Part of these Regulations results in a fraction of a penny, that fraction shall be disregarded.”.
\end{quotation}

\subsection[39. Amendment of regulation 12 of the Arrears, Interest and Adjustment of Maintenance Assessments Regulations]{Amendment of regulation 12 of the Arrears, Interest and Adjustment of Maintenance Assessments Regulations}

39.  In regulation 12 of the Arrears, Interest and Adjustment of Maintenance Assessments Regulations paragraphs (3), (4) and (6) shall be omitted.

\subsection[40. Amendment of regulation 13 of the Arrears, Interest and Adjustment of Maintenance Assessments Regulations]{Amendment of regulation 13 of the Arrears, Interest and Adjustment of Maintenance Assessments Regulations}

40.  In regulation 13 of the Arrears, Interest and Adjustment of Maintenance Assessments Regulations---
\begin{enumerate}\item[]
($a$) in paragraphs (1) and (2) the words “or (3)” shall be omitted;

($b$) paragraph (5) shall be omitted; and

($c$) in paragraph (6) for the words “paragraphs (2) to (5)” there shall be substituted the words “paragraphs (2) to (4)”.
\end{enumerate}

\subsection[41. Amendment of regulation 8 of the Collection and Enforcement Regulations]{Amendment of regulation 8 of the Collection and Enforcement Regulations}

41.  In sub-paragraph ($c$) of paragraph (5) of regulation 8 of the Collection and Enforcement Regulations, for the word “lumps” there shall be substituted the word “lump”.

\subsection[42. Amendment of regulation 24 of the Collection and Enforcement Regulations]{Amendment of regulation 24 of the Collection and Enforcement Regulations}

42.  In regulation 24(2) of the Collection and Enforcement Regulations, in the definition of “Attachment of earnings order”, at the end there shall be inserted the words “or under regulation 37 of the Council Tax (Administration and Enforcement) Regulations 1992\footnote{\frenchspacing  S.I. 1992/613.}”.

\subsection[43. Amendment of regulation 29 of the Collection and Enforcement Regulations]{Amendment of regulation 29 of the Collection and Enforcement Regulations}

43.  In regulation 29(4) of the Collection and Enforcement Regulations, for the words “Part I” there shall be substituted the words “Part II”.

\subsection[44. Amendment of regulation 2 of the Collection and Enforcement of Other Forms of Maintenance Regulations]{Amendment of regulation 2 of the Collection and Enforcement of Other Forms of Maintenance Regulations}

44.  In regulation 2($b$) of the Collection and Enforcement of Other Forms of Maintenance Regulations, after the words “maintenance order” there shall be inserted the words “or, in Scotland, registered minutes of agreement”.

\subsection[45. Amendment of regulation 7 of the Maintenance Arrangements and Jurisdiction Regulations]{Amendment of regulation 7 of the Maintenance Arrangements and Jurisdiction Regulations}

45.  After paragraph (3) of regulation 7 of the Maintenance Arrangements and Jurisdiction Regulations there shall be added the following paragraph---
\begin{quotation}
“(4) Where a parent is treated as an absent parent for the purposes of the Act and of the Maintenance Assessments and Special Cases Regulations by virtue of regulation 20 of those Regulations, he shall be treated as an absent parent for the purposes of paragraphs (1) to (3).”.
\end{quotation}

\subsection[46. Amendment of regulation 8 of the Maintenance Arrangements and Jurisdiction Regulations]{Amendment of regulation 8 of the Maintenance Arrangements and Jurisdiction Regulations}

46.  In regulation 8 of the Maintenance Assessments and Jurisdiction Regulations—
\begin{enumerate}\item[]
($a$) in paragraph (1), after sub-paragraph ($a$) there shall be inserted the following sub-paragraph---
\begin{quotation}
“($aa$) the maintenance order has ceased to have effect by virtue of the provisions of regulation 3;”; and
\end{quotation}

($b$) in paragraph (2)---
\begin{enumerate}\item[]
(i) after sub-paragraph ($a$) there shall be inserted the following sub-paragraph---
\begin{quotation}
“($aa$) the maintenance assessment is cancelled or ceases to have effect;”; and
\end{quotation}

(ii) after the words “shall be treated as not having been cancelled” there shall be inserted the words “or, as the case may be, as not having ceased to have effect”.
\end{enumerate}
\end{enumerate}

\bigskip

Signed by authority of the Secretary of State for Social Security.

{\raggedleft
\emph{Alistair Burt}\\*Parliamentary Under-Secretary of State,\\*Department of Social Security

}

29th March 1993

\clearpage

\part*{S C H E D U L E}

\renewcommand\parthead{--- Schedule}

\part[Schedule --- Schedule to be inserted into the Maintenance Assessments and Special Cases Regulations]{Schedule to be inserted into the Maintenance Assessments and Special Cases Regulations}

\begin{quotation}
\part*{``Schedule 5\\*Provisions applying to cases to which section 43 of the Act and regulation 28 apply}

1.  In this Schedule “relevant decision” means a decision of a child support officer given under section 43 of the Act (contribution to maintenance by deduction from benefit) and regulation 28.

\medskip

2.  A relevant decision may be reviewed by a child support officer, either on application by a relevant person or of his own motion, if it appears to him that the absent parent has at some time after that decision was given satisfied the conditions prescribed by regulation 28(1) or, as the case may be, no longer satisfies those conditions.

\medskip

3.  A relevant decision shall be reviewed by a child support officer when it has been in force for 52 weeks.

\medskip

4.—(1) Before conducting a review under paragraph 6 the child support officer shall---
\begin{enumerate}\item[]
($a$) give 14 days' notice of the proposed review to the relevant persons (within the meaning of regulation 1(2) of the Maintenance Assessment Procedure Regulations); and

($b$) invite representations, either in person or in writing, from the relevant persons on any matter relating to the review and set out the provisions of sub-paragraphs (2) to (4) in relation to such representations.
\end{enumerate}

(2) Subject to sub-paragraph (3), where the child support officer conducting the review does not, within 14 days of the date on which notice of the review was given, receive a request from a relevant person to make representations in person, or receives such a request and arranges for an appointment for such representations to be made but that appointment is not kept, he may complete the review in the absence of such representations from that person.

(3) Where the child support officer conducting the review is satisfied that there was good reason for failure to keep an appointment, he shall provide for a further opportunity for the making of representations by the relevant person concerned before he completes the review.

(4) Where the child support officer conducting the review does not receive written representations from a relevant person within 14 days of the date on which notice of the review was given, he may complete the review in the absence of written representations from that person.

\medskip

5.  After completing a review under paragraph 2, 3 or 6, the child support officer shall notify all relevant persons of the result of the review and---
\begin{enumerate}\item[]
($a$) in the case of a review under paragraph 2 or 3, of the right to apply for a further review under paragraph 6; and

($b$) in the case of a review under that paragraph, of the right of appeal under section 20 of the Act as applied by paragraph 8.
\end{enumerate}

\medskip

6.  Where a child support officer has made a decision under regulation 28 or paragraph 2 or 3, any relevant person may apply to the Secretary of State for a review of that decision and, subject to the modifications set out in paragraph 7, the provisions of section 18(5) to (7) of the Act shall apply to such a review.

\medskip

7.  The modifications to the provisions of section 18(5) to (7) of the Act referred to in paragraph 6 are---
\begin{enumerate}\item[]
($a$) any reference in those provisions to a maintenance assessment shall be read as a reference to a relevant decision; and

($b$) subsection (6) shall apply as if the reference to the cancellation of an assessment was omitted.
\end{enumerate}

8.  The provisions of section 20 of the Act (appeals) shall apply in relation to a review or a refusal to review under paragraph 6.

\medskip

9.  The provisions of paragraphs (1) and (2) of regulation 5 of the Child Support (Collection and Enforcement) Regulations 1992\footnote{\frenchspacing  S.I. 1992/1989.} shall apply to the transmission of payments in place of payments of child support maintenance under section 43 of the Act and regulation 28 as they apply to the transmission of payments of child support maintenance.”. 
\end{quotation}


\part{Explanatory Note}

\renewcommand\parthead{--- Explanatory Note}

\subsection*{(This note is not part of the Order)}

 These Regulations amend the Child Support (Maintenance Assessment Procedure) Regulations 1992, the Child Support (Maintenance Assessments and Special Cases) Regulations 1992, the Child Support (Arrears, Interest and Adjustment of Maintenance Assessments) Regulations 1992, the Child Support (Collection and Enforcement) Regulations 1992, the Child Support (Collection and Enforcement of Other Forms of Maintenance) Regulations 1992 and the Child Support (Maintenance Arrangements and Jurisdiction) Regulations 1992, all of which are made under the Child Support Act 1991 (“the Act”).

 The Child Support (Maintenance Assessment Procedure) Regulations are amended in the following respects---

 (1) regulation 5 is amended to remove the obligation on the Secretary of State, where he is satisfied that a maintenance assessment can be dealt with in the absence of a completed and returned maintenance enquiry form, to give notice of an effective application to the relevant persons other than the applicant and to give or send them maintenance enquiry forms (regulation 2);

 (2) regulation 8 is amended to introduce an additional category of interim maintenance assessment where a child support officer is unable to make a maintenance assessment because the partner of an absent parent or of a parent with care fails to provide information about income (regulation 3) and regulation 9 is amended so that its provisions do not apply to this additional category of interim maintenance assessments (regulation 4);

 (3) regulation 17 is amended so that a fresh assessment following a review under section 18 or 19 of the Act does not affect the date of the next periodical review under section 16 of the Act (regulation 7);

 (4) regulation 20 is amended to remove, in two situations, the requirement that for a fresh maintenance assessment to be made following a change of circumstances review under section 17 of the Act, the difference between the previous and fresh assessment has to exceed a minimum amount (regulation 9);

 (5) regulation 24 is amended to extend the time limits to seek review of a child support officer’s refusal to make a maintenance assessment on the grounds of lack of jurisdiction where, on a subsequent application to a court for a maintenance order, the court refuses to make an order on the grounds of lack of jurisdiction (regulation 10);

 (6) a new regulation 26A is inserted: this provides for treating an application for a review under section 18 of the Act where parentage is an issue, but not the only issue, as two separate applications (regulation 11);

 (7) a new regulation 32A is inserted: this provides for the cancellation of a maintenance assessment made under section 7 of the Act where a child is no longer habitually resident in Scotland (regulation 12);

 (8) regulation 42 is amended to provide that a reduced benefit direction may be reviewed where relevant reasons are provided by a person other than a parent with care; and where a question arises as to whether the welfare of a child is likely to be affected by a reduced benefit direction continuing in force (regulation 14);

 (9) regulation 51 is amended, in conjunction with the insertion of a new regulation 27A into the Child Support (Maintenance Assessments and Special Cases) Regulations 1992, so as to include within the category of persons with care a parent of a child whom the local authority allow the child to live with under section 23(5) of the Children Act 1989 (regulations 15 and 25);

 (10) regulation 57 is added to the Regulations: this permits the Secretary of State to disclose information contained in an application for a review under section 17 or 18 of the Act where a question as to the entitlement to benefit arises, and to delay referring the application to a child support officer in these circumstances (regulation 16).

 The Child Support (Maintenance Assessments and Special Cases) Regulations are amended in the following respects---

 (1) in regulation 1 several definitions are amended or substituted and a new paragraph (2A) is inserted which deals with the calculation of income tax and social security contributions (regulation 19);

 (2) a new sub-paragraph (7) is inserted into paragraph 3 of Schedule 1 to provide for the calculation of the earnings of business partners and share fishermen (regulation 27(4));

 (3) two new paragraphs are inserted into Schedule 2 so as to provide that in calculating a parent’s net income any guardian’s allowance under the Social Security (Contributions and Benefits) Act 1992 and any allowances in respect of duties mentioned in paragraph 1(1)(i) of Schedule 1 (auxiliary coastguard, part-time fireman etc.)\ and which relate to a period of a year or more are to be disregarded (regulation 32); and

 (4) paragraph (1)($d$) of Schedule 3 is amended so as to provide that the sum to be taken into account as eligible housing costs are to include any interest on a loan taken out to meet any service charge imposed to meet the cost of the repairs and improvements mentioned in that sub-paragraph (regulation 33).

 The Child Support (Arrears, Interest and Adjustment of Maintenance Assessments) Regulations are amended in the following respects---

 (1) regulation 4 is amended by providing that there be no liability to pay interest on arrears which are paid within 28 days of the payment being missed (regulation 36);

 (2) regulation 5 is amended to remove the requirement that an arrears agreement be in writing, but obliging the Secretary of State to prepare a payments schedule where an arrears agreement has been entered into (regulation 37);

 (3) regulation 12 is amended to remove the jurisdiction of a child support officer to review the calculation of arrears due under a maintenance assessment or of the interest payable with respect to arrears (regulation 39), and consequential amendments are made to regulation 13 (regulation 40).

  The Child Support (Maintenance Arrangements and Jurisdiction) Regulations are amended to provide that the provisions of regulation 7 (cancellation of a maintenance assessment on grounds of lack of jurisdiction) extend to parents who are treated as absent parents for the purposes of the Act (regulation 45). Other minor amendments are also made to the Regulations mentioned in the first paragraph of this note.


\end{document}