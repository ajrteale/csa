\documentclass[12pt,a4paper]{article}

\newcommand\regstitle{The Child Support (Miscellaneous Amendments) Regulations 2013}

\newcommand\regsnumber{2013/1517}

\title{\regstitle}

\author{S.I.\ 2013 No.\ 1517}

\date{Made
18th June 2013\\
Laid before Parliament
27th June 2013\\
Coming into force
in accordance with regulation 1(2) to (4)
}

%\opt{oldrules}{\newcommand\versionyear{1993}}
%\opt{newrules}{\newcommand\versionyear{2003}}
%\opt{2012rules}{\newcommand\versionyear{2012}}

\usepackage{csa-regs}

\setlength\headheight{42.11603pt}

%\hbadness=10000

\begin{document}

\maketitle

\enlargethispage{\baselineskip}

\noindent
The Secretary of State, in exercise of the powers conferred by sections 28G(3), 42, 51(1), 52(4), 54 and 55(1)($b$)  of, and paragraphs 10(1) and (2)($b$)  and 10C(2)($b$)  of Schedule 1 to, the Child Support Act 1991\footnote{1991 c.~48. Section 28G was substituted by section 7 of the Child Support, Pensions and Social Security Act 2000 (c.~19) (“the 2000 Act”). Section 55 was substituted by section 42 of the Child Maintenance and Other Payments Act 2008 (c.~6) (“the 2008 Act”). Schedule 1 was substituted by section 1(3) of, and Schedule 1 to, the 2000 Act. Paragraph 10(1) and (2) of Schedule 1 was amended by section 16 of, and paragraphs 1, 2 and 9 of Schedule 4 to, the 2008 Act and S.I.~2012/2007. Section 54 is cited for the meaning of “prescribed”.} and sections 55(3) and (4) and 57(2) of the Child Maintenance and Other Payments Act 2008\footnote{2008 c.~6.} makes the following Regulations: 

{\sloppy

\tableofcontents

}

\bigskip

\setcounter{secnumdepth}{-2}

\subsection[1. Citation and commencement]{Citation and commencement}

1.---(1)  These Regulations may be cited as the Child Support (Miscellaneous Amendments) Regulations 2013.

(2) This regulation and regulations 2, 3, 5 to 7, 8(4), (5), (7) and (8), 9 and 10 come into force on 30th September 2013.

(3) Regulation 4 comes into force in relation to a case to which the new calculation rules apply on 30th September 2013.

(4) Regulation 8(1) to (3) and (6) comes into force in relation to a case to which the new calculation rules apply on the day on which paragraph~2 of Schedule 4 to the Child Maintenance and Other Payments Act 2008 (calculation by reference to gross weekly income) comes into force for all purposes.

(5) In this regulation, “a case to which the new calculation rules apply” means a case in which liability to pay child support maintenance is calculated in accordance with Part~I of Schedule 1 to the Child Support Act 1991 as amended by paragraph 2 of Schedule 4 to the Child Maintenance and Other Payments Act 2008.

\subsection[2. Amendment of the Child Support (Maintenance Assessment Procedure) Regulations 1992]{Amendment of the Child Support (Maintenance Assessment Procedure) Regulations 1992}

2.---(1)  The Child Support (Maintenance Assessment Procedure) Regulations 1992\footnote{S.I.~1992/1813. The Regulations have been revoked in relation to particular cases by S.I.~2001/157 (as amended by S.I.~2003/328 and 347) and S.I.~2012/2785. For savings, see S.I.~2001/157 (as amended by S.I.~2004/2415).} are amended as follows.

(2) In Schedule 1 (meaning of “child” for the purposes of the Act)\footnote{Schedule 1 was amended by S.I.~2012/2785.}—
\begin{enumerate}\item[]
($a$) for the heading to paragraph 1 substitute—
\begin{quotation}
“Conditions prescribed for the purposes of section 55(1)”;
\end{quotation}

($b$) after paragraph 7 (education otherwise than at a recognised educational establishment)\footnote{Paragraph 7 was inserted by S.I.~2012/2785.} insert—
\begin{quotation}
\section*{\itshape “Person in respect of whom child benefit is payable}

8.  For the purposes of paragraphs 1(3) and 4(2), a person in respect of whom child benefit is payable includes a person in respect of whom an election has been made under section~13A(1) of the Social Security Administration Act 1992 (election not to receive child benefit)\footnote{1992 c.~5. Section 13A was inserted by section 8 of, and paragraph 3 of Schedule 1 to, the Finance Act 2012 (c.~14).} for payments of child benefit not to be made.”.
\end{quotation}
\end{enumerate}

(3) In paragraph 3 of Schedule 2 (multiple applications), after sub-paragraph (14) insert—
\begin{quotation}
“(15) For the purposes of sub-paragraph (12)($c$), where a person has made an election under section 13A(1) of the Social Security Administration Act 1992 (election not to receive child benefit) for payments of child benefit not to be made in respect of a child, that person is to be treated as the person to whom child benefit is being paid in respect of that child.”.
\end{quotation}

\subsection[3. Amendment of the Child Support (Maintenance Assessments and Special Cases) Regulations 1992]{Amendment of the Child Support (Maintenance Assessments and Special Cases) Regulations 1992}

3.  In regulation 1 of the Child Support (Maintenance Assessments and Special Cases) Regulations 1992 (citation, commencement and interpretation)\footnote{S.I.~1992/1815. The Regulations have been revoked in relation to particular cases by S.I.~2001/155 (as amended by S.I.~2003/347) and S.I.~2012/2785. For savings, see S.I.~2001/155 and S.I.~2000/3186 (as amended by S.I.~2004/2415).}, after paragraph (2A) insert—
\begin{quotation}
“(2B) For the purposes of these Regulations, where a person has made an election under section 13A(1) of the Social Security Administration Act 1992 (election not to receive child benefit) for payments of child benefit not to be made—
\begin{enumerate}\item[]
($a$) that person is to be treated as being in receipt of child benefit; and

($b$) the amount of child benefit that would be otherwise paid in respect of the relevant child is to be treated as being in payment.”.
\end{enumerate}
\end{quotation}

\subsection[4. Amendment of the Child Support (Collection and Enforcement) Regulations 1992]{Amendment of the Child Support (Collection and Enforcement) Regulations 1992}

4.---(1)  The Child Support (Collection and Enforcement) Regulations 1992\footnote{S.I.~1992/1989. Regulations 25A to 25AD were inserted by S.I.~2009/1815. Regulations 25C and 25G were amended by S.I.~2012/2785.} are amended as follows.

(2) In regulation 25A(1) (interpretation)—
\begin{enumerate}\item[]
($a$) after the definition of “assessable income” insert—
\begin{quotation}
““current income” has the meaning given in regulation 37 of the Child Support Maintenance Calculation Regulations 2012 (current income---general)\footnote{S.I.~2012/2677.};”;
\end{quotation}

($b$) after the definition of “garnishee order” insert—
\begin{quotation}
““gross weekly income” means income calculated under Chapter I of Part IV of the Child Support Maintenance Calculation Regulations 2012;”;
\end{quotation}

($c$) omit the definition of “net weekly income”.
\end{enumerate}

(3) In regulation 25A (interpretation), omit paragraph (6)($b$).

(4) In regulation 25C(1)($a$)  (maximum deduction rate), omit “in respect of that period”.

(5) In regulation 25G(2)($d$)  (review of a regular deduction order), for “gross weekly” substitute “current”.

\subsection[5. Amendment of the Child Support Departure Direction and Consequential Amendments Regulations 1996]{Amendment of the Child Support Departure Direction and Consequential Amendments Regulations 1996}

5.  In paragraph (8) of regulation 18 of the Child Support Departure Direction and Consequential Amendments Regulations 1996 (costs incurred in supporting certain children)\footnote{S.I.~1996/2907. The Regulations have been revoked in relation to particular cases by S.I.~2001/156 (as amended by S.I.~2003/347) and S.I.~2012/2785. For savings, see S.I.~2001/156 (as amended by S.I.~2003/347) and S.I.~2000/3/186 (as amended by S.I.~2004/2415). Regulation 18(8) was amended by S.I.~1998/58.}, after sub- paragraph ($b$)  insert—
\begin{quotation}
“($c$) where a person has made an election under section 13A(1) of the Social Security Administration Act 1992 (election not to receive child benefit) for payments of child benefit not to be made, the amount of child benefit that would be otherwise paid in respect of the relevant child is to be treated as being payable.”.
\end{quotation}

\subsection[6. Amendment of the Child Support (Maintenance Calculations and Special Cases) Regulations 2000]{Amendment of the Child Support (Maintenance Calculations and Special Cases) Regulations 2000}

6.---(1)  The Child Support (Maintenance Calculations and Special Cases) Regulations 2000\footnote{S.I.~2001/155. The Regulations have been revoked in relation to particular cases by S.I.~2012/2785.} are amended as follows.

(2) For paragraph (3) of regulation 1 (prescription of “relevant other child”) substitute—
\begin{quotation}
“(3) For the purposes of paragraph 10C(2)($b$)  of Schedule 1 to the Act (which provides for other descriptions of relevant other children to be prescribed) “relevant other child” includes a child, other than a qualifying child, in respect of whom the non-resident parent or the non-resident parent’s partner—
\begin{enumerate}\item[]
($a$) would receive child benefit under Part IX of the Contributions and Benefits Act, but in respect of whom they do not do so, solely because the conditions set out in section~146 of that Act (persons outside Great Britain) are not met; or

($b$) has made an election under section 13A(1) of the Social Security Administration Act 1992 (election not to receive child benefit) for payments of child benefit not to be made.”.
\end{enumerate}
\end{quotation}

(3) In paragraph (3) of regulation 8 (persons treated as non-resident parents), the words from ““child benefit” means” to the end become sub-paragraph ($a$), and after that sub-paragraph insert—
\begin{quotation}
“($b$) where a person has made an election under section 13A(1) of the Social Security Administration Act 1992 (election not to receive child benefit) for payments of child benefit not to be made, that person is to be treated as being in receipt of child benefit.”.
\end{quotation}

\subsection[7. Amendment of the Child Support (Maintenance Calculation Procedure) Regulations 2000]{Amendment of the Child Support (Maintenance Calculation Procedure) Regulations 2000}

7.---(1)  The Child Support (Maintenance Calculation Procedure) Regulations 2000\footnote{S.I.~2001/157. The Regulations have been revoked in relation to particular cases by S.I.~2012/2785.} are amended as follows.

(2) In Schedule 1 (meaning of “child” for the purposes of the Act)\footnote{Schedule 1 was amended by S.I.~2012/2785.}—
\begin{enumerate}\item[]
($a$) for the heading to paragraph 1 substitute—
\begin{quotation}
\section*{\itshape “Conditions prescribed for the purposes of section 55(1)”;}
\end{quotation}

($b$) after paragraph 7 (education otherwise than at a recognised educational establishment)\footnote{Paragraph 7 was inserted by S.I.~2012/2785.} insert—
\begin{quotation}
\section*{\itshape “Person in respect of whom child benefit is payable}

8.  For the purposes of paragraphs 1(3) and 4(2), a person in respect of whom child benefit is payable includes a person in respect of whom an election has been made under section~13A(1) of the Social Security Administration Act 1992 (election not to receive child benefit) for payments of child benefit not to be made.”.
\end{quotation}
\end{enumerate}

(3) In paragraph 3 of Schedule 2 (multiple applications), after sub-paragraph (14) insert—
\begin{quotation}
“(15) For the purposes of sub-paragraph (11), where a person has made an election under section 13A(1) of the Social Security Administration Act 1992 (election not to receive child benefit) for payments of child benefit not to be made in respect of a child, that person is to be treated as the person to whom child benefit is being paid in respect of that child.”.
\end{quotation}

(4) In paragraph 3 of Schedule 3 (multiple applications---transitional provisions), after sub-paragraph (14) insert—
\begin{quotation}
“(15) For the purposes of sub-paragraph (11), where a person has made an election under section 13A(1) of the Social Security Administration Act 1992 (election not to receive child benefit) for payments of child benefit not to be made in respect of a child, that person is to be treated as the person to whom child benefit is being paid in respect of that child.”
\end{quotation}

\subsection[8. Amendment of the Child Support Maintenance Calculation Regulations 2012]{Amendment of the Child Support Maintenance Calculation Regulations 2012}

8.---(1)  The Child Support Maintenance Calculation Regulations 2012 are amended as follows.

(2) In regulation 34 (the general rule for determining gross weekly income)—
\begin{enumerate}\item[]
($a$) in paragraph (2)—
\begin{enumerate}\item[]
(i) in sub-paragraph ($b$)  omit “the amount of historic income is nil or”;

(ii) after sub-paragraph ($b$), insert—
\begin{quotation}
“; or

($c$) the Secretary of State is unable, for whatever reason, to request or obtain the required information from HMRC.”;
\end{quotation}
\end{enumerate}

($b$) after paragraph (2), insert—
\begin{quotation}
“(2A) For the purposes of paragraph (2)($a$), current income is to be treated as differing from historic income by an amount that is at least 25\% of historic income where—
\begin{enumerate}\item[]
($a$) the amount of historic income is nil; and

($b$) the amount of current income is greater than nil.”.
\end{enumerate}
\end{quotation}
\end{enumerate}

(3) In paragraph (1)($a$)  of regulation 42 (estimate of current income where insufficient information available)—
\begin{enumerate}\item[]
($a$) after “by virtue of” insert “regulation 34(2)($a$)  where the amount of historic income is nil or by virtue of”;

($b$) after “34(2)($b$)” insert “or ($c$)”; and

($c$) omit “nil or”.
\end{enumerate}

(4) In regulation 50 (parent treated as a non-resident parent in shared care cases), after paragraph (3) insert—
\begin{quotation}
“(4) For the purposes of paragraph (3), where a person has made an election under section 13A(1) of the Social Security Administration Act 1992 (election not to receive child benefit) for payments of child benefit not to be made, that person is to be treated as receiving child benefit.”.
\end{quotation}

(5) In regulation 54 (care provided for relevant other child by a local authority), the existing provision becomes paragraph (1), and after that paragraph insert—
\begin{quotation}
“(2) For the purposes of paragraph (1), where a person has made an election under section 13A(1) of the Social Security Administration Act 1992 (election not to receive child benefit) for payments of child benefit not to be made, that person is to be treated as receiving child benefit.”.
\end{quotation}

(6) In paragraph (2) of regulation 75 (situations in which a variation previously agreed to may be taken into account in calculating maintenance liability) omit from “and the Secretary of State is satisfied” to “ceased to have effect,”.

(7) For the heading to regulation 77 (relevant other child outside Great Britain), substitute “Meaning of “relevant other child” for the purposes of the 1991 Act”.

(8) In regulation 77, the words from “would receive” to the end become paragraph ($a$), and after that paragraph insert—
\begin{quotation}
“; or

($b$) has made an election under section 13A(1) of the Social Security Administration Act 1992 (election not to receive child benefit) for payments of child benefit not to be made.”.
\end{quotation}

\subsection[9. Amendment of the Child Support (Meaning of Child and New Calculation Rules) (Consequential and Miscellaneous Amendment) Regulations 2012]{Amendment of the Child Support (Meaning of Child and New Calculation Rules) (Consequential and Miscellaneous Amendment) Regulations 2012}

9.  In regulation 1(6) of the Child Support (Meaning of Child and New Calculation Rules) (Consequential and Miscellaneous Amendment) Regulations 2012 (citation, commencement and interpretation)\footnote{S.I.~2012/2785.}, for the definition of “arrears of child support maintenance” substitute—
\begin{quotation}
““arrears of child support maintenance” means any payment of child support maintenance—
\begin{enumerate}\item[]
($a$) 
which has become due in relation to a maintenance assessment, or a maintenance calculation made under 2003 scheme rules, and not paid; and

($b$) 
in respect of which the Secretary of State is arranging collection under section 29 of the 1991 Act;”.
\end{enumerate}
\end{quotation}

\subsection[10. Revocations]{Revocations}

10.  Regulation 9 of the Child Support (Meaning of Child and New Calculation Rules) (Consequential and Miscellaneous Amendment) Regulations 2012 is revoked. 

\bigskip

\pagebreak[3]

Signed 
by authority of the 
Secretary of State for~Work and~Pensions.
%I concur
%By authority of the Lord Chancellor

{\raggedleft
\emph{Steve Webb}\\*
%Secretary
Minister
%Parliamentary Under-Secretary 
of State\\*Department 
for~Work and~Pensions

}

18th June 2013

\small

\part{Explanatory Note}

\renewcommand\parthead{— Explanatory Note}

\subsection*{(This note is not part of the Regulations)}

These Regulations contain provisions amending various sets of Child Support Regulations.

Some of the provisions in these Regulations make amendments to child support provisions following changes to legislation which allow a person to make an election for payments of child benefit not to be made in cases where the person or their partner has income of over £50,000 per year. Regulations 2(2)($b$)  and 2(3), 3, 5, 6(3), 7(2)($b$), (3) and (4), and 8(4) and (5) make amendments to child support provisions to ensure that the effect is that a person who has made an election under section 13A(1) of the Social Security Administration Act 1992 (c.~5) for payments of child benefit not to be made is treated as receiving child benefit (or child benefit is treated as being payable) for the purposes of calculating child maintenance. Regulation 6(2) amends the prescription of relevant other child in the Child Support (Maintenance Calculations and Special Cases) Regulations 2000 (S.I.~2001/155), which apply to the 2003 scheme of child support, so that it includes a child in respect of whom an election not to receive child benefit has been made. Regulation 8(7) and (8) makes the same amendment to the Child Support Maintenance Calculation Regulations 2012 (S.I.~2012/2677) (“the 2012 Regulations”) for the purposes of the 2012 scheme.

Regulation 2(2)($a$)  amends the heading of paragraph 1 of Schedule 1 to the Child Support (Maintenance Assessment Procedure) Regulations 1992 (S.I.~1992/1813) to more accurately reflect the contents of that paragraph. Regulation 7(2)($a$)  makes the same amendment to the Child Support (Maintenance Calculation Procedure) Regulations 2000 (S.I.~2001/157).

Regulation 4 makes amendments to the Child Support (Collection and Enforcement) Regulations 1992 (S.I.~1992/1989), which are consequential on a change made in the Child Support (Meaning of Child and New Calculation Rules) (Consequential and Miscellaneous Amendment) Regulations 2012 (S.I.2012/2785) (“the Consequential Regulations”), for the purposes of the 2012 scheme of child support. Regulation 4(2) inserts a definition of “gross weekly income” and “current income”. Regulation 4(3) omits a paragraph referring to net weekly income. Regulation 4(4) amends regulation 25C so that the maximum deduction rate is 40\% of the person’s gross weekly income as calculated for the purposes of the current maintenance calculation or, where it is an arrears only case, the most recent previous calculation. Regulation 4(5) amends regulation 25G so that in an arrears only case the liable person can request a review of the deduction order where there has been a change to current gross income.

Regulation 8(2) amends regulation 34 of the 2012 Regulations so that the non-resident parent’s gross weekly income can be ($a$)  based on historic income in cases where the amount of historic income is nil, and ($b$)  based on current income where the Secretary of State is unable to request or obtain information from HMRC. Regulation 8(3) makes changes consequential on this.

Regulation 8(6) amends regulation 75 of the 2012 Regulations so that a variation previously agreed to which has ceased to have effect for specified reasons can be taken into account again without the need for an application or needing to consider whether there has been a material change of circumstances.

Regulation 9 makes a technical amendment to a definition in the Consequential Regulations.

Regulation 10 revokes regulation 9 of the Consequential Regulations. 

\end{document}