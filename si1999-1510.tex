\documentclass[12pt,a4paper]{article}

\newcommand\regstitle{The Social Security Act 1998 (Commencement No.\ 7 and Consequential and Transitional Provisions) Order 1999}

\newcommand\regsnumber{1999/1510}

%\opt{newrules}{
\title{\regstitle}
%}

%\opt{2012rules}{
%\title{Child Maintenance and Other Payments Act 2008\\(2012 scheme version)}
%}

\author{S.I. 1999 No. 1510 (C.43)}

\date{Made 27th May 1999}

%\opt{oldrules}{\newcommand\versionyear{1993}}
%\opt{newrules}{\newcommand\versionyear{2003}}
%\opt{2012rules}{\newcommand\versionyear{2012}}

\usepackage{csa-regs}

\setlength\headheight{42.07402pt}

\begin{document}

\maketitle

\noindent
The Secretary of State for Social Security, in exercise of the powers conferred on him by sections 79(3) and (4) and 87(2) and (3) of the Social Security Act 1998\footnote{\frenchspacing 1998 c. 14.} and of all other powers enabling him in that behalf, hereby makes the following Order: 

{\sloppy

\tableofcontents

}

\bigskip

\setcounter{secnumdepth}{-2}

\section{Part I}

\renewcommand\parthead{--- Part I}

\subsection[1. Citation and interpretation]{Citation and interpretation}

1.—(1) This Order may be cited as the Social Security Act 1998 (Commencement No.\ 7 and Consequential and Transitional Provisions) Order 1999.

(2) In Part III of this Order a reference to a numbered regulation is to a regulation of the Child Support (Information, Evidence and Disclosure) Regulations 1992\footnote{\frenchspacing S.I. 1992/1812; the relevant amending instruments are S.I. 1995/123, 1995/1045, 1995/3261, 1996/1945, 1996/2907, 1998/58 and 1999/977.} bearing that number.

(3) In Part IV of this Order a reference to a numbered regulation or Schedule is to a regulation of or, as the case may be, a numbered Schedule to the Child Support (Maintenance Assessments and Special Cases) Regulations 1992\footnote{\frenchspacing S.I. 1992/1815; the relevant amending instruments are S.I. 1993/913, 1995/1045, 1995/3261, 1996/3196 and 1998/58.} bearing that number.

(4) In Part V of this Order, unless the context otherwise requires, a reference to a numbered regulation is to a regulation of the Child Support (Arrears, Interest and Adjustment of Maintenance Assessments) Regulations 1992\footnote{\frenchspacing S.I. 1992/1816; the relevant amending instruments are S.I. 1993/913, 1995/1045, 1995/3261, 1996/2907 and 1998/2799.} bearing that number.

(5) In Part VI of this Order a reference to a numbered regulation is to a regulation of the Child Support (Collection and Enforcement) Regulations 1992\footnote{\frenchspacing S.I. 1992/1989; the relevant amending instruments are S.I. 1995/1045 and 1998/2799.} bearing that number.

(6) In Part VIII of this Order a reference to a numbered regulation is to a regulation of the Child Support (Maintenance Arrangements and Jurisdiction) Regulations 1992\footnote{\frenchspacing S.I. 1992/2645, to which there are amendments not relevant to this Order.} bearing that number.

(7) In Part IX of this Order a reference to a numbered regulation is to a regulation of the Child Support (Miscellaneous Amendments and Transitional Provisions) Regulations 1994\footnote{\frenchspacing S.I. 1994/227; the relevant amending instruments are S.I. 1995/1045 and 1995/3261.} bearing that number.

(8) In Part X of this Order a reference to a numbered regulation is to a regulation of the Child Support and Income Support (Amendment) Regulations 1995\footnote{\frenchspacing S.I. 1995/1045.} bearing that number.

(9) In Part XI of this Order a reference to a numbered regulation is to a regulation of the Child Support (Miscellaneous Amendments) (No.\ 2) Regulations 1995\footnote{\frenchspacing S.I. 1995/3261.} bearing that number.

(10) In Part XII of this Order a reference to a numbered regulation is to a regulation of the Child Support (Compensation for Recipients of Family Credit and Disability Working Allowance) Regulations 1995\footnote{\frenchspacing S.I. 1995/3263.} bearing that number.

(11) In Part XVII of this Order a reference to a numbered regulation is to a regulation of the Child Support (Miscellaneous Amendments) Regulations 1999\footnote{\frenchspacing S.I. 1999/977.} bearing that number.

\subsection[2. Appointed day]{Appointed day}

2.  1st June 1999 is the appointed day for the coming into force of the following provisions of the Social Security Act 1998 (in so far as they are not already in force)—
\begin{enumerate}\item[]
($a$) section 1($c$);

($b$) section 4(1)($b$) and (2)($b$);

($c$) sections 5 to 7 and paragraphs 1 to 9 and 11 to 13 of Schedule 1;

($d$) sections 26(8) and 41 to 44;

($e$) section 86(1) in so far as it relates to the provisions referred to in paragraph ($g$);

($f$) section 86(2) and Schedule 8 in so far as they repeal—
\begin{enumerate}\item[]
(i) the entries in Part III of Schedule 1 to the House of Commons Disqualification Act 1975\footnote{\frenchspacing 1975 c. 24.} to which paragraph ($g$)(ii) below applies;

(ii) words in the Debtors (Scotland) Act 1987\footnote{\frenchspacing 1987 c. 18.};

(iii) provisions in the Child Support Act 1991 and the Child Support Act 1995\footnote{\frenchspacing 1995 c. 34.};

(iv) paragraph 3 of Schedule 2 to the Social Security Administration Act 1992\footnote{\frenchspacing 1992 c. 5.};

(v) paragraph 3 of Schedule 2 to the Tribunals and Inquiries Act 1992\footnote{\frenchspacing 1992 c. 53.};

(vi) the entries relating to chairmen of child support appeal tribunals in Part II of Schedule 1 to, and in Schedule 5 to, the Judicial Pensions and Retirement Act 1993\footnote{\frenchspacing 1993 c. 8.}; and

(vii) paragraph 23(1) of Schedule 6 to the Judicial Pensions and Retirement Act 1993 and paragraph 21(1) of Schedule 8 to that Act; and
\end{enumerate}

($g$) in Schedule 7—
\begin{enumerate}\item[]
(i) paragraphs 1 and 2;

(ii) paragraph 4(2) in so far as it applies to the entries relating to regional or other full-time chairmen of child support appeal tribunals, the Chief Child Support Officer and members of a panel appointed under section 6 of the Tribunals and Inquiries Act 1992 of persons to act as chairmen of child support appeal tribunals;

(iii) paragraphs 4(3) and 18 to 45;

(iv) paragraph 47($a$) and sub-paragraph ($b$) in so far as it relates to the definitions of “Chief Child Support Officer”, “child support appeal tribunal” and “child support officer”;

(v) paragraphs 46($b$), 48 and 50 to 54;

(vi) paragraph 118(1) in so far as it substitutes for the words “paragraph 7” the words “paragraph 7($b$)”;

(vii) paragraphs 121(1), 122, 123(1)($b$) and 124(1)($b$); and

(viii) paragraphs 123(2) and 124(2) in so far as they apply to the entries relating to chairmen of child support appeal tribunals in Part II of Schedule 1 to, and in Schedule 5 to, the Judicial Pensions and Retirement Act 1993.
\end{enumerate}
\end{enumerate}

\subsection[3. Date on which consequential amendments have effect]{Date on which consequential amendments have effect}

3.—(1) Subject to paragraph (2), Part XVIII below and the amendments made by the following provisions of this Order shall have effect on 1st June 1999.

(2) The amendment made by article 11($a$)(iii) below shall have effect on 29th November 1999.

\section[Part II --- Amendment of social security provisions]{Part II\\*Amendment of social security provisions}

\renewcommand\parthead{--- Part II}

\subsection[4. Amendment of Schedule 9 to the Social Security (Claims and Payments) Regulations 1987]{Amendment of Schedule 9 to the Social Security (Claims and Payments) Regulations 1987}

4.  In paragraph 7A(1) of Schedule 9 to the Social Security (Claims and Payments) Regulations 1987\footnote{\frenchspacing S.I. 1987/1968; paragraph 7A was inserted by S.I. 1993/478 and sub-paragraph (1) was substituted by S.I. 1993/2113.}, for the words “a child support officer (within the meaning of section 13 of the Child Support Act 1991)” there shall be substituted the words “the Secretary of State”.

\subsection[5. Amendment of regulation 51A of the Family Credit (General) Regulations 1987 and of regulation 56A of the Disability Working Allowance (General) Regulations 1991]{Amendment of regulation 51A of the Family Credit (General) Regulations 1987 and of regulation 56A of the Disability Working Allowance (General) Regulations 1991}

5.—(1) In paragraph (1)($a$) of regulation 51A of the Family Credit (General) Regulations 1987\footnote{\frenchspacing S.I. 1987/1973; regulation 51A was inserted by S.I. 1993/315.} (“the Family Credit Regulations”) and of regulation 56A of the Disability Working Allowance (General) Regulations 1991\footnote{\frenchspacing S.I. 1991/2887; regulation 56A was inserted by S.I. 1993/315.} (“the Disability Working Allowance Regulations”), the words “by a child support officer” shall be omitted.

(2) Paragraph (2)($a$) of regulation 51A of the Family Credit Regulations and paragraph (2)($a$) of regulation 56A of the Disability Working Allowance Regulations shall be omitted.

\section[Part III --- Amendment of the Child Support (Information, Evidence and Disclosure) Regulations 1992]{Part III\\*Amendment of the Child Support (Information, Evidence and Disclosure) Regulations 1992}

\renewcommand\parthead{--- Part III}

\subsection[6. Amendment of regulation 2]{Amendment of regulation 2}

6.  In regulation 2\footnote{\frenchspacing Regulation 2 was amended by S.I. 1995/123, 1995/1045, 1995/3261 and 1996/1945.} (persons under a duty to furnish information or evidence)—
\begin{enumerate}\item[]
($a$) for paragraph (1) there shall be substituted the following paragraph—
\begin{quotation}
“(1) A person falling within a category listed in paragraph (2) shall furnish such information or evidence—
\begin{enumerate}\item[]
($a$) with respect to the matter or matters specified in that paragraph in relation to that category; and

($b$) which is in his possession or which he can reasonably be expected to acquire,
\end{enumerate}
as is required by the Secretary of State to enable a decision to be made under section 11, 12, 16 or 17 of the Act.”; and
\end{quotation}

($b$) in paragraph (2)—
\begin{enumerate}\item[]
(i) in sub-paragraph ($b$), the words “in respect of which a child support officer is conducting or proposing to conduct a review” shall be omitted; and

(ii) in sub-paragraphs ($c$) and ($cc$), the words “and a child support officer is conducting or proposing to conduct a review of that assessment” shall be omitted.
\end{enumerate}
\end{enumerate}

\subsection[7. Amendment of regulation 3]{Amendment of regulation 3}

7.  In regulation 3(1)\footnote{\frenchspacing The relevant amending instruments are S.I. 1995/3261 and 1996/1945.} (purposes for which information or evidence may be required)—
\begin{enumerate}\item[]
($a$) the words “or a child support officer” shall be omitted; and

($b$) in sub-paragraph ($b$), for the words “a child support officer” there shall be substituted the words “the Secretary of State”.
\end{enumerate}

\subsection[8. Substitution of regulation 3A]{Substitution of regulation 3A}

8.  For regulation 3A\footnote{\frenchspacing Regulation 3A was inserted by S.I. 1995/3261.} (contents of request for information or evidence) there shall be substituted the following regulation—
\begin{quotation}
“3A.  Any request by the Secretary of State in accordance with regulations 2 and 3 for the provision of information or evidence shall set out the possible consequences of failure to provide such information or evidence.”.
\end{quotation}

\subsection[9. Amendment of regulation 5]{Amendment of regulation 5}

9.  In regulation 5\footnote{\frenchspacing Regulation 5 was substituted by S.I. 1995/3261.} (time within which information or evidence is to be furnished)—
\begin{enumerate}\item[]
($a$) in paragraph (1)—
\begin{enumerate}\item[]
(i) the words “paragraph (2) and” shall be omitted; and

(ii) for the words “, 6(1) and 17(5)” there shall be substituted the words “and 6(1)”; and
\end{enumerate}

($b$) paragraph (2) shall be omitted.
\end{enumerate}

\subsection[10. Amendment of regulation 6]{Amendment of regulation 6}

10.  In regulation 6 (continuing duty of persons with care) the words “or a child support officer” shall be omitted.

\subsection[11. Amendment of regulation 8]{Amendment of regulation 8}

11.  In regulation 8\footnote{\frenchspacing Regulation 8 was amended by S.I. 1996/2907 and 1998/58.} (disclosure of information to a court or tribunal)—
\begin{enumerate}\item[]
($a$) in paragraph (1)—
\begin{enumerate}\item[]
(i) the words “or a child support officer” shall be omitted;

(ii) for the word “them” there shall be substituted the word “him”; and

(iii) sub-paragraph ($c$) shall be omitted;
\end{enumerate}

($b$) in paragraph (2), for the words “a child support” there shall be substituted the word “an”; and

($c$) in paragraph (3)—
\begin{enumerate}\item[]
(i) the words “or a child support officer” shall be omitted; and

(ii) for the word “them” there shall be substituted the word “him”.
\end{enumerate}
\end{enumerate}

\subsection[12. Amendment of regulation 9A]{Amendment of regulation 9A}

12.  In regulation 9A\footnote{\frenchspacing Regulation 9A was inserted by S.I. 1995/1045 and amended by S.I. 1995/3261, 1996/2907, 1998/58 and 1999/977.} (disclosure of information to other persons)—
\begin{enumerate}\item[]
($a$) the words “or a child support officer” in each place in which they occur shall be omitted;

($b$) in paragraph (1)($a$), for the words “review under section 17 or 18 of the Act” there shall be substituted the words “revision under section 16 of the Act or a decision under section 17 of the Act superseding an earlier decision”;

($c$) in paragraph (2)—
\begin{enumerate}\item[]
(i) in sub-paragraph ($b$), for the words “3A of the Child Support Appeal Tribunals (Procedure) Regulations 1992” there shall be substituted the words “34 of the Social Security and Child Support (Decisions and Appeals) Regulations 1999\footnote{\frenchspacing S.I. 1999/991.}”; and

(ii) in sub-paragraph ($c$)(i), for the word “review” in both places in which it occurs there shall be substituted the words “revision, supersession”; and
\end{enumerate}

($d$) in paragraph (4)($a$), for the words “the child support officer concerned” there shall be substituted the words “the officer concerned who is exercising functions of the Secretary of State under the Act”.
\end{enumerate}

\subsection[13. Revocation of regulations 10 and 10A]{Revocation of regulations 10 and 10A}

13.  Regulations 10 (disclosure of information by the Secretary of State) and 10A\footnote{\frenchspacing Regulations 10 and 10A were substituted for regulation 10 by S.I. 1995/3261 and amended by S.I. 1996/2907.} (disclosure of information by a child support officer) are hereby revoked.

\section[Part IV --- Amendment of the Child Support (Maintenance Assessments and Special Cases) Regulations 1992]{Part IV\\*Amendment of the Child Support (Maintenance Assessments and Special Cases) Regulations 1992}

\renewcommand\parthead{--- Part IV}

\subsection[14. Amendment of regulation 1]{Amendment of regulation 1}

14.—(1) In regulation 1(2)\footnote{\frenchspacing The relevant amending instruments are S.I. 1993/913, 1995/1045, 1995/3261, 1996/3196 and 1998/58.} (interpretation)—
\begin{enumerate}\item[]
($a$) in the definition of “day to day care”—
\begin{enumerate}\item[]
(i) for the words “child support officer, a period other than 12 months but ending with the relevant week” there shall be substituted the words “Secretary of State, a period other than 12 months”; and

(ii) for heads (iii) and (iv) there shall be substituted the following head—
\begin{quotation}
“(iii) in a case where notification is given under regulation 24 of the Maintenance Assessment Procedure Regulations to the relevant persons on different dates, “relevant week” means the period of seven days immediately preceding the date of the latest notification;”;
\end{quotation}
\end{enumerate}

($b$) in the definition of “home”—
\begin{enumerate}\item[]
(i) for the words “child support officer” there shall be substituted the words “Secretary of State”; and

(ii) for the words “that officer” there shall be substituted the words “the Secretary of State”; and
\end{enumerate}

($c$) for the definition of “relevant week” there shall be substituted the following definition—
\begin{quotation}
    ““relevant week” means—
\begin{enumerate}\item[]
    ($a$)
    in relation to an application for child support maintenance—
\begin{enumerate}\item[]
    (i)
    in the case of the applicant, the period of seven days immediately preceding the date on which the appropriate maintenance assessment application form (being an effective application within the meaning of regulation 2(4) of the Maintenance Assessment Procedure Regulations) is submitted to the Secretary of State;

    (ii)
    in the case of a person to whom a maintenance assessment enquiry form is given or sent as the result of such an application, the period of seven days immediately preceding the date on which that form is given or sent to him or, as the case may be, the date on which it is treated as having been given or sent to him under regulation 1(6)($b$) of the Maintenance Assessment Procedure Regulations;
\end{enumerate}

    ($b$)
    where a decision (“the original decision”) is to be—
\begin{enumerate}\item[]
    (i)
    revised under section 16 of the Act; or

    (ii)
    superseded by a decision under section 17 of the Act on the basis that the original decision was made in ignorance of, or was based upon a mistake as to some material fact or was erroneous in point of law,
\end{enumerate}
    the period of seven days which was the relevant week for the purposes of the original decision;

    ($c$)
    where a decision (“the original decision”) is to be superseded by a decision under section 17 of the Act—
\begin{enumerate}\item[]
    (i)
    on an application made for the purpose on the basis that a material change of circumstances has occurred since the original decision was made, the period of seven days immediately preceding the date on which that application was made;

    (ii)
    subject to paragraph ($b$), in a case where a relevant person is given notice under regulation 24 of the Maintenance Assessment Procedure Regulations\footnote{\frenchspacing Regulation 24 was added by S.I. 1999/1047.}, the period of seven days immediately preceding the date of that notification;
\end{enumerate}
\end{enumerate}
    except that where, under paragraph 15 of Schedule 1 to the Act, the Secretary of State makes separate maintenance assessments in respect of different periods in a particular case, because he is aware of one or more changes of circumstances which occurred after the date which is applicable to that case under paragraph ($a$), ($b$) or ($c$) the relevant week for the purposes of each separate assessment made to take account of each such change of circumstances, shall be the period of seven days immediately preceding the date on which notification was given to the Secretary of State of the change of circumstances relevant to that separate maintenance assessment;”. 
\end{quotation}
\end{enumerate}

\subsection[15. Amendment of regulation 2]{Amendment of regulation 2}

15.  In regulation 2(3)\footnote{\frenchspacing There are amendments to regulation 2 which are not relevant to this Order.} (calculation or estimation of amounts), for the words “A child support officer” there shall be substituted the words “The Secretary of State”.

\subsection[16. Amendment of regulations 7, 10A and 20]{Amendment of regulations 7, 10A and 20}

16.  In regulations 7(4)\footnote{\frenchspacing There is an amendment to regulation 7 which is not relevant to this Order.}, 10A(2)($b$)(ii)\footnote{\frenchspacing Regulation 10A was inserted by S.I. 1996/3196.} and 20(2)($b$)(ii), for the words “child support officer” there shall be substituted the words “Secretary of State”.

\subsection[17. Amendment of regulation 19]{Amendment of regulation 19}

17.  In regulation 19\footnote{\frenchspacing The relevant amending instrument is S.I. 1999/977.} (both parents are absent)—
\begin{enumerate}\item[]
($a$) in paragraph (3), the words “or to a child support officer” shall be omitted; and

($b$) in paragraph (4), for the words “child support officer” there shall be substituted the words “Secretary of State”.
\end{enumerate}

\subsection[18. Amendment of regulation 22]{Amendment of regulation 22}

18.  In regulation 22\footnote{\frenchspacing Paragraph (2B) was inserted by S.I. 1998/58.} (multiple applications relating to an absent parent)—
\begin{enumerate}\item[]
($a$) in paragraph (2B) for sub-paragraph ($c$) there shall be substituted the following sub-paragraph—
\begin{quotation}
“($c$) any of those assessments falls to be replaced by a fresh assessment to be made by virtue of a revision under section 16 of the Act or a decision under section 17 of the Act superseding an earlier decision,”; and
\end{quotation}

($b$) in paragraph (2C) for the words “it is not reviewed under any of the provisions set out in” there shall be substituted the words “not within”.
\end{enumerate}

\subsection[19. Amendment of Schedule 1]{Amendment of Schedule 1}

19.  In Schedule 1\footnote{\frenchspacing The relevant amending instruments are S.I. 1995/1045 and 1996/3196.} (calculation of N and M)—
\begin{enumerate}\item[]
($a$) in paragraph 2—
\begin{enumerate}\item[]
(i) in sub-paragraphs (1), (1A) and (4), for the words “child support officer” in each place in which they occur there shall be substituted the words “Secretary of State”; and

(ii) in sub-paragraph (3A), for the words “a child support officer” there shall be substituted the words “the Secretary of State”;
\end{enumerate}

($b$) in paragraph 5—
\begin{enumerate}\item[]
(i) in sub-paragraph (2A), for the words “the officer” there shall be substituted the words “the Secretary of State”; and

(ii) in sub-paragraphs (2A), (3) and (5), for the words “child support officer” in each place in which they occur there shall be substituted the words “Secretary of State”;
\end{enumerate}

($c$) in paragraph 16(6), for the words “child support officer” in both places in which they occur there shall be substituted the words “Secretary of State”;

($d$) in paragraphs 26, 27 and 30, for the words “a child support officer” in each place in which they occur there shall be substituted the words “the Secretary of State”; and

($e$) in paragraphs 27, 30 and 31, for the words “child support officer” there shall be substituted the words “Secretary of State”.
\end{enumerate}

\subsection[20. Amendment of Schedule 3]{Amendment of Schedule 3}

20.  In paragraph 2($k$)\footnote{\frenchspacing Paragraph 2 was amended by S.I. 1995/1045.} of Schedule 3 (eligible housing costs), for the words “child support officer” there shall be substituted the words “Secretary of State”.

\subsection[21. Amendment of Schedule 3A]{Amendment of Schedule 3A}

21.  In Schedule 3A\footnote{\frenchspacing Schedule 3A was inserted by S.I. 1995/1045. There are amendments to Schedule 3A which are not relevant to this Order.} (amount to be allowed in respect of transfers of property)—
\begin{enumerate}\item[]
($a$) in paragraph 2(2)—
\begin{enumerate}\item[]
(i) for the words “a child support officer” there shall be substituted the words “the Secretary of State”; and

(ii) for the words “the officer” there shall be substituted the word “he”; and
\end{enumerate}

($b$) for paragraph 3 there shall be substituted the following paragraph—
\begin{quotation}
“3.—(1) Where an absent parent has notified the Secretary of State that he wishes him to consider whether an amount should be allowed in respect of the relevant value of a qualifying transfer, the Secretary of State shall—
\begin{enumerate}\item[]
($a$) give notice to the other parent of that application; and

($b$) have regard in determining the application to any representations made by the other parent which are received within the period specified in sub-paragraph (2).
\end{enumerate}

(2) The period specified in this sub-paragraph is one month from the date on which the notice referred to in sub-paragraph (1)($a$) above was sent or such longer period as the Secretary of State is satisfied is reasonable in the circumstances of the case.”.
\end{quotation}
\end{enumerate}

\subsection[22. Amendment of Schedule 3B]{Amendment of Schedule 3B}

22.  In paragraphs 1 (in the definition of “work place”), 4($b$), 8(2), 10, 15(2) and 17 of Schedule 3B\footnote{\frenchspacing Schedule 3B was inserted by S.I. 1995/1045 and amended by S.I. 1995/3261.} (amount to be allowed in respect of travelling costs), for the words “child support officer” there shall be substituted the words “Secretary of State”.

\subsection[23. Amendment of Schedule 5]{Amendment of Schedule 5}

23.  Paragraphs 1 to 8 of Schedule 5\footnote{\frenchspacing Schedule 5 was added by S.I. 1993/913 and amended by S.I. 1993/925 and 1995/1045.} (provisions applying to cases to which section 43 of the Act and regulation 28 apply) shall be omitted.

\section[Part V --- Amendment of the Child Support (Arrears, Interest and Adjustment of Maintenance Assessments) Regulations 1992]{Part V\\*Amendment of the Child Support (Arrears, Interest and Adjustment of Maintenance Assessments) Regulations 1992}

\renewcommand\parthead{--- Part V}

\subsection[24. Amendment of regulation 3]{Amendment of regulation 3}

24.  In regulation 3\footnote{\frenchspacing Regulation 3 was amended by S.I. 1993/913.} (liability to make payments of interest with respect to arrears)—
\begin{enumerate}\item[]
($a$) in paragraph (4), for the words “following a review under section 16, 17, 18 or 19 of the Act or” there shall be substituted the words “by virtue of a revision under section 16 of the Act, a decision under section 17 of the Act superseding an earlier decision or of”;

($b$) in paragraph (5), for the words “following a review under section 16, 17, 18 or 19 of the Act or” there shall be substituted the words “made by virtue of a revision under section 16 of the Act, a decision under section 17 of the Act superseding an earlier decision or of”; and

($c$) in paragraph (6), for the words “the review under section 16, 17, 18 or 19 of the Act or an appeal under section 20 of the Act results in” there shall be substituted the words “by virtue of a revision under section 16 of the Act, a decision under section 17 of the Act superseding an earlier decision or of an appeal under section 20 of the Act there is”.
\end{enumerate}

\subsection[25. Amendment of regulation 4]{Amendment of regulation 4}

25.  In regulation 4(2)($b$)\footnote{\frenchspacing Regulation 4 was amended by S.I. 1993/913 and 1995/1045.} (circumstances in which no liability to pay interest arises), the words “or a child support officer” shall be omitted.

\subsection[26. Amendment of regulation 10]{Amendment of regulation 10}

26.  In regulation 10\footnote{\frenchspacing Regulation 10 was substituted by S.I. 1995/1045 and amended by S.I. 1996/2907 and 1998/2799.} (adjustment of the amount payable under a maintenance assessment)—
\begin{enumerate}\item[]
($a$) subject to paragraph ($b$) below, for the words “a child support officer” in each place in which they occur there shall be substituted the words “the Secretary of State”; and

($b$) in paragraph (2), for the words from “revised as a result” to “and a fresh maintenance assessment made” there shall be substituted the words “replaced by a fresh maintenance assessment made by virtue of a revision under section 16 of the Act or of a decision under section 17 of the Act superseding an earlier decision”.
\end{enumerate}

\subsection[27. Amendment of regulation 11]{Amendment of regulation 11}

27.  In regulation 11\footnote{\frenchspacing Regulation 11 was amended by S.I. 1995/1045.} (notifications following a cancellation or adjustment under the provisions of regulation 10)—
\begin{enumerate}\item[]
($a$) in paragraph (1), for the words “a child support officer” there shall be substituted the words “the Secretary of State”; and

($b$) in paragraph (2), for the words from “regulation 12(1)” (where they first occur) to the end there shall be substituted the words “regulations 12 to 15”.
\end{enumerate}

\subsection[28. Substitution of regulations 12 to 15]{Substitution of regulations 12 to 15}

28.  For regulations 12 to 15\footnote{\frenchspacing Regulation 12 was substituted by S.I. 1995/1045 and amended by S.I. 1995/3261. Regulation 13 was amended by S.I. 1993/913, 1995/1045 and 1995/3261.} there shall be substituted the following regulations—
\begin{quotation}
\subsection*{“Extension of the application of Schedule 4C to the Act}

12.  Schedule 4C to the Act is hereby extended so that it applies to any decision with respect to the adjustment of amounts payable under maintenance assessments for the purpose of taking account of overpayments of child support maintenance.

\subsection*{Revision of decisions}

13.—(1) A decision may be revised by the Secretary of State—
\begin{enumerate}\item[]
($a$) if the Secretary of State receives an application for the revision of a decision under section 16 of the Act as extended by regulation 12 above within one month of the date of notification of the decision or within such longer time as may be allowed by regulation 14;

($b$) if the decision arose from an official error;

($c$) if the Secretary of State commences action leading to the revision of a decision within one month of the date of notification of the decision; or

($d$) if the Secretary of State is satisfied that the original decision was erroneous due to a misrepresentation of, or failure to disclose, a material fact and that the decision was more advantageous to the person who misrepresented or failed to disclose that fact than it would otherwise have been but for that error.
\end{enumerate}

(2) In paragraph (1)—
\begin{enumerate}\item[]
    “decision” means a decision of the Secretary of State—
\begin{enumerate}\item[]
    ($a$)
    adjusting the amount payable under a maintenance assessment; or

    ($b$)
    cancelling an adjustment of an amount payable under a maintenance assessment,
\end{enumerate}
    under regulation 10 and a decision superseding such a decision;

    “official error” means an error made by an officer of the Department of Social Security acting as such which no person outside that Department caused or to which no person outside that Department materially contributed. 
\end{enumerate}

(3) Paragraph (1) shall not apply in respect of a change of circumstances which occurred since the date as from which the decision had effect.

\subsection*{Late application for revision}

14.—(1) The period of one month specified in regulation 13(1)($a$) may be extended where the conditions specified in the following provisions of this regulation are satisfied.

(2) An application for an extension of time shall be made by a relevant person or a person acting on his behalf.

(3) An application for an extension of time under this regulation shall—
\begin{enumerate}\item[]
($a$) be made within 13 months of the date on which notification of the decision which it is sought to have revised was given or sent; and

($b$) contain particulars of the grounds on which the extension of time is sought and shall contain sufficient details of the decision which it is sought to have revised to enable that decision to be identified.
\end{enumerate}

(4) An application for an extension of time shall not be granted unless the person making the application or any person acting for him satisfies the Secretary of State that—
\begin{enumerate}\item[]
($a$) it is reasonable to grant the application;

($b$) the application for a revision has merit; and

($c$) special circumstances are relevant to the application for an extension of time and as a result of those special circumstances, it was not practicable for the application for a decision to be revised to be made within one month of the date of notification of the decision which it is sought to have revised.
\end{enumerate}

(5) In determining whether it is reasonable to grant an application for an extension of time, the Secretary of State shall have regard to the principle that the greater the time that has elapsed between the expiration of one month described in regulation 13(1)($a$) from the date of notification of the decision which it is sought to have revised and the making of the application for an extension of time, the more compelling should be the special circumstances on which the application is based.

(6) In determining whether it is reasonable to grant the application for an extension of time, no account shall be taken of the following—
\begin{enumerate}\item[]
($a$) that the person making the application for an extension of time or any person acting for him was unaware of or misunderstood the law applicable to his case (including ignorance or misunderstanding of the time limits imposed by these Regulations); or

($b$) that a Child Support Commissioner or a court has taken a different view of the law from that previously understood and applied.
\end{enumerate}

(7) An application under this regulation for an extension of time which has been refused may not be renewed.

\subsection*{Date from which revised decision takes effect}

15.  Where the date as from which a decision took effect is found to be erroneous on a revision under section 16 of the Act as extended by regulation 12 above, the revision shall take effect as from the date on which the revised decision would have taken effect had the error not been made.

\subsection*{Supersession of decisions}

16.—(1) For the purposes of section 17 of the Act as extended by regulation 12 above, the cases and circumstances in which a decision adjusting the amount payable under a maintenance assessment may be superseded by a decision under that section as extended are set out in paragraphs (2) to (4).

(2) A decision may be superseded by a decision made by the Secretary of State acting on his own initiative where he is satisfied that the decision—
\begin{enumerate}\item[]
($a$) is one in respect of which there has been a material change of circumstances since the decision was made; or

($b$) was made in ignorance of, or was based upon a mistake as to, some material fact.
\end{enumerate}

(3) A decision may be superseded by a decision made by the Secretary of State where an application is made on the basis that—
\begin{enumerate}\item[]
($a$) there has been a change of circumstances since the decision was made and the Secretary of State is satisfied that the change of circumstances is or would be material; or

($b$) the decision was made in ignorance of, or was based upon a mistake as to, a fact and the Secretary of State is satisfied that the fact is or would be material.
\end{enumerate}

(4) A decision, other than a decision given on appeal, may be superseded by a decision made by the Secretary of State—
\begin{enumerate}\item[]
($a$) acting on his own initiative where he is satisfied that the decision was erroneous in point of law; or

($b$) where an application is made on the basis that the decision was erroneous in point of law.
\end{enumerate}

(5) The cases and circumstances in which a decision may be superseded under section 17 of the Act as extended by regulation 12 above shall not include any case or circumstance in which a decision may be revised.

\subsection*{Application of regulations 1(6), 10(3) and 53 of the Maintenance Assessment Procedure Regulations}

17.—(1) The provisions of regulation 10(3) of the Maintenance Assessment Procedure Regulations shall apply to any notification—
\begin{enumerate}\item[]
($a$) under regulation 11; and

($b$) of a decision under the provisions of regulation 13, 14 or 16.
\end{enumerate}

(2) Regulations 1(6) and 53 of the Maintenance Assessment Procedure Regulations shall apply to the provisions of these Regulations.”.
\end{quotation}

\section[Part VI --- Amendment of the Child Support (Collection and Enforcement) Regulations 1992]{Part VI\\*Amendment of the Child Support (Collection and Enforcement) Regulations 1992}

\renewcommand\parthead{--- Part VI}

\subsection[29. Amendment of regulation 11]{Amendment of regulation 11}

29.  In regulation 11\footnote{\frenchspacing Paragraph (4) of regulation 11 was added by S.I. 1995/1045 and amended by S.I. 1998/2799.} (protected earnings rate), for paragraph (4) there shall be substituted the following paragraph—
\begin{quotation}
“(4) Where there is a liability to make payments of child support maintenance but no maintenance assessment is in force—
\begin{enumerate}\item[]
($a$) in a case where the last maintenance assessment was a Category A or Category C interim maintenance assessment, the protected earnings rate shall be the amount which would be produced by the application of the provisions of paragraph (3) if a Category A or Category C interim maintenance assessment were in force;

($b$) subject to sub-paragraph ($a$), in a case where the absent parent provides sufficient evidence to satisfy the Secretary of State that his circumstances have changed since the last occasion on which his exempt income was calculated for the purposes of a decision under the Act, the protected earnings rate shall be the exempt amount as it would be calculated in consequence of that change of circumstances if regulation 9 of the Child Support (Maintenance Assessments and Special Cases) Regulations 1992\footnote{\frenchspacing S.I. 1992/1815; regulation 9 was amended by S.I. 1993/913, 1995/1045, 1995/3261, 1996/1803, 1996/1945, 1996/2907 and 1998/58.} applied in his case; and

($c$) in any other case, the protected earnings rate shall be the amount of the liable person’s exempt income as it was on the last occasion that amount was calculated for the purposes of a decision under the Act.”.
\end{enumerate}
\end{quotation}

\section[Part VII --- Amendment of the Child Support Act 1991 (Commencement No.\ 3 and Transitional Provisions) Order 1992]{Part VII\\*Amendment of the Child Support Act 1991 (Commencement No.\ 3 and Transitional Provisions) Order 1992}

\renewcommand\parthead{--- Part VII}

\subsection[30. Amendment of the Schedule to the Child Support Act 1991 (Commencement No. 3 and Transitional Provisions) Order 1992]{Amendment of the Schedule to the Child Support Act 1991 (Commencement No. 3 and Transitional Provisions) Order 1992}

30.  In Part II of the Schedule to the Child Support Act 1991 (Commencement No.\ 3 and Transitional Provisions) Order 1992\footnote{\frenchspacing S.I. 1992/2644; the relevant amending instrument is S.I. 1993/966.}—
\begin{enumerate}\item[]
($a$) in paragraph 6, in the definition of “formula amount” and in paragraph 8, for the words “consequent on a review under section 17, 18 or 19 of the Act” there shall be substituted the words “by virtue of a revision under section 16 of the Act or a decision under section 17 of the Act superseding an earlier decision”;

($b$) in paragraph 10, for the words “a child support officer” there shall be substituted the words “the Secretary of State”; and

($c$) in paragraph 12—
\begin{enumerate}\item[]
(i) in sub-paragraph (1), for the words “there is a review of a previous assessment under section 17 of the Act (reviews on change of circumstances)” there shall be substituted the words “a decision is made under section 17 of the Act which supersedes an earlier decision on the ground that there has been a material change of circumstances since the decision took effect”;

(ii) in sub-paragraph (2), for the words “child support officer determines that, were a fresh assessment to be made as a result of the review” there shall be substituted the words “Secretary of State determines that, were a fresh assessment to be made by virtue of a decision under section 17 of the Act superseding an earlier decision”; and

(iii) in sub-paragraph (3), for the words “child support officer” there shall be substituted the words “Secretary of State” and for the words “20 to” there shall be substituted the words “21 and”.
\end{enumerate}
\end{enumerate}

\section[Part VIII --- Amendment of the Child Support (Maintenance Arrangements and Jurisdiction) Regulations 1992]{Part VIII\\*Amendment of the Child Support (Maintenance Arrangements and Jurisdiction) Regulations 1992}

\renewcommand\parthead{--- Part VIII}

\subsection[31. Amendment of regulations 3 to 5, 7 and 8]{Amendment of regulations 3 to 5, 7 and 8}

31.—(1) In regulations 3(4), 4(3), 5(1) and 7(1) and (3), for the words “a child support officer” in each place in which they occur there shall be substituted the words “the Secretary of State”.

(2) In the heading to regulation 5, for the words “child support officers” there shall be substituted the words “the Secretary of State”.

(3) In regulations 7(3) and 8(1)($c$), for the words “the child support officer” there shall be substituted the words “the Secretary of State”.

\section[Part IX --- Amendment of the Child Support (Miscellaneous Amendments and Transitional Provisions) Regulations 1994]{Part IX\\*Amendment of the Child Support (Miscellaneous Amendments and Transitional Provisions) Regulations 1994}

\renewcommand\parthead{--- Part IX}

\subsection[32. Amendment of regulation 6]{Amendment of regulation 6}

32.  In regulation 6—
\begin{enumerate}\item[]
($a$) in the definition of “transitional period” in paragraph (1), for the words “is reviewed” there shall be substituted the words “was reviewed or, as the case may be, a decision is made superseding an earlier decision,”; and

($b$) in paragraph (2), for the words “child support officer” there shall be substituted the words “Secretary of State”.
\end{enumerate}

\subsection[33. Amendment of regulation 10]{Amendment of regulation 10}

33.  In regulation 10 (procedure)—
\begin{enumerate}\item[]
($a$) in paragraph (1), after the words “of the Act” there shall be inserted the words “before 1st June 1999 or an application on or after that date for a decision under section 17 of the Act superseding an earlier decision”; and

($b$) for paragraph (3) there shall be substituted the following paragraph—
\begin{quotation}
“(3) Regulation 10(2) of the Procedure Regulations shall not apply in respect of a decision made solely for the purpose of applying Part III of these Regulations but instead the Secretary of State shall notify the relevant persons (as defined in regulation 1(2) of the Procedure Regulations) of the detail of how the provisions of Part III of these Regulations have been applied in that case.”.
\end{quotation}
\end{enumerate}

\subsection[34. Substitution of regulation 11]{Substitution of regulation 11}

34.  For regulation 11\footnote{\frenchspacing Regulation 11 was amended by S.I. 1995/1045 and 1995/3261.} (reviews) there shall be substituted the following regulation—
\begin{quotation}
\subsection*{“Revision and supersession}

11.—(1) The provisions of the following paragraphs shall apply where the Secretary of State proposes to make a decision under section 16 (revision of decisions) or 17 (decisions superseding earlier decisions) of the Act with respect to a maintenance assessment under which the amount payable was the transitional amount.

(2) Where a fresh maintenance assessment would be made by virtue of a decision under section 16 or 17 of the Act and the amount payable under that assessment (disregarding the provisions of Part III of these Regulations) (in this regulation called “the new formula amount”) would be—
\begin{enumerate}\item[]
($a$) more than the formula amount, the amount of child support maintenance payable shall be the transitional amount plus the difference between the formula amount and the new formula amount;

($b$) less than the formula amount but more than the transitional amount, the amount of the child support maintenance payable shall be the transitional amount;

($c$) less than the transitional amount, the amount of child support maintenance payable shall be the new formula amount.
\end{enumerate}

(3) Regulations 21 and 22 of the Procedure Regulations shall apply as if the new formula amount were the amount which would be fixed in accordance with a decision superseding an earlier decision.

(4) Where the effective date of a fresh maintenance assessment made by virtue of a revision under section 16 of the Act or of a decision under section 17 of the Act superseding an earlier decision would, apart from this regulation, be before 18th April 1995—
\begin{enumerate}\item[]
($a$) the fresh maintenance assessment; and

($b$) the decision under section 16 or, as the case may be, section 17,
\end{enumerate}
shall have effect as from 18th April 1995.”.
\end{quotation}

\subsection[35. Substitution of regulation 12]{Substitution of regulation 12}

35.  For regulation 12 (reviews consequent on the amendments made by Part II) there shall be substituted the following regulation—
\begin{quotation}
\subsection*{“Decisions consequent on the amendments made by Part II}

12.—(1) A fresh maintenance assessment shall not be made by virtue of a decision under section 17 of the Act superseding an earlier decision in consequence only of the amendments made by Part II of these Regulations where the amount of child support maintenance fixed by the assessment currently in force and the amount that would be fixed if a fresh assessment were to be made under that section is less than £1.00 a week.

(2) Except in relation to the amendment made by regulation 4(8) above, where a fresh maintenance assessment is made by virtue of a decision under section 17 of the Act superseding an earlier decision in consequence only of the amendments made by Part II of these Regulations, the date as from which—
\begin{enumerate}\item[]
($a$) the fresh maintenance assessment; and

($b$) the decision under section 16 or, as the case may be, section 17,
\end{enumerate}
shall have effect shall be 7th February 1994.”.
\end{quotation}

\subsection[36. Revocation of regulations 13 and 14]{Revocation of regulations 13 and 14}

36.  Regulations 13 (reviews consequent on the provisions of Part III) and 14 (notification) are hereby revoked.

\section[Part X --- Amendment of the Child Support and Income Support (Amendment) Regulations 1995]{Part X\\*Amendment of the Child Support and Income Support (Amendment) Regulations 1995}

\renewcommand\parthead{--- Part X}

\subsection[37. Amendment of regulation 63]{Amendment of regulation 63}

37.  In regulation 63 (reviews consequent upon the amendments made by these Regulations)—
\begin{enumerate}\item[]
($a$) in paragraph (1) for the words from the beginning to “result of the review is—” there shall be substituted the following words—
\begin{quotation}
    “Subject to paragraph (3), a decision with respect to a maintenance assessment in force on 13th April 1995 or 18th April 1995 shall not be superseded by a decision under section 17 of the Act if the difference between the amount of child support maintenance currently in force and the amount that would be fixed if the fresh assessment were to be made as a result of a supersession is—”; 
\end{quotation}

($b$) in paragraph (3)—
\begin{enumerate}\item[]
(i) for the words “review which is made” there shall be substituted the words “decision under section 17 of the Child Support Act 1991 which falls to be made”; and

(ii) for the word “notifies” there shall be substituted the word “notified”;
\end{enumerate}

($c$) paragraphs (4) and (5) shall be omitted; and

($d$) in paragraph (6)—
\begin{enumerate}\item[]
(i) in sub-paragraph ($a$), for the words “a child support officer to consider the question” there shall be substituted the words “the question to be considered”; and

(ii) for the words “upon a review under section 19 of the Act” there shall be substituted the words “by virtue of a decision under section 17 of the Act superseding an earlier decision”.
\end{enumerate}
\end{enumerate}

\subsection[38. Amendment of regulation 64]{Amendment of regulation 64}

38.  In regulation 64 (transitional provisions)—
\begin{enumerate}\item[]
($a$) in paragraph (3), for the words from “a child support officer” to the end there shall be substituted the words “a relevant person applies for a decision under section 17 of the Child Support Act 1991 superseding an earlier decision on the ground that a qualifying transfer of property has been made or that he has travelling costs.”; and

($b$) in paragraph (4)—
\begin{enumerate}\item[]
(i) in sub-paragraph ($a$), for the words “a review of” there shall be substituted the words “a decision under section 17 of the Child Support Act 1991 superseding a decision with respect to”; and

(ii) for sub-paragraph ($b$), there shall be substituted the following sub-paragraph—
\begin{quotation}
“($b$) a decision under regulation 13 or 16 of the Arrears Regulations is made on an application made by a relevant person.”.
\end{quotation}
\end{enumerate}
\end{enumerate}

\section[Part XI --- Amendment of the Child Support (Miscellaneous Amendments) (No.\ 2) Regulations 1995]{Part XI\\*Amendment of the Child Support (Miscellaneous Amendments) (No.\ 2) Regulations 1995}

\renewcommand\parthead{--- Part XI}

\subsection[39. Substitution of regulation 56]{Substitution of regulation 56}

39.  For regulation 56 (reviews consequent on amendments made by these Regulations) there shall be substituted the following regulation—
\begin{quotation}
\subsection*{“Supersessions consequent on amendments made by these Regulations}

56.  Where a fresh assessment is made by virtue of a decision under section 17 of the Child Support Act 1991 superseding an earlier decision in consequence of the coming into force of regulation 48—
\begin{enumerate}\item[]
($a$) the decision under section 17; and

($b$) that fresh maintenance assessment,
\end{enumerate}
shall have effect as from the first day of the maintenance period following 18th December 1995.”.
\end{quotation}

\subsection[40. Amendment of regulation 57]{Amendment of regulation 57}

40.  In regulation 57 (transitional and consequential provisions)—
\begin{enumerate}\item[]
($a$) for paragraph (1) there shall be substituted the following paragraph—
\begin{quotation}
“(1) A decision with respect to a maintenance assessment shall not be superseded by a decision under section 17 of the Child Support Act 1991 solely to give effect to the provisions set out in paragraph (2).”;
\end{quotation}

($b$) in paragraph (3)—
\begin{enumerate}\item[]
(i) for the words “Where a review” there shall be substituted the words “Where a decision is made under section 17 of the Child Support Act 1991 superseding an earlier decision”; and

(ii) for the words “that review” there shall be substituted the words “that decision”; and
\end{enumerate}

($c$) paragraph (4) shall be omitted.
\end{enumerate}

\section[Part XII --- Amendment of the Child Support (Compensation for Recipients of Family Credit and Disability Working Allowance) Regulations 1995]{\sloppy Part XII\\*Amendment of the Child Support (Compensation for Recipients of Family Credit and Disability Working Allowance) Regulations 1995}

\renewcommand\parthead{--- Part XII}

\subsection[41. Amendment of regulation 5]{Amendment of regulation 5}

41.  For paragraphs (1) to (4) of regulation 5 (calculation of compensation in particular cases) there shall be substituted the following paragraphs—
\begin{quotation}
“(1) Where a revised assessment is replaced by a fresh maintenance assessment of a different amount by virtue of a revision under section 16 of the 1991 Act, the compensation payment calculated under section 24 of the 1995 Act shall be recalculated using the amount due under the fresh maintenance assessment in place of the amount due under the revised assessment.

(2) Subject to paragraph (3), where the earlier assessment is replaced by a fresh assessment—
\begin{enumerate}\item[]
($a$) which was made after the revised assessment; and

($b$) the effective date of that fresh assessment is before the date on which the revised assessment was made,
\end{enumerate}
the amount payable under the fresh assessment shall be ignored for the purposes of the calculation of a compensation payment under section 24 of the 1995 Act.

(3) In a case where the circumstances in paragraphs (1) and (2) apply the compensation payable under section 24 of the 1995 Act shall be recalculated using the amount due under the fresh assessments referred to in paragraphs (1) and (2).”.
\end{quotation}

\section[Part XIII --- Amendment of the Child Benefit, Child Support and Social Security (Miscellaneous Amendments) Regulations 1996]{Part XIII\\*Amendment of the Child Benefit, Child Support and Social Security (Miscellaneous Amendments) Regulations 1996}

\renewcommand\parthead{--- Part XIII}

\subsection[42. Substitution of regulation 49 of the Child Benefit, Child Support and Social Security (Miscellaneous Amendments) Regulations 1996]{Substitution of regulation 49 of the Child Benefit, Child Support and Social Security (Miscellaneous Amendments) Regulations 1996}

42.  For regulation 49 (transitional provision relating to maintenance assessments) of the Child Benefit, Child Support and Social Security (Miscellaneous Amendments) Regulations 1996\footnote{\frenchspacing S.I. 1996/1803.} there shall be substituted the following regulation—
\begin{quotation}
“49.—(1) A decision with respect to a maintenance assessment in force on 7th April 1997 shall not be superseded by a decision under section 17 of the Child Support Act 1991 (“the Act”) solely to give effect to these Regulations.

(2) These Regulations shall apply to a fresh maintenance assessment made by virtue of—
\begin{enumerate}\item[]
($a$) a revision under section 16 of the Act of a decision with respect to a maintenance assessment; or

($b$) a decision under section 17 of the Act which supersedes a decision with respect to a maintenance assessment,
\end{enumerate}
as from the effective date of that revision under section 16 of the Act or, as the case may be, decision under section 17 of the Act.”.
\end{quotation}

\section[Part XIV --- Amendment of the Child Support (Miscellaneous Amendments) Regulations 1996]{Part XIV\\*Amendment of the Child Support (Miscellaneous Amendments) Regulations 1996}

\renewcommand\parthead{--- Part XIV}

\subsection[43. Amendment of regulation 25(5) of the Child Support (Miscellaneous Amendments) Regulations 1996]{Amendment of regulation 25(5) of the Child Support (Miscellaneous Amendments) Regulations 1996}

43.  In regulation 25(5) (transitional provisions) of the Child Support (Miscellaneous Amendments) Regulations 1996\footnote{\frenchspacing S.I. 1996/1945.}, for the words from “and those provisions” to the end there shall be substituted the words “and a decision with respect to a maintenance assessment in force on that date shall not be superseded by a decision under section 17 of the Child Support Act 1991 solely to give effect to the provisions of regulation 19 as amended by regulation 23.”.

\section[Part XV --- Amendment of the Child Support (Miscellaneous Amendments) (No.\ 2) Regulations 1996]{Part XV\\*Amendment of the Child Support (Miscellaneous Amendments) (No.\ 2) Regulations 1996}

\renewcommand\parthead{--- Part XV}

\subsection[44. Amendment of regulation 16 of the Child Support (Miscellaneous Amendments) (No.\ 2) Regulations 1996]{Amendment of regulation 16 of the Child Support (Miscellaneous Amendments) (No.\ 2) Regulations 1996}

44.  In regulation 16 (transitional provision) of the Child Support (Miscellaneous Amendments) (No.\ 2) Regulations 1996\footnote{\frenchspacing S.I. 1996/3196.}—
\begin{enumerate}\item[]
($a$) for paragraph (1) there shall be substituted the following paragraph—
\begin{quotation}
“(1) A decision with respect to a maintenance assessment in force on 13th January 1997 shall not be superseded by a decision under section 17 of the Act solely to give effect to these Regulations.”; and
\end{quotation}

($b$) in paragraph (2), the words “made following a review mentioned in paragraph (1)” shall be omitted.
\end{enumerate}

\section[Part XVI --- Amendment of the Child Support (Miscellaneous Amendments) Regulations 1998]{Part XVI\\*Amendment of the Child Support (Miscellaneous Amendments) Regulations 1998}

\renewcommand\parthead{--- Part XVI}

\subsection[45. Amendment of regulation 59 of the Child Support (Miscellaneous Amendments) Regulations 1998]{Amendment of regulation 59 of the Child Support (Miscellaneous Amendments) Regulations 1998}

45.  For regulation 59 (transitional provisions) of the Child Support (Miscellaneous Amendments) Regulations 1998\footnote{\frenchspacing S.I. 1998/58.} there shall be substituted the following regulation—
\begin{quotation}
“59.—(1) A decision with respect to a maintenance assessment in force on the first commencement day shall not be superseded by a decision under section 17 of the Child Support Act 1991 (“the Act”) solely to give effect to regulation 42(2)($d$), regulation 50 or regulation 56(2).

(2) The regulations specified in paragraph (1) shall apply to a fresh maintenance assessment made by virtue of—
\begin{enumerate}\item[]
($a$) a revision under section 16 of the Act of a decision with respect to a maintenance assessment; or

($b$) a decision under section 17 of the Act which supersedes a decision with respect to a maintenance assessment,
\end{enumerate}
as from whichever is the later of—
\begin{enumerate}\item[]
(i) the date as from which that revision or, as the case may be, supersession takes effect; or

(ii) the first day of the first maintenance period which begins on or after the first commencement day, as the case may be.
\end{enumerate}

(3) A decision with respect to a maintenance assessment in force on the second commencement day shall not be superseded by a decision under section 17 of the Act solely to give effect to regulations 44, 45, 47, 49, 52, 54, 55 and 56(5).

(4) The regulations specified in paragraph (3) shall apply to a fresh maintenance assessment made by virtue of—
\begin{enumerate}\item[]
($a$) a revision under section 16 of the Act of a decision with respect to a maintenance assessment; or

($b$) a decision under section 17 of the Act which supersedes a decision with respect to a maintenance assessment,
\end{enumerate}
as from whichever is the later of—
\begin{enumerate}\item[]
(i) the date as from which that revision or, as the case may be, supersession takes effect; or

(ii) the first day of the first maintenance period which begins on or after the second commencement day, as the case may be.”.
\end{enumerate}
\end{quotation}

\section[Part XVII --- Amendment of the Child Support (Miscellaneous Amendments) Regulations 1999]{Part XVII\\*Amendment of the Child Support (Miscellaneous Amendments) Regulations 1999}

\renewcommand\parthead{--- Part XVII}

\subsection[46. Amendment of regulation 6]{Amendment of regulation 6}

46.  In the amendment to Schedule 1 to the Child Support (Maintenance Assessments and Special Cases) Regulations 1992\footnote{\frenchspacing S.I. 1992/1815.} effected by regulation 6(5), for the words “child support officer” in each place in which they occur there shall be substituted the words “Secretary of State”.

\subsection[47. Substitution of regulation 7]{Substitution of regulation 7}

47.  For regulation 7 (transitional provisions) there shall be substituted the following regulation—
\begin{quotation}
\subsection*{“Transitional provisions}

7.—(1) A decision with respect to a maintenance assessment in force on the first or second commencement day shall not be superseded by a decision under section 17 of the Act solely to give effect to these Regulations.

(2) These Regulations shall apply to a fresh maintenance assessment made by virtue of—
\begin{enumerate}\item[]
($a$) a revision under section 16 of the Act of a decision with respect to a maintenance assessment; or

($b$) a decision under section 17 of the Act which supersedes a decision with respect to a maintenance assessment,
\end{enumerate}
as from whichever is the later of—
\begin{enumerate}\item[]
(i) the date as from which that revision or, as the case may be, supersession takes effect; or

(ii) the first day of the first maintenance period which begins on or after the first or second commencement day, as the case may be.”.
\end{enumerate}
\end{quotation}

\section[Part XVIII --- Transitional provisions]{Part XVIII\\*Transitional provisions}

\renewcommand\parthead{--- Part XVIII}

\subsection[48. Child Support]{Child Support}

48.—(1) Any decision which fell to be made but was not made before 1st June 1999 by a child support officer shall be made by the Secretary of State.

(2) Except for the purposes of paragraph (6) below and any provision as to the time within which an appeal is to be brought, a decision of a child support officer shall be treated as a decision of the Secretary of State made under—
\begin{enumerate}\item[]
($a$) subject to sub-paragraph ($b$) and paragraph (3) below, the provision under which the child support officer made the decision;

($b$) section 17 of the Act where the child support officer made the decision under section 18 or 19 of the Act\footnote{\frenchspacing Section 17 is substituted for sections 17 to 19 by section 41 of the Social Security Act 1998.}.
\end{enumerate}

(3) A fresh maintenance assessment made pursuant to section 16(4) of the Act (periodical reviews) by virtue of the saving in article 3(4) of the Social Security Act 1998 (Commencement No.\ 2) Order 1998\footnote{\frenchspacing S.I. 1998/2780.} shall be treated for the purpose of subsequent decisions as if it were made by virtue of a decision of the Secretary of State under section 17 of the Act (decisions superseding earlier decisions).

(4) For the purposes of a fresh maintenance assessment which falls to be made pursuant to section 16(4) of the Act (periodical reviews) by virtue of the saving in article 3(4) of the Social Security Act 1998 (Commencement No.\ 2) Order 1998, “relevant week” in the Child Support (Maintenance Assessments and Special Cases) Regulations 1992\footnote{\frenchspacing S.I. 1992/1815.} shall mean notwithstanding regulation 1(2) of those Regulations the period of seven days immediately preceding the date on which a request for information or evidence was made under regulation 17(5) of the Child Support (Maintenance Assessment Procedure) Regulations 1992\footnote{\frenchspacing S.I. 1992/1813; regulation 17 was amended by S.I. 1995/3261 and substituted by S.I. 1999/1047.} as that provision was in force when that request was sent.

(5) The date on which the fresh maintenance assessment mentioned in paragraph (4) above made on or after 1st June 1999 takes effect shall be determined in accordance with the provisions of section 16(5) of the Act and regulations made thereunder as those provisions were in force immediately before 1st June 1999.

(6) An application which was not determined before 1st June 1999 for a review of a decision of a child support officer shall be treated—
\begin{enumerate}\item[]
($a$) in a case where the application—
\begin{enumerate}\item[]
(i) is received within one month of the date of notification of the decision which is the subject of the application or such longer period as may be allowed by article 49 below; and

(ii) is made other than on the ground of a relevant change of circumstances,
\end{enumerate}
as an application to the Secretary of State for a revision of the decision under section 16 of the Act; and

($b$) in any other case, as an application to the Secretary of State for a decision under section 17 of the Act superseding the decision.
\end{enumerate}

(7) A revision under section 16 of the Act of a decision made before 22nd January 1996 to cancel a Category B interim maintenance assessment (within the meaning of regulation 8(3)($b$) of the Child Support (Maintenance Assessment Procedure) Regulations 1992 shall have effect as from 22nd January 1996.

(8) For the purposes of paragraph (9) below, this paragraph applies where a decision of the Secretary of State—
\begin{enumerate}\item[]
($a$) supersedes a decision of a child support officer; and

($b$) is made on the basis of information or evidence which was not provided by a relevant person directly.
\end{enumerate}

(9) Where paragraph (8) above applies, a decision which supersedes an earlier decision shall have effect as from the first day of the maintenance period in which that information or evidence was received by—
\begin{enumerate}\item[]
($a$) except where sub-paragraph ($b$) applies, an officer of the Secretary of State exercising functions under the Act; or

($b$) a child support officer.
\end{enumerate}

(10) Where—
\begin{enumerate}\item[]
($a$) a departure direction given under section 28F of the Act (departure directions) has effect on 1st June 1999;

($b$) the applicant in response to whose application that direction was given made a later application before 1st June 1999 for a departure direction—
\begin{enumerate}\item[]
(i) on grounds additional to the grounds in respect of which the earlier direction was given; or

(ii) on the basis that there has been a change of circumstances in respect of any of those grounds; and
\end{enumerate}

($c$) that later application was not determined before 1st June 1999,
\end{enumerate}
that application shall be treated as if it were made under section 17 as extended by paragraph 2 of Schedule 4C to the Act\footnote{\frenchspacing Schedule 4C was inserted by the Social Security Act 1998 (c. 14); Schedule 7, paragraph 54.} for a decision superseding an earlier decision.

(11) A decision made by virtue of paragraph (10) above superseding an earlier decision shall have effect as from the first day of the maintenance period in which the later application was made.

(12) A decision made—
\begin{enumerate}\item[]
($a$) by the Secretary of State on his own initiative under section 17 as extended by paragraph 2 of Schedule 4C to the Act which supersedes an earlier decision with respect to a departure direction; and

($b$) on the basis of information or evidence provided to him before 1st June 1999 by a person who is not the applicant in response to whose application the departure direction was given,
\end{enumerate}
shall have effect as from the first day of the maintenance period in which that information or evidence was provided to the Secretary of State.

(13) A decision—
\begin{enumerate}\item[]
($a$) of the Secretary of State made before 1st June 1999 with respect to a departure direction; or

($b$) of a child support appeal tribunal upon referral under section 28D(1)($b$) of the Act,
\end{enumerate}
may be revised under section 16 as extended by paragraph 1 of Schedule 4C to the Act in consequence of information or evidence—
\begin{enumerate}\item[]
(i) received by the Secretary of State from a relevant person within one month of the date of notification of that decision or such longer period as may be allowed by article 49 below; and

(ii) not acted upon before 1st June 1999.
\end{enumerate}

(14) Except for the purposes of paragraph (16) below, an appeal to a child support appeal tribunal which was not determined before 1st June 1999—
\begin{enumerate}\item[]
($a$) shall be treated as an appeal to an appeal tribunal;

($b$) brought against a decision of a child support officer shall be treated as an appeal brought against a decision of the Secretary of State; and

($c$) may not be withdrawn without either the consent in writing of every other party to the proceedings or the leave of a legally qualified panel member after every other party to the proceedings has been given a reasonable opportunity to make representations.
\end{enumerate}

(15) In paragraph (14) above and paragraph (28) below, “party to the proceedings” means—
\begin{enumerate}\item[]
($a$) the absent parent (within the meaning given to that expression in section 3(2) of the Act);

($b$) the person with care (within the meaning given to that expression in section 3(3) of the Act);

($c$) any child who has made an application for a maintenance assessment to be made under section 7 of the Act; and

($d$) the Secretary of State.
\end{enumerate}

(16) Regulations 3(1A) to (11B) and 15 of the Child Support Appeal Tribunals (Procedure) Regulations 1992\footnote{\frenchspacing S.I. 1992/2641; regulation 3 was amended by S.I. 1995/1045, 1996/2450, 1996/2907, 1996/3196, 1997/827 and 1998/58; regulation 15 was amended by S.I. 1996/2450; the whole instrument is revoked by S.I. 1999/991.} (in this article referred to as “the former Procedure Regulations”) shall continue to apply (notwithstanding their revocation) for the purposes specified in paragraph (17) below subject to the modifications to—
\begin{enumerate}\item[]
($a$) regulation 3 specified in paragraph (18) below; and

($b$) regulation 15 specified in paragraph (19) below.
\end{enumerate}

(17) Paragraph (16) above applies for the purposes of—
\begin{enumerate}\item[]
($a$) any appeal against a decision—
\begin{enumerate}\item[]
(i) of the Secretary of State made before 1st June 1999 on an application for a departure direction; or

(ii) of a child support officer; and
\end{enumerate}

($b$) any application to set aside a decision of a child support appeal tribunal.
\end{enumerate}

(18) The modifications to regulation 3 specified in this paragraph are—
\begin{enumerate}\item[]
($a$) in paragraph (1A), for the words “in paragraph (1)” there shall be substituted the words “in article 48(17) of the Social Security Act 1998 (Commencement No.\ 7 and Consequential and Transitional Provisions) Order 1999\footnote{\frenchspacing S.I. 1999/1510 (C. 43).}”;

($b$) in paragraph (3) for the words “under section 20(1) of the Act” there shall be substituted the words “against a decision of a child support officer”;

($c$) for paragraph (6) there shall be substituted the following paragraph—
\begin{quotation}
“(6) Where an appeal or application is made—
\begin{enumerate}\item[]
($a$) after the specified time has expired; and

($b$) before 1st July 2000,
\end{enumerate}
that time may for special reasons be extended by a legally qualified panel member to the date of the making of the appeal or application.”;
\end{quotation}

($d$) in paragraphs (7), (9A), (11) and (11A), for the word “chairman” in each place in which it occurs there shall be substituted the words “legally qualified panel member”;

($e$) in paragraph (8) for the words “any chairman” there shall be substituted the words “a legally qualified panel member”; and

($f$) after paragraph (11B) there shall be added the following paragraph—
\begin{quotation}
“(11C) In this regulation—
\begin{enumerate}\item[]
    “legally qualified panel member” has the same meaning as in regulation 1(3) of the Social Security and Child Support (Decisions and Appeals) Regulations 1999\footnote{\frenchspacing S.I. 1999/991.}; and

    “tribunal” means an appeal tribunal constituted under section 7 of the Social Security Act 1998\footnote{\frenchspacing 1998 c. 14.}.”.
\end{enumerate} 
\end{quotation}
\end{enumerate}

(19) The modifications to regulation 15 specified in this paragraph are—
\begin{enumerate}\item[]
($a$) in paragraph (1)—
\begin{enumerate}\item[]
(i) after the words “on an application made” there shall be inserted the words “before 1st July 2000”; and

(ii) for the words “the tribunal who gave the decision or by another tribunal” there shall be substituted the words “a tribunal”;
\end{enumerate}

($b$) in paragraph (5), the words “regulation 2 and” shall be omitted; and

($c$) after paragraph (5) there shall be added the following paragraph—
\begin{quotation}
“(6) Except in paragraph (1)($a$), “tribunal” in this regulation means an appeal tribunal constituted under section 7 of the Social Security Act 1998.”.
\end{quotation}
\end{enumerate}

(20) Paragraphs (21) to (24) below shall apply where—
\begin{enumerate}\item[]
($a$) a clerk to a child support appeal tribunal gave a direction under regulation 11(1)\footnote{\frenchspacing Regulation 11(1) was substituted by S.I. 1996/2540 and amended by S.I. 1996/2907 and 1998/58.} of the former Procedure Regulations; and

($b$) notification under that provision was not received by him before 1st June 1999.
\end{enumerate}

(21) A notification in response to a direction given under regulation 11(1) of the former Procedure Regulations shall be in writing and shall be made within 14 days of receipt of the direction or within such other period as the clerk to an appeal tribunal may direct.

(22) An appeal may be struck out by the clerk to an appeal tribunal where a notification such as is referred to in paragraph (21) above is not received within the period specified in that paragraph.

(23) An appeal which has been struck out in accordance with paragraph (22) above shall be treated for the purpose of reinstatement as if it had been struck out under regulation 46 of the Social Security and Child Support (Decisions and Appeals) Regulations 1999.

(24) An oral hearing of the appeal shall be held where—
\begin{enumerate}\item[]
($a$) notification is received by the clerk to the appeal tribunal under paragraph (21) above; or

($b$) the chairman, or in the case of an appeal tribunal which has only one member, that member, is satisfied that such a hearing is necessary to enable the appeal tribunal to reach a decision.
\end{enumerate}

(25) A legally qualified panel member may reinstate an appeal which has been struck out under regulation 6 of the former Procedure Regulations on an application made by any party to the proceedings not later than three months from the date of the order under paragraph (1) of that regulation if he is satisfied that—
\begin{enumerate}\item[]
($a$) the applicant did not receive a notice under paragraph (2) of that regulation; and

($b$) the conditions in paragraph (2A) of that regulation were not satisfied.
\end{enumerate}

(26) Notwithstanding the revocation of the former Procedure Regulations, information such as was mentioned in regulation 17(2) of those Regulations shall not be disclosed if a written notification is received under that regulation within the period specified in that regulation.

(27) A copy of a statement of—
\begin{enumerate}\item[]
($a$) the reasons for a child support appeal tribunal’s decision;

($b$) its findings on questions of fact material thereto; and

($c$) the terms of any—
\begin{enumerate}\item[]
(i) direction under section 20(4) of the Act (given before that provision was substituted); and

(ii) decision made by the tribunal under section 28H(4)($c$) of the Act (before that provision was substituted) or on a referral,
\end{enumerate}
\end{enumerate}
shall be supplied to each party to the proceedings if requested by any of them within 21 days of the date on which notification of the decision was given or sent.

(28) Except for the purposes of—
\begin{enumerate}\item[]
($a$) the Child Support Commissioners (Procedure) Regulations 1999\footnote{\frenchspacing S.I. 1999/1305.};

($b$) paragraphs (16) and (27) above; or

($c$) determining whether any irregularity resulted from failure to comply with the requirements of the former Procedure Regulations,
\end{enumerate}
a decision of a child support appeal tribunal shall be treated as a decision of an appeal tribunal.

(29) An appeal tribunal shall completely rehear any appeal to a child support appeal tribunal which stands adjourned immediately before 1st June 1999.

(30) For the purpose of section 17(1) of the Act, a decision of a Child Support Commissioner on an appeal from a child support appeal tribunal shall be treated as a decision of a Child Support Commissioner on an appeal from an appeal tribunal.

\subsection[49. Late application for a revision]{Late application for a revision}

49.—(1) The period of one month specified in article 48(6)($a$)(i) or (13)(i) above may be extended where the conditions specified in the following provisions of this article are satisfied.

(2) An application for an extension of time under this article shall—
\begin{enumerate}\item[]
($a$) be made—
\begin{enumerate}\item[]
(i) before 1st July 2000; and

(ii) by a relevant person or a person acting on his behalf; and
\end{enumerate}

($b$) contain—
\begin{enumerate}\item[]
(i) particulars of the grounds on which the extension of time is sought; and

(ii) sufficient details of the decision which it is sought to have revised to enable that decision to be identified.
\end{enumerate}
\end{enumerate}

(3) The application for an extension of time shall not be granted unless the person making the application or any person acting for him satisfies the Secretary of State that—
\begin{enumerate}\item[]
($a$) it is reasonable to grant that application;

($b$) the application for a decision to be revised has merit; and

($c$) special circumstances are relevant to the application for an extension of time, and as a result of those special circumstances, it was not practicable for the application for a decision to be revised to be made within one month of the date of notification of the decision which it is sought to have revised.
\end{enumerate}

(4) In determining whether it is reasonable to grant the application for an extension of time, no account shall be taken of the following—
\begin{enumerate}\item[]
($a$) that the person making the application for an extension of time or any person acting for him was unaware of or misunderstood the law applicable to his case (including ignorance or misunderstanding of the time limits imposed by article 48(6)($a$)(i) or (13)(i)) above; or

($b$) that a Child Support Commissioner or a court has taken a different view of the law from that previously understood and applied.
\end{enumerate}

(5) An application under this article for an extension of time which has been refused may not be renewed.

\subsection[50. Transitional functions of staff of appeal tribunals and of the President of appeal tribunals]{Transitional functions of staff of appeal tribunals and of the President of appeal tribunals}

50.  Any appointment under paragraph 6 of Schedule 1 to the Social Security Act 1998 shall be treated during the period commencing on 1st June 1999 and ending on—
\begin{enumerate}\item[]
($a$) 28th November 1999 as being in addition an appointment for—
\begin{enumerate}\item[]
(i) social security appeal tribunals; and

(ii) the President of, and regional and other full-time chairmen of, social security appeal tribunals, medical appeal tribunals and disability appeal tribunals;
\end{enumerate}

($b$) 17th October 1999 as being in addition an appointment for disability appeal tribunals; and

($c$) 5th September 1999 as being in addition an appointment for medical appeal tribunals.
\end{enumerate}

\subsection[51. Interpretation of this Part]{Interpretation of this Part}

51.  In this Part—
\begin{enumerate}\item[]
“the Act” means the Child Support Act 1991\footnote{\frenchspacing 1991 c. 48.};

“legally qualified panel member” has the same meaning as in regulation 1(3) of the Social Security and Child Support (Decisions and Appeals) Regulations 1999; and

“maintenance period” and “relevant person” have the same meaning as in regulation 1(2) of the Child Support (Maintenance Assessment Procedure) Regulations 1992\footnote{\frenchspacing S.I. 1992/1813; there are amendments to regulation 1(2) which are not relevant to this Order.}. 
\end{enumerate}

\bigskip

Signed 
by authority of the Secretary of State for Social Security.

{\raggedleft
\emph{Angela Eagle
}\\*Parliamentary Under-Secretary of State,\\*Department of Social Security

}

27th May 1999

\small

\part{Explanatory Note}

\renewcommand\parthead{--- Explanatory Note}

\subsection*{(This note is not part of the Order)}

This Order provides for the coming into force on 1st June 1999 of the provisions of the Social Security Act 1998 which—
\begin{enumerate}\item[]
($a$) enable the Lord Chancellor to—
\begin{enumerate}\item[]
(i) appoint a President of appeal tribunals; and

(ii) constitute a panel from which the President of appeal tribunals may draw persons to act as members of appeal tribunals;
\end{enumerate}

($b$) enable the Secretary of State to appoint officers and staff for the President of appeal tribunals and for appeal tribunals;

($c$) enable regulations to be made supplementing the provision made by section 26 of the Social Security Act 1998;

($d$) remove the power of the President of social security appeal tribunals, medical appeal tribunals and disability appeal tribunals to appoint officers and staff for—
\begin{enumerate}\item[]
(i) himself;

(ii) regional and other full-time chairmen of social security appeal tribunals, medical appeal tribunals and disability appeal tribunals;

(iii) social security appeal tribunals;

(iv) disability appeal tribunals; and

(v) medical appeal tribunals;
\end{enumerate}

($e$) amend the Judicial Pensions and Retirement Act 1993 (c.\ 8) (“the 1993 Act”) in consequence of provisions of the Social Security Act 1998—
\begin{enumerate}\item[]
(i) by removing references to chairmen of child support appeal tribunals (in view of the replacement of such tribunals by appeal tribunals constituted under section 7 of the Social Security Act 1998);

(ii) so as to provide that the office of President of appeal tribunals is a qualifying judicial office for the purposes of the 1993 Act (new arrangements for judicial pensions);
\end{enumerate}

($f$) substitute new procedures for making decisions and appeals with respect to child support; and

($g$) repeal words in the Debtors (Scotland) Act 1987 (c.\ 18) in consequence of the commencement of paragraph 12 of Schedule 7 to the Social Security Act 1998 by article 2(4)($b$) of the Social Security Act 1998 (Commencement No.\ 4) Order 1999 (S.I.\ 1999/526) (article 2).
\end{enumerate}

This Order amends—
\begin{enumerate}\item[]
($a$) the Social Security (Claims and Payments) Regulations 1987 (S.I.\ 1987/\hspace{0pt}1968);

($b$) the Family Credit (General) Regulations 1987 (S.I.\ 1987/1973);

($c$) the Disability Working Allowance (General) Regulations 1991 (S.I.\ 1991/\hspace{0pt}2887);

($d$) the Child Support (Information, Evidence and Disclosure) Regulations 1992 (S.I.\ 1992/1812);

($e$) the Child Support (Maintenance Assessments and Special Cases) Regulations 1992 (S.I.\ 1992/1815);

($f$) the Child Support (Arrears, Interest and Adjustment of Maintenance Assessments) Regulations 1992 (S.I.\ 1992/1816);

($g$) the Child Support (Collection and Enforcement) Regulations 1992 (S.I.\ 1992/1989);

($h$) the Child Support Act 1991 (Commencement No.\ 3 and Transitional Provisions) Order 1992 (S.I.\ 1992/2644);

($i$) the Child Support (Maintenance Arrangements and Jurisdiction) Regulations 1992 (S.I.\ 1992/2645);

($j$) the Child Support (Miscellaneous Amendments and Transitional Provisions) Regulations 1994 (S.I.\ 1994/227);

($k$) the Child Support and Income Support (Amendment) Regulations 1995 (S.I.\ 1995/1045);

($l$) the Child Support (Miscellaneous Amendments) (No.\ 2) Regulations 1995 (S.I.\ 1995/3261);

($m$) the Child Support (Compensation for Recipients of Family Credit and Disability Working Allowance) Regulations 1995 (S.I.\ 1995/3263);

($n$) the Child Benefit, Child Support and Social Security (Miscellaneous Amendments) Regulations 1996 (S.I.\ 1996/1803);

($o$) the Child Support (Miscellaneous Amendments) Regulations 1996 (S.I.\ 1996/1945);

($p$) the Child Support (Miscellaneous Amendments) (No.\ 2) Regulations 1996 (S.I.\ 1996/3196);

($q$) the Child Support (Miscellaneous Amendments) Regulations 1998 (S.I.\ 1998/58); and

($r$) the Child Support (Miscellaneous Amendments) Regulations 1999 (S.I.\ 1999/977),
\end{enumerate}
in consequence of the amendments to the Child Support Act 1991 (c.\ 48) effected by the Social Security Act 1998 (c.\ 14).

References to “child support officer” are replaced with references to “Secretary of State” and provision referring to reviews is replaced with provision referring to revision and supersession. This Order also revokes provisions rendered obsolete by the Social Security Act 1998.

Provision is made for the revision and supersession of decisions with respect to the interest on arrears of child support maintenance (article 28).

Article 50 provides that appointments of officers and staff by the Secretary of State for the President of appeal tribunals and for appeal tribunals shall be treated as being in addition appointments for the offices and bodies described in sub-paragraph ($b$) above until those offices and bodies cease to exist.

Transitional provision is made for child support purposes in the light of the new procedures for making decisions and appeals which are introduced by virtue of provisions commenced by this Order.

This Order does not impose a charge on business.

\part{Note as to Earlier Commencement Orders}

\renewcommand\parthead{--- Note as to Earlier Commencement Orders}

\subsection*{(This note is not part of the Order)}

The following provisions have been brought into force by the Social Security Act 1998 (Commencement No.\ 1) Order 1998 (S.I.\ 1998/2209), the Social Security Act 1998 (Commencement No.\ 2) Order 1998 (S.I.\ 1998/2780), the Social Security Act 1998 (Commencement No.\ 3) Order 1999 (S.I.\ 1999/418), the Social Security Act 1998 (Commencement No.\ 4) Order 1999 (S.I.\ 1999/526), the Social Security Act 1998 (Commencement No.\ 5) Order 1999 (S.I.\ 1999/528) and the Social Security Act 1998 (Commencement No.\ 6) Order 1999 (S.I.\ 1999/1055).

\medskip


{\footnotesize
\noindent
%\begin{tabulary}{\linewidth}{JJJ}
\begin{longtable}{p{193.78079pt}p{78.37471pt}p{81.83215pt}}
\hline
\itshape Provision of Social Security Act 1998	& \itshape Date of Commencement	& \itshape S.I.\ Number\\
\hline
\endhead
\hline
\endlastfoot
section 2 (except section 2(2)($a$))	&8th September 1998	&1998/2209\\
section 3	&8th September 1998	&1998/2209\\
{}\footnote{\frenchspacing In this note an asterisk indicates that the provision or provisions in the entry to which it relates has or have been commenced only for the purposes of making regulations.}section 6(3)&	4th March 1999&	1999/528\\
{}*section 7(6)	&4th March 1999	&1999/528\\
{}*section 7(7) in so far as it relates to the regulation-making powers in paragraphs 7, 11 and 12 of Schedule 1	&4th March 1999	&1999/528\\
{}*section 9(1), (4) and (6)	&4th March 1999	&1999/528\\
{}*section 10(3) and (6)	&4th March 1999	&1999/528\\
{}*section 11(1)	&4th March 1999	&1999/528\\
{}*section 12(1) in so far as it relates to paragraph 9 of Schedule 2 and paragraphs 1, 4 and 9 of Schedule 3	&4th March 1999	&1999/528\\
{}*section 12(2), (3), (6) and (7)	&4th March 1999	&1999/528\\
{}*section 14(3), (10) and (11)	&4th March 1999	&1999/528\\
{}*section 14(12) in so far as it relates to paragraphs 6 and 8 of Schedule 4	&4th March 1999	&1999/528\\
{}*section 15(2) and (3)	&4th March 1999	&1999/528\\
{}*section 16(1) to (3) and (9) and Schedule 5	&4th March 1999	&1999/528\\
section 16(4) and (5)	&8th September 1998 and 6th April 1999	&1998/2209\\
{}*section 17	&4th March 1999	&1999/528\\
{}*section 18(1)	&4th March 1999	&1999/528\\
{}*sections 20 to 24	&4th March 1999	&1999/528\\
{}*section 25(3)($b$) and (5)($c$)	&4th March 1999	&1999/528\\
{}*section 26(6)($c$)	&4th March 1999	&1999/528\\
{}*section 28	&4th March 1999	&1999/528\\
{}*section 31(2) and (3)	&4th March 1999	&1999/528\\
{}*section 38(1)($a$) and (3)	&4th March 1999	&1999/528\\
section 40	&16th November 1998 and 7th December 1998&	1998/2780\\
{}*section 41 in so far as it substitutes section 17(3) and (5) of the Child Support Act 1991\footnote{\frenchspacing 1991 c. 48.}	&4th March 1999	&1999/528\\
{}*section 42 in so far as it substitutes section 20(4), (5) and (6) of the Child Support Act 1991	&4th March 1999	&1999/528\\
{}*section 43 in so far as it inserts section 28ZA(2)($b$) and (4)($c$) and 28ZB(6)($c$) into the Child Support Act 1991	&4th March 1999	&1999/528\\
{}*section 44 in so far as it inserts section 28ZD into the Child Support Act 1991	&4th March 1999	&1999/528\\
{}*section 45 in so far as it inserts section 3A(1), (3) and (4) into the Vaccine Damage Payments Act 1979\footnote{\frenchspacing 1979 c. 17.}	&4th March 1999	&1999/528\\
{}*section 46 in so far as it substitutes section 4(2) and (3) of the Vaccine Damage Payments Act 1979	&4th March 1999	&1999/528\\
{}*section 47 in so far as it inserts section 7A into the Vaccine Damage Payments Act 1979	&4th March 1999	&1999/528\\
section 48	&8th September 1998	&1998/2209\\
section 49	&8th September 1998	&1998/2209\\
section 50(1) (partially)	&8th September 1998	&1998/2209\\
section 51	&23rd February 1999 and 6th April 1999	&1999/418\\
section 52	&8th September 1998	&1998/2209\\
section 53	&8th September 1998 and 6th April 1999	&1998/2209\\
section 54	&4th March 1999 and 6th April 1999	&1999/526\\
section 55	&8th September 1998	&1998/2209\\
section 56(1)	&6th April 1999	&1999/526\\
section 56(2)	&4th March 1999 and 6th April 1999	&1999/526\\
section 57	&4th March 1999 and 6th April 1999	&1999/526\\
section 59	&8th September 1998	&1998/2209\\
section 60	&4th March 1999 and 6th April 1999	&1999/526\\
section 61 (partially)	&4th March 1999 and 6th April 1999	&1999/526\\
section 62	&6th April 1999	&1999/526\\
section 63	&4th March 1999 and 6th April 1999	&1999/526\\
section 64	&6th April 1999	&1999/526\\
section 65	&8th September 1998 and 6th April 1999	&1998/2209\\
section 68	&8th September 1998	&1998/2209\\
sections 70 and 71	&5th April 1999	&1999/1055\\
section 73	&6th April 1999	&1998/2209\\
section 74	&4th March 1999	&1999/528\\
section 75	&5th October 1998	&1998/2209\\
section 76	&16th November 1998	&1998/2780\\
{}*Schedule 1, paragraphs 7, 11 and 12	&4th March 1999	&1999/528\\
{}*Schedule 2, paragraph 9	&4th March 1999	&1999/528\\
{}*Schedule 3, paragraphs 1, 4 and 9&	4th March 1999&	1999/528\\
{}*Schedule 4, paragraphs 6 and 8&	4th March 1999&	1999/528\\
Schedule 7 in the respects specified below and section 86(1) in so far as it relates to them—		\\
{}*paragraph 8	&4th March 1999	&1999/528\\
{}*paragraph 9	&4th March 1999	&1999/528\\
paragraphs 12 to 14	&6th April 1999	&1999/526\\
paragraph 16	&6th April 1999	&1999/418\\
paragraph 27($a$)	&8th September 1998	&1998/2209\\
{}*paragraph 35(2)	&4th March 1999	&1999/528\\
{}*paragraph 39	&4th March 1999	&1999/528\\
{}*paragraphs 40 and 43(3)	&4th March 1999	&1999/528\\
{}*paragraph 44 in so far as it relates to the insertion of sections 46A(2) and 46B into the Child Support Act 1991	&4th March 1999	&1999/528\\
paragraph 46 (partially)	&16th November 1998 and 4th March 1999	&1998/2780 and 1999/528\\
paragraph 49	&8th September 1998	&1998/2209\\
{}*paragraph 52(4) which substitutes paragraph 6(2) of Schedule 4 to the Child Support Act 1991	&4th March 1999	&1999/528\\
{}*paragraph 53(5)	&4th March 1999	&1999/528\\
{}*paragraph 54	&4th March 1999	&1999/528\\
paragraph 56	&8th September 1998 and 6th April 1999	&1998/2209\\
paragraph 57	&6th April 1999	&1998/2209\\
paragraph 58(1)	&6th April 1999	&1999/418\\
paragraph 58(2)	&6th April 1999	&1998/2209\\
paragraphs 59 to 61	&6th April 1999	&1999/418\\
paragraph 71($b$), ($c$) and ($e$)	&6th April 1999	&1999/418\\
paragraph 71($d$)	&8th September 1998 and 6th April 1999	&1998/2209\\
paragraph 72(3) and (4)	&5th April 1999	&1999/1055\\
paragraphs 74, 75 and 77(2) to (5)	&6th April 1999	&1998/418\\
paragraph 77(1), (6) to (9), (11), (12) and (14) to (16)	&8th September 1998 and 6th April 1999	&1998/2209\\
paragraph 85	&6th April 1999	&1999/526\\
paragraph 86	&6th April 1999	&1998/2209 and 1999/526\\
paragraph 87	&6th April 1999	&1999/526\\
paragraph 90	&6th April 1999	&1999/418\\
paragraph 91	&8th September 1998 and 6th April 1999	&1998/2209\\
paragraphs 92 to 94	&6th April 1999	&1999/418\\
paragraph 99	&8th September 1998 and 6th April 1999	&1998/2209, 1999/418 and 1999/526\\
paragraph 100	&6th April 1999	&1998/2209 and 1999/526\\
paragraph 104	&4th March 1999	&1999/528\\
paragraph 110(1)($a$)	&8th September 1998 and 6th April 1999	&1998/2209\\
paragraph 110(1)($b$)	&6th April 1999	&1999/418\\
paragraph 114	&8th September 1998 and 6th April 1999	&1998/2209\\
{}*paragraph 121	&4th March 1999	&1999/528\\
paragraphs 126 to 128	&6th April 1999	&1999/418\\
{}*paragraph 131	&4th March 1999	&1999/528\\
paragraph 133	&6th April 1999	&1999/418\\
{}*paragraph 149	&4th March 1999	&1999/528\\
section 86(2) and Schedule 8 (partially)	&8th September 1998, 5th and 6th April 1999	&1998/2209, 1999/418, 1999/526 and 1999/1055\\
%\end{tabulary}
\end{longtable}

}


\end{document}