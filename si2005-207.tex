\documentclass[12pt,a4paper]{article}

\newcommand\regstitle{The Social Security and Child Support Commissioners (Procedure) (Amendment) Regulations 2005}

\newcommand\regsnumber{2005/207}

%\opt{newrules}{
\title{\regstitle}
%}

%\opt{2012rules}{
%\title{Child Maintenance and Other Payments Act 2008\\(2012 scheme version)}
%}

\author{S.I.\ 2005 No.\ 207}

\date{Made
3rd February 2005\\
Laid before Parliament
4th February 2005\\
Coming into force
28th February 2005
}

%\opt{oldrules}{\newcommand\versionyear{1993}}
%\opt{newrules}{\newcommand\versionyear{2003}}
%\opt{2012rules}{\newcommand\versionyear{2012}}

\usepackage{csa-regs}

\setlength\headheight{27.57402pt}

\begin{document}

\maketitle

\amendment{
Regs. revoked (3.11.08) by the Tribunals, Courts and Enforcement Act 2007 (Transitional and Consequential Provisions) Order 2008 Sch.~2.
}

%\noindent
%The Lord Chancellor, in exercise of the powers conferred upon him by sections 14(11), 15(3), 16(1), 79(2) and 84 of, and paragraphs 1 and 3 to 5 of Schedule 5 to, the Social Security Act 1998\footnote{1998 c.\ 14. Section 84 is an interpretation provision and is cited because of the meaning assigned to the word “prescribe”.}, sections 14(11), 15(3), and 16(1) of, and paragraphs 1 and 3 to 5 of Schedule 5 to, the Social Security Act 1998 as applied and modified by the Tax Credits (Appeals) Regulations 2002\footnote{S.I.\ 2002/2926, amended by S.I.\ 2004/372.}, sections 22(3), 24(6) and (7) and 25(5) of the Child Support Act 1991\footnote{1991 c.\ 48. Section 24(6) was amended by the Social Security Act 1998 (c.\ 14), section 86(1) and paragraph 30(6) to Schedule 7.}, paragraphs 8(8), 9(3), 10(1), 20(1) and (3) and 23(1) of Schedule 7 to the Child Support, Pensions and Social Security Act 2000\footnote{2000 c.\ 19. Paragraph 23(1) of Schedule 7 is an interpretation provision and is cited because of the meaning assigned to the word “prescribed”.} and section 4(2) of the Forfeiture Act 1982\footnote{1982 c.\ 34. The functions of the Secretary of State in making regulations with respect to proceedings before the Social Security Commissioners under section 4(2) were transferred to the Lord Chancellor by the Transfer of Functions (Social Security Commissioners) Order 1984, article 3 and Schedule (S.I.\ 1984/1818). Section 4(2) was also amended by the Social Security Act 1986 (c.\ 50), section 76(3), the Social Security (Consequential Provisions) Act 1992 (c.\ 6), section 4 and paragraph 63(1) of Schedule 2, and the Social Security Act 1998 (c.\ 14), section 86(1) and paragraph 11(1) of Schedule 7.}, after consultation with the Scottish Ministers\footnote{The functions of the Lord Advocate under section 79(2) of the Social Security Act 1998 (c.\ 14) and sections 22(3), 24(9) and 25(6) of the Child Support Act 1991 (c.\ 48) were transferred to the Secretary of State by the Transfer of Functions (Lord Advocate and Secretary of State) Order 1999 (S.I.\ 1999/678), Article 2(1) and the Schedule. Those functions were then treated as being exercisable in or as regards Scotland, for the purposes of section 63 of the Scotland Act 1998 (c.\ 46), by the Scotland Act 1998 (Functions Exercisable in or as Regards Scotland) Order 1999 (S.I.\ 1999/1748), Article 3 and paragraphs 12 and 19 of Schedule 1, and transferred to the Scottish Ministers by the Scotland Act 1998 (Transfer of Functions to the Scottish Ministers etc.)\ Order 1999 (S.I.\ 1999/1750), Article 2 and Schedule 1.} and, in accordance with section 8(1) of the Tribunals and Inquiries Act 1992\footnote{1992 c.\ 53.}, with the Council on Tribunals, makes the following Regulations: 
%
%{\sloppy
%
%\tableofcontents
%
%}
%
%\bigskip
%
%\setcounter{secnumdepth}{-2}
%
%\subsection[1. Citation, commencement and duration]{Citation, commencement and duration}
%
%1.---(1)  These Regulations may be cited as the Social Security and Child Support Commissioners (Procedure) (Amendment) Regulations 2005 and shall come into force on 28th February 2005.
%
%(2) Regulation 4 of these Regulations shall cease to have effect on such day as is appointed by order made under section 63(1) of the Tax Credits Act 2002 (tax credit appeals etc: temporary modifications).
%
%\subsection[2. Amendments to the Social Security Commissioners (Procedure) Regulations 1999]{Amendments to the Social Security Commissioners (Procedure) Regulations 1999}
%
%2.---(1)  The Social Security Commissioners (Procedure) Regulations 1999\footnote{S.I.\ 1999/1495, amended by S.I.\ 2000/2854 and S.I.\ 2001/1095.} shall be amended in accordance with this regulation.
%
%(2) Under Part I of the Arrangement of Regulations (general provisions), after the entry for regulation 8 insert—
%\begin{quotation}
%“8A.  Funding of legal services”.
%\end{quotation}
%
%(3) In regulation 4 (interpretation)—
%\begin{enumerate}\item[]
%($a$) after the definition of “the chairman” insert—
%\begin{quotation}
%““child benefit” means child benefit under Part IX of the Social Security Contributions and Benefits Act 1992\footnote{1992 c.\ 4.};”;
%\end{quotation}
%
%($b$) after the definition of “forfeiture rule question” insert—
%\begin{quotation}
%““funding notice” means the notice or letter from the Legal Services Commission confirming that legal services are to be funded;
%
%“guardian’s allowance” means guardian’s allowance under section 77 of the Social Security Contributions and Benefits Act 1992;
%
%“legal aid certificate” means the certificate issued by the Scottish Legal Aid Board confirming that legal services are to be funded;”;
%\end{quotation}
%
%($c$) after the definition of “legally qualified” insert—
%\begin{quotation}
%““Legal Services Commission” means the Legal Services Commission established under section 1 of the Access to Justice Act 1999\footnote{1999 c.\ 22.};
%
%“live television link” means a television link or other audio and video facilities which allow a person who is not physically present at an oral hearing to see and hear proceedings and be seen and heard by all others who are present (whether physically present or otherwise);”;
%\end{quotation}
%
%($d$) after the definition of “respondent” insert—
%\begin{quotation}
%““Scottish Legal Aid Board” means the Scottish Legal Aid Board established under section 1 of the Legal Aid (Scotland) Act 1986\footnote{1986 c.\ 47.};”.
%\end{quotation}
%\end{enumerate}
%
%(4) In regulation 8 (manner of and time for service of notices, etc)—
%\begin{enumerate}\item[]
%($a$) after paragraph (1)($b$)  insert—
%\begin{quotation}
%“($ba$) subject to paragraph (1A), sent by email; or”;
%\end{quotation}
%
%($b$) after paragraph (1) insert—
%\begin{quotation}
%“(1A) A document may be served by email on any party if the recipient has informed the person sending the email in writing—
%\begin{enumerate}\item[]
%($a$) that he is willing to accept service by email;
%
%($b$) of the email address to which the documents should be sent; and
%
%($c$) if the recipient wishes to so specify, the electronic format in which documents must be sent.”;
%\end{enumerate}
%\end{quotation}
%
%($c$) in paragraph (2), for “delivered or sent to the office” substitute—
%\begin{quotation}
%    “($a$) 
%    delivered to the office in person;
%
%    ($b$) 
%    sent to the office by prepaid post;
%
%    ($c$) 
%    sent to the office by fax; or
%
%    ($d$) 
%    where the office has given written permission in advance, sent to the office by email”. 
%\end{quotation}
%\end{enumerate}
%
%(5) After regulation 8 insert—
%\begin{quotation}
%\subsection*{“Funding of legal services}
%
%8A.  If a party is granted funding of legal services at any time, he shall—
%\begin{enumerate}\item[]
%($a$) where funding is granted by the Legal Services Commission, send a copy of the funding notice to the office;
%
%($b$) where funding is granted by the Scottish Legal Aid Board, send a copy of the legal aid certificate to the office; and
%
%($c$) notify every other party that funding has been granted.”.
%\end{enumerate}
%\end{quotation}
%
%(6) In regulation 14 (references under the Forfeiture Act 1982)—
%\begin{enumerate}\item[]
%($a$) in paragraph (2)($a$), for “tax credits” substitute “child benefit or guardian’s allowance”;
%
%($b$) in paragraph (3), for “The reference shall” substitute “A reference under this regulation or under regulation 15(2) shall”.
%\end{enumerate}
%
%(7) For regulation 15(2) substitute—
%\begin{quotation}
%“(2) Where the party who referred the forfeiture rule question to a Commissioner under regulation 14(2)—
%\begin{enumerate}\item[]
%($a$) considers that the decision should be superseded; or
%
%($b$) has received a written application for supersession from the person in relation to whom the decision was made,
%\end{enumerate}
%that party shall refer the decision to a Commissioner to determine whether it should be superseded, and shall notify the person to whom the forfeiture rule question relates that the reference has been made.
%
%(3) A Commissioner may supersede any decision on a forfeiture rule question, whether as originally made or as superseded, if—
%\begin{enumerate}\item[]
%($a$) the decision was erroneous in point of law;
%
%($b$) the decision was made in ignorance of, or was based on a mistake as to, some material fact; or
%
%($c$) there has been a relevant change in circumstances since the decision was made.
%\end{enumerate}
%
%(4) A determination by a Commissioner under this regulation shall take effect from the date on which it is made, or from such other date as a Commissioner may direct.”.
%\end{quotation}
%
%(8) Omit regulation 24(6)($e$).
%
%(9) After regulation 24(6) insert—
%\begin{quotation}
%“(6A) Subject to the direction of a Commissioner—
%\begin{enumerate}\item[]
%($a$) any person or organisation entitled to be present and be heard at a hearing; and
%
%($b$) any representatives of such a person or organisation,
%\end{enumerate}
%may be present by means of a live television link.
%
%(6B) Any provision in these Regulations which refers to a party or representative being present is satisfied if the party or representative is present by means of a live television link.”.
%\end{quotation}
%
%(10) In regulation 31(1)—
%\begin{enumerate}\item[]
%($a$) at the end of sub-paragraph ($b$)  omit “or”; and
%
%($b$) omit sub-paragraph ($c$).
%\end{enumerate}
%
%(11) For regulation 33(2) substitute—
%\begin{quotation}
%“(2) Where—
%\begin{enumerate}\item[]
%($a$) any decision or record of a decision is corrected under regulation 30; or
%
%($b$) an application for a decision to be set aside under regulation 31 is refused for reasons other than that the application was made outside the period specified in regulation 31(2),
%\end{enumerate}
%the period specified in paragraph (1) shall run from the date on which written notice of the correction or refusal of the application to set aside is sent to the applicant.”.
%\end{quotation}
%
%(12) In regulation 33(4)($a$)  for “tax credits” substitute “child benefit or guardian’s allowance”.
%
%\subsection[3. Amendments to the Child Support Commissioners (Procedure) Regulations 1999]{Amendments to the Child Support Commissioners (Procedure) Regulations 1999}
%
%3.---(1)  The Child Support Commissioners (Procedure) Regulations 1999\footnote{S.I.\ 1999/1305.} shall be amended in accordance with this regulation.
%
%(2) Under Part I of the Arrangement of Regulations (general provisions), after the entry for regulation 9 insert—
%\begin{quotation}
%“9A.  Funding of legal services”.
%\end{quotation}
%
%(3) In regulation 4 (interpretation)—
%\begin{enumerate}\item[]
%($a$) before the definition of “the Act” insert—
%\begin{quotation}
%““the 1999 Regulations” means the Social Security and Child Support (Decisions and Appeals) Regulations 1999\footnote{S.I.\ 1999/991; frequently amended.};”;
%\end{quotation}
%
%($b$) in the definition of “the chairman”—
%\begin{enumerate}\item[]
%(i) at the end of paragraph (i)  omit “or”; and
%
%(ii) omit paragraph (ii);
%\end{enumerate}
%
%($c$) after the definition of “Commissioner” insert—
%\begin{quotation}
%““funding notice” means the notice or letter from the Legal Services Commission confirming that legal services are to be funded;
%
%“legal aid certificate” means the certificate issued by the Scottish Legal Aid Board confirming that legal services are to be funded;”;
%\end{quotation}
%
%($d$) after the definition of “legally qualified” insert—
%\begin{quotation}
%““Legal Services Commission” means the Legal Services Commission established under section 1 of the Access to Justice Act 1999\footnote{1999 c.\ 22.};
%
%“live television link” means a television link or other audio and video facilities which allow a person who is not physically present at an oral hearing to see and hear proceedings and be seen and heard by all others who are present (whether physically present or otherwise);”;
%\end{quotation}
%
%($e$) after the definition of “office” insert—
%\begin{quotation}
%““panel member” means a person appointed to the panel constituted under section 6 of the Social Security Act 1998 and who—
%\begin{enumerate}\item[]
%(i) 
%has a general qualification (construed in accordance with section 71 of the Courts and Legal Services Act 1990\footnote{1990 c.\ 41. Section 71 was amended by the Access to Justice Act 1999 (c.\ 22) section 43 and paragraphs 4 and 9 to Schedule 6, and section 106 and Part II of Schedule 15.});
%
%(ii) 
%is a member of the Bar of Northern Ireland or a Solicitor of the Supreme Court of Northern Ireland; or
%
%(iii)
%is an advocate or solicitor in Scotland.”;
%\end{enumerate}
%\end{quotation}
%
%($f$) after the definition of “respondent”—
%\begin{enumerate}\item[]
%(i) omit “and”; and
%
%(ii) insert—
%\begin{quotation}
%““Scottish Legal Aid Board” means the Scottish Legal Aid Board established under section 1 of the Legal Aid (Scotland) Act 1986\footnote{1986 c.\ 47.}; and”.
%\end{quotation}
%\end{enumerate}
%\end{enumerate}
%
%(4) In regulation 8 (manner of and time for service of notices, etc)—
%\begin{enumerate}\item[]
%($a$) after paragraph (1)($b$)  insert—
%\begin{quotation}
%“($ba$) subject to paragraph (1A), sent by e-mail; or”;
%\end{quotation}
%
%($b$) after paragraph (1) insert—
%\begin{quotation}
%“(1A) A document may be served by e-mail on any party if the recipient has informed the person sending the e-mail in writing—
%\begin{enumerate}\item[]
%($a$) that he is willing to accept service by e-mail;
%
%($b$) of the e-mail address to which the documents should be sent; and
%
%($c$) if the recipient wishes to so specify, the electronic format in which documents must be sent.”;
%\end{enumerate}
%\end{quotation}
%
%($c$) in paragraph (2), for “delivered or sent to the office” substitute—
%\begin{quotation}
%    “($a$) 
%    delivered to the office in person;
%
%    ($b$) 
%    sent to the office by prepaid post;
%
%    ($c$) 
%    sent to the office by fax; or
%
%    ($d$) 
%    where the office has given written permission in advance, sent to the office by e-mail”. 
%\end{quotation}
%\end{enumerate}
%
%(5) After regulation 9 insert—
%\begin{quotation}
%\subsection*{“Funding of legal services}
%
%9A.  If a party is granted funding of legal services at any time, he shall—
%\begin{enumerate}\item[]
%($a$) where funding is granted by the Legal Services Commission, send a copy of the funding notice to the office;
%
%($b$) where funding is granted by the Scottish Legal Aid Board, send a copy of the legal aid certificate to the office; and
%
%($c$) notify every other party that funding has been granted.”.
%\end{enumerate}
%\end{quotation}
%
%(6) In regulation 10(1), for “An application” substitute “Subject to paragraphs (5) and (7), an application”.
%
%(7) For regulation 10(6) substitute—
%\begin{quotation}
%“(6) Where an application for leave to appeal against a decision of an appeal tribunal is made—
%\begin{enumerate}\item[]
%($a$) if the chairman was a fee-paid panel member, the application may be determined by a salaried panel member; or
%
%($b$) if it is impracticable or would be likely to cause undue delay for the application to be determined by the chairman, the application may be determined by another panel member.”.
%\end{enumerate}
%\end{quotation}
%
%(8) After regulation 10(6) insert—
%\begin{quotation}
%“(7) Where—
%\begin{enumerate}\item[]
%($a$) any decision or the record of a decision is corrected under regulation 56 of the 1999 Regulations; or
%
%($b$) an application for a decision to be set aside under regulation 57 of the 1999 Regulations is refused for reasons other than that the application was made outside the period specified in regulation 57(3) of those Regulations,
%\end{enumerate}
%any time limit specified by this regulation shall run from the date on which notice of the correction or refusal was sent or given to the applicant.”.
%\end{quotation}
%
%(9) After regulation 22(6) insert—
%\begin{quotation}
%“(6A) Subject to the direction of a Commissioner—
%\begin{enumerate}\item[]
%($a$) any person or organisation entitled to be present and be heard at a hearing; and
%
%($b$) any representatives of such a person or organisation,
%\end{enumerate}
%may be present by means of a live television link.
%
%(6B) Any provision in these Regulations which refers to a party or representative being present is satisfied if the party or representative is present by means of a live television link.”.
%\end{quotation}
%
%(10) In regulation 26(5) before “any other information” insert “, so far as practicable,”.
%
%(11) In regulation 28(1)—
%\begin{enumerate}\item[]
%($a$) at the end of sub-paragraph ($b$)  omit “or”; and
%
%($b$) omit sub-paragraph ($c$).
%\end{enumerate}
%
%(12) For regulation 30(2) substitute—
%\begin{quotation}
%“(2) Where—
%\begin{enumerate}\item[]
%($a$) any decision or record of a decision is corrected under regulation 27; or
%
%($b$) an application for a decision to be set aside under regulation 28 is refused for reasons other than that the application was made outside the period specified in regulation 28(2),
%\end{enumerate}
%the period specified in paragraph (1) shall run from the date on which written notice of the correction or refusal of the application to set aside is sent to the applicant.”.
%\end{quotation}
%
%\subsection[4. Amendments to the Social Security Commissioners (Procedure) (Tax Credits Appeals) Regulations 2002]{Amendments to the Social Security Commissioners (Procedure) (Tax Credits Appeals) Regulations 2002}
%
%4.---(1)  The Social Security Commissioners (Procedure) (Tax Credits Appeals) Regulations 2002\footnote{S.I.\ 2002/3237.} shall be amended in accordance with this regulation.
%
%(2) In the Arrangement of Regulations—
%\begin{enumerate}\item[]
%($a$) under Part I (general provisions), after the entry for regulation 6 insert—
%\begin{quotation}
%“6A.  Funding of legal services”.
%\end{quotation}
%
%($b$) under Part II (applications for leave to appeal and appeals)—
%\begin{enumerate}\item[]
%(i) before the entry for regulation 7 insert—
%\begin{quotation}
%“6B.  Application of this Part”; and
%\end{quotation}
%
%(ii) in the entry for regulation 11 omit “after leave obtained”.
%\end{enumerate}
%\end{enumerate}
%
%(3) In regulation 2 (interpretation)—
%\begin{enumerate}\item[]
%($a$) after the definition of “chairman” insert—
%\begin{quotation}
%““funding notice” means the notice or letter from the Legal Services Commission confirming that legal services are to be funded;”;
%\end{quotation}
%
%($b$) after the definition of “joint claimant” insert—
%\begin{quotation}
%““legal aid certificate” means the certificate issued by the Scottish Legal Aid Board confirming that legal services are to be funded;”;
%\end{quotation}
%
%($c$) after the definition of “legally qualified” insert—
%\begin{quotation}
%““Legal Services Commission” means the Legal Services Commission established under section 1 of the Access to Justice Act 1999\footnote{1999 c.\ 22.};
%
%“live television link” means a television link or other audio and video facilities which allow a person who is not physically present at an oral hearing to see and hear proceedings and be seen and heard by all others who are present (whether physically present or otherwise);”;
%\end{quotation}
%
%($d$) after the definition of “respondent” insert—
%\begin{quotation}
%““Scottish Legal Aid Board” means the Scottish Legal Aid Board established under section 1 of the Legal Aid (Scotland) Act 1986\footnote{1986 c.\ 47.};”.
%\end{quotation}
%\end{enumerate}
%
%(4) In regulation 6 (manner of and time for service of notices, etc)—
%\begin{enumerate}\item[]
%($a$) after paragraph (1)($b$)  insert—
%\begin{quotation}
%“($ba$) subject to paragraph (1A), sent by email; or”;
%\end{quotation}
%
%($b$) after paragraph (1) insert—
%\begin{quotation}
%“(1A) A document may be served by email on any party if the recipient has informed the person sending the email in writing—
%\begin{enumerate}\item[]
%($a$) that he is willing to accept service by email;
%
%($b$) of the email address to which the documents should be sent; and
%
%($c$) if the recipient wishes to so specify, the electronic format in which documents must be sent.”;
%\end{enumerate}
%\end{quotation}
%
%($c$) in paragraph (2), for “delivered or sent to the office” substitute—
%\begin{quotation}
%    “($a$) 
%    delivered to the office in person;
%
%    ($b$) 
%    sent to the office by prepaid post;
%
%    ($c$) 
%    sent to the office by fax; or
%
%    ($d$) 
%    where the office has given written permission in advance, sent to the office by email”. 
%\end{quotation}
%\end{enumerate}
%
%(5) After regulation 6 insert—
%\begin{quotation}
%\subsection*{“Funding of legal services}
%
%6A.  If a party is granted funding of legal services at any time, he shall—
%\begin{enumerate}\item[]
%($a$) where funding is granted by the Legal Services Commission, send a copy of the funding notice to the office;
%
%($b$) where funding is granted by the Scottish Legal Aid Board, send a copy of the legal aid certificate to the office; and
%
%($c$) notify every other party that funding has been granted.”.
%\end{enumerate}
%\end{quotation}
%
%(6) In Part II (applications for leave to appeal and appeals), before regulation 7 insert—
%\begin{quotation}
%\subsection*{“Application of this Part}
%
%6B.  In this Part—
%\begin{enumerate}\item[]
%($a$) regulations 7, 8 and 9 apply to appeals other than an appeal against a determination in penalty proceedings;
%
%($b$) regulations 10, 11 and 12 apply to all appeals.”.
%\end{enumerate}
%\end{quotation}
%
%(7) In regulation 10(1)($b$), before “the date” insert “where applicable,”.
%
%(8) In the heading to regulation 11, omit “after leave obtained”.
%
%(9) For regulation 11 substitute—
%\begin{quotation}
%“11.---(1)  In the case of an appeal against a determination in penalty proceedings, the notice of appeal shall not be valid unless it is sent to a Commissioner within one month of the decision of the appeal tribunal being sent to the applicant.
%
%(2) For all other appeals, a notice of appeal shall not be valid unless it is sent to a Commissioner within one month of the date on which the appellant was sent written notice that leave to appeal had been granted.
%
%(3) A Commissioner may for special reasons accept late notice of appeal.”.
%\end{quotation}
%
%(10) After regulation 19(6) insert—
%\begin{quotation}
%“(6A) Subject to the direction of a Commissioner—
%\begin{enumerate}\item[]
%($a$) any person or organisation entitled to be present and be heard at a hearing; and
%
%($b$) any representatives of such a person or organisation,
%\end{enumerate}
%may be present by means of a live television link.
%
%(6B) Any provision in these Regulations which refers to a party or representative being present is satisfied if the party or representative is present by means of a live television link.”.
%\end{quotation}
%
%(11) For regulation 27(2) substitute—
%\begin{quotation}
%“(2) Where—
%\begin{enumerate}\item[]
%($a$) any decision or record of a decision is corrected under regulation 24; or
%
%($b$) an application for a decision to be set aside under regulation 25 is refused for reasons other than that the application was made outside the period specified in regulation 25(2),
%\end{enumerate}
%the period specified in paragraph (1) shall run from the date on which written notice of the correction or refusal of the application to set aside is sent to the applicant.”.
%\end{quotation}
%
%\bigskip
%
%Signed 
%by authority of the 
%%Secretary of State for Work and Pensions.
%Lord Chancellor.
%
%{\raggedleft
%\emph{Cathy Ashton}\\*Parliamentary Under-Secretary of State,\\*Department for Constitutional Affairs
%
%}
%
%%St Andrew's House, Edinburgh
%
%%Dated
%3rd February 2005
%
%\small
%
%\part{Explanatory Note}
%
%\renewcommand\parthead{— Explanatory Note}
%
%\subsection*{(This note is not part of the Regulations)}
%
%These Regulations amend the Social Security Commissioners (Procedure) Regulations 1999 (S.I.\ 1999/1495), the Child Support Commissioners (Procedure) Regulations 1999 (S.I.\ 1999/1305) and the Social Security Commissioners (Procedure) (Tax Credits Appeals) Regulations 2002 (S.I.\ 2002/3237) to:
%\begin{itemize}\item[]
%    provide for service of documents by e-mail (regulations 2(4), 3(4) and 4(4));
%
%    provide for notice to be given where a party receives funding of legal services (regulations 2(5), 3(5) and 4(5));
%
%    provide for people, organisations or their representatives to be present at a hearing by means of a live television link (regulations 2(9), 3(9) and 4(10));
%
%    provide that where a refusal to set aside a decision is due to the application to set aside being out of time, the time during which an appeal must be made is not affected (regulations 2(11), 3(12) and 4(11)). 
%\end{itemize}
%
%Regulation 2 also amends the Social Security Commissioners (Procedure) Regulations 1999 to:
%\begin{enumerate}\item[]
%    remove references to tax credits under the Tax Credits Act 1999 (c.\ 10) which are now repealed and replace them with references to child benefit and guardian’s allowance, both of which are now administered by the Commissioners of Inland Revenue (regulations 2(6)($a$)  and 2(12);
%
%    make express provision for the supersession of decisions on forfeiture (regulation 2(7));
%
%    remove a reference to benefits which are no longer appealed the Social Security Commissioners (regulation 2(8)). 
%\end{enumerate}
%
%Regulation 3 also amends the Child Support Commissioners (Procedure) Regulations 1999 to:
%\begin{enumerate}\item[]
%    prescribe other people in addition to an appeal tribunal chairman who may grant leave to appeal to the Child Support Commissioners (regulation 3(7));
%
%    provide that the time for applying to a chairman for leave to appeal to the Child Support Commissioners does not take account of any time before an error was corrected or before a decision not to set aside (unless because the application was out of time) was taken by the appeal tribunal (regulation 3(8)). 
%\end{enumerate}
%
%Regulation 4 also amends the Social Security Commissioners (Procedure) (Tax Credits Appeals) Regulations 2002 to:
%\begin{enumerate}\item[]
%    clarify which provisions in Part II (applications for leave to appeal and appeals) apply to penalty proceedings (regulation 4(6));
%
%    provide that details concerning leave to appeal are only included in the notice of appeal where applicable, i.e. in all cases apart from those involving penalty proceedings (regulation 4(7));
%
%    substitute regulation 11 to provide for time limits within which an appellant must commence the appeal (regulation 4(9)). 
%\end{enumerate}

\end{document}