\documentclass[12pt,a4paper]{article}

\newcommand\regstitle{The State Pension Credit (Consequential, Transitional and Miscellaneous Provisions) Regulations 2002}

\newcommand\regsnumber{2002/3019}

%\opt{newrules}{
\title{\regstitle}
%}

%\opt{2012rules}{
%\title{Child Maintenance and Other Payments Act 2008\\(2012 scheme version)}
%}

\author{S.I.\ 2002 No.\ 3019}

\date{Made
4th December 2002\\
Laid before Parliament
10th December 2002\\
Coming into force\\
for the purposes of Parts I, II, III and VII
7th April 2003\\
for all other purposes
6th October 2003
}

%\opt{oldrules}{\newcommand\versionyear{1993}}
%\opt{newrules}{\newcommand\versionyear{2003}}
%\opt{2012rules}{\newcommand\versionyear{2012}}

\usepackage{csa-regs}

\setlength\headheight{27.57402pt}

\begin{document}

\maketitle

\noindent
The Treasury, with the concurrence of the Secretary of State, in relation to regulation 23($i$), and the Secretary of State in relation to the remainder of the Regulations, in exercise of the powers conferred upon them by the powers specified in the Schedule to this Instrument, by this Instrument, which contains only regulations made by virtue of or consequential upon sections 1 to 17 of the State Pension Credit Act 2002\footnote{2002 c.\ 16.} and which is made before the end of the period of 6 months beginning with the coming into force of those provisions\footnote{\emph{See} section 173(5)($b$) of the Social Security Administration Act 1992 (c.\ 5).}, hereby make the following Regulations: 

{\sloppy

\tableofcontents

}

\bigskip

\setcounter{secnumdepth}{-2}

\section[Part I --- General]{Part I\\*General}

\subsection[1. Citation, commencement and interpretation]{Citation, commencement and interpretation}

\renewcommand\parthead{--- Part I}

1.---(1)  These Regulations may be cited as the State Pension Credit (Consequential, Transitional and Miscellaneous Provisions) Regulations 2002.

(2) These Regulations shall come into force—
\begin{enumerate}\item[]
($a$) for the purposes of this Part and Parts II, III and VII on 7th April 2003;

($b$) for all other purposes on 6th October 2003.
\end{enumerate}

(3) In these Regulations—
\begin{enumerate}\item[]
“the Act” means the State Pension Credit Act 2002\footnote{2002 c.\ 16.};

“the Administration Act” means the Social Security Administration Act 1992\footnote{1992 c.\ 5.};

“the 1998 Act” means the Social Security Act 1998\footnote{1998 c.\ 14.};

“the Claims and Payments Regulations” means the Social Security (Claims and Payments) Regulations 1987\footnote{S.I.\ 1987/1968.};

“the Decisions and Appeals Regulations” means the Social Security and Child Support (Decisions and Appeals) Regulations 1999\footnote{S.I.\ 1999/991.};

“the appointed day” means the day appointed under section 13(3) of the Act;

“the State Pension Credit Regulations” means the State Pension Credit Regulations 2002\footnote{S.I.\ 2002/1792.}.
\end{enumerate}

\section[Part II --- Amendments to the Claims and Payments Regulations]{Part II\\*Amendments to the Claims and Payments Regulations}

\subsection[2. Interpretation of Part II]{Interpretation of Part II}

\renewcommand\parthead{--- Part II}

2.  The Claims and Payments Regulations shall be amended in accordance with the following provisions of this Part; and in this Part, unless the context otherwise requires, any reference to a regulation or a Schedule is to the regulation or Schedule bearing that number in the Claims and Payments Regulations.

\subsection[3. Amendment of regulation 2]{Amendment of regulation 2}

3.  In regulation 2—
\begin{enumerate}\item[]
($a$) in paragraph (1)—
\begin{enumerate}\item[]
(i) immediately before the definition of “appropriate office” insert—
\begin{quotation}
““the 2002 Act” means the State Pension Credit Act 2002;

“advance period” means the period specified in regulation 4E(2);”;
\end{quotation}

(ii) immediately before the definition of “instrument for benefit payment” insert—
\begin{quotation}
    ““guarantee credit” is to be construed in accordance with sections 1 and 2 of the 2002 Act;”; 
\end{quotation}

(iii) after the definition of “personal pension scheme”\footnote{Inserted by S.I.\ 1995/2303.} insert—
\begin{quotation}
    ““qualifying age” has the same meaning as in the 2002 Act by virtue of section 1(6) of that Act;”; 
\end{quotation}

(iv) after the definition of “retirement annuity contract”\footnote{Inserted by S.I.\ 1995/2303.}, insert—
\begin{quotation}
““state pension credit” means state pension credit under the 2002 Act;

“State Pension Credit Regulations” means the State Pension Credit Regulations 2002\footnote{S.I.\ 2002/1792.};”;
\end{quotation}
\end{enumerate}

($b$) in paragraph (2), in sub-paragraph ($b$), after the words “income support”, insert “, state pension credit”;

($c$) after paragraph (3), insert—
\begin{quotation}
“(4) In these Regulations, references to “beneficiary” include any person entitled to state pension credit.”.
\end{quotation}
\end{enumerate}

\subsection[4. Claims for state pension credit]{Claims for state pension credit}

4.---(1)  In regulation 4 (making a claim for benefit), at the end, insert—
\begin{quotation}
“(10) This regulation shall not apply to a claim for state pension credit.”.
\end{quotation}

(2) In regulation 4B\footnote{Regulations 4A and 4B were inserted by S.I.\ 1999/3108 and regulation 4C by S.I.\ 2002/1789.} (forwarding claims and information), in paragraph (1), in sub-paragraph ($b$), after the word “applies” insert “or for state pension credit”.

(3) After regulation 4C, insert—
\begin{quotation}
\subsection*{“Making a claim for state pension credit}

4D.---(1)  A claim for state pension credit need only be made in writing if the Secretary of State so directs in any particular case.

(2) A claim is made in writing either—
\begin{enumerate}\item[]
($a$) by completing and returning in accordance with the instructions printed on it a form approved or provided by the Secretary of State for the purpose; or

($b$) in such other written form as the Secretary of State accepts as sufficient in the circumstances of the case.
\end{enumerate}

(3) A claim for state pension credit may be made in writing whether or not a direction is issued under paragraph (1) and may also be made by telephone to, or in person at, an appropriate office or other office designated by the Secretary of State for accepting claims for state pension credit.

(4) A claim made in writing may also be made at the offices of—
\begin{enumerate}\item[]
($a$) a local authority administering housing benefit or council tax benefit;

($b$) a person providing services to such an authority; or

($c$) a person authorised to exercise any function of a local authority relating to housing benefit or council tax benefit.
\end{enumerate}

(5) Any claim made in accordance with paragraph (4), together with any information and evidence supplied in connection with making the claim, shall be forwarded as soon as reasonably practicable to the Secretary of State by the person who received the claim.

(6) A claim for state pension credit made in person or by telephone is not a valid claim unless a written statement of the claimant’s circumstances, provided for the purpose by the Secretary of State, is approved by the person making the claim.

(7) A married or unmarried couple may agree between them as to which partner is to make a claim for state pension credit, but in the absence of an agreement, the Secretary of State shall decide which of them is to make the claim.

(8) Where one member of a married or unmarried couple (“the former claimant”) is entitled to state pension credit under an award but a claim for state pension credit is made by the other member of the couple, then, if both members of the couple confirm in writing that they wish the claimant to be the other member, the former claimant’s entitlement shall terminate on the last day of the benefit week specified in paragraph (9).

(9) That benefit week is the benefit week of the former claimant which includes the day immediately preceding the day the partner’s claim is actually made or, if earlier, is treated as made.

(10) If a claim for state pension credit is defective when first received, the Secretary of State is to provide the person making it with an opportunity to correct the defect.

(11) If that person corrects the defect so that the claim then satisfies the requirements of paragraph (2) and does so within 1 month of the date the Secretary of State last drew attention to the defect, the claim shall be treated as having been properly made on the date—
\begin{enumerate}\item[]
($a$) the defective claim was first received by the Secretary of State or the person acting on his behalf; or

($b$) if regulation 4F(3) applies, the person informed an appropriate office of his intention to claim state pension credit.
\end{enumerate}

(12) Paragraph (11) does not apply in a case to which regulation 4E(3) applies.

(13) State pension credit is a relevant benefit for the purposes of section 7A of the Social Security Administration Act 1992\footnote{Section 7A was inserted by the Welfare Reform and Pensions Act 1999 (c.\ 30), section 71.}.

\subsection*{Making a claim before attaining the qualifying age}

4E.---(1)  A claim for state pension credit may be made, and any claim made may be determined, at any time within the advance period.

(2) The advance period begins on the date which falls 4 months before the day on which the claimant attains the qualifying age and ends on the day before he attains that age.

(3) A person who makes a claim within the advance period which is defective may correct the defect at any time before the end of the advance period.

\subsection*{Making a claim after attaining the qualifying age: date of claim}

4F.---(1)  This regulation applies in the case of a person who claims state pension credit on or after attaining the qualifying age.

(2) The date on which a claim is made shall, subject to paragraph (3), be—
\begin{enumerate}\item[]
($a$) where the claim is made in writing and is not defective, the date on which the claim is first received—
\begin{enumerate}\item[]
(i) by the Secretary of State or the person acting on his behalf; or

(ii) in a case to which regulation 4D(4) relates, in the office of a person specified therein;
\end{enumerate}

($b$) where the claim is not made in writing but is otherwise made in accordance with regulation 4D(3) and is not defective, the date the claimant provides details of his circumstances by telephone to, or in person at, the appropriate office or other office designated by the Secretary of State to accept claims for state pension credit; or

($c$) where a claim is initially defective but the defect is corrected under regulation 4D(11), the date the claim is treated as having been made under that regulation.
\end{enumerate}

(3) If a claimant—
\begin{enumerate}\item[]
($a$) informs an appropriate office of his intention to claim state pension credit; and

($b$) subsequently makes the claim in accordance with regulation 4D within 1 month of complying with sub-paragraph ($a$), or within such longer period as the Secretary of State may allow,
\end{enumerate}
the claim may, where in the circumstances of the particular case it is appropriate to do so, be treated as made on the day the claimant first informed the appropriate office of his intention to claim the credit.”.
\end{quotation}

\subsection[5. Evidence and information]{Evidence and information}

5.  In regulation 7 (evidence and information)—
\begin{enumerate}\item[]
($a$) after paragraph (1), insert—
\begin{quotation}
“(1A) A claimant shall furnish such information and evidence as the Secretary of State may require as to the likelihood of future changes in his circumstances which is needed to determine—
\begin{enumerate}\item[]
($a$) whether a period should be specified as an assessed income period under section 6 of the 2002 Act in relation to any decision; and

($b$) if so, the length of the period to be so specified.
\end{enumerate}

(1B) The information and evidence required under paragraph (1A) shall be furnished within 1 month of the Secretary of State notifying the claimant of the requirement, or within such longer period as the Secretary of State considers reasonable in the claimant’s case.

(1C) In the case of a claimant making a claim for state pension credit in the advance period, time begins to run for the purposes of paragraphs (1) and (1B) on the day following the end of that period.”;
\end{quotation}

($b$) in paragraph (4)\footnote{Amended by S.I.\ 1995/2303, 1996/1460 and 1999/2572.}, for the words “or jobseeker’s allowance”, substitute “jobseeker’s allowance or state pension credit”.
\end{enumerate}

\subsection[6. Advance claims and awards of state pension credit]{Advance claims and awards of state pension credit}

6.---(1)  In regulation 13 (advance claims and awards), in paragraph (3)\footnote{Amended by S.I.\ 1991/2741, 1994/2319 and 1999/2572.}, after the words “disabled person’s tax credit”\footnote{Inserted by S.I.\ 1999/2572.} insert “state pension credit”.

(2) After regulation 13C\footnote{Regulations 13A to 13C inserted by S.I.\ 1991/2741.} (further claim for and award of disability living allowance), insert—
\begin{quotation}
\subsection*{\sloppy “Advance claims for and awards of state pension credit}

13D.---(1)  Paragraph (2) applies if—
\begin{enumerate}\item[]
($a$) a person does not satisfy the requirements for entitlement to state pension credit on the date on which the claim is made; and

($b$) the Secretary of State is of the opinion that unless there is a change of circumstances he will satisfy those requirements—
\begin{enumerate}\item[]
(i) where the claim is made in the advance period, when he attains the qualifying age; or

(ii) in any other case, within 4 months of the date on which the claim is made.
\end{enumerate}
\end{enumerate}

(2) Where this paragraph applies, the Secretary of State may—
\begin{enumerate}\item[]
($a$) treat the claim as made for a period beginning on the day (“the relevant day”) the claimant—
\begin{enumerate}\item[]
(i) attains the qualifying age, where the claim is made in the advance period; or

(ii) is likely to satisfy the requirements for entitlement in any other case; and
\end{enumerate}

($b$) if appropriate, award state pension credit accordingly, subject to the condition that the person satisfies the requirements for entitlement on the relevant day.
\end{enumerate}

(3) An award under paragraph (2) may be revised under section 9 of the Social Security Act 1998\footnote{1998 c.\ 14.} if the claimant fails to satisfy the conditions for entitlement to state pension credit on the relevant day.”.
\end{quotation}

\subsection[7. Payability of state pension credit]{Payability of state pension credit}

7.---(1)  In regulation 16 (date of entitlement under an award), in paragraph (4) after the words “income support”, insert “, state pension credit”.

(2) After regulation 16, insert—
\begin{quotation}
\subsection*{“Date of entitlement under an award of state pension credit for the purpose of payability and effective date of change of rate}

16A.---(1)  For the purpose only of determining the day from which state pension credit is to become payable, where the credit is awarded from a day which is not the first day of the claimant’s benefit week, entitlement shall begin on the first day of the benefit week next following.

(2) In the case of a claimant who—
\begin{enumerate}\item[]
($a$) immediately before attaining the qualifying age was entitled to income support or income-based jobseeker’s allowance and is awarded state pension credit from the day on which he attains the qualifying age; or

($b$) was entitled to an income-based jobseeker’s allowance after attaining the qualifying age and is awarded state pension credit from the day which falls after the date that entitlement ends,
\end{enumerate}
entitlement to the guarantee credit shall, notwithstanding paragraph (1), begin on the first day of the award.

(3) Where a change in the rate of state pension credit would otherwise take effect on a day which is not the first day of the claimant’s benefit week, the change shall take effect from the first day of the benefit week next following.

(4) For the purpose of this regulation, “benefit week” means the period of 7 days beginning on the day on which, in the claimant’s case, state pension credit is payable in accordance with regulation 26B.”.
\end{quotation}

\subsection[8. Amendment of regulations 17 and 19]{Amendment of regulations 17 and 19}

8.---(1)  In regulation 17 (duration of awards), in paragraph (3)\footnote{Amended by S.I.\ 1996/1460.}, at the beginning insert “Except in the case of claims for and awards of state pension credit,”.

(2) In regulation 19\footnote{Substituted by S.I.\ 1997/793; paragraph (3) amended by S.I.\ 2000/1483.} (time for claiming benefit), in paragraph (3), after sub-paragraph ($f$), insert—
\begin{quotation}
“($ff$) state pension credit;”.
\end{quotation}

\subsection[9. Payment]{Payment}

9.  After regulation 26A\footnote{Regulation 26A was inserted by S.I.\ 1996/1460.} (jobseeker’s allowance), insert—

\begin{quotation}
\subsection*{“State pension credit}

26B.---(1)  Except where paragraph (2) applies, state pension credit shall be payable on Mondays, but subject, where state pension credit is payable in accordance with paragraph (3)($a$), to the provisions of regulation 21 (direct credit transfer).

(2) State pension credit shall be payable—
\begin{enumerate}\item[]
($a$) if retirement pension is payable to the claimant, on the same day as the retirement pension is payable; or

($b$) on such other day of the week as the Secretary of State may, in the particular circumstances of the case, determine.
\end{enumerate}

(3) Payment of state pension credit shall be made either—
\begin{enumerate}\item[]
($a$) in accordance with regulation 21 (direct credit transfer); or

($b$) by means of an instrument of payment or an instrument for benefit payment at such place as the Secretary of State, after enquiry of the claimant, may from time to time specify.
\end{enumerate}

(4) State pension credit paid in accordance with paragraph (3)($b$)  shall be paid weekly in advance.

(5) Where the amount of state pension credit payable is less than £1$.$00 per week, the Secretary of State may direct that it shall be paid at such intervals, not exceeding 13 weeks, as may be specified in the direction.

(6) Where state pension credit is—
\begin{enumerate}\item[]
($a$) paid by means of a book of serial orders; and

($b$) increased or reduced by an amount which, when added to any previous such increase, is less than 50 pence per week,
\end{enumerate}
the Secretary of State may defer payment of that increase or disregard the reduction until either—
\begin{enumerate}\item[]
(i) the termination of entitlement; or, if earlier,

(ii) the expiration of one week from the date specified for payment of the last order in that book.
\end{enumerate}

(7) Where state pension credit is—
\begin{enumerate}\item[]
($a$) paid by means of a book of serial orders; and

($b$) the amount of state pension credit payable to a third party under Schedule 9 is increased so that the amount of the credit payable to the claimant is reduced by an amount which, with any previous reduction, is less than 50 pence per week,
\end{enumerate}
the Secretary of State may make the payment to the third party and disregard the reduction in the claimant’s state pension credit for the remainder of the period to which the book relates.”.
\end{quotation}

\subsection[10. Amendment of regulation 30]{Amendment of regulation 30}

10.  In regulation 30 (payments on death), in paragraph (5), after the words “income support” insert “, state pension credit”.

\subsection[11. Amendment of regulation 32]{Amendment of regulation 32}

11.  In regulation 32\footnote{Amended by S.I.\ 1992/2595, 1995/2303, 1996/1460 and 1999/2572.} (information to be given when obtaining payment of benefit)—
\begin{enumerate}\item[]
($a$) in paragraph (3), after the words “income support”, insert “, state pension credit”; and

($b$) at the end add—
\begin{quotation}
“(6) This regulation shall apply in the case of state pension credit subject to the following modifications—
\begin{enumerate}\item[]
($a$) at the end of an assessed income period, the information and evidence required to be notified in accordance with this regulation includes information and evidence as to the likelihood of future changes in the claimant’s circumstances needed to determine—
\begin{enumerate}\item[]
(i) whether a period should be specified as an assessed income period under section 6 of the 2002 Act in relation to any decision; and

(ii) if so, the length of the period to be so specified; and
\end{enumerate}

($b$) except to the extent that sub-paragraph ($a$)  applies, changes to an element of the claimant’s retirement provision need not be notified if an assessed income period is current in his case.”.
\end{enumerate}
\end{quotation}
\end{enumerate}

\subsection[12. Amendment of regulation 34A]{Amendment of regulation 34A}

12.  In regulation 34A\footnote{Regulation 34A was inserted by S.I.\ 1992/1026.} (deduction of mortgage interest which shall be made from benefit and paid to qualifying lenders), in paragraph (1), for the words “In relation to cases to which section 51C(1) of the Social Security Act 1986”, substitute “In relation to cases to which section 15A(1) or (1A) of the Social Security Administration Act 1992\footnote{Section 15A(1A) was inserted by the State Pension Credit Act 2002 (c.\ 16), Schedule 2, paragraph 9(2).}”.

\subsection[13. Amendment of regulation 35A]{Amendment of regulation 35A}

13.  In regulation 35A\footnote{Inserted by S.I.\ 1989/1686.} (transitional provisions for persons in hostels or certain residential accommodation), in paragraph (1), in the definition of “specified benefit”, after the words “paragraph 1”, insert “except that it does not include state pension credit”.

\subsection[14. Amendment of Schedules 9 to 9B]{Amendment of Schedules 9 to 9B}

14.---(1)  In Schedule 9 (deductions from benefit and direct payment to third parties)—
\begin{enumerate}\item[]
($a$) in paragraph 1, in sub-paragraph (1)\footnote{Relevant amending Instruments are S.I.\ 1991/2284, 1996/672 and 1996/1460.}—
\begin{enumerate}\item[]
(i) in the definition of “family”, at the end, add “and for the purposes of state pension credit “a family” comprises the claimant, his partner, any additional partner to whom section 12(1)($c$)  of the 2002 Act applies and any person who has not attained the age of 19, is treated as a child for the purposes of section 142 of the Contributions and Benefits Act and lives with the claimant or the claimant’s partner;”;

(ii) in the definition of “housing costs”, at the end, add—
\begin{quotation}
“($c$) Schedule II to the State Pension Credit Regulations but—
\begin{enumerate}\item[]
(i) excludes costs under paragraph 13(1)($f$)  of that Schedule (tents and sites); and

(ii) includes costs under paragraphs 13(1)($a$)  (ground rent and feu duty) and 13(1)($c$)  (rent charges) of that Schedule but only when they are paid with costs under paragraph 13(1)($b$)  of that Schedule (service charges);”;
\end{enumerate}
\end{quotation}

(iii) in the definition of “mortgage payment”, after head ($b$), insert—
\begin{quotation}
    “or

    ($c$) 
    Schedule II to the State Pension Credit Regulations in accordance with paragraph 7 of that Schedule (housing costs to be met in state pension credit) on a loan which qualifies under paragraph 11 or 12 of that Schedule, but less any amount deducted under paragraph 14 of that Schedule (non-dependant deductions),”; 
\end{quotation}

(iv) in the definition of “personal allowance for a single claimant aged not less than 25 years”, after the words “amount specified” insert “in connection with income support and state pension credit” and for the words “as the case may be” substitute “in connection with jobseeker’s allowance”;

(v) in the definition of “specified benefit”, after head ($c$), insert—
\begin{quotation}
“($d$) state pension credit which is either paid alone or paid together with any retirement pension, incapacity benefit or severe disablement allowance in a combined payment in respect of any period;”;
\end{quotation}
\end{enumerate}

($b$) in paragraph 3—
\begin{enumerate}\item[]
(i) in sub-paragraphs (1) and (2A)($a$)\footnote{Relevant amending Instruments are S.I.\ 1992/1026, 1995/1613, 1995/2927, 1996/1460, 1999/2860 and 1999/3178.}, after the words “applicable amount” wherever they occur insert “or appropriate minimum guarantee”;

(ii) in sub-paragraph (2A), after the words “Jobseeker’s Allowance Regulations” in both places in which they occur, insert “or paragraph 5(9) or (12) or paragraph 14 of Schedule II to the State Pension Credit Regulations”;
\end{enumerate}

($c$) in paragraph 5\footnote{Relevant amending Instrument is S.I.\ 1996/1460.} after sub-paragraph (5) insert—
\begin{quotation}
“(5A) In the case of state pension credit, a determination under this paragraph shall not be made without the consent of the beneficiary if the aggregate amount determined in accordance with sub-paragraphs (3) and (6) exceeds a sum equal to 25 per cent. of the appropriate minimum guarantee less any housing costs under Schedule II to the\ State Pension Credit Regulations which may be applicable in the particular case.”;
\end{quotation}

($d$) in paragraph 6\footnote{Relevant amending Instrument is S.I.\ 1996/1460.}—
\begin{enumerate}\item[]
(i) in sub-paragraph (1), for the words “Subject to sub-paragraph (6)”, substitute “Subject to sub-paragraphs (6) and (6A)”;

(ii) after sub-paragraph (6), insert—
\begin{quotation}
“(6A) Subject to paragraph 8, in the case of state pension credit, a determination under this paragraph shall not be made without the consent of the beneficiary if the aggregate amount calculated in accordance with sub-paragraph (2) exceeds a sum equal to 25 per cent.\ of the appropriate minimum guarantee less any housing costs under Schedule II to the State Pension Credit Regulations which may be applicable in the particular case.”;
\end{quotation}
\end{enumerate}

($e$) in paragraph 7\footnote{Paragraph 7 was substituted by S.I.\ 1991/2284 and amended by S.I.\ 1992/2595, 1994/2319, 1996/1460 and 1999/2860 and 3178.}, after sub-paragraph (8), add—
\begin{quotation}
“(9) Subject to paragraph 8, in the case of state pension credit, a determination under this paragraph shall not be made without the consent of the beneficiary if the aggregate amount calculated in accordance with sub-paragraphs (3), (4), (5) and (6) exceeds a sum equal to 25 per cent.\ of the appropriate minimum guarantee less any housing costs under Schedule II to the State Pension Credit Regulations which may be applicable in the particular case.”;
\end{quotation}

($f$) in paragraph 8\footnote{Paragraph 8 was amended by S.I.\ 1990/2205, 1991/2284 and 1996/1460.}, after sub-paragraph (2), insert—
\begin{quotation}
“(2A) In the case of state pension credit, the maximum aggregate amount payable under paragraphs 3(2)($a$), 5, 6, and 7 shall not, without the consent of the beneficiary, exceed a sum equal to 25 per cent.\ of the appropriate minimum guarantee less any housing costs under Schedule II to the State Pension Credit Regulations which may be applicable in the particular case.”.
\end{quotation}
\end{enumerate}

(2) In Schedule 9A\footnote{Schedule 9A was inserted by S.I.\ 1992/1026 and amended by S.I.\ 1995/1613 and 1996/1460.} (deductions of mortgage interest from benefit and payment to qualifying lenders)—
\begin{enumerate}\item[]
($a$) in paragraph 1, in the definition of “relevant benefits”, at the end of sub-paragraph ($c$), insert—
\begin{quotation}
    “and

    ($d$) 
    state pension credit which is either paid alone or paid together with any retirement pension, incapacity benefit or severe disablement allowance in a combined payment in respect of any period;”; 
\end{quotation}

($b$) in paragraph 2 for the words from the beginning of sub-paragraph ($a$)  to “is determined”, substitute—
\begin{quotation}
    “the amount to be met under—
\begin{enumerate}\item[]
    (i) 
    Schedule 3 to the Income Support Regulations; or

    (ii) 
    Schedule 2 to the Jobseeker’s Allowance Regulations; or

    (iii) 
    Schedule II to the State Pension Credit Regulations,”; 
\end{enumerate}
\end{quotation}

($c$) in paragraph 3—
\begin{enumerate}\item[]
(i) after sub-paragraph (1), insert—
\begin{quotation}
“(1A) Subject to the following provisions of this paragraph, the part of state pension credit which, as determined by the Secretary of State in accordance with regulation 34A, shall be paid directly to the qualifying lender, is a sum equal to the amount of mortgage interest to be met under paragraph 7 of Schedule II to the State Pension Credit Regulations.”;
\end{quotation}

(ii) in sub-paragraph (3)—
\begin{enumerate}\item[]
($aa$) after the words “or income-based jobseeker’s allowance” insert “or a relevant beneficiary’s appropriate minimum guarantee in state pension credit” and for the words “sub-paragraph (1)” substitute “sub-paragraph (1) or (1A)” ;

($bb$) in head ($b$), after the words “as the case may be” insert “paragraph 5(9) or (12) or paragraph 14 of Schedule II to the State Pension Credit Regulations or”;

($cc$) in the value “A”, after the words “as the case may be” insert “paragraph 1 of Schedule II to the State Pension Credit Regulations or”;

($dd$) in the value “B”, after the words “as the case may be” insert “paragraph 7 of Schedule II to the State Pension Credit Regulations or”;

($ee$) in the value “C”, after the words “as the case may be” insert “paragraph 5(9) or (12) or paragraph 14 of Schedule II to the State Pension Credit Regulations or”;
\end{enumerate}

(iii) in sub-paragraph (4), at the beginning, insert “Except where the relevant benefit is state pension credit,”;

(iv) after sub-paragraph (9), add—
\begin{quotation}
“(10) In sub-paragraph (1), the relevant benefits do not include in the case of state pension credit so much of any additional amount which is applicable in the claimant’s case under Schedule II to the State Pension Credit Regulations (housing costs) in respect of a period before the decision awarding state pension credit was made.”;
\end{quotation}
\end{enumerate}

($d$) in paragraph 4, in sub-paragraph (2)($a$)\footnote{Relevant amending Instruments are S.I.\ 1995/1613, 1996/1460 and 1997/827.}, after the words “as the case may be” insert “paragraph 9 of Schedule II to the State Pension Credit Regulations or”;

($e$) in paragraph 10—
\begin{enumerate}\item[]
(i) for sub-paragraph (2), substitute—
\begin{quotation}
“(2) Subject to sub-paragraph (4), the information referred to in heads ($a$), ($b$), ($c$)  and ($d$)  of sub-paragraph (1) shall be provided at the request of the Secretary of State when a claim for—
\begin{enumerate}\item[]
($a$) income support or income-based jobseeker’s allowance is made and a sum in respect of mortgage interest is to be brought into account in determining the applicable amount; or

($b$) state pension credit is made and a sum in respect of housing costs is applicable in the claimant’s case in accordance with regulation 6(6)($c$)  of the State Pension Credit Regulations.”;
\end{enumerate}
\end{quotation}

(ii) in sub-paragraph (3), in head ($a$), after the words “income support” insert “, state pension credit”.
\end{enumerate}
\end{enumerate}

(3) In Schedule 9B\footnote{Schedule 9B was inserted by S.I.\ 2001/18.}, in paragraphs 2(1), 3(1), 5(1) and 6(1), after the words “income support” insert “, state pension credit”.

\section[Part III --- Amendments to the Decisions and Appeals Regulations]{Part III\\*Amendments to the Decisions and Appeals Regulations}

\subsection[15. Interpretation of Part III]{Interpretation of Part III}

\renewcommand\parthead{--- Part III}

15.  The Decisions and Appeals Regulations shall be amended in accordance with the following provisions of this Part; and in this Part, unless the context otherwise requires, references to regulations and Schedules are references to regulations of and Schedules to the Decisions and Appeals Regulations.

\subsection[16. Amendment of regulation 1]{Amendment of regulation 1}

16.  In regulation 1(3) (interpretation)—
\begin{enumerate}\item[]
($a$) after the definition of “the Arrears, Interest and Adjustment of Maintenance Assessment Regulations”, insert—
\begin{quotation}
    ““assessed income period” is to be construed in accordance with sections 6 and 9 of the State Pension Credit Act;”; 
\end{quotation}

($b$) in the definition of “claimant” for the words “or section 35(1) of the Jobseekers Act”, substitute “section 35(1) of the Jobseekers Act or section 17(1) of the State Pension Credit Act”;

($c$) after the definition of “relevant credit”\footnote{Inserted by S.I.\ 2000/1596.}, insert the following definitions—
\begin{quotation}
““state pension credit” means the benefit payable under the State Pension Credit Act;

“State Pension Credit Act” means the State Pension Credit Act 2002\footnote{2002 c.\ 16.};

“State Pension Credit Regulations” means the State Pension Credit Regulations 2002\footnote{S.I.\ 2002/1792.};”.
\end{quotation}
\end{enumerate}

\subsection[17. Amendment of regulation 6]{Amendment of regulation 6}

17.  In regulation 6\footnote{Relevant amending Instruments are: S.I.\ 1999/1623, 2570 and 2677, 2000/897, 1596 and 1982, 2001/1711 and 2002/428 and 490.} (supersession of decisions)—
\begin{enumerate}\item[]
($a$) in paragraph (2), after sub-paragraph ($k$), add—
\begin{quotation}
“($l$) is a relevant decision for the purposes of section 6 of the State Pension Credit Act and—
\begin{enumerate}\item[]
(i) on making that decision, the Secretary of State specified a period as the assessed income period; and

(ii) that period has ended or is about to end.”;
\end{enumerate}
\end{quotation}

($b$) after paragraph (7), add—
\begin{quotation}
“(8) In relation to the assessed income period, the only change of circumstance relevant for the purposes of paragraph (2)($a$)  is that the assessed income period ends in accordance with section 9(4) of the State Pension Credit Act or the regulations made under section 9(5) of that Act.”.
\end{quotation}
\end{enumerate}

\subsection[18. Amendment of regulation 7]{Amendment of regulation 7}

18.  In regulation 7 (date from which a decision superseded under section 10 takes effect)—
\begin{enumerate}\item[]
($a$) in paragraph (1), for sub-paragraph ($a$)\footnote{Paragraph (1)($a$) was amended by S.I.\ 1999/3178 and 2000/1596.}, substitute—
\begin{quotation}
“($a$) is, except for paragraphs (2)($b$)  and (29), subject to Schedules 3A and 3B; and”;
\end{quotation}

($b$) in paragraph (2)($b$), in heads (i)  and (ii), for the words “or jobseeker’s allowance”, substitute “, jobseeker’s allowance or state pension credit”;

($c$) in paragraph (3) at the end, add—
\begin{quotation}
    “or regulation 1(2) of the State Pension Credit Regulations.”; 
\end{quotation}

($d$) in paragraph (13) at the end of sub-paragraph ($a$)  for the word “and”, substitute—
\begin{quotation}
    “or

    (iii) 
    paragraph 11 or 12 of Schedule II to the State Pension Credit Regulations; and”; 
\end{quotation}

($e$) after paragraph (17), insert—
\begin{quotation}
“(17A) For the purposes of state pension credit—
\begin{enumerate}\item[]
($a$) paragraph (14) shall apply as if the reference to—
\begin{enumerate}\item[]
(i) “income support and his applicable amount” was a reference to “state pension credit and his appropriate minimum guarantee”;

(ii) “Schedule 3 to the Income Support Regulations” was a reference to “Schedule II to the State Pension Credit Regulations”; and

(iii) “paragraph 15 or 16” was a reference to “paragraph 11 or 12”;
\end{enumerate}

($b$) paragraphs (15) to (17) shall not apply.”;
\end{enumerate}
\end{quotation}

($f$) after paragraph (28)\footnote{Paragraph (28) was inserted by S.I. 2002/490.}, insert—
\begin{quotation}
“(29) A decision to which regulation 6(2)($l$) (state pension credit) refers shall take effect from the day following the day on which the assessed income period ends if that day is the first day of the claimant’s benefit week, but if it is not, from the next following such day.”.
\end{quotation}
\end{enumerate}

\subsection[19. Amendment of regulation 13]{Amendment of regulation 13}

19.  In regulation 13 (income support and social fund determinations on incomplete evidence)—
\begin{enumerate}\item[]
($a$) in paragraph (1), for sub-paragraph ($a$), substitute—
\begin{quotation}
“($a$) a determination falls to be made by the Secretary of State as to what housing costs are to be included in—
\begin{enumerate}\item[]
(i) a claimant’s applicable amount by virtue of regulation 17(1)($e$)  or 18(1)($f$)  of, and Schedule 3 to, the Income Support Regulations; or

(ii) a claimant’s appropriate minimum guarantee by virtue of regulation 6(6)($c$)  and Schedule II to the State Pension Credit Regulations; and”
\end{enumerate}
\end{quotation}
    and for the words “applicable amount are those”, substitute “applicable amount or, as the case may be, appropriate minimum guarantee are those”; and 

($b$) after paragraph (2), add—
\begin{quotation}
“(3) Where, for the purposes of a decision under section 8 or 10—
\begin{enumerate}\item[]
($a$) a determination falls to be made by the Secretary of State as to whether a claimant’s appropriate minimum guarantee includes an additional amount in accordance with regulation 6(4) of, and paragraph 1 of Schedule I to, the State Pension Credit Regulations; and

($b$) it appears to the Secretary of State that he is not in possession of all the evidence or information which is relevant for the purpose of such a determination,
\end{enumerate}
he shall make the determination on the assumption that the relevant evidence or information which is not in his possession is adverse to the claimant.”.
\end{quotation}
\end{enumerate}

\subsection[20. Amendment of regulation 14]{Amendment of regulation 14}

20.  In regulation 14 (effect of alteration of component rates of state pension credit), at the end, add—
\begin{quotation}
“(5) Section 159B of the Administration Act\footnote{Section 159B was inserted by the State Pension Credit Act 2002 (c.\ 16), section 14 and Schedule 2, paragraph 17.} (effect of alterations affecting state pension credit) shall not apply to any award of state pension credit in favour of a person where in relation to that person the appropriate minimum guarantee includes an amount determined under paragraph 6 of Part III of Schedule I to the State Pension Credit Regulations.

(6) Where section 159B of the Administration Act does not apply to an award of state pension credit by virtue of paragraph (5), a decision under section 10 may be made in respect of that award for the sole purpose of giving effect to any change made to an award under section 150 of the Administration Act.”.
\end{quotation}

\subsection[21. Amendment of Schedule 2]{Amendment of Schedule 2}

21.  In Schedule 2 (decision against which no appeal lies), after paragraph 13 (income support), insert—
\begin{quotation}
\subsection*{\itshape “State pension credit}

13A.  A decision of the Secretary of State made in accordance with paragraph (1) or (3) of regulation 13 in relation to state pension credit (determination on incomplete evidence).”.
\end{quotation}

\subsection[22. Change of circumstances]{Change of circumstances}

22.  After Schedule 3A\footnote{Inserted by S.I.\ 2000/1596.}, insert—
\begin{quotation}\noindent\part*{“Schedule 3B\\*Date on which change of circumstances takes effect where claimant entitled to state pension credit}

1.  Where the amount of state pension credit payable under an award is changed by a superseding decision made on the ground that there has been a relevant change of circumstances, that superseding decision shall take effect from the following days—
\begin{enumerate}\item[]
($a$) for the purpose only of determining the day on which an assessed income period begins under section 9 of the State Pension Credit Act, from the day following the day on which the last previous assessed income period ended; and

($b$) except as provided in the following paragraphs, from the day that change occurs or is expected to occur if either of those days is the first day of a benefit week but if it is not from the next following such day.
\end{enumerate}

\medskip

2.  Subject to paragraph 3, where the relevant change is that the claimant’s income (other than deemed income from capital) has changed, the superseding decision shall take effect on the first day of the benefit week in which that change occurs or if that is not practicable in the circumstances of the case, on the first day of the next following benefit week.

\medskip

3.  Paragraph 2 shall not apply where the only relevant change is that working tax credit under the Tax Credits Act 2002\footnote{2002 c.\ 21.} becomes payable or becomes payable at a higher rate.

\medskip

4.  A superseding decision shall take effect from the day the change of circumstances occurs or is expected to occur if—
\begin{enumerate}\item[]
($a$) the person ceases to be or becomes a prisoner, and for this purpose “prisoner” has the same meaning as in regulation 1(2) of the State Pension Credit Regulations; or

($b$) whilst entitled to state pension credit a claimant is awarded another social security benefit and in consequence of that award his benefit week changes or is expected to change.
\end{enumerate}

\medskip

5.  In a case where the relevant change of circumstances is that the claimant ceased for one or more days to be a patient, the superseding decision shall take effect from the first day of the benefit week in which the change occurred.

\medskip

6.  In paragraph 5, “patient” means a person (other than a prisoner) who is regarded as receiving free in-patient treatment within the meaning of the Social Security (Hospital In-Patients) Regulations 1975\footnote{S.I.\ 1975/555.}.”
\end{quotation}

\section[Part IV --- Amendments to the State Pension Credit Regulations]{Part IV\\*Amendments to the State Pension Credit Regulations}

\renewcommand\parthead{--- Part IV}

\subsection[23. Amendment of the State Pension Credit Regulations]{Amendment of the State Pension Credit Regulations}

23.  In the State Pension Credit Regulations—
\begin{enumerate}\item[]
($a$) in regulation 1(2) (citation), after the definition of “close relative”, insert—
\begin{quotation}
“(i) “the Computation of Earnings Regulations” means the Social Security Benefit (Computation of Earnings) Regulations 1996\footnote{S.I.\ 1996/2745; the relevant amending Instruments are S.I.\ 1999/1958, 2422, 2739, 2860 and 3178.};”;

“(ii) “dwelling occupied as the home” means the dwelling together with any garage, garden and outbuildings, normally occupied by the claimant as his home including any premises not so occupied which it is impracticable or unreasonable to sell separately, in particular, in Scotland, any croft land on which the dwelling is situated;”;
\end{quotation}

($b$) regulation 4 (persons receiving treatment outside Great Britain) shall be renumbered paragraph (1) of regulation 4 and after the renumbered paragraph (1) there shall be added—
\begin{quotation}
“(2) Paragraph (1) applies only where—
\begin{enumerate}\item[]
($a$) the “person” is the claimant or his partner; and

($b$) the claimant satisfied the conditions for entitlement to state pension credit immediately before he or, as the case may be, his partner, left Great Britain.”;
\end{enumerate}
\end{quotation}

($c$) in regulation 5 (persons treated as being or not being members of the same household), in paragraph (1), after sub-paragraph ($d$)  add—
\begin{quotation}
“($e$) either he or the claimant is not in Great Britain and is not treated as being in Great Britain in accordance with regulation 4;

($f$) he is not in Great Britain and none of the circumstances specified in either paragraph (2) or (3) of regulation 3 would, had he been the claimant, apply in his case;”;
\end{quotation}

($d$) in regulation 7 (savings credit), after paragraph (3), add—
\begin{quotation}
“(4) If a calculation made for the purposes of paragraph (1)($b$)  or ($c$)  results in a fraction of a penny, that fraction shall, if it would be to the claimant’s advantage, be treated as a penny; otherwise it shall be disregarded.”;
\end{quotation}

($e$) in regulation 10 (assessed income period), in paragraph (7), in sub-paragraph ($a$), for the word “following” substitute “on or after”;

($f$) after regulation 13 (small amounts of state pension credit), insert—
\begin{quotation}
\subsection*{“Part-weeks}

13A.---(1)  The guarantee credit shall be payable for a period of less than a week (“a part-week”) at the rate specified in paragraph (3) if—
\begin{enumerate}\item[]
($a$) the claimant was entitled to income support or an income-based jobseeker’s allowance immediately before the first day on which the conditions for entitlement to the credit are satisfied; and

($b$) the claimant’s entitlement to the credit is likely to continue throughout the first full benefit week which follows the part-week.
\end{enumerate}

(2) For the purpose of determining the amount of the guarantee credit payable in respect of the part-week, no regard shall be had to any income of the claimant and his partner.

(3) The amount of the guarantee credit payable in respect of the part-week shall be determined—
\begin{enumerate}\item[]
($a$) by dividing by 7 the weekly amount of the guarantee credit which, taking into account the requirements of paragraph (2), would be payable in respect of a full week; and then

($b$) multiplying the resulting figure by the number of days in the part-week,
\end{enumerate}
any fraction of a penny being rounded up to the nearest penny.

\subsection*{Date on which benefits are treated as paid}

13B.---(1)  The following benefits shall be treated as paid on the day of the week in respect of which the benefit is payable—
\begin{enumerate}\item[]
($a$) severe disablement allowance;

($b$) short-term and long-term incapacity benefit;

($c$) maternity allowance;

($d$) contribution-based jobseeker’s allowance within the meaning of section 1(4) of the Jobseekers Act 1995\footnote{1995 c.\ 18.}.
\end{enumerate}

(2) All benefits except those mentioned in paragraph (1) shall be treated as paid on the first day of the benefit week in respect of which the benefit is payable.”;
\end{quotation}

($g$) in regulation 15 (income for the purposes of the Act), in paragraph (1) for sub-paragraph ($l$), substitute—
\begin{quotation}
“($l$) housing benefit;

($m$) council tax benefit;

($n$) bereavement payment\footnote{Bereavement payment was introduced by section 54(1) of the Welfare Reform and Pensions Act 1999 (c.\ 30).};

($o$) statutory sick pay;

($p$) statutory maternity pay;

($q$) statutory paternity pay payable under Part XIIZA of the 1992 Act\footnote{Part XIIZA was inserted by section 2 of the Employment Act 2002 (c.\ 22).};

($r$) statutory adoption pay payable under Part XIIZB of the 1992 Act\footnote{Part XIIZB was inserted by section 4 of the Employment Act 2002.};

($s$) any benefit similar to those mentioned in the preceding provisions of this paragraph payable under legislation having effect in Northern Ireland.”;
\end{quotation}

($h$) in regulation 17 (calculation of weekly income)—
\begin{enumerate}\item[]
(i) in paragraph (5) omit the word “and” at the end of sub-paragraph ($a$)  and at the end of sub-paragraph ($b$), add—
\begin{quotation}
    “and

    ($c$) 
    any payment which is made on an occasional basis.”; 
\end{quotation}

(ii) for paragraph (9), substitute—
\begin{quotation}
“(9) The sums specified in Schedule VI shall be disregarded in calculating—
\begin{enumerate}\item[]
($a$) the claimant’s earnings; and

($b$) any amount to which paragraph (5) applies if the claimant or his partner is the first owner of the copyright, patent or trade mark or the author of the book registered under the Public Lending Right Scheme 1982\footnote{The Scheme is set out in the Appendix to S.I.\ 1982/719.}.
\end{enumerate}

(9A) For the purposes of paragraph (9)($b$), and for that purpose only, the amounts specified in paragraph (5) shall be treated as though they were earnings.”;
\end{quotation}

(iii) in paragraph (10), at the beginning, insert “Subject to regulation 17C (deduction of tax and contributions for self-employed earners),”;

(iv) after paragraph (10), add—
\begin{quotation}
“(11) In the case of the earnings of self-employed earners, the amounts specified in paragraph (10) shall be taken into account in accordance with paragraph (4) or, as the case may be, paragraph (10) of regulation 13 of the Computation of Earnings Regulations, as having effect in the case of state pension credit.”;
\end{quotation}
\end{enumerate}

($i$) after regulation 17, insert—
\begin{quotation}
\subsection*{“Earnings of an employed earner}

17A.---(1)  For the purposes of state pension credit, the provisions of this regulation which relate to the earnings of employed earners, shall have effect in place of those prescribed for such earners in the Computation of Earnings Regulations.

(2) Subject to paragraphs (3) and (4), “earnings” in the case of employment as an employed earner, means any remuneration or profit derived from that employment and includes—
\begin{enumerate}\item[]
($a$) any bonus or commission;

($b$) any payment in lieu of remuneration except any periodic sum paid to a claimant on account of the termination of his employment by reason of redundancy;

($c$) any payment in lieu of notice;

($d$) any holiday pay;

($e$) any payment by way of a retainer;

($f$) any payment made by the claimant’s employer in respect of expenses not wholly, exclusively and necessarily incurred in the performance of the duties of the employment, including any payment made by the claimant’s employer in respect of—
\begin{enumerate}\item[]
(i) travelling expenses incurred by the claimant between his home and place of employment;

(ii) expenses incurred by the claimant under arrangements made for the care of a member of his family owing to the claimant’s absence from home;
\end{enumerate}

($g$) the amount of any payment by way of a non-cash voucher which has been taken into account in the computation of a person’s earnings in accordance with Part V of Schedule 3 to the Social Security (Contributions) Regulations 2001\footnote{S.I. 2001/1004.};

($h$) statutory sick pay and statutory maternity pay payable by the employer under the 1992 Act;

($i$) statutory paternity pay payable under Part XIIZA of the 1992 Act;

($j$) statutory adoption pay payable under Part XIIZB of the 1992 Act;

($k$) any sums payable under a contract of service—
\begin{enumerate}\item[]
(i) for incapacity for work due to sickness or injury; or

(ii) by reason of pregnancy or confinement.
\end{enumerate}
\end{enumerate}

(3) “Earnings” shall not include—
\begin{enumerate}\item[]
($a$) subject to paragraph (4), any payment in kind;

($b$) any payment in respect of expenses wholly, exclusively and necessarily incurred in the performance of the duties of the employment;

($c$) any occupational pension;

($d$) any lump sum payment made under the Iron and Steel Re-adaptation Benefits Scheme\footnote{The Scheme is set out in regulation 4 of, and the Schedule to, the European Communities (Iron and Steel Employees Re-adaptation Benefits Scheme) (No.\ 2) (Amendment) Regulations 1996 (S.I.\ 1996/3182).}.
\end{enumerate}

(4) Paragraph (3)($a$)  shall not apply in respect of any non-cash voucher referred to in paragraph (2)($g$) .

(5) In this regulation “employed earner” means a person who is gainfully employed in Great Britain either under a contract of service, or in an office (including elective office) with emoluments chargeable to income tax under Schedule E.

\subsection*{Earnings of self-employed earners}

17B.---(1)  For the purposes of state pension credit, the provisions of the Computation of Earnings Regulations in their application to the earnings of self-employed earners, shall have effect in so far as provided by this regulation.

(2) In their application to state pension credit, regulations 11 to 14 of the Computation of Earnings Regulations shall have effect as if—
\begin{enumerate}\item[]
($a$) “claimant” referred to a person claiming state pension credit and any partner of the claimant;

($b$) “personal pension scheme” referred to a personal pension scheme—
\begin{enumerate}\item[]
(i) as defined in section 1 of the Pension Schemes Act 1993\footnote{1993 c.\ 48.}; or

(ii) as defined in section 1 of the Pension Schemes (Northern Ireland) Act 1993\footnote{1993 c.\ 49.}.
\end{enumerate}
\end{enumerate}

(3) In regulation 11 (calculation of earnings of self-employed earners), paragraph (1) shall have effect, but as if the words “Except where paragraph (2) applies” were omitted.

(4) In regulation 12 (earnings of self-employed earners)—
\begin{enumerate}\item[]
($a$) paragraph (1) shall have effect;

($b$) for paragraph (2), the following provision shall have effect—
\begin{quotation}
“(2) Earnings does not include—
\begin{enumerate}\item[]
($a$) where a claimant occupies a dwelling as his home and he provides in that dwelling board and lodging accommodation for which payment is made, those payments;

($b$) any payment made by a local authority to a claimant—
\begin{enumerate}\item[]
(i) with whom a person is accommodated by virtue of arrangements made under section 23(2)($a$)  of the Children Act 1989\footnote{1989 c.\ 41.} (provision of accommodation and maintenance for a child whom they are looking after) or, as the case may be, section 26(1) of the Children (Scotland) Act 1995\footnote{1995 c.\ 36.}; or

(ii) with whom a local authority foster a child under the Fostering of Children (Scotland) Regulations 1996\footnote{S.I.\ 1996/3263.};
\end{enumerate}

($c$) any payment made by a voluntary organisation in accordance with section 59(1)($a$)  of the Children Act 1989 (provision of accommodation by voluntary organisations);

($d$) any payment made to the claimant or his partner for a person (“the person concerned”) who is not normally a member of the claimant’s household but is temporarily in his care, by—
\begin{enumerate}\item[]
(i) a health authority;

(ii) a local authority;

(iii) a voluntary organisation;

(iv) the person concerned pursuant to section 26(3A) of the National Assistance Act 1948\footnote{11 \& 12 Geo.\ 6 c.\ 29; section 26(3A) was inserted by the National Health Service and Community Care Act 1990 (c.\ 19).}; or

(v) a primary care trust established under section 16A of the National Health Service Act 1977\footnote{1977 c.\ 49; section 16A was inserted by section 2 of the Health Act 1999 (c.\ 8).};
\end{enumerate}

($e$) any sports award.”.
\end{enumerate}
\end{quotation}
\end{enumerate}

(5) In regulation 13 (calculation of net profit of self-employed earners)—
\begin{enumerate}\item[]
($a$) for paragraphs (1) to (3), the following provision shall have effect—
\begin{quotation}
“(1) For the purposes of regulation 11 (calculation of earnings of self-employed earners), the earnings of a claimant to be taken into account shall be—
\begin{enumerate}\item[]
($a$) in the case of a self-employed earner who is engaged in employment on his own account, the net profit derived from that employment;

($b$) in the case of a self-employed earner whose employment is carried on in partnership, his share of the net profit derived from that employment less—
\begin{enumerate}\item[]
(i) an amount in respect of income tax and of social security contributions payable under the Contributions and Benefits Act calculated in accordance with regulation 14 (deduction of tax and contributions for self-employed earners); and

(ii) one half of any premium paid in the period that is relevant under regulation 11 in respect of a retirement annuity contract or a personal pension scheme.”;
\end{enumerate}
\end{enumerate}
\end{quotation}

($b$) paragraphs (4) to (12) shall have effect.
\end{enumerate}

(6) Regulation 14 (deduction of tax and contributions for self-employed earners) shall have effect.”;
\end{quotation}

($j$) in regulation 21 (notional capital), after paragraph (2), add—
\begin{quotation}
“(3) Where a claimant stands in relation to a company in a position analogous to that of a sole owner or partner in the business of that company, he shall be treated as if he were such sole owner or partner and in such a case—
\begin{enumerate}\item[]
($a$) the value of his holding in that company shall, notwithstanding regulation 19 (calculation of capital), be disregarded; and

($b$) he shall, subject to paragraph (4), be treated as possessing an amount of capital equal to the value or, as the case may be, his share of the value of the capital of that company and the foregoing provisions of this Chapter shall apply for the purposes of calculating that amount as if it were actual capital which he does possess.
\end{enumerate}

(4) For so long as a claimant undertakes activities in the course of the business of the company, the amount which he is treated as possessing under paragraph (3) shall be disregarded.

(5) Where under this regulation a person is treated as possessing capital, the amount of that capital shall be calculated in accordance with the provisions of this Part as if it were actual capital which he does possess.”;
\end{quotation}

($k$) after regulation 24 (income paid to third parties), insert—
\begin{quotation}
\subsection*{“Rounding of fractions}

24A.  Where any calculation under this Part results in a fraction of a penny that fraction shall, if it would be to the claimant’s advantage, be treated as a penny; otherwise it shall be disregarded.”;
\end{quotation}

($l$) in Schedule II (housing costs)—
\begin{enumerate}\item[]
(i) in paragraph 5, after sub-paragraph (1), insert—
\begin{quotation}
“(1A) In paragraph (1), “housing benefit expenditure” means expenditure in respect of which housing benefit is payable as specified in regulation 10(1) of the Housing Benefit (General) Regulations 1987\footnote{S.I.\ 1987/1971; the relevant amending Instrument is S.I.\ 1988/1971.} but does not include any such expenditure in respect of which an additional amount is applicable under regulation 6(6)($c$)  (housing costs).”;
\end{quotation}

(ii) in paragraph 7, for sub-paragraph (5), substitute—

\begin{quotation}
“(5) Where in the period since the amount applicable under this Schedule was last determined, there has been a change of circumstances, other than a reduction in the amount of the outstanding loan, which increases or reduces the amount applicable, it shall be recalculated so as to take account of that change.”;
\end{quotation}

(iii) in paragraph 13, at the end, add—
\begin{quotation}
“(6) In this paragraph—
\begin{enumerate}\item[]
($a$) “co-ownership scheme” means a scheme under which a dwelling is let by a housing association and the tenant, or his personal representative, will, under the terms of the tenancy agreement or of the agreement under which he became a member of the association, be entitled, on his ceasing to be a member and subject to any condition stated in either agreement, to a sum calculated by reference directly or indirectly to the value of the dwelling;

($b$) “Crown tenant” means a person who occupies a dwelling under a tenancy or licence where the interest of the landlord belongs to Her Majesty in right of the Crown or to a government department or is held in trust for Her Majesty for the purposes of a government department except (in the case of an interest belonging to Her Majesty in right of the Crown) where the interest is under the management of the Crown Estate Commissioners;

($c$) “housing association” has the meaning assigned to it by section 1(1) of the Housing Associations Act 1985\footnote{1985 c.\ 69.};

($d$) “long tenancy” means a tenancy granted for a term of years certain exceeding twenty one years, whether or not the tenancy is, or may become, terminable before the end of that term by notice given by or to the tenant or by re-entry, forfeiture (or, in Scotland, irritancy) or otherwise and includes a lease for a term fixed by law under a grant with a covenant or obligation for perpetual renewal unless it is a lease by sub-demise from one which is not a long tenancy.”;
\end{enumerate}
\end{quotation}
\end{enumerate}

($m$) in Schedule III—
\begin{enumerate}\item[]
(i) in paragraph 1—
\begin{enumerate}\item[]
($aa$) in sub-paragraph (2), in the words which apply instead of section 3(1), the words “the claimant” on the second occasion on which they occur shall be omitted;

\pagebreak[3]

($bb$) in sub-paragraph (8), after “5,” insert “6(8),”;
\end{enumerate}

(ii) in paragraph 2(5), for the words “shall be determined as if the reductions specified in sub-paragraph (2) do not apply in his case”, substitute
    “shall be determined without regard to any adjustments which fall to be made in accordance with the Social Security (Hospital In-Patients) Regulations 1975\footnote{S.I.\ 1975/555.}.”; 
\end{enumerate}

($n$) in Schedule IV, in paragraph 11—
\begin{enumerate}\item[]
(i) in sub-paragraph (2), omit head ($d$);

(ii) in sub-paragraph (3)($b$), after the words “paragraph 1($a$)  to ($f$)”, insert “or paragraph 7” ;
\end{enumerate}

($o$) in Schedule V, in Part I—
\begin{enumerate}\item[]
(i) after paragraph 9, insert—
\begin{quotation}
“9A.  The assets of any business owned in whole or in part by the claimant if—
\begin{enumerate}\item[]
($a$) he is not engaged as a self-employed earner in that business by reason of some disease or bodily or mental disablement; but

($b$) he intends to become engaged (or, as the case may be, re-engaged) as a self-employed earner in that business as soon as he recovers or is able to become engaged, or re-engaged, in that business,
\end{enumerate}
for a period of 26 weeks from the date on which the claim for state pension credit is made or, if it is unreasonable to expect him to become engaged or re-engaged in that business within that period, for such longer period as is reasonable in the circumstances to enable him to become so engaged or re-engaged.”;
\end{quotation}

(ii) paragraph 16 shall be renumbered sub-paragraph (1) of paragraph 16 and after the renumbered sub-paragraph (1) insert—
\begin{quotation}
“(2) Where the whole or part of the payment is administered—
\begin{enumerate}\item[]
($a$) by the High Court under the provisions of Order 80 of the Rules of the Supreme Court 1965\footnote{S.I.\ 1965/1776.}, the county court under Order 10 of the County Court Rules 1981\footnote{S.I.\ 1981/1687.}, or the Court of Protection;

($b$) in accordance with an order made under Rule 131 of the Act of Sederunt (Rules of the Court, consolidation and amendment) 1965\footnote{S.I.\ 1965/321.}, or under Rule 36.14 of the Ordinary Cause Rules 1993\footnote{First Schedule to the Sheriff Courts (Scotland) Act 1907 (7 Edw.\ 7 c.\ 51) as substituted in respect of causes commenced on or after 1 January 1994 by S.I.\ 1993/1956; the relevant amending Instrument is S.I.\ 1996/2167.} or under Rule 128 of those Rules; or

($c$) in accordance with the terms of a trust established for the benefit of the claimant or his partner,
\end{enumerate}
the whole of the amount so administered.”;
\end{quotation}

(iii) in paragraph 20—
\begin{enumerate}\item[]
($aa$) in sub-paragraph (1), at the end add—
\begin{quotation}
“($d$) by a local authority out of funds provided under either section 93 of the Local Government Act 2000\footnote{2000 c.\ 22.} under a scheme known as “Supporting People” or section 91 of the Housing (Scotland) Act 2001\footnote{2001 asp 10.}”;
\end{quotation}

($bb$) in sub-paragraph (2) at the end, add—
\begin{quotation}
“($j$) council tax benefit;

($k$) social fund payments;

($l$) child benefit;

($m$) working tax credit under the Tax Credits Act 2002\footnote{2002 c.\ 21.};

($n$) child tax credit under the Tax Credits Act 2002.”;
\end{quotation}
\end{enumerate}

(iv) in Part II of Schedule V, for the heading substitute—
\begin{quotation}
\section*{“Capital disregarded only for the purposes of determining deemed income”;}
\end{quotation}
\end{enumerate}
\end{enumerate}

\section[Part V --- Miscellaneous amendments]{Part V\\*Miscellaneous amendments}

\subsection[24. Amendment of the Social Security (Payments on account, Overpayments and Recovery) Regulations 1988]{\sloppy Amendment of the Social Security (Payments on account, Overpayments and Recovery) Regulations 1988}

\renewcommand\parthead{--- Part V}

24.---(1)  The Social Security (Payments on account, Overpayments and Recovery) Regulations 1988\footnote{S.I.\ 1988/664; the relevant amending Instruments are S.I.\ 1993/650 and 846 and 1996/1345.} shall be amended in accordance with the following provisions of this regulation.

(2) In regulation 1 (citation, commencement and interpretation), in paragraph (2)—
\begin{enumerate}\item[]
($a$) in the entry relating to “benefit” for the words “jobseeker’s allowance and”, substitute “jobseeker’s allowance, state pension credit and”; and

($b$) after the entry relating to “severe disablement allowance” insert the following entries—
\begin{quotation}
“state pension credit” means the benefit payable under the State Pension Credit Act 2002;

“the State Pension Credit Regulations” means the State Pension Credit Regulations 2002\footnote{S.I.\ 2002/1792.}”.
\end{quotation}
\end{enumerate}

(3) In regulation 7 (duplication and prescribed income), in paragraph (1)—
\begin{enumerate}\item[]
($a$) for the words “income support and”, substitute “income support, state pension credit and”;

($b$) in sub-paragraph ($a$), after the words “Allowance Regulations” insert “or Part III of the State Pension Credit Regulations”.
\end{enumerate}

(4) In regulation 16 (limitation on deductions from prescribed benefits)—
\begin{enumerate}\item[]
($a$) in paragraph (4A)\footnote{Paragraph (4A) was inserted by S.I.\ 1996/2519.}, at the end insert—
\begin{quotation}
“($d$) state pension credit.”;
\end{quotation}

($b$) in paragraph (6), at the end of sub-paragraph ($b$), insert—
\begin{quotation}
    “or

    ($c$) 
    in the calculation of the income of a person to whom state pension credit is payable, the amount of earnings or other income falling to be taken into account is reduced in accordance with paragraph 1 of Schedule 4 (sums to be disregarded in the calculation of income other than capital), or Schedule 6 (sums disregarded from claimant’s earnings) to the State Pension Credit Regulations,”; 
\end{quotation}

($c$) in paragraph (8)—
\begin{enumerate}\item[]
(i) for the definition of “personal allowance for a single claimant aged not less than 25” substitute—
\begin{quotation}
    ““personal allowance for a single claimant aged not less than 25” means—
\begin{enumerate}\item[]
    ($a$) 
    in the case of a person who is entitled to either income support or state pension credit, the amount for the time being specified in paragraph 1(1)($e$)  of column (2) of Schedule 2 to the Income Support Regulations; or

    ($b$) 
    in the case of a person who is entitled to income-based jobseeker’s allowance, the amount for the time being specified in paragraph 1(1)($e$)  of column (2) of Schedule 1 to the Jobseeker’s Allowance Regulations;”; 
\end{enumerate}
\end{quotation}

(ii) for the definition of “specified benefit”, substitute—
\begin{quotation}
    ““specified benefit” means—
\begin{enumerate}\item[]
    ($a$) 
    a jobseeker’s allowance;

    ($b$) 
    income support when paid alone or together with any incapacity benefit, retirement pension or severe disablement allowance in a combined payment in respect of any period;

    ($c$) 
    if incapacity benefit, retirement pension or severe disablement allowance is paid concurrently with income support in respect of any period but not in a combined payment, income support and such of those benefits as are paid concurrently;

    ($d$) 
    state pension credit when paid alone or together with any retirement pension, incapacity benefit or severe disablement allowance in a combined payment in respect of any period; and

    ($e$) 
    if retirement pension, incapacity benefit or severe disablement allowance is paid concurrently with state pension credit in respect of any period but not in a combined payment, state pension credit and such of those benefits as are paid concurrently,
\end{enumerate}
    but does not include any sum payable by way of child maintenance bonus in accordance with section 10 of the Child Support Act 1995\footnote{1995 c.\ 34.} and the Social Security (Child Maintenance Bonus) Regulations 1996\footnote{S.I.\ 1996/3195.}.”. 
\end{quotation}
\end{enumerate}
\end{enumerate}

(5) In the provisions listed in paragraph (6), after the words “income support” in each place where they occur there shall be inserted the words “, or state pension credit”.

(6) The provisions referred to in paragraph (5) are—
\begin{enumerate}\item[]
($a$) regulation 5(3) (offsetting prior payment against subsequent award);

($b$) regulation 8(2) (duplication and prescribed payments);

($c$) regulation 13(1)($b$)  (sums to be deducted in calculating recoverable amounts);

($d$) regulation 14(1) (quarterly diminution of capital);

($e$) regulation 15(2)($d$)  (recovery by deduction from prescribed benefits); and

($f$) regulation 17 (recovery from couples).
\end{enumerate}

\subsection[25. Amendment of the Social Security (Attendance Allowance) Regulations 1991]{Amendment of the Social Security (Attendance Allowance) Regulations 1991}

25.  In regulation 8(6)($a$)  of the Social Security (Attendance Allowance) Regulations 1991\footnote{S.I.\ 1991/2740; the relevant amending Instrument is S.I.\ 1996/1345.} (exemption from regulations 6 and 7) after the words “income support”, in both places where they occur, insert “, state pension credit”.

\subsection[26. Amendment of the Child Support (Arrears, Interest and Adjustment of Maintenance Assessments) Regulations 1992]{Amendment of the Child Support (Arrears, Interest and Adjustment of Maintenance Assessments) Regulations 1992}

26.---(1)  The Child Support (Arrears, Interest and Adjustment of Maintenance Assessments) Regulations 1992\footnote{S.I.\ 1992/1816; the relevant amending Instruments are S.I.\ 1995/3261, 1996/1345 and 2001/162.} shall be further amended in accordance with the following provisions of this regulation.

(2) In regulation 1 (citation, commencement and interpretation), in paragraph (2), at the end, add—
\begin{quotation}
““state pension credit” means the social security benefit of that name payable under the State Pension Credit Act 2002\footnote{2002 c.\ 16.}.”.
\end{quotation}

(3) In both regulation 10A (reimbursement of a repayment of overpaid child maintenance) and regulation 10B (repayment of a reimbursement of a voluntary payment), after the words “income support” insert “, state pension credit”.

\subsection[27. Amendment of the Child Support (Maintenance Calculations and Special Cases) Regulations 2000]{Amendment of the Child Support (Maintenance Calculations and Special Cases) Regulations 2000}

27.---(1)  The Child Support (Maintenance Calculations and Special Cases) Regulations 2000\footnote{S.I.\ 2001/155; the relevant amending Instrument is S.I.\ 2002/1204.} shall be amended in accordance with the following provisions of this regulation.

(2) In regulation 1 (citation, commencement and interpretation), in paragraph (2), after the definition of “self-employed earner” insert the following definition—
\begin{quotation}
““state pension credit” means the social security benefit of that name payable under the State Pension Credit Act 2002;”.
\end{quotation}

(3) In regulation 4 (flat rate), in paragraph (2), at the end, add—
\begin{quotation}
    “and

    ($c$) 
    state pension credit.”. 
\end{quotation}

(4) In regulation 5 (nil rate), after paragraph ($g$), insert—
\begin{quotation}
“($gg$) a patient in hospital who is in receipt of state pension credit and in respect of whom paragraph 2(1) of Schedule III to the State Pension Credit Regulations\footnote{S.I.\ 2002/1792.} (patient for at least 13 but not exceeding 52 weeks) applies;”.
\end{quotation}

\subsection[28. Amendment of the Social Security (Disability Living Allowance) Regulations 1991]{\sloppy Amendment of the Social Security (Disability Living Allowance) Regulations 1991}

28.  In regulation 10(8)($a$)  of the Social Security (Disability Living Allowance) Regulations 1991\footnote{S.I.\ 1991/2890; the relevant amending Instrument is S.I.\ 1996/1345.} (exemption from regulations 8 and 9), after the words “income support” in each place where they occur insert “, state pension credit”.

\subsection[29. Amendment of the Income Support (General) Regulations 1987]{\sloppy Amendment of the Income Support (General) Regulations 1987}

29.---(1)  The Income Support (General) Regulations 1987\footnote{S.I.\ 1987/1967.} shall be amended in accordance with the following provisions of this regulation.

(2) In regulation 42 (notional income), in paragraph (2C)\footnote{Paragraph (2C) was inserted by S.I.\ 1995/2303.}, for the word “claimant” substitute “person”.

(3) In regulation 53 (calculation of tariff income from capital), in paragraph (1ZA)\footnote{Paragraph (1ZA) was inserted by S.I.\ 2000/2545.}, in sub-paragraph ($a$), omit the words “is aged 60 or over or”.

(4) In Schedule 1B\footnote{Schedule 1B was inserted by S.I.\ 1996/206.} (prescribed categories of persons), omit paragraph 17 (persons aged 60 or over).

(5) In Schedule 2 (applicable amounts)—
\begin{enumerate}\item[]
($a$) for paragraph 9\footnote{Paragraph 9 was substituted by S.I.\ 1989/534.} (pensioner premiums for persons under 75), substitute—
\begin{quotation}
“9.  The condition is that the claimant has a partner aged not less than 60 but less than 75.”;
\end{quotation}

($b$) for paragraph 9A\footnote{Paragraph 9A was inserted by S.I.\ 1989/534.} (pensioner premiums for persons 75 or over), substitute—
\begin{quotation}
“9A.  The condition is that the claimant has a partner aged not less than 75 but less than 80.”;
\end{quotation}

($c$) in paragraph 10\footnote{Paragraph 10 is amended by S.I.\ 1988/663, 1992/468, 1998/2231 and 2000/724.} (higher pensioner premium)—
\begin{enumerate}\item[]
(i) for sub-paragraphs (1) and (2), substitute—
\begin{quotation}
“(1) The condition is that—
\begin{enumerate}\item[]
($a$) the claimant’s partner is aged not less than 80; or

($b$) the claimant’s partner is aged less than 80 but not less than 60 and either—
\begin{enumerate}\item[]
(i) the additional condition specified in paragraph 12(1)($a$)  or ($c$)  is satisfied; or

(ii) the claimant was entitled to, or was treated as being in receipt of, income support and—
\begin{itemize}\item[]
($aa$) the disability premium was or, as the case may be, would have been, applicable to him in respect of a benefit week within eight weeks of his partner’s 60th birthday; and

($bb$) he has, subject to sub-paragraph (3), remained continuously entitled to income support since his partner attained the age of 60.”;
\end{itemize}
\end{enumerate}
\end{enumerate}
\end{quotation}

(ii) in sub-paragraph (3), in head ($b$), for the words “sub-paragraphs (1)($b$)(ii)  and (2)($b$)(ii)  are” substitute “sub-paragraph (1)($b$)(ii)  is” and after the words “includes his” insert “partner's”;
\end{enumerate}

($d$) in paragraph 11 (disability premium)—
\begin{enumerate}\item[]
(i) in sub-paragraph ($a$), omit the words “he is aged less than 60 and”;

(ii) for sub-paragraph ($b$)(i), substitute—

“(i) the claimant satisfies the additional condition specified in paragraph 12(1)($a$), ($b$)  or ($c$); or”;
\end{enumerate}

($e$) in paragraph 12\footnote{The relevant amending Instruments are S.I.\ 1991/2742, 1994/2139 and 1995/482 and 516.} (additional condition for the higher pensioner and disability premium), in sub-paragraph (1) for head ($c$)  substitute—
\begin{quotation}
“($c$) the claimant’s partner was in receipt of long-term incapacity benefit under Part II of the Contributions and Benefits Act when entitlement to that benefit ceased on account of the payment of a retirement pension under that Act and—
\begin{enumerate}\item[]
(i) the claimant has since remained continuously entitled to income support;

(ii) the higher pensioner premium or disability premium has been applicable to the claimant; and

(iii) the partner is still alive;
\end{enumerate}

($d$) except where paragraph (1)($a$), ($b$), ($c$)(ii)  or ($d$)(ii)  of Schedule 7 (patients) applies, the claimant or, as the case may be, his partner was in receipt of attendance allowance or disability living allowance—
\begin{enumerate}\item[]
(i) but payment of that benefit has been suspended under the Social Security (Hospital In-Patients) Regulations 1975\footnote{S.I.\ 1975/555.} or otherwise abated as a consequence of the claimant or his partner becoming a patient within the meaning of regulation 21(3); and

(ii) a higher pensioner premium or disability premium has been applicable to the claimant.”;
\end{enumerate}
\end{quotation}

($f$) in paragraph 13A\footnote{Paragraph 13A was inserted by S.I.\ 2000/2629.} (enhanced disability premium), in sub-paragraph (1) for head ($b$)  and the words which follow that head, substitute—
\begin{quotation}
“($b$) a member of the claimant’s family who is aged less than 60.”;
\end{quotation}

($g$) in paragraph 15\footnote{The relevant amending Instrument is S.I.\ 2002/668.} (premiums)—
\begin{enumerate}\item[]
(i) in column (1)—
\begin{enumerate}\item[]
($aa$) for sub-paragraph (2), substitute—
\begin{quotation}
“(2) Pensioner premium for persons to whom paragraph 9 applies.”;
\end{quotation}

($bb$) for sub-paragraph (2A)\footnote{Sub-paragraph (2A) was inserted by S.I.\ 1989/534.}, substitute—
\begin{quotation}
“(2A) Pensioner premium for persons to whom paragraph 9A applies.”;
\end{quotation}

($cc$) for sub-paragraph (3), substitute—
\begin{quotation}
“(3) Higher pensioner premium for persons to whom paragraph 10 applies.”;
\end{quotation}
\end{enumerate}

(ii) in column (2), in sub-paragraphs (2), (2A) and (3), the entries relating to head ($a$)  shall in each case be omitted.
\end{enumerate}
\end{enumerate}

(6) In Schedule 3 (housing costs)—
\begin{enumerate}\item[]
($a$) in paragraph 6, in sub-paragraph (1B)\footnote{Sub-paragraph (1B) was inserted by S.I.\ 1997/2305.}, after the words “jobseeker’s allowance” insert “or state pension credit”;

($b$) in paragraph 8, in sub-paragraph (1B)\footnote{Sub-paragraph (1B) was inserted by S.I.\ 1997/2305.}, after the words “jobseeker’s allowance” insert “or state pension credit”;

($c$) in paragraph 9, in sub-paragraph (1), for head ($a$), substitute—
\begin{quotation}
“($a$) the claimant’s partner has attained the qualifying age for state pension credit;”;
\end{quotation}

($d$) in paragraph 14, at the end add—
\begin{quotation}
“(14) For the purpose of determining whether the linking rules set out in this paragraph apply in a case where a claimant’s former partner was entitled to state pension credit, any reference to income support in this Schedule shall be taken to include also a reference to state pension credit.”;
\end{quotation}

($e$) in paragraph 18, in sub-paragraph (1), for heads ($a$)  and ($b$)  substitute—
\begin{quotation}
“($a$) in respect of a non-dependant aged 18 or over who is engaged in any remunerative work but is not in receipt of state pension credit, £47$.$75;

($b$) in respect of a non-dependant who is engaged in remunerative work and in receipt of state pension credit, £7$.$40;

($c$) in respect of a non-dependant aged 18 or over to whom neither head ($a$)  nor head ($b$)  applies, £7$.$40.”
\end{quotation}
\end{enumerate}

(7) In Schedule 8 (sums to be disregarded in the calculation of earnings)—
\begin{enumerate}\item[]
($a$) in paragraph 1, in sub-paragraph ($a$), omit head (i);

($b$) after paragraph 1, insert—
\begin{quotation}
“1A.  If the claimant’s partner has been engaged in remunerative work as an employed earner or, had the employment been in Great Britain, would have been so engaged, any earnings paid or due to be paid on termination of that employment by way of retirement but only if the partner has attained the qualifying age for state pension credit on retirement.”;
\end{quotation}

($c$) in paragraph 4—
\begin{enumerate}\item[]
(i) in sub-paragraph (3), head ($b$)  shall be omitted;

(ii) in sub-paragraph (4)—
\begin{enumerate}\item[]
($aa$) for head ($b$), substitute—
\begin{quotation}
“($b$) the claimant’s partner has attained the qualifying age for state pension credit;”;
\end{quotation}

($bb$) in head ($c$)  omit “he or, as the case may be, he or” and “or (3)”;
\end{enumerate}

(iii) in sub-paragraph (7)—
\begin{enumerate}\item[]
($aa$) in head ($a$), for sub-head (i), substitute—
\begin{quotation}
“(i) on or after the date on which the claimant’s partner attained the qualifying age for state pension credit during which the partner was not engaged in part-time employment or the claimant was not entitled to income support; or”;
\end{quotation}

($bb$) in head ($b$), for the words “the claimant or, as the case may be, his partner attained the age of 60”, substitute “the claimant’s partner attains the qualifying age for state pension credit.”;

($cc$) in head ($c$), for the words “the claimant or, if he is a member of a couple, he or his partner attained the age of 60”, substitute “the claimant’s partner, if he is a member of a couple, attained the qualifying age for state pension credit”.
\end{enumerate}
\end{enumerate}
\end{enumerate}

\subsection[30. Amendment of the Jobseeker’s Allowance Regulations 1996]{Amendment of the Jobseeker’s Allowance Regulations 1996}

30.  In Schedule 2 to the Jobseeker’s Allowance Regulations 1996\footnote{S.I.\ 1996/207.} (housing costs)—
\begin{enumerate}\item[]
($a$) in paragraph 6, in sub-paragraph (3), after the words “income support” insert “or state pension credit”;

($b$) in paragraph 13, at the end, add—
\begin{quotation}
“(16) For the purpose of determining whether the linking rules set out in this paragraph apply in a case where a claimant’s former partner was entitled to state pension credit, any reference to income-based jobseeker’s allowance in this Schedule shall be taken to include also a reference to state pension credit.”;
\end{quotation}

($c$) in paragraph 17, in sub-paragraph (1), for heads ($a$)  and ($b$), substitute—
\begin{quotation}
“($a$) in respect of a non-dependant aged 18 or over who is engaged in any remunerative work but is not in receipt of state pension credit, £47$.$75;

($b$) in respect of a non-dependant who is engaged in remunerative work and in receipt of state pension credit, £7$.$40;

($c$) in respect of a non-dependant aged 18 or over to whom neither head ($a$)  nor head ($b$)  applies, £7$.$40.”.
\end{quotation}
\end{enumerate}

\subsection[31. Amendment of regulations relating to the social fund]{Amendment of regulations relating to the social fund}

31.---(1)  The Social Fund Maternity and Funeral Expenses (General) Regulations 1987\footnote{S.I.\ 1987/481.} shall be amended in accordance with paragraph (2).

(2) In both regulation 5(1)($a$)  and regulation 7(1)($a$)(i), after “income support”, insert “state pension credit”.

(3) The Social Fund Cold Weather Payments (General) Regulations 1988\footnote{S.I.\ 1988/1724.} shall be amended in accordance with paragraphs (4) to (6).

(4) Regulation 1A shall be renumbered paragraph (1) of that regulation.

(5) In the renumbered paragraph (1)—
\begin{enumerate}\item[]
($a$) after “income support” insert “state pension credit”;

($b$) after sub-paragraph (ia), insert—
\begin{quotation}
“(ib) the person is entitled to state pension credit and is not resident in a care home;”.
\end{quotation}
\end{enumerate}

(6) After the renumbered paragraph (1), insert—
\begin{quotation}
“(2) In paragraph (1)(ib), the expression “care home” means an establishment which is a care home for the purposes of the Care Standards Act 2000\footnote{2000 c.\ 14; \emph{see} section 3 of that Act.}.”.
\end{quotation}

(7) In the Social Fund (Recovery by Deductions from Benefits) Regulations 1988\footnote{S.I.\ 1988/35.}, in regulation 3, after paragraph ($a$), insert—
\begin{quotation}
“($aa$) state pension credit under the State Pension Credit Act 2002\footnote{2002 c.\ 16.};”.
\end{quotation}

\section[Part VI --- Deductions]{Part VI\\*Deductions}

\subsection[32. Amendment of the Fines (Deductions from Income Support) Regulations 1992]{Amendment of the Fines (Deductions from Income Support) Regulations 1992}

\renewcommand\parthead{--- Part VI}

32.---(1)  The Fines (Deductions from Income Support) Regulations 1992\footnote{S.I.\ 1992/2182.} shall be amended in accordance with the following provisions of this regulation.

(2) In regulation 1, in paragraph (2)—
\begin{enumerate}\item[]
($a$) in the definition of “benefit week”, after the words “as the case may be” insert “regulation 1(2) of the State Pension Credit Regulations 2002\footnote{S.I.\ 2002/1792; the relevant amending instrument is S.I.\ 1996/2344.} or”;

($b$) for the definition of “personal allowance for a single claimant aged not less than 25” substitute—
\begin{quotation}
    ““personal allowance for a single claimant aged not less than 25” means—
\begin{enumerate}\item[]
    ($a$) 
    in the case of a person who is entitled to either income support or state pension credit, the amount for the time being specified in paragraph 1(1)($e$)  of column (2) of Schedule 2 to the Income Support Regulations; or

    ($b$) 
    in the case of a person who is entitled to an income-based jobseeker’s allowance, the amount for the time being specified in paragraph 1(1)($e$)  of column (2) of Schedule 1 to the Jobseeker’s Allowance Regulations 1996;”; 
\end{enumerate}
\end{quotation}

($c$) after the definition of “social security office” insert—
\begin{quotation}
““state pension credit” means the benefit of that name payable under the State Pension Credit Act 2002;”.
\end{quotation}
\end{enumerate}

(3) In regulation 2, in paragraph (1) and in the heading to the regulation, after the words “income support”, wherever they occur, insert “, state pension credit”.

(4) In regulation 4\footnote{Regulation 4 was substituted by S.I.\ 1999/3178.}, in paragraph (1)($a$)  and in the heading to the regulation, after the words “income support” insert “, state pension credit”.

(5) In regulation 7, after the words “income support”, wherever they occur, insert “, state pension credit”.

(6) In Schedule 3, in the form, after the words “by way of income support”, insert “, state pension credit”.

\subsection[33. Amendment of the Council Tax (Deductions from Income Support) Regulations 1993]{Amendment of the Council Tax (Deductions from Income Support) Regulations 1993}

33.---(1)  The Council Tax (Deductions from Income Support) Regulations 1993\footnote{S.I.\ 1993/494; the relevant amending instrument is S.I.\ 1996/2344.} shall be amended in accordance with the following provisions of this regulation.

(2) In regulation 1, in paragraph (2)—
\begin{enumerate}\item[]
($a$) in the definition of “benefit week”, after the words “as the case may be” insert “regulation 1(2) of the State Pension Credit Regulations 2002\footnote{S.I.\ 2002/1792.} or”;

($b$) for the definition of “personal allowance for a single claimant aged not less than 25” substitute—
\begin{quotation}
    ““personal allowance for a single claimant aged not less than 25” means—
\begin{enumerate}\item[]
    ($a$) 
    in the case of a person who is entitled to either income support or state pension credit, the amount for the time being specified in paragraph 1(1)($e$)  of column (2) of Schedule 2 to the Income Support (General) Regulations 1987; or

    ($b$) 
    in the case of a person who is entitled to an income-based jobseeker’s allowance, the amount for the time being specified in paragraph 1(1)($e$)  of column (2) of Schedule 1 to the Jobseeker’s Allowance Regulations 1996;”; 
\end{enumerate}
\end{quotation}

($c$) after the definition of “social security office” insert—
\begin{quotation}
““state pension credit” means the benefit of that name payable under the State Pension Credit Act 2002\footnote{2002 c.\ 16.};”.
\end{quotation}
\end{enumerate}

(3) In regulation 2 and in the heading to the regulation, after the words “income support”, wherever they occur, insert “, state pension credit”.

(4) In regulation 3 and in the heading to the regulation after the words “income support”, wherever they occur, insert “, state pension credit”.

(5) In regulation 4, in paragraph (1)($f$), after the words “income support” insert “, state pension credit”.

(6) In regulation 5\footnote{Regulation 5 substituted by S.I.\ 1999/3178.} in paragraph (1)($a$)  and in the heading to the regulation, after the words “income support” insert “, state pension credit”.

(7) In regulation 8, in paragraphs (1), (2), (3) and (5), after the words “income support”, wherever they occur, insert “, state pension credit”.

\subsection[34. Amendment of the Community Charges (Deductions from Income Support) (Scotland) Regulations 1989]{Amendment of the Community Charges (Deductions from Income Support) (Scotland) Regulations 1989}

34.---(1)  The Community Charges (Deductions from Income Support) (Scotland) Regulations 1989\footnote{S.I.\ 1989/507; the relevant amending Instruments are S.I.\ 1990/113 and 1996/2344.} shall be amended in accordance with the following provisions of this regulation.

(2) In regulation 1, in paragraph (2)—
\begin{enumerate}\item[]
($a$) for the definition of “personal allowance for a single claimant aged not less than 25” substitute—
\begin{quotation}
    ““personal allowance for a single claimant aged not less than 25” means—
\begin{enumerate}\item[]
    ($a$) 
    in the case of a person who is entitled to either income support or state pension credit, the amount for the time being specified in paragraph 1(1)($e$)  of column (2) of Schedule 2 to the Income Support (General) Regulations 1987; or

    ($b$) 
    in the case of a person who is entitled to an income-based jobseeker’s allowance, the amount for the time being specified in paragraph 1(1)($e$)  of column (2) of Schedule 1 to the Jobseeker’s Allowance Regulations 1996;”; 
\end{enumerate}
\end{quotation}

($b$) after the definition of “single debtor”, insert—
\begin{quotation}
““state pension credit” means the benefit of that name payable under the State Pension Credit Act 2002;”.
\end{quotation}
\end{enumerate}

(3) In regulation 2, in paragraphs (1) and (2)($e$), and in the heading, after the words “income support” insert “, state pension credit”.

(4) In regulation 3\footnote{Regulation 3 was inserted by S.I.\ 1999/3178} in paragraph (1) and in the heading to the regulation after the words “income support” insert “, state pension credit”.

(5) In regulation 4, after the words “income support” wherever they occur insert “, state pension credit”.

\subsection[35. Amendment of the Community Charges (Deductions from Income Support) (No.\ 2) Regulations 1990]{Amendment of the Community Charges (Deductions from Income Support) (No.\ 2) Regulations 1990}

35.---(1)  The Community Charges (Deductions from Income Support) (No. 2) Regulations 1990\footnote{S.I.\ 1990/545; the relevant amending Instruments are S.I.\ 1996/2344 and 1999/3178.} shall be amended in accordance with the following provisions of this regulation.

\pagebreak[3]

(2) In regulation 1, in paragraph (2), after the definition of “single debtor”, insert—
\begin{quotation}
““state pension credit” means the benefit of that name payable under the State Pension Credit Act 2002;”.
\end{quotation}

(3) In regulation 2, in paragraphs (1) and (2)($e$), and in the heading to the regulation, after the words “income support” insert “, state pension credit”.

(4) In regulation 3, in paragraph (1) and in the heading to the regulation, after the words “income support”, wherever they occur, insert “, state pension credit”.

(5) In regulation 4, after the words “income support” wherever they occur insert “, state pension credit”.

\section[Part VII --- Transitional provisions]{Part VII\\*Transitional provisions}

\subsection[36. Persons entitled to income support immediately before the appointed day]{Persons entitled to income support immediately before the appointed day}

36.---(1)  This regulation applies in the case of any person (referred to as “the transferee”) who—
\begin{enumerate}\item[]
($a$) immediately before the appointed day, is entitled to income support; and

($b$) attains or has attained the qualifying age on or before the appointed day.
\end{enumerate}

(2) The transferee shall be treated as having made a claim for state pension credit in the period of 6 months immediately preceding the appointed day.

(3) The Secretary of State shall, so far as practicable, decide before the appointed day a claim for state pension credit treated as made under paragraph (2).

(4) A decision of the Secretary of State made in accordance with paragraph (3) may be revised by the Secretary of State at any time within the period of 13 months commencing on the date of notification of the decision if an application is made by the claimant to the Secretary of State or a person acting on his behalf for the decision to be revised.

(5) For the purposes of section 9 (duration of assessed income period), the decision of the Secretary of State takes effect on the appointed day.

(6) Notwithstanding the provisions of regulation 26B(4) of the Claims and Payments Regulations\footnote{Inserted by regulation 9 above.}, state pension credit may in the case of a transferee be payable in arrears if the income support to which he was entitled before the appointed day was paid in arrears.

(7) In the case of a transferee to whom paragraph (6) applies, any decision under section 10 of the 1998 Act which—
\begin{enumerate}\item[]
($a$) supersedes a decision awarding state pension credit to a transferee; and

($b$) is made on the ground that there has been a relevant change of circumstances since the decision was made or that it is anticipated that a relevant change of circumstances will occur,
\end{enumerate}
shall take effect from the first day of the benefit week in which the change occurs or is expected to occur.

(8) For the purpose of paragraph (7), “benefit week” means the period of 7 days ending on the day on which, in the claimant’s case, state pension credit is payable.

(9) Any payment made to a transferee to whom paragraph (10) applies—
\begin{enumerate}\item[]
($a$) in respect of a period falling on or after the appointed day;

($b$) which would have been payable under an award of income support but for the coming into force of the Act,
\end{enumerate}
shall be offset against any state pension credit payable under an award on or after 6th October 2003 on a claim treated as made under paragraph (2).

(10) This paragraph applies to a transferee in respect of whom no decision has been made on his claim for state pension credit which is treated as having been made in accordance with paragraph (2).

(11) If the Secretary of State determines that no state pension credit is payable, or that the amount payable is less than the payments referred to in paragraph (9), he shall determine the amount of the overpayment.

(12) The amount of any overpayment determined in accordance with paragraph (11) shall be recoverable by the Secretary of State by the same procedures and subject to the same conditions as if it were recoverable under section 71 (1) of the Administration Act.

(13) Where the transferee—
\begin{enumerate}\item[]
($a$) has, immediately before the appointed day, an award of income support payable by direct credit transfer in accordance with regulation 21 of the Claims and Payments Regulations; and

($b$) state pension credit is payable or treated as payable to him as from the appointed day,
\end{enumerate}
the state pension credit shall be paid by direct credit transfer into the same bank or other account as the payment of income support; and for this purpose, any application made or treated as made and any consent given or treated as given in relation to the payment of income support shall be treated as made or given in relation to the payment of state pension credit.

(14) Where—
\begin{enumerate}\item[]
($a$) the transferee had immediately before the appointed day an award of income support from which deductions were made or where part of the benefit was paid to a third party in accordance with—
\begin{enumerate}\item[]
(i) regulation 34A of, and Schedule 9A to, the Claims and Payments Regulations (mortgage interest payments); or

(ii) regulation 35 of, and Schedule 9B to, those Regulations (deductions which may be made and payments to third parties); and
\end{enumerate}

($b$) state pension credit is payable or treated as payable to the transferee as from the appointed day,
\end{enumerate}
then as from the appointed day, those deductions shall be made from the transferee’s state pension credit and those payments of part of the benefit shall continue to be made to the third party in accordance with those provisions.

(15) In the case of a transferee who, on the appointed day, has been a patient for more than 6 weeks but not more than 13 weeks, paragraph 2 of Schedule III to the State Pension Credit Regulations shall have effect as if for the references to “13 weeks” there were substituted references to “6 weeks”.

\subsection[37. Assessed income period]{Assessed income period}

37.---(1)  A person to whom paragraph (2) applies shall have an assessed income period allotted to him by the Secretary of State of at least 5 years but not exceeding 7 years beginning on the day the decision takes effect, unless regulation 10(1) of the State Pension Credit Regulations applies in his case.

(2) This paragraph applies to the first assessed income period specified in respect of a person who—
\begin{enumerate}\item[]
($a$) attains or has attained the age of 65 or whose partner attains or has attained that age on or before the appointed day; and

($b$) is awarded state pension credit with effect from the appointed day.
\end{enumerate}

\subsection[38. Claims for state pension credit]{Claims for state pension credit}

38.---(1)  A claim for state pension credit may be made before the appointed day by a person who is not in receipt of income support at the time the claim is made.

(2) Where the Secretary of State is satisfied that unless there is a change in the claimant’s circumstances before the appointed day he will satisfy the conditions for entitlement to state pension credit on that day, then the Secretary of State may—
\begin{enumerate}\item[]
($a$) treat that claim as if made for a period beginning with the appointed day; and

($b$) award benefit accordingly, but subject to the condition that the claimant does in fact satisfy those conditions when benefit becomes payable under the award.
\end{enumerate}

(3) A decision under paragraph (2)($b$)  to award benefit may be revised under section 9 of the 1998 Act if the requirements for entitlement to state pension credit are found not to have been satisfied on the appointed day.

(4) A claim for state pension credit made in the period of 12 months beginning with the appointed day may be treated as made on that day if the claimant satisfied the conditions for entitlement to state pension credit on that day.

(5) A person who does not fall within paragraph (4)—
\begin{enumerate}\item[]
($a$) solely because he does not satisfy the conditions for entitlement to state pension credit on the appointed day; but

($b$) does satisfy those conditions on a day after the appointed day but before the day on which the claim is received by the Secretary of State,
\end{enumerate}
shall be treated as having made the claim on the day the conditions were first satisfied in his case.

(6) A claim for income support made in the period of 6 months preceding the appointed day may be treated also as a claim for state pension credit if the claimant—
\begin{enumerate}\item[]
($a$) is not entitled to income support; and

($b$) has attained the age of 60 on the date the claim is made or will have attained that age on the appointed day.
\end{enumerate}

(7) Paragraphs (2) and (3) shall apply to a claim treated as made under paragraph (6) as they apply to a claim made under paragraph (1).

(8) In the case of a person who—
\begin{enumerate}\item[]
($a$) on the appointed day has attained the qualifying age;

($b$) was, within the period of 6 months preceding the appointed day, entitled to income support; and

($c$) was not entitled to income support on the day immediately preceding the appointed day,
\end{enumerate}
that person shall be treated as having made a claim for state pension credit for a period beginning on the appointed day.

(9) The Secretary of State may treat a claim for state pension credit made in accordance with paragraph (1) as also a claim for income support made on the same day. 

\bigskip

 Signed 
for the purposes of regulation 23(i)  of the Regulations.

{\raggedleft
\emph{John Heppell}\\*\emph{Philip Woolas}\\*Two of the Lords Commissioners of Her Majesty's Treasury

}

%St Andrew's House, Edinburgh

%Dated
4th December 2002

\bigskip

 Signed 
by authority of the Secretary of State for Work and Pensions
both for the purpose of concurring in the making of regulation 23(i)  of the Regulations and for the purposes of the remainder of the Regulations. 

{\raggedleft
\emph{Ian McCartney}\\*Minister of State,\\*Department for Work and Pensions

}

%St Andrew's House, Edinburgh

%Dated
4th December 2002


\small

\part[Schedule --- Provisions conferring powers exercised in making these Regulations]{Schedule\\*Provisions conferring powers exercised in making these Regulations}

{\footnotesize\hbadness=10000
%\begin{tabulary}{\linewidth}{JJJ}
\begin{longtable}{p{66.02693pt}p{59.43776pt}p{228.52734pt}}
\hline
Column (1)	&Column (2)	&Column (3)\\
\itshape Short title	& \itshape Provision	& \itshape Relevant Amendments\\
\hline
\endhead
\hline
\endlastfoot
Social Security Administration Act 1992\footnote{1992 c.\ 5.}	&Section 5(1) ($a$)  to ($e$), ($h$)  to ($l$) and ($p$)  and (3A)	&Section 5(1)($h$)  and ($hh$)  are extended in their application to state pension credit by section 5(3A), inserted by the State Pension Credit Act 2002 (c.\ 16), Schedule 1, paragraph 3(3);\\
&	Section 7A	&Inserted by section 71 of the Welfare Reform and Pensions Act 1999;\footnote{1999 c.\ 30.}\\
	&Section 15A	&Section 15A is inserted by the Social Security (Mortgage Interest Payments) Act 1992\footnote{1992 c.\ 33.} Schedule, and is applied to state pension credit by subsections (1A) and (2)($aa$)  of that section, inserted by the State Pension Credit Act 2002, Schedule 2, paragraph 9;\\
	&Section 159B(1)	&Section 159B is inserted by the State Pension Credit Act 2002, Schedule 2, paragraph 17;\\
	&Section 189(4) to (6)	&Social Security Act 1998 (c.\ 14), Schedule 7, paragraph 109.\\
Social Security Contributions and Benefits Act 1992\footnote{1992 c.\ 4.}	&Section 3(2)	&The Social Security Contributions (Transfer of Functions, etc.)\ Act 1999 (c.\ 2), Schedule 3, paragraph 3;\\
	&Section 138(1)($a$) \newline
Section 175(3) to (5)
&Social Security (Incapacity for Work) Act 1994 (c.\ 18), Schedule 1, paragraph 36; 
Social Security Contributions (Transfer of Functions, etc.)\ Act 1999, Schedule 3, paragraph 29. Section 175 (3) to (5) is applied to provisions of the State Pension Credit Act 2002 by section 19(1) of that Act.\\
Social Security Act 1998\footnote{1998 c.\ 14; Chapter II of Part I is applied to state pension credit by the amendment made to section 8 of the 1998 Act by the State Pension Credit Act 2002, Schedule 1, paragraph 6.}	&Sections 10(3) and (6)	&Chapter II of Part I is applied to State Pension Credit by section 8(3)($bb$)  and (4) as inserted and amended by the State Pension Credit Act 2002, Schedule 1, paragraph 6.\\
&Section 18
\\
&Section 79(4)\\
State Pension Credit Act 2002\footnote{2002 c.\ 16.}&
Section 1(5)($b$) 
\\&
Section 2(3)($b$) 
\\&
Section 3(5)
\\&
Section 7(4)
\\&
Section 12(2)
\\&
Section 13
\\&
Section 15(1)($j$), (3) and (6)($b$) 
\\&
Section 17(2)($a$) 
\\&
Schedule 1, paragraph 13\\
%\end{tabulary}
\end{longtable}

}
	
\part{Explanatory Note}

\renewcommand\parthead{— Explanatory Note}

\subsection*{(This note is not part of the Regulations)}

The Regulations contained in this Instrument are made either by virtue of, or consequential upon, provisions in the State Pension Credit Act 2002 (c.\ 16) (“the 2002 Act”). This Instrument is made before the expiry of the period of 6 months beginning with the coming into force of those provisions; the regulations in it are therefore exempt in accordance with section 173(5) of the Social Security Administration Act 1992 from the requirement in section 172(1) of that Act to refer proposals to make Regulations to the Social Security Advisory Committee and are made without reference to that Committee.

The amendments in Parts II, III, V and VI are consequential on the introduction of state pension credit. Part VII contains transitional matters.

Part I of the Regulations provides for their citation, commencement and interpretation.

Part II amends the Social Security (Claims and Payments) Regulations 1987 so as to make separate provision for claims for, and payment of, state pension credit. Provision is made (regulation 4) for claims to be made by telephone or in person. Claims may be made at any time in the 4 months preceding the day a claimant attains the qualifying age for entitlement to state pension credit.

Claimants may be required to provide information as to the likelihood of future changes in their circumstances (regulation 5).

Regulation 7 provides for the date on which entitlement to state pension credit is to begin and regulation 9 for the day on which, and the method by which, the credit is to be paid.

Regulations 8, 10 to 12 and 14 apply provisions currently in the Social Security (Claims and Payments) Regulations 1987 to state pension credit. Regulation 13 provides that regulation 35A (transitional provisions relating to persons in hostels and certain residential accommodation) is not to apply to state pension credit.

Part III applies provisions of the Social Security and Child Support (Decisions and Appeals) Regulations 1999 to state pension credit. Regulation 20 further provides that, where a claimant’s appropriate minimum guarantee includes a transitional amount, provisions in section 159B of the Social Security Administration Act 1992 (inserted by the 2002 Act) relating to the effect of alterations affecting state pension credit, are not to apply. Regulation 22 makes provision as to the date from which a decision under section 10 of the Social Security Act 1998 (superseding decision) resulting from a change of circumstances takes effect.

Part IV makes changes to the State Pension Credit Regulations 2002. It adds provisions—
\begin{itemize}\item
    relating to part-weeks;
\item
    as to the meaning and calculation of “earnings” for the purposes of state pension credit;
\item
    relating to the rounding of fractions;
\item
    as to the disregard of assets of any business owned by the claimant. 
\end{itemize}

Part V adds references to state pension credit to—
\begin{itemize}\item
    the Social Security (Payments on account, Overpayments and Recovery) Regulations 1988;
\item
    the Social Security (Attendance Allowance) Regulations 1991;
\item
    the Child Support (Arrears, Interest and Adjustment of Maintenance Assessments) Regulations 1992;
\item
    the Child Support (Maintenance Calculations and Special Cases) Regulations 2000;
\item
    the Social Security (Disability Living Allowance) Regulations 1991;
\item
    the Income Support (General) Regulations 1987;
\item
    the Jobseeker’s Allowance Regulations 1996;
\item
    the Social Fund Maternity and Funeral Expenses (General) Regulations 1987;
\item
    the Social Fund Cold Weather Payments (General) Regulations 1988. 
\end{itemize}
It also makes amendments to the Income Support (General) Regulations 1987 consequential upon the 2002 Act which removes from entitlement to income support those who have attained the qualifying age for the purposes of state pension credit.

Part VI adds references to state pension credit to—
\begin{itemize}\item
    the Fines (Deductions from Income Support) Regulations 1992;
\item
    the Council Tax (Deductions from Income Support) Regulations 1993;
\item
    the Community Charges (Deductions from Income Support) (Scotland) Regulations 1989;
\item
    the Community Charges (Deductions from Income Support) (No.\ 2) Regulations 1990. 
\end{itemize}

Part VII makes transitional provisions. Regulation 36 provides for those who were entitled to income support immediately before the 2002 Act comes into force to be treated as making a claim for state pension credit and for the determination of that claim. Regulation 37 makes provision for variable assessed income periods. Regulation 38 provides that a person not in receipt of income support may claim state pension credit before provisions of the 2002 Act relating to entitlement are commenced.

The impact on business of applying these Regulations is minimal and the publication of a regulatory impact assessment is therefore not necessary. A summary of the contents of the assessment made for the State Pension Credit Bill was published at paragraphs 183 and 184 of the Explanatory Notes relating to the Bill. A copy of the summary can be obtained from the Department for Work and Pensions, Regulatory Impact Unit, 3rd Floor, The Adelphi, 1–11 John Adam Street, London \textsc{\lowercase{WC2N 6HT}}. 

\end{document}