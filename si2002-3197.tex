\documentclass[12pt,a4paper]{article}

\newcommand\regstitle{The State Pension Credit (Consequential, Transitional and~Miscellaneous Provisions) (No.~2) Regulations 2002}

\newcommand\regsnumber{2002/3197}

%\opt{newrules}{
\title{\regstitle}
%}

%\opt{2012rules}{
%\title{Child Maintenance and~Other Payments Act 2008\\(2012 scheme version)}
%}

\author{S.I.\ 2002 No.\ 3197}

\date{Made
19th December 2002\\
Laid before Parliament
23rd December 2002\\
Coming into~force\\
for~the purposes of regulations 1, 5, 6 and~7(4) 7th April 2003\\
for~all other purposes 6th October 2003
}

%\opt{oldrules}{\newcommand\versionyear{1993}}
%\opt{newrules}{\newcommand\versionyear{2003}}
%\opt{2012rules}{\newcommand\versionyear{2012}}

\usepackage{csa-regs}

\setlength\headheight{42.11603pt}

%\hbadness=10000

\begin{document}

\maketitle

\enlargethispage{\baselineskip}

\noindent
The Treasury, with the concurrence of the Secretary of State, in relation to~regulation 3, and~the Secretary of State in relation to~the remainder of the Regulations, in exercise of the powers conferred upon them by sections~3(2) and~175(3) to~(5) of the Social Security Contributions and~Benefits Act 1992\footnote{1992 c.~4; section~3(2) is amended by the Social Security Contributions (Transfer of Functions etc.) Act 1999 (c.~2) (“the Transfer of Functions Act”), Schedule~3, paragraph~3, Section~175(3) to~(5) are amended by the Social Security (Incapacity for~Work) Act 1994 (c.~18), Schedule~1, paragraph~36 and~by the Transfer of Functions Act, Schedule~3, paragraph~29. Section~175(3) to~(5) are applied to~the provisions of the State Pension Credit Act 2002 (c.~16) (“the State Pension Credit Act”) by section~19(1) of that Act.}, sections~5(1)($p$), 15A(2)($aa$), 189(4) to~(6) and~191 of the Social Security Administration Act 1992\footnote{1992 c.~5; section~15A is inserted by the Social Security (Mortgage Interest Payments) Act 1992 (c.~33), Schedule~and~is applied to~state pension credit by subsections (1A) and~(2)(aa) of that section, inserted by the State Pension Credit Act, Schedule~2, paragraph~9. Section~189(4) to~(6) is amended by the Social Security Act 1998 (c.~14), Schedule~7, paragraph~109. Section~191 is cited because of the meaning ascribed to~the word “prescribe”.}, sections~26(1) and~(4)($a$), 35(1) and~36(2) of the Jobseekers Act 1995\footnote{1995 c.~18; section~35(1) is cited because of the meaning ascribed to~the words “prescribed” and~“regulations”.}, sections~10(1) and~(5)($a$)  and~26(3) of the Child Support Act 1995\footnote{1995 c.~34.}, section~10(3) and~(6), 79(4) and~84 of the Social Security Act 1998\footnote{Chapter II of Part 1 of the Act is applied to~state pension credit by section~8(3)($bb$) and (4) as inserted and amended by the State Pension Credit Act, Schedule~1, paragraph 6. Section 84 is cited because of the meaning ascribed to~the word “prescribe”.\looseness=-1} and~sections~2(3)($b$)  and~(6), 12(2)($b$), 13, 15(3) and~(6)($a$)  and~($b$),~16(2)($a$)  and~17(2)($a$)  of the State Pension Credit Act 2002, and~of all other powers enabling them in that behalf, by this Instrument, which contains only regulations made by virtue of, or~consequential upon, sections~1 to~17 of the State Pension Credit Act 2002 and~which is made before the end of the period of 6 months beginning with the coming into~force of those provisions\footnote{Paragraph 20($a$) of Schedule~2 to~the State Pension Credit Act added the provisions of that Act to~the list of “the relevant enactments” in respect of which regulations must normally be referred to~the Social Security Advisory Committee. See however section~173(5)($b$) of the Social Security Administration Act 1992.}, make the following Regulations: 

{\sloppy

\tableofcontents

}

\bigskip

\setcounter{secnumdepth}{-2}

\subsection[1. Citation, commencement and~interpretation]{Citation, commencement and~interpretation}

1.---(1)  These Regulations may be cited as the State Pension Credit (Consequential, Transitional and~Miscellaneous Provisions) (No.~2) Regulations 2002 and~shall come into~force—
\begin{enumerate}\item[]
($a$) for~the purposes of this regulation and~regulations 5, 6 and~7(4), on 7th April 2003;

($b$) for~all other purposes, on 6th October 2003.
\end{enumerate}

(2) In these Regulations, “the principal Regulations” means the State Pension Credit Regulations 2002\footnote{S.I.~2002/1792 as amended by S.I.~2002/3019.} and~references in these Regulations to~a regulation or~a Schedule~are, unless the context otherwise requires, to~a regulation of, or~a Schedule~to, those Regulations.

\subsection[2. Amendment of the principal Regulations]{Amendment of the principal Regulations}

2.  The principal Regulations shall be amended in accordance with the Schedule~to~these Regulations.

\subsection[3. Further amendments to~the principal Regulations relating to~earnings]{Further amendments to~the principal Regulations relating to~earnings}

3.---(1)  In regulation 17A (earnings of an employed earner)—
\begin{enumerate}\item[]
($a$) in paragraph~(2), for~“(3) and~(4)”, there shall be substituted “(3),~(4) and~(4A)”;

($b$) after paragraph~(3)($d$), there shall be added the following—
\begin{quotation}
“;

($e$) any payment of compensation made pursuant to~an award by an employment tribunal in respect of unfair dismissal or~unlawful discrimination”;
\end{quotation}

($c$) after paragraph~(4), there shall be inserted the following paragraph—
\begin{quotation}
“(4A) One half of any sum paid by a claimant by way of a contribution towards an occupational pension scheme or~a personal pension scheme shall, for~the purpose of calculating his earnings in accordance with this regulation, be disregarded.”.
\end{quotation}
\end{enumerate}

(2) In regulation 17B (earnings of self-employed earners)—
\begin{enumerate}\item[]
($a$) before paragraph~(2)($a$), there shall be inserted the following sub-paragraph—
\begin{quotation}
“($za$) “board and~lodging accommodation” has the same meaning as in paragraph~8(2) of Schedule~IV;”;
\end{quotation}

($b$) at the end of regulation 12(2)($e$)  of the Social Security Benefit (Computation of Earnings) Regulations 1996\footnote{S.I.~1996/2745.} as given effect by paragraph~(4)($b$), there shall be added the words “being an award made by one of the Sports Councils named in section~23(2) of the National Lottery etc.\ Act 1993\footnote{1993 c.~39.} out of sums allocated to~it for~distribution under that section”.
\end{enumerate}

\subsection[4. Amendment of amounts in the principal Regulations]{Amendment of amounts in the principal Regulations}

4.---(1)  In regulation 6 (amount of the guarantee credit)—
\begin{enumerate}\item[]
($a$) in paragraph~(1)—
\begin{enumerate}\item[]
(i) in sub-paragraph~($a$), for~the sum “£154” there shall be substituted the sum “£155$.$80”;

(ii) in sub-paragraph~($b$), for~the sum “£100” there shall be substituted the sum “£102$.$10”;
\end{enumerate}

($b$) in paragraph~(5)—
\begin{enumerate}\item[]
(i) in sub-paragraph~($a$), for~the sum “£43$.$45” there shall be substituted the sum “£42$.$95”;

(ii) in sub-paragraph~($b$), for~the sum “£86$.$90” there shall be substituted the sum “£85$.$90”;
\end{enumerate}

($c$) in paragraph~(8), for~the sum “£25$.$35” there shall be substituted the sum “£25$.$10”.
\end{enumerate}

(2) In regulation 7(2) (savings credit), for~the sum “£77” there shall be substituted the sum “£77$.$45” and~for~the sum “£123” there shall be substituted the sum “£123$.$80”.

(3) In paragraph~1 of Schedule~III (polygamous marriages)—
\begin{enumerate}\item[]
($a$) in sub-paragraph~(5)—
\begin{enumerate}\item[]
(i) in the substituted regulation 6(1)($a$), for~the sum “£154” there shall be substituted the sum “£155$.$80”;

(ii) in the substituted regulation 6(1)($b$), for~the sum “£54” there shall be substituted the sum “£53$.$70”;
\end{enumerate}

($b$) in sub-paragraph~(7), in the substituted regulation 7(2), for~the sum “£123” there shall be substituted the sum “£123$.$80”.
\end{enumerate}

\subsection[5. Amendment of the Social Security (Claims and~Payments) Regulations 1987]{Amendment of the Social Security (Claims and~Payments) Regulations 1987}

5.---(1)  The Social Security (Claims and~Payments) Regulations 1987\footnote{S.I.~1987/1968.} shall be amended in accordance with the following paragraphs of this regulation.

(2) In regulation 34A\footnote{Regulation 34A is inserted by S.I.~1992/1026 and amended by S.I.~2002/3019.} (deduction of mortgage interest which shall be made from benefit and~paid to~qualifying lenders)—
\begin{enumerate}\item[]
($a$) for~the words “or~(1A)” there shall be substituted the words “or, subject to~paragraph~(1A), section~15A(1A)”;

($b$) after paragraph~(1), there shall be inserted the following paragraph—
\begin{quotation}
“(1A) Paragraph (1) shall only apply in relation to~a relevant beneficiary who is entitled to~state pension credit where he is entitled to~a guarantee credit.”.
\end{quotation}
\end{enumerate}

(3) After regulation 34A, there shall be inserted the following regulation—
\begin{quotation}
\subsection*{“Deductions of mortgage interest which may be made from benefits and~paid to~qualifying lenders in other cases}

34B.---(1)  In relation to~cases to~which section~15A(1A) of the Social Security Administration Act 1992\footnote{Section 15A(1A) is inserted by the State Pension Credit Act 2002 (c.~16), Schedule~2, paragraph 9(2).} applies (other than those referred to~in regulation 34A(1A))—
\begin{enumerate}\item[]
($a$) in the circumstances specified in paragraph~2A(1) of Schedule~9A: and

($b$) in either of the further circumstances specified in paragraph~2A(2) of that Schedule,
\end{enumerate}
such part of any relevant benefits to~which a relevant beneficiary is entitled as may be specified in that Schedule~may be paid by the Secretary of State directly to~the qualifying lender and~shall be applied by that lender towards the discharge of the liability in respect of that interest\footnote{\emph{See} section 15A for~the definition of “relevant beneficiary”, “qualifying lender” and “mortgage interest”.}.

(2) The provisions of Schedule~9A\footnote{Schedule~9A is inserted by S.I.~1992/1026.} shall have effect in relation to~mortgage interest payments made under this regulation.”.
\end{quotation}

(4) In paragraph~3(5)($a$)  of Schedule~9\footnote{Paragraph 3(5) is inserted by S.I.~1992/1026.} (deductions from benefit and~direct payments to~third parties), after “regulation 34A” there shall be inserted “or~34B”.

(5) In Schedule~9A (deductions of mortgage interest from benefit and~made to~qualifying lenders)—
\begin{enumerate}\item[]
($a$) in the heading, for~“Regulation 34A” there shall be substituted “Regulations 34A and~34B”;

($b$) at the end of the heading to~paragraph~2\footnote{Paragraph 2 is substituted by S.I.~1995/1613 and amended by S.I.~1996/1460.}, there shall be added the words “for~the purposes of regulation 34A”;

($c$) after paragraph~2, there shall be inserted the following paragraph—
\begin{quotation}
\section*{\itshape\sloppy\hbadness=3460 “Specified circumstances for~the purposes of regulation 34B}

2A.---(1)  The circumstances referred to~in regulation 34B are that—
\begin{enumerate}\item[]
($a$) the relevant beneficiary is entitled to~a savings credit as construed in accordance with sections~1 and~3 of the 2002 Act and~not to~a guarantee credit; and

($b$) sub-paragraphs ($a$)  and~($b$)  of paragraph~2 apply.
\end{enumerate}

(2) The further circumstances referred to~in that regulation are that—
\begin{enumerate}\item[]
($a$) the relevant beneficiary has requested the Secretary of State in writing to~make such payments to~the qualifying lender; or

($b$) the Secretary of State has determined that it would be in the relevant beneficiary’s interests, or~in the interests of his family, to~make such payments to~the qualifying lender.
\end{enumerate}

(3) In making the determination referred to~in sub-\hspace{0pt}paragraph~(2)($b$), the Secretary of State shall have regard to whether or~not the relevant beneficiary is in arrears with his payments to~the qualifying lender.

(4) For~the purposes of sub-paragraph~(2)($b$), “a family” comprises the relevant beneficiary, his partner, any additional partner to~whom section~12(1)($c$)  of the 2002 Act applies and~any person who has not attained the age of 19, is treated as a child for~the purposes of section~142 of the Contributions and~Benefits Act and~lives with the relevant beneficiary or~the relevant beneficiary’s partner.”.
\end{quotation}

($d$) in paragraph~3\footnote{Sub-paragraphs (1A) and (10) are inserted by S.I.~2002/3019.}—
\begin{enumerate}\item[]
(i) in sub-paragraph~(1A), after the words “qualifying lender” there shall be inserted the words “or, in accordance with regulation 34B, may be paid directly to~the qualifying lender”;

(ii) in sub-paragraph~(10), after the words “state pension credit”, in the first place where those words occur, there shall be inserted the words “but not in a case to~which sub-paragraph~(11) applies,”;

(iii) after sub-paragraph~(10), there shall be added the following sub-paragraph—
\begin{quotation}
“(11) This sub-paragraph~applies where the last day on which either the claimant or~his partner were entitled to~income support or~to~an income-based jobseeker’s allowance was no more than twelve weeks before—
\begin{enumerate}\item[]
($a$) except where paragraph~($b$)  applies, the first day of entitlement to~state pension credit; or

($b$) where the claim for~state pension credit was treated as made on a day earlier than the day on which it was actually made (“the actual date”), the day which would have been the first day of entitlement to~state pension credit had the claim been treated as made on the actual date.”;
\end{enumerate}
\end{quotation}
\end{enumerate}

($e$) in paragraphs 6 and~7, after “regulation 34A” there shall be inserted “or~34B”;

($f$) in paragraph~9(4), for~the words “Regulation 34A shall not” there shall be substituted the words “Neither regulation 34A nor~34B shall”;

($g$) in paragraph~11—
\begin{enumerate}\item[]
(i) in sub-paragraph~(1), after “regulation 34A” there shall be inserted “or~34B”;

(ii) in sub-paragraph~(2)($a$)(i), after the words “Jobseeker’s Allowance Regulations” there shall be inserted the words “or~paragraph~9 of Schedule~II to~the State Pension Credit Regulations”.
\end{enumerate}
\end{enumerate}

\subsection[6. Amendment of the Social Security and~Child Support (Decisions and~Appeals) Regulations 1999]{Amendment of the Social Security and~Child Support (Decisions and~Appeals) Regulations 1999}

6.  In regulation 7 of the Social Security and~Child Support (Decisions and~Appeals) Regulations 1999\footnote{S.I.~1999/991; the relevant amending instrument is S.I.~2002/3019.} (date from which a decision superseded under section~10 takes effect)—
\begin{enumerate}\item[]
($a$) after paragraph~(17A) there shall be inserted the following paragraphs—
\begin{quotation}
“(17B) Subject to~paragraph~(23), where a claimant who is in receipt of state pension credit or~his partner is aged 65 or~over, the claimant’s appropriate minimum guarantee includes an amount determined in accordance with Schedule~II to~the State Pension Credit Regulations and~there is a change of circumstances referred to~in paragraph~(17C), a decision made under section~10 shall take effect—
\begin{enumerate}\item[]
($a$) on the first anniversary of the date on which the claimant’s housing costs were first met under that Schedule; or

($b$) where the change occurred after the first anniversary of the date referred to~in sub-paragraph~($a$), on the next anniversary of that date following the date of the change.
\end{enumerate}

(17C) Paragraph (17B) applies in a case where a non-dependant commences residing with the claimant or~there is an increase in a non-dependant’s income.”.
\end{quotation}

($b$) for~paragraph~(23), there shall be substituted the following paragraph—
\begin{quotation}
(23) Where, in any case to~which paragraph~(14), (17A), (17B) or~(18) applies, a claimant has been continuously in receipt of, or~treated as having been continuously in receipt of income support, a jobseeker’s allowance or~state pension credit, or~one of those benefits followed by the other, and~he or~his partner continues to~receive any of those benefits, the anniversary to~which those paragraphs refer shall be—
\begin{enumerate}\item[]
($a$) in the case of income support or~jobseeker’s allowance, the anniversary of the earliest date on which benefit in respect of those mortgage interest costs became payable;

($b$) in the case of state pension credit, the relevant anniversary date determined in accordance with paragraph~7 of Schedule~II to~the State Pension Credit Regulations.”.
\end{enumerate}
\end{quotation}
\end{enumerate}

\subsection[7. Amendments to~other Regulations]{Amendments to~other Regulations}

7.---(1)  In regulation 8 of the Social Security (Child Maintenance Bonus) Regulations 1996\footnote{S.I.~1996/3195 to~which there are amendments which are not relevant to~these Regulations.} (retirement)—
\begin{enumerate}\item[]
($a$) in paragraph~(2), for~the words “income support”, in both places where they occur, there shall be substituted the words “state pension credit”;

($b$) after paragraph~(2) there shall be inserted the following paragraph—
\begin{quotation}
“(2A) In paragraph~(2), “state pension credit” means the benefit of that name payable under the State Pension Credit Act 2002.”.
\end{quotation}
\end{enumerate}

(2) In regulation 17 of the Social Security (Back to~Work Bonus) (No.~2) Regulations 1996\footnote{S.I.~1996/2570 to~which there are amendments which are not relevant to~these Regulations.} (persons attaining pensionable age)—
\begin{enumerate}\item[]
($a$) in paragraphs (4) and~(6), for~the words “income support”, in all places where they occur, there shall be substituted the words “state pension credit”;

($b$) at the end, there shall be added the following paragraph—
\begin{quotation}
“(8) In this regulation, “state pension credit” means the benefit of that name payable under the State Pension Credit Act 2002.”.
\end{quotation}
\end{enumerate}

(3) In regulation 1(2) of the Community Charges (Deductions from Income Support) (Scotland) Regulations 1989\footnote{S.I.~1989/507; the relevant amending Instruments are S.I.~1990/113, 1996/2344 and 2002/3019.} (interpretation), for~the definition of “personal allowance for~a couple where both members are aged not less than 18” there shall be substituted the following definition—
\begin{quotation}
““personal allowance for~a couple where both members are aged not less than 18” means—
\begin{enumerate}\item[]
($a$) 
in the case of a person who is entitled to~either income support or~state pension credit, the amount for~the time being specified in paragraph~1(3)($c$)  of column (2) of Schedule~2 to~the Income Support (General) Regulations 1987\footnote{S.I.~1987/1967.}; or

($b$) 
in the case of a person who is entitled to~an income-based jobseeker’s allowance, the amount for~the time being specified in paragraph~1(3)($e$)  of column (2) of Schedule~1 to~the Jobseeker’s Allowance Regulations 1996\footnote{S.I.~1996/207.};”.
\end{enumerate}
\end{quotation}

(4) In regulation 36 of the State Pension Credit (Consequential, Transitional and~Miscellaneous Provisions) Regulations 2002\footnote{S.I.~2002/3019.} (persons entitled to~income support immediately before the appointed day)—
\begin{enumerate}\item[]
($a$) in paragraph~(15)—
\begin{enumerate}\item[]
(i) at the beginning there shall be inserted the words “Except where paragraph~(16) applies,”;

(ii) after the words “transferee who”, there shall be inserted the words “or~whose partner”;
\end{enumerate}

($b$) at the end, there shall be added the following paragraphs—
\begin{quotation}
“(16) Paragraphs (17) to~(19) apply only in relation to a transferee whose applicable amount immediately before the appointed day was determined in accordance with paragraph~13(2) of Schedule~7 to~the Income Support (General) Regulations 1987 (“the Income Support Regulations”) (persons in residential accommodation who become patients);

(17) Where a transferee is a patient on the appointed day and~continues to~be a patient after that day, section~2(3) has effect for~so long as the transferee continues to~be a patient with the substitution for~the reference to~the standard minimum guarantee in paragraph~($a$)  of the amount which is for~the time being specified as the applicable amount in column (2) of paragraph~13(2) of Schedule~7 to~the Income Support Regulations less the amount applicable under regulation 17(1)($f$)  or~($g$)  of those Regulations.

(18) Where a transferee—
\begin{enumerate}\item[]
($a$) ceases to~be a patient on or~after the appointed day but again becomes a patient no more than 28 days after the last day on which he was previously a patient; and

($b$) was in residential accommodation (as defined for~the purposes of the Income Support Regulations) immediately before again becoming a patient,
\end{enumerate}
section~2(3) has effect when the transferee again becomes a patient with the substitution for~the reference to~the standard minimum guarantee in paragraph~($a$)  of the amount which is for~the time being specified as the applicable amount in column (2) of paragraph~13(2) of Schedule~7 to~the Income Support Regulations less the amount applicable under regulation 17(1)($f$)  or~($g$)  of those Regulations.

(19) Where a transferee—
\begin{enumerate}\item[]
($a$) ceases to~be a patient on or~after the appointed day but again becomes a patient no more than 28 days after the last day on which he was previously a patient;

\begin{sloppypar}
($b$) was not in accommodation referred to in paragraph~(18)($b$)  immediately before again becoming a patient; and
\end{sloppypar}

($c$) has been a patient for~a total period of more than six weeks,
\end{enumerate}
section~2(3) shall have effect when the transferee again becomes a patient with the substitution for~the reference to~the standard minimum guarantee in paragraph~($a$)  of a reference to~an amount determined by taking the amount for~the time being specified in regulation 6(1)($a$)  of the State Pension Credit Regulations and~reducing it by an amount equal to~20 per cent of the weekly rate of the basic pension for~the time being specified in section~44(4) of the Social Security Contributions and~Benefits Act 1992.”.
\end{quotation}
\end{enumerate}

\bigskip

Signed for~the purposes of regulation 3 of the Regulations. 
%I concur
%By authority of the Lord Chancellor

{\raggedleft
\emph{Nick Ainger}\\*
\emph{Jim Fitzpatrick}\\*
Two of the Lords Commissioners of Her Majesty's Treasury

}

18th December 2002

\bigskip

Signed 
by authority of the 
Secretary of State for~Work and~Pensions
both for~the purpose of concurring in the making of regulation 3 of the Regulations and~for~the purposes of the remainder of the Regulations. 
%I concur
%By authority of the Lord Chancellor

{\raggedleft
\emph{Ian McCartney}\\*
%Secretary
Minister
%Parliamentary Under-Secretary 
of State,\\*Department 
for~Work and~Pensions

}

19th December 2002

\small

\part[Schedule --- Amendment of the principal Regulations]{Schedule\\*Amendment of the principal Regulations}

\renewcommand\parthead{--- Schedule}

1.  In regulation 1(2) (interpretation)—
\begin{enumerate}\item[]
($a$) at the end of the definition of “care home” there shall be added the words “and~in Scotland~means a care home service” and~after that definition there shall be inserted the following definition—
\begin{quotation}
““care home service” has the meaning assigned to~it by section~2(3) of the Regulation of Care (Scotland) Act 2001\footnote{2001 asp.~8.};”;
\end{quotation}

($b$) after the definition of “qualifying person” there shall be inserted the following definition—
\begin{quotation}
““voluntary organisation” means a body, other than a public or~local authority, the activities of which are carried on otherwise than for~profit;”.
\end{quotation}
\end{enumerate}

\medskip

2.  At the end of regulation 5(1) (persons treated as being or~not being members of the same household), there shall be added the following sub-paragraphs—
\begin{quotation}
“($g$) he is not habitually resident in the United Kingdom, the Channel Islands, the Isle of Man or~the Republic of Ireland;

($h$) he is a person subject to~immigration control within the meaning of section~115(9) of the Immigration and~Asylum Act 1999\footnote{1999 c.~33.}”.
\end{quotation}

\medskip

3.  In regulation 13B(2) (date on which benefits are treated as paid), the words “respect of” shall be omitted.

\medskip

4.  In regulation 16 (retirement pension income)—
\begin{enumerate}\item[]
($a$) for~the word “paragraph” there shall be substituted the word “paragraphs”;

($b$) in the new section~16(1)($k$), for~the words “Civil List 1975” there shall be substituted the words “Civil List Act 1975”;

($c$) at the end there shall be added the following—
\begin{quotation}
“;

($l$) any payment, other than a payment ordered by a court or~made in settlement of a claim, made by or~on behalf of a former employer of a person on account of the early retirement of that person on grounds of ill-health or~disability”.
\end{quotation}
\end{enumerate}

\medskip

5.  In regulation 17(10) (calculation of weekly income), for~“17C” there shall be substituted “17B(6)” and~sub-paragraph~($c$)  shall be omitted.

\medskip

6.  For~regulation 21(1) (notional capital), there shall be substituted the following paragraph—
\begin{quotation}
“(1) A claimant shall be treated as possessing capital of which he has deprived himself for~the purpose of securing entitlement to~state pension credit or~increasing the amount of that benefit except to~the extent that the capital which he is treated as possessing is reduced in accordance with regulation 22 (diminishing notional capital rule).”.
\end{quotation}

\medskip

7.  In regulation 22 (diminishing notional capital rule), for~“22(1)”, wherever it occurs, there shall be substituted “21(1)”.

\medskip

8.  In Schedule~I—
\begin{enumerate}\item[]
($a$) in Part I (circumstances in which persons are treated as being or~not being severely disabled)—
\begin{enumerate}\item[]
(i) in paragraph~1(2)($b$), for~“37ZB(3)” there shall be substituted “72(3)”;

(ii) in paragraph~2(6)($a$), for~the word “partners” there shall be substituted the word “partner”;
\end{enumerate}

($b$) in Part II (applicable amount for~carers), in paragraph~4(4), for~the words “sub-paragraph~is” there shall be substituted the words “sub-paragraph~(3) is”;

($c$) in Part III (amount applicable for~former claimants of income support or~income-related jobseeker’s allowance)—
\begin{enumerate}\item[]
(i) in paragraph~6(7)($a$), after the words “Jobseeker’s Allowance”, there shall be inserted the word “Regulations”;

(ii) at the end of paragraph~6, there shall be added the following sub-paragraphs—
\begin{quotation}
\begin{sloppypar}
“(10) This sub-paragraph~applies where the relevant amount included an amount in respect of housing costs relating to a loan—
\end{sloppypar}
\begin{enumerate}\item[]
($a$) which is treated as a qualifying loan by virtue of regulation 3 of the Income Support (General) Amendment and~Transitional Regulations 1995\footnote{S.I.~1995/2287.} or~paragraph~18(2) of Schedule~2 to~the Jobseeker’s Allowance Regulations; or

($b$) the appropriate amount of which was determined in accordance with paragraph~7(6C) of Schedule~3 to~the Income Support Regulations as in force prior~to~10th April 1995 and~maintained in force by regulation 28(1) of the Income-related Benefits Schemes (Miscellaneous Amendments) Regulations 1995\footnote{S.I.~1995/516.}.
\end{enumerate}

(11) Where sub-paragraph~(10) applies, the transitional amount shall be calculated or, as the case may be, recalculated, on the relevant anniversary date determined in accordance with paragraph~7(4C) of Schedule~II (“the relevant anniversary date”) on the basis that the provisional amount on the relevant day included, in respect of housing costs, the amount calculated in accordance with paragraph~7(1) of Schedule~II as applying from the relevant anniversary date and~not the amount in respect of housing costs determined on the basis of the amount of the loan calculated in accordance with paragraph~7(4A) of that Schedule.

(12) The transitional amount as calculated in accordance with sub-paragraph~(11) shall only be applicable from the relevant anniversary date.”.
\end{quotation}
\end{enumerate}
\end{enumerate}

\medskip

9.  In Schedule~II (housing costs)—
\begin{enumerate}\item[]
\begin{sloppypar}
($a$) in paragraph~(iii)  of the definition of “disabled person” in paragraph~1(2)($a$)—
\end{sloppypar}
\begin{enumerate}\item[]
(i) at the end of ($aa$), there shall be inserted the word “and”;

(ii) at the end of ($bb$), there shall be inserted the word “or”;
\end{enumerate}

($b$) in paragraph~1(6)—
\begin{enumerate}\item[]
(i) at the end of head ($a$), there shall be inserted the word “or”;

(ii) head ($b$)  shall be omitted;

(iii) in head ($c$), the words “or~($b$)” shall be omitted;
\end{enumerate}

($c$) at the end of paragraph~2(8), there shall be added—
\begin{quotation}
“,

and~for~the purposes of this sub-paragraph, “sports award” means an award made by one of the Sports Councils named in section~23(2) of the National Lottery etc.\ Act 1993\footnote{1993 c.~39.} out of sums allocated to~it for~distribution under that section”;
\end{quotation}

($d$) in paragraph~5—
\begin{enumerate}\item[]
(i) in sub-paragraph~(5)($a$), for~the word “Intense”, in both places where that word occurs, there shall be substituted the word “Intensive”;

(ii) in sub-paragraph~(8), for~“(13)” there shall be substituted “(12)”;
\end{enumerate}

($e$) in paragraph~7—
\begin{enumerate}\item[]
(i) for~sub-paragraph~(2), there shall be substituted the following—
\begin{quotation}
“(2) For~the purposes of sub-paragraph~(1) and~subject to~sub-paragraphs (3) and~(4A), the amount of the qualifying loan—
\begin{enumerate}\item[]
($a$) except where paragraph~($b$)  applies, shall be determined on the date the housing costs are first met and~thereafter on the anniversary of that date;

($b$) where housing costs are being met in respect of a qualifying loan (“the existing loan”) and~housing costs are subsequently met in respect of one or~more further qualifying loans (“the new loan”), shall be the total amount of those loans determined on the date the housing costs were first met in respect of the new loan and~thereafter on the anniversary of the date housing costs were first met in respect of the existing loan.”;
\end{enumerate}
\end{quotation}

(ii) for~sub-paragraph~(3)(ii), there shall be substituted “recalculated on the relevant date specified in sub-paragraph~(4C)”;

(iii) after sub-paragraph~(4), there shall be inserted the following sub-paragraphs—
\begin{quotation}
“(4A) Where—
\begin{enumerate}\item[]
($a$) the last day on which either the claimant or~his partner were entitled to~income support or~to~an income-based jobseeker’s allowance was no more than twelve weeks before—
\begin{enumerate}\item[]
(i) except where head (ii)  applies, the first day of entitlement to~state pension credit; or

(ii) where the claim for~state pension credit was treated as made on a day earlier than the day on which it was actually made (“the actual date”), the day which would have been the first day of entitlement to~state pension credit had the claim been treated as made on the actual date; and
\end{enumerate}

($b$) sub-paragraph~(4B) applies,
\end{enumerate}
the amount of the qualifying loan shall be the amount last determined for~the purposes of the earlier entitlement and~recalculated on the relevant date specified in paragraph~(4C).

(4B) This sub-paragraph~applies—
\begin{enumerate}\item[]
($a$) where the earlier entitlement was to~income support, if their applicable amount included an amount determined in accordance with Schedule~3 to~the Income Support Regulations as applicable to~them in respect of a loan which qualifies under paragraph~15 or~16 of that Schedule; or

($b$) where the earlier entitlement was to~an income-based jobseeker’s allowance, if their applicable amount included an amount determined in accordance with Schedule~2 to~the Jobseeker’s Allowance Regulations as applicable to~them in respect of a loan which qualifies under paragraph~14 or~15 of that Schedule; and
\end{enumerate}
where the circumstances affecting the calculation of the qualifying loan remain unchanged since the last calculation of that loan and~in this paragraph, “qualifying loan” shall, where the context requires, be construed accordingly.

(4C) The recalculation shall take place—
\begin{enumerate}\item[]
($a$) in a case where sub-paragraph~(3) applies, on each subsequent anniversary of the date on which, for~the purposes of sub-paragraph~(2), housing costs were first met;

($b$) in a case where sub-paragraph~(4A) applies—
\begin{enumerate}\item[]
(i) where housing costs under the earlier entitlement were being met in respect of more than one qualifying loan and~the amounts of those loans were recalculated on different dates, on the first of those dates which falls during the award of state pension credit and~on each subsequent anniversary of that date;

(ii) in any other case, on each subsequent anniversary of the date on which housing costs were first met under the earlier entitlement;
\end{enumerate}

($c$) in the case of claims for state pension credit made between 6th October 2003 and 5th October 2004 and to which sub-paragraph~(4A) does not apply—
\begin{enumerate}\item[]
(i) where there are no housing costs to~be met as at the date of claim but housing costs are to~be met in respect of a qualifying loan taken out after the date of claim, on each subsequent anniversary of the date on which housing costs in respect of that loan were first met;

(ii) in any other case, on each subsequent anniversary of the date on which the decision was made to~award state pension credit.”;
\end{enumerate}
\end{enumerate}
\end{quotation}
\end{enumerate}

($f$) in paragraph~9(1)($a$), for~the words “is 5$.$34 per cent per annum” there shall be substituted the words “shall be the rate specified in paragraph~12(1)($a$)  of Schedule~3 to~the Income Support Regulations”;

($g$) in paragraph~14—
\begin{enumerate}\item[]
(i) in sub-paragraph~(2), for~“(2)($a$)” there shall be substituted “(1)($a$)”;

(ii) after sub-paragraph~(7)($c$), there shall be inserted the following paragraph—
\begin{quotation}
“($cc$) if he is a full-time student and~the claimant or~his partner has attained the age of 65;”.
\end{quotation}
\end{enumerate}
\end{enumerate}

\medskip

10.  In Schedule~III (special groups)—
\begin{enumerate}\item[]
($a$) in the new regulation 7(2) as inserted by paragraph~1(7), after the words “(polygamous marriages)”, there shall be inserted the word “applies,”;

($b$) in paragraph~2(6) for~the word “regulation” there shall be substituted the word “paragraph”.
\end{enumerate}

\medskip

11.  In Schedule~IV (amounts to~be disregarded in the calculation of income other than earnings)—
\begin{enumerate}\item[]
($a$) after paragraph~7, there shall be inserted the following paragraph—
\begin{quotation}
“7A.  £10 of any widowed mother’s allowance to~which the claimant is entitled under section~37 of the 1992 Act.”;
\end{quotation}

($b$) in paragraph~11(3)($b$), after “paragraph~7” there shall be inserted “or~7A”;

($c$) in paragraph~13, for~the words “the partner” there shall be substituted the words “the person”;

($d$) in paragraph~14—
\begin{enumerate}\item[]
(i) the word “final” shall be omitted;

(ii) for~the words “the partner” there shall be substituted the words “that person”.
\end{enumerate}
\end{enumerate}

\medskip

12.  In Schedule~V (income from capital)—
\begin{enumerate}\item[]
($a$) in paragraph~12, for~the word “interment” there shall be substituted the word “internment”;

($b$) in the definition of “training allowance” in paragraph~15(8), the words from “nor~does it include” to~the end of sub-paragraph~(8) shall be omitted;

($c$) in paragraph~20(2)($h$), for~the words “the Act” there shall be substituted the words “the 1992 Act”;

($d$) after paragraph~20, there shall be inserted the following paragraph—
\begin{quotation}
“20A.---(1)  Any payment of £5,000 or~more received by the claimant in full—
\begin{enumerate}\item[]
($a$) no more than 12 months before the day on which he claimed state pension credit; or

($b$) after the day on which he claimed state pension credit,
\end{enumerate}
which is made in order to~rectify, or~to~compensate for, an official error~as defined in regulation 1(3) of the Social Security and~Child Support (Decisions and~Appeals) Regulations 1999\footnote{S.I.~1999/991.} relating to~a benefit, either for~a period of 52 weeks from the date of receipt or, if the payment is received in its entirety during the award of state pension credit, for~the remainder of that award if that is a longer period.

(2) In this paragraph, “benefit” shall have the same meaning as for~the purposes of paragraph~20.”.
\end{quotation}
\end{enumerate}

\medskip

13.  In Schedule~VI (sums disregarded from claimant’s earnings)—
\begin{enumerate}\item[]
($a$) in paragraph~2, at the end of sub-paragraph~(2) there shall be added the following—
\begin{quotation}
“($d$) a member of any territorial or~reserve force prescribed in Part I of Schedule~6 to~the Social Security (Contributions) Regulations 2001\footnote{S.I.~2001/1004.}”;
\end{quotation}

($b$) after paragraph~2 there shall be inserted the following paragraph—
\begin{quotation}
“2A.  Where a person is engaged in one or~more of the employments specified in paragraph~2 but his earnings derived from those employments are less than £20 in any week and~he is also engaged in any other employment, so much of his earnings from that other employment as would not in aggregate with the amount of his earnings disregarded under paragraph~2 exceed £20.”;
\end{quotation}

($c$) in paragraph~4—
\begin{enumerate}\item[]
(i) at the end of sub-paragraph~(1)($a$)  there shall be added the following—
\begin{quotation}
“(vi) the disability element or~the severe disability element of working tax credit under Schedule~2 to~the Working Tax Credit (Entitlement and~Maximum Rate) Regulations 2002\footnote{S.I.~2002/2005.}; or”;
\end{quotation}

(ii) sub-paragraph~(5) shall be omitted;
\end{enumerate}

($d$) after paragraph~4, there shall be inserted the following paragraph—
\begin{quotation}
“4A.---(1)  £20 is the maximum amount which may be disregarded under any of paragraphs 1, 2, 3 or~4 notwithstanding that—
\begin{enumerate}\item[]
($a$) in the case of a claimant with no partner, he satisfies the requirements of more than one of those paragraphs or, in the case of paragraph~4, he satisfies the requirements of more than one of the sub-paragraphs of that paragraph; or

($b$) in the case of married or~unmarried couples, both partners satisfy one or~more of the requirements of paragraphs 2, 3 and~4.
\end{enumerate}

(2) Where, in a case to~which sub-paragraph~(1)($b$)  applies, the amount to~be disregarded in respect of one of the partners (“the first partner”) is less than £20, the amount to~be disregarded in respect of the other partner shall be so much of that other partner’s earnings as would not, in aggregate with the first partner’s earnings, exceed £20.”.
\end{quotation}

($e$) in paragraph~6, after the word “earnings”, there shall be inserted the words “, other than any amount referred to~in regulation 17(9)($b$),”;

($f$) after paragraph~6, there shall be added the following paragraph—
\begin{quotation}
“7.  Any banking charges or~commission payable in converting to~Sterling payments of earnings made in a currency other than Sterling.”.
\end{quotation}
\end{enumerate}

\part{Explanatory Note}

\renewcommand\parthead{— Explanatory Note}

\subsection*{(This note is not part of the Regulations)}

These Regulations are either made by virtue of, or~consequential upon, sections~1 to~17 of the State Pension Credit Act 2002 (c.~16). They are made before the end of the period of six months beginning with the coming into~force of those sections~of that Act and~are therefore exempt in accordance with section~173(5) of the Social Security Administration Act 1992 (c.~5) from the requirement in section~172(1) of that Act to~refer proposals to~make these Regulations to~the Social Security Advisory Committee and~are made without reference to~that Committee.

Regulation 2 and~the Schedule~amend the State Pension Credit Regulations 2002 (S.I.~2002/1792) (“the principal Regulations”). In particular they—
\begin{enumerate}\item[]
    amend the definition of “care home” and~insert a new definition of “voluntary organisation” (paragraph~1);

    provide that persons who are not habitually resident in the United Kingdom, the Channel Islands, the Isle of Man or~the Republic of Ireland~or~who are subject to~immigration control within the meaning of section~115(9) of the Immigration and~Asylum Act 1999 (c.~33) are not to~be treated as being members of the same household as the claimant (paragraph~2);

    add a new category of retirement pension income (paragraph~4($a$)  and~($c$));

    make a change which is consequential on the amendment in regulation 3(1) (paragraph~5);

    remove otiose rules in relation to~the treatment of capital derived from personal injury payments (paragraph~6);

    prescribe how the transitional amount is to~be calculated where a person’s applicable amount in respect of housing costs in income support and~jobseeker’s allowance include an amount for~housing costs calculated on a transitional basis (paragraph~8($c$)(ii));

    prescribe rules as to~when amounts of loans used to~calculate housing costs shall be recalculated (paragraph~9($e$));

    make changes to~the rules on disregarded income and~capital (paragraphs 11 to~13);

    correct minor~errors and~make certain other clarifications (paragraphs 3, 4($b$), 5, 7, 8($a$)  to~($c$)(i), 9($a$)  to~($d$), ($f$)  and~($g$)  and~10). 
\end{enumerate}

Regulation 3 makes further modifications to~the rules applying in state pension credit in relation to~the treatment of earnings of employed and~self-employed earners.

Regulation 4 makes changes to~certain of the amounts prescribed in the principal Regulations in respect of the standard minimum guarantee and~the prescribed additional amounts.

Regulation 5 amends the Social Security (Claims and~Payments) Regulations 1987 (S.I.~1987/1968) so as to~prescribe the cases where payment of mortgage interest payments may be made directly to~qualifying lenders in state pension credit cases and~makes consequential amendments (paragraphs (2) to~(4) and~(5)($a$)  to~($c$), ($e$)  and~($f$)) and~modify the rule as to~when housing costs may be paid directly to~such lenders where the claimant was previously in receipt of income support or~jobseeker’s allowance paragraph~5($d$)).

Regulation 6 amends the Social Security and~Child Support (Decision Making and~Appeals) Regulations 1999 (S.I.~1999/991) so as to~prescribe rules as to~when superseding decisions may take effect where non-dependant deductions for~housing costs purposes become applicable or~the amount of a non-dependant’s income increases.

Regulation 7(1) to~(3) respectively make consequential amendments to~the Social Security (Child Maintenance Bonus) Regulations 1996 (S.I.~1996/3195), to~the Social Security (Back to~Work Bonus) (No.~2) Regulations 1996 (S.I.~1996/2570) and~to~the Community Charges (Deductions from Income Support) (Scotland) Regulations 1989 (S.I.~1989/507).

Regulation 7(4) amends the transitional provisions in the State Pension Credit (Consequential, Transitional and~Miscellaneous Provisions) Regulations 2002 (S.I.~2002/3019) in relation to~those claiming income support immediately before the appointed day, who are patients on that day and~who immediately before becoming patients, were in residential accommodation.

These Regulations do not impose a charge on business. 

\end{document}