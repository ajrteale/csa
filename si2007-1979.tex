\documentclass[12pt,a4paper]{article}

\newcommand\regstitle{The Child Support (Miscellaneous Amendments) Regulations 2007}

\newcommand\regsnumber{2007/1979}

%\opt{newrules}{
\title{\regstitle}
%}

%\opt{2012rules}{
%\title{Child Maintenance and Other Payments Act 2008\\(2012 scheme version)}
%}

\author{S.I.\ 2007 No.\ 1979}

\date{Made
11th July 2007\\
%Laid before Parliament
%16th June 2006\\
Coming into force
1st August 2007
}

%\opt{oldrules}{\newcommand\versionyear{1993}}
%\opt{newrules}{\newcommand\versionyear{2003}}
%\opt{2012rules}{\newcommand\versionyear{2012}}

\usepackage{csa-regs}

\setlength\headheight{42.11603pt}

\hbadness=10000

\begin{document}

\maketitle

\noindent
The Secretary of State for Work and Pensions makes the following Regulations in exercise of the powers conferred by sections 14(1)\footnote{Section 14(1) of the Child Support Act 1991 was amended by section 12 of, and paragraph 11(7) of Schedule 3 to, the Child Support, Pensions and Social Security Act 2000 (c.\ 19). These amendments were commenced in relation to certain cases only on 3rd March 2003 (\emph{see} S.I.\ 2003/192). The power is exercised in these regulations both in its amended and unamended form.}, 32(1), 34(1), 35(7) and 52(4) of, and paragraphs 5(1), 5(2) and 10(1) of Schedule 1\footnote{Section 1 of, and Schedule 1 to, the Child Support, Pensions and Social Security Act 2000 (“the 2000 Act”) substituted a new Part I of Schedule 1 to the Child Support Act 1991. These provisions were commenced in relation to certain cases only on 3rd March 2003 (\emph{see} S.I.\ 2003/192). Paragraphs 5(1) and 5(2) of Schedule 1 to the Child Support Act 1991 relate only to cases for which the amendments in the 2000 Act have not been commenced. Regulations made under these provisions are subject to affirmative resolution procedure (\emph{see} footnote \S{} below). Paragraph 10(1) relates to cases for which the amendments in the 2000 Act have been commenced.} to, the Child Support Act 1991\footnote{1991 c.\ 48.}.

In accordance with section 52(2)\footnote{Section 25 of the Child Support, Pensions and Social Security Act 2000 substituted a new subsection (2) in section 52 of the Child Support Act 1991. This provision was commenced for regulation-making purposes on 10th November 2000 (\emph{see} S.I.\ 2000/2994) and in relation to certain cases only on 3rd March 2003 (\emph{see} S.I.\ 2003/192). In relation to those cases for which this provision has not been commenced, section 52(2) requires that any statutory instrument containing regulations made under Part I of Schedule 1 to the 1991 Act be subject to the affirmative resolution procedure.} of that Act, a draft of this instrument was laid before Parliament and approved by a resolution of each House of Parliament. 

\vfill

{\sloppy

\tableofcontents

}

\bigskip

\setcounter{secnumdepth}{-2}

\subsection[1. Citation and commencement]{Citation and commencement}

1.  These Regulations may be cited as the Child Support (Miscellaneous Amendments) Regulations 2007 and come into force on 1st August 2007.

\subsection[2. Amendment of the Child Support (Collection and Enforcement) Regulations 1992]{Amendment of the Child Support (Collection and Enforcement) Regulations 1992}

2.---(1)  Amend the Child Support (Collection and Enforcement) Regulations 1992\footnote{S.I.\ 1992/1989. Relevant amendments were made by S.I.\ 2001/162 and S.I.\ 2006/1520.} in accordance with this regulation.

(2) In regulation 22 (appeals against deduction from earnings orders)—
\begin{enumerate}\item[]
($a$) in paragraph (1) for “, having jurisdiction in the area in which he resides”, substitute “of the sheriffdom in which he resides”;

($b$) at the beginning of paragraph (2)($b$)  insert “where the liable person is resident in the United Kingdom,”;

($c$) after paragraph (2)($b$)  insert—
\begin{quotation}
“($c$) where the liable person is not resident in the United Kingdom, be made within 56 days of the date on which the matter appealed against arose.”.
\end{quotation}
\end{enumerate}

(3) In regulation 27 (notice of intention to apply for a liability order)—
\begin{enumerate}\item[]
($a$) at the beginning of paragraph (1) insert “Subject to paragraph (1A),”;

($b$) after paragraph (1) insert—
\begin{quotation}
“(1A) Where the liable person is not resident in the United Kingdom, the Secretary of State shall give the liable person at least 28 days notice of his intention to apply for a liability order under section 33(2) of the Act.”.
\end{quotation}
\end{enumerate}

(4) In paragraph (1) of regulation 28 (application for a liability order), omit “having jurisdiction in the area in which the liable person resides”;

(5) In paragraph (1) of regulation 31 (appeals in connection with distress), omit “having jurisdiction in the area in which he resides”;

(6) In paragraph (1) of regulations 33 (application for warrant of commitment) and 35 (disqualification from driving order), omit “having jurisdiction for the area in which the liable person resides”.

\subsection[3. Amendment of the Child Support (Information, Evidence and Disclosure) Regulations 1992]{Amendment of the Child Support (Information, Evidence and Disclosure) Regulations 1992}

3.---(1)  Amend the Child Support (Information, Evidence and Disclosure) Regulations 1992\footnote{S.I.\ 1992/1812. Relevant amendments were made by S.I.\ 1995/1045, S.I.\ 1995/3261, S.I.\ 1996/1945, S.I.\ 1998/58, S.I.\ 1999/977, S.I.\ 1999/1510, S.I.\ 2001/161, S.I.\ 2003/3206, S.I.\ 2005/2877 and S.I.\ 2006/1520.} in accordance with this regulation.

(2) In regulation 1(2) (citation, commencement and interpretation) at end insert—
\begin{quotation}
““taxable profits” means profits calculated in accordance with Part II of the Income Tax (Trading and Other Income) Act 2005\footnote{2005 c.\ 5. This brings the definition of “taxable profits” for child support purposes into line with the definition for income tax purposes. Capital allowances will be deducted from, and balancing charges applied to, gross profits from self-employment to determine a self-employed person’s earnings.}.”
\end{quotation}

(3) In regulation 3(2)($h$)  (purposes for which information or evidence may be required) omit—
\begin{enumerate}\item[]
($a$) “total”; and

($b$) “as submitted to, or as issued to him by, the Inland Revenue,”.
\end{enumerate}

\subsection[4. Amendment of the Child Support (Maintenance Assessments and Special Cases) Regulations 1992]{\sloppy Amendment of the Child Support (Maintenance Assessments and Special Cases) Regulations 1992}

4.---(1)  Amend Schedule 1 to the Child Support (Maintenance Assessments and Special Cases) Regulations 1992\footnote{S.I.\ 1992/1815. Relevant amendments were made by S.I.\ 1993/913, S.I.\ 1995/1045, S.I.\ 1996/3196, S.I.\ 1998/58, S.I.\ 1999/977, S.I.\ 1999/1510 and S.I.\ 2005/785. S.I.\ 1992/1815 was revoked in relation to certain cases by S.I.\ 2001/155.} (calculation of N and M) in accordance with this regulation.

(2) In paragraph 2A—
\begin{enumerate}\item[]
($a$) in sub-paragraph (1) omit “2B,”;

($b$) in sub-paragraph (2) omit—
\begin{enumerate}\item[]
(i) “total”; and

(ii) “as submitted to the Inland Revenue”;
\end{enumerate}

($c$) in sub-paragraph (3) omit “total”; and

($d$) at end insert—
\begin{quotation}
“(5) For the purposes of this paragraph, “taxable profits” means profits calculated in accordance with Part II of the Income Tax (Trading and Other Income) Act 2005.

(6) A self-employed earner who is a person with care or an absent parent shall provide to the Secretary of State on demand a copy of—
\begin{enumerate}\item[]
($a$) any tax calculation notice issued to him by Her Majesty’s Revenue and Customs; and

($b$) any revised notice issued to him by Her Majesty’s Revenue and Customs.”.
\end{enumerate}
\end{quotation}
\end{enumerate}

(3) Omit paragraph 2B.

(4) For paragraph 2C substitute—
\begin{quotation}
“2C.  Where the Secretary of State accepts that it is not reasonably practicable for a self-employed earner to provide any of the information specified in paragraph 2A(6), “earnings” in relation to that earner shall be calculated in accordance with paragraph 3.”.
\end{quotation}

(5) In sub-paragraph (4)($b$)  of paragraph 3 omit sub-paragraphs (i), (iii), (iv), (v) and (vii).

(6) In paragraph 5A—
\begin{enumerate}\item[]
($a$) omit “total” in each place where it appears; and

($b$) omit sub-paragraph (3).
\end{enumerate}

\subsection[5. Amendment of the Child Support (Maintenance Calculations and Special Cases) Regulations 2000]{Amendment of the Child Support (Maintenance Calculations and Special Cases) Regulations 2000
}

5.---(1)  Amend Part III of the Schedule to the Child Support (Maintenance Calculations and Special Cases) Regulations 2000\footnote{S.I.\ 2001/155. Relevant amendments were made by S.I.\ 2002/1204, S.I.\ 2003/328 and S.I.\ 2005/785.} in accordance with this regulation.

(2) In paragraph 7—
\begin{enumerate}\item[]
($a$) for the title “Figures submitted to the Inland Revenue”, substitute “Net weekly income of non-resident parent as a self-employed earner”;

($b$) for sub-paragraph (1) substitute—
\begin{quotation}
“(1) Subject to sub-paragraph (6) and to paragraph 8, the net weekly income of the non-resident parent as a self-employed earner shall be his gross earnings less the deductions to which sub-paragraph (3) applies.”;
\end{quotation}

($c$) after sub-paragraph (1) insert—
\begin{quotation}
“(1A) In this paragraph and paragraph 8 a person’s “gross earnings” are his taxable profits calculated in accordance with Part II of the Income Tax (Trading and Other Income) Act 2005.”;
\end{quotation}

($d$) for sub-paragraph (2) substitute—
\begin{quotation}
“(2) The non-resident parent shall provide to the Secretary of State on demand a copy of—
\begin{enumerate}\item[]
($a$) any tax calculation notice issued to him by Her Majesty’s Revenue and Customs; and

($b$) any revised tax calculation notice issued to him by Her Majesty’s Revenue and Customs.”;
\end{enumerate}
\end{quotation}

and

($e$) omit sub-paragraph (7).
\end{enumerate}

(3) In paragraph 8—
\begin{enumerate}\item[]
($a$) in sub-paragraph (1)—
\begin{enumerate}\item[]
(i) in paragraph ($b$)  omit “ or;” and

(ii) omit paragraph ($c$);
\end{enumerate}

($b$) in sub-paragraph (3)($b$)  omit sub-paragraphs (i), (iii), (iv), (v) and~(vii).
\end{enumerate}

\bigskip

Signed 
by authority of the 
Secretary of State for Work and Pensions.

{\raggedleft
\emph{Bill McKenzie}\\*Parliamentary Under-Secretary of State,\\*Department for Work and Pensions

}

11th July 2007

\small

\part{Explanatory Note}

\renewcommand\parthead{— Explanatory Note}

\subsection*{(This note is not part of the Regulations)}

These Regulations make amendments to:
\begin{enumerate}\item[]
1.  the Child Support (Collection and Enforcement) Regulations 1992 (S.I.\ 1992/1989);

2.  the Child Support (Information, Evidence and Disclosure) Regulations 1992 (S.I.\ 1992/1812);

3.  the Child Support (Maintenance Assessments and Special Cases) Regulations 1992 (S.I.\ 1992/1815); and

4.  the Child Support (Maintenance Calculations and Special Cases) Regulations 2000 (S.I.\ 2001/155).
\end{enumerate}

The powers exercised to make these Regulations are those contained in the Child Support Act 1991 (“the 1991 Act”). Some of those powers are conferred by provisions of the 1991 Act prior to the amendments made to that Act by the Child Support, Pensions and Social Security Act 2000 (“the 2000 Act”), which amendments are not fully in force, and relate to the child support scheme which was in force prior to 3rd March 2003 and which remains in force for the purposes of certain cases (“old scheme cases”). This includes powers contained in Part I of Schedule 1 to the 1991 Act which, by virtue of section 52(2) of the 1991 Act for old scheme casese are subject to affirmative resolution procedure. Other powers are conferred by provisions of the 1991 Act as amended by the 2000 Act, which relate to the child support scheme provided for by those amendments and which came into force for the purposes of specified categories of cases on 3rd March 2003 (\emph{see} the Child Support, Pensions and Social Security Act 2000 (Commencement No 12) Order 2003 S.I.\ 2003/192) (“new scheme cases”). Section 14(1) is exercised in these Regulations in both its unamended form for old scheme cases (in regulations 3 and 4) and as amended by the 2000 Act for new scheme cases (in regulations 3 and 5). The amendments to regulation 3 apply equally to old and new scheme cases.

Regulation 2 amends the Child Support (Collection and Enforcement) Regulations 1992 by removing references to residence as the basis for jurisdiction in relation to liability orders and deduction from earnings orders. This reflects changes to courts’ legislation. It also ensures that rights of appeal are not limited to those who are UK resident and extends periods for appeal and periods of notice where the liable person is resident outside the UK.

Regulation 3 amends the definition of taxable profits in the Child Support (Information, Evidence and Disclosure) Regulations 1992 to bring it into line with the amendments in regulations 4 and 5.

Regulation 4 amends the Child Support (Maintenance Assessment and Special Cases) Regulations 1992 which apply to old scheme cases. It provides a new definition of taxable profits on which the maintenance assessment of a self-employed earner will be based, bringing it into line with the definition for income tax purposes. A self-employed earner’s taxable profits will be calculated for child support purposes as they would be for tax purposes – meaning that capital allowances will be deducted from, and balancing charges applied to, gross profits in line with tax legislation. The figure will in general be derived from information supplied by Her Majesty’s Revenue and Customs. Where tax information is not available, taxable profits are calculated on a different basis as set out in the amended Regulations. Regulation 5 makes similar amendments to the Child Support (Maintenance Calculation and Special Cases) Regulations 2000 for new scheme cases.

A full regulatory impact assessment has not been produced for this instrument as it has no effect on the costs of business, charities or voluntary bodies. 

\end{document}