\documentclass[12pt,a4paper]{article}

\newcommand\regstitle{The Child Support (Decisions and Appeals) (Amendment) Regulations 2000}

\newcommand\regsnumber{2000/3185}

%\opt{newrules}{
\title{\regstitle}
%}

%\opt{2012rules}{
%\title{Child Maintenance and Other Payments Act 2008\\(2012 scheme version)}
%}

\author{S.I. 2000 No. 3185}

\date{Made 4th December 2000\\Laid before Parliament 6th December 2000\\Coming into force as provided in regulation 1(1)}

%\opt{oldrules}{\newcommand\versionyear{1993}}
%\opt{newrules}{\newcommand\versionyear{2003}}
%\opt{2012rules}{\newcommand\versionyear{2012}}

\usepackage{csa-regs}

\setlength\headheight{27.57402pt}

\begin{document}

\maketitle

\noindent
The Secretary of State for Social Security, in exercise of the powers conferred on him by sections 16(1), (4) and (6), 17(3) and (5), 28G(2), 51 and 52(1) and (4) of the Child Support Act 1991\footnote{\frenchspacing 1991 c. 48. Section 28G was substituted by section 7 of the Child Support, Pensions and Social Security Act 2000 (c. 19). Sections 16, 17 and 51 were amended by sections 8 and 9 of, and paragraph 11(19) of Schedule 3 to, that Act.} and of all other powers enabling him in that behalf, after consultation with the Council on Tribunals in accordance with section 8 of the Tribunals and Inquiries Act 1992\footnote{\frenchspacing 1992 c. 53.}, hereby makes the following Regulations: 

{\sloppy

\tableofcontents

}

\bigskip

\setcounter{secnumdepth}{-2}

\subsection[1. Citation, commencement and interpretation]{Citation, commencement and interpretation}

1.---(1)  These Regulations may be cited as the Child Support (Decisions and Appeals) (Amendment) Regulations 2000 and, subject to paragraph (2), shall come into force in relation to a particular case on the date on which sections 16, 17 and 20 of the Child Support Act 1991 as amended by the Child Support, Pensions and Social Security Act 2000\footnote{\frenchspacing 2000 c. 19.} come into force in relation to that type of case.

(2) For the purposes of any revision, supersession or appeal in relation to a decision which is made as provided in regulation 3 of the Child Support (Transitional Provisions) Regulations 2000\footnote{\frenchspacing S.I. 2000/3186.} these Regulations shall come into force on the day on which section 29 of the Child Support, Pensions and Social Security Act 2000 comes fully into force.

(3) In these Regulations “the principal Regulations” means the Social Security and Child Support (Decisions and Appeals) Regulations 1999\footnote{\frenchspacing S.I. 1999/991. The relevant amending instrument is S.I. 2000/1596.}.

(4) In these Regulations any reference to a numbered regulation is to the regulation in the principal Regulations bearing that number and any reference to a numbered Part is to the Part of the principal Regulations bearing that number.

\subsection[2. Amendment of regulation 1]{Amendment of regulation 1}

2.  In regulation 1(3) (citation, commencement and interpretation)—
\begin{enumerate}\item[]
($a$) after the definition of “the 1997 Act” there shall be inserted the following definition—
\begin{quotation}
““the Arrears, Interest and Adjustment of Maintenance Assessments Regulations” means the Child Support (Arrears, Interest and Adjustment of Maintenance Assessments) Regulations 1992\footnote{\frenchspacing S.I. 1992/1816. The relevant amending instruments are S.I. 1995/1045 and S.I. 1999/1501.};”;
\end{quotation}

($b$) after the definition of “legally qualified panel member” there shall be inserted the following definitions—
\begin{quotation}
““the Maintenance Calculation Procedure Regulations” means the Child Support (Maintenance Calculation Procedure) Regulations 2000\footnote{\frenchspacing S.I. 2001/157.};

“the Maintenance Calculations and Special Cases Regulations” means the Child Support (Maintenance Calculations and Special Cases) Regulations 2000\footnote{\frenchspacing S.I. 2001/155.};”;
\end{quotation}

($c$) in the definition of “party to the proceedings” in paragraph ($b$)  the words “as extended by paragraph 3 of Schedule 4C to that Act” shall be omitted;

($d$) in the definition of “referral” for the words “departure direction” there shall be substituted the word “variation”;

($e$) after the definition of “referral” there shall be inserted the following definition—
\begin{quotation}
    “except where otherwise provided “relevant person” means—
\begin{enumerate}\item[]
    ($a$) 
    a person with care;

    ($b$) 
    a non-resident parent;

    ($c$) 
    a parent who is treated as a non-resident parent under regulation 8 of the Maintenance Calculations and Special Cases Regulations;

    ($d$) 
    a child, where the application for a maintenance calculation is made by that child under section 7 of the Child Support Act,
\end{enumerate}
    in respect of whom a maintenance calculation has been applied for, or has been treated as applied for under section 6(3) of that Act, or is or has been in force;”; and 
\end{quotation}

($f$) after the definition of “the Transfer Act”\footnote{\frenchspacing S.I. 1999/1670. This definition was added to the principal Regulations by regulation 2(2).} there shall be added the following definition—
\begin{quotation}
““the Variations Regulations” means the Child Support (Variations) Regulations 2000\footnote{\frenchspacing S.I. 2001/156.};”.
\end{quotation}
\end{enumerate}

\subsection[3. Amendment of regulation 2]{Amendment of regulation 2}

3.  In regulation 2 (service of notices or documents) after the words “of the Act” there shall be inserted the words “, of the Child Support Act”.

\subsection[4. Amendment of the heading to Part II]{Amendment of the heading to Part II}

4.  In the heading to Part II after the words “SOCIAL SECURITY” there shall be added the words “AND CHILD SUPPORT”.

\subsection[5. Insertion of regulation 3A]{Insertion of regulation 3A}

5.  After regulation 3 (revision of decisions) there shall be inserted the following regulation—
\begin{quotation}
\subsection*{“Revision of child support decisions}

3A.---(1)  Subject to paragraph (2), any decision as defined in paragraph (3) may be revised under section 16 of the Child Support Act by the Secretary of State—
\begin{enumerate}\item[]
($a$) if he receives an application for the revision of a decision either—
\begin{enumerate}\item[]
(i) under section 16; or

(ii) by way of an application under section 28G,
\end{enumerate}
of the Child Support Act, within one month of the date of notification of the decision or within such longer time as may be allowed under regulation 4;

($b$) if—
\begin{enumerate}\item[]
(i) he notifies the person who applied for a decision to be revised within the period specified in sub-paragraph ($a$), that the application is unsuccessful because the Secretary of State is not in possession of all of the information or evidence needed to make a decision; and

(ii) that person reapplies for the decision to be revised within one month of the notification described in head (i)  above, or such longer period as the Secretary of State is satisfied is reasonable in the circumstances of the case, and provides in that application sufficient information or evidence to enable a decision to be made;
\end{enumerate}

($c$) if he is satisfied that the decision was erroneous due to a misrepresentation of, or failure to disclose, a material fact and that the decision was more advantageous to the person who misrepresented or failed to disclose that fact than it would have been but for that error;

($d$) if he commences action leading to the revision of the decision within one month of the date of notification of the decision; or

($e$) if the decision arose from an official error.
\end{enumerate}

(2) Paragraph (1)($a$)  to ($d$)  shall not apply in respect of a change of circumstances which—
\begin{enumerate}\item[]
($a$) occurred since the date on which the decision had effect; or

($b$) according to information or evidence which the Secretary of State has, is expected to occur.
\end{enumerate}

(3) Subject to paragraph (7), in paragraphs (1) and (2) and in regulation 4(3) “decision” means a decision of the Secretary of State under sections 11, 12 or 46 of the Child Support Act, or a determination of an appeal tribunal on a referral under section 28D(1)($b$)  of that Act, or any supersession of a decision under section 17 of that Act.

(4) A decision made under section 12(2) of the Child Support Act may be revised at any time before it is replaced by a decision under section 11 of that Act.

(5) Where the Secretary of State revises a decision made under section 12(1) of the Child Support Act in accordance with section 16(1B) of that Act, that decision may be revised under section 16 of that Act at any time.

(6) Section 16 of the Child Support Act shall apply in relation to any decision of the Secretary of State—
\begin{enumerate}\item[]
($a$) under section 41A or 47 of the Child Support Act; or

($b$) that an adjustment shall cease or with respect to the adjustment of amounts payable under maintenance calculations for the purpose of taking account of overpayments of child support maintenance and voluntary payments,
\end{enumerate}
as it applies in relation to any decision of the Secretary of State under sections 11, 12, 17 or 46 of that Act, or the determination of an appeal tribunal on a referral under section 28D(1)($b$)  of that Act.

(7) In paragraph (6)($b$)  and in regulations 6A(9), 6B(4)($d$)  and 30A “voluntary payments” means the same as in the definition in section 28J of the Child Support Act and Regulations made under that section.”.
\end{quotation}

\subsection[6. Amendment of regulation 4]{Amendment of regulation 4}

6.  In regulation 4 (late application for a revision)—
\begin{enumerate}\item[]
($a$) in paragraph (1) after the words “in regulation 3(1) or (3)” there shall be inserted the words “or 3A(1)($a$)”;

($b$) in paragraph (2) after the words “shall be made by” there shall be inserted the words “the relevant person,”;

($c$) in paragraph (4)($c$)  after the words “in regulation 3” there shall be inserted the words “or 3A”;

($d$) in paragraph (5) after the words “in regulation 3(1) and (3)” there shall be added the words “and regulation 3A(1)($a$)”; and

($e$) in paragraph (6)($b$)  after the words “a Commissioner” there shall be inserted the words “, a Child Support Commissioner”.
\end{enumerate}

\subsection[7. Insertion of regulation 5A]{Insertion of regulation 5A}

7.  After regulation 5 (date from which a decision revised under section 9 takes effect) there shall be inserted the following regulation—
\begin{quotation}
\subsection*{“Date from which a decision revised under section 16 of the Child Support Act takes effect}

5A.---(1)  Where the date from which a decision took effect is found to be erroneous on a revision under section 16 of the Child Support Act, the revision shall take effect from the date on which the decision revised would have taken effect had the error not been made.

(2) Where the Secretary of State considers it appropriate to revise a decision under section 12(1) of the Child Support Act as if he were revising a decision under section 11 of that Act, the revision shall take effect from the first day of the maintenance period in which the information required to make a maintenance calculation was provided, except where—
\begin{enumerate}\item[]
($a$) the non-resident parent satisfies the Secretary of State—
\begin{enumerate}\item[]
(i) that he used his best endeavours to obtain the information required by the Secretary of State; and

(ii) the failure to provide the information was not his fault; or
\end{enumerate}

($b$) the decision which is treated as being made under section 11 of the Child Support Act is at a higher rate than the rate of liability which had been imposed by the decision made under section 12(1) of that Act.”.
\end{enumerate}
\end{quotation}

\subsection[8. Insertion of regulations 6A and 6B]{Insertion of regulations 6A and 6B}

8.  After regulation 6 (supersession of decisions) there shall be inserted the following regulations—
\begin{quotation}
\subsection*{“Supersession of child support decisions}

6A.---(1)  Subject to paragraphs (7) and (8), the cases and circumstances in which a decision (“a superseding decision”) may be made by the Secretary of State for the purposes of section 17 of the Child Support Act are set out in paragraphs (2) to (6).

(2) A decision may be superseded by a decision made by the Secretary of State acting on his own initiative where—
\begin{enumerate}\item[]
($a$) there has been a relevant change of circumstances since the decision had effect; or

($b$) the decision was made in ignorance of, or was based upon a mistake as to, some material fact.
\end{enumerate}

(3) Subject to regulation 6B, a decision may be superseded by a decision made by the Secretary of State where—
\begin{enumerate}\item[]
($a$) an application is made on the basis that—
\begin{enumerate}\item[]
(i) there has been a change of circumstances since the date from which the decision had effect; or

(ii) it is expected that a change of circumstances will occur; and
\end{enumerate}

($b$) the Secretary of State is satisfied that the change of circumstances is or would be relevant.
\end{enumerate}

(4) A decision may be superseded by a decision made by the Secretary of State where—
\begin{enumerate}\item[]
($a$) an application is made on the basis that the decision was made in ignorance of, or was based upon a mistake as to, a fact; and

($b$) the Secretary of State is satisfied that the fact is or would be material.
\end{enumerate}

(5) A decision, other than a decision made on appeal, may be superseded by a decision made by the Secretary of State—
\begin{enumerate}\item[]
($a$) acting on his own initiative, where he is satisfied that the decision was erroneous in point of law; or

($b$) where an application is made on the basis that the decision was erroneous in point of law.
\end{enumerate}

(6) A decision may be superseded by a decision made by the Secretary of State where he receives an application for the supersession of a decision by way of an application made under section 28G of the Child Support Act.

(7) The cases and circumstances in which a decision may be superseded shall not include any case or circumstance in which a decision may be revised.

(8) Paragraphs (2) to (6) shall not apply in respect of a decision to refuse an application for a maintenance calculation.

(9) For the purposes of section 17 of the Child Support Act, paragraphs (2) to (6) shall apply in relation to any decision of the Secretary of State that an adjustment shall cease or with respect to the adjustment of amounts payable under a maintenance calculation for the purpose of taking account of overpayments of child support maintenance and voluntary payments, whether as originally made or as revised under section 16 of that Act.

\subsection*{Circumstances in which a child support decision may not be superseded}

6B.---(1)  Except as provided in paragraph (4), and subject to paragraph (3), a decision of the Secretary of State, appeal tribunal or Child Support Commissioner, on an application made under regulation 6A(3), shall not be superseded where the difference between—
\begin{enumerate}\item[]
($a$) the non-resident parent’s net income figure fixed for the purposes of the maintenance calculation in force in accordance with Part I of Schedule 1 to the Child Support Act; and

($b$) the non-resident parent’s net income figure which would be fixed in accordance with a superseding decision,
\end{enumerate}
is less than 5\% of the figure in sub-paragraph ($a$).

(2) In paragraph (1) “superseding decision” means a decision which would supersede the decision subject to the application made under regulation 6A(3) but for the application of this regulation.

(3) Where the application for a supersession is made on more than one ground this regulation shall only apply to the ground relating to the net income of the non-resident parent.

(4) This regulation shall not apply to a decision under regulation 6A(3) where—
\begin{enumerate}\item[]
($a$) the superseding decision is made in consequence of the determination of an application made under section 28G of the Child Support Act;

($b$) the superseding decision affects a variation ground in a decision made under section 11 or 17 of the Child Support Act, whether as originally made or as revised under section 16 of that Act;

($c$) the decision being superseded was made under section 12(2) of the Child Support Act, or was a decision under section 17 of that Act superseding an interim maintenance decision, whether as originally made or as revised under section 16 of that Act;

($d$) the decision being superseded was a decision that an adjustment shall cease or with respect to the adjustment of amounts payable under maintenance calculations for the purpose of taking account of overpayments of child support maintenance and voluntary payments or was a decision under section 17 of the Child Support Act superseding that decision, whether as originally made or as revised under section 16 of that Act; or

($e$) the superseding decision takes effect from the dates prescribed in regulation 7B(1) to (3), (19) or (20).”.
\end{enumerate}
\end{quotation}

\subsection[9. Insertion of regulations 7B and 7C]{Insertion of regulations 7B and 7C}

9.  After regulation 7A\footnote{\frenchspacing Regulation 7A was inserted by S.I. 1999/1623 and the relevant amending instrument is S.I. 2000/1956.} (definitions for the purposes of regulations 3(5)($c$), 6(2)($g$), 7(2)($c$)  and (5) and ancillary provisions) there shall be inserted the following regulations—
\begin{quotation}
\subsection*{“Date from which a decision superseded under section 17 of the Child Support Act takes effect}

7B.---(1)  Subject to paragraphs (17) to (22), where a decision is superseded by a decision made by the Secretary of State in a case to which regulation 6A(2)($a$)  applies on the basis of information or evidence which was also the basis of a decision made under section 8, 9 or 10 of the Act, the decision under section 17 of the Child Support Act shall take effect from the first day of the maintenance period in which that information or evidence was first brought to the attention of an officer exercising the functions of the Secretary of State under the Child Support Act (“the officer”).

(2) Where a decision is superseded by a decision made by the Secretary of State in a case to which regulation 6A(3)($a$)  applies and the relevant circumstance is that the non-resident parent or his partner has notified the officer that he or his partner had made a claim for a relevant benefit and, where the relevant benefit is payable, that the officer was notified within one month of notification of the award, the decision shall take effect from the first day of the maintenance period in which—
\begin{enumerate}\item[]
($a$) the non-resident parent or his partner notified the officer that he or his partner had made a claim for a relevant benefit, where entitlement to that benefit commences on or before the date of notification; or

($b$) entitlement to the relevant benefit commences, where that entitlement commenced after the date of notification.
\end{enumerate}

\pagebreak[3]

(3) Where a decision is superseded by a decision made by the Secretary of State in a case to which regulation 6A(4) applies and the material fact is that the non-resident parent or his partner has notified the officer that he or his partner had made a claim for a relevant benefit before the Secretary of State notified him of an application for a maintenance calculation in accordance with regulation 5 of the Maintenance Calculation Procedure Regulations (notice of an application for a maintenance calculation) and, where the relevant benefit is payable, that the officer was notified within one month of notification of the award, the decision shall take effect from the first day of the maintenance period in which—
\begin{enumerate}\item[]
($a$) the non-resident parent or his partner notified the officer that he or his partner had made a claim for a relevant benefit, where entitlement to that benefit commences on or before the date of notification; or

($b$) entitlement to the relevant benefit commences, where that entitlement commenced after the date of notification.
\end{enumerate}

(4) Subject to paragraphs (17) to (22), where the superseding decision is made in a case to which regulation 6A(3)($a$)(i)  applies and that decision supersedes one which has been made under section 12(2) of the Child Support Act, the decision shall take effect from the first day of the maintenance period in which the change of circumstances occurred.

(5) Where the superseding decision is made in a case to which regulation 6A(3)($a$)(ii)  applies, the decision shall take effect from the first day of the maintenance period in which the change of circumstances is expected to occur.

(6) Where the superseding decision is made in a case to which regulation 6A(6) applies and the relevant circumstance is that a ground for a variation is expected to occur, the decision shall take effect from the first day of the maintenance period in which the ground for the variation is expected to occur.

(7) Except in a case to which paragraph (1) applies, where the superseding decision is made in a case to which regulation 7C applies, that decision shall take effect from the first day of the maintenance period which includes the date which is 28 days after the date on which the Secretary of State gave notice to the relevant persons under that regulation.

(8) For the purposes of paragraph (7)—
\begin{enumerate}\item[]
($a$) where the relevant persons are notified on different dates, the period of 28 days shall be counted from the date of the latest notification;

($b$) notification includes oral and written notification;

($c$) where a person is notified in more than one way, the date on which he is notified is the date on which he was first given notification; and

($d$) the date of written notification is the date on which it was given or sent to the person.
\end{enumerate}

(9) Where—
\begin{enumerate}\item[]
($a$) a decision made by an appeal tribunal or by a Child Support Commissioner is superseded on the ground that it was erroneous due to a misrepresentation of, or that there was a failure to disclose, a material fact; and

($b$) the Secretary of State is satisfied that the decision was more advantageous to the person who misrepresented or failed to disclose that fact than it would otherwise have been but for that error,
\end{enumerate}
the superseding decision shall take effect from the date on which the decision of the appeal tribunal or, as the case may be, the Child Support Commissioner took, or was to take, effect.

(10) Any decision made under section 17 of the Child Support Act in consequence of a determination which is a relevant determination for the purposes of section 28ZC of that Act\footnote{\frenchspacing Section 28ZC was inserted by section 44 of the Social Security Act 1998 (c. 14).} shall take effect from the date of the relevant determination.

(11) Where a decision with respect to a reduced benefit decision is superseded because the decision ceases to be in force in accordance with regulation 16($a$)  of the Maintenance Calculation Procedure Regulations (termination of a reduced benefit decision), the superseding decision shall have effect—
\begin{enumerate}\item[]
($a$) where the decision is in operation immediately before it ceases to be in force, from the last day of the benefit week during the course of which the parent concerned falls within the provisions of section 46(1) of the Child Support Act; or

($b$) where the decision is suspended immediately before it ceases to be in force, from the date on which the parent concerned falls within the provisions of section 46(1) of that Act.
\end{enumerate}

(12) Where a decision with respect to a reduced benefit decision is superseded because the decision ceases to be in force in accordance with regulation 16($b$)  of the Maintenance Calculation Procedure Regulations, the superseding decision shall have effect—
\begin{enumerate}\item[]
($a$) where the decision is in operation immediately before it ceases to be in force, from the last day of the benefit week during the course of which the parent concerned complied with the obligations imposed by section 46(6)($b$)  of the Child Support Act; or

($b$) where the decision is suspended immediately before it ceases to be in force, from the date on which the parent concerned complied with the obligations imposed by section 46(6)($b$)  of the Child Support Act.
\end{enumerate}

(13) Where a decision with respect to a reduced benefit decision is superseded because the decision ceases to be in force in accordance with regulation 16($c$)  of the Maintenance Calculation Procedure Regulations, the superseding decision shall have effect from the last day of the benefit week in which entitlement to benefit ceased.

(14) Where a decision with respect to a reduced benefit decision is superseded because the decision ceases to be in force in accordance with regulation 16($d$)  of the Maintenance Calculation Procedure Regulations, the superseding decision shall have effect—
\begin{enumerate}\item[]
($a$) where the decision is in operation immediately before it ceases to be in force, from the last day of the benefit week during the course of which the Secretary of State is supplied with information that enables him to make the calculation; or

($b$) where the decision is suspended immediately before it ceases to be in force, from the date on which the Secretary of State is supplied with information that enables him to make the calculation.
\end{enumerate}

(15) Where a decision with respect to a reduced benefit decision is superseded because the decision ceases to be in force in accordance with regulation 17(1) of the Maintenance Calculation Procedure Regulations (reduced benefit decisions where there is an additional qualifying child), the superseding decision shall have effect from—
\begin{enumerate}\item[]
($a$) the last day of the benefit week preceding the benefit week which includes, in accordance with the provisions of regulation 11(3) of the Maintenance Calculation Procedure Regulations (amount of and period of reduction of relevant benefit under a reduced benefit decision), the first day on which the further decision comes into operation; or

($b$) the first day on which the further decision would come into operation but for the provisions of regulation 14 of the Maintenance Calculation Procedure Regulations (suspension of a reduced benefit decision when a modified applicable amount is payable (income support)) or 15 (suspension of a reduced benefit decision when a modified applicable amount is payable (income-based jobseeker’s allowance)) of those Regulations.
\end{enumerate}

(16) Where a decision with respect to a reduced benefit decision is superseded because the decision ceases to be in force in accordance with regulation 18(2) of the Maintenance Calculation Procedure Regulations (suspension and termination of a reduced benefit decision where the sole qualifying child ceases to be a child or where the parent concerned ceases to be a person with care), the superseding decision shall have effect from the last day of the benefit week which includes the day on which the child ceases to be a child within the meaning of section 55 of the Child Support Act as supplemented by Schedule 1 to those Regulations, or the parent ceases to be the person with care.

(17) Where a superseding decision is made in a case to which regulation 6A(2)($a$)  or (3) applies and the relevant circumstance is the death of a qualifying child or a qualifying child ceasing to be a qualifying child, the decision shall take effect from the first day of the maintenance period in which the change occurred.

(18) Where a superseding decision is made in a case to which regulation 6A(2)($a$)  or (3) applies and the relevant circumstance is that the non-resident parent, person with care or the qualifying child has moved out of the jurisdiction, the decision shall take effect from the first day of the maintenance period in which the non-resident parent, person with care or qualifying child leaves the jurisdiction and jurisdiction is within the meaning of section 44 of the Child Support Act.

(19) Where a superseding decision is made in a case to which regulation 6A(2)($a$)  or (3) applies and the relevant circumstance is that the maintenance calculation has been made in response to an application which is treated as made under section 6 of the Child Support Act and—
\begin{enumerate}\item[]
($a$) the person on whose application the calculation was made (“the applicant”) asks the Secretary of State to cease acting; and

($b$) the Secretary of State is satisfied that the applicant has ceased to fall within section 6(1) of that Act,
\end{enumerate}
the decision shall take effect from the first day of the maintenance period after the applicant asks the Secretary of State to cease acting.

(20) Where a superseding decision is made in a case to which regulation 6A(2)($a$)  or (3) applies and the relevant circumstance is that both the non-resident parent and the person with care with respect to whom a maintenance calculation was made request the Secretary of State to decide that the maintenance calculation shall cease and he is satisfied that they are living together, the decision shall take effect from the first day of the maintenance period in which the later of the two requests was made.

(21) Where a superseding decision is made in a case to which regulation 6A(2)($a$)  or (3) applies and the relevant circumstance is that—
\begin{enumerate}\item[]
($a$) an application for a maintenance calculation is made under section 4 or 7 of the Child Support Act, or treated as made under section 6(3) of that Act, in respect of a non-resident parent; and

($b$) before the decision as to a maintenance calculation is made at least one other maintenance calculation is in force with respect to the same non-resident parent but to a different person with care and a different child,
\end{enumerate}
the effective date of the maintenance calculation made in respect of the application shall be a date which is not later than 7 days after the date of notification to the non-resident parent and which is the day on which a maintenance period in respect of the maintenance calculation in force begins.

(22) Where a superseding decision is made in a case to which regulation 6A(3) applies and in relation to that decision a maintenance calculation is made to which paragraph 15 of Schedule 1 to the Child Support Act applies, the effective date of the calculation or calculations shall be the beginning of the maintenance period in which the change of circumstance to which the calculation or calculations relates occurred or is expected to occur and where that change occurred before the date of the application for the supersession and was notified after that date, the date of that application.

(23) In this regulation—
\begin{enumerate}\item[]
    “benefit week” in relation to income support has the same meaning as in regulation 2(1) of the Income Support Regulations, and in relation to jobseeker’s allowance has the same meaning as in regulation 1(3) of the Jobseeker’s Allowance Regulations;

    “partner” has the same meaning as in regulation 2 of the Income Support Regulations; and

    “relevant benefit” means a benefit which is prescribed in regulation 4 of the Maintenance Calculations and Special Cases Regulations for the purposes of paragraph 4(1)($b$)  of Part I of Schedule 1 to the Child Support Act, and child benefit as referred to in paragraph 10C(2)($a$)  of Part I of Schedule 1 to that Act. 
\end{enumerate}

\subsection*{Procedure where the Secretary of State proposes to supersede a decision under section 17 of the Child Support Act on his own initiative}

7C.  Where the Secretary of State on his own initiative proposes to make a decision superseding a decision he shall notify the relevant persons who could be materially affected by the decision of that intention.”.
\end{quotation}

\subsection[10. Insertion of regulations 15A, 15B, 15C and 15D]{Insertion of regulations 15A, 15B, 15C and 15D}

10.  After regulation 15 (jobseeker’s allowance determinations on incomplete evidence) there shall be inserted the following regulations—
\begin{quotation}
\subsection*{“Provision of information}

15A.---(1)  Where the Secretary of State has received an application under section 16 or 17 of the Child Support Act in connection with a previously determined variation which has effect on the maintenance calculation in force, he may request further information or evidence from the applicant to enable a decision on that application to be made and any such information or evidence shall be provided within one month of the date of notification of the request, or such longer period as the Secretary of State is satisfied is reasonable in the circumstances of the case.

(2) Where any information or evidence requested in accordance with paragraph (1) is not provided within the time limit specified in that paragraph, the Secretary of State may, where he is able to do so, proceed to make the decision in the absence of that information or evidence.

\subsection*{\sloppy Procedure in relation to an application made under section 16 or 17 of the Child Support Act in connection with a previously determined variation}

15B.---(1)  Subject to paragraph (3), where the Secretary of State has received an application under section 16 or 17 of the Child Support Act in connection with a previously determined variation which has effect on the maintenance calculation in force, he—
\begin{enumerate}\item[]
($a$) shall give notice of the application to the relevant persons, other than the applicant, informing them of the grounds on which the application has been made and any relevant information or evidence the applicant has given, except information or evidence falling within paragraph (2);

($b$) may invite representations, which need not be in writing but shall be in writing if in any case he so directs, from the relevant persons other than the applicant on any matter relating to that application, to be submitted to the Secretary of State within 14 days of notification or such longer period as the Secretary of State is satisfied is reasonable in the circumstances of the case; and

($c$) shall set out the provisions of paragraphs (2)($b$)  and ($c$), (4) and (5) in relation to such representations.
\end{enumerate}

(2) The information or evidence referred to in paragraphs (1)($a$), (4)($a$)  and (7), is—
\begin{enumerate}\item[]
($a$) details of the nature of the long-term illness or disability of the relevant other child which forms the basis of a variation application on the ground in regulation 11 of the Variations Regulations (special expenses — illness or disability of relevant other child) where the applicant requests they should not be disclosed and the Secretary of State is satisfied that disclosure is not necessary in order to be able to determine the application;

($b$) medical evidence or medical advice which has not been disclosed to the applicant or a relevant person and which the Secretary of State considers would be harmful to the health of the applicant or that relevant person if disclosed to him;

($c$) the address of a relevant person or qualifying child, or any other information which could reasonably be expected to lead to that person or child being located, where the Secretary of State considers that there would be a risk of harm or undue distress to that person or that child or any other children living with that person if the address or information were disclosed.
\end{enumerate}

(3) The Secretary of State need not act in accordance with paragraph (1) if—
\begin{enumerate}\item[]
($a$) he is satisfied on the information or evidence available to him, that he will not agree to a variation of the maintenance calculation in force, but if, on further consideration he is minded to do so he shall, before doing so, comply with the provisions of this regulation; and

($b$) were the application to succeed, the decision as revised or superseded would be less advantageous to the applicant than the decision before it was so revised or superseded.
\end{enumerate}

(4) Where the Secretary of State receives representations from the relevant persons he—
\begin{enumerate}\item[]
($a$) may, if he considers it reasonable to do so, send a copy of the representations concerned (excluding material falling within paragraph (2) above) to the applicant and invite any comments he may have within 14 days or such longer period as the Secretary of State is satisfied is reasonable in the circumstances of the case; and

($b$) where the Secretary of State acts under sub-paragraph ($a$), shall not proceed to make a decision in response to the application until he has received such comments or the period referred to in sub-paragraph ($a$)  has expired.
\end{enumerate}

(5) Where the Secretary of State has not received representations from the relevant persons notified in accordance with paragraph (1) within the time limit specified in sub-paragraph ($b$)  of that paragraph, he may proceed to make a decision under section 16 or 17 of the Child Support Act in response to the application, in their absence.

(6) In considering an application for a revision or supersession the Secretary of State shall take into account any representations received at the date upon which he makes a decision under section 16 or 17 of the Child Support Act, from the relevant persons including any representations received in connection with the application in accordance with paragraphs (1)($b$), (4)($a$)  and (7).

(7) Where any information or evidence requested by the Secretary of State under regulation 15A is received after notification has been given under paragraph (1), he may, if he considers it reasonable to do so and except where such information or evidence falls within paragraph (2), send a copy of such information or evidence to the relevant persons and may invite them to submit representations, which need not be in writing unless the Secretary of State so directs in any particular case, on that information or evidence.

(8) Where the Secretary of State is considering making a decision under section 16 or 17 of the Child Support Act in accordance with this regulation, he shall apply the factors to be taken into account for the purposes of section 28F of the Child Support Act set out in regulation 21 of the Variations Regulations (factors to be taken into account and not to be taken into account) as factors to be taken into account and not to be taken into account when considering making a decision under this regulation.

(9) In this regulation “relevant person” means—
\begin{enumerate}\item[]
($a$) a non-resident parent, or a person treated as a non-resident parent under regulation 8 of the Maintenance Calculations and Special Cases Regulations (persons treated as non-resident parents), whose liability to pay child support maintenance may be affected by any variation agreed;

($b$) a person with care, or a child to whom section 7 of the Child Support Act applies, where the amount of child support maintenance payable by virtue of a calculation relevant to that person with care or in respect of that child may be affected by any variation agreed.
\end{enumerate}

\subsection*{Notification of a decision made under section 16 or 17 of the Child Support Act}

15C.---(1)  Subject to paragraphs (2) and (5) to (11), a notification of a decision made following the revision or supersession of a decision made under section 11, 12 or 17 of the Child Support Act, whether as originally made or as revised under section 16 of that Act, shall set out, in relation to the decision in question—
\begin{enumerate}\item[]
($a$) the effective date of the maintenance calculation;

($b$) where relevant, the non-resident parent’s net weekly income;

($c$) the number of qualifying children;

($d$) the number of relevant other children;

($e$) the weekly rate;

($f$) the amounts calculated in accordance with Part I of Schedule 1 to the Child Support Act and, where there has been agreement to a variation or a variation has otherwise been taken into account, the Variations Regulations;

($g$) where the weekly rate is adjusted by apportionment or shared care or both, the amount calculated in accordance with paragraph 6, 7 or 8, as the case may be, of Part I of Schedule 1 to the Child Support Act; and

($h$) where the amount of child support maintenance which the non-resident parent is liable to pay is decreased in accordance with regulation 9 of the Maintenance Calculations and Special Cases Regulations (care provided in part by local authority) or 11 (non-resident parent liable to pay maintenance under a maintenance order) of those Regulations, the adjustment calculated in accordance with that regulation.
\end{enumerate}

(2) A notification of a revision or supersession of a maintenance calculation made under section 12(1) of the Child Support Act shall set out the effective date of the maintenance calculation, the default rate, the number of qualifying children on which the rate is based and whether any apportionment has been applied under regulation 7 of the Maintenance Calculation Procedure Regulations (default rate) and shall state the nature of the information required to enable a decision under section 11 of that Act to be made by way of section 16 of that Act.

(3) Except where a person gives written permission to the Secretary of State that the information in relation to him, mentioned in sub-paragraphs ($a$)  and ($b$), may be conveyed to other persons, any document given or sent under the provisions of paragraph (1) or (2) shall not contain—
\begin{enumerate}\item[]
($a$) the address of any person other than the recipient of the document in question (other than the address of the office of the officer concerned who is exercising functions of the Secretary of State under the Child Support Act) or any other information the use of which could reasonably be expected to lead to any such person being located;

($b$) any other information the use of which could reasonably be expected to lead to any person, other than a qualifying child or a relevant person, being identified.
\end{enumerate}

(4) Where a decision as to the revision or supersession of a decision made under section 11, 12 or 17 of the Child Support Act, whether as originally made or as revised under section 16 of that Act, is made under section 16 or 17 of that Act, a notification under paragraph (1) or (2) shall include information as to the provisions of sections 16, 17 and 20 of that Act.

(5) Where the Secretary of State makes a decision that a maintenance calculation shall cease to have effect—
\begin{enumerate}\item[]
($a$) he shall immediately notify the non-resident parent and person with care, so far as that is reasonably practicable;

($b$) where a decision has been superseded in a case where a child under section 7 of the Child Support Act ceases to be a child for the purposes of that Act, he shall immediately notify the persons in sub-paragraph ($a$)  and the other qualifying children within the meaning of section 7 of that Act; and

($c$) any notice under sub-paragraphs ($a$)  and ($b$)  shall specify the date with effect from which that decision took effect.
\end{enumerate}

(6) Where the Secretary of State, under the provisions of section 16 or 17 of the Child Support Act, has made a decision that an adjustment shall cease, or adjusted the amount payable under a maintenance calculation, he shall immediately notify the relevant persons, so far as that is reasonably practicable, that the adjustment has ceased or of the amount and period of the adjustment, and the amount payable during the period of the adjustment.

(7) Where the Secretary of State has made a decision under section 16 of the Child Support Act, revising a decision under section 41A or 47 of that Act, he shall immediately notify the relevant persons so far as that is reasonably practicable, of the amount of child support maintenance payable, the amount of arrears, the amount of the penalty payment or fees to be paid, as the case may be, the method of payment and the day by which payment is to be made.

(8) Where the non-resident parent appeals against a decision made by the Secretary of State under section 41A or 47 of the Child Support Act and the Secretary of State makes a decision under section 16 of that Act, before the appeal is decided he shall notify the relevant persons, so far as that is reasonably practicable of either the new amount of the penalty payment or the fee to be paid or that the amount is no longer payable, the method of payment and the day by which payment is to be made.

(9) Paragraphs (1) to (3) shall not apply where the Secretary of State has decided not to supersede a decision under section 17 of the Child Support Act, and he shall, so far as that is reasonably practicable, notify the relevant persons of that decision.

(10) A notification under paragraphs (6) to (9) shall include information as to the provisions of sections 16, 17 and 20 of the Child Support Act.

(11) Where paragraph (9) applies, and the Secretary of State decides not to supersede under regulation 6B, he shall notify the relevant person, in relation to the decision in question of—
\begin{enumerate}\item[]
($a$) the fact that regulation 6B applies to the decision;

($b$) the non-resident parent’s net income figure fixed for the purposes of the maintenance calculation in force in accordance with Part I of Schedule 1 to the Child Support Act;

($c$) the non-resident parent’s net income figure provided by that parent to the Secretary of State with the application for supersession under regulation 6A(3);

($d$) the decision of the Secretary of State not to supersede; and

($e$) the right to appeal against the decision under section 20 of the Child Support Act.
\end{enumerate}

(12) Where an appeal lapses in accordance with section 16(6) or 28F(5) of the Child Support Act, the Secretary of State shall, so far as that is reasonably practicable, notify the relevant persons that the appeal has lapsed.%
%
% Reg 10 revoked in part (25.1.10) by SI 2009/3151 Sch
%\subsection*{\sloppy Procedure in relation to the adjustment of the amount payable under a maintenance calculation}
%
%15D.---(1)  Where the Secretary of State has adjusted the amount payable under a maintenance calculation under the provisions of regulation 10(1) and (3A) of the Arrears, Interest and Adjustment of Maintenance Assessments Regulations and that maintenance calculation is subsequently replaced by a fresh maintenance calculation made by virtue of a revision under section 16 of the Child Support Act or of a decision under section 17 of that Act superseding an earlier decision, that adjustment shall, subject to paragraph (2), continue to apply to the amount payable under that fresh maintenance calculation unless the Secretary of State is satisfied that such adjustment would not be appropriate in all the circumstances of the case.
%
%(2) Where the Secretary of State is satisfied that the adjustment referred to in paragraph (1) would not be appropriate, he may make a decision under section 17 of the Child Support Act, superseding an earlier decision making an adjustment, and—
%\begin{enumerate}\item[]
%($a$) the adjustment shall cease; or
%
%($b$) he may adjust the amount payable under that fresh maintenance calculation,
%\end{enumerate}
%as he sees fit, having regard to the matters specified in regulation 10(1)($b$)(i)  to (iii)  of the Arrears, Interest and Adjustment of Maintenance Assessments Regulations.
”.
\end{quotation}

\amendment{
Reg.~10 revoked in so far as it inserts reg.~15D into the principal regulations (25.1.10) by the Child Support (Management of Payments and Arrears) Regulations 2009 Sch.
}

\subsection[11. Amendment of regulation 30]{Amendment of regulation 30}

11.  In regulation 30 (appeal against a decision which has been revised)—
\begin{enumerate}\item[]
($a$) in the heading after the words “which has been” there shall be inserted the words “replaced or”;

($b$) in paragraph (1)—
\begin{enumerate}\item[]
(i) for the words “is revised under section 16 of the Child Support Act” there shall be substituted the words “is treated as replaced by a decision under section 11 of the Child Support Act by section 28F(5) of that Act, or is revised under section 16 of that Act”;

(ii) after the words “and the decision as” there shall be inserted the words “replaced or”; and

(iii) after the words “before it was” there shall be inserted the words “replaced or”;
\end{enumerate}

($c$) in paragraph (3)—
\begin{enumerate}\item[]
(i) for the words “revised under section 16 of the Child Support Act” there shall be substituted “replaced under section 28F(5) of the Child Support Act or revised under section 16 of that Act”;

(ii) after the words “before it was” there shall be inserted the words “replaced or”; and

(iii) after the words “against the decision as” there shall be inserted the words “replaced or”;
\end{enumerate}

($d$) in paragraph (4) after the words “notification of the decision as” there shall be inserted the words “replaced or”; and

($e$) in paragraph (5) after the words “decision before it was” there shall be inserted the words “replaced or”.
\end{enumerate}

\amendment{
Reg.~12 revoked (25.1.10) by the Child Support (Managemement of Payments and Arrears) Regulations 2009 Sch.

\medskip
%}

% Reg 12 revoked (25.1.10) by SI 2009/3151 Sch
%\subsection[12. Insertion of regulation 30A]{Insertion of regulation 30A}
%
%12.  After regulation 30 there shall be inserted the following regulation—
%\begin{quotation}
%\subsection*{\sloppy “Appeals to appeal tribunals in child support cases}
%
%30A.  Section 20 of the Child Support Act shall apply to any decision of the Secretary of State that an adjustment shall cease or with respect to the adjustment of amounts payable under a maintenance calculation for the purpose of taking account of overpayments of child support maintenance and voluntary payments, or a decision under section 17 of that Act, whether as originally made or as revised under section 16 of that Act.”.
%\end{quotation}

%\amendment{
Reg. 13 revoked (3.11.08) by the Tribunals, Courts and Enforcement Act 2007 (Transitional and Consequential Provisions) Order 2008 Sch.~2.
}

% Reg 13 revoked (3.11.08) by SI 2008/2683 Sch 2
%\subsection[13. Substitution of regulation 45]{Substitution of regulation 45}
%
%13.  For regulation 45 (consideration of more than one appeal under section 20 of the Child Support Act) there shall be substituted the following regulation—
%\begin{quotation}
%\subsection*{\sloppy \textls[25]{“Procedure following a referral under section} 28D(1)($b$)  of the Child Support Act}
%
%45.---(1)  On a referral under section 28D(1)($b$)  of the Child Support Act an appeal tribunal may—
%\begin{enumerate}\item[]
%($a$) consider two or more applications for a variation with respect to the same application for a maintenance calculation together; or
%
%($b$) consider two or more applications for a variation with respect to the same maintenance calculation together.
%\end{enumerate}
%
%(2) In this regulation “maintenance calculation” means a decision under section 11 or 17 of the Child Support Act, as calculated in accordance with Part I of Schedule 1 to that Act, whether as originally made or as revised under section 16 of that Act.”.
%\end{quotation}

\subsection[14, 15. Revocation and savings]{Revocation and savings}

14.---(1)  Subject to 
the Child Support (Transitional Provisions) Regulations 2000 and % Words inserted (3.3.03) by SI 2003/347 reg 2(1), (2)(a)
paragraph (2), regulations 10(2) and (3) and 11 to 17 of the Arrears, Interest and Adjustment of Maintenance Assessments Regulations are hereby revoked.

(2) Where on the commencement date—
\begin{enumerate}\item[]
($a$) an appeal has not been decided;

($b$) the time limit for lodging an appeal has not expired;

($c$) the time limit for making an application for the revision of a decision has not expired; or

($d$) an application for a supersession of a decision has not been decided,
\end{enumerate}
the provisions in regulations 10(2) and (3) and 11 to 17 of the Arrears, Interest and Adjustment of Maintenance Assessments Regulations shall continue to apply for the purposes of—
\begin{enumerate}\item[]
(i) the decision of the appeal tribunal referred to in sub-paragraph ($a$);

(ii) the ability to lodge the appeal referred to in sub-paragraph ($b$)  and the decision of the appeal tribunal following the lodging of that appeal;

(iii) the ability to apply for the revision referred to in sub-paragraph ($c$)  and the decision whether to revise following any such application; or

(iv) the decision whether to supersede following the application referred to in sub-paragraph ($d$).
\end{enumerate}

(3) Where on or after the commencement date an adjustment falls to be made in relation to a maintenance assessment, these Regulations shall not apply for the purposes of making the adjustment.

(4) In this regulation—
\begin{enumerate}\item[]
    “commencement date” means, with respect to a particular case, the date on which these Regulations come into force with respect to that type of case;

    “former Act” means the Child Support Act before its amendment by the Child Support, Pensions and Social Security Act 2000; and

    “maintenance assessment” has the meaning given in the former Act. 
\end{enumerate}

\amendment{
Words inserted in reg. 14(1) (3.3.03) by the Child Support (Transitional Provision) (Miscellaneous Amendments) Regulations 2003 reg. 2(1), (2)(a).
}

\medskip

15.---%
% Reg 15(Z1) inserted (3.3.03) by SI 2001/162 reg 2(3), (4)(a)
(Z1) This regulation is subject to the Child Support (Transitional Provisions) Regulations 2000.

(1)  Where—
\begin{enumerate}\item[]
($a$) before the commencement date—
\begin{enumerate}\item[]
(i) an application was made and not determined for a departure direction or a revision or supersession of a decision in respect of a departure direction;

(ii) the Secretary of State had initiated but not completed a revision or supersession of a decision in respect of a departure direction; or

(iii) any appeal was lodged in respect of a departure direction decision which, on the commencement date, had not been decided; or
\end{enumerate}

($b$) on the commencement date any time limit provided for in Regulations for making an application for a departure direction, or revision or, for making an appeal in respect of a departure direction decision, had not expired,
\end{enumerate}
regulation 13 shall not apply for the purposes of any appeal—
\begin{enumerate}\item[]
($aa$) made in consequence of the decision on the application, revision or supersession referred to in paragraph (1)($a$)(i);

($bb$) made in consequence of the revision or supersession referred to in paragraph (1)($a$)(ii);

($cc$) referred to in paragraph (1)($a$)(iii); or

($dd$) made within the time limit referred to in paragraph (1)($b$)  or made in consequence of a decision made on an application for a departure direction or revision made within the time limit referred to in that paragraph.
\end{enumerate}

(2) In this regulation “commencement date” has the same meaning as in regulation 14. 

\amendment{
Reg. 15(Z1) inserted (3.3.03) by the Child Support (Transitional Provision) (Miscellaneous Amendments) Regulations 2003 reg. 2(3), (4)(c).
}

\bigskip

Signed 
by authority of the Secretary of State for Social Security.

{\raggedleft
\emph{Jeff Rooker}\\*Minister of State,\\*Department of Social Security

}

4th December 2000

\small

\part{Explanatory Note}

\renewcommand\parthead{--- Explanatory Note}

\subsection*{(This note is not part of the Regulations)}

These Regulations amend the Social Security and Child Support (Decisions and Appeals) Regulations 1999 (S.I.\ 1999/991) (“the principal Regulations”). The Regulations provide for the decision-making process under the Child Support Act 1991 (c.\ 48) (“the Child Support Act”) for child support and related matters, consequent upon the introduction of changes to the child support system made by the Child Support, Pensions and Social Security Act 2000 (c.\ 19). These Regulations come into force at different times for different cases according to the dates on which provisions of the Child Support, Pensions and Social Security Act 2000 which are relevant to these Regulations are commenced for different types of cases.

Regulation 1 contains provisions relating to citation, commencement and interpretation.

Regulation 2 inserts definitions into the principal Regulations for the purposes of the amendments made by these Regulations.

Regulation 3 amends regulation 2 of the principal Regulations to provide for notices served under the Child Support Act.

Regulation 4 makes a consequential amendment to the principal Regulations.

Regulations 5 and 7 insert regulations into the principal Regulations providing for the revision of child support decisions and when such decisions take effect.

Regulation 6 amends regulation 4 of the principal Regulations.

Regulations 8 and 9 insert regulations into the principal Regulations providing for the supersession of child support decisions, when such decisions take effect and related procedural rules.

Regulation 10 inserts regulations into the principal Regulations which enable the Secretary of State to request further information or evidence and invite representations where he has to make a decision which is connected to a previously determined variation and make provision in respect of notification following decisions made by him.

Regulation 11 amends regulation 30 of the principal Regulations.

Regulation 12 extends appeal rights to decisions adjusting amounts payable under maintenance calculations.

Regulation 13 substitutes a regulation in the principal Regulations providing for the procedure where the Secretary of State has referred an application for a variation to an appeal tribunal under section 28D(1)($b$)  of the Child Support Act.

Regulation 14 provides for the revocation and savings of regulations 10(2) and (3) and 11 to 17 of the Child Support (Arrears, Interest and Adjustment of Maintenance Assessments) Regulations 1992 (S.I.\ 1992/1816).

Regulation 15 provides for the saving of regulation 45 of the principal Regulations.

The impact on business of these Regulations was covered in the Regulatory Impact Assessment (RIA) for the Child Support, Pensions and Social Security Act 2000, in accordance with, and in consequence of which, these Regulations are made. A copy of that RIA has been placed in the libraries of both Houses of Parliament and can be obtained from the Department of Social Security, Regulatory Impact Unit, Adelphi, 1--11 John Adam Street, London \textsc{\lowercase{WC2N 6HT}}. 

\end{document}