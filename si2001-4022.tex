\documentclass[12pt,a4paper]{article}

\newcommand\regstitle{The Social Security (Loss of Benefit) Regulations 2001}

\newcommand\regsnumber{2001/4022}

%\opt{newrules}{
\title{\regstitle}
%}

%\opt{2012rules}{
%\title{Child Maintenance and Other Payments Act 2008\\(2012 scheme version)}
%}

\author{S.I.\ 2001 No.\ 4022}

\date{Made
18th December 2001\\
%Laid before Parliament
%4th March 2008\\
Coming into force
1st April 2002
}

%\opt{oldrules}{\newcommand\versionyear{1993}}
%\opt{newrules}{\newcommand\versionyear{2003}}
%\opt{2012rules}{\newcommand\versionyear{2012}}

\usepackage{csa-regs}

\setlength\headheight{42.11603pt}

\begin{document}

\maketitle

\noindent
Whereas a draft of this instrument was laid before Parliament in accordance with section~11(3) of the Social Security Fraud Act 2001\footnote{2001 c.~11.}, section~80(1) of the Social Security Act 1998\footnote{1998 c.~14.} and section 5A(3) of the Pensions Appeal Tribunals Act 1943\footnote{6 \& 7 Geo.~6 c.~39.} and approved by resolution of each House of Parliament.

Now, therefore, the Secretary of State, in exercise of the powers conferred by sections~7(3) to (6), 8(3) and (4), 9(2) to (5), 10(1) and (2) and~11(1) of the Social Security Fraud Act 2001\footnote{Section 11(1) is cited because of the meaning ascribed to the word “prescribed”.}, section~189(4) of the Social Security Administration Act 1992\footnote{1992 c.~5; section~189 is applied to regulations made under sections~7 to 10 of the Social Security Fraud Act 2001 by section~11(4) of that Act.}, sections~79(4) and 84 of, and paragraph 9 of Schedule 2 to, the Social Security Act 1998\footnote{Section 84 is cited because of the meaning ascribed to the word “prescribe”.} and section 5A(2) of the Pensions Appeal Tribunals Act 1943\footnote{Section 5A was inserted by section 57 of the Child Support, Pensions and Social Security Act 2000 (c.~19).}, and of all other powers enabling him in that behalf, by this Instrument, which is made before the end of the period of 6 months beginning with the coming into force of sections~7 to 13 of the Social Security Fraud Act 2001 and which contains only regulations made by virtue of, or consequential upon, those sections\footnote{Section 12(3) of the Social Security Fraud Act 2001 added sections 7 to 11 of that Act, and paragraph 104 of Schedule 7 to the Social Security Act 1998 added Chapter II of Part I of that Act to the list of “relevant enactments” in respect of which regulations are to be referred to the Social Security Advisory Committee. These Regulations are made within six months of the coming into force of the relevant provisions of the 2001 Act and are therefore exempt from this requirement by virtue of section 173(5)($b$) of the Social Security Administration Act 1992.}, hereby makes the following Regulations:  

{\sloppy

\tableofcontents

}

\bigskip

\setcounter{secnumdepth}{-2}

\section[Part I --- General]{Part I\\*General}

\renewcommand\parthead{--- Part I}

\subsection[1. Citation, commencement and interpretation]{Citation, commencement and interpretation}

1.---(1)  These Regulations may be cited as the Social Security (Loss of Benefit) Regulations 2001 and shall come into force on 1st April 2002.

(2) In these Regulations, unless the context otherwise requires—
\begin{enumerate}\item[]
“the Act” means the Social Security Fraud Act 2001;

“the Benefits Act” means the Social Security Contributions and Benefits Act 1992\footnote{1992 c.~4.};

“the Council Tax Benefit Regulations” means the Council Tax Benefit (General) Regulations 1992\footnote{S.I.~1992/1814.};

“the Housing Benefit Regulations” means the Housing Benefit (General) Regulations 1987\footnote{S.I.~1987/1971.};

“the Income Support Regulations” means the Income Support (General) Regulations 1987\footnote{S.I.~1987/1967.};

“the Jobseekers Act” means the Jobseekers Act 1995\footnote{1995 c.~18.};

“the Jobseeker’s Allowance Regulations” means the Jobseeker’s Allowance Regulations 1996\footnote{S.I.~1996/207.};

“claimant” in a regulation means the person claiming the sanctionable benefit referred to in that regulation;

“disqualification period” means the period in respect of which the restrictions on payment of a relevant benefit apply in respect of an offender in accordance with section~7(6) of the Act and shall be interpreted in accordance with regulation~2; and

“offender” means the person who is subject to the restriction in the payment of his benefit in accordance with section~7 of the Act.
\end{enumerate}

(3) Expressions used in these Regulations which are defined either for the purposes of the Jobseekers Act or for the purposes of the Jobseeker’s Allowance Regulations shall, except where the context otherwise requires, have the same meaning as for the purposes of that Act or, as the case may be, those Regulations.

(4) In these Regulations, unless the context otherwise requires, a reference—
\begin{enumerate}\item[]
($a$) to a numbered regulation is to the regulation in these Regulations bearing that number;

($b$) in a regulation to a numbered paragraph is to the paragraph in that regulation bearing that number.
\end{enumerate}

\subsection[2. Disqualification period]{Disqualification period}

2.---(1)  Subject to paragraph (2), the first day of the disqualification period for the purpose of section~7(6) of the Act shall be—
\begin{enumerate}\item[]
($a$) where, on the determination day—
\begin{enumerate}\item[]
(i) the offender is in receipt of a sanctionable benefit;

(ii) the offender is a member of a joint-claim couple which is in receipt of a joint-claim jobseeker’s allowance; or

(iii) the offender’s family member is in receipt of income support, jobseeker’s allowance, housing benefit or council tax benefit,
\end{enumerate}
the day which is 28 days after the determination day;

($b$) where sub-paragraph ($a$)  does not apply, the day which is 28 days after the first day after the determination day on which the Secretary of State decides to award—
\begin{enumerate}\item[]
(i) a sanctionable benefit to the offender;

(ii) a joint-claim jobseeker’s allowance to a joint-claim couple of which the offender is a member; or

(iii) income support, jobseeker’s allowance, housing benefit or council tax benefit to an offender’s family member.
\end{enumerate}
\end{enumerate}

(2) For the purposes of sub-paragraph (1), the first day of the disqualification period shall be no later than 3 years and 28 days after the date of the conviction of the offender for the benefit offence in the later proceedings referred to in section~7(1) of the Act and section~7(9) of the Act (date of conviction and references to conviction) shall apply for the purposes of this paragraph as it applies for the purposes of section~7 of the Act.

(3) In this regulation, “the determination day” means the day on which the Secretary of State determines that a restriction under—
\begin{enumerate}\item[]
($a$) section~7 of the Act would be applicable to the offender were he in receipt of a sanctionable benefit;

($b$) section~8 of the Act would be applicable to the offender were he a member of a joint-claim couple which is in receipt of a joint-claim jobseeker’s allowance; or

($c$) section 9 of the Act would be applicable to the offender’s family member were that member in receipt of income support, jobseeker’s allowance, housing benefit or council tax benefit.
\end{enumerate}

\section[Part II --- Reductions]{Part II\\*Reductions}

\renewcommand\parthead{--- Part II}

\subsection[3. Reduction of income support]{Reduction of income support}

3.---(1)  Subject to paragraphs (2) to (4), any payment of income support which falls to be made to an offender in respect of any week in the disqualification period, or to an offender’s family member in respect of any week in the relevant period, shall be reduced—
\begin{enumerate}\item[]
($a$) where the claimant or a member of his family is pregnant or seriously ill, by a sum equivalent to 20 per cent.;

($b$) where the applicable amount of the offender used to calculate that payment of income support has been reduced pursuant to regulation~22A of the Income Support Regulations\footnote{Regulation 22A was inserted by S.I.~1996/206 and amended by S.I.~1999/2422, 1999/3109 and 2000/590.} (appeal against a decision embodying an incapacity for work determination), whether or not the appeal referred to in that regulation is successful, by a sum equivalent to 20 per cent.;

($c$) in any other case, by a sum equivalent to 40 per cent.,
\end{enumerate}
of the applicable amount of the offender in respect of a single claimant for income support on the first day of the disqualification period or, as the case may be, on the first day of the relevant period, and specified in paragraph 1(1) of Schedule 2 to the Income Support Regulations.

(2) Payment shall not be reduced under paragraph (1) to below 10 pence per week.

(3) A reduction under paragraph (1) shall, if it is not a multiple of 5p, be rounded to the nearest such multiple or, if it is a multiple of 2$.$5p but not of 5p, to the next lower multiple of 5p.

(4) A payment of income support shall not be reduced as provided in paragraph (1) in respect of any week in the disqualification period in respect of which that payment of income support is subject to a restriction imposed pursuant to section 62 or 63 of the Child Support, Pensions and Social Security Act 2000 (loss of benefit provisions).

(5) Where the rate of income support payable to an offender or an offender’s family member changes, the rules set out above for a reduction in the benefit payable shall be applied to the new rate and any adjustment to the reduction shall take effect from the first day of the first benefit week to start after the date of the change.

(6) In this regulation, “benefit week” shall have the same meaning as in regulation of 2(1) of the Income Support Regulations\footnote{The definition was amended by S.I.~1988/1445.}.

\subsection[4. Reduction of joint-claim jobseeker’s allowance]{Reduction of joint-claim jobseeker’s allowance}

4.  In respect of any part of the disqualification period when section~8(2) of the Act does not apply, the reduced rate of joint-claim jobseeker’s allowance payable to the member of that couple who is not the offender shall be—
\begin{enumerate}\item[]
($a$) in any case in which the member of the couple who is not the offender satisfies the conditions set out in section 2 of the Jobseekers Act (contribution-based conditions), a rate equal to the amount calculated in accordance with section~4(1) of that Act;

($b$) in any case where the couple are a couple in hardship for the purposes of regulation 11, a rate equal to the amount calculated in accordance with regulation 16;

($c$) in any other case, a rate calculated in accordance with section~4(3A) of the Jobseekers Act\footnote{Section 4(3A) was inserted by section 59 of, and paragraph 5(3) of Schedule 7 to, the Welfare Reform and Pensions Act 1999 (c.~30).} save that the applicable amount shall be the amount determined by reference to paragraph 1(1) of Schedule 1 to the Jobseeker’s Allowance Regulations as if the member of the couple who is not the offender were a single claimant.
\end{enumerate}

\section[Part III --- Hardship]{Part III\\*Hardship}

\renewcommand\parthead{--- Part III}

\subsection[5. Meaning of “person in hardship”]{Meaning of “person in hardship”}

5.---(1)  In this Part of these Regulations, a “person in hardship” means, for the purposes of regulation 6, a person, other than a person to whom paragraph (3) or (4) applies, where—
\begin{enumerate}\item[]
($a$) she is a single woman who is pregnant and in respect of whom the Secretary of State is satisfied that, unless a jobseeker’s allowance is paid, she will suffer hardship;

($b$) he is a single person who is responsible for a young person and the Secretary of State is satisfied that, unless a jobseeker’s allowance is paid, the young person will suffer hardship;

($c$) he is a member of a married or unmarried couple where—
\begin{enumerate}\item[]
(i) the woman is pregnant; and

(ii) the Secretary of State is satisfied that, unless a jobseeker’s allowance is paid, the woman will suffer hardship;
\end{enumerate}

($d$) he is a member of a polygamous marriage and—
\begin{enumerate}\item[]
(i) one member of the marriage is pregnant; and

(ii) the Secretary of State is satisfied that, unless a jobseeker’s allowance is paid, that woman will suffer hardship;
\end{enumerate}

($e$) he is a member of a married or unmarried couple or of a polygamous marriage where—
\begin{enumerate}\item[]
(i) one or both members of the couple, or one or more members of the polygamous marriage, are responsible for a child or young person; and

(ii) the Secretary of State is satisfied that, unless a jobseeker’s allowance is paid, the child or young person will suffer hardship;
\end{enumerate}

($f$) he has an award of a jobseeker’s allowance which includes or would, if a claim for a jobseeker’s allowance from him were to succeed, have included in his applicable amount a disability premium and the Secretary of State is satisfied that, unless a jobseeker’s allowance is paid, the person who would satisfy the conditions of entitlement to that premium would suffer hardship;

($g$) he suffers, or his partner suffers, from a chronic medical condition which results in functional capacity being limited or restricted by physical impairment and the Secretary of State is satisfied that—
\begin{enumerate}\item[]
(i) the suffering has already lasted, or is likely to last, for not less than 26 weeks; and

(ii) unless a jobseeker’s allowance is paid to that person, the probability is that the health of the person suffering would, within 2 weeks of the Secretary of State making his decision, decline further than that of a normally healthy adult and that person would suffer hardship;
\end{enumerate}

($h$) he does, or his partner does, or in the case of a person who is married to more than one person under a law which permits polygamy, at least one of those persons does, devote a considerable portion of each week to caring for another person who—
\begin{enumerate}\item[]
(i) is in receipt of an attendance allowance or the care component of disability living allowance at one of the two higher rates prescribed under section~72(4) of the Benefits Act;

(ii) has claimed either attendance allowance or disability living allowance, but only for so long as the claim has not been determined, or for 26 weeks from the date of claiming, whichever is the earlier; or

(iii) has claimed either attendance allowance or disability living allowance and has an award of either attendance allowance or the care component of disability living allowance at one of the two higher rates prescribed under section~72(4) of the Benefits Act for a period commencing after the date on which that claim was made,
\end{enumerate}
and the Secretary of State is satisfied, after taking account of the factors set out in paragraph (5), in so far as they are appropriate to the particular circumstances of the case, that the person providing the care will not be able to continue doing so unless a jobseeker’s allowance is paid to the offender;

($i$) he is a person or is the partner of a person to whom section~16 of the Jobseekers Act applies by virtue of a direction issued by the Secretary of State, except where the person to whom the direction applies does not satisfy the requirements of section~1(2)($a$)  to ($c$)  of that Act;

($j$) he is a person—
\begin{enumerate}\item[]
(i) to whom section 3(1)($f$)(iii)  of the Jobseekers Act (persons under the age of 18) applies, or is the partner of such a person; and

(ii) in respect of whom the Secretary of State is satisfied that the person will, unless a jobseeker’s allowance is paid, suffer hardship; or
\end{enumerate}

($k$) he is a person—
\begin{enumerate}\item[]
(i) who, pursuant to the Children Act 1989\footnote{1989 c.~41.}, was being looked after by a local authority;

(ii) with whom the local authority had a duty, pursuant to that Act, to take reasonable steps to keep in touch; or

(iii) who, pursuant to that Act, qualified for advice and assistance from a local authority,
\end{enumerate}
but in respect of whom head (i), (ii)  or (iii)  above, as the case may be, had not applied for a period of 3 years or less as at the date on which he complies with the requirements of regulation 9; and
\begin{enumerate}\item[]
(iv) who, as at the date on which he complies with the requirements of regulation 9, is under the age of 21.
\end{enumerate}
\end{enumerate}

(2) Except in a case to which paragraph (3) or (4) applies, a person shall, for the purposes of regulation 7, be deemed to be a person in hardship where, after taking account of the factors set out in paragraph (5) in so far as they are appropriate to the particular circumstances of the case, the Secretary of State is satisfied that he or his partner will suffer hardship unless a jobseeker’s allowance is paid to him.

(3) In paragraphs (1) and (2), a person shall not be deemed to be a person in hardship—
\begin{enumerate}\item[]
($a$) where he is entitled, or his partner is entitled, to income support or where he or his partner fall within a category of persons prescribed for the purpose of section~124(1)($e$)  of the Benefits Act;

($b$) during any period in respect of which it has been determined that a jobseeker’s allowance is not payable to him pursuant to section~19 of the Jobseekers Act (circumstances in which a jobseeker’s allowance is not payable); or

($c$) during any week in the disqualification period in respect of which he is subject to a restriction imposed pursuant to section 62 or 63 of the Child Support, Pensions and Social Security Act 2000 (loss of benefit provisions).
\end{enumerate}

(4) Paragraph (1)($h$)  shall not apply in a case where the person being cared for resides in a residential care home or nursing home.

(5) Factors which, for the purposes of paragraphs (1) and (2), the Secretary of State is to take into account in determining whether the person is a person in hardship are—
\begin{enumerate}\item[]
($a$) the presence in that person’s family of a person who satisfies the requirements for a disability premium specified in paragraphs 13 and~14 of Schedule 1 to the Jobseeker’s Allowance Regulations or for a disabled child premium specified in paragraph 16 of that Schedule to those Regulations;

($b$) the resources which, without a jobseeker’s allowance, are likely to be available to the offender’s family, the amount by which these resources fall short of the amount applicable in his case in accordance with regulation 10 (applicable amount in hardship cases), the amount of any resources which may be available to members of the offender’s family from any person in the offender’s household who is not a member of his family and the length of time for which those factors are likely to persist;

($c$) whether there is a substantial risk that essential items, including food, clothing, heating and accommodation, will cease to be available to that person or a member of his family, or will be available at considerably reduced levels and the length of time those factors are likely to persist.
\end{enumerate}

(6) In determining the resources available to that person’s family under paragraph (5)($b$), any training premium or top-up payment paid pursuant to the Employment and Training Act 1973\footnote{1973 c.~50.} shall be disregarded.

\subsection[6. Circumstances in which an income-based jobseeker’s allowance is payable to a person who is a person in hardship]{Circumstances in which an income-based jobseeker’s allowance is payable to a person who is a person in hardship}

6.---(1)  This regulation applies to a person in hardship within the meaning of regulation 5(1) and is subject to the provisions of regulations 8 and 9.

(2) An income-based jobseeker’s allowance shall be payable to a person in hardship even though section~7(2) of the Act prevents payment of a jobseeker’s allowance to the offender or section 9 of the Act prevents payment of a jobseeker’s allowance to an offender’s family member but the allowance shall be payable under this paragraph only if and so long as the claimant satisfies the conditions for entitlement to an income-based jobseeker’s allowance.

{\hbadness=1655
\subsection[7. Further circumstances in which an income-based jobseeker’s allowance is payable to a person who is a person in hardship]{\sloppy Further circumstances in which an income-based jobseeker’s allowance is payable to a person who is a person in hardship}

}

7.---(1)  This regulation applies to a person in hardship within the meaning of regulation 5(2) and is subject to the provisions of regulations 8 and 9.

(2) An income-based jobseeker’s allowance shall be payable to a person in hardship even though section~7(2) of the Act prevents payment of a jobseeker’s allowance to the offender or section 9 of the Act prevents payment of a jobseeker’s allowance to an offender’s family member but the allowance shall not be payable under this paragraph—
\begin{enumerate}\item[]
($a$) where the offender is the claimant, in respect of the first 14 days of the disqualification period;

($b$) where the offender’s family member is the claimant, in respect of the first 14 days of the relevant period,
\end{enumerate}
and shall be payable thereafter only if and so long as the claimant satisfies the conditions for entitlement to an income-based jobseeker’s allowance.

\subsection[8. Conditions for payment of income-based jobseeker’s allowance]{\sloppy Conditions for payment of income-based jobseeker’s allowance}

8.---(1)  An income-based jobseeker’s allowance shall not be payable in accordance with regulation 6 or 7 except where the claimant has—
\begin{enumerate}\item[]
($a$) furnished on a form approved for the purpose by the Secretary of State or in such other form as he may in any particular case approve, a statement of the circumstances he relies upon to establish entitlement under regulation 5(1) or, as the case may be, 5(2); and

($b$) signed the statement.
\end{enumerate}

(2) The completed and signed form shall be delivered by the claimant to such office as the Secretary of State may specify.

\subsection[9. Provision of information]{Provision of information}

9.  For the purpose of section~7(4)($b$)  of the Act, the offender, and for the purpose of section 9(4)($b$)  of the Act, the offender or any member of his family, shall provide to the Secretary of State information as to the circumstances of the person alleged to be in hardship.

\subsection[10. Applicable amount in hardship cases]{Applicable amount in hardship cases}

10.---(1)  The weekly applicable amount of a person to whom an income-based jobseeker’s allowance is payable in accordance with this Part shall be reduced by a sum equivalent to 40 per cent.\ or, in a case where the claimant or any other member of his family is either pregnant or seriously ill, 20 per cent.\ of the following amount—
\begin{enumerate}\item[]
($a$) where the claimant is a single claimant aged not less than 18 but less than 25 or a member of a couple or polygamous marriage where one member is aged not less than 18 but less than 25 and the other member or, in the case of a polygamous marriage each other member, is a person under 18 who is not eligible for an income-based jobseeker’s allowance under section 3(1)($f$)(iii)  of the Jobseekers Act or is not subject to a direction under section~16 of that Act, the amount specified in paragraph 1(1)($d$)  of Schedule 1 to the Jobseeker’s Allowance Regulations;

($b$) where the claimant is a single claimant aged not less than 25 or a member of a couple or a polygamous marriage (other than a member of a couple or polygamous marriage to whom sub-paragraph ($a$)  applies) at least one of whom is aged not less than 18, the amount specified in paragraph 1(1)($e$)  of Schedule 1 to the Jobseeker’s Allowance Regulations.
\end{enumerate}

(2) A reduction under paragraph (1) shall, if it is not a multiple of 5p, be rounded to the nearest such multiple or, if it is a multiple of 2$.$5p but not of 5p, to the next lower multiple of 5p.

\section[Part IV --- Hardship for joint-claim couples]{Part IV\\*Hardship for joint-claim couples}

\renewcommand\parthead{--- Part IV}

\subsection[11. Application of Part and meaning of “couple in hardship”]{Application of Part and meaning of “couple in hardship”}

11.---(1)  This Part of these Regulations applies in respect of any part of the disqualification period when section~8(2) of the Act would otherwise apply.

(2) In this Part of these Regulations, a “couple in hardship” means, for the purposes of regulation 13, a joint-claim couple, other than a couple to whom paragraph (4) or (5) applies, who are claiming a joint-claim jobseeker’s allowance jointly where at least one member of that couple is an offender and where—
\begin{enumerate}\item[]
($a$) the woman member of the joint-claim couple is pregnant and the Secretary of State is satisfied that, unless a joint-claim jobseeker’s allowance is paid, she will suffer hardship;

($b$) one or both members of the couple are members of a polygamous marriage, one member of the marriage is pregnant and the Secretary of State is satisfied that, unless a joint-claim jobseeker’s allowance is paid, she will suffer hardship;

($c$) the award of a joint-claim jobseeker’s allowance includes, or would, if a claim for a jobseeker’s allowance from the couple were to succeed, have included in their applicable amount a disability premium and the Secretary of State is satisfied that, unless a joint-claim jobseeker’s allowance is paid, the member of the couple who would have caused the disability premium to be applicable to the couple would suffer hardship;

($d$) either member of the couple suffers from a chronic medical condition which results in functional capacity being limited or restricted by physical impairment and the Secretary of State is satisfied that—
\begin{enumerate}\item[]
(i) the suffering has already lasted or is likely to last, for not less than 26 weeks; and

(ii) unless a joint-claim jobseeker’s allowance is paid, the probability is that the health of the person suffering would, within two weeks of the Secretary of State making his decision, decline further than that of a normally healthy adult and the member of the couple who suffers from that condition would suffer hardship;
\end{enumerate}

($e$) either member of the couple, or where a member of that couple is married to more than one person under a law which permits polygamy, one member of that marriage, devotes a considerable portion of each week to caring for another person who—
\begin{enumerate}\item[]
(i) is in receipt of an attendance allowance or the care component of disability living allowance at one of the two higher rates prescribed under section~72(4) of the Benefits Act;

(ii) has claimed either attendance allowance or disability living allowance, but only for so long as the claim has not been determined, or for 26 weeks from the date of claiming, whichever is the earlier; or

(iii) has claimed either attendance allowance or disability living allowance and has an award of either attendance allowance or the care component of disability living allowance at one of the two higher rates prescribed under section~72(4) of the Benefits Act for a period commencing after the date on which that claim was made,
\end{enumerate}
and the Secretary of State is satisfied, after taking account of the factors set out in paragraph (6) in so far as they are appropriate to the particular circumstances of the case, that the person providing the care will not be able to continue doing so unless a joint-claim jobseeker’s allowance is paid; or

($f$) section~16 of the Jobseekers Act applies to either member of the couple by virtue of a direction issued by the Secretary of State, except where the member of the joint-claim couple to whom the direction applies does not satisfy the requirements of section~1(2)($a$)  to ($c$)  of that Act;

($g$) section 3A(1)($e$)(ii)  of the Jobseekers Act\footnote{Section 3A was inserted by section 59 of, and paragraph 4(2) of Schedule 7 to, the Welfare Reform and Pensions Act 1999 (c.~30).} (member of joint-claim couple under the age of 18) applies to either member of the couple and the Secretary of State is satisfied that unless a joint-claim jobseeker’s allowance is paid, the couple will suffer hardship; or

($h$) one or both members of the couple is a person—
\begin{enumerate}\item[]
(i) who, pursuant to the Children Act 1989\footnote{1989 c.~41.}, was being looked after by a local authority;

(ii) with whom the local authority had a duty, pursuant to that Act, to take reasonable steps to keep in touch; or

(iii) who, pursuant to that Act, qualified for advice or assistance from a local authority,
\end{enumerate}
but in respect of whom head (i), (ii)  or (iii)  above, as the case may be, had not applied for a period of 3 years or less as at the date on which the requirements of regulation 15 are complied with; and
\begin{enumerate}\item[]
(iv) who, as at the date on which the requirements of regulation 15 are complied with, is under the age of 21.
\end{enumerate}
\end{enumerate}

(3) Except in a case to which paragraph (4) or (5) applies, a joint-claim couple shall, for the purposes of regulation 14, be deemed to be a couple in hardship where the Secretary of State is satisfied, after taking account of the factors set out in paragraph (6) in so far as they are appropriate to the particular circumstances of the case, that the couple will suffer hardship unless a joint-claim jobseeker’s allowance is paid.

(4) In paragraphs (2) and (3), a joint-claim couple shall not be deemed to be a “couple in hardship”—
\begin{enumerate}\item[]
($a$) where one member of the couple is entitled to income support or falls within a category of persons prescribed for the purposes of section~124(1)($e$)  of the Benefits Act; or

($b$) during a period in respect of which it has been determined that both members of the couple are subject to sanctions for the purposes of section 20A of the Jobseekers Act (denial or reduction of joint-claim jobseeker’s allowance).
\end{enumerate}

(5) Paragraph (2)($e$)  shall not apply in a case where the person being cared for resides in a residential care or nursing home.

(6) Factors which, for the purposes of paragraphs (2) and (3), the Secretary of State is to take into account in determining whether a joint-claim couple will suffer hardship are—
\begin{enumerate}\item[]
($a$) the presence in the joint-claim couple of a person who satisfies the requirements for a disability premium specified in paragraphs~20H and 20I of Schedule 1 to the Jobseeker’s Allowance Regulations;

($b$) the resources which, without a joint-claim jobseeker’s allowance, are likely to be available to the joint-claim couple, the amount by which these resources fall short of the amount applicable in their case in accordance with regulation 16 (applicable amount of joint-claim couple in hardship cases), the amount of any resources which may be available to the joint-claim couple from any person in the couple’s household who is not a member of the family and the length of time for which those factors are likely to persist;

($c$) whether there is a substantial risk that essential items, including food, clothing, heating and accommodation, will cease to be available to the joint-claim couple, or will be available at considerably reduced levels, the hardship that will result and the length of time those factors are likely to persist.
\end{enumerate}

(7) In determining the resources available to the offender’s family under paragraph (6)($b$), any training premium or top-up payment paid pursuant to the Employment and Training Act 1973 shall be disregarded.

\subsection[12. Circumstances in which a joint-claim jobseeker’s allowance is payable where a joint-claim couple is a couple in hardship]{Circumstances in which a joint-claim jobseeker’s allowance is payable where a joint-claim couple is a couple in hardship}

12.---(1)  This regulation applies where a joint-claim couple is a couple in hardship within the meaning of regulation 11(2) and is subject to the provisions of regulations 14 and~15.

(2) A joint-claim jobseeker’s allowance shall be payable to a couple in hardship even though section~8(2) of the Act prevents payment of a joint-claim jobseeker’s allowance to the couple or section~8(3) of the Act reduces the amount of a joint-claim jobseeker’s allowance payable to the couple but the allowance shall be payable under this paragraph only if and for so long as—
\begin{enumerate}\item[]
($a$) the joint-claim couple satisfy the other conditions of entitlement to a joint-claim jobseeker’s allowance; or

($b$) one member satisfies those conditions and the other member comes within any paragraph in Schedule A1 to the Jobseeker’s Allowance Regulations (categories of members not required to satisfy conditions in section~1(2B)($b$)  of the Jobseekers Act).
\end{enumerate}

\subsection[13. Further circumstances in which a joint-claim jobseeker’s allowance is payable to a couple in hardship]{Further circumstances in which a joint-claim jobseeker’s allowance is payable to a couple in hardship}

13.---(1)  This regulation applies to a couple in hardship falling within regulation 11(3) and is subject to the provisions of regulations 14 and~15.

(2) A joint-claim jobseeker’s allowance shall be payable to a couple in hardship even though section~8(2) of the Act prevents payment of a joint-claim jobseeker’s allowance to the couple or section~8(3) of the Act reduces the amount of a joint-claim jobseeker’s allowance payable to the couple but the allowance—
\begin{enumerate}\item[]
($a$) shall not be payable under this paragraph in respect of the first 14 days of the prescribed period; and

($b$) shall be payable thereafter only where the conditions of entitlement to a joint-claim jobseeker’s allowance are satisfied or where one member satisfies those conditions and the other member comes within any paragraph in Schedule A1 to the Jobseeker’s Allowance Regulations (categories of members not required to satisfy conditions in section~1(2B)($b$)  of the Jobseekers Act).
\end{enumerate}

\subsection[14. Conditions for payment of a joint-claim jobseeker’s allowance]{\sloppy Conditions for payment of a joint-claim jobseeker’s allowance}

14.---(1)  A joint-claim jobseeker’s allowance shall not be payable in accordance with regulation 12 or 13 except where either member of the couple has—
\begin{enumerate}\item[]
($a$) furnished on a form approved for the purpose by the Secretary of State or in such other form as he may in any particular case approve, a statement of the circumstances he relies upon to establish entitlement under regulation 11(2) or, as the case may be, 11(3); and

($b$) signed the statement.
\end{enumerate}

(2) The completed and signed form shall be delivered by a member of the couple to such office as the Secretary of State may specify.

\subsection[15. Provision of information]{Provision of information}

15.  For the purposes of section~8(4)($b$)  of the Act, a member of the couple shall provide to the Secretary of State information as to the circumstances of the alleged hardship of the couple.

\subsection[16. Applicable amount of joint-claim couple in hardship cases]{Applicable amount of joint-claim couple in hardship cases}

16.---(1)  The weekly applicable amount of a couple to whom a joint-claim jobseeker’s allowance is payable in accordance with this Part shall be reduced by a sum equivalent to 40 per cent.\ or, in a case where a member of the joint-claim couple is either pregnant or seriously ill or where a member of the joint-claim couple is a member of a polygamous marriage and one of those members is either pregnant or seriously ill, 20 per cent.\ of the following amount—
\begin{enumerate}\item[]
($a$) where one member of the joint-claim couple or of the polygamous marriage is aged not less than 18 but less than 25 and the other member or, in the case of a polygamous marriage, each other member, is a person under 18 to whom section 3A(1)($e$)(ii)  of the Jobseekers Act applies or is not subject to a direction under section~16 of that Act, the amount specified in paragraph 1(1)($d$)  of Schedule 1 to the Jobseeker’s Allowance Regulations;

($b$) where one member of the joint-claim couple or at least one member of the polygamous marriage (other than a member of a couple or polygamous marriage to whom sub-paragraph ($a$)  applies) is aged not less than 18, the amount specified in paragraph 1(1)($e$)  of Schedule 1 to the Jobseeker’s Allowance Regulations.
\end{enumerate}

(2) A reduction under paragraph (1) shall, if it is not a multiple of 5p, be rounded to the nearest such multiple or, if it is a multiple of 2$.$5p but not of 5p, to the next lower multiple of 5p.

\section[Part V --- Housing benefit and council tax benefit]{Part V\\*Housing benefit and council tax benefit}

\renewcommand\parthead{--- Part V}

\subsection[17. Circumstances where a reduced amount of housing benefit and council tax benefit is payable]{Circumstances where a reduced amount of housing benefit and council tax benefit is payable}

17.---(1)  Subject to regulation 18, any payment of housing benefit or, as the case may be, council tax benefit which falls to be made to an offender in respect of any week in the disqualification period or to an offender’s family member in respect of any week in the relevant period shall be reduced—
\begin{enumerate}\item[]
($a$) where the claimant or a member of his family is pregnant or seriously ill, by a sum equivalent to 20 per cent.;

($b$) in any other case, by a sum equivalent to 40 per cent.,
\end{enumerate}
of the amount which is or, where he is not the claimant or is not single, would be applicable to the offender in respect of a single claimant for those benefits on the first day of the disqualification period or, where the payment falls to be made to an offender’s family member, on the first day of the relevant period and specified in paragraph 1(1) of Schedule 2 to the Housing Benefit Regulations or, as the case may be, in paragraph 1(1) of Schedule 1 to the Council Tax Benefit Regulations.

(2) A reduction under paragraph (1) shall, if it is not a multiple of 5p, be rounded to the nearest such multiple or, if it is a multiple of 2$.$5p but not of 5p, to the next lower multiple of 5p.

(3) Where the rate of housing benefit or council tax benefit payable to a claimant changes, the rules set out above for a reduction in the benefit payable shall be applied to the new rates and any adjustment to the reduction shall take effect from the beginning of the first benefit week to commence for the claimant following the change and in this paragraph “benefit week” shall have the same meaning as in regulation~2(1) of the Housing Benefit Regulations or, as the case may be, regulation~2(1) of the Council Tax Benefit Regulations.

\subsection[18. Circumstances where housing benefit and council tax benefit is payable]{Circumstances where housing benefit and council tax benefit is payable}

18.  Regulation 17 shall not apply and housing benefit or, as the case may be, council tax benefit shall be payable to an offender or to an offender’s family member—
\begin{enumerate}\item[]
($a$) where the offender is the claimant, he is entitled to either of those benefits during the disqualification period;

($b$) where the offender’s family member is the claimant, he is entitled to either of those benefits during the relevant period,
\end{enumerate}
and the claimant is, at the same time, also entitled to income support or to an income-based jobseeker’s allowance.

\section[Part VI --- Deductions from benefits and disqualifying benefits]{Part VI\\*Deductions from benefits and disqualifying benefits}

\renewcommand\parthead{--- Part VI}

\subsection[19. Social security benefits not to be sanctionable benefits]{Social security benefits not to be sanctionable benefits}

19.  The following social security benefits are to be treated as a disqualifying benefit but not a sanctionable benefit—
\begin{enumerate}\item[]
($a$) constant attendance allowance payable under article~14 of the Naval, Military and Air Forces Etc.\ (Disablement and Death) Service Pensions Order 1983\footnote{S.I.~1983/883.} (“the Order”) or article~14 or 43 of the Personal Injuries (Civilians) Scheme 1983\footnote{S.I.~1983/686.} (“the Scheme”);

($b$) exceptionally severe disablement allowance payable under article~15 of the Order or article~15 or 44 of the Scheme;

($c$) mobility supplement payable under article 26A of the Order or article 25A or 48A of the Scheme;

($d$) constant attendance allowance and exceptionally severe disablement allowance, payable under sections 104 and~105 respectively of the Benefits Act where a disablement pension is payable under section~103 of that Act; and

($e$) a bereavement payment payable under section 36 of the Benefits Act\footnote{Section 36 was substituted by section 54(1) of the Welfare Reform and Pensions Act 1999 (c.~30).}.
\end{enumerate}

\subsection[20. Deductions from benefits]{Deductions from benefits}

20.  Any restriction in section~7, 8 or 9 of the Act shall not apply in relation to payments of benefit to the extent of any deduction from the payments which falls to be made under regulations made under section 5(1)($p$)  of the Social Security Administration Act 1992 for, or in place of, child support maintenance and for this purpose, “child support maintenance” means such maintenance which is payable under the Child Support Act 1991\footnote{1991 c.~48. Section 43 of that Act, which permits deduction in connection with child support maintenance using the powers in section 5 of the Social Security Administration Act 1992, is substituted by section 21 of the Child Support, Pensions and Social Security Act 2000 (c.~19). The Regulations are S.I.~1987/1968 and relevant amending instruments are S.I.~1988/522 and 725, 1992/1026 and 2001/18.}.

\vfill\eject

\section[Part VII --- Other amendments]{Part VII\\*Other amendments}

\renewcommand\parthead{--- Part VII}

\subsection[21. Amendment of the Social Security and Child Support (Decisions and Appeals) Regulations 1999]{Amendment of the Social Security and Child Support (Decisions and Appeals) Regulations 1999}

21.  In Schedule 2 to the Social Security and Child Support (Decisions and Appeals) Regulations 1999\footnote{S.I.~1999/991. Relevant amending instruments are S.I.~1999/3178, 2000/897 and 1596.} (decisions against which no appeal lies), after paragraph 26, there shall be added the following paragraph—
\begin{quotation}
\subsection*{\itshape “Loss of benefit}

27.  A decision of the Secretary of State that a sanctionable benefit as defined in section~7(8) of the Social Security Fraud Act 2001 is not payable, or is to be reduced, pursuant to section~7, 8 or 9 of that Act as a result of convictions for one or more benefit offences in each of two separate sets of proceedings, one offence being committed within 3 years of conviction for another, where the only ground of appeal is that any of the convictions was erroneous.”.
\end{quotation}

\subsection[22. Amendment of the Pensions Appeal Tribunals (Additional Rights of Appeal) Regulations 2001]{Amendment of the Pensions Appeal Tribunals (Additional Rights of Appeal) Regulations 2001}

22.  In the Pensions Appeal Tribunals (Additional Rights of Appeal) Regulations 2001\footnote{S.I.~2001/1031.} after regulation 4 there shall be added the following regulation—
\begin{quotation}
\subsection*{\itshape “Loss of benefit decisions}

5.  A decision of the Secretary of State pursuant to section~7 of the Social Security Fraud Act 2001 (loss of benefit for commission of benefit offences) that a war pension within the meaning of that Act is not payable shall be a specified decision.”.
\end{quotation}

\bigskip

%Signed 
%by authority of the 
%Secretary of State for Work and Pensions.

{\raggedleft
\emph{Alistair Darling}\\*
One of Her Majesty's Principal Secretaries of State
%Minister
%%Parliamentary Under-Secretary 
%of State,\\*Department for Work and Pensions

}

18th December 2001

\small

\part{Explanatory Note}

\renewcommand\parthead{— Explanatory Note}

\subsection*{(This note is not part of the Regulations)}

These Regulations are made by virtue of, or in consequence of, sections~7 to 13 of the Social Security Fraud Act 2001 (c.~11) (“the Act”) and relate to restrictions in payment of certain benefits which apply where a person has been convicted of one or more benefit offences in each of two separate proceedings and one offence is committed within three years of the conviction for another such offence.

The Regulations are made before the end of the period of six months beginning with the coming into force of the relevant provisions in the Act and are therefore exempt from the requirement in section~172(1) of the Social Security Administration Act 1992 (c. 5) to refer proposals to make these Regulations to the Social Security Advisory Committee.

Part I contains provisions relating to citation, commencement and interpretation. The Regulations come into force on 1st April 2002. Regulation 2 prescribes what is to be the disqualification period for the purposes of the imposition of the loss of benefit or reduction in the amount payable.

Part II prescribes what are to be the reductions in income support or joint-claim jobseeker’s allowance when the restrictions apply.

Part III makes provision for an income-based jobseeker’s allowance to be paid where the claimant is a person in hardship and Part IV makes provision for a joint-claim jobseeker’s allowance to be paid where a joint-claim couple are a couple in hardship.

Part V makes provision regarding reductions in housing benefit and council tax benefit during the disqualification period or the relevant period and when those benefits remain payable during those periods.

Part VI prescribes certain benefits which are to be disqualifying but not sanctionable benefits and that the restrictions in sections~7 to 9 of the Act are not to apply to deductions from benefit for, or in place of, child support maintenance.

Part VII amends the Social Security and Child Support (Decisions and Appeals) Regulations 1999 (S.I.~1999/991) so that no appeal to the tribunal lies on the ground that a conviction which led to the restriction was erroneous and the Pensions Appeal Tribunals (Additional Rights of Appeal) Regulations 2001 (S.I.~2001/1031) to give a right of appeal to the Pensions Appeal Tribunal where the restriction affects a war pension.

These Regulations do not impose a charge on business. 

\end{document}