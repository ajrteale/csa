\documentclass[a4paper]{article}

\usepackage[welsh,english]{babel}

\usepackage[utf8]{inputenc}
\usepackage[T1]{fontenc}

\usepackage{textcomp}

%\usepackage[2012rules]{optional}

\usepackage[osf]{mathpazo}

\usepackage{perpage} %the perpage package
\MakePerPage{footnote} %the perpage package command
\renewcommand{\thefootnote}{\fnsymbol{footnote}}

\usepackage[perpage,para,symbol]{footmisc}

%\opt{newrules}{
\title{The Child Support Commissioners (Procedure) (Amendment) Regulations 1996}
%}

%\opt{2012rules}{
%\title{Child Maintenance and Other Payments Act 2008\\(2012 scheme version)}
%}

\author{S.I. 1996 No. 243}

\date{Made 7th February 1996\\Laid before Parliament 9th February 1996\\Coming into force 1st March 1996
}

%\opt{oldrules}{\newcommand\versionyear{1993}}
%\opt{newrules}{\newcommand\versionyear{2003}}
%\opt{2012rules}{\newcommand\versionyear{2012}}

\usepackage{fancyhdr}
\pagestyle{fancy}
\fancyhead[L]{}
\fancyhead[C]{\itshape The Child Support Commissioners (Procedure) (Amendment) Regulations 1996 (S.I.~1996/243) \parthead%\phantom{...}% (\versionyear{} scheme version)
}
\fancyhead[R]{}
\fancyfoot[C]{\thepage}
\newcommand{\parthead}{}

\usepackage{array}
\usepackage{multirow}
\usepackage[debugshow]{tabulary}
\usepackage{longtable}
\usepackage{multicol}
\usepackage{lettrine}

\usepackage[colorlinks=true]{hyperref}
\usepackage{microtype}

\hyphenation{Aw-dur-dod}
\hyphenation{bank-rupt-cy}
\hyphenation{Ec-cles-ton}
\hyphenation{Eux-ton}
\hyphenation{Hogh-ton}
\hyphenation{Pres-ton}
\hyphenation{Pru-den-tial}
\hyphenation{Riv-ing-ton}

\newcolumntype{x}[1]
	{>{\raggedright}p{#1}}
\newcommand{\tn}{\tabularnewline}
\setlength\tymin{50pt}

\newcommand\amendment[1]{\subsubsection*{Notes}{\itshape\frenchspacing\footnotesize #1 \par}}

\setlength\headheight{22.87003pt}

\begin{document}

\maketitle

\begin{center}\small\noindent\itshape
This Statutory Instrument has been made in consequence of a defect in S.I. 1995/2907 and is being issued free of charge to all known recipients of that Statutory Instrument.
\end{center}

\noindent
The Lord Chancellor, in exercise of the powers conferred by sections 22(3), 24(6) and (7) and 25(2), (3) and (5) of, and paragraph 4A of Schedule 4 to, the Child Support Act 1991\footnote{\frenchspacing 1991 c. 48; paragraph 4A was inserted in Schedule 4 by the Child Support Act 1995 (c. 34), section 17(1).}, after consultation with the Lord Advocate and, in accordance with section 8 of the Tribunals and Inquiries Act 1992\footnote{\frenchspacing 1992 c. 53.}, with the Council on Tribunals, hereby makes the following Regulations:

{\sloppy

\tableofcontents

}

\setcounter{secnumdepth}{-2}

\bigskip

1.  These Regulations may be cited as the Child Support Commissioners (Procedure) (Amendment) Regulations 1996 and shall come into force on 1st March 1996.

\medskip

2.  After regulation 23 of the Child Support Commissioners (Procedure) Regulations 1992\footnote{\frenchspacing S.I. 1992/2640.}, there shall be inserted the following regulation—
\begin{quotation}
\subsection*{“Delegation of functions to nominated officers}

23A.—(1) All or any of the following functions of a Commissioner may be exercised by a nominated officer, that is to say:
\begin{enumerate}\item[]
($a$) giving directions under regulation 7(1) and (2) (directions on notice of appeal);

($b$) granting leave under regulation 9 to the Secretary of State to intervene in an appeal;

($c$) making any direction under regulation 10(1), (2) and (3) (other directions);

($d$) making orders for oral hearings under regulation 11(2) and (3);

($e$) summoning witnesses under regulation 14(1) and (2) and setting aside under regulation 14(3) a witness summons made by a nominated officer;

($f$) ordering the postponement of oral hearings under regulation 15(1);

($g$) giving leave for the withdrawal of any appeal under regulation 16(2);

($h$) making any order for the extension or abridgement of time, or for expediting the proceedings, under regulation 23(2)($a$), ($b$) and ($c$).
\end{enumerate}

(2) Any party may, within 10 days of being given the decision of the nominated officer, in writing request a Commissioner to consider, and confirm or replace with his own, that decision, but such a request shall not stop the proceedings unless so ordered by the Commissioner.

(3) In this regulation, “nominated officer” means an officer authorised by the Lord Chancellor (or, in Scotland, by the Secretary of State) in accordance with paragraph 4A of Schedule 4 to the Act.”.
\end{quotation}

\medskip

3.  The Child Support Commissioners (Procedure) (Amendment) Regulations 1995\footnote{\frenchspacing S.I. 1995/2907.} are hereby revoked.


\bigskip

%Signed by authority of the Secretary of State for Social Security.

{\raggedleft
\emph{Mackay of Clashfern, C.}%\\*Parliamentary Under-Secretary of State,\\*Department of Social Security

}

7th February 1996

\part{Explanatory Note}

\renewcommand\parthead{--- Explanatory Note}

\subsection*{(This note is not part of the Regulations)}

These Regulations amend the Child Support Commissioners (Procedure) Regulations 1992 to provide for nominated officers to perform certain functions of a Commissioner and to provide for decisions made by a nominated officer to be considered by a Commissioner. They revoke similar Regulations made in 1995 which were defective.

\end{document}