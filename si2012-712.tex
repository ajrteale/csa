\documentclass[12pt,a4paper]{article}

\newcommand\regstitle{The Child Support (Miscellaneous Amendments) Regulations 2012}

\newcommand\regsnumber{2012/712}

%\opt{newrules}{
\title{\regstitle}
%}

%\opt{2012rules}{
%\title{Child Maintenance and~Other Payments Act 2008\\(2012 scheme version)}
%}

\author{S.I.\ 2012 No.\ 712}

\date{Made
5th March 2012\\
Laid before Parliament
8th March 2012\\
Coming into~force
30th April 2012
}

%\opt{oldrules}{\newcommand\versionyear{1993}}
%\opt{newrules}{\newcommand\versionyear{2003}}
%\opt{2012rules}{\newcommand\versionyear{2012}}

\usepackage{csa-regs}

\setlength\headheight{27.61603pt}

%\hbadness=10000

\begin{document}

\maketitle

\noindent
The Secretary of State for Work and Pensions makes the following Regulations in exercise of the powers conferred by sections 17(3) and (5), 29(2) and (3), 51(1) and (2)($i$), 52(4) and 54 of, and paragraphs 10(1) and (2) of Schedule~1 to, the Child Support Act 1991\footnote{1991 c.~48. Sections 17(3) was substituted by section 41 of the Social Security Act 1998 (c.~14) and further substituted by section 9 of the Child Support, Pensions and Social Security Act 2000 (c.~19) (“the 2000 Act”) in relation to cases other than 1993 scheme cases (a “1993 scheme case” means a case in respect of which the provisions of the 2000 Act have not been brought in to force in accordance with article 3 of the Child Support, Pensions and Social Security Act 2000 (Commencement No.~12) Order 2003 (C.~11)). Section 54 is cited for the meaning given to the word “prescribed”.}. 

{\sloppy

\tableofcontents

}

\bigskip

\setcounter{secnumdepth}{-2}

\subsection[1. Citation, commencement and interpretation]{Citation, commencement and interpretation}

1.---(1)  These Regulations may be cited as the Child Support (Miscellaneous Amendments) Regulations 2012 and shall come into force on 30th April 2012.

(2) In these Regulations—
\begin{enumerate}\item[]
“the Collection and Enforcement Regulations” means the Child Support (Collection and Enforcement) Regulations 1992\footnote{S.I.~1992/1989. References to the Secretary of State in these Regulations are treated as references to the Commission by virtue of paragraph 55(3) of Schedule 3 to the Child Maintenance and Other Payments Act 2008 (c.~6) (“the 2008 Act”), as the functions of the Secretary of State were transferred to the Commission by section 13 of that Act.};

“the Maintenance Assessment Procedure Regulations” means the Child Support (Maintenance Assessment Procedure) Regulations 1992\footnote{S.I.~1992/1813. References to the Secretary of State in these Regulations are treated as references to the Commission by virtue of paragraph 55(3) of Schedule 3 to the 2008 Act, as the functions of the Secretary of State were transferred to the Commission by section 13 of that Act.};

“the Maintenance Assessment and Special Cases Regulations” means the Child Support (Maintenance Assessments and Special Cases) Regulations 1992\footnote{S.I.~1992/1815. References to the Secretary of State in these Regulations are treated as references to the Commission by virtue of paragraph 55(3) of Schedule 3 to the 2008 Act, as the functions of the Secretary of State were transferred to the Commission by section 13 of that Act.};

“the Maintenance Calculations and Special Cases Regulations” means the Child Support (Maintenance Calculations and Special Cases) Regulations 2000\footnote{S.I.~2001/155. References to the Secretary of State in these Regulations are treated as references to the Commission by virtue of paragraph 55(3) of Schedule 3 to the 2008 Act, as the functions of the Secretary of State were transferred to the Commission by section 13 of that Act.}; and

“the Management of Payments and Arrears Regulations” means the Child Support (Management of Payments and Arrears) Regulations 2009\footnote{S.I.~2009/3151.}.
\end{enumerate}

\subsection[2. Amendment of Regulation 5 of the Collection and Enforcement Regulations]{Amendment of Regulation 5 of the Collection and Enforcement Regulations}

2.  For regulation 5(1) of the Collection and Enforcement Regulations (transmission of payments) substitute---
\begin{quotation}
“(1) Payments of child support maintenance made through the Secretary of State or other specified person shall be transmitted to the person entitled to receive them---
\begin{enumerate}\item[]
($a$) by transfer of credit to an account nominated by the person entitled to receive the payments; or

($b$) by means other than by transfer of credit as determined by the Secretary of State, where it appears to the Secretary of State to be necessary to do so in the circumstances of the particular case.”.
\end{enumerate}
\end{quotation}

\subsection[3. Amendments of Regulation 3 of the Management of Payments and Arrears Regulations]{Amendments of Regulation 3 of the Management of Payments and Arrears Regulations}

3.  In regulation 3(3) of the Management of Payments and Arrears Regulations (arrears notices), for sub-paragraph~($a$)  substitute---
\begin{quotation}
“($a$) include the amount of all outstanding arrears of child support maintenance due and not paid;”.
\end{quotation}

\subsection[4. Amendment of the Maintenance Assessment Procedure Regulations]{Amendment of the Maintenance Assessment Procedure Regulations}

4.  In regulation 23\footnote{Regulation 23 was substituted by S.I.~1999/1047 and revoked with savings by S.I.~2001/157 (as amended by S.I.~2003/328 and 347 and 2004/2415). The relevant amending instruments are S.I.~2000/1596, 2003/1050, 2005/785, 2008/2543 and 2683, 2009/2909 and 2011/1464.} of the Maintenance Assessment Procedure Regulations (date from which a decision is superseded), for paragraph~(19) substitute—
\begin{quotation}
“(19) Where a superseding decision is made in a case to which regulation 20(2)($a$)  or (3) applies and the material circumstance is---
\begin{enumerate}\item[]
($a$) a qualifying child dies or ceases to be a qualifying child;

($b$) a relevant child dies or ceases to be a relevant child; or

($c$) a child who is a member of the family of the absent parent for the purposes of regulation 11(1)($g$)  of the Child Support (Maintenance Assessments and Special Cases) Regulations 1992, dies or ceases to be a member of the family of the absent parent for those purposes,
\end{enumerate}
the decision shall take effect as from the first day of the maintenance period in which the change occurred.”.
\end{quotation}

\subsection[5. Amendment of the Maintenance Assessments and Special Cases Regulations]{Amendment of the Maintenance Assessments and Special Cases Regulations}

5.  In Schedule~1 to the Maintenance Assessments and Special Cases Regulations (calculation of $N$ and $M$---earnings), after paragraph 5A, insert—
\begin{quotation}
\subsection*{“Chapter III\\*Estimate of earnings where insufficient information available}

5B.---(1)  Where the Commission is calculating earnings of an employed earner or a self-employed earner under Part~I of Schedule~1 and the information available in relation to those earnings is insufficient or unreliable, the Commission may estimate those earnings and, in doing so, may make any assumptions as to any fact.

(2) Where the Commission is satisfied that the person is engaged in a particular occupation, whether as an employee or a self-employed person, the assumptions referred to in sub-paragraph~(1) may include an assumption that the person has the average weekly earnings of a person engaged in that occupation in the United Kingdom or in any part of the United Kingdom.”.
\end{quotation}

\subsection[6. Amendment of the Maintenance Calculations and Special Cases Regulations]{Amendment of the Maintenance Calculations and Special Cases Regulations}

6.---(1)  The Maintenance Calculations and Special Cases Regulations are amended as follows.

(2) In regulation 1(2) (interpretation), in the definition of “employed earner”---
\begin{enumerate}\item[]
($a$) in sub-paragraph~($a$)  omit “and”; and

($b$) at the end of sub-paragraph~($b$)  add—
\begin{quotation}
“and

a person gainfully employed outside the United Kingdom if the person’s income from that employment is chargeable to tax under the Income Tax (Earnings and Pensions) Act 2003 or would be were it not for any double taxation arrangements under Part~II of the Taxation (International and Other Provisions) Act 2010.”.
\end{quotation}
\end{enumerate}

(3) In regulation 1(2), for the definition of “self-employed earner” substitute—
\begin{quotation}
\begin{sloppypar}
““self-employed earner” has the same meaning as in section~2(1)($b$)  of the Contributions and Benefits Act except that it includes a person gainfully employed otherwise than in employed earner’s employment (whether or not he is also employed in such employment)---
\end{sloppypar}
\begin{enumerate}\item[]
($a$) 
in Northern Ireland; or

($b$) 
outside the United Kingdom if the person’s income from that gainful employment is chargeable to tax under the Income Tax (Trading and Other Income) Act 2005 or would be were it not for any double taxation arrangements made under Part~II of the Taxation (International and Other Provisions) Act 2010.”.
\end{enumerate}
\end{quotation}

(4) In the Schedule (net weekly income), in paragraph 5(2), after “For the purposes of sub-paragraph~(1)($a$),” insert “except for cases falling within sub-paragraph~(3),”.

(5) In the Schedule, after paragraph 5(2), insert the following sub-paragraphs—
\begin{quotation}
“(3) For the purposes of sub-paragraph~(1)($a$), where an employed earner is gainfully employed outside the United Kingdom, amounts deducted by way of income tax shall be---
\begin{enumerate}\item[]
($a$) the amounts actually deducted in respect of income tax applicable to the income in question, whether that is paid in full in Great Britain or outside Great Britain, or partly paid both in Great Britain and outside Great Britain; or

($b$) where insufficient or unreliable evidence or information is provided by the non-resident parent as to the actual amounts deducted, the amounts that would have been deducted had that employed earner been gainfully employed in Great Britain.
\end{enumerate}

(4) For the purposes of sub-paragraph~(1)($b$), where an employed earner is gainfully employed outside the United Kingdom, amounts deducted by way of primary Class 1 contributions\footnote{Primary Class 1 contributions are defined for Great Britain in Part~I of the Social Security Contributions and Benefits Act 1992 (c.~4) and for Northern Ireland in the Social Security Contributions and Benefits (Northern Ireland) Act 1992 (c.~7). These are payments for National Insurance paid by those that are employed.} shall be the amounts actually deducted under the Contributions and Benefits Act or under the Contributions and Benefits (Northern Ireland) Act and amounts actually deducted outside the United Kingdom for payments of a similar nature.”.
\end{quotation}

(6) In the Schedule, after paragraph 6, insert—
\begin{quotation}
\subsection*{\sloppy\itshape “Estimate of net weekly income of employed earner where insufficient information available}

6A.---(1)  Where the Commission is calculating net weekly income of an employed earner under Part II of the Schedule and the information available in relation to that income is insufficient or unreliable, the Commission may estimate that income and, in doing so, may make any assumptions as to any fact.

(2) Where the Commission is satisfied that the non-resident parent is engaged in a particular occupation as an employee, the assumptions referred to in sub-paragraph~(1) may include an assumption that the non-resident parent has the average net weekly income of a person engaged in that occupation in the United Kingdom or any part of the United Kingdom.”.
\end{quotation}

(7) In the Schedule, after paragraph 9, insert—
\begin{quotation}
\subsection*{\sloppy\itshape “Estimate of net weekly income of self-\hspace{0pt}employed earner where insufficient information available}

9A.---(1)  Where the Commission is calculating net weekly income of a self-employed earner under Part III of the Schedule and the information available in relation to that income is insufficient or unreliable, the Commission may estimate that income and, in doing so, may make any assumptions as to any fact.

(2) Where the Commission is satisfied that the non-resident parent is engaged in a particular occupation as a self-employed earner, the assumptions referred to in sub-paragraph~(1) may include an assumption that the non-resident parent has the average net weekly income of a person engaged in that occupation in the United Kingdom or any part of the United Kingdom.”.
\end{quotation}

\bigskip

\pagebreak[3]

Signed 
by authority of the 
Secretary of State for~Work and~Pensions.
%I concur
%By authority of the Lord Chancellor

{\raggedleft
\emph{Maria Miller}\\*
%Secretary
%Minister
Parliamentary Under-Secretary 
of State\\*Department 
for~Work and~Pensions

}

5th March 2012

\small

\part{Explanatory Note}

\renewcommand\parthead{— Explanatory Note}

\subsection*{(This note is not part of the Regulations)}

These Regulations are made under powers in the Child Support Act 1991 (c.48) (“the 1991 Act”) and come into force on 30th April 2012. They amend the Child Support (Collection and Enforcement) Regulations 1992 (“the Collection and Enforcement Regulations”), the Child Support (Maintenance Assessment Procedure) Regulations 1992 (“the Maintenance Assessment Procedure Regulations”), the Child Support (Maintenance Assessment and Special Cases) Regulations 1992 (“the Maintenance Assessment and Special Cases Regulations”), the Child Support (Maintenance Calculations and Special Cases) Regulations 2000 (“the Maintenance Calculations and Special Cases Regulations”) and the Child Support (Management of Payments and Arrears) Regulations 2009 (“the Management of Payments and Arrears Regulations”).

Regulation 2 amends the Collection and Enforcement Regulations. This regulation substitutes regulation 5(1) of those Regulations to provide that payments of child support maintenance made through the Secretary of State shall be made to the person entitled to receive those payments by transfer of credit. Other methods of payment will only be used where it appears to the Secretary of State necessary in the circumstances of the particular case.

Regulation 3 amends the Management of Payments and Arrears Regulations. Regulation 3 of those Regulations makes provision with respect to arrears notices. This regulation removes the requirement to itemise the payments of child support maintenance due and not paid and replaces this with a requirement that any arrears notice must include the amount of all outstanding arrears of child support maintenance due and not paid.

Regulation 4 amends the Maintenance Assessment Procedure Regulations. This substitutes a new paragraph~(19) of regulation 23 of those Regulations (date from which a decision is superseded). It provides an effective date for a superseding decision made in a case to which regulation 20(2)($a$)  or (3) of those Regulations applies (that is, where the Secretary of State is satisfied that the decision is one in respect of which there has been a material change of circumstances since the decision was made) and the material change of circumstance is any of the following, namely a qualifying child dies or ceases to be a qualifying child; a relevant child dies or ceases to be a relevant child; or a child, who is a member of the family of the absent parent, dies or ceases to be a member of the family of the absent parent.

Regulation 5 amends Schedule~1 to the Maintenance Assessment and Special Cases Regulations. This regulation inserts a new paragraph 5B into Schedule~1. This provides that where the Commission is calculating earnings of an employed earner or self-employed earner and the information in relation to those earnings is insufficient or unreliable, the Commission may estimate those earnings and in doing so may make any assumption as to any fact.

Regulation 6 amends the Maintenance Calculations and Special Cases Regulations. First, it amends the definitions of “employed earner” and “self-employed earner” in regulation 1(2) of those Regulations to include a person gainfully employed outside the United Kingdom in certain specified circumstances. Secondly, it amends the Schedule to those Regulations to make similar provision to that made by regulation 5 of these Regulations (estimation of net weekly income where insufficient information available).

A full impact assessment has not been produced for this instrument as it has no impact on the private sector and civil society organisations. 

\end{document}