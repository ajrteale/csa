\documentclass[12pt,a4paper]{article}

\newcommand\regstitle{The Social Security and Child Support (Supersession of 
Appeal Decisions) Regulations 2012}

\newcommand\regsnumber{2012/1267}

%\opt{newrules}{
\title{\regstitle}
%}

%\opt{2012rules}{
%\title{Child Maintenance and~Other Payments Act 2008\\(2012 scheme version)}
%}

\author{S.I.\ 2012 No.\ 1267}

\date{Made
8th May 2012\\
Laid before Parliament
14th May 2012\\
Coming into~force
4th June 2012
}

%\opt{oldrules}{\newcommand\versionyear{1993}}
%\opt{newrules}{\newcommand\versionyear{2003}}
%\opt{2012rules}{\newcommand\versionyear{2012}}

\usepackage{csa-regs}

\setlength\headheight{27.61603pt}

%\hbadness=10000

\begin{document}

\maketitle

\begin{center}\itshape 
This Statutory Instrument has been printed to correct errors in S.I.~2008/2683 and is being issued free of charge to all known recipients of that Statutory Instrument 
\end{center}

\noindent
The Secretary of State for Work and Pensions makes the following Regulations in exercise of the powers conferred by sections 17(3) and (5) and 54 of the Child Support Act 1991\footnote{1991 c.~48 (“the 1991 Act”). Section 54 is cited for the meaning of “prescribe”.}, sections 10(3) and (6) and 84 of the Social Security Act 1998\footnote{1998 c.~14. (“the 1998 Act”). Section 84 is cited for the meaning of “prescribe”.}, paragraphs 4(4) and (6) and 23(1) of Schedule 7 to the Child Support, Pensions and Social Security Act 2000\footnote{2000 c.~19 (“the 2000 Act”). Paragraph 23 is cited for the meaning of “prescribe”.} and section~103(2)($b$)  of the Welfare Reform Act 2012\footnote{2012 c.~5 (“the Welfare Reform Act”).}.

In respect of provisions in these Regulations relating to housing benefit and council tax benefit, organisations appearing to the Secretary of State to be representative of the authorities concerned have agreed that consultations need not be undertaken\footnote{\emph{See} section~176(2) of the Social Security Administration Act 1992 (c.~5) (“the 1992 Act”).}.

This Instrument contains only regulations made by virtue of, or consequential upon, section~103 of the Welfare Reform Act 2012 and is made before the end of the period of 6 months beginning with the coming into force of that section\footnote{\emph{See} section~173(5) of the 1992 Act.}. 

{\sloppy

\tableofcontents

}

\bigskip

\setcounter{secnumdepth}{-2}

\subsection[1. Citation, commencement and effect]{Citation, commencement and effect}

1.---(1)  These Regulations may be cited as the Social Security and Child Support (Supersession of Appeal Decisions) Regulations 2012 and come into force on 4th June 2012.

(2) These Regulations have effect as if they had come into force on 3rd November 2008.

\subsection[2. Amendment of the Child Support (Maintenance Assessment Procedure) Regulations 1992]{Amendment of the Child Support (Maintenance Assessment Procedure) Regulations 1992}

2.---(1)  The Child Support (Maintenance Assessment Procedure) Regulations 1992\footnote{S.I.~1992/1813. Revoked, with savings for certain purposes, by S.I.~2001/157. \emph{See} section~17(6) of the 1991 Act, as inserted by paragraph 2 of Schedule 12 to, the Welfare Reform Act for the meaning of “appeal tribunal” and “Child Support Commissioner”.} are amended as follows.

(2) In regulation 20(4A) (supersession of decisions)\footnote{Regulation 20(4A) was inserted by S.I.~2003/1050 and amended by paragraphs 60 and 61 of Schedule 1 to S.I.~2008/2683.} for “the First-tier Tribunal or Upper Tribunal” substitute, “an appeal tribunal, the First-tier Tribunal, the Upper Tribunal or of a Child Support Commissioner”.

(3) In regulation 23 (date from which a decision is superseded)—
\begin{enumerate}\item[]
($a$) in paragraph (10)—
\begin{enumerate}\item[]
(i) in sub-paragraph ($a$), for “the First-Tier Tribunal under section~20 of the Act or the Upper Tribunal” substitute “an appeal tribunal or the First-tier Tribunal under section~20 of the Act or the Upper Tribunal or a Child Support Commissioner”;

(ii) for “the First-tier Tribunal or, as the case may be, the Upper Tribunal” substitute “an appeal tribunal, the First-tier Tribunal, the Upper Tribunal or the Child Support Commissioner”.
\end{enumerate}

($b$) in paragraph (20)\footnote{Regulation 23(20) was also inserted by S.I.~2003/1050 and amended by paragraphs 60 and 61 of Schedule 1 to S.I.~2008/2683.}—
\begin{enumerate}\item[]
(i) for “First-tier Tribunal or the Upper Tribunal’s decision” substitute, “the decision of the appeal tribunal, the First-tier Tribunal, the Upper Tribunal or the Child Support Commissioner”;

(ii) after “the Upper Tribunal” insert “or the Child Support Commissioner”.
\end{enumerate}
\end{enumerate}

\subsection[3. Amendment of the Child Support Departure Direction and Consequential Amendments Regulations 1996]{Amendment of the Child Support Departure Direction and Consequential Amendments Regulations 1996}

3.  In regulation 32E(6)($a$)  (date from which a superseding decision takes effect)\footnote{Regulation 32E was inserted by regulation 44 of S.I.~1999/1047. Regulation 32E(6)($a$)  was amended by paragraph 78 of Schedule 1 to S.I.~2008/2683.} of the Child Support Departure Direction and Consequential Amendments Regulations 1996\footnote{S.I.~1996/2907. Revoked, with savings for certain purposes, by S.I.~2001/156 as amended by S.I.~2003/347.}, before “the First-tier Tribunal” insert “an appeal tribunal or”.

\subsection[4. Amendment of the Social Security and Child Support (Decisions and Appeals) Regulations 1999]{Amendment of the Social Security and Child Support (Decisions and Appeals) Regulations 1999}

4.---(1)  The Social Security and Child Support (Decisions and Appeals) Regulations 1999\footnote{S.I.~1999/991. Relevant amending instruments are S.I.~S.I.~2000/1596, S.I.~2000/3185, S.I.~2000/1596, S.I.~2003/916, S.I.~2003/1050, 2008/2683 and S.I.~2009/396. See section~17(7) of the 1998 Act, as inserted by paragraph 4 of Schedule 12 to, the Welfare Reform Act for the meaning of “appeal tribunal” and “Commissioner”.} are amended as follows.

(2) In regulation 6 (supersession of decisions)—
\begin{enumerate}\item[]
($a$) in paragraph (2)($c$)  for “the First-tier Tribunal or of the Upper Tribunal” substitute “an appeal tribunal, the First-tier Tribunal, the Upper Tribunal or of a Commissioner”;

($b$) in sub-paragraph (2)($n$)—
\begin{enumerate}\item[]
(i) before “the First-tier Tribunal” in the first place it occurs insert “an appeal tribunal or”;

(ii) for “decision of the First-tier Tribunal” substitute “decision of an appeal tribunal or the First-tier Tribunal”.
\end{enumerate}
\end{enumerate}

(3) In regulation 6A(4A) (supersession of child support decisions) for “the First-tier Tribunal or of the Upper Tribunal” substitute “an appeal tribunal, the First-tier Tribunal, the Upper Tribunal or of a Child Support Commissioner”\footnote{See section~17(6) of the 1991 Act, as inserted by paragraph 2 of Schedule 12 to, the Welfare Reform Act for the meaning of “Child Support Commissioner”. Relevant amending instruments are S.I.~2000/3185, 2003/1050 and S.I.~2009/396.}.

(4) In regulation 6B(1) (circumstances in which a child support decision may not be superseded) for “the First-tier Tribunal or the Upper Tribunal” substitute “an appeal tribunal, the First-tier Tribunal, the Upper Tribunal or a Child Support Commissioner”.

(5) In regulation 7 (date from which a decision superseded under section~10 takes effect)—
\begin{enumerate}\item[]
($a$) in paragraph (5) for “the First-tier Tribunal or the Upper Tribunal” in both places it occurs substitute “an appeal tribunal, the First-tier Tribunal, the Upper Tribunal or a Commissioner”;

($b$) in paragraph (33)—
\begin{enumerate}\item[]
(i) for “the First-tier Tribunal or the Upper Tribunal’s decision” substitute “the decision of the appeal tribunal, the First-tier Tribunal, the Upper Tribunal or the Commissioner”;

(ii) after “the Upper Tribunal” insert ``or the Commissioner”.
\end{enumerate}
\end{enumerate}

(6) In paragraph 11 of Schedule 3D (effective dates for supersession of decisions of child support decisions)\footnote{Schedule 3D was inserted by S.I.~2009/396.}—
\begin{enumerate}\item[]
($a$) in sub-paragraph ($a$), for “the First-tier Tribunal or the Upper Tribunal” substitute “an appeal tribunal, the First-tier Tribunal, the Upper Tribunal or of a Child Support Commissioner”;

($b$) for “the First-tier Tribunal or, as the case may be, the Upper Tribunal” substitute “an appeal tribunal, the First-tier Tribunal, the Upper Tribunal or a Child Support Commissioner (as the case may be)”.
\end{enumerate}

\subsection[5. Amendment of the Housing and Council Tax Benefit (Decisions and Appeals) Regulations 2001]{Amendment of the Housing and Council Tax Benefit (Decisions and Appeals) Regulations 2001}

5.---(1)  The Housing and Council Tax Benefit (Decisions and Appeals) Regulations 2001\footnote{S.I.~2001/1002. Regulation 7(2)($d$)  was substituted, and regulation 8(11) was inserted, by S.I.~2003/1050. See paragraph 4(6) of Schedule 7 to the 2000 Act, as inserted by paragraph 5 of Schedule 12 to, the Welfare Reform Act for the meaning of “appeal tribunal” and “Commissioner”.} are amended as follows.

(2) In regulation 7(2)($d$)  (decisions superseding earlier decisions) for “the First-tier Tribunal or of the Upper Tribunal” substitute “an appeal tribunal, the First-tier Tribunal, the Upper Tribunal or of a Commissioner”.

(3) In regulation 8 (date from which a decision superseding an earlier decision takes effect)—
\begin{enumerate}\item[]
($a$) in paragraph (7) for “the First-tier Tribunal or of the Upper Tribunal” substitute “an appeal tribunal, the First-tier Tribunal, the Upper Tribunal or of a Commissioner””;

($b$) in paragraph (11)—
\begin{enumerate}\item[]
(i) for “the First-tier Tribunal or the Upper Tribunal’s decision” substitute, “the decision of the appeal tribunal, the First-tier Tribunal, the Upper Tribunal or the Commissioner”;

(ii) after “the Upper Tribunal” insert “, the Commissioner”.
\end{enumerate}
\end{enumerate}

\bigskip

\pagebreak[3]

Signed 
by authority of the 
Secretary of State for~Work and~Pensions.
%I concur
%By authority of the Lord Chancellor

{\raggedleft
\emph{Freud}\\*
%Secretary
%Minister
Parliamentary Under-Secretary 
of State\\*Department 
for~Work and~Pensions

}

8th May 2012

\small

\part{Explanatory Note}

\renewcommand\parthead{— Explanatory Note}

\subsection*{(This note is not part of the Regulations)}

This instrument is made to correct errors in the Tribunals, Courts and Enforcement (Transitional and Consequential Provisions) Order 2008 (S.I.~2008/2683).

The functions of the former appeal tribunals, the Child Support Commissioner and the Social Security Commissioner were transferred to the First-tier Tribunal and the Upper Tribunal on 3rd November 2008.

Whilst the legislation ensured that a decision maker could supersede decisions which had been appealed to the First-tier Tribunal or Upper Tribunal, the need to supersede decisions made under the old appeals system was inadvertently overlooked---references to the former appeal bodies were substituted with references to the new appeal bodies, when the legislation should have retained a reference to both.

Section 103 of the Welfare Reform Act 2012 amends provisions in Acts, with retrospective effect, so as to make it clear that the power to supersede earlier decisions includes decisions made under the old appeals regime as well as the present one.

These Regulations, which are made as a consequence of those amendments, insert references to the former appeal bodies in subordinate legislation relating to the cases or circumstances in which supersession decisions can be made and to the date on which those decisions take effect.

These Regulations have effect as though they came into force on 3rd November 2008. Authority for retrospective provision is conferred by section~103(2)($b$)  of the Welfare Reform Act 2012.

A full impact assessment has not been produced for this instrument as it has no impact on the private sector or civil society organisations. 

\end{document}