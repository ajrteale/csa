\documentclass[12pt,a4paper]{article}

\newcommand\regstitle{The Child Benefit and~Guardian's Allowance (Decisions and~Appeals) Regulations 2003}

\newcommand\regsnumber{2003/916}

%\opt{newrules}{
\title{\regstitle}
%}

%\opt{2012rules}{
%\title{Child Maintenance and~Other Payments Act 2008\\(2012 scheme version)}
%}

\author{S.I.\ 2003 No.\ 916}

\date{Made
27th March 2003\\
%Laid before Parliament
%15th October 2008\\
Coming into~force
7th April 2003
}

%\opt{oldrules}{\newcommand\versionyear{1993}}
%\opt{newrules}{\newcommand\versionyear{2003}}
%\opt{2012rules}{\newcommand\versionyear{2012}}

\usepackage{csa-regs}

\setlength\headheight{42.11603pt}

%\hbadness=10000

\begin{document}

\maketitle

\noindent
Whereas a draft of this instrument was laid before Parliament in accordance with section~80(1) of the Social Security Act 1998\footnote{1998 c.~14.} and~Article~75(1A) of the Social Security (Northern Ireland) Order 1998\footnote{S.I.~1998/1506 (N.I.~10). Paragraph (1A) was inserted in Article~75 by paragraph~18(3) of Schedule~4 to~the Tax Credits Act 2002 (c.~21).} and~approved by resolution of each House of Parliament;

Now, therefore, the Commissioners of Inland~Revenue, in exercise of the powers conferred upon them by the provisions set out in Schedule~1 and, in accordance with section~8 of the Tribunals and~Inquiries Act 1992\footnote{1992 c.~53. Section~8 was amended by article 335 of S.I.~2001/3949.}, after consultation with the Council on Tribunals, hereby make the following Regulations: \looseness=-1

{\sloppy

\tableofcontents

}

\bigskip

\setcounter{secnumdepth}{-2}

\section[Part~I --- General]{Part~I\\*General}

\renewcommand\parthead{--- Part~I}

\subsection[1. Citation, commencement and~effect]{Citation, commencement and~effect}

1.---(1)  These Regulations may be cited as the Child Benefit and~Guardian’s Allowance (Decisions and~Appeals) Regulations 2003 and~shall come into~force on 7th April 2003 immediately after the commencement of section~50 of the Tax Credits Act 2002 for the purposes of entitlement to~payment of child benefit and~guardian’s allowance.

(2) These Regulations have effect only in relation to—
\begin{enumerate}\item[]
($a$) child benefit and~guardian’s allowance under the Contributions and~Benefits Act; and

($b$) child benefit and~guardian’s allowance under the Contributions and~Benefits (NI) Act.
\end{enumerate}

\subsection[2. Interpretation]{Interpretation}

2.---(1)  In these Regulations—
\begin{enumerate}\item[]
“the 1998 Act” means the Social Security Act 1998;

“the Administration Act” means the Social Security Administration Act 1992\footnote{1992 c.~5.};

“the Administration (NI) Act” means the Social Security Administration (Northern Ireland) Act 1992\footnote{1992 c.~8.};

\begin{sloppypar}\hbadness=1242
“the Administration Regulations” means the Child Benefit and Guardian’s Allowance (Administration) Regulations 2003\footnote{S.I.~2003/492.};
\end{sloppypar}

“appeal tribunal” means—
\begin{enumerate}\item[]
($a$) 
in relation to~child benefit or guardian’s allowance under the Contributions and~Benefits Act, an appeal tribunal constituted under Chapter I of Part~I of the 1998 Act;

($b$) 
in relation to~child benefit or guardian’s allowance under the Contributions and~Benefits (NI) Act, an appeal tribunal constituted under Chapter I of Part~II of the 1998 Order;
\end{enumerate}

“the appropriate office” means—
\begin{enumerate}\item[]
($a$) 
in relation to~child benefit or guardian’s allowance under the Contributions and~Benefits Act, the Child Benefit Office, Waterview Park, Washington, Tyne and~Wear;

($b$) 
in relation to~child benefit or guardian’s allowance under the Contributions and~Benefits (NI) Act, the Child Benefit Office (Northern Ireland), Windsor House, Bedford Street, Belfast;

($c$) 
any Inland~Revenue Enquiry Centre;
\end{enumerate}

“the Board” means the Commissioners of Inland~Revenue;

“claimant” means a person who has claimed child benefit or guardian’s allowance and~includes, in relation to~an award or decision, a beneficiary under the award or a person affected by the decision;

\enlargethispage{-\baselineskip}

“clerk to~the appeal tribunal” means—
\begin{enumerate}\item[]
($a$) 
in relation to~child benefit or guardian’s allowance under the Contributions and~Benefits Act, a clerk assigned to~the appeal tribunal in accordance with regulation~37 of the Decisions and~Appeals Regulations;

($b$) 
in relation to~child benefit or guardian’s allowance under the Contributions and~Benefits (NI) Act, a clerk assigned to~the appeal tribunal in accordance with regulation~37 of the Decisions and~Appeals Regulations (NI);
\end{enumerate}

“Commissioner” means—
\begin{enumerate}\item[]
($a$) 
in relation to~child benefit or guardian’s allowance under the Contributions and~Benefits Act, the Chief Social Security Commissioner or any other Social Security Commissioner appointed under the 1998 Act and~includes a tribunal of three or more Commissioners constituted under section~16(7);

($b$) 
in relation to~child benefit or guardian’s allowance under the Contributions and~Benefits (NI) Act, the Chief Social Security Commissioner or any other Social Security Commissioner appointed under the 1998 Order and~includes a tribunal of two or more Commissioners constituted under Article~16(7);
\end{enumerate}

\enlargethispage{-\baselineskip}

“the Contributions and~Benefits Act” means the Social Security Contributions and~Benefits Act 1992\footnote{1992 c.~4.};

“the Contributions and~Benefits (NI) Act” means the Social Security Contributions and~Benefits (Northern Ireland) Act 1992\footnote{1992 c.~7.};

“the Decisions and~Appeals Regulations” means the Social Security and~Child Support (Decisions and~Appeals) Regulations 1999\footnote{S.I.~1999/991.};

“the Decisions and~Appeals Regulations (NI)” means the Social Security and~Child Support (Decisions and~Appeals) Regulations (Northern Ireland) 1999\footnote{S.R.~1999 No.~162.};

“family” has—
\begin{enumerate}\item[]
($a$) 
in relation to~child benefit and~guardian’s allowance under the Contributions and~Benefits Act, the meaning given by section~137 of that Act;

($b$) 
in relation to~child benefit and~guardian’s allowance under the Contributions and~Benefits (NI) Act, the meaning given by section~133 of that Act;
\end{enumerate}

“legally qualified panel member” means—
\begin{enumerate}\item[]
($a$) 
in relation to~child benefit or guardian’s allowance under the Contributions and~Benefits Act, a panel member who satisfies the requirements of paragraph~1 of Schedule~3 to~the Decisions and~Appeals Regulations;

($b$) 
in relation to~child benefit or guardian’s allowance under the Contributions and~Benefits (NI) Act, a panel member who satisfies the requirements of paragraph~1 of Schedule~2 to~the Decisions and~Appeals Regulations (NI);
\end{enumerate}

“the Northern Ireland~Department” means the Department for Social Development in Northern Ireland;

“the 1998 Order” means the Social Security (Northern Ireland) Order 1998;

“panel” means the panel constituted under section~6\footnote{Section~6 was amended by paragraph~4 of Schedule~3 to~S.I.~1999/1042 and~paragraph~3 of Schedule~5 to~S.I.~2000/253.} or Article~7;

“panel member” means a person appointed to~the panel;

“party to~the proceedings” means the Board and~any other person who—
\begin{enumerate}\item[]
($a$) 
is one of the principal parties for the purposes of sections 13 and~14\footnote{Sections~13 and~14 were amended by paragraphs~26 and~27 of Schedule~7, and~Part~I of Schedule~10, to~the Social Security Contributions (Transfer of Functions, etc.)\ Act 1999 (c.~2) (“the Transfer Act 1999”).} or Articles 14 and~15\footnote{Articles 14 and~15 were amended by paragraphs~20 and~21 of Schedule~6, and~Part~I of Schedule~9, to~the Social Security Contributions (Transfer of Functions, etc.)\ (Northern Ireland) Order 1999 (S.I.~1999/671) (“the Transfer Order 1999”).}; or

($b$) 
has a right of appeal to~an appeal tribunal under section~12(2)\footnote{Section~12(2) was substituted by paragraph~25(3) of Schedule~7 to~the Transfer Act 1999.} or Article~13(2)\footnote{Article~13(3) was substituted by paragraph~19(3) of Schedule~6 to~the Transfer Order 1999.};
\end{enumerate}

“relevant benefit” means child tax credit under the Tax Credits Act 2002 and—
\begin{enumerate}\item[]
($a$) 
in relation to~child benefit or guardian’s allowance under the Contributions and~Benefits Act, any of the benefits mentioned in section~8(3)\footnote{Section~8(3) was amended by paragraphs 1 and 6($q$)  of Schedule~1 to~the Tax Credits Act 1999 (c.~10), Part VI of Schedule~13 to~the Welfare Reform and Pensions Act 1999 (c.~30), paragraph~6(2) of Schedule~1 to~the State Pension Credit Act 2002 (c.~16) and Schedule~6 to~the Tax Credits Act 2002.};

($b$) 
in relation to~child benefit or guardian’s allowance under the Contributions and~Benefits (NI) Act, any of the benefits mentioned in Article~9(3)\footnote{Article~9(3) was amended by paragraphs 1 and 6($r$)  of Schedule~1 to~the Tax Credits Act 1999, Part VII of Schedule~13 to~the Welfare Reform and Pensions Act 1999 and Schedule~6 to~the Tax Credits Act 2002.};
\end{enumerate}

“superseding decision” has the meaning given by regulation~13(1);

“writing” includes writing produced by electronic communications used in accordance with regulation~4.
\end{enumerate}

(2) In these Regulations—
\begin{enumerate}\item[]
($a$) a reference to~a numbered section~without more is a reference to~the section~of the 1998 Act bearing that number;

($b$) a reference to~a numbered Article~without more is a reference to~the Article~of the 1998 Order bearing that number.
\end{enumerate}

\subsection[3. Service of notices or documents]{Service of notices or documents}

3.---(1)  Where, under any provision of these Regulations—
\begin{enumerate}\item[]
($a$) a notice or other document is required to~be given or sent to~the clerk to~the appeal tribunal or the Board, the notice or document is to~be treated as having been so given or sent on the day that it is received by the clerk or the Board;

($b$) a notice (including notification of a decision of the Board) or other document is required to~be given or sent to~any person other than clerk to~the appeal tribunal or the Board, the notice or document is, if sent by post to~that person’s last known address, to~be treated as having been given or sent on the day that it was posted.
\end{enumerate}

(2) In these Regulations, “the date of notification”, in relation to~a decision of the Board, means the date on which notification of the decision is treated under paragraph~(1)($b$)  as having been given or sent.

\subsection[4. Use of electronic communications]{Use of electronic communications}

4.---(1)  Schedule~2 to~the Administration Regulations (the use of electronic communications) applies to~the delivery of information to~or by the Board which is authorised or required by these Regulations in the same manner as it applies to~the delivery of information to~or by the Board which is authorised or required by the Administration Regulations.

(2) References in paragraph~(1) to~the delivery of information shall be construed in accordance with section~132(8) of the Finance Act 1999\footnote{1999 c.~16.}.

\section[Part II --- Revision of decisions]{Part II\\*Revision of decisions}

\renewcommand\parthead{--- Part II}

\subsection[5. Revision of decisions within a prescribed period or on an application]{Revision of decisions within a prescribed period or on an application}

5.---(1)  Subject to~paragraph~(3), if the conditions specified in paragraph~(2) are satisfied—
\begin{enumerate}\item[]
($a$) a decision under section~8\footnote{The amendments to~subsection (3) of section 8 have been noted previously. Subsections~(1) and (5) of that section were amended by paragraph~22 of Schedule~7, and Part I of Schedule~10, to~the Transfer Act 1999 and subsection (4) by paragraph~6(3) of Schedule~1 to~the State Pension Credit Act 2002.} or 10\footnote{Section~10 was amended by paragraph~23 of Schedule~7, and Part I of Schedule~10, to~the Transfer Act 1999.} may be revised by the Board under section~9; and

($b$) a decision under Article~9\footnote{The amendments to~subsection (3) of Article~9 have been noted previously. Subsections~(1) and (5) of that section were amended by paragraph~16 of Schedule~6, and Part I of Schedule~9, to~the Transfer Order 1999.} or 11\footnote{Article~11 was amended by paragraph~17 of Schedule~6, and Part I of Schedule~9, to~the Transfer Order 1999.} may be revised by them under Article~10.
\end{enumerate}

(2) The conditions specified in this paragraph~are that—
\begin{enumerate}\item[]
($a$) the Board commence action leading to~the revision within one month of the date of notification of the decision; or

($b$) subject to~regulation~6, an application for the revision was received by the Board at the appropriate office—
\begin{enumerate}\item[]
(i) within one month of the date of notification of the decision;

(ii) if a written statement of the reasons for the decision requested under regulation~26(1)($b$)  was provided within the period specified in paragraph~(i), within 14 days of the expiry of that period; or

(iii) if such a statement was provided after the period specified in paragraph~(i), within 14 days of the date on which the statement was provided.
\end{enumerate}
\end{enumerate}

(3) Paragraph (1) does not apply in respect of a relevant change of circumstances which occurred since the decision was made or where the Board have evidence or information which indicates that a relevant change of circumstances will occur.

\subsection[6. Late applications for revision of decisions]{Late applications for revision of decisions}

6.---(1)  The Board may extend the time limits specified in regulation~5(2)($b$)(i)  to~(iii)  if the first and~second conditions are satisfied.

(2) The first condition is that an application for an extension of time must be made to~the Board by the claimant or a person acting on his behalf.

(3) The second condition is that the application for the extension of time must—
\begin{enumerate}\item[]
($a$) contain particulars of the grounds on which the extension is sought;

($b$) contain sufficient details of the decision which it is sought to~have revised so as to~enable it to~be identified; and

($c$) be made within 13 months of the latest date by which the application for revision should have been received by the Board in accordance with regulation~5(2)($b$).
\end{enumerate}

(4) An application for an extension of time must not be granted unless the Board are satisfied that—
\begin{enumerate}\item[]
($a$) it is reasonable to~grant it;

($b$) the application for revision has merit; and

($c$) special circumstances are relevant to~the application for an extension of time as a result of which it was not practicable for the application for revision to~be made within the time limits specified in regulation~5(2)($b$)(i)  to~(iii).
\end{enumerate}

(5) In determining whether it is reasonable to~grant an application for an extension of time, the Board must have regard to~the principle that the greater the amount of time that has elapsed between the expiration of the time limits specified in regulation~5(2)($b$)(i)  to~(iii)  and~the making of the application, the more compelling the special circumstances mentioned in paragraph~(4)($c$)  should be.

(6) In determining whether it is reasonable to~grant an application for an extension of time, the Board must take no account of the following—
\begin{enumerate}\item[]
($a$) that the applicant or any person acting for him was unaware of, or misunderstood, the law applicable to~his case (including being unaware of, or misunderstanding, the time limits imposed by these Regulations); or

($b$) that a Commissioner or a court has taken a different view of the law from that previously understood and~applied.
\end{enumerate}

(7) An application for an extension of time which has been refused may not be renewed.

\subsection[7. Procedure for revision of decisions on an application]{Procedure for revision of decisions on an application}

7.---(1)  The Board may treat—
\begin{enumerate}\item[]
($a$) an application for a decision under section~10 as an application for a revision under section~9;

($b$) an application for a decision under Article~11 as an application for a revision under Article~10.
\end{enumerate}

(2) Paragraph (3) applies where, in order to~consider all the issues raised by an application for such a revision, the Board require further evidence or information from the applicant.

(3) Where this paragraph~applies, the Board must notify the applicant that further evidence or information is required and—
\begin{enumerate}\item[]
($a$) if the applicant provides relevant further evidence or information within one month of the date of notification or such longer period of time as the Board may allow, the decision may be revised;

($b$) if the applicant does not provide such evidence or information within that time, the decision may be revised on the basis of the application.
\end{enumerate}

\subsection[8. Revision of decisions against which there has been an appeal]{Revision of decisions against which there has been an appeal}

8.---(1)  In the circumstances prescribed by paragraph~(2), any of the following decisions may be revised by the Board at any time—
\begin{enumerate}\item[]
($a$) a decision under section~8 or 10;

($b$) a decision under Article~9 or 11.
\end{enumerate}

(2) The circumstances prescribed by this paragraph~are circumstances where there is an appeal to~an appeal tribunal against the decision within the time prescribed by regulation~28, or in a case to~which regulation~29 applies within the time prescribed by that regulation, but the appeal has not been determined.

(3) If—
\begin{enumerate}\item[]
($a$) the Board make one of the following decisions (“the original decision”)—
\begin{enumerate}\item[]
(i) a decision under section~8 or 10 or one under section~9(1) revising such a decision; or

(ii) a decision under Article~9 or 11 or one under Article~10(1) revising such a decision;
\end{enumerate}

($b$) the claimant appeals to~an appeal tribunal against the original decision;

($c$) after the appeal has been made, but before it results in a decision by the appeal tribunal, the Board make a second decision which—
\begin{enumerate}\item[]
(i) supersedes the original decision in accordance with section~10 or Article~11; or

(ii) decides a further claim for child benefit or guardian’s allowance by the claimant; and
\end{enumerate}

($d$) the Board would have made their second decision differently if, at the time they made it, they had been aware of the decision subsequently made by the appeal tribunal,
\end{enumerate}
the second decision may be revised by the Board at any time.

\subsection[9. Revision of decisions against which no appeal lies]{Revision of decisions against which no appeal lies}

9.---(1)  In the case prescribed by paragraph~(2), any of the following decisions may be revised by the Board at any time—
\begin{enumerate}\item[]
($a$) a decision under section~8 or 10;

($b$) a decision under Article~9 or 11.
\end{enumerate}

(2) The case prescribed by this paragraph~is the case of decisions which—
\begin{enumerate}\item[]
($a$) are specified in—
\begin{enumerate}\item[]
(i) Schedule~2 to~the 1998 Act\footnote{Schedule~2 was amended by paragraph~87 of Schedule~12 to~the Welfare Reform and Pensions Act 1999, paragraph~11 of Schedule~1 to~the State Pension Credit Act 2002 and paragraph~3($b$)  of the Schedule~to~S.I.~2002/1457.}; or

(ii) Schedule~2 to~the 1998 Order\footnote{Schedule~2 was amended by paragraph~61 of Schedule~9 to~the Welfare Reform and Pensions (Northern Ireland) Order 1999 (S.I.~1999/3147 (N.I.~11)).}; or
\end{enumerate}

($b$) are prescribed by regulation~25 (decisions against which no appeal lies).
\end{enumerate}

\subsection[10. Revision of decisions arising from official error etc.]{Revision of decisions arising from official error etc.}

10.---(1)  In the circumstances prescribed by paragraph~(2), any of the following decisions may be revised by the Board at any time—
\begin{enumerate}\item[]
($a$) a decision under section~8 or 10;

($b$) a decision under Article~9 or 11.
\end{enumerate}

(2) The circumstances prescribed by this paragraph~are circumstances where the decision—
\begin{enumerate}\item[]
($a$) arose from an official error; or

($b$) was made in ignorance of, or was based upon a mistake as to, some material fact and, as a result of that ignorance of, or mistake as to, that fact, is more advantageous to~the claimant than it would otherwise have been.
\end{enumerate}

(3) “Official error” means an error made by—
\begin{enumerate}\item[]
($a$) an officer of the Board acting as such, which no person outside the Inland~Revenue caused or to~which no such person materially contributed; or

($b$) a person employed by a person providing services to~the Board and~acting as such which no other person who was not so employed caused or to~which no such other person materially contributed,
\end{enumerate}
but does not include an error of law which is shown to~have been an error by virtue of a subsequent decision of a Commissioner or the court.

\subsection[11. Revision of decisions following the award of another relevant benefit]{Revision of decisions following the award of another relevant benefit}

11.---(1)  In the circumstances prescribed by paragraph~(2), any of the following decisions may be revised by the Board at any time—
\begin{enumerate}\item[]
($a$) a decision under section~8 or 10;

($b$) a decision under Article~9 or 11.
\end{enumerate}

(2) The circumstances prescribed by this paragraph~are circumstances where—
\begin{enumerate}\item[]
($a$) the decision awards child benefit or guardian’s allowance to~a person; and

($b$) an award of another relevant benefit, or of an increase in the rate of another relevant benefit, is made to~that person or a member of his family for a period which includes the date on which the decision took effect.
\end{enumerate}

\subsection[12. Date as from which revised decisions take effect]{Date as from which revised decisions take effect}

12.  If the Board decide that—
\begin{enumerate}\item[]
($a$) on a revision under section~9, the date as from which the decision under section~8 or 10 took effect was erroneous; or

($b$) on a revision under Article~10, the date as from which the decision under Article~9 or 11 took effect was erroneous,
\end{enumerate}
the revision shall take effect as from the date from which the decision would have taken effect had the error not been made.

\section[Part III --- Superseding decisions]{Part III\\*Superseding decisions}

\renewcommand\parthead{--- Part III}

\subsection[13. Cases and~circumstances in which superseding decisions may be made]{Cases and~circumstances in which superseding decisions may be made}

13.---(1)  Subject to~regulation~15, the Board may make a decision under section~10 or Article~11 (“a superseding decision”), either on their own initiative or on an application received by them at an appropriate office, in any of the cases and~circumstances prescribed by paragraph~(2).

(2) The cases and~circumstances prescribed by this paragraph~are cases and~circumstances where the decision to~be superseded is—
\begin{enumerate}\item[]
($a$) a decision in respect of which—
\begin{enumerate}\item[]
(i) there has been a relevant change of circumstances since it had effect; or

(ii) it is anticipated that there will be such a change;
\end{enumerate}

($b$) a decision (other than one to~which sub-paragraph~($d$)  refers)—
\begin{enumerate}\item[]
(i) which was erroneous in point of law, or was made in ignorance of, or was based upon a mistake as to, some material fact; and

(ii) in relation to~which an application for a superseding decision was received by the Board, or a decision by the Board to~act on their own initiative was taken, more than one month after the date of notification of the decision to~be superseded or after the expiry of such longer period of time as may have been allowed under regulation~6;
\end{enumerate}

($c$) a decision of an appeal tribunal or a Commissioner which—
\begin{enumerate}\item[]
(i) was made in ignorance of, or was based upon a mistake as to, some material fact;

(ii) in a case to~which subsection~(5) of section~26 applies, was dealt with in accordance with subsection~(4)($b$)  of that section; or

(iii) in a case to~which paragraph~(5) of Article~26 applies, was dealt with in accordance with paragraph~(4)($b$)  of that Article;
\end{enumerate}

($d$) a decision—
\begin{enumerate}\item[]
(i) specified in Schedule~2 to~the 1998 Act;

(ii) specified in Schedule~2 to~the 1998 Order; or

(iii) prescribed by regulation~25 (decisions against which no appeal lies); or
\end{enumerate}

($e$) a decision where—
\begin{enumerate}\item[]
(i) the claimant has been awarded entitlement to~child benefit or guardian’s allowance; and

(ii) subsequent to~the first day of the period to~which that entitlement relates, the claimant or a member of his family becomes entitled to, or to~an increase in the rate of, another relevant benefit.
\end{enumerate}
\end{enumerate}

\subsection[14. Procedure for making superseding decisions on an application]{Procedure for making superseding decisions on an application}

14.---(1)  The Board may treat—
\begin{enumerate}\item[]
($a$) an application for a revision under section~9 as an application for a decision under section~10;

($b$) an application for a revision under Article~10 as an application for a decision under Article~11.
\end{enumerate}

(2) Paragraph (3) applies where, in order to~consider all the issues raised by an application for a superseding decision, the Board require further evidence or information from the applicant.

(3) Where this paragraph~applies, the Board must notify the applicant that further evidence or information is required and—
\begin{enumerate}\item[]
($a$) if the applicant provides further relevant evidence or information within one month of the date of notification or such longer period of time as the Board may allow, the decision to~be superseded may be superseded;

($b$) if the applicant does not provide such evidence or information within that period, the decision to~be superseded may be superseded on the basis of the application.
\end{enumerate}

\subsection[15. Interaction of revisions and~superseding decisions]{Interaction of revisions and~superseding decisions}

15.---(1)  This regulation~applies to~any decision in relation to~which circumstances arise in which the decision may be revised under section~9 or Article~10.

(2) A decision to~which this regulation~applies may not be superseded by a superseding decision unless—
\begin{enumerate}\item[]
($a$) circumstances arise in which the Board may revise the decision in accordance with Part~II; and

($b$) further circumstances arise in relation to~the decision which—
\begin{enumerate}\item[]
(i) are not specified in any of the regulations in Part~II; but

(ii) are prescribed by regulation~13(2) or are ones where a superseding decision may be made in accordance with regulation~14(3).
\end{enumerate}
\end{enumerate}

\subsection[16. Date as from which superseding decisions take effect]{Date as from which superseding decisions take effect}

16.---(1)  This regulation~prescribes cases or circumstances in which a superseding decision shall take effect as from a prescribed date other than the date on which it was made or, where applicable, the date on which the application for it was made.

(2) If a superseding decision is made on the basis that—
\begin{enumerate}\item[]
($a$) there has been a relevant change of circumstances since the decision to~be superseded had effect; or

($b$) it is anticipated there will be such a change,
\end{enumerate}
it shall take effect as from the earliest date prescribed by paragraphs~(3) to~(8).

(3) In any case where the superseding decision is advantageous to~the claimant and~notification of the change was given in accordance with any enactment or subordinate legislation under which that notification was required, the date prescribed by this paragraph~is—
\begin{enumerate}\item[]
($a$) if the notification was given within one month of the change occurring or such longer period as may be allowed under regulation~17, the date the change occurred or, if later, the first date on which the change has effect; or

($b$) if the notification was given after the period mentioned in sub-paragraph~($a$), the date of notification of the change.
\end{enumerate}

(4) In any case where the superseding decision is advantageous to~the claimant and~is made on the Board’s own initiative, the date prescribed by this paragraph~is the date on which the Board commenced action with a view to~the supersession.

(5) In any case where the superseding decision is not advantageous to~the claimant, the date prescribed by this paragraph~is the date of the change.

(6) Decisions which are advantageous to~claimants include those mentioned in regulation~27(5).

(7) If—
\begin{enumerate}\item[]
($a$) the Board supersede a decision made by an appeal tribunal or a Commissioner in accordance with paragraph~(i)  of regulation~13(2)($c$); and

($b$) as a result of the ignorance or mistake referred to~in that paragraph, the decision to~be superseded was more advantageous to~the claimant than it would otherwise have been,
\end{enumerate}
the superseding decision shall take effect as from the date on which the decision of the appeal tribunal or the Commissioner took, or was to~take, effect.

(8) If the Board supersede a decision made by an appeal tribunal or a Commissioner in accordance with paragraph~(ii)  or (iii)  of regulation~13(2)($c$), the superseding decision shall take effect as from the date on which it would have taken effect had it been decided in accordance with the determination of the Commissioner or the court in the appeal referred to~in section~26(1)($b$)  or Article~26(1)($b$).

(9) If a superseding decision is made in consequence of a decision which is a relevant determination for the purposes of section~27\footnote{Section~27 was amended by paragraph~9 of Schedule~1 to the State Pension Credit Act 2002.} or Article~27, it shall take effect as from the date of the relevant determination.

(10) If the Board supersede a decision in accordance with sub-paragraph~($e$)  of regulation~13(2), the superseding decision shall take effect as from the date on which entitlement arises—
\begin{enumerate}\item[]
($a$) to~the other relevant benefit referred to~in paragraph~(ii)  of that sub-paragraph; or

($b$) to~an increase in the rate of that benefit.
\end{enumerate}

\subsection[17. Effective date for late notifications of change of circumstances]{Effective date for late notifications of change of circumstances}

17.---(1)  For the purposes of paragraph~(3) of regulation~16, the Board may allow a longer period of time than the period of one month mentioned in sub-paragraph~($a$)  of that paragraph~for the notification of a change of circumstances if the first and~second conditions are satisfied.

(2) The first condition is that an application for the purposes of regulation~16(3) must be made by the claimant or a person acting on his behalf.

(3) The second condition is that the application for the purposes of regulation~16(3) must—
\begin{enumerate}\item[]
($a$) contain particulars of the relevant change of circumstances and~the reasons for the failure to~notify the change on an earlier date; and

($b$) be made within 13 months of the date on which the change occurred.
\end{enumerate}

(4) An application under this regulation~must not be granted unless the Board are satisfied that—
\begin{enumerate}\item[]
($a$) it is reasonable to~grant it;

($b$) the change of circumstances notified by the applicant is relevant to~the decision which is to~be superseded; and

($c$) special circumstances are relevant to~the application as a result of which it was not practicable for the applicant to~notify the change of circumstances within one month of the change occurring.
\end{enumerate}

(5) In determining whether it is reasonable to~grant an application for the purposes of regulation~16(3), the Board must have regard to~the principle that the greater the amount of time that has elapsed between the date one month after the change of circumstances occurred and~the date the application is made, the more compelling the special circumstances mentioned in paragraph~(4)($c$)  should be.

(6) In determining whether it is reasonable to~grant an application for the purposes of regulation~16(3), the Board must take no account of the following—
\begin{enumerate}\item[]
($a$) that the applicant or any person acting for him was unaware of, or misunderstood, the law applicable to~his case (including being unaware of, or misunderstanding, the time limits imposed by these Regulations); or

($b$) that a Commissioner or a court has taken a different view of the law from that previously understood and~applied.
\end{enumerate}

(7) An application for the purposes of regulation~16(3) which has been refused may not be renewed.

\section[Part IV --- Suspension and termination]{Part IV\\*Suspension and termination}

\renewcommand\parthead{--- Part IV}

\subsection[18. Suspension in prescribed cases]{Suspension in prescribed cases}

18.---(1)  The Board may suspend payment of child benefit or guardian’s allowance, in whole or in part, in the circumstances prescribed by paragraph~(2) or (3).

(2) The circumstances prescribed by this paragraph~are circumstances where it appears to~the Board that—
\begin{enumerate}\item[]
($a$) an issue arises as to~whether the conditions for entitlement to~the benefit or allowance are or were fulfilled;

($b$) an issue arises as to~whether a decision relating to~an award of the benefit or allowance should be—
\begin{enumerate}\item[]
(i) revised under section~9 or Article~10; or

(ii) superseded under section~10 or Article~11;
\end{enumerate}

($c$) an issue arises as to~whether any amount paid or payable to~a person by way of, or in connection with a claim for, the benefit or allowance is recoverable under—
\begin{enumerate}\item[]
(i) section~71 of the Administration Act;

(ii) section~69 of the Administration (NI) Act; or

(iii) regulations made under either of those sections;
\end{enumerate}

($d$) the last address notified to~them of a person who is in receipt of the benefit or allowance is not the address at which that person is residing; or

($e$) the details of a bank account or other account which has been notified to~them and~to~which payment of the benefit or allowance by way of a credit is to~be made to~a person are incorrect.
\end{enumerate}

(3) The circumstances prescribed by this paragraph~are where—
\begin{enumerate}\item[]
($a$) an appeal is pending against a decision of an appeal tribunal, a Commissioner or a court; or

($b$) an appeal is pending against a decision given in a different case by a Commissioner or a court (whether or not relating to~child benefit or guardian’s allowance) and~it appears to~the Board that, if the appeal were to~be determined in a particular way, an issue would arise as to~whether the award of child benefit or guardian’s allowance should be revised or superseded.
\end{enumerate}

(4) For the purposes of section~21(3)($c$)  and~Article~21(3)($c$), the prescribed circumstances are circumstances where an appeal tribunal, a Commissioner or a court has made a decision and~the Board—
\begin{enumerate}\item[]
($a$) are awaiting receipt of the decision or, in the case of an appeal tribunal decision, are considering whether to~apply for a statement of the reasons for it;

($b$) in the case of an appeal tribunal decision, have applied for, and~are awaiting receipt of, such a statement; or

($c$) have received the decision, or, in the case of an appeal tribunal decision, such a statement, and~are considering—
\begin{enumerate}\item[]
(i) whether to~apply for leave to~appeal; or

(ii) where leave to~appeal has been granted, whether to~appeal.
\end{enumerate}
\end{enumerate}

(5) In the circumstances prescribed by paragraph~(4), the Board must give written notice, as soon as reasonably practicable, to~the person in respect of whom payment has been or is to~be suspended of their proposal—
\begin{enumerate}\item[]
($a$) to~make a request for a statement of the reasons for an appeal tribunal decision;

($b$) to~apply for leave to~appeal; or

($c$) to~appeal.
\end{enumerate}

\subsection[19. Provision of information or evidence]{Provision of information or evidence}

19.---(1)  This regulation~applies where the Board require information or evidence for a determination whether a decision awarding child benefit or guardian’s allowance should be—
\begin{enumerate}\item[]
($a$) revised under section~9 or Article~10; or

($b$) superseded under section~10 or Article~11.
\end{enumerate}

(2) A person to~whom this paragraph~applies must—
\begin{enumerate}\item[]
($a$) supply the information or evidence within—
\begin{enumerate}\item[]
(i) the period of one month beginning with the date on which the notification under paragraph~(4) was sent to~him; or

(ii) such longer period as he satisfies the Board is necessary in order to~enable him to~comply with the requirement; or
\end{enumerate}

($b$) satisfy the Board within the period of time specified in sub-\hspace{0pt}paragraph~($a$)(i)  that—
\begin{enumerate}\item[]
(i) the information or evidence required of him does not exist; or

(ii) it is not possible for him to~obtain it.
\end{enumerate}
\end{enumerate}

(3) A person to~whom paragraph~(2) applies is any of the following—
\begin{enumerate}\item[]
($a$) a person in respect of whom payment of the benefit or allowance has been suspended in the circumstances prescribed by regulation~18(2);

($b$) a person who has made an application for the decision to~be revised or superseded;

($c$) a person who fails to~comply with the provisions of regulation~23 of the Administration Regulations in so far as they relate to~information, facts or evidence required by the Board.
\end{enumerate}

(4) The Board must notify a person to~whom paragraph~(2) applies of the requirements of that paragraph.

(5) The Board may suspend the payment of benefit or allowance, in whole or in part, to~a person falling within paragraph~(3)($b$)  or ($c$)  who fails to~satisfy the requirements of paragraph~(2).

\subsection[20. Termination in cases of failure to~furnish information or evidence]{Termination in cases of failure to~furnish information or evidence}

20.---(1)  Subject to~paragraph~(3), this regulation~applies where—
\begin{enumerate}\item[]
($a$) a person whose benefit or allowance has been suspended under regulation~18 subsequently fails to~comply with a requirement for information or evidence under regulation~19 and~more than one month has elapsed since the requirement was made; or

($b$) a person’s benefit or allowance has been suspended under regulation~19(5) and~more than one month has elapsed since the first payment was so suspended.
\end{enumerate}

(2) The Board must decide that the person ceases to~be entitled to~the benefit or allowance from the date on which payment was suspended except where entitlement to~the benefit or allowance ceases on an earlier date.

(3) This regulation~does not apply where benefit or allowance has been suspended in part under regulation~18 or 19.

\subsection[21. Making of payments which have been suspended]{Making of payments which have been suspended}

21.---(1)  Payment of benefit or allowance suspended in accordance with regulation~18 or 19 must be made in any of the circumstances prescribed by paragraphs~(2) to~(5).

(2) The circumstances prescribed by this paragraph~are circumstances where—
\begin{enumerate}\item[]
($a$) in a case to~which regulation~18(2)($a$), ($b$)  or ($c$)  applies, the Board are satisfied that—
\begin{enumerate}\item[]
(i) the benefit or allowance suspended is properly payable; and

(ii) no outstanding issues remain to~be resolved;
\end{enumerate}

($b$) in a case to~which regulation~18(2)($d$)  applies, the Board are satisfied that they have been notified of the address at which the person is residing;

($c$) in a case to~which regulation~18(2)($e$)  applies, the Board are satisfied that they have been notified of the correct details of the bank account or other account to~which payment of the benefit or allowance by way of a credit is to~be made to~the person.
\end{enumerate}

(3) The circumstances prescribed by this paragraph~are circumstances where, in a case to~which regulation~18(3)($a$)  applies, the Board—
\begin{enumerate}\item[]
($a$) in the case of a decision of an appeal tribunal, do not apply for a statement of the reasons for that decision within the period of one month specified in—
\begin{enumerate}\item[]
(i) in relation to~child benefit and~guardian’s allowance under the Contributions and~Benefits Act, regulation~53(4) of the Decisions and~Appeals Regulations\footnote{Regulation~53(4) was substituted by regulation 16 of S.I.~2002/1379.};

(ii) in relation to~child benefit and~guardian’s allowance under the Contributions and~Benefits (NI) Act, regulation~53(4) of the Decisions and~Appeals Regulations (NI)\footnote{Regulation~53(4) was amended by regulation 6(15)($b$) of S.R.~2000 No.~215 and regulation 2(15) of S.R.~2002 No.~189};
\end{enumerate}

($b$) in the case of a decision of an appeal tribunal, a Commissioner or a court—
\begin{enumerate}\item[]
(i) do not make an application for leave to~appeal within the time prescribed for the making of such an application; or

(ii) where leave to~appeal is granted, do not make the appeal within the time prescribed for the making of it;
\end{enumerate}

($c$) withdraw an application for leave to~appeal or the appeal; or

($d$) are refused leave to~appeal in circumstances where it is not open to~them to~renew the application, or to~make a further application, for such leave.
\end{enumerate}

(4) The circumstances prescribed by this paragraph~are circumstances where, in a case to~which regulation~18(3)($b$)  applies, the Board, in relation to~the decision of the Commissioner or the court in the different case—
\begin{enumerate}\item[]
($a$) do not make an application for leave to~appeal within the time prescribed for the making of such an application;

($b$) where leave to~appeal is granted, do not make the appeal within the time prescribed for the making of it;

($c$) withdraw an application for leave to~appeal or the appeal; or

($d$) are refused leave to~appeal in circumstances where it is not open to~them to~renew the application, or to~make a further application, for such leave.
\end{enumerate}

(5) The circumstances prescribed by this paragraph~are circumstances where, in a case to~which paragraph~(5) of regulation~19 applies, the Board are satisfied that—
\begin{enumerate}\item[]
($a$) the benefit or allowance suspended is properly payable; and

($b$) the requirements of paragraph~(2) of that regulation~have been satisfied.
\end{enumerate}

\section[Part V --- Other matters]{Part V\\*Other matters}

\renewcommand\parthead{--- Part V}

\subsection[22. Decisions involving issues that arise on appeal in other cases]{Decisions involving issues that arise on appeal in other cases}

22.---(1)  A case which satisfies the condition specified in paragraph~(2) is a prescribed case for the purposes of section~25(3)($b$)  and~Article~25(3)($b$)  (prescribed cases and~circumstances in which a decision may be made on a prescribed basis).

(2) The condition specified in this paragraph~is that the claimant would be entitled to~the benefit or allowance to~which the decision which falls to~be made relates, even if the appeal in the other case referred to~in section~25(1)($b$)  or Article~25(1)($b$)  were decided in a way which is the most unfavourable to~him.\looseness=-1

(3) For the purposes of subsection~(3)($b$)  of section~25 and~paragraph~(3)($b$)  of Article~25, the prescribed basis on which the Board may make the decision is as if—
\begin{enumerate}\item[]
($a$) the appeal in the other case which is referred to~in subsection~(1)($b$)  of that section, or paragraph~(1)($b$)  of that Article, had already been determined; and

($b$) that appeal had been decided in a way which is the most unfavourable to~the claimant.
\end{enumerate}

(4) For the purposes of subsection~(5)($c$)  of section~25 and~paragraph~(5)($c$)  of Article~25 (prescribed circumstances in which, for the purposes of the section~or the Article, an appeal is pending against a decision), the prescribed circumstances are circumstances where the Board—
\begin{enumerate}\item[]
($a$) certify in writing that they are considering appealing against that decision; and

($b$) consider that, if such an appeal were to~be determined in a particular way—
\begin{enumerate}\item[]
(i) there would be no entitlement to~the benefit or allowance in a case to~which subsection~(1)($a$)  of that section, or paragraph~(1)($a$)  of that Article, refers; or

(ii) the appeal would affect the decision in that case in some other way.
\end{enumerate}
\end{enumerate}

\subsection[23. Appeals involving issues that arise on appeal in other cases]{Appeals involving issues that arise on appeal in other cases}

23.  For the purposes of subsection~(6)($c$)  of section~26 and~paragraph~(6)($c$)  of Article~26 (prescribed circumstances in which an appeal against a decision which has not been brought, or an application for leave to~appeal has not been made, but the time for so doing has not yet expired, is pending for the purposes of the section~or the Article), the prescribed circumstances are circumstances where the Board—
\begin{enumerate}\item[]
($a$) certify in writing that they are considering appealing against that decision; and

($b$) consider that, if such an appeal were already determined, it would affect the determination of the appeal described in subsection~(1)($a$)  of that section~or paragraph~(1)($a$)  of that Article.
\end{enumerate}

\section[Part VI --- Rights of appeal and procedure for bringing appeals]{Part VI\\*Rights of appeal and procedure for bringing appeals}

\renewcommand\parthead{--- Part VI}

\subsection[24. Other persons with a right of appeal]{Other persons with a right of appeal}

24.  For the purposes of section~12(2) and~Article~13(2), the following persons are prescribed—
\begin{enumerate}\item[]
($a$) any person appointed by the Board under regulation~28(1) of the Administration Regulations to~act on behalf of another who is unable to~act;\looseness=-1

($b$) any person appointed by the Board under regulation~29(1) of those regulations to~proceed with the claim of a person who has made a claim for benefit or allowance and~subsequently died;

($c$) any person who, having been appointed by the Board under paragraph~(2) of regulation~31 of those regulations to~claim on behalf of a deceased person, makes a claim in accordance with that regulation.
\end{enumerate}

\subsection[25. Decisions against which no appeal lies]{Decisions against which no appeal lies}

25.---(1)  Subject to~paragraph~(2), for the purposes of section~12(2) and~Article~13(2), the decisions set out in Schedule~2 are prescribed as decisions against which no appeal lies to~an appeal tribunal.

(2) Paragraph (1) shall not have the effect of prescribing any decision that relates to~the conditions of entitlement to~child benefit or guardian’s allowance for which a claim has been validly made or for which no claim is required.\looseness=-1

(3) In this regulation~and~Schedule~2, “decision” includes any determination embodied in or necessary to~a decision.

\subsection[26. Notice of decision against which appeal lies]{Notice of decision against which appeal lies}

26.---(1)  A person with a right of appeal under the 1998 Act, the 1998 Order or these Regulations against a decision of the Board must—
\begin{enumerate}\item[]
($a$) be given written notice of the decision against which the appeal lies;

($b$) be informed that, in a case where that written notice does not include a statement of the reasons for that decision, he may, within one month of the date of notification of that decision, request that the Board provide him with a written statement of the reasons for that decision; and

($c$) be given written notice of his right of appeal against that decision.
\end{enumerate}

(2) If the Board are requested under paragraph~(1)($b$)  to~provide a written statement of the reasons for the decision, they must provide the statement within 14 days of receipt of the request.

\subsection[27. Appeals against decisions which have been revised]{Appeals against decisions which have been revised}

27.---(1)  This regulation~applies where—
\begin{enumerate}\item[]
($a$) a decision—
\begin{enumerate}\item[]
(i) under section~8 or 10 is revised under section~9; or

(ii) under Article~9 or 11 is revised under Article~10,
\end{enumerate}
before an appeal against that decision is determined; and

($b$) the decision as revised is not more advantageous to~the appellant than the decision before it was revised.
\end{enumerate}

(2) The appeal shall not lapse and~is to~be treated as though it had been brought against the decision as revised.

(3) The appellant shall have a period of one month from the date of notification of the decision as revised to~make further representations as to~the appeal.

(4) After the expiration of the period specified in paragraph~(3), or within that period if the appellant consents in writing, the appeal shall proceed unless, in the light of the further representations from the appellant, the Board further revise their decision and~that decision is more advantageous to~the appellant than the decision before it was revised.

(5) Decisions which are more advantageous to~the appellant include those in consequence of which—
\begin{enumerate}\item[]
($a$) child benefit or guardian’s allowance paid to~him is greater or is awarded for a longer period;

($b$) the amount of benefit or allowance in payment would have been greater but for the operation of—
\begin{enumerate}\item[]
(i) any provision of the Administration Act or the Administration (NI) Act; or

(ii) any provision of the Contributions and~Benefits Act, or any provision of the Contributions and~Benefits (NI) Act, restricting or suspending the payment of, or disqualifying a claimant from receiving, some or all of the benefit or allowance;
\end{enumerate}

($c$) a denial or disqualification for the receiving of benefit or allowance is lifted wholly or in part;

($d$) a decision to~pay benefit or allowance to~a third party is reversed;

($e$) benefit or allowance paid is not recoverable under—
\begin{enumerate}\item[]
(i) section~71 of the Administration Act\footnote{Section~71 was amended by section 32(1) of, and paragraph~48 of Schedule~2 and Schedule~3 to, the Jobseekers Act 1995 (c.~18), section 1 of the Social Security (Overpayments) Act 1996 (c.~51), paragraph~81 of Schedule~7 to the Social Security Act 1998, paragraphs 1 and 3($c$) of Schedule~1 to the Tax Credits Act 1999, paragraph~10 of Schedule~2 to the State Pension Credit Act 2002 and paragraph~2 of Schedule~4, and Schedule~6, to the Tax Credits Act 2002.} or section~69 of the Administration (NI) Act\footnote{Section~71 was amended by Article 33(1) of, and paragraph~31 of Schedule~2 and Schedule~3 to, the Jobseekers (Northern Ireland) Order 1995 (S.I.~1995/2705 (N.I.~15), section 2 of the Social Security (Overpayments) Act 1996, paragraph~62 of Schedule~6 to S.I.~1998/1506 (N.I.~10), paragraphs 1 and 5($c$) of Schedule~1 to the Tax Credits Act 1999 and paragraph~8 of Schedule~4, and Schedule~6, to the Tax Credits Act 2002.}; or

(ii) regulations made under either of those sections;
\end{enumerate}

($f$) the amount of benefit or allowance paid which is recoverable as mentioned in sub-paragraph~($e$)  is reduced; or

($g$) a financial gain accrues or will accrue to~the appellant in consequence of the decision.
\end{enumerate}

\subsection[28. Time within which an appeal is to~be brought]{Time within which an appeal is to~be brought}

28.---(1)  Subject to~the following provisions of this Part, where an appeal lies from a decision of the Board to~an appeal tribunal, the time within which that appeal must be brought is—
\begin{enumerate}\item[]
($a$) within one month of the date of notification of the decision against which the appeal is brought;

($b$) if a written statement of the reasons for that decision is requested and~provided within the period mentioned in sub-paragraph~($a$), within 14 days of the expiry of that period; or

($c$) if a written statement of the reasons for that decision is requested but is not provided within the period mentioned in sub-paragraph~($a$), within 14 days of the date on which the statement is provided.
\end{enumerate}

(2) If the Board—
\begin{enumerate}\item[]
($a$) revise a decision under section~9 or Article~10;

($b$) make a superseding decision; or

($c$) following an application for a revision, do not revise a decision under section~9 or Article~10,
\end{enumerate}
the period of one month specified in paragraph~(1) shall begin to~run from the date of notification of the revision or supersession or the date the Board issue a notice that they are not revising the decision.

(3) If a dispute arises as to~whether an appeal was brought within the time limit specified in this regulation, the dispute must be referred to, and~be determined by, a legally qualified panel member.

(4) The time limit specified in this regulation~for bringing an appeal may be extended in accordance with regulation~29.

\subsection[29. Late appeals]{Late appeals}

29.---(1)  The time within which an appeal must be brought may be extended in accordance with this regulation, but no appeal shall in any event be brought more than one year after the expiration of the last day for appealing under regulation~28.

(2) An application for an extension of time under this regulation~must—
\begin{enumerate}\item[]
($a$) be made in accordance with regulation~31; and

($b$) be determined by a legally qualified panel member except where the Board consider that the application satisfies paragraph~(5)($b$).
\end{enumerate}

(3) If the Board consider that an application under this regulation~satisfies paragraph~(5)($b$), they may grant it.

(4) An application under this regulation~must contain particulars of the grounds on which the extension of time is sought, including details of any relevant special circumstances specified in regulation~30(2).

(5) An application under this regulation~must not be granted unless---
\begin{enumerate}\item[]
($a$) the panel member is satisfied that if the application is granted, there are reasonable prospects that the appeal will be successful; or

($b$) the panel member is satisfied, or the Board are satisfied, that it is in the interests of justice for the application to~be granted (see generally regulation~30).
\end{enumerate}

(6) An application under this regulation~which has been refused may not be renewed.

(7) The panel member who determines an application under this regulation~must record a summary of his decision in such written form as has been approved by the President.

(8) “The President” means—
\begin{enumerate}\item[]
($a$) in relation to~child benefit or guardian’s allowance under the Contributions and~Benefits Act, the President of appeals tribunals appointed under section~5;

($b$) in relation to~child benefit or guardian’s allowance under the Contributions and~Benefits (NI) Act, the President of appeals tribunals appointed under Article~6.
\end{enumerate}

(9) As soon as practicable after the decision is made a copy of the decision must be sent or given to~every party to~the proceedings.

\subsection[30. Interests of justice]{Interests of justice}

30.---(1)  For the purposes of paragraph~(5)($b$)  of regulation~29, it is not in the interests of justice to~grant an application under that regulation~unless the panel member is satisfied, or the Board are satisfied, that—
\begin{enumerate}\item[]
($a$) the special circumstances specified in paragraph~(2) are relevant to~the application; or

($b$) some other special circumstances exist which are wholly exceptional and~relevant to~the application,
\end{enumerate}
and, as a result of those special circumstances, it was not practicable for the appeal to~be brought within the time limit specified in regulation~28.

(2) The special circumstances specified in this paragraph~are that—
\begin{enumerate}\item[]
($a$) the applicant or a partner or dependant of the applicant has died or suffered serious illness;

($b$) the applicant is not resident in the United Kingdom; or

($c$) normal postal services were disrupted.
\end{enumerate}

(3) “Partner” means—
\begin{enumerate}\item[]
($a$) where a person is a member of a married couple or an unmarried couple, the other member of that couple; or

($b$) where a person is polygamously married to~two or more members of his household, any such member.
\end{enumerate}

(4) In determining whether it is in the interests of justice to~grant an application under regulation~29, the panel member or the Board must have regard to~the principle that the greater the amount of time that has elapsed between the expiration of the time within which the appeal is to~be brought under regulation~28 and~the making of the application, the more compelling the special circumstances mentioned in paragraph~(1) should be.

(5) In determining whether it is in the interests of justice to~grant an application under regulation~29, the panel member or the Board must take no account of the following—
\begin{enumerate}\item[]
($a$) that the applicant or any person acting for him was unaware of or misunderstood the law applicable to~his case (including ignorance or misunderstanding of the time limits imposed by these Regulations); or

($b$) that a Commissioner or a court has taken a different view of the law from that previously understood and~applied.
\end{enumerate}

\subsection[31. Making of appeals and~applications]{Making of appeals and~applications}

31.---(1)  Subject to~the following provisions of this regulation, an appeal, or an application for an extension of time for making an appeal, to~an appeal tribunal must—
\begin{enumerate}\item[]
($a$) be in writing—
\begin{enumerate}\item[]
(i) on a form approved for the purpose by the Board (“the approved form”); or

(ii) in such other format as the Board may accept as sufficient for the purpose;
\end{enumerate}

($b$) be signed by–
\begin{enumerate}\item[]
(i) the person who has a right of appeal under section~12(2) or Article~13(2); or

(ii) if that person has provided written authority to~a representative to~act on his behalf, that representative;
\end{enumerate}

($c$) be sent or delivered to~an appropriate office;

($d$) contain particulars of the grounds on which it is made; and

($e$) contain sufficient particulars of the decision or the subject of the application, to~enable that decision or subject to~be identified.
\end{enumerate}

(2) An approved form which is not completed in accordance with the instructions on it—
\begin{enumerate}\item[]
($a$) subject to~paragraph~(3), does not satisfy the requirements of paragraph~(1), and

($b$) may be returned by the Board to~the sender for completion in accordance with those instructions.
\end{enumerate}

(3) If the Board are satisfied that an approved form, although not completed in accordance with the instructions on it, includes sufficient information to~enable the appeal or application to~proceed, they may treat it as satisfying the requirements of paragraph~(1).

(4) If an appeal or application made in writing otherwise than on the approved form includes sufficient information to~enable the appeal or application to~proceed, the Board may treat it as satisfying the requirements of paragraph~(1).

(5) If an appeal or application made in writing otherwise than on the approved form does not include sufficient information to~enable the appeal or application to~proceed, the Board may request further information in writing from the appellant or applicant.

(6) If an appellant or applicant to~whom an approved form is returned, or from whom further information is requested, duly completes and~returns the form or sends the further information and~that form or further information is received by the Board—
\begin{enumerate}\item[]
($a$) within 14 days of the date on which the form was returned to~him by them, the time for making the appeal shall be extended by 14 days from the date on which the form was returned;

($b$) within 14 days of the date on which the further information was requested by them, the time for making the appeal shall be extended by 14 days from the date of the request;

($c$) within such longer period as they may direct, the time for making the appeal shall be extended by a period equal to~that longer period.
\end{enumerate}

(7) If an appellant or applicant to~whom an approved form is returned, or from whom further information is requested, does not complete and~return the form or send further information within the period of time specified in paragraph~(6), the Board must forward a copy of the appeal or application, together with any other relevant documents or evidence, to~a legally qualified panel member who must—
\begin{enumerate}\item[]
($a$) determine whether the appeal or application satisfies the requirement of paragraph~(1), and

($b$) inform the appellant or applicant and~the Board of his determination.
\end{enumerate}

(8) If—
\begin{enumerate}\item[]
($a$) an approved form is duly completed and~returned or further information is sent after the expiry of the period of time specified in paragraph~(6); and

($b$) no determination has been made under paragraph~(7) at the time the form or the further information is received by the Board,
\end{enumerate}
the Board must forward the duly completed form or further information to~the legally qualified panel member who must take into~account any further information or evidence set out in that form or the further information.

\subsection[32. Discontinuing action on appeals]{Discontinuing action on appeals}

32.  The Board may discontinue action on an appeal to~an appeal tribunal if—\looseness=-1
\begin{enumerate}\item[]
($a$) the appeal has not been forwarded to~the clerk to~an appeal tribunal or to~a legally qualified panel member; and

($b$) the appellant or an authorised representative of the appellant has given written notice that he does not wish the appeal to~continue.
\end{enumerate}

\subsection[33. Death of a party to~an appeal]{Death of a party to~an appeal}

33.---(1)  In any proceedings, on the death of a party to~those proceedings (other than a member of the Board), the Board may appoint such person as they think fit to~proceed with the appeal in the place of such deceased party.

(2) A grant of probate, confirmation or letters of administration to~the estate of the deceased party, whenever taken out, shall have no effect on an appointment made under paragraph~(1).

(3) If a person appointed under paragraph~(1) has, prior to~the date of such appointment, taken any action in relation to~the appeal on behalf of the deceased party, the effective date of appointment by the Board shall be the day immediately prior to~the first day on which such action was taken.

\section[Part VII --- Revocations, transitional provisions and consequential amendments]{Part VII\\*Revocations, transitional provisions and consequential amendments}

\renewcommand\parthead{--- Part VII}

\subsection[34. Revocations]{Revocations}

34.  The following provisions are hereby revoked—
\begin{enumerate}\item[]
($a$) in so far as they relate to~child benefit or guardian’s allowance under the Contributions and~Benefits Act, Parts II, III and~IV of, and~Schedule~2 to, the Decisions and~Appeals Regulations;

($b$) in so far as they relate to~child benefit or guardian’s allowance under the Contributions and~Benefits (NI) Act, Parts II, III and~IV of, and~Schedule~1 to, the Decisions and~Appeals Regulations (NI).
\end{enumerate}

\subsection[35. Transitional provisions]{Transitional provisions}

35.  Anything done or commenced under any provision revoked by regulation~34, so far as relating to~child benefit or guardian’s allowance, is to~be treated as having been done or as being continued under the corresponding provision of these Regulations.

\subsection[36. Consequential amendments to~the Decisions and~Appeals Regulations]{Consequential amendments to~the Decisions and~Appeals Regulations}

36.---(1)  The Decisions and~Appeals Regulations (so far as relating to~child benefit and~guardian’s allowance under the Contributions and~Benefits Act) are amended as follows.

(2) In regulation~1(3) in the definition of “out of jurisdiction appeal” for “regulation~27” substitute “regulation~25 of the Child Benefit and~Guardian’s Allowance (Decisions and~Appeals) Regulations 2003”.

\subsection[37. Consequential amendments to~the Decisions and~Appeals (NI) Regulations]{Consequential amendments to~the Decisions and~Appeals (NI) Regulations}

37.---(1)  The Decisions and~Appeals Regulations (NI) (so far as relating to~child benefit and~guardian’s allowance under the Contributions and~Benefits (NI) Act) are amended as follows.

(2) In regulation~1(2) in the definition of “out of jurisdiction appeal” for “regulation~27” substitute “regulation~25 of the Child Benefit and~Guardian’s Allowance (Decisions and~Appeals) Regulations 2003”. 

\bigskip

%Signed 
%by authority of the 
%Secretary of State for~Work and~Pensions.
%I concur
%By authority of the Lord Chancellor

{\raggedleft
\emph{Nick Montagu}\\*
\emph{Tim Flesher}\\*
%Secretary
%Minister
%Parliamentary Under-Secretary 
%of State\\%*Department 
%%for~Work and~Pensions
%Ministry of Justice
Two of the Commissioners of Inland~Revenue

}

27th March 2003


\small

\part[Schedule~1 --- Powers exercised in making these regulations]{Schedule~1\\*Powers exercised in making these regulations}

\renewcommand\parthead{--- Schedule~1}

1.  Section~5(1)($hh$)  of the Administration Act\footnote{Section~5(1)($hh$) was inserted by section 74 of the Social Security Act 1998.}.

\medskip

2.  Section~5(1)($hh$)  of the Administration (NI) Act\footnote{Section 5(1)($hh$) was inserted by Article 70 of S.I.~1998/1506 (N.I.~10).}.

\medskip

3.  The following provisions of the 1998 Act—
\begin{enumerate}\item[]
($a$) section~9(1), (4) and~(6);

($b$) section~10(3) and~(6);

($c$) section~12(2), (3), (6) and~(7);

($d$) section~16(1) and~paragraphs~1 to~4 and~6 of Schedule~5;

($e$) section~21\footnote{Section~21 was amended by paragraph 32 of Schedule~7, and Part I of Schedule~10, to the Transfer Act 1999.};

($f$) section~22;

($g$) section~23;

($h$) section~25(3)($b$)  and~(5)($c$);

($i$) section~26(6)($c$);

($j$) section~79(1), (2A) and~(4) to~(7)\footnote{Subsection (1) of section 79 was amended, and subsection (2A) of that section inserted, by paragraph 13 of Schedule~4 to the Tax Credits Act 2002.};

($k$) section~84\footnote{Section~84 is cited because of the definition of “prescribe”.}.
\end{enumerate}

\medskip

4.  The following provisions of the 1998 Order—
\begin{enumerate}\item[]
($a$) Article~2(2)\footnote{Article 2(2) is cited because of the definition of “prescribe”.};

($b$) Article~10(1), (4) and~(6);

($c$) Article~11(3) and~(6);

($d$) Article~13(2), (3), (6) and~(7);

($e$) Article~16(1) and~paragraphs~1 to~4 and~6 of Schedule~4;

($f$) Article~21\footnote{Article 21 was amended by Schedule~9 to the Transfer Order 1999.};

($g$) Article~22;

($h$) Article~23;

(i) Article~25(3)($b$)  and~(5)($c$);

($j$) Article~26(6)($c$);

($k$) Article~74(1) and~(3) to~(6)\footnote{Paragraph (1) of Article 74 was amended, and paragraph (2A) of that Article inserted, by paragraph 17 of Schedule~4 to the Tax Credits Act 2002.}.
\end{enumerate}

\medskip

5.  Sections~132 and~133(1) and~(2) of the Finance Act 1999.

\medskip

6.  The following provisions of the Tax Credits Act 2002—
\begin{enumerate}\item[]
($a$) section~50(1) and~(2)($e$)  and~($f$);

($b$) section~54(2);

($c$) paragraphs~15 and~19 of Schedule~4.
\end{enumerate}

\part[Schedule~2 --- Decisions against which no appeal lies]{Schedule~2\\*Decisions against which no appeal lies}

\section[Part I --- Decisions made under primary legislation]{Part I\\*Decisions made under primary legislation}

\renewcommand\parthead{--- Schedule~2 Part I}

1.  A decision of the Board whether to~recognise, for the purposes of Part~IX of the Contributions and~Benefits Act or Part~X of the Contributions and~Benefits (NI) Act—
\begin{enumerate}\item[]
($a$) an educational establishment; or

($b$) education provided otherwise than at a recognised educational establishment.
\end{enumerate}

\medskip

2.  A decision of the Board whether to~pay expenses to~any person under—
\begin{enumerate}\item[]
($a$) sections 180 and~180A of the Administration Act\footnote{Section 180 was amended by paragraph 71 of Schedule~2 to the Jobseekers Act 1995, paragraph 9 of Schedule~3 to the Social Security (Recovery of Benefits) Act 1997 (c.~27), paragraph 108 of Schedule~7 to the Social Security Act 1998 and paragraph 22 of Schedule~2 to the State Pension Credit Act 2002. Section 180A was inserted by paragraph 16 of Schedule~7 to the Transfer Act 1999.}; or

($b$) section~156 of the Administration (NI) Act\footnote{Section 156 was amended by paragraph 49 of Schedule 2 to the Jobseekers (Northern Ireland) Order 1995 and paragraph 8 of Schedule 3 to the Social Security (Recovery of Benefits) (Northern Ireland) Order 1997 (S.I.~1997/1183 (N.I.~12)).}.
\end{enumerate}

\medskip

3.  A decision of the Treasury relating to~the up-rating of child benefit or guardian’s allowance under—
\begin{enumerate}\item[]
($a$) Part~X of the Administration Act; or

($b$) Part~IX of the Administration (NI) Act.
\end{enumerate}

\medskip

4.  A decision of the Board under—
\begin{enumerate}\item[]
($a$) section~25 or 26; or

($b$) Article~25 or 26.
\end{enumerate}

\section[Part II --- Decisions made under secondary legislation]{Part II\\*Decisions made under secondary legislation}

\renewcommand\parthead{--- Schedule 2 Part II}

5.  A decision of the Board relating to—
\begin{enumerate}\item[]
($a$) the suspension of child benefit or allowance under Part~IV; or

($b$) the payment of such a benefit or allowance which has been so suspended.
\end{enumerate}

\medskip

6.  A decision of the Board under any of the following provisions of the Administration Regulations—
\begin{enumerate}\item[]
($a$) regulation~5 (decision as to~making a claim for benefit or allowance);

($b$) regulation~7 (decision as to~evidence and~information required);

($c$) regulation~10 (decision as to~defective applications);

($d$) regulation~11 (decision as to~claims for child benefit treated as claims for guardian’s allowance and~vice versa);

($e$) regulation~18 (decision as to~the time of payments);

($f$) regulation~19 (decision as to~elections to~have child benefit paid weekly);

($g$) regulation~23 (decision as to~information to~be given);

($h$) regulation~26 (decision as to~extinguishment of right to~payment if payment is not obtained within the prescribed period) other than a decision under paragraph~(5) (decision as to~payment request after expiration of prescribed period);

($i$) regulation~28 (decision as to~appointments where person unable to~act);

($j$) regulations 29 to~32 (decisions as to~claims or payments after death of claimant);

($k$) regulation~33 (decision as to~paying a person on behalf of another);

($l$) regulation~34 (decision as to~paying partner as alternative payee);

($m$) Part~V other than a decision under—
\begin{enumerate}\item[]
(i) regulation~35(1) (decision as to~whether a payment in excess of entitlement has been credited to~a bank or other account);

(ii) regulation~37 (decision as to~the sums to~be deducted in calculating recoverable amounts);

(iii) regulation~38 (decision as to~the offsetting of a prior payment of child benefit or guardian’s allowance against arrears of child benefit or guardian’s allowance payable by virtue of a subsequent determination);

(iv) regulation~39 (decision as to~the offsetting of a prior payment of income support or jobseeker’s allowance against arrears of child benefit or guardian’s allowance payable by virtue of a subsequent determination);

(v) regulation~41(1) (decision as to~bringing interim payments into~account);

(vi) regulation~42(1) (decision as to~the overpayment of an interim payment).
\end{enumerate}
\end{enumerate}

\medskip

7.  A decision of the Board made in accordance with the discretion conferred upon them by the following regulations of the Child Benefit (General) Regulations 2003\footnote{S.I.~2003/493.}—
\begin{enumerate}\item[]
($a$) regulation~4(1) or (4) (provisions relating to~contributions and~expenses in respect of a child);

($b$) regulation~24(1)($c$)  or 28(1)($c$)  (decisions relating to~a child’s temporary absence abroad).
\end{enumerate}

\medskip

8.  A decision of the Board relating to~the giving of a notice under regulation~8(2) of the Guardian’s Allowance (General) Regulations 2003 (children whose surviving parents are in prison or legal custody)\footnote{S.I.~2003/495.}.

\medskip

9.  A decision of the Board made in accordance with an Order made under—
\begin{enumerate}\item[]
($a$) section~179 of the Administration Act (reciprocal agreements with countries outside the United Kingdom)\footnote{Section 179 was amended by paragraph 70 of Schedule 2 to the Jobseekers Act 1995, paragraph 107 of Schedule 7 to the Social Security Act 1998, paragraph 15 of Schedule 7 to the Transfer Act 1999, paragraph 2 of Schedule 6 to the Transfer Order 1999, paragraphs 1 and 3($g$) of Schedule 1 to the Tax Credits Act 1999, paragraph 21 of Schedule 2 to the State Pension Credit Act 2002 and Schedule 6 to the Tax Credits Act 2002.}; or

($b$) section~155 of the Administration (NI) Act (reciprocal agreements with countries outside the United Kingdom)\footnote{Section 155 was amended by paragraph 48 of Schedule 2 to the Jobseekers (Northern Ireland) Order 1995 and Schedule 6 to the Tax Credits Act 2002.}.
\end{enumerate}

\section[Part III --- Other decisions]{Part III\\*Other decisions}

\renewcommand\parthead{--- Schedule 2 Part III}

10.  An authorization given by the Board in accordance with Article~22(1) or 55(1) of Council Regulation~(EEC) No.~1408/71\footnote{\frenchspacing O.J. No. L149/2, 5.7.1971 (O.J./S.E. 1971 (II) p. 416.} on the application of social security schemes to~employed persons, to~self-employed persons and~to~members of their families moving within the Community. 

\part{Explanatory Note}

\renewcommand\parthead{— Explanatory Note}

\subsection*{(This note is not part of the Regulations)}

These Regulations make provision in relation to~the administration of child benefit and~guardian’s allowance which is to~be transferred from the Department for Work and~Pensions (in Northern Ireland, the Department for Social Development) to~the Commissioners of Inland~Revenue (“the Board”) by Part~II of the Tax Credits Act 2002 (c.~21) with effect from 1st April 2003.

The Regulations are modelled closely on the provisions, so far as relating to~child benefit and~guardian’s allowance, that are contained in—
\begin{enumerate}\item[]
($a$) for Great Britain, Parts II to~IV of the Social Security (Decisions and~Appeals) Regulations 1999 (S.I.~1987/1968); and

($b$) for Northern Ireland, Parts II to~IV of the Social Security and~Child Support (Decisions and~Appeals) Regulations (Northern Ireland) 1999 (S.R.~1999 No.~162).
\end{enumerate}

The opportunity is being taken to~combine the regulations in a single set relating to~child benefit and~guardian’s allowance that extends to~both Great Britain and~Northern Ireland.

The Regulations are made by virtue of, or in consequence of, provisions in the Social Security Act 1998 (c.~14) (“the Act”) and~the decision-making process introduced by that Act. The Regulations also make provision in relation to~the unified appeals system introduced by the Act and~are made after consultation with the Council on Tribunals in accordance with section~8 of the Tribunals and~Inquiries Act 1992 (c.~53).

Part~I (regulations 1 to~4) provides for citation, commencement and~effect and~interpretation. It contains also provisions relating to~the service of notices or documents and~the use of electronic communications.

Parts II and~III (regulations 5 to~17) provide for the circumstances in which the Board may revise or supersede decisions, when such decisions take effect and~related procedural rules.

Part~IV (regulations 18 to~21) makes provision in relation to~the suspension and~termination of benefit and~for dealing with decisions where there are related appeals or decisions.

Part~V (regulations 22 to~23) makes provision in relation to~decisions involving issues that arise on appeal in other cases and~appeals involving issues that arise on appeal in other cases.

Part~VI (regulations 24 to~33) makes provision in relation to~rights of appeal and~the procedure for bringing appeals. In particular, it provides for additional persons to~have a right of appeal and, together with Schedule~2, prescribes decisions against which there is no right of appeal. It provides also for procedural rules for bringing appeals.

Part~VII (regulations 34 to~37) provides for revocations, transitional provisions and~consequential amendments. 

\end{document}