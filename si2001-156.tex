\documentclass[12pt,a4paper]{article}

\newcommand\regstitle{The Child Support (Variations) Regulations 2000}

\newcommand\regsnumber{2001/156}

%\opt{newrules}{
\title{\regstitle}
%}

%\opt{2012rules}{
%\title{Child Maintenance and~Other Payments Act 2008\\(2012 scheme version)}
%}

\author{S.I.\ 2001 No.\ 156}

\date{Made
18th January 2001\\
%Laid before Parliament
%27th July 2000\\
Coming into~force
as provided in regulation~1(1)
}

%\opt{oldrules}{\newcommand\versionyear{1993}}
%\opt{newrules}{\newcommand\versionyear{2003}}
%\opt{2012rules}{\newcommand\versionyear{2012}}

\usepackage{csa-regs}

\setlength\headheight{27.61603pt}

\begin{document}

\maketitle

\amendment{
Regs. revoked (10.12.12 for 2012 scheme cases only) by the Child Support (Meaning of Child and New Calculation Rules) (Consequential and Miscellaneous Amendment) Regulations 2012 reg.~10(f).
}

\medskip

\noindent
Whereas a draft of this Instrument was laid before Parliament in accordance with section 52(2) of the Child Support Act 1991\footnote{1991 c.\ 48.} and~approved by a resolution of each House of Parliament:

Now, therefore, the Secretary of State for Social Security, in exercise of the powers conferred upon him by sections 28A(5), 28B(2)($c$), 28C(2)($b$)  and~(5), 28E(1) and~(5), 28F(2)($b$)  and~(3)($b$), 28G(3), 51, 52(4) and~54 of, and~parargaphs~1, 2($a$), 4 and~5(1) of Schedule~4A, and~parargaphs~2(2) to~(5), 3(1) and~(2), 4, 5(1) and~(3) to~(5) and~6 of Schedule~4B to~the Child Support Act 1991\footnote{Section~28A, 28B, 28C and~28F and~Schedules 4A and~4B are substituted by section 5 of, and~Schedule~2 to, the Child Support, Pensions and~Social Security Act 2000 c.\ 19. Section~28E was inserted by the Child Support Act 1995 c.\ 34 and~is amended by section 5 of the Child Support, Pensions and~Social Security Act 2000. Section~54 is cited because of the meaning assigned to~the word “prescribed” and~Schedule~4A paragraph~1 is cited because of the meaning assigned to~the word “regulations”.} hereby makes the following Regulations: 

{\sloppy

\tableofcontents

}

\bigskip

\setcounter{secnumdepth}{-2}

\section[Part I --- General]{Part I\\*General}

\renewcommand\parthead{--- Part I}

\subsection[1. Citation, commencement and~interpretation]{Citation, commencement and~interpretation}

1.---(1)  These Regulations may be cited as the Child Support (Variations) Regulations 2000 and~shall come into~force in relation to~a particular case on the day on which section 5 of the Child Support, Pensions and~Social Security Act 2000 which substitutes or amends sections 28A to~28F of the Act is commenced in relation to~that type of case.

(2) In these Regulations, unless the context otherwise requires—
\begin{enumerate}\item[]
“the Act” means the Child Support Act 1991;

“capped amount” means the amount of income for the purposes of paragraph~10(3) of Schedule~1 to~the Act;

% Definition of ``the Commission'' inserted (6.4.09) by SI 2009/736 reg 4(2), omitted (1.8.12) by SI 2012/2007 Sch para 115(a)
%“the Commission” means the Child Maintenance and~Enforcement Commission;

“Contributions and~Benefits Act” means the Social Security Contributions and~Benefits Act 1992\footnote{1992 c.\ 4.};

“couple” has the same meaning as in paragraph~10C(5) of Schedule~1 to~the Act;

“date of notification” means the date upon which notification is given in person or communicated by telephone to~the recipient or, where this is not possible, the date of posting;

“date of receipt” means the day on which the information or document is actually received;

“home” has the meaning given in regulation~1(2) of the Maintenance Calculations and~Special Cases Regulations;

“Maintenance Calculation Procedure Regulations” means the Child Support (Maintenance Calculation Procedure) Regulations 2000\footnote{S.I.\ 2001/157.};

“Maintenance Calculations and~Special Cases Regulations” means the Child Support (Maintenance Calculations and~Special Cases) Regulations 2000\footnote{S.I.\ 2001/155.};

% Definition of ``partner'' inserted (16.9.04) by SI 2004/2416 reg 9(2)
“partner” has the same meaning as in paragraph~10C(4) of Schedule~1 to~the Act;

“qualifying child” means the child with respect to~whom the maintenance calculation falls to~be made;

“relevant person” means—
\begin{enumerate}\item[]
($a$) 
a non-resident parent, or a person treated as a non-resident parent under regulation~8 of the Maintenance Calculations and~Special Cases Regulations, whose liability to~pay child support maintenance may be affected by any variation agreed;

($b$) 
a person with care, or a child to~whom section 7 of the Act applies, where the amount of child support maintenance payable by virtue of a calculation relevant to~that person with care or in respect of that child may be affected by any variation agreed; and
\end{enumerate}

“Transitional Regulations” means the Child Support (Transitional Provisions) Regulations 2000\footnote{S.I.\ 2000/3186.}.
\end{enumerate}

(3) In these Regulations, unless the context otherwise requires, a reference—
\begin{enumerate}\item[]
($a$) to~a numbered Part, is to~the Part of these Regulations bearing that number;

($b$) to~the Schedule, is to~the Schedule~to~these Regulations;

($c$) to~a numbered regulation, is to~the regulation~in these Regulations bearing that number;

($d$) in a regulation, or the Schedule, to~a numbered paragraph, is to~the paragraph~in that regulation~or the Schedule~bearing that number; and

($e$) in a paragraph~to~a lettered or numbered sub-paragraph, is to~the sub-paragraph~in that paragraph~bearing that letter or number.
\end{enumerate}

\amendment{
Definition of ``partner'' inserted in reg. 1(2) (16.9.04) by the Child Support (Miscellaneous Amendments) Regulations 2004 reg. 9(2).

Definition of ``the Commission'' inserted in reg.~1(2) 
(6.4.09) by the Child Support (Miscellaneous and~Consequential Amendments) Regulations 2009 reg.~4(2).

Definition of ``the Commission'' in reg.~1(2) omitted (1.8.12) by the Public Bodies (Child Maintenance and Enforcement Commission: Abolition and Transfer of Functions) Order 2012 Sch. para.~115(a).
}

\subsection[2. Documents]{Documents}

2.  Except where otherwise stated, where—
\begin{enumerate}\item[]
($a$) any document is given or sent to~the Secretary of State, that document shall be treated as having been so given or sent on the date of receipt by the Secretary of State; and

($b$) any document is given or sent to~any other person, that document shall, if sent by post to~that person’s last known or notified address, be treated as having been given or sent on the date that it is posted.
\end{enumerate}

\subsection[3. Determination of amounts]{Determination of amounts}

3.---(1)  Where any amount is required to~be determined for the purposes of these Regulations, it shall be determined as a weekly amount and, except where the context otherwise requires, any reference to~such an amount shall be construed accordingly.

(2) Where any calculation made under these Regulations results in a fraction of a penny, that fraction shall be treated as a penny if it is either one half or exceeds one half and~shall be otherwise disregarded.

\vfill

\section[Part II --- Application and~determination procedure]{Part II\\*Application and~determination procedure}

\renewcommand\parthead{--- Part II}

\subsection[4. Application for a variation]{Application for a variation}

4.---(1)  Where an application for a variation is made other than in writing and~the Secretary of State directs that the application be made in writing, the application shall be made either on an application form provided by the Secretary of State and~completed in accordance with the Secretary of State’s instructions or in such other written form as the Secretary of State may accept as sufficient in the circumstances of any particular case.

(2) An application for a variation which is made other than in writing shall be treated as made on the date of notification from the applicant to~the Secretary of State that he wishes to~make such an application.

(3) Where an application for a variation is made in writing other than in the circumstances to~which paragraph~(1) applies, the application shall be treated as made on the date of receipt by the Secretary of State.

(4) Where paragraph~(1) applies and~the Secretary of State receives the application within 14 days of the date of the direction, or at a later date but in circumstances where the Secretary of State is satisfied that the delay was unavoidable, the application shall be treated as made on the date of notification from the applicant to~the Secretary of State that he wishes to~make an application for a variation.

(5) Where paragraph~(1) applies and~the Secretary of State receives the application more than 14 days from the date of the direction and~in circumstances where he is not satisfied that the delay was unavoidable, the application shall be treated as made on the date of receipt.

(6) An application for a variation is duly made when it has been made in accordance with this regulation~and~section 28A(4) of the Act.

\subsection[5. Amendment or withdrawal of application]{Amendment or withdrawal of application}

5.---(1)  A person who has made an application for a variation may amend or withdraw his application at any time before a decision under section 11,~16 or 17 of the Act, or a decision not to~revise or supersede under section 16 or 17 of the Act, is made in response to~the variation application and~such amendment or withdrawal need not be in writing unless, in any particular case, the Secretary of State requires it to~be.

(2) No amendment under paragraph~(1) shall relate to~any change of circumstances arising after what would be the effective date of a decision in response to~the variation application.

\subsection[6. Rejection of an application following preliminary consideration]{Rejection of an application following preliminary consideration}

6.---(1)  The Secretary of State may, on completing the preliminary consideration, reject an application for a variation (and~proceed to~make his decision on the application for a maintenance calculation, or to~revise or supersede a decision under section 16 or 17 of the Act, without the variation, or not to~revise or supersede a decision under section 16 or 17 of the Act, as the case may be) if one of the circumstances in paragraph~(2) applies.

(2) The circumstances are—
\begin{enumerate}\item[]
($a$) the application has been made in one of the circumstances to~which regulation~7 applies;

($b$) the application is made—
\begin{enumerate}\item[]
(i) on a ground in paragraph~2 of Schedule~4B to~the Act (special expenses) and~the amount of the special expenses, or the aggregate amount of those expenses, as the case may be, does not exceed the relevant threshold provided for in regulation~15;

(ii) on a ground in paragraph~3 of that Schedule~(property or capital transfers) and~the value of the property or capital transferred does not exceed the minimum value in regulation~16(4); or

(iii) on a ground referred to~in regulation~18 (assets) and~the value of the assets does not exceed the figure in regulation~18(3)($a$), or on a ground in regulation~19(1) 
or (1A)  % Words inserted (16.3.05) by SI 2005/785 reg 8(2)
(income not taken into~account) and~the amount of the income does not exceed the figure in regulation~19(2);
\end{enumerate}

($c$) a request under regulation~8 has not been complied with by the applicant and~the Secretary of State is not able to~determine the application without the information requested; or

($d$) the Secretary of State is satisfied, on the information or evidence available to~him, that the application would not be agreed to, including where, although a ground is stated, the facts alleged in the application would not bring the case within the prescription of the relevant ground in these Regulations.
\end{enumerate}

\amendment{
Words inserted in reg. 6(2)(b)(iii) (6.4.05) by the Child Support (Miscellaneous Amendments) Regulations 2005 reg. 8(2).
}

\subsection[7. Prescribed circumstances]{Prescribed circumstances}

7.---(1)  This regulation~applies where an application for a variation is made under 
%section 28G 
section 28A or 28G  % Words substituted (30.4.02) by SI 2002/1204 reg 9(2)(a)
of the Act and—
\begin{enumerate}\item[]
($a$) the application is made by a relevant person and~a circumstance set out in paragraph~(2) applies at the relevant date;

($b$) the application is made by a non-resident parent and~a circumstance set out in paragraph~(3) or (4) applies at the relevant date;

($c$) the application is made by a person with care, or a child to~whom section 7 of the Act applies, on a ground in paragraph~4 of Schedule~4B to~the Act (additional cases) and~a circumstance set out in paragraph~(5) applies at the relevant date; or

($d$) the application is made by a non-resident parent on a ground in paragraph~2 of Schedule~4B to~the Act (special expenses) and~a circumstance set out in paragraph~(6) applies at the relevant date.
\end{enumerate}

(2) The circumstances for the purposes of this paragraph~are that—
\begin{enumerate}\item[]
($a$) a default maintenance decision is in force with respect to~the non-resident parent;

($b$) the non-resident parent is liable to~pay the flat rate of child support maintenance owing to~the application of paragraph~4(1)($c$)  of Schedule~1 to~the Act, or would be so liable but is liable to~pay less than that amount, or nil, owing to~the application of paragraph~8 of Schedule~1 to~the Act, or the Transitional Regulations; or

($c$) the non-resident parent is liable to~pay child support maintenance at a flat rate of a prescribed amount owing to~the application of paragraph~4(2) of Schedule~1 to~the Act, or would be so liable but is liable to~pay less than that amount, or nil, owing to~the application of paragraph~8 of Schedule~1 to~the Act, or the Transitional Regulations.
\end{enumerate}

(3) The circumstances for the purposes of this paragraph~are that the non-resident parent is liable to~pay child support maintenance—
\begin{enumerate}\item[]
($a$) at the nil rate owing to~the application of paragraph~5 of Schedule~1 to~the Act;

($b$) at a flat rate owing to~the application of paragraph~4(1)($a$)  of Schedule~1 to~the Act, including where the net weekly income of the non-resident parent which is taken into~account for the purposes of a maintenance calculation in force in respect of him is £100 per week or less owing to~a variation being taken into~account or to~the application of regulation~18, 19 or 21 of the Transitional Regulations (reduction for relevant departure direction or relevant property transfer); or

($c$) at a flat rate owing to~the application of paragraph~4(1)($b$)  of Schedule~1 to~the Act, or would be so liable but is liable to~pay less than that amount, or nil, owing to~the application of paragraph~8 of Schedule~1 to~the Act, or the Transitional Regulations.
\end{enumerate}

(4) The circumstances for the purposes of this paragraph~are that the non-resident parent is liable to~pay an amount of child support maintenance at a rate—
\begin{enumerate}\item[]
($a$) of £5 per week or such other amount as may be prescribed owing to~the application of paragraph~7(7) of Schedule~1 to~the Act (shared care); or

($b$) equivalent to~the flat rate provided for in, or prescribed for the purposes of, paragraph~4(1)($b$)  of Part I of Schedule~1 to~the Act owing to~the application of—
\begin{enumerate}\item[]
(i) regulation~27(5);

(ii) regulation~9 of the Maintenance Calculations and~Special Cases Regulations (care provided in part by a local authority); or

(iii) regulation~23(5) of the Transitional Regulations.
\end{enumerate}
\end{enumerate}

(5) The circumstances for the purposes of this paragraph~are that—
\begin{enumerate}\item[]
($a$) the amount of the net weekly income of the non-resident parent to~which the Secretary of State had regard when making the maintenance calculation was the capped amount; or

($b$) the non-resident parent or a partner of his is in receipt of 
%working families' tax credit (as defined in section 128 of the Contributions and~Benefits Act) or disabled person’s tax credit (as defined in section 129\footnote{Sections 128 and~129 were amended by section 1(2) of Schedule~1 to~the Tax Credits Act 1999 (c.\ 10).} of that Act) 
working tax credit under section 10 of the Tax Credits Act 2002%  % Words substituted (6.4.03) by SI 2003/328 reg 10
%and~for this purpose “partner” has the same meaning as in paragraph~10C(4) of Schedule~1 to~the Act  % Words omitted (16.9.04) by SI 2004/2415 reg 9(3)
.
\end{enumerate}

(6) The circumstances for the purposes of this paragraph~are that the amount of the net weekly income of the non-resident parent to~which the Secretary of State would have regard after deducting the amount of the special expenses would exceed the capped amount.

(7) For the purposes of paragraph~(1), the “relevant date” means the date from which, if the variation were agreed
and~the application had been made under section 28G of the Act%  % Words inserted (30.4.02) by SI 2002/1204 reg 9(2)(b)(i)
, the decision under section 16 or 17 of the Act, as the case may be, would take effect
and~if the variation were agreed, and~the application had been made under section 28A of the Act, the decision under section 11 of the Act would take effect%  % Words inserted (30.4.02) by SI 2002/1204 reg 9(2)(b)(ii)
.

\amendment{
Words inserted in reg. 7(7) and~words substituted in reg. 7(1) (30.4.02) by the Child Support (Miscellaneous Amendments) Regulations 2002 reg. 9(2).

Words substituted in reg. 7(5)(b) (6.4.03) by the Child Support (Miscellaneous Amendments) Regulations 2003 reg. 10.

Words omitted in reg. 7(5)(b) (16.9.04) by the Child Support (Miscellaneous Amendments) Regulations 2004 reg. 9(3).
}

\subsection[8. Provision of information]{Provision of information}

8.---(1)  Where an application has been duly made, the Secretary of State may request further information or evidence from the applicant to~enable that application to~be determined and~any such information or evidence requested shall be provided within one month of the date of notification of the request or such longer period as the Secretary of State is satisfied is reasonable in the circumstances of the case.

(2) Where any information or evidence requested in accordance with paragraph~(1) is not provided in accordance with the time limit specified in that paragraph, the Secretary of State may, where he is able to~do so, proceed to~determine the application in the absence of the requested information or evidence.

\subsection[9. Procedure in relation to~the determination of an application]{Procedure in relation to~the determination of an application}

9.---(1)  Subject to~paragraph~(3), where the Secretary of State has given the preliminary consideration to~an application and~not rejected it he—
\begin{enumerate}\item[]
($a$) shall give notice of the application to~the relevant persons other than the applicant, informing them of the grounds on which the application has been made and~any relevant information or evidence the applicant has given, except information or evidence falling within paragraph~(2);

($b$) may invite representations, which need not be in writing but shall be in writing if in any case he so directs, from the relevant persons other than the applicant on any matter relating to~that application, to~be submitted to~the Secretary of State within 14 days of the date of notification or such longer period as the Secretary of State is satisfied is reasonable in the circumstances of the case; and

($c$) shall set out the provisions of parargaphs~(2)($b$)  and~($c$), (4) and~(5) in relation to~such representations.
\end{enumerate}

(2) The information or evidence referred to~in parargaphs~(1)($a$), (4)($a$)  and~(7), are—
\begin{enumerate}\item[]
($a$) details of the nature of the long-term illness or disability of the relevant other child which forms the basis of a variation application on the ground in regulation~11 where the applicant requests they should not be disclosed and~the Secretary of State is satisfied that disclosure is not necessary in order to~be able to~determine the application;

($b$) medical evidence or medical advice which has not been disclosed to~the applicant or a relevant person and~which the Secretary of State considers would be harmful to~the health of the applicant or that relevant person if disclosed to~him; or

($c$) the address of a relevant person or qualifying child, or any other information which could reasonably be expected to~lead to~that person or child being located, where the Secretary of State considers that there would be a risk of harm or undue distress to~that person or that child or any other children living with that person if the address or information were disclosed.
\end{enumerate}

(3) The Secretary of State need not act in accordance with paragraph~(1)—
\begin{enumerate}\item[]
($a$) where regulation~29 applies (variation may be taken into~account notwithstanding that no application has been made);

($b$) where the variation agreed is one falling within paragraph~3 of Schedule~4B to~the Act (property or capital transfer), the Secretary of State ceases to~have jurisdiction to~make a maintenance calculation and~subsequently acquires jurisdiction in respect of the same non-resident parent, person with care and~any child in respect of whom the earlier calculation was made;

($c$) if he is satisfied on the information or evidence available to~him that the application would not be agreed to, but if, on further consideration of the application, he is minded to~agree to~the variation he shall, before doing so, comply with the provisions of this regulation; or

($d$) where—
\begin{enumerate}\item[]
(i) a variation has been agreed in relation to~a maintenance calculation;

(ii) the decision as to~the maintenance calculation is replaced with a default maintenance decision under section 12(1)($b$)  of the Act;

(iii) the default maintenance decision is revised in accordance with section 16(1B) of the Act,
\end{enumerate}
and~the Secretary of State is satisfied, on the information or evidence available to~him, that there has been no material change of circumstances relating to~the variation since the date from which the maintenance calculation referred to~in head (i)  ceased to~have effect.
\end{enumerate}

(4) Where the Secretary of State receives representations from the relevant persons—
\begin{enumerate}\item[]
($a$) he may, if he considers it reasonable to~do so, send a copy of the representations concerned (excluding material falling within paragraph~(2)) to~the applicant and~invite any comments he may have within 14 days or such longer period as the Secretary of State is satisfied is reasonable in the circumstances of the case; and

($b$) where the Secretary of State acts under sub-paragraph~($a$)  he shall not proceed to~determine the application until he has received such comments or the period referred to~in that sub-paragraph~has expired.
\end{enumerate}

(5) Where the Secretary of State has not received representations from the relevant persons notified in accordance with paragraph~(1) within the time limit specified in sub-paragraph~($b$)  of that paragraph, he may proceed to~agree or not (as the case may be) to~a variation in their absence.

(6) In considering an application for a variation, the Secretary of State shall take into~account any representations received at the date upon which he agrees or not (as the case may be) to~the variation from the relevant persons, including any representation received in accordance with parargaphs~(1)($b$),~%
%4($a$)  
(4)($a$)  % Words substituted (30.4.02) by SI 2002/1204 reg 9(3)
and~(7).

(7) Where any information or evidence requested by the Secretary of State under regulation~8 is received after notification has been given under paragraph~(1), the Secretary of State may, if he considers it reasonable to~do so, and~except where such information or evidence falls within paragraph~(2), send a copy of such information or evidence to~the relevant persons and~may invite them to~submit representations, which need not be in writing unless the Secretary of State so directs in any particular case, on that information or evidence.

(8) The Secretary of State may, if he considers it appropriate, treat an application for a variation made on one ground as if it were an application made on a different ground, and, if he does intend to~do so, he shall include this information in the notice and~invitation to~make representations referred to~in parargaphs~(1), (4) and~(7).

(9) Two or more applications for a variation with respect to~the same maintenance calculation or application for a maintenance calculation, made or treated as made, may be considered together.

\amendment{
Words substituted in reg. 9(6) (30.4.02) by the Child Support (Miscellaneous Amendments) Regulations 2002 reg. 9(3).
}

\section[Part III --- Special expenses]{Part III\\*Special expenses}

\subsection[10. Special expenses—contact costs]{Special expenses—contact costs}

\renewcommand\parthead{--- Part III}

10.---(1)  Subject to~the following parargaphs~of this regulation, and~to~regulation~15, the following costs incurred or reasonably expected to~be incurred by the non-resident parent, whether in respect of himself or the qualifying child or both, for the purpose of maintaining contact with that child, shall constitute expenses for the purposes of paragraph~2(2) of Schedule~4B to~the Act—
\begin{enumerate}\item[]
($a$) the cost of purchasing a ticket for travel;

($b$) the cost of purchasing fuel where travel is by a vehicle which is not carrying fare-paying passengers;

($c$) the taxi fare for a journey or part of a journey where the Secretary of State is satisfied that the disability or long-term illness of the non-resident parent or the qualifying child makes it impracticable for any other form of transport to~be used for that journey or part of that journey;

($d$) the cost of car hire where the cost of the journey would be less in total than it would be if public transport or taxis or a combination of both were used;

($e$) where the Secretary of State considers a return journey on the same day is impracticable, or the established or intended pattern of contact with the child includes contact over two or more consecutive days, the cost of the non-resident parent's, or, as the case may be, the child's, accommodation for the number of nights the Secretary of State considers appropriate in the circumstances of the case; and

($f$) any minor incidental costs such as tolls or fees payable for the use of a particular road or bridge incurred in connection with such travel, including breakfast where it is included as part of the accommodation cost referred to~in sub-paragraph~($e$).
\end{enumerate}

(2) The costs to~which paragraph~(1) applies include the cost of a person to~travel with the non-resident parent or the qualifying child, if the Secretary of State is satisfied that the presence of another person on the journey, or part of the journey, is necessary including, but not limited to, where it is necessary because of the young age of the qualifying child or the disability or long-term illness of the non-resident parent or that child.

(3) The costs referred to~in parargaphs~(1) and~(2)—
\begin{enumerate}\item[]
($a$) shall be expenses for the purposes of paragraph~2(2) of Schedule~4B to~the Act only to~the extent that they are—
\begin{enumerate}\item[]
(i) incurred in accordance with a set pattern as to~frequency of contact between the non-resident parent and~the qualifying child which has been established at or, where at the time of the variation application it has ceased, which had been established before, the time that the variation application is made; or

(ii) based on an intended set pattern for such contact which the Secretary of State is satisfied has been agreed between the non-resident parent and~the person with care of the qualifying child; and
\end{enumerate}

($b$) shall be—
\begin{enumerate}\item[]
(i) where head (i)  of sub-paragraph~($a$)  applies and~such contact is continuing, calculated as an average weekly amount based on the expenses actually incurred over the period of 12 months, or such lesser period as the Secretary of State may consider appropriate in the circumstances of the case, ending immediately before the first day of the maintenance period from which a variation agreed on this ground would take effect;

(ii) where head (i)  of sub-paragraph~($a$)  applies and~such contact has ceased, calculated as an average weekly amount based on the expenses actually incurred during the period from the first day of the maintenance period from which a variation agreed on this ground would take effect to~the last day of the maintenance period in relation to~which the variation would take effect; or

(iii) where head (ii)  of sub-paragraph~($a$)  applies, calculated as an average weekly amount based on anticipated costs during such period as the Secretary of State considers appropriate.
\end{enumerate}
\end{enumerate}

(4) For the purposes of this regulation, costs of contact shall not include costs which relate to~periods where the non-resident parent has care of a qualifying child overnight as part of a shared care arrangement for which provision is made under parargaphs~7 and~8 of Schedule~1 to~the Act and~regulation~7 of the Maintenance Calculations and~Special Cases Regulations.

(5) Where the non-resident parent has at the date he makes the variation application received, or at that date is in receipt of, or where he will receive, any financial assistance, other than a loan, from any source to~meet, wholly or in part, the costs of maintaining contact with a child as referred to~in paragraph~(1), only the amount of the costs referred to~in that paragraph, after the deduction of the financial assistance, shall constitute special expenses for the purposes of paragraph~2(2) of Schedule~4B to~the Act.

\subsection[11. Special expenses—illness or disability of relevant other child]{\sloppy Special expenses—illness or disability of relevant other child}

11.---(1)  Subject to~the following parargaphs~of this regulation, expenses necessarily incurred by the non-resident parent in respect of the items listed in sub-parargaphs~($a$)  to~($m$)  due to~the long-term illness or disability of a relevant other child shall constitute special expenses for the purposes of paragraph~2(2) of Schedule~4B to~the Act—
\begin{enumerate}\item[]
($a$) personal care and~attendance;

($b$) personal communication needs;

($c$) mobility;

($d$) domestic help;

($e$) medical aids where these cannot be provided under the health service;

($f$) heating;

($g$) clothing;

($h$) laundry requirements;

($i$) payments for food essential to~comply with a diet recommended by a medical practitioner;

($j$) adaptations required to~the non-resident parent’s home;

($k$) day care;

($l$) rehabilitation; or

($m$) respite care.
\end{enumerate}

(2) For the purposes of this regulation~and~regulation~10—
\begin{enumerate}\item[]
($a$) a person is “disabled” for a period in respect of which—
\begin{enumerate}\item[]
(i) either an attendance allowance, disability living allowance% 
, personal independence payment%  % Words inserted by SI 2013/388 Sch para 24(a)
%or a mobility supplement 
, a mobility supplement or armed forces independence payment under the Armed Forces and Reserve Forces (Compensation Scheme) Order 2011  % Words substituted by SI 2013/591 Sch para 17(2)(a)
is paid to or in respect of him;

(ii) he would receive an attendance allowance or disability living allowance if it were not for the fact that he is a patient, though remaining part of the applicant’s family; 
%or  % Word omitted by SI 2013/388 Sch para 24(b)

(iii) he is registered blind or treated as blind within the meaning of paragraph~12(1)($a$)(iii)  and~(2) of Schedule~2 to the Income Support (General) Regulations 1987\footnote{S.I.\ 1987/1967.};
%
% Reg 11(2)(a)(iv) inserted by SI 2013/388 Sch para 24(c)
or

(iv) he would receive personal independence payment but for regulations under section 86(1) (hospital in-patients) of the Welfare Reform Act 2012 and he remains part of the applicant’s family;
\end{enumerate}
and~for this purpose—
\begin{enumerate}\item[]
(i) “attendance allowance” means an allowance payable under section 64 of the Contributions and Benefits Act or an increase of disablement pension under section 104 of that Act, or an award under article 14 of the Naval, Military and Air Forces etc., (Disablement and Death) Service Pensions Order 1983\footnote{S.I.\ 1983/883.} or any analogous allowance payable in conjunction with any other war disablement pension within the meaning of section 150(2) of the Contributions and Benefits Act;

(ii) “disability living allowance” means an allowance payable under section 72 of the Contributions and~Benefits Act;

% Definition (iia) inserted by SI 2013/388 Sch para 19(d)
(iia) “personal independence payment” means an allowance payable under section 78 of the Welfare Reform Act 2012 (daily living component);

(iii) “mobility supplement” means an award under article 26A of the Naval, Military and~Air Forces etc., (Disablement and~Death) Service Pensions Order 1983 or any analogous allowance payable in conjunction with any other war disablement pension within the meaning of section 150(2) of the Contributions and~Benefits Act; and

(iv) “patient” means a person (other than a person who is serving a sentence of imprisonment or detention in a young offenders institution within the meaning of the Criminal Justice Act 1982\footnote{1982 c.\ 48. The Act is amended by the Criminal Justice Act 1988 c.\ 33.}) who is regarded as receiving free in-patient treatment within the meaning of the Social Security (Hospital In-Patients) Regulations 1975\footnote{S.I.\ 1975/555. Relevant amendments are made by S.I.\ 1992/2595 and~1999/1326.};
\end{enumerate}

($b$) “the health service” has the same meaning as in section 128 of the National Health Service Act 1977\footnote{1977 c.\ 49.} or in section 108(1) of the National Health Service (Scotland) Act 1978\footnote{1978 c.\ 29.};

($c$) “long-term illness” means an illness from which the non-resident parent or child is suffering at the date of the application or the date from which the variation, if agreed, would take effect and~which is likely to~last for at least 52 weeks from that date, or, if likely to~be shorter than 52 weeks, for the remainder of the life of that person; and

($d$) “relevant other child” has the meaning given in paragraph~10C(2) of Schedule~1 to~the Act and~Regulations made under that paragraph.
\end{enumerate}

%(3) Where the non-resident parent has, at the date he makes the variation application, received, or at that date is in receipt of, or where he will receive any financial assistance from any source in respect of the long-term illness or disability of the relevant other child or a disability living allowance is received by the non-resident parent on behalf of the relevant other child, only the net amount of the costs incurred in respect of the items listed in paragraph~(1), after the deduction of the financial assistance or the amount of the allowance, shall constitute special expenses for the purposes of paragraph~2(2) of Schedule~4B to~the Act.

% Reg 11(3) substituted (16.3.05) by SI 2005/785 reg 8(3)
(3) Where, at the date on which the non-resident parent makes the variation application—
\begin{enumerate}\item[]
($a$) he or a member of his household has received, or at that date is in receipt of, or where he or the member of his household will receive any financial assistance from any source in respect of the long-term illness or disability of the relevant other child; or

($b$) a disability living allowance 
or armed forces independence payment under the Armed Forces and Reserve Forces (Compensation Scheme) Order 2011  % Words inserted by SI 2013/591 Sch para 17(2)(b)
or personal independence payment  % Words inserted by SI 2013/388 Sch para 24(e)
is received by the non-resident parent or the member of his household on behalf of the relevant other child,
\end{enumerate}
only the net amount of the costs incurred in respect of the items listed in paragraph~(1), after the deduction of the financial assistance or the amount of the allowance, shall constitute special expenses for the purposes of paragraph~2(2) of Schedule~4B to~the Act.

\amendment{
Reg. 11(3) substituted (16.3.05) by the Child Support (Miscellaneous Amendments) Regulations 2005 reg. 8(3).

Words inserted in reg. 11(2)(a)(i), reg. 11(2)(a)(iv) inserted, definition (iia) inserted in reg. 11(2)(a) and words inserted in reg. 11(3)(b) (8.4.13) by the Personal Independence Payment (Supplementary Provisions and Consequential Amendments) Regulations 2013 Sch. para. 24.

Words substituted in reg. 11(2)(a)(i) and words inserted in reg. 64(3)(b) (8.4.13) by the Armed Forces and Reserve Forces Compensation Scheme (Consequential Provisions: Subordinate Legislation) Order 2013 Sch. para. 17.
}

\subsection[12. Special expenses—prior debts]{Special expenses—prior debts}

12.---(1)  Subject to~the following parargaphs~of this regulation~and~regulation~15, the repayment of debts to~which paragraph~(2) applies shall constitute expenses for the purposes of paragraph~2(2) of Schedule~4B to~the Act where those debts were incurred—
\begin{enumerate}\item[]
($a$) before the non-resident parent became a non-resident parent in relation to~the qualifying child; and

($b$) at the time when the non-resident parent and~the person with care in relation to~the child referred to~in sub-paragraph~($a$)  were a couple.
\end{enumerate}

(2) This paragraph~applies to~debts incurred—
\begin{enumerate}\item[]
($a$) for the joint benefit of the non-resident parent and~the person with care;

($b$) for the benefit of the person with care where the non-resident parent remains legally liable to~repay the whole or part of the debt;

($c$) for the benefit of any person who is not a child but who at the time the debt was incurred—
\begin{enumerate}\item[]
(i) was a child;

(ii) lived with the non-resident parent and~the person with care; and

(iii) of whom the non-resident parent or the person with care is the parent, or both are the parents;
\end{enumerate}

($d$) for the benefit of the qualifying child referred to~in paragraph~(1); or

($e$) for the benefit of any child, other than the qualifying child referred to~in paragraph~(1), who, at the time the debt was incurred—
\begin{enumerate}\item[]
(i) lived with the non-resident parent and~the person with care; and

(ii) of whom the person with care is the parent.
\end{enumerate}
\end{enumerate}

(3) Paragraph (1) shall not apply to~repayment of—
\begin{enumerate}\item[]
($a$) a debt which would otherwise fall within paragraph~(1) where the non-resident parent has retained for his own use and~benefit the asset in connection with the purchase of which he incurred the debt;

($b$) a debt incurred for the purposes of any trade or business;

($c$) a gambling debt;

($d$) a fine imposed on the non-resident parent;

%($e$) unpaid legal costs in respect of separation or divorce from the person with care;

% Reg 12(3)(e) substituted (5.12.05) by SI 2005/2877 Sch 4 para 8
($e$) unpaid legal costs in respect of—
\begin{enumerate}\item[]
(i) separation or divorce from the person with care;

(ii) separation from the person with care or the dissolution of a civil partnership that had been formed with the person with care;
\end{enumerate}

($f$) amounts due after use of a credit card;

($g$) a debt incurred by the non-resident parent to~pay any of the items listed in sub-parargaphs~($c$)  to~($f$)  and~($j$);

($h$) amounts payable by the non-resident parent under a mortgage or loan taken out on the security of any property except where that mortgage or loan was taken out to~facilitate the purchase of, or to~pay for repairs or improvements to, any property which is the home of the person with care and~any qualifying child;

($i$) amounts payable by the non-resident parent in respect of a policy of insurance except where that policy of insurance was obtained or retained to~discharge a mortgage or charge taken out to~facilitate the purchase of, or to~pay for repairs or improvements to, any property which is the home of the person with care and~the qualifying child;

($j$) a bank overdraft except where the overdraft was at the time it was taken out agreed to~be for a specified amount repayable over a specified period;

($k$) a loan obtained by the non-resident parent other than a loan obtained from a qualifying lender or the non-resident parent’s current or former employer;

($l$) a debt in respect of which a variation has previously been agreed and~which has not been repaid during the period for which the maintenance calculation which took account of the variation was in force; or

($m$) any other debt which the Secretary of State is satisfied it is reasonable to~exclude.
\end{enumerate}

(4) Except where the repayment is of an amount which is payable under a mortgage or loan or in respect of a policy of insurance which falls within the exception set out in sub-paragraph~($h$)  or ($i$)  of paragraph~(3), repayment of a debt shall not constitute expenses for the purposes of paragraph~(1) where the Secretary of State is satisfied that the non-resident parent has taken responsibility for repayment of that debt as, or as part of, a financial settlement with the person with care or by virtue of a court order.

(5) Where an applicant has incurred a debt partly to~repay a debt repayment of which would have fallen within paragraph~(1), the repayment of that part of the debt incurred which is referable to~the debt repayment of which would have fallen within that paragraph~shall constitute expenses for the purposes of paragraph~2(2) of Schedule~4B to~the Act.

(6) For the purposes of this regulation~and~regulation~14—
\begin{enumerate}\item[]
($a$) “qualifying lender” has the meaning given to~it in section 376(4) of the Income and~Corporation Taxes Act 1988\footnote{1988 c.\ 1.}; and

($b$) “repairs or improvements” means major repairs necessary to~maintain the fabric of the home and~any of the following measures—
\begin{enumerate}\item[]
(i) installation of a fixed bath, shower, wash basin or lavatory, and~necessary associated plumbing;

(ii) damp-proofing measures;

(iii) provision or improvement of ventilation and~natural light;

(iv) provision of electric lighting and~sockets;

(v) provision or improvement of drainage facilities;

(vi) improvement of the structural condition of the home;

(vii) improvements to~the facilities for the storing, preparation and~cooking of food;

(viii) provision of heating, including central heating;

(ix) provision of storage facilities for fuel and~refuse;

(x) improvements to~the insulation of the home; or

(xi) other improvements which the Secretary of State considers reasonable in the circumstances.
\end{enumerate}
\end{enumerate}

\amendment{
Reg. 12(3)(e) substituted (5.12.05) by the Civil Partnership (Pensions, Social Security and~Child Support) (Consequential, etc. Provisions) Order 2005 Sch. 4 para. 8.
}

\subsection[13. Special expenses—boarding school fees]{Special expenses—boarding school fees}

13.---(1)  Subject to~the following parargaphs~of this regulation~and~regulation~15, the maintenance element of the costs, incurred or reasonably expected to~be incurred, by the non-resident parent for the purpose of the attendance at a boarding school of the qualifying child shall constitute expenses for the purposes of paragraph~2(2) of Schedule~4B to~the Act.

(2) Where the Secretary of State considers that the costs referred to~in paragraph~(1) cannot be distinguished with reasonable certainty from other costs incurred in connection with the attendance at boarding school by the qualifying child, he may instead determine the amount of those costs and~any such determination shall not exceed 35\% of the total costs.

(3) Where—
\begin{enumerate}\item[]
($a$) the non-resident parent has at the date the variation application is made, received, or at that date is in receipt of, financial assistance from any source in respect of the boarding school fees; or

($b$) the boarding school fees are being paid in part by the non-resident parent and~in part by another person,
\end{enumerate}
a portion of the costs incurred by the non-resident parent in respect of the boarding school fees shall constitute special expenses for the purposes of paragraph~2(2) of Schedule~4B to~the Act being the same proportion as the maintenance element of the costs bears to~the total amount of the costs.

(4) No variation on this ground shall reduce by more than 50\% the income to~which the Secretary of State would otherwise have had regard in the calculation of maintenance liability.

(5) For the purposes of this regulation, “boarding school fees” means the fees payable in respect of attendance at a recognised educational establishment providing full-time education which is not advanced education for children under the age of 19 and~where some or all of the pupils, including the qualifying child, are resident during term time.

\subsection[14. Special expenses—payments in respect of certain mortgages, loans or insurance policies]{\sloppy Special expenses—payments in respect of certain mortgages, loans or insurance policies}

14.---(1)  Subject to~regulation~15, the payments to~which paragraph~(2) applies shall constitute expenses for the purposes of paragraph~2(2) of Schedule~4B to~the Act.

(2) This paragraph~applies to~payments, whether made to~the mortgagee, lender, insurer or the person with care—
\begin{enumerate}\item[]
($a$) in respect of a mortgage or loan where—
\begin{enumerate}\item[]
(i) the mortgage or loan was taken out to~facilitate the purchase of, or repairs or improvements to, a property (“the property”) by a person other than the non-resident parent;

(ii) the payments are not made under a debt incurred by the non-resident parent or do not arise out of any other legal liability of his for the period in respect of which the variation is applied for;

(iii) the property was the home of the applicant and~the person with care when they were a couple and~remains the home of the person with care and~the qualifying child; and

(iv) the non-resident parent has no legal or equitable interest in and~no charge or right to~have a charge over the property; or
\end{enumerate}

($b$) of amounts payable in respect of a policy of insurance taken out for the discharge of a mortgage or loan referred to~in sub-paragraph~($a$), including an endowment policy, except where the non-resident parent is entitled to~any part of the proceeds on the maturity of that policy.
\end{enumerate}

\subsection[15. Thresholds for and~reduction of amount of special expenses]{Thresholds for and~reduction of amount of special expenses}

15.---(1)  Subject to~parargaphs~(2) to~(4), the costs or repayments referred to~in regulations 10 and~12 to~14 shall be special expenses for the purposes of paragraph~2(2) of Schedule~4B to~the Act where and~to~the extent that they exceed the threshold amount, which is—
\begin{enumerate}\item[]
($a$) £15 per week where the expenses fall within only one description of expenses and, where the expenses fall within more than one description of expenses, £15 per week in respect of the aggregate of those expenses, where the relevant net weekly income of the non-resident parent is £200 or more; or

($b$) £10 per week where the expenses fall within only one description of expenses, and, where the expenses fall within more than one description of expenses, £10 per week in respect of the aggregate of those expenses, where the relevant net weekly income is below £200.
\end{enumerate}

(2) Subject to~paragraph~(3), where the Secretary of State considers any expenses referred to~in regulations 10 to~14 to~be unreasonably high or to~have been unreasonably incurred he may substitute such lower amount as he considers reasonable, including an amount which is below the threshold amount or a nil amount.

(3) Any lower amount substituted by the Secretary of State under paragraph~(2) in relation to~contact costs under regulation~10 shall not be so low as to~make it impossible, in the Secretary of State’s opinion, for contact between the non-resident parent and~the qualifying child to~be maintained at the frequency specified in any court order made in respect of the non-resident parent and~that child where the non-resident parent is maintaining contact at that frequency.

(4) For the purposes of this regulation, “relevant net weekly income” means the net weekly income taken into~account for the purposes of the maintenance calculation before taking account of any variation on the grounds of special expenses.

\section[Part IV --- Property or capital transfers]{Part IV\\*Property or capital transfers}

\renewcommand\parthead{--- Part IV}

\subsection[16. Prescription of terms]{Prescription of terms}

16.---(1)  For the purposes of parargaphs~3(1)($a$)  and~($b$)  of Schedule~4B to~the Act—
\begin{enumerate}\item[]
($a$) a court order means an order made—
\begin{enumerate}\item[]
(i) under one or more of the enactments listed in or prescribed under section 8(11) of the Act; and

(ii) in connection with the transfer of property of a kind defined in paragraph~(2); and
\end{enumerate}

($b$) an agreement means a written agreement made in connection with the transfer of property of a kind defined in paragraph~(2).
\end{enumerate}

(2) Subject to~parargaphs~(3) and~(4), for the purposes of paragraph~3(2) of Schedule~4B to~the Act, a transfer of property is a transfer by the non-resident parent of his beneficial interest in any asset to~the person with care, to~the qualifying child, or to~trustees where the object or one of the objects of the trust is the provision of maintenance.

(3) Where a transfer of property would not have fallen within paragraph~(2) when made but the Secretary of State is satisfied that some or all of the amount of that property was subsequently transferred to~the person currently with care of the qualifying child, the transfer of that property to~the person currently with care shall constitute a transfer of property for the purposes of paragraph~3 of Schedule~4B to~the Act.

(4) The minimum value for the purposes of paragraph~3(2) of Schedule~4B to~the Act is the threshold amount which is 
%£5000
£4999$.$99%  % Figure substituted (30.4.02) by SI 2002/1204 reg 9(4)
.

\amendment{
Figure substituted in reg. 16(4) (30.4.02) by the Child Support (Miscellaneous Amendments) Regulations 2002 reg. 9(4).
}

\subsection[17. Value of a transfer of property—equivalent weekly value]{Value of a transfer of property—equivalent weekly value}

17.---(1)  Where the conditions specified in paragraph~3 of Schedule~4B to~the Act are satisfied, the value of a transfer of property for the purposes of that paragraph~shall be that part of the transfer made by the non-resident parent (making allowances for any transfer by the person with care to~the non-resident parent) which the Secretary of State is satisfied is in lieu of periodical payments of maintenance.

(2) The Secretary of State shall, in determining the value of a transfer of property in accordance with paragraph~(1), assume that, unless evidence to~the contrary is provided to~him—
\begin{enumerate}\item[]
($a$) the person with care and~the non-resident parent had equal beneficial interests in the asset in relation to~which the court order or agreement was made;

($b$) where the person with care was married to~the non-resident parent, one half of the value of the transfer was a transfer for the benefit of the person with care; and

($c$) where the person with care has never been married to~the non-resident parent, none of the value of the transfer was for the benefit of the person with care.
\end{enumerate}

(3) The equivalent weekly value of a transfer of property shall be determined in accordance with the provisions of the Schedule.

(4) For the purposes of regulation~16 and~this regulation, the term “maintenance” means the normal day-to-day living expenses of the qualifying child.

(5) A variation falling within paragraph~(1) shall cease to~have effect at the end of the number of years of liability, as defined in paragraph~1 of the Schedule, for the case in question.

\section[Part V --- Additional cases]{Part V\\*Additional cases}

\subsection[18. Assets]{Assets}

\renewcommand\parthead{--- Part V}

18.---(1)  Subject to~parargaphs~(2) and~(3), a case shall constitute a case for the purposes of paragraph~4(1) of Schedule~4B to~the Act where the Secretary of State is satisfied there is an asset—
\begin{enumerate}\item[]
($a$) in which the non-resident parent 
%has the beneficial interest
has a beneficial interest%  % Words substituted (30.4.02) by SI 2002/1204 reg 9(5)(a)
, or which the non-resident parent has the ability to~control;

($b$) which has been transferred by the non-resident parent to~trustees, and~the non-resident parent is a beneficiary of the trust so created, in circumstances where the Secretary of State is satisfied the non-resident parent has made the transfer to~reduce the amount of assets which would otherwise be taken into~account for the purposes of a variation under paragraph~4(1) of Schedule~4B to~the Act; or

($c$) which has become subject to~a trust created by legal implication of which the non-resident parent is a beneficiary.
\end{enumerate}

(2) For the purposes of this regulation~“asset” means—
\begin{enumerate}\item[]
($a$) money, whether in cash or on deposit, including any which, in Scotland, is monies due or an obligation owed, whether immediately payable or otherwise and~whether the payment or obligation is secured or not and~the Secretary of State is satisfied that requiring payment of the monies or implementation of the obligation would be reasonable;

($b$) a legal estate or beneficial interest in land~and~rights in or over land;

($c$) shares as defined in section 744 of the Companies Act 1985\footnote{1985 c.\ 6.}, stock and~unit trusts as defined in section 6 of the Charging Orders Act 1979\footnote{1979 c.\ 53.}, gilt-edged securities as defined in Part I of Schedule~9 to~the Taxation of Chargeable Gains Act 1992\footnote{1992 c.\ 12.}, and~other similar financial instruments; or

($d$) a chose in action which has not been enforced when the Secretary of State is satisfied that such enforcement would be reasonable,
\end{enumerate}
and~includes any such asset located outside Great Britain.

(3) Paragraph (2) shall not apply—
\begin{enumerate}\item[]
%($a$) where the total value of the assets referred to~in that paragraph~does not exceed £65,000 after deduction of the amount owing under any mortgage or charge on those assets;

% Reg 18(3)(a) substituted (16.3.05) by SI 2005/785 reg 8(4)
($a$) where the total value of the assets referred to~in that paragraph~does not exceed £65,000 after deduction of—
\begin{enumerate}\item[]
(i) the amount owing under any mortgage or charge on those assets;

(ii) the value of any asset in respect of which income has been taken into~account under regulation~19(1A);
\end{enumerate}

($b$) in relation to~any asset which the Secretary of State is satisfied is being retained by the non-resident parent to~be used for a purpose which the Secretary of State considers reasonable in all the circumstances of the case;

($c$) to~any asset received by the non-resident parent as compensation for personal injury suffered by him;

($d$) 
except where the asset is of a type specified in paragraph~(2)($b$)  and~produces income which does not form part of the net weekly income of the non-resident parent as calculated or estimated under Part III of the Schedule~to~the Maintenance Calculations and~Special Cases Regulations,  % Words inserted (30.4.02) by SI 2002/1204 reg 9(5)(b)(i)
to~any asset used in the course of a trade or business; or

($e$) to~property which is the home of the non-resident parent or any child of his%.
%
% Reg 18(3)(f) added (30.4.02) by SI 2002/1204 reg 9(5)(b)(ii)
; or

    ($f$) 
    where, were the non-resident parent a claimant, paragraph~22 (treatment of payments from certain trusts) or 64 (treatment of relevant trust payments) of Schedule~10 to~the Income Support (General) Regulations 1987\footnote{Paragraph 22 was substituted by regulation~5(8) of S.I.\ 1991/1175 and~amended by regulation~6(8) of S.I.\ 1992/1101, regulation~2(3) of S.I.\ 1993/963 and~regulation~4(5) of S.I.\ 1993/1249.} would apply to~the asset referred to~in that paragraph.
\end{enumerate}

(4) For the purposes of this regulation, where any asset is held in the joint names of the non-resident parent and~another person the Secretary of State shall assume, unless evidence to~the contrary is provided to~him, that the asset is held by them in equal shares.

(5) Where a variation is agreed on the ground that the non-resident parent has assets for which provision is made in this regulation, the Secretary of State shall calculate the weekly value of the assets by applying the statutory rate of interest to~the value of the assets and~dividing by 52, and~the resulting figure, aggregated with any benefit, pension or allowance 
prescribed for the purposes of paragraph~4(1)($b$)  of Schedule~1 to~the Act  % Words inserted (30.4.02) by SI 2002/1204 reg 9(6)
which the non-resident parent receives, other than any benefits referred to~in regulation~26(3), shall be taken into~account as additional income under regulation~25.

(6) For the purposes of this regulation, the “statutory rate of interest” means interest at the statutory rate prescribed for a judgment debt or, in Scotland, the statutory rate in respect of interest included in or payable under a decree in the Court of Session, which in either case applies on the date from which the maintenance calculation which takes account of the variation takes effect.

\amendment{
Words inserted in reg. 18(3)(d), (5), words substituted in reg. 18(1)(a) and~reg. 18(3)(f) substituted (30.4.02) by the Child Support (Miscellaneous Amendments) Regulations 2002 reg. 9(5), (6).

Reg. 18(3)(a) substituted (6.4.05) by the Child Support (Miscellaneous Amendments) Regulations 2005 reg. 8(4).
}

\subsection[19. Income not taken into~account and~diversion of income]{Income not taken into~account and~diversion of income}

19.---(1)  Subject to~paragraph~(2), a case shall constitute a case for the purposes of paragraph~4(1) of Schedule~4B to~the Act where—
\begin{enumerate}\item[]
($a$) the non-resident parent’s liability to~pay child support maintenance under the maintenance calculation which is in force or has been applied for% 
%or treated as applied for  % Words omitted (27.10.08) by SI 2008/2543 reg 8(2)
, is, or would be, as the case may be—
\begin{enumerate}\item[]
(i) the nil rate owing to~the application of paragraph~5($a$)  of Schedule~1 to~the Act; or

(ii) a flat rate, owing to~the application of paragraph~4(1)($b$)  of Schedule~1 to~the Act, or would be a flat rate but is less than that amount, or nil, owing to~the application of paragraph~8 of Schedule~1 to~the Act; and
\end{enumerate}

($b$) the Secretary of State is satisfied that the non-resident parent is in receipt of income which would fall to~be taken into~account under the Maintenance Calculations and~Special Cases Regulations but for the application to~the non-resident parent of paragraph~4(1)($b$)  or 5($a$)  of Schedule~1 to~the Act.
\end{enumerate}

% Reg 19(1A) inserted (16.3.05) by SI 2005/785 reg 8(5)(a)
(1A) Subject to~paragraph~(2), a case shall constitute a case for the purposes of paragraph~4(1) of Schedule~4B to~the Act where—
\begin{enumerate}\item[]
($a$) the non-resident parent has the ability to~control the amount of income he receives from a company or business, including earnings from employment or self-employment; and

($b$) the Secretary of State is satisfied that the non-resident parent is receiving income from that company or business which would not otherwise fall to~be taken into~account under the Maintenance Calculations and~Special Cases Regulations.
\end{enumerate}

%(2) Paragraph (1) shall apply where the income referred to~in sub-paragraph~($b$)  of that paragraph~is a net weekly income of over £100.

% Reg 19(2) substituted (16.3.05) by SI 2005/785 reg 8(5)(b)
(2) Paragraphs (1) and~(1A) shall apply where—
\begin{enumerate}\item[]
($a$) the income referred to~in paragraph~(1)($b$)  is net weekly income of over £100; or

($b$) the income referred to~in paragraph~(1A)($b$)  is over £100; or

($c$) the aggregate of the net weekly income referred to~in sub-\hspace{0pt}paragraph~($a$)  and~the income referred to~in sub-paragraph~($b$)  is over £100,
\end{enumerate}
as the case may be.

(3) Net weekly income for the purposes of paragraph~(2), in relation to~earned income of a non-resident parent who is a student, shall be calculated by aggregating the income for the year ending with the relevant week (which for this purpose shall have the meaning given in the Maintenance Calculations and~Special Cases Regulations) and~dividing by 52, or, where the Secretary of State does not consider the result to~be representative of the student’s earned income, over such other period as he shall consider representative and~dividing by the number of weeks in that period.

%(4) A case shall constitute a case for the purposes of paragraph~4(1) of Schedule~4B to~the Act where—
%\begin{enumerate}\item[]
%($a$) the non-resident parent has the ability to~control the amount of income he receives, including earnings from employment or self-employment, whether or not the whole of that income is derived from the company or business from which his earnings are derived, and
%
%($b$) the Secretary of State is satisfied that the non-resident parent has unreasonably reduced the amount of his income which would otherwise fall to~be taken into~account under the Maintenance Calculations and~Special Cases Regulations 
%or paragraph~(1A)  % Words inserted (16.3.05) by SI 2005/785 reg 8(5)(c)(i)
%by diverting it to~other persons or for purposes other than the provision of such income for himself% 
%%in order to~reduce his liability to~pay child support maintenance  % Words omitted (16.3.05) by SI 2005/785 reg 8(5)(c)(ii)
%.
%\end{enumerate}

% Reg 19(4), (4A) substituted for reg 19(4) (6.4.09) by SI 2009/736 reg 4(3)
(4) A case shall constitute a case for the purposes of paragraph~4(1) of Schedule 4B to the Act where—
\begin{enumerate}\item[]
($a$) the non-resident parent (“$\mathcal{P}$”) has the ability to control the amount of income that—
\begin{enumerate}\item[]
(i) $\mathcal{P}$ receives, or

(ii) is taken into account as $\mathcal{P}$’s net weekly income,
\end{enumerate}
including earnings from employment or self-employment, whether or not the whole of that income is derived from the company or business from which those earnings are derived; and

($b$) the 
%Commission 
Secretary of State  % Words substituted (1.8.12) by SI 2012/2007 Sch para 115(b)
is satisfied that $\mathcal{P}$ has unreasonably reduced the amount of $\mathcal{P}$’s income which would otherwise fall to be taken into account under the Maintenance Calculations and~Special Cases Regulations or paragraph~(1A) by diverting it to other persons or for purposes other than the provision of such income for $\mathcal{P}$.
\end{enumerate}

(4A) In paragraph~(4), “net weekly income” has the same meaning as in the Maintenance Calculations and~Special Cases Regulations.

(5) Where a variation on this ground is agreed to—
\begin{enumerate}\item[]
($a$) in a case to~which paragraph~(1) applies, the additional income taken into~account under regulation~25 shall be the whole of the income referred to~in paragraph~(1)($b$), aggregated with any benefit, pension or allowance 
prescribed for the purposes of paragraph~4(1)($b$)  of Schedule~1 to~the Act  % Words inserted (30.4.02) by SI 2002/1204 reg 9(6)
which the non-resident parent receives other than any benefits referred to~in regulation~26(3); and

($b$) in a case to~which paragraph~(4) applies, the additional income taken into~account under regulation~25 shall be the whole of the amount by which the Secretary of State is satisfied the non-resident parent has unreasonably reduced his income%
%
% Reg 19(5)(c) added (16.3.05) by SI 2005/785 reg 8(5)(d)
; and

($c$) in a case to~which paragraph~(1A) applies, the additional income taken into~account under regulation~25 shall be the whole of the income referred to~in paragraph~(1A)($b$).
\end{enumerate}

\amendment{
Words inserted in reg. 19(5)(a) (30.4.02) by the Child Support (Miscellaneous Amendments) Regulations 2002 reg. 9(6).

Words inserted and~omitted in reg. 19(4)(b), reg. 19(1A), (5)(c) inserted and~reg. 19(2) substituted (6.4.05) by the Child Support (Miscellaneous Amendments) Regulations 2005 reg. 8(5).

Words omitted in reg. 19(1)(a) (27.10.08) by the Child Support (Consequential Provisions) Regulations 2008 reg. 8(2).

Reg.~19(4), (4A) substituted for reg.~19(4) 
(6.4.09) by the Child Support (Miscellaneous and~Consequential Amendments) Regulations 2009 reg.~4(3).

Words substituted in reg.~19(4)(b) (1.8.12) by the Public Bodies (Child Maintenance and Enforcement Commission: Abolition and Transfer of Functions) Order 2012 Sch. para.~115(b).
}

\subsection[20. Life-style inconsistent with declared income]{Life-style inconsistent with declared income}

20.---(1)  Subject to~paragraph~(3), a case shall constitute a case for the purposes of paragraph~4(1) of Schedule~4B to~the Act where—
\begin{enumerate}\item[]
($a$) the non-resident parent’s liability to~pay child support maintenance under the maintenance calculation which is in force, or which has been applied for% 
%or treated as applied for  % Words omitted (27.10.08) by SI 2008/2543 reg 8(3)
, is, or would be, as the case may be—
\begin{enumerate}\item[]
(i) the basic rate,

(ii) the reduced rate,

(iii) a flat rate owing to~the application of paragraph~4(1)($a$)  of Schedule~1 to~the Act, including where the net weekly income of the non-resident parent taken into~account for the purposes of the maintenance calculation is, or would be, £100 per week or less owing to~a variation being taken into~account, or to~the application of regulation~18, 19 or 21 of the Transitional Regulations (deduction for relevant departure direction or relevant property transfer);

(iv) £5 per week or such other amount as may be prescribed owing to~the application of paragraph~7(7) of Schedule~1 to~the Act (shared care);

(v) equivalent to~the flat rate provided for in, or prescribed for the purposes of, paragraph~4(1)($b$)  of Schedule~1 to~the Act owing to~the application of—
\begin{enumerate}\item[]
($aa$) regulation~27(5);

($bb$) regulation~9 of the Maintenance Calculations and~Special Cases Regulations (care provided in part by a local authority); or

($cc$) regulation~23(5) of the Transitional Regulations; or
\end{enumerate}

(vi) the nil rate owing to~the application of paragraph~5($b$)  of Schedule~1 to~the Act; and
\end{enumerate}

($b$) the Secretary of State is satisfied that the income which has been, or would be, taken into~account for the purposes of the maintenance calculation is substantially lower than the level of income required to~support the overall life-style of the non-resident parent.
\end{enumerate}

(2) Subject to~paragraph~(4), a case shall constitute a case for the purposes of paragraph~4(1) of Schedule~4B to~the Act where the non-resident parent’s liability to~pay child support maintenance under the maintenance calculation which is in force, or which has been applied for% 
%or treated as applied for  % Words omitted (27.10.08) by SI 2008/2543 reg 8(3)
, is, or would be, as the case may be—
\begin{enumerate}\item[]
($a$) a flat rate owing to~the application of paragraph~4(1)($b$)  of Schedule~1 to~the Act, or would be a flat rate but is less than that amount, or nil, owing to~the application of paragraph~8 of Schedule~1 to~the Act; or

($b$) the nil rate owing to~the application of paragraph~5($a$)  of Schedule~1 to~the Act,
\end{enumerate}
and~the Secretary of State is satisfied that the income which would otherwise be taken into~account for the purposes of the maintenance calculation is substantially lower than the level of income required to~support the overall life-style of the non-resident parent.

(3) Paragraph (1) shall not apply where the Secretary of State is satisfied that the life-style of the non-resident parent is paid for from—
\begin{enumerate}\item[]
($a$) income which is or would be disregarded for the purposes of a maintenance calculation under the Maintenance Calculations and~Special Cases Regulations;

% Reg 20(3)(aa) inserted (16.3.05) by SI 2005/785 reg 8(6)
($aa$) income which falls to~be considered under regulation~19(1A) (income not taken into~account);

($b$) income which falls to~be considered under regulation~19(4) (diversion of income);

($c$) assets as defined for the purposes of regulation~18, or income derived from those assets;

($d$) the income of any partner of the non-resident parent, except where the non-resident parent is able to~influence or control the amount of income received by that partner; or

($e$) assets as defined for the purposes of regulation~18 of any partner of the non-resident parent, or any income derived from such assets, except where the non-resident parent is able to~influence or control the assets, their use, or income derived from them.
\end{enumerate}

(4) Paragraph (2) shall not apply where the Secretary of State is satisfied that the life-style of the non-resident parent is paid for—
\begin{enumerate}\item[]
($a$) from a source referred to~in paragraph~(3);

($b$) from net weekly income of £100 or less; or

($c$) from income which falls to~be considered under regulation~19(1).
\end{enumerate}

(5) Where a variation on this ground is agreed to, the additional income taken into~account under regulation~25 shall be the difference between the income which the Secretary of State is satisfied the non-resident parent requires to~support his overall life-style and~the income which has been or, but for the application of paragraph~4(1)($b$)  or 5($a$)  of Schedule~1 to~the Act, would be taken into~account for the purposes of the maintenance calculation, aggregated with any benefit, pension or allowance 
prescribed for the purposes of paragraph~4(1)($b$)  of Schedule~1 to~the Act  % Words inserted (30.4.02) by SI 2002/1204 reg 9(6)
which the non-resident parent receives other than any benefits referred to~in regulation~26(3).

\amendment{
Words inserted in reg. 20(5) (30.4.02) by the Child Support (Miscellaneous Amendments) Regulations 2002 reg. 9(6).

Reg. 20(3)(aa) inserted (6.4.05) by the Child Support (Miscellaneous Amendments) Regulations 2005 reg. 8(6).

Words omitted in reg. 20(1)(a), (2) (27.10.08) by the Child Support (Consequential Provisions) Regulations 2008 reg. 8(3).
}

\section[Part VI --- Factors to~be taken into~account for the purposes of section 28F of the Act]{Part VI\\*Factors to~be taken into~account for the purposes of section 28F of the Act}

\renewcommand\parthead{--- Part VI}

\subsection[21. Factors to~be taken into~account and~not to~be taken into~account]{Factors to~be taken into~account and~not to~be taken into~account}

21.---(1)  The factors to~be taken into~account in determining whether it would be just and~equitable to~agree to~a variation in any case shall include—
\begin{enumerate}\item[]
($a$) where the application is made on any ground—
\begin{enumerate}\item[]
(i) whether, in the opinion of the Secretary of State, agreeing to~a variation would be likely to~result in a relevant person ceasing paid employment;

(ii) if the applicant is the non-resident parent, the extent, if any, of his liability to~pay child maintenance under a court order or agreement in the period prior to~the effective date of the maintenance calculation; and
\end{enumerate}

($b$) where an application is made on the ground that the case falls within regulations 10 to~14 (special expenses), whether, in the opinion of the Secretary of State—
\begin{enumerate}\item[]
(i) the financial arrangements made by the non-resident parent could have been such as to~enable the expenses to~be paid without a variation being agreed; or

(ii) the non-resident parent has at his disposal financial resources which are currently utilised for the payment of expenses other than those arising from essential everyday requirements and~which could be used to~pay the expenses.
\end{enumerate}
\end{enumerate}

(2) The following factors are not to~be taken into~account in determining whether it would be just and~equitable to~agree to~a variation in any case—
\begin{enumerate}\item[]
($a$) the fact that the conception of the qualifying child was not planned by one or both of the parents;

($b$) whether the non-resident parent or the person with care of the qualifying child was responsible for the breakdown of the relationship between them;

($c$) the fact that the non-resident parent or the person with care of the qualifying child has formed a new relationship with a person who is not a parent of that child;

($d$) the existence of particular arrangements for contact with the qualifying child, including whether any arrangements made are being adhered to;

($e$) the income or assets of any person other than the non-resident parent, other than the income or assets of a partner of the non-resident parent taken into~account under regulation~20(3);

($f$) the failure by a non-resident parent to~make payments of child support maintenance, or to~make payments under a maintenance order or a written maintenance agreement; or

($g$) representations made by persons other than the relevant persons.
\end{enumerate}

\section[Part VII --- Effect of a variation on the maintenance calculation and~effective dates]{Part VII\\*Effect of a variation on the maintenance calculation and~effective dates}

\renewcommand\parthead{--- Part VII}

\subsection[22. Effective dates]{Effective dates}

22.---(1)  Subject to~paragraph~(2), where the application for a variation is made in the circumstances referred to~in section 28A(3) of the Act (before the Secretary of State has reached a decision under section 11 or 12(1) of the Act) and~the application is agreed to, the effective date of the maintenance calculation which takes account of the variation shall be—
\begin{enumerate}\item[]
($a$) where the ground giving rise to~the variation existed from the effective date of the maintenance calculation as provided for in the Maintenance Calculation Procedure Regulations, that date; or

($b$) where the ground giving rise to~the variation arose after the effective date referred to~in sub-paragraph~($a$), the first day of the maintenance period in which the ground arose.
\end{enumerate}

(2) Where the ground for the variation applied for under section 28A(3) of the Act is a ground in regulation~12 (prior debts) or 14 (special expenses—payments in respect of certain mortgages, loans or insurance policies) and~payments falling within regulation~12 or 14 which have been made by the non-resident parent constitute voluntary payments for the purposes of section 28J of the Act and~Regulations made under that section, the date from which the maintenance calculation shall take account of the variation on this ground shall be the date on which the maintenance period begins which immediately follows the date on which the non-resident parent is notified under the Maintenance Calculation Procedure Regulations of the amount of his liability to~pay child support maintenance.

(3) Where the ground for the variation applied for under section 28A(3) of the Act has ceased to~exist by the date the maintenance calculation is made, that calculation shall take account of the variation for the period ending on the last day of the maintenance period in which the ground existed.

\subsection[23. Effect on maintenance calculation—special expenses]{Effect on maintenance calculation—special expenses}

23.---(1)  Subject to~paragraph~(2) and~regulations 26 and~27, where the variation agreed to~is one falling within regulation~10 to~14 (special expenses) effect shall be given to~the variation in the maintenance calculation by deducting from the net weekly income of the non-resident parent the weekly amount of those expenses.

(2) Where the income which is taken into~account in the maintenance calculation is the capped amount and~the variation agreed to~is one falling within regulation~10 to~14 then—
\begin{enumerate}\item[]
($a$) the weekly amount of the expenses shall first be deducted from the actual net weekly income of the non-resident parent;

($b$) the amount by the which the capped amount exceeds the figure calculated under sub-paragraph~($a$)  shall be calculated; and

($c$) effect shall be given to~the variation in the maintenance calculation by deducting from the capped amount the amount calculated under sub-paragraph~($b$).
\end{enumerate}

\subsection[24. Effect on maintenance calculation—property or capital transfer]{\sloppy \textls[25]{Effect on maintenance calculation—property or capital} transfer}

24.  Subject to~regulation~27, where the variation agreed to~is one falling within regulation~16 (property or capital transfers)—
\begin{enumerate}\item[]
($a$) the maintenance calculation shall be carried out in accordance with Part I of Schedule~1 to~the Act and~Regulations made under that Part; and

($b$) the equivalent weekly value of the transfer calculated as provided in regulation~17 shall be deducted from the amount of child support maintenance which he would otherwise be liable to~pay to~the person with care with respect to~whom the transfer was made.
\end{enumerate}

\subsection[25. Effect on maintenance calculation—additional cases]{Effect on maintenance calculation—additional cases}

25.  Subject to~regulations 26 and~27, where the variation agreed to~is one falling within regulations 18 to~20 (additional cases), effect shall be given to~the variation in the maintenance calculation by increasing the net weekly income of the non-resident parent which would otherwise be taken into~account by the weekly amount of the additional income except that, where the amount of net weekly income calculated in this way would exceed the capped amount, the amount of net weekly income taken into~account shall be the capped amount.

\subsection[26. Effect on maintenance calculation—maximum amount payable where the variation is on additional cases ground]{\sloppy Effect on maintenance calculation—maximum amount payable where the variation is on additional cases ground}

26.---(1)  Subject to~regulation~27, where this regulation~applies the amount of child support maintenance which the non-resident parent shall be liable to~pay shall be whichever is the lesser of—
\begin{enumerate}\item[]
($a$) a weekly amount calculated by aggregating an amount equivalent to~the flat rate stated in or prescribed for the purposes of paragraph~4(1)($b$)  of Schedule~1 to~the Act with the amount calculated by applying that Schedule~to~the Act to~the additional income arising under the variation, other than the weekly amount of any benefit, pension or allowance the non-resident parent receives which is prescribed for the purposes of that paragraph; or

($b$) a weekly amount calculated by applying Part I of Schedule~1 to~the Act to~the additional income arising under the variation.
\end{enumerate}

(2) This regulation~applies where the variation agreed to~is one to~which regulation~25 applies and~the non-resident parent’s liability calculated as provided in Part I of Schedule~1 to~the Act and~Regulations made under that Schedule~would, but for the variation, be—
\begin{enumerate}\item[]
($a$) a flat rate under paragraph~4(1)($b$)  of that Schedule;

($b$) a flat rate but is less than that amount or nil, owing to~the application of paragraph~8 of that Schedule; or

($c$) a flat rate under paragraph~4(1)($b$)  of that Schedule~but for the application of paragraph~5($a$)  of that Schedule.
\end{enumerate}

(3) For the purposes of paragraph~(1)—
\begin{enumerate}\item[]
($a$) any benefit, pension or allowance taken into~account in the additional income referred to~in sub-paragraph~($b$)  shall not include—
\begin{enumerate}\item[]
(i) in the case of industrial injuries benefit under section 94 of the Contributions and~Benefits Act, any increase in that benefit under section 104 (constant attendance) or 105 (exceptionally severe disablement) of that Act;

(ii) in the case of a war disablement pension within the meaning in section 150(2) of the Contributions and~Benefits Act, any award under the following articles of the Naval, Military and~Air Forces Etc., (Disablement and~Death) Service Pensions Order 1983 (“the Service Pensions Order”): article 14 (constant attendance allowance), 15 (exceptionally severe disablement allowance), 16 (severe disablement occupational allowance) or 26A (mobility supplement)\footnote{S.I.\ 1983/883. Article 26A was inserted by article 4 of S.I.\ 1983/1116 and~amended by S.I.\ 1983/1821, 1986/592, 1990/1308, 1991/766, 1992/710, 1995/766 and~1997/286.} or any analogous allowances payable in conjunction with any other war disablement pension; and

(iii) any award under article 18 of the Service Pensions Order (unemployability allowances) which is an additional allowance in respect of a child of the non-resident parent where that child is not living with the non-resident parent;
\end{enumerate}

($b$) “additional income” for the purposes of sub-parargaphs~($a$)  and~($b$)  means such income after the application of a variation falling within regulations 10 to~14 (special expenses); and

($c$) “weekly amount” for the purposes of sub-parargaphs~($a$)  and~($b$)  means the aggregate of the amounts referred to~in the relevant sub-paragraph—
\begin{enumerate}\item[]
(i) adjusted as provided in regulation~27(3) as if the reference in that regulation~to~child support maintenance were to~the weekly amount; and

(ii) after any deduction provided for in regulation~27(4) as if the reference in that regulation~to~child support maintenance were to~the weekly amount.
\end{enumerate}
\end{enumerate}

\subsection[27. Effect on maintenance calculation—general]{Effect on maintenance calculation—general}

27.---(1)  Subject to~parargaphs~(4) and~(5), where more than one variation is agreed to~in respect of the same period regulations 23 to~26 shall apply and~the results shall be aggregated as appropriate.

(2) Paragraph 7(2) to~(7) of Schedule~1 to~the Act (shared care) shall apply where the rate of child support maintenance is affected by a variation which is agreed to~and~paragraph~7(2) shall be read as if after the words “as calculated in accordance with the preceding parargaphs~of this Part of this Schedule” there were inserted the words “, Schedule~4B and~Regulations made under that Schedule”.

(3) Subject to~parargaphs~(4) and~(5), where the non-resident parent shares the care of a qualifying child within the meaning in Part I of Schedule~1 to~the Act, or where the care of such a child is shared in part by a local authority, the amount of child support maintenance the non-resident parent is liable to~pay the person with care, calculated to~take account of any variation, shall be reduced in accordance with the provisions of paragraph~7 of that Part or regulation~9 of the Maintenance Calculations and~Special Cases Regulations, as the case may be.

(4) Subject to~paragraph~(5), where the variation agreed to~is one falling within regulation~16 (property or capital transfers) the equivalent weekly value of the transfer calculated as provided in regulation~17 shall be deducted from the amount of child support maintenance the non-resident parent would otherwise be liable to~pay the person with care in respect of whom the transfer was made after aggregation of the effects of any other variations as provided in paragraph~(1) or deduction for shared care as provided in paragraph~(3).

(5) If the application of regulation~24, or paragraph~(3) or (4), would decrease the weekly amount of child support maintenance (or the aggregate of all such amounts) payable by the non-resident parent to~the person with care (or all of them) to~less than a figure equivalent to~the flat rate of child support maintenance payable under 
%paragraph~4(1)($b$)  
paragraph~4(1)  % Words substituted (16.9.04) by SI 2004/2415 reg 9(4)
of Schedule~1 to~the Act, he shall instead be liable to~pay child support maintenance at a rate equivalent to~that rate apportioned (if appropriate) as provided in paragraph~6 of Schedule~1 to~the Act.

(6) The effect of a variation shall not be applied for any period during which a circumstance referred to~in regulation~7 applies.

(7) For the purposes of regulations 23 and~25 “net weekly income” means as calculated or estimated under the Maintenance Calculations and~Special Cases Regulations.

\vfill

\amendment{
Words substituted in reg. 27(5) (16.9.04) by the Child Support (Miscellaneous Amendments) Regulations 2004 reg. 9(4).
}

\subsection[28. Transitional provisions—conversion decisions]{Transitional provisions—conversion decisions}

28.  
%Where 
Subject to~regulation~17(10) of the Transitional Regulations, where  % Words substituted (5.11.03) by SI 2003/2779 reg 8
the variation is being applied for in connection with a subsequent decision within the meaning given in the Transitional Regulations, and~the decision to~be revised or superseded under section 16 or 17 of the Act, as the case may be, takes into~account a relevant property transfer as defined and~provided for in those Regulations—
\begin{enumerate}\item[]
($a$) for the purposes of regulations 23 and~25 “capped amount” shall mean the income for the purposes of paragraph~10(3) of Schedule~1 to~the Act less any deduction in respect of the relevant property transfer;

($b$) for the purposes of regulation~26(3)($b$)  the additional income for the purposes of paragraph~(1) of that regulation~shall be after deduction in respect of the relevant property transfer;

($c$) regulation~27(4) shall be read as if the aggregation referred to~included any deduction in respect of the relevant property transfer; and

($d$) regulation~27(5) shall be read as if after the reference to~paragraph~(3) or (4) there were a reference to~any deduction in respect of the relevant property transfer.
\end{enumerate}

\amendment{
Words substituted in reg. 28 (5.11.03) by the Child Support (Miscellaneous Amendments) (No. 2) Regulations 2003 reg. 8.
}

\subsection[29. Situations in which a variation previously agreed to~may be taken into~account in calculating maintenance liability]{Situations in which a variation previously agreed to~may be taken into~account in calculating maintenance liability}

29.---(1)  This regulation~applies where a variation has been agreed to~in relation to~a maintenance calculation.

(2) In the circumstances set out in paragraph~(3), the Secretary of State may take account of the effect of such a variation upon the rate of liability for child support maintenance notwithstanding the fact that an application has not been made.

(3) The circumstances are—
\begin{enumerate}\item[]
($a$) that the decision as to~the maintenance calculation is superseded under section 17 of the Act on a change of circumstances so that the non-resident parent becomes liable to~pay child support maintenance at the nil rate, or another rate which means that the variation cannot be taken into~account; and

($b$) that the superseding decision referred to~in sub-paragraph~($a$)  is itself superseded under section 17 of the Act on a change of circumstances so that the non-resident parent becomes liable to~pay a rate of child support maintenance which can be adjusted to~take account of the variation.
\end{enumerate}

\subsection[30. Circumstances for the purposes of section 28F(3) of the Act]{Circumstances for the purposes of section 28F(3) of the Act}

30.  The circumstances prescribed for the purposes of section 28F(3) of the Act (Secretary of State shall not agree to~a variation) are—
\begin{enumerate}\item[]
($a$) the prescribed circumstances in regulation~6(2) or 7; and

($b$) where the Secretary of State considers it would not be just and~equitable to~agree to~the variation having regard to~any of the factors referred to~in regulation~21.
\end{enumerate}

\section[Part VIII --- Miscellaneous]{Part VIII\\*Miscellaneous}

\subsection[31. Regular payments condition]{Regular payments condition}

\renewcommand\parthead{--- Part VIII}

31.---(1)  For the purposes of section 28C(2)($b$)  of the Act (payments of child support maintenance less than those specified in the interim maintenance decision) the payments shall be those fixed by the interim maintenance decision or the maintenance calculation in force, as the case may be, adjusted to~take account of the variation applied for by the non-resident parent as if that variation had been agreed.

(2) The Secretary of State may refuse to~consider the application for a variation where a regular payments condition has been imposed and~the non-resident parent has failed to~comply with it in the circumstances to~which paragraph~(3) applies.

(3) This paragraph~applies where the non-resident parent has failed to~comply with the regular payments condition and~fails to~make such payments which are due and~unpaid within one month of being required to~do so by the Secretary of State or such other period as the Secretary of State may in the particular case decide.

\subsection[32. Meaning of “benefit” for the purposes of section 28E of the Act]{Meaning of “benefit” for the purposes of section 28E of the Act}

32.  For the purposes of section 28E of the Act, “benefit” means income support, income-based jobseeker’s allowance, 
income-related employment and~support allowance under Part~I of the Welfare Reform Act 2007
%,  % Words inserted (27.10.08) by SI 2008/1554 reg 62
%housing benefit and~council tax benefit
and housing benefit%  % Words substituted by SI 2013/458 Sch 2 para 5
.

\amendment{
Words inserted in reg.~32
(27.10.08) by the Employment and Support Allowance (Consequential Provisions) (No.~2) Regulations 2008 reg.~62.

Words substituted in reg. 32 (1.4.13) by the Council Tax Benefit Abolition (Consequential Provision) Regulations 2013 Sch. 2 para. 5.
}

\section[Part IX --- Revocation]{Part IX\\*Revocation}

\renewcommand\parthead{--- Part IX}

\subsection[33. Revocation and~savings]{Revocation and~savings}

33.---(1)  Subject to~
the Transitional Regulations and~ % Words inserted (3.3.03) by SI 2003/347 reg 2(5)
paragraph~(2), the Child Support Departure Direction and~Consequential Amendments Regulations 1996\footnote{S.I.\ 1996/2907.} shall be revoked with respect to~a particular case with effect from the date that these Regulations come into~force with respect to~that type of case (“the commencement date”).

(2) Where before the commencement date in respect of a particular case—
\begin{enumerate}\item[]
($a$) an application was made and~not determined for—
\begin{enumerate}\item[]
(i) a maintenance assessement;

(ii) a departure direction; or

(iii) a revision or supersession of a decision;
\end{enumerate}

($b$) the Secretary of State had begun but not completed a revision or supersession of a decision on his own initiative;

($c$) any time limit provided for in Regulations for making an application for a revision or a departure direction had not expired; or

($d$) any appeal was made but not decided or any time limit for making an appeal had not expired,
\end{enumerate}
the provisions of the Child Support Departure Direction and~Consequential Amendments Regulations 1996 shall continue to~apply for the purposes of—
\begin{enumerate}\item[]
($aa$) the decision on the application referred to~in sub-paragraph~($a$);

($bb$) the revision or supersession referred to~in sub-paragraph~($b$);

($cc$) the ability to~apply for the revision or the departure direction referred to~in sub-paragraph~($c$)  and~the decision whether to~revise or to~give a departure direction following any such application;

($dd$) any appeal outstanding or made during the time limit referred to~in sub-paragraph~($d$); or

\pagebreak[3]

($ee$) any revision, supersession or appeal or application for a departure direction in relation to~a decision, ability to~apply or appeal referred to~in sub-parargaphs~($aa$)  to~($dd$).
\end{enumerate}

(3) Where, after the commencement date, a decision with respect to~a departure direction is revised from a date which is prior to~the commencement date, the provisions of the Child Support Departure Direction and~Consequential Amendments Regulations 1996 shall continue to~apply for the purposes of that revision.

(4) Where, under regulation~28(1) of the Transitional Regulations, an application for a maintenance calculation is treated as an application for a maintenance assessment, the provisions of the Child Support Departure Direction and~Consequential Amendments Regulations 1996 shall continue to~apply for the purposes of an application for a departure direction in relation to~any such assessment made.

(5) For the purposes of this regulation—
\begin{enumerate}\item[]
($a$) “departure direction” and~“maintenance assessment” means as provided in section 54 of the Act before its amendment by the 2000 Act;

($b$) “revision or supersession” means a revision or supersession of a decision under section 16 or 17 of the Act before its amendment by the 2000 Act and~“any time limit for making an application for a revision” means any time limit provided for in Regulations made under section 16 of the Act; and

($c$) “2000 Act” means the Child Support, Pensions and~Social Security Act 2000. 
\end{enumerate}

\amendment{
Words inserted in reg. 33(1) (3.3.03) by the Child Support (Transitional Provision) (Miscellaneous Amendments) Regulations 2003 reg. 2(5).
}

\bigskip

Signed 
by authority of the Secretary of State for Social Security.

{\raggedleft
\emph{P.~Hollis}\\*Parliamentary Under-Secretary of State,\\*Department of Social Security

}

18th January 2001

\small

\part[Schedule~--- Equivalent weekly value of a transfer of property]{Schedule\\*Equivalent weekly value of a transfer of property}

\renewcommand\parthead{--- Schedule}

1.---(1)  Subject to~paragraph~3, the equivalent weekly value of a transfer of property shall be calculated by multiplying the value of a transfer of property determined in accordance with regulation~17 by the relevant factor specified in the Table set out in paragraph~2 (“the Table”).

(2) For the purposes of sub-paragaph (1), the relevant factor is the number in the Table at the intersection of the column for the statutory rate and~of the row for the number of years of liability.

(3) In sub-paragraph~(2)—
\begin{enumerate}\item[]
($a$) “the statutory rate” means interest at the statutory rate prescribed for a judgment debt or, in Scotland, the statutory rate in respect of interest included in or payable under a decree in the Court of Session, which in either case applies at the date of the court order or written agreement relating to~the transfer of the property;

($b$) “the number of years of liability” means the number of years, beginning on the date of the court order or written agreement relating to~the transfer of property and~ending on—
\begin{enumerate}\item[]
(i) the date specified in that order or agreement as the date on which maintenance for the youngest child in respect of whom that order or agreement was made shall cease; or

(ii) if no such date is specified, the date on which the youngest child specified in the order or agreement reaches the age of 18,
\end{enumerate}
and~where that period includes a fraction of a year, that fraction shall be treated as a full year if it is either one half or exceeds one half of a year, and~shall otherwise be disregarded.
\end{enumerate}

\medskip

2.  The Table referred to~in paragraph~1(1) is set out below—

{\footnotesize
\noindent
\begin{longtable}{p{33pt} llll llll}
%\begin{tabulary}{\linewidth}{Jllllllll}
\hline
\itshape \sloppyword{Number of years of liability}&\multicolumn{7}{l}{\itshape Statutory rate}\\
&7$.$0\%& 8$.$0\%&10$.$0\%&11$.$0\%&12$.$0\%&12$.$5\%&14$.$0\%&15$.$0\%\\
\hline
\endhead
\hline
\endlastfoot
1&$.$02058&$.$02077&$.$02115&$.$02135&$.$02154&$.$02163&$.$02192&$.$02212\\
2&$.$01064&$.$01078&$.$01108&$.$01123&$.$01138&$.$01145&$.$01168&$.$01183\\
3&$.$00733&$.$00746&$.$00773&$.$00787&$.$00801&$.$00808&$.$00828&$.$00842\\	
4&$.$00568&$.$00581&$.$00607&$.$00620&$.$00633&$.$00640&$.$00660&$.$00674\\
5&$.$00469&$.$00482&$.$00507&$.$00520&$.$00533&$.$00540&$.$00560&$.$00574\\
6&$.$00403&$.$00416&$.$00442&$.$00455&$.$00468&$.$00474&$.$00495&$.$00508\\
7&$.$00357&$.$00369&$.$00395&$.$00408&$.$00421&$.$00428&$.$00448&$.$00462\\
8&$.$00322&$.$00335&$.$00360&$.$00374&$.$00387&$.$00394&$.$00415&$.$00429\\
9&$.$00295&$.$00308&$.$00334&$.$00347&$.$00361&$.$00368&$.$00389&$.$00403\\
10&$.$00274&$.$00287&$.$00313&$.$00327&$.$00340&$.$00347&$.$00369&$.$00383\\
11&$.$00256&$.$00269&$.$00296&$.$00310&$.$00324&$.$00331&$.$00353&$.$00367\\
12&$.$00242&$.$00255&$.$00282&$.$00296&$.$00310&$.$00318&$.$00340&$.$00355\\
13&$.$00230&$.$00243&$.$00271&$.$00285&$.$00299&$.$00307&$.$00329&$.$00344\\
14&$.$00220&$.$00233&$.$00261&$.$00275&$.$00290&$.$00298&$.$00320&$.$00336\\
15&$.$00211&$.$00225&$.$00253&$.$00267&$.$00282&$.$00290&$.$00313&$.$00329\\
16&$.$00204&$.$00217&$.$00246&$.$00261&$.$00276&$.$00283&$.$00307&$.$00323\\
17&$.$00197&$.$00211&$.$00240&$.$00255&$.$00270&$.$00278&$.$00302&$.$00318\\
18&$.$00191&$.$00205&$.$00234&$.$00250&$.$00265&$.$00273&$.$00297&$.$00314\\
\end{longtable}
%\end{tabulary}

}

\medskip

3.  The Secretary of State may determine a lower equivalent weekly value than that determined in accordance with parargaphs~1 and~2 where the amount of child support maintenance that would be payable in consequence of agreeing to a variation of that value is lower than the amount of the periodical payments of maintenance which were payable under the court order or written agreement referred to in regulation~16. 

\vfill

\part{Explanatory Note}

\renewcommand\parthead{— Explanatory Note}

\subsection*{(This note is not part of the Regulations)}

These Regulations provide for variations to~the rate of child maintenance payable under the Child Support Act 1991 (c.~48) (“the Act”) consequent upon the introduction of changes to the child support system made by the Child Support, Pensions and Social Security Act 2000 (c.~19). Subject to savings for transitional purposes these Regulations revoke the Child Support Departure Direction and Consequential Amendments Regulations 1996 (S.I.~1996/2907).

These Regulations come into force at different times for different cases according to the dates on which provisions of the Child Support, Pensions and Social Security Act 2000 which are relevant to these Regulations are commenced for different types of cases.

Part I (regulations 1 to 3) contains general provisions relating to citation, commencement and interpretation, the giving or sending of documents and the determination of amounts as weekly amounts.

Part II (regulations 4 to 9) sets out the procedure for making and determining applications for a variation. This includes the ability to reject an application on a preliminary consideration. Among the reasons for this may be because the application is for an amount which does not exceed a threshold applicable to the ground, or because a circumstance prescribed for the purposes of section 28E of the Act, in regulation 7, applies. The Secretary of State is enabled to request further information (regulation 8) and may invite representations (regulation 9).

Part III (regulations 10 to 15) gives details of what constitutes expenses for the purposes of the ground for a variation in paragraph 2 of Schedule 4B to the Act. Part IV (regulations 16 and 17 and the Schedule) is concerned with a variation under paragraph 3 of Schedule 4B, in relation to a property or capital transfer, and Part V (regulations 18 to 20) concerns the additional cases for which provision is made in paragraph 4 of that Schedule. The factors to be taken into account in determining whether it would be just and equitable to agree a variation, referred to in section 28F of the Act, are set out in Part VI (regulation 21).

Part VII (regulations 22 to 30) provides for the effective date of maintenance calculations which take account of a variation, for the way in which each type of variation is to affect the non-resident parent’s liability, for situations where a variation may be applied without an application and for the circumstances, for the purposes of section 28F(3) of the Act, in which a variation is not to be agreed to.

Part VIII (regulations 31 and 32) prescribes the amount payable under the regular payments condition for the purposes of section 28C(2)($b$)  of the Act and benefit for the purposes of section 28E of the Act.

Part IX (regulation~33) provides for the revocation, with savings for transitional purposes, of the Child Support Departure Direction and Consequential Amendments Regulations 1996.

The impact on business of these Regulations was covered in the Regulatory Impact Assessment (RIA) for the Child Support, Pensions and Social Security Act 2000, in accordance with, and in consequence of which, these Regulations are made. A copy of that RIA has been placed in the libraries of both Houses of Parliament and can be obtained from the Department of Social Security, Regulatory Impact Unit, Adelphi, 1--11 John Adam Street, London \textsc{\lowercase{WC2N 6HT}}. 

\end{document}