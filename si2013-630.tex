\documentclass[12pt,a4paper]{article}

\newcommand\regstitle{The Universal Credit (Consequential, Supplementary, Incidental and Miscellaneous Provisions) Regulations 2013}

\newcommand\regsnumber{2013/630}

\title{\regstitle}

\author{S.I.\ 2013 No.\ 630}

\date{Made
13th March 2013\\
Laid before Parliament
18th March 2013\\
Coming into force
29th April 2013
}

%\opt{oldrules}{\newcommand\versionyear{1993}}
%\opt{newrules}{\newcommand\versionyear{2003}}
%\opt{2012rules}{\newcommand\versionyear{2012}}

\usepackage{soul}

\usepackage{csa-regs}

\setlength\headheight{42.11603pt}

%\hbadness=10000

\begin{document}

\maketitle

\enlargethispage{\baselineskip}

\noindent
The Secretary of State for Work and Pensions makes the following Regulations in exercise of the powers conferred by paragraph 5(4) of Schedule 1 to the Child Support Act 1991 (as it has effect apart from section 1 of the Child Support, Pensions and Social Security Act 2000)\footnote{1991 c.~48. Section 1 of the Child Support, Pensions and Social Security Act 2000 (c.~19) substituted a new schedule for Schedule 1 to the Child Support Act 1991 (“the 1991 Act”). Paragraph 5(4) of Schedule 1 (as it has effect apart from section 1 of the 1991 Act) was amended by Schedule 2, paragraph (2), to the Welfare Reform Act 2012 (c.~5).}, section 6J(2)($a$)  of the Jobseekers Act 1995\footnote{1995 c.~18. Section 6J was inserted by section 49 of the Welfare Reform Act 2012.} and sections 2(2), 4(6), 6(1) and (3), 26(2)($a$), 32, 40, 42(1), (2) and (3) and 96(4)($c$)  of, and paragraph 4(3) of Schedule 1 to, the Welfare Reform Act 2012\footnote{2012 c.~5.} (“the Act”).

This instrument has not been referred to the Social Security Advisory Committee because it contains only regulations made under provisions of the Act, and provisions of the Child Support Act 1991 and of the Jobseekers Act 1995 inserted into those Acts by the Act, and is made before the end of the period of 6 months beginning with the coming into force of those provisions\footnote{\emph{See} section 173(5) of the Social Security Administration Act 1992. The requirement to refer regulations to the Social Security Advisory Committee does not apply where regulations are contained in a statutory instrument made before the end of the period of six months beginning with the coming into force of the enactment under which the regulations were made.}.

In accordance with section 176(2)($b$)  of the Social Security Administration Act 1992, the Secretary of State has obtained the agreement of organisations appearing to him to be representative of the authorities concerned that proposals in respect of these Regulations should not be referred to them. 

{\sloppy

\tableofcontents

}

\bigskip

\setcounter{secnumdepth}{-2}

\section[Part I --- General]{Part~I\\*General}

\renewcommand\parthead{--- Part I}

\subsection[1. Citation, commencement, extent and application]{Citation, commencement, extent and application}

1.—(1) These Regulations may be cited as the Universal Credit (Consequential, Supplementary, Incidental and Miscellaneous Provisions) Regulations 2013.

(2) These Regulations come into force on 29th April 2013.

(3) Subject to paragraphs (4) to (6), each of the amendments made by these Regulations has the same extent and application as the provision amended.

(4) The amendments made by regulations 18, 55, 56, 57, 68 and 69 extend to England and Wales but apply in relation to England only.

(5) The amendment made by regulation 21 extends to England and Wales and Scotland only.

(6) The amendment made by regulation 51 extends to England and Wales only.

\section[Part II --- Amendments of primary legislation]{Part II\\*Amendments of primary legislation}

\renewcommand\parthead{--- Part II}

\subsection[2. Amendment of the Maintenance Orders Act 1950]{Amendment of the Maintenance Orders Act 1950}

2.—(1) The Maintenance Orders Act 1950\footnote{14 Geo.~6 c.~37. Sections 4(1)($d$)  and 9(1)($d$)  were inserted by paragraphs 35 and 36 of Schedule 10 to the Social Security Act 1986 (c.~50) and were amended by paragraph 3 of Schedule 2 to the Social Security (Consequential Provisions) Act 1992 (c.~6).} is amended as follows.

(2) In section 4(1)($d$)  (contributions under the Children and Young Persons Act 1933 and the National Assistance Act 1948), after “income support” insert “or universal credit”.

(3) In section 9(1)($d$)  (contributions under the Children and Young Persons (Scotland) Act 1937 and the National Assistance Act 1948), after “income support” insert “or universal credit”.

\subsection[3. Amendment of the Rent Act 1977]{Amendment of the Rent Act 1977}

3.  In section 72A of the Rent Act 1977 (amounts attributable to services)\footnote{1977 c.~42. Section 72A was inserted by paragraph 47 of Schedule 2 to the Social Security (Consequential Provisions) Act 1992.}, after “Benefits Act 1992” insert “or to assist the Secretary of State in the administration of universal credit”.

\subsection[4. Amendment of the Magistrates’ Courts Act 1980]{Amendment of the Magistrates’ Courts Act 1980}

4.—(1) The Magistrates’ Courts Act 1980\footnote{1980 c.~43. Sections 89(2A) and 90(3A) were inserted by section 47 of the Criminal Justice and Public Order Act 1994 (c.~33). They were amended by S.I.~2006/1737.} is amended as follows.

(2) In section 89(2A) (transfer of fine order), after “power to deduct fines etc from” insert “universal credit and”.

(3) In section 90(3A) (transfer of fines to Scotland or Northern Ireland), after “power to deduct fines from” insert “universal credit and”.

\subsection[5. Amendment of the Transport Act 1982]{Amendment of the Transport Act 1982}

5.  In section 70(2)($b$)  of the Transport Act 1982 (payments in respect of applicants for exemption from wearing seat belts)\footnote{1982 c.~49. Section 70(2)($b$)  was amended by paragraph 57 of Schedule 10 to the Social Security Act 1986 (c.~50), paragraph 3 of Schedule 3 to the Welfare Reform Act 2007 (c.~5) and paragraph 9 of Schedule 3 to the Tax Credits Act 2002 (c.~21).}, after “those in receipt of” insert “universal credit,”.

\subsection[6. Amendment of the Housing Act 1988]{Amendment of the Housing Act 1988}

6.  In section 41A of the Housing Act 1988 (amounts attributable to services)\footnote{1988 c.~50. Section 41A was inserted by paragraph 103 of Schedule 2 to the Social Security (Consequential Provisions) Act 1992.}, after “Benefits Act 1992” insert “or to assist the Secretary of State in the administration of universal credit”.

\subsection[7. Amendment of the Employment Act 1989]{Amendment of the Employment Act 1989}

7.  In section 8(4) of the Employment Act 1989 (power to exempt discrimination in favour of lone parents in connection with training)\footnote{1989 c.~38.}, for paragraph ($b$)  substitute—
\begin{quotation}
“($b$) “couple” has the meaning given by section 39(1) of the Welfare Reform Act 2012; and

($c$) “lone parent” means a person who—
\begin{enumerate}\item[]
(i) is not a member of a couple, and

(ii) is responsible for, and a member of the same household as, a child.”.
\end{enumerate}
\end{quotation}

\subsection[8. Amendment of the Criminal Justice Act 1991]{Amendment of the Criminal Justice Act 1991}

8.  In section 24 of the Criminal Justice Act 1991 (recovery of fines etc by deductions from income support)\footnote{1991 c.~53. Relevant amendments were made by paragraph 31 of Schedule 2 to the State Pension Credit Act 2002 (c.~16) and paragraph 8 of Schedule 3 to the Welfare Reform Act 2007 (c.~5).}—
\begin{enumerate}\item[]
($a$) in the heading, after “by deductions from” insert “universal credit and”; and

($b$) in subsections (1) and (2)($d$), before “income support” insert “universal credit,”.
\end{enumerate}

\subsection[9. Amendment of the Value Added Tax Act 1994]{Amendment of the Value Added Tax Act 1994}

9.—(1) The Value Added Tax Act 1994\footnote{1994 c.~23. Schedule 7A was inserted by Schedule 31 to the Finance Act 2001 (c.~9). Group 3 and the Notes to that Group were amended by paragraph 48 of Schedule 3 to the Tax Credits Act 2002 (c.~21) and S.I.s 2002/1100, 2011/1043 and 2013/601. Note (1D) of Schedule 8 was inserted by S.I.~2000/805 and substituted by paragraph 49 of Schedule 3 to the Tax Credits Act 2002.} is amended as follows.

(2) In paragraph (2) of Note 6 to Group 3 in Part~II of Schedule 7A (charge at reduced rate; meaning of qualifying person for the purposes of Group 3), after sub-paragraph ($h$)  insert—
\begin{quotation}
“($i$) universal credit under Part~I of the Welfare Reform Act 2012.”.
\end{quotation}

(3) In Note (1D) to Group 15 in Part~II of Schedule 8 (zero rating)—
\begin{enumerate}\item[]
($a$) omit “and” at the end of paragraph ($e$); and

($b$) after paragraph ($f$), insert—
\begin{quotation}
“and

($g$) universal credit under Part~I of the Welfare Reform Act 2012.”.
\end{quotation}
\end{enumerate}

\subsection[10. Amendment of the Jobseekers Act 1995]{Amendment of the Jobseekers Act 1995}

10.  In paragraph 2 of Schedule 1 to the Jobseekers Act 1995 (supplementary provisions: limited capability for work)\footnote{1995 c.~18. Paragraph 2 was substituted by paragraph 12 of Schedule 3 to the Welfare Reform Act 2007.}—
\begin{enumerate}\item[]
($a$) in sub-paragraph (1), insert at the end “or Part~I of the Welfare Reform Act 2012 (universal credit) as the Secretary of State considers appropriate in the person’s case”, and

($b$) after sub-paragraph (2), insert—
\begin{quotation}
“(3) References in Part~I of the Welfare Reform Act 2012 to the purposes of that Part are to be construed, where the provisions of that Part have effect for the purposes of this Act, as references to the purposes of this Act.”.
\end{quotation}
\end{enumerate}

\subsection[11. Amendment of the Employment Tribunals Act 1996]{Amendment of the Employment Tribunals Act 1996}

11.—(1) The Employment Tribunals Act 1996\footnote{1996 c.~17. Section 16(5)($cc$)  was inserted by paragraph 147 of Schedule 7 to the Social Security Act 1998. Sections 16(3)($a$), ($b$) and ($c$) and (5)($cc$)  and ($e$) and 17(1) were amended by paragraph 15 of Schedule 3 to the Welfare Reform Act 2007. Section 16(3)($b$) and ($c$) and (5)($c$) was amended by section 1 of the Employment Rights (Dispute Resolution) Act 1998 (c.~8).} is amended as follows.

(2) In section 16 (power to provide for recoupment of benefits) in the following provisions, before “jobseeker’s allowance” insert “universal credit,”—
\begin{enumerate}\item[]
($a$) subsection (3)($a$), ($b$)  and ($c$); and

($b$) subsection (5)($e$).
\end{enumerate}

(3) In section 16(5)($cc$), before “a jobseeker’s allowance” insert “universal credit,”.

(4) In section 17 (recoupment: further provisions), in the introductory words in subsection (1) and in paragraph ($b$)  of that subsection, before “jobseeker’s allowance” insert “universal credit,”.

\subsection[12. Amendment of the Housing Act 1996]{Amendment of the Housing Act 1996}

12.—(1) The Housing Act 1996\footnote{1996 c.~52. Section 160A was inserted by section 14(2) of the Homelessness Act 2002 (c.~7). Section 185(2A) was inserted by section 117(4) of the Immigration Act 1999 (c.~33) and substituted by paragraph 7 of Schedule 1 to the Homelessness Act 2002.} is amended as follows.

(2) In the heading to Part IV (Housing Benefit and Related Matters), before “Housing Benefit” insert “Universal Credit,”.

(3) In section 160ZA(3) (allocation only to eligible and qualifying persons: England), after “from entitlement to” insert “universal credit or”.

(4) In section 160A(4) (allocation only to eligible persons), before “housing benefit” insert “universal credit or”.

(5) In section 185(2A) (persons from abroad not eligible for housing assistance), before “housing benefit” insert “universal credit or”.

(6) In section 231(2) (extent), before “housing benefit” insert “universal credit,”.

\subsection[13. Amendment of the Housing Grants, Construction and Regeneration Act 1996]{Amendment of the Housing Grants, Construction and Regeneration Act 1996}

13.—(1) Section 3 of the Housing Grants, Construction and Regeneration Act 1996\footnote{1996 c.~53.} (ineligible applicants) is amended as follows.

(2) In subsection (4), after “subsection (3)” insert “made by the Welsh Ministers”.

(3) After subsection (4) insert—
\begin{quotation}
“(4A) Regulations under subsection (3) made by the Secretary of State may proceed wholly or in part by reference to the provisions relating to entitlement to—
\begin{enumerate}\item[]
($a$) housing benefit;

($b$) universal credit; or

($c$) any other form of assistance,
\end{enumerate}
as they have effect from time to time.”.
\end{quotation}

\subsection[14. Amendment of the Data Protection Act 1998]{Amendment of the Data Protection Act 1998}

14.—(1) The Data Protection Act 1998\footnote{1998 c.~29. Paragraph 2($e$) of the Table in section 56(6) was amended by S.I.~2011/2425. Section 75(5A) was inserted by S.I.~2011/2425.} is amended as follows.

(2) In paragraph 2($e$)  of the Table in section 56(6) (prohibition of requirement as to production of certain records), for “or Part~I of the Welfare Reform Act 2007” substitute “,~Part~I of the Welfare Reform Act 2007 or Part~I of the Welfare Reform Act 2012”.

(3) In section 75(5A) (extent), after “Welfare Reform Act 2007” insert “and Part~I of the Welfare Reform Act 2012”.

\subsection[15. Amendment of the Welfare Reform and Pensions Act 1999]{Amendment of the Welfare Reform and Pensions Act 1999}

15.  In section 72(3) of the Welfare Reform and Pensions Act 1999 (supply of information for certain purposes)\footnote{1999 c.~30. Section 72(3) was amended by Schedules 3 and 8 to the Welfare Reform Act 2007, section 2(5) of the Welfare Reform Act 2009 and Part I of Schedule 14 to the Welfare Reform Act 2012.}—

($a$) at the end of paragraph ($c$), delete “or”; and

($b$) after paragraph ($d$)  insert—
\begin{quotation}
“,~or

($e$) Part~I of the Welfare Reform Act 2012.”.
\end{quotation}

\subsection[16. Amendment of the Income Tax (Earnings and Pensions) Act 2003]{Amendment of the Income Tax (Earnings and Pensions) Act 2003}

16.—(1) The Income Tax (Earnings and Pensions) Act 2003\footnote{2003 c.~1.} is amended as follows.

(2) At the end of section 318D(2) (childcare)\footnote{Section 318D was inserted by paragraph 1 of Schedule 13 to the Finance Act 2004 (c.~12) and amended by paragraph 6 of Schedule 8 to the Finance Act 2011 (c.~11).}, insert “or section 12 of the Welfare Reform Act 2012 relating to amounts in respect of childcare costs that may be included in the calculation of an award of universal credit”.

(3) In section 660(1) (taxable benefits: UK benefits---Table A) in column 1 of Table A, for “Contributory employment and support allowance” substitute “Employment and support allowance”.

(4) In section 675(1) (interpretation)\footnote{Section 675 was amended by S.I.~2005/3229.}, for the definitions of “contribution-based jobseeker’s allowance” and “income-based jobseeker’s allowance” substitute—
\begin{quotation}
\begin{sloppypar}
““contribution-based jobseeker’s allowance” means a jobseeker’s allowance entitlement to which is based on the claimant’s satisfying conditions which include those set out in section 2 of JSA 1995;
\end{sloppypar}

“income-based jobseeker’s allowance” means a jobseeker’s allowance entitlement to which is based on the claimant’s satisfying conditions which include those set out in section 3 of JSA 1995 or a joint-claim jobseeker’s allowance (which means a jobseeker’s allowance entitlement to which arises by virtue of section 1(2B) of JSA 1995);”.
\end{quotation}

\subsection[17. Amendment of the Courts Act 2003]{Amendment of the Courts Act 2003}

17.—(1) The Courts Act 2003\footnote{2003 c.~39.} is amended as follows.

(2) In Part~III of Schedule 5 (collection of fines and other sums imposed on conviction: attachment of earnings orders and applications for benefit deductions), in paragraph 10($a$)  (meaning of relevant benefit) before “income support” insert “universal credit and”.

(3) In paragraph 2(1)($a$)(v)  of Schedule 6 (discharge of fines by unpaid work), before “income support” insert “universal credit and”.

\subsection[18. Amendment of the Housing Act 2004]{Amendment of the Housing Act 2004}

18.—(1) The Housing Act 2004\footnote{2004 c.~34.} is amended as follows.

(2) In section 73 (other consequences of operating unlicensed HMOs: rent repayment orders)—
\begin{enumerate}\item[]
($a$) in subsection (5), before “housing benefit” insert “relevant award or awards of universal credit or the”;

($b$) in subsection (6), for paragraph ($b$)  substitute—
\begin{quotation}
“($b$) that—
\begin{enumerate}\item[]
(i) one or more relevant awards of universal credit have been paid (to any person); or

(ii) housing benefit has been paid (to any person) in respect of periodical payments payable in connection with the occupation of a part or parts of the HMO,
\end{enumerate}
during any period during which it appears to the tribunal that such an offence was being committed,”;
\end{quotation}

($c$) after subsection (6), insert—
\begin{quotation}
“(6A) In subsection (6)($b$)(i), “relevant award of universal credit” means an award of universal credit the calculation of which included an amount under section 11 of the Welfare Reform Act 2012, calculated in accordance with Schedule~4 to the Universal Credit Regulations 2013 (housing costs element for renters)\footnote{S.I.~2013/376.} or any corresponding provision replacing that Schedule, in respect of periodical payments payable in connection with the occupation of a part or parts of the HMO.”;
\end{quotation}

($d$) in subsection (8)($a$), for the words from “housing benefit” to the end substitute—
\begin{quotation}
“(i) one or more relevant awards of universal credit, or

(ii) housing benefit paid in connection with occupation of a part or parts of the HMO,”;
\end{quotation}

($e$) in subsection (10)—
\begin{enumerate}\item[]
(i) in the definition of “the appropriate person”, before “housing benefit” insert “universal credit or”;

(ii) for the definition of “periodical payments” substitute—
\begin{quotation}
““periodical payments” means—
\begin{enumerate}\item[]
($a$) 
payments in respect of which an amount under section 11 of the Welfare Reform Act 2012 may be included in the calculation of an award of universal credit, as referred to in paragraph 3 of Schedule 4 to the Universal Credit Regulations 2013 (“relevant payments”) or any corresponding provision replacing that paragraph; and

($b$) 
periodical payments in respect of which housing benefit may be paid by virtue of regulation 12 of the Housing Benefit Regulations 2006 or any corresponding provision replacing that regulation;”; and
\end{enumerate}
\end{quotation}
\end{enumerate}

($f$) in subsection (11)($b$), before “housing benefit” insert “universal credit or”.
\end{enumerate}

(3) In section 74 (further provision about rent repayment orders)—
\begin{enumerate}\item[]
($a$) in subsection (2)—
\begin{enumerate}\item[]
(i) for paragraph ($b$)  substitute—
\begin{quotation}
“($b$) that—
\begin{enumerate}\item[]
(i) one or more relevant awards of universal credit (as defined in section 73(6A)) were paid (whether or not to the appropriate person), or

(ii) housing benefit was paid (whether or not to the appropriate person) in respect of periodical payments payable in connection with occupation of a part or parts of the HMO,
\end{enumerate}
during any period during which it appears to the tribunal that such an offence was being committed in relation to the HMO in question,”; and
\end{quotation}

(ii) in the closing words, for “an amount equal to the total amount of housing benefit paid as mentioned in paragraph ($b$)” substitute “the amount mentioned in subsection (2A)”;
\end{enumerate}

($b$) after subsection (2) insert—
\begin{quotation}
“(2A) The amount referred to in subsection (2) is—
\begin{enumerate}\item[]
($a$) an amount equal to—
\begin{enumerate}\item[]
(i) where one relevant award of universal credit was paid as mentioned in subsection (2)($b$)(i), the amount included in the calculation of that award under section 11 of the Welfare Reform Act 2012, calculated in accordance with Schedule 4 to the Universal Credit Regulations 2013 (housing costs element for renters) or any corresponding provision replacing that Schedule, or the amount of the award if less; or

(ii) if more than one such award was paid as mentioned in subsection (2)($b$)(i), the sum of the amounts included in the calculation of those awards as referred to in sub-paragraph (i), or the sum of the amounts of those awards if less, or
\end{enumerate}

($b$) an amount equal to the total amount of housing benefit paid as mentioned in subsection (2)($b$)(ii), (as the case may be)”.
\end{enumerate}
\end{quotation}

($c$) in subsection (3), for “total amount of housing benefit paid as mentioned in that paragraph” substitute “amount mentioned in subsection~(2A)”;

($d$) in subsection (6)($b$)(i), after “payments of” insert “relevant awards of universal credit or”;

($e$) in subsection (7)—
\begin{enumerate}\item[]
(i) in paragraph ($a$), before “housing benefit” insert “relevant awards of universal credit,”; and

(ii) in paragraph ($b$), for the words from “any amount” to the end substitute—
\begin{quotation}
“(i) where one or more relevant awards of universal credit were payable during the period in question, the amount mentioned in subsection (2A)($a$)  in respect of the award or awards that related to the occupation of the part of the HMO occupied by him during that period; or

(ii) any amount of housing benefit payable in respect of the occupation of the part of the HMO occupied by him during the period in question”; and
\end{quotation}
\end{enumerate}

($f$) in subsections (9)($a$)  and (15)($a$), before “housing benefit” insert “universal credit or”.
\end{enumerate}

(4) In section 96 (other consequences of operating unlicensed houses: rent repayment orders)—
\begin{enumerate}\item[]
($a$) in subsection (5), before “housing benefit” insert “relevant award or awards of universal credit or the”;

($b$) in subsection (6), for paragraph ($b$)  substitute—
\begin{quotation}
“($b$) that—
\begin{enumerate}\item[]
(i) one or more relevant awards of universal credit have been paid (to any person); or

(ii) housing benefit has been paid (to any person) in respect of periodical payments payable in connection with the occupation of the whole or any part or parts of the house,
\end{enumerate}
during any period during which it appears to the tribunal that such an offence was being committed,”;
\end{quotation}

($c$) after subsection (6), insert—
\begin{quotation}
“(6A) In subsection (6)($b$)(i), “relevant award of universal credit” means an award of universal credit the calculation of which included an amount under section 11 of the Welfare Reform Act 2012, calculated in accordance with Schedule~4 to the Universal Credit Regulations 2013 (housing costs element for renters) or any corresponding provision replacing that Schedule, in respect of periodical payments payable in connection with the occupation of the whole or any part or parts of the house.”;
\end{quotation}

($d$) in subsection (8)($a$), for the words from “housing benefit” to the end substitute—
\begin{quotation}
“(i) one or more relevant awards of universal credit, or

(ii) housing benefit paid in connection with occupation of the whole or any part or parts of the house”;
\end{quotation}

($e$) in subsection (10)—
\begin{enumerate}\item[]
(i) in the definition of “the appropriate person”, before “housing benefit” insert “universal credit or”;

(ii) for the definition of “periodical payments” substitute—
\begin{quotation}
““periodical payments” means—
\begin{enumerate}\item[]
($a$) 
payments in respect of which an amount under section 11 of the Welfare Reform Act 2012 may be included in the calculation of an award of universal credit, as referred to in paragraph 3 of Schedule 4 to the Universal Credit Regulations 2013 (“relevant payments”) or any corresponding provision replacing that paragraph; and

($b$) 
periodical payments in respect of which housing benefit may be paid by virtue of regulation 12 of the Housing Benefit Regulations 2006 or any corresponding provision replacing that regulation;”; and
\end{enumerate}
\end{quotation}
\end{enumerate}

($f$) in subsection (11)($b$), before “housing benefit” insert “universal credit or”.
\end{enumerate}

(5) In section 97 (further provision about rent repayment orders)—
\begin{enumerate}\item[]
($a$) in subsection (2)—
\begin{enumerate}\item[]
(i) for paragraph ($b$)  substitute—
\begin{quotation}
“($b$) that—
\begin{enumerate}\item[]
(i) one or more relevant awards of universal credit (as defined in section 96(6A)) were paid (whether or not to the appropriate person), or

(ii) housing benefit was paid (whether or not to the appropriate person) in respect of periodical payments payable in connection with occupation of the whole or any part or parts of the house,
\end{enumerate}
during any period during which it appears to the tribunal that such an offence was being committed in relation to the house,”; and
\end{quotation}

(ii) in the closing words, for “an amount equal to the total amount of housing benefit paid as mentioned in paragraph ($b$)” substitute “the amount mentioned in subsection (2A)”;
\end{enumerate}

($b$) after subsection (2) insert—
\begin{quotation}
“(2A) The amount referred to in subsection (2) is—
\begin{enumerate}\item[]
($a$) an amount equal to—
\begin{enumerate}\item[]
(i) where one relevant award of universal credit was paid as mentioned in subsection (2)($b$)(i), the amount included in the calculation of that award under section 11 of the Welfare Reform Act 2012, calculated in accordance with Schedule 4 to the Universal Credit Regulations 2013 (housing costs element for renters) or any corresponding provision replacing that Schedule, or the amount of the award if less; or

(ii) if more than one such award was paid as mentioned in subsection (2)($b$)(i), the sum of the amounts included in the calculation of those awards as referred to in sub-paragraph (i), or the sum of the amounts of those awards if less, or
\end{enumerate}

($b$) an amount equal to the total amount of housing benefit paid as mentioned in subsection (2)($b$)(ii), (as the case may be)”.
\end{enumerate}
\end{quotation}

($c$) in subsection (3), for “total amount of housing benefit paid as mentioned in that paragraph” substitute “amount mentioned in subsection~(2A);

($d$) in subsection (6)($b$)(i), after “payments of” insert “relevant awards of universal credit or”;

($e$) in subsection (7)—
\begin{enumerate}\item[]
(i) in paragraph ($a$), before “housing benefit” insert “relevant awards of universal credit,”; and

(ii) in paragraph ($b$), for the words from “any amount” to the end substitute—
\begin{quotation}
“(i) where one or more relevant awards of relevant universal credit were payable during the period in question, the amount mentioned in subsection (2A)($a$)  in respect of the award or awards that related to the occupation of the part of the HMO occupied by him during that period; or

(ii) any amount of housing benefit payable in respect of the occupation of the part of the HMO occupied by him during the period in question”; and
\end{quotation}
\end{enumerate}

($f$) in subsections (9)($a$)  and (15)($a$), before “housing benefit” insert “universal credit or”.
\end{enumerate}

\subsection[19. Amendment of the Childcare Act 2006]{Amendment of the Childcare Act 2006}

19.—(1) The Childcare Act 2006\footnote{2006 c.~21.} is amended as follows.

(2) After section 6(2)($a$)(i)  (duty to secure sufficient childcare for working parents), insert—
\begin{quotation}
“(ia) the provision of childcare in respect of which an amount in respect of childcare costs may be included under section 12 of the Welfare Reform Act 2012 in the calculation of an award of universal credit, and”.
\end{quotation}

(3) In section 83 (supply of information to HMRC and local authorities)—
\begin{enumerate}\item[]
($a$) in the heading, after “Supply of information to” insert “the Secretary of State,”,

($b$) in subsections (1) and (2), after “prescribed information to” insert “the Secretary of State,”, and

($c$) in subsection (3), after paragraph ($a$)  insert—
\begin{quotation}
“($aa$) in the case of information to be provided to the Secretary of State, information which the Secretary of State may require for the purposes of the Secretary of State’s functions in relation to universal credit;”.
\end{quotation}
\end{enumerate}

\section[Part III --- Amendments of secondary legislation]{Part III\\*Amendments of secondary legislation}

\subsection[Chapter I --- Social security benefits]{Chapter I\\*Social security benefits}

\renewcommand\parthead{--- Part III Chapter I}

\subsubsection[20. Amendment of the Social Security (Benefit) (Married Women and Widows Special Provisions) Regulations 1974]{Amendment of the Social Security (Benefit) (Married Women and Widows Special Provisions) Regulations 1974}

20.—(1) The Social Security (Benefit) (Married Women and Widows Special Provisions) Regulations 1974\footnote{S.I.~1974/2010.} are amended as follows.

(2) In regulation 1(2) (interpretation)\footnote{Regulation 1(2) was amended by S.I.~1984/458 and 2008/1544.}—
\begin{enumerate}\item[]
($a$) omit the definition of “contributory employment and support allowance”; and

($b$) after the definition of “the determining authority”, insert—

““employment and support allowance” means an allowance under Part~I of the Welfare Reform Act 2007 as amended by the provisions of Schedule 3, and Part~I of Schedule 14, to the Welfare Reform Act 2012 that remove references to an income-related allowance, and a contributory allowance under Part~I of the Welfare Reform Act 2007 as that Part has effect apart from those provisions;”.
\end{enumerate}

(3) In the title of regulation 3 (modifications in relation to widows, of provisions with respect to unemployment and short-term incapacity benefit, contributory employment and support allowance, maternity allowance and Category A retirement pension) and in paragraphs (1)($a$)  and ($b$)  and (5)($a$)  and ($b$)  of that regulation, omit “contributory” in all places where it occurs.

\subsubsection[21. Amendment of the Social Security (Benefit) (Members of the Forces) Regulations 1975]{\sloppy Amendment of the Social Security (Benefit) (Members of the Forces) Regulations 1975}

21.—(1) The Social Security (Benefit) (Members of the Forces) Regulations 1975\footnote{S.I.~1975/493.} are amended as follows.

(2) In regulation 3 (unemployment benefit)\footnote{Regulation 3 was amended by S.I.~1996/207, 2000/1982 and 2009/2054.}—
\begin{enumerate}\item[]
($a$) in paragraph (1)—

(i) after “for the purposes of” insert “section 6J,”; and

\begin{sloppypar}
(ii) after “(circumstances in which a jobseeker’s allowance is not payable)” insert “or section 26 of the Welfare Reform Act 2012 (higher-level sanctions)”; and
\end{sloppypar}

($b$) in paragraph (3), after “a jobseeker’s allowance” insert “or universal credit under Part~I of the Welfare Reform Act 2012”.
\end{enumerate}

(3) In regulation 5 (application of the Act, the Northern Ireland Act and regulations)\footnote{Regulation 5 was amended by S.I.~1996/1345.}, after “or of the Jobseekers Act 1995 and regulations made under it,” insert “or Part~I of the Welfare Reform Act 2012 and regulations made under it,”.

\subsubsection[22. Amendment of the Social Security (Mariners’ Benefits) Regulations 1975]{Amendment of the Social Security (Mariners’ Benefits) Regulations 1975}

22.—(1) The Social Security (Mariners’ Benefits) Regulations 1975\footnote{S.I.~1975/529.} are amended as follows.

(2) In regulation 1(2) (interpretation)\footnote{Regulation 1(2) was amended by S.I.~2008/1554.}—
\begin{enumerate}\item[]
($a$) after the definition of “British ship” insert—
\begin{quotation}
““contribution-based jobseeker’s allowance” means an allowance under the Jobseekers Act 1995 as amended by the provisions of Part~I of Schedule 14 to the Welfare Reform Act 2012 that remove references to an income-based allowance, and a contribution-based allowance under the Jobseekers Act 1995 as that Act has effect apart from those provisions;”;
\end{quotation}

($b$) for the definition of “contributory employment and support allowance” substitute—
\begin{quotation}
““contributory employment and support allowance” means an allowance under Part~I of the Welfare Reform Act as amended by the provisions of Schedule 3, and Part~I of Schedule 14, to the Welfare Reform Act 2012 that remove references to an income-related allowance, and a contributory allowance under Part~I of the Welfare Reform Act as that Part has effect apart from those provisions;”; and
\end{quotation}

($c$) after the definition of “mariner”, insert—
\begin{quotation}
““new style JSA” means a jobseeker’s allowance under the Jobseekers Act 1995 as amended by the provisions of Part~I of Schedule 14 to the 2012 Act that remove references to an income-based allowance;”.
\end{quotation}
\end{enumerate}

(3) In regulation 2 (days in periods of paid leave not to be treated as days of unemployment)\footnote{Regulation 2 was amended by S.I.~1996/207 and 1996/1516.}, after “that period of leave” insert “and, in relation to new style JSA, shall not be regarded as entitled to a jobseeker’s allowance for any day in that period”.

(4) In regulation 6 (special provisions relating to days of unemployment etc.)\footnote{Regulation 6 was amended by S.I.~1984/1303, 1995/829, 1996/207, 1997/563 and 2008/1554.}, after paragraph (1), insert—
\begin{quotation}
“(1A) In relation to new style JSA, a mariner or share fisherman employed as such on board any ship or vessel shall be treated as complying with the work-related requirements referred to in section 6(2) of the Jobseekers Act 1995 during any period when he is absent from Great Britain if he would comply with those requirements but for the fact that he is absent from Great Britain.”.
\end{quotation}

\subsubsection[23. Amendment of the Social Security Benefit (Persons Abroad) Regulations 1975]{Amendment of the Social Security Benefit (Persons Abroad) Regulations 1975}

23.—(1) The Social Security Benefit (Persons Abroad) Regulations 1975\footnote{S.I.~1975/563; regulation 1(2) has been amended in ways not relevant to these Regulations.} are amended as follows.

(2) In regulation 1(2) (interpretation), after the definition of “the Industrial Injuries Employment Regulations” insert—
\begin{quotation}
““jobseeker’s allowance” means an allowance under the Jobseekers Act 1995 as amended by the provisions of Part~I of Schedule~14 to the Welfare Reform Act 2012 that remove references to an income-based allowance, and a contribution-based allowance under the Jobseekers Act 1995 as that Act has effect apart from those provisions;”.
\end{quotation}

(3) In regulation 11(1A) (modification of the Act in relation to employment on the Continental Shelf\footnote{Regulation 11(1A) was inserted by S.I.~1996/207.} omit “contribution-based” in both places where it occurs.

\subsubsection[24. Amendment of the Social Security (Medical Evidence) Regulations 1976]{Amendment of the Social Security (Medical Evidence) Regulations 1976}

24.  In regulation 1 of the Social Security (Medical Evidence) Regulations 1976 (interpretation)\footnote{S.I.~1976/615; regulation 1(2) has been amended in ways that are not relevant to these Regulations.}—
\begin{enumerate}\item[]
($a$) in paragraph (2)—
\begin{enumerate}\item[]
(i) for the definition of “limited capability for work” substitute—
\begin{quotation}
““limited capability for work” has the meaning—
\begin{enumerate}\item[]
($a$) 
for the purposes of employment and support allowance, given in section 1(4) of the Welfare Reform Act 2007; and

($b$) 
for the purposes of universal credit, given in section~37 of the Welfare Reform Act 2012;”; and
\end{enumerate}
\end{quotation}

(ii) for the definition of “limited capability for work assessment” substitute—
\begin{quotation}
““limited capability for work assessment” means the assessment of whether a person has limited capability for work—
\begin{enumerate}\item[]
($a$) 
for the purposes of old style ESA, under Part~V of the Employment and Support Allowance Regulations;

($b$) 
for the purposes of new style ESA, under Part~IV of the Employment and Support Allowance Regulations 2013;

($c$) 
for the purposes of universal credit, under Part~V of the Universal Credit Regulations 2013;”;
\end{enumerate}
\end{quotation}
\end{enumerate}

($b$) after paragraph (4), insert—
\begin{quotation}
“(5) For the purposes of the definition of “limited capability for work assessment” in paragraph (2)—
\begin{enumerate}\item[]
($a$) “old style ESA” means an allowance under Part~I of the Welfare Reform Act 2007 as that Part has effect apart from the amendments made by Schedule 3, and Part~I of Schedule 14, to the Welfare Reform Act 2012 that remove references to an income-related allowance; and

($b$) “new style ESA” means an allowance under Part~I of the Welfare Reform Act 2007 as amended by the provisions of Schedule 3, and Part~I of Schedule 14, to the Welfare Reform Act 2012 that remove references to an income-related allowance.”.
\end{enumerate}
\end{quotation}
\end{enumerate}

\subsubsection[25. Amendment of the Social Security (Overlapping Benefits) Regulations 1979]{Amendment of the Social Security (Overlapping Benefits) Regulations 1979}

25.  In regulation 2(1) of the Social Security (Overlapping Benefits) Regulations 1979 (interpretation)\footnote{S.I.~1979/597. Regulation 2(1) has been amended by S.I.~2013/388; there are other amendments that are not relevant to these Regulations.}—
\begin{enumerate}\item[]
($a$) after the definition of “child benefit” insert—
\begin{quotation}
““contribution-based jobseeker’s allowance” means an allowance under the Jobseekers Act as amended by the provisions of Part~I of Schedule 14 to the 2012 Act that remove references to an income-based allowance, and a contribution-based allowance under the Jobseekers Act as that Act has effect apart from those provisions;”;
\end{quotation}

($b$) after the definition of “contributory benefit” insert—
\begin{quotation}
““contributory employment and support allowance” means an allowance under Part~I of the Welfare Reform Act as amended by the provisions of Schedule 3, and Part~I of Schedule 14, to the 2012 Act that remove references to an income-related allowance, and a contributory allowance under Part~I of the Welfare Reform Act as that Part has effect apart from those provisions;”;
\end{quotation}

($c$) after the definition of “disablement pension” insert—
\begin{quotation}
““income-based jobseeker’s allowance” means an income-based allowance under the Jobseekers Act;

\begin{sloppypar}
“income-related employment and support allowance” means an income-related allowance under Part~I of the Welfare Reform Act;”; and
\end{sloppypar}
\end{quotation}

($d$) in the definition of “personal benefit”, after the words “dependency benefit” insert “or universal credit under Part~I of the 2012 Act”.
\end{enumerate}

\subsubsection[26. Amendment of the Social Security (Widow’s Benefit and Retirement Pensions) Regulations 1979]{Amendment of the Social Security (Widow’s Benefit and Retirement Pensions) Regulations 1979}

26.  Regulation 4 of the Social Security (Widow’s Benefit and Retirement Pensions) Regulations 1979 (days to be treated as days of increment)\footnote{S.I.~1979/642. Regulation 4(1)($e$) was inserted by S.I.~2011/634, regulation 4(2)($b$) was amended by S.I.~1996/1345 and regulation 4(5) was inserted by S.I.~1996/1345.}—
\begin{enumerate}\item[]
($a$) in paragraph (1)($e$), omit “or” after paragraph (iii)  and after paragraph (iv)  insert—
\begin{quotation}
“or

(v)  universal credit under Part~I of the Welfare Reform Act 2012;”;
\end{quotation}

($b$) in paragraph (2)($b$)(ii), omit “contribution-based”; and

($c$) for paragraph (5), substitute—
\begin{quotation}
“(5) In this regulation—
\begin{enumerate}\item[]
($a$) in paragraph (1), “couple” has the meaning—
\begin{enumerate}\item[]
(i) apart from in relation to universal credit, given by section 137(1) of the Social Security Contributions and Benefits Act 1992 (interpretation of Part VII and supplementary provisions);

(ii) in relation to universal credit, given by section~39 of the Welfare Reform Act 2012 (couples);
\end{enumerate}

($b$) in paragraph (2), “jobseeker’s allowance” means an allowance under the Jobseekers Act 1995 as amended by the provisions of Part~I of Schedule 14 to the Welfare Reform Act 2012 that remove references to an income-based allowance, and a contribution-based allowance under the Jobseekers Act 1995 as that Act has effect apart from those provisions; and

($c$) “universal credit” means universal credit under Part~I of the Welfare Reform Act 2012.”.
\end{enumerate}
\end{quotation}
\end{enumerate}

\subsubsection[27. Amendment of the Social Security (General Benefit) Regulations 1982]{Amendment of the Social Security (General Benefit) Regulations 1982}

27.  In regulation 9 of the Social Security (General Benefit) Regulations 1982 (payments of benefit and suspension of payments pending a decision on appeals or references, arrears and repayments)\footnote{S.I.~1982/1408. Paragraphs (5A) to (5D) of regulation 9 were inserted by S.I.~1996/2538.}—
\begin{enumerate}\item[]
\begin{sloppypar}
($a$) in paragraph (5A), in sub-paragraphs ($a$), ($b$)  and ($c$), omit “contribution-based” in each place where it occurs;
\end{sloppypar}

($b$) in paragraph (5B), in sub-paragraphs ($a$)  and ($c$), omit “contribution-based” in each place where it occurs;

($c$) in paragraphs (5C) and (5D), omit “contribution-based” in each place where it occurs; and

($d$) after paragraph (5D), insert—
\begin{quotation}
“(5E) In this regulation, “jobseeker’s allowance” means an allowance under the Jobseekers Act 1995 as amended by the provisions of Part~I of Schedule 14 to the Welfare Reform Act 2012 that remove references to an income-based allowance, and a contribution-based allowance under the Jobseekers Act 1995 as that Act has effect apart from those provisions.”.
\end{quotation}
\end{enumerate}

\subsubsection[28. Amendment of the Income Support (General) Regulations 1987]{Amendment of the Income Support (General) Regulations 1987}

28.—(1) The Income Support (General) Regulations 1987\footnote{S.I.~1987/1967.} are amended as follows.

(2) In regulation 2(1) (interpretation)\footnote{Regulation 2(1) was amended by S.I.~2008/1554 and S.I.~2013/388; there are other amendments that are not relevant to these Regulations.}—
\begin{enumerate}\item[]
($a$) after the definition of “the Contributions and Benefits Act” insert—
\begin{quotation}
““contribution-based jobseeker’s allowance” means an allowance under the Jobseekers Act 1995 as amended by the provisions of Part~I of Schedule 14 to the 2012 Act that remove references to an income-based allowance, and a contribution-based allowance under the Jobseekers Act 1995 as that Act has effect apart from those provisions;”;
\end{quotation}

($b$) for the definition of “contributory employment and support allowance” substitute—
\begin{quotation}
““contributory employment and support allowance” means an allowance under Part~I of the Welfare Reform Act as amended by the provisions of Schedule 3, and Part~I of Schedule 14, to the 2012 Act that remove references to an income-related allowance, and a contributory allowance under Part~I of the Welfare Reform Act as that Part has effect apart from those provisions;”; and
\end{quotation}

($c$) after the definition of “training allowance”, insert—
\begin{quotation}
““universal credit” means universal credit under Part~I of the 2012 Act;”.
\end{quotation}
\end{enumerate}

(3) In regulation 14(2) (persons of a prescribed description)\footnote{Regulation 14 was amended by S.I.~2001/3070, 2006/718 and 2008/1554.}—
\begin{enumerate}\item[]
($a$) after sub-paragraph ($c$), omit “or”; and

($b$) after sub-paragraph ($d$), insert—
\begin{quotation}
“; or

($e$) entitled to universal credit”.
\end{quotation}
\end{enumerate}

(4) In regulation 31(2) (date on which income is treated as paid)—
\begin{enumerate}\item[]
($a$) for “or employment and support allowance” substitute “,~employment and support allowance or universal credit”; and

($b$) for “on the day of the benefit week” substitute “on any day”.
\end{enumerate}

(5) In regulation 40(6)($c$)  (calculation of income other than earnings)\footnote{Regulation 40(6) was inserted by S.I.~2008/1554.}, after “Employment and Support Allowance Regulations” insert “or section~11J of the Welfare Reform Act\footnote{Section 11J was inserted by section 57 of the Welfare Reform Act 2012 (c.~5).} as the case may be”.

(6) In regulation 75 (modifications in the calculation of income)\footnote{Regulation 75($b$) was amended by S.I.~1988/1445, 1995/482, 1996/206 and 2008/1554.}, in paragraph ($b$), for “or employment and support allowance” substitute “,~employment and support allowance or universal credit”.

(7) In Schedule 3 (housing costs)\footnote{In Schedule 3, paragraph 1(3) was amended by S.I.~1995/2927, 2006/2378, 2008/1554 and 2012/913; paragraph 18(7) was amended by S.I.~2004/2327 and 2008/1554. There are other amendments to paragraph 18(7) that are not relevant to these Regulations.}—
\begin{enumerate}\item[]
($a$) in paragraph 1, after sub-paragraph (3)($d$)  insert—
\begin{quotation}
“; or

($e$) who is entitled to an award of universal credit the calculation of which includes an amount under regulation~27(1) of the Universal Credit Regulations 2013 in respect of the fact that he has limited capability for work or limited capability for work and work-related activity, or would include such an amount but for regulation 27(4) or 29(4) of those Regulations”; and
\end{quotation}

($b$) in paragraph 18—
\begin{enumerate}\item[]
(i) after sub-paragraph (7)($i$)  insert—
\begin{quotation}
\looseness=1
“($j$) if he is aged less than 25 and is entitled to an award of universal credit where the award is calculated on the basis that he does not have any earned income”; and
\end{quotation}

(ii) after sub-paragraph (8) insert—
\begin{quotation}
“(9) For the purposes of sub-paragraph (7)($j$), “earned income” has the meaning given in regulation 52 of the Universal Credit Regulations 2013\footnote{S.I.~2013/376.}.”.
\end{quotation}
\end{enumerate}
\end{enumerate}

(8) In Schedule 9 (sums to be disregarded in the calculation of income other than earnings), in paragraph 7\footnote{Schedule 9 paragraph 7 was amended by S.I.~2008/3157; there are other amendments not relevant to these Regulations.}—
\begin{enumerate}\item[]
($a$) omit “or” after sub-paragraph ($c$); and

($b$) after sub-paragraph ($d$), insert—
\begin{quotation}
“; or

($e$) universal credit”.
\end{quotation}
\end{enumerate}

(9) In Schedule 10 (capital to be disregarded)\footnote{Paragraph 7(1) of Schedule 10 was amended by S.I.~1991/2742, 2002/2380, 2008/698 and 2008/1554; paragraph 7(3) was amended by S.I.~2008/1554. There are other amendments that are not relevant to these Regulations.}, in paragraph 7—
\begin{enumerate}\item[]
($a$) in sub-paragraph (1), after paragraph ($d$), insert—
\begin{quotation}
“;

($e$) universal credit”;
\end{quotation}

($b$) in sub-paragraph (3)—
\begin{enumerate}\item[]
(i) for “or of an income-based jobseeker’s allowance”, in both places where it occurs, substitute “,~an income-based jobseeker’s allowance or universal credit”;

(ii) in paragraph ($a$), for “either of” substitute “of any of”;

(iii) in paragraph ($b$), for “either” substitute “any”; and

(iv) after paragraph ($b$)(iii), insert—
\begin{quotation}
“,~or

(iv) in a case where universal credit is awarded to the claimant and another person as joint claimants, either the claimant or the other person, or both of them, received the relevant sum”.
\end{quotation}
\end{enumerate}
\end{enumerate}

\subsubsection[29. Amendment of the Social Security (Claims and Payments) Regulations 1987]{Amendment of the Social Security (Claims and Payments) Regulations 1987}

29.—(1) The Social Security (Claims and Payments) Regulations 1987\footnote{S.I.~1987/1968.} are amended as follows.

(2) In regulation 2(1) (interpretation)\footnote{Regulation 2(1) has been amended in ways not relevant to these Regulations.}, after the definition of “State Pension Credit Regulations” insert—
\begin{quotation}
““universal credit” means universal credit under Part~I of the Welfare Reform Act 2012;”.
\end{quotation}

(3) In regulation 16A(2) (date of entitlement under an award of state pension credit for the purpose of payability and effective date of change of rate)\footnote{Regulation 16A was inserted by S.I.~2002/3019 and paragraph (2) of that regulation was amended by S.I.~2008/1554.}—
\begin{enumerate}\item[]
($a$) in sub-paragraph ($a$), after “income support,” insert “universal credit,”; and

($b$) in sub-paragraph ($b$), after income-based jobseeker’s allowance” insert “or universal credit”.
\end{enumerate}

\subsubsection[30. Amendment of the Jobseeker’s Allowance Regulations 1996]{Amendment of the Jobseeker’s Allowance Regulations 1996}

30.—(1) The Jobseeker’s Allowance Regulations 1996\footnote{S.I.~1996/207; regulation 1(3) was amended by S.I.~2013/388.} are amended as follows.

(2) In regulation 1 (citation, commencement and interpretation)\footnote{Regulation 1(3) was amended by S.I.~2013/388; there are other amendments that are not relevant to these Regulations.}—
\begin{enumerate}\item[]
($a$) for the heading substitute “Citation, commencement, interpretation and application”;

($b$) after paragraph (2) insert—
\begin{quotation}
“(2A) These Regulations do not apply to a particular case on any day on which section 33(1)($a$)  of the 2012 Act (abolition of income-based jobseeker’s allowance) is in force and applies in relation to that case.”; and
\end{quotation}

($c$) in paragraph (3)—
\begin{enumerate}\item[]
(i) after the definition of “the Contributions Regulations” insert—
\begin{quotation}
““contributory employment and support allowance” means an allowance under Part~I of the Welfare Reform Act as amended by the provisions of Schedule 3, and Part~I of Schedule 14, to the 2012 Act that remove references to an income-related allowance, and a contributory allowance under Part~I of the Welfare Reform Act as that Part has effect apart from those provisions;”; and
\end{quotation}

(ii) after the definition of “training allowance” insert—
\begin{quotation}
““universal credit” means universal credit under Part~I of the 2012 Act;”.
\end{quotation}
\end{enumerate}
\end{enumerate}

(3) In regulation 76(2) (persons of a prescribed description)\footnote{Regulation 76(2) was amended by S.I.~2001/3070, 2006/718 and 2008/1554.}, after sub-paragraph ($e$)  insert—
\begin{quotation}
“; or

($f$) entitled to universal credit”.
\end{quotation}

(4) In regulation 96(2) (date on which income is treated as paid)\footnote{Regulation 96(2) was amended by S.I.~2008/1554.}—
\begin{enumerate}\item[]
($a$) for “or employment and support allowance” substitute “,~employment and support allowance or universal credit”; and

($b$) for “on the day of the benefit week” substitute “on any day”.
\end{enumerate}

(5) In regulation 103(5B) (calculation of income other than earnings)\footnote{Paragraph (5B) was inserted by S.I.~2008/1554.}, after “Employment and Support Allowance Regulations” insert “or section~11J of the Welfare Reform Act as the case may be”.

(6) In regulation 153($b$)  (modification in the calculation of income)\footnote{Regulation 153(b) was amended by S.I.~2008/1554.}, after “the Benefits Act” insert “,~universal credit”.

(7) In Schedule 2 (housing costs)\footnote{Paragraph 1(3) was substituted by S.I.~2012/913. Paragraph 17(7) was amended by S.I.~1997/827 and 2008/1554; there are other amendments that are not relevant to these Regulations.}—
\begin{enumerate}\item[]
($a$) in paragraph 1(3), after paragraph ($e$)  insert—
\begin{quotation}
“; or

($f$) who is entitled to an award of universal credit the calculation of which includes an amount under regulation~27(1) of the Universal Credit Regulations 2013 in respect of the fact that he has limited capability for work or limited capability for work and work-related activity, or would include such an amount but for regulation 27(4) or 29(4) of those Regulations”; and
\end{quotation}

($b$) in paragraph 17—
\begin{enumerate}\item[]
(i) after sub-paragraph (7)($i$)  insert—
\begin{quotation}
“; or

($j$) if he is aged less than 25 and is entitled to an award of universal credit which is calculated on the basis that he does not have any earned income”; and
\end{quotation}

(ii) after sub-paragraph (8) insert—
\begin{quotation}
“(9) For the purposes of sub-paragraph (7)($j$), “earned income” has the meaning given in regulation 52 of the Universal Credit Regulations 2013.”.
\end{quotation}
\end{enumerate}
\end{enumerate}

(8) In Schedule 7 (sums to be disregarded in the calculation of income other than earnings)\footnote{Paragraph 8 was amended by S.I.~2008/3157.}, omit “or” after paragraph 8($c$)  and after paragraph~8($d$)  insert—
\begin{quotation}
“; or

($e$) universal credit”.
\end{quotation}

(9) In Schedule 8, in paragraph 12(1)($b$) (capital to be disregarded)\footnote{Paragraph 12(1)($b$) was amended by S.I.~2003/455, 2005/574, 2008/698, 2008/1554 and 2008/3157.}, after “working tax credit” insert “,~universal credit”.

\subsubsection[31. Amendment of the Social Security (Immigration and Asylum) Consequential Amendments Regulations 2000]{Amendment of the Social Security (Immigration and Asylum) Consequential Amendments Regulations 2000}

31.—(1) The Social Security (Immigration and Asylum) Consequential Amendments Regulations 2000\footnote{S.I.~2000/636.} are amended as follows.

(2) In regulation 1(3) (interpretation)\footnote{Regulation 1(3) has been amended in ways that are not relevant to these Regulations.}, after the definition of “income-related employment and support allowance” insert—
\begin{quotation}
“;

“universal credit” means universal credit under Part~I of the Welfare Reform Act 2012”.
\end{quotation}

(3) In regulation 2 (persons not excluded from specified benefits under section 115 of the Immigration and Asylum Act 1999)\footnote{Regulation 2 has been amended in ways that are not relevant to these Regulations.}—
\begin{enumerate}\item[]
($a$) after paragraph (1) insert—
\begin{quotation}
“(1A) For the purposes of entitlement to universal credit, a person falling within a category or description of persons specified in paragraphs 2, 3 and 4 of Part I of the Schedule is a person to whom section 115 of the Act does not apply.”; and
\end{quotation}

($b$) in paragraph (5)—
\begin{enumerate}\item[]
(i) after “entitlement to” insert “universal credit,”; and

(ii) for “a jobseeker’s allowance” substitute “an income-based jobseeker’s allowance under the Jobseekers Act 1995”; and

(iii) before “employment and support allowance” insert “income-related”.
\end{enumerate}
\end{enumerate}

(4) In the heading to Part I of the Schedule (persons not excluded from certain benefits under section 115 of the Immigration and Asylum Act 1999)\footnote{Part I of the Schedule was amended by S.I.~2008/1554.}, after “entitlement to” insert “universal credit,”.

\subsubsection[32. Amendment of the Occupational and Personal Pension Schemes (Bankruptcy) (No.~2) Regulations 2002]{Amendment of the Occupational and Personal Pension Schemes (Bankruptcy) (No.~2) Regulations 2002}

32.—(1) The Occupational and Personal Pension Schemes (Bankruptcy) (No.~2) Regulations 2002\footnote{S.I.~2002/836.} are amended as follows.

(2) In regulation 5(3)($b$)  (exclusion orders: England and Wales), after “an income-related benefit” insert “or universal credit under Part~I of the Welfare Reform Act 2012”.

(3) In regulation 14(3)($b$)  (exclusion orders: Scotland), after “an income-related benefit” insert “or universal credit under Part~I of the Welfare Reform Act 2012”.

\subsubsection[33. Amendment of the State Pension Credit Regulations 2002]{Amendment of the State Pension Credit Regulations 2002}

33.—(1) The State Pension Credit Regulations 2002\footnote{S.I.~2002/1792.} are amended as follows.

(2) In regulation 1(2) (interpretation)\footnote{Regulation 1(2) was amended by S.I.~2013/388; there are other amendments that are not relevant to these Regulations.}—
\begin{enumerate}\item[]
($a$) after the definition of “close relative”, insert—
\begin{quotation}
““contribution-based jobseeker’s allowance” means an allowance under the Jobseekers Act 1995 as amended by the provisions of Part~I of Schedule 14 to the 2012 Act that remove references to an income-based allowance, and a contribution-based allowance under the Jobseekers Act 1995 as that Act has effect apart from those provisions;”;
\end{quotation}

($b$) after the definition of “the Health Service (Wales) Act”, insert—
\begin{quotation}
““income-based jobseeker’s allowance” means an income-based allowance under the Jobseekers 
Act 1995;”;
\end{quotation}

($c$) after the definition of “the Skipton Fund”, insert—
\begin{quotation}
““universal credit” means universal credit under Part~I of the 2012 Act;”; and
\end{quotation}

($d$) for the definition of “contributory employment and support allowance” substitute—
\begin{quotation}
““contributory employment and support allowance” means an allowance under Part~I of the Welfare Reform Act as amended by the provisions of Schedule 3, and Part~I of Schedule 14, to the 2012 Act that remove references to an income-related allowance, and a contributory allowance under Part~I of the Welfare Reform Act as that Part has effect apart from those provisions;”.
\end{quotation}
\end{enumerate}

(3) In regulation 9 (qualifying income for the purposes of savings credits), in paragraph ($c$), omit “within the meaning of section 1(4) of the Jobseekers Act 1995”.

(4) In regulation 13A(1)($a$)  (part-weeks)\footnote{Regulation 13A was inserted by S.I.~2002/3019.} before “income support” insert “universal credit,”.

(5) In regulation 13B(1)($d$)  (date on which benefits are treated as paid)\footnote{Regulation 13B was inserted by S.I.~2002/3019 and amended by S.I.~2008/1554.}, omit “within the meaning of section 1(4) of the Jobseekers Act 1995”.

(6) In Schedule 2 (housing costs)\footnote{Paragraph 1(2)($a$)(iii) was amended by S.I.~2002/3197, 2005/3360, 2006/718, 2008/1554, 2013/388 and 2013/591.}—
\begin{enumerate}\item[]
($a$) in paragraph 1(2)($a$)(iii), after paragraph ($ee$)  insert—
\begin{quotation}
“; or

($ff$)  is entitled to an award of universal credit the calculation of which includes an amount under regulation 27(1) of the Universal Credit Regulations 2013 in respect of the fact that he has limited capability for work or limited capability for work and work-related activity, or would include such an amount but for regulation 27(4) or 29(4) of those Regulations”; and
\end{quotation}

($b$) in paragraph 14 (persons residing with the claimant)\footnote{Paragraph 14(7) was amended by S.I.~2002/3197, 2003/1195, 2004/2327, 2005/3360, 2006/2378, 2008/1554 and 2008/2767.}—
\begin{enumerate}\item[]
(i) after sub-paragraph (7)($g$), insert—
\begin{quotation}
“; or

($h$) if he is aged less than 25 and is entitled to an award of universal credit which is calculated on the basis that he does not have any earned income”; and
\end{quotation}

(ii) after sub-paragraph (8), insert—
\begin{quotation}
“(9) For the purposes of sub-paragraph (7)($h$), “earned income” has the meaning given in regulation 52 of the Universal Credit Regulations 2013.”.
\end{quotation}
\end{enumerate}
\end{enumerate}

(7) In Part~I of Schedule 5 (capital disregarded for the purpose of calculating income)\footnote{Paragraph 20(2) was amended by S.I.~2002\slash 3019, 2003/2774, 2008/1554 and 2008/3157. Paragraph 20A(2) was amended by S.I.~2002/2380, 2008/1554 and 2008/3157.}—
\begin{enumerate}\item[]
($a$) in paragraph 20(2), after paragraph ($o$)  insert—
\begin{quotation}
“;

($p$) universal credit”; and
\end{quotation}

($b$) in paragraph 20A(2), after paragraph ($h$)  omit “or” and after paragraph ($i$)  insert—
\begin{quotation}
“or 

($j$)  paragraph 17 of Schedule 10 to the Universal Credit Regulations 2013;”.
\end{quotation}
\end{enumerate}

\subsubsection[34. Amendment of the Social Security (Deferral of Retirement Pensions) Regulations 2005]{Amendment of the Social Security (Deferral of Retirement Pensions) Regulations 2005}

34.  In regulation 3 of the Social Security (Deferral of Retirement Pensions) Regulations 2005 (amount of retirement pension not included in the calculation of the lump sum)\footnote{S.I.~2005/453. Paragraph (1)($aa$) of regulation 3 was amended by S.I.~2011/634; paragraph (5A) was inserted by S.I.~2011/634.}—
\begin{enumerate}\item[]
($a$) in paragraph (1)($aa$), at the end of paragraph (iii)  omit “or” and after paragraph (iv)  insert—
\begin{quotation}
“or

(v) universal credit under Part~I of the Welfare Reform Act 2012;”; and
\end{quotation}

($b$) for paragraph (5A) substitute—
\begin{quotation}
“(5A) In paragraph (1), ``couple'' has the meaning—
\begin{enumerate}\item[]
($a$) in relation to universal credit, given by section 39 of the Welfare Reform Act 2012; and

($b$) in relation to the other benefits referred to in paragraph (1)($a$)  or ($aa$), given by section 137(1) of the Act.”.
\end{enumerate}
\end{quotation}
\end{enumerate}

\subsubsection[35. Amendment of the Housing Benefit Regulations 2006]{Amendment of the Housing Benefit Regulations 2006}

35.—(1) The Housing Benefit Regulations 2006\footnote{S.I.~2006/213.} are amended as follows.

(2) In regulation 2(1) (interpretation)\footnote{Regulation 2(1) was amended by S.I.~2013/388; there are other amendments that are not relevant to these Regulations.}—
\begin{enumerate}\item[]
($a$) for the definition of “contributory employment and support allowance”, substitute—
\begin{quotation}
““contributory employment and support allowance” means an allowance under Part~I of the Welfare Reform Act as amended by the provisions of Schedule 3, and Part~I of Schedule 14, to the 2012 Act that remove references to an income-related allowance, and a contributory allowance under Part~I of the Welfare Reform Act as that Part has effect apart from those provisions;”; and
\end{quotation}

($b$) after the definition of “training allowance”, insert—
\begin{quotation}
““universal credit” means universal credit under Part~I of the 2012 Act;”.
\end{quotation}
\end{enumerate}

(3) In regulation 19 (persons of a prescribed description)\footnote{Regulation 19(2) was amended by S.I.~2006/718 and 2008/1082.}, omit “or” after paragraph (2)($a$)  and after paragraph (2)($c$)  insert—
\begin{quotation}
“; or

($d$) entitled to an award of universal credit”.
\end{quotation}

(4) In regulation 28(11) (treatment of child care charges)\footnote{Regulation 28(11)($ba$)  and ($ca$)  were inserted by S.I.~2008/1082.}, in sub-\hspace{0pt}paragraphs ($ba$)  and ($ca$), after “Employment and Support Allowance Regulations” insert “or the Employment and Support Allowance Regulations 2013\footnote{S.I.~2013/379.}”.

(5) In regulation 40(5A) (calculation of income other than earnings)\footnote{Paragraph (5A) was inserted by S.I.~2008/1082; section 11J was inserted by section 57(2) of the Welfare Reform Act 2012 (c.~5).}, after “Employment and Support Allowance Regulations” insert “or section~11J of the Welfare Reform Act\footnote{Section 11J was inserted by 57(2) of the Welfare Reform Act 2012 (c.~5).}”.

(6) In regulation 56(2)($ea$)  (full-time students to be treated as not liable to make payments)\footnote{Sub-paragraph ($ea$)  was inserted by S.I.~2008/1082.}, after “Employment and Support Allowance Regulations” insert “or the Employment and Support Allowance Regulations 2013”.

(7) In regulation 74 (non-dependant deductions)—
\begin{enumerate}\item[]
($a$) for paragraph (8), substitute—
\begin{quotation}
“(8) No deduction shall be made in calculating the amount of a rent rebate or allowance in respect of a non-dependant aged less than 25 who is—
\begin{enumerate}\item[]
($a$) on income support, an income-based jobseeker’s allowance or an income-related employment and support allowance which does not include an amount under section 4(2)($b$)  of the Welfare Reform Act (the support component and the work-related activity component); or

($b$) entitled to an award of universal credit where the award is calculated on the basis that the non-dependant does not have any earned income.”; and
\end{enumerate}
\end{quotation}

($b$) after paragraph (10) insert—
\begin{quotation}
“(11) For the purposes of paragraph (8), “earned income” has the meaning given in regulation 52 of the Universal Credit Regulations 2013.”.
\end{quotation}
\end{enumerate}

(8) In regulation 102(4) (method of recovery)\footnote{Paragraph (4) was amended by S.I.~2009/2608.}, after “Employment and Support Allowance Regulations” insert “,~regulation 39(1)($a$)  of the Employment and Support Allowance Regulations 2013”.

(9) In Schedule 3 (applicable amounts)—
\begin{enumerate}\item[]
($a$) in Part~V (the components), in paragraph 21($c$)(ii), after “Employment and Support Allowance Regulations” insert “or regulation 7 of the Employment and Support Allowance Regulations 2013”; and

($b$) in Part~VII (transitional addition)—
\begin{enumerate}\item[]
(i) in paragraph 27(1)($b$)(i), after “Employment and Support Allowance Regulations” insert “or regulation 26 of the Employment and Support Allowance Regulations 2013, in either case”; and

(ii) in paragraph 29(1)($c$), after “Employment and Support Allowance Regulations” insert “or regulation 86 of the Employment and Support Allowance Regulations 2013”.
\end{enumerate}
\end{enumerate}

(10) In Schedule 4 (sums to be disregarded in the calculation of earnings)\footnote{Paragraph 10A was inserted by S.I.~2009/2608.}, paragraph 10A(6)($a$), after “Employment and Support Allowance Regulations” insert “or regulation 39(1)($a$), ($b$)  or ($c$)  of the Employment and Support Allowance Regulations 2013”.

(11) In Schedule 5 (sums to be disregarded in the calculation of income other than earnings)\footnote{Paragraph 7 was amended by S.I.~2005/2502, 2008/698 and 2008/1082.}, in paragraph 7, after sub-paragraph ($d$)  insert—
\begin{quotation}
“($e$) universal credit”.
\end{quotation}

(12) In Schedule 6 (capital to be disregarded), in paragraph 9(1), after paragraph ($f$)  insert—
\begin{quotation}
“($g$) universal credit,”.
\end{quotation}

\subsubsection[36. Amendment of the Housing Benefit (Persons who have attained the qualifying age for state pension credit) Regulations 2006]{Amendment of the Housing Benefit (Persons who have attained the qualifying age for state pension credit) Regulations 2006}

36.—(1) The Housing Benefit (Persons who have attained the qualifying age for state pension credit) Regulations 2006\footnote{S.I.~2006/214; regulation 2 was amended by S.I.~2013/388 and paragraph 21(2) was amended by S.I.~2013/443, 2013/388 and 2013/591.} are amended as follows.

(2) In regulation 2 (interpretation)\footnote{Regulation 2 was amended by S.I.~2013/388; there are other amendments that are not relevant to these Regulations.}—
\begin{enumerate}\item[]
($a$) for the definition of “contributory employment and support allowance”, substitute—
\begin{quotation}
““contributory employment and support allowance” means an allowance under Part~I of the Welfare Reform Act as amended by the provisions of Schedule 3, and Part~I of Schedule 14, to the 2012 Act that remove references to an income-related allowance, and a contributory allowance under Part~I of the Welfare Reform Act as that Part has effect apart from those provisions;”; and
\end{quotation}

($b$) after the definition of “training allowance”, insert—
\begin{quotation}
““universal credit” means universal credit under Part I of the 2012 Act;”.
\end{quotation}
\end{enumerate}

(3) In regulation 19 (persons of a prescribed description)\footnote{Regulation 19(2) was amended by S.I.~2006/718 and 2008/1082.}, omit “or” after paragraph (2)($a$)  and after paragraph (2)($c$), insert—
\begin{quotation}
“; or

($d$) entitled to an award of universal credit”.
\end{quotation}

\begin{sloppypar}\looseness=-1
(4) In regulation 31(11) (treatment of child care charges)\footnote{Regulation 31(11)($ba$) and ($ca$) were inserted by S.I.~2008/1082.}, in sub-paragraphs~($ba$)  and ($ca$), after “Employment and Support Allowance Regulations” insert “or the Employment and Support Allowance Regulations 2013”.
\end{sloppypar}

(5) In regulation 55 (non-dependant deductions)\footnote{Paragraph (8) was amended by S.I.~2008/1082.}—
\begin{enumerate}\item[]
($a$) in paragraph (8), after “work-related activity component)” insert “or who is entitled to an award of universal credit where the award is calculated on the basis that the person does not have any earned income”; and

($b$) after paragraph (10) insert—
\begin{quotation}
“(11) For the purposes of paragraph (8), “earned income” has the meaning given in regulation 52 of the Universal Credit Regulations 2013.”.
\end{quotation}
\end{enumerate}

(6) In regulation 83(4) (method of recovery)\footnote{Paragraph (4) was amended by S.I.~2008/1042 and 2009/2608.}, after “Employment and Support Allowance Regulations” insert “,~regulation 39(1)($a$)  of the Employment and Support Allowance Regulations 2013”.

(7) In Schedule 4 (sums disregarded from claimant’s earnings)\footnote{Paragraph 5(1)($d$) was added by S.I.~2009/583.}—
\begin{enumerate}\item[]
($a$) in paragraph 5(1)($d$)(ii), after “Employment and Support Allowance Regulations” insert “or regulation 7 of the Employment and Support Allowance Regulations 2013”; and

($b$) in paragraph 5A(6)($a$), after “Employment and Support Allowance Regulations” insert “or regulation 39(1)($a$), ($b$)  or ($c$)  of the Employment and Support Allowance Regulations 2013”.
\end{enumerate}

(8) In Schedule 6 (capital to be disregarded)\footnote{Paragraph 21(2) was amended by S.I.~2005/2502, 2008/1082, 2013/388, 2013/443 and 2013/591.}, in paragraph 21(2), omit “or” after sub-paragraph ($m$)  and after sub-paragraph ($n$)  insert—
\begin{quotation}
“; or

($o$) universal credit”.
\end{quotation}

\subsubsection[37. Amendment of the Employment and Support Allowance Regulations 2008]{Amendment of the Employment and Support Allowance Regulations 2008}

37.—(1) The Employment and Support Allowance Regulations 2008\footnote{S.I.~2008/794; regulation 2(1) was amended by S.I.~2013/388.} are amended as follows.

(2) In regulation 1—
\begin{enumerate}\item[]
($a$) for the heading substitute “Citation, commencement and application”; and

($b$) at the beginning, insert “(1)” and after paragraph (1) insert—
\begin{quotation}
“(2) These Regulations do not apply to a particular case on any day on which section 33(1)($b$)  of the 2012 Act (abolition of income-related employment and support allowance) is in force and applies in relation to that case.”.
\end{quotation}
\end{enumerate}

(3) In regulation 2(1) (interpretation)\footnote{Regulation 2(1) was amended by S.I.~2013/388; there are other amendments that are not relevant to these Regulations.}—
\begin{enumerate}\item[]
($a$) after the definition of “confinement” insert—
\begin{quotation}
““contribution-based jobseeker’s allowance” means an allowance under the Jobseekers Act as amended by the provisions of Part~I of Schedule 14 to the 2012 Act that remove references to an income-based allowance, and a contribution-based allowance under the Jobseekers Act as that Act has effect apart from those provisions;”;
\end{quotation}

($b$) after the definition of “New Deal Options”, insert—
\begin{quotation}
““new style ESA” means an allowance under Part~I of the Act as amended by the provisions of Schedule 3, and Part~I of Schedule 14, to the 2012 Act that remove references to an income-related allowance;”; and
\end{quotation}

($c$) after the definition of “training allowance” insert—
\begin{quotation}
““universal credit” means universal credit under Part~I of the 2012 Act;”.
\end{quotation}
\end{enumerate}

\begin{sloppypar}
(4) In regulation 63(5) (reduction of employment and support allowance)\footnote{Section 11J was inserted by the Welfare Reform Act 2012 (c.~5), section 57.}—
\end{sloppypar}
\begin{enumerate}\item[]
($a$) after “contributory allowance” insert “including new style ESA”; and

($b$) after “this regulation”, insert “or section 11J of the Act respectively”.
\end{enumerate}

(5) In regulation 93(2) (date on which income is treated as paid)—
\begin{enumerate}\item[]
($a$) for “or severe disablement allowance” substitute “severe disablement allowance or universal credit”; and

($b$) for “the day of the benefit week” substitute “on any day”.
\end{enumerate}

(6) In regulation 167($b$)  (modification in the calculation of income), after “income support” insert “,~universal credit”.

(7) In Schedule 6 (housing costs)—
\begin{enumerate}\item[]
($a$) in paragraph 1(3) (housing costs---meaning of disabled person)\footnote{Paragraph 1(3) was amended by S.I.~2012/913.}, omit “or” after paragraph ($c$)  and, after paragraph ($d$), insert—
\begin{quotation}
“; or

($e$) who is entitled to an award of universal credit the calculation of which includes an amount under regulation~27(1) of the Universal Credit Regulations 2013 in respect of the fact that that person has limited capability for work or limited capability for work and work-related activity, or would include such an amount but for regulation 27(4) or~29(4) of those Regulations;”; and
\end{quotation}

($b$) in paragraph 19 (non-dependant deductions)\footnote{Paragraph 19(7) was amended by S.I.~2008/2428.}—
\begin{enumerate}\item[]
(i) omit “or” after sub-paragraph (7)($g$)  and, after sub-paragraph (7)($h$), insert—
\begin{quotation}
“or

($ha$) if the non-dependant is aged less than 25 and is entitled to an award of universal credit which is calculated on the basis that the non-dependant does not have any earned income;”; and
\end{quotation}

(ii) after sub-paragraph (7) insert—
\begin{quotation}
\begin{sloppypar}
“(7A) For the purposes of sub-paragraph (7)($ha$), “earned income” has the meaning given in regulation 52 of the Universal Credit Regulations 2013.”.
\end{sloppypar}
\end{quotation}
\end{enumerate}
\end{enumerate}

(8) In Schedule 8, paragraph 9($b$)  (sums to be disregarded in the calculation of income other than earnings)\footnote{Paragraph 9($b$) was amended by S.I.~2008/2428.}, after “income support” insert “,~universal credit”.

(9) In Schedule 9 paragraph 11 (capital to be disregarded)—
\begin{enumerate}\item[]
($a$) in sub-paragraph (1)($b$), after “income-based jobseeker’s allowance” insert “,~universal credit”;

($b$) in sub-paragraph (3)($a$)  and ($b$), after “income support” insert “,~universal credit”; and

($c$) in sub-paragraph (3), omit “or” after paragraph ($b$)(ii)  and, after paragraph ($b$)(iii), insert—
\begin{quotation}
“; or

(iv) in a case where universal credit is awarded to the claimant and another person as joint claimants, either the claimant or the other person, or both of them, received the relevant sum”.
\end{quotation}
\end{enumerate}

\subsubsection[38. Amendment of the Universal Credit Regulations 2013]{Amendment of the Universal Credit Regulations 2013}

38.—(1) The Universal Credit Regulations 2013\footnote{S.I.~2013/376.} are amended as follows—

(2) In regulation 3 (couples)—
\begin{enumerate}\item[]
($a$) in paragraph (1) after “whose partner does not meet all the basic conditions” insert “or is otherwise excluded from entitlement to universal credit”; and

($b$) in paragraph (3) at the end of sub-paragraph ($c$)  omit “or” and at the end of sub-paragraph ($d$)  insert—
\begin{quotation}
“; or

($e$) is a person to whom section 115 of the Immigration and Asylum Act 1999\footnote{1999 c.~33.} (exclusion from benefits) applies,”.
\end{quotation}
\end{enumerate}

(3) In regulation 13(1) (meaning of “undertaking a course”) after “education” insert “,~study”.

(4) In regulation 19(2)($a$)  (restrictions on entitlement---prisoners etc.)\ after “universal credit” insert “as a single person”.

(5) In regulation 68(1) (person treated as having a student loan), in the first line, after “course” insert “of education, study or training”.

(6) In regulation 83(1)($h$)  (exceptions from the benefit cap) for “sub-paragraph ($b$), ($c$), ($d$)  or ($e$)” substitute “sub-paragraphs ($b$)  to ($g$)”.

(7) In regulation 111(4) (daily reduction rate) for “paragraphs (1) to (3)” substitute “paragraphs (1) and (2)”.

(8) For regulation 114 (sanctionable failures under section 26---work placement) substitute—
\begin{quotation}
\subsubsection*{\sloppy “Sanctionable failures under section 26---work placements}

114.—(1) A placement on the Mandatory Work Activity Scheme is a prescribed placement for the purpose of section 26(2)($a$)  of the Act (sanctionable failure not to comply with a work placement).

(2) In paragraph (1) “the Mandatory Work Activity Scheme” means a scheme provided pursuant to arrangements made by the Secretary of State and known by that name that is designed to provide work or work-related activity for up to 30 hours per week over a period of 4 consecutive weeks with a view to assisting claimants to improve their prospects of obtaining employment.”.
\end{quotation}

(9) For regulation 117 substitute—
\begin{quotation}
\subsubsection*{“The period of hardship payments}

117.—(1) A hardship payment is to be made in respect of a period which—
\begin{enumerate}\item[]
($a$) begins with the date on which all the conditions in regulation 116(1) are met; and

($b$) unless paragraph (2) applies, ends with the day before the normal payment date for the assessment period in which those conditions are met.
\end{enumerate}

(2) If the period calculated in accordance with paragraph (1) would be 7 days or less, it does not end on the date referred to in paragraph (1)($b$)  but instead ends on the normal payment date for the following assessment period or, if earlier, the last day on which the award is to be reduced under section 26 or 27 of the Act or under section 6B(5A), 7(2A) or 9(2A) of the Social Security Fraud Act 2001.

(3) In this regulation “the normal payment date” for an assessment period is the date on which the Secretary of State would normally expect to make a regular payment of universal credit in respect of an assessment period in a case where payments of universal credit are made monthly in arrears.”.
\end{quotation}

(10) In paragraph 2(1)($a$)  of Schedule 8 for “an in-patient” substitute “a patient”.

\subsubsection[39. Amendment of the Jobseeker’s Allowance Regulations 2013]{Amendment of the Jobseeker’s Allowance Regulations 2013}

39.  For paragraphs (1) and (2) of regulation 29 of the Jobseeker’s Allowance Regulations 2013\footnote{S.I.~2013/378.} (sanctionable failures under section 6J of the Act: work placements) substitute---
\begin{quotation}
\begin{sloppypar}
“29.—(1) A placement on the Mandatory Work Activity Scheme is a prescribed placement for the purpose of section~6J(2)($a$)  of the Act\footnote{Section 6J was inserted by section 46 of the Welfare Reform Act 2012 (c.~5).} (sanctionable failure not to comply with a work placement).
\end{sloppypar}

(2) In paragraph (1) “the Mandatory Work Activity Scheme” means a scheme provided pursuant to arrangements made by the Secretary of State and known by that name that is designed to provide work or work-related activity for up to 30 hours per week over a period of 4 consecutive weeks with a view to assisting claimants to improve their prospects of obtaining employment.”.
\end{quotation}

\subsection[Chapter II --- Child support]{Chapter II\\*Child support}

\renewcommand\parthead{--- Part III Chapter II}

\subsubsection[40. Amendment of the Child Support (Maintenance Assessment Procedure) Regulations 1992]{Amendment of the Child Support (Maintenance Assessment Procedure) Regulations 1992}

40.—(1) The Child Support (Maintenance Assessment Procedure) Regulations 1992\footnote{S.I.~1992/1813. The Regulations are revoked in certain cases by S.I.~2001/157 (amended by S.I.~2003/328 and 347) and 2012/2785.} are amended as follows.

(2) In regulation 1(2) (interpretation), after the definition of “relevant person”, insert—
\begin{quotation}
“;

“universal credit” means universal credit under Part~I of the Welfare Reform Act 2012”.
\end{quotation}

(3) In regulation 23 (date from which a decision is superseded)\footnote{The regulation was substituted by S.I.~1999/1047 and amended by 2000/1596, 2012/1267 and 2012/2683.}—
\begin{enumerate}\item[]
($a$) at the beginning of paragraphs (3), (5), (10), (12), (13) and (20), insert “Subject to paragraph (25),”;

($b$) at the beginning of paragraph (1), after “paragraph (2)” insert “or~(25)”;

($c$) in paragraph (4), for “paragraph (19)” substitute “paragraphs (19) and (25)”; and

($d$) after paragraph (24) insert—
\begin{quotation}
“(25) Where a superseding decision is made under regulation 20(2) or (3) with respect to the circumstance that a parent with care or an absent parent—
\begin{enumerate}\item[]
($a$) has been awarded universal credit on the basis that they have no earned income;

($b$) was awarded universal credit on that basis and their award has been revised or superseded on the basis of their having, at the time the award was made or after that time, earned income; or

($c$) was awarded universal credit on the basis that they had earned income and their award has been revised or superseded on the basis of their not having, at the time the award was made or after that time, earned income,
\end{enumerate}
the decision takes effect from the first day of the maintenance period in which the award of universal credit, or the revision or supersession of such an award, as the case may be, took effect or is due to take effect.

(26) For the purposes of paragraph (25), “earned income” has the meaning given in regulation 52 of the Universal Credit Regulations 2013\footnote{S.I.~2013/376.}.”.
\end{quotation}
\end{enumerate}

\subsubsection[41. Amendment of the Child Support (Maintenance Assessments and Special Cases) Regulations 1992]{Amendment of the Child Support (Maintenance Assessments and Special Cases) Regulations 1992}

41.—(1) The Child Support (Maintenance Assessments and Special Cases) Regulations 1992\footnote{S.I.~1992/1815; the Regulations were revoked in certain cases by S.I.~2001/157 (amended by S.I.~2003/347) and 2012/2785. Regulation 1(2) was amended by S.I.~2013/235; there are other amendments of that provision that are not relevant to these Regulations.} are amended as follows.

(2) In regulation 1(2) (interpretation)—
\begin{enumerate}\item[]
($a$) after the definition of “clinical commissioning group” insert—
\begin{quotation}
““contribution-based jobseeker’s allowance” means an allowance under the Jobseekers Act as amended by the provisions of Part~I of Schedule 14 to the Welfare Reform Act 2012 that remove references to an income-based allowance, and a contribution-based allowance under the Jobseekers Act as that Act has effect apart from those provisions;”; and
\end{quotation}

($b$) after the definition of “training allowance”, insert—
\begin{quotation}
““universal credit” means universal credit under Part~I of the Welfare Reform Act 2012;”.
\end{quotation}
\end{enumerate}

(3) After regulation 10B (assessable income: state pension credit paid to or in respect of a parent with care or an absent parent)\footnote{Regulation 10B was inserted by S.I.~2003/2779.} insert—
\begin{quotation}
\subsubsection*{“Assessable income: universal credit paid to or in respect of the parent concerned}

10C.—(1) The circumstances prescribed for the purpose of the reference to universal credit in sub-paragraph (4) of paragraph 5 of Schedule 1 to the Child Support Act 1991 (as that paragraph has effect apart from section 1 of the Child Support, Pensions and Social Security Act 2000)\footnote{1991 c.~48. Paragraph 5(4) was amended by paragraph 2 of Schedule 2 to the Welfare Reform Act 2012 (c.~5).} are where the universal credit that is paid to or in respect of the parent concerned is calculated on the basis that the parent has no earned income.

(2) In paragraph (1), “earned income” has the meaning given in regulation 52 of the Universal Credit Regulations 2013.”.
\end{quotation}

(4) In Schedule 2, after paragraph 7 (amounts to be disregarded when calculating or estimating $N$ and $M$) insert—
\begin{quotation}
“7A.  Any payment of universal credit.”.
\end{quotation}

\subsubsection[42. Amendment of the Child Support Departure Direction and Consequential Amendments Regulations 1996]{Amendment of the Child Support Departure Direction and Consequential Amendments Regulations 1996}

42.—(1) The Child Support Departure Direction and Consequential Amendments Regulations 1996\footnote{S.I.~1996/2907. The Regulations were revoked in certain cases by S.I.~2001/157 (amended by S.I.~2003/347) and 2012/2785.} are amended as follows.

(2) In regulation 1 (interpretation)\footnote{Regulation 1(2) has been amended in ways that are not relevant to these Regulations.}—
\begin{enumerate}\item[]
($a$) in paragraph (2), after the definition of “relevant person” insert—
\begin{quotation}
“;

“relevant universal credit” means, in relation to an absent parent or parent with care, an award of universal credit made to the parent in question, where the award is calculated on the basis that the parent does not have any earned income;

“universal credit” means universal credit under Part~I of the Welfare Reform Act 2012”;
\end{quotation}

($b$) after paragraph (2) insert—
\begin{quotation}
“(2A) For the purposes of the definition of “relevant universal credit” in paragraph (2), “earned income” has the meaning given in regulation 52 of the Universal Credit Regulations 2013.”.
\end{quotation}
\end{enumerate}

(3) In regulation 9 (departure directions and persons in receipt of income support etc.)\footnote{Regulation 9 was substituted by S.I.~1998/58 and amended by S.I.~2003/328, 2003/2779 and 2008/1554.}—
\begin{enumerate}\item[]
($a$) in the heading to the regulation, after “jobseeker’s allowance,” insert “universal credit”;

($b$) in paragraph (1)—
\begin{enumerate}\item[]
(i) in sub-paragraphs ($a$)  and ($c$), for “or income-based jobseeker’s allowance” substitute “,~income-based jobseeker’s allowance or relevant universal credit”; and

(ii) in sub-paragraph ($b$), for “or working tax credit” substitute “,~working tax credit or relevant universal credit”;
\end{enumerate}

($c$) in sub-paragraphs ($a$)  and ($b$)  of paragraph (2), for “or income-based jobseeker’s allowance” substitute “,~income-based jobseeker’s allowance or relevant universal credit”; and

($d$) in paragraph (3)—
\begin{enumerate}\item[]
(i) in sub-paragraphs ($a$)  and ($c$), for “or income-based jobseeker’s allowance” substitute “,~income-based jobseeker’s allowance or relevant universal credit”; and

(ii) in sub-paragraph ($b$), for “or working tax credit” substitute “,~working tax credit or relevant universal credit”.
\end{enumerate}
\end{enumerate}

(4) In regulation 12 (meaning of “benefit” for the purposes of section 28E of the Child Support Act 1991)\footnote{Regulation 12 was amended by S.I.~2003/328, 2003/2779 and 2008/1554.}, for “and council tax benefit” substitute “council tax benefit and relevant universal credit”.

\subsubsection[43. Amendment of the Child Support (Maintenance Calculations and Special Cases) Regulations 2000]{Amendment of the Child Support (Maintenance Calculations and Special Cases) Regulations 2000}

43.—(1) The Child Support (Maintenance Calculations and Special Cases) Regulations 2000\footnote{S.I.~2001\slash 155. The Regulations were revoked in certain cases by S.I.~2012/2785.} are amended as follows.

(2) In regulation 1(2) (interpretation)\footnote{Regulation 1(2) has been amended in ways that are not relevant to these Regulations.}—
\begin{enumerate}\item[]
($a$) after the definition of “child tax credit”, insert—
\begin{quotation}
““contribution-based jobseeker’s allowance” means an allowance under the Jobseekers Act as amended by the provisions of Part~I of Schedule 14 to the Welfare Reform Act 2012 that remove references to an income-based allowance, and a contribution-based allowance under the Jobseekers Act as that Act has effect apart from those provisions;”; and
\end{quotation}

($b$) after the definition of “Contributions and Benefits (Northern Ireland) Act”, insert—
\begin{quotation}
““contributory employment and support allowance” means an allowance under Part~I of the Welfare Reform Act as amended by the provisions of Schedule 3, and Part~I of Schedule 14, to the Welfare Reform Act 2012 that remove references to an income-related allowance, and a contributory allowance under Part~I of the Welfare Reform Act as that Part has effect apart from those provisions;”.
\end{quotation}
\end{enumerate}

(3) In regulation 4 (flat rate)\footnote{Regulation 4(2) was amended by S.I.~2002/3019 and 2008/1554.}—
\begin{enumerate}\item[]
($a$) in paragraph (2), omit “and” after sub-paragraph ($c$)  and, after sub-paragraph ($d$)  insert—
\begin{quotation}
“; and

($e$) universal credit under Part~I of the Welfare Reform Act 2012, where the award of universal credit is calculated on the basis that the non-resident parent does not have any earned income”; and
\end{quotation}

($b$) after paragraph (3), insert—
\begin{quotation}
“(4) For the purposes of paragraph (2)($e$)  and regulation~5($d$), “earned income” has the meaning given in regulation 52 of the Universal Credit Regulations 2013 (earned income).”.
\end{quotation}
\end{enumerate}

(4) In regulation 5($d$)  (nil rate)\footnote{Regulation 5($d$) was amended by S.I.~2008/1554.} omit “or” after sub-paragraph (i)  and after sub-paragraph (ii)  insert—
\begin{quotation}
“;

(iii) in receipt of universal credit under Part~I of the Welfare Reform Act 2012, where the award of universal credit is calculated on the basis that they do not have any earned income; or

(iv) in a case not covered by paragraph (iii), a member of a couple where their partner is in receipt of universal credit under Part~I of the Welfare Reform Act 2012 and the award of universal credit is calculated on the basis that the non-resident parent does not have any earned income”.
\end{quotation}

\subsubsection[44. Amendment of the Child Support Maintenance Calculation Regulations 2012]{Amendment of the Child Support Maintenance Calculation Regulations 2012}

44.—(1) The Child Support Maintenance Calculation Regulations 2012\footnote{S.I.~2012/2677.} are amended as follows.

(2) In regulation 2 (interpretation)—
\begin{enumerate}\item[]
($a$) after the definition of “the 1991 Act” insert—
\begin{quotation}
““contribution-based jobseeker’s allowance” means an allowance under the Jobseekers Act 1995 as amended by the provisions of Part~I of Schedule 14 to the Welfare Reform Act 2012 that remove references to an income-based allowance, and a contribution-based allowance under the Jobseekers Act 1995 as that Act has effect apart from those provisions;”; and
\end{quotation}

($b$) for the definition of “contributory employment and support allowance” substitute—
\begin{quotation}
““contributory employment and support allowance” means an allowance under Part~I of the Welfare Reform Act 2007 as amended by the provisions of Schedule 3, and Part~I of Schedule 14, to the Welfare Reform Act 2012 that remove references to an income-related allowance, and a contributory allowance under Part~I of the Welfare Reform Act 2007 as that Part has effect apart from those provisions;”.
\end{quotation}
\end{enumerate}

(3) In regulation 44 (flat rate)—
\begin{enumerate}\item[]
($a$) in paragraph (2), omit “and” after sub-paragraph ($c$)  and, after sub-paragraph ($d$), insert—
\begin{quotation}
“and

($e$) universal credit under Part~I of the Welfare Reform Act 2012, where the award of universal credit is calculated on the basis that the non-resident parent does not have any earned income”; and
\end{quotation}

($b$) after paragraph (4), insert—
\begin{quotation}
“(5) For the purposes of paragraph (2)($e$)  and regulation~45(1)($c$), “earned income” has the meaning given in regulation~52 of the Universal Credit Regulations 2013.”.
\end{quotation}
\end{enumerate}

(4) In regulation 45(1)($c$)  (nil rate), omit “or” after paragraph (i)  and, after paragraph (ii), insert—
\begin{quotation}
“;

(iii) in receipt of universal credit under Part~I of the Welfare Reform Act 2012, where the award of universal credit is calculated on the basis that they do not have any earned income; or

(iv) in a case not covered by paragraph (iii), a member of a couple where their partner is in receipt of universal credit under Part~I of the Welfare Reform Act 2012 and the award of universal credit is calculated on the basis that the non-resident parent does not have any earned income”.
\end{quotation}

\subsection[Chapter III --- Children]{Chapter III\\*Children}

\renewcommand\parthead{--- Part III Chapter III}

\subsubsection[45. Amendment of the Adoption Support Services Regulations 2005]{Amendment of the Adoption Support Services Regulations 2005}

45.—(1) The Adoption Support Services Regulations 2005 are amended as follows.

(2) In regulation 2(1) (interpretation), after the definition of “tax credit” insert—
\begin{quotation}
“;

“universal credit” means universal credit under Part~I of the Welfare Reform Act 2012”.
\end{quotation}

(3) In regulation 11($c$)  (cessation of financial support), after “qualifies for” insert “universal credit,”.

\subsubsection[46. Amendment of the Special Guardianship Regulations 2005]{Amendment of the Special Guardianship Regulations 2005}

46.—(1) The Special Guardianship Regulations 2005 are amended as follows.

(2) In regulation 2(1) (Interpretation), after the definition of “relevant child” insert—
\begin{quotation}
“;

“universal credit” means universal credit under Part~I of the Welfare Reform Act 2012”.
\end{quotation}

(3) In regulation 9($c$)  (cessation of financial support), after “qualifies for” insert “universal credit,”.

\subsubsection[47. Amendment of the Childcare (Supply and Disclosure of Information) (England) Regulations 2007]{Amendment of the Childcare (Supply and Disclosure of Information) (England) Regulations 2007}

47.  Regulation 4 of the Childcare (Supply and Disclosure of Information) (England) Regulations 2007 (supply of information to Her Majesty’s Revenue and Customs) is amended as follows—
\begin{enumerate}\item[]
($a$) in the heading, after “Supply of information to” insert “the Secretary of State and”; and

($b$) in paragraphs (1) and (2), after “provided to” insert “the Secretary of State and”.
\end{enumerate}

\subsubsection[48. Amendment of the Childcare Providers (Information, Advice and Training) Regulations 2007]{Amendment of the Childcare Providers (Information, Advice and Training) Regulations 2007}

48.  Regulation 4 of the Childcare Providers (Information, Advice and Training) Regulations 2007 (information, advice and training) are amended as follows—
\begin{enumerate}\item[]
($a$) in paragraph (2)($d$), after “tax credit” insert “,~or the childcare costs element of universal credit,”; and

($b$) in paragraph (3), after sub-paragraph ($a$)  insert—
\begin{quotation}
“($aa$) childcare costs element of universal credit” means an amount included in the calculation of an award of universal credit under regulation 31 of the Universal Credit Regulations 2013;”.
\end{quotation}
\end{enumerate}

\subsection[Chapter IV --- Defence]{Chapter IV\\*Defence}

\renewcommand\parthead{--- Part III Chapter IV}

\subsubsection[49. Amendment of the Naval, Military and Air Forces etc (Disablement and Death) Service Pensions Order 2006]{Amendment of the Naval, Military and Air Forces etc (Disablement and Death) Service Pensions Order 2006}

49.—(1) The Naval, Military and Air Forces etc (Disablement and Death) Service Pensions Order 2006 is amended as follows.

(2) In article 10 (severe disablement occupational allowance)—
\begin{enumerate}\item[]
($a$) in paragraph (2)($c$), omit “under Part~I of the Welfare Reform Act 2007 or the corresponding provisions of the Welfare Reform Act (Northern Ireland) 2007”;

($b$) after paragraph (3), insert—
\begin{quotation}
“(4) In this article, “employment and support contributory allowance” means—
\begin{enumerate}\item[]
($a$) an allowance under Part~I of the Welfare Reform Act 2007 (“the 2007 Act”) as amended by the provisions of Schedule 3, and Part~I of Schedule 14, to the Welfare Reform Act 2012 that remove references to an income-related allowance, and a contributory allowance under Part~I of the 2007 Act as that Part has effect apart from those provisions; or

($b$) a contributory allowance under the provisions of the Welfare Reform Act (Northern Ireland) 2007 that correspond to Part~I of the Welfare Reform Act 2007.”.
\end{enumerate}
\end{quotation}
\end{enumerate}

(3) In article 15(2)($b$)(iii)  (allowance for lowered standard of occupation), after sub-paragraph ($bb$)  omit “or”, and after sub-paragraph ($cc$)  insert—
\begin{quotation}
“,~or

($dd$) an award of universal credit under Part~I of the Welfare Reform Act 2012, the calculation of which includes an amount under regulation 27(1) of the Universal Credit Regulations 2013 in respect of the fact that the member has limited capability for work or limited capability for work and work-related activity, or would include such an amount but for regulation 27(4) or 29(4) of those Regulations”.
\end{quotation}

(4) In article 50(3) (payment of public claims out of pensions), after sub-paragraph ($aa$)  omit “or”, and insert—
\begin{quotation}
“($ab$) universal credit under Part~I of the Welfare Reform Act 2012; or”.
\end{quotation}

(5) In article 56(3) (abatement of awards in respect of social security benefits) after sub-paragraph ($h$), insert—
\begin{quotation}
“;

($i$) Part~I of the Welfare Reform Act 2012”.
\end{quotation}

\subsection[Chapter V --- Education and employment]{Chapter V\\*Education and employment}

\renewcommand\parthead{--- Part III Chapter V}

\subsubsection[50. Amendment of the Employment Protection (Recoupment of Jobseeker’s Allowance and Income Support) Regulations 1996]{Amendment of the Employment Protection (Recoupment of Jobseeker’s Allowance and Income Support) Regulations 1996}

50.—(1) The Employment Protection (Recoupment of Jobseeker’s Allowance and Income Support) Regulations 1996 are amended as follows.

(2) In the title of the Regulations, for “Jobseeker’s Allowance and Income Support” substitute “Benefits”.

(3) In regulation 1 (citation and commencement), for “Jobseeker’s Allowance and Income Support” substitute “Benefits”.

(4) In regulation 2(1) (interpretation)—
\begin{enumerate}\item[]
($a$) in the definition of “recoupable benefit”, after “income-related employment and support allowance” insert “,~universal credit”;

($b$) after the definition of “Secretary of the Tribunals” insert—
\begin{quotation}
““universal credit” means universal credit under Part~I of the Welfare Reform Act 2012;”.
\end{quotation}
\end{enumerate}

(5) In regulation 4 (duties of the employment tribunals and of the Secretary of the Tribunals in respect of monetary awards), in paragraphs (1) and (8), after “income-related employment and support allowance” insert “,~universal credit”.

(6) In regulation 8 (recoupment of benefit)—
\begin{enumerate}\item[]
($a$) in paragraph (1), after “income-related employment and support allowance” insert “,~universal credit”;

($b$) in paragraph (2)($b$)  at the beginning insert “(i)”, and after “is attributable” insert—
\begin{quotation}
“; or

(ii) in the case of an employee entitled to an award of universal credit for any period (“the UC period”) which coincides with any part of the period to which the prescribed element is attributable, any amount paid by way of or on account of universal credit for the UC period that would not have been paid if the person’s earned income for that period was the same as immediately before the period to which the prescribed element is attributable”;
\end{quotation}

($c$) in paragraph (3)($b$), at the beginning insert “(i)”, and after “described in ($a$)  above” insert—
\begin{quotation}
“; or

(ii) in the case of an employee entitled to an award of universal credit for any period (“the UC period”) which coincides with any part of the protected period falling before the date described in ($a$)  above, any amount paid by way of or on account of universal credit for the UC period that would not have been paid if the person’s earned income for that period was the same as immediately before the protected period”; and
\end{quotation}

($d$) after paragraph (11) insert—
\begin{quotation}
\begin{sloppypar}
\sloppyword{``(12) For the purposes of paragraphs (2)($b$)(ii)  and~(3)($b$)(ii), ``earned income'' has the meaning given in regulation 52 of the Universal Credit Regulations 2013.''.}
\end{sloppypar}
\end{quotation}
\end{enumerate}

(7) In regulation 10 (provisions relating to determination of amount paid by way of or paid as on account of benefit), in paragraphs (1) and (2), after “income-related employment and support allowance” insert “,~universal credit”.

\subsubsection[51. Amendment of the Education (Student Loans) Regulations 1998]{Amendment of the Education (Student Loans) Regulations 1998}

51.  In Schedule 2, paragraph 1, to the Education (Student Loans) Regulations 1998 (terms of loans), for the definition of “disability related benefits” substitute—
\begin{quotation}
““disability related benefits” means—
\begin{enumerate}\item[]
($a$) 
long term incapacity benefit or short term incapacity benefit at the higher rate, severe disablement allowance, disability living allowance, industrial injuries benefit and disabled person’s tax credit, all payable under the Social Security Contributions and Benefits Act 1992;

($b$) 
personal independence payment under Part~IV of the Welfare Reform Act 2012;

($c$) 
armed forces independence payment under the Armed Forces and Reserve Forces (Compensation Scheme) Order 2011;

($d$) 
the amount of any disability premium and severe disability premium included in the applicable amount in calculating the income support payable under the Income Support (General) Regulations 1987;

($e$) 
any amount that is included in the calculation of an award of universal credit, under regulation 27(1) of the Universal Credit Regulations 2013, in respect of the fact that the borrower has limited capability for work or limited capability for work and work-related activity; or

($f$) 
any other statutory disability related benefit which replaces any of those benefits and which the lender gives the borrower details;”.
\end{enumerate}
\end{quotation}

\subsubsection[52. Amendment of the National Minimum Wage Regulations 1999]{Amendment of the National Minimum Wage Regulations 1999}

52.  In regulation 12(11) of the National Minimum Wage Regulations 1999 (workers who do not qualify for the national minimum wage)—
\begin{enumerate}\item[]
($a$) in sub-paragraph ($b$)(i), after “or entitled to,” insert “universal credit under Part~I of the Welfare Reform Act 2012,”; and

($b$) in sub-paragraph ($b$)(ii), for “either” substitute “any”.
\end{enumerate}

\subsubsection[53. Amendment of the Education (Student Support) (European University Institute) Regulations 2010]{Amendment of the Education (Student Support) (European University Institute) Regulations 2010}

53.  In regulation 27(2) of the Education (Student Support) (European University Institute) Regulations 2010 (interpretation), after sub-paragraph~($f$)  omit “and” and after sub-paragraph ($g$)  insert—
\begin{quotation}
“; and

($h$) in the case of a dependant who is entitled to an award of universal credit under Part~I of the Welfare Reform Act 2012—
\begin{enumerate}\item[]
(i) any amount that is included in the calculation of the award, under regulation 27(1) of the Universal Credit Regulations 2013, in respect of the fact that the dependant has limited capability for work or limited capability for work and work-related activity; and

(ii) any amount or additional amount that is included in the calculation of the award under regulation 24 of those Regulations (the child element)”.
\end{enumerate}
\end{quotation}

\subsubsection[54. Amendment of the Education (Student Support) Regulations 2011]{Amendment of the Education (Student Support) Regulations 2011}

54.—(1) The Education (Student Support) Regulations 2011 are amended as follows.

(2) In regulation 2(1) (interpretation), after the definition of “type 3 teacher training student” insert—
\begin{quotation}
““universal credit” means universal credit under Part~I of the Welfare Reform Act 2012;”.
\end{quotation}

\begin{sloppypar}
(3) In regulation 42(2) (interpretation of Chapter IV), after sub-paragraph~($g$)  omit “and” and after sub-paragraph ($h$)  insert—
\end{sloppypar}
\begin{quotation}
“($i$) in the case of a dependant who is entitled to an award of universal credit—
\begin{enumerate}\item[]
(i) any amount that is included in the calculation of the award, under regulation 27(1) of the Universal Credit Regulations 2013, in respect of the fact that the dependant has limited capability for work or limited capability for work and work-related activity;

(ii) any amount or additional amount that is included in the calculation of the award under regulation 24 of those Regulations (the child element)”.
\end{enumerate}
\end{quotation}

(4) In regulation 45(3) (childcare grant), for the words from “has elected” to the end substitute—
\begin{quotation}
“—

($a$) has elected to receive the childcare element of the working tax credit under Part~I of the Tax Credits Act 2002; or

($b$) is entitled to an award of universal credit the calculation of which includes an amount under regulation 31 of the Universal Credit Regulations 2013 (childcare costs element)”.
\end{quotation}

(5) In regulation 61(2) (qualifying conditions for the special support grant), after sub-paragraph ($a$)  omit “or”, and in paragraph ($b$)  after “of that Act” insert—
\begin{quotation}
“; or

($c$) under regulation 25(3) of the Universal Credit Regulations 2013 is liable or treated as being liable to make payments in respect of the accommodation they occupy as their home”.
\end{quotation}

(6) In regulation 125(1)($a$)  (amount of support for designated distance learning courses), after paragraph (ii)  omit “or”, and after paragraph (iii)  insert—
\begin{quotation}
“or

(iv) to universal credit;”.
\end{quotation}

(7) In regulation 142(3)($a$)  (amount of assistance in respect of courses beginning before 1st September 2012), after paragraph (ii)  omit “or”, and after paragraph (iii)  insert—
\begin{quotation}
“or

(iv) to universal credit;”.
\end{quotation}

\subsection[Chapter VI --- Housing and council tax]{Chapter VI\\*Housing and council tax}

\renewcommand\parthead{--- Part III Chapter VI}

\subsubsection[55. Amendment of the Council Tax (Discount Disregards) Order 1992]{Amendment of the Council Tax (Discount Disregards) Order 1992}

55.  In article 3(2) of the Council Tax (Discount Disregards) Order 1992 (the severely mentally impaired), after sub-paragraph ($m$)  insert—
\begin{quotation}
“($n$) universal credit under Part~I of the Welfare Reform Act the calculation of which includes an amount under regulation 27(1) of the Universal Credit Regulations 2013 in respect of the fact that the person in question has limited capability for work or limited capability for work and work-related activity or would include such an amount but for regulation 27(4) or 29(4) of those Regulations”.
\end{quotation}

\subsubsection[56. Amendment of the Council Tax (Administration and Enforcement) Regulations 1992]{Amendment of the Council Tax (Administration and Enforcement) Regulations 1992}

56.—(1) The Council Tax (Administration and Enforcement) Regulations 1992 are amended as follows.

(2) In regulation 1(2) (interpretation), after the definition of “managing agent”, omit “and” and, after the definition of “premium”, insert—
\begin{quotation}
“; and

“universal credit” means universal credit under Part~I of the Welfare Reform Act 2012”.
\end{quotation}

(3) In regulation 32 (interpretation and application of Part VI), after paragraph (iiia) of the definition of “earnings” insert—
\begin{quotation}
“(iiib) universal credit;”.
\end{quotation}

(4) In regulation 52(2)($b$)  (relationship between remedies), after “income support” insert “,~universal credit”.

(5) In regulation 54(5)($d$)  and (6A) (joint and several liability: enforcement), after “income support” insert “or universal credit”.

(6) In Schedule 3, in the Form of Attachment of Earnings Order, in the copy of regulation 32 of the Council Tax (Administration and Enforcement) Regulations 1992, after paragraph (iii)  insert—
\begin{quotation}
“(iiia) universal credit;”.
\end{quotation}

\subsubsection[57. Amendment of the Housing Renewal Grants Regulations 1996]{Amendment of the Housing Renewal Grants Regulations 1996}

57.—(1) The Housing Renewal Grants Regulations 1996 are amended as follows.

(2) Regulation 2(1) (interpretation) is amended as follows—
\begin{enumerate}\item[]
($a$) for the definition of ``contributory employment and support allowance'' substitute—
\begin{quotation}
““contributory employment and support allowance” means an allowance under Part~I of the Welfare Reform Act 2007 (“the 2007 Act”) as amended by the provisions of Schedule~3, and Part~I of Schedule 14, to the 2012 Act that remove references to an income-related allowance, and a contributory allowance under Part~I of the 2007 Act as that Part has effect apart from those provisions;”; and
\end{quotation}

($b$) after the definition of “training allowance” insert—
\begin{quotation}
““universal credit” means universal credit under Part~I of the 2012 Act;”.
\end{quotation}
\end{enumerate}

(3) In regulation 10 (the applicable amount)—
\begin{enumerate}\item[]
($a$) in the first paragraph that is numbered “(3)”—
\begin{enumerate}\item[]
\begin{sloppypar}
\sloppyword{(i) after sub-paragraph ($a$)(iv)  omit ``or'' and, after sub-paragraph~($a$)(v)  insert---}
\end{sloppypar}
\begin{quotation}
“or

(vi) universal credit;”; and
\end{quotation}

(ii) after sub-paragraph ($b$)  omit “or” and after sub-paragraph ($c$)  insert—
\begin{quotation}
“; or

($d$) subject to paragraph (5), a relevant person who has a partner, where the partner is entitled to universal credit”;
\end{quotation}
\end{enumerate}

($b$) the second paragraph that is numbered “(3)” is re-numbered “(4)”; and

($c$) after paragraph (4), insert—
\begin{quotation}
“(5) For the purposes of paragraph (3)($d$)  and regulation~11(2)($b$), where the relevant person and a partner of that person are parties to a polygamous marriage, the fact that they are partners will be disregarded if—
\begin{enumerate}\item[]
($a$) one of them is a party to an earlier marriage that still subsists; and

($b$) the other party to that earlier marriage is living in the same household.”.
\end{enumerate}
\end{quotation}
\end{enumerate}

(4) In regulation 11 (financial resources)—
\begin{enumerate}\item[]
($a$) at the beginning insert “(1) Subject to paragraph (2),”; and

($b$) after paragraph (1) insert—
\begin{quotation}
“(2) Subject to regulation 10(5), where a relevant person in the case of the application—
\begin{enumerate}\item[]
($a$) is entitled to universal credit; or

($b$) is not entitled to universal credit but their partner is so entitled,
\end{enumerate}
then the income of that relevant person for the purposes of paragraph (1) shall be taken to be nil.”.
\end{quotation}
\end{enumerate}

(5) In regulation 19 (treatment of child care charges), in paragraphs (3)($b$)  and (3)($c$)(ii), after “Employment and Support Allowance Regulations 2008” insert “or the Employment and Support Allowance Regulations 2013”.

(6) In regulation 31(10A)($b$)(i)  (notional income) for the words from “in accordance with” to the end substitute “approved by the Secretary of State”.

\subsubsection[58. Amendment of the Rent Repayment Orders (Supplementary Provisions) (England) Regulations 2007]{Amendment of the Rent Repayment Orders (Supplementary Provisions) (England) Regulations 2007}

58.—(1) The Rent Repayment Orders (Supplementary Provisions) (England) Regulations 2007 are amended as follows.

(2) For regulation 1(3) substitute—
\begin{quotation}
“(3) In these Regulations—
\begin{enumerate}\item[]
“the Act” means the Housing Act 2004;

“relevant award of universal credit” means an award as referred to in section 73(6A) of the Housing Act 2004.”.
\end{enumerate}
\end{quotation}

(3) In regulation 2—
\begin{enumerate}\item[]
($a$) in paragraph (1), after “housing benefit”, insert “or of a relevant award of universal credit”; and

($b$) in paragraph (2), for the words from “for the total amount” to the end substitute—
\begin{quotation}
“($a$) in the case of housing benefit, for the total amount of housing benefit paid, such part of that amount as they believe is properly payable;

($b$) in the case of a relevant award of universal credit, for the amount referred to in section 74(2A)($a$)  of the Act that was originally believed to apply, the amount that is now believed to apply (if different)”.
\end{quotation}
\end{enumerate}

(4) in paragraph (3), after sub-paragraph ($a$)  omit “and” and, after sub-paragraph ($b$)  insert—
\begin{quotation}
“,~and ($c$)  a relevant award of universal credit is properly payable if the person to whom, or in respect of whom, it is paid is entitled to it under the Universal Credit Regulations 2013 (whether on the initial decision or as subsequently revised or superseded or further revised or superseded)”.
\end{quotation}

\subsection[Chapter VII --- Immigration and asylum]{Chapter VII\\*Immigration and asylum}

\renewcommand\parthead{--- Part III Chapter VII}

\subsubsection[59. Amendment of Asylum Support Regulations 2000]{Amendment of Asylum Support Regulations 2000}

59.  In regulation 4(6)($a$)  of the Asylum Regulations 2000 (persons excluded from support)—
\begin{enumerate}\item[]
($a$) in paragraph (ii), after the words “Act 1992;”, omit the word “or”; and

($b$) after paragraph (iii)  insert—
\begin{quotation}
“or

(iv) universal credit under Part~I of the Welfare Reform Act 2012;”.
\end{quotation}
\end{enumerate}

\subsubsection[60. Amendment of the Displaced Persons (Temporary Protection) Regulations 2005]{Amendment of the Displaced Persons (Temporary Protection) Regulations 2005}

60.  In regulation 14 of the Displaced Persons (Temporary Protection) Regulations 2005 (housing: rent liability) omit from “,~in relation to” to the end and substitute—
\begin{quotation}
“in relation to—
\begin{enumerate}\item[]
($a$) any claim for housing benefit by virtue of regulation 3, such payments shall be regarded as rent for the purposes of regulation 14(1)($a$)  of the Housing Benefit Regulations 2006, regulation 12(1)($a$)  of the Housing Benefit (Persons who have attained the qualifying age for state pension credit) Regulations 2006 and regulation 10(1)($a$)  of the Housing Benefit (General) Regulations (Northern Ireland) 1987;

($b$) any claim for universal credit by virtue of regulation 3, such payments shall be regarded as rent for the purposes of regulation 25(2) of, and paragraph 2 of Schedule 1 to, the Universal Credit Regulations 2013”.
\end{enumerate}
\end{quotation}

\subsection[Chapter VIII --- Justice]{Chapter VIII\\*Justice}

\renewcommand\parthead{--- Part III Chapter VIII}

\subsubsection[61. Amendment of the Magistrates’ Courts Rules 1981]{Amendment of the Magistrates’ Courts Rules 1981}

61.  In rule 65(2)($ff$)  of the Magistrates’ Courts Rules 1981 (particulars of fine enforcement to be entered in register), omit “from income support”.

\subsubsection[62. Amendment of the Community Legal Service (Financial) Regulations 2000]{Amendment of the Community Legal Service (Financial) Regulations 2000}

62.  In regulation 4(2) of the Community Legal Service (Financial) Regulations 2000 (financial eligibility), after paragraph ($c$)  omit “or” and after paragraph~($d$)  insert—
\begin{quotation}
“;

($e$) universal credit under Part~I of the Welfare Reform Act 2012”.
\end{quotation}

\subsubsection[63. Amendment of the Criminal Defence Service (Recovery of Defence Costs Orders) Regulations 2001]{Amendment of the Criminal Defence Service (Recovery of Defence Costs Orders) Regulations 2001}

63.  In regulation 4(3)($b$)  of the Criminal Defence Service (Recovery of Defence Costs Orders) Regulations 2001, after paragraph (iii)  omit “or” and after paragraph (iv)  insert—
\begin{quotation}
“; or

(v) universal credit under Part~I of the Welfare Reform Act 2012”.
\end{quotation}

\subsubsection[64. Amendment of the Criminal Defence Service (General) (No.~2) Regulations 2001]{Amendment of the Criminal Defence Service (General) (No.~2) Regulations 2001}

64.  In regulation 5(8) of the Criminal Defence Service (General) (No.~2) Regulations 2001 (advice and assistance—financial eligibility), after paragraph~($e$)  omit “and” and after paragraph ($f$)  insert—
\begin{quotation}
“;

($g$) universal credit under Part~I of the Welfare Reform Act 2012”.
\end{quotation}

\subsubsection[65. Amendment of the Criminal Defence Service (Financial Eligibility) Regulations 2006]{Amendment of the Criminal Defence Service (Financial Eligibility) Regulations 2006}

65.  In regulation 5(4) of the Criminal Defence Service (Financial Eligibility) Regulations 2006 (assessment by representation authority), after sub-paragraph~($d$)  insert—
\begin{quotation}
“;

($e$) universal credit under Part~I of the Welfare Reform Act 2012”.
\end{quotation}

\subsubsection[66. Amendment of the Fines Collection (Disclosure of Information) (Prescribed Benefits) Regulations 2008]{Amendment of the Fines Collection (Disclosure of Information) (Prescribed Benefits) Regulations 2008}

66.—(1) The Fines Collection (Disclosure of Information) (Prescribed Benefits) Regulations 2008 are amended as follows.

(2) In regulation 1(2) (interpretation)—
\begin{enumerate}\item[]
($a$) before the definition of “contribution-based jobseeker’s allowance” insert—
\begin{quotation}
““contributory employment and support allowance” means an allowance under Part~I of the Welfare Reform Act 2007 (“the 2007 Act”) as amended by the provisions of Schedule~3, and Part~I of Schedule 14, to the Welfare Reform Act 2012 that remove references to an income-related allowance, and a contributory allowance under Part~I of the 2007 Act as that Part has effect apart from those provisions”;
\end{quotation}

($b$) for the definition of “contribution-based jobseeker’s allowance” substitute—
\begin{quotation}
““contribution-based jobseeker’s allowance”, except in a case to which paragraph ($b$)  of the definition of income-based jobseeker’s allowance applies, means an allowance under the Jobseekers Act 1995 (“the 1995 Act”) as amended by the provisions of Part~I of Schedule 14 to the Welfare Reform Act 2012 that remove references to an income-based allowance, and a contribution-based allowance under the 1995 Act as that Act has effect apart from those provisions;”; and
\end{quotation}

($c$) omit “and” after the definition of “income support” and after the definition of “state pension credit” insert—
\begin{quotation}
“; and 

“universal credit” means universal credit under Part~I of the Welfare Reform Act 2012”.
\end{quotation}
\end{enumerate}

(3) In regulation 2 (prescribed benefits), after paragraph ($d$)  omit “and”, and after paragraph ($e$)  insert—
\begin{quotation}
“; and

($f$) universal credit”.
\end{quotation}

\subsubsection[67. Amendment of the Criminal Defence Service (Contribution Orders) Regulations 2009]{Amendment of the Criminal Defence Service (Contribution Orders) Regulations 2009}

67.  In regulation 2(1) of the Criminal Defence Service (Contribution Orders) Regulations 2009 (interpretation), in the definition of “qualifying benefit” after paragraph ($d$)  insert—
\begin{quotation}
“;

($e$) universal credit under Part~I of the Welfare Reform Act 2012”.
\end{quotation}

\subsection[Chapter IX --- Landlord and tenant]{Chapter IX\\*Landlord and tenant}

\renewcommand\parthead{--- Part III Chapter IX}

\subsubsection[68. Rent Book (Forms of Notice) Regulations 1982]{Rent Book (Forms of Notice) Regulations 1982}

68.  In the Schedule to the Rent Book (Forms of Notice) Regulations 1982—
\begin{enumerate}\item[]
($a$) for paragraph 11 of Part I (Form for Rent Book for restricted contract) of the Schedule to the Regulations substitute—
\begin{quotation}
“You may be entitled to get help to pay your rent through the housing benefit scheme or through Universal Credit. Apply to your local council or to the Department for Work and Pensions for details. The Gov.uk website provides further advice: \url{http://www.gov.uk}”;
\end{quotation}

($b$) for paragraph 13 of Part II (Form for Rent Book for Protected or Statutory Tenancy) of the Schedule to the Regulations substitute—
\begin{quotation}
“You may be entitled to get help to pay your rent through the housing benefit scheme or through Universal Credit. Apply to your local council or to the Department for Work and Pensions for details. The Gov.uk website provides further advice: \url{http://www.gov.uk}”;
\end{quotation}

($c$) for paragraph 11 of Part III (Form for Rent Book for Tenancy under the Rent (Agriculture) Act 1976) of the Schedule to the Regulations substitute—
\begin{quotation}
“You may be entitled to get help to pay your rent through the housing benefit scheme or through Universal Credit. Apply to your local council or to the Department for Work and Pensions for details. The Gov.uk website provides further advice: \url{http://www.gov.uk}”; and
\end{quotation}

($d$) for paragraph 8 of Part IV (Form for Rent Book for Assured Tenancy or Assured Agricultural Occupancy) of the Schedule to the Regulations substitute—
\begin{quotation}
“You may be entitled to get help to pay your rent through the housing benefit scheme or through Universal Credit. Apply to your local council or to the Department for Work and Pensions for details. The Gov.uk website provides further advice: \url{http://www.gov.uk}”.
\end{quotation}
\end{enumerate}

\subsubsection[69. Assured Tenancies and Agricultural Occupancies (Forms) Regulations 1997]{Assured Tenancies and Agricultural Occupancies (Forms) Regulations 1997}

69.  In the Schedule to the Assured Tenancies and Agricultural Occupancies (Forms) Regulations 1997 (forms prescribed for the purposes of Part I of the Housing Act 1988)—
\begin{enumerate}\item[]
($a$) in paragraph 2 of the Guidance Notes to Form 4B (Landlord’s Notice proposing a new rent under an Assured Periodic Tenancy of premises situated in England) omit from “You should also notify” to “claiming benefit” and substitute—
\begin{quotation}
“You should also notify your Housing Benefit office in your local authority if you are claiming a Benefit or the Department for Work and Pensions if you are claiming Universal Credit. The Gov.UK website provides further advice: \url{http://www.gov.uk}”; and
\end{quotation}

($b$) in paragraph 2 of the Guidance Notes to Form 4C (Landlord’s or Licensor’s Notice proposing a new rent or licence fee under an Assured Agricultural Occupancy of premises situated in England) omit from “You should also notify” to “claiming benefit” and substitute—
\begin{quotation}
“You should also notify your Housing Benefit office in your local authority if you are claiming a Benefit or the Department for Work and Pensions if you are claiming Universal Credit. The Gov.uk website provides further advice: \url{http://www.gov.uk}”.
\end{quotation}
\end{enumerate}

\subsection[Chapter X --- National insurance contributions and credits]{Chapter X\\*National insurance contributions and credits}

\renewcommand\parthead{--- Part III Chapter X}

\subsubsection[70. Amendment of the Social Security (Credits) Regulations 1975]{Amendment of the Social Security (Credits) Regulations 1975}

70.—(1) The Social Security (Credits) Regulations 1975 are amended as follows.

(2) In regulation 2(1) (interpretation)—
\begin{enumerate}\item[]
($a$) after the definition of “the Act” insert—
\begin{quotation}
““the 2012 Act” means the Welfare Reform Act 2012;”;
\end{quotation}

($b$) for the definition of “contribution-based jobseeker’s allowance” substitute—
\begin{quotation}
““contribution-based jobseeker’s allowance” means an allowance under the Jobseekers Act 1995 as amended by the provisions of Part~I of Schedule 14 to the 2012 Act that remove references to an income-based allowance, and a contribution-based allowance under the Jobseekers Act 1995 as that Act has effect apart from those provisions;”;
\end{quotation}

($c$) for the definition of “contributory employment and support allowance” substitute—
\begin{quotation}
““contributory employment and support allowance” means an allowance under Part~I of the Welfare Reform Act as amended by the provisions of Schedule 3, and Part~I of Schedule 14, to the 2012 Act that remove references to an income-related allowance, and a contributory allowance under Part~I of the Welfare Reform Act as that Part has effect apart from those provisions;”;
\end{quotation}

($d$) after the definition of “relevant past year” insert—
\begin{quotation}
““universal credit” means universal credit under Part~I of the 2012 Act;”.
\end{quotation}
\end{enumerate}

(3) In regulation 7 (credits for approved training)—
\begin{enumerate}\item[]
($a$) in paragraph (1), for “(2) and (3)” substitute “(2) to (4)”; and

($b$) after paragraph (3), insert—
\begin{quotation}
“(4) Paragraph (1) shall not apply to a person in respect of any week in any part of which that person was entitled to universal credit.”.
\end{quotation}
\end{enumerate}

(4) In regulation 8A (credits for unemployment)—
\begin{enumerate}\item[]
($a$) for paragraph (2)($b$), substitute—
\begin{quotation}
“($b$) a week for the whole of which the person in relation to old style JSA—
\begin{enumerate}\item[]
(i) satisfied or was treated as having satisfied the conditions set out in paragraphs ($a$), ($c$)  and ($e$)  to ($h$)  of section 1(2) of the Jobseekers Act 1995 (conditions for entitlement to a jobseeker’s allowance); and

\begin{sloppypar}
(ii) satisfied the further condition specified in paragraph~(3) below; or
\end{sloppypar}
\end{enumerate}

($ba$) a week for the whole of which the person in relation to new style JSA—
\begin{enumerate}\item[]
(i) satisfied or was treated as having satisfied the conditions set out in paragraphs ($e$)  to ($h$)  of section 1(2) of the Jobseekers Act 1995 (conditions for entitlement to a jobseeker’s allowance);

(ii) satisfied or was treated as having satisfied the work-related requirements under section 6D and 6E of the Jobseekers Act 1995 (work search and work availability requirements); and

\begin{sloppypar}
(iii) satisfied the further condition specified in paragraph~(3) below; or”;
\end{sloppypar}
\end{enumerate}
\end{quotation}

($b$) in paragraph (2)($c$), after “sub-paragraph ($b$)” insert “or ($ba$)”;

($c$) in paragraph (3), after “paragraph (2)($b$)” insert “and ($ba$)”;

($d$) in paragraph (3)($b$), after “paragraph (2)($b$)” insert “or the conditions and requirements in paragraph (2)($ba$)”;

($e$) for paragraph (5)($c$), substitute—
\begin{quotation}
“($c$) a week in respect of which, in relation to the person concerned—
\begin{enumerate}\item[]
(i) an old style JSA was reduced in accordance with section 19 or 19A, or regulations made under section 19B, of the Jobseekers Act 1995; or

(ii) a new style JSA was reduced in accordance with section 6J or 6K of the Jobseekers Act 1995; or”;
\end{enumerate}
\end{quotation}

($f$) after paragraph (5)($dd$)  insert—
\begin{quotation}
“($de$) a week where paragraph (2)($b$), ($ba$)  or ($c$)  apply and the person concerned was entitled to universal credit for any part of that week; or”;
\end{quotation}

($g$) after paragraph (5) insert—
\begin{quotation}
“(6) In this regulation—
\begin{enumerate}\item[]
“new style JSA” means a jobseeker’s allowance under the Jobseekers Act 1995 as amended by the provisions of Part~I of Schedule 14 to the 2012 Act that remove references to an income-based allowance;

“old style JSA” means a jobseeker’s allowance under the Jobseekers Act 1995 as it has effect apart from the amendments made by Part~I of Schedule 14 to the 2012 Act that remove references to an income-based allowance.”.
\end{enumerate}
\end{quotation}
\end{enumerate}

(5) In regulation 8B (credits for incapacity for work or limited capability for work)—
\begin{enumerate}\item[]
($a$) in paragraph (2), after “paragraphs” insert “(2A),”;

($b$) after paragraph (2) insert—
\begin{quotation}
“(2A) This regulation shall not apply to a week where—
\begin{enumerate}\item[]
($a$) under paragraph (2)($a$)(i)  the person concerned was not entitled to incapacity benefit, severe disablement allowance or maternity allowance;

($b$) paragraph (2)($a$)(ii), (iva) or (v)  apply; or

($c$) under paragraph (2)($a$)(iv)  the person concerned was not entitled to an employment and support allowance by virtue of section 1(2)($a$)  of the Welfare Reform Act,
\end{enumerate}
and the person concerned was entitled to universal credit for any part of that week.”.
\end{quotation}
\end{enumerate}

\begin{sloppypar}
(6) After regulation 8F (credits for the purposes of entitlement to contribution-based jobseeker’s allowance following official error) insert—
\end{sloppypar}
\begin{quotation}
\subsubsection*{“Credits for persons entitled to universal credit}

8G.—(1) For the purposes of entitlement to a benefit to which this regulation applies, a person shall be credited with a Class 3 contribution in respect of a week if that person is entitled to universal credit under Part~I of the Welfare Reform Act 2012 for any part of that week.

(2) This regulation applies to—
\begin{enumerate}\item[]
($a$) a Category A retirement pension;

($b$) a Category B retirement pension;

($c$) a widowed parent’s allowance;

($d$) a bereavement allowance.”.
\end{enumerate}
\end{quotation}

\subsubsection[71. Amendment of the Social Security (Crediting and Treatment of Contributions, and National Insurance Numbers) Regulations 2001]{Amendment of the Social Security (Crediting and Treatment of Contributions, and National Insurance Numbers) Regulations 2001}

71.—(1) The Social Security (Crediting and Treatment of Contributions, and National Insurance Numbers) Regulations 2001 are amended as follows.

(2) In regulation 1(2) (interpretation)—
\begin{enumerate}\item[]
($a$) for the definition of “contribution-based jobseeker’s allowance” and “income-based jobseeker’s allowance” substitute—
\begin{quotation}
““contribution-based jobseeker’s allowance” means an allowance under the Jobseekers Act 1995 as amended by the provisions of Part~I of Schedule 14 to the Welfare Reform Act 2012 that remove references to an income-based allowance, and a contribution-based allowance under the Jobseekers Act 1995 as that Act has effect apart from those provisions;”;
\end{quotation}

($b$) for the definition of “contributory employment and support allowance” substitute—
\begin{quotation}
““contributory employment and support allowance” means an allowance under Part~I of the Welfare Reform Act as amended by the provisions of Schedule 3, and Part~I of Schedule 14, to the Welfare Reform Act 2012 that remove references to an income-related allowance, and a contributory allowance under Part~I of the Welfare Reform Act as that Part has effect apart from those provisions;”; and
\end{quotation}

($c$) after the definition of “earnings factor”, insert—
\begin{quotation}
““income-based jobseeker’s allowance” has the same meaning as in the Jobseekers Act 1995;”.
\end{quotation}
\end{enumerate}

\subsubsection[72. Amendment of the Social Security (Contributions) Regulations 2001]{Amendment of the Social Security (Contributions) Regulations 2001}

72.—(1) The Social Security (Contributions) Regulations 2001 are amended as follows.

(2) In regulation 1(2) (interpretation) for the definition of “a contribution-based jobseeker’s allowance” substitute—
\begin{quotation}
\begin{sloppypar}
““contribution-based jobseeker’s allowance” means an allowance under the Jobseekers Act 1995 as amended by the provisions of Part~I of Schedule 14 to the Welfare Reform Act 2012 that remove references to an income-based allowance, and a contribution-based allowance under the Jobseekers Act 1995 as that Act has effect apart from those provisions;”.
\end{sloppypar}
\end{quotation}

\subsubsection[73. Amendment of the Additional Pension and Social Security Pensions (Home Responsibilities) (Amendment) Regulations 2001]{Amendment of the Additional Pension and Social Security Pensions (Home Responsibilities) (Amendment) Regulations 2001}

73.—(1) The Additional Pension and Social Security Pensions (Home Responsibilities) (Amendment) Regulations 2001 are amended as follows.

(2) After regulation 5A (earnings factor credits eligibility for pensioners to whom employment and support allowance was payable) insert—
\begin{quotation}
\subsubsection*{“Earnings factor credits eligibility for certain persons entitled to universal credit}

5B.—(1) For the purposes of subsection (3) of section 44C (earnings factor credits) of the Contributions and Benefits Act, a pensioner is eligible for earnings factor enhancement in respect of a week if that pensioner was a person entitled to an award of universal credit under Part~I of the Welfare Reform Act 2012 in respect of any part of that week which includes—
\begin{enumerate}\item[]
($a$) if the person satisfies the condition in paragraph (2), an amount under regulation 27(1)($a$)  of the Universal Credit Regulations 2013 in respect of the fact that the person has limited capability for work;

($b$) an amount under regulation 27(1)($b$)  of those Regulations in respect of the fact that the person has limited capability for work and work-related activity; or

($c$) an amount under regulation 29(1) of those Regulations where the person has regular and substantial caring responsibilities for a severely disabled person,
\end{enumerate}
or would include any of those amounts but for regulation 27(4) or~29(4) of those Regulations.

(2) The condition referred to in paragraph (1)($a$)  is that for each of the 52 weeks immediately prior to that week—
\begin{enumerate}\item[]
\looseness=1
($a$) the person was entitled to universal credit in respect of the fact that the person had limited capability for work or would have included an amount in respect of the fact that the person had limited capability for work but for regulation~27(4) or~29(4) of the Universal Credit Regulations 2013; or

($b$) employment and support allowance under Part~I (employment and support allowance) of the Welfare Reform Act 2007 (“the 2007 Act”)—
\begin{enumerate}\item[]
(i) was payable to the person;

(ii) would have been payable to the person but for the fact that the person did not satisfy the contribution condition in paragraph 1 or paragraph 2 of Schedule 1 to the 2007 Act;

(iii) would have been payable to the person but for the fact that the person had been entitled to it for the relevant maximum number of days under section 1A of the 2007 Act; or

(iv) would have been payable to the person but for the fact that under regulations the amount was reduced to nil because of—
\begin{enumerate}\item[]
($aa$) receipt of other benefits; or

($bb$) receipt of payments from an occupational pension scheme or personal pension scheme.
\end{enumerate}
\end{enumerate}
\end{enumerate}

(3) Paragraph (2)($b$)  of this regulation is satisfied in respect of a week which falls between periods which are linked by virtue of regulations under paragraph 4 (linking periods) of Schedule 2 to the 2007 Act.”.
\end{quotation}

\subsubsection[74. Amendment of the Transfer of State Pensions and Benefits Regulations 2007]{Amendment of the Transfer of State Pensions and Benefits Regulations 2007}

74.—(1) The Transfer of State Pensions and Benefits Regulations 2007 are amended as follows.

(2) In regulation 1(2) (interpretation), in the definition of “relevant benefit”—
\begin{enumerate}\item[]
($a$) for paragraph ($c$)  substitute—
\begin{quotation}
“($c$) a jobseeker’s allowance under the Jobseekers Act 1995 as amended by the provisions of Part~I of Schedule 14 to the Welfare Reform Act 2012 that remove references to an income-based allowance, and a contribution-based allowance under the Jobseekers Act 1995 as it has effect apart from those provisions;

($ca$) a contribution-based jobseeker’s allowance under Part~II of the Jobseekers (Northern Ireland) Order 1995;”; and
\end{quotation}

($b$) for paragraph ($d$)  substitute—
\begin{quotation}
“($d$) employment and support allowance under Part~I of the Welfare Reform Act 2007 as amended by the provisions of Schedule 3, and Part~I of Schedule 14, to the Welfare Reform Act 2012 that remove references to an income-related allowance, and a contributory allowance under Part~I of the Welfare Reform Act 2007 as that Part has effect apart from those provisions;

($da$) contributory employment and support allowance under Part~I of the Welfare Reform Act (Northern Ireland) 2007;”.
\end{quotation}
\end{enumerate}

\subsection[Chapter XI --- Police]{Chapter XI\\*Police}

\renewcommand\parthead{--- Part III Chapter XI}

\subsubsection[75. Amendment of Police (Injury Benefit) Regulations 2006]{Amendment of Police (Injury Benefit) Regulations 2006}

\looseness=1
75.  In Schedule 4, paragraph 1(1), to the Police (Injury Benefit) Regulations 2006) (reduction in child’s special allowance during full time remunerated training etc), for the words from “applicable amount” to “1987” substitute—
\begin{quotation}
“standard allowance included in an award of universal credit, for a single claimant aged under 25 years, as specified in regulation~36 of the Universal Credit Regulations 2013”.
\end{quotation}

\subsubsection[76. Amendment of Police Pensions Regulations 2006]{Amendment of Police Pensions Regulations 2006}

76.  In regulation 42(5) of the Police Pensions Regulations 2006 (calculation of child survivors’ pensions), for the words from “applicable amount” to “1987” substitute—
\begin{quotation}
“standard allowance included in an award of universal credit, for a single claimant aged under 25 years, as specified in regulation~36 of the Universal Credit Regulations 2013”.
\end{quotation}

\subsection[Chapter XII --- Tax, child benefit, guardian's allowance and tax credits]{Chapter XII\\*Tax, child benefit, guardian's allowance and tax credits}

\renewcommand\parthead{--- Part III Chapter XII}

\subsubsection[77. Amendment of the Working Tax Credit (Entitlement and Maximum Rate) Regulations 2002]{Amendment of the Working Tax Credit (Entitlement and Maximum Rate) Regulations 2002}

77.—(1) The Working Tax Credit (Entitlement and Maximum Rate) Regulations 2002 are amended as follows.

(2) In regulation 2 (interpretation)---
\begin{enumerate}\item[]
($a$) in paragraph (1), in the definition of “contributory employment and support allowance”, after “Welfare Reform Act” insert “(“the 2007 Act”) as amended by the provisions of Schedule 3, and Part~I of Schedule~14, to the Welfare Reform Act 2012 that remove references to an income-related allowance, and a contributory allowance under Part~I of the 2007 Act as that Part has effect apart from those provisions”; and

($b$) in paragraph (5)($a$), after “2008” insert “or regulation 86 of the Employment and Support Allowance Regulations 2013”.
\end{enumerate}

(3) After regulation 9(7) (disability element and workers who are to be treated as at a disadvantage in getting a job) insert—
\begin{quotation}
“(7A) In paragraph (7)($b$)(iv), the reference to contributory employment and support allowance is a reference to an allowance under Part~I of the Welfare Reform Act 2007 (“the 2007 Act”) as amended by the provisions of Schedule 3, and Part~I of Schedule~14, to the Welfare Reform Act 2012 that remove references to an income-based allowance, and a contributory allowance under Part~I of the 2007 Act as that Part has effect apart from those provisions.”.
\end{quotation}

(4) After regulation 13(6) (entitlement to child care element of working tax credit) insert—
\begin{quotation}
“(6A) In paragraph (6)($h$), the reference to contributory employment and support allowance is a reference to an allowance under Part~I of the Welfare Reform Act 2007 (“the 2007 Act”) as amended by the provisions of Schedule 3, and Part~I of Schedule~14, to the Welfare Reform Act 2012 that remove references to an income-related allowance, and a contributory allowance under Part~I of the 2007 Act as that Part has effect apart from those provisions.”.
\end{quotation}

\subsubsection[78. Amendment of the Tax Credits (Definition and Calculation of Income) Regulations 2002]{Amendment of the Tax Credits (Definition and Calculation of Income) Regulations 2002}

78.—(1) The Tax Credits (Definition and Calculation of Income) Regulations 2002 are amended as follows.

(2) In Table 3 in regulation 7(3) (social security income), in the entry in row 16, after “Jobseeker’s Act 1995” insert “as amended by the provisions of Part~I of Schedule 14 to the Welfare Reform Act 2012 that remove references to an income-based allowance, and a contribution-based allowance under the Jobseekers Act 1995 as that Act has effect apart from those provisions”.

(3) In regulation 17(2)($b$)  (claimants providing services to other persons for less than full earnings), for paragraph (i)  substitute—
\begin{quotation}
“(i) in Great Britain, which is approved by the Secretary of State;”.
\end{quotation}

\subsubsection[79. Amendment of the Child Tax Credit Regulations 2002]{Amendment of the Child Tax Credit Regulations 2002}

79.  In regulation 5(4)($c$)  of the Child Tax Credit Regulations 2002 (maximum age and prescribed conditions for a qualifying young person)—
\begin{enumerate}\item[]
($a$) for the “or” after “Welfare Reform Act 2007” substitute “,”; and

($b$) after “Jobseekers Act 1995”, insert “or universal credit under Part~I of the Welfare Reform Act 2012”.
\end{enumerate}

\subsubsection[80. Amendment of the Tax Credits (Administrative Arrangements) Regulations 2002]{Amendment of the Tax Credits (Administrative Arrangements) Regulations 2002}

80.  In regulation 5(6) of the Tax Credits (Administrative Arrangements) Regulations 2002 (recording, verification and holding, and forwarding, of claims etc.\ received by relevant authorities), remove the “or” between sub-paragraphs ($a$)  and ($b$)  and after sub-paragraph ($b$)  insert—
\begin{quotation}
“; or

($c$) universal credit under Part~I of the Welfare Reform Act 2012”.
\end{quotation}

\subsubsection[81. Amendments to the Child Benefit and Guardian’s Allowance (Administration) Regulations 2003]{Amendments to the Child Benefit and Guardian’s Allowance (Administration) Regulations 2003}

81.  In regulation 19(1)($b$)  of the Child Benefit and Guardian’s Allowance (Administration) Regulations 2003 (persons who may elect to have child benefit paid weekly), remove the “or” at the end of paragraph (iii)  and after paragraph (iv)  insert—
\begin{quotation}
“; or

(v) universal credit under Part~I of the Welfare Reform Act 2012”.
\end{quotation}

\subsubsection[82. Amendment of the Child Benefit and Guardian’s Allowance (Administrative Arrangements) Regulations 2003]{Amendment of the Child Benefit and Guardian’s Allowance (Administrative Arrangements) Regulations 2003}

82.  In regulation 5(8) of the Child Benefit and Guardian’s Allowance (Administrative Arrangements) Regulations 2003 (recording, verification and holding, and forwarding, of claims etc.\ received by relevant authorities), omit the “or” between sub-paragraphs ($a$)  and ($b$)  and after sub-paragraph ($b$)  insert—
\begin{quotation}
“; or

($c$) universal credit under Part~I of the Welfare Reform Act 2012”.
\end{quotation}

\subsubsection[83. Amendment of the Income Tax (Pay As You Earn) Regulations 2003]{Amendment of the Income Tax (Pay As You Earn) Regulations 2003}

83.  In regulation 148 of the Income Tax (Pay As You Earn) Regulations 2003 (interpretation of Chapters I and II), in the entry for “Chapter~II claimant”, for paragraph ($b$)  substitute—
\begin{quotation}
“($b$) a claimant who is a share fisherman—
\begin{enumerate}\item[]
(i) where the JSA Regulations apply, as defined in regulation~156 of those Regulations; and

(ii) where the Jobseeker’s Allowance Regulations 2013 apply, as defined in regulation 67 of those Regulations;”.
\end{enumerate}
\end{quotation}

\subsubsection[84. Amendments to the Child Benefit (General) Regulations 2006]{Amendments to the Child Benefit (General) Regulations 2006}

84.  In regulation 8 of the Child Benefit (General) Regulations 2006 (child benefit not payable in respect of qualifying young person: other financial support), omit “or” at the end of sub-paragraph ($d$)  and after sub-paragraph~($e$)  insert—
\begin{quotation}
“; or

($f$) universal credit under Part~I of the Welfare Reform Act 2012”.
\end{quotation}

\subsection[Chapter XIII --- Transport]{Chapter XIII\\*Transport}

\renewcommand\parthead{--- Part III Chapter XIII}

\subsubsection[85. Amendment of the Bus Service Operators Grant (England) Regulations 2002]{Amendment of the Bus Service Operators Grant (England) Regulations 2002}

85.  In regulation 3(4) of the Bus Service Operators Grant (England) Regulations 2002 (eligibility for grant), after sub-paragraph ($c$)  insert—
\begin{quotation}
“($ca$) persons in receipt of universal credit under Part~I of the Welfare Reform Act 2012;”.
\end{quotation}

\subsection[Chapter XIV --- Water]{Chapter XIV\\*Water}

\renewcommand\parthead{--- Part III Chapter XIV}

\subsubsection[86. Water Industry (Charges) (Vulnerable Groups) Regulations 1999]{Water Industry (Charges) (Vulnerable Groups) Regulations 1999}

86.  In regulation 2(4) of the Water Industry (Charges) (Vulnerable Groups) Regulations 1999 (special provision to be included in charges schemes), insert after sub-paragraph ($e$)—
\begin{quotation}
“($f$) universal credit under Part~I of the Welfare Reform Act 2012”.
\end{quotation}

\bigskip

\pagebreak[3]

Signed 
by authority of the 
Secretary of State for~Work and~Pensions.
%I concur
%By authority of the Lord Chancellor

{\raggedleft
\emph{Freud}\\*
%Secretary
%Minister
Parliamentary Under Secretary 
of State\\*Department 
for~Work and~Pensions

}

13th March 2013

\small

\part{Explanatory Note}

\renewcommand\parthead{— Explanatory Note}

\subsection*{(This note is not part of the Regulations)}

These Regulations make consequential, supplementary, incidental and miscellaneous provision in relation to the provisions of Part~I of the Welfare Reform Act 2012 (c.~5) (“the Act”) that relate to the introduction of universal credit (“universal credit provisions”) and the abolition of income-related employment and support allowance and income-based jobseeker’s allowance.

As a result of this abolition, employment and support allowance will no longer consist of separate contributory and income-related allowances, but only of a contributory allowance to be known simply as “employment and support allowance”.

Also, jobseeker’s allowance will no longer consist of separate contribution-based and income-based allowances, but of a contribution-based allowance to be known as “jobseeker’s allowance”.

The universal credit provisions and the provisions abolishing income-related employment and support allowance and income-based jobseeker’s allowance are to be commenced in stages, such that for a period of time, the old forms of employment and support allowance and jobseeker’s allowance (“old style ESA” and “old style JSA”) will apply to some people and the new forms (“new style ESA” and “new style JSA”) to other people.

Part~II of the Regulations amends provisions of primary legislation. The Part makes amendments to 18 pieces of primary legislation, consequential on the coming into force of Part~I of the 2012 Act.

The majority of the amendments made by these Regulations add a reference to universal credit to existing primary legislation. A few of the amendments insert a reference to particular elements of Universal Credit, for example the Income Tax (Earnings and Pensions) Act 2003 (c.~1) is amended to refer to Universal Credit paid in respect of childcare costs (see section 12 of the Act). The amendments to the State Pension Credit Act 2002 (c.~16) and the Employment Act 1989 (c.38) are minor consequential amendments.

Regulation 18 amends sections 73, 74, 96 and 97 of the Housing Act 2004 (c.34) and provides that, in relation to applications to a residential property tribunal for a rent repayment order (RRO) where a person has failed to obtain a licence for a house in multiple occupation or a house in an area of selective licensing, the tribunal may make an order–
\begin{enumerate}\item[]
a) in the case of an application by the local housing authority, in the amount of the housing element of an award of universal credit in respect of rent, or the amount of the award if less;

b) in the case of an application by the occupier, in an amount that takes account of the rent paid and the amount of the award of universal credit the calculation of which that includes the housing element of an award of universal credit in respect of rent.
\end{enumerate}

Part~III contains amendments to secondary legislation.

Chapter~I contains amendments relating to social security benefits.

The regulations in this Part insert references to universal credit where there are already references to other income-related benefits. They also provide for definitions of “contributory employment and support allowance” and “contribution-based jobseeker’s allowance” that include both the old style ESA and JSA contributory allowances and the new style contributory-only ESA and JSA allowances.

\begin{sloppypar}
Regulation 28 amends the Income Support (General) Regulations 1987 (S.I.~1987\slash 1967). In addition to the changes as referred to above it provides that–
\end{sloppypar}
\begin{enumerate}\item[]
a) payments of universal credit that do not relate to a period for which income support is payable are disregarded;

b) the definition of a “disabled person” includes where a person is entitled to an award of universal credit the calculation of which includes an amount in respect of the fact that they have limited capability for work (LCW) or limited capability for work and work-related activity (LCWRA) (or would include such an amount but for regulation 27(4) (couples) or 29(4) (Carer’s allowance) of the Universal Credit Regulations 2013 (S.I.~2013/376);

c) the exceptions from the rule that provides for a deduction to be made from the housing costs element of income support in respect of a non-dependant of the claimant include the situation where the non-dependant is aged less than 25 and is entitled to universal credit on the basis that the non-dependant does not have any earned income as defined in the Universal Credit Regulations 2013 (S.I.~2013/376).
\end{enumerate}

Regulations 30, 33, 35, 36 and 37 make similar provision in relation to Jobseeker’s Allowance, State Pension Credit, Housing Benefit and Employment and Support Allowance.

Regulations 38 and 39 make miscellaneous amendments to the Universal Credit Regulations 2013 and the Jobseeker’s Allowance Regulations 2013. In particular:
\begin{enumerate}\item[]
    regulation 38(2) provides for an extra category of claimant who may claim universal credit as single person when their partner is not eligible to claim;

    regulation 38(9) makes provision for the period of a hardship payment where an award has been reduced for a sanctionable failure.
\end{enumerate}

Chapter~II contains amendments to secondary legislation relating to child support.

Regulation 42 amends the Child Support (Maintenance Assessments and Special Cases) Regulations 1992 (S.I.~1992/1815) which relates to maintenance assessments under the “old scheme”, under the Child Support Act 1991 as it has effect apart from section 1 of the Child Support, Pensions and Social Security Act 2000. The amendment provides that, where a parent with care or absent parent is awarded universal credit on the basis that they have no earned income, as defined in the Universal Credit Regulations 2013, they will be treated as having “no assessable income” for the purposes of a maintenance assessment.

Regulations 43 and 44 amend the Child Support (Maintenance Calculations and Special Cases) Regulations 2000 (S.I.~2000/155) and Child Support Maintenance Calculation Regulations 2012 S.I.~2012/2677) which relate to the “current scheme” and the “future scheme” respectively under the Child Support Act 1991 as it has effect as amended by section 1 of the Child Support, Pensions and Social Security Act 2000. The amendments provide that, where a non-resident parent or their partner is awarded universal credit on the basis that the non-resident parent has no “earned income”, then they will be liable to pay the flat rate of maintenance unless the conditions for payment of the nil rate of maintenance apply. They also provide that the latter conditions include a reference to the situation where a non-resident parent or their partner is awarded universal credit on the above basis.

Chapters~III to~XIV contain amendments to secondary legislation relating to other legal regimes. Again, these Chapters insert references to universal credit where there are already references to other income-related benefits and provide for definitions of “contributory employment and support allowance” and “contribution-based jobseeker’s allowance” that include both the old style ESA and JSA contributory allowances and the new style contributory-only ESA and JSA allowances.

Regulation 50 amends the Employment Protection (Recoupment of Jobseeker’s Allowance and Income Support) Regulations 1996 (S.I.~1996/2349) and provides for the recoupment of an award if universal credit where the award was paid for a period in respect of which an employment tribunal has made an order, where the award would not have been paid if the person’s earnings had not been reduced or stopped.

Regulation 57 amends the Housing Renewal Grants Regulations (S.I.~1996\slash 2890) to provide that, in relation to a “relevant person” with respect to whom an application for a housing renewal grant is made, where the person or their partner (excluding a partner to a polygamous marriage that is not the earliest marriage with respect to partners living in one household) is entitled to universal credit, then they are to be regarded as having no income, and as having an “applicable amount” of £1, with the result that there will be no reduction in grant with respect to that person.

Regulation 58 amends the Rent Repayment Orders (Supplementary Provision) Regulations 2007 (S.I.~2007/572) and (complementing the amendment of sections 72, 73, 96 and 97 of the Housing Act 2004, referred to above) provides for the tribunal to be able to alter the amount of a rent repayment order where an award of universal credit that included the housing element with respect to occupation of part of the house in question has been altered in a material way.

Regulation 70 amends the Social Security (Credits) Regulations 1975 (S.I.~1975\slash 556) and provides that a person entitled to universal credit will be credited with a class 3 national insurance contribution.

An impact assessment has been made of the impact of universal credit. Copies of the impact assessment may be obtained from the Better Regulation Unit of the Department of Work and Pensions, 2D Caxton House, Tothill Street, London, \textsc{\lowercase{SW1P~9NA}} or from the DWP website at: \href{http://www.dwp.gov.uk/policy/welfare-reform/legislation-and-key-documents/welfare-reform-act-2012/impact-assessments-and-equality/}{\sloppyword{http://www.dwp.gov.uk/policy/welfare-reform/legislation-and-key-documents/welfare-reform-act-2012/impact-\hbox{assessments}-and-equality/}}.

\end{document}