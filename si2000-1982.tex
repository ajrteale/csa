\documentclass[12pt,a4paper]{article}

\newcommand\regstitle{The Social Security (Joint Claims: Consequential Amendments) Regulations 2000}

\newcommand\regsnumber{2000/1982}

%\opt{newrules}{
\title{\regstitle}
%}

%\opt{2012rules}{
%\title{Child Maintenance and Other Payments Act 2008\\(2012 scheme version)}
%}

\author{S.I. 2000 No. 1982}

\date{Made
24th July 2000\\
Laid before Parliament
27th July 2000\\
Coming into force
19th March 2001
}

%\opt{oldrules}{\newcommand\versionyear{1993}}
%\opt{newrules}{\newcommand\versionyear{2003}}
%\opt{2012rules}{\newcommand\versionyear{2012}}

\usepackage{csa-regs}

\setlength\headheight{27.57402pt}

\begin{document}

\maketitle

\noindent
The Secretary of State for Social Security, in exercise of the powers conferred on him by sections 2A(6)($a$), 5(1)($a$), ($b$), ($c$)  and ($p$), 189(1) and (4) and 191 of the Social Security Administration Act 1992\footnote{1992 c.\ 5; section 2A was inserted by section 57 of the Welfare Reform and Pensions Act 1999 (c.\ 30). Section 191 is an interpretation provision and is cited because of the meaning ascribed to the word “prescribe”.}, sections 123(1)($d$)  and ($e$), 137(1) and 175(1), (3) and (4) of the Social Security Contributions and Benefits Act 1992\footnote{1992 c.\ 4; section 123(1)($e$) was substituted by the Local Government Finance Act 1992 (c.\ 14), Schedule 9, paragraph 1(1); section 137(1) is an interpretation provision and is cited because of the meaning ascribed to the word “prescribed”.}, sections 9(1), 10(3) and (6), 79(1) and 84 of the Social Security Act 1998\footnote{1998 c.\ 14; section 84 is an interpretation provision and is cited because of the meaning ascribed to the word “prescribe”.} and sections 35(1), 36 and 40 of the Jobseekers Act 1995\footnote{1995 c.\ 18; section 35(1) is an interpretation provision and is cited because of the meaning ascribed to the words “prescribed” and “regulations”.}, and of all other powers enabling him in that behalf, after consultation, in respect of provisions in these Regulations relating to housing benefit and council tax benefit, with organisations appearing to the Secretary of State to be representative of the authorities concerned\footnote{\emph{See} section 176(1) of the Social Security Administration Act 1992 (c.\ 5).} by this Instrument, which contains only regulations made consequential upon section 59 of, and Schedule 7 to, the Welfare Reform and Pensions Act 1999\footnote{1999 c.\ 30.} and which is made before the end of the period of six months beginning with the coming into force of those provisions\footnote{\emph{See} section 173(5)($b$) of the Social Security Administration Act 1992.}, hereby makes the following Regulations: 

\enlargethispage{\baselineskip}

{\sloppy

\tableofcontents

}

\bigskip

\setcounter{secnumdepth}{-2}

\subsection[1. Citation and commencement]{Citation and commencement}

1.  These Regulations may be cited as the Social Security (Joint Claims: Consequential Amendments) Regulations 2000 and shall come into force on 19th March 2001.

\subsection[2. Amendment of the Social Security (Claims and Payments) Regulations 1987]{Amendment of the Social Security (Claims and Payments) Regulations 1987}

2.---(1)  The Social Security (Claims and Payments) Regulations 1987\footnote{S.I.\ 1987/1968.} shall be amended in accordance with the following paragraphs of this regulation.

(2) In regulation 2(1) (interpretation), after the definition of “the Jobseeker’s Allowance Regulations” there shall be inserted the following definition—
\begin{quotation}
\begin{sloppypar}
    \textls[25]{““joint-claim couple” and “joint-claim jobseeker’s al\-}lowance” have the same meaning in these Regulations as they have in the Jobseekers Act by virtue of section 1(4) of that Act\footnote{Those definitions were inserted into section 1(4) by section 59 of, and paragraph 2(4) of Schedule 7 to, the Welfare Reform and Pensions Act 1999 (c.\ 30).};”. 
\end{sloppypar}
\end{quotation}

(3) In regulation 4 (making a claim for benefit)—
\begin{enumerate}\item[]
($a$) in paragraph (1B)\footnote{Paragraph (1B) was inserted by S.I.\ 1997/793.}—
\begin{enumerate}\item[]
(i) at the beginning of sub-paragraph ($a$)  there shall be inserted the words “subject to paragraph (1BA),”;

(ii) in sub-paragraph ($e$), after the words “making the claim” there shall be inserted the words “or, in the case of a claim for a jobseeker’s allowance by a joint-claim couple, either member of that couple,”;
\end{enumerate}

($b$) after paragraph (1B) there shall be inserted the following paragraph—
\begin{quotation}
“(1BA) In the case of a joint-claim couple claiming a jobseeker’s allowance jointly, paragraph (1B)($a$)  shall not apply to the extent that it is reasonably practicable for a member of a joint-claim couple to whom that sub-paragraph applies to obtain assistance from the other member of that couple.”;
\end{quotation}

($c$) in paragraph (3B)\footnote{Paragraph (3B) was inserted by S.I.\ 1996/1460.}, for sub-paragraph ($b$)  there shall be substituted the following sub-paragraph—
\begin{quotation}
“($b$) where there is no entitlement to a contribution-based jobseeker’s allowance on a claim made—
\begin{enumerate}\item[]
(i) by a member of a joint-claim couple, he subsequently claims a joint-claim jobseeker’s allowance with the other member of that couple, the claim made by the couple shall be treated as having been made on the date on which the member of that couple made the claim for a jobseeker’s allowance in respect of which there was no entitlement to contribution-based jobseeker’s allowance;

(ii) by one partner and the other partner wishes to claim income-based jobseeker’s allowance, the claim made by that other partner shall be treated as having been made on the date on which the first partner made his claim;”;
\end{enumerate}
\end{quotation}

($d$) in paragraph (5)\footnote{Paragraph (5) was substituted by S.I.\ 1997/793.}, after the word “he” there shall be inserted the words “, or if he is a member of a joint-claim couple, either member of that couple”;

($e$) in paragraph (6)($a$)\footnote{Paragraph (6) was substituted by S.I.\ 1996/1460 and amended by S.I.\ 1999/3108.}, for the words “notice under regulation 23” there shall be substituted the words “notification under regulation 23 or 23A”;

($f$) for paragraph (7A)\footnote{Paragraph (7A) was inserted by S.I.\ 1997/793.}, there shall be substituted the following paragraphs—
\begin{quotation}
“(7A) In the case of a claim for income support, if a defective claim is received, the Secretary of State shall advise the person making the claim of the defect and of the relevant provisions of regulation 6(1A) relating to the date of claim.

(7B) In the case of a claim for a jobseeker’s allowance, if a defective claim is received, the Secretary of State shall advise—
\begin{enumerate}\item[]
($a$) in the case of a claim made by a joint-claim couple, each member of the couple of the defect and of the relevant provisions of regulation 6(4ZA) relating to the date of the claim;

($b$) in any other case, the person making the claim of the defect and of the relevant provisions of regulation 6(4A) relating to the date of claim.”.
\end{enumerate}
\end{quotation}
\end{enumerate}

(4) In regulation 6 (date of claim)—
\begin{enumerate}\item[]
($a$) after paragraph (4) there shall be inserted the following paragraphs—
\begin{quotation}
“(4ZA) Where a member of a joint-claim couple notifies the employment officer (by whatever means) that he wishes to claim a jobseeker’s allowance jointly with the other member of that couple, the claim shall be treated as made on the relevant date specified in accordance with paragraphs (4ZB) to (4ZD).

(4ZB) Where each member of a joint-claim couple is required to attend under regulation 4(6)($a$)—
\begin{enumerate}\item[]
($a$) if each member subsequently attends for the purpose of jointly claiming a jobseeker’s allowance at the time and place specified by the employment officer and complies with the requirements of paragraph (4AA)($a$), the claim shall be treated as made on whichever is the later of the first notification of intention to make that claim and the first day in respect of which the claim is made;

($b$) if, without good cause, either member fails to attend for the purpose of jointly claiming a jobseeker’s allowance at either the time or place so specified or does not comply with the requirements of paragraph (4AA)($a$), the claim shall be treated as made on the first day on which a member of the couple attends at the specified place and complies with the requirements of paragraph (4AA)($a$) .
\end{enumerate}

(4ZC) Where only one member of the couple is required to attend under regulation 4(6)($a$)—
\begin{enumerate}\item[]
($a$) subject to the following sub-paragraphs, the date on which the claim is made shall be the date on which a properly completed claim is received in an appropriate office or the first day in respect of which the claim is made, if later, provided the member of the couple who is required to attend under regulation 4(6)($a$)  does so attend;

($b$) where a properly completed form is received in an appropriate office within one month of first notification of intention to make that claim, the date of claim shall be the date of that notification;

($c$) if, without good cause, the member of the couple who is required to attend under regulation 4(6)($a$)  fails to attend for the purpose of making a claim at either the time or place so specified or does not comply with the requirements of paragraph (4AA), the claim shall be treated as made on the first day on which that member does attend at that place and does provide a properly completed claim.
\end{enumerate}

(4ZD) Where, as at the day on which a member of a joint-claim couple (“the first member”) notifies the employment officer in accordance with paragraph (4ZA), the other member of that couple is temporarily absent from Great Britain in the circumstances specified in regulation 50(6B) of the Jobseeker’s Allowance Regulations, the date on which the claim is made shall be the relevant date specified in paragraph (4ZB) or (4ZC) but nothing in this paragraph shall treat the claim as having been made on a day which is more than three months after the day on which the first member notified the employment officer in accordance with paragraph (4ZA).”;
\end{quotation}

($b$) in paragraph (4A)\footnote{Regulation 6(4A) was substituted by S.I.\ 1997/793.}—
\begin{enumerate}\item[]
(i) after the words “a person” there shall be inserted the words “who is not a member of a joint-claim couple”;

(ii) in sub-paragraph ($a$), after the words “paragraph (4AA)” in both places where those words occur, there shall be inserted “($b$)”;
\end{enumerate}

($c$) for paragraph (4AA)\footnote{Regulation 6(4AA) was inserted by S.I.\ 1997/793.}, there shall be substituted the following paragraph—
\begin{quotation}
“(4AA) Unless the Secretary of State otherwise directs, a properly completed claim form shall be provided—
\begin{enumerate}\item[]
($a$) in a case to which paragraph (4ZA) applies, at or before the time when a member of the joint-claim couple is first required to attend for the purpose of making a claim for a jobseeker’s allowance;

($b$) in any other case, at or before the time when the person making the claim for a jobseeker’s allowance is required to attend for the purpose of making a claim.”.
\end{enumerate}
\end{quotation}
\end{enumerate}

(5) In regulation 19 (time for claiming benefit), after paragraph (7)($h$)\footnote{Regulation 19 was substituted by S.I.\ 1997/793; paragraph (7)($h$) was added by S.I.\ 1997/2290.}, there shall be added the following sub-paragraph—
\begin{quotation}
“($i$) in the case of a claim for a jobseeker’s allowance by a member of a joint-claim couple where the other member of that couple failed to attend at the time and place specified by the Secretary of State for the purposes of regulation 6.”.
\end{quotation}

(6) In regulation 21 (direct credit transfer), after paragraph (5), there shall be inserted the following paragraph—
\begin{quotation}
“(5A) In relation to payment of a joint-claim jobseeker’s allowance, references in this regulation to the person entitled to benefit shall be construed as references to the member of the joint-claim couple who is the nominated member for the purposes of section 3B of the Jobseekers Act.”.
\end{quotation}

(7) In regulation 30 (payments on death)—
\begin{enumerate}\item[]
($a$) in paragraph (2), for the words “paragraph (4)” there shall be substituted the words “paragraphs (4) and (4A)”;

($b$) after paragraph (4) there shall be inserted the following paragraph—
\begin{quotation}
\begin{sloppypar}
“(4A) In a case where a joint-claim jobseeker’s allowance has been awarded to a joint-claim couple and one member of that couple dies, the amount payable under that award shall be payable to the other member of that couple.”.
\end{sloppypar}
\end{quotation}
\end{enumerate}

(8) Regulation 34 (payment to another person on the beneficiary’s behalf) shall be renumbered regulation 34(1) and—
\begin{enumerate}\item[]
($a$) at the beginning of the renumbered regulation 34(1) there shall be inserted the words “Except in a case to which paragraph (2) applies,”;

($b$) after the renumbered paragraph (1) there shall be added the following paragraph—
\begin{quotation}
“(2) The Secretary of State may direct that a joint-claim jobseeker’s allowance shall be paid wholly or in part to a natural person who is not the member of the joint-claim couple who is the nominated member for the purposes of section 3B of the Jobseekers Act if such a direction as to payment appears to the Secretary of State to be necessary for protecting the interests of the other member of that couple or, as the case may be, both members of that couple.”.
\end{quotation}
\end{enumerate}

\subsection[3. Amendment of the Social Security (Work-focused Interviews) Regulations 2000]{\sloppy Amendment of the Social Security (Work-focused Interviews) Regulations 2000}

3.  In regulation 5 of the Social Security (Work-focused Interviews) Regulations 2000\footnote{S.I.\ 2000/897.} (exemptions)—
\begin{enumerate}\item[]
($a$) in paragraph (1), at the beginning of both sub-paragraphs ($b$)  and ($c$), there shall be inserted the words “except in a case to which paragraph (1A) applies,”;

($b$) after paragraph (1) there shall be inserted the following paragraph—
\begin{quotation}
“(1A) Notwithstanding paragraph (1)($b$)  and ($c$), a claim for a specified benefit shall give rise to an interview under regulation 4 where—
\begin{enumerate}\item[]
($a$) at the time the claim is made, the person making the claim is a member of a joint-claim couple as defined for the purposes of the Jobseeker’s Allowance Regulations 1996\footnote{S.I.\ 1996/207; Schedule A1 was inserted by S.I.\ 2000/1978.}; and

($b$) it has been decided that that person is a person to whom a paragraph of Schedule A1 to those Regulations applies (categories of members of joint-claim couples who are not required to satisfy the conditions in section 1(2B)($b$)  of the Jobseekers Act 1995).”.
\end{enumerate}
\end{quotation}
\end{enumerate}

\subsection[4. Amendment of the Housing Benefit (General) Regulations 1987 and of the Council Tax Benefit (General) Regulations 1992]{Amendment of the Housing Benefit (General) Regulations 1987 and of the Council Tax Benefit (General) Regulations 1992}

4.  In regulation 2(3A) of both the Housing Benefit (General) Regulations 1987\footnote{S.I.\ 1987/1971; regulation 2(3A) was inserted by S.I.\ 1996/1510.} and the Council Tax Benefit (General) Regulations 1992\footnote{S.I.\ 1992/1814; regulation 2(3A) was inserted by S.I.\ 1996/1510.} (interpretation)—
\begin{enumerate}\item[]
($a$) after “19” in both places where that figure occurs there shall be inserted the words “or 20A”;

($b$) after sub-paragraph ($b$), there shall be added the following sub-paragraph—
\begin{quotation}
“($c$) in respect of which he is a member of a joint-claim couple for the purposes of the Jobseekers Act 1995 and no joint-claim jobseeker’s allowance is payable in respect of that couple as a consequence of either member of that couple being subject to sanctions for the purposes of section 20A of that Act.”.
\end{quotation}
\end{enumerate}

\subsection[5. Amendment of the Social Security and Child Support (Decisions and Appeals) Regulations 1999]{Amendment of the Social Security and Child Support (Decisions and Appeals) Regulations 1999}

5.  In the Social Security and Child Support (Decisions and Appeals) Regulations 1999\footnote{S.I.\ 1999/991; regulations 6(2)($f$) and 7(8) were substituted by S.I.\ 1999/2677.}—
\begin{enumerate}\item[]
($a$) in regulation 3(6) (revision of decisions), after “19” there shall be inserted “or 20A”;

($b$) in regulation 6(2)($f$)  (supersession of decisions), after the word “Act” there shall be added the words “or ceases to be payable or is reduced by virtue of section 20A(5) of that Act”;

($c$) in regulation 7(8) (date from which a decision superseded under section 10 takes effect)—
\begin{enumerate}\item[]
(i) in sub-paragraph ($a$), after “19(2)” there shall be inserted the words “or 20A(3)”;

(ii) in sub-paragraph ($b$) ; after “19(3)” there shall be inserted the words “or 20A(4)”.
\end{enumerate}
\end{enumerate}

\subsection[6. Amendment of the Social Security Benefit (Members of the Forces) Regulations 1975]{Amendment of the Social Security Benefit (Members of the Forces) Regulations 1975}

6.  In regulation 3 of the Social Security Benefit (Members of the Forces) Regulations 1975\footnote{S.I.\ 1975/493; the relevant amending instrument is S.I.\ 1996/207.} (unemployment benefit)—
\begin{enumerate}\item[]
($a$) in paragraph (1), after “19” there shall be inserted the words “or 20A”;

($b$) in paragraph (2), after the words “Section 19(6)($b$)  and ($d$)” there shall be inserted the words “or section 20A(2)($e$)  and ($g$)”.
\end{enumerate}

\bigskip

Signed 
by authority of the Secretary of State for Social Security.

{\raggedleft
\emph{Angela Eagle}\\*Parliamentary Under-Secretary of State,\\*Department of Social Security

}

24th July 2000

\small

\part{Explanatory Note}

\renewcommand\parthead{— Explanatory Note}

\subsection*{(This note is not part of the Regulations)}

The Regulations contained in this Instrument are made in consequence of provisions in section 59 of, and Schedule 7 to, the Welfare Reform and Pensions Act 1999 (c.\ 30). The Instrument is made before the end of the period of six months beginning with the coming into force of those provisions; the regulations in it are therefore exempted from the requirement in section 172(1) of the Social Security Administration Act 1992 (c.\ 5) to refer proposals to make these Regulations to the Social Security Advisory Committee and are made without reference to that Committee.

In particular, regulation 2 of these Regulations amends the Social Security (Claims and Payments) Regulations 1987 (S.I.\ 1987/1968) to reflect new provisions in the Jobseekers Act 1995 (c.\ 18) whereby certain couples (“joint-claim couples”) may only be entitled to a jobseeker’s allowance if they make a claim for it jointly and both satisfy the conditions for entitlement in that Act. Regulation 2(2) inserts definitions for the purpose of those Regulations. Regulation 2(3) prescribes rules relating to the making of claims by joint-claim couples and regulation 2(4) prescribes the date on which such claims are to be treated as made. Regulation 2(5) prescribes when a member of a joint-claim couple may have a claim backdated and regulation 2(6) to (8) prescribes special rules relating to the payment of jobseeker’s allowance to a joint-claim couple.

Regulation 3 amends the Social Security (Work-focused Interviews) Regulations 2000 (S.I.\ 2000/897) so as to provide that a member of a joint-claim couple who is claiming a benefit specified for the purposes of those Regulations, shall be subject to an interview where it has been decided that that member is a person to whom Schedule A1 to the Jobseeker’s Allowance Regulations 1996 (S.I.\ 1996/207) applies categories of members of joint-claim couples not required to comply with the jobseeking conditions).

Regulation 4 makes consequential amendments to the Housing Benefit (General) Regulations 1987 (S.I.\ 1987/1971) and to the Council Tax Benefit (General) Regulations 1992 (S.I.\ 1992/1814), regulation 5 makes consequential amendments to the Social Security and Child Support (Decisions and Appeals) Regulations 1999 (S.I.\ 1999/991) and regulation 6 makes consequential amendments to the Social Security Benefit (Members of the Forces) Regulations 1975 (S.I.\ 1975/493).

These Regulations do not impose a charge on businesses. 

\end{document}