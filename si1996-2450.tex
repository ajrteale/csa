\documentclass[a4paper]{article}

\usepackage[welsh,english]{babel}

\usepackage[utf8]{inputenc}
\usepackage[T1]{fontenc}

\usepackage{textcomp}

%\usepackage[2012rules]{optional}

\usepackage[sc,osf]{mathpazo}

\usepackage{perpage} %the perpage package
\MakePerPage{footnote} %the perpage package command
\renewcommand{\thefootnote}{\fnsymbol{footnote}}

\usepackage[perpage,para,symbol]{footmisc}

%\opt{newrules}{
\title{The Social Security (Adjudication) and Child Support Amendment (No. 2) Regulations 1996}
%}

%\opt{2012rules}{
%\title{Child Maintenance and Other Payments Act 2008\\(2012 scheme version)}
%}

\author{S.I. 1996 No. 2450}

\date{Made 23rd September 1996\\Laid before Parliament 27th September 1996\\Coming into force 21st October 1996}

%\opt{oldrules}{\newcommand\versionyear{1993}}
%\opt{newrules}{\newcommand\versionyear{2003}}
%\opt{2012rules}{\newcommand\versionyear{2012}}

\usepackage{fancyhdr}
\pagestyle{fancy}
\fancyhead[L]{}
\fancyhead[C]{\itshape The Social Security (Adjudication) and Child Support Amendment (No. 2) Regulations 1996 (S.I.~1996/2450) \parthead%\phantom{...}% (\versionyear{} scheme version)
}
\fancyhead[R]{}
\fancyfoot[C]{\thepage}
\newcommand{\parthead}{}

\usepackage{array}
\usepackage{multirow}
\usepackage[debugshow]{tabulary}
\usepackage{longtable}
\usepackage{multicol}
\usepackage{lettrine}

\usepackage[colorlinks=true]{hyperref}
\usepackage{microtype}

\hyphenation{Aw-dur-dod}
\hyphenation{bank-rupt-cy}
\hyphenation{Ec-cles-ton}
\hyphenation{Eux-ton}
\hyphenation{Hogh-ton}
\hyphenation{Pres-ton}
\hyphenation{Pru-den-tial}
\hyphenation{Riv-ing-ton}

\newcolumntype{x}[1]
	{>{\raggedright}p{#1}}
\newcommand{\tn}{\tabularnewline}
\setlength\tymin{50pt}

\newcommand\amendment[1]{\subsubsection*{Notes}{\itshape\frenchspacing\footnotesize #1 \par}}

\setlength\headheight{22.87003pt}

\newcommand\fnote[1]{\footnote{\frenchspacing #1}}

\begin{document}

\maketitle

\noindent
The Secretary of State for Social Security, in exercise of the powers conferred by sections 21(2) and (3), 51 and 52 of the Child Support Act 1991\footnote{\frenchspacing 1991 c. 48. Section 52 was amended by paragraph 15 of Schedule 3 to the Child Support Act 1995 (c. 34).} and sections 22(2), 22(4), 33(2), 46(2), 59(1), 189 and 191 of, and paragraphs 2 to 5 of Schedule 3 to, the Social Security Administration Act 1992\footnote{\frenchspacing 1992 c. 5.}, and of all other powers enabling him in that behalf, after consultation with the Council on Tribunals in accordance with section 8(1) of the Tribunals and Inquiries Act 1992\footnote{\frenchspacing 1992 c. 53.}, hereby makes the following Regulations:

{\sloppy

\tableofcontents

}

\setcounter{secnumdepth}{-2}

\subsection[1. Citation, commencement and interpretation]{Citation, commencement and interpretation}

1.—(1) These Regulations may be cited as the Social Security (Adjudication) and Child Support Amendment (No.\ 2) Regulations 1996 and shall come into force on 21st October 1996.

(2) In these Regulations:
\begin{enumerate}\item[]
“the Adjudication Regulations” means the Social Security (Adjudication) Regulations 1995\footnote{\frenchspacing S.I. 1995/1801; the relevant amending instrument is S.I. 1996/182.};

“the Appeal Regulations” means the Child Support Appeal Tribunals (Procedure) Regulations 1992\footnote{\frenchspacing S.I. 1992/2641; the relevant amending instruments are S.I. 1995/1045 and 1996/182.}.
\end{enumerate}

\subsection[2. Amendment of regulation 1 of the Adjudication Regulations]{Amendment of regulation 1 of the Adjudication Regulations}

2.  In paragraph (2) of regulation 1 of the Adjudication Regulations (citation, commencement and interpretation) after the definition of “claimant” there shall be inserted the following definition—
\begin{quotation}
““clerk to the tribunal” means, as the case may be, a clerk to a social security appeal tribunal, a clerk to a disability appeal tribunal or a clerk to a medical appeal tribunal appointed in accordance with section 41, 43, or 50 of and paragraph 3 of Schedule 2 to, the Administration Act, or a person acting as the clerk to a medical board or a special medical board constituted in accordance with these Regulations;”.
\end{quotation}

\subsection[3. Amendment of regulation 2 of the Adjudication Regulations]{Amendment of regulation 2 of the Adjudication Regulations}

3.  After sub-paragraph ($a$) of paragraph (1) of regulation 2 of the Adjudication Regulations (procedure in connection with determinations; and right to representation) there shall be inserted the following sub-paragraphs—
\begin{quotation}
“($aa$) the chairman of a tribunal or board may give directions requiring any party to the proceedings to comply with any provision of these Regulations and may further at any stage of the proceedings either of his own motion or on a written application made to the clerk to the tribunal by any party to the proceedings give such directions as he may consider necessary or desirable for the just, effective and efficient conduct of the proceedings and may direct any party to provide such further particulars or to produce such documents as may reasonably be required;

($ab$) where under these Regulations the clerk to the tribunal is authorised to take steps in relation to the procedure of the tribunal or board, he may give directions requiring any party to the proceedings to comply with any provision of these Regulations;”.
\end{quotation}

\subsection[4. Amendment of regulation 3 of the Adjudication Regulations]{Amendment of regulation 3 of the Adjudication Regulations}

4.—(1) Regulation 3 of the Adjudication Regulations (manner of making applications, appeals or references; and time limits) shall be amended in accordance with the following provisions of this regulation.

(2) In paragraph (1) after the words “in writing” there shall be inserted the words “and, in the case of an appeal, shall be on a form approved by the Secretary of State”.

(3) For paragraph (3) there shall be substituted the following paragraph—
\begin{quotation}
“(3) The time specified by this regulation and Schedule 2 for the making of any application, appeal or reference (except an application to the chairman of an appeal tribunal, a medical appeal tribunal or a disability appeal tribunal for leave to appeal to a Commissioner) may be extended, even though the time so specified may already have expired—
\begin{enumerate}\item[]
($a$) in the case of an application or reference, for special reasons;

($b$) in the case of an appeal, provided the conditions set out in paragraphs (3A) to (3E) are satisfied; and any application for an extension of time under this paragraph shall be made to and determined by the person or body to whom the application, appeal or reference is sought to be made or, in the case of a tribunal or board, its chairman.”.
\end{enumerate}
\end{quotation}

(4) For paragraph (5)\footnote{\frenchspacing Paragraph (5) as previously substituted by regulation 2 of S.I. 1996/182.} there shall be substituted the following paragraph—
\begin{quotation}
“(5) Any application, appeal or reference under these Regulations shall contain the following particulars—
\begin{enumerate}\item[]
($a$) in the case of an appeal, the date of the notification of the decision against which the appeal is made, the claim or question under the Acts to which the decision relates, and a summary of the arguments relied on by the person making the appeal to support his contention that the decision was wrong;

($b$) in the case of an application under paragraph (3) for an extension of time in which to appeal, in relation to the appeal which it is proposed to bring, the particulars required under sub-paragraph ($a$) together with particulars of the special reasons on which the application is based;

($c$) in the case of any other application or any reference, the grounds on which it is made or given.”.
\end{enumerate}
\end{quotation}

(5) At the end of paragraph (5) there shall be inserted the following paragraph—
\begin{quotation}
“(5A) Where an appeal is not made on the form approved for the time being, but is made in writing and contains all the particulars required under paragraph (5), the chairman of the tribunal may treat that appeal as duly made.”.
\end{quotation}

(6) For paragraph (6) there shall be substituted the following paragraphs—
\begin{quotation}
“(6) Where it appears—
\begin{enumerate}\item[]
($a$) to the chairman of a tribunal or board or the clerk to the tribunal that an application, appeal or reference which is made to him or to the tribunal or board; or

($b$) to the Secretary of State or an adjudication officer that an application or reference which is made to him,
\end{enumerate}
does not contain the particulars required under paragraph (5), he may direct the person making the application, appeal or reference to provide such particulars.

(6A) Where further particulars are required under paragraph (6), the chairman of the tribunal or board, the clerk to the tribunal, the Secretary of State or the adjudication officer, as the case may be, may extend the time specified by this regulation and Schedule 2 for making the application, appeal or reference by a period of not more than 14 days.

(6B) Where further particulars are required under paragraph (6), in the case of an appeal they shall be sent or delivered to the clerk to the tribunal within such period as the chairman or the clerk to the tribunal may direct.

(6C) The date of an appeal shall be the date on which all the particulars required under paragraph (5) are received by the clerk to the tribunal.”.
\end{quotation}

\subsection[5. Amendment of regulation 4 of the Adjudication Regulations]{Amendment of regulation 4 of the Adjudication Regulations}

5.—(1) Regulation 4 of the Adjudication Regulations (oral hearings and inquiries) shall be amended in accordance with the following provisions of this regulation.

(2) In paragraph (2) for the words “Reasonable notice (being not less than 10 days” there shall be substituted the words “Except where paragraph (2C) applies, not less than 7 days notice”.

(3) After paragraph (2) there shall be inserted the following paragraphs—
\begin{quotation}
“(2A) The chairman of an appeal tribunal, a medical appeal tribunal or a disability appeal tribunal may give notice for the determination forthwith, in accordance with the provisions of these Regulations, of an appeal notwithstanding that a party to the proceedings has failed to indicate his availability for a hearing or to provide all the information which may have been requested, if the chairman is satisfied that such party—
\begin{enumerate}\item[]
($a$) has failed to comply with a direction regarding his availability or requiring information under regulation 2(1)($aa$) or ($ab$); and

($b$) has not given any explanation for his failure to comply with such a direction; provided that the chairman is satisfied that the tribunal has sufficient particulars in order for the appeal to be determined.
\end{enumerate}

(2B) The chairman of an appeal tribunal, a medical appeal tribunal or a disability appeal tribunal may give notice for the determination forthwith, in accordance with the provisions of these Regulations, of an appeal which he believes has no reasonable prospect of success.

(2C) Any party to the proceedings may waive his right to receive not less than 7 days notice of the time and place of any oral hearing as specified in paragraph (2).”.
\end{quotation}

(4) In paragraph (3)—
\begin{enumerate}\item[]
($a$) for the words “shall fail to appear” there shall be substituted the words “fails to appear”;

($b$) after the words “including any explanation offered for the absence” there shall be inserted the words “and where applicable the circumstances set out in sub-paragraphs ($a$) or ($b$) of paragraph (2A)”.

($c$) for the words “proceed with the case of inquiry” there shall be substituted the words “proceed with the hearing or inquiry”.
\end{enumerate}

(5) After paragraph (3) there shall be inserted the following paragraph—
\begin{quotation}
“(3A) If a party to the proceedings has waived his right to be given notice under paragraph (2C) the adjudicating authority or the person holding the inquiry or hearing may proceed with the hearing or inquiry notwithstanding his absence.”.
\end{quotation}

\subsection[6. Amendment of regulation 5 of the Adjudication Regulations]{Amendment of regulation 5 of the Adjudication Regulations}

6.—(1) Regulation 5 of the Adjudication Regulations (postponement and adjournment) shall be amended in accordance with the following provisions of this regulation.

(2) For paragraph (1) there shall be substituted the following paragraph—
\begin{quotation}
“(1) Where a person to whom notice of an oral hearing or inquiry has been given wishes to request a postponement of that hearing or inquiry—
\begin{enumerate}\item[]
($a$) in the case of an oral hearing by an adjudicating authority, he shall do so in writing to the clerk to the tribunal stating his reasons for the request, and the clerk to the tribunal may grant or refuse the request as he thinks fit or may pass the request to the chairman, who may grant or refuse the request as he thinks fit;

($b$) in the case of an inquiry, he shall do so in writing to the person appointed to hold the inquiry stating his reasons for the request, and the person appointed may grant or refuse the request as he thinks fit.”.
\end{enumerate}
\end{quotation}

(3) In paragraph (2), there shall be inserted after the words “A chairman” the words “or the clerk to the tribunal”.

\subsection[7. Amendment of regulation 6 of the Adjudication Regulations]{Amendment of regulation 6 of the Adjudication Regulations}

7.—(1) Regulation 6 of the Adjudication Regulations (withdrawal of applications, appeals and references) shall be amended in accordance with the following provisions of this regulation.

(2) For sub-paragraph ($a$) of paragraph (2) there shall be substituted the following sub-paragraph—
\begin{quotation}
“($a$) before the hearing begins, provided that, in the case of a tribunal or board, the clerk to the tribunal has not received any notice under paragraph (2A), by giving written notice of intention to withdraw to the adjudicating authority to whom the appeal was made and with the consent in writing of any other party to the proceedings other than—
\begin{enumerate}\item[]
(i) in a case which originated in a decision of an adjudication officer, an adjudication officer;

(ii) in any other case, the Secretary of State; or”.
\end{enumerate}
\end{quotation}

(3) After paragraph (2) there shall be inserted the following paragraph—
\begin{quotation}
“(2A) An appeal to a tribunal or board shall not be withdrawn under sub-paragraph ($a$) of paragraph (2) if the clerk to the tribunal has previously received notice opposing a withdrawal of such appeal from—
\begin{enumerate}\item[]
($a$) in a case which originated in a decision of an adjudication officer, an adjudication officer; or

($b$) in any other case, the Secretary of State.”.
\end{enumerate}
\end{quotation}

\subsection[8. Amendment of regulation 7 of the Adjudication Regulations]{Amendment of regulation 7 of the Adjudication Regulations}

8.—(1) Regulation 7 of the Adjudication Regulations (striking-out of proceedings for want of prosecution) shall be amended in accordance with the following provisions of this regulation.

(2) In paragraph (1)—
\begin{enumerate}\item[]
($a$) there shall be inserted after the words “a direction given by the chairman” the words “or the clerk to the tribunal”;

($b$) there shall be substituted for the words “regulation 2(1)($a$)” the words “regulation 2(1)($aa$) or ($ab$)”.
\end{enumerate}

(3) After paragraph (1) there shall be inserted the following paragraphs—
\begin{quotation}
“(1A) Where the chairman decides not to strike out an appeal under paragraph (1) he shall consider whether the appeal should be determined forthwith in accordance with these Regulations.

(1B) Where the chairman decides that an appeal should not be determined forthwith under paragraph (1A) he shall consider whether he should make further directions with a view to expediting the hearing of the appeal.”.
\end{quotation}

(4) After paragraph (2) there shall be inserted the following paragraph—
\begin{quotation}
“(2A) Paragraph (2) shall not apply where the address of the person against whom it is proposed that an order under paragraph (1) should be made is unknown to the chairman or to the clerk to the tribunal and cannot be ascertained by reasonable enquiry.”.
\end{quotation}

(5) In paragraph (3)—
\begin{enumerate}\item[]
($a$) there shall be substituted for the words “12 months” the words “3 months”;

($b$) after the words “paragraph (1)” where they appear for the second time there shall be added the words “, if he is satisfied that the party concerned did not receive a notice under paragraph (2) and that the conditions in paragraph (2A) were not met”.
\end{enumerate}

\subsection[9. Amendment of regulation 10 of the Adjudication Regulations]{Amendment of regulation 10 of the Adjudication Regulations}

9.  After paragraph (1) of regulation 10 of the Adjudication Regulations (setting aside decisions on certain grounds) there shall be inserted the following paragraph—
\begin{quotation}
“(1A) In determining whether it is just to set aside a decision on the ground set out in paragraph (1)($b$), the adjudicating authority shall determine whether the party making the application gave notice that he wished an oral hearing to be held, and if that party did not give such notice the adjudicating authority shall not set the decision aside unless it is satisfied that the interests of justice manifestly so require.”.
\end{quotation}

\subsection[10. Amendment of regulation 22 of the Adjudication Regulations]{Amendment of regulation 22 of the Adjudication Regulations}

10.  For paragraph (1) of regulation 22 of the Adjudication Regulations (oral hearing of appeals and references) there shall be substituted the following paragraphs—
\begin{quotation}
“(1) Where an appeal or reference is made to an appeal tribunal, the clerk to the tribunal shall direct every party to the proceedings to notify him if that party wishes an oral hearing of that appeal or reference to be held.

(1A) A notification under paragraph (1) shall be in writing and shall be made within 10 days of receipt of the direction from the clerk to the tribunal or within such other period as the clerk to the tribunal or the chairman of the tribunal may direct.

(1B) Where the clerk to the tribunal receives notification in accordance with paragraph (1A) the appeal tribunal shall hold an oral hearing.

(1C) The chairman of an appeal tribunal may of his own motion require an oral hearing to be held if he is satisfied that such a hearing is necessary to enable the tribunal to reach a decision.”.
\end{quotation}

\subsection[11. Amendment of regulation 23 of the Adjudication Regulations]{Amendment of regulation 23 of the Adjudication Regulations}

11.—(1) Regulation 23 of the Adjudication Regulations (decisions of appeal tribunals) shall be amended in accordance with the following provisions of this regulation.

(2) For paragraph (2) there shall be substituted the following paragraph—
\begin{quotation}
“(2) Every decision of an appeal tribunal shall be recorded in summary by the chairman in such written form of decision notice as shall have been approved by the President, and such decision notice shall be signed by the chairman.”.
\end{quotation}

(3) For paragraph (3) there shall be substituted the following paragraphs—
\begin{quotation}
“(3) As soon as may be practicable after a case has been decided by an appeal tribunal, a copy of the decision notice made in accordance with paragraph (2) shall be sent or given to every party to the proceedings who shall also be informed of—
\begin{enumerate}\item[]
($a$) his right under paragraph (3C); and

($b$) the conditions governing appeals to a Commissioner.
\end{enumerate}

(3A) A statement of the reasons for the tribunal’s decision and of its findings on questions of fact material thereto may be given—
\begin{enumerate}\item[]
($a$) orally at the hearing; or

($b$) in writing at such later date as the chairman may determine.
\end{enumerate}

(3B) Where the statement referred to in paragraph (3A) is given orally, it shall be recorded in such medium as the chairman may determine.

(3C) A copy of the statement referred to in paragraph (3A) shall be supplied to the parties to the proceedings if requested by any of them within 21 days after the decision notice has been sent or given, and if the statement is one to which sub-paragraph ($a$) of that paragraph applies, that copy shall be supplied in such medium as the chairman may direct.

(3D) If a decision is not unanimous, the statement referred to in paragraph (3A) shall record that one of the members dissented and the reasons given by him for dissenting.”.
\end{quotation}

(4) In paragraph (4)\footnote{\frenchspacing Paragraph (4) was inserted by regulation 2 of S.I. 1996/182.} the words “(which may take the form of a transcript or tape)” shall be omitted.

\subsection[12. Amendment of regulation 29 of the Adjudication Regulations]{Amendment of regulation 29 of the Adjudication Regulations}

12.—(1) Regulation 29 of the Adjudication Regulations (procedure for disability appeal tribunals) shall be amended in accordance with the following provisions of this regulation.

(2) For paragraph (1) there shall be substituted the following paragraphs—
\begin{quotation}
“(1) Where an appeal is made to a disability appeal tribunal, the clerk to the tribunal shall direct every party to the proceedings to notify him if that party wishes an oral hearing of that appeal to be held.

(1A) A notification under paragraph (1) shall be in writing and shall be made within 10 days of receipt of the direction from the clerk to the tribunal or within such other period as the clerk to the tribunal or the chairman of the tribunal may direct.

(1B) Where the clerk to the tribunal receives notification in accordance with paragraph (1A) the disability appeal tribunal shall hold an oral hearing.

(1C) The chairman of a disability appeal tribunal may of his own motion require an oral hearing to be held if he is satisfied that such a hearing is necessary to enable the tribunal to reach a decision.”.
\end{quotation}

(3) For paragraph (5) there shall be substituted the following paragraph—
\begin{quotation}
“(5) Every decision of a disability appeal tribunal shall be recorded in summary by the chairman in such written form of decision notice as shall have been approved by the President, and such decision notice shall be signed by the chairman.”.
\end{quotation}

(4) For paragraph (6) there shall be substituted the following paragraphs—
\begin{quotation}
“(6) As soon as may be practicable after a case has been decided by a disability appeal tribunal, a copy of the decision notice made in accordance with paragraph (5) shall be sent or given to every party to the proceedings who shall also be informed of—
\begin{enumerate}\item[]
($a$) his right under paragraph (6C); and

($b$) the conditions governing appeals to a Commissioner.
\end{enumerate}

(6A) A statement of the reasons for the tribunal’s decision and of its findings on questions of fact material thereto may be given—
\begin{enumerate}\item[]
($a$) orally at the hearing; or

($b$) in writing at such later date as the chairman may determine.
\end{enumerate}

(6B) Where the statement referred to in paragraph (6A) is given orally, it shall be recorded in such medium as the chairman may determine.

(6C) A copy of the statement referred to in paragraph (6A) shall be supplied to the parties to the proceedings if requested by any of them within 21 days after the decision notice has been sent or given, and if the statement is one to which sub-paragraph ($a$) of that paragraph applies, that copy shall be supplied in such medium as the chairman may direct.

(6D) If a decision is not unanimous, the statement referred to in paragraph (6A) shall record that one of the members dissented and the reasons given by him for dissenting.”.
\end{quotation}

(5) In paragraph (7)\footnote{\frenchspacing Paragraph (7) was inserted by regulation 2 of S.I. 1996/182.} the words “(which may take the form of a transcript or a tape)” shall be omitted.

\subsection[13. Amendment of regulation 38 of the Adjudication Regulations]{Amendment of regulation 38 of the Adjudication Regulations}

13.—(1) Regulation 38 of the Adjudication Regulations (medical appeal tribunals) shall be amended in accordance with the following provisions of this regulation.

(2) For paragraph (1) there shall be substituted the following paragraphs—
\begin{quotation}
“(1) Where an appeal or reference is made to a medical appeal tribunal, the clerk to the tribunal shall direct every party to the proceedings to notify him if that party wishes an oral hearing of that appeal or reference to be held.

(1A) A notification under paragraph (1) shall be in writing and shall be made within 10 days of receipt of the direction from the clerk to the tribunal or within such other period as the clerk to the tribunal or the chairman of the tribunal may direct.

(1B) Where the clerk to the tribunal receives notification in accordance with paragraph (1A) the medical appeal tribunal shall hold an oral hearing.

(1C) The chairman of a medical appeal tribunal may of his own motion require an oral hearing to be held if he is satisfied that such a hearing is necessary to enable the tribunal to reach a decision.”.
\end{quotation}

(3) For paragraph (4) there shall be substituted the following paragraph—
\begin{quotation}
\begin{sloppypar}
“(4) Every decision of a medical appeal tribunal shall be recorded in summary by the chairman in such written form of decision notice as shall have been approved by the President, and such decision notice shall be signed by the chairman.”.
\end{sloppypar}
\end{quotation}

(4) For paragraph (5) there shall be substituted the following paragraphs—
\begin{quotation}
“(5) As soon as may be practicable after a case has been decided by a medical appeal tribunal, a copy of the decision notice made in accordance with paragraph (4) shall be sent or given to every party to the proceedings who shall also be informed of—
\begin{enumerate}\item[]
($a$) his right under paragraph (5C); and

($b$) the conditions governing appeals to a Commissioner.
\end{enumerate}

(5A) A statement of the reasons for the tribunal’s decision and of its findings on questions of fact material thereto may be given—
\begin{enumerate}\item[]
($a$) orally at the hearing; or

($b$) in writing at such later date as the chairman may determine.
\end{enumerate}

(5B) Where the statement referred to in paragraph (5A) is given orally, it shall be recorded in such medium as the chairman may determine.

(5C) A copy of the statement referred to in paragraph (5A) shall be supplied to the parties to the proceedings if requested by any of them within 21 days after the decision notice has been sent or given, and if the statement is one to which sub-paragraph ($a$) of that paragraph applies, that copy shall be supplied in such medium as the chairman may direct.

(5D) If a decision is not unanimous, the statement referred to in paragraph (5A) shall record that one of the members dissented and the reasons given by him for dissenting.”.
\end{quotation}

(5) In paragraph (6) the words “(which may take the form of a transcript or a tape)” shall be omitted.

\subsection[14. Amendment of regulation 3 of the Appeal Regulations]{Amendment of regulation 3 of the Appeal Regulations}

14.—(1) Regulation 3 of the Appeal Regulations (making an appeal or application and time limits) shall be amended in accordance with the following provisions of this regulation.

(2) For paragraph (1A)\footnote{\frenchspacing Paragraph (1A) was inserted by regulation 3 of S.I. 1995/1045.} there shall be substituted the following paragraph—
\begin{quotation}
“(1A) An appeal or application of a kind mentioned in paragraph (1) shall be by notice in writing, and, in the case of an appeal, shall be on a form approved by the Secretary of State and shall be signed by the person making it, or by his representative where it appears to a chairman that he was unable to sign it personally, or by a barrister, advocate or solicitor on his behalf.”.
\end{quotation}

(3) For paragraph (9) there shall be substituted the following paragraphs—
\begin{quotation}
“(9) A notice of appeal shall contain particulars of the date of the notification of the decision against which the appeal is made, the subject matter of the decision and a summary of the arguments relied on by the person making the appeal to support his contention that the decision was wrong.

(9A) Where the notice referred to in paragraph (9) is not made on the form approved for the time being, but is made in writing and contains all the particulars required by paragraph (9), a chairman may treat that appeal as duly made.”.
\end{quotation}

(4) In paragraph (10) the words “appeal or” shall be omitted.

(5) For paragraph (11) there shall be substituted the following paragraphs—
\begin{quotation}
“(11) Where it appears to a chairman or the clerk to the tribunal that the notice of appeal does not contain the particulars required under paragraph (9), or that the notice of application does not contain the particulars required under paragraph (10), he may direct the person making the appeal or application to furnish those further particulars.

(11A) Where further particulars are required under paragraph (11), in the case of an appeal they shall be sent or delivered to the clerk to the tribunal at the Central Office within such period as a chairman or the clerk to the tribunal may direct.

(11B) The date of an appeal or application shall be the date on which all the particulars required under paragraph (9) are received in the Central Office.”.
\end{quotation}

\subsection[15. Amendment of regulation 5 of the Appeal Regulations]{Amendment of regulation 5 of the Appeal Regulations}

15.  Regulation 5 of the Appeal Regulations (directions) shall be renumbered paragraph (1) of that Regulation and after that paragraph there shall be added the following paragraph—
\begin{quotation}
“(2) Where under these Regulations the clerk to the tribunal is authorised to take steps in relation to the procedure of the tribunal, he may give directions requiring any party to the proceedings to comply with any provision of these Regulations.”.
\end{quotation}

\subsection[16. Amendment of regulation 6 of the Appeal Regulations]{Amendment of regulation 6 of the Appeal Regulations}

16.—(1) Regulation 6 of the Appeal Regulations (striking out of proceedings) shall be amended in accordance with the following provisions of this regulation.

(2) In paragraph (1) for the words “a direction under regulation 3(11) or 5 or to reply to an enquiry from the clerk to the tribunal about his availability to attend a hearing” there shall be substituted the words “a direction under regulation 3(11), 5(1) or (2)”.

(3) After paragraph (1) there shall be inserted the following paragraphs—
\begin{quotation}
“(1A) Where a chairman decides not to strike out an appeal or application under paragraph (1) he shall consider whether the appeal or application should be determined forthwith in accordance with these Regulations.

(1B) Where a chairman decides that an appeal or application should not be determined forthwith under paragraph (1A) he shall consider whether he should make further directions with a view to expediting the hearing of the appeal or application.”.
\end{quotation}

(4) After paragraph (2) there shall be inserted the following paragraph—
\begin{quotation}
“(2A) Paragraph (2) shall not require a notice to be sent to a party, including a person against whom it is proposed that an order under paragraph (1) should be made, where his address is unknown to the chairman or the clerk to the tribunal and cannot be ascertained by reasonable enquiry.”.
\end{quotation}

(5) In paragraph (3)—
\begin{enumerate}\item[]
($a$) for the words “one year” there shall be substituted the words “3 months”;

($b$) after the words “that order” there shall be added the words “if he is satisfied that the party concerned did not receive a notice under paragraph (2) and that the conditions in paragraph (2A) were not met”.
\end{enumerate}

\subsection[17. Amendment of regulation 7 of the Appeal Regulations]{Amendment of regulation 7 of the Appeal Regulations}

17.—(1) Regulation 7 of the Appeal Regulations (withdrawal of appeals and applications) shall be amended in accordance with the following provisions of this regulation.

(2) For sub-paragraph ($b$) of paragraph (1) there shall be substituted the following sub-paragraph—
\begin{quotation}
“($b$) at any other time, provided that the clerk to the tribunal has not received notice under paragraph (1A), by giving written notice of intention to withdraw to the clerk to the tribunal and either—
\begin{enumerate}\item[]
(i) with the consent in writing of every other party to the proceedings other than the child support officer; or

(ii) with the leave of the chairman after every other party to the proceedings other than the child support officer has had a reasonable opportunity to make representations.”.
\end{enumerate}
\end{quotation}

(3) After paragraph (1) there shall be inserted the following paragraph—
\begin{quotation}
“(1A) An appeal shall not be withdrawn under sub-paragraph ($b$) of paragraph (1) if the clerk to the tribunal has previously received notice opposing a withdrawal of such appeal from the child support officer.”.
\end{quotation}

\subsection[18. Amendment of regulation 8 of the Appeal Regulations]{Amendment of regulation 8 of the Appeal Regulations}

18.—(1) Regulation 8 of the Appeal Regulations (postponement) shall be amended in accordance with the following provisions of this regulation.

(2) For paragraph (1) there shall be substituted the following paragraph—
\begin{quotation}
“(1) Where a person to whom notice of a hearing has been given wishes to request a postponement of that hearing he shall do so in writing to the clerk to the tribunal stating his reasons for the application, and the clerk to the tribunal may grant or refuse the application as he thinks fit or may pass the application to a chairman, who may grant or refuse the application as he thinks fit.”.
\end{quotation}

(3) In paragraph (2) there shall be inserted after the words “A chairman” the words “or the clerk to the tribunal”.

\subsection[19. Amendment of regulation 11 of the Appeal Regulations]{Amendment of regulation 11 of the Appeal Regulations}

19.—(1) Regulation 11 of the Appeal Regulations (hearings) shall be amended in accordance with the following provisions of this regulation.

(2) For paragraph (1) there shall be substituted the following paragraphs—
\begin{quotation}
“(1) Where an appeal or application is made to a tribunal, the clerk to the tribunal shall direct every party to the proceedings to notify him if that party wishes an oral hearing of that appeal or application to be held.

(1A) A notification under paragraph (1) shall be in writing and shall be made within 21 days of receipt of the direction from the clerk to the tribunal or within such other period as the clerk to the tribunal or a chairman may direct.

(1B) Where the clerk to the tribunal receives notification in accordance with paragraph (1A) the tribunal shall hold an oral hearing.

(1C) A chairman may of his own motion require an oral hearing to be held if he is satisfied that such a hearing is necessary to enable the tribunal to reach a decision.

(1D) Subject to the provisions of the Act and of these Regulations the procedure in connection with an oral hearing shall be such as the chairman shall determine.”.
\end{quotation}

(3) In paragraph (2)—
\begin{enumerate}\item[]
($a$) at the beginning, there shall be inserted the words “Except where paragraph (2C) applies,”;

($b$) for the words “the time and place of any hearing” there shall be substituted the words “the time and place of any oral hearing”.
\end{enumerate}

(4) After paragraph (2) there shall be inserted the following paragraphs—
\begin{quotation}
“(2A) A chairman may give notice for the determination forthwith, in accordance with the provisions of the Act and these Regulations, of an appeal or application notwithstanding that a party to the proceedings has failed to indicate his availability for a hearing or to provide all the information which may have been requested, if the chairman is satisfied that such party—
\begin{enumerate}\item[]
($a$) has failed to comply with a direction regarding his availability or requiring information under regulation 3(11), 5(1) or (2); and

($b$) has not given any explanation for his failure to comply with such a direction; provided that the chairman is satisfied that the tribunal has sufficient particulars in order for the appeal or application to be determined.
\end{enumerate}

(2B) A chairman may give notice for the determination forthwith, in accordance with the provisions of these Regulations, of an appeal or application which he believes has no reasonable prospect of success.

(2C) Any party to the proceedings may waive his right to receive not less than 10 days notice of the time and place of any oral hearing as specified in paragraph (2).”.
\end{quotation}

(5) In paragraph (6) after the words “including any explanation offered for the absence” there shall be inserted the words “and where applicable the circumstances set out in sub-paragraphs ($a$) or ($b$) of paragraph (2A)”.

(6) After paragraph (6) there shall be inserted the following paragraph—
\begin{quotation}
“(6A) Where any party to the proceedings has waived his right to be given notice under paragraph (2C) the tribunal may proceed with the hearing notwithstanding his absence.”.
\end{quotation}

\subsection[20. Amendment of regulation 13 of the Appeal Regulations]{Amendment of regulation 13 of the Appeal Regulations}

20.—(1) Regulation 13 of the Appeal Regulations (decisions) shall be amended in accordance with the following provisions of this regulation.

(2) For paragraph (2) there shall be substituted the following paragraph—
\begin{quotation}
“(2) Every decision of a tribunal shall be recorded in summary by the chairman in such written form of decision notice as shall have been approved by the President, and such decision notice shall be signed by the chairman.”.
\end{quotation}

(3) For paragraphs (3) and (3A)\footnote{\frenchspacing Paragraph (3A) was inserted by regulation 3 of S.I. 1996/182.} there shall be substituted the following paragraphs—
\begin{quotation}
“(3) As soon as may be practicable after a case has been decided by a tribunal, a copy of the decision notice made in accordance with paragraph (2) shall be sent or given to every party to the proceedings who shall also be informed of—
\begin{enumerate}\item[]
($a$) his right under paragraph (3C); and

($b$) the conditions governing appeals to a Commissioner.
\end{enumerate}

(3A) A statement of the reasons for the tribunal’s decision, of its findings on questions of fact material thereto and of the terms of any direction under section 20(4) of the Act may be given—
\begin{enumerate}\item[]
($a$) orally at the hearing; or

($b$) in writing at such later date as the chairman may determine.
\end{enumerate}

(3B) Where the statement referred to in paragraph (3A) is given orally, it shall be recorded in such medium as the chairman may determine.

(3C) A copy of the statement referred to in paragraph (3A) shall be supplied to the parties to the proceedings if requested by any of them within 21 days after the decision notice has been sent or given and if the statement is one to which sub-paragraph ($a$) of that paragraph applies, that copy shall be supplied in such medium as the chairman may direct.

(3D) If a decision is not unanimous, the statement referred to in paragraph (3A) shall record that one of the members dissented and the reasons given by him for dissenting.

(3E) A record of the proceedings at the hearing may be made by the chairman in such medium as he may direct and preserved by the clerk to the tribunal for 18 months, and a copy of such record shall be supplied to the parties if requested by any of them within that period.”.
\end{quotation}

\subsection[21. Amendment of regulation 15 of the Appeal Regulations]{Amendment of regulation 15 of the Appeal Regulations}

21.  After paragraph (1) of regulation 15 of the Appeal Regulations (setting aside) there shall be inserted the following paragraph—
\begin{quotation}
“(1A) In determining whether it is just to set aside a decision on the ground set out in paragraph (1)($b$) the tribunal shall determine whether the party making the application gave notice that he wished an oral hearing to be held, and if the party did not give such notice the tribunal shall not set the decision aside unless it is satisfied that there has been some procedural irregularity or mishap.”.
\end{quotation}

\subsection[22. Saving Provision]{Saving Provision}

22.  In a case where an appeal, application or reference was made before the date on which these Regulations come into force, regulations 3, 7(3), 22, 29(1) and 38(1) of the Adjudication Regulations and regulations 3, 6(3), and 11(1) of the Appeal Regulations shall apply as if these Regulations had not been made.

\bigskip

Signed by authority of the Secretary of State for Social Security.

{\raggedleft
\emph{Roger Evans}\\*Parliamentary Under-Secretary of State,\\*Department of Social Security

}

23rd September 1996

\bigskip

\part{Explanatory Note}

\renewcommand\parthead{--- Explanatory Note}

\subsection*{(This note is not part of the Regulations)}

These Regulations amend the Social Security (Adjudication) Regulations 1995 and the Child Support Appeal Tribunals (Procedure) Regulations 1992 to make certain changes to the procedure of social security appeal tribunals, disability appeal tribunals, medical appeal tribunals and child support appeal tribunals.

 The Social Security (Adjudication) Regulations 1995 are amended in respect of social security appeal tribunals, disability appeal tribunals and medical appeal tribunals to—
\begin{enumerate}\item[]
 ($a$) insert a new definition of “clerk to the tribunal” (regulation 2);

 ($b$) specify the circumstances in which the chairman of a tribunal or board or the clerk to the tribunal may give directions as to procedure (regulation 3);

 ($c$) include new requirements for the information to be provided in connection with an appeal (regulation 4);

 ($d$) reduce the minimum period for notice of an oral hearing, provide that in certain circumstances a tribunal chairman may direct that an appeal be determined forthwith, and provide that a party to the proceedings may waive the right to be given notice (regulation 5);

 ($e$) allow the clerk to the tribunal to deal with requests for hearings to be postponed and to postpone hearings of his own motion (regulation 6);

 ($f$) allow a person who has made an appeal to withdraw it before a hearing without consent in certain circumstances (regulation 7);

 ($g$) amend the circumstances in which a tribunal chairman may strike out an appeal for want of prosecution, the procedure for striking out, and the circumstances in which an appeal which has been struck out may be reinstated (regulation 8);

 ($h$) amend the provisions on setting aside tribunal decisions to reflect the new provisions on oral hearings (regulation 9);

 ($i$) provide that, unless the chairman of a tribunal or board orders an oral hearing, an oral hearing of an appeal or reference shall be held only on the request of a party to the proceedings, and specify how such a request is to be made (regulations 10, 12(2) and 13(2));

 ($j$) amend the provisions on the form and promulgation of tribunal decisions (regulations 11(2) and (3), 12(3) and (4) and 13(3) and (4)).
\end{enumerate}

  The Child Support Appeal Tribunals (Procedure) Regulations 1992 are amended to make broadly equivalent changes in respect of child support appeal tribunals in relation to: time and manner of making an appeal (regulation 14); directions by the chairman or the clerk to the tribunal (regulation 15); striking out (regulation 16); withdrawal (regulation 17); postponement (regulation 18); hearings including procedure for requesting an oral hearing (regulation 19); form and promulgation of decisions (regulation 20); setting aside of decisions (regulation 21).

  Regulation 22 makes a saving provision in respect of certain provisions of the Social Security (Adjudication) Regulations 1995 and the Child Support Appeal Tribunals (Procedure) Regulations 1992.

  These Regulations do not impose any costs to business.

\end{document}