\documentclass[12pt,a4paper]{article}

\newcommand\regstitle{The Child Support, Pensions and Social Security Act 2000 (Commencement No.\ 10) Order 2001}

\newcommand\regsnumber{2001/2619}

%\opt{newrules}{
\title{\regstitle}
%}

%\opt{2012rules}{
%\title{Child Maintenance and Other Payments Act 2008\\(2012 scheme version)}
%}

\author{S.I.\ 2001 No.\ 2619 (C.\ 86)}

\date{Made
18th July 2001\\
%Laid before Parliament
%10th May 2001\\
%Coming into force
%31st May 2001
}

%\opt{oldrules}{\newcommand\versionyear{1993}}
%\opt{newrules}{\newcommand\versionyear{2003}}
%\opt{2012rules}{\newcommand\versionyear{2012}}

\usepackage{csa-regs}

\setlength\headheight{27.57402pt}

\begin{document}

\maketitle

\noindent
The Secretary of State for Work and Pensions, in exercise of the powers conferred by section 86(2), (3)($a$)  and (4) of the Child Support, Pensions and Social Security Act 2000\footnote{2000 c.\ 19.} and of all other powers enabling him in that behalf, hereby makes the following Order:  

{\sloppy

\tableofcontents

}

\bigskip

\setcounter{secnumdepth}{-2}

\subsection[1. Citation and interpretation]{Citation and interpretation}

1.---(1)  This Order may be cited as the Child Support, Pensions and Social Security Act 2000 (Commencement No.\ 10) Order 2001.

(2) In this Order, unless the context otherwise requires, any reference to a numbered section is a reference to the section bearing that number in the Child Support, Pensions and Social Security Act 2000.

\subsection[2. Appointed day]{Appointed day}

2.---(1)  The day appointed for the coming into force—
\begin{enumerate}\item[]
($a$) of section 64(1) (explanation by court about the consequences of failure to comply with a relevant community order) for the purposes of its application in the case of any person of a description specified in paragraph (2)($a$)  below, and

($b$) of—
\begin{enumerate}\item[]
(i) section 62 (loss of benefit for breach of community order), except for subsection (11),

(ii) section 63 (loss of joint-claim jobseekers' allowance),

(iii) subsections (2), (4)($a$), (5), (6), (7)($a$)  to ($c$), (8) and (10) of section 64 (information provision),

(iv) section 65 (loss of benefit regulations), except for subsection (7), and

(v) section 66 (appeals relating to loss of benefit),
\end{enumerate}
\end{enumerate}
for the purposes of their application in the case of any person of a description specified in paragraph (2)($b$)  below, so far as not already brought into force, is 15th October 2001.

(2) The persons referred to in paragraph (1) above are—
\begin{enumerate}\item[]
($a$) any person who, as a result of a relevant community order (as defined in section 62(8)) being made in relation to him, falls to be supervised by an officer of the local probation board for any of the probation areas of Derbyshire, Hertfordshire, Teesside and West Midlands;

($b$) any person to whom sub-paragraph ($a$)  above applies and in relation to whom a relevant community order (as defined in section 62(8)) has been made in accordance with section 64(1).
\end{enumerate}

\bigskip

Signed 
by authority of the Secretary of State for Work and Pensions.

{\raggedleft
\emph{Malcolm Wicks}\\*Parliamentary Under-Secretary of State,\\*Department of Work and Pensions

}

%Dated
18th July 2001

\small

\part{Explanatory Note}

\renewcommand\parthead{— Explanatory Note}

\subsection*{(This note is not part of the Order)}

This Order brings into force further provisions of the Child Support, Pensions and Social Security Act 2000 (c.\ 19).

Sections 62 to 66 (loss of benefit for breach of community order) are brought fully into force on 15th October 2001 for the purposes of their application to persons in relation to whom specified community orders are made, and who fall to be supervised in the probation areas of Derbyshire, Hertfordshire, Teesside or West Midlands (article 2(1) and (2)).

The impact on business of provisions in the Child Support, Pensions and Social Security Act 2000 is detailed in the Regulatory Impact Assessment relating to the Child Support, Pensions and Social Security Bill (which was introduced in the House of Commons on 1st December 1999). A copy of that Assessment has been placed in the libraries of both Houses of Parliament and can be obtained from the Department for Work and Pensions, Regulatory Impact Unit, 3rd Floor, The Adelphi, 1--11 John Adam Street, London \textsc{\lowercase{WC2N 6HT}}. 

\end{document}