\documentclass[12pt,a4paper]{article}

\newcommand\regstitle{The Social Security (Child Maintenance Premium and Miscellaneous Amendments) Regulations 2000}

\newcommand\regsnumber{2000/3176}

%\opt{newrules}{
\title{\regstitle}
%}

%\opt{2012rules}{
%\title{Child Maintenance and Other Payments Act 2008\\(2012 scheme version)}
%}

\author{S.I. 2000 No. 3176}

\date{Made 30th November 2000\\Laid before Parliament 6th December 2000\\Coming into force in accordance with regulation 1}

%\opt{oldrules}{\newcommand\versionyear{1993}}
%\opt{newrules}{\newcommand\versionyear{2003}}
%\opt{2012rules}{\newcommand\versionyear{2012}}

\usepackage{csa-regs}

\setlength\headheight{27.57402pt}

\begin{document}

\maketitle

\noindent
The Secretary of State for Social Security, in exercise of the powers conferred upon him by sections 123(1)($a$), ($d$)  and ($e$), 136(3) and (5)($b$), 137(1) and 175(1) and (3) of the Social Security Contributions and Benefits Act 1992\footnote{\frenchspacing 1992 c. 4; section 123(1)($e$)  was substituted by the Local Government Finance Act 1992 (c. 14), Schedule 9, paragraph 1(1); section 137(1) is an interpretation provision and is cited because of the meaning ascribed to the word “prescribed”.}, sections 12(1) and (4)($b$), 35(1) and 36(1), (2) and (4) of the Jobseekers Act 1995\footnote{\frenchspacing 1995 c. 18; section 35(1) is an interpretation provision and is cited because of the meaning ascribed to the words “prescribed” and “regulations”.}, sections 10 and 26(1) to (3) of the Child Support Act 1995\footnote{\frenchspacing 1995 c. 34; the meaning ascribed to the word “prescribed” is given in section 54 of the Child Support Act 1991 (c. 48) which is applied to section 10 of the Child Support Act 1995 by section 27(2) of that Act.} and section 87(4) of the Northern Ireland Act 1998\footnote{\frenchspacing 1998 c. 47.}, and of all other powers enabling him in that behalf, after consultation, in respect of regulation 3 of these Regulations, with organisations appearing to him to be representative of the authorities concerned\footnote{\frenchspacing \emph{See} section 176(1) of the Social Security Administration Act 1992 (c. 5).} and after agreement by the Social Security Advisory Committee that proposals in respect of these Regulations should not be referred to it\footnote{\frenchspacing \emph{See} sections 170 and 173(1)($b$) of the Social Security Administration Act 1992 (c. 5); paragraph 67 of Schedule 2 to the Jobseekers Act 1995 added that Act to the list of “relevant enactments” in respect of which regulations must normally be referred to the Committee; paragraph 20 of Schedule 3 to the Child Support Act 1995 added section 10 of that Act to that list.}, hereby makes the following Regulations:  

{\sloppy

\tableofcontents

}

\bigskip

\setcounter{secnumdepth}{-2}

\subsection[1. Citation and commencement]{Citation and commencement}

1.  These Regulations shall be cited as the Social Security (Child Maintenance Premium and Miscellaneous Amendments) Regulations 2000 and shall come into force, in relation to any particular case, on the date on which section~23 of the Child Support, Pensions and Social Security Act 2000\footnote{\frenchspacing 2000 c. 19.} comes into force in relation to that type of case (“the commencement date”).

\subsection[2. Child maintenance: income support and jobseeker’s allowance]{\sloppy Child maintenance: income support and jobseeker’s allowance}

2.---(1)  In the Income Support (General) Regulations 1987\footnote{\frenchspacing S.I. 1987/1967.}—
\begin{enumerate}\item[]
($a$) at the end of the definition of “child support maintenance” in regulation 60A\footnote{\frenchspacing Regulation 60A was inserted by S.I. 1993/846.} (child support: interpretation) there shall be added the words “and shall include any payments made by the Secretary of State in lieu of such payments”; and

($b$) at the end of Schedule 9\footnote{\frenchspacing Paragraph 72 was inserted by S.I. 2000/724.} (sums to be disregarded in the calculation of income other than earnings) there shall be added the following paragraph—
\begin{quotation}
“73.---(1)  Subject to sub-paragraph (3), any payment of child maintenance, whether under a court order or not, which is made or due to be made by the parent of a child or young person where that child or young person is a member of the claimant’s family except where that parent is the claimant or the claimant’s partner.

(2) For the purposes of sub-paragraph (1), where more than one payment of child maintenance falls to be taken into account in any week, all such payments shall be aggregated and treated as if they were a single payment.

(3) No more than £10 shall be disregarded in any week pursuant to this paragraph.

(4) In this paragraph, “child maintenance” shall have the same meaning as that prescribed for the purposes of section 74A of the Social Security Administration Act 1992\footnote{\frenchspacing 1992 c. 5. Section 74A was inserted by section 25 of the Child Support Act 1995 (c. 34). “Child maintenance” is defined for the purposes of section 74A in regulation 2 of the Social Security Benefits (Maintenance Payments and Consequential Amendments) Regulations 1996 (S.I. 1996/940).} and shall include any payment made by the Secretary of State in lieu of such maintenance.”.
\end{quotation}
\end{enumerate}

(2) In the Jobseeker’s Allowance Regulations 1996\footnote{\frenchspacing S.I. 1996/207.}—
\begin{enumerate}\item[]
($a$) at the end of the definition of “child support maintenance” in regulation 125 (child support: interpretation) there shall be added the words “and shall include any payments made by the Secretary of State in lieu of such payments”; and

($b$) at the end of Schedule 7\footnote{\frenchspacing Paragraph 69 was inserted by S.I. 2000/724.} (sums to be disregarded in the calculation of income other than earnings) there shall be added the following paragraph—
\begin{quotation}
“70.---(1)  Subject to sub-paragraph (3), any payment of child maintenance, whether under a court order or not, which is made or due to be made by the parent of a child or young person where that child or young person is a member of the claimant’s family except where that parent is the claimant or the claimant’s partner.

(2) For the purposes of sub-paragraph (1), where more than one payment of child maintenance falls to be taken into account in any week, all such payments shall be aggregated and treated as if they were a single payment.

(3) No more than £10 shall be disregarded in any week pursuant to this paragraph.

(4) In this paragraph, “child maintenance” shall have the same meaning as that prescribed for the purposes of section~74A of the Administration Act and shall include any payment made by the Secretary of State in lieu of such maintenance.”.
\end{quotation}
\end{enumerate}

\amendment{
Reg. 3 revoked (6.3.06) by the Housing Benefit and Council Tax Benefit (Consequential Provisions) Regulations 2006 Sch. 1.
}

% Reg 3 revoked by SI 2006/217 Sch 1
%\subsection[3. Housing benefit and council tax benefit]{Housing benefit and council tax benefit}
%
%3.  At the end of both paragraph 46 of Schedule 4 to the Council Tax Benefit (General) Regulations 1992\footnote{\frenchspacing S.I. 1992/1814; paragraph 46 was amended by S.I. 1996/803.} and paragraph 47 of Schedule 4 to the Housing Benefit (General) Regulations 1987\footnote{\frenchspacing S.I. 1987/1971; paragraph 47 was inserted by S.I. 1991/2695 and amended by S.I. 1996/1803.} (sums to be disregarded in the calculation of income other than earnings), there shall be added the following sub-paragraph—
%\begin{quotation}
%“(3) A payment made by the Secretary of State in lieu of maintenance shall, for the purposes of sub-paragraph (1), be treated as a payment of maintenance made by a person specified in paragraph ($a$)  or ($b$)  of that sub-paragraph.”.
%\end{quotation}

\subsection[4. Revocations and transitional provisions]{Revocations and transitional provisions}

4.---(1)  Subject to paragraphs (2) to (4) below—
\begin{enumerate}\item[]
($a$) regulations 2 to 13 of the Social Security (Child Maintenance Bonus) Regulations 1996\footnote{\frenchspacing S.I. 1996/3195.} (“the Child Maintenance Bonus Regulations”);

($b$) the Child Maintenance Bonus (Northern Ireland Reciprocal Arrangements) Regulations 1997\footnote{\frenchspacing S.I. 1997/645.} (“the Reciprocal Arrangements Regulations”);

($c$) regulation 8 of the Social Security (Miscellaneous Amendments) Regulations 1997\footnote{\frenchspacing S.I. 1997/454.}; and

($d$) regulation 2 of the Social Security (Miscellaneous Amendments) Regulations 1998\footnote{\frenchspacing S.I. 1998/563.},
\end{enumerate}
are hereby revoked.

(2) Subject to paragraph (3) below, regulations 2 to 13 of the Child Maintenance Bonus Regulations and the Reciprocal Arrangements Regulations shall continue to have effect as if paragraph (1) above had not been made in relation to a person who—
\begin{enumerate}\item[]
($a$) claimed a child maintenance bonus before the commencement date but whose claim was not determined until on or after that date; or

($b$) claims a child maintenance bonus on or after the commencement date but within the time specified in regulations 3(1)($f$)\footnote{\frenchspacing Regulation 3(1)($f$) was substituted by S.I. 1998/563.}, 10(1) and, where applicable, 11(4) of the Child Maintenance Bonus Regulations.
\end{enumerate}

(3) For the purposes of paragraph (2) above, regulation 3(1)($f$)(iii) of the Child Maintenance Bonus Regulations shall have effect as if for the words “14 days” there were substituted the words “one month”.

(4) Nothing in this regulation shall prevent the Secretary of State from issuing a written statement pursuant to regulation 6(1) of the Child Maintenance Bonus Regulations to a person who appears to him to satisfy the requirements of regulation 3 of those Regulations. 

\bigskip

Signed 
by authority of the Secretary of State for Social Security.

{\raggedleft
\emph{P.~Hollis}\\*Parliamentary Under-Secretary of State,\\*Department of Social Security

}

30th November 2000

\small

\part{Explanatory Note}

\renewcommand\parthead{--- Explanatory Note}

\subsection*{(This note is not part of the Regulations)}

These Regulations come into force at different times for different cases according to the dates on which section 23 of the Child Support, Pensions and Social Security Act 2000 (c.\ 19) which is relevant to these Regulations, is commenced for different types of cases.

Regulations 2 and 3 of these Regulations amend the Income Support (General) Regulations 1987 (S.I.\ 1987/1967), the Jobseeker’s Allowance Regulations 1996 (S.I.\ 1996/207), the Housing Benefit (General) Regulations 1987 (S.I.\ 1987/1971) and the Council Tax Benefit (General) Regulations 1992 (S.I.\ 1992/1814).

In particular, regulation 2(1)($b$)  and (2)($b$)  of these Regulations provides that for the purpose of ascertaining entitlement to income support and jobseeker’s allowance, up to £10 of a payment of child maintenance shall be disregarded. That disregarded amount is known as the child maintenance premium. Child maintenance is defined for these purposes as a payment prescribed for the purposes of section 74A of the Social Security Administration Act 1992 (c.\ 5) and payments by the Secretary of State in lieu of such maintenance.

Regulation 2(1)($a$)  and (2)($a$)  provides that for the purpose of calculating the weekly amount of child support maintenance, payments by the Secretary of State in lieu of periodical payments of child support maintenance shall be treated as payments of child support maintenance.

Regulation 3 provides that payments by the Secretary of State in lieu of child maintenance shall, for the purpose of ascertaining entitlement to housing benefit and council tax benefit, be treated as if they were payments of maintenance paid by a former partner of the claimant or his partner or by a parent of a child or young person.

Regulation 4 revokes, with transitional provisions, regulations 2 to 13 of the Social Security (Child Maintenance Bonus) Regulations 1996 (S.I.\ 1996/3195), the Child Maintenance Bonus (Northern Ireland Reciprocal Arrangements) Regulations 1997 (S.I.\ 1997/645), regulation 8 of the Social Security (Miscellaneous Amendments) Regulations 1997 (S.I.\ 1997/454) and regulation 2 of the Social Security (Miscellaneous Amendments) Regulations 1998 (S.I.\ 1998/563).

These Regulations do not impose a charge on business.  

\end{document}