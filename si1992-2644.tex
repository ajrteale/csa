\documentclass[a4paper]{article}

\usepackage[welsh,english]{babel}

\usepackage[utf8]{inputenc}
\usepackage[T1]{fontenc}

\usepackage{textcomp}

%\usepackage[2012rules]{optional}

\usepackage[osf]{mathpazo}
\usepackage{cfr-lm}

\usepackage{perpage} %the perpage package
\MakePerPage{footnote} %the perpage package command
\renewcommand{\thefootnote}{\fnsymbol{footnote}}

\usepackage[perpage,para,symbol]{footmisc}

%\opt{newrules}{
\title{The Child Support Act 1991 (Commencement No.\ 3 and Transitional Provisions) Order 1992}
%}

%\opt{2012rules}{
%\title{Child Maintenance and Other Payments Act 2008\\(2012 scheme version)}
%}

\author{S.I. 1992 No. 2644 (C.83)}

\date{Made 26th October 1992}

%\opt{oldrules}{\newcommand\versionyear{1993}}
%\opt{newrules}{\newcommand\versionyear{2003}}
%\opt{2012rules}{\newcommand\versionyear{2012}}

\usepackage{fancyhdr}
\pagestyle{fancy}
\fancyhead[L]{}
\fancyhead[C]{\itshape The Child Support Act 1991 (Commencement No.\ 3 and Transitional Provisions) Order 1992 (S.I.~1992/2644) \parthead%\phantom{...}% (\versionyear{} scheme version)
}
\fancyhead[R]{}
\fancyfoot[C]{\thepage}
\newcommand{\parthead}{}

\usepackage{array}
\usepackage{multirow}
\usepackage[debugshow]{tabulary}
\usepackage{longtable}
\usepackage{multicol}
\usepackage{lettrine}

\usepackage[colorlinks=true]{hyperref}
\usepackage{microtype}

\hyphenation{Aw-dur-dod}
\hyphenation{bank-rupt-cy}
\hyphenation{Ec-cles-ton}
\hyphenation{Eux-ton}
\hyphenation{Hogh-ton}
\hyphenation{Pres-ton}
\hyphenation{Pru-den-tial}
\hyphenation{Riv-ing-ton}

\newcolumntype{x}[1]
	{>{\raggedright}p{#1}}
\newcommand{\tn}{\tabularnewline}
\setlength\tymin{50pt}

\newcommand\amendment[1]{\subsubsection*{Notes}{\itshape\frenchspacing\footnotesize #1 \par}}

\setlength\headheight{22.91502pt}

\begin{document}

\maketitle

\noindent
The Secretary of State for Social Security, in exercise of the powers conferred upon him by section 58(2) to (6) of the Child Support Act 1991\footnote{\frenchspacing 1991 c. 48.}, hereby makes the following Order:

{\sloppy

\tableofcontents

}

\setcounter{secnumdepth}{-2}

\subsection[1. Citation]{Citation}

1.  This Order may be cited as the Child Support Act 1991 (Commencement No.\ 3 and Transitional Provisions) Order 1992.

\subsection[2. Date appointed for the coming into force of certain provisions of the Child Support Act 1991]{Date appointed for the coming into force of certain provisions of the Child Support Act 1991}

2.  Subject to the following provisions of this Order, the date appointed for the coming into force of all the provisions of the Child Support Act 1991, in so far as they are not already in force, except sections 19(3), 30(2), 34(2), 37(2) and (3) and 58(12), is 5th April 1993.

\subsection[3. Transitional Provisions]{Transitional Provisions}

3.  The transitional provisions set out in the Schedule to this Order shall have effect.

\bigskip

Signed by authority of the Secretary of State for Social Security.

{\raggedleft
\emph{Alistair Burt}\\*Parliamentary Under-Secretary of State,\\*Department of Social Security

}

26th October 1992

\clearpage

\part[Schedule]{S C H E D U L E}

%\section[Part I --- Phased take-on of cases]{Part I\\*Phased take-on of cases}
%
%\renewcommand\parthead{--- Schedule Part I}

%1.  In this Part of this Schedule—
%\begin{enumerate}\item[]
%“the Act” means the Child Support Act 1991;
%
%“benefit” means income support, family credit, or disability working allowance under Part VII of the Social Security Contributions and Benefits Act 1992\footnote{\frenchspacing 1992 c. 4. Disability working allowance is prescribed for the purposes of section 6(1) of the Child Support Act by regulation 34 of S.I. 1992/1813.}, or any other benefit prescribed under section 6(1) of the Act (applications by persons receiving benefit); and
%
%“transitional period” means the period beginning with 5th April 1993 and ending with 6th April 1997.
%\end{enumerate}
%
%\medskip
%
%2.  Subject to paragraph 4 below, during the transitional period no application under section 4 of the Act (applications for child support maintenance) in relation to a qualifying child or any qualifying children may be made at any time when—
%\begin{enumerate}\item[]
%($a$) there is in force a maintenance order or maintenance agreement in respect of that qualifying child or those qualifying children and the absent parent, or there is pending before any court an application for such a maintenance order; or
%
%($b$) benefit is being paid to a parent of that child or those children.
%\end{enumerate}
%
%\medskip
%
%3.  Subject to paragraph 4 below, during the transitional period no application under section 7 of the Act (right of child in Scotland to apply for assessment) may be made by a qualifying child at any time when there is in force a maintenance order or maintenance agreement in respect of that child and the absent parent, or there is pending before any court an application for such a maintenance order.
%
%\medskip
%
%4.—(1) Paragraphs 2 and 3 above do not apply to an application made—
%\begin{enumerate}\item[]
%($a$) in that part of the transitional period beginning with 8th April 1996, if the surname of the person with care begins with any of the letters A to D inclusive;
%
%($b$) in that part of the transitional period beginning with 1st July 1996, if the surname of the person with care begins with any of the letters E to K inclusive;
%
%($c$) in that part of the transitional period beginning with 7th October 1996, if the surname of the person with care begins with any of the letters L to R inclusive; and
%
%($d$) in that part of the transitional period beginning with 6th January 1997, if the surname of the person with care begins with any of the letters S to Z inclusive.
%\end{enumerate}
%
%(2) Where paragraph 2 or 3 applies to a case because there is pending before a court an application for a maintenance order, and that application was made before 5th April 1993, those paragraphs shall not prevent the making of an application for a maintenance assessment under section 4 or, as the case may be, section 7 of the Act; but in such a case section 8(3) of the Act shall not have effect until such an application is actually made.
%
%\medskip
%
%5.  For so long as paragraph 2 or 3 above operates in a case so as to prevent an application being made under section 4 of the Act or, as the case may be, section 7 of the Act, and no application has been made under section 6 of the Act, then in relation to that case section 8(3) of the Act (role of the courts with respect to maintenance orders) shall be modified so as to have effect as if the word “vary” was omitted.

%Sch 1 Part I substituted (31.3.93) by SI 1993/966 art 2(1)

\amendment{
%Pt. I substituted (31.3.93) by the Child Support Act 1991 (Commencement No.\ 3 and Transitional Provisions) Amendment Order 1993 art. 2(1).

Pt. I revoked (4.9.95) by the Child Support Act 1995 s. 18(8).
}

%\medskip
%
%1.—(1)  In this Part of this Schedule—
%\begin{enumerate}\item[]
%“the Act” means the Child Support Act 1991;
%
%“benefit” means income support, family credit or disability working allowance under Part VII of the Social Security Contributions and Benefits Act 1992\footnote{\frenchspacing 1992 c. 4.}, or any other benefit prescribed under section 6(1) of the Act (applications by parents receiving benefit);
%
%“parent with care” means a person who, in respect of the same child or children, is both a parent and a person with care; and
%“transitional period” means the period beginning with 5th April 1993 and ending with 6th April 1997.
%\end{enumerate}
%
%(2) For the purposes of paragraph 5 below, in England and Wales, an application for a maintenance order is pending before a court if—
%\begin{enumerate}\item[]
%(i) notice of the application has been filed, in accordance with rules of court, before 5th April 1993;
%
%(ii) in the case of an application contained in a petition for divorce, nullity or judicial separation, or the answer to it, notice of intention to proceed with it was given, in the form required by rules of court, before 5th April 1993.
%\end{enumerate}
%
%\medskip
%
%2.  Subject to paragraph 4 below, during the transitional period no application under section 4 of the Act (applications for child support maintenance) in relation to a qualifying child or any qualifying children may be made at any time when—
%\begin{enumerate}\item[]
%($a$) there is in force a maintenance order or written maintenance agreement (being an agreement made before 5th April 1993) in respect of that qualifying child or those qualifying children and the absent parent; or
%
%($b$) benefit is being paid to a parent with care of that child or those children.
%\end{enumerate}
%
%\medskip
%
%3.  Subject to paragraph 4 below, during the transitional period no application under section 7 of the Act (right of child in Scotland to apply for assessment) may be made by a qualifying child at any time when there is in force a maintenance order or written maintenance agreement (being an agreement made before 5th April 1993) in respect of that child and the absent parent.
%
%\medskip
%
%4.  Paragraphs 2 and 3 above do not apply to an application made—
%\begin{enumerate}\item[]
%($a$) in that part of the transitional period beginning with 8th April 1996, if the surname of the person with care begins with any of the letters A to D inclusive;
%
%($b$) in that part of the transitional period beginning with 1st July 1996, if the surname of the person with care begins with any of the letters E to K inclusive;
%
%($c$) in that part of the transitional period beginning with 7th October 1996, if the surname of the person with care begins with any of the letters L to R inclusive; and
%
%($d$) in that part of the transitional period beginning with 6th January 1997, if the surname of the person with care begins with any of the letters S to Z inclusive.
%\end{enumerate}
%
%5.—(1) For so long as either—
%\begin{enumerate}\item[]
%($a$) paragraph 2 or 3 above operates in a case so as to prevent an application being made under section 4 of the Act or, as the case may be, section 7 of the Act, and no application has been made under section 6 of the Act; or
%
%($b$) an application has been made under section 6 of the Act but no maintenance assessment has yet been made pursuant to that application,
%\end{enumerate}
%then in relation to that case—
%\begin{enumerate}\item[]
%(i) section 8(3) of the Act (role of the courts with respect to maintenance orders) shall be modified so as to have effect as if the word “vary” were omitted;
%
%(ii) in a case falling within sub-paragraph ($a$) above, section 9(3) of the Act shall not apply; and
%
%(iii) section 9(5) of the Act shall be modified so as to have effect as if paragraph ($b$) were omitted.
%\end{enumerate}
%
%(2) In a case where there is, at any time during the transitional period, pending before a court an application for a maintenance order or an application for an order varying a written maintenance agreement, section 8(3) or, as the case may be, section 9(5)($b$) of the Act, shall not apply in relation to that case.

\section[Part II --- Modification of maintenance assessment in certain cases]{Part II\\*Modification of maintenance assessment in certain cases}

\renewcommand\parthead{--- Schedule Part II}

6.  In this Part of this Schedule—
\begin{enumerate}\item[]
“the Act” means the Child Support Act 1991;

“formula amount” means the amount of child support maintenance that would, but for the provisions of this Part of this Schedule, be payable under an original assessment, or any fresh assessment made during the period specified in paragraph 8 consequent on a review under section 17, 18 or 19 of the Act;

“the Maintenance Assessment Procedure Regulations” means the Child Support (Maintenance Assessment Procedure) Regulations 1992\footnote{\frenchspacing S.I. 1992/1813.};

“modified amount” means an amount which is £20 greater than the aggregate weekly amount which was payable under the orders, agreements or arrangements mentioned in paragraph 7(1)($a$) below; and

“original assessment” means a maintenance assessment made in respect of a qualifying child where no previous such assessment has been made or, where the assessment is made in respect of more than one child, where no previous such assessment has been made in respect of any of those children.
\end{enumerate}

\medskip

7.—(1) Subject to sub-paragraph (2), the provisions of this Part of this Schedule apply to cases where—
\begin{enumerate}\item[]
($a$) on 4th April 1993%
, and at all times thereafter until the date when a maintenance assessment is made under the Act, % Words inserted (31.3.93) by SI 1993/966 art 2(2)
 there is in force, in respect of all the qualifying children in respect of whom an application for a maintenance assessment is made under the Act and the absent parent concerned, one or more—
\begin{enumerate}\item[]
(i) maintenance orders;

(ii) orders under section 151 of the Army Act 1955\footnote{\frenchspacing 3 \& 4 Eliz 2 c. 18.} (deductions from pay for maintenance of wife or child) or section 151 of the Air Force Act 1955\footnote{\frenchspacing 3 \& 4 Eliz 2 c. 19.} (deductions from pay for maintenance of wife or child) or arrangements corresponding to such an order and made under Article 1($b$) or 3 of the Naval and Marine Pay and Pensions (Deductions for Maintenance) Order 1959\footnote{\frenchspacing This Order in Council is not a statutory instrument but copies may be obtained from the Ministry of Defence, Naval Pay (Pensions and Conditions of Service) Branch, Old Admiralty Building, Spring Gardens, London, \textsc{sw1a 2be}.}; or

(iii) maintenance agreements (being agreements which are made or evidenced in writing); and
\end{enumerate}

($b$) the absent parent is responsible for maintaining a child or children residing with him other than the child or children in respect of whom the application is made; and

($c$) the formula amount is not more than £60; and

($d$) the formula amount exceeds the aggregate weekly amount which was payable under the orders, agreements or arrangements mentioned in sub-paragraph ($a$) above by more than £20 a week.
\end{enumerate}

(2) Nothing in this Part of this Schedule applies to 
%an interim maintenance assessment 
a Category A interim maintenance assessment within the meaning of regulation 8(1B) of the Child Support (Maintenance Assessment Procedure) Regulations 1992\footnote{\frenchspacing S.I. 1992/1813; the relevant amending instrument is S.I. 1993/913.} % Words substituted (31.3.93) by SI 1993/966 art 2(3)
made under section 12 of the Act.

\amendment{
Words inserted in para. 7(1)($a$) (31.3.93) by the Child Support Act 1991 (Commencement No.\ 3 and Transitional Provisions) Amendment Order 1993 art. 2(2).

Words substituted in para. 7(2) (31.3.93) by the Child Support Act 1991 (Commencement No.\ 3 and Transitional Provisions) Amendment Order 1993 art. 2(3). 
}

\medskip

8.  In a case to which this Part of this Schedule applies, the amount payable under an original assessment, or any fresh assessment made consequent on a review under section 17, 18 or 19 of the Act, during the period of one year beginning with the date on which the original assessment takes effect or, if shorter, until any of the conditions specified in paragraph 7(1) is no longer satisfied, shall, instead of being the formula amount, be the modified amount.

\medskip

9.  For the purpose of determining the aggregate weekly amount payable under the orders, agreements or arrangements mentioned in paragraph 7(1)($a$) above any payments in kind and any payments made to a third party on behalf of or for the benefit of the qualifying child or qualifying children or the person with care shall be disregarded.

\medskip

10.  If, in making a maintenance assessment, a child support officer has applied the provisions of this Part of this Schedule, regulation 10(2) of the Maintenance Assessment Procedure Regulations shall have effect as if there was added at the end—
\begin{quotation}
“($g$) the aggregate weekly amount which was payable under the orders, agreements or arrangements specified in paragraph 7(1)($a$) of the Schedule to the Child Support Act 1991 (Commencement No.\ 3 and Transitional Provisions) Order 1992 (modification of maintenance assessment in certain cases).”.
\end{quotation}

\medskip

11.  The first review of an original assessment under section 16 of the Act (periodical reviews) shall be conducted on the basis that the amount payable under the assessment immediately before the review takes place was the formula amount.

\medskip

12.—(1) The provisions of the following sub-paragraphs shall apply where there is a review of a previous assessment under section 17 of the Act (reviews on change of circumstances) at any time when the amount payable under that assessment is the modified amount.

(2) Where the child support officer determines that, were a fresh assessment to be made as a result of the review, the amount payable under it (disregarding the provisions of this Part of this Schedule) (in this paragraph called “the reviewed formula amount”) would be—
\begin{enumerate}\item[]
($a$) more than the formula amount, the amount of child support maintenance payable shall be the modified amount plus the difference between the formula amount and the reviewed formula amount;

($b$) less than the formula amount but more than the modified amount, the amount of child support maintenance payable shall be the modified amount;

($c$) less than the modified amount, the amount of child support maintenance payable shall be the reviewed formula amount.
\end{enumerate}

(3) The child support officer shall, in determining the reviewed formula amount, apply the provisions of regulations 20 to 22 of the Maintenance Assessment Procedure Regulations.

\part{Explanatory Note}

\renewcommand\parthead{--- Explanatory Note}

\subsection*{(This note is not part of the Order)}

 This Order brings into force on 5th April 1993 all the provisions of the Child Support Act 1991 which are not already in force, or not fully in force, except for sections 19(3) (which relates to the giving of notice to prescribed persons before making a fresh maintenance assessment), 30(2) (which relates to the Secretary of State arranging for the collection of certain payments for the benefit of a child even though he is not arranging for the collection of child support maintenance for that child), 34(2) (which relates to the definition of “relevant information”), 37(2) and (3) (which relates to the duty of a person in Scotland against whom a liability order has been made to supply relevant information to the Secretary of State) and 58(12) (which provides for paragraph 1(1) of Schedule 3 to the Act to have effect in a modified form until Schedule 1 to the Disability Living Allowance and Disability Working Allowance Act 1991 (c.\ 21) comes into force. That Schedule is already in force.). However, Article 3 and the Schedule contain transitional provisions which delay the full operation of sections 4 and 7 of the Act until 7th April 1997 and varies in some cases the way in which the amount of child support maintenance payable during the first year of an assessment is calculated.

\part{Note as to Earlier Commencement Orders}

\renewcommand\parthead{--- Note as to Earlier Commencement Orders}

\subsection*{(This note is not part of the Order)}

 The provisions of the Act specified in column 1 have been brought into force on the date specified in column 2 by the instrument specified in column 3.
\begin{longtable}{p{134pt}p{100pt}p{75pt}}
\hline
Column 1 & Column 2 & Column 3\\
\itshape Provision & \itshape Date in force & \itshape Commencement\\
&&\itshape Order\\
\hline
\endhead
\hline
\endlastfoot
Section 3(3)($c$)&17th June 1992&S.I. 1992/1431.\\
Section 4(4), (7) and (8)&17th June 1992&S.I. 1992/1431.\\
Section 5(3)&17th June 1992&S.I. 1992/1431.\\
Section 6(1) (partially), (9), (10) and (13)&17th June 1992&S.I. 1992/1431.\\
Section 7(5), (8) and (9)&17th June 1992&S.I. 1992/1431.\\
Section 8(5), (9) and (11)($f$)&17th June 1992&S.I. 1992/1431.\\
Section 10&17th June 1992&S.I. 1992/1431.\\
Section 12(2), (3) and (5)&17th June 1992&S.I. 1992/1431.\\
Section 13&1st September 1992&S.I. 1992/1938.\\
Section 14(1) and (3)&17th June 1992&S.I. 1992/1431.\\
Section 16(1), (2), (5) and (6)&17th June 1992&S.I. 1992/1431.\\
Section 17(4) and (6)($b$)&17th June 1992&S.I. 1992/1431.\\
Section 18(8) and (11)&17th June 1992&S.I. 1992/1431.\\
Section 21(2) and (3)&17th June 1992&S.I. 1992/1431.\\
Section 21(1) and (4)&1st September 1992&S.I. 1992/1938.\\
Section 22(3) and (4)&17th June 1992&S.I. 1992/1431.\\
Section 22(1), (2) and (5)&1st September 1992&S.I. 1992/1938.\\
Section 23&1st September 1992&S.I. 1992/1938.\\
Section 24(6) and (7)&17th June 1992&S.I. 1992/1431.\\
Section 24(9)&1st September 1992&S.I. 1992/1938.\\
Section 25(2)($a$), (3)($c$), (5) and (6)&17th June 1992&S.I. 1992/1431.\\
Section 29(2) and (3)&17th June 1992&S.I. 1992/1431.\\
Section 30(1), (4) and (5)&17th June 1992&S.I. 1992/1431.\\
Section 31(8)&17th June 1992&S.I. 1992/1431.\\
Section 32(1) to (5) and (7) to (9)&17th June 1992&S.I. 1992/1431.\\
Section 34(1)&17th June 1992&S.I. 1992/1431.\\
Section 35(2)($b$), (7) and (8)&17th June 1992&S.I. 1992/1431.\\
Section 39&17th June 1992&S.I. 1992/1431.\\
Section 40(4)($a$)(ii), (8) and (11)&17th June 1992&S.I. 1992/1431.\\
Section 41(2), (3) and (4)&17th June 1992&S.I. 1992/1431.\\
Section 42&17th June 1992&S.I. 1992/1431.\\
Section 43(1)($b$) and (2)($a$)&17th June 1992&S.I. 1992/1431.\\
Section 44(3)&17th June 1992&S.I. 1992/1431.\\
Section 45&17th June 1992&S.I. 1992/1431.\\
Section 46(11)&17th June 1992&S.I. 1992/1431.\\
Section 47&17th June 1992&S.I. 1992/1431.\\
Section 49&17th June 1992&S.I. 1992/1431.\\
Section 50(5) and (7)($d$)&17th June 1992&S.I. 1992/1431.\\
Section 51&17th June 1992&S.I. 1992/1431.\\
Section 52&17th June 1992&S.I. 1992/1431.\\
Section 54&17th June 1992&S.I. 1992/1431.\\
Section 55&17th June 1992&S.I. 1992/1431.\\
Section 56(2), (3) and (4)&17th June 1992&S.I. 1992/1431.\\
Section 57&17th June 1992&S.I. 1992/1431.\\
Schedule 1 and section 11 (partially)&17th June 1992&S.I. 1992/1431.\\
Paragraph 2(4) of Schedule 2 and section 14(4) so far as it relates to that sub-paragraph&17th June 1992&S.I. 1992/1431.\\
Paragraph 3(3) of Schedule 3 and section 21(4) so far as it relates to that sub-paragraph&17th June 1992&S.I. 1992/1431.\\
Schedule 3 so far as not already in force and section 21(4) so far as it relates to those provisions&1st September 1992&S.I. 1992/1938.\\
Schedule 4&1st September 1992&S.I. 1992/1938.\\
Paragraphs 1 to 4 of Schedule 5 and section 58(13) so far as it relates to those paragraphs&1st September 1992&S.I. 1992/1938.\\
\end{longtable}

\end{document}