\documentclass[a4paper]{article}

\usepackage[welsh,english]{babel}

\usepackage[utf8]{inputenc}
\usepackage[T1]{fontenc}

\usepackage{textcomp}

%\usepackage[2012rules]{optional}

\usepackage[osf]{mathpazo}

%\opt{newrules}{
\title{The Child Support (Collection and Enforcement) Regulations 1992}
%}

%\opt{2012rules}{
%\title{Child Maintenance and Other Payments Act 2008\\(2012 scheme version)}
%}

\author{S.I. 1992 No. 1989}

\date{Made 17th August 1992\\Laid before Parliament 26th August 1992\\Coming into force 5th April 1993}

%\opt{oldrules}{\newcommand\versionyear{1993}}
%\opt{newrules}{\newcommand\versionyear{2003}}
%\opt{2012rules}{\newcommand\versionyear{2012}}

\usepackage{fancyhdr}
\pagestyle{fancy}
\fancyhead[L]{}
\fancyhead[C]{\itshape The Child Support (Collection and Enforcement) Regulations 1992 (S.I.~1992/1989) \parthead%\phantom{...}% (\versionyear{} scheme version)
}
\fancyhead[R]{}
\fancyfoot[C]{\thepage}
\newcommand{\parthead}{}

\usepackage{array}
\usepackage{multirow}
\usepackage[debugshow]{tabulary}
\usepackage{longtable}
\usepackage{multicol}
\usepackage{lettrine}

\usepackage[colorlinks=true]{hyperref}
\usepackage{microtype}

\hyphenation{Aw-dur-dod}
\hyphenation{bank-rupt-cy}
\hyphenation{Ec-cles-ton}
\hyphenation{Eux-ton}
\hyphenation{Hogh-ton}
\hyphenation{Pres-ton}
\hyphenation{Pru-den-tial}
\hyphenation{Riv-ing-ton}

\newcolumntype{x}[1]
	{>{\raggedright}p{#1}}
\newcommand{\tn}{\tabularnewline}
\setlength\tymin{50pt}

\newcommand\amendment[1]{\subsubsection*{Notes}{\itshape\frenchspacing\footnotesize #1 \par}}

\usepackage{perpage} %the perpage package
\MakePerPage{footnote} %the perpage package command
\renewcommand{\thefootnote}{\fnsymbol{footnote}}

\usepackage[perpage,para,symbol]{footmisc}

\begin{document}

\maketitle

\noindent
The Secretary of State for Social Security, in exercise of the powers conferred by sections 29(2) and (3), 31(8), 32(1) to (5) and (7) to (9), 34(1), 35(2), (7) and (8), 39(1), (3) and (4), 40(4), (8) and (11), 51, 52 and 54 of the Child Support Act 1991\footnote{\frenchspacing 1991 c. 48. Section 54 is cited because of the meaning ascribed to the word “prescribed”.} and of all other powers enabling him in that behalf, hereby makes the following Regulations:

{\sloppy

\tableofcontents

}

\setcounter{secnumdepth}{-2}

\section[Part I --- General]{Part I\\*General}

\subsection[1. Citation, commencement and interpretation]{Citation, commencement and interpretation}

\renewcommand\parthead{--- Part I}

1.—(1) These Regulations may be cited as the Child Support (Collection and Enforcement) Regulations 1992 and shall come into force on 5th April 1993.

(2) In these Regulations “the Act” means the Child Support Act 1991.

(3) Where under any provision of the Act or of these Regulations—
\begin{enumerate}\item[]
($a$) any document or notice is given or sent to the Secretary of State, it shall be treated as having been given or sent on the day it is received by the Secretary of State; and

($b$) any document or notice is given or sent to any other person, it shall, if sent by post to that person’s last known or notified address, be treated as having been given or sent on the second day after the day of posting, excluding any Sunday or any day which is a bank holiday under the Banking and Financial Dealings Act 1971\footnote{\frenchspacing 1971 c. 80.}.
\end{enumerate}

(4) In these Regulations, unless the context otherwise requires, a reference—
\begin{enumerate}\item[]
($a$) to a numbered Part is to the Part of these Regulations bearing that number;

($b$) to a numbered regulation is to the regulation in these Regulations bearing that number;

($c$) in a regulation to a numbered or lettered paragraph or sub-paragraph is to the paragraph or sub-paragraph in that regulation bearing that number or letter;

($d$) in a paragraph to a lettered or numbered sub-paragraph is to the sub-paragraph in that paragraph bearing that letter or number;

($e$) to a numbered Schedule is to the Schedule to these Regulations bearing that number.
\end{enumerate}

\section[Part II --- Collection of child support maintenance]{Part II\\*Collection of child support maintenance}

\renewcommand\parthead{--- Part II}

\subsection[2. Payment of child support maintenance]{Payment of child support maintenance}

2.—(1) Where a maintenance assessment has been made under the Act and the case is one to which section 29 of the Act applies, the Secretary of State may specify that payments of child support maintenance shall be made by the liable person—
\begin{enumerate}\item[]
($a$) to the person caring for the child or children in question or, where an application has been made under section 7 of the Act, to the child who made the application;

($b$) to, or through, the Secretary of State; or

($c$) to, or through, such other person as the Secretary of State may, from time to time, specify.
\end{enumerate}

(2) In paragraph (1) and in the rest of this Part, “liable person” means a person liable to make payments of child support maintenance.

\subsection[3. Method of payment]{Method of payment}

3.—(1) Payments of child support maintenance shall be made by the liable person by whichever of the following methods the Secretary of State specifies as being appropriate in the circumstances—
\begin{enumerate}\item[]
($a$) by standing order;

($b$) by any other method which requires one person to give his authority for payments to be made from an account of his to an account of another’s on specific dates during the period for which the authority is in force and without the need for any further authority from him;

($c$) by an arrangement whereby one person gives his authority for payments to be made from an account of his, or on his behalf, to another person or to an account of that other person;

($d$) by cheque or postal order;

($e$) in cash.
\end{enumerate}

(2) The Secretary of State may direct a liable person to take all reasonable steps to open an account from which payments under the maintenance assessment may be made in accordance with the method of payment specified under paragraph (1).

\subsection[4. Interval of payment]{Interval of payment}

4.—(1) The Secretary of State shall specify the day and interval by reference to which payments of child support maintenance are to be made by the liable person and may from time to time vary such day or interval.

%(2) In specifying the day and interval of payment the Secretary of State shall have regard to all the circumstances and in particular to—
%\begin{enumerate}\item[]
%($a$) the needs of the person entitled to receive payment and the day and interval by reference to which any other income is normally received by that person;
%
%($b$) the day and interval by reference to which the liable person’s income is normally received; and
%
%($c$) any period necessary to enable the clearance of cheques or otherwise necessary to enable the transmission of payments to the person entitled to receive them.
%\end{enumerate}

% Reg 4(2) substituted (18.4.95) by SI 1995/1045 reg 12
(2) In specifying the day and interval of payment the Secretary of State shall have regard to the following factors—
\begin{enumerate}\item[]
($a$) the circumstances of the person liable to make the payments and in particular the day upon which and the interval at which any income is payable to that person;

($b$) any preference indicated by that person;

($c$) any period necessary to enable the clearance of cheques or otherwise necessary to enable the transmission of payments to the person entitled to receive them,
\end{enumerate}
and, subject to those factors, to any other matter which appears to him to be relevant in the particular circumstances of the case.

\amendment{
Reg. 4(2) substituted (18.4.95) by the Child Support and Income Support (Amendment) Regulations 1995 reg. 12.
}

\subsection[5. Transmission of payments]{Transmission of payments}

5.—(1) Payments of child support maintenance made through the Secretary of State or other specified person shall be transmitted to the person entitled to receive them in whichever of the following ways the Secretary of State specifies as being appropriate in the circumstances—
\begin{enumerate}\item[]
($a$) by a transfer of credit to an account nominated by the person entitled to receive the payments;

($b$) by cheque, girocheque or other payable order;

($c$) in cash.
\end{enumerate}

(2) 
%The Secretary of State 
Subject to paragraph (3), the Secretary of State  % Words substituted (18.4.95) by SI 1995/1045 reg 13(2)
shall specify the interval by reference to which the payments referred to in paragraph (1) are to be transmitted to the person entitled to receive them.

%(3) The interval referred to in paragraph (2) may differ from the interval referred to in regulation 4 and may from time to time be varied by the Secretary of State.
%
%(4) In specifying the interval for transmission of payments the Secretary of State shall have regard to all the circumstances and in particular to—
%\begin{enumerate}\item[]
%($a$) the needs of the person entitled to receive payment and the interval by reference to which any other income is normally received by that person;
%
%($b$) any period necessary to enable the clearance of cheques or otherwise necessary to enable the transmission of payments to the person entitled to receive them.
%\end{enumerate}

%Reg 5(3), (4) substituted (18.4.95) by SI 1995/1045 reg 18(3)
(3) Except where the Secretary of State is satisfied in the circumstances of the case that it would cause undue hardship to either the person liable to make the payments or the person entitled to receive them, the interval referred to in paragraph (2) shall not differ from the interval referred to in regulation 4.

(4) Subject to paragraph (3) and regulation 4(2), the interval referred to in paragraph (2) and that referred to in regulation 4 may be varied from time to time by the Secretary of State.

\amendment{
Words substituted in reg. 5(2) and reg. 5(3), (4) substituted (18.4.95) by the Child Support and Income Support (Amendment) Regulations 1995 reg. 13.
}

\subsection[6. Representations about payment arrangements]{Representations about payment arrangements}

6.  The Secretary of State shall, insofar as is reasonably practicable, provide the liable person and the person entitled to receive the payments of child support maintenance with an opportunity to make representations with regard to the matters referred to in regulations 2 to 5 and the Secretary of State shall have regard to those representations in exercising his powers under those regulations.

\subsection[7. Notice to liable person as to requirements about payment]{Notice to liable person as to requirements about payment}

7.—(1) The Secretary of State shall send the liable person a notice stating—
\begin{enumerate}\item[]
($a$) the amount of child support maintenance payable;

($b$) to whom it is to be paid;

($c$) the method of payment; and

($d$) the day and interval by reference to which payments are to be made.
\end{enumerate}

(2) A notice under paragraph (1) shall be sent to the liable person as soon as is reasonably practicable after—
\begin{enumerate}\item[]
($a$) the making of a maintenance assessment, and

($b$) after any change in the requirements referred to in any previous such notice.
\end{enumerate}

\section[Part III --- Deduction from earnings orders]{Part III\\*Deduction from earnings orders}

\renewcommand\parthead{--- Part III}

\subsection[8. Interpretation of this Part]{Interpretation of this Part}

8.—(1) For the purposes of this Part—
\begin{enumerate}\item[]
%Definition of ``defective'' inserted (18.4.95) by SI 1995/1045 reg 14(2)
“defective” means in relation to a deduction from earnings order that it does not comply with the requirements of regulations 9 to 11 and such failure to comply has made it impracticable for the employer to comply with his obligations under the Act and these Regulations;

“disposable income” means the amount determined under 
%regulation 12(1) 
regulation 12(1)($a$)  % Words substituted (18.4.95) by SI 1995/1045 reg 14(3)
of the Child Support (Maintenance Assessments and Special Cases) Regulations 1992\footnote{\frenchspacing S.I. 1992/1815.};

“earnings” shall be construed in accordance with paragraphs (3) and (4);

“exempt income” means the amount determined under regulation 9 of the Child Support (Maintenance Assessments and Special Cases) Regulations 1992;

%Definition of ``interim maintenance assessment'' inserted (18.4.95) by SI 1995/1045 reg 14(4)
“interim maintenance assessment” means a Category A, Category B, Category C or Category D interim maintenance assessment within the meaning of 
%regulation 8(1B) 
regulation 8(3)  % Words substituted (7.10.96) by SI 1996/1945 reg 3
of the Child Support (Maintenance Assessment Procedure) Regulations 1992\footnote{\frenchspacing S.I. 1992/1813.};

“net earnings” shall be construed in accordance with paragraph (5);

“normal deduction rate” means the rate specified in a deduction from earnings order (expressed as a sum of money per week, month or other period) at which deductions are to be made from the liable person’s net earnings;

“pay-day” in relation to a liable person means an occasion on which earnings are paid to him or the day on which such earnings would normally fall to be paid;

“prescribed minimum amount” means the minimum amount prescribed in regulation 13 of the Child Support (Maintenance Assessments and Special Cases) Regulations 1992;

“protected earnings rate” means the level of earnings specified in a deduction from earnings order (expressed as a sum of money per week, month or other period) below which deductions of child support maintenance shall not be made for the purposes of this Part;

“protected income level” means the level of protected income determined in accordance with 
paragraphs (1) to (5) of  % Words inserted (18.4.95) by SI 1995/1045 reg 14(5)
regulation 11 of the Child Support (Maintenance Assessments and Special Cases) Regulations 1992.
\end{enumerate}

(2) For the purposes of this Part the relationship of employer and employee shall be treated as subsisting between two persons if one of them, as a principal and not as a servant or agent, pays to the other any sum defined as earnings under paragraph (1) and “employment”, “employer” and “employee” shall be construed accordingly.

(3) Subject to paragraph (4), “earnings” are any sums payable to a person—
\begin{enumerate}\item[]
($a$) by way of wages or salary (including any fees, bonus, commission, overtime pay or other emoluments payable in addition to wages or salary or payable under a contract of service);

($b$) by way of pension (including an annuity in respect of past service, whether or not rendered to the person paying the annuity, and including periodical payments by way of compensation for the loss, abolition or relinquishment, or diminution in the emoluments, of any office or employment);

($c$) by way of statutory sick pay.
\end{enumerate}

(4) “Earnings” shall not include—
\begin{enumerate}\item[]
($a$) sums payable by any public department of the Government of Northern Ireland or of a territory outside the United Kingdom;

($b$) pay or allowances payable to the liable person as a member of Her Majesty’s forces;

($c$) pension, allowances or benefit payable under any enactment relating to social security;

($d$) pension or allowances payable in respect of disablement or disability;

($e$) guaranteed minimum pension within the meaning of the Social Security Pensions Act 1975\footnote{\frenchspacing 1975 c. 60.}.
\end{enumerate}

(5) “Net earnings” means the residue of earnings after deduction of—
\begin{enumerate}\item[]
($a$) income tax;

($b$) primary class I contributions under Part I of the Contributions and Benefits Act 1992\footnote{\frenchspacing 1992 c. 4.};

($c$) amounts deductible by way of contributions to a superannuation scheme which provides for the payment of annuities or 
%lumps 
lump % Word substituted (5.4.93) by SI 1993/913 reg 41.
sums—
\begin{enumerate}\item[]
(i) to the employee on his retirement at a specified age or on becoming incapacitated at some earlier age; or

(ii) on his death or otherwise, to his personal representative, widow, relatives or dependants.
\end{enumerate}
\end{enumerate}

\amendment{
Word substituted in reg. 8(5)($c$) (5.4.93) by the Child Support (Miscellaneous Amendments) Regulations 1993 reg. 41.

Words substituted in definition of ``disposable income'' in reg. 8(1), words inserted in definition of ``protected income level'' in reg. 8(1) and definitions of ``defective'', ``interim maintenance assessment'' inserted in reg. 8(1) (18.4.95) by the Child Support and Income Support (Amendment) Regulations 1995 reg. 14.

Words substituted in definition of ``interim maintenance assessment'' in reg. 8(1) (7.10.96) by the Child Support (Miscellaneous Amendments) Regulations 1996 reg. 3.
}

\subsection[9. Deduction from earnings orders]{Deduction from earnings orders}

9.  A deduction from earnings order shall specify—
\begin{enumerate}\item[]
($a$) the name and address of the liable person;

($b$) the name of the employer at whom it is directed;

($c$) where known, the liable person’s place of work, the nature of his work and any works or pay number;

% Reg 9($cc$) inserted (22.1.96) by SI 1995/3261 reg 6
($cc$) where known, the liable person’s national insurance number;

%($d$) the normal deduction rate;
%
%($e$) 
%except in the case of a Category A or Category B interim maintenance assessment within the meaning of regulation 8(1A) and (1B) of the Child Support (Maintenance Assessment Procedure) Regulations 1992, % Words inserted in reg 9($e$) (7.2.94) by SI 1994/227 reg 3(1)
%the protected earnings rate;

% Reg 9($d$), ($e$) substituted (18.4.95) by SI 1995/1045 reg 15
($d$) the normal deduction rate or rates and the date upon which each is to take effect;

($e$) the protected earnings rate;

($f$) the address to which amounts deducted from earnings are to be sent.
\end{enumerate}

\amendment{
%Words inserted in reg. 9($e$) (7.2.94) by the Child Support (Miscellaneous Amendments and Transitional Provisions) Regulations 1994 reg. 3(1) (subject to transitional provisions in reg. 12).

Reg. 9($d$), ($e$) substituted (18.4.95) by the Child Support and Income Support (Amendment) Regulations 1995 reg. 15.

Reg. 9($cc$) inserted (22.1.96) by the Child Support (Miscellaneous Amendments) (No.\ 2) Regulations 1995 reg. 6.
}

\subsection[10. Normal deduction rate]{Normal deduction rate}

10.—(1) The period by reference to which 
%the normal deduction rate 
a normal deduction rate  % Words substituted (18.4.95) by SI 1995/1045 reg 16(2)
is set shall be the period by reference to which the liable person’s earnings are normally paid or, if none, such other period as the Secretary of State may specify.

(2) The Secretary of State, in specifying the normal deduction rate, shall not include any amount in respect of arrears or interest% 
, in a case where there is a current assessment,  % Words inserted (18.4.95) by SI 1995/1045 reg 16(3)($a$)
if, 
%at the date of making of the current assessment—
at the date of making of any current maintenance assessment other than an interim maintenance assessment---  % Words substituted (18.4.95) by SI 1995/1045 reg 16(3)($b$)
\begin{enumerate}\item[]
($a$) the liable person’s disposable income was below the level specified in paragraph (3); or

($b$) the deduction of such an amount from the liable person’s disposable income would have reduced his disposable income below the level specified in paragraph (3).
\end{enumerate}

(3) The level referred to in paragraph (2) is the liable person’s protected income level less the prescribed minimum amount.

\amendment{
Words substituted in reg. 10(1), (2) and words inserted in reg. 10(2) (18.4.95) by the Child Support and Income Support (Amendment) Regulations 1995 reg. 16.
}

\subsection[11. Protected earnings rate]{Protected earnings rate}

11.—(1) The period by reference to which the protected earnings rate is set shall be the same as the period by reference to which the normal deduction rate is set under regulation 10(1).

(2) The amount to be specified as the protected earnings rate in respect of any period shall%
, except where paragraph (3) or paragraph (4) applies,  % Words inserted (18.4.95) by SI 1995/1045 reg 17(2)
be an amount equal to the liable person’s exempt income in respect of that period as calculated at the date of the current assessment.

%Reg 11(3), (4) inserted (18.4.95) by SI 1995/1045 reg 17(3)
(3) Where an interim maintenance assessment%
, except a Category B interim maintenance assessment,  % Words inserted (5.8.96) by SI 1996/1945 reg 4
is in force the protected earnings rate shall be—
\begin{enumerate}\item[]
($a$) where there is some knowledge of the liable person’s circumstances, the aggregate of the following amounts at the date of the making of the assessment—
\begin{enumerate}\item[]
(i) the personal allowance applicable by virtue of paragraph 1(1)($e$) of Schedule 2 to the Income Support (General) Regulations 1987\footnote{\frenchspacing S.I. 1987/1967. Relevant amending instruments are 1988/663, 1989/1678.} (in this paragraph referred to as “the relevant Schedule”) or if he is known to have a partner, that applicable for a couple under paragraph 1(3)($c$) of that Schedule;

(ii) the personal allowance applicable by virtue of the relevant Schedule in respect of any child or young person who is known to be living with the relevant person (and where the age of the child or young person is not known it shall be assumed to be less than 11);

(iii) the amount of any premium applicable by virtue of the relevant Schedule which is known to be applicable in the circumstances of the case; and

(iv) £30;
\end{enumerate}

($b$) in any other case the personal allowance specified in paragraph 1(1)($e$) of the relevant Schedule at the date mentioned in sub-paragraph ($a$), plus £30.
\end{enumerate}

(4) Where there is a liability to make payments of child support maintenance but no maintenance assessment is in force, the protected earnings rate shall be—
\begin{enumerate}\item[]
($a$) except where the last maintenance assessment was an interim maintenance assessment of Category A or Category C—
\begin{enumerate}\item[]
(i) where the absent parent produces evidence sufficient to satisfy the child support officer that his circumstances have changed since the last assessment or review under section 16, 17, 18 or 19 of the Act, a figure equal to the figure that would be his exempt income if the assessment were then being reviewed; or

(ii) in any other case an amount equal to the amount of exempt income produced by the last assessment or review under section 16, 17, 18 or 19 of the Act applicable in his case;
\end{enumerate}

($b$) in the case of an interim maintenance assessment of Category A or Category C, the amount produced by the application of the provisions of paragraph (3) above in his case.
\end{enumerate}

\amendment{
Words inserted in reg. 11(2) and reg. 11(3), (4) inserted (18.4.95) by the Child Support and Income Support (Amendment) Regulations 1995 reg. 17.

Words inserted in reg. 11(3) (5.8.96) by the Child Support (Miscellaneous Amendments) Regulations 1996 reg. 4.
}

\subsection[12. Amount to be deducted by employer]{Amount to be deducted by employer}

12.—(1) Subject to the provisions of this regulation, an employer who has been served with a copy of a deduction from earnings order in respect of a liable person in his employment shall, each pay-day, make a deduction from the net earnings of that liable person of an amount equal to the normal deduction rate.

(2) Where the deduction of the normal deduction rate would reduce the liable person’s net earnings below the protected earnings rate the employer shall deduct only such amount as will leave the liable person with net earnings equal to the protected earnings rate.

(3) Where the liable person receives a payment of earnings at an interval greater or lesser than the interval specified in relation to the normal deduction rate and the protected earnings rate (“the specified interval”) the employer shall, for the purpose of such payments, take as the normal deduction rate and the protected earnings rate such amounts (to the nearest whole penny) as are in the same proportion to the interval since the last pay-day as the normal deduction rate and the protected earnings rate bear to the specified interval.

(4) Where, on any pay-day, the employer fails to deduct an amount due under the deduction from earnings order or deducts an amount less than the amount of the normal deduction rate the shortfall shall, subject to the operation of paragraph (2), be deducted in addition to the normal deduction rate at the next available pay-day or days.

(5) Where, on any pay-day, the liable person’s net earnings are less than his protected earnings rate the amount of the difference shall be carried forward to his next pay-day and treated as part of his protected earnings in respect of that pay-day.

(6) Where, on any pay-day, an employer makes a deduction from the earnings of a liable person in accordance with the deduction from earnings order he may also deduct an amount not exceeding £1 in respect of his administrative costs and such deduction for administrative costs may be made notwithstanding that it may reduce the liable person’s net earnings below the protected earnings rate.

\subsection[13. Employer to notify liable person of deduction]{Employer to notify liable person of deduction}

13.—(1) An employer making a deduction from earnings for the purposes of this Part shall notify the liable person in writing of the amount of the deduction, including any amount deducted for administrative costs under regulation 12(6).

(2) Such notification shall be given not later than the pay-day on which the deduction is made or, where that is impracticable, not later than the following pay-day.

\subsection[14. Payment by employer to Secretary of State]{Payment by employer to Secretary of State}

14.—(1) Amounts deducted by an employer under a deduction from earnings order (other than any administrative costs deducted under regulation 12(6)) shall be paid to the Secretary of State by the 19th day of the month following the month in which the deduction is made.

(2) Such payment may be made—
\begin{enumerate}\item[]
($a$) by cheque;

($b$) by automated credit transfer; or

($c$) by such other method as the Secretary of State may specify.
\end{enumerate}

\subsection[15. Information to be provided by liable person]{Information to be provided by liable person}

15.—(1) The Secretary of State may, in relation to the making or operation of a deduction from earnings order, require the liable person to provide the following details—
\begin{enumerate}\item[]
($a$) the name and address of his employer;

($b$) the amount of his earnings and anticipated earnings;

($c$) his place of work, the nature of his work and any works or pay number;
\end{enumerate}
and it shall be the duty of the liable person to comply with any such requirement within 7 days of being given written notice to that effect.

(2) A liable person in respect of whom a deduction from earnings order is in force shall notify the Secretary of State in writing within 7 days of every occasion on which he leaves employment or becomes employed or re-employed.

\subsection[16. Duty of employers and others to notify Secretary of State]{Duty of employers and others to notify Secretary of State}

16.—(1) Where a deduction from earnings order is served on a person on the assumption that he is the employer of a liable person but the liable person to whom the order relates is not in his employment, the person on whom the order was served shall notify the Secretary of State of that fact in writing, at the address specified in the order, within 10 days of the date of service on him of the order.

(2) Where an employer is required to operate a deduction from earnings order and the liable person to whom the order relates ceases to be in his employment the employer shall notify the Secretary of State of that fact in writing, at the address specified in the order, within 10 days of the liable person ceasing to be in his employment.

(3) Where an employer becomes aware that a deduction from earnings order is in force in relation to a person who is an employee of his he shall, within 7 days of the date on which he becomes aware, notify the Secretary of State of that fact in writing at the address specified in the order.

\subsection[17. Requirement to review deduction from earnings orders]{Requirement to review deduction from earnings orders}

%17.  The Secretary of State shall review a deduction from earnings order in the following circumstances—
%\begin{enumerate}\item[]
%($a$) where there is a change in the amount of the maintenance assessment;
%
%($b$) where any arrears and interest on arrears payable under the order are paid off.
%\end{enumerate}

% Reg 17 substituted (18.4.95) by SI 1995/1045 reg 18
17.—(1) Subject to paragraph (2), the Secretary of State shall review a deduction from earnings order in the following circumstances—
\begin{enumerate}\item[]
($a$) where there is a change in the amount of the maintenance assessment;

($b$) where any arrears and interest on arrears payable under the order are paid off.
\end{enumerate}

(2) There shall be no obligation to review a deduction from earnings order under paragraph (1) where the normal deduction rates specified in the order take account of the changes which will arise as a result of the circumstances specified in sub-paragraph ($a$) or ($b$) of that paragraph.

\amendment{
Reg. 17 substituted (18.4.95) by the Child Support and Income Support (Amendment) Regulations 1995 reg. 18.
}

\subsection[18. Power to vary deduction from earnings orders]{Power to vary deduction from earnings orders}

18.—(1) The Secretary of State may (whether on a review under regulation 17 or otherwise) vary a deduction from earnings order so as to—
\begin{enumerate}\item[]
($a$) include any amount which may be included in such an order or exclude or decrease any such amount;

($b$) substitute a subsequent employer for the employer at whom the order was previously directed.
\end{enumerate}

(2) The Secretary of State shall serve a copy of any deduction from earnings order, as varied, on the liable person’s employer and on the liable person.

\subsection[19. Compliance with deduction from earnings order as varied]{Compliance with deduction from earnings order as varied}

19.—(1) Where a deduction from earnings order has been varied and a copy of the order as varied has been served on the liable person’s employer it shall, subject to paragraph (2), be the duty of the employer to comply with the order as varied.

(2) The employer shall not be under any liability for non-compliance with the order, as varied, before the end of the period of 7 days beginning with the date on which a copy of the order, as varied, was served on him.

\subsection[20. Discharge of deduction from earnings orders]{Discharge of deduction from earnings orders}

20.—%(1) The Secretary of State may discharge a deduction from earnings order where—
%\begin{enumerate}\item[]
%($a$) no further payments under it are due; or
%
%($b$) it appears to him that the order is ineffective or that some other way of securing that payments are made would be more effective.
%\end{enumerate}
%
%Reg 20(1) substituted (18.4.95) by SI 1995/1045 reg 19
(1) The Secretary of State may discharge a deduction from earnings order where it appears to him that—
\begin{enumerate}\item[]
($a$) no further payments are due under it;

($b$) the order is ineffective or some other way of securing that payments are made would be more effective;

($c$) the order is defective;

($d$) the order fails to comply in a material respect with any procedural provision of the Act or regulations made under it other than provision made in regulation 9, 10 or 11;

($e$) at the time of the making of the order he did not have, or subsequently ceased to have, jurisdiction to make a deduction from earnings order; or

($f$) in the case of an order made at a time when there is in force an interim maintenance assessment, it is inappropriate to continue deductions under the order having regard to the compliance or the attempted compliance with the maintenance assessment by the liable person.
\end{enumerate}


(2) The Secretary of State shall give written notice of the discharge of the deduction from earnings order to the liable person and to the liable person’s employer.

\amendment{
Reg. 20(1) substituted (18.4.95) by the Child Support and Income Support (Amendment) Regulations 1995 reg. 19.
}

\subsection[21. Lapse of deduction from earnings orders]{Lapse of deduction from earnings orders}

21.—(1) A deduction from earnings order shall lapse (except in relation to any deductions made or to be made in respect of the employment not yet paid to the Secretary of State) where the employer at whom it is directed ceases to have the liable person in his employment.

(2) The order shall lapse from the pay-day coinciding with, or, if none, the pay-day following, the termination of the employment.

(3) A deduction from earnings order which has lapsed under this regulation shall nonetheless be treated as remaining in force for the purposes of regulations 15 and 24.

(4) Where a deduction from earnings order has lapsed under paragraph (1) and the liable person recommences employment (whether with the same or another employer), the order may be revived from such date as may be specified by the Secretary of State.

(5) Where a deduction from earnings order is revived under paragraph (4), the Secretary of State shall give written notice of that fact to, and serve a copy of the notice on, the liable person and the liable person’s employer.

(6) Where an order is revived under paragraph (4), no amount shall be carried forward under regulation 12(4) or (5) from a time prior to the revival of the order.

\subsection[22. Appeals against deduction from earnings orders]{Appeals against deduction from earnings orders}

22.—(1) A liable person in respect of whom a deduction from earnings order has been made may appeal to the magistrates' court, or in Scotland the sheriff, having jurisdiction in the area in which he resides.

(2) Any appeal shall—
\begin{enumerate}\item[]
($a$) be by way of complaint for an order or, in Scotland, by way of application;

($b$) be made within 28 days of the date on which the matter appealed against arose.
\end{enumerate}

(3) An appeal may be made only on one or both of the following grounds—
\begin{enumerate}\item[]
($a$) that the deduction from earnings order is defective;

($b$) that the payments in question do not constitute earnings.
\end{enumerate}

(4) Where the court or, as the case may be, the sheriff is satisfied that the appeal should be allowed the court, or sheriff, may—
\begin{enumerate}\item[]
($a$) quash the deduction from earnings order; or

($b$) specify which, if any, of the payments in question do not constitute earnings.
\end{enumerate}

\subsection[23. Crown employment]{Crown employment}

23.  Where a liable person is in the employment of the Crown and a deduction from earnings order is made in respect of him then for the purposes of this Part—
\begin{enumerate}\item[]
($a$) the chief officer for the time being of the Department, office or other body in which the liable person is employed shall be treated as having the liable person in his employment (any transfer of the liable person from one Department, office or body to another being treated as a change of employment); and

($b$) any earnings paid by the Crown or a minister of the Crown, or out of the public revenue of the United Kingdom, shall be treated as paid by that chief officer.
\end{enumerate}

\subsection[24. Priority as between orders]{Priority as between orders}

24.—(1) Where an employer would, but for this paragraph, be obliged, on any pay-day, to make deductions under two or more deduction from earnings orders he shall—
\begin{enumerate}\item[]
($a$) deal with the orders according to the respective dates on which they were made, disregarding any later order until an earlier one has been dealt with;

($b$) deal with any later order as if the earnings to which it relates were the residue of the liable person’s earnings after the making of any deduction to comply with any earlier order.
\end{enumerate}

(2) Where an employer would, but for this paragraph, be obliged to comply with one or more deduction from earnings orders and one or more attachment of earnings orders he shall—
\begin{enumerate}\item[]
($a$) in the case of an attachment of earnings order which was made either wholly or in part in respect of the payment of a judgment debt or payments under an administration order, deal first with the deduction from earnings order or orders and thereafter with the attachment of earnings order as if the earnings to which it relates were the residue of the liable person’s earnings after the making of deductions to comply with the deduction from earnings order or orders;

($b$) in the case of any other attachment of earnings order, deal with the orders according to the respective dates on which they were made in like manner as under paragraph (1).
\end{enumerate}

“Attachment of earnings order” in this paragraph means an order made under the Attachment of Earnings Act 1971\footnote{\frenchspacing 1971 c. 32.} or under regulation 32 of the Community Charge (Administration and Enforcement) Regulations 1989\footnote{\frenchspacing S.I. 1989/438.}
or under regulation 37 of the Council Tax (Administration and Enforcement) Regulations 1992\footnote{\frenchspacing  S.I. 1992/613.}. % Words inserted (5.4.93) by SI 1993/913 reg 42

(3) Paragraph (2) does not apply to Scotland.

(4) In Scotland, where an employer would, but for this paragraph, be obliged to comply with one or more deduction from earnings orders and one or more diligences against earnings he shall deal first with the deduction from earnings order or orders and thereafter with the diligence against earnings as if the earnings to which the diligence relates were the residue of the liable person’s earnings after the making of deductions to comply with the deduction from earnings order or orders.

\amendment{
Words inserted in the definition of ``attachment of earnings order'' in reg. 24(2) (5.4.93) by the Child Support (Miscellaneous Amendments) Regulations 1993 reg. 42.
}

\subsection[25. Offences]{Offences}

25.  The following regulations are designated for the purposes of section 32(8) of the Act (offences relating to deduction from earnings orders)—
\begin{enumerate}\item[]
($a$) regulation 15(1) and (2);

($b$) regulation 16(1), (2) and (3);

($c$) regulation 19(1).
\end{enumerate}

\section[Part IV --- Liability orders]{Part IV\\*Liability orders}

\renewcommand\parthead{--- Part IV}

\subsection[26. Extent of this Part]{Extent of this Part}

26.  This Part, except regulation 29(2), does not apply to Scotland.

\subsection[27. Notice of intention to apply for a liability order]{Notice of intention to apply for a liability order}

27.—(1) The Secretary of State shall give the liable person at least 7 days notice of his intention to apply for a liability order under section 33(2) of the Act.

(2) Such notice shall set out the amount of child support maintenance which it is claimed has become payable by the liable person and has not been paid and the amount of any interest in respect of arrears payable under section 41(3) of the Act\footnote{\frenchspacing See The Child Support (Arrears, Interest and Adjustment of Maintenance Assessments) Regulations 1992, S.I. 1992/1816.}.

(3) Payment by the liable person of any part of the amounts referred to in paragraph (2) shall not require the giving of a further notice under paragraph (1) prior to the making of the application.

\subsection[28. Application for a liability order]{Application for a liability order}

28.—(1) An application for a liability order shall be by way of complaint for an order to the magistrates' court having jurisdiction in the area in which the liable person resides.

(2) An application under paragraph (1) may not be instituted more than 6 years after the day on which payment of the amount in question became due.

(3) A warrant shall not be issued under section 55(2) of the Magistrates' Courts Act 1980\footnote{\frenchspacing 1980 c. 43.} in any proceedings under this regulation.

\subsection[29. Liability orders]{Liability orders}

29.—(1) A liability order shall be made in the form prescribed in Schedule 1.

(2) A liability order made by a court in England or Wales or any corresponding order made by a court in Northern Ireland may be enforced in Scotland as if it had been made by the sheriff.

(3) A liability order made by the sheriff in Scotland or any corresponding order made by a court in Northern Ireland may, subject to paragraph (4), be enforced in England and Wales as if it had been made by a magistrates' court in England and Wales.

(4) A liability order made by the sheriff in Scotland or a corresponding order made by a court in Northern Ireland shall not be enforced in England or Wales unless registered in accordance with the provisions of 
%Part I 
Part II % Words substituted (5.4.93) by SI 1993/913 reg 43
of the Maintenance Orders Act 1950\footnote{\frenchspacing 14 Geo. 6 c. 37.} and for this purpose—
\begin{enumerate}\item[]
($a$) a liability order made by the sheriff in Scotland shall be treated as if it were a decree to which section 16(2)($b$) of that Act applies (decree for payment of aliment);

($b$) a corresponding order made by a court in Northern Ireland shall be treated as if it were an order to which section 16(2)($c$) of that Act applies (order for alimony, maintenance or other payments).
\end{enumerate}

\amendment{
Words substituted in reg. 29(4) (5.4.93) by the Child Support (Miscellaneous Amendments) Regulations 1993 reg. 43.
}

\subsection[30. Enforcement of liability orders by distress]{Enforcement of liability orders by distress}

30.—(1) A distress made pursuant to section 35(1) of the Act may be made anywhere in England and Wales.

(2) The person levying distress on behalf of the Secretary of State shall carry with him the written authorisation of the Secretary of State, which he shall show to the liable person if so requested, and he shall hand to the liable person or leave at the premises where the distress is levied—
\begin{enumerate}\item[]
($a$) copies of this regulation, regulation 31 and Schedule 2;

($b$) a memorandum setting out the amount which is the appropriate amount for the purposes of section 35(2) of the Act;

($c$) a memorandum setting out details of any arrangement entered into regarding the taking of possession of the goods distrained; and

($d$) a notice setting out the liable person’s rights of appeal under regulation 31 giving the Secretary of State’s address for the purposes of any appeal.
\end{enumerate}

(3) A distress shall not be deemed unlawful on account of any defect or want of form in the liability order.

(4) If, before any goods are seized, the appropriate amount (including charges arising up to the time of the payment or tender) is paid or tendered to the Secretary of State, the Secretary of State shall accept the amount and the levy shall not be proceeded with.

(5) Where the Secretary of State has seized goods of the liable person in pursuance of the distress, but before sale of those goods the appropriate amount (including charges arising up to the time of the payment or tender) is paid or tendered to the Secretary of State, the Secretary of State shall accept the amount, the sale shall not be proceeded with and the goods shall be made available for collection by the liable person.

\subsection[31. Appeals in connection with distress]{Appeals in connection with distress}

31.—(1) A person aggrieved by the levy of, or an attempt to levy, a distress may appeal to the magistrates' court having jurisdiction in the area in which he resides.

(2) The appeal shall be by way of complaint for an order.

(3) If the court is satisfied that the levy was irregular, it may—
\begin{enumerate}\item[]
($a$) order the goods distrained to be discharged if they are in the possession of the Secretary of State;

($b$) order an award of compensation in respect of any goods distrained and sold of an amount equal to the amount which, in the opinion of the court, would be awarded by way of special damages in respect of the goods if proceedings under section 35(6) of the Act were brought in trespass or otherwise in connection with the irregularity.
\end{enumerate}

(4) If the court is satisfied that an attempted levy was irregular, it may by order require the Secretary of State to desist from levying in the manner giving rise to the irregularity.

\subsection[32. Charges connected with distress]{Charges connected with distress}

32.  Schedule 2 shall have effect for the purpose of determining the amounts in respect of charges in connection with the distress for the purposes of section 35(2)($b$) of the Act.

\subsection[33. Application for warrant of commitment]{Application for warrant of commitment}

33.—(1) For the purposes of enabling an inquiry to be made under section 40 of the Act as to the liable person’s conduct and means, a justice of the peace having jurisdiction for the area in which the liable person resides may—
\begin{enumerate}\item[]
($a$) issue a summons to him to appear before a magistrates' court and (if he does not obey the summons) issue a warrant for his arrest; or

($b$) issue a warrant for his arrest without issuing a summons.
\end{enumerate}

(2) In any proceedings under section 40 of the Act, a statement in writing to the effect that wages of any amount have been paid to the liable person during any period, purporting to be signed by or on behalf of his employer, shall be evidence of the facts there stated.

(3) Where an application under section 40 of the Act has been made but no warrant of commitment is issued or term of imprisonment fixed, the application may be renewed on the ground that the circumstances of the liable person have changed.

\subsection[34. Warrant of commitment]{Warrant of commitment}

34.—(1) A warrant of commitment shall be in the form specified in Schedule 3, or in a form to the like effect.

(2) The amount to be included in the warrant under section 40(4)($a$)(ii) of the Act in respect of costs shall be such amount as in the view of the court is equal to the costs reasonably incurred by the Secretary of State in respect of the costs of commitment.

(3) A warrant issued under section 40 of the Act may be executed anywhere in England and Wales by any person to whom it is directed or by any constable acting within his police area.

(4) A warrant may be executed by a constable notwithstanding that it is not in his possession at the time but such warrant shall, on the demand of the person arrested, be shown to him as soon as possible.

(5) Where, after the issue of a warrant, part-payment of the amount stated in it is made, the period of imprisonment shall be reduced proportionately so that for the period of imprisonment specified in the warrant there shall be substituted a period of imprisonment of such number of days as bears the same proportion to the number of days specified in the warrant as the amount remaining unpaid under the warrant bears to the amount specified in the warrant.

(6) Where the part-payment is of such an amount as would, under paragraph (5), reduce the period of imprisonment to such number of days as have already been served (or would be so served in the course of the day of payment), the period of imprisonment shall be reduced to the period already served plus one day.

\bigskip

Signed by authority of the Secretary of State for Social Security.

{\raggedleft
\emph{Ann Widdecombe}\\*Parliamentary Under-Secretary of State,\\*Department of Social Security

}

17th August 1992

\clearpage

\part*{S C H E D U L E S}

\part[Schedule 1 --- Liability order prescribed form]{Schedule 1\\*Liability order prescribed form}

\renewcommand\parthead{--- Schedule 1}

\noindent
Section 33 of the Child Support Act 1991 and regulation 29(1) of the Child Support (Collection and Enforcement) Regulations 1992

\medskip

{\raggedleft \hspace{0.5\linewidth}\dotfill Magistrates' Court

}

\medskip

Date:

\medskip

Defendant:

\medskip

Address:

\medskip

On the complaint of the Secretary of State for Social Security that the sums specified below are due from the defendant under the Child Support Act 1991 and Part IV of the Child Support (Collection and Enforcement) Regulations 1992 and are outstanding, it is adjudged that the defendant is liable to pay the aggregate amount specified below.

\medskip

\noindent
\begin{tabulary}{0.9\linewidth}{lJJ}
Sum payable and outstanding \hspace{0.075\linewidth} &  --- & child support maintenance\\
&--- & interest\\
& --- & other periodical pay\-\textls[25]{ments collected by} virtue of section 30 of the Child Support Act 1991\\
\end{tabulary}

\medskip

Aggregate amount in respect of which the liability order is made:

\medskip

{\raggedleft Justice of the Peace

\medskip

[\emph{or} by order of the Court\\*Clerk of the Court]

}


\part[Schedule 2 --- Charges connected with distress]{Schedule 2\\*Charges connected with distress}

\renewcommand\parthead{--- Schedule 2}

1.  The sum in respect of charges connected with the distress which may be aggregated under section 35(2)($b$) of the Act shall be set out in the following Table—

\noindent
\begin{longtable}{p{135.43257pt}p{185.55615pt}}
\hline
(1)&(2)\\
\itshape Matter connected with distress&\itshape Charge\\
\hline
\endhead
\hline
\endlastfoot
\textls[75]{A {} For making a visit to} premises with a view to levying distress (whether the levy is made or not):&
Reasonable costs and fees incurred, but not exceeding an amount which, when aggregated with charges under this head for any previous visits made with a view to levying distress in relation to an amount in respect of which the liability order concerned was made, is not greater than the relevant amount calculated under paragraph 2(1) with respect to the visit.\\
B {} For levying distress:&
An amount (if any) which, when aggregated with charges under head A for any visits made with a view to levying distress in relation to an amount in respect of which the liability order concerned was made, is equal to the relevant amount calculated under paragraph 2(1) with respect to the levy.\\
%Head BB inserted (7.2.94) by SI 1994/227 reg 3(2)($a$)
BB {} For preparing and sending a letter advising the liable person that the written authorisation of the Secretary of State is with the person levying the distress and requesting the total sum due: & £10.00.\\
C {} For the removal and storage of goods for the purposes of sale:&
Reasonable costs and fees incurred.\\
\textls[75]{D {}  For the possession of} \textls[25]{goods as described in para\-}graph 2(3)—\\
\hspace{12pt}(i) for close possession (the person in possession on behalf of the Secretary of State to provide his own board):&
£4.50 per day.\\
\hspace{12pt}(ii) for walking possession:&
%45p per day.\\
10p per day.\\ % Words in head D(ii) substituted (7.2.94) by SI 1994/227 reg 3(2)($b$)
E {} For appraisement of an item distrained, at the request in writing of the liable person:&
Reasonable fees and expenses of the broker appraising.\\
F {} For other expenses of, and commission on, a sale by auction—\\
\hspace{12pt}(i) where the sale is held on the auctioneer’s premises:&
The auctioneer’s commission fee and out-of-pocket expenses (but not exceeding in aggregate 15 per cent.\ of the sum realised), together with reasonable costs and fees incurred in respect of advertising.\\
\hspace{12pt}(ii) where the sale is held on the liable person’s premises:&
The auctioneer’s commission fee (but not exceeding 7\textonehalf{} per cent.\ of the sum realised), together with the auctioneer’s out-of-pocket expenses and reasonable costs and fees incurred in respect of advertising.\\
G {} For other expenses incurred in connection with a proposed sale where there is no buyer in relation to it:&
Reasonable costs and fees incurred.\\
\end{longtable}

\amendment{
Words substituted in Head D(ii) of the Table and Head BB inserted into the Table in para. 1 (7.2.94) by the Child Support (Miscellaneous Amendments and Transitional Provisions) Regulations 1994 reg. 3(2) (subject to transitional provisions in reg. 12).
}

\medskip

2.—(1) In heads A and B of the Table to paragraph 1, “the relevant amount” with respect to a visit or a levy means—
\begin{enumerate}\item[]
($a$) where the sum due at the time of the visit or of the levy (as the case may be) does not exceed £100, £12.50;

($b$) where the sum due at the time of the visit or of the levy (as the case may be) exceeds £100, 12\textonehalf{} per cent.\ on the first £100 of the sum due, 4 per cent.\ on the next £400, 2\textonehalf{} per cent.\ on the next £1,500, 1 per cent.\ on the next £8,000 and \textonequarter{} per cent. on any additional sum;
\end{enumerate}
and the sum due at any time for these purposes means so much of the amount in respect of which the liability order concerned was made as is outstanding at the time.

(2) Where a charge has arisen under head B with respect to an amount, no further charge may be aggregated under heads A or B in respect of that amount.

(3) The Secretary of State takes close or walking possession of goods for the purposes of head D of the Table to paragraph 1 if he takes such possession in pursuance of an agreement which is made at the time that the distress is levied and which (without prejudice to such other terms as may be agreed) is expressed to the effect that, in consideration of the Secretary of State not immediately removing the goods distrained upon from the premises occupied by the liable person and delaying the sale of the goods, the Secretary of State may remove and sell the goods after a later specified date if the liable person has not by then paid the amount distrained for (including charges under this Schedule); and the Secretary of State is in close possession of goods on any day for these purposes if during the greater part of the day a person is left on the premises in physical possession of the goods on behalf of the Secretary of State under such an agreement.

\medskip

3.—(1) Where the calculation under this Schedule of a percentage of a sum results in an amount containing a fraction of a pound, that fraction shall be reckoned as a whole pound.

(2) In the case of dispute as to any charge under this Schedule, the amount of the charge shall be taxed.

(3) Such a taxation shall be carried out by the district judge of the county court for the district in which the distress is or is intended to be levied, and he may give such directions as to the costs of the taxation as he thinks fit; and any such costs directed to be paid by the liable person to the Secretary of State shall be added to the sum which may be aggregated under section 35(2) of the Act.

(4) References in the Table in paragraph 1 to costs, fees and expenses include references to amounts payable by way of value added tax with respect to the supply of goods or services to which the costs, fees and expenses relate.

\part[Schedule 3 --- Form of warrant of commitment]{Schedule 3\\*Form of warrant of commitment}

\renewcommand\parthead{--- Schedule 3}

\noindent
Section 40 of the Child Support Act 1991 and regulation 34(1) of the Child Support (Collection and Enforcement) Regulations 1992

\medskip

{\raggedleft \hspace{0.5\linewidth}\dotfill Magistrates' Court

}

\medskip

Date:

\medskip

Liable Person:

\medskip

Address:

\medskip

A liability order (``the order'') was made against the liable person by the [\phantom{Bolton}] Magistrates' Court on [\phantom{\today}] under section 33 of the Child Support Act 1991 (``the Act'') in respect of an amount of [\phantom{£100.00}].

The court is satisfied---
\begin{enumerate}
\item[]
(i) that the Secretary of state sought under section 35 of the Act to levy by distress the amount then outstanding in respect of which the order was made;
\end{enumerate}
[and/or]
\begin{enumerate}\item[]
that the Secretary of State sought under section 36 of the Act to recover through the [\phantom{Bolton}] County Court, by means of [garnishee proceedings] or [a charging order], the amount then outstanding in respect of which the order was made;

(ii) that such amount, or any portion of it, remains unpaid; and

(iii) having inquired in the liable person's presence as to his means and as to whether there was been [wilful refusal] or [culpable neglect] on his part, the court is of the opinion that there has been [wilful refusal] or [culpable neglect] on his part.
\end{enumerate}

The decision of the Court is that the liable person be [committed to prison] [detained] for [\phantom{7 days}] unless the aggregate amount mentioned below in respect of which this warrant is made is sooner paid.*

\medskip

This warrant is made in respect of---

Amount outstanding (including any interest, costs and charges):

Costs of commitment of the Secretary of State:

\medskip

Aggregate amount:

\medskip

And you [\emph{name of person or persons to whom warrant is directed}] are hereby required to take the liable person and convey him to [\emph{name of prison or place of detention}] and there deliver him to the [governor] [officer in charge] thereof; and you, the [governor] [officer in charge], to receive the liable person into your custody and keep him for [\emph{period of imprisonment}] from the date of his arrest under this warrant or until he be sooner discharged in due course of law.

\medskip

{\raggedleft Justice of the Peace

\medskip

[\emph{or} by order of the Court\\*Clerk of the Court]

}

\medskip

*\emph{Note:} The period of imprisonment will be reduced as provided by regulation 34(5) and (6) of the Child Support (Collection and Enforcement) Regulations 1992 if part-payment is made of the aggregate amount.


\part{Explanatory Note}

\renewcommand\parthead{--- Explanatory Note}

\subsection*{(This note is not part of the Regulations)}

 These Regulations make provision in relation to the collection and enforcement of child support maintenance under the Child Support Act 1991.

  Part I contains interpretation provisions and provisions relating to the service and receipt of notices and other documents.

  Part II (regulations 2 to 7) deals with the collection of child support maintenance and, in particular, makes provision in relation to the method and interval of payment and the notification of such matters to the liable person.

  Part III (regulations 8 to 25) makes provision in relation to deduction from earnings orders under which payments in respect of child support maintenance are to be deducted by employers from the earnings of liable persons. Regulation 25 creates offences where certain of the requirements are contravened.

  Part IV (regulations 26 to 34) and Schedules 1 to 3 make provision in relation to the obtaining of liability orders in the magistrates' courts and in relation to the enforcement of those orders by distraining the liable person’s goods or by commitment to prison.

\end{document}