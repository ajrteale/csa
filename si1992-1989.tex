\documentclass[12pt,a4paper]{article}

\newcommand\regstitle{The Child Support (Collection and Enforcement) Regulations 1992}

\newcommand\regsnumber{1992/1989}

\usepackage[oldrules]{optional}

\title{\regstitle}

%\opt{2012rules}{
%\title{Child Maintenance and~Other Payments Act 2008\\(2012 scheme version)}
%}

\author{S.I.~1992 No.~1989}

\date{Made 17th August 1992\\Laid before Parliament 26th August 1992\\Coming into~force 5th April 1993}

%\opt{oldrules}{\newcommand\versionyear{1993}}
%\opt{newrules}{\newcommand\versionyear{2003}}
%\opt{2012rules}{\newcommand\versionyear{2012}}

\usepackage{csa-regs}

\setlength\headheight{27.61603pt}

\begin{document}

\maketitle

\noindent
The Secretary of State for Social Security, in exercise of the powers conferred by sections~29(2) and~(3), 31(8), 32(1) to~(5) and~(7) to~(9), 34(1), 35(2), (7) and~(8), 39(1), (3) and~(4), 40(4), (8) and~(11), 51, 52 and~54 of the Child Support Act 1991\footnote{1991 c.~48. Section~54 is cited because of the meaning ascribed to the word “prescribed”.} and~of all other powers enabling him in that behalf, hereby makes the following Regulations:

{\sloppy

\tableofcontents

}

\setcounter{secnumdepth}{-2}

\section[Part I --- General]{Part I\\*General}

\subsection[1. Citation, commencement and~interpretation --- \emph{1993 scheme version}]{Citation, commencement and~interpretation\\*\emph{1993 scheme version}}

\renewcommand\parthead{--- Part I}

1.—(1) These Regulations may be cited as the Child Support (Collection and~Enforcement) Regulations 1992 and shall come into~force on 5th April 1993.

(2) In these Regulations “the Act” means the Child Support Act 1991.

(3) Where under any provision of the Act or of these Regulations—
\begin{enumerate}\item[]
($a$) any document or notice is given or sent to the Secretary of State, it shall be treated as having been given or sent on the day it is received by the Secretary of State; and

($b$) any document or notice is given or sent to~any other person, it shall, if sent by post to that person’s last known or notified address, be treated as having been given or sent on the second day after the day of posting, excluding any Sunday or any day which is a bank holiday under the Banking and~Financial Dealings Act 1971\footnote{\frenchspacing 1971 c. 80.}.
\end{enumerate}

(4) In these Regulations, unless the context otherwise requires, a reference—
\begin{enumerate}\item[]
($a$) to~a numbered Part is to the Part of these Regulations bearing that number;

($b$) to~a numbered regulation is to the regulation~in these Regulations bearing that number;

($c$) in a regulation~to~a numbered or lettered paragraph~or sub-paragraph is to the paragraph~or sub-paragraph~in that regulation~bearing that number or letter;

($d$) in a paragraph~to~a lettered or numbered sub-paragraph~is to the sub-paragraph~in that paragraph~bearing that letter or number;

($e$) to~a numbered Schedule~is to the Schedule~to these Regulations bearing that number.
\end{enumerate}

\subsection[1. Citation, commencement and~interpretation --- \emph{2003 scheme version}]{Citation, commencement and~interpretation\\*\emph{2003 scheme version}}

\renewcommand\parthead{--- Part I}

1.—(1) These Regulations may be cited as the Child Support (Collection and~Enforcement) Regulations 1992 and shall come into~force on 5th April 1993.

(2) In these Regulations—
\begin{enumerate}\item[]
“the Act” means the Child Support Act 1991;

“the 2000 Act” means the Child Support, Pensions and~Social Security Act 2000\footnote{2000 c.~19.};

“interest” means interest which has become payable under section~41 of the Act before its amendment by the 2000 Act; and

“voluntary payment” means a payment as defined in section~28J of the Act and~Regulations made under that section.
\end{enumerate}

% Reg 1(2A) inserted (3.3.03 for new-rules cases only) by SI 2001/162 reg 2(2)(b)
(2A) Except in relation to regulation~8(3)($a$)  and~Schedule~2, in these Regulations “fee” means an assessment fee or a collection fee, which for these purposes have the same meaning as in the Child Support Fees Regulations 1992\footnote{The definition of collection fee was amended by S.I.~1994/227.} prior to their revocation by the Child Support (Collection and~Enforcement and~Miscellaneous Amendments) Regulations 2000\footnote{S.I.~2001/162.}.

(3) Where under any provision of the Act or of these Regulations—
\begin{enumerate}\item[]
($a$) any document or notice is given or sent to the Secretary of State, it shall be treated as having been given or sent on the day it is received by the Secretary of State; and

($b$) any document or notice is given or sent to~any other person, it shall, if sent by post to that person’s last known or notified address, be treated as having been given or sent on 
%the second day after the day of posting, excluding any Sunday or any day which is a bank holiday under the Banking and~Financial Dealings Act 1971\footnote{\frenchspacing 1971 c. 80.}.
the day that it is posted.  % Words substituted (3.3.03 for new-rules cases only) by SI 2001/162 reg 2(2)(c)
\end{enumerate}

(4) In these Regulations, unless the context otherwise requires, a reference—
\begin{enumerate}\item[]
($a$) to~a numbered Part is to the Part of these Regulations bearing that number;

($b$) to~a numbered regulation is to the regulation~in these Regulations bearing that number;

($c$) in a regulation~to~a numbered or lettered paragraph~or sub-paragraph is to the paragraph~or sub-paragraph~in that regulation~bearing that number or letter;

($d$) in a paragraph~to~a lettered or numbered sub-paragraph~is to the sub-paragraph~in that paragraph~bearing that letter or number;

($e$) to~a numbered Schedule~is to the Schedule~to these Regulations bearing that number.
\end{enumerate}

\amendment{
Words substituted in reg.~1(3)(b), reg.~1(2A) inserted and reg.~1(2) substituted (3.3.03 for new-rules cases only) by the Child Support (Collection and~Enforcement and~Miscellaneous Amendments) Regulations 2000 reg.~2(2).
}

\section[Part II --- Collection of child support maintenance]{Part II\\*Collection of child support maintenance}

\renewcommand\parthead{--- Part II}

\subsection[2. Payment of child support maintenance]{Payment of child support maintenance}

2.—(1) Where a maintenance 
%assessment
\emph{calculation}  % Word substituted (3.3.03 for new-rules cases only) by SI 2001/162 reg 2(3)(a)
has been made under the Act and the case is one to which section~29 of the Act applies, the Secretary of State may specify that payments of child support maintenance shall be made by the liable person—
\begin{enumerate}\item[]
($a$) to the person caring for the child or children in question or, where an application has been made under section~7 of the Act, to the child who made the application;

($b$) to, or through, the Secretary of State; or

($c$) to, or through, such other person as the Secretary of State may, from time to time, specify.
\end{enumerate}

(2) In paragraph~(1) and~in the rest of this Part, “liable person” means a person liable to make payments of child support maintenance.

\amendment{
Word substituted in reg.~2(1) (3.3.03 for new-rules cases only) by the Child Support (Collection and~Enforcement and~Miscellaneous Amendments) Regulations 2000 reg.~2(3)(a) (subject to~savings in reg.~6).

For 1993 scheme cases, in reg.~2(1) for ``calculation'' read ``assessment''.
}

\subsection[3. Method of payment]{Method of payment}

3.—(1) Payments of child support maintenance% 
\emph{, penalty payments, interest and fees}  % Words inserted (3.3.03 for new-rules cases only) by SI 2001/162 reg 2(3)(b)(i)
shall be made by the liable person by whichever of the following methods the Secretary of State specifies as being appropriate in the circumstances—
\begin{enumerate}\item[]
($a$) by standing order;

($b$) by any other method which requires one person to~give his authority for payments to be made from an account of his to~an account of another’s on specific dates during the period for which the authority is in force and without the need for any further authority from him;

($c$) by an arrangement whereby one person gives his authority for payments to be made from an account of his, or on his behalf, to~another person or to~an account of that other person;

($d$) by cheque or postal order;

($e$) in cash;

% Reg 3(1)(f) inserted (3.3.03 for new-rules cases only, 12.7.06 for all other purposes) by SI 2001/162 reg 2(3)(b)(ii)
($f$) by debit card;

% Reg 3(1)(g), (h) inserted (12.7.06) by SI 2006/1520 reg 3(2)(a)
($g$) by credit card;

% Reg 3(1)(h) omitted (27.10.08) by SI 2008/2544 reg 2(2)(a)(i)
%($h$) by a voluntary deduction from earnings arrangement.

% Reg 3(1)(i) added (27.10.08) by SI 2008/2544 reg 2(2)(a)(ii)
($i$) by deduction from earnings order.
\end{enumerate}

% Reg 3(1A) inserted (3.3.03 for new-rules cases only, 12.7.06 for all other purposes) by SI 2001/162 reg 2(3)(c)
%(1A) In paragraph~(1), “debit card” means a card, operating as a substitute for a cheque, that can be used to~obtain cash or to make a payment at a point of sale whereby the card holder’s bank or building society account is debited without deferment of payment.

% Reg 3(1A) substituted (12.7.06) by SI 2006/1520 reg 3(2)(b)
(1A) In paragraph~(1)—
\begin{enumerate}\item[]
($a$) “debit card” means a card, operating as a substitute for a cheque, that can be used to~obtain cash or to make a payment at a point of sale whereby the card holder’s bank or building society account is debited without deferment of payment;

($b$) “credit card” means a card which is a credit-token within the meaning of section~14(1)($b$)  of the Consumer Credit Act 1974\footnote{1974 c.~39.}%;
%
% Reg 3(1A)(c) omitted (27.10.08) by SI 2008/2544 reg 2(2)(b)
%($c$) “voluntary deduction from earnings arrangement” means an arrangement under which the liable person and~his employer agree that payments of child support maintenance are to be deducted from the liable person’s earnings and~paid to the Secretary of State
.
\end{enumerate}

(2) The Secretary of State may direct a liable person to take all reasonable steps to~open an account from which payments under the maintenance 
%assessment
\emph{calculation}  % Word substituted (3.3.03 for new-rules cases only) by SI 2001/162 reg 2(3)(a)
may be made in accordance with the method of payment specified under paragraph~(1).

% Reg 3(3)--(9) added (27.10.08) by SI 2008/2544 reg 2(2)(c)
(3) Where the Secretary of State is considering specifying a deduction from earnings order by virtue of paragraph~(1)($i$), that method of payment is not to be used in any case where there is good reason not to use it.

(4) For the purposes of paragraph~(3) the matters which are to be taken into account in determining whether there is good reason not to use that method of payment are whether the making of a deduction from earnings order is likely to result in the disclosure of the parentage of a child and the impact of that disclosure on—
\begin{enumerate}\item[]
($a$) the liable person’s employment;

($b$) any relationship between the liable person and~a third party.
\end{enumerate}

(5) For the purposes of paragraph~(3) the circumstances in which good reason not to use that method of payment is to be regarded as existing are—
\begin{enumerate}\item[]
($a$) a member of the liable person’s or~parent with care’s family is employed by the same relevant employer as the liable person;

($b$) that family member’s employment requires knowledge of the relevant employer’s functions in giving effect to the deduction from earnings order; and

($c$) as a consequence of these circumstances the liable person’s employment status or family relationships may be adversely affected by the use of a deduction from earnings order as a method of payment.
\end{enumerate}

(6) For the purposes of paragraph~(3) the matters which are not to be taken into account in determining whether there is good reason not to use that method of payment are—
\begin{enumerate}\item[]
($a$) the liable person’s preference for a different method of payment;

($b$) the liable person’s preference for a relevant employer not to be informed about that parent’s maintenance liability;

($c$) that a third party would become aware of the liable person’s maintenance liability,
\end{enumerate}
unless they are relevant to any matter falling within paragraph~(4) or~circumstance falling within paragraph~(5).

(7) Where the Secretary of State is considering specifying the method of payment set out in paragraph~(1)($i$)  and~decides that there is no good reason not to use it, that method is not to be specified until—
\begin{enumerate}\item[]
($a$) the time within which an appeal against that decision may ordinarily be brought (including any period during which a further appeal may ordinarily be brought) has ended; or

($b$) if an appeal is brought on the grounds set out in regulation~22(3A), the time at which proceedings on the appeal (including any proceedings on a further appeal) have been concluded.
\end{enumerate}

(8) Nothing in this regulation is to prevent the Secretary of State exercising his powers under section~31 of the Act to make a deduction from earnings order where the Secretary of State considers it is appropriate in the circumstances of the case, unless he has specified a deduction from earnings order as a method of payment by virtue of paragraph~(1)($i$).

(9) In this regulation—
\begin{enumerate}\item[]
“couple” means—
\begin{enumerate}\item[]
($a$) 
a man and woman who are married to each other and~are members of the same household;

($b$) 
a man and woman who are not married to each other but are living together as husband~and wife;

($c$) 
two people of the same sex who are civil partners of each other and~are members of the same household; or

($d$) 
two people of the same sex who are not civil partners of each other but are living together as if they were civil partners,
\end{enumerate}
and~for the purposes of paragraph~($d$), two people of the same sex are to be regarded as living together as if they were civil partners if, but only if, they would be regarded as living together as husband~and wife were they instead two people of the opposite sex;

“family” means partner, parent, parent-in-law, son, son-in-law, daughter, daughter-in-law, step-parent, step-son, step-daughter, brother, sister, grand-parent, grand-child, uncle, aunt, nephew, niece, or if any of the preceding persons is one member of a couple, the other member of that couple;

“partner” means where a person is a member of a couple the other member of that couple; and

“relevant employer” means the employer of a liable person in respect of whom the order under section~31 of the Act would be made but for~paragraph~(3).
\end{enumerate}

\amendment{
Words inserted in reg.~3(1), word substituted in reg.~3(2) (3.3.03 for new-rules cases only) and reg. 3(1)(f), (1A) inserted (3.3.03 for new-rules cases only, 12.7.06 for all other purposes) by the Child Support (Collection and~Enforcement and~Miscellaneous Amendments) Regulations 2000 reg.~2(3)(a), (b), (c) (subject to~savings in reg.~6).

Reg. 3(1)(g), (h) inserted and reg.~3(1A) substituted (12.7.06) by the Child Support (Miscellaneous Amendments) Regulations 2006 reg.~3(2).

Reg.~3(1)(i), (3)--(9) added and reg.~3(1)(h), (1A)(c) omitted (27.10.08) by the Child Support (Miscellaneous Amendments) (No.~2) Regulations 2008 reg.~2(2).

For 1993 scheme cases---
\begin{enumerate}\item[]
the words ``, penalty payments, interest and fees'' in reg.~3(1) do not apply; and

the reference to a calculation in reg.~3(2) should be read as a reference to an assessment.
\end{enumerate}

}

\subsection[4. Interval of payment --- \emph{1993 and 2003 schemes version}]{Interval of payment\\*\emph{1993 and 2003 schemes version}}

4.—(1) The Secretary of State shall specify the day and~interval by reference to which payments of child support maintenance are to be made by the liable person and~may from time to time vary such day or interval.

%(2) In specifying the day and~interval of payment the Secretary of State shall have regard to~all the circumstances and~in particular to—
%\begin{enumerate}\item[]
%($a$) the needs of the person entitled to receive payment and the day and~interval by reference to which any other income is normally received by that person;
%
%($b$) the day and~interval by reference to which the liable person’s income is normally received; and
%
%($c$) any period necessary to enable the clearance of cheques or otherwise necessary to enable the transmission of payments to the person entitled to receive them.
%\end{enumerate}

% Reg 4(2) substituted (18.4.95) by SI 1995/1045 reg 12
(2) In specifying the day and~interval of payment the Secretary of State shall have regard to the following factors—
\begin{enumerate}\item[]
($a$) the circumstances of the person liable to make the payments and~in particular the day upon which and the interval at which any income is payable to that person;

($b$) any preference indicated by that person;

($c$) any period necessary to enable the clearance of cheques or otherwise necessary to enable the transmission of payments to the person entitled to receive them,
\end{enumerate}
and, subject to those factors, to~any other matter which appears to him to be relevant in the particular circumstances of the case.

\amendment{
Reg. 4(2) substituted (18.4.95) by the Child Support and~Income Support (Amendment) Regulations 1995 reg.~12.
}

\subsection[4. Payments to be scheduled over reference period --- \emph{2012 scheme version}]{Payments to be scheduled over reference period\\*\emph{2012 scheme version}}

4.---(1)  The Secretary of State may, for the purposes of determining the frequency and amount of the payments of child support maintenance required to be made by a liable person—
\begin{enumerate}\item[]
($a$) determine the total amount payable for the reference period on the assumption that the weekly rate of child support maintenance will not change over that period; and

($b$) require that amount to be paid by equal instalments over that period at intervals determined by the Secretary of State.
\end{enumerate}

(2) The reference period in relation to the maintenance calculation is, subject to paragraph (3), the period of 52 weeks mentioned in section 29(3A) of the Act beginning with—
\begin{enumerate}\item[]
($a$) the initial effective date (where it is the first such period in relation to the maintenance calculation); or

($b$) the review date.
\end{enumerate}

(3) In this regulation “initial effective date” and “review date” have the meanings given by regulations 12 and 19 of the Child Support Maintenance Calculation Regulations 2012\footnote{S.I.~2012/2677.} respectively.

\amendment{
Reg. 4 substituted (10.12.12 for 2012 scheme cases only) by the Child Support (Meaning of Child and New Calculation Rules) (Consequential and Miscellaneous Amendment) Regulations 2012 reg.~4(2).
}

\subsection[5. Transmission of payments]{Transmission of payments}

5.—%(1) Payments of child support maintenance made through the Secretary of State or other specified person shall be transmitted to the person entitled to receive them in whichever of the following ways the Secretary of State specifies as being appropriate in the circumstances—
%\begin{enumerate}\item[]
%($a$) by a transfer of credit to~an account nominated by the person entitled to receive the payments;
%
%($b$) by cheque, girocheque or other payable order;
%
%($c$) in cash.
%\end{enumerate}
%
% Reg 5(1) substituted (30.4.12) by SI 2012/712 reg 2
(1) Payments of child support maintenance made through the Secretary of State or other specified person shall be transmitted to the person entitled to receive them---
\begin{enumerate}\item[]
($a$) by transfer of credit to an account nominated by the person entitled to receive the payments; or

($b$) by means other than by transfer of credit as determined by the Secretary of State, where it appears to the Secretary of State to be necessary to do so in the circumstances of the particular case.
\end{enumerate}

(2) 
%The Secretary of State 
Subject to paragraph~(3), the Secretary of State  % Words substituted (18.4.95) by SI 1995/1045 reg 13(2)
shall specify the interval by reference to which the payments referred to in paragraph~(1) are to be transmitted to the person entitled to receive them.

%(3) The interval referred to in paragraph~(2) may differ from the interval referred to in regulation~4 and~may from time to time be varied by the Secretary of State.
%
%(4) In specifying the interval for transmission of payments the Secretary of State shall have regard to~all the circumstances and~in particular to—
%\begin{enumerate}\item[]
%($a$) the needs of the person entitled to receive payment and the interval by reference to which any other income is normally received by that person;
%
%($b$) any period necessary to enable the clearance of cheques or otherwise necessary to enable the transmission of payments to the person entitled to receive them.
%\end{enumerate}

%Reg 5(3), (4) substituted (18.4.95) by SI 1995/1045 reg 18(3)
(3) Except where the Secretary of State is satisfied in the circumstances of the case that it would cause undue hardship to either the person liable to make the payments or the person entitled to receive them, the interval referred to in paragraph~(2) shall not differ from the interval referred to in regulation~4.

(4) Subject to paragraph~(3) and regulation~4(2), the interval referred to in paragraph~(2) and that referred to in regulation~4 may be varied from time to time by the Secretary of State.

\amendment{
Words substituted in reg.~5(2) and reg.~5(3), (4) substituted (18.4.95) by the Child Support and~Income Support (Amendment) Regulations 1995 reg.~13.

Reg. 5(1) substituted (30.4.12) by the Child Support (Miscellaneous Amendments) Regulations 2012 reg.~2.
}

% Reg 5A inserted (3.3.03 for new-rules cases only) by SI 2001/162 reg 2(3)(d)
\subsection[5A. Voluntary payments]{Voluntary payments\\*\emph{This regulation does not apply to 1993 scheme cases.}}

5A.---(1)  Regulation 5(1) shall apply in relation to~voluntary payments as if—
\begin{enumerate}\item[]
($a$) for the words “Payment of child support maintenance” there were substituted the words “Voluntary payments”; and

($b$) the words “or other specified person” were omitted.
\end{enumerate}

(2) In determining when the Secretary of State shall transmit a voluntary payment to the person entitled to it, the Secretary of State shall have regard to the factor in regulation~4(2)($c$).

\amendment{
Reg. 5A inserted (3.3.03 for new-rules cases only) by the Child Support (Collection and~Enforcement and~Miscellaneous Amendments) Regulations 2000 reg.~2(3)(d) (subject to~savings in reg.~6).
}

\subsection[6. Representations about payment arrangements]{Representations about payment arrangements}

6.  The Secretary of State shall, insofar as is reasonably practicable, provide the liable person and the person entitled to receive the payments of child support maintenance with an opportunity to make representations with regard to the matters referred to in regulations 2 to~5 and the Secretary of State shall have regard to those representations in exercising his powers under those regulations.

\subsection[7. Notice to liable person as to requirements about payment --- \emph{1993 scheme version}]{Notice to liable person as to requirements about payment\\*\emph{1993 scheme version}}

7.—(1) The Secretary of State shall send the liable person a notice stating—
\begin{enumerate}\item[]
($a$) the amount of child support maintenance payable;

($b$) to whom it is to be paid;

($c$) the method of payment; and

($d$) the day and~interval by reference to which payments are to be made.
\end{enumerate}

(2) A notice under paragraph~(1) shall be sent to the liable person as soon as is reasonably practicable after—
\begin{enumerate}\item[]
($a$) the making of a maintenance assessment, and

($b$) after any change in the requirements referred to in any previous such notice.
\end{enumerate}

\subsection[7. Notice to liable person as to requirements about payment --- \emph{2003 scheme version}]{Notice to liable person as to requirements about payment\\*\emph{2003 scheme version}}

7.—(1) 
In the case of child support maintenance,  % Words inserted (3.3.03 for new-rules cases only) by SI 2001/162 reg 2(3)(e)(i)(aa)
the Secretary of State shall send the liable person a notice stating—
\begin{enumerate}\item[]
($a$) the amount of child support maintenance payable;

($b$) to whom it is to be paid;

($c$) the method of payment; 
%and

($d$) the day and~interval by reference to which payments are to be made%
; and

% Reg 7(1)(e) added (3.3.03 for new-rules cases only) by SI 2001/162 reg 2(3)(e)(i)(bb)
($e$) the amount of any payment of child support maintenance which is overdue and which remains outstanding.
\end{enumerate}

% Reg 7(1A) inserted (3.3.03 for new-rules cases only) by SI 2001/162 reg 2(3)(e)(ii)
(1A) In the case of penalty payments, interest or fees, the Secretary of State shall send the liable person a notice stating—
\begin{enumerate}\item[]
($a$) the amount of child support maintenance payable;

($b$) the amount of arrears;

($c$) the amount of the penalty payment, interest or fees to be paid, as the case may be;

($d$) the method of payment;

($e$) the day by which payment is to be made; and

($f$) information as to the provisions of sections~16 and~20 of the Act.
\end{enumerate}

(2) A notice under paragraph~(1) shall be sent to the liable person as soon as is reasonably practicable after—
\begin{enumerate}\item[]
($a$) the making of a maintenance 
%assessment
calculation%  % Word substituted (3.3.03 for new-rules cases only) by SI 2001/162 reg 2(3)(a)
, and

($b$) after any change in the requirements referred to in any previous such notice.
\end{enumerate}

% Reg 7(3) added (3.3.03 for new-rules cases only) by SI 2001/162 reg 2(3)(e)(iii)
(3) A notice under paragraph~(1A) shall be sent to the liable person as soon as reasonably practicable after the decision to require a payment of the penalty payment, interest or fees has been made.

\amendment{
Words inserted in reg.~7(1), word substituted in reg.~7(2) and reg.~7(1)(e), (1A), (3) inserted (3.3.03 for new-rules cases only) by the Child Support (Collection and~Enforcement and~Miscellaneous Amendments) Regulations 2000 reg.~2(3)(a), (e) (subject to~savings in reg.~6).
}

% Pt IIA inserted (3.3.03 for new-rules cases only) by SI 2001/162 reg 2(4)
\section[Part IIA --- Collection of penalty payments]{Part IIA\\*Collection of penalty payments}

\renewcommand\parthead{--- Part IIA}

\subsection[7A. Payment of a financial penalty]{Payment of a financial penalty\\*\emph{This regulation does not apply to 1993 scheme cases.}}

7A.---(1)  This regulation~applies where a maintenance calculation is, or has been, in force, the liable person is in arrears with payments of child support maintenance, and the Secretary of State requires the liable person to pay penalty payments to him.

(2) For the purposes of regulation~7(1)($e$)  a payment will be overdue if it is not received by the time that the next payment of child support maintenance is due.

(3) The Secretary of State may require a penalty payment to be made if the outstanding amount is not received within 7 days of the notification in regulation~7(1)($e$)  or if the liable person fails to pay all outstanding amounts due on dates and~of amounts as agreed between the liable person and the Secretary of State.

(4) Payments of a penalty payment shall be made within 14 days of the notification referred to in regulation~7(1A).

(5) In this Part a “liable person” means a person liable to make a penalty payment and~in Part II and~in this Part “penalty payment” is to be construed in accordance with section~41A of the Act.

\amendment{
Reg. 7A inserted (3.3.03 for new-rules cases only) by the Child Support (Collection and~Enforcement and~Miscellaneous Amendments) Regulations 2000 reg.~2(4) (subject to~savings in reg.~6).
}

\section[Part III --- Deduction from earnings orders]{Part III\\*Deduction from earnings orders}

\renewcommand\parthead{--- Part III}

\subsection[8. Interpretation of this Part --- \emph{1993 scheme version}]{Interpretation of this Part\\*\emph{1993 scheme version}}

8.—(1) For the purposes of this Part—
\begin{enumerate}\item[]
%Definition of ``defective'' inserted (18.4.95) by SI 1995/1045 reg 14(2)
“defective” means in relation to~a deduction from earnings order that it does not comply with the requirements of regulations 9 to~11 and such failure to comply has made it impracticable for the employer to comply with his obligations under the Act and these Regulations;

\begin{sloppypar}
“disposable income” means the amount determined under 
%regulation~12(1) 
regulation~12(1)($a$)  % Words substituted (18.4.95) by SI 1995/1045 reg 14(3)
of the Child Support (Maintenance Assessments and~Special Cases) Regulations 1992\footnote{\frenchspacing S.I.~1992/1815.};
\end{sloppypar}

“earnings” shall be construed in accordance with paragraphs (3) and~(4);

“exempt income” means the amount determined under regulation~9 of the Child Support (Maintenance Assessments and~Special Cases) Regulations 1992;

%Definition of ``interim maintenance assessment'' inserted (18.4.95) by SI 1995/1045 reg 14(4)
\pagebreak[3]

“interim maintenance assessment” means a Category A, Category B, Category C or Category D interim maintenance assessment within the meaning of 
%regulation~8(1B) 
regulation~8(3)  % Words substituted (7.10.96) by SI 1996/1945 reg 3
of the Child Support (Maintenance Assessment Procedure) Regulations 1992\footnote{\frenchspacing S.I.~1992/1813.};

“net earnings” shall be construed in accordance with paragraph~(5);

“normal deduction rate” means the rate specified in a deduction from earnings order (expressed as a sum of money per week, month or other period) at which deductions are to be made from the liable person’s net earnings;

“pay-day” in relation to~a liable person means an occasion on which earnings are paid to him or the day on which such earnings would normally fall to be paid;

“prescribed minimum amount” means the minimum amount prescribed in regulation~13 of the Child Support (Maintenance Assessments and Special Cases) Regulations 1992;

“protected earnings rate” means the level of earnings specified in a deduction from earnings order (expressed as a sum of money per week, month or other period) below which deductions of child support maintenance shall not be made for the purposes of this Part;

“protected income level” means the level of protected income determined in accordance with 
paragraphs (1) to~(5) of  % Words inserted (18.4.95) by SI 1995/1045 reg 14(5)
regulation~11 of the Child Support (Maintenance Assessments and~Special Cases) Regulations 1992.
\end{enumerate}

(2) For the purposes of this Part the relationship of employer and~employee shall be treated as subsisting between two persons if one of them, as a principal and~not as a servant or agent, pays to the other any sum defined as earnings under paragraph~(1) and~“employment”, “employer” and~“employee” shall be construed accordingly.

(3) Subject to paragraph~(4), “earnings” are any sums payable to a person—
\begin{enumerate}\item[]
($a$) by way of wages or salary (including any fees, bonus, commission, overtime pay or other emoluments payable in addition to wages or salary or payable under a contract of service);

($b$) by way of pension (including an annuity in respect of past service, whether or not rendered to the person paying the annuity, and~including periodical payments by way of compensation for the loss, abolition or relinquishment, or diminution in the emoluments, of any office or employment);

($c$) by way of statutory sick pay.
\end{enumerate}

(4) “Earnings” shall not include—
\begin{enumerate}\item[]
($a$) sums payable by any public department of the Government of Northern Ireland~or of a territory outside the United Kingdom;

($b$) pay or allowances payable to the liable person as a member of Her Majesty’s forces
other than pay or allowances payable by his employer to him as a special member of a reserve force (within the meaning of the Reserve Forces Act 1996\footnote{\frenchspacing 1996 c. 14.});  % Words inserted (6.4.99) by SI 1999/977 reg 2(2)

($c$) pension, allowances or benefit payable under any enactment relating to~social security;

($d$) pension or allowances payable in respect of disablement or disability;

($e$) guaranteed minimum pension within the meaning of the Social Security Pensions Act 1975\footnote{\frenchspacing 1975 c. 60.};

% Reg 8(4)(f) added (6.4.03) by SI 2003/328 reg 2
($f$) working tax credit payable under section~10 of the Tax Credits Act 2002\footnote{2002 c.\ 21.}.
\end{enumerate}

(5) “Net earnings” means the residue of earnings after deduction of—
\begin{enumerate}\item[]
($a$) income tax;

($b$) primary class I contributions under Part I of the Contributions and~Benefits Act 1992\footnote{\frenchspacing 1992 c. 4.};

($c$) amounts deductible by way of contributions to~a superannuation scheme which provides for the payment of annuities or 
%lumps 
lump % Word substituted (5.4.93) by SI 1993/913 reg 41.
sums—
\begin{enumerate}\item[]
(i) to the employee on his retirement at a specified age or on becoming incapacitated at some earlier age; or

(ii) on his death or otherwise, to his personal representative, widow, 
surviving civil partner,  % Words inserted (5.12.05) by SI 2005/2877 Sch 4 para 3
relatives or dependants.
\end{enumerate}
\end{enumerate}

\amendment{
Word substituted in reg.~8(5)($c$) (5.4.93) by the Child Support (Miscellaneous Amendments) Regulations 1993 reg.~41.

Words substituted in definition of ``disposable income'' in reg.~8(1), words inserted in definition of ``protected income level'' in reg.~8(1) and~definitions of ``defective'', ``interim maintenance assessment'' inserted in reg.~8(1) (18.4.95) by the Child Support and~Income Support (Amendment) Regulations 1995 reg.~14.

Words substituted in definition of ``interim maintenance assessment'' in reg.~8(1) (7.10.96) by the Child Support (Miscellaneous Amendments) Regulations 1996 reg.~3.

Words inserted in reg.~8(4)(b) (6.4.99) by the Child Support (Miscellaneous Amendments) Regulations 1999 reg.~2(2).

Reg. 8(4)(f) added (6.4.03) by the Child Support (Miscellaneous Amendments) Regulations 2003 reg.~2.

Words inserted in reg.~8(5)(c)(ii) (5.12.05) by the Civil Partnership (Pensions, Social Security and~Child Support) (Consequential, etc. Provisions) Order 2005 Sch. 4 para.~3.
}

\subsection[8. Interpretation of this Part --- \emph{2003 and 2012 scheme version}]{Interpretation of this Part\\*\emph{2003 and 2012 scheme version}}

8.—(1) For the purposes of this Part—
\begin{enumerate}\item[]
%Definition of ``defective'' inserted (18.4.95) by SI 1995/1045 reg 14(2)
“defective” means in relation to~a deduction from earnings order that it does not comply with the requirements of regulations 9 to~11 and such failure to comply has made it impracticable for the employer to comply with his obligations under the Act and these Regulations;

% Definition of ``disposable income'' omitted (3.3.03 for new-rules cases only) by SI 2001/162 reg 2(5)(a)(i)
%\begin{sloppypar}
%“disposable income” means the amount determined under 
%%regulation~12(1) 
%regulation~12(1)($a$)  % Words substituted (18.4.95) by SI 1995/1045 reg 14(3)
%of the Child Support (Maintenance Assessments and~Special Cases) Regulations 1992\footnote{\frenchspacing S.I.~1992/1815.};
%\end{sloppypar}

“earnings” shall be construed in accordance with paragraphs (3) and~(4);

% Definition of ``exempt income'' omitted (3.3.03 for new-rules cases only) by SI 2001/162 reg 2(5)(a)(i)
%“exempt income” means the amount determined under regulation~9 of the Child Support (Maintenance Assessments and~Special Cases) Regulations 1992;

%Definition of ``interim maintenance assessment'' inserted (18.4.95) by SI 1995/1045 reg 14(4), omitted (3.3.03 for new-rules cases only) by SI 2001/162 reg 2(5)(a)(i)
%\pagebreak[3]
%
%“interim maintenance assessment” means a Category A, Category B, Category C or Category D interim maintenance assessment within the meaning of 
%%regulation~8(1B) 
%regulation~8(3)  % Words substituted (7.10.96) by SI 1996/1945 reg 3
%of the Child Support (Maintenance Assessment Procedure) Regulations 1992\footnote{\frenchspacing S.I.~1992/1813.};

“net earnings” shall be construed in accordance with paragraph~(5);

“normal deduction rate” means the rate specified in a deduction from earnings order (expressed as a sum of money per \emph{week, month or other period}) at which deductions are to be made from the liable person’s net earnings;

“pay-day” in relation to~a liable person means an occasion on which earnings are paid to him or the day on which such earnings would normally fall to be paid;

% Definition of ``protected earnings proportion'' inserted (3.3.03 for new-rules cases only) by SI 2001/162 reg 2(5)(a)(ii)
“protected earnings proportion” means the proportion referred to in regulation~11(2).

% Definition of ``prescribed minimum amount'' omitted (3.3.03 for new-rules cases only) by SI 2001/162 reg 2(5)(a)(i)
%“prescribed minimum amount” means the minimum amount prescribed in regulation~13 of the Child Support (Maintenance Assessments and Special Cases) Regulations 1992;

% Definition of ``protected earnings rate'' omitted (3.3.03 for new-rules cases only) by SI 2001/162 reg 2(5)(a)(i)
%“protected earnings rate” means the level of earnings specified in a deduction from earnings order (expressed as a sum of money per week, month or other period) below which deductions of child support maintenance shall not be made for the purposes of this Part;

% Definition of ``protected income level'' omitted (3.3.03 for new-rules cases only) by SI 2001/162 reg 2(5)(a)(i)
%“protected income level” means the level of protected income determined in accordance with 
%paragraphs (1) to~(5) of  % Words inserted (18.4.95) by SI 1995/1045 reg 14(5)
%regulation~11 of the Child Support (Maintenance Assessments and~Special Cases) Regulations 1992.
\end{enumerate}

(2) For the purposes of this Part the relationship of employer and~employee shall be treated as subsisting between two persons if one of them, as a principal and~not as a servant or agent, pays to the other any sum defined as earnings under paragraph~(1) and~“employment”, “employer” and~“employee” shall be construed accordingly.

(3) Subject to paragraph~(4), “earnings” are any sums payable to a person—
\begin{enumerate}\item[]
($a$) by way of wages or salary (including any fees, bonus, commission, overtime pay or other emoluments payable in addition to wages or salary or payable under a contract of service);

($b$) by way of pension (including an annuity in respect of past service, whether or not rendered to the person paying the annuity, and~including periodical payments by way of compensation for the loss, abolition or relinquishment, or diminution in the emoluments, of any office or employment);

($c$) by way of statutory sick pay.
\end{enumerate}

(4) “Earnings” shall not include—
\begin{enumerate}\item[]
($a$) sums payable by any public department of the Government of Northern Ireland~or of a territory outside the United Kingdom;

($b$) pay or allowances payable to the liable person as a member of Her Majesty’s forces
other than pay or allowances payable by his employer to him as a special member of a reserve force (within the meaning of the Reserve Forces Act 1996\footnote{\frenchspacing 1996 c. 14.});  % Words inserted (6.4.99) by SI 1999/977 reg 2(2)

($c$) pension, allowances or benefit payable under any enactment relating to~social security;

($d$) pension or allowances payable in respect of disablement or disability;

($e$) guaranteed minimum pension within the meaning of the Social Security Pensions Act 1975\footnote{\frenchspacing 1975 c. 60.};

% Reg 8(4)(f) added (6.4.03) by SI 2003/328 reg 2
($f$) working tax credit payable under section~10 of the Tax Credits Act 2002\footnote{2002 c.\ 21.}.
\end{enumerate}

(5) “Net earnings” means the residue of earnings after deduction of—
\begin{enumerate}\item[]
($a$) income tax;

($b$) primary class I contributions under Part I of the Contributions and~Benefits Act 1992\footnote{\frenchspacing 1992 c. 4.};

($c$) amounts deductible by way of contributions to~a superannuation scheme which provides for the payment of annuities or 
%lumps 
lump % Word substituted (5.4.93) by SI 1993/913 reg 41.
sums—
\begin{enumerate}\item[]
(i) to the employee on his retirement at a specified age or on becoming incapacitated at some earlier age; or

(ii) on his death or otherwise, to his personal representative, widow, 
surviving civil partner,  % Words inserted (5.12.05) by SI 2005/2877 Sch 4 para 3
relatives or dependants.
\end{enumerate}
\end{enumerate}

\amendment{
Word substituted in reg.~8(5)($c$) (5.4.93) by the Child Support (Miscellaneous Amendments) Regulations 1993 reg.~41.

Words substituted in definition of ``disposable income'' in reg.~8(1), words inserted in definition of ``protected income level'' in reg.~8(1) and~definitions of ``defective'', ``interim maintenance assessment'' inserted in reg.~8(1) (18.4.95) by the Child Support and~Income Support (Amendment) Regulations 1995 reg.~14.

Words substituted in definition of ``interim maintenance assessment'' in reg.~8(1) (7.10.96) by the Child Support (Miscellaneous Amendments) Regulations 1996 reg.~3.

Words inserted in reg.~8(4)(b) (6.4.99) by the Child Support (Miscellaneous Amendments) Regulations 1999 reg.~2(2).

Definition of ``protected earnings proportion'' inserted in reg.~8(1) and~definitions of ``disposable income'', ``exempt income'', ``interim maintenance assessment'', ``prescribed minimum amount'', ``protected earnings rate'' and~``protected income level'' omitted in reg.~8(1) (3.3.03 for new-rules cases only) by the Child Support (Collection and~Enforcement and~Miscellaneous Amendments) Regulations 2000 reg.~2(5)(a) (subject to~savings in reg.~6).

Reg. 8(4)(f) added (6.4.03) by the Child Support (Miscellaneous Amendments) Regulations 2003 reg.~2.

Words inserted in reg.~8(5)(c)(ii) (5.12.05) by the Civil Partnership (Pensions, Social Security and~Child Support) (Consequential, etc. Provisions) Order 2005 Sch. 4 para.~3.

Words substituted in definition of ``normal deduction rate'' in reg.~8(1) (10.12.12 for 2012 scheme cases only) by the Child Support (Meaning of Child and New Calculation Rules) (Consequential and Miscellaneous Amendment) Regulations 2012 reg.~4(3).

For 2012 scheme cases the definition of ``normal deduction rate'' in reg.~8(1) should be read as follows:
\begin{quotation}
“normal deduction rate” means the rate specified in a deduction from earnings order (expressed as a sum of money per %week, month or other period
month and the equivalent of that sum for a 1, 2 and 4 week period%  % Words substituted (10.12.12 for 2012 scheme cases only) by SI 2012/2785 reg 4(3)
) at which deductions are to be made from the liable person’s net earnings;
\end{quotation}
}

\subsection[9. Deduction from earnings orders]{Deduction from earnings orders}

9.  A deduction from earnings order shall specify—
\begin{enumerate}\item[]
($a$) the name and~address of the liable person;

($b$) the name of the employer at whom it is directed;

($c$) where known, the liable person’s place of work, the nature of his work and~any works or pay number;

% Reg 9($cc$) inserted (22.1.96) by SI 1995/3261 reg 6
($cc$) where known, the liable person’s national insurance number;

%($d$) the normal deduction rate;
%
%($e$) 
%except in the case of a Category A or Category B interim maintenance assessment within the meaning of regulation~8(1A) and~(1B) of the Child Support (Maintenance Assessment Procedure) Regulations 1992, % Words inserted in reg 9($e$) (7.2.94) by SI 1994/227 reg 3(1)
%the protected earnings rate;

% Reg 9($d$), ($e$) substituted (18.4.95) by SI 1995/1045 reg 15
($d$) the normal deduction rate or rates and the date upon which each is to take effect;

($e$) the 
%protected earnings rate
\emph{protected earnings proportion};  % Words substituted (3.3.03 for new-rules cases only) by SI 2001/162 reg 2(5)(b)

($f$) the address to which amounts deducted from earnings are to be sent.
\end{enumerate}

\amendment{
%Words inserted in reg.~9($e$) (7.2.94) by the Child Support (Miscellaneous Amendments and~Transitional Provisions) Regulations 1994 reg.~3(1) (subject to transitional provisions in reg.~12).

Reg. 9($d$), ($e$) substituted (18.4.95) by the Child Support and~Income Support (Amendment) Regulations 1995 reg.~15.

Reg. 9($cc$) inserted (22.1.96) by the Child Support (Miscellaneous Amendments) (No.~2) Regulations 1995 reg.~6.

Words substituted in reg.~9(e) (3.3.03 for new-rules cases only) by the Child Support (Collection and~Enforcement and~Miscellaneous Amendments) Regulations 2000 reg.~2(5)(b) (subject to~savings in reg.~6).

For 1993 scheme cases the reference in reg.~9(e) to the protected earnings proportion should be read as a reference to the protected earnings rate.
}

\subsection[10. Normal deduction rate --- \emph{1993 scheme version}]{Normal deduction rate\\*\emph{1993 scheme version}}

10.—(1) The period by reference to which 
%the normal deduction rate 
a normal deduction rate  % Words substituted (18.4.95) by SI 1995/1045 reg 16(2)
is set shall be the period by reference to which the liable person’s earnings are normally paid or, if none, such other period as the Secretary of State may specify.

(2) The Secretary of State, in specifying the normal deduction rate, shall not include any amount in respect of arrears or interest% 
, in a case where there is a current assessment,  % Words inserted (18.4.95) by SI 1995/1045 reg 16(3)($a$)
if, 
%at the date of making of the current assessment—
at the date of making of any current maintenance assessment other than an interim maintenance assessment---  % Words substituted (18.4.95) by SI 1995/1045 reg 16(3)($b$)
\begin{enumerate}\item[]
($a$) the liable person’s disposable income was below the level specified in paragraph~(3); or

($b$) the deduction of such an amount from the liable person’s disposable income would have reduced his disposable income below the level specified in paragraph~(3).
\end{enumerate}

(3) The level referred to in paragraph~(2) is the liable person’s protected income level less the prescribed minimum amount.

\amendment{
Words substituted in reg.~10(1), (2) and words inserted in reg.~10(2) (18.4.95) by the Child Support and~Income Support (Amendment) Regulations 1995 reg.~16.
}

\subsection[10. Normal deduction rate --- \emph{2003 scheme version}]{Normal deduction rate\\*\emph{2003 scheme version}}

10.—(1) The period by reference to which 
%the normal deduction rate 
a normal deduction rate  % Words substituted (18.4.95) by SI 1995/1045 reg 16(2)
is set shall be the period by reference to which the liable person’s earnings are normally paid or, if none, such other period as the Secretary of State may specify.

% Reg 10(2), (3) omitted (3.3.03 for new-rules cases only) by SI 2001/162 reg 2(5)(c)
%(2) The Secretary of State, in specifying the normal deduction rate, shall not include any amount in respect of arrears or interest% 
%, in a case where there is a current assessment,  % Words inserted (18.4.95) by SI 1995/1045 reg 16(3)($a$)
%if, 
%%at the date of making of the current assessment—
%at the date of making of any current maintenance assessment other than an interim maintenance assessment---  % Words substituted (18.4.95) by SI 1995/1045 reg 16(3)($b$)
%\begin{enumerate}\item[]
%($a$) the liable person’s disposable income was below the level specified in paragraph~(3); or
%
%($b$) the deduction of such an amount from the liable person’s disposable income would have reduced his disposable income below the level specified in paragraph~(3).
%\end{enumerate}
%
%(3) The level referred to in paragraph~(2) is the liable person’s protected income level less the prescribed minimum amount.

\amendment{
Words substituted in reg.~10(1), (2) and words inserted in reg.~10(2) (18.4.95) by the Child Support and~Income Support (Amendment) Regulations 1995 reg.~16.

Reg. 10(2), (3) omitted (3.3.03 for new-rules cases only) by the Child Support (Collection and~Enforcement and~Miscellaneous Amendments) Regulations 2000 reg.~2(5)(c) (subject to~savings in reg.~6).
}

\subsection[10. Normal deduction rate --- \emph{2012 scheme version}]{Normal deduction rate\\*\emph{2012 scheme version}}

% Reg 10 substituted (10.12.12 for 2012 scheme cases only) by SI 2012/2785 reg 4(4)
10.---(1)  The period by reference to which the normal deduction rate is set must be the period by reference to which the liable person is normally paid where that period is a 1, 2 or 4 weekly or monthly period.

(2) The employer must select the normal deduction rate which applies depending on the period by reference to which the liable person’s earnings are normally paid.

(3) Where the liable person is paid by reference to a period other than at a 1, 2 or 4 weekly or monthly period, the Secretary of State must discharge the deduction from earnings order in accordance with regulation 20.

\amendment{
Reg,~10 substituted (10.12.12 for 2012 scheme cases only) by the Child Support (Meaning of Child and New Calculation Rules) (Consequential and Miscellaneous Amendment) Regulations 2012 reg.~4(4).
}

\subsection[11. Protected earnings rate --- \emph{1993 scheme version}]{Protected earnings rate\\*\emph{1993 scheme version}}

11.—(1) The period by reference to which the protected earnings rate is set shall be the same as the period by reference to which the normal deduction rate is set under regulation~10(1).

(2) The amount to be specified as the protected earnings rate in respect of any period shall, except where 
%paragraph~(3) or paragraph~(4) 
paragraph~(3), paragraph~(4) or~paragraph~(5)  % Words substituted (27.10.08) by SI 2008/2544 reg 2(3)(a)
applies,  % Words inserted (18.4.95) by SI 1995/1045 reg 17(2), omitted (3.3.03 for new-rules cases only) by SI 2001/162 reg 2(5)(d)(iii)(aa)
be an amount equal to the liable person’s exempt income in respect of that period as calculated at the date of the current assessment.

%Reg 11(3), (4) inserted (18.4.95) by SI 1995/1045 reg 17(3)
(3) Where an interim maintenance assessment%
, except a Category B interim maintenance assessment,  % Words inserted (5.8.96) by SI 1996/1945 reg 4
is in force the protected earnings rate shall be—
\begin{enumerate}\item[]
($a$) where there is some knowledge of the liable person’s circumstances, the aggregate of the following amounts at the date of the making of the assessment—
\begin{enumerate}\item[]
(i) the personal allowance applicable by virtue of paragraph~1(1)($e$) of Schedule~2 to the Income Support (General) Regulations 1987\footnote{\frenchspacing S.I.~1987/1967. Relevant amending instruments are 1988/663, 1989/1678.} (in this paragraph~referred to~as “the relevant Schedule”) or if he is known to have a partner, that applicable for a couple under paragraph~1(3)($c$) of that Schedule;

(ii) the personal allowance applicable by virtue of the relevant Schedule~in respect of any child or young person who is known to be living with the relevant person (and where the age of the child or young person is not known it shall be assumed to be less than 11);

(iii) the amount of any premium applicable by virtue of the relevant Schedule~which is known to be applicable in the circumstances of the case; and

(iv) £30;
\end{enumerate}

($b$) in any other case the personal allowance specified in paragraph~1(1)($e$) of the relevant Schedule~at the date mentioned in sub-\hspace{0pt}paragraph~($a$), plus £30.
\end{enumerate}

%(4) Where there is a liability to make payments of child support maintenance but no maintenance assessment is in force, the protected earnings rate shall be—
%\begin{enumerate}\item[]
%($a$) except where the last maintenance assessment was an interim maintenance assessment of Category A or Category C—
%\begin{enumerate}\item[]
%(i) where the absent parent produces evidence sufficient to~satisfy the child support officer that his circumstances have changed since the last assessment or review under section~16
%by a child support officer of a maintenance assessment the effective date of which is on or before 8th December 1996 or a revision by the Secretary of State under that section~after 6th December 1998, or a review under section~ % Words inserted (7.12.98) by SI 1998/2799 reg 3(3)
%17, 18 or 19 of the Act, a figure equal to the figure that would be his exempt income if the assessment were then being reviewed; or
%
%(ii) in any other case an amount equal to the amount of exempt income produced by the last assessment or review under section~16
%by a child support officer of a maintenance assessment the effective date of which is on or before 8th December 1996 or a revision by the Secretary of State under that section~after 6th December 1998, or a review under section~ % Words inserted (7.12.98) by SI 1998/2799 reg 3(3)
%17, 18 or 19 of the Act applicable in his case;
%\end{enumerate}
%
%($b$) in the case of an interim maintenance assessment of Category A or Category C, the amount produced by the application of the provisions of paragraph~(3) above in his case.
%\end{enumerate}

% Reg 11(4) substituted (1.6.99) by SI 1999/1510 reg 29
(4) Where there is a liability to make payments of child support maintenance but no maintenance assessment is in force—
\begin{enumerate}\item[]
($a$) in a case where the last maintenance assessment was a Category A or Category C interim maintenance assessment, the protected earnings rate shall be the amount which would be produced by the application of the provisions of paragraph~(3) if a Category A or Category C interim maintenance assessment were in force;

($b$) subject to~sub-paragraph~($a$), in a case where the absent parent provides sufficient evidence to~satisfy the Secretary of State that his circumstances have changed since the last occasion on which his exempt income was calculated for the purposes of a decision under the Act, the protected earnings rate shall be the exempt amount as it would be calculated in consequence of that change of circumstances if regulation~9 of the Child Support (Maintenance Assessments and~Special Cases) Regulations 1992\footnote{\frenchspacing S.I.~1992/1815; regulation~9 was amended by S.I.~1993/913, 1995/1045, 1995/3261, 1996/1803, 1996/1945, 1996/2907 and~1998/58.} applied in his case; and

($c$) in any other case, the protected earnings rate shall be the amount of the liable person’s exempt income as it was on the last occasion that amount was calculated for the purposes of a decision under the Act.
\end{enumerate}

% Reg 11(5)--(8) added (27.10.08) by SI 2008/2544 reg 2(3)(b)
(5) This paragraph~applies where the liable person—
\begin{enumerate}\item[]
($a$) has more than one employer; and

($b$) the Secretary of State makes an order under section~31 of the Act (“an order”) against that person in respect of more than one employer.
\end{enumerate}

(6) Where paragraph~(5) applies, the protected earnings rate for~each order is to be divided proportionately between the earnings of the liable person with each employer in accordance with paragraph~(7).

(7) The amount to be specified as the protected earnings rate in respect of any period in an order is an amount equal to the percentage of the liable person’s exempt income which is the same as the amounts earned with an employer, as a percentage of the total earnings with the employers.

(8) Any reference to an “employer” in paragraphs~(6) and~(7) is to be construed as a reference to an employer subject to an order made in respect of a liable person.

\amendment{
Words inserted in reg.~11(2) and reg.~11(3), (4) inserted (18.4.95) by the Child Support and~Income Support (Amendment) Regulations 1995 reg.~17.

Words inserted in reg.~11(3) (5.8.96) by the Child Support (Miscellaneous Amendments) Regulations 1996 reg.~4.

%Words inserted in reg.~11(4)(a)(i), (ii) (7.12.98) by the Child Support (Miscellaneous Amendments) (No.~2) Regulations 1998 reg.~3(3).

Reg. 11(4) substituted (1.6.99) by the Social Security Act 1998 (Commencement No.~7 and~Consequential and~Transitional Provisions) Order 1999 reg.~29.

\begin{sloppypar}
Words substituted in reg.~11(2) and reg.~11(5)--(8) added (27.10.08 for old-rules cases only) by the Child Support (Miscellaneous Amendments) (No.~2) Regulations 2008 reg.~2(3).
\end{sloppypar}
}

\subsection[11. 
%Protected earnings rate
Protected earnings proportion
--- \emph{2003 scheme version}
]{%
%Protected earnings rate
Protected earnings proportion%
\\*\emph{2003 scheme version}}  % Words substituted in heading to reg 11 (3.3.03 for new-rules cases only) by SI 2001/162 reg 2(5)(d)(i)

11.—(1) The period by reference to which the 
%protected earnings rate
protected earnings proportion  % Words substituted (3.3.03 for new-rules cases only) by SI 2001/162 reg 2(5)(d)(ii)
is set shall be the same as the period by reference to which the normal deduction rate is set under regulation~10(1).

(2) The amount to be specified as the 
%protected earnings rate
protected earnings proportion  % Words substituted (3.3.03 for new-rules cases only) by SI 2001/162 reg 2(5)(d)(ii)
in respect of any period shall
%, except where paragraph~(3) or paragraph~(4) applies,  % Words inserted (18.4.95) by SI 1995/1045 reg 17(2), omitted (3.3.03 for new-rules cases only) by SI 2001/162 reg 2(5)(d)(iii)(aa)
be an amount equal to
%the liable person’s exempt income
60\% of the liable person’s net earnings  % Words substituted (3.3.03 for new-rules cases only) by SI 2001/162 reg 2(5)(d)(iii)(bb)
in respect of that period 
%as calculated at the date of the current 
%assessment%
%maintenance calculation
%.%  % Words substituted (3.3.03 for new-rules cases only) by SI 2001/162 reg 2(5)(d)(iii)(cc)
as calculated—
\begin{enumerate}\item[]
    ($a$) 
    at the date of the current maintenance calculation; or

    ($b$) 
    if the deduction from earnings order relates only to~arrears of child support maintenance, at the date on which the order is made or varied.
\end{enumerate}  % Words substituted (12.7.06 for new-rules cases only) by SI 2006/1520 reg 3(3)

%Reg 11(3), (4) inserted (18.4.95) by SI 1995/1045 reg 17(3), omitted (3.3.03 for new-rules cases only) by SI 2001/162 reg 2(5)(d)(iv)
%(3) Where an interim maintenance assessment%
%, except a Category B interim maintenance assessment,  % Words inserted (5.8.96) by SI 1996/1945 reg 4
%is in force the protected earnings rate shall be—
%\begin{enumerate}\item[]
%($a$) where there is some knowledge of the liable person’s circumstances, the aggregate of the following amounts at the date of the making of the assessment—
%\begin{enumerate}\item[]
%(i) the personal allowance applicable by virtue of paragraph~1(1)($e$) of Schedule~2 to the Income Support (General) Regulations 1987\footnote{\frenchspacing S.I.~1987/1967. Relevant amending instruments are 1988/663, 1989/1678.} (in this paragraph~referred to~as “the relevant Schedule”) or if he is known to have a partner, that applicable for a couple under paragraph~1(3)($c$) of that Schedule;
%
%(ii) the personal allowance applicable by virtue of the relevant Schedule~in respect of any child or young person who is known to be living with the relevant person (and where the age of the child or young person is not known it shall be assumed to be less than 11);
%
%(iii) the amount of any premium applicable by virtue of the relevant Schedule~which is known to be applicable in the circumstances of the case; and
%
%(iv) £30;
%\end{enumerate}
%
%($b$) in any other case the personal allowance specified in paragraph~1(1)($e$) of the relevant Schedule~at the date mentioned in sub-\hspace{0pt}paragraph~($a$), plus £30.
%\end{enumerate}

%(4) Where there is a liability to make payments of child support maintenance but no maintenance assessment is in force, the protected earnings rate shall be—
%\begin{enumerate}\item[]
%($a$) except where the last maintenance assessment was an interim maintenance assessment of Category A or Category C—
%\begin{enumerate}\item[]
%(i) where the absent parent produces evidence sufficient to~satisfy the child support officer that his circumstances have changed since the last assessment or review under section~16
%by a child support officer of a maintenance assessment the effective date of which is on or before 8th December 1996 or a revision by the Secretary of State under that section~after 6th December 1998, or a review under section~ % Words inserted (7.12.98) by SI 1998/2799 reg 3(3)
%17, 18 or 19 of the Act, a figure equal to the figure that would be his exempt income if the assessment were then being reviewed; or
%
%(ii) in any other case an amount equal to the amount of exempt income produced by the last assessment or review under section~16
%by a child support officer of a maintenance assessment the effective date of which is on or before 8th December 1996 or a revision by the Secretary of State under that section~after 6th December 1998, or a review under section~ % Words inserted (7.12.98) by SI 1998/2799 reg 3(3)
%17, 18 or 19 of the Act applicable in his case;
%\end{enumerate}
%
%($b$) in the case of an interim maintenance assessment of Category A or Category C, the amount produced by the application of the provisions of paragraph~(3) above in his case.
%\end{enumerate}
%
%% Reg 11(4) substituted (1.6.99) by SI 1999/1510 reg 29
%(4) Where there is a liability to make payments of child support maintenance but no maintenance assessment is in force—
%\begin{enumerate}\item[]
%($a$) in a case where the last maintenance assessment was a Category A or Category C interim maintenance assessment, the protected earnings rate shall be the amount which would be produced by the application of the provisions of paragraph~(3) if a Category A or Category C interim maintenance assessment were in force;
%
%($b$) subject to~sub-paragraph~($a$), in a case where the absent parent provides sufficient evidence to~satisfy the Secretary of State that his circumstances have changed since the last occasion on which his exempt income was calculated for the purposes of a decision under the Act, the protected earnings rate shall be the exempt amount as it would be calculated in consequence of that change of circumstances if regulation~9 of the Child Support (Maintenance Assessments and~Special Cases) Regulations 1992\footnote{\frenchspacing S.I.~1992/1815; regulation~9 was amended by S.I.~1993/913, 1995/1045, 1995/3261, 1996/1803, 1996/1945, 1996/2907 and~1998/58.} applied in his case; and
%
%($c$) in any other case, the protected earnings rate shall be the amount of the liable person’s exempt income as it was on the last occasion that amount was calculated for the purposes of a decision under the Act.
%\end{enumerate}

% Reg 11(5)--(8) added (27.10.08 for old-rules cases only) by SI 2008/2544 reg 2(3)(b)
%(5) This paragraph~applies where the liable person—
%\begin{enumerate}\item[]
%($a$) has more than one employer; and
%
%($b$) the Secretary of State makes an order under section~31 of the Act (“an order”) against that person in respect of more than one employer.
%\end{enumerate}
%
%(6) Where paragraph~(5) applies, the protected earnings rate for~each order is to be divided proportionately between the earnings of the liable person with each employer in accordance with paragraph~(7).
%
%(7) The amount to be specified as the protected earnings rate in respect of any period in an order is an amount equal to the percentage of the liable person’s exempt income which is the same as the amounts earned with an employer, as a percentage of the total earnings with the employers.
%
%(8) Any reference to an “employer” in paragraphs~(6) and~(7) is to be construed as a reference to an employer subject to an order made in respect of a liable person.

\amendment{
Words inserted in reg.~11(2) and reg.~11(3), (4) inserted (18.4.95) by the Child Support and~Income Support (Amendment) Regulations 1995 reg.~17.

Words inserted in reg.~11(3) (5.8.96) by the Child Support (Miscellaneous Amendments) Regulations 1996 reg.~4.

%Words inserted in reg.~11(4)(a)(i), (ii) (7.12.98) by the Child Support (Miscellaneous Amendments) (No.~2) Regulations 1998 reg.~3(3).

Reg. 11(4) substituted (1.6.99) by the Social Security Act 1998 (Commencement No.~7 and~Consequential and~Transitional Provisions) Order 1999 reg.~29.

Words substituted in reg.~11(1), (2), words omitted in reg.~11(2), words in heading to reg.~11 substituted and reg.~11(3), (4) omitted (3.3.03 for new-rules cases only) by the Child Support (Collection and~Enforcement and~Miscellaneous Amendments) Regulations 2000 reg.~2(5)(d) (subject to~savings in reg.~6).

Words substituted in reg.~11(2) (12.7.06 for new-rules cases only) by the Child Support (Miscellaneous Amendments) Regulations 2006 reg.~3(3).
}

% Reg 11 substituted (10.12.12 for 2012 scheme cases only) by SI 2012/2785 reg 4(5)
\subsection[11. Protected earnings proportion --- \emph{2012 scheme version}]{Protected earnings proportion\\*\emph{2012 scheme version}}

11.---(1)  The period by reference to which the protected earnings proportion is set must be the same as the period by reference to which the normal deduction rate is set in accordance with regulation 10(1).

(2) The protected earnings proportion in respect of any period shall be 60\% of the liable person’s net earnings in respect of that period as calculated at the pay-day of the liable person by the employer.

\amendment{
Reg,~11 substituted (10.12.12 for 2012 scheme cases only) by the Child Support (Meaning of Child and New Calculation Rules) (Consequential and Miscellaneous Amendment) Regulations 2012 reg.~4(5).
}

\subsection[12. Amount to be deducted by employer --- \emph{1993 scheme version}]{Amount to be deducted by employer\\*\emph{1993 scheme version}}

12.—(1) Subject to the provisions of this regulation, an employer who has been served with a copy of a deduction from earnings order in respect of a liable person in his employment shall, each pay-day, make a deduction from the net earnings of that liable person of an amount equal to the normal deduction rate.

(2) Where the deduction of the normal deduction rate would reduce the liable person’s net earnings below the protected earnings rate the employer shall deduct only such amount as will leave the liable person with net earnings equal to the protected earnings rate.

(3) Where the liable person receives a payment of earnings at an interval greater or lesser than the interval specified in relation to the normal deduction rate and the protected earnings rate (“the specified interval”) the employer shall, for the purpose of such payments, take as the normal deduction rate and the protected earnings rate such amounts (to the nearest whole penny) as are in the same proportion to the interval since the last pay-day as the normal deduction rate and the protected earnings rate bear to the specified interval.

%  Reg 12(3A) inserted (18.1.98) by SI 1998/58 reg 6
(3A) Where on any pay-day the liable person receives a payment of earnings covering a period longer than the period by reference to which the normal deduction rate is set, the employer shall, subject to paragraph~(2), make a deduction from the net earnings paid to that liable person on that pay-day of an amount which is in the same proportion to the normal deduction rate as that longer period is to the period by reference to which that normal deduction rate is set.

(4) Where, on any pay-day, the employer fails to~deduct an amount due under the deduction from earnings order or deducts an amount less than the amount of the normal deduction rate the shortfall shall, subject to the operation of paragraph~(2), be deducted in addition to the normal deduction rate at the next available pay-day or days.

(5) Where, on any pay-day, the liable person’s net earnings are less than his protected earnings rate the amount of the difference shall be carried forward to his next pay-day and treated as part of his protected earnings in respect of that pay-day.

(6) Where, on any pay-day, an employer makes a deduction from the earnings of a liable person in accordance with the deduction from earnings order he may also deduct an amount not exceeding £1 in respect of his administrative costs and such deduction for administrative costs may be made notwithstanding that it may reduce the liable person’s net earnings below the protected earnings rate.

\amendment{
Reg. 12(3A) inserted (18.1.98) by the Child Support (Miscellaneous Amendments) Regulations 1998 reg.~6.
}

\subsection[12. Amount to be deducted by employer --- \emph{2003 scheme version}]{Amount to be deducted by employer\\*\emph{2003 scheme version}}

12.—(1) Subject to the provisions of this regulation, an employer who has been served with a copy of a deduction from earnings order in respect of a liable person in his employment shall, each pay-day, make a deduction from the net earnings of that liable person of an amount equal to the normal deduction rate.

(2) Where the deduction of the normal deduction rate would reduce the liable person’s net earnings below the 
%protected earnings rate
protected earnings proportion  % Words substituted (3.3.03 for new-rules cases only) by SI 2001/162 reg 2(5)(e)(i)
the employer shall deduct only such amount as will leave the liable person with net earnings equal to the 
%protected earnings rate
protected earnings proportion.  % Words substituted (3.3.03 for new-rules cases only) by SI 2001/162 reg 2(5)(e)(i)

(3) Where the liable person receives a payment of earnings at an interval greater or lesser than the interval specified in relation to the normal deduction rate and the 
%protected earnings rate
protected earnings proportion  % Words substituted (3.3.03 for new-rules cases only) by SI 2001/162 reg 2(5)(e)(i)
(“the specified interval”) the employer shall, for the purpose of such payments, take as the normal deduction rate and the  
%protected earnings rate
protected earnings proportion  % Words substituted (3.3.03 for new-rules cases only) by SI 2001/162 reg 2(5)(e)(i)
such amounts (to the nearest whole penny) as are in the same proportion to the interval since the last pay-day as the normal deduction rate and the  
%protected earnings rate
protected earnings proportion  % Words substituted (3.3.03 for new-rules cases only) by SI 2001/162 reg 2(5)(e)(i)
bear to the specified interval.

%  Reg 12(3A) inserted (18.1.98) by SI 1998/58 reg 6
(3A) Where on any pay-day the liable person receives a payment of earnings covering a period longer than the period by reference to which the normal deduction rate is set, the employer shall, subject to paragraph~(2), make a deduction from the net earnings paid to that liable person on that pay-day of an amount which is in the same proportion to the normal deduction rate as that longer period is to the period by reference to which that normal deduction rate is set.

(4) Where, on any pay-day, the employer fails to~deduct an amount due under the deduction from earnings order or deducts an amount less than the amount of the normal deduction rate the shortfall shall, subject to the operation of paragraph~(2), be deducted in addition to the normal deduction rate at the next available pay-day or days.

% Reg 12(5) omitted (3.3.03 for new-rules cases only) by SI 2001/162 reg 2(5)(e)(ii)
%(5) Where, on any pay-day, the liable person’s net earnings are less than his protected earnings rate the amount of the difference shall be carried forward to his next pay-day and treated as part of his protected earnings in respect of that pay-day.

(6) Where, on any pay-day, an employer makes a deduction from the earnings of a liable person in accordance with the deduction from earnings order he may also deduct an amount not exceeding £1 in respect of his administrative costs and such deduction for administrative costs may be made notwithstanding that it may reduce the liable person’s net earnings below the  
%protected earnings rate
protected earnings proportion.  % Words substituted (3.3.03 for new-rules cases only) by SI 2001/162 reg 2(5)(e)(i)

\amendment{
Reg. 12(3A) inserted (18.1.98) by the Child Support (Miscellaneous Amendments) Regulations 1998 reg.~6.

Words substituted in reg.~12(2), (3), (6) and reg.~12(5) omitted (3.3.03 for new-rules cases only) by the Child Support (Collection and~Enforcement and~Miscellaneous Amendments) Regulations 2000 reg.~2(5)(e) (subject to~savings in reg.~6).
}

\subsection[13. Employer to notify liable person of deduction]{Employer to notify liable person of deduction}

13.—(1) An employer making a deduction from earnings for the purposes of this Part shall notify the liable person in writing of the amount of the deduction, including any amount deducted for administrative costs under regulation~12(6).

(2) Such notification shall be given not later than the pay-day on which the deduction is made or, where that is impracticable, not later than the following pay-day.

\subsection[14. Payment by employer to~Secretary of State]{Payment by employer to~Secretary of State}

14.—(1) Amounts deducted by an employer under a deduction from earnings order (other than any administrative costs deducted under regulation~12(6)) shall be paid to the Secretary of State by the 19th day of the month following the month in which the deduction is made.

(2) Such payment may be made—
\begin{enumerate}\item[]
($a$) by cheque;

($b$) by automated credit transfer; or

($c$) by such other method as the Secretary of State may specify.
\end{enumerate}

\subsection[15. Information to be provided by liable person]{Information to be provided by liable person}

%15.—(1) The Secretary of State may, in relation to the making or operation of a deduction from earnings order, require the liable person to provide the following details—
%\begin{enumerate}\item[]
%($a$) the name and~address of his employer;
%
%($b$) the amount of his earnings and~anticipated earnings;
%
%($c$) his place of work, the nature of his work and~any works or pay number;
%\end{enumerate}
%and it shall be the duty of the liable person to comply with any such requirement within 7 days of being given written notice to that effect.
%
%(2) A liable person in respect of whom a deduction from earnings order is in force shall notify the Secretary of State in writing within 7 days of every occasion on which he leaves employment or becomes employed or re-employed.

% Reg 15 substituted (6.4.08) by SI 2008/536 reg 3
15.---(1)  A liable person in respect of whom a deduction from earnings order is in force must notify the Secretary of State in writing within 7 days of each occasion on which he leaves employment or becomes employed, or re-employed.

(2) If a liable person becomes employed or re-employed, such notification must include the following details—
\begin{enumerate}\item[]
($a$) the name and~address of his employer;

($b$) the amount of his earnings and~expected earnings; and

($c$) his place of work, nature of his work and~any works or pay number.
\end{enumerate}

\amendment{
Reg. 15 substituted (6.4.08) by the Child Support (Miscellaneous Amendments) Regulations 2008 reg.~3.
}

\subsection[16. Duty of employers and~others to notify Secretary of State]{Duty of employers and~others to notify Secretary of State}

16.—(1) Where a deduction from earnings order is served on a person on the assumption that he is the employer of a liable person but the liable person to whom the order relates is not in his employment, the person on whom the order was served shall notify the Secretary of State of that fact in writing, at the address specified in the order, within 10 days of the date of service on him of the order.

(2) Where an employer is required to~operate a deduction from earnings order and the liable person to whom the order relates ceases to be in his employment the employer shall notify the Secretary of State of that fact in writing, at the address specified in the order, within 10 days of the liable person ceasing to be in his employment.

(3) Where an employer becomes aware that a deduction from earnings order is in force in relation to~a person who is an employee of his he shall, within 7 days of the date on which he becomes aware, notify the Secretary of State of that fact in writing at the address specified in the order.

\subsection[17. Requirement to review deduction from earnings orders --- \emph{1993 scheme version}]{Requirement to review deduction from earnings orders\\*\emph{1993 scheme version}}

%17.  The Secretary of State shall review a deduction from earnings order in the following circumstances—
%\begin{enumerate}\item[]
%($a$) where there is a change in the amount of the maintenance assessment;
%
%($b$) where any arrears and~interest on arrears payable under the order are paid off.
%\end{enumerate}

% Reg 17 substituted (18.4.95) by SI 1995/1045 reg 18
17.—(1) Subject to paragraph~(2), the Secretary of State shall review a deduction from earnings order in the following circumstances—
\begin{enumerate}\item[]
($a$) where there is a change in the amount of the maintenance assessment;

($b$) where any arrears and interest on arrears payable under the order are paid off.
\end{enumerate}

(2) There shall be no obligation to review a deduction from earnings order under paragraph~(1) where the normal deduction rates specified in the order take account of the changes which will arise as a result of the circumstances specified in sub-paragraph~($a$) or ($b$) of that paragraph.

\amendment{
Reg. 17 substituted (18.4.95) by the Child Support and~Income Support (Amendment) Regulations 1995 reg.~18.
}

\subsection[17. Requirement to review deduction from earnings orders --- \emph{2003 scheme version}]{Requirement to review deduction from earnings orders\\*\emph{2003 scheme version}}

%17.  The Secretary of State shall review a deduction from earnings order in the following circumstances—
%\begin{enumerate}\item[]
%($a$) where there is a change in the amount of the maintenance assessment;
%
%($b$) where any arrears and~interest on arrears payable under the order are paid off.
%\end{enumerate}

% Reg 17 substituted (18.4.95) by SI 1995/1045 reg 18
17.—(1) Subject to paragraph~(2), the Secretary of State shall review a deduction from earnings order in the following circumstances—
\begin{enumerate}\item[]
($a$) where there is a change in the amount of the maintenance 
%assessment
calculation;  % Word substituted (3.3.03 for new-rules cases only) by SI 2001/162 reg 2(5)(f)(i)

($b$) where any arrears% 
% and~interest on arrears
, penalty payment, interest or fees  % Words substituted (3.3.03 for new-rules cases only) by SI 2001/162 reg 2(5)(f)(ii)
payable under the order are paid off.
\end{enumerate}

(2) There shall be no obligation to review a deduction from earnings order under paragraph~(1) where the normal deduction rates specified in the order take account of the changes which will arise as a result of the circumstances specified in sub-paragraph~($a$) or ($b$) of that paragraph.

\amendment{
Reg. 17 substituted (18.4.95) by the Child Support and~Income Support (Amendment) Regulations 1995 reg.~18.

Words substituted in reg.~17(1)(a), (b) (3.3.03 for new-rules cases only) by the Child Support (Collection and~Enforcement and~Miscellaneous Amendments) Regulations 2000 reg.~2(5)(f) (subject to~savings in reg.~6).
}

\subsection[18. Power to~vary deduction from earnings orders]{Power to~vary deduction from earnings orders}

18.—(1) The Secretary of State may (whether on a review under regulation~17 or otherwise) vary a deduction from earnings order so as to—
\begin{enumerate}\item[]
($a$) include any amount which may be included in such an order or exclude or decrease any such amount;

($b$) substitute a subsequent employer for the employer at whom the order was previously directed.
\end{enumerate}

(2) The Secretary of State shall serve a copy of any deduction from earnings order, as varied, on the liable person’s employer and~on the liable person.

\subsection[19. Compliance with deduction from earnings order as varied]{Compliance with deduction from earnings order as varied}

19.—(1) Where a deduction from earnings order has been varied and~a copy of the order as varied has been served on the liable person’s employer it shall, subject to paragraph~(2), be the duty of the employer to comply with the order as varied.

(2) The employer shall not be under any liability for non-compliance with the order, as varied, before the end of the period of 7 days beginning with the date on which a copy of the order, as varied, was served on him.

\subsection[20. Discharge of deduction from earnings orders]{Discharge of deduction from earnings orders}

20.—%(1) The Secretary of State may discharge a deduction from earnings order where—
%\begin{enumerate}\item[]
%($a$) no further payments under it are due; or
%
%($b$) it appears to him that the order is ineffective or that some other way of securing that payments are made would be more effective.
%\end{enumerate}
%
%Reg 20(1) substituted (18.4.95) by SI 1995/1045 reg 19
(1) The Secretary of State may discharge a deduction from earnings order where it appears to him that—
\begin{enumerate}\item[]
($a$) no further payments are due under it;

($b$) the order is ineffective or some other way of securing that payments are made would be more effective;

($c$) the order is defective;

($d$) the order fails to comply in a material respect with any procedural provision of the Act or regulations made under it other than provision made in regulation~9, 10 or 11;

($e$) at the time of the making of the order he did not have, or subsequently ceased to have, jurisdiction to make a deduction from earnings order; 
\emph{or}  % Word omitted (10.12.12 for 2012 scheme cases only) by SI 2012/2785 reg 4(6)(a)

($f$) in the case of an order made at a time when there is in force 
%an interim maintenance assessment
\emph{a default or interim maintenance decision},  % Words substituted (3.3.03 for new-rules cases only) by SI 2001/162 reg 2(5)(g)(i)
it is inappropriate to continue deductions under the order having regard to the compliance or the attempted compliance with the 
%maintenance assessment
\emph{maintenance calculation}  % Words substituted (3.3.03 for new-rules cases only) by SI 2001/162 reg 2(5)(g)(ii)
by the liable person%
\emph{%
% Reg 20(1)(g) inserted (10.12.12 for 2012 scheme cases only) by SI 2012/2785 reg 4(6)(b)
; or
}

\emph{
($g$) the circumstances in regulation 10(3) apply.
}
\end{enumerate}

(2) The Secretary of State shall give written notice of the discharge of the deduction from earnings order to the liable person and to the liable person’s employer.

\amendment{
Reg. 20(1) substituted (18.4.95) by the Child Support and~Income Support (Amendment) Regulations 1995 reg.~19.

Words substituted in reg.~20(1)(f) (3.3.03 for new-rules cases only) by the Child Support (Collection and~Enforcement and~Miscellaneous Amendments) Regulations 2000 reg.~2(5)(g) (subject to~savings in reg.~6).

Reg.~20(1)(g) inserted (10.12.12 for 2012 scheme cases only) by the Child Support (Meaning of Child and New Calculation Rules) (Consequential and Miscellaneous Amendment) Regulations 2012 reg.~4(6).

For 1993 scheme cases reg. 20(1)(f) should be read as follows:
\begin{quotation}
($f$) in the case of an order made at a time when there is in force an interim maintenance assessment it is inappropriate to continue deductions under the order having regard to the compliance or the attempted compliance with the maintenance assessment by the liable person.
\end{quotation}

Reg.~20(1)(g) applies only to 2012 scheme cases.
}

\subsection[21. Lapse of deduction from earnings orders]{Lapse of deduction from earnings orders}

21.—(1) A deduction from earnings order shall lapse (except in relation to~any deductions made or to be made in respect of the employment not yet paid to the Secretary of State) where the employer at whom it is directed ceases to have the liable person in his employment.

(2) The order shall lapse from the pay-day coinciding with, or, if none, the pay-day following, the termination of the employment.

(3) A deduction from earnings order which has lapsed under this regulation~shall nonetheless be treated as remaining in force for the purposes of regulations 15 and~24.

(4) Where a deduction from earnings order has lapsed under paragraph~(1) and the liable person recommences employment (whether with the same or another employer), the order may be revived from such date as may be specified by the Secretary of State.

(5) Where a deduction from earnings order is revived under paragraph~(4), the Secretary of State shall give written notice of that fact to, and serve a copy of the notice on, the liable person and the liable person’s employer.

(6) Where an order is revived under paragraph~(4), no amount shall be carried forward under regulation~12(4)  
\emph{or (5)}   % Words omitted (3.3.03 for new-rules cases only) by SI 2001/162 reg 2(5)(h)
from a time prior to the revival of the order.

\amendment{
Words ``or (5)'' omitted in reg.~21(6) (3.3.03 for new-rules cases only) by the Child Support (Collection and~Enforcement and~Miscellaneous Amendments) Regulations 2000 reg.~2(5)(h) (subject to~savings in reg.~6).
}

\subsection[22. Appeals against deduction from earnings orders]{Appeals against deduction from earnings orders}

22.—(1) A liable person in respect of whom a deduction from earnings order has been made may appeal to the magistrates' court, or in Scotland the sheriff
%, having jurisdiction in the area in which he resides
of the sheriffdom in which he resides%  % Words substituted (1.8.07) by SI 2007/197 reg 2(2)(a)
.

(2) 
Subject to paragraph~(2A),  % Words inserted (27.10.08) by SI 2008/2544 reg 2(4)(a)
any appeal shall—
\begin{enumerate}\item[]
($a$) be by way of complaint for an order or, in Scotland, by way of application;

($b$) 
where the liable person is resident in the United Kingdom,  % Words inserted (1.8.07) by SI 2007/1979 reg 2(2)(b)
be made within 28 days of the date on which the matter appealed against arose;

% Reg 22(2)(c) inserted (1.8.07) by SI 2007/1979 reg 2(2)(c)
($c$) where the liable person is not resident in the United Kingdom, be made within 56 days of the date on which the matter appealed against arose.
\end{enumerate}

% Reg 22(2A) inserted (27.10.08) by SI 2008/2544 reg 2(4)(b)
(2A) Any appeal against a decision of the Secretary of State that the exclusion required by regulation~3(3) does not apply is—
\begin{enumerate}\item[]
($a$) where the liable person is resident in the United Kingdom, to be made within 28 days of the date on which that decision is given or~sent to the liable person;

($b$) where the liable person is not resident in the United Kingdom, to be made within 56 days of the date on which that decision is given or~sent to the liable person.
\end{enumerate}

(3) 
Subject to paragraph~(3A),  % Words inserted (27.10.08) by SI 2008/2544 reg 2(4)(c)
an appeal may be made only on one or both of the following grounds—\looseness=-1
\begin{enumerate}\item[]
($a$) that the deduction from earnings order is defective;

($b$) that the payments in question do not constitute earnings.
\end{enumerate}

% Reg 22(3A) inserted (27.10.08) by SI 2008/2544 reg 2(4)(d)
(3A) Where the Secretary of State is considering specifying a deduction from earnings order as a method of payment under regulation~3(1)($i$)  an appeal may also be made against a decision of the Secretary of State that the exclusion required by regulation~3(3) does not apply.

(4) 
Subject to paragraph~(5),  % Words inserted (27.10.08) by SI 2008/2544 reg 2(4)(e)
where the court or, as the case may be, the sheriff is satisfied that the appeal should be allowed the court, or sheriff, may—
\begin{enumerate}\item[]
($a$) quash the deduction from earnings order; or

($b$) specify which, if any, of the payments in question do not constitute earnings.\end{enumerate}

% Reg 22(5) added (27.10.08) by SI 2008/2544 reg 2(4)(f)
(5) Where an appeal is brought on the grounds set out in paragraph~(3A), and the court, or as the case may be, the sheriff, is satisfied that the appeal should be allowed the court or the sheriff is to refer the case to the Secretary of State for him to specify whichever of the methods of payment set out in regulation~3(1) he considers to be appropriate in the circumstances.

\amendment{
Words inserted in reg.~22(2)(b), words substituted in reg.~22(1) and reg.~22(2)(c) inserted (1.8.07) by the Child Support (Miscellaneous Amendments) Regulations 2007 reg.~2(2).

Words inserted in reg.~22(2), (3), (4) and reg.~22(2A), (3A), (5) inserted (27.10.08) by the Child Support (Miscellaneous Amendments) (No.~2) Regulations 2008 reg.~2(4).
}

\subsection[23. Crown employment]{Crown employment}

23.  Where a liable person is in the employment of the Crown and~a deduction from earnings order is made in respect of him then for the purposes of this Part—
\begin{enumerate}\item[]
($a$) the chief officer for the time being of the Department, office or other body in which the liable person is employed shall be treated as having the liable person in his employment (any transfer of the liable person from one Department, office or body to~another being treated as a change of employment); and

($b$) any earnings paid by the Crown or a minister of the Crown, or out of the public revenue of the United Kingdom, shall be treated as paid by that chief officer.
\end{enumerate}

\subsection[24. Priority as between orders --- \emph{1993 scheme version}]{Priority as between orders\\*\emph{1993 scheme version}}

24.—(1) Where an employer would, but for this paragraph, be obliged, on any pay-day, to make deductions under two or more deduction from earnings orders he shall—
\begin{enumerate}\item[]
($a$) deal with the orders according to the respective dates on which they were made, disregarding any later order until an earlier one has been dealt with;

($b$) deal with any later order as if the earnings to which it relates were the residue of the liable person’s earnings after the making of any deduction to comply with any earlier order.
\end{enumerate}

(2) Where an employer would, but for this paragraph, be obliged to comply with one or more deduction from earnings orders and~one or more attachment of earnings orders he shall—
\begin{enumerate}\item[]
($a$) in the case of an attachment of earnings order which was made either wholly or in part in respect of the payment of a judgment debt or payments under an administration order, deal first with the deduction from earnings order or orders and thereafter with the attachment of earnings order as if the earnings to which it relates were the residue of the liable person’s earnings after the making of deductions to comply with the deduction from earnings order or orders;

($b$) in the case of any other attachment of earnings order, deal with the orders according to the respective dates on which they were made in like manner as under paragraph~(1).\end{enumerate}

“Attachment of earnings order” in this paragraph~means an order made under the Attachment of Earnings Act 1971\footnote{\frenchspacing 1971 c. 32.} or under regulation~32 of the Community Charge (Administration and~Enforcement) Regulations~1989\footnote{\frenchspacing S.I.~1989/438.}
or under regulation~37 of the Council Tax (Administration and~Enforcement) Regulations 1992\footnote{\frenchspacing  S.I.~1992/613.}. % Words inserted (5.4.93) by SI 1993/913 reg 42

(3) Paragraph (2) does not apply to~Scotland.

(4) In Scotland, where an employer would, but for this paragraph, be obliged to comply with one or more deduction from earnings orders and one or more diligences against earnings he shall deal first with the deduction from earnings order or orders and thereafter with the diligence against earnings as if the earnings to which the diligence relates were the residue of the liable person’s earnings after the making of deductions to comply with the deduction from earnings order or orders.

\amendment{
Words inserted in the definition of ``attachment of earnings order'' in reg.~24(2) (5.4.93) by the Child Support (Miscellaneous Amendments) Regulations 1993 reg.~42.
}

\subsection[24. Priority as between orders --- \emph{2003 scheme version}]{Priority as between orders\\*\emph{2003 scheme version}}

% Reg 24(1) omitted (3.3.03 for new-rules cases only) by SI 2001/162 reg 2(5)(i)(i)
24.—%(1) Where an employer would, but for this paragraph, be obliged, on any pay-day, to make deductions under two or more deduction from earnings orders he shall—
%\begin{enumerate}\item[]
%($a$) deal with the orders according to the respective dates on which they were made, disregarding any later order until an earlier one has been dealt with;
%
%($b$) deal with any later order as if the earnings to which it relates were the residue of the liable person’s earnings after the making of any deduction to comply with any earlier order.
%\end{enumerate}
%
(2) Where an employer would, but for this paragraph, be obliged to comply with 
%one or more deduction from earnings orders
a deduction from earnings order  % Words substituted (3.3.03 for new-rules cases only) by SI 2001/162 reg 2(5)(i)(iii)(aa)
and one or more attachment of earnings orders he shall—
\begin{enumerate}\item[]
($a$) in the case of an attachment of earnings order which was made either wholly or in part in respect of the payment of a judgment debt or payments under an administration order, deal first with the deduction from earnings order
%or orders  % Words omitted (3.3.03 for new-rules cases only) by SI 2001/162 reg 2(5)(i)(iii)(bb)
and thereafter with the attachment of earnings order as if the earnings to which it relates were the residue of the liable person’s earnings after the making of deductions to comply with the deduction from earnings order%
%or orders  % Words omitted (3.3.03 for new-rules cases only) by SI 2001/162 reg 2(5)(i)(iii)(bb)
;

($b$) in the case of any other attachment of earnings order
%, deal with the orders according to the respective dates on which they were made in like manner as under paragraph~(1).
he shall—
\begin{enumerate}\item[]
(i) deal with the orders according to the respective dates on which they were made, disregarding any later order until an earlier one has been dealt with;

(ii) deal with any later order as if the earnings to which it relates were the residue of the liable person’s earnings after the making of any deduction to comply with any earlier order.
\end{enumerate}  % Words substituted (3.3.03 for new-rules cases only) by SI 2001/162 reg 2(5)(i)(ii)
\end{enumerate}

“Attachment of earnings order” in this paragraph~means an order made under the Attachment of Earnings Act 1971\footnote{\frenchspacing 1971 c. 32.} or under regulation~32 of the Community Charge (Administration and~Enforcement) Regulations~1989\footnote{\frenchspacing S.I.~1989/438.}
or under regulation~37 of the Council Tax (Administration and~Enforcement) Regulations 1992\footnote{\frenchspacing  S.I.~1992/613.}. % Words inserted (5.4.93) by SI 1993/913 reg 42

(3) Paragraph (2) does not apply to~Scotland.

(4) In Scotland, where an employer would, but for this paragraph, be obliged to comply with 
%one or more deduction from earnings orders
a deduction from earnings order  % Words substituted (3.3.03 for new-rules cases only) by SI 2001/162 reg 2(5)(i)(iii)(aa)
and~one or more diligences against earnings he shall deal first with the deduction from earnings order
%or orders  % Words omitted (3.3.03 for new-rules cases only) by SI 2001/162 reg 2(5)(i)(iii)(bb)
and thereafter with the diligence against earnings as if the earnings to which the diligence relates were the residue of the liable person’s earnings after the making of deductions to comply with the deduction from earnings order% 
%or orders  % Words omitted (3.3.03 for new-rules cases only) by SI 2001/162 reg 2(5)(i)(iii)(bb)
.

\amendment{
Words inserted in the definition of ``attachment of earnings order'' in reg.~24(2) (5.4.93) by the Child Support (Miscellaneous Amendments) Regulations 1993 reg.~42.

Words substituted in reg.~24(2), (4), words omitted in reg.~24(2), (4) and reg.~24(1) omitted (3.3.03 for new-rules cases only) by the Child Support (Collection and~Enforcement and~Miscellaneous Amendments) Regulations 2000 reg.~2(5)(i) (subject to~savings in reg.~6).
}

\subsection[25. Offences]{Offences}

25.  The following regulations are designated for the purposes of section~32(8) of the Act (offences relating to~deduction from earnings orders)—
\begin{enumerate}\item[]
% Reg 25(aa) inserted (6.4.99) by SI 1999/977 reg 2(3)(a)
($aa$) regulation~14(1);

%($a$) 
($ab$)  % Reg 25(a) renumbered (6.4.99) by SI 1999/977 reg 2(3)(b)
regulation~15(1) and~(2);

($b$) regulation~16(1), (2) and~(3);

($c$) regulation~19(1).
\end{enumerate}

\amendment{
Reg. 25(a) renumbered as reg.~25(ab) and reg.~25(aa) inserted (6.4.99) by the Child Support (Miscellaneous Amendments) Regulations 1999 reg.~2(3).
}

% Pt IIIA inserted (3.8.09) by SI 2009/1815 reg 2

\section[Part IIIA --- Deduction orders]{Part IIIA\\*Deduction orders}

\amendment{
Pt. IIIA inserted (3.8.09) by the Child Support Collection and Enforcement (Deduction Orders) Amendment Regulations 2009 reg.~2.
}

\subsection[Chapter I --- Interpretation]{Chapter I\\*Interpretation}

\renewcommand\parthead{--- Part IIIA Chapter I}

\subsubsection[25A. Interpretation of this Part --- \emph{1993 and 2003 schemes version}]{Interpretation of this Part\\*\emph{1993 and 2003 schemes version}}

25A.---(1)  In this Part—
\begin{enumerate}\item[]
“assessable income” means the amount calculated in accordance with paragraph~5 of Schedule 1 to the Act as it applies to~a 1993 scheme case and regulations made for the purposes of that paragraph;

“deduction period” means the period of a week, a month or other period at which deductions are to be made from the amount (if any) standing to the credit of the account specified in a regular deduction order;

“garnishee order” means an order made in accordance with the provisions of order 30 of the County Court Rules 1981\footnote{S.I.~1981/1687, these Rules are replaced by the Civil Procedure Rules 1998, except to the extent that Rule 2.1(2) of the Civil Procedure Rules also provides that those Rules do not apply to~family proceedings and specifies enactments under which rules may be made for the purposes of such proceedings.} or order 49 of the Rules of the Supreme Court 1965\footnote{S.I.~1965/1976, these Rules are replaced by the Civil Procedure Rules 1998, except to the extent that Rule 2.1(2) of the Civil Procedure Rules also provides that those Rules do not apply to~family proceedings and specifies enactments under which rules may be made for the purposes of such proceedings.};

“net weekly income” has the meaning given in the Schedule to the Child Support (Maintenance Calculations and~Special Cases) Regulations 2000\footnote{S.I.~2001/155, relevant amending instruments are S.I.~2002/1204, 2003/328, 2004/2415 and~3168, 2005/785, 2060 and~2929, 2007/1979 and~2008/2544.};

“lump sum deduction order” means an order under section~32E(1) or, as the case may be, 32F(1) of the Act;

“regular deduction order” means an order under section 32A(1) of the Act;

“third party debt order” means an order made in accordance with the provisions of Part LXXII of the Civil Procedure Rules 1998\footnote{S.I.~1998/3132, relevant amending instruments are S.I.~2001/2792 and~2005/2292.};

“working day” means any day other than a Saturday, a Sunday, Christmas Day, Good Friday or a day which is a bank holiday within the meaning of the Banking and~Financial Dealings Act 1971\footnote{1971 c. 80.} in the part of the United Kingdom where a copy of a regular deduction order or a lump sum deduction order is served or a notification sent by the %Commission 
Secretary of State  % Words substituted (1.8.12) by SI 2012/2007 Sch para 111(2)(a)
is received.
\end{enumerate}

(2) Any person against whom an order under section~32A(1) of the Act may be made by the  %Commission 
Secretary of State  % Words substituted (1.8.12) by SI 2012/2007 Sch para 111(2)(b)
is referred to in this Chapter and~Chapters~II and~IV as “the liable person”.

(3) Where a copy of a regular deduction order or a lump sum deduction order is served by the  
%Commission 
Secretary of State  % Words substituted (1.8.12) by SI 2012/2007 Sch para 111(2)(b)
in accordance with section~32A(7), 32E(6) or~32F(6) of the Act—
\begin{enumerate}\item[]
($a$) on a deposit-taker—
\begin{enumerate}\item[]
(i) where that copy of the order is sent by electronic communication or fax to the deposit-taker’s last notified address for~electronic communication or, as the case may be, fax number, it is to be treated as having been served at the end of the first working day after the day it was sent by the 
%Commission 
Secretary of State%  % Words substituted (1.8.12) by SI 2012/2007 Sch para 111(2)(b)
, or

(ii) where that copy of the order is sent by post to the deposit-taker’s last notified address, it is to be treated as having been served at the end of the second working day after the day it was posted by the 
%Commission 
Secretary of State%  % Words substituted (1.8.12) by SI 2012/2007 Sch para 111(2)(b)
; or
\end{enumerate}

($b$) on a liable person, where that copy of the order is sent by post to that person’s last known or~notified address, it is to be treated as having been served at the end of the day on which the copy of the order is posted.
\end{enumerate}

(4) Any notification sent by the 
%Commission 
Secretary of State  % Words substituted (1.8.12) by SI 2012/2007 Sch para 111(2)(b)
in accordance with this Part to~a deposit-taker or a liable person is to be treated as having been received at the same time as an order is treated as having been served in accordance with the provisions of paragraph~(3).

(5) Where a copy of a regular deduction order or a lump sum deduction order or any notification has been sent by electronic communication in accordance with paragraph~(3)($a$)(i)  the record held on an official computer system is conclusive (or in Scotland, sufficient) evidence—
\begin{enumerate}\item[]
($a$) that a copy of that order has been sent; and

($b$) of the content of that order.
\end{enumerate}

(6) This Part applies to~a 1993 scheme case in the same way as it applies to~a 2003 scheme case and—
\begin{enumerate}\item[]
($a$) any references to expressions in the Act (including “maintenance calculation”) or~to regulations made under the Act are to be read, in relation to~a 1993 scheme case, with the necessary modifications; and

($b$) any reference in this Part to~“net weekly income” is to be read as if it were a reference to~“assessable income” where these Regulations apply to~a 1993 scheme case.
\end{enumerate}

(7) In this regulation—
\begin{enumerate}\item[]
($a$) “electronic communication” has the meaning given in section~15(1) of the Electronic Communications Act 2000\footnote{2000 c.~7.};

($b$) “an official computer system” means a computer system maintained by or on behalf of the 
%Commission 
Secretary of State  % Words substituted (1.8.12) by SI 2012/2007 Sch para 111(2)(b)
for~sending an order or any notification;

($c$) “1993 scheme case” means a case in respect of which the provisions of the Child Support, Pensions and~Social Security Act 2000\footnote{2000 c.~19.} have not been brought into~force in accordance with article 3 of the Child Support, Pensions and~Social Security Act 2000 (Commencement No.~12) Order 2003\footnote{S.I.~2003/192 (C.~11).}; and

($d$) “2003 scheme case” means a case in respect of which those provisions have been brought into~force.
\end{enumerate}

\amendment{
Words substituted in reg. 25A(1), (2), (3), (4), (7) (1.8.12) by the Public Bodies (Child Maintenance and Enforcement Commission: Abolition and Transfer of Functions) Order 2012 Sch. para. 111(2).
}

\subsubsection[25A. Interpretation of this Part -- \emph{2012 scheme version}]{Interpretation of this Part\\*\emph{2012 scheme version}}

25A.---(1)  In this Part—
\begin{enumerate}\item[]
“assessable income” means the amount calculated in accordance with paragraph~5 of Schedule 1 to the Act as it applies to~a 1993 scheme case and regulations made for the purposes of that paragraph;

% Definition of ``current income'' inserted (30.9.13 for 2012 scheme cases only) by SI 2013/1517 reg 4(2)(a)
“current income” has the meaning given in regulation 37 of the Child Support Maintenance Calculation Regulations 2012 (current income---general)\footnote{S.I.~2012/2677.};

“deduction period” means the period of a week, a month or other period at which deductions are to be made from the amount (if any) standing to the credit of the account specified in a regular deduction order;

“garnishee order” means an order made in accordance with the provisions of order 30 of the County Court Rules 1981\footnote{S.I.~1981/1687, these Rules are replaced by the Civil Procedure Rules 1998, except to the extent that Rule 2.1(2) of the Civil Procedure Rules also provides that those Rules do not apply to~family proceedings and specifies enactments under which rules may be made for the purposes of such proceedings.} or order 49 of the Rules of the Supreme Court 1965\footnote{S.I.~1965/1976, these Rules are replaced by the Civil Procedure Rules 1998, except to the extent that Rule 2.1(2) of the Civil Procedure Rules also provides that those Rules do not apply to~family proceedings and specifies enactments under which rules may be made for the purposes of such proceedings.};

% Definition of ``gross weekly income'' inserted (30.9.13 for 2012 scheme cases only) by SI 2013/1517 reg 4(2)(b)
“gross weekly income” means income calculated under Chapter I of Part~IV of the Child Support Maintenance Calculation Regulations 2012;

% Definition of ``net weekly income'' omitted (30.9.13 for 2012 scheme cases only) by SI 2013/1517 reg 4(2)(c)
%“net weekly income” has the meaning given in the Schedule to the Child Support (Maintenance Calculations and~Special Cases) Regulations 2000\footnote{S.I.~2001/155, relevant amending instruments are S.I.~2002/1204, 2003/328, 2004/2415 and~3168, 2005/785, 2060 and~2929, 2007/1979 and~2008/2544.};

“lump sum deduction order” means an order under section~32E(1) or, as the case may be, 32F(1) of the Act;

“regular deduction order” means an order under section 32A(1) of the Act;

“third party debt order” means an order made in accordance with the provisions of Part LXXII of the Civil Procedure Rules 1998\footnote{S.I.~1998/3132, relevant amending instruments are S.I.~2001/2792 and~2005/2292.};

“working day” means any day other than a Saturday, a Sunday, Christmas Day, Good Friday or a day which is a bank holiday within the meaning of the Banking and~Financial Dealings Act 1971\footnote{1971 c. 80.} in the part of the United Kingdom where a copy of a regular deduction order or a lump sum deduction order is served or a notification sent by the %Commission 
Secretary of State  % Words substituted (1.8.12) by SI 2012/2007 Sch para 111(2)(a)
is received.
\end{enumerate}

(2) Any person against whom an order under section~32A(1) of the Act may be made by the  %Commission 
Secretary of State  % Words substituted (1.8.12) by SI 2012/2007 Sch para 111(2)(b)
is referred to in this Chapter and~Chapters~II and~IV as “the liable person”.

(3) Where a copy of a regular deduction order or a lump sum deduction order is served by the  
%Commission 
Secretary of State  % Words substituted (1.8.12) by SI 2012/2007 Sch para 111(2)(b)
in accordance with section~32A(7), 32E(6) or~32F(6) of the Act—
\begin{enumerate}\item[]
($a$) on a deposit-taker—
\begin{enumerate}\item[]
(i) where that copy of the order is sent by electronic communication or fax to the deposit-taker’s last notified address for~electronic communication or, as the case may be, fax number, it is to be treated as having been served at the end of the first working day after the day it was sent by the 
%Commission 
Secretary of State%  % Words substituted (1.8.12) by SI 2012/2007 Sch para 111(2)(b)
, or

(ii) where that copy of the order is sent by post to the deposit-taker’s last notified address, it is to be treated as having been served at the end of the second working day after the day it was posted by the 
%Commission 
Secretary of State%  % Words substituted (1.8.12) by SI 2012/2007 Sch para 111(2)(b)
; or
\end{enumerate}

($b$) on a liable person, where that copy of the order is sent by post to that person’s last known or~notified address, it is to be treated as having been served at the end of the day on which the copy of the order is posted.
\end{enumerate}

(4) Any notification sent by the 
%Commission 
Secretary of State  % Words substituted (1.8.12) by SI 2012/2007 Sch para 111(2)(b)
in accordance with this Part to~a deposit-taker or a liable person is to be treated as having been received at the same time as an order is treated as having been served in accordance with the provisions of paragraph~(3).

(5) Where a copy of a regular deduction order or a lump sum deduction order or any notification has been sent by electronic communication in accordance with paragraph~(3)($a$)(i)  the record held on an official computer system is conclusive (or in Scotland, sufficient) evidence—
\begin{enumerate}\item[]
($a$) that a copy of that order has been sent; and

($b$) of the content of that order.
\end{enumerate}

(6) This Part applies to~a 1993 scheme case in the same way as it applies to~a 2003 scheme case and—
\begin{enumerate}\item[]
($a$) any references to expressions in the Act (including “maintenance calculation”) or~to regulations made under the Act are to be read, in relation to~a 1993 scheme case, with the necessary modifications%; and
%
% Reg 25A(6)(b) omitted (30.9.13 for 2012 scheme cases only) by SI 2013/1517 reg 4(3)
%($b$) any reference in this Part to~“net weekly income” is to be read as if it were a reference to~“assessable income” where these Regulations apply to~a 1993 scheme case
.
\end{enumerate}

(7) In this regulation—
\begin{enumerate}\item[]
($a$) “electronic communication” has the meaning given in section~15(1) of the Electronic Communications Act 2000\footnote{2000 c.~7.};

($b$) “an official computer system” means a computer system maintained by or on behalf of the 
%Commission 
Secretary of State  % Words substituted (1.8.12) by SI 2012/2007 Sch para 111(2)(b)
for~sending an order or any notification;

($c$) “1993 scheme case” means a case in respect of which the provisions of the Child Support, Pensions and~Social Security Act 2000\footnote{2000 c.~19.} have not been brought into~force in accordance with article 3 of the Child Support, Pensions and~Social Security Act 2000 (Commencement No.~12) Order 2003\footnote{S.I.~2003/192 (C.~11).}; and

($d$) “2003 scheme case” means a case in respect of which those provisions have been brought into~force.
\end{enumerate}

\amendment{
Words substituted in reg. 25A(1), (2), (3), (4), (7) (1.8.12) by the Public Bodies (Child Maintenance and Enforcement Commission: Abolition and Transfer of Functions) Order 2012 Sch. para. 111(2).

Definitions of ``current income'', ``gross weekly income'' inserted in reg.~25A(1), definition of ``net weekly income'' in reg.~25A(1) omitted and reg.~25A(6)(b) omitted (30.9.13 for 2012 scheme cases only) by the Child Support (Miscellaneous Amendments) Regulations 2013 reg.~4(2), (3).
}

\subsection[Chapter II --- Regular deduction orders]{Chapter II\\*Regular deduction orders}

\renewcommand\parthead{--- Part IIIA Chapter II}

\subsubsection[25B. Regular deduction orders]{Regular deduction orders}

25B.---(1)  A regular deduction order must specify—
\begin{enumerate}\item[]
($a$) the amount of the regular deduction; and

($b$) the dates on which regular deductions (referred to in this Chapter as “deduction dates”) are due to be made.
\end{enumerate}

(2) Where the date on which the regular deduction is due to be made is not a working day, the deduction must be made on the first working day after the date specified in the order.

\subsubsection[25C. Maximum deduction rate]{Maximum deduction rate}

25C.---(1)  The deduction rate under a regular deduction order in respect of any deduction period—
\begin{enumerate}\item[]
\emph{
($a$) is not to exceed 40\% of the liable person’s 
%net 
gross  % Word substituted (10.12.12 for 2012 scheme cases only) by SI 2012/2785 reg 4(7)
weekly income 
%in respect of that period  % Words omitted (30.9.13 for 2012 scheme cases only) by SI 2013/1517 reg 4(4)
as calculated—}
\begin{enumerate}\item[]
(i) at the date of the current maintenance calculation, or

(ii) where a maintenance calculation has been in force and there are arrears of child support maintenance, at the date of the most recent previous maintenance calculation; or
\end{enumerate}

($b$) where a default maintenance decision has been made, is not to exceed £80 per week.
\end{enumerate}

(2) In this Chapter “previous maintenance calculation” means a maintenance calculation which is no longer in force.

\amendment{
Word substituted in reg.~25C(1)(a) (10.12.12 for 2012 scheme cases only) by the Child Support (Meaning of Child and New Calculation Rules) (Consequential and Miscellaneous Amendment) Regulations 2012 reg.~4(7).

Words omitted in reg.~25C(1)(a) (30.9.13 for 2012 scheme cases only) by the Child Support (Miscellaneous Amendments) Regulations 2013 reg.~4(4).

For 1993 and 2003 scheme cases, reg.~25C(1)(a) should be read as follows:
\begin{quotation}
($a$) is not to exceed 40\% of the liable person’s net weekly income in respect of that period as calculated—
\end{quotation}

}

\subsubsection[25D. Minimum amount]{Minimum amount}

25D.---(1)  A deduction must not be made where the amount standing to the credit of the account specified in the regular deduction order is below the minimum amount on the date a deduction is due to be made.

(2) The minimum amount (for the purposes of this Chapter) is, where the deduction period is—
\begin{enumerate}\item[]
($a$) monthly, £40;

($b$) weekly, £10; or

($c$) for any other period, £10 for~each whole week in that period plus £1 for~each additional day in that period,
\end{enumerate}
plus the amount of administrative costs authorised by regulation~25Z($a$)  (administrative costs).

\subsubsection[25E. Notification by the deposit-taker to the 
%Commission 
Secretary of State%  % Words substituted (1.8.12) by SI 2012/2007 Sch para 111(3)
]{Notification by the deposit-taker to the 
%Commission 
Secretary of State%  % Words substituted (1.8.12) by SI 2012/2007 Sch para 111(3)
}

25E.---(1)  A deposit-taker at which a regular deduction order is directed must notify the 
%Commission 
Secretary of State  % Words substituted (1.8.12) by SI 2012/2007 Sch para 111(3)
in writing, within 7 days—
\begin{enumerate}\item[]
($a$) of a copy of the order or the order as varied being served; or

($b$) of notification being received by the deposit-taker that an order has been revived,
\end{enumerate}
of the matters set out in paragraph~(2).

(2) The matters are—
\begin{enumerate}\item[]
($a$) if the account specified in the order does not exist; and

($b$) where the name of the liable person specified in the order is different to the name in which the account specified in the order is held—
\begin{enumerate}\item[]
(i) whether the account was previously held in the name of the liable person specified in the order, and

(ii) if so, the new name in which the account is held,
\end{enumerate}
only where the liable person named in the order is the same person as the person in whose name the account specified in the order is held.
\end{enumerate}

(3) A deposit-taker at which a regular deduction order is directed must notify the 
%Commission 
Secretary of State  % Words substituted (1.8.12) by SI 2012/2007 Sch para 111(3)
within 7 days of notification being received that an order has lapsed or has been discharged—
\begin{enumerate}\item[]
($a$) if the account specified in the order does not exist; and

($b$) where the name of the liable person specified in the order is different to the name in which the account specified in the order is held—
\begin{enumerate}\item[]
(i) whether the account was previously held in the name of the liable person specified in the order, and

(ii) if so, the new name in which the account is held,
\end{enumerate}
only where the liable person named in the order is the same person as the person in whose name the account specified in the order is held.
\end{enumerate}

(4) The deposit-taker at which a regular deduction order is directed must notify the 
%Commission 
Secretary of State  % Words substituted (1.8.12) by SI 2012/2007 Sch para 111(3)
within 7 days starting on the date on which a deduction is due to be made—
\begin{enumerate}\item[]
($a$) if the account specified in the order has been closed;

($b$) if the amount standing to the credit of the account specified in the order is less than the minimum amount; and

($c$) where the name of the liable person specified in the order is different to the name in which the account specified in the order is held—
\begin{enumerate}\item[]
(i) whether the account was previously held in the name of the liable person specified in the order, and

(ii) if so, the new name in which the account is held,
\end{enumerate}
only where the liable person named in the order is the same person as the person in whose name the account specified in the order is held.
\end{enumerate}

(5) The deposit-taker at which a regular deduction order is directed must notify the 
%Commission 
Secretary of State  % Words substituted (1.8.12) by SI 2012/2007 Sch para 111(3)
within 7 days of receipt of a request made by the 
%Commission 
Secretary of State  % Words substituted (1.8.12) by SI 2012/2007 Sch para 111(3)
of the details of any other account held by the liable person with that deposit-taker and the details of that account, including—
\begin{enumerate}\item[]
($a$) the number and sort code of that account; and

($b$) the type of account.
\end{enumerate}

(6) The requirements of this regulation~apply only in so far as the deposit-taker has the information or~can reasonably be expected to~acquire it.

\amendment{
Words substituted in reg. 25E(1), (3), (4), (5) and heading (1.8.12) by the Public Bodies (Child Maintenance and Enforcement Commission: Abolition and Transfer of Functions) Order 2012 Sch. para. 111(3).
}

\subsubsection[25F. Notification by the 
%Commission 
Secretary of State  % Words substituted (1.8.12) by SI 2012/2007 Sch para 111(4)
to the deposit-taker]{Notification by the 
%Commission 
Secretary of State  % Words substituted (1.8.12) by SI 2012/2007 Sch para 111(4)
to the deposit-taker}

25F.  The 
%Commission 
Secretary of State  % Words substituted (1.8.12) by SI 2012/2007 Sch para 111(4)
must notify the deposit-taker within 7 days of making a decision that a regular deduction order has—
\begin{enumerate}\item[]
($a$) been varied by virtue of regulation~25I (variation of a regular deduction order);

($b$) lapsed under regulation~25J (lapse of a regular deduction order);

($c$) been revived under regulation~25K (revival of a regular deduction order); or

($d$) ceased to have effect by virtue of regulation~25L (discharge of a regular deduction order).
\end{enumerate}

\amendment{
Words substituted in reg. 25F and heading (1.8.12) by the Public Bodies (Child Maintenance and Enforcement Commission: Abolition and Transfer of Functions) Order 2012 Sch. para. 111(4).
}

\subsubsection[25G. Review of a regular deduction order]{Review of a regular deduction order}

25G.---(1)  A deposit-taker at which a regular deduction order is directed or the liable person against whom the order is made may apply to the 
%Commission 
Secretary of State  % Words substituted (1.8.12) by SI 2012/2007 Sch para 111(5)(a)
for a review of the order.

(2) The circumstances in which an application may be made under paragraph~(1) are that—
\begin{enumerate}\item[]
($a$) the liable person or the deposit-taker satisfies the 
%Commission 
Secretary of State  % Words substituted (1.8.12) by SI 2012/2007 Sch para 111(5)(a)
that some or all of the amount standing to the credit of the account specified in the order is not an amount in which the liable person has a beneficial interest;

($b$) there has been a change in the amount of the maintenance calculation in question;

($c$) any amounts payable under the order have been paid;

($d$) the maximum deduction rate has been calculated in accordance with regulation~25C(1)($a$)(ii)  (maximum deduction rate) and there has been a change in the liable person's 
%net 
%gross  % Word substituted (10.12.12 for 2012 scheme cases only) by SI 2012/2785 reg 4(7)
%weekly 
\emph{current}  % Word substituted (30.9.13 for 2012 scheme cases only) by SI 2013/1517 reg 4(5)
income since the date of the most recent previous maintenance calculation;

($e$) due to~an official error, an incorrect amount has been specified in the order; or

($f$) the order does not comply with the requirements of section~32A(5) of the Act or~regulation~25B(1) or~25C.
\end{enumerate}

(3) Following a review of an order under this regulation—
\begin{enumerate}\item[]
($a$) where the 
%Commission 
Secretary of State  % Words substituted (1.8.12) by SI 2012/2007 Sch para 111(5)(b)
changes the amount to be deducted by the deposit-taker under the order, 
%it 
the Secretary of State  % Words substituted (1.8.12) by SI 2012/2007 Sch para 111(5)(b)
may vary the order; or

($b$) where the 
%Commission 
Secretary of State  % Words substituted (1.8.12) by SI 2012/2007 Sch para 111(5)(b)
extinguishes the amount to be deducted by the deposit-taker under the order, 
%it 
the Secretary of State  % Words substituted (1.8.12) by SI 2012/2007 Sch para 111(5)(b)
must discharge the order.
\end{enumerate}

(4) In paragraph~(2)($e$)  “official error” has the same meaning as in regulation~1(3) of the Social Security and~Child Support (Decisions and~Appeals) Regulations 1999 (interpretation)\footnote{S.I.~1999/991, relevant amending instruments are S.I.~2002/1379, 2008/2656 and 2008/2683.}.

\amendment{
Words substituted in reg. 25G(1), (2)(a), (3) (1.8.12) by the Public Bodies (Child Maintenance and Enforcement Commission: Abolition and Transfer of Functions) Order 2012 Sch. para. 111(5).

Word substituted in reg.~25G(2)(d) (10.12.12 for 2012 scheme cases only) by the Child Support (Meaning of Child and New Calculation Rules) (Consequential and Miscellaneous Amendment) Regulations 2012 reg.~4(7).

Words substituted in reg.~25G(2)(d) (30.9.13 for 2012 scheme cases only) by the Child Support (Miscellaneous Amendments) Regulations 2013 reg.~4(5).

For 1993 and 2003 scheme cases, the reference in reg.~25G(2)(d) to current income should be read as a reference to net weekly income.
}

\subsubsection[25H. Priority as between orders---regular deduction orders]{Priority as between orders---regular deduction orders}

25H.---(1)  Paragraphs (2) to~(5) apply where one or~more third party debt orders or~garnishee orders provide for~deductions to be made from the same account as that specified in a regular deduction order.

(2) Where—
\begin{enumerate}\item[]
($a$) one or~more third party debt orders or~garnishee orders are served on a deposit-taker before or on the day a payment is due to be made under a regular deduction order; and

($b$) the regular deduction order was served on the same deposit-taker before those orders,
\end{enumerate}
the deposit-taker must make that payment except where the deposit-taker has taken action to comply with the obligations under any third party debt order or~garnishee order.

(3) Where a regular deduction order is served after an interim third party debt order or a garnishee order nisi the deposit-taker must take action to comply with any of those orders before making a deduction under the regular deduction order.

(4) Where paragraph~(2) or~(3) applies, the deposit-taker must take action to comply with any third party debt orders or~garnishee orders before making further deductions under the regular deduction order.

(5) Where a decision to revive a regular deduction order takes effect on the same day as or any day after a third party debt order or~garnishee order has been served, the deposit-taker must take action to comply with any of those orders before making a deduction under the regular deduction order.

(6) Paragraphs (1) to~(5) do not apply to~Scotland.

(7) In Scotland, paragraphs (8) to~(10) apply where a deposit-taker receives one or~more arrestment schedules (“arrestments”) and~a regular deduction order which apply to the same account.

(8) Where—
\begin{enumerate}\item[]
($a$) one or~more arrestments are served on a deposit-taker before or on the day a payment is due to be made under a regular deduction order; and

($b$) the regular deduction order was served on the same deposit-taker before any of those arrestments,
\end{enumerate}
the deposit-taker must make that payment except where the deposit-taker has taken action to comply with the obligations under any of the arrestments.

(9) Where paragraph~(8) applies, the deposit-taker must take action to comply with any of those arrestments before making further deductions under the regular deduction order.

(10) Where a decision to revive a regular deduction order takes effect on the same day as or any day after any arrestments have been served, the deposit-taker must take action to comply with any of those arrestments before making a deduction under the regular deduction order.

\subsubsection[25I. Variation of a regular deduction order]{Variation of a regular deduction order}

25I.---(1)  The 
%Commission 
Secretary of State  % Words substituted (1.8.12) by SI 2012/2007 Sch para 111(6)
may vary a regular deduction order by changing the amount to be deducted in the circumstances set out in paragraph~(2).

(2) The circumstances are that—
\begin{enumerate}\item[]
($a$) the 
%Commission 
Secretary of State  % Words substituted (1.8.12) by SI 2012/2007 Sch para 111(6)
has accepted—
\begin{enumerate}\item[]
(i) that a payment of arrears has been made by the liable person, and

(ii) no alternative method of payment of child support maintenance has been arranged;
\end{enumerate}

($b$) a decision has been made under section~11, 12, 16 or~17 of the Act or there has been an appeal against a maintenance calculation;

($c$) the 
%Commission 
Secretary of State  % Words substituted (1.8.12) by SI 2012/2007 Sch para 111(6)
has reviewed the order under regulation~25G (review of a regular deduction order); or

($d$) there has been an appeal under regulation~25AB(1)($a$)  or~($b$)  (appeals).
\end{enumerate}

(3) The 
%Commission 
Secretary of State  % Words substituted (1.8.12) by SI 2012/2007 Sch para 111(6)
may from time to time vary the deduction period.

(4) Where—
\begin{enumerate}\item[]
($a$) a regular deduction order has been varied under this regulation; and

($b$) a copy of the order as varied has been served on the deposit-taker at which it is directed,
\end{enumerate}
that deposit-taker must comply with the order; but the deposit-taker is not to be under any liability for~non-compliance before the end of the period of 7 days beginning on the day on which the copy of the order as varied is served on the deposit-taker.

\amendment{
Words substituted in reg. 25I(1), (2), (3) (1.8.12) by the Public Bodies (Child Maintenance and Enforcement Commission: Abolition and Transfer of Functions) Order 2012 Sch. para. 111(6).
}

\subsubsection[25J. Lapse of a regular deduction order]{Lapse of a regular deduction order}

25J.---(1)  A regular deduction order is to lapse in the circumstances set out in  paragraph~(2).

(2) The circumstances are where—
\begin{enumerate}\item[]
($a$) the 
%Commission 
Secretary of State  % Words substituted (1.8.12) by SI 2012/2007 Sch para 111(7)
has agreed with the liable person an alternative method of payment of the child support maintenance due under the maintenance calculation; or

($b$) there is an insufficient amount standing to the credit of the account specified in the order to enable a deduction to be made on two consecutive deduction dates, unless the 
%Commission 
Secretary of State  % Words substituted (1.8.12) by SI 2012/2007 Sch para 111(7)
has decided that the order is to continue for a greater number of deduction dates,
\end{enumerate}
and the 
%Commission 
Secretary of State  % Words substituted (1.8.12) by SI 2012/2007 Sch para 111(7)
considers it is reasonable in all the circumstances that the order is to lapse.

(3) A regular deduction order lapses on the day on which the deposit-taker receives notification that the order has lapsed from the 
%Commission 
Secretary of State%  % Words substituted (1.8.12) by SI 2012/2007 Sch para 111(7)
.

(4) A regular deduction order which has lapsed under this regulation is to be treated as remaining in force for the purposes of regulations 25E (notification by the deposit-taker to the 
%Commission 
Secretary of State%  % Words substituted (1.8.12) by SI 2012/2007 Sch para 111(7)
), 25G (review of a regular deduction order) and~25AB (appeals).

\amendment{
Words substituted in reg. 25J(2), (3), (4) (1.8.12) by the Public Bodies (Child Maintenance and Enforcement Commission: Abolition and Transfer of Functions) Order 2012 Sch. para. 111(7).
}

\subsubsection[25K. Revival of a regular deduction order]{Revival of a regular deduction order}

25K.---(1)  Where a regular deduction order has lapsed it may be revived by the 
%Commission 
Secretary of State  % Words substituted (1.8.12) by SI 2012/2007 Sch para 111(8)
where—
\begin{enumerate}\item[]
($a$) the liable person has failed to comply with any agreement reached under regulation~25J(2)($a$)  (lapse of a regular deduction order); or

($b$) the 
%Commission 
Secretary of State  % Words substituted (1.8.12) by SI 2012/2007 Sch para 111(8)
has reason to believe that following the lapse of an order under regulation~25J(2)($b$)  there is sufficient amount standing to the credit of the account specified in the order to enable a deduction to be made.
\end{enumerate}

(2) Where the 
%Commission 
Secretary of State  % Words substituted (1.8.12) by SI 2012/2007 Sch para 111(8)
decides to revive a regular deduction order that decision is to take effect on the day notification that the order has been revived is received by the deposit-taker.

\amendment{
Words substituted in reg. 25K (1.8.12) by the Public Bodies (Child Maintenance and Enforcement Commission: Abolition and Transfer of Functions) Order 2012 Sch. para. 111(8).
}

\subsubsection[25L. Discharge of a regular deduction order]{Discharge of a regular deduction order}

25L.---(1)  A regular deduction order must be discharged by the 
%Commission 
Secretary of State  % Words substituted (1.8.12) by SI 2012/2007 Sch para 111(9)(a)
where—
\begin{enumerate}\item[]
($a$) the account specified in the order has been closed;

($b$) the maintenance calculation in question is no longer in force and the amount of child support maintenance due under that calculation has been paid in full in accordance with regulation~2 (payment of child support maintenance);

($c$) the liable person has complied with any agreement reached under regulation~25J(2)($a$)  for such period as the Commission considers appropriate in the circumstances of the case;

($d$) the 
%Commission 
Secretary of State  % Words substituted (1.8.12) by SI 2012/2007 Sch para 111(9)(a)
has reviewed the order under regulation~25G and 
%it
the Secretary of State  % Words substituted (1.8.12) by SI 2012/2007 Sch para 111(9)(a)
has extinguished the amount to be deducted by the deposit-taker under the order;

($e$) on an appeal under regulation~25AB(1)($a$)  (appeals) the court has set aside the order;

($f$) unless sub-paragraph~($g$)  applies, a regular deduction order has lapsed under regulation~25J(2) and~6 months have passed beginning on the day the lapse took effect;

\begin{sloppypar}
($g$) an appeal is brought by virtue of regulation~25AB(1)($a$)  or~($b$), against a regular deduction order which has lapsed under regulation~25J(2) and~1 month has passed beginning on—
\end{sloppypar}
\begin{enumerate}\item[]
(i) the day proceedings on the appeal (including any further appeal) concluded, or

(ii) the end of any period during which a further appeal may ordinarily be brought,
\end{enumerate}
whichever is the later; or

($h$) the liable person has died.
\end{enumerate}

(2) A regular deduction order may be discharged where the 
%Commission 
Secretary of State  % Words substituted (1.8.12) by SI 2012/2007 Sch para 111(9)(b)
considers it is appropriate to do so in the circumstances of the case.

(3) Where a regular deduction order is discharged that discharge takes effect immediately after the payment of the last regular deduction prior~to~discharge.

\amendment{
Words substituted in reg. 25L(1), (2) (1.8.12) by the Public Bodies (Child Maintenance and Enforcement Commission: Abolition and Transfer of Functions) Order 2012 Sch. para. 111(9).
}

\subsection[Chapter III --- Lump sum deduction orders]{Chapter III\\*Lump sum deduction orders}

\renewcommand\parthead{--- Part IIIA Chapter III}

\subsubsection[25M. Period in which representations may be made]{Period in which representations may be made}

25M.  Where a lump sum deduction order has been made under section~32E(1) of the Act the period for~making representations to the 
%Commission 
Secretary of State  % Words substituted (1.8.12) by SI 2012/2007 Sch para 111(10)
in respect of the proposal specified in that order is 14 days beginning on the day a copy of the order was served.

\amendment{
Words substituted in reg. 25M (1.8.12) by the Public Bodies (Child Maintenance and Enforcement Commission: Abolition and Transfer of Functions) Order 2012 Sch. para. 111(10).
}

\subsubsection[25N. Disapplication of sections~32G(1) and~32H(2)($b$)  of the Act]{Disapplication of sections~32G(1) and~32H(2)($b$)  of the Act}

25N.---(1)  Something that would otherwise be in breach of sections~32G(1) and~32H(2)($b$)  of the Act may, with the consent of the 
%Commission 
Secretary of State%  % Words substituted (1.8.12) by SI 2012/2007 Sch para 111(11)
, be done in the following circumstances—
\begin{enumerate}\item[]
($a$) the liable person, the liable person’s partner or any relevant other child is suffering hardship in meeting ordinary living expenses;

($b$) the liable person is under a written contractual obligation, agreed before the lump sum deduction order was made, to make a payment;

($c$) the deposit-taker has a right of set off and satisfies the 
%Commission 
Secretary of State  % Words substituted (1.8.12) by SI 2012/2007 Sch para 111(11)
that an intention to exercise that right was formed within 30 days before the date the lump sum deduction order under section~32E of the Act was served;

($d$) the deposit-taker and the liable person have made a written agreement in which the availability of an amount standing to the credit of the account specified in the lump sum deduction order was required as security for that agreement; or

($e$) any other circumstances the 
%Commission 
Secretary of State  % Words substituted (1.8.12) by SI 2012/2007 Sch para 111(11)
considers appropriate in the particular case.
\end{enumerate}

(2) The liable person or the deposit-taker at which a lump sum deduction order is directed may apply to the 
%Commission 
Secretary of State  % Words substituted (1.8.12) by SI 2012/2007 Sch para 111(11)
for consent.

(3) When deciding whether to give consent, the 
%Commission 
Secretary of State  % Words substituted (1.8.12) by SI 2012/2007 Sch para 111(11)
must take into account—
\begin{enumerate}\item[]
($a$) any adverse impact the decision may have on the liable person or any other person; and

($b$) any alternative arrangements which may be made by the liable person or the deposit-taker.
\end{enumerate}

(4) Where the 
%Commission 
Secretary of State  % Words substituted (1.8.12) by SI 2012/2007 Sch para 111(11)
gives consent it is to take effect on the day on which the deposit-taker receives notification from the 
%Commission 
Secretary of State  % Words substituted (1.8.12) by SI 2012/2007 Sch para 111(11)
to~disapply section~32G(1) or~32H(2)($b$)  of the Act.

(5) Something that would otherwise be in breach of section~32G(1) and~32H(2)($b$)  of the Act may be done where—
\begin{enumerate}\item[]
($a$) the amount standing to the credit of the account specified in the lump sum deduction order is less than the amount specified in that order, except in respect of any amount dealt with in compliance with section~32G(1) of the Act; or

($b$) the deposit-taker has made a payment in accordance with section~32H(1)($a$)  of the Act.
\end{enumerate}

(6) Paragraph (5) has effect until the 
%Commission 
Secretary of State  % Words substituted (1.8.12) by SI 2012/2007 Sch para 111(11)
gives notice to the deposit-taker that paragraph~(5) has ceased to have effect in a particular case and that notification is to take effect on the day on which the deposit-taker receives notification from the 
%Commission 
Secretary of State%  % Words substituted (1.8.12) by SI 2012/2007 Sch para 111(11)
.

(7) In this regulation—
\begin{enumerate}\item[]
\begin{sloppypar}
“partner” has the same meaning as in regulation~3(9) (method of payment) and the definition of “couple” in that regulation is to apply accordingly; and
\end{sloppypar}

“relevant other child” is to be interpreted in accordance with paragraph~10C(2) of Schedule 1 to the Act\footnote{Paragraph 10C was inserted by section~1(3) of, and~Schedule 1 to, the Child Support, Pensions and~Social Security Act 2000 (c. 34).} and regulations made for the purposes of that paragraph.
\end{enumerate}

\amendment{
Words substituted in reg. 25N(1), (2), (3), (4), (6) (1.8.12) by the Public Bodies (Child Maintenance and Enforcement Commission: Abolition and Transfer of Functions) Order 2012 Sch. para. 111(11).
}

\subsubsection[25O. Information]{Information}

25O.---(1)  A deposit-taker at which a lump sum deduction order is directed must supply to the 
%Commission 
Secretary of State  % Words substituted (1.8.12) by SI 2012/2007 Sch para 111(12)
in writing, within 7 days—
\begin{enumerate}\item[]
($a$) of a copy of the order or order as varied being served; or

($b$) of notification being received by the deposit-taker that an order has been revived,
\end{enumerate}
the information set out in paragraph~(2).

(2) The information is—
\begin{enumerate}\item[]
($a$) if the account specified in the order—
\begin{enumerate}\item[]
(i) does not exist,

(ii) cannot be traced, or

(iii) has been closed;
\end{enumerate}

($b$) whether the amount standing to the credit of the account specified in the order—
\begin{enumerate}\item[]
(i) on the day the order is served, or

(ii) where an order is revived, on the day the decision to revive the order takes effect,
\end{enumerate}
is at least the same or less than the amount specified in the order and where it is less, that amount; and

($c$) where the name of the liable person specified in the order is different to the name in which the account specified in the order is held—
\begin{enumerate}\item[]
(i) whether the account was previously held in the name of the liable person specified in the order, and

(ii) if so, the new name in which the account is held,
\end{enumerate}
only where the liable person named in the order is the same person as the person in whose name the account specified in the order is held.
\end{enumerate}

(3) A deposit-taker at which a lump sum deduction order is directed must notify the 
%Commission 
Secretary of State  % Words substituted (1.8.12) by SI 2012/2007 Sch para 111(12)
within 7 days of notification being received that an order has lapsed or has been discharged—
\begin{enumerate}\item[]
($a$) if the account specified in the order cannot be traced; or

($b$) where the name of the liable person specified in the order is different to the name in which the account specified in the order is held—
\begin{enumerate}\item[]
(i) whether the account was previously held in the name of the liable person specified in the order, and

(ii) if so, the new name in which the account is held,
\end{enumerate}
only where the liable person named in the order is the same person as the person in whose name the account specified in the order is held.
\end{enumerate}

(4) A deposit-taker at which a lump sum deduction order is directed, must supply to the 
%Commission 
Secretary of State  % Words substituted (1.8.12) by SI 2012/2007 Sch para 111(12)
within 7 days of receipt of a request being made by the 
%Commission 
Secretary of State%  % Words substituted (1.8.12) by SI 2012/2007 Sch para 111(12)
, the following information—
\begin{enumerate}\item[]
($a$) whether the liable person holds another account or has opened an account with that deposit-taker or~with another deposit-taker and, if so, the details of that account, including—
\begin{enumerate}\item[]
(i) the number and sort code of that account, and

(ii) the type of account; and
\end{enumerate}

($b$) whether the amount standing to the credit of the account specified in the order on the day the request is received is at least the same or less than the amount specified in the order or the remaining amount and where it is less, that amount.
\end{enumerate}

(5) In so far as a deposit-taker at which a lump sum deduction order is directed (“$\mathcal{A}$”) has the information, the details of an account held with another deposit-taker (“$\mathcal{B}$”) must be supplied to the Commission in accordance with paragraph~(4) only if—
\begin{enumerate}\item[]
($a$) the liable person has—
\begin{enumerate}\item[]
(i) closed the account specified in the order and~held with $\mathcal{A}$,

(ii) opened an account with $\mathcal{B}$, and

(iii) transferred the amount standing to the credit of the account held with $\mathcal{A}$ to the account held with $\mathcal{B}$;
\end{enumerate}

($b$) either—
\begin{enumerate}\item[]
(i) a lump sum deduction order has lapsed, or

\begin{sloppypar}
(ii) $\mathcal{A}$ has notified the 
%Commission 
Secretary of State  % Words substituted (1.8.12) by SI 2012/2007 Sch para 111(12)
in accordance with paragraph~(2)($a$)(iii), that the account specified in the order has been closed; and
\end{sloppypar}
\end{enumerate}

($c$) the 
%Commission 
Secretary of State  % Words substituted (1.8.12) by SI 2012/2007 Sch para 111(12)
has made a request for the information within 1 month of the order lapsing or, as the case may be, notification being received by the 
%Commission 
Secretary of State  % Words substituted (1.8.12) by SI 2012/2007 Sch para 111(12)
that the account has been closed.
\end{enumerate}

(6) The requirements of paragraphs (1) to~(3) and~paragraph~(4) as it applies to~a deposit-taker at which a lump sum deduction order is directed, apply only in so far as the deposit-taker has the information or~can reasonably be expected to~acquire it.

(7) In paragraph~(4)($b$)  and regulation~25T(1)($b$)  and~($c$)  “remaining amount” has the same meaning as in section~32H(6) of the Act.

\amendment{
Words substituted in reg. 25O(1), (3), (4), (5) (1.8.12) by the Public Bodies (Child Maintenance and Enforcement Commission: Abolition and Transfer of Functions) Order 2012 Sch. para. 111(12).
}

\subsubsection[25P. Priority as between orders---lump sum deduction orders]{Priority as between orders---lump sum deduction orders}

25P.---(1)  Where a deposit-taker would, but for this paragraph, be obliged to comply with an order under section~32F of the Act, and~one or~more interim third party debt orders or~garnishee orders nisi, it must take action to comply with the orders according to the order in which they were served on the deposit-taker.

(2) Paragraph (1) does not apply where an order under section~32E of the Act was served after an interim third party debt order or a garnishee order nisi except where there remains an amount standing to the credit of the account specified in the order under section~32F of the Act after any third party debt orders or~garnishee orders have been complied with by the deposit-taker (referred to in this regulation~as “an outstanding amount”).

(3) Where there is an outstanding amount section~32G(1) of the Act applies in respect of that amount.

(4) Where a decision to revive a lump sum deduction order takes effect on the same day as or any day after a third party debt order or~garnishee order has been served, the deposit-taker must take action to comply with any of those orders before making a deduction under the lump sum deduction order.

(5) Paragraphs (1) to~(4) do not apply to~Scotland.

(6) In Scotland, where a deposit-taker would, but for this paragraph, be obliged to comply with an order under section~32F of the Act, and~one or~more arrestment schedules (“arrestments”) it must give preference to that order and those arrestments according to the order in which they were served on the deposit-taker.

(7) Where there remains an amount standing to the credit of the account specified in the order under section~32F of the Act after any arrestments have been complied with by the deposit-taker, section~32G(1) of the Act applies in respect of that amount.

(8) Where a decision to revive a lump sum deduction order takes effect on the same day as or any day after any arrestments have been served, the deposit-taker must take action to comply with any of those arrestments before making a deduction under the lump sum deduction order.

\subsubsection[25Q. Minimum amount]{Minimum amount}

25Q.---(1)  A deduction must not be made where the amount standing to the credit of the account specified in the lump sum deduction order is below the minimum amount on the date the deduction is due to be made.

(2) The minimum amount is £55 plus the amount of administrative costs authorised by regulation~25Z($b$)  (administrative costs).

\subsubsection[25R. Variation of a lump sum deduction order]{Variation of a lump sum deduction order}

25R.---(1)  The 
%Commission 
Secretary of State  % Words substituted (1.8.12) by SI 2012/2007 Sch para 111(13)
may, in the circumstances set out in paragraph~(2), vary a lump sum deduction order by reducing the amount specified in that order.

(2) The circumstances are that—
\begin{enumerate}\item[]
($a$) the 
%Commission 
Secretary of State  % Words substituted (1.8.12) by SI 2012/2007 Sch para 111(13)
accepts the liable person’s agreement to make a payment;

($b$) a decision has been made under section~11, 12, 16 or~17 of the Act or there has been an appeal against a maintenance calculation;

\begin{sloppypar}
($c$) the 
%Commission 
Secretary of State  % Words substituted (1.8.12) by SI 2012/2007 Sch para 111(13)
has consented to the doing of things that would otherwise be in breach of sections~32G(1) and~32H(2)($b$)  of the Act;
\end{sloppypar}

\begin{sloppypar}
($d$) there has been an appeal made under regulation 25AB(1)($c$)  or ($d$)  (appeals); or
\end{sloppypar}

($e$) representations made in respect of the proposals specified in the order made under section~32E of the Act have been accepted by the 
%Commission 
Secretary of State%  % Words substituted (1.8.12) by SI 2012/2007 Sch para 111(13)
.
\end{enumerate}

(3) Where—
\begin{enumerate}\item[]
($a$) a lump sum deduction order has been varied under this regulation; and

($b$) a copy of the order as varied has been served on the deposit-taker at which it is directed,
\end{enumerate}
that deposit-taker must comply with the order when that order is served.

\amendment{
Words substituted in reg. 25R(1), (2) (1.8.12) by the Public Bodies (Child Maintenance and Enforcement Commission: Abolition and Transfer of Functions) Order 2012 Sch. para. 111(13).
}

\subsubsection[25S. Lapse of a lump sum deduction order]{Lapse of a lump sum deduction order}

25S.---(1)  A lump sum deduction order is to lapse in the circumstances set out in paragraph~(2).

(2) The circumstances are where—
\begin{enumerate}\item[]
($a$) the amount in the account specified in the order under section~32E of the Act is nil;

($b$) in consequence of the consent given by the 
%Commission 
Secretary of State  % Words substituted (1.8.12) by SI 2012/2007 Sch para 111(14)
under regulation~25N(1) (disapplication of section~32G(1) and~32H(2)($b$)  of the Act) the amount in the account specified in the lump sum deduction order is reduced to nil; or

($c$) the 
%Commission 
Secretary of State  % Words substituted (1.8.12) by SI 2012/2007 Sch para 111(14)
has agreed with the liable person an alternative method of payment of the child support maintenance due under the maintenance calculation,
\end{enumerate}
and the 
%Commission 
Secretary of State  % Words substituted (1.8.12) by SI 2012/2007 Sch para 111(14)
considers it is reasonable in all the circumstances that the order is to lapse.

(3) A lump sum deduction order lapses on the day on which the deposit-taker receives notification that the order has lapsed from the 
%Commission 
Secretary of State%  % Words substituted (1.8.12) by SI 2012/2007 Sch para 111(14)
.

(4) A lump sum deduction order which has lapsed under this regulation is to be treated as remaining in force for the purposes of regulations 25M (period in which representations may be made),~25O (information) and~25AB (appeals).

\amendment{
Words substituted in reg. 25S(2), (3) (1.8.12) by the Public Bodies (Child Maintenance and Enforcement Commission: Abolition and Transfer of Functions) Order 2012 Sch. para. 111(14).
}

\subsubsection[25T. Revival of a lump sum deduction order]{Revival of a lump sum deduction order}

25T.---(1)  Where a lump sum deduction order has lapsed it may be revived by the 
%Commission 
Secretary of State  % Words substituted (1.8.12) by SI 2012/2007 Sch para 111(15)
where—
\begin{enumerate}\item[]
($a$) in the case of an order under section~32E of the Act, the amount standing to the credit of the account specified in that order was nil and the 
%Commission 
Secretary of State  % Words substituted (1.8.12) by SI 2012/2007 Sch para 111(15)
is informed in accordance with the requirement in regulation~25O(4)($b$)  that there is an amount at least the same as or less than the amount specified in the order standing to the credit of the account specified in the order;

($b$) a lump sum deduction order has lapsed under regulation~25S(2)($b$)  (lapse of a lump sum deduction order) and the 
%Commission 
Secretary of State  % Words substituted (1.8.12) by SI 2012/2007 Sch para 111(15)
is informed in accordance with the requirement in regulation~25O(4)($b$)  that there is an amount at least the same as or less than the amount specified in the order, or the remaining amount, standing to the credit of the account specified in the order; or

($c$) in the case of an order under section~32F of the Act, there is a remaining amount and the liable person has failed to comply with the agreement referred to in regulation~25S(2)($c$).
\end{enumerate}

(2) Where the 
%Commission 
Secretary of State  % Words substituted (1.8.12) by SI 2012/2007 Sch para 111(15)
decides to revive a lump sum deduction order that decision is to take effect on the day notification that the order has been revived is received by the deposit-taker.

\amendment{
Words substituted in reg. 25T (1.8.12) by the Public Bodies (Child Maintenance and Enforcement Commission: Abolition and Transfer of Functions) Order 2012 Sch. para. 111(15).
}

\subsubsection[25U. Discharge of a lump sum deduction order]{Discharge of a lump sum deduction order}

25U.---(1)  A lump sum deduction order must be discharged where—
\begin{enumerate}\item[]
($a$) the account specified in the order has been closed;

($b$) the amount of arrears of child support maintenance specified in the order has been paid in full in accordance with regulation~2 (payment of child support maintenance);

($c$) the liable person has paid the total amount of arrears of child support maintenance specified in the order by an alternative method agreed between the 
%Commission 
Secretary of State  % Words substituted (1.8.12) by SI 2012/2007 Sch para 111(16)(a)
and the liable person;

($d$) the 
%Commission 
Secretary of State  % Words substituted (1.8.12) by SI 2012/2007 Sch para 111(16)(a)
has considered representations made in respect of an order under section~32E of the Act and 
%it 
the Secretary of State  % Words substituted (1.8.12) by SI 2012/2007 Sch para 111(16)(b)
has decided not to make an order under section~32F of the Act;

($e$) unless sub-paragraph~($f$)  applies—
\begin{enumerate}\item[]
(i) an order under section~32F of the Act has lapsed under regulation~25S(2) and 6 months have passed beginning on the day on which the deposit-taker received notification that the order had lapsed from the 
%Commission 
Secretary of State%  % Words substituted (1.8.12) by SI 2012/2007 Sch para 111(16)(a)
, or

(ii) regulation~25N(5) applies and 6 months have passed beginning on the day on which payment was made under section~32H(1)($a$)  of the Act;
\end{enumerate}

($f$) an appeal is brought by virtue of regulation~25AB(1)($d$)  and 1 month has passed beginning on—
\begin{enumerate}\item[]
(i) the day proceedings on the appeal (including any further appeal) concluded, or

(ii) the end of any period during which a further appeal may ordinarily be brought,
\end{enumerate}
whichever is the later; or

($g$) the liable person has died.
\end{enumerate}

(2) A lump sum deduction order may be discharged where the 
%Commission 
Secretary of State  % Words substituted (1.8.12) by SI 2012/2007 Sch para 111(16)(a)
considers it is appropriate to do so in the circumstances of the case.

(3) A lump sum deduction order is discharged on the day notification that the order has been discharged is received by the deposit-taker.

\amendment{
Words substituted in reg. 25U(1), (2) (1.8.12) by the Public Bodies (Child Maintenance and Enforcement Commission: Abolition and Transfer of Functions) Order 2012 Sch. para. 111(16).
}

\subsubsection[25V. Time at which a lump sum deduction order under section~32E of the Act ceases to be in force]{Time at which a lump sum deduction order under section~32E of the Act ceases to be in force}

25V.  For the purposes of section~32E(8)($a$)  of the Act the prescribed period is—
\begin{enumerate}\item[]
($a$) unless paragraph~($b$)  applies, 6 months beginning on—
\begin{enumerate}\item[]
(i) the day the order under section~32E of the Act was served on the deposit-taker, or

(ii) where that order has lapsed under regulation~25S, the day on which the deposit-taker received notification that the order had lapsed from the 
%Commission 
Secretary of State%  % Words substituted (1.8.12) by SI 2012/2007 Sch para 111(17)
; or
\end{enumerate}

\begin{sloppypar}
($b$) where an appeal is brought by virtue of regulation 25AB(1)($c$)  (appeal against the withholding of consent), 1 month beginning on—
\end{sloppypar}
\begin{enumerate}\item[]
(i) the day proceedings on the appeal (including any further appeal) concluded, or

(ii) the end of any period during which a further appeal may ordinarily be brought,
\end{enumerate}
whichever is the later.
\end{enumerate}

\amendment{
Words substituted in reg. 25V(a)(ii) (1.8.12) by the Public Bodies (Child Maintenance and Enforcement Commission: Abolition and Transfer of Functions) Order 2012 Sch. para. 111(17).
}

\subsubsection[25W. Meaning of “the relevant time”]{Meaning of “the relevant time”}

25W.  For the purposes of the meaning of “the relevant time” in section~32H(6) of the Act the prescribed circumstances are that—
\begin{enumerate}\item[]
($a$) unless to paragraph~($b$)  applies, 6 months have passed beginning on the day the order under section~32F of the Act was served on the deposit-taker; or

\begin{sloppypar}
($b$) where an appeal is brought by virtue of regulation 25AB(1)($d$), 1 month has passed beginning on—
\end{sloppypar}
\begin{enumerate}\item[]
(i) the day proceedings on the appeal (including any further appeal) concluded, or

(ii) the end of any period during which a further appeal may ordinarily be brought,
\end{enumerate}
whichever is the later.
\end{enumerate}

\subsection[Chapter IV --- General matters for deduction orders]{Chapter IV\\*General matters for deduction orders}

\renewcommand\parthead{--- Part IIIA Chapter IV}

\subsubsection[25X. Accounts of a prescribed description]{Accounts of a prescribed description}

25X.---(1)  A regular deduction order or a lump sum deduction order may not be made in respect of an account which—
\begin{enumerate}\item[]
($a$) the liable person operates solely for the purposes of exercising the function of a trustee or office holder and the account is one in which all the funds are held on behalf of other persons or for the purposes of that office; or

($b$) is used wholly or in part for business purposes.
\end{enumerate}

(2) For the purposes of paragraph~(1)($b$), whether an account is used wholly or in part for business purposes is to be decided by the 
%Commission 
Secretary of State%  % Words substituted (1.8.12) by SI 2012/2007 Sch para 111(18)
.

(3) Paragraph (1)($b$)  does not apply where a regular deduction order is made in respect of an account which is used by the liable person as a sole trader.

\amendment{
Words substituted in reg. 25X(2) (1.8.12) by the Public Bodies (Child Maintenance and Enforcement Commission: Abolition and Transfer of Functions) Order 2012 Sch. para. 111(18).
}

\subsubsection[25Y. Circumstances in which amounts standing to the credit of an account are to be disregarded]{Circumstances in which amounts standing to the credit of an account are to be disregarded}

25Y.  The circumstances in which amounts standing to the credit of an account are to be disregarded for the purposes of sections~32A,~32E, 32G and~32H of the Act are where the liable person has no beneficial interest in the amount.

\subsubsection[25Z. Administrative costs]{Administrative costs}

25Z.  A deposit-taker at which an order under section~32A or~32F of the Act is directed may deduct from the amount standing to the credit of the account specified in the order an amount towards its administrative costs for~each deduction made, not exceeding—
\begin{enumerate}\item[]
($a$) in the case of a regular deduction order, £10; or

($b$) in the case of a lump sum deduction order under section~32F of the Act, £55,
\end{enumerate}
before making any payment to the 
%Commission 
Secretary of State  % Words substituted (1.8.12) by SI 2012/2007 Sch para 111(19)
required by section~32A or, as the case may be, section~32H of the Act.

\amendment{
Words substituted in reg. 25Z (1.8.12) by the Public Bodies (Child Maintenance and Enforcement Commission: Abolition and Transfer of Functions) Order 2012 Sch. para. 111(19).
}

\subsubsection[25AA. Payment by deposit-taker to the 
%Commission 
Secretary of State%  % Words substituted (1.8.12) by SI 2012/2007 Sch para 111(20)
]{Payment by deposit-taker to the 
%Commission 
Secretary of State%  % Words substituted (1.8.12) by SI 2012/2007 Sch para 111(20)
}

25AA.---(1)  Amounts deducted by a deposit-taker at which a regular deduction order or a lump sum deduction order under section~32F of the Act is directed must be paid to the 
%Commission 
Secretary of State  % Words substituted (1.8.12) by SI 2012/2007 Sch para 111(20)
within—
\begin{enumerate}\item[]
($a$) in the case of a regular deduction order, 10 days of the date the regular deduction is due to be made; or

($b$) in the case of a lump sum deduction order under section~32F of the Act, 10 days of the end of the relevant period.
\end{enumerate}

(2) The payment to the 
%Commission 
Secretary of State  % Words substituted (1.8.12) by SI 2012/2007 Sch para 111(20)
of amounts deducted under that order may be made by—
\begin{enumerate}\item[]
($a$) cheque;

($b$) automated credit transfer; or

($c$) such other method as the Commission may specify.
\end{enumerate}

(3) In this regulation~“the relevant period” has the same meaning as in section~32G(5) and~(6) of the Act.

\amendment{
Words substituted in reg. 25AA(1), (2) and heading (1.8.12) by the Public Bodies (Child Maintenance and Enforcement Commission: Abolition and Transfer of Functions) Order 2012 Sch. para. 111(20).
}

\subsubsection[25AB. Appeals]{Appeals}

25AB.---(1)  A qualifying person has a right of appeal to~a county court or in Scotland the sheriff of the sheriffdom in which that person resides, against—
\begin{enumerate}\item[]
($a$) the making of a regular deduction order;

($b$) any decision made by the 
%Commission 
Secretary of State  % Words substituted (1.8.12) by SI 2012/2007 Sch para 111(21)
on an application made under regulation~25G (review of a regular deduction order);

($c$) the withholding of the consent to be obtained in accordance with regulation~25N (disapplication of sections~32G(1) and~32H(2)($b$)  of the Act);

($d$) the making of an order under section~32F of the Act.
\end{enumerate}

(2) In this regulation~a “qualifying person” means—
\begin{enumerate}\item[]
($a$) in relation to paragraph~(1)($a$)  and~($b$), any person affected by—
\begin{enumerate}\item[]
(i) a regular deduction order, or, as the case may be,

(ii) the decision referred to in paragraph~(1)($b$);
\end{enumerate}

($b$) in relation to paragraph~(1)($c$), the persons prescribed in regulation~25N(2); and

($c$) in relation to paragraph~(1)($d$), any person affected by an order under section~32F of the Act.
\end{enumerate}

\amendment{
Words substituted in reg. 25AB(1)(b) (1.8.12) by the Public Bodies (Child Maintenance and Enforcement Commission: Abolition and Transfer of Functions) Order 2012 Sch. para. 111(21).
}

\subsubsection[25AC. Offences]{Offences}

25AC. The following regulations are designated for the purposes of sections~32D(1)($b$)  and~32K(1)($b$)  of the Act—
\begin{enumerate}\item[]
($a$) regulation~25E(1) to~(5) (notification by the deposit-taker to the 
%Commission 
Secretary of State%  % Words substituted (1.8.12) by SI 2012/2007 Sch para 111(22)
);

($b$) regulation~25I(4) (variation of a regular deduction order);

($c$) regulation~25O(1) to~(5) (information);

($d$) regulation~25R(3) (variation of a lump sum deduction order); and

($e$) regulation~25AA(1) (payment by deposit-taker to the 
%Commission 
Secretary of State%  % Words substituted (1.8.12) by SI 2012/2007 Sch para 111(22)
).
\end{enumerate}

\amendment{
Words substituted in reg. 25AC(a), (e) (1.8.12) by the Public Bodies (Child Maintenance and Enforcement Commission: Abolition and Transfer of Functions) Order 2012 Sch. para. 111(22).
}

\subsubsection[25AD. 
%Commission 
Secretary of State  % Words substituted (1.8.12) by SI 2012/2007 Sch para 111(21)
to warn of consequences of failing to comply with an order or~to provide information]{%
%Commission 
Secretary of State  % Words substituted (1.8.12) by SI 2012/2007 Sch para 111(21)
to warn of consequences of failing to comply with an order or~to provide information}

25AD.  Where information is required by virtue of regulation~25E or~25O, the 
%Commission 
Secretary of State  % Words substituted (1.8.12) by SI 2012/2007 Sch para 111(21)
must set out in writing the possible consequences of failure to—
\begin{enumerate}\item[]
($a$) comply with a regular deduction order or lump sum deduction order; and

($b$) provide the information required under the regulations designated by regulation~25AC($a$)  and~($b$)  (offences),
\end{enumerate}
including details of the offences provided for by virtue of sections~32D and~32K of the   Act, as the case may be.

\amendment{
Words substituted in reg. 25AD and heading (1.8.12) by the Public Bodies (Child Maintenance and Enforcement Commission: Abolition and Transfer of Functions) Order 2012 Sch. para. 111(23).
}

\section[Part IV --- Liability orders]{Part IV\\*Liability orders}

\renewcommand\parthead{--- Part IV}

\subsection[26. Extent of this Part]{Extent of this Part}

26.  This Part, except 
%regulation~29(2)
regulations 29(2) and~35(5)%  % Words substituted (12.7.06) by SI 2006/1520 reg 3(4)
, does not apply to~Scotland.

\amendment{
Words substituted in reg.~26 (12.7.06) by the Child Support (Miscellaneous Amendments) Regulations 2006 reg.~3(4).
}

\subsection[27. Notice of intention to~apply for a liability order]{Notice of intention to~apply for a liability order}

27.—(1) 
Subject to paragraph~(1A),  % Words inserted (1.8.07) by SI 2007/1979 reg 2(3)(a)
the Secretary of State shall give the liable person at least 7 days notice of his intention to~apply for a liability order under section~33(2) of the Act.

% Reg 27(1A) inserted (1.8.07) by SI 2007/1979 reg 2(3)(b)
(1A) Where the liable person is not resident in the United Kingdom, the Secretary of State shall give the liable person at least 28 days notice of his intention to~apply for a liability order under section~33(2) of the Act.

(2) Such notice shall set out the amount of child support maintenance which it is claimed has become payable by the liable person and~has not been paid and the amount of any interest%
% in respect of arrears payable under section~41(3) of the Act\footnote{\frenchspacing See The Child Support (Arrears, Interest and~Adjustment of Maintenance Assessments) Regulations 1992, S.I.~1992/1816.}
\emph{, penalty payments or fees which have become payable and~have not been paid}.  % Words substituted (3.3.03 for new-rules cases only) by SI 2001/162 reg 2(6)(a)

(3) Payment by the liable person of any part of the amounts referred to in paragraph~(2) shall not require the giving of a further notice under paragraph~(1) prior to the making of the application.

\amendment{
Words substituted in reg.~27(2) (3.3.03 for new-rules cases only) by the Child Support (Collection and~Enforcement and~Miscellaneous Amendments) Regulations 2000 reg.~2(6)(a) (subject to~savings in reg.~6).

Words inserted in reg.~27(1) and reg.~27(1A) inserted (1.8.07) by the Child Support (Miscellaneous Amendments) Regulations 2007 reg.~2(3).

For 1993 scheme cases reg.~27(2) should be read as follows:
\begin{quotation}
(2) Such notice shall set out the amount of child support maintenance which it is claimed has become payable by the liable person and~has not been paid and the amount of any interest in respect of arrears payable under section~41(3) of the Act\footnote{\frenchspacing \emph{See} the Child Support (Arrears, Interest and~Adjustment of Maintenance Assessments) Regulations 1992, S.I.~1992/1816.}.
\end{quotation}

}

\subsection[28. Application for a liability order]{Application for a liability order}

28.—(1) An application for a liability order shall be by way of complaint for an order to the magistrates' court% 
%having jurisdiction in the area in which the liable person resides  % Words omitted (1.8.07) by SI 2007/1979 reg 2(4)
.

%(2) An application under paragraph~(1) may not be instituted more than 6 years after the day on which payment of the amount in question became due.

% Reg 28(2) substituted (12.7.06) by SI 2006/1520 reg 3(5)(a)
(2) Subject to paragraph~(2A), there is no period of limitation in relation to~an application under paragraph~(1).

% Reg 28(2A) inserted (12.7.06) by SI 2006/1520 reg 3(5)(b)
(2A) An application under paragraph~(1) may not be instituted in respect of an amount payment of which became due on or before 12th July 2000.

(3) A warrant shall not be issued under section~55(2) of the Magistrates' Courts Act 1980\footnote{\frenchspacing 1980 c. 43.} in any proceedings under this regulation.

\amendment{
Reg. 28(2A) inserted and reg.~28(2) substituted (12.7.06) by the Child Support (Miscellaneous Amendments) Regulations 2006 reg.~3(5).

Words omitted in reg.~28(1) (1.8.07) by the Child Support (Miscellaneous Amendments) Regulations 2007 reg.~2(4).
}

\subsection[29. Liability orders]{Liability orders}

29.—(1) A liability order shall be made in the form prescribed in Schedule~1.

(2) A liability order made by a court in England~or Wales or any corresponding order made by a court in Northern Ireland~may be enforced in Scotland~as if it had been made by the sheriff.

(3) A liability order made by the sheriff in Scotland~or any corresponding order made by a court in Northern Ireland~may, subject to paragraph~(4), be enforced in England~and~Wales as if it had been made by a magistrates' court in England~and~Wales.

(4) A liability order made by the sheriff in Scotland~or a corresponding order made by a court in Northern Ireland shall not be enforced in England~or Wales unless registered in accordance with the provisions of 
%Part I 
Part II % Words substituted (5.4.93) by SI 1993/913 reg 43
of the Maintenance Orders Act 1950\footnote{\frenchspacing 14 Geo. 6 c. 37.} and~for this purpose—
\begin{enumerate}\item[]
($a$) a liability order made by the sheriff in Scotland shall be treated as if it were a decree to which section~16(2)($b$) of that Act applies (decree for payment of aliment);

($b$) a corresponding order made by a court in Northern Ireland shall be treated as if it were an order to which section~16(2)($c$) of that Act applies (order for alimony, maintenance or other payments).
\end{enumerate}

\amendment{
Words substituted in reg.~29(4) (5.4.93) by the Child Support (Miscellaneous Amendments) Regulations 1993 reg.~43.
}

\subsection[30. Enforcement of liability orders by distress]{Enforcement of liability orders by distress}

30.—(1) A distress made pursuant to~section~35(1) of the Act may be made anywhere in England~and~Wales.

(2) The person levying distress on behalf of the Secretary of State shall carry with him the written authorisation of the Secretary of State, which he shall show to the liable person if so requested, and~he shall hand to the liable person or leave at the premises where the distress is levied—
\begin{enumerate}\item[]
($a$) copies of this regulation, regulation~31 and~Schedule~2;

($b$) a memorandum setting out the amount which is the appropriate amount for the purposes of section~35(2) of the Act;

($c$) a memorandum setting out details of any arrangement entered into regarding the taking of possession of the goods distrained; and

($d$) a notice setting out the liable person’s rights of appeal under regulation~31 giving the Secretary of State’s address for the purposes of any appeal.
\end{enumerate}

(3) A distress shall not be deemed unlawful on account of any defect or want of form in the liability order.

(4) If, before any goods are seized, the appropriate amount (including charges arising up to the time of the payment or tender) is paid or tendered to the Secretary of State, the Secretary of State shall accept the amount and the levy shall not be proceeded with.

(5) Where the Secretary of State has seized goods of the liable person in pursuance of the distress, but before sale of those goods the appropriate amount (including charges arising up to the time of the payment or tender) is paid or tendered to the Secretary of State, the Secretary of State shall accept the amount, the sale shall not be proceeded with and the goods shall be made available for collection by the liable person.

\subsection[31. Appeals in connection with distress]{Appeals in connection with distress}

31.—(1) A person aggrieved by the levy of, or an attempt to levy, a distress may appeal to the magistrates' court% 
%having jurisdiction in the area in which he resides  % Words omitted (1.8.07) by SI 2007/1979 reg 2(5)
.

(2) The appeal shall be by way of complaint for an order.

(3) If the court is satisfied that the levy was irregular, it may—
\begin{enumerate}\item[]
($a$) order the goods distrained to be discharged if they are in the possession of the Secretary of State;

($b$) order an award of compensation in respect of any goods distrained and sold of an amount equal to the amount which, in the opinion of the court, would be awarded by way of special damages in respect of the goods if proceedings under section~35(6) of the Act were brought in trespass or otherwise in connection with the irregularity.
\end{enumerate}

(4) If the court is satisfied that an attempted levy was irregular, it may by order require the Secretary of State to~desist from levying in the manner giving rise to the irregularity.

\amendment{
Words omitted in reg.~31(1) (1.8.07) by the Child Support (Miscellaneous Amendments) Regulations 2007 reg.~2(5).
}

\subsection[32. Charges connected with distress]{Charges connected with distress}

32.  Schedule~2 shall have effect for the purpose of determining the amounts in respect of charges in connection with the distress for the purposes of section~35(2)($b$) of the Act.

\subsection[33. Application for warrant of commitment --- \emph{1993 scheme version}]{Application for warrant of commitment\\*\emph{1993 scheme version}}

33.—(1) For the purposes of enabling an inquiry to be made under section~40 of the Act as to the liable person’s conduct and~means, a justice of the peace 
%having jurisdiction for the area in which the liable person resides   % Words omitted (1.8.07) by SI 2007/1979 reg 2(6)
may—
\begin{enumerate}\item[]
($a$) issue a summons to him to~appear before a magistrates' court and~(if he does not obey the summons) issue a warrant for his arrest; or

($b$) issue a warrant for his arrest without issuing a summons.
\end{enumerate}

(2) In any proceedings under section~40 of the Act, a statement in writing to the effect that wages of any amount have been paid to the liable person during any period, purporting to be signed by or on behalf of his employer, shall be evidence of the facts there stated.

(3) Where an application under section~40 of the Act has been made but no warrant of commitment is issued or term of imprisonment fixed, the application may be renewed on the ground that the circumstances of the liable person have changed.

\amendment{
Words omitted in reg.~33(1) (1.8.07) by the Child Support (Miscellaneous Amendments) Regulations 2007 reg.~2(6).
}

\subsection[33. Application for warrant of commitment --- \emph{2003 scheme version}]{Application for warrant of commitment\\*\emph{2003 scheme version}}

33.—(1) For the purposes of enabling an inquiry to be made under section~%
%40
39A  % Word substituted (3.3.03 for new-rules cases only) by SI 2001/162 reg 2(6)(b)(i) 
of the Act as to the liable person’s conduct and~means, a justice of the peace 
%having jurisdiction for the area in which the liable person resides   % Words omitted (1.8.07) by SI 2007/1979 reg 2(6)
may—
\begin{enumerate}\item[]
($a$) issue a summons to him to~appear before a magistrates' court and~(if he does not obey the summons) issue a warrant for his arrest; or

($b$) issue a warrant for his arrest without issuing a summons.
\end{enumerate}

(2) In any proceedings under 
%section~40
sections~39A and~40  % Words substituted (3.3.03 for new-rules cases only) by SI 2001/162 reg 2(6)(b)(ii) 
of the Act, a statement in writing to the effect that wages of any amount have been paid to the liable person during any period, purporting to be signed by or on behalf of his employer, shall be evidence of the facts there stated.

(3) Where an application under section~%
%40
39A  % Word substituted (3.3.03 for new-rules cases only) by SI 2001/162 reg 2(6)(b)(i) 
of the Act has been made but no warrant of commitment is issued or term of imprisonment fixed, the application may be renewed on the ground that the circumstances of the liable person have changed.

\amendment{
Words substituted in reg.~33(1), (2), (3) (31.5.01) by the Child Support (Collection and~Enforcement and~Miscellaneous Amendments) Regulations 2000 reg.~2(6)(b) (as modified by the Child Support (Miscellaneous Amendments) Regulations 2001 reg.~2 and subject to~savings in reg.~6).

Words omitted in reg.~33(1) (1.8.07) by the Child Support (Miscellaneous Amendments) Regulations 2007 reg.~2(6).
}

\subsection[34. Warrant of commitment]{Warrant of commitment}

34.—(1) A warrant of commitment shall be in the form specified in Schedule~3, or in a form to the like effect.

(2) The amount to be included in the warrant under section~40(4)($a$)(ii) of the Act in respect of costs shall be such amount as in the view of the court is equal to the costs reasonably incurred by the Secretary of State in respect of the costs of commitment.

(3) A warrant issued under section~40 of the Act may be executed anywhere in England~and~Wales by any person to whom it is directed or by any constable acting within his police area.

(4) A warrant may be executed by a constable notwithstanding that it is not in his possession at the time but such warrant shall, on the demand~of the person arrested, be shown to him as soon as possible.

(5) Where, after the issue of a warrant, part-payment of the amount stated in it is made, the period of imprisonment shall be reduced proportionately so that for the period of imprisonment specified in the warrant there shall be substituted a period of imprisonment of such number of days as bears the same proportion to the number of days specified in the warrant as the amount remaining unpaid under the warrant bears to the amount specified in the warrant.

(6) Where the part-payment is of such an amount as would, under paragraph~(5), reduce the period of imprisonment to~such number of days as have already been served (or would be so served in the course of the day of payment), the period of imprisonment shall be reduced to the period already served plus one day.

% Reg 35 added (2.4.01) by SI 2001/162 reg 2(6)(c)
\subsection[35. Disqualification from driving order]{Disqualification from driving order}

35.---(1)  For the purposes of enabling an enquiry to be made under section~39A of the Act as to the liable person’s livelihood, means and~conduct, a justice of the peace 
%having jurisdiction for the area in which the liable person resides   % Words omitted (1.8.07) by SI 2007/1979 reg 2(6)
may issue a summons to him to~appear before a magistrates' court and to produce any driving licence held by him, and, where applicable, its counterpart, and, if he does not appear, may issue a warrant for his arrest.

(2) In any proceedings under sections~39A and~40B of the Act, a statement in writing to the effect that wages of any amount have been paid to the liable person during any period, purporting to be signed for or on behalf of his employer, shall be evidence of the facts there stated.

(3) Where an application under section~39A of the Act has been made but no disqualification order is made, the application may be renewed on the ground that the circumstances of the liable person have changed.

(4) A disqualification order shall be in the form prescribed in Schedule~4.

(5) The amount to be included in the disqualification order under section~40B(3)($b$)  of the Act in respect of the costs shall be such amount as in the view of the court is equal to the costs reasonably incurred by the Secretary of State in respect of the costs of the application for the disqualification order.

(6) An order made under section~40B(4) of the Act may be executed anywhere in England~and~Wales by any person to whom it is directed or by any constable acting within his police area, if the liable person fails to~appear or produce or surrender his driving licence or its counterpart to the court.

(7) An order may be executed by a constable notwithstanding that it is not in his possession at the time but such order shall, if demanded, be shown to the liable person as soon as reasonably practicable.

(8) In this regulation~“driving licence” means a licence to~drive a motor vehicle granted under Part III of the Road Traffic Act 1988\footnote{1998 c.\ 52, section~108(1).}.

\amendment{
Reg. 35 added (2.4.01) by the Child Support (Collection and~Enforcement and~Miscellaneous Amendments) Regulations 2000 reg.~2(6)(c).

Words omitted in reg.~35(1) (1.8.07) by the Child Support (Miscellaneous Amendments) Regulations 2007 reg.~2(6).
}

\bigskip

Signed by authority of the Secretary of State for Social Security.

{\raggedleft
\emph{Ann Widdecombe}\\*Parliamentary Under-Secretary of State,\\*Department of Social Security

}

17th August 1992

\small

\part[Schedule~1 --- Liability order prescribed form]{Schedule~1\\*Liability order prescribed form}

\renewcommand\parthead{--- Schedule~1}

\noindent
Section~33 of the Child Support Act 1991 and regulation~29(1) of the Child Support (Collection and~Enforcement) Regulations 1992

\medskip

{\raggedleft \hspace{0.5\linewidth}\dotfill Magistrates' Court

}

\medskip

Date:

\medskip

Defendant:

\medskip

Address:

\medskip

On the complaint of the Secretary of State for Social Security that the sums specified below are due from the defendant under the Child Support Act 1991 and~Part IV of the Child Support (Collection and~Enforcement) Regulations 1992 and~are outstanding, it is adjudged that the defendant is liable to pay the aggregate amount specified below.

\medskip

\noindent
\begin{tabulary}{0.9\linewidth}{lJJ}
Sum payable and~outstanding \hspace{0.075\linewidth} &  --- & child support maintenance\\
&--- & interest\\
&--- & \emph {penalty payments}\\
&--- & \emph {fees}\\
& --- & other periodical payments collected by virtue of section~30 of the Child Support Act 1991\\
\end{tabulary}

\medskip

Aggregate amount in respect of which the liability order is made:

\medskip

{\raggedleft Justice of the Peace

\medskip

[\emph{or} by order of the Court\\*Clerk of the Court]

}

\amendment{
Words inserted in Sch. 1 (3.3.03 for new-rules cases only) by the Child Support (Collection and~Enforcement and~Miscellaneous Amendments) Regulations 2000 reg.~2(7) (subject to~savings in reg.~6).

The words in italics do not apply to 1993 scheme cases.
}

\part[Schedule~2 --- Charges connected with distress]{Schedule~2\\*Charges connected with distress}

\renewcommand\parthead{--- Schedule~2}

1.  The sum in respect of charges connected with the distress which may be aggregated under section~35(2)($b$) of the Act shall be set out in the following Table—

{\noindent\footnotesize
\begin{longtable}{p{183pt}p{183pt}}
%\begin{tabulary}{\linewidth}{JJ}
\hline
(1)&(2)\\
\itshape Matter connected with distress&\itshape Charge\\
\hline
\endhead
\hline
\endlastfoot
A {} For making a visit to premises with a view to levying distress (whether the levy is made or not):&
Reasonable costs and~fees incurred, but not exceeding an amount which, when aggregated with charges under this head for any previous visits made with a view to levying distress in relation to~an amount in respect of which the liability order concerned was made, is not greater than the relevant amount calculated under paragraph~2(1) with respect to the visit.\\
B {} For levying distress:&
An amount (if any) which, when aggregated with charges under head A for any visits made with a view to levying distress in relation to~an amount in respect of which the liability order concerned was made, is equal to the relevant amount calculated under paragraph~2(1) with respect to the levy.\\
%Head BB inserted (7.2.94) by SI 1994/227 reg 3(2)($a$)
BB {} For preparing and sending a letter advising the liable person that the written authorisation of the Secretary of State is with the person levying the distress and requesting the total sum due: & £10$.$00.\\
C {} For the removal and storage of goods for the purposes of sale:&
Reasonable costs and~fees incurred.\\
D {}  For the possession of goods as described in paragraph~2(3)—\\
\hspace{12pt}(i) for close possession (the person in\linebreak\hspace*{12pt}possession on behalf of the Secretary of\linebreak\hspace*{12pt}State to provide his own board):&
£4$.$50 per day.\\
\hspace{12pt}(ii) for walking possession:&
%45p per day.\\
10p per day.\\ % Words in head D(ii) substituted (7.2.94) by SI 1994/227 reg 3(2)($b$)
E {} For appraisement of an item distrained, at the request in writing of the liable person:&
Reasonable fees and~expenses of the broker appraising.\\
F {} For other expenses of, and~commission on, a sale by auction—\\
\hspace{12pt}\textls[25]{(i) where the sale is held on the}\linebreak\hspace*{12pt}auctioneer’s premises:&
\textls[25]{The auctioneer’s commission fee and} out-of-pocket expenses (but not exceeding in aggregate 15 per cent.\ of the sum realised), together with reasonable costs and~fees incurred in respect of advertising.\\
\hspace{12pt}(ii) where the sale is held on the liable\linebreak\hspace*{12pt}person’s premises:&
The auctioneer’s commission fee (but not exceeding 7\textonehalf{} per cent.\ of the sum realised), together with the auctioneer’s out-of-pocket expenses and reasonable costs and~fees incurred in respect of advertising.\\
G {} For other expenses incurred in connection with a proposed sale where there is no buyer in relation to it:&
Reasonable costs and~fees incurred.\\
\end{longtable}
%\end{tabulary}

}

\amendment{
Words substituted in Head D(ii) of the Table and~Head BB inserted into the Table in para.~1 (7.2.94) by the Child Support (Miscellaneous Amendments and~Transitional Provisions) Regulations 1994 reg.~3(2) (subject to transitional provisions in reg.~12).
}

\medskip

2.—(1) In heads A and~B of the Table to paragraph~1, “the relevant amount” with respect to~a visit or a levy means—
\begin{enumerate}\item[]
($a$) where the sum due at the time of the visit or of the levy (as the case may be) does not exceed £100, £12$.$50;

($b$) where the sum due at the time of the visit or of the levy (as the case may be) exceeds £100, 12\textonehalf{} per cent.\ on the first £100 of the sum due, 4 per cent.\ on the next £400, 2\textonehalf{} per cent.\ on the next £1,500, 1 per cent.\ on the next £8,000 and~\textonequarter{} per cent. on any additional sum;
\end{enumerate}
and the sum due at any time for these purposes means so much of the amount in respect of which the liability order concerned was made as is outstanding at the time.

(2) Where a charge has arisen under head B with respect to~an amount, no further charge may be aggregated under heads A or B in respect of that amount.

(3) The Secretary of State takes close or walking possession of goods for the purposes of head D of the Table to paragraph~1 if he takes such possession in pursuance of an agreement which is made at the time that the distress is levied and which (without prejudice to~such other terms as may be agreed) is expressed to the effect that, in consideration of the Secretary of State not immediately removing the goods distrained upon from the premises occupied by the liable person and~delaying the sale of the goods, the Secretary of State may remove and sell the goods after a later specified date if the liable person has not by then paid the amount distrained for (including charges under this Schedule); and the Secretary of State is in close possession of goods on any day for these purposes if during the greater part of the day a person is left on the premises in physical possession of the goods on behalf of the Secretary of State under such an agreement.

\medskip

3.—(1) Where the calculation under this Schedule~of a percentage of a sum results in an amount containing a fraction of a pound, that fraction shall be reckoned as a whole pound.

(2) In the case of dispute as to~any charge under this Schedule, the amount of the charge shall be taxed.

(3) Such a taxation shall be carried out by the district judge of the county court for the district in which the distress is or is intended to be levied, and~he may give such directions as to the costs of the taxation as he thinks fit; and~any such costs directed to be paid by the liable person to the Secretary of State shall be added to the sum which may be aggregated under section~35(2) of the Act.

(4) References in the Table in paragraph~1 to costs, fees and~expenses include references to~amounts payable by way of value added tax with respect to the supply of goods or services to which the costs, fees and~expenses relate.

\part[Schedule~3 --- Form of warrant of commitment --- \emph{1993 scheme version}]{Schedule~3\\*Form of warrant of commitment\\*\emph{1993 scheme version}}

\renewcommand\parthead{--- Schedule~3}

\noindent
Section~40 of the Child Support Act 1991 and regulation~34(1) of the Child Support (Collection and~Enforcement) Regulations 1992

\medskip

{\raggedleft \hspace{0.5\linewidth}\dotfill Magistrates' Court

}

\medskip

Date:

\medskip

Liable Person:

\medskip

Address:

\medskip

A liability order (``the order'') was made against the liable person by the [\phantom{Bolton}] Magistrates' Court on [\phantom{\today}] under section~33 of the Child Support Act 1991 (``the Act'') in respect of an amount of [\phantom{£100.00}].

The court is satisfied---
\begin{enumerate}
\item[]
(i) that the Secretary of state sought under section~35 of the Act to levy by distress the amount then outstanding in respect of which the order was made;
\end{enumerate}
[and/or]
\begin{enumerate}\item[]
that the Secretary of State sought under section~36 of the Act to recover through the [\phantom{Bolton}] County Court, by means of [garnishee proceedings] or [a charging order], the amount then outstanding in respect of which the order was made;

(ii) that such amount, or any portion of it, remains unpaid; and

(iii) having inquired in the liable person's presence as to his means and~as to whether there was been [wilful refusal] or [culpable neglect] on his part, the court is of the opinion that there has been [wilful refusal] or [culpable neglect] on his part.
\end{enumerate}

The decision of the Court is that the liable person be [committed to prison] [detained] for [\phantom{7 days}] unless the aggregate amount mentioned below in respect of which this warrant is made is sooner paid.*

\medskip

This warrant is made in respect of---

Amount outstanding (including any interest, costs and~charges):

Costs of commitment of the Secretary of State:

\medskip

Aggregate amount:

\medskip

And you [\emph{name of person or persons to whom warrant is directed}] are hereby required to take the liable person and~convey him to~[\emph{name of prison or place of detention}] and there deliver him to the [governor] [officer in charge] thereof; and~you, the [governor] [officer in charge], to receive the liable person into~your custody and~keep him for [\emph{period of imprisonment}] from the date of his arrest under this warrant or until he be sooner discharged in due course of law.

\medskip

{\raggedleft Justice of the Peace

\medskip

[\emph{or} by order of the Court\\*Clerk of the Court]

}

\medskip

*\emph{Note:} The period of imprisonment will be reduced as provided by regulation~34(5) and~(6) of the Child Support (Collection and~Enforcement) Regulations 1992 if part-payment is made of the aggregate amount.

\part[Schedule~3 --- Form of warrant of commitment --- \emph{2003 scheme version}]{Schedule~3\\*Form of warrant of commitment\\*\emph{2003 scheme version}}

\noindent
Section~40 of the Child Support Act 1991 and regulation~34(1) of the Child Support (Collection and~Enforcement) Regulations 1992

\medskip

{\raggedleft \hspace{0.5\linewidth}\dotfill Magistrates' Court

}

\medskip

Date:

\medskip

Liable Person:

\medskip

Address:

\medskip

A liability order (``the order'') was made against the liable person by the [\phantom{Bolton}] Magistrates' Court on [\phantom{\today}] under section~33 of the Child Support Act 1991 (``the Act'') in respect of an amount of [\phantom{£100.00}].

The court is satisfied---
\begin{enumerate}
\item[]
(i) that the Secretary of state sought under section~35 of the Act to levy by distress the amount then outstanding in respect of which the order was made;
\end{enumerate}
[and/or]
\begin{enumerate}\item[]
that the Secretary of State sought under section~36 of the Act to recover through the [\phantom{Bolton}] County Court, by means of [garnishee proceedings] 
%or  % Word omitted (3.3.03) by SI 2001/162 reg 2(8)(b)
[a charging order], the amount then outstanding in respect of which the order was made;

(ii) that such amount, or any portion of it, remains unpaid; and

(iii) having inquired in the liable person's presence as to his means and~as to whether there was been [wilful refusal] 
%or  % Word omitted (3.3.03) by SI 2001/162 reg 2(8)(c)
[culpable neglect] on his part, the court is of the opinion that there has been [wilful refusal] 
%or  % Word omitted (3.3.03) by SI 2001/162 reg 2(8)(c)
[culpable neglect] on his part.
\end{enumerate}

The decision of the Court is that the liable person be [committed to prison] [detained] for [\phantom{7 days}] unless the aggregate amount mentioned below in respect of which this warrant is made is sooner paid.*

\medskip

This warrant is made in respect of---

Amount outstanding (including any interest, 
penalty payments, fees,  % Words inserted (3.3.03) by SI 2001/162 reg 2(8)(a)
costs and~charges):

Costs of commitment of the Secretary of State:

\medskip

Aggregate amount:

\medskip

And you [\emph{name of person or persons to whom warrant is directed}] are hereby required to take the liable person and~convey him to~[\emph{name of prison or place of detention}] and there deliver him to the [governor] [officer in charge] thereof; and~you, the [governor] [officer in charge], to receive the liable person into~your custody and~keep him for [\emph{period of imprisonment}] from the date of his arrest under this warrant or until he be sooner discharged in due course of law.

\medskip

{\raggedleft Justice of the Peace

\medskip

[\emph{or} by order of the Court\\*Clerk of the Court]

}

\medskip

*\emph{Note:} The period of imprisonment will be reduced as provided by regulation~34(5) and~(6) of the Child Support (Collection and~Enforcement) Regulations 1992 if part-payment is made of the aggregate amount.

\amendment{
Words inserted and~omitted in Sch. 3 (3.3.03 for new-rules cases only) by the Child Support (Collection and~Enforcement and~Miscellaneous Amendments) Regulations 2000 reg.~2(8) (subject to~savings in reg.~6).
}

\part[Schedule~4 --- Form of order of disqualification from holding or obtaining a driving licence]{Schedule~4\\*Form of order of disqualification from holding or obtaining a driving licence}

\renewcommand\parthead{--- Schedule~4}

\amendment{
Sch. 4 added (2.4.01) by the Child Support (Collection and~Enforcement and~Miscellaneous Amendments) Regulations 2000 reg.~2(9) and~Sch.
}

\medskip

\noindent
Sections 39A and~40B of the Child Support Act 1991 and regulation~35 of the Child Support (Collection and~Enforcement) Regulations 1992

\medskip

{\raggedleft \hspace{0.5\linewidth}\dotfill Magistrates' Court

}

\medskip

Date:

\medskip

Liable Person:

\medskip

Address:

\medskip

A liability order (``the order'') was made against the liable person by the [\phantom{Bolton}] Magistrates' Court on [\phantom{\today}] under section~33 of the Child Support Act 1991 (``the Act'') in respect of an amount of [\phantom{£100.00}].

The court is satisfied---
\begin{enumerate}
\item[]
(i) that the Secretary of state sought under section~35 of the Act to levy by distress the amount then outstanding in respect of which the order was made;

[and/or]

that the Secretary of State sought under section~36 of the Act to recover through [\phantom{Bolton}] County Court, by means of [garnishee proceedings] [a charging order], the amount then outstanding in respect of which the order was made;

(ii) that such amount, or any portion of it, remains unpaid; and

(iii) having inquired in the liable person's presence as to his means and~as to whether there was been [wilful refusal] or [culpable neglect] on his part.
\end{enumerate}

The decision of the Court is that the liable person be disqualified from [holding or obtaining] a driving licence from [date] for [period] unless the aggregate amount in respect of which this order is made is sooner paid.*

\medskip

This order is made in respect of---

Amount outstanding (including any interest, fees, penalty payments, costs and~charges):

\medskip

Aggregate amount:

\medskip

And you [the liable person] shall surrender to the court any driving licence and~counterpart held.

\medskip

{\raggedleft Justice of the Peace

\medskip

[\emph{or} by order of the Court\\*Clerk of the Court]

}

\medskip

*\emph{Note:} The period of disqualification may be reduced as provided by section~40B(5)($a$) of the Act if part payment is made of the aggregate amount.  The order will be revoked by section~40B(5)($b$) of the Act if full payment is made of the aggregate amount.

\part{Explanatory Note}

\renewcommand\parthead{--- Explanatory Note}

\subsection*{(This note is not part of the Regulations)}

 These Regulations make provision in relation to the collection and~enforcement of child support maintenance under the Child Support Act 1991.

  Part I contains interpretation provisions and~provisions relating to the service and receipt of notices and~other documents.

  Part II (regulations 2 to~7) deals with the collection of child support maintenance and, in particular, makes provision in relation to the method and~interval of payment and the notification of such matters to the liable person.

  Part III (regulations 8 to~25) makes provision in relation to~deduction from earnings orders under which payments in respect of child support maintenance are to be deducted by employers from the earnings of liable persons. Regulation 25 creates offences where certain of the requirements are contravened.

  Part IV (regulations 26 to~34) and~Schedules 1 to~3 make provision in relation to the obtaining of liability orders in the magistrates' courts and~in relation to the enforcement of those orders by distraining the liable person’s goods or by commitment to prison.

\end{document}