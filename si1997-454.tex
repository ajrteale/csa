\documentclass[12pt,a4paper]{article}

\newcommand\regstitle{The Social Security (Miscellaneous Amendments) Regulations 1997}

\newcommand\regsnumber{1997/454}

%\opt{newrules}{
\title{\regstitle}
%}

%\opt{2012rules}{
%\title{Child Maintenance and Other Payments Act 2008\\(2012 scheme version)}
%}

\author{S.I. 1997 No. 454}

\date{Made 24th February 1997\\Laid before Parliament 28th February 1997\\Coming into force \\for the purposes of regulations 1, 4 and 5 21st March 1997\\for the purposes of regulation 8 6th April 1997\\for all other purposes 7th April 1997
}

%\opt{oldrules}{\newcommand\versionyear{1993}}
%\opt{newrules}{\newcommand\versionyear{2003}}
%\opt{2012rules}{\newcommand\versionyear{2012}}

\usepackage{csa-regs}

\setlength\headheight{27.57402pt}

\begin{document}

\maketitle

\noindent
The Secretary of State for Education and Employment, in relation to regulation 2(3), (6) and (7) of these Regulations and the Secretary of State for Social Security, in relation to the remainder of these Regulations, in exercise of the powers conferred by sections 4(1)($b$)  and (5), 12(1) and (2), 19(8)($b$)(i), 26(4)($c$), ($d$), ($i$) and ($l$), 35(1), 36(1) to (4) and 40 of, and paragraphs 1(1) and 3 of Schedule 1 to, the Jobseekers Act 1995\footnote{\frenchspacing 1995 c. 18; section 35(1) is an interpretation provision and is cited because of the meanings ascribed to the words “prescribed”, “regulations” and “week”.}, sections 123(1)($a$), 136(3), 137(1) and 175(1) and (3) of the Social Security Contributions and Benefits Act 1992\footnote{\frenchspacing 1992 c. 4; section 137(1) is cited because of the meaning ascribed to the word “prescribed”.}, sections 22(4), 61 and 189(1) and (5) of the Social Security Administration Act 1992\footnote{\frenchspacing 1992 c. 5.} and sections 10 and 26(1) to (3) of the Child Support Act 1995\footnote{\frenchspacing 1995 c. 34.} and of all other powers enabling each of them in that behalf, after agreement by the Social Security Advisory Committee that proposals in respect of regulations 2 to 7 and 9 of these Regulations should not be referred to it\footnote{\frenchspacing \emph{See} the Social Security Administration Act 1992 (c. 5), sections 170 and 173(1)($b$) and the Jobseekers Act 1995, Schedule 2, paragraph 67($a$).} and whereas this instrument is made, for the purposes of regulation 8, before the end of the period of 6 months beginning with the coming into force of section 10 of the Child Support Act 1995\footnote{\frenchspacing \emph{See} section 173(5)($a$) of the Social Security Administration Act 1992 (c. 5).}, hereby make the following Regulations: 


{\sloppy

\tableofcontents

}

\setcounter{secnumdepth}{-2}

\subsection[1. Citation and commencement]{Citation and commencement}

1.  These Regulations may be cited as the Social Security (Miscellaneous Amendments) Regulations 1997 and shall come into force—
\begin{enumerate}\item[]
($a$) for the purposes of this regulation and of regulations 4 and 5, on 21st March 1997;

($b$) for the purposes of regulation 8, on 6th April 1997; and

($c$) for all other purposes, on 7th April 1997.
\end{enumerate}

\subsection[2. Amendment of the Jobseeker’s Allowance Regulations 1996]{Amendment of the Jobseeker’s Allowance Regulations 1996}

2.---(1)  The Jobseeker’s Allowance Regulations 1996\footnote{\frenchspacing S.I. 1996/207.} shall be amended in accordance with the following paragraphs of this regulation.

(2) In paragraph (3) of regulation 1 (interpretation), in the definition of “week”\footnote{\frenchspacing The definition of “week” was amended by S.I. 1996/1517.} after the word “in” there shall be inserted the words “the definitions of “full-time course of advanced education” and of “full-time student” and”.

(3) In paragraph ($b$)(ii)  of regulation 42 (appeals to the Social Security Appeal Tribunal) for “(S.I.\ 1996/)” there shall be substituted “(S.I.\ 1996/207)”.

(4) In paragraph (2) of regulation 48 (linking periods), in sub-paragraph ($e$)\footnote{\frenchspacing Regulation 48(2)($e$) was inserted by S.I. 1996/2538.}, after the words “6th October” there shall be inserted the figure “1996”.

(5) In paragraph (1) of regulation 51 (remunerative work), for the words “work is work” there shall be substituted the words ““work” is work”.

(6) In regulation 72 (good cause for the purposes of section 19(5)($a$)  and (6)($c$)  and ($d$))—
\begin{enumerate}\item[]
($a$) in paragraph (5), sub-paragraph ($c$)  shall be omitted;

($b$) after paragraph (5) there shall be inserted the following paragraph—
\begin{quotation}
“(5A) A person is to be regarded as having good cause for any act or omission for the purposes of section 19(6)($c$)  and ($d$)  if—
\begin{enumerate}\item[]
($a$) in a case where it has been agreed that the claimant may restrict his hours of availability to less than 24 hours a week, the employment in question is for less than 16 hours a week; or

($b$) in a case not falling within sub-paragraph ($a$), the employment is for less than 24 hours a week.”.
\end{enumerate}
\end{quotation}
\end{enumerate}

(7) In regulation 75 (interpretation for the purposes of section 19 and Part V)—
\begin{enumerate}\item[]
($a$) in paragraph (2), after the words “regulation 69” there shall be inserted the words “and the first occasion on which the word occurs in regulation 72(5A)($a$)\footnote{\frenchspacing Regulation 72(5A) was inserted by S.I. 1997/454.}”;

($b$) in paragraph (3), for the words “and regulation 69” there shall be substituted the words “, regulation 69 and the first occasion on which the word occurs in regulation 72(5A)($a$)”.
\end{enumerate}

(8) In paragraph (1) of regulation 80 (deductions in respect of earnings), for the word “week” there shall be substituted the words “benefit week”.

(9) In paragraph (2) of regulation 81\footnote{\frenchspacing Regulation 81(2) was amended by S.I. 1996/1517.} (payments by way of pensions), for sub-paragraph ($c$), there shall be substituted the following sub-paragraph—
\begin{quotation}
“($c$) any payments from a personal pension scheme, an occupational pension scheme or a public service pension scheme which are payable to him and which arose in accordance with the terms of such a scheme on the death of a person who was a member of the scheme in question.”.
\end{quotation}

(10) In paragraph (1) of regulation 85 (applicable amounts in special cases), for the words “capital is calculated” there shall be substituted the words “capital, if calculated”.

(11) In paragraph (1) of regulation 97\footnote{\frenchspacing Regulation 97 was amended by S.I. 1997/65.} (calculation of weekly amount of income), for the word “(6)”, there shall be substituted the word “(7)”.

(12) In paragraph (2) of regulation 98\footnote{\frenchspacing Regulation 98 was amended by S.I. 1996/1517.} (earnings of employed earners), after sub-paragraph ($f$)  there shall be added the following sub-paragraph—
\begin{quotation}
“($g$) any lump sum payment made under the Iron and Steel Re-adaptation Benefits Scheme\footnote{\frenchspacing The Scheme is set out in regulation 4 of, and the Schedule to, the European Communities (Iron and Steel Employees Re-adaptation Benefits Scheme) (No. 2) (Amendment) Regulations 1996 (S.I. 1996/3182).}”.
\end{quotation}

(13) In paragraph (5) of regulation 103 (calculation of income other than earnings), for the word “week” there shall be substituted the words “benefit week”.

(14) In paragraph ($e$)  of Schedule 3 (the Greater London area), for the word “Old” there shall be substituted the word “old”.

(15) In paragraph 11 of Schedule 5 (applicable amount for couple or member of couple taking child or young person abroad for treatment), in sub-paragraph (2)($a$)  of Column (1), for the words “available for an actively seeking” there shall be substituted the words “available for and actively seeking”.

(16) In paragraph 38 of Schedule 7 (disregard of sums in the calculation of income other than earnings), for the word “week” there shall be substituted the words “benefit week”.

\subsection[3. Amendment of the Social Security (Back to Work Bonus) (No.\ 2) Regulations 1996]{Amendment of the Social Security (Back to Work Bonus) (No.\ 2) Regulations 1996}

3.---(1)  The Social Security (Back to Work Bonus) (No.\ 2) Regulations 1996\footnote{\frenchspacing S.I. 1996/2570.} shall be amended in accordance with the following paragraphs of this regulation.

(2) In regulation 5 (periods of entitlement which do not qualify)—
\begin{enumerate}\item[]
($a$) in paragraph (1), for the words “paragraph (3)” there shall be substituted the words “paragraph (4)”;

($b$) in paragraph (4)—
\begin{enumerate}\item[]
(i) in sub-paragraph ($b$), for the words “paragraph (4)” there shall be substituted the words “paragraph (5)”;

(ii) in sub-paragraph ($c$)—
\begin{enumerate}\item[]
($aa$) after the word “(hardship)” there shall be inserted the word “because”,

($bb$) in head (ii)  for the words “paragraph (5)” there shall be substituted the words “paragraph (6)”;
\end{enumerate}
\end{enumerate}

($c$) in paragraph (5), for the words “Paragraph (3)($b$)” there shall be substituted the words “Paragraph (4)($b$)”.
\end{enumerate}

(3) In paragraph (4) of regulation 13 (single persons who become couples) the words “to the members of the polygamous marriage and references to each member of the couple were references” shall be omitted.

(4) In paragraph (3) of regulation 18 (death) after the words “regulation 6” there shall be inserted the words “where that partner becomes entitled to a qualifying benefit no more than 12 weeks after the date of death”.

(5) In paragraph (6) of regulation 19 (trade disputes) for the words “the partner’s earnings” there shall be substituted the words “that partner’s or that person’s earnings”.

\subsection[4. Amendment of the Jobseeker’s Allowance (Transitional Provisions) Regulations 1996]{Amendment of the Jobseeker’s Allowance (Transitional Provisions) Regulations 1996}

4.---(1)  The Jobseeker’s Allowance (Transitional Provisions) Regulations 1996\footnote{\frenchspacing S.I. 1996/2567.} shall be amended in accordance with the following paragraphs of this regulation.

(2) In paragraph (2) of regulation 3 (linking periods), in sub-paragraph ($a$), for the words “paragraph (3)” there shall be substituted the words “paragraph (5)”.

(3) In paragraph (6) of regulation 8 (claims for entitlement to a jobseeker’s allowance)—
\begin{enumerate}\item[]
($a$) for the words “paragraph (3)” there shall be substituted the words “paragraph (4)”;

($b$) after the word “applies” there shall be inserted the words “or 156 days in a case to which paragraph (5) applies”.
\end{enumerate}

(4) In paragraph (3) of regulation 9 (further provisions applying to a continuing entitlement to a jobseeker’s allowance), in sub-paragraph ($b$)—
\begin{enumerate}\item[]
($a$) at the end of head (i)  there shall be inserted the word “or”;

($b$) in head (ii), after the words “that day” there shall be inserted the words “is after the last day of the transitionally protected period and”;

($c$) at the end of head (ii), “; or” shall be omitted;

($d$) head (iii) shall be omitted.
\end{enumerate}

(5) In paragraph (7) of regulation 10 (transitionally protected period), after the words “paragraph 1 of” there shall be inserted the words “Part I of”.

\subsection[5. Amendment of the Social Security (Graduated Retirement Benefit) (No.\ 2) Regulations 1978]{Amendment of the Social Security (Graduated Retirement Benefit) (No.\ 2) Regulations 1978}

5.  In section 36(9) of the National Insurance Act 1965\footnote{\frenchspacing 1965 c. 51.}, as continued in force by paragraph (3) of regulation 3 of, and Schedule 1 to, the Social Security (Graduated Retirement Benefit) (No.\ 2) Regulations 1978\footnote{\frenchspacing S.I. 1978/393; the relevant amending instrument is S.I. 1996/1345.}, for the words “regulation 9(5) of the Jobseeker’s Allowance (Transitional Provisions) Regulations 1995” there shall be substituted the words “regulation 10(6) of the Jobseeker’s Allowance (Transitional Provisions) Regulations 1996\footnote{\frenchspacing S.I. 1996/2567.}”.

\subsection[6. Amendment of the Social Security (General Benefit) Regulations 1982]{Amendment of the Social Security (General Benefit) Regulations 1982}

6.  In regulation 9 of the Social Security (General Benefit) Regulations 1982\footnote{\frenchspacing S.I. 1982/1408; the relevant amending instrument is S.I. 1984/1259.} (payment of benefit and suspension of payments pending a decision on appeals or references, arrears and repayments), paragraphs (5), (6) and (6A) shall be omitted.

\subsection[7. Amendment of the Income Support (General) Regulations 1987]{Amendment of the Income Support (General) Regulations 1987}

7.  In regulation 35 of the Income Support (General) Regulations 1987\footnote{\frenchspacing S.I. 1987/1967; the relevant amending instruments are S.I. 1989/1323, 1990/774 and 1993/2119.} (earnings of employed earners)—
\begin{enumerate}\item[]
($a$) in paragraph (2), after sub-paragraph ($d$)  there shall be added the following sub-paragraph—
\begin{quotation}
“($e$) any lump sum payment made under the Iron and Steel Re-adaptation Benefits Scheme\footnote{\frenchspacing The Scheme is set out in regulation 4 of, and the Schedule to, the European Communities (Iron and Steel Employees Re-adaptation Benefits Scheme) (No. 2) (Amendment) Regulations 1996 (S.I. 1996/3182).}”;
\end{quotation}

($b$) in paragraph (3), in sub-paragraph ($a$)(ii), for the word “($d$)” there shall be substituted the word “($e$)”.
\end{enumerate}

% Reg 8 revoked (3.3.03) by SI 2000/3176 reg 4(1)(c)
%\subsection[8. Amendment of the Social Security (Child Maintenance Bonus) Regulations 1996]{\sloppy Amendment of the Social Security (Child Maintenance Bonus) Regulations 1996}
%
%8.---(1)  The Social Security (Child Maintenance Bonus) Regulations 1996\footnote{\frenchspacing S.I. 1996/3195.} shall be amended in accordance with the following paragraphs of this regulation.
%
%(2) In regulation 1 (citation, commencement and interpretation)—
%\begin{enumerate}\item[]
%($a$) in paragraph (2), in the definition of “child maintenance” in sub-paragraph ($b$), for the words “to a parent with care” there shall be substituted the words “to a person with care”; and
%
%($b$) in paragraph (7) after the words “these Regulations” there shall be inserted the words “other than regulation 4 (bonus period)”.
%\end{enumerate}
%
%(3) In regulation 3 (entitlement to a bonus), after paragraph (1) there shall be inserted the following paragraph—
%\begin{quotation}
%“(1A) In the case of an applicant who satisfies the requirements of paragraph (1)($f$)  but whose entitlement, or whose partner’s entitlement, to a qualifying benefit ceased otherwise than as a result of satisfying the work condition, for sub-paragraph ($d$)  of paragraph (1) there shall be substituted the following sub-paragraph—
%\begin{quotation}
%“($d$) had the work condition been satisfied on the day she, or her partner, was last entitled to a qualifying benefit, that entitlement would as a consequence have ceased.””.
%\end{quotation}
%\end{quotation}
%
%(4) In regulation 4 (bonus period)—
%\begin{enumerate}\item[]
%($a$) in paragraph (1), after the words “7th April 1997” there shall be inserted the words “, other than days to which paragraph (9) applies,” and in sub-paragraph ($c$), for head (i)  there shall be substituted the following head—
%\begin{quotation}
%“(i) taken into account in determining, for the purposes of the qualifying benefit, the amount of the claimant’s income;”.
%\end{quotation}
%
%($b$) in paragraph (3), in sub-paragraph ($b$), head (i)  shall be omitted;
%
%($c$) in paragraph (7), for sub-paragraph ($b$), there shall be substituted the following sub-paragraph—
%\begin{quotation}
%“($b$) on the date of death of a person with care of a qualifying child to whom child maintenance is payable.”; and
%\end{quotation}
%
%($d$) after paragraph (7), there shall be added the following paragraphs—
%\begin{quotation}
%“(8) In paragraphs (1)($c$)(i)  and (9) “claimant”—
%\begin{enumerate}\item[]
%($a$) where the qualifying benefit is income support, means a person who claims income support; and
%
%($b$) where the qualifying benefit is a jobseeker’s allowance, means a person who claims a jobseeker’s allowance.
%\end{enumerate}
%
%\begin{sloppypar}
%(9) This paragraph applies to days on which the claimant is a person to whom—
%\end{sloppypar}
%\begin{enumerate}\item[]
%($a$) regulation 70 of the Income Support (General) Regulations 1987\footnote{\frenchspacing S.I. 1987/1967.} (urgent cases) applies other than by virtue of paragraph (2)($a$)  of that regulation (certain persons from abroad), or
%
%($b$) regulation 147 of the Jobseeker’s Allowance Regulations 1996\footnote{\frenchspacing S.I. 1996/207.} applies other than by virtue of paragraph (2)($a$)  of that regulation.”.
%\end{enumerate}
%\end{quotation}
%\end{enumerate}
%
%(5) In regulation 5 (amount payable) paragraphs (2) and (4) shall be omitted.
%
%(6) In regulation 6 (Secretary of State estimates) in paragraph (1), for the words “who is in receipt of a qualifying benefit” there shall be substituted the words “with care”, and for the words “the person in receipt of the qualifying benefit”, there shall be substituted the words “that person”.
%
%(7) In regulation 7 (death of a person with care of a child), in paragraph (1), in sub-paragraph ($a$), for the words “child in respect of whom child maintenance is payable dies;” there shall be substituted the words “qualifying child to whom child maintenance is payable dies;”.
%
%(8) In regulation 10 (claiming a bonus), in paragraph (1)—
%\begin{enumerate}\item[]
%($a$) in sub-paragraph ($b$)  the words “or entitlement to the qualifying benefit” shall be omitted;
%
%($b$) for sub-paragraph ($c$)  there shall be substituted the following sub-paragraph—
%\begin{quotation}
%“($c$) where—
%\begin{enumerate}\item[]
%(i) a person with care cares for one child only; and
%
%(ii) that child dies,
%\end{enumerate}
%in the period not exceeding 12 months which begins on the date the child died and throughout which she, or where she has a partner, her partner is entitled to a qualifying benefit; or”.
%\end{quotation}
%\end{enumerate}
%
%(9) In regulation 15 (capital to be disregarded)—
%\begin{enumerate}\item[]
%($a$) in paragraph ($a$)  for the words “paragraph 48” there shall be substituted the words “paragraph 50”;
%
%($b$) in paragraph ($b$)  for the words “paragraph 49” there shall be substituted the words “paragraph 51”;
%
%($c$) in paragraph ($c$)  for the words “paragraph 50” there shall be substituted the words “paragraph 52”; and
%
%($d$) in paragraph ($d$)  for the words “paragraph 50” there shall be substituted the words “paragraph 52”.
%\end{enumerate}
%
%(10) In regulation 16 (no deduction from bonus), which adds words to and inserts a regulation in other Regulations, where the words added or the regulation inserted include the words “Child Maintenance Bonus” for those words there shall be substituted the words “Social Security (Child Maintenance Bonus)”.

\amendment{
Reg. 8 revoked (3.3.03) by the Social Security (Child Maintenance Premium and Miscellaneous Amendments) Regulations 2000 reg. 4(1)(c).
}

\subsection[9. Revocations]{Revocations}

9.  The Social Security (General Benefit) Amendment Regulations 1984\footnote{\frenchspacing S.I. 1984/1259.} and regulation 2(13)($b$)(iii) of the Social Security and Child Support (Jobseeker’s Allowance) (Miscellaneous Amendments) Regulations 1996\footnote{\frenchspacing S.I. 1996/2538.} are hereby revoked. 

\bigskip

Signed in connection with regulation 2(3), (6) and (7) of these Regulations by authority of the Secretary of State for Education and Employment.

{\raggedleft
\emph{Eric Forth}\\*Minister of
State,\\*Department for Education and Employment

}

24th February 1997

\bigskip

Signed in connection with the remainder of these Regulations by authority of the Secretary of State for Social Security.

{\raggedleft
\emph{Roger Evans}\\*Parliamentary Under-Secretary of
State,\\*Department of Social Security

}

24th February 1997

\small

\part{Explanatory Note}

\renewcommand\parthead{--- Explanatory Note}

\subsection*{(This note is not part of the Regulations)}

These Regulations primarily amend the Jobseeker’s Allowance Regulations 1996 (S.I.\ 1996/207), the Social Security (Back to Work Bonus) (No.\ 2) Regulations 1996 (S.I.\ 1996/2570) and the Jobseeker’s Allowance (Transitional Provisions) Regulations 1996 (S.I.\ 1996/2567) by both correcting a number of technical errors (regulations 2, 3 and 4) and making certain other amendments. In particular, those amendments—
\begin{itemize}
\item    clarify what constitutes good cause for the purpose of any sanction imposed in accordance with Section 19(6)($c$)  and ($d$)  of the Jobseekers Act 1995 (c.\ 18) (regulation 2(6) and (7));

\item    allow any pension payments made to a jobseeker as a result of the death of a person who was a member of a pension scheme, to be disregarded in the calculation of a contribution-based jobseeker’s allowance (regulation 2(9));

\item    provide that payments under the Iron and Steel Employees Re-adaptation Benefits Scheme should not be treated as earnings or as a compensation payment for jobseeker’s allowance purposes (regulation 2(12));

\item    provide that an income support claimant’s earnings should count towards a Back to Work Bonus if they are involved in a trade dispute and become incapable of work or enter their maternity pay period (regulation 3(5));

\item    clarify the formula whereby days of unemployment benefit count as days of entitlement to a contribution-based jobseeker’s allowance at the end of a transitionally protected period; and provide that the formula should not be used where a new award of a contribution-based jobseeker’s allowance which is transitionally protected is based on the same tax years as a previous non-linking award of unemployment benefit (regulation 4(4)). 
\end{itemize}

Regulation 5 amends a technical error in the Social Security (Graduated Retirement Benefit) (No.\ 2) Regulations 1978 (S.I.\ 1978/393) which also relates to jobseeker’s allowance paid during the transitional period.

Regulation 6 amends the Social Security (General Benefit) Regulations 1982 (S.I.\ 1982/1408) by omitting certain provisions which are now rendered otiose with the introduction of contribution-based jobseeker’s allowance.

Regulation 7 amends the Income Support (General) Regulations 1987 (S.I.\ 1987/1967) by providing similar amendments in relation to income support to those made in relation to jobseeker’s allowance by regulation 2(12) of these Regulations.

The amendments in regulation 8 to the Social Security (Child Maintenance Bonus) Regulations 1996 (S.I.\ 1996/3195), are made under section 10 of the Child Support Act 1995 (c.\ 34) and are made before the end of the period of 6 months beginning with the coming into force of that section. Regulation 8 is therefore exempt from the requirement in section 172(1) of the Social Security Administration Act 1992 (c.\ 5) to refer proposals to make the regulation to the Social Security Advisory Committee and is made without reference to that Committee. This regulation makes a number of minor textual amendments to those Regulations.

Regulation 9 revokes the Social Security (General Benefit) Amendment Regulations 1984 (S.I.\ 1984/1259), together with a provision in the Social Security and Child Support (Jobseeker’s Allowance) (Miscellaneous Amendments) Regulations 1996 (S.I.\ 1996/2538) which had been inserted erroneously.

These Regulations do not impose a charge on businesses. 

\end{document}