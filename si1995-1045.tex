\documentclass[12pt,a4paper]{article}

\newcommand\regstitle{The Child Support and Income Support (Amendment) Regulations 1995}

\newcommand\regsnumber{1995/1045}

%\opt{newrules}{
\title{\regstitle}
%}

%\opt{2012rules}{
%\title{Child Maintenance and Other Payments Act 2008\\(2012 scheme version)}
%}

\author{S.I. 1995 No. 1045}

\date{Made 10th January 1995\\Coming into force\\Regulations 1 and 58 13th April 1995\\Remainder 18th April 1995}

%\opt{oldrules}{\newcommand\versionyear{1993}}
%\opt{newrules}{\newcommand\versionyear{2003}}
%\opt{2012rules}{\newcommand\versionyear{2012}}

\usepackage{csa-regs}

\setlength\headheight{27.57402pt}

\begin{document}

\maketitle

\noindent
Whereas a draft of this instrument was laid before Parliament in accordance with section 52(2) of the Child Support Act 1991\footnote{\frenchspacing 1991 c. 48.} and approved by a resolution of each House of Parliament:

 Now, therefore, the Secretary of State for Social Security, in exercise of the powers conferred by sections 8(11), 10(1), 12(2) and (3), 14(1) and (3), 16(1), 17(4), 18(11), 21(2), 29(2), 32, 41(3), 42, 43(1), 46(11), 47, 51, 52, 54 and 57 of, and paragraphs 4(3), 5(1) and (2), 6(2), (4) and (5), 7(1), 8, 9($a$) and 11 of Schedule 1 to, the Child Support Act 1991\footnote{\frenchspacing Section 54 is cited because of the meaning ascribed to the word “prescribed”.} and by sections 135(1), 137(1) and 175(1), (3) and (4) of the Social Security Contributions and Benefits Act 1992\footnote{\frenchspacing 1992 c. 4. Section 137(1) is cited because of the meaning ascribed in the word “prescribed”.} and of all other powers enabling him in that behalf, after consultation with the Council on Tribunals in accordance with section 8 of the Tribunals and Inquiries Act 1992\footnote{\frenchspacing 1992 c. 53.} and after agreement by the Social Security Advisory Committee that proposals in respect of regulation 62 should not be referred to it\footnote{\frenchspacing See section 173(1)($b$) of the Social Security Administration Act 1992 (c. 5).}, hereby makes the following Regulations:

{\sloppy

\tableofcontents

}

\setcounter{secnumdepth}{-2}

\subsection[1. Citation, commencement and interpretation]{Citation, commencement and interpretation}

1.—(1) These Regulations may be cited as the Child Support and Income Support (Amendment) Regulations 1995.

(2) This regulation and regulation 58 of these Regulations shall come into force on 13th April 1995 and all other regulations shall come into force on 18th April 1995.

(3) In these Regulations—
\begin{enumerate}\item[]
“the Appeals Regulations” means the Child Support Appeal Tribunals (Procedure) Regulations 1992\footnote{\frenchspacing S.I. 1992/2641.};

“the Arrears Regulations” means the Child Support (Arrears, Interest and Adjustment of Maintenance Assessments) Regulations 1992\footnote{\frenchspacing S.I. 1992/1816. Regulation 4 was amended by S.I. 1993/913.};

“the Collection and Enforcement Regulations” means the Child Support (Collection and Enforcement) Regulations 1992\footnote{\frenchspacing S.I. 1992/1989. Regulations 8 was amended by S.I. 1993/913 and regulation 9 by S.I. 1994/227.};

“the Fees Regulations” means the Child Support Fees Regulations 1992\footnote{\frenchspacing S.I. 1992/3094. Regulations 1, 3 and 4 were amended by S.I. 1994/227.};

“the Information, Evidence and Disclosure Regulations” means the Child Support (Information, Evidence and Disclosure) Regulations 1992\footnote{\frenchspacing S.I. 1992/1812. Regulation 2 was amended by S.I. 1995/123.};

“the Maintenance Arrangements and Jurisdiction Regulations” means the Child Support (Maintenance Arrangements and Jurisdiction) Regulations 1992\footnote{\frenchspacing S.I. 1992/2645. Regulation 3 was amended by S.I. 1995/123.};

“the Maintenance Assessment Procedure Regulations” means the Child Support (Maintenance Assessment Procedure) Regulations 1992\footnote{\frenchspacing S.I. 1992/1813. Regulation 1 was amended by S.I. 1995/123; regulation 8 was amended by S.I. 1993/913 and S.I. 1995/123; regulation 10 was amended by S.I. 1995/123; regulation 19, 40 and 42 were amended by S.I. 1993/913; regulation 30 was amended by S.I. 1995/123; regulation 31 was amended by S.I. 1994/227 and S.I. 1995/123.};

“the Maintenance Assessments and Special Cases Regulations” means the Child Support (Maintenance Assessments and Special Cases) Regulations 1992\footnote{\frenchspacing S.I. 1992/1815. Regulation 1 and paragraph 3 of Schedule 1 were amended by S.I. 1993/913 and regulation 11 by S.I. 1994/227.};

“the Miscellaneous Amendment Regulations” means the Child Support (Miscellaneous Amendment 
and Transitional Provisions) Regulations 1994\footnote{\frenchspacing S.I. 1994/227.}.
\end{enumerate}

\subsection[2. Amendment of regulation 2 of the Appeals Regulations]{Amendment of regulation 2 of the Appeals Regulations}

2.  After paragraph (2) of regulation 2 of the Appeals Regulations (service of notices or documents) there shall be added the following paragraph—
\begin{quotation}
“(3) The provisions of paragraph (2) shall apply to a summons or a citation issued under regulation 10.”.
\end{quotation}

\subsection[3. Amendment of regulation 3 of the Appeals Regulations]{Amendment of regulation 3 of the Appeals Regulations}

3.  For paragraph (1) of regulation 3 of the Appeals Regulations (making an appeal or application and time limits) there shall be substituted the following paragraphs—
\begin{quotation}
“(1) This regulation applies to an appeal to a tribunal under—
\begin{enumerate}\item[]
($a$) section 20(1) or 46(7) of the Act; or

($b$) regulation 42(9) of the Child Support (Maintenance Assessment Procedure) Regulations 1992,
\end{enumerate}
and to an application to a tribunal to set aside its decision under regulation 15.

(1A) An appeal or application of a kind mentioned in paragraph (1) shall be by notice in writing, signed by the person making it, or by his representative where it appears to a chairman that he was unable to sign personally, or by a barrister, advocate or solicitor on his behalf.”.
\end{quotation}

\subsection[4. Insertion of regulation 3A in the Appeals Regulations]{Insertion of regulation 3A in the Appeals Regulations}

4.  After regulation 3 of the Appeals Regulations there shall be inserted the following regulation—
\begin{quotation}
\subsection*{“Death of a party to an appeal}

3A.—(1) In any proceedings, on the death of a party to those proceedings, the Secretary of State may appoint such person as he thinks fit to proceed with the appeal in the place of such deceased party.

(2) A grant of probate, confirmation or letters of administration to the estate of the deceased party, whenever taken out, shall have no effect on an appointment made under paragraph (1).

(3) Where a person appointed under paragraph (1) has, prior to the date of such appointment, taken any action in relation to the appeal on behalf of the deceased party, the effective date of the appointment by the Secretary of State shall be the day immediately prior to the first day on which such action was taken.”.
\end{quotation}

\subsection[5. Amendment of regulation 6 of the Appeals Regulations]{Amendment of regulation 6 of the Appeals Regulations}

5.  In paragraph (1) of regulation 6 of the Appeals Regulations, (striking out of proceedings) for the words “because of” there shall be substituted the words “for want of prosecution which term includes”.

\subsection[6. Amendment of regulation 11 of the Appeals Regulations]{Amendment of regulation 11 of the Appeals Regulations}

6.  After paragraph (5) of regulation 11 of the Appeals Regulations (hearings) there shall be inserted the following paragraph—
\begin{quotation}
“(5A) A tribunal may require any person who is, with leave of the tribunal, acting as an interpreter for any person entitled to be heard or for a witness, to swear or affirm that he will carry out his functions correctly and to the best of his skill and understand and, for that purpose, there may be administered an oath or affirmation in due form.”.
\end{quotation}

\amendment{
Regs.~7--11 revoked (25.1.10) by the Child Support (Management of Payments and Arrears) Regulations 2009 Sch.
}

% Regs 7--11 revoked (25.1.10) by SI 2009/3151 Sch
%\subsection[7. Amendment of regulation 4 of the Arrears Regulations]{Amendment of regulation 4 of the Arrears Regulations}
%
%7.—(1) Regulation 4 of the Arrears Regulations (circumstances in which no liability to pay interest arises)\footnote{\frenchspacing Paragraph (3) of regulation 4 of the Arrears Regulations was inserted by regulation 36 of S.I. 1993/913.} shall be amended in accordance with the following provisions of this regulation.
%
%(2) In paragraph (1) for the words from “with respect to arrears” to the end of the paragraph there shall be substituted the words “with respect to arrears—
%\begin{enumerate}\item[]
%($a$) in respect of any day which falls after 17th April 1995; or
%
%($b$) in respect of any period if either of the conditions set out in paragraph (2) is satisfied in relation to that period.”
%\end{enumerate}
%
%(3) In paragraph (3) the words “of interest” where they first occur shall be omitted.
%
%\subsection[8. Substitution of regulation 10 of the Arrears Regulations]{Substitution of regulation 10 of the Arrears Regulations}
%
%8.  For regulation 10 of the Arrears Regulations, there shall be substituted the following regulation—
%
%\begin{quotation}
%\subsection*{“Adjustment of the amount payable under a maintenance assessment}
%
%10.—(1) Where for any reason, including the retrospective effect of a new or fresh maintenance assessment, there has been an overpayment of child support maintenance, a child support officer may, for the purpose of taking account of that overpayment—
%\begin{enumerate}\item[]
%($a$) apply the amount overpaid to reduce any arrears of child support maintenance due under any previous maintenance assessment made in respect of the same relevant persons; or
%
%($b$) where there is no previous relevant maintenance assessment or an overpayment remains after the application for sub-paragraph ($a$), and subject to paragraph (4), adjust the amount payable under a current maintenance assessment by such amount as he considers appropriate in all the circumstances of the case having regard in particular to—
%\begin{enumerate}\item[]
%(i) the circumstances of the absent parent and the person with care;
%
%\begin{sloppypar}
%(ii) the amount of the overpayment in relation to the amount due under the current maintenance assessment; and
%\end{sloppypar}
%
%(iii) the period over which it would be reasonable for the overpayment to be rectified.
%\end{enumerate}
%\end{enumerate}
%
%\begin{sloppypar}
%(2) Where a child support officer has adjusted the amount payable under a maintenance assessment under the provisions of paragraph (1) and that maintenance assessment is subsequently reviewed under section 16, 17, 18 or 19 of the Act and a fresh maintenance assessment made, that adjustment shall, subject to paragraph (3), continue to apply to the amount payable under that fresh maintenance assessment unless a child support officer is satisfied that such adjustment would not be appropriate in all the circumstances of the case.
%\end{sloppypar}
%
%(3) Where a child support officer is satisfied that the adjustment referred to in paragraph (2) would not be appropriate, he may cancel that adjustment or he may adjust the amount payable under that fresh maintenance assessment as he sees fit, having regard to the matters specified in heads (i) to (iii) of sub-paragraph ($b$) of paragraph (1).
%
%(4) Any adjustment under the provisions of paragraph (1), (2) or (3) shall not reduce the amount payable under a maintenance assessment to less than the minimum amount prescribed under paragraph 7 of Schedule 1 to the Act.”.
%\end{quotation}
%
%\subsection[9. Amendment of regulation 11 of the Arrears Regulations]{Amendment of regulation 11 of the Arrears Regulations}
%
%9.—(1) Regulation 11 of the Arrears Regulations shall be amended in accordance with the following provisions of this regulation.
%
%(2) For the heading, there shall be substituted the following heading—
%\begin{quotation}
%\subsection*{“Notifications following a cancellation or adjustment under the provisions of regulation 10”.} {}
%\end{quotation}
%
%(3) For paragraph (1) there shall be substituted the following paragraph—
%
%\begin{quotation}
%“(1) Where a child support officer has, under the provisions of regulation 10, cancelled an adjustment in accordance with the provisions of paragraph (3) of that regulation or adjusted the amount payable under a maintenance assessment, he shall immediately notify the relevant persons, so far as is reasonably practicable, of the cancellation or, of the amount and period of the adjustment, and the amount payable during the period of the adjustment.”.
%\end{quotation}
%
%\subsection[10. Substitution of regulation 12 of the Arrears Regulations]{Substitution of regulation 12 of the Arrears Regulations}
%
%10.  For regulation 12 of the Arrears Regulations there shall be substituted the following regulation—
%\begin{quotation}
%\subsection*{\sloppy “Review of cancellations or adjustments under regulation 10}
%
%\begin{sloppypar}
%12.—(1) Where an adjustment made under regulation 10 has been cancelled under paragraph (3) of that regulation or where the amount payable under a maintenance assessment has been adjusted under the provisions of that regulation, a relevant person may apply to the Secretary of State for a review of that cancellation or adjustment as if it were a case falling within section 18 of the Act and—
%\end{sloppypar}
%\begin{enumerate}\item[]
%($a$) section 18(5), (7), (8) and regulations made under section 18(11); and
%
%($b$) subject to the modifications set out in paragraph (2), section 18(6) and (9),
%\end{enumerate}
%shall apply to such a review.
%
%(2) The modifications referred to in paragraph (1) are—
%\begin{enumerate}\item[]
%($a$) section 18(6) of the Act shall have effect as if for the words “the refusal, assessment or cancellation in question” there are substituted the words “the adjustment of the amount payable, or the cancellation of the adjustment of the amount payable, under regulation 10 of the Child Support (Arrears, Interest and Adjustment of Maintenance Assessments) Regulations 1992”;
%
%($b$) section 18(9) of the Act shall have effect as if for the words “a maintenance assessment or (as the case may be) a fresh maintenance assessment should be made” there are substituted the words “a cancelled adjustment should be reinstated or a revised adjustment of the amount payable under regulation 10 of the Child Support (Arrears, Interest and Adjustment of Maintenance Assessments) Regulations 1992 should be made.”.
%\end{enumerate}
%
%(3) Where an adjustment has been cancelled or the amount payable under a maintenance assessment has been adjusted under the provisions of regulation 10, a child support officer may reinstate that cancelled adjustment or revise that adjustment if he is satisfied that one or more of the circumstances set out in paragraphs ($a$) to ($c$) of section 19(1) of the Act apply to that cancellation or that adjustment.”.
%\end{quotation}
%
%\subsection[11. Amendment of regulation 13 of the Arrears Regulations]{Amendment of regulation 13 of the Arrears Regulations}
%
%11.—(1) Regulation 13 of the Arrears Regulations shall be amended in accordance with the following provisions of this regulation.
%
%(2) In paragraph (3) for the words “or (5)” there shall be substituted the words “or (3)”.
%
%(3) After paragraph (4) there shall be inserted the following paragraph\footnote{\frenchspacing The original paragraph (5) of regulation 13 of the Arrears Regulations was revoked by regulation 40 of S.I. 1993/913.}—
%\begin{quotation}
%“(5) Where a child support officer refuses to reinstate, or reinstates, a cancelled adjustment following a review under regulation 12(1), he shall immediately notify the relevant persons, so far as that is reasonably practicable, of the refusal or reinstatement, as the case may be, and shall give reasons for his refusal in writing.”.
%\end{quotation}

\subsection[12. Amendment of regulation 4 of the Collection and Enforcement Regulations]{Amendment of regulation 4 of the Collection and Enforcement Regulations}

12.  For paragraph (2) of regulation 4 of the Collection and Enforcement Regulations (interval of payment) there shall be substituted the following paragraph—
\begin{quotation}
“(2) In specifying the day and interval of payment the Secretary of State shall have regard to the following factors—
\begin{enumerate}\item[]
($a$) the circumstances of the person liable to make the payments and in particular the day upon which and the interval at which any income is payable to that person;

($b$) any preference indicated by that person;

($c$) any period necessary to enable the clearance of cheques or otherwise necessary to enable the transmission of payments to the person entitled to receive them,
\end{enumerate}
and, subject to those factors, to any other matter which appears to him to be relevant in the particular circumstances of the case.''
\end{quotation}

\subsection[13. Amendment of regulation 5 of the Collection and Enforcement Regulations]{Amendment of regulation 5 of the Collection and Enforcement Regulations}

13.—(1) Regulation 5 of the Collection and Enforcement Regulations (transmission of payments) shall be amended in accordance with the following provisions of this regulation.

(2) In paragraph (2), for the words “The Secretary of State” there shall be substituted the words “Subject to paragraph (3), the Secretary of State”.

(3) For paragraphs (3) and (4) there shall be substituted the following paragraphs—
\begin{quotation}
“(3) Except where the Secretary of State is satisfied in the circumstances of the case that it would cause undue hardship to either the person liable to make the payments or the person entitled to receive them, the interval referred to in paragraph (2) shall not differ from the interval referred to in regulation 4.

(4) Subject to paragraph (3) and regulation 4(2), the interval referred to in paragraph (2) and that referred to in regulation 4 may be varied from time to time by the Secretary of State.”.
\end{quotation}

\subsection[14. Amendment of regulation 8 of the Collection and Enforcement Regulations]{Amendment of regulation 8 of the Collection and Enforcement Regulations}

14.—(1) Paragraph (1) of regulation 8 of the Collection and Enforcement Regulations shall be amended in accordance with the following provisions of this regulation.

(2) Before the definition of “disposable income”, there shall be inserted the following definition—
\begin{quotation}
““defective” means in relation to a deduction from earnings order that it does not comply with the requirements of regulations 9 to 11 and such failure to comply has made it impracticable for the employer to comply with his obligations under the Act and these Regulations;”.
\end{quotation}

(3) In the definition of “disposable income”, for the words “regulation 12(1)”, there shall be substituted the words “regulation 12(1)($a$)”.

(4) After the definition of “exempt income” there shall be inserted the following definition—
\begin{quotation}
““interim maintenance assessment” means a Category A, Category B, Category C or Category D interim maintenance assessment within the meaning of regulation 8(1B) of the Child Support (Maintenance Assessment Procedure) Regulations 1992;”.
\end{quotation}

(5) In the definition of “protected income level”, after the words “in accordance with” there shall be inserted the words “paragraphs (1) to (5) of”.

\subsection[15. Amendment of regulation 9 of the Collection and Enforcement Regulations]{Amendment of regulation 9 of the Collection and Enforcement Regulations}

15.  For paragraphs ($d$) and ($e$) of regulation 9 of the Collection and Enforcement Regulations\footnote{\frenchspacing Paragraph ($e$) was amended by regulation 3(1) of S.I. 1994/227.}, there shall be substituted the following paragraphs—
\begin{quotation}
“($d$) the normal deduction rate or rates and the date upon which each is to take effect;

($e$) the protected earnings rate;”.
\end{quotation}

\subsection[16. Amendment of regulation 10 of the Collection and Enforcement Regulations]{Amendment of regulation 10 of the Collection and Enforcement Regulations}

16.—(1) Regulation 10 of the Collection and Enforcement Regulations (normal deduction rate) shall be amended in accordance with the following provisions of this regulation.

(2) In paragraph (1), for the words “the normal deduction rate” there shall be substituted the words “a normal deduction rate”.

(3) In paragraph (2)—
\begin{enumerate}\item[]
($a$) after the words “arrears or interest” there shall be inserted the words “,~in a case where there is a current assessment,”;

($b$) for the words “at the date of making of the current assessment” there shall be substituted the words “at the date of making of any current maintenance assessment other than an interim maintenance assessment”.
\end{enumerate}

\subsection[17. Amendment of regulation 11 of the Collection and Enforcement Regulations]{Amendment of regulation 11 of the Collection and Enforcement Regulations}

17.—(1) Regulation 11 of the Collection and Enforcement Regulations (protected earnings rate) shall be amended in accordance with the following provisions of this regulation.

(2) In paragraph (2), after the word “shall” there shall be inserted the words ``,~except where paragraph (3) or paragraph (4) applies,”

(3) At the end of the regulation there shall be added the following paragraphs—
\begin{quotation}
“(3) Where an interim maintenance assessment is in force the protected earnings rate shall be—
\begin{enumerate}\item[]
($a$) where there is some knowledge of the liable person’s circumstances, the aggregate of the following amounts at the date of the making of the assessment—
\begin{enumerate}\item[]
(i) the personal allowance applicable by virtue of paragraph 1(1)($e$) of Schedule 2 to the Income Support (General) Regulations 1987\footnote{\frenchspacing S.I. 1987/1967. Relevant amending instruments are 1988/663, 1989/1678.} (in this paragraph referred to as “the relevant Schedule”) or if he is known to have a partner, that applicable for a couple under paragraph 1(3)($c$) of that Schedule;

(ii) the personal allowance applicable by virtue of the relevant Schedule in respect of any child or young person who is known to be living with the relevant person (and where the age of the child or young person is not known it shall be assumed to be less than 11);

(iii) the amount of any premium aplicable by virtue of the relevant Schedule which is known to be applicable in the circumstances of the case; and

(iv) £30;
\end{enumerate}

($b$) in any other case the personal allowance specified in paragraph 1(1)($e$) of the relevant Schedule at the date mentioned in sub-paragraph ($a$), plus £30.
\end{enumerate}

(4) Where there is a liability to make payments of child support maintenance but no maintenance assessment is in force, the protected earnings rate shall be—
\begin{enumerate}\item[]
($a$) except where the last maintenance assessment was an interim maintenance assessment of Category A or Category C—
\begin{enumerate}\item[]
(i) where the absent parent produces evidence sufficient to satisfy the child support officer that his circumstances have changed since the last assessment or review under section 16, 17, 18 or 19 of the Act, a figure equal to the figure that would be his exempt income if the assessment were then being reviewed; or

(ii) in any other case an amount equal to the amount of exempt income produced by the last assessment or review under section 16, 17, 18 or 19 of the Act applicable in his case;
\end{enumerate}

($b$) in the case of an interim maintenance assessment of Category A or Category C, the amount produced by the application of the provisions of paragraph (3) above in his case.”.
\end{enumerate}
\end{quotation}

\subsection[18. Substitution of regulation 17 of the Collection and Enforcement Regulations]{Substitution of regulation 17 of the Collection and Enforcement Regulations}

18.  For regulation 17 of the Collection and Enforcement Regulations there shall be substituted the following regulation—
\begin{quotation}
\subsection*{\sloppy “Requirement to review deduction from earning orders}

17.—(1) Subject to paragraph (2), the Secretary of State shall review a deduction from earnings order in the following circumstances—
\begin{enumerate}\item[]
($a$) where there is a change in the amount of the maintenance assessment;

($b$) where any arrears and interest on arrears payable under the order are paid off.
\end{enumerate}

(2) There shall be no obligation to review a deduction from earnings order under paragraph (1) where the normal deduction rates specified in the order take account of the changes which will arise as a result of the circumstances specified in sub-paragraph ($a$) or ($b$) of that paragraph.”.
\end{quotation}

\subsection[19. Amendment of regulation 20 of the Collection and Enforcement Regulations]{Amendment of regulation 20 of the Collection and Enforcement Regulations}

19.  For paragraph (1) of regulation 20 of the Collection and Enforcement Regulations (discharge of deduction from earnings orders) there shall be substituted the following paragraph—
\begin{quotation}
“(1) The Secretary of State may discharge a deduction from earnings order where it appears to him that—
\begin{enumerate}\item[]
($a$) no further payments are due under it;

($b$) the order is ineffective or some other way of securing that payments are made would be more effective;

($c$) the order is defective;

($d$) the order fails to comply in a material respect with any procedural provision of the Act or regulations made under it other than provision made in regulation 9, 10 or 11;

($e$) at the time of the making of the order he did not have, or subsequently ceased to have, jurisdiction to make a deduction from earnings order; or

($f$) in the case of an order made at a time when there is in force an interim maintenance assessment, it is inappropriate to continue deductions under the order having regard to the compliance or the attempted compliance with the maintenance assessment by the liable person.”.
\end{enumerate}
\end{quotation}

\subsection[20. Amendment of regulation 3 of the Fees Regulations]{Amendment of regulation 3 of the Fees Regulations}

20.—(1) Regulation 3 of the Fees Regulations (liability to pay fees)\footnote{\frenchspacing Regulation 3 has been amended; paragraph (3) was substituted by regulation 5(3) of S.I. 1994/227.} shall be amended in accordance with the following provisions of this regulation.

(2) In paragraph (1), for the words “paragraphs (4) and (5)” there shall be substituted the words “paragraphs (3A) to (5)”.

(3) After paragraph (3) there shall be inserted the following paragraph—
\begin{quotation}
“(3A) No person shall be liable to pay an assessment fee or a collection fee which would otherwise become payable on or after 18th April 1995 and before 6th April 1997, and for the purposes of this paragraph a fee becomes payable—
\begin{enumerate}\item[]
($a$) in the case of a collection fee, upon the date upon which the Secretary of State arranges for the collection of, and enforcement of the obligation to pay, child support maintenance in accordance with the assessment or the anniversary of the date upon which he so arranges;

($b$) in the case of an assessment fee upon the date upon which the maintenance assessment in the case in question is made, or the anniversary thereof.”.
\end{enumerate}
\end{quotation}

\subsection[21. Amendment of regulation 4 of the Fees Regulations]{Amendment of regulation 4 of the Fees Regulations}

21.—(1) Regulation 4 of the Fees Regulations (fees)\footnote{\frenchspacing Regulation 4 has been amended; paragraph (2) was substituted by regulation 5(4) of S.I. 1994/227.} shall be amended in accordance with the following provisions of this regulation.

(2) In paragraph (4), for the words from “the amount of that fee shall” to the end of the paragraph there shall be substituted—
\begin{quotation}
“the amount of that fee shall be—
\begin{enumerate}\item[]
($a$) in a case where the Secretary of State arranges for the enforcement of the obligation to pay child support maintenance in accordance with the assessment whichever is the less of the following—
\begin{enumerate}\item[]
(i) the amount specified in paragraph (3); or

(ii) that amount multiplied by the number of complete weeks between the first date in respect of which arrears are due and the date the assessment fee next becomes payable divided by 52;
\end{enumerate}

($b$) in any other case an amount equal to the collection fee specified in paragraph (3), multiplied by the number of complete weeks between the first collection date and the date the assessment fee next becomes payable, and divided by 52.”.
\end{enumerate}
\end{quotation}

(3) In paragraph (5), after the word “terminated” there shall be inserted the words “except by virtue of regulation 3(3A) above”.

\subsection[22. Amendment of regulation 2 of the Information, Evidence and Disclosure Regulations]{Amendment of regulation 2 of the Information, Evidence and Disclosure Regulations}

22.  After sub-paragraph ($c$) of paragraph (2) of regulation 2 of the Information, Evidence and Disclosure Regulations, (persons under a duty to furnish information or evidence) there shall be inserted the following sub-paragraphs—
\begin{quotation}
“($cc$) persons employed in the service of the Crown or otherwise in the discharge of Crown functions, where they are the current or recent employer of the absent parent or the parent with care in relation to whom an application for a maintenance assessment has been made, with respect to the matters listed in sub-paragraphs ($d$), ($e$), ($f$), ($h$) and ($j$) of regulation 3(1);

($cd$) persons employed in the service of the Crown or otherwise in the discharge of Crown functions, where they are the current or recent employer of a person falling within sub-paragraph ($b$), with respect to the matters listed in sub-paragraphs ($d$) and ($e$) of regulation 3(1);”.
\end{quotation}

\subsection[23. Amendment of regulation 3 of the Information, Evidence and Disclosure Regulations]{Amendment of regulation 3 of the Information, Evidence and Disclosure Regulations}

23.  Regulation 3 of the Information, Evidence and Disclosure Regulations (purposes for which information or evidence may be required) shall be amended by the addition at the end of paragraph (2) (which specifies matters as to which information or evidence may be required) of the following sub-paragraph—
\begin{quotation}
“($s$) the making of, and the amount of, any qualifying transfer or compensating transfer within the meaning of Schedule 3A to the Maintenance Assessments and Special Cases Regulations.”.
\end{quotation}

\subsection[24. Insertion of regulation 9A into the Information, Evidence and Disclosure Regulations]{Insertion of regulation 9A into the Information, Evidence and Disclosure Regulations}

24.  After regulation 9 of the Information, Evidence and Disclosure Regulations there shall be inserted the following regulation—
\begin{quotation}
\subsection*{“Disclosure of information to other persons}

9A.—(1) The Secretary of State or a child support officer may disclose information given to him by one party to a maintenance assessment to another party to that assessment where, in the opinion of the Secretary of State or a child support officer, such information is essential to inform the party to whom it would be given as to—
\begin{enumerate}\item[]
($a$) why an application for a maintenance assessment under section 4, 6 or 7 of the Act, or an application for a review under section 17 or 18 of the Act has been rejected;

($b$) why, although an application for a maintenance assessment referred to in sub-paragraph ($a$) has been accepted, that assessment cannot, at the time in question, be proceeded with or why a maintenance assessment will not be made following that application;

($c$) why a maintenance assessment has ceased to have effect or has been cancelled, or;

($d$) how a maintenance assessment has been calculated, in so far as the matter has not been dealt with by the notification given under regulation 10 of the Maintenance Assessment Procedure Regulations.
\end{enumerate}

(2) For the purposes of this regulation, “party to a maintenance assessment” means—
\begin{enumerate}\item[]
($a$) a relevant person;

($b$) a person appointed by the Secretary of State under regulation 3A of the Child Support Appeal Tribunals (Procedure) Regulations 1992\footnote{\frenchspacing S.I. 1992/2641. Regulation 3A is inserted by regulation 4 of these Regulations.};

($c$) the personal representative of a relevant person where a review or appeal was pending at the date of death of that person and the personal representative is dealing with that review or appeal on behalf of that person.
\end{enumerate}

(3) Any application for information under this regulation shall be made to the Secretary of State or a child support officer in writing setting out the reasons for the application.

(4) Except where a person gives written permission to the Secretary of State or a child support officer that the information in relation to him mentioned in sub-paragraphs ($a$) and ($b$) below may be conveyed to other persons, any information given under the provisions of paragraph (1) shall not contain—
\begin{enumerate}\item[]
($a$) the address of any person other than the recipient of the information in question (other than the address of the office of the child support officer concerned) or any other information the use of which could reasonably be expected to lead to any such person being located;

($b$) any other information the use of which could reasonably be expected to lead to any person, other than a qualifying child or a relevant person, being identified.”.
\end{enumerate}
\end{quotation}

\subsection[25. Amendment of regulation 1 of the Maintenance Arrangement and Jurisdiction Regulations]{Amendment of regulation 1 of the Maintenance Arrangement and Jurisdiction Regulations}

25.  In paragraph (2) of regulation 1 of the Maintenance Arrangements and Jurisdiction Regulations, after the definition of “the Act” there shall be inserted the following definition—
\begin{quotation}
““Maintenance Assessment Procedure Regulations” means the Child Support (Maintenance Assessment Procedure) Regulations 1992\footnote{\frenchspacing S.I. 1992/1813. Regulation 5 was amended by S.I. 1993/913.};”.
\end{quotation}

\subsection[26. Substitution of regulation 2 of the Maintenance Arrangements and Jurisdiction Regulations]{Substitution of regulation 2 of the Maintenance Arrangements and Jurisdiction Regulations}

26.  For regulation 2 of the Maintenance Arrangements and Jurisdiction Regulations (prescription of enactments for the purposes of section 8(11) of the Act), there shall be substituted the following regulation—
\begin{quotation}
\subsection*{“Prescription of enactments for the purposes of section 8(11) of the Act}

2.  The following enactments are prescribed for the purposes of section 8(11)($f$) of the Act—
\begin{enumerate}\item[]
($a$) the Conjugal Rights (Scotland) Amendment Act 1861\footnote{\frenchspacing 24 \& 25 Vict. c. 86.};

($b$) the Court of Session Act 1868\footnote{\frenchspacing 31 \& 32 Vict. c. 100.};

($c$) the Sheriff Courts (Scotland) Act 1907\footnote{\frenchspacing 7 Edw. 7 c. 51.};

($d$) the Guardianship of Infants Act 1925\footnote{\frenchspacing 15 \& 16 Geo. 5 c. 45.};

($e$) the Illegitimate Children (Scotland) Act 1930\footnote{\frenchspacing 20 \& 21 Geo. 5 c. 33.};

($f$) the Children and Young Persons (Scotland) Act 1932\footnote{\frenchspacing 22 \& 23 Geo. 5 c. 47.};

($g$) the Children and Young Persons (Scotland) Act 1937\footnote{\frenchspacing 1 Edw. 8 \& 1 Geo. 6 c. 37.};

($h$) the Custody of Children (Scotland) Act 1939\footnote{\frenchspacing 2 \& 3 Geo. 6 c. 4.};

($i$) the National Assistance Act 1948\footnote{\frenchspacing 11 \& 12 Geo. 6 c. 29.};

($j$) the Affiliation Orders Act 1952\footnote{\frenchspacing 15 \& 16 Geo. 6 \& 1 Eliz. 2 c. 41.};

($k$) the Affiliation Proceedings Act 1957\footnote{\frenchspacing 5 \& 6 Eliz. 2 c. 55.};

($l$) the Matrimonial Proceedings (Children) Act 1958\footnote{\frenchspacing 6 \& 7 Eliz. 2 c. 40.};

($m$) the Guardianship of Minors Act 1971\footnote{\frenchspacing 1971 c. 3.};

($n$) the Guardianship Act 1973\footnote{\frenchspacing 1973 c. 29.};

($o$) the Children Act 1975\footnote{\frenchspacing 1975 c. 72.};

($p$) the Supplementary Benefits Act 1976\footnote{\frenchspacing 1976 c. 71.};

($q$) the Social Security Act 1986\footnote{\frenchspacing 1986 c. 50.};

($r$) the Social Security Administration Act 1992\footnote{\frenchspacing 1992 c. 5.}.”.
\end{enumerate}
\end{quotation}

\subsection[27. Amendment of regulation 3 of the Maintenance Arrangements and Jurisdiction Regulations]{Amendment of regulation 3 of the Maintenance Arrangements and Jurisdiction Regulations}

27.—(1) Regulation 3 of the Maintenance Arrangements and Jurisdiction Regulations (relationship between maintenance assessments and certain court orders) shall be amended in accordance with the following provisioins of this regulation.

(2) For paragraph (1), there shall be substituted the following paragraph—
\begin{quotation}
“(1) Orders made under the following enactments are of a kind prescribed for the purposes of section 10(1) of the Act—
\begin{enumerate}\item[]
($a$) the Conjugal Rights (Scotland) Amendment Act 1861;

($b$) the Court of Session Act 1868;

($c$) the Sheriff Courts (Scotland) Act 1907;

($d$) the Guardianship of Infants Act 1925;

($e$) the Illegitimate Children (Scotland) Act 1930;

($f$) the Children and Young Persons (Scotland) Act 1932;

($g$) the Children and Young Persons (Scotland) Act 1937;

($h$) the Custody of Children (Scotland) Act 1939;

($i$) the National Assistance Act 1948;

($j$) the Affiliation Orders Act 1952;

($k$) the Affiliation Proceedings Act 1957;

($l$) the Matrimonial Proceedings (Children) Act 1958;

($m$) the Guardianship of Minors Act 1971;

($n$) the Guardianship Act 1973;

($o$) Part II of the Matrimonial Causes Act 1973\footnote{\frenchspacing 1973 c. 18.};

($p$) the Children Act 1975;

($q$) the Supplementary Benefits Act 1976;

\begin{sloppypar}
($r$) the Domestic Proceedings and Magistrates Courts Act 1978\footnote{\frenchspacing 1978 c. 22.};
\end{sloppypar}

($s$) Part III of the Matrimonial and Family Proceedings Act 1984\footnote{\frenchspacing 1984 c. 42.};

($t$) the Family Law (Scotland) Act 1985\footnote{\frenchspacing 1985 c. 37.};

($u$) the Social Security Act 1986;

($v$) Schedule 1 to the Children Act 1989\footnote{\frenchspacing 1989 c. 41.};

($w$) the Social Security Administration Act 1992.”.
\end{enumerate}
\end{quotation}

(3) After paragraph (7), there shall be added the following paragraph—
\begin{quotation}
“(8) Where—
\begin{enumerate}\item[]
($a$) a maintenance assessment is made in accordance with Part I of Schedule 1 to the Act in respect of children with respect to whom an order falling within paragraph (1) was in force; and

($b$) that order ceases to have effect on or after 18th April 1995, for reasons other than the making of an interim maintenance assessment, but prior to the date on which the maintenance assessment is made and after—
\begin{enumerate}\item[]
(i) the date on which a maintenance enquiry form referred to in regulation 5(2) of the Maintenance Assessment Procedure Regulations was given or sent to the absent parent, where the application for a maintenance assessment was made by a person with care or a child under section 7 of the Act; or

(ii) the date on which a maintenance application which complies with the provisions of regulation 2 of the Maintenance Assessment Procedure Regulations was received by the Secretary of State from an absent parent,
\end{enumerate}
\end{enumerate}
the effective date of that maintenance assessment shall be the day following that on which the court order ceased to have effect.”.
\end{quotation}

\subsection[28. Amendment of regulation 8 of the Maintenance Assessment Procedure Regulations]{\sloppy Amendment of regulation 8 of the Maintenance Assessment Procedure Regulations}

28.—(1) Regulation 8 of the Maintenance Assessment Procedure Regulations (amount and duration of an interim maintenance assessment) shall be amended in accordance with the following provisions of this regulation.

(2) In paragraph (1A), for the word “two” there shall be substituted the word “four” and for the words “Category B interim maintenance assessments.”\ there shall be substituted the words “Category B interim maintenance assessments, Category C interim maintenance assessments and Category D interim maintenance assessments.”.

(3) In sub-paragraph ($a$) of paragraph (1B), for the words “the information that is required by him as to the income of the absent parent” there shall be substituted the words “any information, other than information referred to in sub-paragraph ($b$), that is required by him”.

(4) For sub-paragraph ($b$) of paragraph (1B), there shall be substituted the following sub-paragraph—
\begin{quotation}
“($b$) a Category B interim maintenance assessment, where the information that is required by him as to the income of the partner or other member of the family of the absent parent or parent with care for the purposes of the calculation of the income of that partner or other member of the family under regulation 9(2), 10, 11(2) or 12(1) of the Maintenance Assessments and Special Cases Regulations—
\begin{enumerate}\item[]
(i) has not been provided by that partner or other member of the family, and that partner or other member of the family has that information in his possession or can reasonably be expected to acquire it; or

(ii) has been provided by that partner or other member of the family to the absent parent or parent with care, but the absent parent or parent with care has not provided it to the Secretary of State or the child support officer;”.
\end{enumerate}
\end{quotation}

(5) In paragraph (1B) after sub-paragraph ($b$), there shall be inserted the following sub-paragraphs—
\begin{quotation}
“($c$) a Category C interim maintenance assessment where—
\begin{enumerate}\item[]
(i) the absent parent is a self-employed earner as defined in regulation 1(2) of the Maintenance Assessments and Special Cases Regulations; and

(ii) the absent parent is currently unable to provide, but has indicated that he expects within a reasonable time to be able to provide, information to enable a child support officer to determine the earnings of that absent parent in accordance with paragraphs 3 to 5 of Schedule 1 to the Maintenance Assessments and Special Cases Regulations; and

(iii) no maintenance order as defined in section 8(11) of the Act or written maintenance agreement as defined in section 9(1) of the Act is in force with respect to the children in respect of whom the Category C interim maintenance assessment would be made; or
\end{enumerate}

($d$) a Category D interim maintenance assessment where it appears to a child support officer, on the basis of information available to him as to the income of the absent parent, that the amount of any maintenance assessment made in accordance with Part I of Schedule 1 of the Act applicable to that absent parent may be higher than the amount of a Category A interim maintenance assessment in force in respect of him.”.
\end{quotation}

(6) In paragraph (2A), for the words “The amount” there shall be substituted the words “Subject to paragraph (2D), the amount”.

(7) In paragraph (2C), after the words “absent parent” where they first appear there shall be inserted the words “calculated in accordance with regulation 12(1)($a$) of the Maintenance Assessments and Special Cases Regulations”.

(8) After paragraph (2C), there shall be inserted the following paragraphs—
\begin{quotation}
“(2D) Where the application of the provisions of paragraph (2B) or (2C) would result in the amount of a Category B interim maintenance assessment being more than 30 per centum of the net income of the absent parent as calculated in accordance with regulation 7 of the Maintenance Assessments and Special Cases Regulations, those provisions shall not apply to that absent parent and instead, the amount of that Category B interim maintenance assessment shall be 30 per centum of his net income as so calculated and where that calculation results in a fraction of a penny, that fraction shall be disregarded.

(2E) The amount of child support maintenance fixed by a category C interim maintenance assessment shall be £30.00 but a child support officer may set a lower amount, including a nil amount, if he thinks it reasonable to do so in all the circumstances of the case.

(2F) Paragraph 6 of Schedule 1 to the Act shall not apply to Category C interim maintenance assessments.

(2G) A child support officer shall notify the person with care where he is considering setting a lower amount for a Category C interim maintenance assessment in accordance with paragraph (2E), and shall take into account any relevant representations made by that person with care in deciding the amount of that Category C interim maintenance assessment.

(2H) The amount of child support maintenance fixed by a Category D interim maintenance assessment shall be calculated or estimated by applying to the absent parent’s income, in so far as the child support officer is able to determine it at the time of the making of that Category D interim maintenance assessment, the provisions of Part I of Schedule 1 to the Act and regulations made under it, subject to the modification that—
\begin{enumerate}\item[]
($a$) paragraphs 6 and 8 of that Schedule shall not apply; and

($b$) only paragraphs (1)($a$) and (5) of regulation 9 of the Maintenance Assessments and Special Cases Regulations shall apply; and

($c$) heads ($b$) and ($c$) of sub-paragraph (3) of paragraph 1 of Schedule 1 to the Maintenance Assessments and Special Cases Regulations shall not apply.
\end{enumerate}

(2I) Where the absent parent referred to in paragraph (2H) is an employed earner as defined in regulation 1 of the Maintenance Assessments and Special Cases Regulations and the child support officer is unable to calculate the net income of that absent parent, his net income shall be estimated under the provisions of paragraph (2A)($a$) and ($b$) of regulation 1 of the Maintenance Assessments and Special Cases Regulations.”.
\end{quotation}

(9) For paragraph (3), there shall be substituted the following paragraph—
\begin{quotation}
“(3) Except where regulation 3(5) of the maintenance Arrangements and Jurisdiction Regulations (effective date of maintenance assessment where court order in force) or paragraph (3A), (3B), (3C), (3D), (7) (7A) or (7E) applies, the effective date of an interim maintenance assessment shall be—
\begin{enumerate}\item[]
($a$) in respect of a Category A, Category C or Category D interim maintenance assessment, subject to sub-paragraph ($c$), such date, being not earlier than the first and not later than the seventh day following the date upon which that interim maintenance assessment was made, as falls on the same day of the week as the date determined in accordance with regulation 30(2)($a$)(ii);

($b$) in respect of a Category B interim maintenance assessment—
\begin{enumerate}\item[]
(i) subject to head (ii) and sub-paragraph ($c$), such date, being not earlier than the first and not later than the seventh day following the expiry of the period of 14 days specified in paragraph (1), as falls on the same day of the week as the date determined in accordance with regulation 30(2)($a$)(ii);

(ii) where that Category B interim maintenance assessment is made after a Category A, Category C or Category D interim maintenance assessment has been in force, the date upon which that Category A, Category C or Category D interim maintenance assessment ceased to have effect in accordance with paragraph (9A);
\end{enumerate}

($c$) in respect of a Category A, Category B, Category C or Category D interim maintenance assessment, where the application of the provisions of sub-paragraph ($a$) or ($b$)(i) would otherwise set an effective date for an interim maintenance assessment earlier than the end of a period of eight weeks from the date upon which—
\begin{enumerate}\item[]
(i) the maintenance enquiry form referred to in regulation 30(2)($a$)(i) was given or sent to an absent parent; or

(ii) the application made by in absent parent referred to in regulation 30(2)($b$)(i) was received by the Secretary of State,
\end{enumerate}
in circumstances where that absent parent has complied with the provisions of regulation 30(2)($a$)(i) or ($b$)(i) or paragraph (2A) of that regulation applies, the date determined in accordance with regulation 30(2)($a$)(i) or ($b$)(i).”.
\end{enumerate}
\end{quotation}

(10) In paragraph (3A), for the words “52 weeks” there shall be substituted the words “104 weeks” and in paragraphs (3A) to (3C), for the words “Category A” where they occur there shall be substituted the words “Category A or Category D”.

(11) In paragraph (4), for the words “or (3D)” there shall be substituted the words “,~(3D), (7), (7A) or (7E)”.

(12) In paragraph (6), for the words “an interim maintenance assessment” there shall be substituted the words “a Category A, Category B or Category D interim maintenance assessment”.

(13) For paragraph (7), there shall be substituted the following paragraphs—
\begin{quotation}
“(7) Where a child support officer cancels a Category A, Category B or Category D interim maintenance assessment in accordance with the provisions of paragraph (6), and he is satisfied that there was unavoidable delay for only part of the period during which that assessment was in force, and that another Category A, Category B or Category D interim maintenance assessment should be made, the effective date of that other Category A, Category B or Category D interim maintenance assessment shall, subject to paragraph (7A), be the first day of the maintenance period following the date upon which, in the opinion of the child support officer, the delay became avoidable.

(7A) Where the Category A or Category B interim maintenance assessment cancelled in accordance with the provisions of paragraph (6) was made prior to 18 April 1995 and the effective date of any new Category A or Category B interim maintenance assessment would, by virtue of paragraph (7), be prior to 18 April 1995, the effective date of that new Category A or Category B interim maintenance assessment shall be the first day of the maintenance period which begins on or after 18 April 1995.

(7B) Where in respect of any Category A or Category B interim maintenance assessment in force before 18 April 1995 the delay referred to in paragraph (6) became avoidable before 18 April 1995, that Category A or Category B interim maintenance assessment may not be cancelled with effect from a date earlier than the date the delay became avoidable.

(7C) Subject to paragraph (6), where a child support officer is satisfied that it would be appropriate to make an interim maintenance assessment the Category of which is different from that of the interim maintenance assessment in force, he may cancel the interim maintenance assessment which is in force with effect from whichever is the later of the first day of the maintenance period in which he becomes so satisfied or the first day of the maintenance period which begins on or after 18 April 1995.

(7D) In paragraph (7C), “Category” in relation to an interim maintenance assessment means Category A, Category B, Category C or Category D, as the case may be.

(7E) Where a child support officer makes an interim maintenance assessment following the cancellation of an interim maintenance assessment in accordance with paragraph (7C), the effective date for the fresh interim maintenance assessment shall be the date upon which that cancellation took effect.

(7F) A child support officer may cancel an interim maintenance assessment which is in force with effect from such date as he considers appropriate in all the circumstances on the grounds that—
\begin{enumerate}\item[]
($a$) there was a material procedural error in connection with the making of the assessment; or

($b$) he is satisfied that he did not, or has subsequently ceased to have jurisdiction to make that interim maintenance assessment.”.
\end{enumerate}
\end{quotation}

(14) For paragraph (8), there shall be substituted the following paragraph—
\begin{quotation}
“(8) Where a maintenance assessment calculated in accordance with Part I of Schedule 1 to the Act is made following an interim maintenance assessment, the amount of child support maintenance payable in respect of the period after 18 April 1995 during which that interim maintenance assessment was in force shall be that fixed by the maintenance assessment.”.
\end{quotation}

(15) At the beginning of paragraph (9), for the words “An interim maintenance assessment” there shall be substituted the words “Subject to paragraph (9A), an interim maintenance assessment”.

(16) After paragraph (9) there shall be inserted the following paragraph—
\begin{quotation}
“(9A) A Category A, Category C or Category D interim maintenance assessment shall cease to have effect on the first day of the maintenance period in which the Secretary of State has received all the information that is required by him to enable a child support officer to make an assessment in accordance with the provisions of Part I of Schedule 1 to the Act with the exception of the information referred to in paragraph (1B)($b$) or the first day of the maintenance period after 18 April 1995 whichever is the later.”.
\end{quotation}

(17) Paragraph (10) shall be omitted.

(18) In paragraph (11), for the words “for the purposes of sections 17 and 18 of the Act” there shall be substituted the words “for the purposes of sections 17, 18 and 19(2) of the Act” and after the words “Category A” there shall be inserted the words “or Category D”.

(19) In paragraph (12), after the words “Category A” there shall be inserted the words “or Category D”.

\subsection[29. Amendment of regulation 9 of the Maintenance Assessment Procedure Regulations]{\sloppy Amendment of regulation 9 of the Maintenance Assessment Procedure Regulations}

29.  In paragraph (1) of regulation 9 of the Maintenance Assessment Procedure Regulations, after the words “Category A” there shall be inserted the words “or Category D”.

\subsection[30. Amendment of regulation 10 of the Maintenance Assessment Procedure Regulations]{Amendment of regulation 10 of the Maintenance Assessment Procedure Regulations}

30.—(1) Regulation 10 of the Maintenance Assessment Procedure Regulations (notification of a new or a fresh maintenance assessment) shall be amended in accordance with the following provisions of this regulation.

(2) For paragraph (1) there shall be substituted the following paragraph—
\begin{quotation}
“(1) Where a child support officer—
\begin{enumerate}\item[]
($a$) makes a new or fresh maintenance assessment following an application under section 4, 6 or 7 of the Act or a review under section 16, 17, 18 or 19 of the Act;

($b$) substitutes an interim maintenance assessment for one which is in force in accordance with regulation 8; or

($c$) makes a maintenance assessment calculated in accordance with Part I of Schedule 1 to the Act where an interim maintenance assessment is or has been in force,
\end{enumerate}
he shall immediately notify the relevant persons, so far as that is reasonably practicable, of the amount of the child support maintenance under that assessment.”.
\end{quotation}

(3) In paragraph (2) for the words “paragraph (2A)” there shall be substituted the words “paragraphs (2A) and (2B)”.

(4) In paragraph (2A), after the words “Category A” wherever they appear there shall be inserted the words “Category C or D”.

(5) After paragraph (2A) there shall be inserted the following paragraph—
\begin{quotation}
“(2B) A notification under paragraph (1) in relation to a Category B interim maintenance assessment shall set out in relation to it—
\begin{enumerate}\item[]
($a$) the matters listed in sub-paragraphs ($a$), ($b$) and ($d$) to ($f$) of paragraph (2); and

($b$) where known, the absent parent’s assessable income.”.
\end{enumerate}
\end{quotation}

(6) In paragraph (5), after the words “Category A” wherever they appear there shall be inserted the words “or Category D”.

\subsection[31. Amendment of regulation 11 of the Maintenance Assessment Procedure Regulations]{Amendment of regulation 11 of the Maintenance Assessment Procedure Regulations}

31.  For paragraph (2) of regulation 11 of the Maintenance Assessment Procedure Regulations (notification of a refusal to conduct a review) there shall be substituted the following paragraph—
\begin{quotation}
“(2) A notification under paragraph (1) shall include information as to the following provisions—
\begin{enumerate}\item[]
($a$) where the refusal is on the grounds set out in section 17(3) of the Act, sections 16 and 18 of the Act and regulations 24(1) and 31(7);

($b$) except where sub-paragraph ($c$) applies, where the refusal is on the grounds set out in section 18(6) of the Act, sections 16, 17 and 20 of the Act;

($c$) where the refusal is on the grounds set out in section 18(6) of the Act and relates to a decision made under regulation 9(6), sections 16 and 20 of the Act.”.
\end{enumerate}
\end{quotation}

\subsection[32. Amendment of regulation 13 of the Maintenance Assessment Procedure Regulations]{Amendment of regulation 13 of the Maintenance Assessment Procedure Regulations}

32.  After sub-paragraph ($b$) of paragraph (2) of regulation 13 of the Maintenance Assessment Procedure Regulations (notification of refusal to cancel a maintenance assessment) there shall be added the following sub-paragraph—
\begin{quotation}
“($c$) where the refusal is of an application for the cancellation of a Category A or a Category D interim maintenance assessment under regulation 9, sections 16 and 20 of the Act.”.
\end{quotation}

\subsection[33. Amendment of regulation 14 of the Maintenance Assessment Procedure Regulations]{Amendment of regulation 14 of the Maintenance Assessment Procedure Regulations}

33.  In paragraph (1) of regulation 14 of the Maintenance Assessment Procedure Regulations (notification of cancellation of maintenance assessment), after the word “assessment,” there shall be inserted the words “except a Category A or Category D interim maintenance assessment falling within regulation 9,”.

\subsection[34. Amendment of regulation 17 of the Maintenance Assessment Procedure Regulations]{Amendment of regulation 17 of the Maintenance Assessment Procedure Regulations}

34.—(1) Regulation 17 of the Maintenance Assessment Procedure Regulations (intervals between periodical reviews and notice of a periodical review) shall be amended in accordance with the following provisions of this regulation.

(2) In paragraph (1), for the words “after it has been in force for a period of 52 weeks” there shall be substituted the words “after it has been in force for a period of—
\begin{enumerate}\item[]
(i) in the case of an assessment the effective date of which is on or before 18th April 1994, 52 weeks;

(ii) in the case of an assessment the effective date of which is after 18th April 1994, 104 weeks.”
\end{enumerate}

(3) For paragraph (2), there shall be substituted the following paragraph—
\begin{quotation}
“(2) Where a maintenance assessment in force is a fresh assessment following a review under section 18 or 19 of the Act, that assessment shall be reviewed by a child support officer under section 16 of the Act after it has been in force for a period of—
\begin{enumerate}\item[]
($a$) in the case of an assessment the effective date of which is on or before 18th April 1994, 52 weeks;

($b$) in the case of an assessment the effective date of which is after 18th April 1994, 104 weeks,
\end{enumerate}
less, in either case, the period between the effective date of the previous assessment falling within paragraph (1) above and the effective date of the fresh assessment following the review under section 18 or 19 of the Act.”.
\end{quotation}

\subsection[35. Amendment of regulation 19 of the Maintenance Assessment Procedure Regulations]{Amendment of regulation 19 of the Maintenance Assessment Procedure Regulations}

35.  For paragraph (3) of regulation 19 of the Maintenance Assessment Procedure Regulations (conduct of a review on a change of circumstances), there shall be substituted the following paragraph—
\begin{quotation}
“(3) The provisions of paragraph (2) shall not apply in relation to any person to whom or in respect of whom income support is payable or to a person with care where income support is payable to or in respect of the absent parent.”.
\end{quotation}

\subsection[36. Amendment of regulation 30 of the Maintenance Assessment Procedure Regulations]{Amendment of regulation 30 of the Maintenance Assessment Procedure Regulations}

36.—(1) Regulation 30 of the Maintenance Assessment Procedure Regulations (effective dates of new assessments) shall be amended in accordance with the following provisions of this regulation.

(2) In paragraph (1) for the words “and (7)” there shall be substituted the words “, (7) and (8)”.

(3) For paragraph (2) there shall be substituted the following paragraph—
\begin{quotation}
“(2) Where no maintenance assessment made in accordance with Part I of Schedule 1 to the Act is in force with respect to the person with care and absent parent, the effective date of a new assessment shall be—
\begin{enumerate}\item[]
($a$) in a case where the application for a maintenance assessment is made by a person with care or by a child under section 7 of the Act—
\begin{enumerate}\item[]
(i) eight weeks from the date on which a maintenance enquiry form has been given or sent to an absent parent, where such date is on or after 18th April 1995 and where within four weeks of the date that form was given or sent, it has been returned by the absent parent to the Secretary of State and it contains his name, address and written confirmation that he is the parent of the child or children in respect of whom the application for a maintenance assessment was made;

(ii) in all other circumstances, the date a maintenance enquiry form is given or sent to an absent parent;
\end{enumerate}

($b$) in a case where the application for a maintenance assessment is made by an absent parent—
\begin{enumerate}\item[]
(i) eight weeks from the date on which an application made by an absent parent was received by the Secretary of State, where such date is on or after 18 April 1995 and where, on, or within four weeks of, the date of receipt of that maintenance application, the absent parent has provided his name, address and written confirmation that he is the parent of the child or children in respect of whom the application was made;

(ii) in all other circumstances, the date an effective maintenance application form is received by the Secretary of State.”.
\end{enumerate}
\end{enumerate}
\end{quotation}

(4) After paragraph (2), there shall be inserted the following paragraph—
\begin{quotation}
“(2A) Where a child support officer is satisfied that there was unavoidable delay by the absent parent in providing the information listed in sub-paragraphs ($a$)(i) or ($b$)(i) of paragraph (2) within the time specified in those sub-paragraphs, he may apply the provisions of those sub-paragraphs for the purpose of setting the effective date of a maintenance assessment even though that information was not provided within the time specified in those sub-paragraphs.”.
\end{quotation}

\subsection[37. Amendment of regulation 31 of the Maintenance Assessment Procedure Regulations]{Amendment of regulation 31 of the Maintenance Assessment Procedure Regulations}

37.—(1) Regulation 31 of the Maintenance Assessment Procedure Regulations (effective dates of maintenance assessments following a review under sections 16 to 19 of the Act) shall be amended in accordance with the following provisions of this regulation.

(2) In paragraph (1), for the words “52 weeks” there shall be substituted the words “104 weeks”.

(3) In paragraph (6) for the words “Subject to paragraphs (7), (10) and (11)” there shall be substituted the words “Subject to paragraphs (6A), (6B), (6C), (9) and (10)”.

(4) After paragraph (6) there shall be inserted the following paragraphs—
\begin{quotation}
“(6A) Subject to paragraph (6C), where an application is made under section 18(2) of the Act for a review of a maintenance assessment in force following notification being given to the relevant person that the child support officer does not propose to review the assessment in consequence of the coming into force of the provisions mentioned in paragraph (6B), the effective date of a fresh assessment (if one is made) following such a review shall be—
\begin{enumerate}\item[]
($a$) where the application is received within 28 days of the Secretary of State notifying the relevant person of the child support officer’s decision, or on a later date where the Secretary of State is satisfied that there was unavoidable delay, the effective date as determined on the review;

($b$) subject to sub-paragraph ($a$), where the application is received by the Secretary of State later than 28 days after the date of the notification of the child support officer’s decision, the first day of the maintenance period in which the application is received.
\end{enumerate}

(6B) Paragraph (6A) applies to the following provisions of the Income Support and Child Support (Amendment) Regulations 1995—
\begin{enumerate}\item[]
($a$) regulation 44(2);

($b$) regulation 45;

($c$) regulation 46(2)($d$) and ($e$);

($d$) regulation 51.
\end{enumerate}

(6C) Where the application made under section 18(2) is made following notification being given to the relevant person that the child support officer has determined that the amount to be allowed in the computation of the relevant person’s exempt income in accordance with Schedule 3A to the Child Support (Maintenance Assessments and Special Cases) Regulations is nil by reason of the failure of the relevant person to furnish within a reasonable time the evidence required by paragraph 2 of that Schedule—
\begin{enumerate}\item[]
($a$) where the Secretary of State is satisfied that there was good cause for the delay in furnishing the evidence the effective date of any assessment made in consequence of the review shall be the effective date which would have been applicable to the assessment had the evidence been furnished timeously;

($b$) where the Secretary of State is not satisfied that there was good cause for the delay, the effective date of any revised assessment shall be the first day of the maintenance period in which the relevant person provides that evidence.”.
\end{enumerate}
\end{quotation}

(5) For paragraph (14), there shall be substituted the following paragraph—
\begin{quotation}
“(14) Where a child support officer following a review under section 19(1) of the Act makes a fresh maintenance assessment or on a review under section 19(2) of the Act is satisfied that if an application were to be made under section 18 of the Act it would be appropriate to make a fresh maintenance assessment, and does so, the effective date of that fresh assessment shall—
\begin{enumerate}\item[]
($a$) be determined in accordance with paragraph (5) or (8); or

($b$) be determined in accordance with paragraph (7), subject to the modification that that paragraph shall have effect as if for “the date determined under paragraph (2)” there is substituted “the first day of the maintenance period in which the child support officer is first satisfied that a review under section 19(1) of the Act should be undertaken or the first day of the maintenance period following 18th April 1995, whichever is the later”; or

($c$) (subject to paragraphs (9) or (10)), be the first day of the maintenance period in which the child support officer is satisfied that a review under section 19 of the Act should be undertaken or the first day of the maintenance period following 18th April 1995 whichever is the later.”.
\end{enumerate}
\end{quotation}

\subsection[38. Amendment of regulation 36 of the Maintenance Assessment Procedure Regulations]{Amendment of regulation 36 of the Maintenance Assessment Procedure Regulations}

38.—(1) Regulation 36 of the Maintenance Assessment Procedure Regulations (amount of and period of reduction of relevant benefit under a reduced benefit direction) shall be amended in accordance with the following provisions of this regulation.

(2) In paragraph (4), for the words “Subject to paragraph (5)”, there shall be substituted the words “Subject to paragraphs (5), (5A) and (5B)”.

(3) After paragraph (5), there shall be inserted the following paragraphs—
\begin{quotation}
“(5A) Where the relevant benefit is family credit or disability working allowance and, at the time a direction is given, a lump sum payment has already been made under the provisions of regulation 27(1A) of the Social Security (Claims and Payments) Regulations 1987\footnote{\frenchspacing S.I. 1987/1968. Regulation 27(1A) was inserted by S.I. 1993/2113.} (payment of family credit or disability working allowance by lump sum) the direction shall, subject to paragraph (5B), come into operation on the first day of any benefit week which immediately follows the period in respect of which the lump sum payment was made, or the first day of any benefit week which immediately follows 18th April 1995 if later.

(5B) Where the period in respect of which the lump sum payment was made is not immediately followed by a benefit week, but family credit or disability working allowance again becomes payable, or income support becomes payable, during a period of 52 weeks from the date the direction was given, the direction shall come into operation on the first day of the second benefit week which immediately follows the expiry of a period of 14 days from service of the notice specified in paragraph (5C).

(5C) Where paragraph (5B) applies, the parent to or in respect of whom family credit or disability working allowance has again become payable, or income support has become payable, shall be notified in writing by a child support officer that the amount of family credit, disability working allowance or income support paid to or in respect of her will be reduced in accordance with the provisions of paragraph (5B) if she continues to fail to comply with the obligations imposed by section 6 of the Act.

(5D) Where—
\begin{enumerate}\item[]
($a$) family credit or disability working allowance has been paid by lump sum under the provisions of regulation 27(1A) of the Social Security (Claims and Payments) Regulations 1987 (whether or not a benefit week immediately follows the period in respect of which the lump sum payment was made); and

($b$) where income support becomes payable to or in respect of a parent to or in respect of whom family credit or disability working allowance was payable at the time the direction referred to in paragraph (5A) was made, 
\end{enumerate}
income support shall become a relevant benefit for the purposes of that direction and the amount payable by way of income support shall be reduced in accordance with that direction.

(5E) In circumstances to which paragraph (5A) or (5B) applies, where no relevant benefit has become payable during a period of 52 weeks from that date on which a direction was given, it shall lapse.”.
\end{quotation}

\subsection[39. Amendment of regulation 40 of the Maintenance Assessment Procedure Regulations]{Amendment of regulation 40 of the Maintenance Assessment Procedure Regulations}

39.—(1) Regulation 40 of the Maintenance Assessment Procedure Regulations (suspension of a reduced benefit direction where a modified applicable amount is payable) shall be amended in accordance with the following provisions of this regulation.

(2) After paragraph (1) there shall be inserted the following paragraph—
\begin{quotation}
“(1A) Where a direction is given or is in operation at a time when income support is payable to or in respect of the parent concerned, but her applicable amount includes a residential allowance under regulation 17 of, and paragraph 2A of Schedule 2 to, the Income Support Regulations\footnote{\frenchspacing Regulation 17 was amended and paragraph 2A added by S.I. 1992/3147. Paragraph 2A(1) was substituted by S.I. 1994/542.} (applicable amounts for those in residential care or nursing homes), that direction shall be suspended for as long as her applicable amount includes a residential allowance under regulation 17 and paragraph 2A of Schedule 2, or 52 weeks, whichever period is the shorter.”.
\end{quotation}

(3) In paragraph (2), after the words “paragraph (1)” the words “or (1A)” shall be inserted.

\subsection[40. Amendment of regulation 42 of the Maintenance Assessment Procedure Regulations]{Amendment of regulation 42 of the Maintenance Assessment Procedure Regulations}

40.  In paragraph (5) of regulation 42 of the Maintenance Assessment Procedure Regulations (review of a reduced benefit direction), after the words “were given” there shall be added the words “or the Secretary of State or a child support officer becomes aware of a question of a kind mentioned in paragraph (2A) or (2B)”.

\subsection[41. Amendment of regulation 1 of the Maintenance Assessments and Special Cases Regulations]{\sloppy Amendment of regulation 1 of the Maintenance Assessments and Special Cases Regulations}

\begin{sloppypar}
41.—(1) Regulation 1 of the Maintenance Assessments and Special Cases Regulations (citation, commencement and interpretation)\footnote{\frenchspacing There are amendments to regulation 1 which are not relevant for the purpose of these regulations.} shall be amended in accordance with the following provisions of this regulation.
\end{sloppypar}

(2) In paragraph (2)—
\begin{enumerate}\item[]
(i) for the definition of “day to day care” there shall be substituted the following definition—
\begin{quotation}
““day to day care” means—
\begin{enumerate}\item[]
($a$) care of not less than 104 nights in total during the 12 month period ending with the relevant week; or

($b$) where, in the opinion of the child support officer, a period other than 12 months but ending with the relevant week is more representative of the current arrangements for the care of the child in question, care during that period is not less in total than the number of nights which bears the same ratio to 104 nights as that period bears to 12 months,
\end{enumerate}
and for the purpose of this definition—
\begin{enumerate}\item[]
(i) where a child is a boarder at a boarding school, or is an in-patient in a hospital, the person who, but for those circumstances, would otherwise provide day to day care of the child shall be treated as providing day to day care during the periods in question;

(ii) “relevant week” shall have the meaning ascribed to it in head (ii) of sub-paragraph ($a$) of the definition of “relevant week” in this paragraph;”;
\end{enumerate}
\end{quotation}

(ii) after the definition of the word “prisoner” there shall be inserted the following definition—
\begin{quotation}
““qualifying transfer” has the meaning assigned to it in Schedule 3A;” and
\end{quotation}

(iii) in the definition of the word “student” for the words “Education (Mandatory Awards) Regulations 1988” there shall be substituted the words
“Education (Mandatory Awards) (No.\ 2) Regulations 1993”\footnote{\frenchspacing S.I. 1993/2914.}.
\end{enumerate}

(3) In paragraph (2A)—
\begin{enumerate}\item[]
($a$) after the words “personal pension scheme, then” there shall be inserted the words “subject to sub-paragraph ($e$)”;

($b$) in sub-paragraph ($a$), for the words “effective date” there shall be substituted the words “relevant week”;

($c$) in sub-paragraph ($c$), the words “or personal” shall be omitted and for the word “paid.” there shall be substituted the words “paid; and”;

($d$) after sub-paragraph ($c$), the following sub-paragraphs shall be added—
\begin{quotation}
“($d$) the amount to be deducted in respect of contributions towards a personal pension scheme shall be one half of the contributions paid by that person or, where that scheme is intended partly to provide a capital sum to discharge a mortgage secured on that person’s home, 37.5 per centum of those contributions;

($e$) in relation to any bonus or commission which may be included in that person’s income—
\begin{enumerate}\item[]
(i) the amount to be deducted in respect of income tax shall be calculated by applying to the gross amount of that bonus or commission the rate or rates of income tax applicable in the relevant week;

(ii) the amount to be deducted in respect of primary Class 1 contributions under the Contributions and Benefit Act or under the Social Security Contributions and Benefits (Northern Ireland) Act 1992 shall be calculated by applying to the gross amount of that bonus or commission the appropriate main primary percentage applicable in the relevant week; and

(iii) the amount to be deducted in respect of contributions paid by that person in respect of the gross amount of that bonus or commission towards an occupational pension scheme shall be one half of any sum so paid.”.
\end{enumerate}
\end{quotation}
\end{enumerate}

\subsection[42. Amendment of regulation 2 of the Maintenance Assessments and Special Cases Regulations]{\sloppy Amendment of regulation 2 of the Maintenance Assessments and Special Cases Regulations}

42.  In paragraph (2) of regulation 2 of the Maintenance Assessments and Special Cases Regulations (calculation or estimation of amounts), for the words “regulation 13(2)” there shall be substituted the words “regulations 11(6) and (7) and 13(2) and regulation 8(2C) of the Maintenance Assessment Procedure Regulations” and before the words “these Regulations” there shall be inserted the words “the Act or”.

\subsection[43. Amendment of regulation 6 of the Maintenance Assessments and Special Cases Regulations]{\sloppy Amendment of regulation 6 of the Maintenance Assessments and Special Cases Regulations}

43.  In sub-paragraph ($a$) of paragraph (2) of regulation 6 of the Maintenance Assessments and Special Cases Regulations (value of Z in calculating the additional element)\footnote{\frenchspacing There is an amendment to regulation 6 which is not relevant for the purposes of these regulations.}, for “3” there shall be substituted “1.5”.

\subsection[44. Amendment of regulation 9 of the Maintenance Assessments and Special Cases Regulations]{\sloppy Amendment of regulation 9 of the Maintenance Assessments and Special Cases Regulations}

44.—(1) Regulation 9 of the Maintenance Assessments and Special Cases Regulations (exempt income: calculation or estimation of E) shall be amended in accordance with the following provisions of this regulation.

(2) Paragraph (1) shall be amended—
\begin{enumerate}\item[]
($a$) by the insertion, after sub-paragraph ($b$), of the following sub-\hspace{0pt}paragraph—
\begin{quotation}
“($bb$) where applicable, an amount in respect of a qualifying transfer of property determined in accordance with Schedule 3A;”; and
\end{quotation}

($b$) by the addition, at the end of the paragraph of the following sub-paragraph—
\begin{quotation}
“($i$) where applicable, an amount in respect of travelling costs determined in accordance with Schedule 3B.”.
\end{quotation}
\end{enumerate}

(3) In sub-paragraph ($c$) of paragraph (2)—
\begin{enumerate}\item[]
(i) at the end of head (iii) the word “and” shall be inserted; and

(ii) head (v) and the word “; and” preceding it shall be omitted.
\end{enumerate}

\subsection[45. Amendment of regulation 10 of the Maintenance Assessments and Special Cases Regulations]{Amendment of regulation 10 of the Maintenance Assessments and Special Cases Regulations}

45.  In regulation 10 of the Maintenance Assessments and Special Cases Regulations (exempt income of parent with care)\footnote{\frenchspacing Regulation 10 was amended by regulation 21 of S.I. 1993/913.} for the words from “except” to the end of the regulation there shall be substituted the words “expect that—
\begin{enumerate}\item[]
($a$) sub-paragraph ($bb$) of paragraph (1) of that regulation shall not apply unless at the time of the making of the qualifying transfer the parent with care would have been the absent parent had the Child Support Act 1991 been in force at the date of the making of the transfer; and

($b$) paragraphs (3) and (4) of that regulation shall apply only where the parent with care shares day to day care of the child mentioned in those paragraphs with one or more other persons.”.
\end{enumerate}

\subsection[46. Amendment of regulation 11 of the Maintenance Assessments and Special Cases Regulations]{Amendment of regulation 11 of the Maintenance Assessments and Special Cases Regulations}

46.—(1) Regulation 11 of the Maintenance Assessments and Special Cases Regulations (protected income) shall be amended in accordance with the following provisions of this regulation.

(2) In paragraph (1)—
\begin{enumerate}\item[]
($a$) for the words “subject to paragraphs (3) and (4),” there shall be substituted the words “subject to paragraphs (3), (4) and (6),”;

($b$) in sub-paragraph ($b$), for the words “regulation 15(5),” there shall be substituted the words “paragraphs (1), (2) and (9) of regulation 63 of the Housing Benefit Regulations (non-dependant deductions) if he were a non-dependant in respect of whom a calculation were to be made under those paragraphs (disregarding any other provision of that regulation)”;

($c$) for sub-paragraph ($j$) there shall be substituted the following sub-paragraph—
\begin{quotation}
“($j$) where—
\begin{enumerate}\item[]
(i) the absent parent is, or that absent parent and any partner of his are, the only person or persons resident in, and liable to pay council tax in respect of, the home of which housing costs are included under sub-paragraph ($b$), the amount of weekly council tax for which he is liable in respect of that home, less any applicable council tax benefit;

(ii) where other persons are resident with the absent parent in, and liable to pay council tax in respect of, the home for which housing costs are included under sub-paragraph ($b$), an amount representing the share of the weekly council tax in respect of that home applicable to the absent parent, determined by dividing the total amount of council tax due in that week by the number of persons liable to pay it, less any council tax benefit applicable to that share, provided that, if the absent parent is required to pay and pays more than that share because of default by one or more of those other persons, the amount of the purposes of this regulation shall be the amount of weekly council tax the absent parent pays, less any council tax benefit applicable to such amount;”
\end{enumerate}
\end{quotation}

($d$) after sub-paragraph ($k$) there shall be inserted the following sub-paragraph—
\begin{quotation}
“($kk$) an amount in respect of travelling costs determined in accordance with Schedule 3B;”; and
\end{quotation}

($e$) in sub-paragraph ($l$) for the words “($a$) to ($k$)” there shall be substituted the words “($a$) to ($kk$)”.
\end{enumerate}

(3) At the end of head (ii) of sub-paragraph ($a$) of paragraph (2), the word “and” shall be omitted.

(4) In paragraph (2), after head (ii) of sub-paragraph ($a$), there shall be inserted the following head—
\begin{quotation}
“(iii) to the extent that it falls under sub-paragraph ($b$), the income of any child in that family shall not be treated as the income of the parent or his partner and Part IV of Schedule 1 shall not apply; and”.
\end{quotation}

(5) In sub-paragraph ($b$) of paragraph (2) for the words “that child’s income” there shall be substituted the words “that child’s relevant income (within the meaning of paragraph 23 of Schedule 1), there being disregarded any maintenance in payment to or in respect of him,”.

(6) After paragraph (5) there shall be added the following paragraphs—
\begin{quotation}
“(6) If the application of the above provisions of this regulation would result in the protected income level of an absent parent being less than 70 per centum of his net income, as calculated in accordance with regulation 7, those provisions shall not apply in his case and instead his protected income level shall be 70 per centum of his net income as so calculated.

(7) Where any calculation under paragraph (6) results in a fraction of a penny, that fraction shall be treated as a penny.”.
\end{quotation}

\subsection[47. Amendment of regulation 12 of the Maintenance Assessments and Special Cases Regulations]{Amendment of regulation 12 of the Maintenance Assessments and Special Cases Regulations}

47.  For paragraph (1) of regulation 12 of the Maintenance Assessments and Special Cases Regulations (disposable income), there shall be substituted the following paragraph—
\begin{quotation}
“(1) For the purposes of paragraph 6(4) of Schedule 1 to the Act (protected income), the disposable income of an absent parent shall be—
\begin{enumerate}\item[]
($a$) except in a case to which regulation 11(6) applies, the aggregate of his income and any income of any member of his family calculated in like manner as under regulation 11(2); and

($b$) in a case to which regulation 11(6) applies, his net income as calculated in accordance with regulation 7.”.
\end{enumerate}
\end{quotation}

\subsection[48. Amendment of regulation 15 of the Maintenance Assessments and Special Cases Regulations]{Amendment of regulation 15 of the Maintenance Assessments and Special Cases Regulations}

48.—(1) Regulation 15 of the Maintenance Assessments and Special Cases Regulations\footnote{\frenchspacing There are amendments to regulation 15 which are not relevant for the purposes of this instrument.} (amount of housing costs) shall be amended in accordance with the following provisions of this regulation.

(2) In paragraph (1) for the words “regulations 16 to 18” there shall be substituted the words “regulations 16 and 18”. 

(3) Paragraphs (4) to (9) shall be omitted.

(4) For paragraph (10) there shall be substituted the following paragraph—
\begin{quotation}
“(10) A parent shall be treated as having no housing costs where he is a non-dependant member of a household and is not responsible for meeting housing costs except to another member, or other members, of that household.”.
\end{quotation}

\subsection[49. Amendment of regulation 16 of the Maintenance Assessments and Special Cases Regulations]{Amendment of regulation 16 of the Maintenance Assessments and Special Cases Regulations}

49.  In regulation 16 of the Maintenance Assessments and Special Cases Regulations, after paragraph ($b$) there shall be inserted the following paragraph—
\begin{quotation}
“($bb$) by way of rent payable to a housing association, as defined in section 1(1) of the Housing Associations Act 1985\footnote{\frenchspacing 1985 c. 69.}, which is registered in accordance with section 5 of that Act, or to a local authority, on a free week basis, that is to say the basis that he pays an amount by way of rent for a given number of weeks in a 52 week period, with a lesser number of weeks in which there is no liability to pay (“free weeks”), the amount of such housing costs shall be the amount which he pays---
\begin{enumerate}\item[]
(i) in the relevant week if it is not a free week; or

(ii) in the last week before the relevant week which is not a free week, if the relevant week is a free week;”.
\end{enumerate}
\end{quotation}

\subsection[50. Revocation of regulation 17 of the Maintenance Assessments and Special Cases Regulations]{\sloppy Revocation of regulation 17 of the Maintenance Assessments and Special Cases Regulations}

50.  Regulation 17 of the Maintenance Assessments and Special Cases Regulations is hereby revoked.

\subsection[51. Amendment of regulation 22 of the Maintenance Assessments and Special Cases Regulations]{Amendment of regulation 22 of the Maintenance Assessments and Special Cases Regulations}

51.  For paragraph (2) of regulation 22 of the Maintenance Assessments and Special Cases Regulations there shall be substituted the following paragraph—
\begin{quotation}
“(2) For the purposes of assessing the amount of child support maintenance payable in respect of each application where paragraph (1) applies, for references to the assessable income of an absent parent in the Act and in these Regulations there shall be substituted references to the amount calculated by the formula—
\[ \left( (\mathrm{A} + \mathrm{T}) \times \frac{\mathrm{B}}{\mathrm{D}}\right) - \mathrm{CS}\]
where—
\begin{enumerate}\item[]
A is the absent parent’s assessable income;

T is the sum of the amounts allowable in the calculation or estimation of his exempt income by virtue of Schedule 3A;

B is the maintenance requirement calculated in respect of the application in question;

D is the sum of the maintenance requirements as calculated for the purposes of each assessment relating to the absent parent in question; and

CS is the amount (if any) allowable by virtue of Schedule 3A in calculating or estimating the absent parent’s exempt income in respect of a relevant qualifying transfer of property in respect of the assessment in question.”
\end{enumerate}
\end{quotation}

\subsection[52. Amendment of Regulation 25 of the Maintenance Assessments and Special Cases Regulations]{Amendment of Regulation 25 of the Maintenance Assessments and Special Cases Regulations}

52.—(1) Regulation 25 of the Maintenance Assessments and Special Cases Regulations (care provided in part by a local authority) shall be amended in accordance with the following provisions of this regulation.

(2) In paragraph (2), for the words “In a case where this regulation applies” there shall be substituted the words “Subject to paragraph (3), in a case where this regulation applies”.

(3) After paragraph (2), there shall be added the following paragraph—
\begin{quotation}
“(3) In a case where more than one qualifying child is included in a child support maintenance assessment application and where this regulation applies to at least one of those children, child support maintenance shall be a calculated by applying the formula—
\[ \mathrm{S} \times \left( \frac{\mathrm{A}}{7 \times \mathrm{B}} \right)  \]
where—
\begin{enumerate}\item[]
S is the total amount of child support maintenance in respect of all qualifying children included in that maintenance assessment application, calculated as if this regulation did not apply;

A is the aggregate of the number of nights of day to day care for all qualifying children included in that maintenance assessment application provided in each week by a person other than the local authority;

B is the number of qualifying children in respect of whom the maintenance assessment application has been made.”.
\end{enumerate}
\end{quotation}

\subsection[53. Amendment of regulation 26 of the Maintenance Assessments and Special Cases Regulations]{Amendment of regulation 26 of the Maintenance Assessments and Special Cases Regulations}

53.  In head (ii) of sub-paragraph ($b$) of paragraph (1) of regulation 26 of the Maintenance Assessments and Special Cases Regulations, after the word “estimating” there shall be inserted the words “under paragraphs (1) to (5) of regulation 11,”.

\subsection[54. Amendment of Schedule 1 to the Maintenance Assessments and Special Cases Regulations]{Amendment of Schedule 1 to the Maintenance Assessments and Special Cases Regulations}

54.—(1) Schedule 1 to the Maintenance Assessments and Special Cases Regulations (Calculation of N and M)\footnote{\frenchspacing Schedule 1 has been amended: the relevant amending instrument is S.I. 1993/913.} shall be amended in accordance with the following provisions of this regulation.

(2) In head ($a$) of sub-paragraph (1) of paragraph 1 after the words “bonus, commission,” there shall be inserted the words “payment in respect of overtime,”.

(3) In sub-paragraph (3) of paragraph 1—
\begin{enumerate}\item[]
($a$) in head ($a$)(ii), after the words “Contributions and Benefits Act” there shall be inserted the words “or under the Social Security Contributions and Benefits (Northern Ireland) Act 1992”;

($b$) in head ($b$) for the words “occupational or personal pension scheme” there shall be substituted the words “occupational pension scheme”; and

($c$) after head ($b$) there shall be inserted the following head—
\begin{quotation}
“($c$) one half of any sums paid by the parent towards a personal pension scheme, or, where that scheme is intended partly to provide a capital sum to discharge a mortgage secured upon the parent’s home, 37.5 per centum of any such sums.”.
\end{quotation}
\end{enumerate}

(4) For sub-paragraph (1) of paragraph 2 (calculation of income of employed earners) there shall be substituted the following sub-paragraph—
\begin{quotation}
“(1) Subject to sub-paragraphs (2) to (4), the amount of the earnings to be taken into account for the purpose of calculating N and M shall be calculated or estimated by reference to the average earnings at the relevant week having regard to such evidence as is available in relation to that person’s earnings during such period as appears appropriate to the child support officer beginning not earlier than eight weeks before the relevant week and ending not later than the date of the assessment and for the purpose of that calculation or estimate he may consider evidence of that person’s cumulative earnings during the period beginning with the start of the year of assessment (within the meaning of section 832 of the Income and Corporation Taxes Act 1988\footnote{\frenchspacing 1988 c. 1.}) in which the relevant week falls and ending with a date no later than the date of the assessment.”.
\end{quotation}

(5) In sub-paragraph (3) of paragraph 3 (earnings of self-employed earner)—
\begin{enumerate}\item[]
($a$) in head ($b$) after the words “paragraph 4” there shall be inserted the words “or 5(2)”;

($b$) in head ($e$) after the words “pension scheme” there shall be inserted the words “, or, where that scheme is intended partly to provide a capital sum to discharge a mortgage or charge secured upon the parent’s home, 37.5 per centum of the contributions payable”.
\end{enumerate}

(6) For sub-paragraph (5) of paragraph 3, there shall be substituted the following sub-paragraph—
\begin{quotation}
“(5) For the purposes of sub-paragraph (3)($c$), the amount of income tax to be allowed against earnings shall be calculated—
\begin{enumerate}\item[]
($a$) where the earnings are determined over a period of 12 months on the basis of chargeable earnings and as if those earnings, less any personal allowance applicable to the earner under Chapter I of Part VII of the Income and Corporation Taxes Act 1988 (Personal Relief), were assessable to income tax at the rates of tax applicable at the effective date; or

($b$) where the earnings are determined over a period of less than 12 months on the basis of chargeable earnings and as if those earnings, less a proportionate part of any personal allowance applicable to the earner under Chapter I of Part VII of the Income and Corporation Taxes Act 1988 (Personal Relief), were assessable to income tax at the rates of tax applicable at the effective date, but the amount of the earnings to which each tax rate applies shall be determined on the basis that the ratio of that amount to the full amount of chargeable earnings to which each tax rate applies is the same as the ratio of the proportionate part of the personal allowance to the full personal allowance.”.
\end{enumerate}
\end{quotation}

(7) In sub-paragraph (6)($a$) of paragraph 3, for “(4)” there shall be substituted “(3)”.

(8) For sub-paragraph (7) of paragraph 3, there shall be substituted the following sub-paragraph—
\begin{quotation}
“(7) In the case of a self-employed earner whose employment is carried on in partnership or is that of a share fisherman within the meaning of the Social Security (Mariners' Benefits) Regulations 1975\footnote{\frenchspacing S.I. 1975/470.}, sub-paragraph (3) shall have effect as though it requires—
\begin{enumerate}\item[]
($a$) a deduction from the earner’s estimated or, where appropriate, actual share of the gross receipts of the partnership or fishing boat, of his share of the sums likely to be deducted or, where appropriate, deducted from those gross receipts under heads ($a$) and ($b$) of that sub-paragraph; and

($b$) a deduction from the amount so calculated of the sums mentioned in heads ($c$) to ($e$) of that sub-paragraph.”.
\end{enumerate}
\end{quotation}

(9) Paragraph 5 (calculation of income of self-employed earner) shall be amended as follows—
\begin{enumerate}\item[]
($a$) in sub-paragraph (1) for the words “(2) and (3)” there shall be substituted the words “(2) to (3)”;

($b$) in sub-paragraph (2)—
\begin{enumerate}\item[]
(i) at the beginning for the word “Where” there shall be substituted the words “Subject to sub-paragraph (2A), where”;

(ii) for the words “12 months” there shall be substituted the words “24 months”; and
\end{enumerate}

($c$) after sub-paragraph (2) there shall be inserted the following sub-paragraph—
\begin{quotation}
“(2A) Where the child support officer is satisfied that, in relation to the person referred to in sub-paragraph (2) there is more than one profit and loss account, each in respect of different periods, both or all of which satisfy the conditions mentioned in that sub-paragraph, the provisions of that sub-paragraph shall apply only to the account which relates to the latest such period, unless the officer is satisfied that the latest such account is not available for reasons beyond the control of that person, in which case he may have regard to any such other account which satisfies the requirements of that sub-paragraph.”.
\end{quotation}
\end{enumerate}

(10) In paragraph 20, in sub-paragraph ($b$) for the words “three times” there shall be substituted the words “one-and-a-half times”.

(11) In paragraph 23 after the words “that child” there shall be added the following sub-paragraph—
\begin{quotation}
“;

($e$) the first £10 of any other income of that child”.
\end{quotation}

\subsection[55.Amendment of Schedule 2 to the Maintenance Assessments and Special Cases Regulations]{Amendment of Schedule 2 to the Maintenance Assessments and Special Cases Regulations}

55.—(1) Schedule 2 to the Maintenance Assessments and Special Cases Regulations shall be amended in accordance with the following provisions of this regulation.

(2) In paragraph 24, for the words “Where a parent provides”, there shall be substituted the words “For each week in which a parent provides”.

(3) In sub-paragraph ($a$) of paragraph 24, after the word “by” there shall be inserted the words, “, on behalf or in respect of”.

(4) In paragraph 27, for the words “does not in any period exceed” there shall be substituted the word “exceeds”.

\subsection[56. Amendment of Schedule 3 to the Maintenance Assessments and Special Cases Regulations]{Amendment of Schedule 3 to the Maintenance Assessments and Special Cases Regulations}

56.—(1) Schedule 3 to the Maintenance Assessments and Special Cases Regulations shall be amended in accordance with the following provisions of this regulation.

(2) In sub-paragraph ($t$) of paragraph 1 (eligible housing costs for purpose of determining exempt income and protected income) the words from “and only to the extent to which” to the end of the paragraph shall be omitted.

(3) In paragraph 2 (loans for repairs and improvements to the home) for the words “For the purposes of” there shall be substituted the words “Subject to paragraph 2A (loans for repairs and improvements in transitional cases), for the purposes of”.

(4) After paragraph 2 there shall be inserted the following paragraph—
\begin{quotation}
\subsection*{\sloppy “Loans for repairs and improvements in transitional cases}

2A.  In the case of a loan entered into before the first date upon which a maintenance application or enquiry form is given or sent or treated as given or sent to the relevant person, for the purposes of paragraph 1($d$) “repairs and improvements” means repairs and improvements of any description whatsoever.”.
\end{quotation}

(5) Paragraph 3 shall be amended—
\begin{enumerate}\item[]
($a$) by adding at the end of sub-paragraph (4) the words “including for the avoidance of doubt such a policy of insurance whose purpose is to secure the payment of monies due under the mortgage or charge in the event of the unemployment, sickness or disability of the insured.”;

($b$) by inserting, after sub-paragraph (5)\footnote{\frenchspacing Sub-paragraph (5) was substituted by regulation 4(8) of S.I. 1994/227.} the following sub-\hspace{0pt}paragraphs—
\begin{quotation}
“(5A) Where a plan within the meaning of regulation 4 of the Personal Equity Plans Regulations 1989\footnote{\frenchspacing S.I. 1989/469; relevant amendments were made by S.I. 1990/678 and 1991/733.} has been obtained and retained for the purpose of discharging a mortgage or charge on the home of the parent in question and also for the purpose of accruing profits upon the realisation of the plan, there shall be eligible to be taken into account as a housing cost—
\begin{enumerate}\item[]
($a$) where the sum secured by the mortgage or charge does not exceed £60,000, the whole of the premiums payable in respect of the plan; and

($b$) where the sum secured by the mortgage or charge exceeds £60,000, that part of the premiums payable in respect of the plan which is necessarily incurred for the purpose of discharging the mortgage or charge or, where that part cannot be ascertained, 0.0277 per centum of the amount secured by the mortgage or charge.
\end{enumerate}

(5B) Where a personal pension plan has been obtained and retained for the purpose of discharging a mortgage or charge on the home of the parent in question and also for the purpose of securing the payment of a pension to him, there shall be eligible to be taken into account as a housing cost 25 per centum of the contributions payable in respect of that personal pension plan.”;
\end{quotation}

($c$) in sub-paragraph (6) there shall be substituted—
\begin{enumerate}\item[]
(i) in head ($a$) for the words “any payment of arrears or any payments in excess of those required” the words “any payments in excess of those required”;

(ii) in head ($b$) for the words “they are attributable to arrears or would otherwise not be eligible” the words “they would not be eligible”.
\end{enumerate}
\end{enumerate}

(6) Paragraph 6 shall be amended as follows—
\begin{enumerate}\item[]
($a$) in sub-paragraph ($a$) for the words “paragraph 1” there shall be substituted the words “paragraph 1($a$)(i)”;

($b$) after sub-paragraph ($a$) there shall be inserted the following sub-paragraph—
\begin{quotation}
“($aa$) where the costs are inclusive of charges, other than those which are not to be included by virtue of sub-paragraph ($a$), that part of those charges which exceeds the greater of the following amounts—
\begin{enumerate}\item[]
(i) the total of the charges other than those which are ineligible service charges within the meaning of paragraph 1 of Schedule 1 to the Housing Benefit Regulations (housing costs);

(ii) 25 per centum of the total amount of eligible housing costs;”.
\end{enumerate}
\end{quotation}
\end{enumerate}

\subsection[57. Insertion of Schedules 3A and 3B into the Maintenance Assessments and Special Cases Regulations]{Insertion of Schedules 3A and 3B into the Maintenance Assessments and Special Cases Regulations}

57.  After Schedule 3 to the Maintenance Assessments and Special Cases Regulations there shall be inserted, as Schedules 3A and 3B, the Schedules set out in Schedules 1 and 2 respectively to these Regulations.

\subsection[58. Amendment of Schedule 4 to the Maintenance Assessments and Special Cases Regulations]{Amendment of Schedule 4 to the Maintenance Assessments and Special Cases Regulations}

58.  In paragraph ($a$) of Schedule 4 to the Maintenance Assessments and Special Cases Regulations—
\begin{enumerate}\item[]
($a$) for sub-paragraphs (i) to (iii), there shall be substituted the following sub-paragraphs—
\begin{quotation}
“(i) incapacity benefit under section 30A\footnote{\frenchspacing Section 30A was inserted by section 1 of, and sections 40 and 41 substituted by paragraphs 8 and 9 of Schedule 1 to, the Social Security (Incapacity for Work) Act 1994 (c. 18).};

(ii) long-term incapacity benefit for widows under section 40;

(iii) long-term incapacity benefit for widowers under section 41;”; and
\end{quotation}

($b$) sub-paragraph (v) shall be omitted.
\end{enumerate}

\subsection[59. Amendment of Schedule 5 to the Maintenance Assessments and Special Cases Regulations]{Amendment of Schedule 5 to the Maintenance Assessments and Special Cases Regulations}

59.—(1) Schedule 5 to the Maintenance Assessments and Special Cases Regulations shall be amended in accordance with the following provisions of this regulation.

(2) For paragraph 2, there shall be substituted the following paragraph—
\begin{quotation}
“2.  A relevant decision may be reviewed by a child support officer, either on application by a relevant person or of his own motion—
\begin{enumerate}\item[]
($a$) if it appears to him that the absent parent has at some time after that decision was given satisfied the conditions prescribed by regulation 28(1) or, as the case may be, no longer satisfies those conditions; or

($b$) if it appears to him that the relevant decision was wrong in law or was made in ignorance of, or based on a mistake as to, a material fact.”.
\end{enumerate}
\end{quotation}

(3) In paragraph 3, after the word “decision” there shall be inserted the words “made on or before 18th April 1994” and for the word “when” there shall be substituted the word “after”.

(4) After paragraph 3, there shall be inserted the following paragraph—
\begin{quotation}
“3A.  A relevant decision made after 18th April 1994 shall be reviewed by a child support officer after it has been in force for 104 weeks.”.
\end{quotation}

\subsection[60. Amendment of regulation 7 of the Miscellaneous Amendments Regulations]{Amendment of regulation 7 of the Miscellaneous Amendments Regulations}

60.  In sub-paragraph ($a$) of paragraph (2) of regulation 7 of the Miscellaneous Amendments Regulations (scope of Part III), after the words “Category A” there shall be inserted the words “or Category D”.

\subsection[61. Amendment of regulation 11 of the Miscellaneous Amendments Regulations]{\sloppy Amendment of regulation 11 of the Miscellaneous Amendments Regulations}

61.—(1) Regulation 11 of the Miscellaneous Amendments Regulations (reviews on change of circumstances) shall be amended in accordance with the following provision of this regulation.

(2) For the heading, there shall be substituted the following heading—
\begin{quotation}
\subsection*{“Reviews”.}
\end{quotation}

(3) For paragraph (1), there shall be substituted the following paragraph—
\begin{quotation}
“(1) The provisions of the following paragraphs shall apply where there is a review of a previous assessment under section 17, 18 or 19 of the Act (reviews) at any time when the amount payable under that assessment is the transitional amount.”.
\end{quotation}

(4) In paragraph (3), after the words “the reviewed formula amount” there shall be inserted the words “on a review under section 17 of the Act”.

(5) After paragraph (3), there shall be added the following paragraph—
\begin{quotation}
“(4) Where a child support officer makes a fresh maintenance assessment following a review under section 18 or 19 of the Act, the effective date of that fresh maintenance assessment shall be the date prescribed under regulation 31 of the Maintenance Assessment Procedure Regulations or the first day of the maintenance period following 18th April 1995, whichever is the later.”.
\end{quotation}

\subsection[62. Amendment of Schedule 3 to the Income Support (General) Regulations 1987]{\sloppy Amendment of Schedule 3 to the Income Support (General) Regulations 1987}

62.  Schedule 3 of the Income Support (General) Regulations 1987\footnote{\frenchspacing S.I. 1987/1967. Relevant amending instruments are S.I. 1988/663 and S.I. 1989/1678.} shall be amended in the following respects—
\begin{enumerate}\item[]
($a$) in sub-paragraph 7(9) at the beginning for the words “Subject to sub-paragraphs (10) and (11)” there shall be substituted the words “Subject to sub-paragraphs (10) to (12)”;

($b$) after sub-paragraph 7(11) the following sub-paragraph shall be added—
\begin{quotation}
“(12) Where a claimant, with the care of a child, has ceased to be in receipt of income support in consequence of the payment of child support maintenance under the Child Support Act 1991\footnote{\frenchspacing 1991 c. 48.} and immediately before ceasing to be so in receipt an amount under sub-paragraph (1)($b$)(i) was applicable to him, then—
\begin{enumerate}\item[]
($a$) if the child support maintenance assessment concerned is terminated or replaced on review by a lower assessment in consequence of the coming into force on or after 18th April 1995 of regulations made under the Child Support Act 1991; or

($b$) where the child support maintenance assessment concerned is an interim maintenance assessment and, in circumstances other than those referred to in sub-paragraph ($a$), it is terminated or replaced after termination by another interim maintenance assessment or by a maintenance assessment made in accordance with Part I of Schedule 1 to the Child Support Act 1991, in either case of a lower amount than the assessment concerned,
\end{enumerate}
sub-paragraph (9)($a$)(ii) shall apply to him as if for the words “any period of eight weeks or less” there were substituted the words “any period of 26 weeks or less”.”
\end{quotation}
\end{enumerate}

\subsection[63. Reviews consequent upon the amendments made by these regulations]{Reviews consequent upon the amendments made by these regulations}

63.—(1) 
%Subject to paragraph (3), where a child support officer reviews a maintenance assessment in consequence only of the amendments specified in paragraph (2) he shall not make a fresh assessment if the difference between the amount of child support maintenance fixed by the assessment currently in force and the amount that would be fixed if the fresh assessment were to be made as a result of the review is—
Subject to paragraph (3), a decision with respect to a maintenance assessment in force on 13th April 1995 or 18th April 1995 shall not be superseded by a decision under section 17 of the Act if the difference between the amount of child support maintenance currently in force and the amount that would be fixed if the fresh assessment were to be made as a result of a supersession is—  % Words substituted (1.6.99) by SI 1999/1510 reg 37(a)
\begin{enumerate}\item[]
($a$) less than £1.00 per week where the amount fixed by the assessment currently in force is more than the amount that would be fixed by the fresh assessment; or

($b$) less than £10.00 per week in all other cases.
\end{enumerate}

(2) Paragraph (1) applies to the following provisions—
\begin{enumerate}\item[]
($a$) regulation 28(8);

($b$) regulation 43;

($c$) regulation 44(2);

($d$) regulation 45;

($e$) regulation 46(2)($d$) and ($e$), (4) and (6);

($f$) regulation 47;

($g$) regulation 50;

($h$) regulation 51;

($i$) regulation 54(10) and (11).
\end{enumerate}

(3) Paragraph (1) shall not apply to a 
%review which is made 
decision under section 17 of the Child Support Act 1991 which falls to be made  % Words substituted (1.6.99) by SI 1999/1510 reg 37(b)(i)
in consequence only of the amendments made by regulations 44(2), 45, 46(2)($d$) and ($e$) and 51 unless the person to whom the assessment relates 
%notifies 
notified  % Word substituted (1.6.99) by SI 1999/1510 reg 37(b)(ii)
the Secretary of State before 18th July 1995 that he wishes a child support officer to consider whether the assessment in his case should be reviewed; but the Secretary of State may accept a later notification for the purposes of this paragraph if he is satisfied that there is good cause for the delay in giving it.

% Reg 63(4), (5) omitted (1.6.99) by SI 1999/1510 reg 37(c)
%(4) For the purposes of regulation 17(2) (intervals between periodical reviews and notice of a periodical review) and 31 (effective date of maintenance assessments following a review under section 16 to 19 of the Act) of the Maintenance Assessment Procedure Regulations, a review such as is mentioned in paragraph (1) shall be disregarded.
%
%(5) Notwithstanding any provision of regulation 31 of the Maintenance Assessment Procedure Regulations except, in the cases to which they apply, paragraphs (6A) to (6C)\footnote{\frenchspacing These paragraphs are inserted by regulation 37(4) supra.} the effective date of a maintenance assessment made on such a review as is mentioned in paragraph (1) shall be 18th April 1995.

(6) Where a maintenance assessment is in force on 18th April 1995 and—
\begin{enumerate}\item[]
($a$) the relevant person notifies the Secretary of State on or after 18th July 1995 that he wishes 
%a child support officer to consider the question 
the question to be considered  % Words substituted (1.6.99) by SI 1999/1510 reg 37(d)(i)
of whether an amount should be allowed in the computation of the relevant person’s exempt income or protected income in respect of travelling costs or his exempt income in respect of a qualifying transfer of property; and

($b$) the Secretary of State is not satisfied that there was good cause for the delay on the part of the relevant person in giving the notification,
\end{enumerate}
the effective date of any assessment made 
%upon a review under section 19 of the Act 
by virtue of a decision under section 17 of the Act superseding an earlier decision  % Words substituted (1.6.99) by SI 1999/1510 reg 37(d)(ii)
shall be the first day of the maintenance period in which the Secretary of State is so notified.

\amendment{
Words substituted in reg. 63(1), (3), (6) and reg. 63(4), (5) omitted (1.6.99) by the Social Security Act 1998 (Commencement No. 7 and Consequential and Transitional Provisions) Order 1999 reg. 37.
}

\subsection[64. Transitional Provisions]{Transitional Provisions}

64.—(1) Where a maintenance assessment, other than an interim maintenance assessment, is in force on 18th April 1995 
%or on that date there is in force a decision of a child support officer under section 43 of the Act (contribution to maintenance by deduction from benefit) and that decision or 
and  % Words substituted (22.1.96) by SI 1995/3261 reg 50(1)
the amount of child support payable under that assessment would be affected by the provisions of these Regulations, only the provisions mentioned in paragraph (2) and (3) shall apply to that assessment until that assessment is reviewed under section 16, 17 or 18 of the Act.

(2) The provisions of these Regulations to which paragraph (1) refers are—
\begin{enumerate}\item[]
($a$) regulation 34;

($b$) regulation 43;

($c$) regulation 46(6);

($d$) regulation 47;

($e$) regulation 50;

($f$) regulation 54(10) and (11).

%($g$) regulation 59.  % Reg 64(2)($g$) omitted (22.1.96) by SI 1995/3261 reg 50(2)
\end{enumerate}

(3) The provisions of regulations 44(2), 45, 46(2)($d$) and ($e$) and 51 and Schedules 1 and 2 to these Regulations shall not apply in a case where there is a maintenance assessment in force on the 18th April 1995 until 
%a child support officer considers the question whether that assessment should be reviewed under section 19 of the Act as a result of the relevant person having notified the Secretary of State that he wishes a child support officer to consider that question because of the making of a qualifying transfer of property or because he has travelling costs.
a relevant person applies for a decision under section 17 of the Child Support Act 1991 superseding an earlier decision on the ground that a qualifying transfer of property has been made or that he has travelling costs.  % Words substituted (1.6.99) by SI 1999/1510 reg 38(a)

(4) Where on 18th April 1995 in any particular case there is in force a maintenance assessment which is subject to an adjustment made under the provisions of regulation 10 of the Arrears Regulations as in force prior to that date that adjustment shall continue until whichever is the earlier of—
\begin{enumerate}\item[]
($a$) 
%a review of 
a decision under section 17 of the Child Support Act 1991 superseding a decision with respect to  % Words substituted (1.6.99) by SI 1999/1510 reg 38(b)(i)
that assessment on grounds other than the coming into force of these Regulations; or

%($b$) a decision made by a child support officer on a request by a relevant person for reconsideration of that adjustment.

% Reg 64(4)(b) substituted (1.6.99) by SI 1999/1510 reg 38(b)(ii)
($b$) a decision under regulation 13 or 16 of the Arrears Regulations is made on an application made by a relevant person.
\end{enumerate}

(5) Regulations 12 and 13 shall not apply to a case in which there is an existing assessment until the Secretary of State first reviews the period by reference to which payments are to be made after these Regulations come into force.

\amendment{
Words substituted in reg. 64(1) and reg. 64(2)($g$) omitted (22.1.96) by the Child Support (Miscellaneous Amendments) (No. 2) Regulations 1995 reg. 50.

Words substituted in reg. 64(3), (4)(a) and reg. 64(4)(b) substituted (1.6.99) by the Social Security Act 1998 (Commencement No. 7 and Consequential and Transitional Provisions) Order 1999 reg. 38.
}

\bigskip

Signed by authority of the Secretary of State for Social Security.

{\raggedleft
\emph{Alistair Burt}\\*Parliamentary Under-Secretary of State,\\*Department of Social Security

}

10th April 1995

\clearpage

\part*{S C H E D U L E S}

\part[Schedule 1 --- Schedule to be inserted into the Maintenance Assessments and Special Cases Regulations as Schedule 3A to those regulations]{Schedule 1\\*Schedule to be inserted into the Maintenance Assessments and Special Cases Regulations as Schedule 3A to those regulations}

\renewcommand\parthead{--- Schedule 1}

\begin{quotation}
\part*{``Schedule 3A\\*Amount to be allowed in respect of transfer of property}

\subsection*{Interpretation}

1.—(1) In this Schedule—
\begin{enumerate}\item[]
“property” means—
\begin{enumerate}\item[]
($a$) a legal estate or an equitable interest in land; or

($b$) a sum of money which is derived from or represents capital, whether in cash or in the form of a deposit with—
\begin{enumerate}\item[]
(i) the Bank of England;

(ii) an authorised institution or an exempted person within the meaning of the Banking Act 1987\footnote{\frenchspacing 1987 c. 22.};

(iii) a building society incorporated or deemed to be incorporated under the Building Societies Act 1986\footnote{\frenchspacing 1986 c. 53.};
\end{enumerate}

\begin{sloppypar}
($c$) any business asset as defined in sub-paragraph (2) (whether in the form of money or an interest in land or otherwise);
\end{sloppypar}

($d$) any policy of insurance which has been obtained and retained for the purpose of providing a capital sum to discharge a mortgage or charge secured upon an estate or interest in land which is also the subject of the transfer (in this schedule referred to as an endowment policy);
\end{enumerate}

“qualifying transfer” means a transfer of property—
\begin{enumerate}\item[]
($a$) which was made in pursuance of a court order made, or a written maintenance agreement executed, before 5th April 1993;

($b$) which was made between the absent parent and either the parent with care or a relevant child;

($c$) which was made at a time when the absent parent and the parent with care were living separate and apart;

($d$) the effect of which is that the parent with care or a relevant child is beneficially entitled (subject to any mortgage or charge) to the whole of the asset transferred; and 

($e$) which was not made expressly for the purpose only of compensating the parent with care for the loss of any right to apply for or receive periodical payments or a capital sum in respect of herself;
\end{enumerate}

“compensating transfer” means a transfer of property which would be a qualifying transfer (disregarding the requirement of paragraph ($e$) of the definition of “qualifying transfer”) if it were made by the absent parent, but which is made by the parent with care in favour of the absent parent or a relevant child;

“relevant date” means the date of the making of the court order or the execution of the written maintenance agreement in pursuance of which the qualifying transfer was made.
\end{enumerate}

(2) For the purposes of sub-paragraph (1) “business asset” means an asset, whether in the form of money or an interest in land or otherwise which, prior to the date of transfer was use in the course of a trade or business carried on—
\begin{enumerate}\item[]
($a$) by the absent parent as a sole trader;

($b$) by the absent parent in partnership, whether with the parent with care or not;

($c$) by a close company within the meaning of sections 414 and 415 of the Income and Corporation Taxes Act 1988\footnote{\frenchspacing 1986 c. 1.} in which the absent parent was a participator at the date of the transfer.
\end{enumerate}

(3) Where the condition specified in regulation 10($a$) is satisfied this Schedule shall apply as if references—
\begin{enumerate}\item[]
($a$) to the parent with care were references to the absent parent; and

($b$) to the absent parent were references to the parent with care.
\end{enumerate}

\subsection*{\sloppy Evidence to be produced in connection with the allowance for transfers of property}

2.—(1) Where the absent parent produces to the Secretary of State—
\begin{enumerate}\item[]
($a$) contemporaneous evidence in writing of the making of a court order or of the execution of a written maintenance agreement, which requires the relevant person to make a qualifying transfer of property;

($b$) evidence in writing and whether contemporaneous or not as to—
\begin{enumerate}\item[]
(i) the fact of the transfer;

(ii) the value of the property transferred at the relevant date;

(iii) the amount of any mortgage or charge outstanding at the relevant date,
\end{enumerate}
\end{enumerate}
an amount in respect of the relevant value of the transfer determined in accordance with the following provisions of this Schedule shall be allowed in calculating or estimating the exempt income of the absent parent.

(2) Whether the evidence specified in sub-paragraph (1) is not produced within a reasonable time after the Secretary of State has been notified of the wish of the absent parent that a child support officer consider the question, the officer shall determine the question on the basis that the relevant value of the transfer is nil.

\subsection*{Consideration of evidence produced by other parent}

3.  Where an absent parent has notified the Secretary of State that he wishes a child support officer to consider whether an amount should be allowed in respect of the relevant value of a qualifying transfer, the Secretary of State shall give notice to the other parent that he proposes to refer the question to a child support officer for consideration and shall transmit to the child support officer any representations made by the other parent in considering the question.

\subsection*{\textls[25]{Computation of qualifying value—business assets} and land}

4.—(1) Subject to paragraph 6, where the property which is the subject of the transfer by the absent parent is, or includes an estate or interest in land, or a business asset, the qualifying value of that estate, interest or asset shall be determined in accordance with the formula—
\[\mathrm{QV} = \frac{\mathrm{VT} - \mathrm{MC}}{2}\]
where—
\begin{enumerate}\item[]
(i) QV is the qualifying value,

(ii) VT is the value of the estate or interest in land or the value of the asset (as the case may be) calculated at the relevant date, and

(iii) MC is the amount of the principal outstanding at the relevant date under any mortgage or charge on the estate, interest or asset.
\end{enumerate}

(2) For the purposes of sub-paragraph (1) the value of an estate or interest in land is to be determined upon the basis that the parent with care and any relevant child, if in occupation of the land, would quit on completion of the sale.

\subsection*{Computation of qualifying value—cash, deposits and endowment policies}

5.  Subject to paragraph 6, where the property which is the subject of the qualifying transfer is, or includes—
\begin{enumerate}\item[]
(i) a sum of money whether in cash or in the form of a deposit with the Bank of England, and authorised institution or exempted person within the meaning of the Banking Act 1987, or a building society incorporated or deemed to be incorporated under the Building Societies Act 1986, derived from or representing capital; or

(ii) an endowment policy,
\end{enumerate}
the amount of the qualifying value shall be determined by applying the formula—
\[ \mathrm{QV} = \frac{\mathrm{VT}}{2}\]
where—
\begin{enumerate}\item[]
QV is the qualifying value; and

VT is the amount of cash, the balance of the account or the surrender value of the endowment policy on the relevant date.
\end{enumerate}

\subsection*{Transfer wholly in lieu of periodical payments for relevant child}

6.  Where the evidence produced in relation to a transfer to, or in respect of, a relevant child, shows expressly that the whole of that transfer was made exclusively in lieu of periodical payments in respect of that child—
\begin{enumerate}\item[]
($a$) in a case to which paragraph 4 applies, for the formula given in that paragraph there shall be substituted the following—
\[\mathrm{QV} = \mathrm{VT} - \mathrm{MC};\]
and

($b$) in a case to which paragraph 5 applies, the qualifying value shall be the value of the transfer.
\end{enumerate}

\subsection*{Multiple transfers to related persons}

7.—(1) Where there has been more than one qualifying transfer from the absent parent—
\begin{enumerate}\item[]
($a$) to the same parent with care;

($b$) to or for the benefit of the same relevant child;

($c$) to or for the benefit of two or more relevant children with respect to all of whom the same persons are respectively the parent with care and the absent parent;
\end{enumerate}
or any combination thereof, the relevant value by reference to which the allowance is to be calculated in accordance with paragraph 10 shall be the aggregate of the qualifying transfers calculated individually in accordance with the preceding paragraphs of this Schedule, less the value of any compensating transfer or where there has been more than one, the aggregate of the values of the compensating transfers so calculated.

(2) Except as provided by sub-paragraph (1), the values of transfers shall not be aggregated for the purposes of this Schedule.

\subsection*{Computation of the value of compensation transfers}

8.  The value of a compensation transfer shall be determined in accordance with paragraph 4 to 7 above, but as if any reference in those paragraphs—
\begin{enumerate}\item[]
($a$) to the absent parent were a reference to the parent with care;

($b$) to the parent with care were a reference to the absent parent; and

($c$) to a qualifying transfer were a reference to a compensating transfer.
\end{enumerate}

\subsection*{\sloppy Computation of relevant value of a qualifying transfer}

9.  The relevant value of a qualifying transfer shall be calculated by deducting from the qualifying value of the qualifying transfer the qualifying value of any compensating transfer between the same persons as are parties to the qualifying transfer.

\subsection*{Amount to be allowed in respect of a qualifying transfer}

10.  For the purposes of regulation 9(1)($bb$), the amount to be allowed in the computation of E, or in the case where regulation 10($a$) applies, F, shall be—
\begin{enumerate}\item[]
($a$) where the relevant value calculated in accordance with paragraph 9 is less than £5,000, nil;

($b$) where the relevant value calculated in accordance with paragraph 9 is at least £5,000, but less than £10,000, £20.00 per week;

($c$) where the relevant value calculated in accordance with paragraph 9 is at least £10,000, but less than £25,000, £40.00 per week;

($d$) where the relevant value calculated in accordance with paragraph 9 is not less than £25,000, £60.00 per week.
\end{enumerate}

\medskip

11.  This Schedule in its application to Scotland shall have effect as if—
\begin{enumerate}\item[]
($a$) in paragraph 1 for the words “legal estate or equitable interest in land” there were substituted the words “an interest in land within the meaning of section 2(6) of the Conveyancing and Feudal Reform (Scotland) Act 1970\footnote{\frenchspacing 1970 c. 35.}”;

($b$) in paragraph 4 the word “estate” and the words “estate or” in each place where they respectively occur were omitted.''
\end{enumerate}
\end{quotation}

\part[Schedule 2 --- Schedule to be inserted into the Maintenance Assessments and Special Cases Regulations as Schedule 3B to those regulations]{Schedule 2\\*Schedule to be inserted into the Maintenance Assessments and Special Cases Regulations as Schedule 3B to those regulations}

\renewcommand\parthead{--- Schedule 2}

\begin{quotation}
\part*{``Schedule 3B\\*Amount to be allowed in respect of travelling costs}

\subsection*{Interpretation}

1.  In this Schedule—
\begin{enumerate}\item[]
“day” means, in relation to a person who attends at a work place for one period of work which commences before midnight of one day and concludes the following day, the first of those days;

“journey” means a single journey, and “pair of journeys” means two journeys in opposing directions, between the same two places;

“relevant employment” means an employed earner’s employment in which the relevant person is employed and in the course of which he is required to attend at a work place, and “relevant employer” means the employer of the relevant person in that employment;

“relevant person” means—
\begin{enumerate}\item[]
($a$) in the application of the provisions of this Schedule to regulation 9, the absent parent or the parent with care; and

($b$) in the application of the provisions of this Schedule to regulation 11, the absent parent;
\end{enumerate}

“straight-line distance” means the straight-line distance measured in miles and calculated to 2 decimal places, and, where that distance is not a whole number of miles, rounded to the nearest whole number of miles, a distance which exceeds a whole number of miles by 0.50 of a mile being rounded up;

“travelling costs” means the costs of—
\begin{enumerate}\item[]
($a$) purchasing either fuel or a ticket for the purpose of travel;

($b$) contributing to the costs borne by a person other than a relevant employer in providing transport; or

($c$) paying another to provide transport,
\end{enumerate}
which are incurred by the relevant person in travelling between the relevant person’s home and his work place, and where he has more than one relevant employment between any of his work places in those employments;

“work place” means the relevant person’s normal place of employment in a relevant employment, and “deemed work place” means a place which has been selected by the child support officer, pursuant either to paragraph 8(2) or 15(2) for the purpose of calculating the amount to be allowed in respect of the relevant person’s travelling costs.
\end{enumerate}

\subsection*{Computation of amount allowable in respect of travelling costs}

2.  For the purpose of regulation 9 and regulation 11 an amount in respect of the travelling costs of the relevant person shall be determined in accordance with the following provisions of this Schedule if the relevant person—
\begin{enumerate}\item[]
($a$) has travelling costs; and

($b$) provides the information required to enable the amount of the allowance to be determined.
\end{enumerate}

\subsection*{Computation in cases where there is one relevant employment and one work place in that employment}

3.  Subject to paragraphs 21 to 23, where the relevant persons has one relevant employment and is normally required to attend at only one work place in the course of that employment the amount to be allowed in respect of travelling costs shall be determined in accordance with paragraphs 4 to 7 below.

\medskip

4.  There shall be calculated or, if this is impracticable, estimated—
\begin{enumerate}\item[]
($a$) the straight-line distance between the relevant person’s home and his work place;

($b$) the number of journeys between the relevant person’s home and this work place which he makes during a period comprising a whole number of weeks which appears to the child support officer to be representative of his normal pattern of work, there being disregarded any pair of journeys between his work place and his home and where the first journey is from his work place to his home and where the time which elapses between the start of the first journey and the conclusion of the second is not more than two hours.
\end{enumerate}

\medskip

5.  The results of the calculation or estimate produced by sub-paragraph ($a$) of paragraph 4 shall be multiplied by the result of the calculation or estimate required by sub-paragraph ($b$) of that paragraph.

\medskip


6.  The product of the multiplication required by paragraph 5 shall be divided by the number of weeks in the period.

\medskip

7.  Where the result of the division required by paragraph 6 is less than or equal to 150, the amount to be allowed in respect of the relevant person’s travelling costs shall be nil, and where it is greater than 150 the weekly allowance to be made in respect of the relevant person’s travelling costs shall be 10 pence multiplied by the number by which that number exceeds 150.

\subsection*{Computation in cases where there is more than one work place but only one relevant employment}

8.—(1) Subject to sub-paragraph (2) and paragraphs 21 to 23 below, where the relevant person has one relevant employment but attends at more than one work place the amount to be allowed in respect of travelling costs for the purposes of regulations 9 and 11 shall be determined in accordance with paragraphs 9 to 13.

(2) Where it appears that the relevant person works at more than one work place but his pattern of work is not sufficiently regular to enable the calculation of the amounts to be allowed in respect of his travelling costs to be made readily, the child support officer may—
\begin{enumerate}\item[]
($a$) select a place which is either one of the relevant person’s work places or some other place which is connected with the relevant employment; and

($b$) apply the provisions of paragraphs 4 to 7 above to calculate the amount of the allowance to be made in respect of travelling costs upon the basis that the relevant person makes one journey from his home to the deemed work place and one journey from the deemed work place to home on each day on which he attends at a work place in connection with relevant employment,
\end{enumerate}
and the provision of paragraphs 9 to 13 shall not apply.

(3) For the purposes of sub-paragraph (2)($b$) there shall be disregarded any day upon which the relevant person attends at a work place and in order to travel to or from that work place he undertakes a journey in respect of which—
\begin{enumerate}\item[]
($a$) the travelling costs are borne wholly or in part by the relevant employer; or

($b$) the relevant employer provides transport for any part of the journey for the use of the relevant person,
\end{enumerate}
and where he attends at more than one work place on the same day that day shall be disregarded only if the condition specified in this sub-paragraph is satisfied in respect of all the work places at which he attends on that day,

\medskip

9.  There shall be calculated, or if that is impracticable, estimated—
\begin{enumerate}\item[]
($a$) the straight-line distances between the relevant person’s home and each work place; and

($b$) the straight-line distances between each of the relevant person’s work places, other than those between which he does not ordinarily travel.
\end{enumerate}

\medskip

10.  Subject to paragraph 11, there shall be calculated for each pair of places referred to in paragraph 9 the number of journeys which the relevant person makes between them during a period comprising a whole number of weeks which appears to the child support officer to be representative of the normal working pattern of the relevant person.

\medskip

11.  For the purposes of the calculation required by paragraph 10 there shall be disregarded—
\begin{enumerate}\item[]
($a$) any pair of journeys between the same work place and the relevant person’s home where the first journey is from his work place to his home and the time which elapses between the start of the first journey and the conclusion of the second is not more than two hours; and

($b$) any journey in respect of which—
\begin{enumerate}\item[]
(i) the travelling costs are borne wholly or in part by the relevant employer; or

(ii) the relevant employer provides transport for any part of the journey for the use of the relevant person.
\end{enumerate}
\end{enumerate}

\medskip

12.  The result of the calculation of the number of journeys made between each pair of places required by paragraph 10 shall be multiplied by the result of the calculation or estimate of the straight-line distance between them required by paragraph 9.

\medskip

13.  All the products of the multiplications required by paragraph 12 shall be added together and the resulting sum divided by the number of weeks in the period.

\medskip

14.  Where the result of the division required by paragraph 13 is less than or equal to 150, the amount to be allowed in respect of travelling costs shall be nil, and where it is greater than 150, the weekly allowance to be made in respect of the relevant person’s travelling costs shall be 10 pence multiplied by the number by which that number exceeds 150.

\subsection*{Computation in cases where there is more than one relevant employment}

15.—(1) Subject to sub-paragraph (2) and paragraphs 21 to 23, where the relevant person has more than one relevant employment the amount to be allowed in respect of travelling costs for the purposes of regulations 9 and 11 shall be determined in accordance with paragraphs 16 to 20.

(2) Where it appears that in respect of any of his relevant employments, whilst the relevant person works at more than one work place, his pattern or work is not sufficiently regular to enable the calculations of the amount to be allowed in respect of his travelling costs to be made readily, the child support officer—
\begin{enumerate}\item[]
($a$) may select a place which is either one of the relevant person’s work places in that relevant employment or some other place which is connected with that relevant employment;

($b$) may calculate the weekly average distance travelled in the course of his journeys made in connection with the relevant employment upon the basis that—
\begin{enumerate}\item[]
(i) the relevant person makes one journey from his home, or from another work place or deemed work place in another relevant employment, to the deemed work place and one journey from the deemed work place to his home, or to another work place or deemed work place in another relevant employment, on each day on which he attends at a work place in connection with the relevant employment in relation to which the deemed work place has been selected, and

(ii) the distance he travels between those places is the straight-line distance between them; and
\end{enumerate}

($c$) shall disregard any journeys made between work places in the relevant employment in respect of which a deemed work place has been selected.
\end{enumerate}

(3) For the purposes of sub-paragraph (2)($b$) there shall be disregarded any day upon which the relevant person attends at a work place and in order to travel to or from that work place he undertakes a journey in respect of which—
\begin{enumerate}\item[]
($a$) the travelling costs are borne wholly or in part by the relevant employer; or

($b$) the relevant employer provides transport for any part of the journey for the use of the relevant person,
\end{enumerate}
and where in the course of the particular relevant employment he attends at more than one work place on the same day, that day shall be disregarded only if the condition specified in this paragraph is satisfied in respect of all the work places at which he attends on that day in the course of that employment.

\medskip

16.  There shall be calculated, or if that is impracticable, estimated—
\begin{enumerate}\item[]
($a$) the straight-line distances between the relevant person’s home and each work place; and

($b$) the straight-line distances between each of the relevant person’s work places, except—
\begin{enumerate}\item[]
(i) those between which he does not ordinarily travel, and

(ii) those for which a calculation of the distance from the relevant person’s home is not required by virtue of paragraph 15($c$).
\end{enumerate}
\end{enumerate}

\medskip

17.  There shall be calculated, or if that is impracticable, estimated for each pair of places referred to in paragraph 16 between which straight-line distances are required to be calculated or estimated the number of journeys which the relevant person makes between them during a period comprising a whole number of weeks which appears to the child support officer to be representative of the normal working pattern of the relevant person, thee being disregarded any pair of journeys between the same work place and his home where the first journey is from his work place to his home and the time which elapses between the start of the first journey and the conclusion of the second is not more than two hours.

\medskip

18.  The result of the calculation or estimate of the number of journeys made between each pair of places required by paragraph 17 shall be multiplied by the result of the calculation or estimate of the straight-line distance between them required by paragraph 16.

\medskip

19.  All the products of the multiplications required by paragraph 18, shall be added together and the resulting sum divided by the number of weeks in the period.

\medskip

20.  Where the result of the division required by paragraph 19, plus where appropriate the result of the calculation required by paragraph 15 in respect of a relevant employment in which a deemed work place has been selected, is less than or equal to 150 the amount to be allowed in respect of travelling costs shall be nil, and where it is greater than 150, the weekly allowance to be made in respect of the relevant person’s travelling costs shall be 10 pence multiplied by the number by which that number exceeds 150.

\subsection*{\sloppy Relevant employments in respect of which no amount is to be allowed}

21.—(1) No allowance shall be made in respect of travelling costs in respect of journeys between the relevant person’s home and his work place or between his work place and his home in a particular relevant employment if the condition set out in paragraph 22 or 23 is satisfied in respect of that employment.

(2) The condition mentioned in paragraph 22, or as the case may be 23, is satisfied in relation to a case where the relevant person has more than one work place in a relevant employment only where the employer provides assistance of the kind mentioned in that paragraph in respect of all of the work places to or from which the relevant person travels in the course of that employment, but those journeys in respect of which that assistance is provided shall be disregarded in computing the total distance travelled by the relevant person in the course of the relevant employment.

\medskip

22.  The condition is that the relevant employer provides transport of any description in connection with the employment which is available to the relevant person for any part of the journey between his home and his work place or between his work place or between his work place and his home.

\medskip

23.  The condition is that the relevant employer bears any part of the travelling costs arising from the relevant person travelling between his home and his work place or between his work place and his home in connection with that employment, and for the purposes of this paragraph he does not bear any part of that cost where he does no more than—
\begin{enumerate}\item[]
($a$) make a payment to the relevant person which would fail to be taken into account in determining the amount of the relevant person’s net income;

($b$) make a loan to the relevant person;

($c$) pay to the relevant person an increased amount of remuneration,
\end{enumerate}
to enable the relevant person to meet those costs himself.''
\end{quotation}

\part{Explanatory Note}

\renewcommand\parthead{--- Explanatory Note}

\subsection*{(This note is not part of the Regulations)}

These Regulations amend various regulations made under the Child Support Act 1991 (“the Act”). They also amend, in one respect, the Income Support (General) Regulations 1987.

  The Child Support Appeal Tribunals (Procedure) Regulations 1992 are amended to make provision for a pending appeal to continue when a party thereto dies (regulation 4).

  The Child Support (Arrears, Interest and Adjustment of Maintenance Assessments) Regulations 1992 are amended to remove the liability for interest in respect of any day after 17th April 1995 (regulation 7). They are also amended to make new provisions for overpaid maintenance to be set off against arrears of maintenance and against current maintenance payable (regulation 8).

\begin{sloppypar}
  The Child Support (Collection and Enforcement) Regulations 1992 are amended to provide for a protected earnings rate in deduction from earnings orders in respect of interim maintenance assessments and in cases where arrears are due under a previous assessment but there is no current assessment in existence (regulation 17). The regulations are also amended to define the grounds upon which magistrates may discharge a deduction from earnings order as defective (regulation 14) and to make further provision for discharge of a deduction from earnings order by the Secretary of State (regulation 19).
\end{sloppypar}

  The Child Support Fees Regulations 1992 are amended to provide that no assessment fee or collection fee shall be payable where it would otherwise have become payable on or after 18th April 1995 and before 6th April 1997 (regulation 20).

  The Child Support (Information, Evidence and Disclosure) Regulations 1992 are amended to require Crown servants to provide information in certain circumstances (regulation 22) and to make provision for information given by one party to a maintenance assessment to be disclosed to the other in certain circumstances (regulation 24).

  The Child Support (Maintenance Arrangements and Jurisdiction) Regulations 1992 are amended to provide for maintenance orders under certain enactments to be relevant for the purposes of sections 8 and 10 of the Act (regulations 26 and 27).

  The Child Support (Maintenance Assessment Procedure) Regulations 1992 are amended to make provision for two further categories of interim maintenance assessment (regulation 28); for periodical reviews under section 16 of the Act to take place every 104 weeks (regulation 34); and for the effective date of a maintenance assessment in certain circumstances to be 8 weeks after a maintenance enquiry form has been sent to an absent parent (regulation 36).

  The Child Support (Maintenance Assessments and Special Cases) Regulations 1992 are amended to require a child support officer, in certain circumstances, to make allowances for travel to work costs and for certain property transfers in the calculation of child support maintenance (regulations 44 and 45 and Schedules 1 and 2); and to provide that an absent parent must be left with not less than 70 per cent of his net income after deduction of the amount payable under a maintenance assessment (regulation 46). The requirement to apportion housing costs has been removed (regulation 50).

  The Income Support (General) Regulations 1987 are amended so that if a person loses income support in respect of mortgage interest on receiving child support maintenance and that maintenance in specified cases is later reduced or terminated, the claimant’s previous entitlement to income support on account of mortgage interest will be restored provided the period since he was last entitled to benefit does not exceed 26 weeks (regulation 62).

  The Child Support (Miscellaneous Amendments and Transitional Provisions) Regulations 1994 are amended to provide that the transitional provisions will not apply to Category D interim maintenance assessments and that the special provision for the amount of a maintenance assessment made following a review under section 17 of the Act where the transitional provisions apply shall also apply to reviews under sections 18 or 19 of the Act (regulations 60 and 61).

  Other amendments are of a minor, technical, or consequential nature.

  An assessment of the cost to business of applying these Regulations has been placed in the libraries of both Houses of Parliament. Copies can be obtained from: DSS, Room 9/03, Adelphi, 1 11 John Adam Street, London \textsc{\lowercase{WC2N 6HT}}.


\end{document}