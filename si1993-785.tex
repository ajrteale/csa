\documentclass[a4paper]{article}

\usepackage[welsh,english]{babel}

\usepackage[utf8]{inputenc}
\usepackage[T1]{fontenc}

\usepackage{textcomp}

%\usepackage[2012rules]{optional}

\usepackage[osf]{mathpazo}
\usepackage{cfr-lm}

\usepackage{perpage} %the perpage package
\MakePerPage{footnote} %the perpage package command
\renewcommand{\thefootnote}{\fnsymbol{footnote}}

\usepackage[perpage,para,symbol]{footmisc}

%\opt{newrules}{
\title{The Child Support Act 1991 (Consequential Amendments) Order 1993}
%}

%\opt{2012rules}{
%\title{Child Maintenance and Other Payments Act 2008\\(2012 scheme version)}
%}

\author{S.I. 1993 No. 785}

\date{Made 17th March 1993\\Laid before Parliament 22nd March 1993\\Coming into force 12th April 1993}

%\opt{oldrules}{\newcommand\versionyear{1993}}
%\opt{newrules}{\newcommand\versionyear{2003}}
%\opt{2012rules}{\newcommand\versionyear{2012}}

\usepackage{fancyhdr}
\pagestyle{fancy}
\fancyhead[L]{}
\fancyhead[C]{\itshape The Child Support Act 1991 (Consequential Amendments) Order 1993 (S.I.~1993/785) \parthead%\phantom{...}% (\versionyear{} scheme version)
}
\fancyhead[R]{}
\fancyfoot[C]{\thepage}
\newcommand{\parthead}{}

\usepackage{array}
\usepackage{multirow}
\usepackage[debugshow]{tabulary}
\usepackage{longtable}
\usepackage{multicol}
\usepackage{lettrine}

\usepackage[colorlinks=true]{hyperref}
\usepackage{microtype}

\hyphenation{Aw-dur-dod}
\hyphenation{bank-rupt-cy}
\hyphenation{Ec-cles-ton}
\hyphenation{Eux-ton}
\hyphenation{Hogh-ton}
\hyphenation{Pres-ton}
\hyphenation{Pru-den-tial}
\hyphenation{Riv-ing-ton}

\newcolumntype{x}[1]
	{>{\raggedright}p{#1}}
\newcommand{\tn}{\tabularnewline}
\setlength\tymin{50pt}

\newcommand\amendment[1]{\subsubsection*{Notes}{\itshape\frenchspacing\footnotesize #1 \par}}

\begin{document}

\maketitle

\noindent
The Secretary of State for Social Services, in exercise of the powers conferred by sections 52(1) and 58(7) of the Child Support Act 1991\footnote{\frenchspacing 1991 c. 48.}, and of all other powers enabling him in that behalf, hereby makes the following Order:

{\sloppy

\tableofcontents

}

\setcounter{secnumdepth}{-2}

\subsection[1. Citation and commencement]{Citation and commencement}

1.  This Order may be cited as the Child Support Act 1991 (Consequential Amendments) Order 1993 and shall come into force on 12th April 1993.

\subsection[2. Amendments relating to the Army]{Amendments relating to the Army}

2.—(1) The following section shall be inserted in the Army Act 1955\footnote{\frenchspacing 3 \& 4 Eliz. 2 c. 18. Section 152 was amended by the Armed Forces Act 1971 (c. 33), sections 59(2) and 77(1) and Schedule 4, Part I.} after section 150—
\begin{quotation}
\subsection*{“150A. Enforcement of maintenance assessment by deductions from pay}

(1) Subsection (2) applies where any officer, warrant officer, non-commissioned officer or soldier of the regular forces (“the liable person”) is required to make periodical payments in respect of any child in accordance with a maintenance assessment made under the Child Support Act 1991.

(2) The Defence Council or an officer authorised by them may order such sum to be deducted from the pay of the liable person and appropriated in or towards satisfaction of any obligation of his—
\begin{enumerate}\item[]
($a$) to make periodical payments in accordance with the maintenance assessment; or

($b$) to pay interest (by virtue of regulations made under section 41(3) of the Act of 1991) with respect to arrears of child support maintenance payable in accordance with the assessment,
\end{enumerate}
as they, or the authorised officer, thinks fit.

(3) Where a child support officer—
\begin{enumerate}\item[]
($a$) makes or cancels a maintenance assessment or a fresh maintenance assessment; and

($b$) has reason to believe that the person against whom the assessment is, or was, made is an officer, warrant officer, non-commissioned officer or soldier of the regular forces,
\end{enumerate}
the Secretary of State shall inform the Defence Council or an officer authorised by them of the terms of the assessment or (as the case may be) that it has been cancelled.

(4) This section applies whether or not the liable person was a member of the regular forces when the maintenance assessment was made.”.
\end{quotation}

(2) The following subsection shall be inserted in section 151 of the Act of 1955 (deductions from pay for maintenance of wife or child)—
\begin{quotation}
“(3A) Where an order is in force under section 150A of this Act for deductions to be made from the pay of any member of the regular forces with respect to the maintenance of a child of his, no order may be made under this section for the deductions of any sums from the pay of that person with respect to the maintenance of that child.”.
\end{quotation}

(3) In section 152 of the Act of 1955 (limit on deductions under sections 150 and 151 and effect of forfeiture), after “150”, in each case, there shall be inserted “,~150A”.

\subsection[3. Amendments relating to the Royal Air Force]{Amendments relating to the Royal Air Force}

3.—(1) The following section shall be inserted in the Air Force Act 1955\footnote{\frenchspacing 3 \& 4 Eliz. 2 c. 19. Section 152 was amended by the Armed Forces Act 1971 (c. 33), sections 59(2) and 77(1) and Schedule 4, Part I.} after section 150—
\begin{quotation}
\subsection*{“150A. Enforcement of maintenance assessment by deductions from pay.}

(1) Subsection (2) applies where any officer, warrant officer, non-commissioned officer or airman of the regular air force (“the liable person”) is required to make periodical payments in respect of any child in accordance with a maintenance assessment made under the Child Support Act 1991.

(2) The Defence Council or an officer authorised by them may order such sum to be deducted from the pay of the liable person and appropriated in or towards satisfaction of any obligation of his—
\begin{enumerate}\item[]
($a$) to make periodical payments in accordance with the maintenance assessment; or

($b$) to pay interest (by virtue of regulations made under section 41(3) of the Act of 1991) with respect of arrears of child support maintenance payable in accordance with the assessment,
\end{enumerate}
as they, or the authorised officer, thinks fit.

(3) Where a child support officer—
\begin{enumerate}\item[]
($a$) makes or cancels a maintenance assessment or a fresh maintenance assessment; and

($b$) has reason to believe that the person against whom the assessment is, or was, made is an officer, warrant officer, non-commissioned officer or airman of the regular air force,
\end{enumerate}
the Secretary of State shall inform the Defence Council or an officer authorised by them of the terms of the assessment or (as the case may be) that it has been cancelled.

(4) This section applies whether or not the liable person was a member of the regular air force when the maintenance assessment was made.”.
\end{quotation}

(2) The following subsection shall be inserted in section 151 of the Act of 1955 (deductions from pay for maintenance of wife or child)—
\begin{quotation}
“(3A) Where an order is in force under section 150A of this Act for deductions to be made from the pay of any member of the regular air force with respect to the maintenance of a child of his, no order may be made under this section for the deduction of any sums from the pay of that person with respect to the maintenance of that child.”.
\end{quotation}

(3) In section 152 of the Act of 1955 (limit on deductions under sections 150 and 151 and effect of forfeiture), after “150”, in each case, there shall be inserted “,~150A”.

\subsection[4. Amendments relating to the naval forces]{Amendments relating to the naval forces}

4.  In section 1(1) of the Naval Forces (Enforcement of Maintenance Liabilities) Act 1947\footnote{\frenchspacing 10 \& 11 Geo. 6, c. 24. Section 1 was amended by Schedule 6 of the Naval Discipline Act 1957 (c. 53) and sections 15(1) to (4) and 26(2) of, and Schedule 3 to, the Armed Forces Act 1991 (c. 62).} (deductions from pay in respect of liabilities for maintenance etc.), the following paragraph shall be inserted after paragraph ($aa$)—
\begin{quotation}
“($aaa$) for the payment of interest (by virtue of regulations made under section 41(3) of the Child Support Act 1991) with respect to arrears of child support maintenance payable in accordance with any maintenance assessment made under that Act;”.
\end{quotation}

\subsection[5. Amendments relating to the merchant navy]{Amendments relating to the merchant navy}

5.  In section 11 of the Merchant Shipping Act 1970\footnote{\frenchspacing 1970 c. 36.} (restriction on assignment of and charges upon wages), the following subsection shall be added at the end—
\begin{quotation}
“(4) Subsection (1)($a$) of this section is subject to any provision made by or under sections 31 or 32 of the Child Support Act 1991 (deductions from earnings orders).”.
\end{quotation}

\bigskip

Signed by authority of the Secretary of State for Social Security.

{\raggedleft
\emph{Alistair Burt}\\*Parliamentary Under-Secretary of State,\\*Department of Social Security

}

17th March 1993

\part{Explanatory Note}

\renewcommand\parthead{--- Explanatory Note}

\subsection*{(This note is not part of the Order)}

This Order makes amendments to the Army Act 1955, the Air Force Act 1955, the Naval Forces (Enforcement of Maintenance Liabilities) Act 1947 and the Merchant Shipping Act 1970 consequential on the coming into force of the Child Support Act 1991 so as to make provision for the recovery of child support maintenance by deductions from the pay of soldiers, airmen, sailors and merchant seamen.

\end{document}