\documentclass[12pt,a4paper]{article}

\newcommand\regstitle{The Child Support Commissioners (Procedure) (Amendment) Regulations 1997}

\newcommand\regsnumber{1997/802}

%\opt{newrules}{
\title{\regstitle}
%}

%\opt{2012rules}{
%\title{Child Maintenance and Other Payments Act 2008\\(2012 scheme version)}
%}

\author{S.I. 1997 No. 802}

\date{Made 12th March 1997\\Laid before Parliament 13th March 1997\\Coming into force 14th April 1997}

%\opt{oldrules}{\newcommand\versionyear{1993}}
%\opt{newrules}{\newcommand\versionyear{2003}}
%\opt{2012rules}{\newcommand\versionyear{2012}}

\usepackage{csa-regs}

\begin{document}

\maketitle

\noindent
The Lord Chancellor, in exercise of the powers conferred by sections 22(3), 24(6) and (7) of the Child Support Act 1991\footnote{\frenchspacing 1991 c. 48.} and of all other powers enabling him in that behalf, after consultation with the Lord Advocate and, in accordance with section 8 of the Tribunals and Inquiries Act 1992\footnote{\frenchspacing 1992 c. 53.}, with the Council on Tribunals, hereby makes the following Regulations:—


%{\sloppy
%
%\tableofcontents
%
%}

\bigskip

\setcounter{secnumdepth}{-2}

1.—(1) These Regulations may be cited as the Child Support Commissioners (Procedure) (Amendment) Regulations 1997 and shall come into force on 14th April 1997.

(2) In these Regulations a regulation referred to by number means the regulation so numbered in the Child Support Commissioners (Procedure) Regulations 1992\footnote{\frenchspacing S.I. 1992/2640.}.

\medskip

2.  After regulation 1(3), there shall be inserted the following paragraph:—
\begin{quotation}
“(4) In these Regulations, for the purposes of any proceedings relating to an application for a departure direction which has been decided by an appeal tribunal under section 28D(1)($b$)\footnote{\frenchspacing Section 28D of the Child Support Act 1991 was inserted by the Child Support Act 1995, section 4.} of the Act, the term ‘party to the proceedings’ shall include the Secretary of State.”.
\end{quotation}

\medskip

3.  Regulation 3(1) shall be amended by substituting, for the words “tribunal at the Central Office of Child Support Appeal Tribunals” the words “appeal tribunal”.

\medskip

4.  Regulation 3(4) shall be amended by substituting, for the words “a child support officer he”, the words “a child support officer or by the Secretary of State under section 24(1A)\footnote{\frenchspacing Subsection (1A) of section 24 of the Child Support Act 1991 was inserted by the Child Support Act 1995, section 30(5), Schedule 3, paragraph 7.} of the Act, the child support officer or the Secretary of State, as the case may be,”.

\medskip

5.  At the beginning of regulation 9, there shall be inserted the words “Except where he is already a party to the proceedings by virtue of regulation 1(4) or of regulation 1 of the Child Support Appeal Tribunals (Procedure) Regulations 1992\footnote{\frenchspacing S.I. 1992/2641.},”.

\medskip

6.  For regulation 22, there shall be substituted the following regulation:—
\begin{quotation}
\subsection*{“Confidentiality}

22.—(1) No information such as is mentioned in paragraph (2), and which has been provided for the purposes of any proceedings to which these Regulations apply, shall be disclosed if, before the expiry of the period of 21 days specified in paragraph (3), written notification has been received from the person to whom the information relates that he does not consent to such disclosure.

(2) The information referred to in paragraph (1) is—
\begin{enumerate}\item[]
($a$) the address of the person referred to in that paragraph; and

($b$) any other information the use of which could reasonably be expected to lead to that person being located.
\end{enumerate}

(3) Except where the proceedings relate to an application for leave to appeal to a Commissioner or to an appeal in either case made under section 46(7) of the Act (Failure to comply with obligations imposed by section 6) or regulation 42(9) of the Child Support (Maintenance Assessment Procedure) Regulations 1992\footnote{\frenchspacing S.I. 1992/1813.} (Review of a reduced benefit direction), the Office of the Commissioner shall notify the person to whom the information referred to in paragraphs (1) and (2) relates of the provisions of those paragraphs and that disclosure of that information may be made unless the written notification specified in paragraph (1) is received before the expiry of the period of 21 days beginning with the date the notification by the Office of the Commissioner was given or sent to that person.”.
\end{quotation}

\bigskip

%Signed by authority of the Secretary of State for Social Security.

{\raggedleft
\emph{Mackay of Clashfern, C.}%\\*Parliamentary Under-Secretary of State,\\*Department of Social Security

}

Dated 12th March 1997

\small

\part{Explanatory Note}

\renewcommand\parthead{--- Explanatory Note}

\subsection*{(This note is not part of the Regulations)}

These Regulations amend the Child Support Commissioners (Procedure) Regulations 1992 so as to:—
\begin{enumerate}\item[]
 ($a$) provide that the Secretary of State shall be included as a party to certain proceedings (\emph{regulation 2});

 (b) provide that in a notice of an application for leave to appeal to a Commissioner, the shorter form of wording “appeal tribunal” shall be used instead of the words “tribunal at the Central Office of Child Support Appeal Tribunals” (\emph{regulation 3});

 ($c$) provide that, where an application for leave to appeal is made by a child support officer or by the Secretary of State under section 24(1A) of the Child Support Act 1991, the child support officer or the Secretary of State, as the case may be, shall send a copy of the application to each person specified in regulation 3(4) of the 1992 Regulations (\emph{regulation 4});

 ($d$) provide that, except where the Secretary of State is already a party to the proceedings, he may at any time apply to a Commissioner for leave to intervene in an appeal pending before a Commissioner (\emph{regulation 5}); and

 ($e$) provide that information which has been provided for the purposes of any proceedings to which the 1992 Regulations apply may be disclosed, unless within 21 days the person to which that information relates gives written notification that he does not consent to such disclosure (\emph{regulation 6}).
\end{enumerate}

\end{document}