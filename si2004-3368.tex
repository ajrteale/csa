\documentclass[12pt,a4paper]{article}

\newcommand\regstitle{The Social Security, Child Support and Tax Credits (Decisions and Appeals) Amendment Regulations 2004}

\newcommand\regsnumber{2004/3368}

%\opt{newrules}{
\title{\regstitle}
%}

%\opt{2012rules}{
%\title{Child Maintenance and Other Payments Act 2008\\(2012 scheme version)}
%}

\author{S.I.\ 2004 No.\ 3368}

\date{Made
20th December 2004\\
%Laid before Parliament
%15th December 2003\\
Coming into force
in accordance with regulation 1
}

%\opt{oldrules}{\newcommand\versionyear{1993}}
%\opt{newrules}{\newcommand\versionyear{2003}}
%\opt{2012rules}{\newcommand\versionyear{2012}}

\usepackage{csa-regs}

\setlength\headheight{27.57402pt}

\begin{document}

\maketitle

\noindent
Whereas a draft of this Instrument was laid before Parliament in accordance with section 80(1) of the Social Security Act 1998\footnote{1998 c.\ 14.}, and paragraph 20(4) of Schedule 7 to the Child Support, Pensions and Social Security Act 2000\footnote{2000 c.\ 19.}, and approved by a resolution of each House of Parliament;

Now, therefore, the Secretary of State for Work and Pensions in exercise of the powers conferred by section 4(2) and (3) of the Vaccine Damage Payments Act 1979\footnote{1979 c.\ 17; section 4 was substituted by the Social Security Act 1998, section 46.}, section 20(4), (5) and (6) of the Child Support Act 1991\footnote{1991 c.\ 48; section 20 was substituted by the Social Security Act 1998, section 42 and was further substituted by the Child Support, Pensions and Social Security Act 2000 (c.\ 19), section 10.}, sections 5(1)($a$)  and ($h$), 6(1)($a$)  and ($h$), 189(1), (4) and (6) and 191 of the Social Security Administration Act 1992\footnote{1992 c.\ 5; section 6(1) was substituted by the Local Government Finance Act 1992 (c.\ 14), Schedule 9, paragraph 12(1)($a$); section 191 is cited because of the meaning it gives to “prescribe”.}, section 11(5) of the Social Security (Recovery of Benefits) Act 1997\footnote{1997 c.\ 27.}, sections 7(6)($a$), 12(2), 16(1), 79(1), (4), (6) and (7) and 84 of, and paragraphs 1 to 4 of Schedule 5 to, the Social Security Act 1998\footnote{Section 12(2) was substituted by the Social Security Contributions (Transfer of Functions, etc.)\ Act 1999 (c.\ 2), Schedule 7, paragraph 25; the powers in sections 16(1) and paragraphs 1 to 4 and 6 of Schedule 5, which are exercised in these Regulations in respect of Tax Credits, are those which have been applied and modified by S.I.\ 2002/2926 under powers contained in section 63(8) of the Tax Credits Act 2002 (c.\ 21); section 84 is cited because of the meaning it gives to “prescribe”.} and section 68 of, and paragraphs 6(2)($e$), 10(1), 20(1) and (3) and 23(1) of Schedule 7 to, the Child Support, Pensions and Social Security Act 2000\footnote{Paragraph 23(1) is cited because of the meaning it gives to “prescribed”.} and of all other powers enabling him in that behalf, after consultation with the Council on Tribunals in accordance with section 8 of the Tribunals and Inquiries Act 1992\footnote{1992 c.\ 53.}, after agreement by the Social Security Advisory Committee that the proposals to make these Regulations should not be referred to it\footnote{See the Social Security Administration Act 1992, sections 172 and 173(1)($b$) and Schedule 7, paragraph 9.}, and so far as they concern housing benefit and council tax benefit after consultation with organisations appearing to the Secretary of State to be representative of the authorities concerned\footnote{\emph{See} the Social Security Administration Act 1992, section 176(1)($a$).}, hereby makes the following Regulations: 

{\sloppy

\tableofcontents

}

\bigskip

\setcounter{secnumdepth}{-2}

\subsection[1. Citation and commencement]{Citation and commencement}

1.  These Regulations may be cited as the Social Security, Child Support and Tax Credits (Decisions and Appeals) Amendment Regulations 2004 and shall come into force on the day after the day on which they are made.

\subsection[2. Amendment of the Social Security and Child Support (Decisions and Appeals) Regulations 1999]{Amendment of the Social Security and Child Support (Decisions and Appeals) Regulations 1999}

2.---(1)  The Social Security and Child Support (Decisions and Appeals) Regulations 1999\footnote{S.I.\ 1999/991.} shall be amended in accordance with this regulation.

(2) In regulation 1(3) (interpretation) the definition of “misconceived appeal” shall be omitted.

(3) In regulation 25\footnote{Regulation 25 was amended by S.I.\ 1999/2570 and 2002/1379.} (other persons with a right of appeal) for “section 12(2)($b$)”\footnote{Section 12(2) of the Social Security Act 1998 (c.\ 14) was substituted by the Social Security Contributions (Transfer of Functions, etc.)\ Act 1999 (c.\ 2), Schedule 7, paragraph 25.} there shall be substituted “section 12(2)”.

% Reg 2(4)--(7) revoked (3.11.08) by SI 2008/2683 Sch 2
%(4) In regulation 36\footnote{Regulation 36 was amended by S.I.\ 1999/1466 and 2000/1596.} (composition of appeal tribunals)—
%\begin{enumerate}\item[]
%($a$) in paragraph (2) for “, (8) and (9)” substitute “and (8)”,
%
%($b$) in paragraph (5) for “, (3) or (9)” substitute “or (3)”, and
%
%($c$) omit paragraph (9) (misconceived appeals).
%\end{enumerate}
%
%(5) In regulation 39 (directions concerning oral hearings) for the heading and paragraphs (1) to (4) substitute—
%\begin{quotation}
%\subsection*{“Choice of hearing}
%
%39.---(1)  Where an appeal or a referral is made to an appeal tribunal the appellant and any other party to the proceedings shall notify the clerk to the appeal tribunal, on a form approved by the Secretary of State, whether he wishes to have an oral hearing of the appeal or whether he is content for the appeal or referral to proceed without an oral hearing.
%
%(2) Except in the case of a referral, the form shall include a statement informing the appellant that, if he does not notify the clerk to the appeal tribunal as required by paragraph (1) within the period specified in paragraph (3), the appeal may be struck out in accordance with regulation 46(1).
%
%(3) Notification in accordance with paragraph (1)—
%\begin{enumerate}\item[]
%($a$) if given by the appellant or a party to the proceedings other than the Secretary of State, must be sent or given to the clerk to the appeal tribunal within 14 days of the date on which the form is issued to him; or
%
%($b$) if given by the Secretary of State, must be sent or given to the clerk—
%\begin{enumerate}\item[]
%(i) in the case of an appeal, within 14 days of the date on which the form is issued to the appellant; or
%
%(ii) in the case of a referral, on the date of referral,
%\end{enumerate}
%or within such longer period as the clerk may direct.
%\end{enumerate}
%
%(4) Where an oral hearing is requested in accordance with paragraphs (1) and (3) the appeal tribunal shall hold an oral hearing unless the appeal is struck out under regulation 46(1).”.
%\end{quotation}
%
%(6) In regulation 46 (appeals which may be struck out)—
%\begin{enumerate}\item[]
%($a$) in paragraph (1)—
%\begin{enumerate}\item[]
%(i) at the end of sub-paragraph ($b$)  omit “or”,
%
%(ii) in sub-paragraph ($c$)  omit “subject to regulation 39(4),” and after “struck out” add “; or”, and
%
%(iii) after sub-paragraph ($c$)  add—
%\begin{quotation}
%“($d$) for failure of the appellant to notify the clerk to the appeal tribunal, in accordance with regulation 39, whether or not he wishes to have an oral hearing of his appeal.”, and
%\end{quotation}
%\end{enumerate}
%
%($b$) omit paragraph (4) (misconceived appeals).
%\end{enumerate}
%
%(7) In regulation 47\footnote{Regulation 47 was amended by S.I.\ 2000/1596 and 2002/1379.} (reinstatement of struck out appeals)—
%\begin{enumerate}\item[]
%($a$) in paragraph (1) for “46(1)($c$)” substitute “46(1)($d$)”,
%
%($b$) in paragraph (2)—
%\begin{enumerate}\item[]
%(i) omit “or 48”, and
%
%(ii) omit sub-paragraph ($b$) .
%\end{enumerate}
%\end{enumerate}

(8) Omit regulation 48 (misconceived appeals).

(9) In Schedule 2 (decisions against which no appeal lies), in paragraph 5\footnote{Paragraph 5 was substituted by S.I.\ 2002/1379 and amended by S.I.\ 2003/1581.} (claims and payments)—
\begin{enumerate}\item[]
($a$) for sub-paragraph ($a$)  substitute—
\begin{quotation}
“($a$) regulation 4(3) or (3B)\footnote{Paragraph (3) was amended by S.I.\ 1996/2431 and paragraph (3B) was inserted by S.I.\ 1996/1460.} (which partner should make a claim for income support or jobseeker’s allowance);”,
\end{quotation}

($b$) omit sub-paragraphs ($b$), ($c$), ($d$)  and ($e$), and

($c$) for sub-paragraph ($bb$)\footnote{Sub-paragraph ($bb$)  was inserted by S.I.\ 2003/1581.} substitute—
\begin{quotation}
“($bb$) regulation 4D(7)\footnote{Regulation 4D was inserted by S.I.\ 2002/3019.} (which partner should make a claim for state pension credit);”.
\end{quotation}
\end{enumerate}

\amendment{
Reg. 2(4)--(7) revoked (3.11.08) by the Tribunals, Courts and Enforcement Act 2007 (Transitional and Consequential Provisions) Order 2008 Sch.~2.
}

\subsection[3. Amendment of the Housing Benefit and Council Tax Benefit (Decisions and Appeals) Regulations 2001]{Amendment of the Housing Benefit and Council Tax Benefit (Decisions and Appeals) Regulations 2001}

3.---(1)  The Housing Benefit and Council Tax Benefit (Decisions and Appeals) Regulations 2001\footnote{S.I.\ 2001/1002.} shall be amended in accordance with this regulation.

% Reg 3(2), (3) revoked (3.11.08) by SI 2008/2683 Sch 2
%(2) In regulation 22 (composition of appeal tribunals) omit paragraph (3) (misconceived appeals).
%
%(3) In regulation 23\footnote{Regulation 23 was amended by S.I.\ 2002/1379.} (procedure in connection with appeals)—
%\begin{enumerate}\item[]
%($a$) in paragraph (1) for “the Social Security and Child Support (Decisions and Appeals) (Miscellaneous Amendments) Regulations 2002” substitute “the Social Security, Child Support and Tax Credits (Decisions and Appeals) Amendment Regulations 2004\footnote{S.I.\ 2004/3368.}”, and
%
%($b$) in paragraph (3) after “except in regulations” insert “39(1)\footnote{Regulation 39 was amended by S.I.\ 2004/3368.} (choice of hearing),”.
%\end{enumerate}

(4) In the Schedule (decisions against which no appeal lies)—
\begin{enumerate}\item[]
($a$) in paragraph 1 (no right of appeal: exceptions) for sub-paragraph ($a$)  substitute—
\begin{quotation}
“($a$) regulations 72\footnote{Relevant amendments to regulation 72 were made by S.I.\ 1996/2432, 2000/897, and 2001/1605.} (time and manner in which claims are to be made), 72A(1)\footnote{Regulation 72A was inserted by S.I.\ 1999/1539 and amended by S.I.\ 2002/1397.} and 72B(1) and (4)\footnote{Regulation 72B was substituted by S.I.\ 2000/897.} (date of claim).”, and
\end{quotation}

($b$) in paragraph 2 (no right of appeal: exceptions) for sub-paragraph ($a$)  substitute—
\begin{quotation}
“($a$) regulations 62\footnote{Relevant amendments to regulation 62 were made by S.I.\ 1996/462, 1510 and 2432, 2000/897, 2001/1605 and 2003/48, 325 and 1632.} (time and manner in which claims are to be made), 62A(1)\footnote{Regulation 62A was inserted by S.I.\ 1999/1539 and amended by 2000/897.} and 62B(1) and (4)\footnote{Regulation 62B was substituted by S.I.\ 2000/897.} (date of claim).”.
\end{quotation}
\end{enumerate}

\amendment{
Reg. 3(2), (3) revoked (3.11.08) by the Tribunals, Courts and Enforcement Act 2007 (Transitional and Consequential Provisions) Order 2008 Sch.~2.

\medskip

Regs. 4, 5 revoked (6.3.06) by the Housing Benefit and Council Tax Benefit (Consequential Provisions) Regulations 2006 Sch. 1.

\medskip
%}
%
% Regs 4, 5 revoked by SI 2006/217 Sch 1
%\subsection[4. Amendment of the Housing Benefit (General) Regulations 1987]{\sloppy Amendment of the Housing Benefit (General) Regulations 1987}
%
%4.---(1)  The Housing Benefit (General) Regulations 1987\footnote{S.I.\ 1987/1971.} shall be amended in accordance with this regulation.
%
%(2) In regulation 72 (time and manner in which claims are to be made)—
%\begin{enumerate}\item[]
%($a$) in paragraph (1)\footnote{Paragraph (1) was amended by S.I.\ 1996/2432, 2000/897 and 2001/1605.} for “and be accompanied by or supplemented by such certificates, documents, information and evidence as are required in accordance with regulation 73(1) (evidence and information) or paragraph 5 of Schedule A1 (treatment of claims for housing benefit by refugees)”, substitute “having regard to the sufficiency of the written information and evidence”,
%
%($b$) in paragraph (7)\footnote{Paragraph (7) was amended by S.I.\ 2001/1605.}—
%\begin{enumerate}\item[]
%(i) in sub-paragraph ($a$)  for “in a written form sufficient in the circumstances of the case” substitute “properly completed”,
%
%(ii) in sub-paragraph ($b$)  after “circumstances of the case” insert “having regard to the sufficiency of the written information and evidence”,
%
%(iii) for “refer the defective claim to the claimant” substitute “request the claimant to complete the defective claim” and
%
%(iv) after “supply the claimant with the approved form” add “or request further information or evidence”,
%\end{enumerate}
%
%($c$) for paragraph (8)\footnote{Paragraph (8) was amended by S.I.\ 2001/1605.} substitute—
%\begin{quotation}
%“(8) The relevant authority shall treat a defective claim as if it had been validly made in the first instance if—
%\begin{enumerate}\item[]
%($a$) where paragraph (7)($a$)  applies, the authority receives at the designated office the properly completed claim or the information requested to complete it within 4 weeks of the request to complete, or such longer period as the relevant authority may consider reasonable; or
%
%($b$) where paragraph (7)($b$)  applies—
%\begin{enumerate}\item[]
%(i) the approved form sent to the claimant is received at the designated office properly completed within 4 weeks of it having been sent to him; or, as the case may be,
%
%(ii) the claimant supplies whatever information or evidence was requested under paragraph (7) within 4 weeks of the request,
%\end{enumerate}
%or within such longer period as the relevant authority may consider reasonable.”, and
%\end{enumerate}
%\end{quotation}
%
%($d$) in paragraph (9) after “instructions on the form” add “, including any instructions to provide information and evidence in connection with the claim”.
%\end{enumerate}
%
%(3) In regulation 76\footnote{Relevant amendments to regulation 76 were made by S.I.\ 1996/2432, 2001/1605, 2003/1338 and 2004/14.} (who is to make a decision) for the heading and paragraphs (1) to (3) substitute—
%\begin{quotation}
%\subsection*{“Decisions by a relevant authority}
%
%76.---(1)  Unless provided otherwise by these Regulations, any matter required to be determined under these Regulations shall be determined in the first instance by the relevant authority.
%
%(2) The relevant authority shall make a decision on each claim within 14 days of the provisions of regulations 72 and 73 being satisfied or as soon as reasonably practicable thereafter.”.
%\end{quotation}
%
%\subsection[5. Amendment of the Council Tax Benefit (General) Regulations 1992]{Amendment of the Council Tax Benefit (General) Regulations 1992}
%
%5.---(1)  The Council Tax Benefit (General) Regulations 1992\footnote{S.I.\ 1992/1814.} shall be amended in accordance with this regulation.
%
%(2) In regulation 62 (time and manner in which claims are to be made)—
%\begin{enumerate}\item[]
%($a$) in paragraph (1)\footnote{Paragraph (1) was amended by S.I.\ 1996/2432, 2000/897 and 2001/1605.} for “and be accompanied by or supplemented by such certificates, documents, information and evidence as are required in accordance with regulation 63(1) (evidence and information) or paragraph 4 of Schedule A1 (treatment of claims for council tax benefit by refugees)”, substitute “having regard to the sufficiency of the written information and evidence”,
%
%($b$) in paragraph (7)\footnote{Paragraph (7) was amended by S.I.\ 2001/1605.}—
%\begin{enumerate}\item[]
%(i) in sub-paragraph ($a$)  for “in a written form sufficient in the circumstances of the case” substitute “properly completed”,
%
%(ii) in sub-paragraph ($b$)  after “circumstances of the case” insert “having regard to the sufficiency of the written information and evidence”,
%
%(iii) for “refer the defective claim to the claimant” substitute “request the claimant to complete the defective claim”, and
%
%(iv) after “supply the claimant with the approved form” add “or request further information or evidence”,
%\end{enumerate}
%
%($c$) for paragraph (8)\footnote{Paragraph (8) was amended by S.I.\ 2001/1605.} substitute—
%\begin{quotation}
%“(8) The relevant authority shall treat a defective claim as if it had been validly made in the first instance if—
%\begin{enumerate}\item[]
%($a$) where paragraph (7)($a$)  applies, the authority receives at the designated office the properly completed claim or the information requested to complete it within 4 weeks of the request to complete, or such longer period as the relevant authority may consider reasonable; or
%
%($b$) where paragraph (7)($b$)  applies—
%\begin{enumerate}\item[]
%(i) the approved form sent to the claimant is received at the designated office properly completed within 4 weeks of it having been sent to him; or, as the case may be,
%
%(ii) the claimant supplies whatever information or evidence was requested under paragraph (7) within 4 weeks of the request,
%\end{enumerate}
%or within such longer period as the relevant authority may consider reasonable.”, and
%\end{enumerate}
%\end{quotation}
%
%($d$) in paragraph (9) after “instructions on the form” add “, including any instructions to provide information and evidence in connection with the claim”.
%\end{enumerate}
%
%(3) In regulation 66\footnote{Relevant amendments to regulation 66 were made by S.I.\ 1993/688, 1996/2432, 2001/1605 and 2004/14.} (who is to make a decision) for the heading and paragraphs (1) to (3) substitute—
%\begin{quotation}
%\subsection*{“Decisions by a relevant authority}
%
%66.---(1)  Unless provided otherwise by these Regulations, any matter required to be determined under these Regulations shall be determined in the first instance by the relevant authority.
%
%(2) The relevant authority shall make a decision on each claim within 14 days of the provisions of regulations 62 and 63 being satisfied or as soon as reasonably practicable thereafter.”.
%\end{quotation}
%
%\amendment{
Reg. 6 revoked (3.11.08) by the Tribunals, Courts and Enforcement Act 2007 (Transitional and Consequential Provisions) Order 2008 Sch.~2.
}

%\subsection[6. Amendment of the Tax Credits (Appeals) (No.\ 2) Regulations 2002]{Amendment of the Tax Credits (Appeals) (No.\ 2) Regulations 2002}
%
%6.---(1)  The Tax Credits (Appeals) (No.\ 2) Regulations 2002\footnote{S.I.\ 2002/3196.} shall be amended in accordance with this regulation.
%
%(2) In regulation 12 (directions concerning oral hearings), for the heading and paragraphs (1) to (4) substitute—
%\begin{quotation}
%\subsection*{“Choice of hearing}
%
%12.---(1)  Where an appeal or an application for a direction is made to an appeal tribunal the appellant or applicant and any other party to the proceedings shall notify the clerk to the appeal tribunal, on a form approved by the Secretary of State, whether he wishes to have an oral hearing or whether he is content for the appeal or application to proceed without an oral hearing.
%
%(2) The form shall include a statement informing the appellant or applicant that, if he does not notify the clerk to the appeal tribunal as required by paragraph (1) within the period specified in paragraph (3), the appeal or, as the case may be, the application may be struck out in accordance with regulation 16(1).
%
%(3) Notification in accordance with paragraph (1)—
%\begin{enumerate}\item[]
%($a$)  if given by the appellant, the applicant or any party to the proceedings other than the Board, must be given or sent to the clerk to the appeal tribunal within 14 days of the date on which the form is issued to him; or
%
%($b$) if given by the Board, must be given or sent to the clerk to the appeal tribunal within 14 days of the date on which the form is issued to the appellant or applicant,
%\end{enumerate}
%or within such longer period as the clerk may direct.
%
%(4) Where an oral hearing is requested in accordance with paragraphs (1) and (3) the appeal tribunal shall hold an oral hearing unless the case is struck out under regulation 16(1).”.
%\end{quotation}
%
%(3) In regulation 16(1) (cases which may be struck out)—
%\begin{enumerate}\item[]
%($a$) at the end of sub-paragraph ($a$)  omit “or”,
%
%($b$) in sub-paragraph ($b$)—
%\begin{enumerate}\item[]
%(i) omit “subject to regulation 12(4),”, and
%
%(ii) after “struck out” add “; or”, and
%\end{enumerate}
%
%($c$) after sub-paragraph ($b$)  add—
%\begin{quotation}
%“($c$) for failure of the appellant or applicant to notify the clerk to the appeal tribunal, in accordance with regulation 12, whether he wishes to have an oral hearing of his case.”.
%\end{quotation}
%\end{enumerate}
%
%(4) In regulation 17(1) (reinstatement of struck out cases) for “16(1)($b$)” substitute “16(1)($c$)”. 

\bigskip

Signed 
by authority of the Secretary of State for Work and Pensions.

{\raggedleft
\emph{Jane Kennedy}\\*Minister of State,\\*Department of Work and Pensions

}

%St Andrew's House, Edinburgh

%Dated
20th December 2004

\small

\part{Explanatory Note}

\renewcommand\parthead{— Explanatory Note}

\subsection*{(This note is not part of the Regulations)}

These Regulations amend the Social Security and Child Support (Decisions and Appeals) Regulations 1999 (“the 1999 Regulations”), the Housing Benefit and Council Tax Benefit (Decisions and Appeals) Regulations 2001 (“the 2001 Regulations”) and the Tax Credits (Appeals) (No.\ 2) Regulations 2002 (“the 2002 Regulations”) in respect of appeals to an appeal tribunal. They make consequential amendments to the Housing Benefit (General) Regulations 1987 and the Council Tax Benefit (General) Regulations 1992.

Regulation 2 amends the 1999 Regulations. Paragraphs (2), (4) and (8) remove the power of appeal tribunals to strike out misconceived appeals. Paragraph (5) provides for a form on which appellants and parties to proceedings must notify the clerk to the appeal tribunal if they want an oral hearing; and it gives mandatory right to an oral hearing after such notification, unless the appeal is struck out. Paragraph (6) allows the tribunal clerk to strike out an appeal if an appellant does not give the notification as prescribed; and he may reinstate the appeal under paragraph (7). Paragraph (9) confirms that there is a right of appeal against a decision that a benefit claim is defective.

Regulation 3 amends the 2001 Regulations in respect of housing benefit and council tax benefit appeals to make similar provision to that made by regulation 2 in respect of other social security benefit appeals.

Regulations 4 and 5 make amendments to the Housing Benefit (General) Regulations 1987 and the Council Tax Benefit (General) Regulations 1992 respectively. The amendments are consequential upon, or supplementary to, the provisions in regulation 3(4) relating to appeals against decisions about defective claims.

Regulation 6 amends the 2002 Regulations in respect of tax credit appeals to make similar provision to that made by regulation 2(5) to (7) in respect of social security benefit appeals.

A regulatory impact assessment has not been produced for this instrument as it has no impact on the costs of business. 

\end{document}