\documentclass[12pt,a4paper]{article}

\newcommand\regstitle{The Child Support (Miscellaneous Amendments) (No.\ 2) Regulations 1999}

\newcommand\regsnumber{1999/1047}

%\opt{newrules}{
\title{\regstitle}
%}

%\opt{2012rules}{
%\title{Child Maintenance and Other Payments Act 2008\\(2012 scheme version)}
%}

\author{S.I. 1999 No. 1047}

\date{Made 29th March 1999\\Coming into force\\---except for regulation 26($a$) 1st June 1999\\---regulation 26($a$) 29th November 1999}

%\opt{oldrules}{\newcommand\versionyear{1993}}
%\opt{newrules}{\newcommand\versionyear{2003}}
%\opt{2012rules}{\newcommand\versionyear{2012}}

\usepackage{csa-regs-draft}

\setlength\headheight{27.57402pt}

\begin{document}

\maketitle

\noindent
Whereas a draft of this Instrument was laid before Parliament in accordance with section 52(2) of the Child Support Act 1991\footnote{\frenchspacing 1991 c. 48. Section 28A to 28I of and Schedules 4A and 4B to the Child Support Act 1991 (“the 1991 Act”) were inserted by sections 1 and 9 of the Child Support Act 1995 (c. 34); sections 16, 17 and 20 were substituted for sections 16 to 20 of the 1991 Act and Schedule 4C to the 1991 Act was inserted by the Social Security Act 1998 (c. 14).}, and approved by a resolution of each House of Parliament:

 Now, therefore, the Secretary of State for Social Security, in exercise of the powers set out in the Schedule to this Instrument and of all other powers enabling him in that behalf, hereby makes the following Regulations:

{\sloppy

\tableofcontents

}

\bigskip

\setcounter{secnumdepth}{-2}

\subsection[1. Citation, commencement and interpretation]{Citation, commencement and interpretation}

1.—(1) These Regulations may be cited as the Child Support (Miscellaneous Amendments) (No.\ 2) Regulations 1999 and shall, subject to paragraph (2), come into force on 1st June 1999.

(2) Regulation 26($a$) shall come into force on 29th November 1999.

(3) In these Regulations---
\begin{enumerate}\item[]
“the Departure Direction Regulations” means the Child Support Departure Direction and Consequential Amendments Regulations 1996\footnote{\frenchspacing S.I. 1996/2907; the relevant amending instruments are S.I. 1998/58 and S.I. 1998/2799.}; and

“the Maintenance Assessment Procedure Regulations” means the Child Support (Maintenance Assessment Procedure) Regulations 1992\footnote{\frenchspacing S.I. 1992/1813; the relevant amending instruments are S.I. 1993/913, S.I. 1994/227, S.I. 1995/123, S.I. 1995/1045, S.I. 1995/3261, S.I. 1995/3265, S.I. 1996/1345, S.I. 1996/1945, S.I. 1996/2907, S.I. 1996/3196, S.I. 1998/58, S.I. 1998/2799 and S.I. 1999/977.}.
\end{enumerate}

(4) In these Regulations–
($a$)in Part I, any reference to a regulation or a Schedule is a reference to a regulation of, or a Schedule to, the Maintenance Assessment Procedure Regulations; and
(b)In Part II, any reference to a regulation or a Schedule is a reference to a regulation of, or a Schedule to, the Departure Direction Regulations.
(5) In the Schedule to this Instrument, “the Act” means the Social Security Act 1998(4).
PART IAmendment of the Maintenance Assessment Procedure RegulationsAmendment of regulation 12.  In regulation 1 (citation, commencement and interpretation)–
($a$)in paragraph (2), after the definition of “obligation imposed by section 6 of the Act” there shall be inserted the following definition–
““official error” means an error made by an officer of the Department of Social Security acting as such which no person outside that Department caused or to which no person outside that Department materially contributed;”;
(b)in paragraph (7) for the words “8(6), 24(2), 29(3) or 31(6)($a$)” there shall be substituted the words “9(1) or 18(4)”.
Amendment of regulation 33.  In regulation 3(2)($a$) (applications on the termination of a maintenance assessment), for the words “a child support officer” there shall be substituted the words “the Secretary of State”.
Amendment of regulation 74.  In regulation 7 (death of a qualifying child)–
($a$)in paragraph (1), for the words “the child support officer concerned” there shall be substituted the words “the Secretary of State”;
(b)in paragraph (2), for the words “the child support officer” there shall be substituted the words “the Secretary of State”.
Amendment of regulation 85.  In regulation 8(5) (categories of interim maintenance assessment)–
($a$)in paragraph (1), for the words “a child support officer” there shall be substituted the words “the Secretary of State”;
(b)in paragraph (3)–
(i)for the words “a child support officer” in each place in which they occur there shall be substituted the words “the Secretary of State”;
(ii)the words “or the child support officer” shall be omitted.
Amendment of regulation 8A6.  In regulation 8A(6) (amount of an interim maintenance assessment) for the words–
($a$)“a child support officer”; and
(b)“the child support officer”,
in each place in which they occur there shall be substituted the words “the Secretary of State”.
Revocation of regulation 8B7.  Regulation 8B(7) (review of an interim maintenance assessment) is hereby revoked.
Amendment of regulation 8C8.  In regulation 8C(8) (effective date of an interim maintenance assessment)–
($a$)in paragraph (1), the words “9(9) or” shall be omitted;
(b)in paragraph (1)($a$), the words “regulations 8B, 9(2) and (3) and” shall be omitted;
(c)in paragraph (1)(b), the words “and to regulations 31 to 31C” shall be omitted;
(d)in paragraph (1)(c), the words “and regulations 31 to 31C” shall be omitted;
(e)in paragraph (2)–
(i)for the words from the beginning to “regulations 31 to 31C”, there shall be substituted the words “The effective date of an interim maintenance assessment made under section 12(1)(b) of the Act shall, subject to regulation 33(7)”;
(ii)for the words “is being reviewed” there shall be substituted the words “the Secretary of State is proposing to supersede with a decision under section 17 of the Act”;
(f)in paragraph (3), the words “, regulation 8B or 9(2), (3) or (9),” shall be omitted.
Amendment of regulation 8D9.  In regulation 8D(9) (miscellaneous provisions in relation to interim maintenance assessments)–
($a$)paragraph (3) shall be omitted;
(b)in paragraph (4), for the words “29, 31 to 31C, 32, 33(5) and 55” there shall be substituted the words “32 and 33(5)”;
(c)in paragraph (5)–
(i)the words “and regulation 9(15)” shall be omitted; and
(ii)for the words “a child support officer” there shall be substituted the word “him”;
(d)in paragraph (6)–
(i)the words “Subject to regulation 9(15)” shall be omitted; and
(ii)for the words “a child support officer” there shall be substituted the words “the Secretary of State”;
(e)in paragraph (7), for the words “a child support officer” there shall be substituted the words “the Secretary of State”.
Substitution of regulations 9 and 9A10.  For regulations 9 and 9A(10) there shall be substituted the following regulation–
“Interim maintenance assessments which follow other interim maintenance assessments9.—(1) Where an interim maintenance assessment is being revised on the ground specified in regulation 17(1)(b) and the Secretary of State is satisfied–
($a$)that another Category A, Category B or Category D maintenance assessment should be made, and
(b)that there has been unavoidable delay for part of the period during which the assessment which is being revised was in force,
the effective date of that other–
(i)Category A or Category D interim maintenance assessment shall be the first day of the maintenance period following the date upon which, in the opinion of the Secretary of State, the delay became avoidable;
(ii)Category B interim maintenance assessment shall be the date set out in regulation 8C(1)(b).
(2) Where an interim maintenance assessment is revised on either of the grounds set out in regulation 17(4) or (5), payments made under that interim maintenance assessment before the revision shall be treated as payments made under the Category B interim maintenance assessment which replaces it.
(3) Subject to paragraphs (5) and (6), where the Secretary of State makes a Category B interim maintenance assessment following the revision of an interim maintenance assessment in accordance with regulation 17(4), the effective date of that Category B interim maintenance assessment shall be the date determined in accordance with regulation 8C(1)(b).
(4) Where the Secretary of State makes a fresh interim maintenance assessment following the supersession of an interim maintenance assessment in accordance with regulation 20(7), the effective date of that fresh interim maintenance assessment shall be the date from which that supersession took effect.
(5) Where the Secretary of State cancels upon a revision an interim maintenance assessment in accordance with regulation 17(4) which caused a court order to cease to have effect in accordance with regulation 3(6) of the Maintenance Arrangements and Jurisdiction Regulations, the effective date of the Category B interim maintenance assessment referred to in regulation 17(4) shall be the date on which that revision took effect.
(6) Where the revision of an interim maintenance assessment in accordance with regulation 17(5) caused a court order to cease to have effect in accordance with regulation 3(6) of the Maintenance Arrangements and Jurisdiction Regulations, the effective date of the Category B interim maintenance assessment referred to in regulation 17(4) shall be the date on which that revision took effect.”.
Amendment of regulation 1011.  In regulation 10(11) (notification of a new or a fresh maintenance assessment)–
($a$)for paragraph (1) there shall be substituted the following paragraph–
“(1) A person with a right of appeal to an appeal tribunal under–
($a$)section 20 of the Act(12); and
(b)section 20 of the Act as extended by paragraph 3(1)(b) of Schedule 4C to the Act(13),
shall be given notice of that right and of the decision to which that right relates.”;
(b)paragraphs (1A) to (1C) shall be omitted;
(c)in paragraph (2)–
(i)after the words “paragraph (1)” there shall be inserted the words “of a new or fresh maintenance assessment made under section 11, 16 or 17”; and
(ii)in sub-paragraph (i), the words “or (b)” shall be omitted;
(d)in paragraph (2A), for the words “a review under section 16 of the Act of a maintenance assessment the effective date of which is on or before 8th December 1996 or a review under section 19(1)” there shall be substituted the words “a revision of a maintenance assessment under section 16 of the Act or a supersession of a maintenance assessment under section 17 of the Act”;
(e)in paragraph (3)($a$), for the words “of the child support officer concerned” there shall be substituted the words “of the officer concerned who is exercising functions of the Secretary of State under the Act”;
(f)for paragraphs (4) to (6) there shall be substituted the following paragraph–
“(4) Where a decision as to a maintenance assessment is made under section 11, 12, 16 or 17 of the Act, a notification under paragraph (1) shall include information as to the provisions of sections 16 and 17 of the Act.”.
Amendment of regulation 10A12.  In regulation 10A(14) (notification of increase or reduction in the amount of a maintenance assessment)–
($a$)in paragraph (1), for the words–
(i)“a child support officer” there shall be substituted the words “the Secretary of State”;
(ii)“section 18” there shall be substituted the words “sections 16 and 17”;
(b)in paragraph (2)($a$) for the words “of the child support officer concerned” there shall be substituted the words “of the officer concerned who is exercising functions of the Secretary of State under the Act”.
Revocation of regulations 11 to 15A13.  Regulations 11 to 15A(15) are hereby revoked.
Amendment of regulation 1614.  In regulation 16(16) (periodical reviews), for the words “a child support officer” there shall be substituted the words “the Secretary of State”.
Substitution of regulation 16A15.  For regulation 16A(17) (notification that an appeal has lapsed) there shall be substituted the following regulation–
“Notification that an appeal has lapsed16A.  Where an appeal lapses in accordance with section 16(6) of the Act, the Secretary of State shall, so far as is reasonably practicable, notify the relevant persons that that appeal has lapsed.”.
Substitution of Parts V to VII16.  For Parts V to VII of the Maintenance Assessment Procedure Regulations(18) there shall be substituted the following Part–
“PART VRevisions and SupersessionsRevision of decisions17.—(1) Subject to paragraphs (6) and (8), any decision may be revised by the Secretary of State–
($a$)if the Secretary of State receives an application for the revision of a decision under section 16 of the Act within one month of the date of notification of the decision or within such longer time as may be allowed by regulation 18;
(b)if–
(i)the Secretary of State notifies a person, who applied for a decision to be revised within the period specified in sub-paragraph ($a$), that the application is unsuccessful because the Secretary of State is not in possession of all of the information or evidence needed to make a decision; and
(ii)that person reapplies for a decision to be revised within one month of the notification described in head (i) above, or such longer period as the Secretary of State is satisfied is reasonable in the circumstances of the case, and provides in that application sufficient evidence or information to enable a decision to be made;
(c)if the decision arose from an official error;
(d)if the Secretary of State is satisfied that the original decision was erroneous due to a misrepresentation of, or failure to disclose, a material fact and that the decision was more advantageous to the person who misrepresented or failed to disclose that fact than it would otherwise have been but for that error; or
(e)if the Secretary of State commences action leading to the revision of a decision within one month of the date of notification of the decision.
(2) A decision may be revised by the Secretary of State in consequence of a departure direction where that departure direction takes effect on the effective date.
(3) Subject to regulation 20(6) a decision of the Secretary of State under section 12 of the Act may be revised where–
($a$)the Secretary of State receives information which enables him to make a maintenance assessment calculated in accordance with Part I of Schedule 1 to the Act for the whole of the period beginning with the effective date applicable to a particular case; or
(b)the Secretary of State is satisfied that there was unavoidable delay by the absent parent in–
(i)completing and returning a maintenance enquiry form under the provisions of regulation 6(1);
(ii)providing information or evidence that is required by him for the determination of an application for a maintenance assessment; or
(iii)providing information or evidence that is required by him to enable him to revise a decision under section 16 of the Act or supersede a decision under section 17 of the Act.
(4) Where an interim maintenance assessment is in force which is not a Category B interim maintenance assessment and the Secretary of State is satisfied that it would be appropriate to make a Category B interim maintenance assessment, he may revise the interim maintenance assessment which is in force.
(5) Where the Secretary of State revises an interim maintenance assessment in accordance with paragraph (4) and that interim maintenance assessment was made immediately following a previous interim maintenance assessment, he may also revise that previous interim maintenance assessment.
(6) Paragraph (1) shall apply neither–
($a$)in respect of a material change of circumstances which–
(i)occurred since the date as from which the decision had effect; or
(ii)is expected, according to information or evidence which the Secretary of State has, to occur; nor
(b)where–
(i)an appeal against a decision has been brought but not determined; and
(ii)from the point of view of the appellant, a revision of that decision, if made, would be less to his advantage than the original decision.
(7) In paragraphs (1), (2) and (6) and regulation 18(3) “decision” means a decision of the Secretary of State under section 11 or 12 of the Act and any supersession of such a decision.
(8) Paragraph (1) shall apply in relation to–
($a$)any decision of the Secretary of State with respect to a reduced benefit direction or a person’s liability under section 43 of the Act; and
(b)the supersession of any such decision under section 17 as extended by paragraph 2 of Schedule 4C to the Act,
as it applies in relation to any decision of the Secretary of State under sections 11, 12 or 17 of the Act.
Late applications for a revision18.—(1) The period of one month specified in regulation 17(1)($a$) may be extended where the requirements specified in the following provisions of this regulation are met.
(2) An application for an extension of time shall be made by a relevant person or a person acting on his behalf.
(3) An application for an extension of time under this regulation shall–
($a$)be made within 13 months of the date on which notification of the decision which it is sought to have revised was given or sent; and
(b)contain particulars of the grounds on which the extension of time is sought and shall contain sufficient details of the decision which it is sought to have revised to enable that decision to be identified.
(4) The application for an extension of time shall not be granted unless the person making the application or any person acting for him satisfies the Secretary of State that–
($a$)it is reasonable to grant that application;
(b)the application for a decision to be revised has merit; and
(c)special circumstances are relevant to the application for an extension of time,
and as a result of those special circumstances, it was not practicable for the application for a decision to be revised to be made within one month of the date of notification of the decision which it is sought to have revised.
(5) In determining whether it is reasonable to grant an application for an extension of time, the Secretary of State shall have regard to the principle that the greater the time that has elapsed between the expiration of the period of one month described in regulation 17(1)($a$) from the date of notification of the decision which it is sought to have revised and the making of the application for an extension of time, the more compelling should be the special circumstances on which the application is based.
(6) In determining whether it is reasonable to grant the application for an extension of time, no account shall be taken of the following–
($a$)that the person making the application for an extension of time or any person acting for him was unaware of or misunderstood the law applicable to his case (including ignorance or misunderstanding of the time limits imposed by these Regulations); or
(b)that a Child Support Commissioner or a court has taken a different view of the law from that previously understood and applied.
(7) An application under this regulation for an extension of time which has been refused may not be renewed.
Date from which revised decision takes effect19.  Where the date from which a decision took effect is found to be erroneous on a revision under section 16 of the Act, the revision shall take effect from the date on which the revised decision would have taken effect had the error not been made.
Supersession of decisions20.—(1) Subject to paragraphs (9) and (10), for the purposes of section 17 of the Act, the cases and circumstances in which a decision (“a superseding decision”) may be made under that section are set out in paragraphs (2) to (7).
(2) A decision may be superseded by a decision made by the Secretary of State acting on his own initiative–
($a$)where he is satisfied that the decision is one in respect of which there has been a material change of circumstances since the decision was made;
(b)where he is satisfied that the decision was made in ignorance of, or was based upon a mistake as to, some material fact; or
(c)in consequence of a departure direction or of a revision or supersession of a decision with respect to a departure direction.
(3) Except where paragraph (8) applies, a decision may be superseded by a decision made by the Secretary of State where–
($a$)an application is made on the basis that–
(i)there has been a change of circumstances since the decision was made; or
(ii)it is expected that a change of circumstances will occur; and
(b)the Secretary of State is satisfied that the change of circumstances is or would be material.
(4) A decision may be superseded by a decision made by the Secretary of State where–
($a$)an application is made on the basis that the decision was made in ignorance of, or was based upon a mistake as to, a fact; and
(b)the Secretary of State is satisfied that the fact is or would be material.
(5) A decision, other than a decision given on appeal, may be superseded by a decision made by the Secretary of State–
($a$)acting on his own initiative where he is satisfied that the decision was erroneous in point of law; or
(b)where an application is made on the basis that the decision was erroneous in point of law.
(6) An interim maintenance assessment may be superseded by a decision made by the Secretary of State where he receives information which enables him to make a maintenance assessment calculated in accordance with Part I of Schedule 1 to the Act for a period beginning after the effective date of that interim maintenance assessment.
(7) Subject to paragraphs (4) and (5) of regulation 17, where the Secretary of State is satisfied that it would be appropriate to make an interim maintenance assessment the category of which is different from that of the interim maintenance assessment which is in force, he may make a decision which supersedes the interim maintenance assessment which is in force.
(8) This paragraph applies–
($a$)where any paragraph of regulation 21 applies; and
(b)in the case of a Category A or Category D interim maintenance assessment.
(9) The cases and circumstances in which a decision may be superseded shall not include any case or circumstance in which a decision may be revised.
(10) Paragraphs (2) to (6) shall apply neither in respect of–
($a$)a decision to refuse an application for a maintenance assessment; nor
(b)a decision to cancel a maintenance assessment.
(11) For the purposes of section 17 of the Act as extended by paragraph 2 of Schedule 4C to the Act, paragraphs (2) to (5) shall apply in relation to–
($a$)a decision with respect to a reduced benefit direction or a person’s liability under section 43 of the Act; and
(b)any decision of the Secretary of State under section 17 of the Act as extended by paragraph 2 of Schedule 4C to the Act, whether as originally made or as revised under section 16 of the Act as extended by paragraph 1 of Schedule 4C to the Act, as they apply in relation to any decision as to a maintenance assessment save that paragraph (8) shall not apply in respect of such a decision.
Circumstances in which a decision may not be superseded21.—(1) A decision of the Secretary of State shall not be superseded in any of the circumstances specified in the following paragraphs of this regulation.
(2) Except where paragraph (3) or (4) applies and subject to paragraph (5) and regulation 22, this paragraph applies where the difference between–
($a$)the amount of child support maintenance (“the amount”) fixed in accordance with the original decision; and
(b)the amount which would be fixed in accordance with a superseding decision,
is less than £10.00 per week.
(3) Subject to paragraph (5), this paragraph applies where the circumstances of the absent parent are such that the provisions of paragraph 6 of Schedule 1 to the Act would apply and either–
($a$)the amount fixed in accordance with the original decision is less than the amount that would be fixed in accordance with a superseding decision and the difference between the two amounts is less than £5.00 per week; or
(b)the amount fixed in accordance with the original decision is more than the amount that would be fixed in accordance with the superseding decision and the difference between the two amounts is less than £1.00 per week.
(4) Subject to paragraph (5), this paragraph applies where–
($a$)the children, in respect of whom child support maintenance would be fixed in accordance with a superseding decision, are not the same children for whom child support maintenance was fixed in accordance with the original decision; and
(b)the difference between–
(i)the amount of child support maintenance (“the amount”) fixed in accordance with the original decision; and
(ii)the amount which would be fixed in accordance with a superseding decision, is less than £1.00 per week.
(5) This regulation shall not apply where–
($a$)the absent parent is, by virtue of paragraph 5(4) of Schedule 1 to the Act, to be taken for the purposes of that Schedule to have no assessable income;
(b)the case falls within paragraph 7(2) of Schedule 1 to the Act; or
(c)it appears to the Secretary of State that the case no longer falls within paragraph 5(4) of Schedule 1 to the Act.
(6) In this regulation–
“original decision” means the decision which would be superseded but for the application of this regulation; and
“superseding decision” means a decision which would supersede the original decision but for the application of this regulation.
Special cases and circumstances for which regulation 21 is modified22.  Where an application is made for a supersession on the basis of a change of circumstances which is relevant to more than one maintenance assessment, regulation 21 shall apply with the following modifications–
($a$)before the word “amount” in each place it occurs there shall be inserted the word “aggregate”; and
(b)for the word “decision” in each place it occurs there shall be substituted the word “decisions”.
Date from which a decision is superseded23.—(1) Except in a case to which paragraph (2) applies, where notice is given under regulation 24 in the period which begins 28 days before an application for a supersession is made and ends 28 days after that application is made, the superseding decision of which notice was given under regulation 24 shall take effect as from the first day of the maintenance period in which that application was made.
(2) Where a decision is superseded by a decision made by the Secretary of State in a case to which regulation 20(2)($a$) applies on the basis of evidence or information which was also the basis of a decision made under section 9 or 10 of the Social Security Act 1998 the superseding decision under section 17 shall take effect as from the first day of the maintenance period in which that evidence or information was first brought to the attention of an officer exercising the functions of the Secretary of State under the Act.
(3) Where a superseding decision is made in a case to which either paragraph (2)(b) or (5)($a$) of regulation 20 applies, the decision shall take effect as from the first day of the maintenance period in which the decision was made.
(4) Where a superseding decision is made in a case to which regulation 20(3)($a$)(i), (4) or (5)(b) applies, the decision shall take effect as from the first day of the maintenance period in which the application for a supersession was made.
(5) Where a superseding decision is made in a case to which regulation 20(3)($a$)(ii) applies, the decision shall take effect as from the first day of the maintenance period in which the change of circumstances is due to occur.
(6) Subject to paragraphs (1), (3) and (14), in a case to which regulation 24 applies, a superseding decision shall take effect as from the first day of the maintenance period in which falls the date which is 28 days after the date on which the Secretary of State gave notice to the relevant persons under that regulation.
(7) For the purposes of paragraph (6), where the relevant persons are notified on different dates, the period of 28 days shall be counted from the date of the latest notification.
(8) For the purposes of paragraphs (6) and (7)–
($a$)notification includes oral and written notification;
(b)where a person is notified in more than one way, the date on which he is notified is the date on which he was first given notification; and
(c)the date of written notification is the date on which it was handed or sent to the person.
(9) Regulation 1(6) shall not apply in a case to which paragraph (8)(c) applies.
(10) Where–
($a$)a decision made by an appeal tribunal under section 20 of the Act or by a Child Support Commissioner is superseded on the ground that it was erroneous due to a misrepresentation of, or that there was a failure to disclose, a material fact; and
(b)the Secretary of State is satisfied that the decision was more advantageous to the person who misrepresented or failed to disclose that fact than it would otherwise have been but for that error,
the superseding decision shall take effect as from the date the decision of the appeal tribunal or, as the case may be, the Child Support Commissioner took, or was to take effect.
(11) Any decision given under section 17 of the Act in consequence of a determination which is a relevant determination for the purposes of section 28ZC of the Act(19) (restrictions on liability in certain cases of error) shall take effect as from the date of the relevant determination.
(12) Where the Secretary of State supersedes a decision in accordance with regulation 20(6), the superseding decision shall take effect as from the first day of the maintenance period in which the Secretary of State has received the information referred to in that paragraph.
(13) Where the Secretary of State supersedes a decision in accordance with regulation 20(7), the superseding decision shall take effect as from the first day of the maintenance period in which the Secretary of State became satisfied that it would be appropriate to make an interim maintenance assessment the category of which is different from that of the maintenance assessment which is in force.
(14) Where a decision is superseded in consequence of a departure direction or a revision or supersession of a decision with respect to a departure direction–
($a$)paragraph (6) above shall not apply; and
(b)the superseding decision shall take effect as from the date on which the departure direction or, as the case may be, the revision or supersession, took effect.
(15) Where a decision with respect to a reduced benefit direction is superseded because the direction ceases to be in force in accordance with regulation 41($a$), the superseding decision shall have effect as from–
($a$)where the direction is in operation immediately before it ceases to be in force, the last day of the benefit week during the course of which the parent concerned complied with the obligations imposed by section 6 of the Act; or
(b)where the direction is suspended immediately before it ceases to be in force, the date on which the parent concerned complied with the obligations imposed by section 6 of the Act.
(16) Where a decision with respect to a reduced benefit direction is superseded because the direction ceases to be in force in accordance with regulation 41(b), the superseding decision shall have effect as from–
($a$)where the direction is in operation immediately before it ceases to be in force, the last day of the benefit week during the course of which the application under regulation 41(b) was made; or
(b)where the direction is suspended immediately before it ceases to be in force, the date on which the application under regulation 41(b) was made.
(17) Where a decision with respect to a reduced benefit direction is superseded because the direction ceases to be in force in accordance with regulation 41(c) or (d), the superseding decision shall have effect as from–
($a$)where the direction is in operation immediately before it ceases to be in force, the last day of the benefit week during the course of which the Secretary of State is supplied with information that enables him to make the assessment;
(b)where the direction is suspended immediately before it ceases to be in force, the date on which the Secretary of State is supplied with information that enables him to make the assessment.
(18) Where a decision with respect to a reduced benefit direction is superseded because the direction ceases to be in force in accordance with regulation 47(1), the superseding decision shall have effect as from the last day of the benefit week preceding the benefit week on the first day of which, in accordance with the provisions of regulation 36(4), the further direction comes into operation, or would come into operation but for the provisions of regulation 40 or 40ZA.
Procedure where the Secretary of State proposes to supersede a decision on his own initiative24.  Where the Secretary of State on his own initiative proposes to make a decision superseding a decision other than in consequence of a decision with respect to a departure direction or a revision or supersession of such a decision he shall notify the relevant persons who could be materially affected by the decision of that intention.”.
Amendment of regulation 3017.  In paragraphs (2A) and (4) of regulation 30(20) (effective dates of new maintenance assessments), for the words “a child support officer” in each place in which they occur there shall be substituted the words “the Secretary of State”.
Amendment of regulation 30A18.  In regulation 30A(21) (effective dates of new maintenance assessments in particular cases)–
($a$)in paragraphs (2), (4)(c) and (6), for the words “a child support officer” in each place in which they occur there shall be substituted the words “the Secretary of State”;
(b)in paragraph (5), for the words “a child support officer” there shall be substituted the word “him”.
Revocation of regulations 31 to 31C19.  Regulations 31 to 31C(22) are hereby revoked.
Amendment of regulation 3220.  In regulation 32 (cancellation of a maintenance assessment)–
($a$)for the words “a child support officer” there shall be substituted the words “the Secretary of State”;
(b)for the words “the child support officer” there shall be substituted the word “he”.
Amendment of regulation 32A21.  In regulation 32A(23) (cancellation of maintenance assessments made under section 7 of the Act where the child is no longer habitually resident in Scotland)–
($a$)in paragraph (1), for the words “a child support officer” there shall be substituted the words “the Secretary of State”;
(b)in paragraph (2), for the words “the child support officer” there shall be substituted the words “the Secretary of State”.
Amendment of regulation 32B22.  In regulation 32B(24) (notification of intention to cancel a maintenance assessment under paragraph 16(4A) of Schedule 1 to the Act), for the words “a child support officer” in each place in which they occur there shall be substituted the words “the Secretary of State”.
Amendment of regulation 3323.  In regulation 33(3)(25) (maintenance periods), for the words from “following a review under section 16” to the words “17, 18 or 19 of the Act” there shall be substituted the words “made upon the supersession of a decision under section 17 of the Act”.
Substitution of regulation 3524.  For regulation 35(26) (periods for compliance with obligations imposed by section 6 of the Act) there shall be substituted–
“Periods for compliance with obligations imposed by section 6 of the Act35.  The period specified for the purposes of section 46(2) of the Act is–
($a$)except where paragraph (b) applies, four weeks from the date on which the Secretary of State serves notice under that subsection; or
(b)eight weeks from that date where the Secretary of State has received, within two weeks of serving that notice, a statement in writing from the parent with care which sets out the reasons why she believes that, if she were to be required to comply with an obligation imposed by section 6 of the Act, there would be a risk, as a result of that compliance, of her or any child or children living with her suffering harm or undue distress.”.
Amendment of regulation 35A25.  In regulation 35A(27) (circumstances in which a reduced benefit direction shall not be given), for the words “A child support officer” there shall be substituted the words “The Secretary of State”.
Amendment of regulation 3626.  In regulation 36(28) (amount of and period of reduction of relevant benefit under a reduced benefit direction)–
($a$)in paragraph (4), for the words “the adjudication officer” there shall be substituted the words “the Secretary of State”;
(b)in paragraph (5C), for the words “a child support officer” there shall be substituted the words “the Secretary of State”.
Amendment of regulation 3827.  In regulation 38 (suspension of a reduced benefit direction when relevant benefit ceases to be payable), in paragraph (6), for the words “a child support officer” there shall be substituted the words “the Secretary of State”.
Substitution of regulations 41 to 4628.  For regulations 41 to 46(29) there shall be substituted the following regulation–
“Termination of reduced benefit direction41.  A reduced benefit direction shall cease to be in force–
($a$)where a parent with care, with respect to whom such a direction is in force, complies with the obligations imposed by section 6 of the Act;
(b)upon an application made for the purpose where the Secretary of State is satisfied that a parent with care, with respect to whom such a direction is in force, should not be required to comply with the obligations imposed by section 6 of the Act;
(c)where a qualifying child of a parent with respect to whom a direction is in force applies for a maintenance assessment to be made with respect to him under section 7 of the Act and an assessment is made in response to that application in respect of all of the qualifying children in relation to whom the parent concerned failed to comply with the obligations imposed by section 6 of the Act; or
(d)where–
(i)an absent parent applies for a maintenance assessment to be made under section 4 of the Act with respect to all of his qualifying children in relation to whom the other parent of those children is a person with care;
(ii)a direction is in force with respect to that other parent following her failure to comply with the obligations imposed by section 6 of the Act in relation to those qualifying children; and
(iii)an assessment is made in response to that application by the absent parent for a maintenance assessment.”.
Amendment of regulation 4729.  In regulation 47(30) (reduced benefit directions where there is an additional qualifying child)–
($a$)in paragraph (1)–
(i)for the words “a child support officer” there shall be substituted the words “the Secretary of State”;
(ii)the words from “on the last day” to the end shall be omitted;
(b)in paragraph (3)(b), for the words “a child support officer” there shall be substituted the words “the Secretary of State”.
Substitution of regulation 4930.  For regulation 49 (notice of termination of a reduced benefit direction) there shall be substituted the following regulation–
“Notice of termination of a reduced benefit direction49.  Where a direction ceases to be in force under the provisions of regulations 41, 47 or 48, or is suspended under the provisions of regulation 48, the Secretary of State shall serve notice of such a termination or suspension, as the case may be, on the parent concerned and shall specify the date on which the direction ceases to be in force or is suspended, as the case may be.”.
Revocation of regulations 52 and 54 to 5731.  Regulations 52 and 54 to 57(31) are hereby revoked.
Amendment of Schedule 132.  In paragraph 4(1) of Schedule 1 (meaning of “child” for the purposes of the Act) for the words “a child support officer” there shall be substituted the words “the Secretary of State”.
Amendment of Schedule 233.  In Schedule 2(32) (multiple applications)–
($a$)in paragraph 3–
(i)in sub-paragraph (1), for the words from “refer each such application” to “the child support officer shall” there shall be substituted the words “, if no maintenance assessment has been made in relation to any of the applications,”;
(ii)in sub-paragraphs (2) to (14), for the words “the child support officer” in each place in which they occur there shall be substituted the words “the Secretary of State”; and
(b)in paragraph 4, the words from “unless the Secretary of State” to the end shall be omitted.
PART IIAmendment of the Departure Direction RegulationsAmendment of regulation 134.  In regulation 1(2) (citation, commencement and interpretation)–
($a$)in the definition of “application” after the word “means” there shall be inserted the words “, except in regulations 32A to 32G,”;
(b)after the definition of “non-applicant” there shall be inserted the following definition–
““official error” means an error made by an officer of the Department of Social Security acting as such which no person outside the Department caused or to which no person outside the Department materially contributed;”.
Amendment of regulation 435.  Paragraphs (11) to (14) of regulation 4 (application for departure direction) shall be omitted.
Amendment of regulation 636.  In regulation 6(2) (provision of information) for the words “14 days” there shall be substituted the words “one month, or such longer period as the Secretary of State is satisfied is reasonable in the circumstances of the case,”.
Amendment of regulation 837.  In regulation 8 (procedure in relation to the determination of an application)–
($a$)in paragraph (3)(b) the words “of 14 days” shall be omitted;
(b)in paragraph (8)(b)(i)–
(i)the words “or by a child support officer,” shall be omitted;
(ii)for the words “for a review of” there shall be substituted the words “for a revision or a supersession of”;
(c)in paragraph (9)(b), for the words “refer the case to a child support officer” there shall be substituted the words “make a decision in accordance with regulation 17(2) or 20(2)(c) of the Maintenance Assessment Procedure Regulations”;
(d)paragraph (11) shall be omitted.
Insertion of regulation 8A38.  After regulation 8 there shall be inserted the following regulation–
“Procedure in relation to determination of an application for a revision or a supersession of a decision with respect to a departure direction8A.—(1) Subject to the modifications described in paragraph (2), regulation 8 shall apply to any application for a revision or a supersession of a decision with respect to a departure direction as it applies to an application for a departure direction.
(2) The modifications described in this paragraph are–
($a$)for paragraph (1) there shall be substituted the following paragraphs–
“(1) Except where paragraph (1A) applies, the Secretary of State shall–
($a$)give notice of an application for a revision or a supersession of a decision with respect to a departure direction to the relevant persons other than the applicant;
(b)inform them of the grounds on which the application has been made and any relevant information or evidence the applicant has given, except details, information or evidence falling within paragraph (2);
(c)invite representations from the relevant persons other than the applicant on any matter relating to that application; and
(d)explain the provisions of paragraphs (2), (5) and (6) in relation to such representations.
(1A) This paragraph applies where an application for a revision or a supersession has been made and the Secretary of State is satisfied on the information or evidence available to him that either–
($a$)a revision or supersession of a departure direction is unlikely to be made; or
(b)in a case where the applicant was the applicant for the decision which is to be revised or superseded, a ground on which the decision to be revised or superseded was made no longer applies.”;
(b)paragraphs (3), (4) and (7) shall be omitted;
(c)in paragraph (4A) for the words from “that a departure direction” to the words “in that case” there shall be substituted the words “that a decision revising or superseding a decision with respect to a departure direction was unlikely to be made, but on further consideration of the application he is minded to make such a decision”;
(d)in paragraph (5)–
(i)for the words “(1), (6) or (7)” there shall be substituted the words “(1) or (6)”;
(ii)after the word “application” there shall be added the words “for a decision revising or superseding a decision”;
(e)in paragraph (8)–
(i)for the words “In deciding whether to give a departure direction” there shall be substituted the words “Before deciding whether or not to make a decision revising or, as the case may be, superseding a decision as to a departure direction in consequence of an application for such a decision”; and
(ii)in sub-paragraph ($a$), for the words “by the applicant for that direction” there shall be substituted the words “in connection with the application”;
(f)for paragraphs (9) and (10) there shall be substituted the following paragraph–
“(9) Where the Secretary of State has determined an application made for the purpose of revising or superseding a decision he shall, as soon as is reasonably practicable, notify the relevant persons of–
($a$)that determination;
(b)the reasons for it; and
(c)where appropriate, the basis on which the amount of child support maintenance is to be fixed by any fresh assessment made in consequence of that determination.””
Revocation of regulation 1139.  Regulation 11 (departure application and review under section 17 of the Act) is hereby revoked.
Substitution of regulation 11A40.  For regulation 11A(33) (meaning of “current assessment” for the purposes of the Act) there shall be substituted–
“11A.  Where–
($a$)an application under section 28A of the Act has been made in respect of a current assessment; and
(b)after the making of that application, a fresh maintenance assessment has been made upon a revision of a decision as to a maintenance assessment under section 16 of the Act,
references to the current assessment in sections 28B(3), 28C(2)($a$) and 28F(5) of, and in paragraph 8 of Schedule 4A and paragraphs 2, 3 and 4 of Schedule 4B to, the Act shall have effect as if they were references to the fresh maintenance assessment.”.
Amendment of regulation 1441.  After regulation 14(7) (contact costs) there shall be added the following paragraph–
“(8) This regulation shall apply in relation to an application made for the purpose of superseding a decision with respect to a departure direction as though–
($a$)for the words “at the time a departure direction is applied for” in paragraphs (1) and (7) there were substituted the words “at the time an application is made for a decision superseding a decision with respect to a departure direction”;
(b)in paragraph (5), after the words “an application” there were inserted the words “for the supersession of a decision with respect to a departure direction.””.
Amendment of regulation 1542.  In regulation 15(4)($a$)(34) (illness or disability), for the words, “adjudicating authority” there shall be substituted the words “Secretary of State”.
Amendment of regulation 3243.  In regulation 32(35) (effective date of a departure direction)–
($a$)in paragraphs (1) and (2), for the words “28 days” in each place where they occur there shall be substituted the words “one month”;
(b)in paragraph (3A), for the word “where” there shall be substituted the words “subject to paragraph (3B), where”;
(c)after paragraph (3A) there shall be inserted the following paragraph–
“(3B) For the purposes of paragraph (3A), paragraph (8) of regulation 14 shall not apply.”;
(d)paragraphs (7) and (8) shall be omitted.
Insertion of regulations 32A to 32G44.  After regulation 32 there shall be inserted the following regulations–
“Revision of decisions32A.—(1) Subject to paragraphs (2) and (3), a decision of the Secretary of State or any decision upon referral under section 28D(1)(b) of an appeal tribunal with respect to a departure direction may be revised by the Secretary of State under section 16 of the Act as extended by paragraph 1 of Schedule 4C to the Act–
($a$)if the Secretary of State receives an application for the revision of a decision under section 16 of the Act as extended within one month of the date of notification of the decision or within such longer time as may be allowed by regulation 32B;
(b)if–
(i)the Secretary of State notifies a person, who applied for a decision to be revised within the period specified in sub-paragraph ($a$), that the application is unsuccessful because the Secretary of State is not in possession of all of the information or evidence needed to make a decision; and
(ii)that person reapplies for a decision to be revised within one month of the notification described in head (i) above or such longer period as the Secretary of State is satisfied is reasonable in the circumstances of the case, and provides in that application sufficient information or evidence to enable a decision to be made;
(c)if the decision arose from an official error;
(d)if the Secretary of State is satisfied that the original decision was erroneous due to a misrepresentation of, or failure to disclose, a material fact and that the decision was more advantageous to the person who misrepresented or failed to disclose that fact than it would otherwise have been but for that error;
(e)where a departure direction takes effect in the circumstances described in regulation 35(3); or
(f)if the Secretary of State commences action leading to the revision of a decision within one month of the date of notification of the decision.
(2) Paragraph (1) shall apply neither–
($a$)in respect of a material change of circumstances which–
(i)occurred since the date from which the decision had effect; or
(ii)is expected, according to information or evidence which the Secretary of State has, to occur; nor
(b)where–
(i)an appeal against the original decision has been brought but not determined; and
(ii)from the point of view of the appellant, a revision, if made, would be less to his advantage than the original decision.
Late applications for a revision32B.—(1) The period of one month specified in regulation 32A(1)($a$) may be extended where the requirements specified in the following provisions of this regulation are met.
(2) An application for an extension of time shall be made by a relevant person or a person acting on his behalf.
(3) An application for an extension of time under this regulation shall–
($a$)be made within 13 months of the date on which notification of the decision which it is sought to have revised was given or sent; and
(b)contain particulars of the grounds on which the extension of time is sought and shall contain sufficient details of the decision which it is sought to have revised to enable that decision to be identified.
(4) The application for an extension of time shall not be granted unless the person making the application, or any person acting for him, satisfies the Secretary of State that–
($a$)it is reasonable to grant that application;
(b)the application for the decision to be revised has merit; and
(c)special circumstances are relevant to the application for an extension of time,
and as a result of those special circumstances, it was not practicable for the application for a decision to be revised to be made within one month of the date of notification of the decision which it is sought to have revised.
(5) In determining whether it is reasonable to grant an application for an extension of time, the Secretary of State shall have regard to the principle that the greater the time that has elapsed between the expiration of the period of one month described in regulation 32A(1)($a$) from the date of notification of the decision which it is sought to have revised and the making of the application for an extension of time, the more compelling should be the special circumstances on which the application is based.
(6) In determining whether it is reasonable to grant an application for an extension of time, no account shall be taken of the following–
($a$)that the person making the application for an extension of time or any person acting for him was unaware of or misunderstood the law applicable to his case (including ignorance or misunderstanding of the time limits imposed by these Regulations);
(b)that a Child Support Commissioner or a court has taken a different view of the law from that previously understood and applied.
(7) An application under this regulation for an extension of time which has been refused may not be renewed.
Date from which a revision of a decision takes effect32C.  Where the date from which a decision took effect is found to be erroneous on a revision, the revision shall take effect from the date on which the revised decision would have taken effect had the error not been made.
Supersession of decisions32D.—(1) For the purposes of section 17 of the Act as it applies in relation to decisions with respect to departure directions by virtue of paragraph 2 of Schedule 4C to the Act and subject to paragraphs (6), (9) and (10), the cases and circumstances in which a decision with respect to a departure direction may be made under that section are set out in paragraphs (2) to (5).
(2) A decision may be superseded by a decision made by the Secretary of State acting on his own initiative where he is satisfied that–
($a$)there has been a material change of circumstances since the decision was made; or
(b)the decision was made in ignorance of, or was based upon a mistake as to, some material fact.
(3) A decision may be superseded by a decision made by the Secretary of State where–
($a$)an application is made on the basis that–
(i)there has been a change of circumstances since the decision was made; or
(ii)it is expected that a change of circumstances will occur; and
(b)the Secretary of State is satisfied that the change of circumstances is or would be material.
(4) A decision may be superseded by a decision made by the Secretary of State where–
($a$)an application is made on the basis that the decision was made in ignorance of, or was based upon a mistake as to, a fact; and
(b)the Secretary of State is satisfied that the fact is or would be material.
(5) A decision, other than a decision given on appeal, may be superseded by a decision made by the Secretary of State–
($a$)where an application is made on the basis that the decision was erroneous in point of law; or
(b)acting on his own initiative where he is satisfied that the decision was erroneous in point of law.
(6) Subject to paragraph (7), paragraphs (2)($a$) and (3) shall not apply where, if a decision were to be superseded in accordance with section 17 of the Act, the difference between the current amount and the revised amount would be less than £1.00 per week.
(7) Paragraph (6) shall not apply where the Secretary of State is satisfied on the information or evidence available to him that a ground on which the decision to be superseded was made no longer applies.
(8) In paragraph (6) “revised amount” means the amount of child support maintenance which would be fixed if a decision with respect to a maintenance assessment were to be superseded by a decision made by the Secretary of State in accordance with paragraphs (2)($a$) and (3) but for the operation of paragraph (6).
(9) The cases and circumstances in which a decision may be superseded by a decision made by the Secretary of State shall not include any case or circumstance in which a decision may be revised.
(10) Subject to paragraph (11), paragraphs (2) to (5) shall apply in respect of neither–
($a$)a decision to reject or refuse an application for a departure direction; nor
(b)a decision to cancel a departure direction.
(11) Paragraph (10) above shall not apply in a case to which either paragraph (2) or (3) of regulation 35 applies.
Date from which a superseding decision takes effect32E.—(1) This regulation contains exceptions to the provisions of section 17(4) of the Act, as it applies in relation to decisions with respect to departure directions by virtue of paragraph 2 of Schedule 4C to the Act, as to the date from which decisions which supersede earlier decisions are to take effect.
(2) Subject to paragraphs (3) and (5), where–
($a$)a decision is made by the Secretary of State which supersedes an earlier decision in consequence of an application having been made under section 17 of the Act as it applies in relation to decisions with respect to departure directions by virtue of paragraph 2 of Schedule 4C to the Act; and
(b)the date on which the application is made is not the first day in a maintenance period, the decision shall take effect as from the first day of the maintenance period in which the application is made.
(3) Where a decision is superseded by a decision made by the Secretary of State in a case to which regulation 32D(2)($a$) applies on the basis of evidence or information which was also the basis of a decision made under section 9 or 10 of the Social Security Act 1998 the superseding decision under section 17 of the Act as extended by paragraph 2 of Schedule 7 to the Act shall take effect as from the first day of the maintenance period in which that evidence or information was first brought to the attention of an officer exercising the functions of the Secretary of State under the Act.
(4) Where a decision is superseded by a decision made by the Secretary of State under regulation 32D(3) in consequence of an application made on the basis that a material change of circumstances is expected to occur, the superseding decision shall take effect as from the first day of the maintenance period which immediately succeeds the maintenance period in which the material change of circumstances is expected to occur.
(5) Where the Secretary of State makes, on his own initiative, a decision superseding a decision in consequence of evidence or information contained in an unsuccessful application for a revision of that decision, the superseding decision shall take effect as from the first day of the maintenance period in which that application was made.
(6) Where–
($a$)a decision made by an appeal tribunal under section 20 of the Act as extended by paragraph 3 of Schedule 4C to the Act is superseded on the ground that it was erroneous due to a misrepresentation of, or that there was a failure to disclose, a material fact; and
(b)the Secretary of State is satisfied that the decision was more advantageous to the person who misrepresented or failed to disclose that fact than it would otherwise have been but for that error,
the superseding decision shall take effect as from the date the decision it superseded took, or was to take, effect.
(7) Any decision given under section 17 of the Act as extended by paragraph 2 of Schedule 4C to the Act in consequence of a decision which is a relevant determination for the purposes of section 28ZC of the Act(36) (restrictions on liability in certain cases of error) shall take effect as from the date of the relevant determination.
(8) Where a decision with respect to a departure direction is superseded by a decision under section 17 of the Act as extended by paragraph 2 of Schedule 4C to the Act because the departure direction ceases to have effect in accordance with regulation 35(1), the superseding decision shall have effect as from the date on which the decision that the maintenance assessment is cancelled or ceases to have effect, takes or took effect.
(9) Where the superseding decision referred to in paragraph (8) above is itself superseded by a further decision made under section 17 of the Act as extended by paragraph 2 of Schedule 4C to the Act in the circumstances described in regulation 35(2), that further decision shall have effect as from the effective date of the fresh maintenance assessment.
(10) Where a decision with respect to a departure direction is superseded by a decision under section 17 of the Act as extended by paragraph 2 of Schedule 4C to the Act because the departure direction is suspended in accordance with regulation 35(4), the superseding decision shall have effect as from the effective date of the later interim maintenance assessment or, as the case may be, the interim maintenance assessment which replaces a maintenance assessment.
(11) Where the superseding decision referred to in paragraph (10) above is itself superseded by a further decision under section 17 as extended because the interim maintenance assessment referred to in regulation 35(4)(c) is followed by a maintenance assessment made in accordance with the provisions of Part I of Schedule 1 to the Act or by an interim maintenance assessment to which regulation 10 does not apply, that further decision shall have effect as from the effective date of the fresh maintenance assessment or, as the case may be, interim maintenance assessment.
Cancellation of departure directions32F.  The Secretary of State may cancel a departure direction where–
($a$)regulation 32A(1) applies and he is satisfied that it was not appropriate to have given it; or
(b)regulation 32D applies and he is satisfied that it is no longer appropriate for it to continue to have effect.
Notification of right of appeal, decision and reasons for decision32G.—(1) The Secretary of State shall notify a person with a right of appeal under the Act against the decision under section 16 or 17 of the Act as those sections apply in relation to decisions with respect to departure directions by virtue of paragraphs 1 and 2 of Schedule 4C to the Act with respect to a departure direction of–
($a$)that right;
(b)that decision; and
(c)the reasons for that decision.
(2) A written notice provided under paragraph (1)–
($a$)shall also contain sufficient information to enable a relevant person to exercise a right of appeal; and
(b)shall not contain any information which it is not necessary for a person to have in order to understand how the decision was reached.”.
Revocation of regulations 33 and 3445.  Regulations 33(37) (cancellation of a departure direction following a review under section 16, 17, 18 or 19 of the Child Support Act) and 34 (cancellation of a departure direction on recognition of an error) are hereby revoked.
Amendment of regulation 34A46.  In regulation 34A(3)(38) (correction of accidental errors in departure directions), for the words “section 28H(3) of the Act” there shall be substituted the words “regulation 31(1) (time within which an appeal is to be brought) or, as the case may be, regulation 32(1) (late appeals) of the Social Security and Child Support (Decisions and Appeals) Regulations 1999(39)”.
Amendment of regulation 3547.  In regulation 35 (termination and suspension of departure directions)–
($a$)in paragraph (2)–
(i)for the words “a child support officer” there shall be substituted the words “the Secretary of State”;
(ii)the words from “from the effective date” to the end shall be omitted;
(b)in paragraph (3), the words “from the date that maintenance assessment was cancelled or ceased to have effect” shall be omitted;
(c)in paragraph (4), for the words from “from the effective date” to the end there shall be substituted the words “where the interim maintenance assessment referred to in sub-paragraph (c) is followed by a maintenance assessment made in accordance with the provisions of Part I of Schedule 1 to the Act or by an interim maintenance assessment to which regulation 10 does not apply”.
Amendment of regulation 4148.  In regulation 41(40) (child support maintenance payable where effect of a departure direction would be to decrease an absent parent’s assessable income but case still fell within paragraph 2(3) of Schedule 1 to the Act)–
($a$)in paragraph (5), for the words “the child support officer” there shall be substituted the words “the Secretary of State”;
(b)in paragraph (6), for the words “a review under section 16 of the Act by a child support officer of a maintenance assessment the effective date of which is on or before 8th December 1996 or a revision by the Secretary of State under that section after 6th December 1998, or a review under section 17, 18 or 19 of the Act”, there shall be substituted the words “a decision under section 16 of the Act revising a decision as to a maintenance assessment or a decision under section 17 of the Act superseding a decision as to a maintenance assessment”.
Amendment of regulation 4249.  In regulation 42(4)(41) (application of regulation 41 where there is a transfer of property falling within paragraph 3 of Schedule 4B to the Act) for the words “the child support officer” there shall be substituted the words “the Secretary of State”.
Amendment of regulation 4450.  In regulation 44(5)(42) (maintenance assessment following a departure direction where there is a phased maintenance assessment)–
($a$)for the words from the beginning to “of the Act” there shall be substituted the words “Where the Secretary of State is satisfied that, were a decision as to a fresh maintenance assessment to be made under section 16 or, as the case may be, section 17 of the Act”;
(b)for the words “the reviewed unadjusted departure amount” in each place they occur there shall be substituted the words “the fresh unadjusted departure amount”.
Amendment of regulation 46A51.  In regulation 46A(1)(43) (cases to which regulation 11A applies), for the words “to (c)” there shall be substituted the words “and (b)”.
Amendment of regulation 4752.  In regulation 47 (transitional provisions—application before 2nd December 1996), in paragraph (6)($a$), for the words “to (11)” there shall be substituted the words “to (10)”.
Amendment of regulation 5053.  In regulation 50 (transitional provisions—new maintenance assessment made before 2nd December 1996 whose effective date is on or after 2nd December 1996), for the words “28 days” in each place where they occur there shall be substituted the words “one month”.

\bigskip

Signed 
by authority of the Secretary of State for Social Security.

{\raggedleft
\emph{Angela Eagle}\\*Parliamentary Under-Secretary of State,\\*Department of Social Security

}

29th March 1999

\small

\part[Schedule --- Provisions conferring powers exercised in making these Regulations]{S C H E D U L E\\*Provisions conferring powers exercised in making these Regulations}

\renewcommand\parthead{--- Schedule}

{\footnotesize
\begin{tabulary}{\linewidth}{JJ}
%\begin{longtable}{p{150pt}p{102pt}p{102pt}}
\hline
\itshape Column (1) & \itshape Column (2)\\
\itshape Provision of the Child Support Act 1991(44) & \itshape Relevant Amendments\\
\hline
%\endhead
%\hline
%\endlastfoot
Section 12(2)\\
Section 14(1)\\
Section 16(1) and (4)&The Act, Section 40.\\
Section 17(3) and (5)&The Act, Section 41.\\
Section 20(4)&The Act, Section 42.\\
Section 28G(3) and (4)&The Child Support Act 1995(45), Section 7.\\
Section 46(2)\\
Section 51&The Act, paragraph 46 of Schedule 7.\\
Section 54(46)\\
Schedule 4A, paragraphs 4(2) and 10&The Child Support Act 1995, Schedule 1.\\
Schedule 4B, paragraph 2&The Child Support Act 1995, Schedule 2.\\
Schedule 4C, paragraphs 1 and 2&The Act, paragraph 54 of Schedule 7.\\
\end{tabulary}
%\end{longtable}

}

\part{Explanatory Note}

\renewcommand\parthead{--- Explanatory Note}

\subsection*{(This note is not part of the Regulations)}

These Regulations amend the Child Support (Maintenance Assessment Procedure) Regulations 1992 (S.I. 1992/1813) (“the 1992 Regulations”) and the Child Support Departure Direction and Consequential Amendments Regulations 1996 (S.I. 1996/2907) (“the 1996 Regulations”). They are made in consequence of the amendments to the Child Support Act 1991 (c.\ 48) effected by the Social Security Act 1998 (c.\ 14) which alter the way child support decisions are made and changed at first tier.

  These Regulations amend the 1992 and 1996 Regulations consequential upon the transfer to the Secretary of State of the functions of child support officers under section 13 of the Child Support Act 1991. Regulations 16 and 43 insert regulations into the 1992 Regulations and the 1996 Regulations respectively which describe when and how the Secretary of State may revise or supersede such a decision. The inserted regulations also make provision as to the date a revision or supersession takes effect. These Regulations remove or replace provisions in the 1992 and 1996 Regulations which deal with reviews of decisions and the effective date of assessments made following a review.

  These Regulations do not impose a charge on business.


\end{document}