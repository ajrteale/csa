\documentclass[12pt,a4paper]{article}

\newcommand\regstitle{The Social Security (National Insurance Credits) Amendment Regulations 2007}

\newcommand\regsnumber{2007/2582}

%\opt{newrules}{
\title{\regstitle}
%}

%\opt{2012rules}{
%\title{Child Maintenance and~Other Payments Act 2008\\(2012 scheme version)}
%}

\author{S.I.\ 2007 No.\ 2582}

\date{Made
4th September 2007\\
Laid before Parliament
7th September 2007\\
Coming into~force
1st October 2007
}

%\opt{oldrules}{\newcommand\versionyear{1993}}
%\opt{newrules}{\newcommand\versionyear{2003}}
%\opt{2012rules}{\newcommand\versionyear{2012}}

\usepackage{csa-regs}

\setlength\headheight{27.61603pt}

%\hbadness=10000

\begin{document}

\maketitle

\noindent
The Secretary of State for Work and Pensions, with the concurrence of the Commissioners for Her Majesty’s Revenue and Customs in relation to regulation~4, makes the following Regulations in exercise of the powers conferred by sections 22(5), 122(1) and 175(1) and (3) of, and paragraph~8(1)($d$)  and (1A) of Schedule 1 to, the Social Security Contributions and Benefits Act 1992\footnote{1992 c.~4. Section 122(1) is cited because of the meanings ascribed to “prescribe” and “Inland Revenue”. Section 175(1) is amended by the Social Security Contributions (Transfer of Functions, etc.)\ Act 1999 (c.~2), Schedule 3, paragraph 29(3). Paragraph 8(1A) is inserted by that Act, Schedule 3, paragraph 39(3) and amended by the Welfare Reform and Pensions Act 1999 (c.~30), Schedule 11, paragraph 3. Under section 50(1) of the Commissioners for Revenue and Customs Act 2005 (c.~11), references to the Commissioners of Inland Revenue in enactments are to be taken as references to the Commissioners for Her Majesty’s Revenue and Customs.} and sections 9(1)($a$), 79(4) and 84 of the Social Security Act 1998\footnote{1998 c.~14. Section 84 is cited because of the meaning ascribed to the word “prescribe”.}.

The Social Security Advisory Committee has agreed that proposals in respect of these Regulations should not be referred to it\footnote{\emph{See} section 173(1)($b$)  of the Social Security Administration Act 1992 (c.~5). Paragraph 104($a$)  of Schedule 7 to the Social Security Act 1998 added that Act to the list of “relevant enactments” in respect of which regulations must normally be referred to the Committee.}. 

{\sloppy

\tableofcontents

}

\bigskip

\setcounter{secnumdepth}{-2}

\subsection[1. Citation and commencement]{Citation and commencement}

1.  These Regulations may be cited as the Social Security (National Insurance Credits) Amendment Regulations 2007 and shall come into force on 1st October 2007.

\subsection[2. Amendment of the Social Security (Credits) Regulations 1975]{Amendment of the Social Security (Credits) Regulations 1975}

2.  After regulation~8C of the Social Security (Credits) Regulations 1975\footnote{S.I.~1975/556; relevant amending instruments are S.I.~1977/788, 1978/409, 1982/96, 1983/197, 1987/414, 1987/687, 1988/516, 1988/1545, 1991/387, 1991/2772, 1992/726, 1994/1837, 1995/829, 1995/2558, 1996/2367, 2000/3120 and 2003/521. Credits for approved training are made under regulation~7. Credits for incapacity for work are made under regulation~8B and were made under regulation~9 before the amendments made by S.I.~1996/2367.}, insert—
\begin{quotation}
\subsection*{“Credits for the purposes of entitlement to incapacity benefit following official error}

8D.---(1)  This regulation~applies for the purpose only of enabling a person who was previously entitled to incapacity benefit to satisfy the condition referred to in paragraph~2(3)($a$)  of Schedule 3 to the Contributions and Benefits Act in respect of a subsequent claim for incapacity benefit where his period of incapacity for work is, together with a previous period of incapacity for work, to be treated as one period of incapacity for work under section 30C of that Act.

(2) Where—
\begin{enumerate}\item[]
($a$) a person was previously entitled to incapacity benefit;

($b$) the award of incapacity benefit was as a result of satisfying the condition referred to in paragraph~(1) by virtue of being credited with earnings for incapacity for work or approved training in the tax years from 1993--94 to 2007--08;

($c$) some or all of those credits were credited by virtue of official error derived from the failure to transpose correctly information relating to those credits from the Department for Work and Pensions’ Pension Strategy Computer System to Her Majesty’s Revenue and Customs’ computer system (\lowercase{\textsc{NIRS2}}) or from related clerical procedures;

($d$) that person makes a further claim for incapacity benefit; and

($e$) his period of incapacity for work is, together with the period of incapacity for work to which his previous entitlement referred to in sub-paragraph~($a$)  related, to be treated as one period of incapacity for work under section 30C of the Contributions and Benefits Act,
\end{enumerate}
that person shall be credited with such earnings as may be required to enable the condition referred to in paragraph~(1) to be satisfied.

(3) In this regulation~and in regulations 8E and 8F, “official error” means an error made by—
\begin{enumerate}\item[]
($a$) an officer of the Department for Work and Pensions or an officer of Revenue and Customs acting as such which no person outside the Department or Her Majesty’s Revenue and Customs caused or to which no person outside the Department for Work and Pensions or Her Majesty’s Revenue and Customs materially contributed; or

($b$) a person employed by a service provider and to which no person who was not so employed materially contributed,
\end{enumerate}
but excludes any error of law which is shown to have been an error by virtue of a subsequent decision of a Commissioner or the court.

(4) In paragraph~(3)—
\begin{enumerate}\item[]
“Commissioner” means the Chief Social Security Commissioner or any other Social Security Commissioner and includes a tribunal of three or more Commissioners constituted under section 16(7) of the Social Security Act 1998;

“service provider” means a person providing services to the Secretary of State for Work and Pensions or to Her Majesty’s Revenue and Customs.
\end{enumerate}

\subsection*{Credits for the purposes of entitlement to retirement pension following official error}

8E.---(1)  This regulation~applies for the purpose only of enabling the condition referred to in paragraph~5(3)($a$)  of Schedule 3 to the Contributions and Benefits Act to be satisfied in respect of a claim for retirement pension made by a person (“the claimant”)—
\begin{enumerate}\item[]
($a$) who would attain pensionable age no later than 31st May 2008;

($b$) not falling within sub-paragraph~($a$)  but based on the satisfaction of that condition by another person—
\begin{enumerate}\item[]
(i) who would attain, or would have attained, pensionable age no later than 31st May 2008; or

(ii) in respect of whose death the claimant received a bereavement benefit.
\end{enumerate}
\end{enumerate}

(2) Where—
\begin{enumerate}\item[]
($a$) a person claims retirement pension;

($b$) the satisfaction of the condition referred to in paragraph~(1) would be based on earnings credited for incapacity for work or approved training in the tax years from 1993--94 to 2007--08; and

($c$) some or all of those credits were credited by virtue of official error derived from the failure to transpose correctly information relating to those credits from the Department for Work and Pensions’ Pension Strategy Computer System to Her Majesty’s Revenue and Customs’ computer system (\textsc{\lowercase{NIRS2}}) or from related clerical procedures,
\end{enumerate}
those earnings shall be credited.

(3) In this regulation, “bereavement benefit” means a bereavement allowance, a widowed mother’s allowance, a widowed parent’s allowance or a widow’s pension.

\subsection*{Credits for the purposes of entitlement to contribution\hspace{0pt}-based jobseeker’s allowance following official error}

8F.---(1)  This regulation~applies for the purpose only of enabling a person to satisfy the condition referred to in section 2(1)($b$)  of the Jobseekers Act 1995\footnote{1995 c.~18.}.

(2) Where—
\begin{enumerate}\item[]
($a$) a person claims a jobseeker’s allowance;

($b$) the satisfaction of the condition referred to in paragraph~(1) would be based on earnings credited for incapacity for work or approved training in the tax years from 1993--94 to 2007--08; and

($c$) some or all of those credits were credited by virtue of official error derived from the failure to transpose correctly information relating to those credits from the Department for Work and Pensions’ Pension Strategy Computer System to Her Majesty’s Revenue and Customs’ computer system (\textsc{\lowercase{NIRS2}}) or from related clerical procedures,
\end{enumerate}
that person shall be credited with those earnings.”.
\end{quotation}

\subsection[3. Amendment of the Social Security and Child Support (Decisions and Appeals) Regulations 1999]{Amendment of the Social Security and Child Support (Decisions and Appeals) Regulations 1999}

3.---(1)  Regulation 3 of the Social Security and Child Support (Decisions and Appeals) Regulations 1999\footnote{S.I.~1999/991; relevant amending instruments are S.I.~1999/1662, 2570 and 2677, 2000/897 and 1982, 2001/1711, 2002/428, 490, 1379 and 1703, 2003/1886, 2005/337 and 2677 and 2006/832.} (revision of decisions) is amended in accordance with the following paragraphs.

(2) At the beginning of paragraph~(5)($a$)  insert “except where paragraph~(5ZA) applies”.

(3) After paragraph~(5) insert—
\begin{quotation}
“(5ZA) This paragraph~applies where—
\begin{enumerate}\item[]
($a$) the decision which would otherwise fall to be revised is a decision to award a benefit specified in paragraph~(5ZB), whether or not the award has already been put in payment;

($b$) that award was based on the satisfaction by a person of the contribution conditions, in whole or in part, by virtue of credits of earnings for incapacity for work or approved training in the tax years from 1993--94 to 2007--08;

($c$) the official error derives from the failure to transpose correctly information relating to those credits from the Department for Work and Pensions’ Pension Strategy Computer System to Her Majesty’s Revenue and Customs’ computer system (\textsc{\lowercase{NIRS2}}) or from related clerical procedures; and

($d$) that error has resulted in an award to the claimant which is more advantageous to him than if the error had not been made.
\end{enumerate}

(5ZB) The specified benefits are—
\begin{enumerate}\item[]
($a$) bereavement allowance;

($b$) contribution-based jobseeker’s allowance;

($c$) incapacity benefit;

($d$) retirement pension;

($e$) widowed mother’s allowance;

($f$) widowed parent’s allowance; and

($g$) widow’s pension.
\end{enumerate}

(5ZC) In paragraph~(5ZA)($b$), “tax year” has the meaning ascribed to it by section 122(1) of the Contributions and Benefits Act.”.
\end{quotation}

\subsection[4. Amendment of the Social Security (Crediting and Treatment of Contributions, and National Insurance Numbers) Regulations 2001]{Amendment of the Social Security (Crediting and Treatment of Contributions, and National Insurance Numbers) Regulations 2001}

4.---(1)  The Social Security (Crediting and Treatment of Contributions, and National Insurance Numbers) Regulations 2001\footnote{S.I.~2001/769; the relevant amending instrument is S.I.~2004/1361.} is amended in accordance with the following paragraphs.

(2) After regulation~1(2) (interpretation), insert—
\begin{quotation}
“(3) In these Regulations, “official error” means an error made by—
\begin{enumerate}\item[]
($a$) an officer of the Department for Work and Pensions or an officer of Revenue and Customs acting as such which no person outside the Department or Her Majesty’s Revenue and Customs caused or to which no person outside the Department or Her Majesty’s Revenue and Customs materially contributed; or

($b$) a person employed by a service provider and to which no person who was not so employed materially contributed,
\end{enumerate}
but excludes any error of law which is shown to have been an error by virtue of a subsequent decision of a Commissioner or the court.

(4) In paragraph~(3)—
\begin{enumerate}\item[]
“Commissioner” means the Chief Social Security Commissioner or any other Social Security Commissioner and includes a tribunal of three or more Commissioners constituted under section 16(7) of the Social Security Act 1998;

“service provider” means a person providing services to the Secretary of State for Work and Pensions or to Her Majesty’s Revenue and Customs.”.
\end{enumerate}
\end{quotation}

(3) In regulation~4 (treatment for the purpose of any contributory benefit of late paid contributions)—
\begin{enumerate}\item[]
($a$) in paragraph~(1), for “6A” substitute “6B”;

($b$) after paragraph~(1) insert—
\begin{quotation}
“(1A) Any relevant contribution which is paid—
\begin{enumerate}\item[]
($a$) by virtue of an official error; and

($b$) more than six years after the end of the year in which the contributor was first advised of that error,
\end{enumerate}
shall be treated as not paid.”.
\end{quotation}
\end{enumerate}

(4) After regulation~6A insert—
\begin{quotation}
\subsection*{“Treatment for the purpose of any contributory benefit of certain Class 2 or Class 3 contributions}

6B.  For the purpose of entitlement to any contributory benefit, a Class 2 or a Class 3 contribution paid after the due date—
\begin{enumerate}\item[]
($a$) which would otherwise under regulation~4 (apart from paragraph~(1A) of that regulation)—
\begin{enumerate}\item[]
(i) have been treated as paid on a day other than the day on which it was actually paid; or

(ii) have been treated as not paid; and
\end{enumerate}

($b$) which was paid after the due date by virtue of an official error,
\end{enumerate}
shall be treated as paid on the day on which it is paid.”.
\end{quotation}

\pagebreak[3]

\bigskip

Signed 
by authority of the 
Secretary of State for~Work and~Pensions.
%I concur
%By authority of the Lord Chancellor

{\raggedleft
\emph{Mike O'Brien}\\*
%Secretary
Minister
%Parliamentary Under-Secretary 
of State,\\*Department 
for~Work and~Pensions

}

1st September 2007

\pagebreak[3]

\bigskip

The Commissioners for Her Majesty’s Revenue and Customs hereby concur in relation to regulation~4. 
%I concur
%By authority of the Lord Chancellor

{\raggedleft
\emph{Mike Eland}\\*
\emph{Paul Gray}\\*
%Secretary
Two of the Commissioners for Her Majesty’s Revenue and Customs.

}

4th September 2007

\small

\part{Explanatory Note}

\renewcommand\parthead{— Explanatory Note}

\subsection*{(This note is not part of the Regulations)}

These Regulations amend three sets of regulations—
\begin{enumerate}\item[]
($a$) the Social Security (Credits) Regulations 1975 (S.I.~1975/556) so that certain claimants—
\begin{enumerate}\item[]
(i) previously entitled to incapacity benefit as a result of satisfying the contribution condition referred to in paragraph~2(3)($a$)  of Schedule 3 to the Social Security Contributions and Benefits Act 1992 (c.~4); or

(ii) who wish to satisfy the condition referred to in paragraph~5(3)($a$)  of Schedule 3 to that Act for a retirement pension or in section 2 of the Jobseekers Act 1995 (c.~18) for contribution-based jobseeker’s allowance,
\end{enumerate}
and whose National Insurance contributions record had previously contained earnings credited by virtue of official error deriving from a failure to transpose correctly certain information from the Department for Work and Pensions’ computer system to Her Majesty’s Revenue and Customs’ computer system (NIRS2) will be credited with earnings to enable them to be entitled to incapacity benefit, a retirement pension or a contribution-based jobseeker’s allowance (regulation~2);

($b$) the Social Security and Child Support (Decisions and Appeals) Regulations 1999 (S.I.~1999/991) so that decisions to award retirement pensions, contribution-based jobseeker’s allowance, incapacity benefit and bereavement benefits, need not be revised for the official error referred to above (regulation~3);

($c$) the Social Security (Crediting and Treatment of Contributions, and National Insurance Numbers) Regulations 2001 (S.I.~2001/769) so that certain Class 2 and Class 3 contributions which may have been paid after the due date as a result of official error but less than six years after the end of the year in which they were advised of the error, are to be treated as paid on the day on which they are paid so as to give rise to entitlement to contributory benefits (regulation~4).
\end{enumerate}

An impact assessment has not been produced for this instrument as no impact on the private or voluntary sectors is foreseen. 

\end{document}