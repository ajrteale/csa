\documentclass[12pt,a4paper]{article}

\newcommand\regstitle{The Social Security (Work-focused Interviews) Regulations 2000}

\newcommand\regsnumber{2000/897}

%\opt{newrules}{
\title{\regstitle}
%}

%\opt{2012rules}{
%\title{Child Maintenance and Other Payments Act 2008\\(2012 scheme version)}
%}

\author{S.I. 2000 No. 897}

\date{Made
28th March 2000\\
%Laid before Parliament
%27th January 2000\\
Coming into force
3rd April 2000}

%\opt{oldrules}{\newcommand\versionyear{1993}}
%\opt{newrules}{\newcommand\versionyear{2003}}
%\opt{2012rules}{\newcommand\versionyear{2012}}

\usepackage{csa-regs}

\setlength\headheight{27.57402pt}

\begin{document}

\maketitle

\noindent
Whereas a draft of this Instrument was laid before Parliament in accordance with section 190(1)($aa$)\footnote{\frenchspacing 1992 c. 5; paragraph ($aa$) was inserted in sub-section (1) by the Welfare Reform and Pensions Act 1999 (c. 30), Schedule 12, paragraph 83.} of the Social Security Administration Act 1992 and approved by resolution of each House of Parliament:

Now therefore the Secretary of State for Social Security, in exercise of the powers conferred upon him by sections 2A, 2B(6) and (7), 5(1)($a$)  and ($b$), 6(1)($a$)  and ($b$), 7A, 189(1), (4) to (7A) and 191 of the Social Security Administration Act 1992\footnote{\frenchspacing Sections 2A, 2B and 7A were all inserted by the Welfare Reform and Pensions Act 1999, sections 57 and 71 respectively; section 191 is an interpretation provision and is cited because of the meaning ascribed to the word “prescribe”; section 189(7A) was inserted by the Welfare Reform and Pensions Act 1999. Schedule 12, paragraph 82.} and of all other powers enabling him in that behalf, after consultation with the Council on Tribunals in accordance with section 8(1) of the Tribunals and Inquiries Act 1992\footnote{\frenchspacing 1992 c. 53.} and in respect of provisions in these Regulations relating to housing benefit and council tax benefit with organisations appearing to him to be representative of the authorities concerned\footnote{\frenchspacing \emph{See} section 176(1)($a$) of the Social Security Administration Act 1992.}, by this Instrument, which contains only regulations made by virtue of or consequential upon sections 57 and 71 of the Welfare Reform and Pensions Act 1999 and which is made before the end of a period of 6 months beginning with the coming into force of those provisions\footnote{\frenchspacing \emph{See} section 173(5)($b$) of the Social Security Administration Act 1992.}, hereby make the following Regulations: 

{\sloppy

\tableofcontents

}

\bigskip

\setcounter{secnumdepth}{-2}

\section[Part I — General]{Part I\\*General}

\renewcommand\parthead{— Part I}

\subsection[1. Citation and commencement]{Citation and commencement}

1.  These Regulations may be cited as the Social Security (Work-focused Interviews) Regulations 2000 and shall come into force on 3rd April 2000.

\subsection[2. Interpretation]{Interpretation}

2.—(1)  In these Regulations—
\begin{enumerate}\item[]
“the Act” means the Welfare Reform and Pensions Act 1999;

“the 1998 Act” means the Social Security Act 1998\footnote{\frenchspacing 1998 c. 14.};

“benefit week”—
\begin{enumerate}\item[]
($a$) 
in relation to housing benefit and council tax benefit, means a period of 7 days beginning on a Monday;

($b$) 
in relation to any other specified benefit, means any period of 7 days corresponding to the week in respect of which the relevant social security benefit is due to be paid;
\end{enumerate}

“the designated authority” means any of the following—
\begin{enumerate}\item[]
($a$) 
the Secretary of State;

($b$) 
a person providing services to the Secretary of State;

($c$) 
a local authority;

($d$) 
a person providing services to, or authorised to exercise any functions of, any such authority;
\end{enumerate}

“the Careers Service” means a person of any description with whom the Secretary of State has made an arrangement under section 10(1) of the Employment and Training Act 1973\footnote{\frenchspacing 1973 c. 50; section 10 was inserted by section 45 of the Trade Union Reform and Employment Rights Act 1993 (c. 19).} and any person to whom he has given a direction under section 10(2) of that Act;

“the Council Tax Benefit Regulations” means the Council Tax Benefit (General) Regulations 1992\footnote{\frenchspacing S.I. 1992/1814.};

“the Claims and Payments Regulations” means the Social Security (Claims and Payments) Regulations 1987\footnote{\frenchspacing S.I. 1987/1968.};

“the Housing Benefit Regulations” means the Housing Benefit (General) Regulations 1987\footnote{\frenchspacing S.I. 1987/1971.};

“interview” means a work-focused interview;

“specified benefit” means a benefit other than widow’s payment to which section 2A of the Administration Act applies by virtue of subsection (2) of that section;

“work-focused interview” has the meaning given in regulation 3.
\end{enumerate}

(2) In these Regulations, a “relevant person” is a person who resides in an area identified in Schedule 1.

(3) For the purposes of these Regulations—
\begin{enumerate}\item[]
($a$) “remunerative work” has the meaning prescribed in regulation 4 of the Housing Benefit Regulations; and

($b$) “part-time work” means work for which payment is made and which is not remunerative work.
\end{enumerate}

(4) Where a claim for benefit is made by a person (“the appointee”) on behalf of another, references in these Regulations to a person claiming benefit shall be treated as a reference to the person on whose behalf the claim is made and not to the appointee.

(5) These Regulations apply in respect of a specified benefit claimed on or after 3rd April 2000.

(6) In these Regulations, unless the context otherwise requires, a reference—
\begin{enumerate}\item[]
($a$) to a numbered section is to the section of the Act bearing that number;

($b$) to a numbered regulation or Schedule is to the regulation in or Schedule to these Regulations bearing that number;

($c$) in a regulation or Schedule to a numbered paragraph is to the paragraph in that regulation or Schedule bearing that number;

($d$) in a paragraph to a lettered or numbered sub-paragraph is to the sub-paragraph in that paragraph bearing that letter or number.
\end{enumerate}

\subsection[3. Work-focused interview]{Work-focused interview}

3.  In these Regulations, a “work-focused interview” means an interview with a relevant person conducted for any or all of the following purposes—
\begin{enumerate}\item[]
($a$) assessing a person’s prospects for existing or future employment (whether paid or voluntary);

($b$) assisting or encouraging a person to enhance his prospects of such employment;

\begin{sloppypar}
($c$) identifying activities which the person may undertake to strengthen his existing or future prospects of such employment;
\end{sloppypar}

($d$) identifying current or future employment or training opportunities suitable to the person’s needs; and

($e$) identifying educational opportunities connected with the existing or future employment prospects or needs of the person.
\end{enumerate}

\section[Part II — Work-focused interviews]{Part II\\*Work-focused interviews}

\renewcommand\parthead{— Part II}

\subsection[4. Persons required to take part in an interview]{Persons required to take part in an interview}

4.—(1)  This regulation is subject to the provisions of regulations 5, 7, 8 and 9.

(2) A relevant person who—
\begin{enumerate}\item[]
($a$)  makes a claim for a specified benefit to a designated authority;

($b$) has not attained the age of 60 at the time of making the claim; and

($c$) is not in remunerative work,
\end{enumerate}
is required to take part in an interview.

(3) A designated authority to whom a claim is made shall arrange for the person to whom the claim relates and who is required in accordance with these Regulations to take part in an interview to have a personal adviser.

(4) A personal adviser shall except where paragraph (6) applies conduct the interview.

(5) The interview shall take place at an office of the designated authority or at such other place as may be notified to that person by a personal adviser.

(6) Where the claimant has not attained the age of 18, the designated authority shall arrange for the claimant to have an interview with an officer of the Careers Service.

\subsection[5. Exemptions]{Exemptions}

5.—(1)  The following claims for a specified benefit do not give rise to an interview under regulation 4—
\begin{enumerate}\item[]
($a$) claims by persons who at the time the claim is made are engaged in remunerative work;

($b$) 
except in a case to which paragraph (1A) applies,  % Words inserted (19.3.01) by SI 2000/1982 reg 3(a)
claims for a specified benefit where the person making the claim is also claiming a jobseeker’s allowance;

($c$) 
except in a case to which paragraph (1A) applies,  % Words inserted (19.3.01) by SI 2000/1982 reg 3(a)
claims for a specified benefit where, at the time the claim is made, the person making the claim is entitled to a jobseeker’s allowance;

($d$) claims by persons who are not present in, and who do not normally reside in, Great Britain.
\end{enumerate}

% Reg 5(1A) inserted (19.3.01) by SI 2000/1982 reg 3(b)
(1A) Notwithstanding paragraph (1)($b$)  and ($c$), a claim for a specified benefit shall give rise to an interview under regulation 4 where—
\begin{enumerate}\item[]
($a$) at the time the claim is made, the person making the claim is a member of a joint-claim couple as defined for the purposes of the Jobseeker’s Allowance Regulations 1996\footnote{S.I.\ 1996/207; Schedule A1 was inserted by S.I.\ 2000/1978.}; and

($b$) it has been decided that that person is a person to whom a paragraph of Schedule A1 to those Regulations applies (categories of members of joint-claim couples who are not required to satisfy the conditions in section 1(2B)($b$)  of the Jobseekers Act 1995).
\end{enumerate}

(2) The following claims for housing benefit and council tax benefit do not give rise to an interview under regulation 4—
\begin{enumerate}\item[]
($a$) claims made on the expiry of a benefit period;

($b$) claims made in consequence of the claimant moving from one dwelling to another within the same local authority area.
\end{enumerate}

\amendment{
Words inserted in reg. 5(1)(b), (c) and reg. 5(1A) inserted (19.3.01) by the Social Security (Joint Claims: Consequential Amendments) Regulations 2000 reg. 3.
}

\subsection[6. Continuing entitlement dependent upon an interview]{Continuing entitlement dependent upon an interview}

6.—(1)  A relevant person who has not attained the age of 60 and who is entitled to a specified benefit shall be required to take part in an interview as a condition of his continuing to be entitled to the full amount of benefit which is payable apart from these Regulations where paragraph (2) applies and—
\begin{enumerate}\item[]
($a$) in the case of a lone parent who is not entitled to either incapacity benefit or severe disablement allowance, any of the circumstances specified in paragraph (3) apply; or

($b$) in any other case, any of the circumstances specified in paragraph (4) apply.
\end{enumerate}

(2) This paragraph applies in the case of a person who has taken part in a work-focused interview, or who would have taken part in such an interview but for the requirement being waived in accordance with regulation 7 or deferred in accordance with regulation 8.

(3) The circumstances specified in this paragraph are that the lone parent—
\begin{enumerate}\item[]
($a$)  has been entitled to a specified benefit for more than a year except where one of the benefits to which the person was entitled during the previous 12 months was incapacity benefit or severe disablement allowance; and

($b$) has not taken part in an interview for at least a year.
\end{enumerate}

(4) The circumstances specified in this paragraph are those where—
\begin{enumerate}\item[]
($a$) a person is entitled to incapacity benefit or severe disablement allowance following a personal capability assessment;

($b$) a person’s entitlement to an invalid care allowance ceases whilst entitlement to another specified benefit continues;

($c$) a person becomes engaged or ceases to be engaged in part-time work;

($d$) a person has been undergoing education or training arranged by a personal adviser and that education or training comes to an end; and

($e$) a person who has not attained the age of 18 and who has previously undertaken an interview attains the age of 18.
\end{enumerate}

(5) In this regulation—
\begin{enumerate}\item[]
“lone parent” means a person who has no partner and who is responsible for, and a member of the same household as, a child;

“personal capability assessment” means an assessment in accordance with regulations made under section 171C of the Contributions and Benefits Act\footnote{\frenchspacing Section 171C was inserted by section 61 of the Welfare Reform and Pensions Act 1999 (c. 30).}.
\end{enumerate}

\subsection[7. Waiver]{Waiver}

7.---(1)  A requirement to take part in an interview imposed by these Regulations shall not apply where the designated authority determines in the case of any particular person making a claim that the interview—
\begin{enumerate}\item[]
($a$) would not be of assistance to that person, or

($b$) would not be appropriate in the circumstances of that case.
\end{enumerate}

(2) A person in relation to whom the requirement to take part in an interview has been waived shall be treated for the purposes of any claim for or entitlement to a specified benefit as having complied with that requirement.

\subsection[8. Deferment of interview]{Deferment of interview}

8.---(1)  Except in a case to which paragraph (2) refers, a personal adviser shall arrange for an interview to take place as soon as reasonably practicable after the claim is made or the event which under regulation 6(3) or (4) gives rise to the interview occurs.

(2) This paragraph applies where the designated authority determines in the case of any particular person that the requirement to take part in an interview shall not apply at the time the claim is made or the event occurs because an interview would not at that time—
\begin{enumerate}\item[]
($a$) be of assistance to that person, or

($b$) be appropriate in the circumstances of that case.
\end{enumerate}

(3) A designated authority who determines in accordance with the preceding provisions of this regulation that the requirement to take part in an interview shall not apply shall also determine either when that determination is made or later, the time when the requirement to take part in an interview is to apply in the claimant’s case.

(4) Where an interview has been deferred in accordance with this regulation, then until both—
\begin{enumerate}\item[]
($a$) a determination has been made that the claimant is to take part in an interview, and

($b$) a determination has been made as to whether the claimant in fact took part in an interview,
\end{enumerate}
he shall be treated for the purposes of any claim for or entitlement to a specified benefit as having complied with any requirement to take part in an interview.

\subsection[9. Claims for two or more specified benefits]{Claims for two or more specified benefits}

9.  A person who would otherwise be required under these Regulations to take part in interviews relating to more than one specified benefit—
\begin{enumerate}\item[]
($a$) is only required to take part in one interview; and

($b$) that interview counts for the purposes of all those benefits.
\end{enumerate}

\subsection[10. The interview]{The interview}

10.---(1)  The relevant person’s personal adviser shall inform the claimant of the place and time of the interview.

(2) The personal adviser may determine that the interview is to take place in the home of the claimant or elsewhere where it would in the opinion of the personal adviser be unreasonable to expect the claimant to attend the office of a designated authority because his personal circumstances are such that attending the office would cause him undue inconvenience or endanger his health.

\subsection[11. Taking part in an interview]{Taking part in an interview}

11.---(1)  The designated authority shall determine whether a person has taken part in an interview.

(2) A person who has attained the age of 18 shall be regarded as having taken part in an interview if, and only if—
\begin{enumerate}\item[]
($a$) he attends at the place and time notified to him by the personal adviser for the interview; and

($b$) he provides answers (where asked) and appropriate information to questions about—
\begin{enumerate}\item[]
(i) the level to which he has pursued any educational qualifications;

(ii) his employment history;

(iii) any vocational training he has undertaken;

(iv) any skills he has acquired which fit him for employment;

(v) any paid or unpaid employment he is engaged in;

(vi) any medical condition which in the opinion of that person puts him at a disadvantage in obtaining employment; and

(vii) any caring or childcare responsibilities he has.
\end{enumerate}
\end{enumerate}

(3) A person who has not attained the age of 18 shall be regarded as having taken part in an interview if, and only if, he attends an interview with the Careers Service at the time and place notified to him by the personal adviser.

\subsection[12. Failure to take part in an interview]{Failure to take part in an interview}

12.---(1)  A person who—
\begin{enumerate}\item[]
($a$) has been notified of any interview in accordance with these Regulations;

($b$) fails to take part in that interview; and

($c$) fails to show before the end of 5 working days following the day on which the interview was to take place good cause for his failure to take part in the interview,
\end{enumerate}
shall, subject to paragraph (12), suffer the consequences set out below.

(2) Those consequences are—
\begin{enumerate}\item[]
($a$) where the interview arose in connection with a claim for a specified benefit, that the person to whom the claim relates is to be regarded as not having made a claim for a specified benefit;

($b$) where an interview which arose in connection with a claim for a specified benefit was deferred and benefit became payable in accordance with regulation 8(4), the person’s entitlement to that benefit shall terminate as from the first day of the next benefit week following the date the decision was made;

($c$) where the claimant has an award of benefit and the requirement for the interview arose under regulation 6, the claimant’s benefit shall be reduced as from the first day of the next benefit week following the day the decision was made, by a sum equal (but subject to paragraphs (3) and (4)) to 20 per cent.\ of the amount applicable on the date the deduction commences in respect of a single claimant for income support aged not less than 25.
\end{enumerate}

(3) Benefit reduced in accordance with paragraph (2)($c$)  shall not be reduced below—
\begin{enumerate}\item[]
($a$) 50 pence per week in the case of housing benefit; or

($b$) 10 pence per week in the case of any other specified benefit.
\end{enumerate}

(4) Where two or more specified benefits are in payment to a claimant, a deduction made in accordance with this regulation shall be applied, except in a case to which paragraph (5) applies, to the specified benefits in the following order of priority—
\begin{enumerate}\item[]
($a$) income support;

($b$) incapacity benefit;

($c$) widow’s benefits;

($d$) invalid care allowance;

($e$) severe disablement allowance;

($f$) council tax benefit;

($g$) housing benefit.
\end{enumerate}

(5) Where the amount of the reduction is greater than some (but not all) of the specified benefits listed in paragraph (4), the reduction shall be made against the first benefit in that list which is the same as or greater than the amount of the reduction.

(6) For the purpose of determining whether a specified benefit is the same as or greater than the amount of the reduction for the purposes of paragraph (5), the amount set out in paragraph 3($a$)  or as the case may be ($b$)  shall be added to the amount of the reduction.

(7) In a case where the whole of the reduction cannot be applied against any one specified benefit because no one benefit is the same as or greater than the amount of the reduction, the reduction shall be applied against the first benefit in payment in the list of priorities at paragraph (4) and so on against each benefit in turn until the whole of the reduction is exhausted or, if this is not possible, the whole of the specified benefits are exhausted, subject in each case to the minimum sums specified in paragraph (3) remaining in payment.

(8) Where the rate of any specified benefit payable to a claimant changes, the rules set out above for a reduction in the benefit payable shall be applied to the new rates and any adjustments to the benefits against which the reductions are made shall take effect from the beginning of the first benefit week to commence for that claimant following the change.

(9) Where a claimant whose benefit has been reduced in accordance with this regulation subsequently takes part in an interview, the reduction shall cease to have effect on the first day of the benefit week in which the requirement to take part in an interview was met.

(10) For the avoidance of doubt, a person who is regarded as not having made a claim for any benefit because he failed to take part in a work-focused interview shall be required to make a new claim in order to establish entitlement to any benefit.

(11) For the purposes of determining the amount of any benefit payable, a claimant shall be treated as receiving the amount of any specified benefit which would have been payable but for a reduction made in accordance with this regulation.

(12) The consequences set out in this regulation shall not apply in the case of a person who brings new facts to the notice of the personal adviser within 1 month of the date on which the decision was notified and—
\begin{enumerate}\item[]
($a$) those facts could not reasonably have been brought to the personal adviser’s notice within 5 working days of the day on which the interview was to take place; and

($b$) those facts show that he had good cause for his failure to take part in the interview.
\end{enumerate}

(13) In paragraphs (2) and (12), the “decision” means the decision that the person failed without good cause to take part in an interview.

\subsection[13. Circumstances where regulation 12 does not apply]{Circumstances where regulation 12 does not apply}

13.  The consequences of a failure to take part in an interview set out in regulation 12 shall not apply where the person—
\begin{enumerate}\item[]
($a$) ceases to reside in an area specified in Schedule 1; or

($b$) attains the age of 60.
\end{enumerate}

\subsection[14. Good cause]{Good cause}

14.  Matters to be taken into account in determining whether a person has shown good cause for his failure to take part in an interview include—
\begin{enumerate}\item[]
($a$) that the person misunderstood the requirement to take part in the interview due to any learning, language or literacy difficulties of the person or any misleading information given to the person by an officer of a designated authority;

($b$) that the person was attending a medical or dental appointment, or accompanying a person for whom the claimant has caring responsibilities to such an appointment, and that it would have been unreasonable, in the circumstances, to rearrange the appointment;

($c$)  that the person had difficulties with his normal mode of transport and that no reasonable alternative was available;

($d$) that the established customs and practices of the religion to which the person belongs prevented him attending on that day or at that time;

($e$) that the person was attending an interview with an employer with a view to obtaining employment;

($f$) that the person was actually pursuing employment opportunities as a self-employed earner;

($g$) that the person or a dependant of his or a person for whom he provides care suffered an accident, sudden illness or relapse of a chronic condition;

($h$) that he was attending the funeral of a close friend or relative on the day fixed for the interview;

($i$) that a disability from which the person suffers made it impracticable for him to attend at the time fixed for the interview.
\end{enumerate}

\subsection[15. Appeals]{Appeals}

15.---(1)  This regulation applies to any relevant decision of a designated authority or any decision under section 10 of the 1998 Act superseding such a decision.

(2) This regulation applies—
\begin{enumerate}\item[]
($a$) whether the decision is as originally made or as revised under section 9 of the 1998 Act; and

($b$) as if any decision made, superseded or revised otherwise than by the Secretary of State was a decision made, superseded or revised by him.
\end{enumerate}

(3) In the case of a decision to which this regulation applies, the person in respect of whom the decision was made shall have a right of appeal under section 12 of the 1998 Act to an appeal tribunal.

\subsection[16. Consequential changes]{Consequential changes}

16.---(1)  Schedule 2, which makes changes to the Housing Benefit Regulations which are consequential upon the making of decisions under these Regulations on the work-focused interview, shall have effect.

(2) Schedule 3, which makes corresponding changes to the Council Tax Benefit Regulations, shall have effect.

(3) Schedule 4, which makes changes relating to the sending and delivering of claims for housing benefit and council tax benefit, shall have effect.

(4) Schedule 5, which makes changes to the Claims and Payments Regulations, shall have effect.

(5) Schedule 6, which makes changes to the procedure relating to decisions and appeals, shall have effect.

\subsection[17. Amendments to Social Security Regulations]{Amendments to Social Security Regulations}

17.---(1)  In regulation 4A(1) of the Claims and Payments Regulations\footnote{\frenchspacing Regulation 4A was inserted by S.I. 1999/3108.}, after the words “to any office” there shall be inserted the words “of a relevant authority”.

(2) In regulation 6(1) of the Social Security (Claims and Information) Regulations 1999\footnote{\frenchspacing S.I. 1999/3108.} for the words “Part I or II of Schedule 1 to these Regulations” there shall be substituted the words “paragraph (3)”. 

\bigskip

Signed 
by authority of the Secretary of State for Social Security.

{\raggedleft
\emph{Angela Eagle
}\\*Parliamentary Under-Secretary of State,\\*Department of Social Security

}

28th March 2000

\small

\part[Schedule 1 --- Areas in which “relevant persons” reside]{Schedule 1\\*Areas in which ``relevant persons'' reside}

\renewcommand\parthead{--- Schedule 1}

For the purposes of regulation 2(2), the areas are—
\begin{enumerate}\item[]
($a$) the areas of the local authorities listed below (all of which are within the area of Buckinghamshire County Council)—
\begin{enumerate}\item[]
    Aylesbury Vale District Council;

    Chiltern District Council;

    Wycombe District Council;

    South Buckinghamshire District Council;

    Milton Keynes District Council; 
\end{enumerate}

($b$) the areas of the local authorities listed below (all of which are within the area of Somerset County Council)—
\begin{enumerate}\item[]
    Sedgemoor District Council;

    Taunton District Council;

    South Somerset District Council;

    West Somerset District Council;

    Mendip District Council; 
\end{enumerate}

($c$) the area of Warwickshire County Council, except for the areas of the Parish Councils listed below—
\begin{enumerate}\item[]
    Alcester;

    Arrow;

    Aston Cantlow;

    Bidford on Avon;

    Bagington;

    Bubbenhall;

    Coughton;

    Earlswood;

    Exhall;

    Great Alne;

    Haselor;

    Hockley Heath;

    Kinwarton;

    Morton Bagot;

    Oldberrow;

    Packwood;

    Portway;

    Salford Priors;

    Sambourne;

    Spernall;

    Stoneleigh;

    Studely;

    Weethley;

    Wixford; 
\end{enumerate}

($d$) the following postcode districts—

\newcommand\postcode[1]{\textsc{\lowercase{#1}}}

\begin{enumerate}\item[]
    \postcode{BD3}, \postcode{BD4 6} to \postcode{BD4 9, BD11, BD12 0, BD12 8, BD12 9, BD19}

    \postcode{CB8 0, CB8 7} to \postcode{CB8 9, CB9 9, CO10 0, CO10 1, CO10 5, CO10 7} to \postcode{CO10 9}

    \postcode{CF3} and \postcode{CF8}

    \postcode{CM0} to \postcode{CM6, CM8, CM9, CM11} and \postcode{CM16}

    \postcode{CO5 0RX}

    \postcode{DE55 1} to \postcode{DE55 5, DE55 7}

    \postcode{E4, E10, E11, E17} and \postcode{E18}

    \postcode{G78 1} to \postcode{G78 4}

    \postcode{GL15} and \postcode{GL16}

    \postcode{HD1} to \postcode{HD8}

    \postcode{HR2} and \postcode{HR9}

    \postcode{IG1} to \postcode{IG10}

    \postcode{IP1, IP2 0, IP2 9, IP3 0, IP3 8, IP2 4, IP4 5, IP5 1, IP5 3, IP6 0, IP6~8, IP6 9, IP7 5} to \postcode{IP7 7, IP8 3, IP8 4, IP9 1, IP9 2, IP10 0, IP11~0, IP11 7} to \postcode{IP11 9, IP12 1} to \postcode{IP12 3, IP13 0, IP13 6} to \postcode{IP13 9, IP14 1} to \postcode{IP14 6, IP15 5, IP16 4, IP17 1} to \postcode{IP17 3, IP18 6, IP19 0, IP19 8, IP19~9, IP27 0, IP27 9, IP28 6} to \postcode{IP28 8, IP29 4, IP29 5, IP30 0, IP30 9, IP31 1} to \postcode{IP31 3, IP32 6, IP32 7, IP33 2} and \postcode{IP33 3}

    \postcode{KA28} to \postcode{KA30}

    \postcode{LS1} to \postcode{LS19, LS21, LS22, LS26} to \postcode{LS28}

    \postcode{NG14 7, NG16 5, NG16 6, NG17} to \postcode{NG22, NG23 5, NG23 6} and \postcode{NG25}

    \postcode{NP1, NP4} to \postcode{NP11, NP15, NP16, NP18, NP20, NP25, NP26} and \postcode{NP44}

    \postcode{NR32 3} to \postcode{NR32 5, NR33 7} to \postcode{NR33 9, NR34 0} and \postcode{NR34 7} to \postcode{NR34 9}

    \postcode{OL14}

    \postcode{PA1--PA5} and \postcode{PA7} to \postcode{PA27}

    \postcode{RM6}, except for the following parts:
\begin{enumerate}\item[]
    \postcode{5AA, 5BH, 5HD, 5HB, 5HH, 5HP, 5EP, 5ER, 5EL, 5QT, 6DU, 6DX, 5TJ, 5SB, 6RH, 5RA, 5QX, 6RJ, 6RL} and \postcode{6RB}
\end{enumerate}

    \postcode{RM8}, except for the following parts:
\begin{enumerate}\item[]
    \postcode{3UH, 3UL, 3UB, 3UA, 3HX, 3HR, 3HA, 3HB, 3HD, 3JA, 3HU, 3JP, 3XX, 3YA, 3YB, 3YH, 3YJ, 3YL, 1UT, 1XA, 1DB, 1DD, 1DH, 1DJ, 1YR, 1YP, 1BX, 1BU, 1BT, 1BP, 3RP, 3RR, 3SR, 3UD, 3UX, 1XJ} and \postcode{1XL}
\end{enumerate}

    \postcode{SS0} to \postcode{SS6} and \postcode{SS9}

    \postcode{SS11}, but only the following parts:
\begin{enumerate}\item[]
    \postcode{7EE, 7PR, 7BS, 7NW, 7NP, 7NS, 7NJ, 7NR, 7NX, 7PD, 7PB, 7PE, 7PA, 7PT, 7BL, 7JG, 7HU, 7JE, 7PX, 7HS, 7QH, 7BJ, 7NB, 7ND, 7EY, 7HY, 7HZ, 7JD, 7JF, 7DP, 7DN, 7JQ, 7BQ, 7JG, 7BG, 7NA, 7LY, 7LX, 7BH, 7BW, 7EX, 7ET, 7LZ, 7EP, 7BE, 7LR, 7LP, 7HX, 7PP , 7PY, 7DX, 7DY, 7HB, 7HA, 7BN, 7ES, 7PU, 7QD, 7QA, 7QB, 7PZ, 7DW, 7HP, 7PS, 7QF, 7PN, 7HT, 7QG, 7EU, 7DR, 7DT, 7DA, 7DB, 7NU, 7JB, 7JA, 7LN, 7LW, 7LS} and \postcode{7BP}
\end{enumerate}

    \postcode{SS12}

    \postcode{WA1} to \postcode{WA5, WA7, WA8} and \postcode{WA11} to \postcode{WA13}

    \postcode{WF3} and \postcode{WF12} to \postcode{WF17}. 
\end{enumerate}
\end{enumerate}

\part[Schedule 2 --- Housing benefit amendments]{Schedule 2\\*Housing benefit amendments}

\renewcommand\parthead{--- Schedule 2}

1.  The Housing Benefit Regulations shall be amended in accordance with the following paragraphs of this Schedule.

\medskip

2.  In regulation 2(1) (interpretation), after the entry relating to “water charges”, there shall be inserted the following entries—
\begin{quotation}
    ““work-focused interview” has the meaning it has in regulation 3 of the Work-focused Interviews Regulations;

    “the Work-focused Interviews Regulations” means the Social Security (Work-focused Interviews) Regulations 2000\footnote{\frenchspacing S.I. 2000/897.};”. 
\end{quotation}

\medskip

3.---(1)  In regulation 68 (date on which change of circumstances is to take effect), in paragraph (1), the word “either” shall be omitted and for the word “applies” there shall be substituted the words “or regulation 68A applies”.

(2) After regulation 68 there shall be inserted the following regulation—
\begin{quotation}
\subsection*{“Date of change of circumstances following decision as to whether person took part in a work-focused interview}

68A.---(1)  Where the relevant change of circumstances is a decision made in accordance with regulation 11 of the Work-focused Interviews Regulations as to whether a person took part in a work-focused interview, the date on which the change of circumstances is to take effect shall be determined in accordance with the following paragraphs of this regulation.

(2) Where the relevant change of circumstances is that the consequences specified in regulation 12(2)($b$)  or ($c$)  of the Work-focused Interviews Regulations apply, the change shall take effect as from the first day of the next benefit week following the date of the decision that the claimant failed without good cause to take part in a work-focused interview.

(3) Where the relevant change of circumstances is that the claimant attains the age of 60 or ceases to reside in an area specified in Schedule 1 to the Work-focused Interviews Regulations, the date on which the change of circumstances is to take effect is the first day of the next benefit week to commence for that person following the date the decision was made or the circumstance occurred.

(4) Where the relevant change of circumstances is a decision that the consequences specified in paragraph (2) which applied to the claimant no longer apply, the date on which the change of circumstances is to take effect is the day on which it would have had effect had the revised decision been made on the date of the decision it revised.

(5) Where a person—
\begin{enumerate}\item[]
($a$) has been held not to have taken part in a work-focused interview;

($b$) in consequence of that decision suffers a reduction in benefit; and

($c$) subsequently takes part in a work-focused interview,
\end{enumerate}
the date on which the change of circumstances is to have effect is the first day of the benefit week in which the requirement to take part in the interview was met.”.
\end{quotation}

\medskip

4.  In regulation 79 (review of determinations)—
\begin{enumerate}\item[]
($a$) in paragraph (1), for the words “Subject to paragraph 1A\footnote{\frenchspacing The relevant amending Instrument is S.I. 1993/1150.}”, there shall be substituted the words “Subject to paragraphs (1A) and (1B)”; and

($b$) after paragraph (1A), there shall be inserted the following paragraph—
\begin{quotation}
“(1B) A determination or decision that a person did or did not take part in a work-focused interview and if he did not whether he had good cause for not doing so, shall not be reviewed.”.
\end{quotation}
\end{enumerate}

\medskip

5.  In regulation 81 (further review of determinations)—
\begin{enumerate}\item[]
($a$) in paragraph (3), at the beginning, there shall be inserted the words “Subject to paragraph (5)”;

($b$) after paragraph (4), there shall be added the following paragraph—
\begin{quotation}
“(5) A determination or a decision that a person did or did not take part in a work-focused interview and if he did not whether he had good cause for not doing so, shall not be reviewed by a Review Board.”.
\end{quotation}
\end{enumerate}

\medskip

6.  In Schedule 6 (matters to be included in the Notice of Determination), the following Part shall be added at the end—
\begin{quotation}
\section*{\sloppy “Part VIII\\*Notice following a decision on a work-focused interview}

15.---(1)  This Part applies in a case where a decision has been made in accordance with regulation 11 of the Work-focused Interviews Regulations that a person has failed to take part in a work-focused interview.

(2) In a case where one of the consequences specified in sub-paragraphs (3) and (4) apply, the notice of determination shall include a statement as to the person’s right of appeal against the decision that he failed to take part in a work-focused interview.

(3) In a case where the consequence of the failure to take part is that the entitlement to housing benefit terminates, the notice of determination shall include a statement as to—
\begin{enumerate}\item[]
($a$) the last date of the entitlement to housing benefit;

($b$) the reason entitlement terminated.
\end{enumerate}

(4) In a case where the consequence of the failure to take part is that the amount of housing benefit payable is reduced, the notice of determination shall include a statement as to—
\begin{enumerate}\item[]
($a$) the amount by which the housing benefit is reduced;

($b$) the date from which the reduction takes effect; and

($c$) the reason for the reduction.
\end{enumerate}

(5) In a case where a new decision is made reversing an earlier decision that a person failed to take part in a work-focused interview, the notice of determination shall include a statement as to—
\begin{enumerate}\item[]
($a$) the date from which the consequences of the failure cease to apply; and

($b$) the reason for the new decision.”.
\end{enumerate}
\end{quotation}

\part[Schedule 3 --- Council tax benefit amendments]{Schedule 3\\*Council tax benefit amendments}

\renewcommand\parthead{--- Schedule 3}

1.  The Council Tax Benefit Regulations shall be amended in accordance with the following paragraphs of this Schedule.

\medskip

2.  In regulation 2(1) (interpretation), after the entry relating to “water charges”, there shall be inserted the following entries—
\begin{quotation}
    ““work-focused interview” has the meaning it has in regulation 3 of the Work-focused Interviews Regulations;

    “the Work-focused Interviews Regulations” means the Social Security (Work-focused Interviews) Regulations 2000\footnote{\frenchspacing S.I. 2000/897.};”. 
\end{quotation}

\medskip

3.---(1)  In regulation 59 (date on which change of circumstances is to take effect), in paragraph (1), for the word “applies” there shall be substituted the words “and regulation 59A applies”.

(2) After regulation 59 there shall be inserted the following regulation—
\begin{quotation}
\subsection*{“Date of change of circumstances following a decision as to whether a person took part in a work-focused interview}

59A.---(1)  Where the relevant change of circumstances is a decision made in accordance with regulation 11 of the Work-focused Interviews Regulations as to whether a person took part in a work-focused interview, the date on which the change of circumstances is to take effect shall be determined in accordance with the following paragraphs of this regulation.

(2) Where the relevant change of circumstances is that the consequences specified in regulation 12(2)($b$)  or ($c$)  of the Work-focused Interviews Regulations apply, the change shall take effect as from the first day of the next benefit week following the date of the decision that the claimant failed without good cause to take part in a work-focused interview.

(3) Where the relevant change of circumstances is that the claimant attains the age of 60 or ceases to reside in an area specified in Schedule 1 to the Work-focused Interviews Regulations, the date on which the change of circumstances is to take effect is the first day of the next benefit week to commence for that person following the date the decision was made or the circumstance occurs.

(4) Where the relevant change of circumstances is a decision that the consequences specified in paragraph (2) which applied to the claimant no longer apply, the date on which the change of circumstances is to take effect is the day on which it would have had effect had the revised decision been made on the date of the decision it revised.

(5) Where a person—
\begin{enumerate}\item[]
($a$) has been held not to have taken part in a work-focused interview;

($b$) in consequence of that decision suffers a reduction in benefit;

($c$) subsequently takes part in a work-focused interview,
\end{enumerate}
the date on which the change of circumstances is to have effect is the first day of the benefit week in which the requirement to take part in the interview was met.”.
\end{quotation}

\medskip

4.  In regulation 69 (review of determinations)—
\begin{enumerate}\item[]
($a$) in paragraph (1), for the words “Subject to paragraph (1A)\footnote{\frenchspacing The relevant amending Instrument is S.I. 1995/511.}”, there shall be substituted the words “Subject to paragraphs (1A) and (1B),”; and

\goodbreak[3]

($b$) after paragraph (1A), there shall be inserted the following paragraph—
\begin{quotation}
“(1B) A determination or decision that a person did or did not take part in a work-focused interview and if he did not whether he had good cause for not doing so, shall not be reviewed.”.
\end{quotation}
\end{enumerate}

\medskip

5.  In regulation 70 (further review of determinations)—
\begin{enumerate}\item[]
($a$) in paragraph (3) for the words “Subject to paragraph (4)”, there shall be substituted the words “Subject to paragraphs (4) and (6)”;

($b$) after paragraph (5), there shall be added the following paragraph—
\begin{quotation}
“(6) A determination or a decision that a person did or did not take part in a work-focused interview and if he did not whether he had good cause for not doing so, shall not be reviewed by a Review Board.”.
\end{quotation}
\end{enumerate}

\medskip

6.  In Schedule 6 (matters to be included in the notice of determination), the following Part shall be added at the end—
\begin{quotation}
\section*{\sloppy “Part VIII\\*Notice following a decision on a work-focused interview}

15.---(1)  This Part applies in a case where a decision has been made in accordance with regulation 11 of the Work-focused Interviews Regulations that a person has failed to take part in a work-focused interview.

(2) In a case where one of the consequences specified in sub-paragraphs (3) and (4) apply, the notice of determination shall include a statement as to the person’s right of appeal against the decision that he failed to take part in a work-focused interview.

(3) In a case where the consequence of the failure to take part is that the entitlement to council tax benefit terminates, the notice of determination shall include a statement as to—
\begin{enumerate}\item[]
($a$) the last date of the entitlement to council tax benefit;

($b$) the reason entitlement terminated.
\end{enumerate}

(4) In a case where the consequence of the failure to take part is that the amount of council tax benefit is reduced, the notice of determination shall include a statement as to—
\begin{enumerate}\item[]
($a$) the amount by which the council tax benefit is reduced;

($b$) the date from which the reduction takes effect; and

($c$) the reason for the reduction.
\end{enumerate}

(5) In a case where a new decision is made reversing an earlier decision that a person failed to take part in a work-focused interview, the notice of determination shall include a statement as to—
\begin{enumerate}\item[]
($a$) the date from which the consequences of the failure cease to apply; and

($b$) the reason for the new decision.”.
\end{enumerate}
\end{quotation}

\part[Schedule 4 --- Amendments to the Housing Benefit Regulations and Council Tax Benefit Regulations relating to claims]{Schedule 4\\*Amendments to the Housing Benefit Regulations and Council Tax Benefit Regulations relating to claims}

\section[Part I --- Housing benefit amendments]{Part I\\*Housing benefit amendments}

\renewcommand\parthead{--- Schedule 4 Part I}

1.  The Housing Benefit Regulations shall be amended in accordance with the following provisions of this Part.

\medskip

2.  In regulation 2 (interpretation), in paragraph (1)—
\begin{enumerate}\item[]
($a$) after the entry relating to “date of claim” there shall be inserted the following entry—
\begin{quotation}
    “the designated authority” means any of the following—
\begin{enumerate}\item[]
    ($a$) 
    the Secretary of State;

    ($b$) 
    a person providing services to the Secretary of State;

    ($c$) 
    a local authority;

    ($d$) 
    a person providing services to, or authorised to exercise any functions of, any such authority;
\end{enumerate}
\end{quotation}

    ($b$) 
    the entry relating to “relevant authority” shall be omitted\footnote{\frenchspacing The relevant amending Instrument is S.I. 1999/3108.\label{fn:18}}. 
\end{enumerate}

\medskip

3.  In regulation 71 (who may claim), in paragraph (7), for the words “Part I or II of Schedule 2 to the Social Security (Claims and Information) Regulations 1999” there shall be substituted the words “Schedule 1 to the Work-focused Interviews Regulations” and for the words “relevant authority”, there shall be substituted the words “designated authority”.

\medskip

4.  In regulation 72 (time and manner in which claims are to be made)—
\begin{enumerate}\item[]
($a$) in paragraph (1), for the words “Subject to regulation 72B, every”, there shall be substituted the word “Every”;

($b$) in paragraph (4)—
\begin{enumerate}\item[]
(i) in sub-paragraph ($d$), for the words “and is not engaged in remunerative work”, there shall be substituted the words “and is neither engaged in remunerative work nor residing in an area identified in Schedule 1 to the Work-focused Interviews Regulations”;

(ii) at the end there shall be added the following sub-paragraph—
\begin{quotation}
“($e$) may be sent or delivered to the office of a designated authority where the claimant—
\begin{enumerate}\item[]
(i) has attained the age of 16 but not the age of 60; and

(ii) resides in an area identified in Schedule 1 to the Work-focused Interviews Regulations.”.
\end{enumerate}
\end{quotation}
\end{enumerate}
\end{enumerate}

\medskip

5.  In regulation 72A\footnote{\frenchspacing The relevant amending Instrument is S.I. 1999/1539.} (date of claim where claim sent or delivered to a gateway office) in paragraph (3) at the end there shall be added the words “or are made at an office of a designated authority in accordance with regulation 72(4)($c$).”.

\medskip

6.  For regulations 72B and 72C\footnote{\frenchspacing The relevant amending Instrument is S.I. 1999/3108.}, there shall be substituted the following regulation—
\begin{quotation}
\subsection*{“Date of claim where claim sent or delivered to an office of a designated authority}

72B.---(1)  Where a claim for housing benefit has been sent or delivered to an office of a designated authority in accordance with regulation 72(4)($e$), the date on which the claim is made shall be—
\begin{enumerate}\item[]
($a$) except where paragraph ($b$)  applies, the date the claim is received at an office of the designated authority; or

($b$) where in the 4 weeks before the claim is received in an office of a designated authority, the person making the claim or a person acting on his behalf had notified an office of a designated authority of his intention to make such a claim, the date the notification was given.
\end{enumerate}

(2) A notification of intention to make a claim is deemed to be given on the date on which notification of the intention to claim housing benefit is received, in whatever form, from the claimant, or the person acting on his behalf, at an office of a designated authority.

(3) Paragraph (2) applies where neither income support nor a jobseeker’s allowance is claimed in conjunction with housing benefit.

(4) Where the person claiming housing benefit in accordance with regulation 72(4)($e$), or the partner of that person,—
\begin{enumerate}\item[]
($a$) has an award of income support or income-based jobseeker’s allowance; or

($b$) has claimed such a benefit but no award has been made,
\end{enumerate}
the date on which the claim for housing benefit is made shall be determined as if sub-paragraphs ($a$), ($b$), ($c$)  and ($e$)  of paragraph (1) of regulation 72A applied to that claim as they apply to claims under regulation 72(4)($d$).”.
\end{quotation}

\medskip

7.  In regulation 73 (evidence and information) paragraph (7) shall be omitted.

\medskip

8.  In regulation 75 (duty to notify change of circumstances), in paragraph (4), for the words “relevant authority” there shall be substituted the words “designated authority”.

\section[Part II --- Council tax benefit amendments]{Part II\\*Council tax benefit amendments}

\renewcommand\parthead{--- Schedule 4 Part II}

1.  The Council Tax Benefit Regulations shall be amended in accordance with the following provisions of this Part.

\medskip

2.  In regulation 2 (interpretation), in paragraph (1)—
\begin{enumerate}\item[]
($a$) after the entry relating to “date of claim” there shall be inserted the following entry—
\begin{quotation}
    ““the designated authority” means any of the following—
\begin{enumerate}\item[]
    ($a$) 
    the Secretary of State;

    ($b$) 
    a person providing services to the Secretary of State;

    ($c$) 
    a local authority;

    ($d$) 
    a person providing services to, or authorised to exercise any functions of, any such authority;” 
\end{enumerate}
\end{quotation}

($b$) the entry relating to “relevant authority” shall be omitted\footnote{\frenchspacing The relevant amending Instrument is S.I. 1999/3108.}.
\end{enumerate}

\medskip

3.  In regulation 61 (who may claim), in paragraph (7), for the words “relevant authority”, there shall be substituted the words “designated authority”.

\medskip

4.  In regulation 62 (time and manner in which claims are made)—
\begin{enumerate}\item[]
($a$) in paragraph (1), for the words “Subject to regulation 62B, every” there shall be substituted the word “Every”;

($b$) in paragraph (4)—
\begin{enumerate}\item[]
(i) in sub-paragraph ($d$), for the words “and is not engaged in remunerative work”, there shall be substituted the words “and is neither engaged in remunerative work nor residing in an area identified in Schedule 1 to the Work-focused Interviews Regulations”;

(ii) at the end there shall be added the following sub-paragraph—
\begin{quotation}
“($e$) may be sent or delivered to the office of a designated authority where the claimant—
\begin{enumerate}\item[]
(i) has attained the age of 16 but not the age of 60; and

(ii) resides in an area identified in Schedule 1 to the Work-focused Interviews Regulations.”.
\end{enumerate}
\end{quotation}
\end{enumerate}
\end{enumerate}

\medskip

5.  In regulation 62A\footnote{\frenchspacing The relevant amending Instruments are S.I. 1999/1539 and 3108.} (date of claim where claim sent or delivered to a gateway office) in paragraph (3) at the end there shall be added the words “or are made at an office of a designated authority in accordance with regulation 62(4)($e$).”.

\medskip

6.  For regulations 62B and 62C\footnote{\frenchspacing The relevant amending Instruments are S.I. 1999/1539 and 3108.}, there shall be substituted the following regulation—
\begin{quotation}
\subsection*{“Date of claim where claim sent or delivered to an office of a designated authority}

62B.---(1)  Where a claim for council tax benefit has been sent or delivered to an office of a designated authority in accordance with regulation 62(4)($e$), the date on which the claim is made shall be—
\begin{enumerate}\item[]
($a$) except where paragraph ($b$)  applies, the date the claim is received at an office of the designated authority; or

($b$) where in the 4 weeks before the claim is received in an office of the designated authority, the person making the claim or a person acting on his behalf had notified an office of a \pagebreak[3] designated authority of his intention to make such a claim, the date the notification was given.
\end{enumerate}

(2) A notification of intention to make a claim is deemed to be given on the date on which notification of the intention to claim council tax benefit is received, in whatever form, from the claimant, or the person acting on his behalf, at an office of a designated authority.

(3) Paragraph (2) applies where neither income support nor a jobseeker’s allowance is claimed in conjunction with council tax benefit.

(4) Where the person claiming council tax benefit in accordance with regulation 62(4)($e$), or the partner of that person,—
\begin{enumerate}\item[]
($a$) has an award of income support or income-based jobseeker’s allowance; or

($b$) has claimed such a benefit but no award has been made,
\end{enumerate}
the date on which the claim for council tax benefit is made shall be determined as if sub-paragraphs ($a$), ($b$), ($c$)  and ($e$)  of paragraph (1) of regulation 62A applied to that claim as they apply to claims under regulation 62(4)($d$).”.
\end{quotation}

\medskip

7.  In regulation 63 (evidence and information), paragraph (7) shall be omitted.

\medskip

8.  In regulation 65 (duty to notify changes of circumstances), in paragraph (4), for the words “relevant authority” there shall be substituted the words “designated authority”.

\part[Schedule 5 --- Consequential amendments to the Claims and Payments Regulations]{Schedule 5\\*Consequential amendments to the Claims and Payments Regulations}

\renewcommand\parthead{--- Schedule 5}

1.  The Claims and Payments Regulations shall be amended in accordance with the following provisions of this Schedule.

\medskip

2.  In regulation 4 (making a claim for benefit) after paragraph (1C)\footnote{\frenchspacing The relevant amending Instrument is S.I. 1997/793.} there shall be inserted the following paragraph—
\begin{quotation}
“(1D) In calculating any period of one month for the purposes of paragraph (7) and regulation 6(1A)($b$), there shall be disregarded any period commencing on a day on which a person is first notified of a decision that he failed to take part in a work-focused interview and ending on a day on which he was notified that that decision has been revised so that the decision as revised is that he did take part.”.
\end{quotation}

\medskip

3.  In paragraph (1) of regulation 6 (date of claim)\footnote{\frenchspacing The relevant amending Instruments are S.I. 1990/725 and 1997/793.} after the words “this regulation” there shall be inserted the words “or regulation 6A (claims by persons subject to work-focused interviews)”.

\medskip

4.  After regulation 6, there shall be inserted the following regulation—
\begin{quotation}
\subsection*{“Claims by persons subject to work-focused interviews}

6A.---(1)  This regulation applies to any person who is required to take part in a work-focused interview in accordance with regulation 4 of the Social Security (Work-focused Interviews) Regulations 2000 (“the Work-focused Interviews Regulations”).

(2) Subject to the following provisions of this regulation, where a person takes part in a work-focused interview, the date on which the claim is made shall be—
\begin{enumerate}\item[]
($a$) in a case where—
\begin{enumerate}\item[]
(i) the claim made by the claimant meets the requirements of regulation 4(1), or

(ii) the claim made by the claimant is for income support and meets the requirements of regulation 4(1A),
\end{enumerate}
the date on which the claim is received in the appropriate office;

($b$) in a case where a claim does not meet the requirements of regulation 4(1) but is treated, under regulation 4(7), as having been duly made, the date on which the claim was treated as received in the appropriate office in the first instance;

($c$) in a case where—
\begin{enumerate}\item[]
(i) first notification of intention to claim income support is made to an appropriate office, or

(ii) a claim for income support is received in an appropriate office which does not meet the requirements of regulation 4(1A),
\end{enumerate}
the date of notification or, as the case may be, the date the claim is first received where the properly completed claim form is received within 1 month of notification or the date the claim is first received, or the day on which a properly completed claim form is received where these requirements are not met.
\end{enumerate}

(3) In a case where a decision is made that a person is regarded as not having made a claim for any benefit because he failed to take part in a work-focused interview but subsequently claims such a benefit, in applying paragraph (2) to that claim no regard shall be had to any claim regarded as not having been made in consequence of that decision.

(4) Paragraph (2) shall not apply in any case where a decision has been made that the claimant has failed to take part in a work-focused interview.

(5) In regulation 4 and this regulation, “work-focused interview” has the meaning it has in regulation 3 of the Work-focused Interviews Regulations and in this regulation “designated authority” has the meaning it has in regulation 2(1) of the Work-focused Interviews Regulations.”.
\end{quotation}

\part[Schedule 6 --- Decisions and appeals]{Schedule 6\\*Decisions and appeals}

\renewcommand\parthead{--- Schedule 6}

1.  The Social Security and Child Support (Decisions and Appeals) Regulations 1999\footnote{\frenchspacing S.I. 1999/991; the relevant amending Instruments are S.I. 1999/1623 and 1670.} shall be amended in accordance with paragraphs 2 to 7.

\medskip

2.  In regulation 1(3) (interpretation)—
\begin{enumerate}\item[]
($a$) after the entry relating to “the date of notification” there shall be inserted the following entry—
\begin{quotation}
    ““designated authority” has the meaning it has in regulation 2(1) of the Work-focused Interviews Regulations;”; 
\end{quotation}

($b$) for the entry relating to “official error”, there shall be substituted the following entry—
\begin{quotation}
    ““official error” means an error made by—
\begin{enumerate}\item[]
    ($a$) 
    an officer of the Department of Social Security, the Board or the Department for Education and Employment acting as such which no person outside any of those Departments caused or to which no person outside any of those Departments materially contributed;

    ($b$) 
    a person employed by a designated authority acting on behalf of the authority, which no person outside that authority caused or to which no person outside that authority materially contributed;”; 
\end{enumerate}
\end{quotation}

($c$) after the entry relating to “the Transfer Act”, there shall be inserted the following entries—
\begin{quotation}
    ““work-focused interview” has the meaning it has in regulation 3 of the Work-focused Interviews Regulations;

    “the Work-focused Interviews Regulations” means the Social Security (Work-focused Interviews) Regulations 2000\footnote{\frenchspacing S.I. 2000/897.};”. 
\end{quotation}
\end{enumerate}

\medskip

3.  In regulation 3 (revision of decisions)—
\begin{enumerate}\item[]
($a$) after paragraph (6), there shall be inserted the following paragraph—
\begin{quotation}
“(6A) A relevant decision within the meaning of section 2B(2) of the Administration Act\footnote{\frenchspacing Section 2B was inserted by Section 57 of the Welfare Reform Act 1999 (c. 30).} may be revised at any time if it contains an error.”;
\end{quotation}

($b$) in paragraph (11), at the end of sub-paragraph ($e$)  there shall be added—
\begin{quotation}
“or

($f$) in the case of a relevant person within the meaning of regulation 2(2) of the Work-focused Interviews Regulations, an office of any designated authority which displays the ONE logo.”
\end{quotation}
\end{enumerate}

\medskip

4.  In regulation 6 (supersession of decisions), in paragraph (2), at the end there shall be added the following—
\begin{quotation}
“and

($h$) is one in respect of a person who—
\begin{enumerate}\item[]
(i) is subsequently the subject of a separate decision or determination as to whether or not he took part in a work-focused interview;

(ii) had been held not to have taken part in a work-focused interview but who had, subsequent to the decision to be superseded, attained the age of 60 or ceased to reside in an area in which there is a requirement to take part in a work-focused interview.”.
\end{enumerate}
\end{quotation}

\medskip

5.  In regulation 7 (date from which decision superseded takes effect)\footnote{\frenchspacing The relevant amending Instrument is S.I. 1999/1623.}, at the end there shall be added the following paragraphs—
\begin{quotation}
“(25) In a case where a decision (“the first decision”) has been made that a person failed without good cause to take part in a work-focused interview, the decision under section 10 shall take effect as from the first day of the benefit week to commence for that person following the date of the first decision.

(26) In paragraph (25), “benefit week” means any period of 7 days corresponding to the week in respect of which the relevant social security benefit is due to be paid.”.
\end{quotation}

\medskip

6.  In regulation 33\footnote{\frenchspacing The relevant amending Instruments are S.I. 1999/1662 and 2570.}, in paragraph (2), after sub-paragraph ($dd$)  there shall be inserted the following sub-paragraph—
\begin{quotation}
“($ddd$) in a case where the decision appealed against was a decision arising from a claim to a designated office, an office of a designated authority;”.
\end{quotation}

\medskip

7.  At the end of Schedule 2 there shall be added the following paragraph—
\begin{quotation}
“26.  Any decision treated as a decision of the Secretary of State whether or not to waive or defer a work-focused interview.”.
\end{quotation}

\medskip

8.  In regulation 2 of the Child Support (Maintenance Assessment Procedure) Regulations 1992\footnote{\frenchspacing S.I. 1992/1813; the relevant amending Instrument is S.I. 1999/1047.}, in paragraph (2)—
\begin{enumerate}\item[]
($a$) the following entry shall be inserted in the appropriate place—
\begin{quotation}
    ““designated authority” has the meaning it has in regulation 2(1) of the Social Security (Work-focused Interviews) Regulations 2000.”; 
\end{quotation}

($b$) for the definition of “official error” there shall be substituted the following definition—
\begin{quotation}
    ““official error” means an error made by—
\begin{enumerate}\item[]
    ($a$) 
    an officer of the Department of Social Security acting as such which no person outside that Department caused or to which no person outside that Department materially contributed;

    ($b$) 
    a person employed by a designated authority acting on behalf of the authority, which no person outside that authority caused or to which no person outside that authority materially contributed;”. 
\end{enumerate}
\end{quotation}
\end{enumerate}

\medskip

9.  In regulation 1 of the Child Support Departure Direction and Consequential Amendment Regulations 1996\footnote{\frenchspacing S.I. 1996/2907; the relevant amending Instrument is S.I. 1999/1047.} in paragraph (2)—
\begin{enumerate}\item[]
($a$) after the definition of “departure direction application form” there shall be inserted the following definition—
\begin{quotation}
    ““designated authority” has the meaning it has in regulation 2(1) of the Social Security (Work-focused Interviews) Regulations 2000.”; 
\end{quotation}

($b$) for the definition of “official error” there shall be substituted the following definition—
\begin{quotation}
    ““official error” means an error made by—
\begin{enumerate}\item[]
    ($a$) 
    an officer of the Department of Social Security acting as such which no person outside that Department caused or to which no person outside that Department materially contributed;

    ($b$) 
    a person employed by a designated authority acting on behalf of the authority, which no person outside that authority caused or to which no person outside that authority materially contributed;”.
\end{enumerate}
\end{quotation}
\end{enumerate}

\part{Explanatory Note}

\renewcommand\parthead{— Explanatory Note}

\subsection*{(This note is not part of the Regulations)}

The Regulations contained in this Instrument are made either by virtue of, or in consequence of, provisions in the Welfare Reform and Pensions Act 1999 (c.\ 30) (“the 1999 Act”). This Instrument is made before the expiry of the period of 6 months beginning with the coming into force of those provisions; the regulations in it are therefore exempt from the requirement in section 172(2) of the Social Security Administration Act 1992 (c.\ 5) to refer proposals to make Regulations to the Social Security Advisory Committee and are made without reference to that Committee.

Part I of these Regulations contains general provisions relating to their citation, commencement and interpretation (regulations 1 to 3 and Schedule 1). They also provide for the Regulations to apply in certain areas of the country only.

Part II relates to the work-focused interview. Regulation 4 specifies those persons claiming social security benefits who are required to take part in a work-focused interview. Regulation 5 specifies a number of exemptions. Regulation 6 specifies circumstances in which a claimant’s continuing entitlement to the full amount of benefit is to be dependent upon his taking part in a work-focused interview.

Regulations 7 and 8 contain provisions as to waiver and deferment. Regulation 9 specifies when a requirement to take part in 2 or more work-focused interviews is satisfied by the person taking part in a single interview. Regulation 10 provides for the claimant to be advised of the time and place of the interview.

Regulation 11 sets out the requirements for taking part in a work-focused interview and regulation 12 details the consequences of a failure to take part in the interview. Regulation 13 specifies circumstances where those consequences do not apply. Regulation 14 specifies the matters to be taken into account in determining whether a person had good cause for his failure to take part in an interview.

Regulation 15 provides a right of appeal against a decision that a person did not take part in a work-focused interview.

Regulation 16 and Schedules 2 to 6 contain amendments consequential upon these changes and regulation 17 contains amendments to the Social Security (Claims and Information) Regulations 1999.

These Regulations do not impose a charge on businesses. 

\end{document}