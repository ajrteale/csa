\documentclass[12pt,a4paper]{article}

\newcommand\regstitle{The Child Support (Voluntary Payments) Regulations 2000}

\newcommand\regsnumber{2000/3177}

%\opt{newrules}{
\title{\regstitle}
%}

%\opt{2012rules}{
%\title{Child Maintenance and Other Payments Act 2008\\(2012 scheme version)}
%}

\author{S.I. 2000 No. 3177}

\date{Made 30th November 2000\\Laid before Parliament 6th December 2000\\Coming into force as provided in regulation 1(1)}

%\opt{oldrules}{\newcommand\versionyear{1993}}
%\opt{newrules}{\newcommand\versionyear{2003}}
%\opt{2012rules}{\newcommand\versionyear{2012}}

\usepackage{csa-regs}

\setlength\headheight{27.57402pt}

\begin{document}

\maketitle

\amendment{
These regulations do not apply to 1993 scheme cases - see the Child Support, Pensions and Social Security Act 2000 (Commencement No. 12) Order 2003 art. 5.
}

\medskip

\noindent
The Secretary of State for Social Security, in exercise of the powers conferred upon him by sections 28J(5), 52(1) and (4) and 54 of the Child Support Act 1991\footnote{\frenchspacing 1991 c. 48. Section 28J is inserted by, and sections 52 and 54 are amended by, respectively, sections 20(1) and 25 of, and paragraph 11(20) of Schedule 3 to, the Child Support, Pensions and Social Security Act 2000 (c. 19). Section 54 is cited because of the meaning ascribed to the word “prescribed”.} and of all other powers enabling him in that behalf, hereby makes the following Regulations: 

{\sloppy

\tableofcontents

}

\bigskip

\setcounter{secnumdepth}{-2}

\subsection[1. Citation, commencement and interpretation]{Citation, commencement and interpretation}

1.---(1)  These Regulations may be cited as the Child Support (Voluntary Payments) Regulations 2000 and shall come into force on the day on which section 28J of the Act as inserted by the Child Support, Pensions and Social Security Act 2000\footnote{\frenchspacing 2000 c. 19.} comes into force.

(2) In these Regulations—
\begin{enumerate}\item[]
“the Act” means the Child Support Act 1991;

“debit card” means a card, operating as a substitute for a cheque, that can be used to obtain cash or to make a payment at a point of sale whereby the card holder’s bank or building society account is debited without deferment of payment;

“the Maintenance Calculations and Special Cases Regulations” means the Child Support (Maintenance Calculations and Special Cases) Regulations 2000\footnote{\frenchspacing S.I. 2001/155.};

“the qualifying child’s home” means the home in which the qualifying child resides with the person with care and “home” has the meaning given in regulation 1 of the Maintenance Calculations and Special Cases Regulations; and

“relevant person” means—
\begin{enumerate}\item[]
($a$) 
a person with care;

($b$) 
a non-resident parent;

($c$) 
a parent who is treated as a non-resident parent under regulation 8 of the Maintenance Calculations and Special Cases Regulations;

($d$) 
where the application for a maintenance calculation is made by a child under section 7 of the Act, that child,
\end{enumerate}
in respect of whom a maintenance calculation has been applied for, or has been treated as applied for, under section 6(3) of the Act, or is or has been in force.
\end{enumerate}

\subsection[2. Voluntary payment]{Voluntary payment}

2.---(1)  A payment counts as a voluntary payment if it is—
\begin{enumerate}\item[]
($a$) made in accordance with section 28J(2) and (4) of the Act;

($b$) of a type to which regulation 3 applies;

($c$) made on or after the effective date of the maintenance calculation made, or which would be made but for the Secretary of State’s decision not to make one, and for this purpose “effective date” means the effective date as determined in accordance with the Child Support (Maintenance Calculation Procedure) Regulations 2000\footnote{\frenchspacing S.I. 2001/157.}; and

($d$) a payment in relation to which evidence or verification of a type to which regulation 4 applies is provided, if the Secretary of State so requires.
\end{enumerate}

(2) Where the Secretary of State is considering whether a payment is a voluntary payment, he may invite representations from a relevant person.

\subsection[3. Types of payment]{Types of payment}

3.  This regulation applies to a payment made by the non-resident parent—
\begin{enumerate}\item[]
($a$) by any of the following methods—
\begin{enumerate}\item[]
(i) in cash;

(ii) by standing order;

(iii) by any other method which requires one person to give his authority for payments to be made from an account of his to an account of another on specific dates during the period for which the authority is in force and without the need for any further authority from him;

(iv) by an arrangement whereby one person gives his authority for payments to be made from an account of his, or on his behalf, to another person or to an account of that other person;

(v) by cheque or postal order; or

(vi) by debit card, and
\end{enumerate}

($b$) which is, or is in respect of,—
\begin{enumerate}\item[]
(i) a payment in lieu of child support maintenance and which is paid to the person with care;

(ii) a mortgage or loan taken out on the security of the property which is the qualifying child’s home where that mortgage or loan was taken out to facilitate the purchase of, or to pay for essential repairs or improvements to, that property;

(iii) rent on the property which is the qualifying child’s home;

(iv) mains-supplied gas, water or electricity charges at the qualifying child’s home;

(v) council tax payable by the person with care in relation to the qualifying child’s home;

(vi) essential repairs to the heating system in the qualifying child’s home; or

(vii) repairs which are essential to maintain the fabric of the qualifying child’s home.
\end{enumerate}
\end{enumerate}

\subsection[4. Evidence or verification of payment]{Evidence or verification of payment}

4.  This regulation applies to—
\begin{enumerate}\item[]
($a$) evidence provided by the non-resident parent in the form of—
\begin{enumerate}\item[]
(i) a bank statement;

(ii) a duplicate of a cashed cheque;

(iii) a receipt from the payee; or

(iv) a receipted bill or invoice; or
\end{enumerate}

($b$) verification orally or in writing from the person with care.
\end{enumerate}

\bigskip

Signed 
by authority of the Secretary of State for Social Security.

{\raggedleft
\emph{P.~Hollis}\\*Parliamentary Under-Secretary of State,\\*Department of Social Security

}

30th November 2000

\small

\part{Explanatory Note}

\renewcommand\parthead{--- Explanatory Note}

\subsection*{(This note is not part of the Regulations)}

These Regulations are made pursuant to section 28J of the Child Support Act 1991 (c.\ 48) (“the Act”) as inserted by the Child Support, Pensions and Social Security Act 2000 (c.\ 19) (“the 2000 Act”). Section 28J of the Act provides for the Secretary of State to offset against child support maintenance arrears, or to adjust a maintenance calculation to take account of, voluntary payments.

Regulation 1 contains provisions relating to citation, commencement and interpretation. These Regulations come into force according to the date on which section 28J of the 2000 Act is commenced.

Regulation 2 defines a “voluntary payment” by reference to regulations 3 and 4 and provides that, to be a voluntary payment, a payment must be made on or after the effective date of the maintenance calculation, which is governed by the Child Support (Maintenance Calculation Procedure) Regulations 2000.

Regulation 3 defines the types of payments which may count as voluntary payments and regulation 4 defines the evidence or verification of such payments which the Secretary of State may require to be provided.

The impact on business of these Regulations was covered in the Regulatory Impact Assessment (RIA) for the 2000 Act, in accordance with which, and in consequence of which, these Regulations are made. A copy of that RIA has been placed in the libraries of both Houses of Parliament and can be obtained from the Department of Social Security, Regulatory Impact Unit, Adelphi, 1--11 John Adam Street, London, \textsc{\lowercase{WC2N 6HT}}. 

\end{document}