\documentclass[a4paper]{article}

\usepackage[welsh,english]{babel}

\usepackage[utf8]{inputenc}
\usepackage[T1]{fontenc}

\usepackage{textcomp}

%\usepackage[2012rules]{optional}

\usepackage[osf]{mathpazo}

%\opt{newrules}{
\title{The Child Support Appeal Tribunals (Procedure) Regulations 1992}
%}

%\opt{2012rules}{
%\title{Child Maintenance and Other Payments Act 2008\\(2012 scheme version)}
%}

\author{S.I. 1992 No. 2641}

\date{Made 26th October 1992\\Laid before Parliament 29th October 1992\\Coming into force 5th April 1993}

%\opt{oldrules}{\newcommand\versionyear{1993}}
%\opt{newrules}{\newcommand\versionyear{2003}}
%\opt{2012rules}{\newcommand\versionyear{2012}}

\usepackage{fancyhdr}
\pagestyle{fancy}
\fancyhead[L]{}
\fancyhead[C]{\itshape The Child Support Appeal Tribunals (Procedure) Regulations 1992 (S.I.~1992/2641) \parthead%\phantom{...}% (\versionyear{} scheme version)
}
\fancyhead[R]{}
\fancyfoot[C]{\thepage}
\newcommand{\parthead}{}

\usepackage{array}
\usepackage{multirow}
\usepackage[debugshow]{tabulary}
\usepackage{longtable}
\usepackage{multicol}
\usepackage{lettrine}

\usepackage[colorlinks=true]{hyperref}
\usepackage{microtype}

\hyphenation{Aw-dur-dod}
\hyphenation{bank-rupt-cy}
\hyphenation{Ec-cles-ton}
\hyphenation{Eux-ton}
\hyphenation{Hogh-ton}
\hyphenation{Pres-ton}
\hyphenation{Pru-den-tial}
\hyphenation{Riv-ing-ton}

\newcolumntype{x}[1]
	{>{\raggedright}p{#1}}
\newcommand{\tn}{\tabularnewline}
\setlength\tymin{50pt}

\newcommand\amendment[1]{\subsubsection*{Notes}{\itshape\frenchspacing\footnotesize #1 \par}}

\usepackage{perpage} %the perpage package
\MakePerPage{footnote} %the perpage package command
\renewcommand{\thefootnote}{\fnsymbol{footnote}}

\usepackage[perpage,para,symbol]{footmisc}

\begin{document}

\maketitle

\noindent
The Secretary of State for Social Security, in exercise of the powers conferred by sections 21(2) and (3) and 51(1) of the Child Support Act 1991\footnote{\frenchspacing 1991 c. 48.} and of all other powers enabling him in that behalf, after consultation with the Council on Tribunals in accordance with section 8 of the Tribunals and Inquiries Act 1992\footnote{\frenchspacing 1992 c. 53.}, hereby makes the following Regulations:

{\sloppy

\tableofcontents

}

\setcounter{secnumdepth}{-2}

\subsection[1. Citation, commencement and interpretation]{Citation, commencement and interpretation}

1.—(1) These Regulations may be cited as the Child Support Appeal Tribunals (Procedure) Regulations 1992 and shall come into force on 5th April 1993.

(2) In these Regulations, unless the context otherwise requires:–
\begin{enumerate}\item[]
“absent parent” has the meaning assigned to it in section 3(2) of the Act;

“the Act” means the Child Support Act 1991;

“Central Office” means the Central Office of Child Support Appeal Tribunals at Anchorage Two, Anchorage Quay, Salford Quays, Manchester, M5 2YN;

“chairman”, subject to paragraph (3), means a person nominated under paragraph 3 of Schedule 3 to the Act and includes the President and any full-time chairman;

“clerk to the tribunal” means a person appointed under paragraph 6 of Schedule 3 to the Act;

“Commissioner” means the Chief or any other Child Support Commissioner appointed under section 22 of the Act;

“full-time chairman” means a regional or other full-time chairman of a child support appeal tribunal appointed under paragraph 4 of Schedule 3 to the Act;

“party to the proceedings” means–
\begin{enumerate}\item[]
($a$) the person with care;

($b$) the absent parent;

($c$) any child who has made an application for a maintenance assessment under section 7 of the Act;

($d$) the child support officer;

($e$) any other person, who on an application made by him, appears to the chairman of the tribunal to be interested in the proceedings;
\end{enumerate}

“person with care” has the meaning assigned to it by section 3(3) of the Act;

“President” has the meaning assigned to it in paragraph 1(1) of Schedule 3 to the Act;

“proceedings” means proceedings on an appeal or application to which these Regulations apply; and

“tribunal” means a child support appeal tribunal constituted in accordance with section 21 of the Act.
\end{enumerate}

(3) Unless otherwise provided, where by these Regulations anything is required to be done by, or any power is conferred on, a chairman, then---
\begin{enumerate}\item[]
($a$) if that thing is to be done or the power is to be exercised at the hearing of an appeal or application, it shall be done or exercised by the chairman of the tribunal hearing the appeal or application; and

($b$) otherwise, shall be done or exercised by a person who is eligible to be nominated to act as a chairman of a child support appeal tribunal under paragraph 3(2) of Schedule 3 to the Act.
\end{enumerate}

(4) In these Regulations, unless the context otherwise requires, a reference---
\begin{enumerate}\item[]
($a$) to a numbered regulation is to the regulation in these Regulations bearing that number; and

($b$) in a regulation to a numbered paragraph is to the paragraph in that regulation bearing that number.
\end{enumerate}

\subsection[2. Service of notices or documents]{Service of notices or documents}

2.—(1) Where by any provision of the Act or of these Regulations any notice or other document is required to be given or sent to the clerk to the tribunal that notice or document shall be treated as having been so given or sent on the day that it is received by the clerk to the tribunal.

(2) Where by any provision of the Act or of these Regulations any notice or other document is required to be given or sent to any person other than the clerk to the tribunal that notice or document shall, if sent by post to that person’s last known address, be treated as having been given or sent on the day that it was posted.

%Reg 2(3) inserted (18.4.95) by SI 1995/1045 reg 2.
(3) The provisions of paragraph (2) shall apply to a summons or a citation issued under regulation 10.

\amendment{
Reg. 2(3) inserted (18.4.95) by the Child Support and Income Support (Amendment) Regulations 1995 reg. 2.
}

\subsection[3. Making an appeal or application and time limits]{Making an appeal or application and time limits}

3.—%(1) An appeal to a tribunal under section 20(1) of the Act or an application to a tribunal to set aside its decision under regulation 15 shall be by notice in writing signed by the person making it or by his representative where it appears to a chairman that he was unable to sign personally, or by a barrister, advocate or solicitor on his behalf.
%Reg 3(1), (1A) substituted for reg 3(1) (18.4.95) by SI 1995/1045 reg 3
(1) This regulation applies to an appeal to a tribunal under—
\begin{enumerate}\item[]
($a$) section 20(1) or 46(7) of the Act; or

($b$) regulation 42(9) of the Child Support (Maintenance Assessment Procedure) Regulations 1992\footnote{\frenchspacing S.I. 1992/1813.},
\end{enumerate}
and to an application to a tribunal to set aside its decision under regulation 15.

%(1A) An appeal or application of a kind mentioned in paragraph (1) shall be by notice in writing, signed by the person making it, or by his representative where it appears to a chairman that he was unable to sign personally, or by a barrister, advocate or solicitor on his behalf.

% Reg. 3(1A) substituted (21.10.96) by SI 1996/2450 reg. 14(2)
(1A) An appeal or application of a kind mentioned in paragraph (1) shall be by notice in writing, and, in the case of an appeal, shall be on a form approved by the Secretary of State and shall be signed by the person making it, or by his representative where it appears to a chairman that he was unable to sign it personally, or by a barrister, advocate or solicitor on his behalf.

(2) The notice shall be made or given by sending or delivering it to the clerk to the tribunal at the Central Office.

(3) An appeal under section 20(1) of the Act shall be brought within the period of 28 days beginning with the date on which notification of the decision in question was given or sent to the appellant.

(4) An application under regulation 15 shall be made within the period of 3 months beginning with the date when a copy of the record of the decision was given or sent to the applicant.

(5) In paragraphs (6) and (7) “the specified time” means the time specified in paragraph (3) or, as the case may be, paragraph (4).

(6) When an appeal or application is made after the specified time has expired, that time may for special reasons be extended by the chairman to the date of the making of the appeal or application.

(7) Any appeal or application made after the specified time has expired which does not include an application for an extension of time shall be deemed to include such an application, and if it appears to a chairman that an application for an extension of time does not state reasons for the appeal or application being made after the specified time the chairman may before determining it give the person making the application for an extension of time a reasonable opportunity to provide reasons.

(8) An application for an extension of time which has been refused may not be renewed, but any chairman may set aside a refusal if it appears to him just to do so on any of the grounds set out in regulation 15(1).

%(9) In the case of an appeal the notice shall contain sufficient particulars of the decision under appeal to enable that decision to be identified.

% Reg 3(9), (9A) substituted for reg. 3(9) (21.10.96) by SI 1996/2450 reg 14(3)
(9) A notice of appeal shall contain particulars of the date of the notification of the decision against which the appeal is made, the subject matter of the decision and a summary of the arguments relied on by the person making the appeal to support his contention that the decision was wrong.

(9A) Where the notice referred to in paragraph (9) is not made on the form approved for the time being, but is made in writing and contains all the particulars required by paragraph (9), a chairman may treat that appeal as duly made.

(10) Any notice of 
%appeal or   % Words omitted (21.10.96) by SI 1996/2450 reg 14(4)
application other than an application for an extension of time shall state the grounds on which it is made.

%(11) If it appears to a chairman that the notice of appeal does not enable the decision under appeal to be identified or that the notice of appeal or application does not state the grounds on which it is made the chairman may direct the person making it to provide such particulars as the chairman may reasonably require.

% Reg 3(11)--(11B) substituted for reg. 3(11) (21.10.96) by SI 1996/2450 reg 14(5)
(11) Where it appears to a chairman or the clerk to the tribunal that the notice of appeal does not contain the particulars required under paragraph (9), or that the notice of application does not contain the particulars required under paragraph (10), he may direct the person making the appeal or application to furnish those further particulars.

(11A) Where further particulars are required under paragraph (11), in the case of an appeal they shall be sent or delivered to the clerk to the tribunal at the Central Office within such period as a chairman or the clerk to the tribunal may direct.

(11B) The date of an appeal or application shall be the date on which all the particulars required under paragraph (9) are received in the Central Office.

\amendment{
Reg. 3(1), (1A) substituted for reg. 3(1) (18.4.95) by the Child Support and Income Support (Amendment) Regulations 1995 reg. 3.

Words omitted in reg. 3(10), reg. 3(9), (9A) substituted for reg. 3(9), reg. 3(11)--(11B) substituted for reg. 3(11) and reg. 3(1A) substituted (21.10.96) by the Social Security (Adjudication) and Child Support Amendment (No.\ 2) Regulations 1996 reg. 14.

Under the Social Security (Adjudication) and Child Support Amendment (No.\ 2) Regulations 1996 reg. 22, the following version of reg. 3 applies to cases where appeals or applications are made before 21.10.96:

\begin{quotation}
3.—%(1) An appeal to a tribunal under section 20(1) of the Act or an application to a tribunal to set aside its decision under regulation 15 shall be by notice in writing signed by the person making it or by his representative where it appears to a chairman that he was unable to sign personally, or by a barrister, advocate or solicitor on his behalf.
%Reg 3(1), (1A) substituted for reg 3(1) (18.4.95) by SI 1995/1045 reg 3
(1) This regulation applies to an appeal to a tribunal under—
\begin{enumerate}\item[]
($a$) section 20(1) or 46(7) of the Act; or

($b$) regulation 42(9) of the Child Support (Maintenance Assessment Procedure) Regulations 1992\footnote{\frenchspacing S.I. 1992/1813.},
\end{enumerate}
and to an application to a tribunal to set aside its decision under regulation 15.

(1A) An appeal or application of a kind mentioned in paragraph (1) shall be by notice in writing, signed by the person making it, or by his representative where it appears to a chairman that he was unable to sign personally, or by a barrister, advocate or solicitor on his behalf.

(2) The notice shall be made or given by sending or delivering it to the clerk to the tribunal at the Central Office.

(3) An appeal under section 20(1) of the Act shall be brought within the period of 28 days beginning with the date on which notification of the decision in question was given or sent to the appellant.

(4) An application under regulation 15 shall be made within the period of 3 months beginning with the date when a copy of the record of the decision was given or sent to the applicant.

(5) In paragraphs (6) and (7) “the specified time” means the time specified in paragraph (3) or, as the case may be, paragraph (4).

(6) When an appeal or application is made after the specified time has expired, that time may for special reasons be extended by the chairman to the date of the making of the appeal or application.

(7) Any appeal or application made after the specified time has expired which does not include an application for an extension of time shall be deemed to include such an application, and if it appears to a chairman that an application for an extension of time does not state reasons for the appeal or application being made after the specified time the chairman may before determining it give the person making the application for an extension of time a reasonable opportunity to provide reasons.

(8) An application for an extension of time which has been refused may not be renewed, but any chairman may set aside a refusal if it appears to him just to do so on any of the grounds set out in regulation 15(1).

(9) In the case of an appeal the notice shall contain sufficient particulars of the decision under appeal to enable that decision to be identified.

(10) Any notice of appeal or application other than an application for an extension of time shall state the grounds on which it is made.

(11) If it appears to a chairman that the notice of appeal does not enable the decision under appeal to be identified or that the notice of appeal or application does not state the grounds on which it is made the chairman may direct the person making it to provide such particulars as the chairman may reasonably require.
\end{quotation}
}

%Reg 3A inserted (18.4.95) by SI 1995/1045 reg 4
\subsection[3A. Death of a party to an appeal]{Death of a party to an appeal}

3A.—(1) In any proceedings, on the death of a party to those proceedings, the Secretary of State may appoint such person as he thinks fit to proceed with the appeal in the place of such deceased party.

(2) A grant of probate, confirmation or letters of administration to the estate of the deceased party, whenever taken out, shall have no effect on an appointment made under paragraph (1).

(3) Where a person appointed under paragraph (1) has, prior to the date of such appointment, taken any action in relation to the appeal on behalf of the deceased party, the effective date of the appointment by the Secretary of State shall be the day immediately prior to the first day on which such action was taken.

\amendment{
Reg. 3A inserted (18.4.95) by the Child Support and Income Support (Amendment) Regulations 1995 reg. 4.
}

\subsection[4. Lack of jurisdiction]{Lack of jurisdiction}

4.  When a chairman is satisfied that the tribunal does not have jurisdiction to entertain a purported appeal he may make a declaration to that effect and such declaration shall dispose of the purported appeal.

\subsection[5. Directions]{Directions}

5.%
---(1)  % Reg 5 renumbered as reg 5(1) (21.10.96) by SI 1996/2450 reg 15
  At any stage of the proceedings a chairman may either of his own motion or on a written application made to the clerk to the tribunal by any party to the proceedings give such directions as he may consider necessary or desirable for the just, effective and efficient conduct of the proceedings and may direct any party to provide such further particulars or to produce such documents as may reasonably be required.

% Reg 5(2) inserted (21.10.96) by SI 1996/2450 reg 15
(2) Where under these Regulations the clerk to the tribunal is authorised to take steps in relation to the procedure of the tribunal, he may give directions requiring any party to the proceedings to comply with any provision of these Regulations.

\amendment{
Reg. 5 renumbered as reg. 5(1) and reg. 5(2) inserted (21.10.96) by the Social Security (Adjudication) and Child Support Amendment (No.\ 2) Regulations 1996 reg. 15.
}

\subsection[6. Striking out of proceedings]{Striking out of proceedings}

6.—(1) Subject to paragraph (2), a chairman may, either of his own motion or on the application of any party to the proceedings, order that the appeal or application be struck out 
%because of 
for want of prosecution which term includes  % Words substituted (18.4.95) by SI 1995/1045 reg 5
the failure of the appellant or applicant to comply with 
%a direction under regulation 3(11) or 5 or to reply to an enquiry from the clerk to the tribunal about his availability to attend a hearing.
a direction under regulation 3(11), 5(1) or (2).  % Words substituted (21.10.96) by SI 1996/2450 reg 16(2)

% Reg 6(1A), (1B) inserted (21.10.96) by SI 1996/2450 reg 16(3)
(1A) Where a chairman decides not to strike out an appeal or application under paragraph (1) he shall consider whether the appeal or application should be determined forthwith in accordance with these Regulations.

(1B) Where a chairman decides that an appeal or application should not be determined forthwith under paragraph (1A) he shall consider whether he should make further directions with a view to expediting the hearing of the appeal or application.

(2) Before making an order under paragraph (1) the chairman shall send notice to the person against whom it is proposed that any such order should be made and any other party to the proceedings giving each of them a reasonable opportunity to show cause why such an order should not be made.

% Reg 6(2A) inserted (21.10.96) by SI 1996/2450 reg 16(4)
(2A) Paragraph (2) shall not require a notice to be sent to a party, including a person against whom it is proposed that an order under paragraph (1) should be made, where his address is unknown to the chairman or the clerk to the tribunal and cannot be ascertained by reasonable enquiry.

(3) The chairman may, on application by any party to the proceedings made not later than 
%one year 
3 months  % Words substituted (21.10.96) by SI 1996/2450 reg 16(5)(a)
beginning with the date of the order made under paragraph (1), give leave to reinstate any appeal or application which has been struck out in accordance with that order
if he is satisfied that the party concerned did not receive a notice under paragraph (2) and that the conditions in paragraph (2A) were not met.  % Words inserted (21.10.96) by SI 1996/2450 reg 16(5)(b)

\amendment{
Words substituted in reg. 6(1) (18.4.95) by the Child Support and Income Support (Amendment) Regulations 1995 reg. 5.

Words inserted in reg. 6(3), words substituted in reg. 6(1), (3) and reg. 6(1A), (1B), (2A) inserted (21.10.96) by the Social Security (Adjudication) and Child Support Amendment (No.\ 2) Regulations 1996 reg. 16.

Under the Social Security (Adjudication) and Child Support Amendment (No.\ 2) Regulations 1996 reg. 22 the following version of reg. 6(3) continues to apply where an appeal or application was made before 21.10.96:
\begin{quotation}
(3) The chairman may, on application by any party to the proceedings made not later than one year beginning with the date of the order made under paragraph (1), give leave to reinstate any appeal or application which has been struck out in accordance with that order.
\end{quotation}
}

\subsection[7. Withdrawal of appeals and applications]{Withdrawal of appeals and applications}

7.—(1) Any appeal to a tribunal may be withdrawn by the person making the appeal---
\begin{enumerate}\item[]
($a$) at a hearing with the leave of the chairman; or

%($b$) at any other time, by giving written notice of intention to withdraw to the clerk to the tribunal and either---
%\begin{enumerate}\item[]
%(i) with the consent in writing of every other party to the proceedings; or
%
%(ii) with the leave of the chairman after every other party to the proceedings has had a reasonable opportunity to make representations.
%\end{enumerate}

% Reg 7(1)(b) substituted (21.10.96) by SI 1996/2450 reg 17(2)
($b$) at any other time, provided that the clerk to the tribunal has not received notice under paragraph (1A), by giving written notice of intention to withdraw to the clerk to the tribunal and either—
\begin{enumerate}\item[]
(i) with the consent in writing of every other party to the proceedings other than the child support officer; or

(ii) with the leave of the chairman after every other party to the proceedings other than the child support officer has had a reasonable opportunity to make representations.
\end{enumerate}
\end{enumerate}

% Reg 7(1A) inserted (21.10.96) by SI 1996/2450 reg 17(3)
(1A) An appeal shall not be withdrawn under sub-paragraph ($b$) of paragraph (1) if the clerk to the tribunal has previously received notice opposing a withdrawal of such appeal from the child support officer.

(2) A person who has made an application to a tribunal to set aside their decision under regulation 15 may withdraw it at any time before the application is determined by giving written notice of withdrawal to the clerk to the tribunal.

\amendment{
Reg. 7(1A) inserted and reg. 7(1)($b$) substituted (21.10.96) by the Social Security (Adjudication) and Child Support Amendment (No.\ 2) Regulations 1996 reg. 17.
}

\subsection[8. Postponement]{Postponement}

8.—%(1) Where a person to whom notice of a hearing has been given wishes to request a postponement of that hearing he shall give notice in writing to the clerk to the tribunal stating his reasons for the request and a chairman may grant or refuse the request as he thinks fit.
%
% Reg 8(1) substituted (21.10.96) by SI 1996/2450 reg 18(2)
(1) Where a person to whom notice of a hearing has been given wishes to request a postponement of that hearing he shall do so in writing to the clerk to the tribunal stating his reasons for the application, and the clerk to the tribunal may grant or refuse the application as he thinks fit or may pass the application to a chairman, who may grant or refuse the application as he thinks fit.

(2) A chairman 
or the clerk to the tribunal  % Words inserted (21.10.96) by SI 1996/2450 reg 18(3)
may of his own motion at any time before the beginning of the hearing postpone the hearing.

\amendment{
Words inserted in reg. 8(2) and reg. 8(1) substituted (21.10.96) by the Social Security (Adjudication) and Child Support Amendment (No.\ 2) Regulations 1996 reg. 18.
}

\subsection[9. Representation of parties to the proceedings]{Representation of parties to the proceedings}

9.  Any party to the proceedings may be accompanied and (whether or not the party himself attends) may be represented by another person whether having a professional qualification or not, and for the purposes of any proceedings any such representative shall have all the rights and powers to which the person represented is entitled under these Regulations, except that a representative who is not a barrister, advocate or solicitor shall not have the power to sign the notice of appeal or application.

\subsection[10. Summoning of witnesses]{Summoning of witnesses}

10.—(1) A chairman may by summons or, in Scotland, citation require any person in Great Britain to attend as a witness at a hearing of an appeal or application at such time and place as shall be specified in the summons or citation and, subject to paragraph (2), at the hearing to answer any question or produce any documents in his custody or under his control which relate to any matter in question in the appeal or application, but---
\begin{enumerate}\item[]
($a$) no person shall be required to attend in obedience to such a summons or citation unless he has been given at least 10 days' notice of the hearing or, if less than 10 days' notice is given, he has informed the tribunal that he accepts that notice as sufficient; and

($b$) no person shall be required to attend and give evidence or to produce any document in obedience to such a summons or citation unless the necessary expenses of attendance are paid or tendered to him.
\end{enumerate}

(2) No person shall be compelled to give any evidence or produce any document or other material that he could not be compelled to give or produce on a trial of an action in a court of law in that part of Great Britain where the hearing takes place.

(3) In exercising the powers conferred by this regulation, the chairman shall take into account the need to protect any matter that relates to intimate personal or financial circumstances, is commercially sensitive, consists of information communicated or obtained in confidence or concerns national security.

(4) Every summons or citation issued under this regulation shall contain a statement to the effect that the person in question may apply in writing to a chairman to vary or set aside the summons or citation.

\subsection[11. Hearings]{Hearings}

11.—%(1) A tribunal shall hold an oral hearing of every appeal, and may hold an oral hearing of an application, and subject to the provisions of the Act and of these Regulations the procedure in connection with the hearing shall be such as the chairman shall determine.
%
% Reg 11(1)--(1D) substituted for reg. 11(1) (21.10.96) by SI 1996/2450 reg 19(2)
(1) Where an appeal or application is made to a tribunal, the clerk to the tribunal shall direct every party to the proceedings to notify him if that party wishes an oral hearing of that appeal or application to be held.

(1A) A notification under paragraph (1) shall be in writing and shall be made within 21 days of receipt of the direction from the clerk to the tribunal or within such other period as the clerk to the tribunal or a chairman may direct.

(1B) Where the clerk to the tribunal receives notification in accordance with paragraph (1A) the tribunal shall hold an oral hearing.

(1C) A chairman may of his own motion require an oral hearing to be held if he is satisfied that such a hearing is necessary to enable the tribunal to reach a decision.

(1D) Subject to the provisions of the Act and of these Regulations the procedure in connection with an oral hearing shall be such as the chairman shall determine.

(2) 
Except where paragraph (2C) applies,  % Words inserted (21.10.96) by SI 1996/2450 reg 19(3)(a)
not less than 10 days' notice (beginning with the day on which it is given and ending on the day before the hearing) of 
%the time and place of any hearing
the time and place of any oral hearing  % Words substituted (21.10.96) by SI 1996/2450 reg 19(3)(b)
shall be given to every party to the proceedings, and if such notice has not been given to a person to whom it should have been given under the provisions of this paragraph the hearing may proceed only with the consent of that person.

% Reg 11(2A)--(2C) inserted (21.10.96) by SI 1996/2450 reg 19(4)
(2A) A chairman may give notice for the determination forthwith, in accordance with the provisions of the Act and these Regulations, of an appeal or application notwithstanding that a party to the proceedings has failed to indicate his availability for a hearing or to provide all the information which may have been requested, if the chairman is satisfied that such party—
\begin{enumerate}\item[]
($a$) has failed to comply with a direction regarding his availability or requiring information under regulation 3(11), 5(1) or (2); and

($b$) has not given any explanation for his failure to comply with such a direction; provided that the chairman is satisfied that the tribunal has sufficient particulars in order for the appeal or application to be determined.
\end{enumerate}

(2B) A chairman may give notice for the determination forthwith, in accordance with the provisions of these Regulations, of an appeal or application which he believes has no reasonable prospect of success.

(2C) Any party to the proceedings may waive his right to receive not less than 10 days notice of the time and place of any oral hearing as specified in paragraph (2).

(3) At any hearing any party to the proceedings shall be entitled to be present and be heard.

(4) Any person entitled to be heard at a hearing may address the tribunal, give evidence, call witnesses and put questions directly to any other party to the proceedings, to any representative of the child support officer or to any other person called as a witness.

(5) A tribunal may require any witness to give evidence on oath or affirmation and for that purpose there may be administered an oath or affirmation in due form.

%Reg 11(5A) inserted (18.4.95) by SI 1995/1045 reg 6
(5A) A tribunal may require any person who is, with leave of the tribunal, acting as an interpreter for any person entitled to be heard or for a witness, to swear or affirm that he will carry out his functions correctly and to the best of his skill and understand and, for that purpose, there may be administered an oath or affirmation in due form.

(6) If a party to the proceedings to whom notice has been given under paragraph (2) fails to appear at the hearing the tribunal may, having regard to all the circumstances including any explanation offered for the absence
and where applicable the circumstances set out in sub-paragraphs ($a$) or ($b$) of paragraph (2A),  % Words inserted (21.10.96) by SI 1996/2450 reg 19(5)
proceed with the appeal notwithstanding his absence or give such directions with a view to the determination of the appeal as it may think proper.

% Reg 11(6A) inserted (21.10.96) by SI 1996/2450 reg 19(6)
(6A) Where any party to the proceedings has waived his right to be given notice under paragraph (2C) the tribunal may proceed with the hearing notwithstanding his absence.

(7) Any hearing before the tribunal shall be in private unless the chairman directs that the hearing, or part of it, shall be in public.

(8) The following persons shall also be entitled to be present at a hearing even though it is in private---
\begin{enumerate}\item[]
($a$) the President, any full-time chairman and the clerk to the tribunal;

($b$) any person undergoing training as a chairman or other member of the tribunal or as a clerk to the tribunal;

($c$) any person acting on behalf of the President in the training or supervision of clerks to tribunals;

($d$) a member of the Council on Tribunals or of the Scottish Committee of the Council;

($e$) any person undergoing training as a child support officer or as the representative of a child support officer and any person acting on behalf of the Chief Child Support Officer or the Secretary of State in the training or supervision of child support officers or representatives of child support officers or in the monitoring of standards of adjudication by child support officers;

($f$) with leave of the chairman and the consent of every party to the proceedings actually present, any other person.
\end{enumerate}

(9) For the purposes of arriving at its decision a tribunal shall, and for the purposes of discussing any question of procedure may, notwithstanding anything contained in these Regulations, order all persons to withdraw from the sitting of the tribunal other than the members of the tribunal, any of the persons mentioned in sub-paragraphs ($a$), ($b$) and ($d$) of paragraph (8) and, with the leave of the chairman and if no party to the proceedings actually present objects, any of the persons mentioned in sub-paragraphs ($c$) and ($f$) of that paragraph.

(10) None of the persons mentioned in paragraph (8) shall take any part in the hearing or (where entitled or permitted to remain) in the deliberations of the tribunal.

\amendment{
Reg. 11(5A) inserted (18.4.95) by the Child Support and Income Support (Amendment) Regulations 1995 reg. 6.

Words inserted in reg. 11(2), (6), words substituted in reg. 11(2), reg. 11(2A)--(2C), (6A) inserted and reg. 11(1)--(1D) substituted for reg. 11(1) (21.10.96) by the Social Security (Adjudication) and Child Support Amendment (No.\ 2) Regulations 1996 reg. 19.

Under the Social Security (Adjudication) and Child Support Amendment (No.\ 2) Regulations 1996 reg. 22 the following version of reg. 11(1) continues to apply, and reg. 11(1A)--(1D) do not apply, where an appeal or application was made before 21.10.96:
\begin{quotation}
(1) A tribunal shall hold an oral hearing of every appeal, and may hold an oral hearing of an application, and subject to the provisions of the Act and of these Regulations the procedure in connection with the hearing shall be such as the chairman shall determine.
\end{quotation}

}

\subsection[12. Adjournments]{Adjournments}

12.—(1) A hearing may be adjourned by the tribunal at any time on the application of any party to the proceedings or of its own motion.

(2) Where a hearing has been adjourned and it is not practicable, or would cause undue delay, for it to be resumed before a tribunal consisting of the same members, the appeal or application shall be heard by a tribunal none of the members of which was a member of the original tribunal and the proceedings shall be by way of a complete re-hearing of the case.

\subsection[13. Decisions]{Decisions}

13.—(1) A decision of the tribunal may be taken by a majority.

%(2) The chairman shall---
%\begin{enumerate}\item[]
%($a$) record in writing the decision of the tribunal;
%
%($b$) include in the record of every decision a statement of the reasons for it, the findings of the tribunal on questions of fact material to the decision and the terms of any direction given under section 20(4) of the Act; and
%
%($c$) if a decision is not unanimous, record a statement that one of the members dissented and the reasons given by him for so dissenting.
%\end{enumerate}

% Reg 13(2) substituted (21.10.96) by SI 1996/2450 reg 20(2)
(2) Every decision of a tribunal shall be recorded in summary by the chairman in such written form of decision notice as shall have been approved by the President, and such decision notice shall be signed by the chairman.

%(3) As soon as may be practicable after the decision of the tribunal a copy of the record of the decision made in accordance with this regulation shall be sent to every party to the proceedings who shall also be informed of the conditions governing appeals to a Commissioner.
%
%% Reg 13(3A) inserted (28.2.96) by SI 1996/182 reg 3
%(3A) A record of the proceedings at the hearing shall be made by the chairman in such medium as he may direct and preserved by the clerk to the tribunal for 18 months, and a copy of such record (which may take the form of a transcript or a tape) shall be supplied to the parties if requested by any of them within that period.

% Reg 13(3)--(3E) substituted for reg 13(3), (3A) (21.10.96) by SI 1996/2450 reg 20(3)
(3) As soon as may be practicable after a case has been decided by a tribunal, a copy of the decision notice made in accordance with paragraph (2) shall be sent or given to every party to the proceedings who shall also be informed of—
\begin{enumerate}\item[]
($a$) his right under paragraph (3C); and

($b$) the conditions governing appeals to a Commissioner.
\end{enumerate}

(3A) A statement of the reasons for the tribunal’s decision, of its findings on questions of fact material thereto and of the terms of any direction under section 20(4) of the Act may be given—
\begin{enumerate}\item[]
($a$) orally at the hearing; or

($b$) in writing at such later date as the chairman may determine.
\end{enumerate}

(3B) Where the statement referred to in paragraph (3A) is given orally, it shall be recorded in such medium as the chairman may determine.

(3C) A copy of the statement referred to in paragraph (3A) shall be supplied to the parties to the proceedings if requested by any of them within 21 days after the decision notice has been sent or given and if the statement is one to which sub-paragraph ($a$) of that paragraph applies, that copy shall be supplied in such medium as the chairman may direct.

(3D) If a decision is not unanimous, the statement referred to in paragraph (3A) shall record that one of the members dissented and the reasons given by him for dissenting.

(3E) A record of the proceedings at the hearing may be made by the chairman in such medium as he may direct and preserved by the clerk to the tribunal for 18 months, and a copy of such record shall be supplied to the parties if requested by any of them within that period.

(4) If a child support officer to whom a case is referred by the Secretary of State under section 20(3) of the Act (procedure following a successful appeal) is uncertain, having regard to the terms of the decision and of any directions contained in it, how he should deal with the case, he may apply to the tribunal or another tribunal for directions or further directions, and the tribunal may give such directions or further directions as it thinks fit.

(5) Upon receiving an application from a child support officer under paragraph (4) the clerk to the tribunal shall send a copy of it to all the other parties to the case, and the tribunal shall not give any directions or further directions on the application until those other parties have had a reasonable opportunity of making representations on it.

\amendment{
%Reg. 13(3A) inserted (28.2.96) by the Social Security (Adjudication) and Child Support Amendment Regulations 1996 reg. 3.

Reg. 13(3)--(3E) substituted for reg. 13(3), (3A) and reg. 13(2) substituted (21.10.96) by the Social Security (Adjudication) and Child Support Amendment (No.\ 2) Regulations 1996 reg. 20.
}

\subsection[14. Corrections]{Corrections}

14.—(1) Subject to regulation 16 (provisions common to regulations 14 and 15) accidental errors (whether of omission or commission) in any decision or record of any decision may at any time be corrected by the tribunal who gave the decision or by another tribunal.

(2) A correction made to a decision or to the record of a decision shall be deemed to be part of the decision or of the record thereof and written notice of it shall be given as soon as practicable to every party to the proceedings.

\subsection[15. Setting aside]{Setting aside}

15.—(1) Subject to regulation 16 (provisions common to regulations 14 and 15) on an application made by a party to the proceedings a decision may be set aside by the tribunal who gave the decision or by another tribunal in a case where it appears just to do so on the grounds that---
\begin{enumerate}\item[]
($a$) a document relating to the proceedings in which the decision was given was not sent to, or was not received at an appropriate time by, a party to the proceedings or the party’s representative or was not received at an appropriate time by the tribunal who gave the decision;

($b$) a party to the proceedings in which the decision was given or the party’s representative was not present at the hearing notice of which had been given under regulation 11(2); or

($c$) there has been some other procedural irregularity or mishap.
\end{enumerate}

% Reg 15(1A) inserted (21.10.96) by SI 1996/2450 reg 21
(1A) In determining whether it is just to set aside a decision on the ground set out in paragraph (1)($b$) the tribunal shall determine whether the party making the application gave notice that he wished an oral hearing to be held, and if the party did not give such notice the tribunal shall not set the decision aside unless it is satisfied that there has been some procedural irregularity or mishap.

(2) An application under this regulation shall be made in accordance with regulation 3.

(3) Where an application to set aside a decision is made under paragraph (1) every party to the proceedings shall be sent a copy of the application and shall be afforded a reasonable opportunity of making representations on it before the application is decided.

(4) Notice in writing of a decision on an application to set aside a decision shall be given to every party to the proceedings as soon as may be practicable and the notice shall contain a statement giving the reasons for the decision.

(5) For the purpose of deciding an application to set aside a decision under these Regulations there shall be disregarded regulation 2 and any provision in any enactment or instrument to the effect that any notice or other document required or authorised to be given or sent to any person shall be deemed to have been given or sent if it was sent by post to the person’s last known address.

\amendment{
Reg. 15(1A) inserted (21.10.96) by the Social Security (Adjudication) and Child Support Amendment (No.\ 2) Regulations 1996 reg. 21.
}

\subsection[16. Provisions common to regulations 14 and 15]{Provisions common to regulations 14 and 15}

16.—(1) In calculating time under regulation 2(1) of the Child Support Commissioners (Procedure) Regulations 1992\footnote{\frenchspacing S.I. 1992/2640.} (applications for leave to appeal to a Commissioner) there shall be disregarded any day falling before the day on which notice was given of a correction of a decision or record thereof pursuant to regulation 14 or on which notice is given of a decision that a prior decision shall not be set aside following an application made under regulation 15, as the case may be.

(2) Notwithstanding anything contained in these Regulations, there shall be no appeal against a correction made under regulation 14, or a refusal to make such a correction, or against a decision given under regulation 15.

(3) Nothing in these Regulations shall be construed as derogating from any power to correct errors or set aside decisions which is exercisable apart from these Regulations.

%\subsection[17. Confidentiality]{Confidentiality}
%
%17.—(1) No information such as is mentioned in paragraph (2), and which has been provided for the purposes of any proceedings to which these Regulations apply, shall be disclosed except with the written consent of the person to whom the information relates.
%
%(2) The information referred to in paragraph (1) is---
%\begin{enumerate}\item[]
%($a$) any address, other than the address of the Central Office and the place where the oral hearing is to be held; and
%
%($b$) any other information the use of which could reasonably be expected to lead to a person being located.
%\end{enumerate}

% Reg 17 substituted (7.10.96) by SI 1996/1945 reg 2
\subsection[17. Confidentiality]{Confidentiality}

17.—(1) No information such as is mentioned in paragraph (2), and which has been provided for the purposes of any proceedings to which these Regulations apply, shall be disclosed if, before the expiry of the period of 21 days specified in paragraph (3), written notification has been received from the person to whom the information relates that he does not consent to such disclosure.

(2) The information referred to in paragraph (1) is—
\begin{enumerate}\item[]
($a$) the address of the person referred to in that paragraph; and

($b$) any other information the use of which could reasonably be expected to lead to that person being located.
\end{enumerate}

(3) Except where the appeal is made under section 46(7) of the Act or is one to which regulation 3(1)($b$) applies, the clerk to the tribunal shall notify the person to whom the information referred to in paragraphs (1) and (2) relates of the provisions of those paragraphs and that disclosure of that information may be made, unless the written notification specified in paragraph (1) is received before the expiry of the period of 21 days, beginning with the date the notification by the clerk to the tribunal was given or sent to that person.

\amendment{
Reg. 17 substituted (7.10.96) by the Child Support (Miscellaneous Amendments) Regulations 1996 reg. 2.
}

\bigskip

Signed by authority of the Secretary of State for Social Security.

{\raggedleft
\emph{Alistair Burt}\\*Parliamentary Under-Secretary of State,\\*Department of Social Security

}

26th October 1992

\part{Explanatory Note}

\renewcommand\parthead{--- Explanatory Note}

\subsection*{(This note is not part of the Regulations)}

 These Regulations provide for the procedure to be followed by Child Support Appeal Tribunals established under section 21 of the Child Support Act 1991.

  The principal matters dealt with include the service of notices or other documents (regulation 2); the time and manner of making an appeal or application (regulation 3); the powers of a chairman of a tribunal to give directions for the conduct or disposal of proceedings and to strike out proceedings (regulations 4, 5 and 6); the summoning of witnesses to hearings and the conduct of hearings (regulations 10 and 11) and the giving of decisions and the correction and setting aside of decisions (regulations 13 to 16).


\end{document}