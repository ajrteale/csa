\documentclass[12pt,a4paper]{article}

\newcommand\regstitle{The Social Security (Adjudication) and Child Support Amendment Regulations 1996}

\newcommand\regsnumber{1996/182}

%\opt{newrules}{
\title{\regstitle}
%}

%\opt{2012rules}{
%\title{Child Maintenance and Other Payments Act 2008\\(2012 scheme version)}
%}

\author{S.I. 1996 No. 182}

\date{Made 31st January 1996\\Laid before Parliament 7th February 1996\\Coming into force 28th February 1996}

%\opt{oldrules}{\newcommand\versionyear{1993}}
%\opt{newrules}{\newcommand\versionyear{2003}}
%\opt{2012rules}{\newcommand\versionyear{2012}}

\usepackage{csa-regs}

\setlength\headheight{27.57402pt}

\begin{document}

\maketitle

\noindent
The Secretary of State for Social Security, in exercise of the powers conferred by section 21(2) of the Child Support Act 1991\footnote{\frenchspacing 1991 c. 48.} and sections 59(1) and 189 of, and paragraphs 2 and 5 of Schedule 3 to, the Social Security Administration Act 1992\footnote{\frenchspacing 1992 c. 5.} and of all other powers enabling him in that behalf, after consultation with the Council on Tribunals in accordance with section 8 of the Tribunals and Inquiries Act 1992\footnote{\frenchspacing 1992 c. 53.}, hereby makes the following Regulations:

{\sloppy

\tableofcontents

}

\setcounter{secnumdepth}{-2}

\subsection[1. Citation and commencement]{Citation and commencement}

1.  These Regulations may be cited as the Social Security (Adjudication) and Child Support Amendment Regulations 1996 and shall come into force on 28th February 1996.

\amendment{
Reg. 2 revoked (1.6.99) by the Social Security and Child Support (Decisions and Appeals) Regulations 1999 reg. 59, Sch. 4.
}

% Reg 2 revoked by SI 1999/991 reg 59 Sch 4
%
%\subsection[2. Amendment of the Social Security (Adjudication) Regulations 1995]{Amendment of the Social Security (Adjudication) Regulations 1995}
%
%2.—(1) The Social Security (Adjudication) Regulations 1995\footnote{\frenchspacing S.I. 1995/1801.} are amended in accordance with the following provisions of this regulation.
%
%(2) In regulation 3—
%\begin{enumerate}\item[]
%($a$) in paragraph (3), for the words “for special reasons” there are substituted the words “in the case of an application or reference, for special reasons, and in the case of an appeal, provided the conditions set out in paragraphs (3A) to (3E) are satisfied”;
%
%($b$) after paragraph (3) there are inserted the following paragraphs—
%\begin{quotation}
%“(3A) Where the time specified for the making of an appeal has already expired, an application for an extension of time for making an appeal shall not be granted unless the applicant has satisfied the person considering the application that—
%\begin{enumerate}\item[]
%\begin{sloppypar}
%\textls[25]{($a$) if the application is granted there are reasonable} prospects that such an appeal will be successful; and
%\end{sloppypar}
%
%($b$) it is in the interests of justice that the application be granted.
%\end{enumerate}
%
%(3B) For the purposes of paragraph (3A) it shall not be considered to be in the interests of justice to grant an application unless the person considering the application is satisfied that—
%\begin{enumerate}\item[]
%($a$) special reasons exist, which are wholly exceptional and which relate to the history or facts of the case; and
%
%($b$) such special reasons have existed throughout the period beginning with the day following the expiration of the time specified by Schedule 2 for the making of an appeal and ending with the day on which the application for an extension of time is made; and
%
%($c$) such special reasons manifestly constitute a reasonable excuse of compelling weight for the applicant’s failure to make an appeal within the time specified.
%\end{enumerate}
%
%(3C) In determining whether there are special reasons for granting an application for an extension of time for making an appeal under paragraph (3) the person considering the application shall have regard to the principle that the greater the amount of time that has elapsed between the expiration of the time specified for the making of the appeal and the making of the application for an extension of time, the more cogent should be the special reasons on which the application is based.
%
%(3D) In determining whether facts constitute special reasons for granting an application for an extension of time for making an appeal under paragraph (3) no account shall be taken of the following—
%\begin{enumerate}\item[]
%($a$) that the applicant or anyone acting for him or advising him was unaware of or misunderstood the law applicable to his case (including ignorance or misunderstanding of any time limits imposed by Schedule 2);
%
%($b$) that a Commissioner or a court has taken a different view of the law from that previously understood and applied.
%\end{enumerate}
%
%(3E) Notwithstanding paragraph (3), no appeal may in any event be brought later than 6 years after the beginning of the period specified in column (3) of Schedule 2.”;
%\end{quotation}
%
%($c$) for paragraph (5) there is substituted the following paragraph—
%\begin{quotation}
%“(5) Any application, appeal or reference under these Regulations shall contain particulars of the grounds on which it is made or given and—
%\begin{enumerate}\item[]
%($a$) in the case of an appeal, shall include sufficient particulars of the decision under appeal to enable that decision to be identified; and
%
%($b$) in the case of an application under paragraph (3) for an extension of time in which to appeal, shall state the grounds on which it is proposed to bring the appeal.”;
%\end{enumerate}
%\end{quotation}
%
%(d) after paragraph (7) there are added the following paragraphs—
%\begin{quotation}
%“(8) In the case of an application under paragraph (3) for an extension of time for making an appeal, the person who determines that application shall record his decision in writing together with a statement of the reasons for the decision.
%
%(9) As soon as practicable after the decision has been made it shall be communicated to the applicant and to every other party to the proceedings and if within 3 months of such communication being sent the applicant or any other party to the proceedings so requests in writing, a copy of the record referred to in paragraph (8) shall be supplied to him.”.
%\end{quotation}
%\end{enumerate}
%
%(3) In regulation 23, after paragraph (3) there is added the following paragraph—
%\begin{quotation}
%“(4) A record of the proceedings at the hearing shall be made by the chairman in such medium as he may direct and preserved by the clerk to the tribunal for 18 months, and a copy of such record (which may take the form of a transcript or tape) shall be supplied to the parties if requested by any of them within that period.”.
%\end{quotation}
%
%(4) In regulation 29, after paragraph (6) there is added the following paragraph—
%\begin{quotation}
%“(7) A record of the proceedings at the hearing shall be made by the chairman in such medium as he may direct and preserved by the clerk to the tribunal for 18 months, and a copy of such record (which may take the form of a transcript or a tape) shall be supplied to the parties if requested by any of them within that period.”.
%\end{quotation}
%
%(5) In regulation 38, after paragraph (5) there is added the following paragraph—
%\begin{quotation}
%“(6) A record of the proceedings at the hearing shall be made by the chairman in such medium as he may direct and preserved by the clerk to the tribunal for 18 months, and a copy of such record (which may take the form of a transcript or a tape) shall be supplied to the parties if requested by any of them within that period.”.
%\end{quotation}

\subsection[3. Amendment of the Child Support Appeal Tribunals (Procedure) Regulations 1992]{Amendment of the Child Support Appeal Tribunals (Procedure) Regulations 1992}

3.  In regulation 13 of the Child Support Appeal Tribunals (Procedure) Regulations 1992\footnote{\frenchspacing S.I. 1992/2641.}, after paragraph (3) there shall be added the following paragraph—
\begin{quotation}
“(3A) A record of the proceedings at the hearing shall be made by the chairman in such medium as he may direct and preserved by the clerk to the tribunal for 18 months, and a copy of such record (which may take the form of a transcript or a tape) shall be supplied to the parties if requested by any of them within that period.”.
\end{quotation}

\bigskip

Signed by authority of the Secretary of State for Social Security.

{\raggedleft
\emph{Roger Evans}\\*Parliamentary Under-Secretary of State,\\*Department of Social Security

}

31st January 1996

\small

\part{Explanatory Note}

\renewcommand\parthead{--- Explanatory Note}

\subsection*{(This note is not part of the Regulations)}

These Regulations amend the Social Security (Adjudication) Regulations 1995 and the Child Support Appeal Tribunals (Procedure) Regulations 1992.

  Regulation 3 of the Social Security (Adjudication) Regulations 1995 provides for the manner of making applications, appeals and references and the time limits for so doing. Regulation 3(3) provides that the time limit may be extended for special reasons. These Regulations amend regulation 3 by inserting new paragraphs into that regulation which make further provision about an application for an extension of time for making an appeal.

  New paragraph (3B) provides that special reasons for granting an extension must exist throughout the period between the end of the specified time limit and the time when the application for an extension is made. New paragraphs (3C) and (3D) set out criteria to be taken into account when deciding whether special reasons exist and specify certain factors that are not to be taken into account. New paragraph (3E) imposes an absolute time limit of 6 years for making an appeal. An amendment is made to paragraph (5) so as to require that an application for an extension of time shall state the grounds on which it is made, and a new paragraph (8) is added which requires the person deciding the application to record his decision in writing together with the reasons for it. New paragraph (9) provides that as soon as practicable a copy of the decision shall be sent to every party to the proceedings and that a copy of the record should be supplied to any party on a request being made within 3 months.

  A new paragraph is inserted into regulations 23, 29 and 38 of the Social Security (Adjudication) Regulations and also into regulation 13 of the Child Support Appeal Tribunals (Procedure) Regulations to provide that the chairman of, respectively, a social security appeal tribunal, a disability appeal tribunal, a medical appeal tribunal and a child support appeal tribunal shall make a record of the proceedings at the hearing of a case and that the clerk to the tribunal shall preserve that record for 18 months during which time copies are to be available to the parties to the hearing on request. This removes the practice of the automatic issue of the record of proceedings to every appellant.

  These Regulations do not impose any costs on business.


\end{document}