\documentclass[12pt,a4paper]{article}

\newcommand\regstitle{The State Pension Credit (Miscellaneous Amendments) Regulations 2004}

\newcommand\regsnumber{2004/647}

%\opt{newrules}{
\title{\regstitle}
%}

%\opt{2012rules}{
%\title{Child Maintenance and~Other Payments Act 2008\\(2012 scheme version)}
%}

\author{S.I.\ 2004 No.\ 647}

\date{Made
8th March 2004\\
Laid before Parliament
15th March 2004\\
Coming into~force
5th April 2004
}

%\opt{oldrules}{\newcommand\versionyear{1993}}
%\opt{newrules}{\newcommand\versionyear{2003}}
%\opt{2012rules}{\newcommand\versionyear{2012}}

\usepackage{csa-regs}

\setlength\headheight{42.11603pt}

%\hbadness=10000

\begin{document}

\maketitle

\enlargethispage{\baselineskip}

\noindent
The Secretary of State for Work and~Pensions, in exercise of the powers conferred upon him by section~175(1), (3) and~(4) of the Social Security Contributions and~Benefits Act 1992\footnote{1992 c.~4; section~175(1) and~(4) were amended by the Social Security Contributions (Transfer of Functions, etc.) Act 1999 (c.~11), Schedule 2, paragraph 29; section~175(1), (3) and~(4) are applied to provisions of the State Pension Credit Act (c.~16) by section~19(1) of that Act.}, sections 10(6), 79(1), (4) and~(6) and~84 of the Social Security Act 1998\footnote{1998 c.~14; section~84 is cited because of the meaning it gives to “prescribe”.} and~sections 7(4), 15(3), (4), (6)($a$)  and~($b$), and~17(1) of the State Pension Credit Act 2002\footnote{2002 c.~16; section~17(1) is cited because of the meaning it gives to “prescribed”.}, and~all other powers enabling him in that behalf, and~after agreement by the Social Security Advisory Committee that proposals to make these Regulations should not be referred to it\footnote{\emph{See} the Social Security Administration Act 1992 (c.~5) section~173(1)($b$).}, hereby makes the following Regulations: 

{\sloppy

\tableofcontents

}

\bigskip

\setcounter{secnumdepth}{-2}

\subsection[1. Citation and~commencement]{Citation and~commencement}

1.  These Regulations may be cited as the State Pension Credit (Miscellaneous Amendments) Regulations 2004 and~shall come into force on 5th April 2004.

\subsection[2. Amendment of the Social Security and~Child Support (Decisions and~Appeals) Regulations 1999]{Amendment of the Social Security and~Child Support (Decisions and~Appeals) Regulations 1999}

2.  In the Social Security and~Child Support (Decisions and~Appeals) Regulations 1999\footnote{S.I.~1999/991.}, in regulation 7 (date from which a decision superseded under section~10 takes effect)—
\begin{enumerate}\item[]
($a$) for paragraphs (17B) and~(17C)\footnote{Paragraphs (17B) and (17C) were inserted by S.I.~2002/3197.} substitute—
\begin{quotation}
“(17B) Paragraph (17C) applies where—
\begin{enumerate}\item[]
($a$) a claimant is awarded state pension credit;

($b$) the claimant or his partner is aged 65 or over;

($c$) his appropriate minimum guarantee (as defined by the State Pension Credit Act) includes housing costs determined in accordance with Schedule II to the State Pension Credit Regulations; and

($d$) after the date from which sub-paragraph ($c$)  applies—
\begin{enumerate}\item[]
(i) a non-dependant (as defined in that Schedule) begins to reside with the claimant; or

(ii) a non-dependant’s income increases and~this affects the applicable amount of the claimant’s housing costs.
\end{enumerate}
\end{enumerate}

(17C) In the circumstances specified in paragraph (17B) a decision made under section~10 shall take effect—
\begin{enumerate}\item[]
($a$) where there is more than one change of the kind specified in paragraph (17B)($d$)  in respect of the same non-dependant within the same 26 week period, 26 weeks after the date on which the first such change occurred; and

($b$) in any other circumstances, 26 weeks after the date on which a change specified in paragraph (17B)($d$)  occurred.”; and
\end{enumerate}
\end{quotation}

($b$) in paragraph (23)\footnote{Paragraph (23) was substituted by S.I.~2002/3197.} omit “, (17B)”.
\end{enumerate}

\subsection[3. Amendment of the State Pension Credit Regulations 2002]{Amendment of the State Pension Credit Regulations 2002}

3.---(1)  The State Pension Credit Regulations 2002\footnote{S.I.~2002/1792.} shall be amended in accordance with the following provisions of this regulation.

(2) For regulation 10(5) (which provides how the date specified in paragraph (4) is to be determined), substitute—
\begin{quotation}
“(5) The day referred to in this paragraph is—
\begin{enumerate}\item[]
($a$) in a case to which paragraph (5A) applies—
\begin{enumerate}\item[]
(i) where the first increased payment date is the day on which the benefit week begins, that day;

(ii) where head (i)  does not apply, the first day of the next benefit week which begins after that increased payment date;
\end{enumerate}

($b$) in a case to which paragraph (5A) does not apply—
\begin{enumerate}\item[]
(i) where the second increased payment date is the day on which the benefit week begins, that day;

(ii) where head (i)  does not apply, the first day of the next benefit week following that increased payment date.
\end{enumerate}
\end{enumerate}

(5A) This paragraph applies where the period which—
\begin{enumerate}\item[]
($a$) begins on the date from which the increase in the assessed amount is to accrue; and

($b$) ends on the first increased payment date,
\end{enumerate}
is a period of the same length as the period in respect of which the last payment of the pre-increase assessed amount was made.

(5B) In paragraphs (5) and~(5A)—
\begin{enumerate}\item[]
“increased payment date” means a date on which the increase in the assessed amount referred to in paragraph (4) is paid as part of a periodic payment under the claimant’s retirement pension scheme or annuity contract; and

“pre-increase assessed amount” means the assessed amount prior to that increase.”.
\end{enumerate}
\end{quotation}

(3) For regulation 10(6)($a$)  (which provides that an assessed amount is deemed to increase by reference to the terms of an Order made under section~150 of the Social Security Administration Act 1992), substitute—
\begin{quotation}
“($a$) on the day in April each year on which increases under section~150(1)($c$)  of the Administration Act come into force if that is the first day of a benefit week but if it is not from the next following such day; and”.
\end{quotation}

(4) After regulation 17\footnote{Regulations 17A and 17B were inserted by S.I.~2002/3019.} (calculation of weekly income) insert—
\begin{quotation}
\subsection*{“Treatment of final payments of income}

17ZA.---(1)  Save where regulation 13B\footnote{Regulation~13B was inserted by S.I.~2002/3019.} applies, this regulation applies where—
\begin{enumerate}\item[]
($a$) a claimant has been receiving a regular payment of income;

($b$) that payment is coming to an end or has ended; and

($c$) the claimant receives a payment of income whether as the last of the regular payments or following the last of them (“the final payment”).
\end{enumerate}

(2) For the purposes of regulation 17(1)—
\begin{enumerate}\item[]
($a$) where the amount of the final payment is less than or equal to the amount of the preceding, or the last, regular payment, the whole amount shall be treated as being paid in respect of a period of the same length as that in respect of which that regular payment was made;

($b$) where the amount of the final payment is greater than the amount of that regular payment—
\begin{enumerate}\item[]
(i) to the extent that it comprises (whether exactly or with an excess remaining) one or more multiples of that amount, each such multiple shall be treated as being paid in respect of a period of the same length as that in respect of which that regular payment was made; and

(ii) any excess shall be treated as paid in respect of a further period of the same length as that in respect of which that regular payment was made.
\end{enumerate}
\end{enumerate}

(3)  A final payment referred to in paragraph (2)($a$)  shall, where not in fact paid on the date on which a regular payment would have been paid had it continued in payment, be treated as paid on that date.

(4) Each multiple and~any excess referred to in paragraph (2)($b$)  shall be treated as paid on the dates on which a corresponding number of regular payments would have been made had they continued in payment.

(5) For the purposes of this regulation, a “regular payment” means a payment of income made in respect of a period—
\begin{enumerate}\item[]
($a$) referred to in regulation 17(1)($a$)  or ($b$)  on a regular date; or

($b$) which is subject to the provisions of regulation 17(2).”.
\end{enumerate}
\end{quotation}

(5) For regulation 21(2) (which deals with the effect of disposing of capital) substitute—
\begin{quotation}
“(2) A person who disposes of a capital resource for the purpose of—
\begin{enumerate}\item[]
($a$) reducing or paying a debt owed by the claimant; or

($b$) purchasing goods or services if the expenditure was reasonable in the circumstances of the claimant’s case,
\end{enumerate}
shall be regarded as not depriving himself of it.”.
\end{quotation}

\subsection[4. Saving]{Saving}

4.  Regulation~2 shall not apply where—
\begin{enumerate}\item[]
($a$) a person’s claim for state pension credit is backdated in accordance with regulation 38 of the State Pension Credit (Consequential, Transitional and~Miscellaneous Provisions) Regulations 2002\footnote{S.I.~2002/3019.} (claims for state pension credit); and

($b$) a change of a kind specified in the paragraph (17C) substituted by regulation 2 or, as the case may be, the first such change occurred before 5th April 2004.
\end{enumerate}

\bigskip

Signed 
by authority of the 
Secretary of State for~Work and~Pensions.
%I concur
%By authority of the Lord Chancellor

{\raggedleft
\emph{Malcolm Wicks}\\*
%Secretary
Minister
%Parliamentary Under-Secretary 
of State,\\*Department 
for~Work and~Pensions

}

8th March 2004

\small

\part{Explanatory Note}

\renewcommand\parthead{— Explanatory Note}

\subsection*{(This note is not part of the Regulations)}

These Regulations further amend the Social Security and~Child Support (Decisions and~Appeals) Regulations 1999 and~the State Pension Credit Regulations 2002 (“the 2002 Regulations”) in respect of state pension credit.

Regulation~2 provides that, where a state pension credit claimant aged 65 or over is awarded housing costs, a change in a non-dependant’s circumstances which reduces the award shall take effect 26 weeks after the change occurred. Regulation~4 makes a consequential saving.

Regulation~3 amends the 2002 Regulations. It amends the provisions relating to the date which occupational pensions increase (regulation 10(5) of the 2002 Regulations) and~the provisions relating to the treatment of deemed increases to the assessed amount made under the annual Uprating Order (regulation 10(6) of the 2002 Regulations). Regulation~3 also inserts a new provision in respect of the treatment of final payments of income (regulation 17ZA) and~amends regulation 21 of the 2002 Regulations to provide that a person shall not be treated as depriving himself of a capital resource if he disposes of it for the purpose of paying or reducing a debt or, in certain cases, purchasing goods or services.

A regulatory impact assessment has not been produced for this instrument as it has no impact on the costs of business. 

\end{document}