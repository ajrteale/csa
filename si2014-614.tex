\documentclass[12pt,a4paper]{article}

\newcommand\regstitle{The Child Support (Ending Liability in Existing Cases and Transition to New Calculation Rules) Regulations 2014}

\newcommand\regsnumber{2014/614}

\title{\regstitle}

\author{S.I.\ 2014 No.\ 614}

\date{Made
12th March 2014\\
%%%Laid before Parliament
%%%27th June 2013\\
Coming into force
in accordance with regulation 1(1)
}

%\opt{oldrules}{\newcommand\versionyear{1993}}
%\opt{newrules}{\newcommand\versionyear{2003}}
%\opt{2012rules}{\newcommand\versionyear{2012}}

\usepackage{csa-regs}

\setlength\headheight{27.61603pt}

%\hbadness=10000

\begin{document}

\maketitle

\enlargethispage{\baselineskip}

\noindent
The Secretary of State for Work and Pensions makes the following Regulations in exercise of the powers conferred by sections 6(2)($g$)  and 55(3) and (4) of, and paragraphs 2, 3, 5, 6 and 7 of Schedule 5 to, the Child Maintenance and Other Payments Act 2008\footnote{2008 c.~6. Paragraph 7 is cited for the meaning of “interested parties” and “prescribed”. Schedule 5 to the Child Maintenance and Other Payments Act 2008 (“the 2008 Act”) was amended by the Public Bodies (Child Maintenance and Enforcement Commission: Abolition and Transfer of Functions) Order 2012 (S.I.~2012/2007) (“the 2012 Order”).}.

A draft of this instrument was laid before and approved by a resolution of each House of Parliament in accordance with section 55(5) of that Act\footnote{Section 55(5) of the 2008 Act was amended by the 2012 Order.}. 

{\sloppy

\tableofcontents

}

\bigskip

\setcounter{secnumdepth}{-2}

\subsection[1. Citation, commencement and interpretation]{Citation, commencement and interpretation}

1.—(1) These Regulations may be cited as the Child Support (Ending Liability in Existing Cases and Transition to New Calculation Rules) Regulations 2014 and come into force on the day on which section 19 (transfer of cases to new rules) of the Child Maintenance and Other Payments Act 2008 comes into force for all purposes.

(2) In these Regulations—
\begin{enumerate}\item[]
“the 1991 Act” means the Child Support Act 1991\footnote{1991 c. 48.};

“the 1992 Regulations” means the Child Support (Maintenance Assessments and Special Cases) Regulations 1992\footnote{S.I.~1992/1815. The 1992 Regulations were revoked with savings by S.I.~2001/155 and 2012/2785.};

“the 2008 Act” means the Child Maintenance and Other Payments Act 2008;

“absent parent” has the meaning given in section 3(2) (meaning of certain terms) of the 1991 Act\footnote{The substitution of the term “absent parent” with “non-resident parent” by section 26 of, and paragraph 11(1) and (2) of Schedule 3 to, the Child Support, Pensions and Social Security Act 2000 (“the 2000 Act”) was partially commenced for the types of cases specified in article 3 of the Child Support, Pensions and Social Security Act 2000 (Commencement No.~12) Order 2003 (S.I.~2003/192) (“the 2003 Order”).};

“liability end date” has the meaning given in regulation 6;

“non-resident parent” has the meaning given in section 3(2) of the 1991 Act\footnote{See footnote \ddag{} above.};

“person with care” has the meaning given in section 3(3) of the 1991 Act;

“partner” has the meaning given in paragraph 10C(4) (references to various terms) of Schedule 1 (maintenance calculations) to the 1991 Act\footnote{The substitution of Part~I of Schedule 1 to the 1991 Act by section 1(3) of, and Schedule 1 to, the 2000 Act was partially commenced for the types of cases specified in article 3 of the 2003 Order.};

“prescribed benefit” means a benefit prescribed by regulations made under paragraph 4(1)($c$)  (flat rate) of Schedule 1 to the 1991 Act;

“qualifying child” has the meaning given in section 3(1) of the 1991 Act\footnote{The amendment of the definition of “qualifying child” by section 26 of, and paragraph 11(1) and (2) of Schedule 3 to, the 2000 Act was partially commenced for the types of cases specified in article 3 of the 2003 Order.};

“the scheme” means the scheme prepared by the Secretary of State under regulation 3(1);

“transition period” has the meaning given in regulation 3(2)
and (3)%  % Words inserted by SI 2014/1386 reg 8(2)
.
\end{enumerate}

(3) For the purposes of these Regulations an existing case is related to an application made under section 4(1) (child support maintenance) or 7(1) (right of a child in Scotland to apply for assessment) of the 1991 Act if—
\begin{enumerate}\item[]
($a$) the non-resident parent in relation to that application is also the non-resident parent or absent parent in relation to the existing case and the person with care in relation to that application is not the person with care in relation to the existing case; or

($b$) the non-resident parent in relation to that application is a partner of a non-resident parent or absent parent and either or both are in receipt of a prescribed benefit.
\end{enumerate}

\amendment{
Word inserted in definition of ``transition period'' in reg. 1(2) (30.6.14) by the Child support (Consequential and Miscellaneous Amendments) Regulations 2014 reg. 8(2).
}

\subsection[2. Meaning of “interested parties”]{Meaning of “interested parties”}

2.  For the purposes of Schedule 5 to the 2008 Act “interested parties” means, in relation to an existing case—
\begin{enumerate}\item[]
($a$) the absent parent or the non-resident parent;

($b$) the person with care; and

($c$) in the case of an application made by a qualifying child under section~7(1) of the 1991 Act\footnote{The amendment of section 7(1) of the 1991 Act by sections 1(2) and 26 of, and paragraph 11(1), (2) and (4)($a$)  of Schedule 3 to the 2000 Act was partially commenced for the types of cases specified in article 3 of the 2003 Order. Section 7(1) was also amended by section 58 of, and Schedule 8 to, the 2008 Act and by the 2012 Order.}, or a maintenance assessment or a maintenance calculation made in response to an application under that section, the child in question.
\end{enumerate}

\subsection[3. Scheme in relation to ending liability in existing cases]{Scheme in relation to ending liability in existing cases}

3.—(1) The power under paragraph 1(1) of Schedule 5 to the 2008 Act (power to require the parties to an existing case to choose whether or not to stay in the statutory scheme) must be exercised in accordance with a scheme prepared by the Secretary of State.

(2) The transition period during which the power in paragraph 1(1) of Schedule 5 to the 2008 Act may be exercised shall begin on the day on which these Regulations come into force.

(3) The scheme prepared by the Secretary of State shall state the date on which the transition period ends.

(4) The scheme may be revised from time to time by the Secretary of State.

\subsection[4. Staging under the scheme]{Staging under the scheme}

4.—(1) The scheme prepared by the Secretary of State must make provision for the exercise of the power in paragraph 1(1) of Schedule 5 to the 2008 Act in stages, applying the principles in paragraphs (2) to (4).

(2) Where an application is made under section 4(1) or 7(1) of the 1991 Act during the transition period the power is to be exercised in relation to any existing case that is related to that application.

(3) The power is not to be exercised in relation to an existing case where the youngest, or only, qualifying child will have reached the age of 20 before the end of the transition period unless that case is related to an application referred to under paragraph (2).

(4) Cases other than those to which either paragraphs (2) or (3) apply, are to be selected in tranches and, in making that selection priority may be given to cases where—
\begin{enumerate}\item[]
($a$) the nil rate is payable under regulation 26 (cases where child support maintenance is not to be payable) of the 1992 Regulations\footnote{Regulation 26 was amended by S.I.~1995/1045 and 1998/58 and was partially revoked by S.I.~2001/155 and revoked with savings by 2012/2785.};

($b$) regulation 28 (amount payable where absent parent is in receipt of income support or other prescribed benefit) of the 1992 Regulations\footnote{Regulation 28 was amended by S.I.~1993/913, 1993/925, 1996/1345, 1998/58 and was partially revoked by S.I.~2001/155 and revoked with savings by 2012/2785.} applies;

($c$) the nil rate is payable under regulation 5 (nil rate) of the Child Support (Maintenance Calculations and Special Cases) Regulations 2000\footnote{S.I.~2001/155. Regulation 5 was amended by S.I.~2003/2779, 2004/2415, 2008/1554, 2009/396, 2012/2785 and 2013/630.}.
\end{enumerate}

\subsection[5. Exercise of the choice as to whether or not to stay in the statutory scheme]{Exercise of the choice as to whether or not to stay in the statutory scheme}

5.—(1) The right to make a choice required under paragraph 1(1) of Schedule~5 to the 2008 Act must be exercised in accordance with this regulation.

(2) An interested party must make a choice following receipt of written notice given by the Secretary of State.

(3) The notice to the interested parties must specify—
\begin{enumerate}\item[]
($a$) the liability end date; and

($b$) the manner in which they are to exercise a choice to remain in the statutory scheme.
\end{enumerate}

(4) A party is taken to have received written notice in accordance with paragraph (2) on the second day after the notice is sent by post to that party’s last known or notified address.

(5) A choice to remain in the statutory scheme must be made—
\begin{enumerate}\item[]
($a$) by way of an application to the Secretary of State for a maintenance calculation; and

($b$) before the liability end date.
\end{enumerate}

(6) The Secretary of State may require information to be provided in an application made under paragraph (5) and may do so despite such information having been notified for the purposes of the existing case.

(7) The Secretary of State may withdraw a notice given under paragraph~(2) where—
\begin{enumerate}\item[]
($a$) in the Secretary of State’s opinion the notice was given in error; and

($b$) the date of withdrawal is earlier than 30 days from the liability end date.
\end{enumerate}

(8) Where a decision is made under paragraph (7) the Secretary of State shall reimburse any application fee paid under regulations made under section~6(1) of the 2008 Act.

(9) In paragraph (8), “application fee” means any fee payable to the Secretary of State by the person making an application for child support maintenance under section 4(1) or 7(1) of the 1991 Act.

\subsection[6. Liability end date]{Liability end date}

6.—(1) Subject to paragraph (2), the date determined for the purposes of paragraph 5(1) and (2) of Schedule 5 to the 2008 Act (beyond which no further liability accrues in relation to the case) is—
\begin{enumerate}\item[]
($a$) if an existing case is related to an application under section 4(1) or~7(1) of the 1991 Act\footnote{Section 4(1) was amended by sections 1(2) and 26 of, and paragraph 11(1) and (2) of Schedule 3 to, the 2000 Act in relation to certain cases. Section 7(1) was amended by sections 1(2) and 26 of, and paragraph 11(1) and (2) of Schedule 3 to, the 2000 Act in relation to certain cases, by schedule 8 to the 2012 Act and by the 2012 Order.}, 
%the day falling 
a date specified by the Secretary of State which shall be no less than  % Words substituted by SI 2014/1386 reg 8(3)(a)
30 days after the date on which notice is given under regulation 5(2);

($b$) in any other case, a date specified by the Secretary of State which shall be no less than 180 days but no more than 272 days after which that notice is given.
\end{enumerate}

(2) Where an existing case becomes related to an application under section~4(1) or 7(1) of the 1991 Act after a notice has been given under regulation~5(2), the Secretary of State may revise the liability end date by issuing a further notice to the interested parties with a liability end date falling 
at least  % Words inserted by SI 2014/1386 reg 8(3)(b)
30 days after such further notice is given.

(3) Paragraph (1) shall not apply where the Secretary of State withdraws a notice under regulation 5(7).

\amendment{
Words substituted in reg. 6(1)(a) and words inserted in reg. 6(2) (30.6.14) by the Child support (Consequential and Miscellaneous Amendments) Regulations 2014 reg. 8(3).
}

\subsection[7. Effect of an application exercising the choice to remain in the statutory scheme]{Effect of an application exercising the choice to remain in the statutory scheme}

7.—(1) The 1991 Act and regulations made under that Act apply in relation to an application under regulation 5(5) as if it were an application made under section 4(1) or section 7(1) of that Act.

(2) Subject to paragraph (3), the 1991 Act and regulations made under that Act apply in relation to a maintenance calculation made in response to an application under regulation 5(5) as if it were a maintenance calculation made in response to an application made under section 4(1) or section 7(1) of that Act.

(3) Where an application under regulation 5(5) is made, the maintenance calculation made in response to that application is to be calculated by reference to the information applicable at the date the absent parent or non-resident parent is notified of that application but takes effect from the day after the liability end date.

(4) Where an application under regulation 5(5) is made in any given case, in sections 31 (deduction from earnings orders) and 32A (orders for regular deductions from accounts) of the 1991 Act, references to “the maintenance calculation in question” and “the calculation” (or “the maintenance assessment in question” or “assessment” where applicable) apply in relation to that particular case to the maintenance calculation made in response to that application as though it were a continuation of the maintenance calculation (or maintenance assessment) for which liability ceased to accrue in accordance with regulation 6\footnote{In section 31 the terms “maintenance assessment” and “assessment” were substituted by “maintenance calculation” and “calculation” by section 1(2) of the 2000 Act in relation to cases specified in article 3 of S.I.~2003/192.}.

\subsection[8. Treating applications for a maintenance assessment or for a maintenance calculation falling to be made under existing rules as withdrawn]{Treating applications for a maintenance assessment or for a maintenance calculation falling to be made under existing rules as withdrawn}

8.  Where the power in paragraph 1(1) of Schedule 5 to the 2008 Act is exercised in relation to a case mentioned in paragraph 1(2)($b$)  or ($d$)  of that Schedule (application for a maintenance assessment or for a maintenance calculation falling to be made under existing rules), if none of the interested parties exercises a choice to remain in the statutory scheme before the liability end date, the Secretary of State may treat that application as withdrawn. 

\bigskip

\pagebreak[3]

Signed 
by authority of the 
Secretary of State for~Work and~Pensions.
%I concur
%By authority of the Lord Chancellor

{\raggedleft
\emph{Steve Webb}\\*
%Secretary
Minister
%Parliamentary Under Secretary 
of State\\*Department 
for~Work and~Pensions

}

12th March 2014

\small

\part{Explanatory Note}

\renewcommand\parthead{— Explanatory Note}

\subsection*{(This note is not part of the Regulations)}

These Regulations make provision for the cases that are existing cases under the statutory child support maintenance scheme that are not subject to the new calculation rules.

The new calculation rules are the provisions in Schedule 1 to the Child Support Act 1991 (“the 1991 Act”) as amended by the Child Maintenance and Other Payments Act 2008 (“the 2008 Act”). The calculation rules that applied to an existing case before the new rules came into force continue to apply.

Regulation 3 makes provision for the power in paragraph 1 of Schedule 5 to the 2008 Act (power to require the parties to an existing case to make a choice as to whether to stay in the statutory child support scheme) to be exercised in accordance with a scheme prepared by the Secretary of State. The scheme must provide for a transition period’s end date. The scheme can be revised by the Secretary of State.

Regulation 4 provides that the scheme prepared by the Secretary of State must make provision for the exercise of the power in stages and lists the principles that must be applied in making such provision.

Regulation 5 provides that at a time determined in accordance with the scheme, the parties to the case must be notified of the date on which liability in the existing case will end and by which they must choose whether to remain in the statutory child support maintenance scheme. The choice to remain in the statutory child support maintenance scheme is to be exercised by way of a new application for a calculation of child support maintenance. Paragraphs (7) and (8) make provision for when a notice under this regulation can be withdrawn and includes provision for reimbursing any application fee paid.

Regulation 6 provides for the meaning of “liability end date” referred to in regulation 5 which is also the date by which an interested party must choose whether to remain in the statutory child support maintenance scheme. The notice period will be 30 days where a case is related to a new application and between 180--272 days in all other cases. Where a case becomes related to a new application after a notice has been issued, the Secretary of State may revise the liability end date to 30 days’ notice.

Regulation 7 provides for the application of the 1991 Act, and regulations made under it, to an application made under regulation 5, and a calculation made in response to an application under regulation 5, as if it were an application made under section 4 (or, where appropriate, section 7) of that Act. The exception to this is that the maintenance calculation made in response to the application under regulation 5 is to be calculated by reference to the information applicable at the date the absent parent or non-resident parent is notified of the application but the effective date will be the day after the liability end date. There is also provision for any references to a maintenance calculation (or maintenance assessment) in relation to deduction from earnings orders and regular deductions from accounts to be treated as a continuation of the maintenance calculation (or maintenance assessment) for which liability ends under regulation 6.

Regulation 8 provides for the treatment of an existing case where no calculation of maintenance has yet been made. If the choice to stay in the statutory scheme is not exercised the application may be treated as withdrawn.

An assessment of the impact of these Regulations on the costs of business and the voluntary sector is available from the Department for Work and Pensions, Child Support, Level 7, Caxton House, Tothill Street, London, \textsc{\lowercase{SW1H 9NA}} and is annexed to the Explanatory Memorandum to these Regulations which is available alongside the instrument on \url{www.legislation.gov.uk}. 

\end{document}