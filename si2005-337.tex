\documentclass[12pt,a4paper]{article}

\newcommand\regstitle{The Social Security, Child Support and Tax Credits (Miscellaneous Amendments) Regulations 2005}

\newcommand\regsnumber{2005/337}

%\opt{newrules}{
\title{\regstitle}
%}

%\opt{2012rules}{
%\title{Child Maintenance and Other Payments Act 2008\\(2012 scheme version)}
%}

\author{S.I.\ 2005 No.\ 337}

\date{Made
15th February 2005\\
Laid before Parliament
18th February 2005\\
Coming into force
18th March 2005
}

%\opt{oldrules}{\newcommand\versionyear{1993}}
%\opt{newrules}{\newcommand\versionyear{2003}}
%\opt{2012rules}{\newcommand\versionyear{2012}}

\usepackage{csa-regs}

\setlength\headheight{27.61603pt}

\begin{document}

\maketitle

\noindent
The Secretary of State for Work and Pensions, in exercise of the powers conferred by the enactments set out in the Schedule to this Instrument, and of all other powers enabling him in that behalf, with the concurrence of the Lord Chancellor in so far as the Regulations are made under section 6(3) of the Social Security Act 1998, after consultation with the Council on Tribunals in accordance with section 8 of the Tribunals and Inquiries Act 1992\footnote{1992 c.\ 53.}, after agreement by the Social Security Advisory Committee that the proposals to make these Regulations should not be referred to it\footnote{\emph{See} the Social Security Administration Act 1992 (c.\ 5), sections 172 and 173(1)($b$)  and Schedule 7, paragraph 9.}, and so far as they concern housing benefit and council tax benefit after consultation with organisations appearing to the Secretary of State to be representative of the authorities concerned\footnote{\emph{See} the Social Security Administration Act 1992, section 176(1)($a$).}, hereby makes the following Regulations: 

{\sloppy

\tableofcontents

}

\bigskip

\setcounter{secnumdepth}{-2}

\subsection[1. Citation and commencement]{Citation and commencement}

1.  These Regulations may be cited as the Social Security, Child Support and Tax Credits (Miscellaneous Amendments) Regulations 2005 and shall come into force on 18th March 2005.

\subsection[2. Amendment of the Social Security and Child Support (Decisions and Appeals) Regulations 1999]{Amendment of the Social Security and Child Support (Decisions and Appeals) Regulations 1999}

2.---(1)  The Social Security and Child Support (Decisions and Appeals) Regulations 1999\footnote{S.I.\ 1999/991.} shall be amended in accordance with this regulation.

(2) In regulation 3 (revision of decisions)—
\begin{enumerate}\item[]
($a$) after paragraph (7)\footnote{Paragraph (7) was substituted by S.I.\ 2002/428} (award of another relevant benefit) insert—
\begin{quotation}
“(7ZA) Where—
\begin{enumerate}\item[]
($a$) the Secretary of State makes a decision under section 8 or 10 awarding income support or state pension credit to a claimant (“the original award”);

($b$) the claimant has a non-dependant within the meaning of regulation 3 of the Income Support Regulations or a person residing with him within the meaning of paragraph 1(1)($a$)(ii), ($b$)(ii)  or ($c$)(iii)  of Schedule I to the State Pension Credit Regulations (“the non-dependant”);

($c$) but for the non-dependant—
\begin{enumerate}\item[]
(i) a severe disability premium would be applicable to the claimant under regulation 17(1)($d$)  of the Income Support Regulations; or

(ii)  an additional amount would be applicable to the claimant as a severe disabled person under regulation 6(4) of the State Pension Credit Regulations; and
\end{enumerate}

($d$) after the original award the non-dependant is awarded benefit which—
\begin{enumerate}\item[]
(i) is for a period which includes the date on which the original award took effect; and

(ii)  is such that a severe disability premium becomes applicable to the claimant under paragraph 13(3)($a$)  of Schedule 2 to the Income Support Regulations or an additional amount for severe disability becomes applicable to him under paragraph 2(2)($a$)  of Schedule I to the State Pension Credit Regulations,
\end{enumerate}
\end{enumerate}
the Secretary of State may revise the original award.”,
\end{quotation}

($b$) after paragraph (7A)\footnote{Paragraph (7A) was inserted by S.I.\ 2002/1379.} insert—
\begin{quotation}
“(7B) A decision under regulation 22A\footnote{Regulation 22A was inserted by S.I.\ 1996/206 and was amended by S.I.\ 1999/2422 and 3109, 2000/590 and 2001/3767.} of the Income Support Regulations (reduction in applicable amount where the claimant is appealing against a decision which embodies a determination that he is not incapable of work) may be revised if the appeal is successful.

(7C) Where a person’s entitlement to income support is terminated because of a determination that he is not incapable of work and he subsequently appeals the decision that embodies that determination and is entitled to income support under regulation 22A of the Income Support Regulations, the decision to terminate entitlement may be revised.”, and
\end{quotation}

($c$) in paragraph (9)($a$)\footnote{Paragraph (9) was substituted by S.I.\ 1999/2677 and amended by S.I.\ 2003/1050.} after “since the decision had effect” insert “or, in the case of an advance award under regulation 13\footnote{Regulation 13 was amended by S.I.\ 1991/2284 and 2741, 1992/247, 1994/2319, 1999/2422, 2572 and 3178 and 2002/3019.}, 13A or 13C\footnote{Regulations 13A and 13C were inserted by S.I.\ 1991/2741 and amended by S.I.\ 1999/2860 and 3178.} of the Claims and Payments Regulations, since the decision was made,”.
\end{enumerate}

(3) In regulation 4\footnote{Relevant amendments to regulation 4 were made by S.I.\ 2000/3185.} (late application for a revision), in paragraph (3)($b$)  after “revised” add—
\begin{quotation}
“, but if the applicant has requested a statement of the reasons in accordance with regulation 28(1)($b$)  the 13 month period shall be extended by—
\begin{enumerate}\item[]
(i) if the statement is provided within one month of the notification, an additional 14 days; or

(ii) if it is provided after the elapse of a period after the one month ends, the length of that period and an additional 14 days.”.
\end{enumerate}
\end{quotation}

(4) In regulation 6 (supersession of decisions)—
\begin{enumerate}\item[]
($a$) in paragraph (2)($a$)(i)\footnote{Sub-paragraph ($a$)(i) was amended by S.I.\ 2003/1050.} after “since the decision had effect” insert “or, in the case of an advance award under regulation 13\footnote{Regulation 13 was amended by S.I.\ 1991/2284 and 2741, 1992/247, 1994/2319, 1999/2422, 2572 and 3178 and 2002/3019.}, 13A or 13C\footnote{Regulations 13A and 13C were inserted by S.I.\ 1991/2741 and amended by S.I.\ 1999/2860 and 3178.} of the Claims and Payments Regulations, since the decision was made”,

($b$) after paragraph (2)($e$)\footnote{Sub-paragraph ($e$) was substituted by S.I.\ 2000/1596 and amended by S.I.\ 2002/428.} insert—
\begin{quotation}
“($ee$) is an original award within the meaning of regulation 3(7ZA) and sub-paragraphs ($a$)  to ($c$)  and ($d$)(ii)  of regulation 3(7ZA) apply but not sub-paragraph ($d$)(i);”,
\end{quotation}

($c$) after paragraph (2)($m$)\footnote{Sub-paragraph ($m$) was inserted by S.I.\ 2003/2274.} add—
\begin{quotation}
“($n$) is a decision by an appeal tribunal confirming a decision by the Secretary of State terminating a claimant’s entitlement to income support because he no longer falls within the category of person specified in paragraph 7 of Schedule 1B to the Income Support Regulations (persons incapable of work) and a further appeal tribunal subsequently determines that he is incapable of work.”, and
\end{quotation}

($d$) omit paragraph (6)($b$)  (person in receipt of income support or a jobseeker’s allowance temporarily absent from a nursing home or residential care home).
\end{enumerate}

(5) In regulation 7 (date from which a decision superseded under section 10 takes effect)—

\begin{enumerate}\item[]
($a$) in paragraph (2)\footnote{Paragraph (2) was amended by S.I.\ 1999/1623 and 3178, 2000/1596, 2002/3019 and 2003/1050.}, after “since the decision had effect” insert “or, in the case of an advance award, since the decision was made”,

($b$) after paragraph (6) insert—
\begin{quotation}
“(6A) Where—
\begin{enumerate}\item[]
($a$) there is a decision which is a relevant determination for the purposes of section 27 and the Secretary of State makes a benefit decision of the kind specified in section 27(1)($b$);

($b$) there is an appeal against the determination;

($c$) after the benefit decision payment is suspended in accordance with regulation 16(1) and (3)($b$)(ii); and

($d$) on appeal a court, within the meaning of section 27, reverses the determination in whole or in part,
\end{enumerate}
a consequential decision by the Secretary of State under section 10 which supersedes his earlier decision under sub-paragraph ($a$)  shall take effect from the date on which the earlier decision took effect.”,
\end{quotation}

($c$) in paragraph (7)\footnote{Paragraph (7) was substituted by S.I.\ 2002/428.} after “6(2)($e$)” and “6(2)($e$)(ii)” insert “or ($ee$)”, and

($d$) after paragraph (33)\footnote{Paragraph (33) was added by S.I.\ 2003/1050.} add—
\begin{quotation}
“(34) A decision which supersedes a decision specified in regulation 6(2)($n$)  shall take effect from the effective date of the Secretary of State’s decision to terminate income support which was confirmed by the decision specified in regulation 6(2)($n$).”.
\end{quotation}
\end{enumerate}

(6) In regulation 28 (notice of decision against which appeal lies), in paragraph (2) after “request” add “or as soon as practicable afterwards.”.

(7) In regulation 30 (appeal against a decision which has been replaced or revised), in paragraph (2) after sub-paragraph ($d$)  add—
\begin{quotation}
“($dd$) it reverses a decision under section 29(2) that an accident is not an industrial accident;”.
\end{quotation}

% Reg 2(8)--(18) revoked (3.11.08) by SI 2008/2683 Sch 2
%(8) In regulation 31(2)\footnote{Relevant amendments were made to paragraph (2) by S.I.\ 2002/1204.} (time within which an appeal is to be brought) for the words “or 3A(1)” in both places where they occur substitute “, 3A(1) or regulation 17(1)($a$)  of the Child Support (Maintenance Assessment Procedure) Regulations 1992\footnote{S.I. 1992/1813; regulation 17(1)($a$) was made under section 16 of the Child Support Act 1991 (c.\ 48) before that section was amended by the Child Support, Pensions and Social Security Act 2000 (c.\ 19), section 8.}”.
%
%(9) In regulation 49 (procedure at oral hearings), in paragraph (7)($b$)\footnote{Paragraph (7) was substituted by S.I.\ 2002/1379.} omit “and the appellant consents”.
%
%(10) In regulation 53 (decisions of appeal tribunals)—
%\begin{enumerate}\item[]
%($a$) in paragraph (3), omit “prepared in accordance with paragraphs (1) and (2)”,
%
%($b$) in paragraph (4)\footnote{Paragraph (4) was substituted by S.I.\ 2002/1379.} before “A party to the proceedings may apply” insert “Subject to paragraph (4A),”, and
%
%($c$) after paragraph (4) insert—
%\begin{quotation}
%“(4A) Where—
%\begin{enumerate}\item[]
%($a$) the decision notice is corrected in accordance with regulation 56; or
%
%($b$) an application under regulation 57 for the decision to be set aside is refused for reasons other than a refusal to extend the time for making the application,
%\end{enumerate}
%the period specified in paragraph (4) shall run from the date on which notice of the correction or the refusal of the application for setting aside is sent to the applicant.”.
%\end{quotation}
%\end{enumerate}
%
%(11) In regulation 54 (late applications for a statement of reasons of tribunal decision)—
%\begin{enumerate}\item[]
%($a$) in paragraph (1) for “paragraph (13)” substitute “regulation 53(4A)”, and
%
%($b$) omit paragraph (13)\footnote{Paragraph (13) was inserted by S.I.\ 2002/1379.}.
%\end{enumerate}
%
%(12) In regulation 55 (record of tribunal proceedings), for paragraph (2) substitute—
%\begin{quotation}
%“(2) The clerk to the appeal tribunal shall preserve—
%\begin{enumerate}\item[]
%($a$) the record of proceedings;
%
%($b$) the decision notice; and
%
%($c$) any statement of the reasons for the tribunal’s decision,
%\end{enumerate}
%for the period specified in paragraph (3).
%
%(3) That period is six months from the date of—
%\begin{enumerate}\item[]
%($a$) the decision made by the appeal tribunal;
%
%($b$) any statement of reasons for the tribunal’s decision;
%
%($c$) any correction of the decision in accordance with regulation 56;
%
%($d$) any refusal to set aside the decision in accordance with regulation 57; or
%
%($e$) any determination of an application under regulation 58 for leave to appeal against the decision,
%\end{enumerate}
%or until the date on which those documents are sent to the office of the Social Security and Child Support Commissioners in connection with an appeal against the decision or an application to a Commissioner for leave to appeal, if that occurs within the six months.
%
%(4) Any party to the proceedings may within the time specified in paragraph (3) apply in writing for a copy of the record of proceedings and a copy shall be supplied to him.”.
%\end{quotation}
%
%(13) In regulation 56\footnote{Regulation 56 was amended by S.I.\ 2000/1596.} (correction of accidental errors)—
%\begin{enumerate}\item[]
%($a$) in paragraph (1), for “any decision, or the record of any such decision,” substitute “the notice of any decision”, and
%
%($b$) for paragraph (2) substitute—
%\begin{quotation}
%“(2) A correction made to a decision notice shall be deemed to be part of the decision notice and written notice of the correction shall be given as soon as practicable to every party to the proceedings.”.
%\end{quotation}
%\end{enumerate}
%
%(14) In regulation 57\footnote{Regulation 57 was amended by S.I.\ 2002/1379.} (setting aside decisions on certain grounds) after paragraph (4) insert—
%\begin{quotation}
%“(4A) Where a legally qualified panel member refuses to set aside a decision he may treat the application to set aside the decision as an application under regulation 53(4) for a statement of the reasons for the tribunal’s decision, subject to the time limits set out in regulation 53(4) and (4A).”.
%\end{quotation}
%
%(15) In regulation 57A\footnote{Regulation 57A was inserted by S.I.\ 2000/1596 and substituted by S.I.\ 2002/1397.} (provisions common to regulation 56 and 57) omit paragraph (1).
%
%(16) After regulation 57A insert—
%\begin{quotation}
%\subsection*{“Service of decision notice by electronic mail}
%
%57AA.  For the purposes of the time limits in regulations 53 to 57, a properly addressed copy of a decision notice sent by electronic mail is effective from the date it is sent.”.
%\end{quotation}
%
%(17) For regulation 57B\footnote{Regulation 57B was inserted by S.I.\ 2000/1596.} substitute—
%\begin{quotation}
%\subsection*{“Interpretation of Chapter V}
%
%57B.---(1)  In Chapter V, except in regulations 58 and 58A—
%\begin{enumerate}\item[]
%“Commissioner” includes Child Support Commissioner;
%
%“decision” includes a determination on a referral.
%\end{enumerate}
%
%(2) In Chapter V—
%\begin{enumerate}\item[]
%“decision notice” has the meaning given in regulation 53(1) and (2).”.
%\end{enumerate}
%\end{quotation}
%
%(18) In regulation 58\footnote{Relevant amendments to regulation 58 were made by S.I.\ 2002/1379.} (application for leave to appeal to a Commissioner from an appeal tribunal)—
%\begin{enumerate}\item[]
%($a$) in paragraph (1)\footnote{Paragraph (1) was amended by S.I.\ 2002/1379.}—
%\begin{enumerate}\item[]
%(i) before “An application for leave” insert “Subject to paragraph (1A),”, and
%
%(ii) for sub-paragraph ($b$)  substitute—
%\begin{quotation}
%“($b$) be in writing and signed by the applicant or, where he has given written authority to a representative to make the application on his behalf, by that representative;
%
%($c$) contain particulars of the grounds on which the applicant intends to rely;
%
%($d$) contain sufficient particulars of the decision of the appeal tribunal to enable the decision to be identified; and
%
%($e$) if the application is made late, contain the grounds for seeking late acceptance.”,
%\end{quotation}
%\end{enumerate}
%
%($b$) after paragraph (1) insert—
%\begin{quotation}
%“(1A) Where after the written statement of the reasons for the decision has been sent to the parties to the proceedings—
%\begin{enumerate}\item[]
%($a$) the decision notice is corrected in accordance with regulation 56; or
%
%($b$) an application under regulation 57 for the decision to be set aside is refused for reasons other than a refusal to extend the time for making the application,
%\end{enumerate}
%the period specified in paragraph (1)($a$)  shall run from the date on which notice of the correction or the refusal of the application for setting aside is sent to the applicant.”, and
%\end{quotation}
%
%($c$) in paragraph (5) after “(1)($a$)” insert “or (1A)”.
%\end{enumerate}

(19) In Chapter V of Part V, after regulation 58 insert—
\begin{quotation}
\subsection*{“Appeal to a Commissioner by a partner}

58A.  A partner within the meaning of section 2AA(7) of the Administration Act\footnote{Section 2AA was inserted by the Employment Act 2002 (c.\ 22), section 49.} (full entitlement to certain benefits conditional on work-focused interview for partner) may appeal to a Commissioner under section 14 from a decision of an appeal tribunal in respect of a decision specified in section 2B(2A) and (6)\footnote{Section 2B was inserted by the Welfare Reform and Pensions Act 1999 (c.\ 30), section 57; subsection (2A) was inserted by the Employment Act 2002, Schedule 7, paragraph 9(4).} of the Administration Act.”.
\end{quotation}

% Reg 2(20) revoked (3.11.08) by SI 2008/2683 Sch 2
%(20) In Schedule 3 (qualifications of persons appointed to the panel)—
%\begin{enumerate}\item[]
%($a$) in paragraph 2 (medical qualifications)—
%\begin{enumerate}\item[]
%(i) omit “Fully”,
%
%(ii) for sub-paragraph ($a$)  substitute—
%\begin{quotation}
%“($a$) the practitioner is a citizen of an \textsc{\lowercase{EEA}} state and his name appears on a medical specialist register maintained in an \textsc{\lowercase{EEA}} state in accordance with the Medical Directive, or he is a Swiss citizen with equivalent qualifications; or”,
%    and 
%\end{quotation}
%
%(iii) for sub-paragraph ($c$)  substitute—
%\begin{quotation}
%“($c$) the practitioner does not satisfy the requirements of sub-paragraph ($a$)  or ($b$)  above, but has not less than 10 years experience in clinical practice, or as a medical disability analyst in disciplines which are the same or similar to those undertaken by practitioners to whom those sub-paragraphs apply.”,
%    and 
%\end{quotation}
%\end{enumerate}
%
%($b$) in paragraph 3, in the definition of “Medical Directive” after “1997” add “, or any directive which replaces Directive 93/16/\textsc{\lowercase{EEC}}”.
%\end{enumerate}
%
%(21) In Schedule 3A\footnote{Schedule 3A was inserted by S.I.\ 2000/1596.} (date on which change of circumstances takes effect in certain cases where a claimant is in receipt of income support or jobseeker’s allowance)—
%\begin{enumerate}\item[]
%($a$) in the Schedule heading for “on which change of circumstances takes effect in certain cases” substitute “from which superseding decision takes effect”,
%
%($b$) omit paragraphs 3($c$)  and 8($c$)  (cases where the claimant or his partner enters a nursing home or residential care home for not more than 8 weeks),
%
%($c$) in paragraph 5—
%\begin{enumerate}\item[]
%(i) in sub-paragraph ($a$)  after “change of circumstances” insert “or change specified in paragraphs 12 and 13”, and
%
%(ii) in sub-paragraph ($b$)  after “prescribed in” insert “paragraph 12 or”,
%\end{enumerate}
%
%($d$) in paragraph 11—
%\begin{enumerate}\item[]
%(i) after “Where a relevant change of circumstances” insert “or change specified in paragraphs 12 and 13”, and
%
%(ii) after “accordance with” insert “paragraph 12 or”, and
%\end{enumerate}
%
%($e$) after paragraph 11 add—
%\begin{quotation}
%\subsection*{\sloppy\itshape “Changes other than changes of circumstances}
%
%12.  Where an amount of income support or jobseeker’s allowance payable under an award is changed by a superseding decision specified in paragraph 13 the superseding decision shall take effect—
%\begin{enumerate}\item[]
%($a$) in the case of a change in respect of income support, from the day specified in paragraph 1($a$)  or ($b$)  for a change of circumstances; and
%
%($b$) in the case of a change in respect of jobseeker’s allowance, from the day specified in paragraph 7 for a change of circumstances.
%\end{enumerate}
%
%\medskip
%
%13.  The following are superseding decisions for the purposes of paragraph 12—
%\begin{enumerate}\item[]
%($a$) a decision which supersedes a decision specified in regulation 6(2)($b$)  to ($ee$); and
%
%($b$) a superseding decision which would, but for paragraph 12, take effect from a date specified in regulation 7(5) to (7), (12) to (16), (18) to (20), (22), (24) and (33).”.
%\end{enumerate}
%\end{quotation}
%\end{enumerate}

\amendment{
Reg. 2(8)--(18), (20) revoked (3.11.08) by the Tribunals, Courts and Enforcement Act 2007 (Transitional and Consequential Provisions) Order 2008 Sch.~2.
}

\subsection[3. Amendment of the Housing Benefit and Council Tax Benefit (Decisions and Appeals) Regulations 2001]{Amendment of the Housing Benefit and Council Tax Benefit (Decisions and Appeals) Regulations 2001}

3.---(1)  The Housing Benefit and Council Tax Benefit (Decisions and Appeals) Regulations 2001\footnote{S.I.\ 2001/1002.} shall be amended in accordance with this regulation.

(2) In regulation 4 (revision of decisions) in paragraph (10) for “was made” substitute “had effect”.

(3) In regulation 5 (late application for a revision), in paragraph (3)($b$)  before “be made” insert “subject to regulation 4(4),”.

(4) In regulation 23 (procedure in connection with appeals), in paragraph (1) for “Social Security and Child Support (Decisions and Appeals) Amendment Regulations 2004\footnote{S.I.\ 2004/3368.}” substitute “Social Security, Child Support and Tax Credits (Miscellaneous Amendments) Regulations 2005\footnote{S.I.\ 2005/337.}”.

\subsection[4. Amendment of the Tax Credits (Appeals) (No.\ 2) Regulations 2002]{Amendment of the Tax Credits (Appeals) (No.\ 2) Regulations 2002}

4.---(1)  The Tax Credits (Appeals) (No.\ 2) Regulations 2002\footnote{S.I.\ 2002/3196.} shall be amended in accordance with this regulation.

(2) In regulation 3 (other persons with a right of appeal or a right to make an application for a direction), in paragraph (ii)  for “tutor, curator or other guardian acting or appointed in terms of law” substitute “judicial factor, or guardian acting or appointed under the Adults with Incapacity (Scotland) Act 2000\footnote{2000 asp 4.} who has power to claim, or as the case may be, receive a tax credit on his behalf”.

% Reg 4(3)--(11) revoked (3.11.08) by SI 2008/2683 Sch 2
%(3) In regulation 18 (procedure at oral hearings), in paragraph (8)($b$)  omit “and the appellant, the applicant for a direction or the person who is subject to penalty proceedings consents”.
%
%(4) In regulation 21 (decisions of appeal tribunals)—
%\begin{enumerate}\item[]
%($a$) in paragraph (3), omit “prepared in accordance with paragraphs (1) and (2)”,
%
%($b$) in paragraph (4) before “A party to the proceedings may apply” insert “Subject to paragraph (4A)”, and
%
%($c$) after paragraph (4) insert—
%\begin{quotation}
%“(4A) Where—
%\begin{enumerate}\item[]
%($a$) the decision notice is corrected in accordance with regulation 24; or
%
%($b$) an application under regulation 25 for the decision to be set aside is refused for reasons other than a refusal to extend the time for making the application,
%\end{enumerate}
%the period specified in paragraph (4) shall run from the date on which notice of the correction or the refusal of the application for setting aside is sent to the applicant.”.
%\end{quotation}
%\end{enumerate}
%
%(5) In regulation 22 (late applications for a statement of reasons of tribunal decision)—
%\begin{enumerate}\item[]
%($a$) in paragraph (1), for “paragraph (13)” substitute “regulation 21(4A)”, and
%
%($b$) omit paragraph (13).
%\end{enumerate}
%
%(6) In regulation 23 (record of tribunal proceedings), for paragraph (2) substitute—
%\begin{quotation}
%“(2) The clerk to the appeal tribunal shall preserve—
%\begin{enumerate}\item[]
%($a$) the record of proceedings;
%
%($b$) the decision notice; and
%
%($c$) any statement of the reasons for the tribunal’s decision,
%\end{enumerate}
%for the period specified in paragraph (3).
%
%(3) That period is six months from the date of—
%\begin{enumerate}\item[]
%($a$) the decision made by the appeal tribunal;
%
%($b$) any statement of reasons for the tribunal’s decision;
%
%($c$) any correction of the decision in accordance with regulation 24;
%
%($d$) any refusal to set aside the decision in accordance with regulation 25; or
%
%($e$) any determination of an application under regulations 27 for leave to appeal against the decision,
%\end{enumerate}
%or until the date on which those documents are sent to the office of the Social Security Commissioners in connection with an appeal against the decision or an application to a Commissioner for leave to appeal, if that occurs within the six months.
%
%(4) Any party to the proceedings may within the time specified in paragraph (3) apply in writing for a copy of the record of proceedings and a copy shall be supplied to him.”.
%\end{quotation}
%
%(7) In regulation 24 (correction of accidental errors)—
%\begin{enumerate}\item[]
%($a$) in paragraph (1), for “any decision, or the record of any such decision,” substitute “the notice of any decision”, and
%
%($b$) for paragraph (2) substitute—
%\begin{quotation}
%“(2) A correction made to a decision notice shall be deemed to be part of the decision notice and written notice of the correction shall be given as soon as practicable to every party to the proceedings.”.
%\end{quotation}
%\end{enumerate}
%
%(8) In regulation 25 (setting aside decisions on certain grounds) after paragraph (4) insert—
%\begin{quotation}
%“(4A) Where a legally qualified panel member refuses to set aside a decision he may treat the application to set aside the decision as an application under regulation 21(4) for a statement of the reasons for the tribunal’s decision, subject to the time limits set out in regulation 21(4) and (4A).”.
%\end{quotation}
%
%(9) In regulation 26 (provisions common to regulations 24 and 25) omit paragraph (1).
%
%(10) After regulation 26 insert—
%\begin{quotation}
%\subsection*{“Service of decision notice by electronic mail}
%
%26A.  For the purposes of the time limits in regulations 21 to 25, a properly addressed copy of a decision notice sent by electronic mail is effective from the date it is sent.”.
%\end{quotation}
%
%(11) In regulation 27 (application for leave to appeal to a Commissioner from a decision of an appeal tribunal)—
%\begin{enumerate}\item[]
%($a$) in paragraph (1)—
%\begin{enumerate}\item[]
%(i) after “Appeals Regulations)” in the first place it occurs insert “and paragraph (1A)”, and
%
%(ii) for sub-paragraph ($b$)  substitute—
%\begin{quotation}
%“($b$) be in writing and signed by the applicant or, where he has given written authority to a representative to make the application on his behalf, by that representative;
%
%($c$) contain particulars of the grounds on which the applicant intends to rely;
%
%($d$) contain sufficient particulars of the decision of the appeal tribunal to enable the decision to be identified; and
%
%($e$) if the application is made late, contain the grounds for seeking late acceptance.”,
%\end{quotation}
%\end{enumerate}
%
%($b$) after paragraph (1) insert—
%\begin{quotation}
%“(1A) Where after the written statement of the reasons for the decision has been sent to the parties to the proceedings—
%\begin{enumerate}\item[]
%($a$) the decision notice is corrected in accordance with regulation 24; or
%
%($b$) an application under regulation 25 for the decision to be set aside is refused for reasons other than a refusal to extend the time for making the application,
%\end{enumerate}
%the period specified in paragraph (1)($a$)  shall run from the date on which notice of the correction or the refusal of the application for setting aside is sent to the applicant.”,
%    and 
%\end{quotation}
%
%($c$) in paragraph (4) after “(1)($a$)” insert “or (1A)”.
%\end{enumerate}

\amendment{
Reg. 4(3)--(11) revoked (3.11.08) by the Tribunals, Courts and Enforcement Act 2007 (Transitional and Consequential Provisions) Order 2008 Sch.~2.
}

\subsection[5. Amendment of the Social Security (Industrial Injuries) (Prescribed Diseases) Regulations 1985]{Amendment of the Social Security (Industrial Injuries) (Prescribed Diseases) Regulations 1985}

5.  In the Social Security (Industrial Injuries) (Prescribed Diseases) Regulations 1985\footnote{S.I.\ 1985/967.}, regulation 5 (development of disease) shall be re-numbered paragraph (1) of regulation 5, and immediately after re-numbered paragraph (1) add—
\begin{quotation}
“(2) Where a person claims benefit under Part V of the Contributions and Benefits Act and it is decided that he is not entitled on the basis of a finding that he was not suffering from a prescribed disease, the finding shall be conclusive for the purpose of a decision on a subsequent claim of that kind in respect of the same disease and the same person.”.
\end{quotation}

\subsection[6. Amendment of the Income Support (General) Regulations 1987]{\sloppy Amendment of the Income Support (General) Regulations 1987}

6.  In the Income Support (General) Regulations 1987\footnote{S.I.\ 1987/1967.}, in Schedule 1B\footnote{Schedule 1B was inserted by S.I.\ 1996/206.} (prescribed categories of person) in paragraph 25\footnote{Paragraph 25 was amended by S.I.\ 1999/2422 and 3109.} (persons appealing against a decision which embodies a determination that they are not incapable of work) for “prior to” substitute “beginning with the date on which that determination takes effect until”.

\subsection[7. Amendment of the Social Security (Claims and Payments) Regulations 1987]{Amendment of the Social Security (Claims and Payments) Regulations 1987}

7.---(1)  The Social Security (Claims and Payments) Regulations 1987\footnote{S.I.\ 1987/1968.} shall be amended in accordance with this regulation.

(2) In regulation 4 (making a claim for benefit), after paragraph (6C)\footnote{Paragraph (6C) was inserted by S.I.\ 2003/1632.} insert—
\begin{quotation}
“(6CC) Paragraphs (6C)($b$)  to ($e$)  apply in respect of information, evidence and advice relating to any claim by a person to whom paragraph (6A) applies, whether the claim is made in accordance with paragraph (6B)($b$)  or otherwise.”.
\end{quotation}

(3) In regulation 4D\footnote{Regulation 4D was inserted by S.I.\ 2002/3019 and was amended by S.I.\ 2003/1632.} (making a claim for state pension credit)—
\begin{enumerate}\item[]
($a$) in paragraph (3) omit “or other office designated by the Secretary of State for accepting claims for state pension credit”,

($b$) after paragraph (3) insert—
\begin{quotation}
“(3A) A claim made in writing may also be made at an office designated by the Secretary of State for accepting claims for state pension credit.”,
\end{quotation}

($c$) for paragraph (5) substitute—
\begin{quotation}
“(5) Where a claim is made in accordance with paragraph (4), the local authority or other specified person—
\begin{enumerate}\item[]
($a$) shall forward the claim to the Secretary of State as soon as reasonably practicable;

($b$) may receive information or evidence relating to the claim supplied by the person making, or who has made, the claim or another person, and shall forward it to the Secretary of State as soon as reasonably practicable;

($c$) may obtain information or evidence relating to the claim from the person who has made the claim and shall forward it to the Secretary of State as soon as reasonably practicable;

($d$) may record information or evidence relating to the claim supplied or obtained in accordance with sub-\hspace{0pt}paragraph ($b$)  or ($c$)  and may hold the information or evidence (whether as supplied or obtained or as recorded) for the purpose of forwarding it to the Secretary of State; and

($e$) may give information and advice with respect to the claim to the person who makes, or has made, the claim.”,
    and 
\end{enumerate}
\end{quotation}

($d$) after paragraph (5) insert—
\begin{quotation}
“(5A) Paragraph (5)($b$)  to ($e$)  applies in respect of information, evidence and advice relating to any claim for state pension credit, whether it is made in accordance with paragraph (4) or otherwise.”.
\end{quotation}
\end{enumerate}

(4) In regulation 6 (date of claim)—
\begin{enumerate}\item[]
($a$) in paragraph (20) for “or appeal” substitute “, appeal or termination of an award for a fixed period”,

($b$) in paragraph (21)($b$)\footnote{Paragraph (21) was amended by S.I.\ 2002/428.} for “or appeal” substitute “, appeal or further claim when an award for a fixed period expires, whether benefit is re-awarded when the further claim is decided or following a revision of, or an appeal against, such a decision”, and

($c$) in paragraph (22)\footnote{Paragraph (22) was amended by S.I.\ 2002/428 and 2497.}—
\begin{enumerate}\item[]
(i) in the definition of “relevant benefit”, after sub-paragraph ($e$)  add “($f$)  state pension credit”, and

(ii) in the definition of “family” after “Jobseekers Act” add “, and in the case of state pension credit “member of his family” means the other member of a couple where the claimant is a member of a married or unmarried couple”.
\end{enumerate}
\end{enumerate}

(5) In regulation 13C\footnote{Regulation 13C was inserted by S.I.\ 1991/2741.} for the heading and paragraph (1) substitute—
\begin{quotation}
\subsection*{“Further claim for and award of disability living allowance or attendance allowance}

13C.---(1)  A person entitled to an award of disability living allowance or attendance allowance may make a further claim for disability living allowance or attendance allowance, as the case may be, during the period of 6 months immediately before the existing award expires.”.
\end{quotation}

(6) In regulation 30 (payments on death) in paragraph (1) after “proceed with the claim” add “and any related issue of revision, supersession or appeal”.

(7) In regulation 33 (persons unable to act)—
\begin{enumerate}\item[]
($a$) in paragraph (1)($d$)  for “tutor, curator or other guardian acting or appointed in terms of law” substitute “a judicial factor or any guardian acting or appointed under the Adults with Incapacity (Scotland) Act 2000\footnote{2000 asp 4.} who has power to claim or, as the case may be, receive benefit on his behalf”,

($b$) after paragraph (1) insert—
\begin{quotation}
“(1A) Where a person has been appointed under regulation 71(3) of the Housing Benefit (General) Regulations 1987\footnote{S.I.\ 1987/1971.} or regulation 61(3) of the Council Tax Benefit (General) Regulations 1992\footnote{S.I.\ 1992/1814.} by a relevant authority within the meaning of those Regulations to act on behalf of another in relation to a benefit claim or award, the Secretary of State may, if the person agrees, treat him as if he had appointed him under paragraph (1).”,
\end{quotation}

($c$) in paragraph (2) after “an appointment” insert “, or treated an appointment as made,”, and

($d$) in paragraph (3) for “tutor, curator, or other” substitute “judicial factor or”.
\end{enumerate}

(8) In regulation 38(1)\footnote{Regulation 38(1) was amended by S.I.\ 1989/1686 and 1996/672.} (extinguishment of right to payment of sums by way of benefit where payment is not obtained within the prescribed period)—
\begin{enumerate}\item[]
($a$) after sub-paragraph ($b$)  insert—
\begin{quotation}
“($bb$) in relation to any such sum which the person entitled to it and the Secretary of State have arranged to be paid by means of direct credit transfer into a bank or other account, on the due date for payment of the sum;”, and
\end{quotation}

($b$) in sub-paragraph ($c$)  for “or ($b$)” substitute “, ($b$)  or ($bb$)”.
\end{enumerate}

(9) In regulation 43\footnote{Regulation 43 was amended by S.I.\ 1991/2741 and 2002/2469.} (children)—
\begin{enumerate}\item[]
($a$) in paragraph (2)($b$)  after “18” insert “or, if the person is a parent of the child and living with him, be over the age of 16”, and

($b$) in paragraph (7), before the definition of “child’s father” insert—
\begin{quotation}
““child” means a person under the age of 16;”.
\end{quotation}
\end{enumerate}

(10) In regulation 44\footnote{Regulation 44 was amended by S.I.\ 1990/2208 and 1991/2741.} (payment of disability living allowance on behalf of a beneficiary), in paragraph (3)—
\begin{enumerate}\item[]
($a$) omit “original” in each place where it occurs,

($b$) in sub-paragraph ($b$)  after “term of hire” add “, other than where sub-paragraph ($d$)  applies,”, and

($c$) after sub-paragraph ($c$)  insert—
\begin{quotation}
“; or

($d$) where the original term of hire is extended by an agreed variation of the agreement, the period of the extended term.”.
\end{quotation}
\end{enumerate}

(11) In Schedule 7 (manner and time of payment, and commencement of entitlement in income support cases), in paragraph 2($c$)\footnote{Paragraph 2($c$) was amended by S.I.\ 2000/1483.} omit “registering or required to register as available for work or”.

\amendment{
Regs. 8, 9 revoked (6.3.06) by the Housing Benefit and Council Tax Benefit (Consequential Provisions) Regulations 2006 Sch. 1.
}

% Regs 8, 9 revoked by SI 2006/217 Sch 1
%\subsection[8. Amendment of the Housing Benefit (General) Regulations 1987]{\sloppy Amendment of the Housing Benefit (General) Regulations 1987}
%
%8.  In the Housing Benefit (General) Regulations 1987\footnote{S.I.\ 1987/1971.}, in regulation 71\footnote{Relevant amendments to regulation 71 were made by S.I.\ 2001/1605.} (who may claim)—
%\begin{enumerate}\item[]
%($a$) in paragraph (2)—
%\begin{enumerate}\item[]
%(i) in sub-paragraph ($b$)  for “tutor, curator or other guardian acting or appointed in terms of law” substitute “judicial factor or any guardian acting or appointed under the Adults with Incapacity (Scotland) Act 2000\footnote{2000 asp 4.} who has power to claim or, as the case may be, receive benefit on his behalf”, and
%
%(ii) for “tutor, curator, other” substitute “judicial factor,”,
%\end{enumerate}
%
%($b$) in paragraph (4) after “paragraph (3)” insert “or treated a person as an appointee under paragraph (5)”,
%
%($c$) in paragraph (5) for “so requests in writing” substitute “agrees”, and
%
%($d$) in paragraph (6) for “tutor, curator, other” substitute “judicial factor,”.
%\end{enumerate}
%
%\subsection[9. Amendment of the Council Tax Benefit (General) Regulations 1992]{Amendment of the Council Tax Benefit (General) Regulations 1992}
%
%9.  In the Council Tax Benefit (General) Regulations 1992\footnote{S.I.\ 1992/1814.}, in regulation 61\footnote{Regulation 61 was amended by S.I.\ 1993/688, 1999/3108 and 2001/1605.} (who may claim)—
%\begin{enumerate}\item[]
%($a$) in paragraph (2)—
%\begin{enumerate}\item[]
%(i) in sub-paragraph ($b$)  for “curator, judicial factor or other guardian acting or appointed in terms of law” substitute “judicial factor or any guardian acting or appointed under the Adults with Incapacity (Scotland) Act 2000\footnote{2000 asp 4.} who has power to claim or, as the case may be, receive benefit on his behalf”, and
%
%(ii) for “curator, other” substitute “judicial factor,”,
%\end{enumerate}
%
%($b$) in paragraph (4) after “paragraph (3)” insert “or treated a person as an appointee under paragraph (5)”, and
%
%($c$) in paragraph (5) for “so requests in writing” substitute “agrees”.
%\end{enumerate}

\subsection[10. Amendment of the Social Security (Payments on account, Overpayments and Recovery) Regulations 1988]{\sloppy Amendment of the Social Security (Payments on account, Overpayments and Recovery) Regulations 1988}

10.---(1)  The Social Security (Payments on account, Overpayments and Recovery) Regulations 1988\footnote{S.I.\ 1988/664.} shall be amended in accordance with this regulation.

(2) In regulation 1(2) (interpretation) after the definition of “the Act” insert—
\begin{quotation}
““the Administration Act” means the Social Security Administration Act 1992\footnote{1992 c.\ 5.};”.
\end{quotation}

(3) In regulation 2 (making of interim payments)—
\begin{enumerate}\item[]
($a$) in paragraph (1)\footnote{Paragraph (1) was amended by S.I.\ 1996/30, 1999/1958, 2422, 2571, 2739, 2860 and 3178.}—
\begin{enumerate}\item[]
(i) after “may be entitled” insert “(or, where sub-paragraph ($a$)  applies, entitled apart from satisfying the condition of making a claim)”, and

(ii) in sub-paragraph ($a$), after “immediately” insert “, including where it is impracticable to satisfy immediately the national insurance number requirements in section 1(1A) and (1B) of the Administration Act”, and
\end{enumerate}

($b$) for paragraph (1A)\footnote{Paragraph (1A) was inserted by S.I.\ 1996/30 and amended by S.I.\ 1999/2571.} substitute—
\begin{quotation}
“(1A) Paragraph (1) shall not apply pending the determination of an appeal.”.
\end{quotation}
\end{enumerate}

(4) In regulation 8(1)\footnote{Regulation 8(1) was amended by S.I.\ 1991/387, 1999/2571 and 2000/1483.} (duplication and prescribed payments)—
\begin{enumerate}\item[]
($a$) for “27(2) of the Act” substitute “74(2) of the Administration Act”, and

($b$) after sub-paragraph ($i$)  add—
\begin{quotation}
“($j$) any contribution-based jobseeker’s allowance within the meaning of section 1(4) of the Jobseekers Act 1995\footnote{1995 c.\ 18.}.”.
\end{quotation}
\end{enumerate}

(5) In regulation 12\footnote{Regulation 12 was amended by S.I.\ 1999/3178 and 2003/492.} (circumstances in which determination need not be reversed, varied, revised or superseded) for “Section 53(4) of the Act” substitute “Section 71(5) or (5A)\footnote{Subsection (5A) was inserted by the Social Security (Overpayments) Act 1996 (c.\ 51), section 1(4).} of the Administration Act”.

(6) In regulation 16(3) (limitations on deductions from prescribed benefits) for “regulation 37(1) of the Claims and Payments Regulations (suspension of payments)” substitute “regulation 20 of the Social Security and Child Support (Decisions and Appeals) Regulations 1999\footnote{S.I.\ 1999/991.} (making of payments which have been suspended)”.

\subsection[11. Amendment of the Social Security (Overlapping Benefits) Regulations 1979]{Amendment of the Social Security (Overlapping Benefits) Regulations 1979}

11.  In regulation 2(1) of the Social Security (Overlapping Benefits) Regulations 1979\footnote{S.I.\ 1979/597.} (interpretation), in the definition of “training allowance”\footnote{The definition was amended by S.I.\ 1988/1446 and 1991/387.}—
\begin{enumerate}\item[]
($a$) after “Highlands and Islands Enterprise” in the first place where the words occur insert “, the Learning and Skills Council for England, the National Assembly for Wales”, and

($b$) after “Highlands and Islands Enterprise” in the second place where the words occur insert “, the National Assembly for Wales”.
\end{enumerate}

\bigskip

Signed 
by authority of the 
Secretary of State for Work and Pensions.

{\raggedleft
\emph{P.~Hollis}\\*Parliamentary Under-Secretary of State,\\*Department for Work and Pensions

}

10th February 2005

\bigskip

I concur

Signed 
by authority of the 
Lord Chancellor.

{\raggedleft
\emph{Catherine M.~Ashton}\\*Parliamentary Secretary\\*Department for Constitutional Affairs

}

%St Andrew's House, Edinburgh

%Dated
15th February 2005

\small

\part[Schedule --- Enactments conferring powers exercised in making these Regulations]{Schedule\\*Enactments conferring powers exercised in making these Regulations}

{\noindent\hbadness=10000
%\begin{tabulary}{\linewidth}{JJ}
\begin{longtable}{p{94.09729pt}p{271.91016pt}}
\hline
\endhead
\hline
\endlastfoot
Vaccine Damage Payments Act 1979\footnote{1979 c.\ 17; section 4 was substituted by the Social Security Act 1998 (c.\ 14), section 46.}	&Section 4(2) and (3)\\
Child Support Act 1991\footnote{1991 c.\ 48; section 20 was substituted by the Social Security Act 1998, section 42, and was further substituted by the Child Support, Pensions and Social Security Act 2000 (c.\ 19), section 10 (for the purposes of certain cases only, \emph{see} S.I.\ 2003/192 (C.\ 11)).}	&Section 20(4), (5) and (6)\\
Social Security Contributions and Benefits Act 1992\footnote{1992 c.\ 4; paragraph (1)($e$)  of section 124 was added by the Jobseekers Act 1995 (c.\ 18), Schedule 2, paragraph 30; section 137(1) is cited because of the meaning it gives to “prescribed”.}	&Sections 124(1)($e$), 137(1) and 175(1) and (3)\\
Social Security Administration Act 1992\footnote{1992 c.\ 5; section 6(1) was amended by the Local Government Finance Act 1992 (c.\ 14), Schedule 9, paragraph 12(1)($a$); section 7A was inserted by the Welfare Reform and Pensions Act 1999 (c.\ 30), section 71; subsection (5A) of section 71 was inserted by the Social Security (Overpayments) Act 1996 (c.\ 51), section 1(4), and subsections (5) and (5A) of section 71 were amended by the Social Security Act 1998, Schedule 7, paragraph 81; section 73(1) was amended by to the Jobseekers Act 1995, Schedule 2, paragraph 49(2); section 74(2) was amended by the State Pension Credit Act 2002 (c.\ 16), Schedule 2, paragraph 11; section 191 is cited because of the meaning it gives to “prescribe”.}	&Sections 5(1)($a$)  to ($c$), ($g$), ($i$), ($m$), ($p$)  and ($r$), 6(1)($g$), 7A(2), 71(5), (5A) and (8), 73(1)($a$), 74(2), 189(1), (4) to (6) and 191\\
Social Security (Recovery of Benefits) Act 1997\footnote{1997 c.\ 27.}	&Section 11(5)($a$)  and ($b$)\\ 
Social Security Act 1998\footnote{1998 c.\ 14; section 14(3) was amended by Social Security Contributions (Transfer of Functions, etc.)\ Act 1999 (c.\ 11), Schedule 7, paragraph 27($b$); section 28 was amended by the State Pension Credit Act 2002, Schedule 1, paragraph 10; section 84 is cited because of the meaning it gives to “prescribe”: the powers in sections 12(2), 14(11), 16(1) and 28(1) and Schedule 5, paragraphs 1, 3, 4 and 6, which are exercised in these Regulations in respect of tax credits, are those which have been applied and modified by S.I.\ 2002/2926 under powers in the Tax Credits Act 2002 (c.\ 21), section 63(8).}	&Sections 6(3), 9(1) and (6), 10(3) and (6), 12(2), (6) and (7), 14(3)($b$)  and (11), 16(1), 17(2), 28(1), 79(1) and (4) to (7) and 84; and Schedule 5, paragraphs 1, 3, 4 and 6\\
Child Support, Pensions and Social Security Act 2000\footnote{2000 c.\ 19; paragraph 23(1) of Schedule 7 is cited because of the meaning it gives to “prescribed”.}	&Section 68 and Schedule 7, paragraphs 3(1), 6(8), 10(1), 20(1) and (3) and 23(1)\\
%\end{tabulary}
\end{longtable}

}

\part{Explanatory Note}

\renewcommand\parthead{— Explanatory Note}

\subsection*{(This note is not part of the Regulations)}

Regulation 2 amends the Social Security and Child Support (Decisions and Appeals) Regulations 1999.
\begin{itemize}\item[]
    Paragraph (2) adds further circumstances in which a social security benefit decision may be revised, and paragraph (3) clarifies the time for applying for revision.

    Paragraph (4) adds further circumstances in which a benefit decision may be superseded, and paragraphs (5) and (21) specify the date from which such decisions and other decisions take effect.

    Paragraphs (6) to (8) make further provision for making appeals: paragraph (7) adds a situation in which an appeal lapses, and paragraph (8) provides for the time within which an appeal may be brought when a child support maintenance assessment is revised.

    Paragraph (9) removes the need for the appellant’s consent to the use of a live television link at an appeal hearing.

    Paragraphs (10) to (17) make further, more detailed, provision in relation to decisions of appeal tribunals: paragraph (10) clarifies the time for applying for a statement of the tribunal’s reasons for its decision, paragraph (12) specifies more documents which the tribunal must preserve, paragraph (14) provides for a tribunal to treat a refusal to set aside a decision as an application for a statement of reasons for the decision and paragraph (16) provides for decision notices to be sent by electronic mail.

    Paragraph (18) clarifies the procedure and time for applying for leave to appeal to a Commissioner and paragraph (19) prescribes a further person who may appeal to a Commissioner.

    Paragraph (20) clarifies the medical qualifications required for a person to be appointed to panel so that he may act as a member of an appeal tribunal. 
\end{itemize}

Regulation 3 amends the Housing Benefit and Council Tax Benefit (Decisions and Appeals) Regulations 2001 in respect of revision of decisions and appeal procedures to mirror amendments made to the Social Security and Child Support (Decisions and Appeals) Regulations 1999.

Regulation 4 amends the Tax Credits (Appeals) (No.\ 2) Regulations 2002 making provision similar to that made by regulations 2(9) to (16) and (18) and 7(7)($a$).

Regulation 5 amends the Social Security (Industrial Injuries) (Prescribed Diseases) Regulations 1985 to provide for the finality of a specified determination necessary to a benefit decision.

Regulation 6 makes an amendment to the Income Support (General) Regulations 1987 consequential to the amendment made by regulation 2(2)($b$).

Regulation 7 amends the Social Security (Claims and Payments) Regulations 1987.
\begin{enumerate}\item[]
    Paragraphs (2) and (3) make further provision in respect of specified benefit claims made to a designated office or an authorised local authority.

    Paragraph (4) makes further provision in respect of the date of claim in relation to different benefits.

    Paragraph (5) provides for a further claim for an attendance allowance before an existing claim expires.

    Paragraphs (6) and (7) make further provision in respect of those appointed to act for a benefit claimant.

    Paragraph (8) makes further provision in respect of extinguishment of the right to payment of benefit.

    Paragraphs (9) and (10) amend provisions governing payment of disability living allowance in respect of child beneficiaries and beneficiaries who have hired a vehicle.

    Paragraph (11) makes a minor amendment in respect of the payment of income support. 
\end{enumerate}

Regulations 8 and 9 amend the Housing Benefit (General) Regulations 1987 and the Council Tax Benefit (General) Regulations 1992 respectively, making provision similar to that made by regulation 7(7).

Regulation 10 amends the Social Security (Payments on account, Overpayments and Recovery) Regulations 1988.
\begin{enumerate}\item[]
    Paragraph (3) provides for an interim payment where it is impractical to satisfy national insurance number requirements.

    Paragraph (4) provides for the recovery of duplicate payments of benefit from payments of contribution-based jobseeker’s allowance.

    Paragraph (5) makes further provision for the recovery of a benefit overpayment without prior revision or supersession of the original decision awarding benefit.

    Paragraph (6) clarifies a limitation on the right to deduct a recoverable overpayment from prescribed benefits. 
\end{enumerate}

Regulation 11 amends the Social Security (Overlapping Benefits) Regulations 1979 to update a definition.

A regulatory impact assessment has not been produced for this instrument as it has no impact on the costs of business. 

\end{document}