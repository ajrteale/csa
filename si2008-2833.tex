\documentclass[12pt,a4paper]{article}

\newcommand\regstitle{The Transfer of Tribunal Functions Order 2008}

\newcommand\regsnumber{2008/2833}

%\opt{newrules}{
\title{\regstitle}
%}

%\opt{2012rules}{
%\title{Child Maintenance and~Other Payments Act 2008\\(2012 scheme version)}
%}

\author{S.I.\ 2008 No.\ 2833}

\date{Made
29th October 2008\\
%Laid before Parliament
%10th October 2008\\
Coming into~force
3rd November 2008
}

%\opt{oldrules}{\newcommand\versionyear{1993}}
%\opt{newrules}{\newcommand\versionyear{2003}}
%\opt{2012rules}{\newcommand\versionyear{2012}}

\usepackage{csa-regs}

\setlength\headheight{42.11603pt}

%\hbadness=10000

\begin{document}

\maketitle

\noindent
The Lord Chancellor~makes the following Order in exercise of the powers conferred by sections~30(1) and~(4), 31(1), (2) and~(9), 32(3) and~(5), 33(2) and~(3), 34(2) and~(3), 37(1), 38 and~145 of, and~paragraph~30 of Schedule~5 to, the Tribunals, Courts and~Enforcement Act 2007\footnote{2007 c.~15.}. The Scottish Ministers have consented to~the making of this order in so far as their consent is required by section~30(7) of that Act.

A draft of this Order was laid before Parliament and~approved by a resolution of each House of Parliament in accordance with section~49(5) of that Act. 

{\sloppy

\tableofcontents

}

\bigskip

\setcounter{secnumdepth}{-2}

\subsection[1. Citation, commencement, interpretation and~extent]{Citation, commencement, interpretation and~extent}

1.---(1)  This Order may be cited as the Transfer of Tribunal Functions Order 2008 and~comes into~force on 3rd November 2008.

(2) A reference in this Order to~a Schedule~by a number alone is a reference to~the Schedule~so numbered in this Order.

(3) Subject as follows, this Order extends to~England~and~Wales, Scotland~and~Northern Ireland.

(4) Except as provided by paragraph~(5) or~(6), an amendment, repeal or~revocation of any enactment by any provision of Schedule~3 extends to~the part or~parts of the United Kingdom to~which the enactment extends.

(5) For~the purposes of article~3(3)($a$)  and~($b$)  the following amendments, repeals and~revocations made by the provisions of that Schedule~do not extend to~Scotland—
\begin{enumerate}\item[]
($a$) paragraphs~145 to~147;

($b$) paragraph~150;

($c$) paragraph~151($d$);

($d$) paragraph~152;

($e$) paragraph~154;

($f$) paragraphs~167 to~173; and

($g$) paragraph~228($h$), ($l$), ($n$)  and~($r$).
\end{enumerate}

(6) The amendments and~repeals made by paragraphs~198 to~201 of Schedule~3 do not extend to~Scotland.

\subsection[2. Additions to~the list of tribunals in Schedule~6]{Additions to~the list of tribunals in Schedule~6}

2.  In Part~IV of Schedule~6 to~the Tribunals, Courts and~Enforcement Act 2007 (tribunals for~the purposes of section~30), insert the following entries at the appropriate places—

{\footnotesize\noindent
%\begin{tabulary}{\linewidth}{JJ}
\begin{longtable}{p{150.86967pt}p{215.1561pt}}
\hline
\endhead
\hline
\endlastfoot
“Claims Management Services Tribunal	&Section~12 of the Compensation Act 2006 (c.~29)”\\
“Gender Recognition Panel	&Section~1(3) of the Gender Recognition Act 2004 (c.~7)”\\
“Tribunal	&Section~704 of the Income Tax Act 2007 (c.~3)”\\
%\hline
%\end{tabulary}
\end{longtable}

}

\subsection[3. Transfer of functions of certain tribunals]{Transfer of functions of certain tribunals}

3.---(1)  Subject to~paragraph~(3), the functions of the tribunals listed in Table~1 of Schedule~1 are transferred to~the First-tier Tribunal.

(2) Subject to~paragraph~(3), the functions of the tribunals listed in Table~2 of Schedule~1 are transferred to~the Upper Tribunal.

(3) The following functions are not transferred—
\begin{enumerate}\item[]
($a$) the determination by an appeal tribunal constituted under Chapter~I of Part~I of the Social Security Act 1998\footnote{1998 c.~14.} of an appeal which is referred to~such tribunal by the Scottish Ministers, or~the Secretary of State on their behalf, pursuant to~section~158 (appeal tribunals) of the Health and~Social Care (Community Health and~Standards) Act 2003\footnote{2003 c.~43. This function of Scottish Ministers can be exercised by the Secretary of State pursuant to~Scotland~Act 1998 (Agency Arrangements) (Specifications) (No.~3) Order 2006 (S.I.~2006/3338).} (“the 2003 Act”); and

($b$) the determination by a Social Security Commissioner of an appeal made under section~159 (appeal to~social security commissioner) of the 2003 Act against a decision falling within sub-paragraph~($a$).
\end{enumerate}

\subsection[4. Abolition of tribunals transferred under section~30(1)]{Abolition of tribunals transferred under section~30(1)}

4.  The tribunals listed in Table~1 and~Table~2 of Schedule~1 are abolished except for—
\begin{enumerate}\item[]
($a$) appeal tribunals constituted under Chapter~I of Part~I of the Social Security Act 1998 in respect of Scotland~for~the purposes of the function described in article~3(3)($a$); and

($b$) the Social Security Commissioners in respect of Scotland~for~the purposes of the function described in article~3(3)($b$).
\end{enumerate}

\subsection[5. Transfer of persons into~the First-tier Tribunal and~the Upper Tribunal]{Transfer of persons into~the First-tier Tribunal and~the Upper Tribunal}

5.---(1)  A person holding an office listed in a table in Schedule~2 who was, was a member of, or~was an authorised decision-maker for, a tribunal listed in the corresponding table in Schedule~1 immediately before the functions of that tribunal were transferred under article~3 shall hold the corresponding office or~offices.

(2) In paragraph~(1) “corresponding” means appearing in the corresponding entry in the table below.

{\footnotesize\noindent\hbadness=10000
%\begin{tabulary}{\linewidth}{JJJ}
\begin{longtable}{p{53.78494pt}p{54.8609pt}p{245.3638pt}}
\hline
\itshape Table in Schedule 1	& \itshape Table~in Schedule~2	& \itshape Office or~offices\\
\hline
\endhead
\hline
\endlastfoot
Table~1	&Table~1	&Transferred-in judge of the First-tier Tribunal\\
Table~1	&Table~2	&Transferred-in other member of the First-tier Tribunal\\
Table~1	&Table~3	&Transferred in judge of the First-tier Tribunal and~deputy judge of the Upper Tribunal\\
Table~2	&Table~4	&Transferred-in judge of the Upper Tribunal\\
Table~1 or~2	&Table~5	&Transferred-in other member of the Upper Tribunal\\
%\hline
%\end{tabulary}
\end{longtable}

}

\subsection[6. Appeal to~Upper Tribunal from tribunals in Wales]{Appeal to~Upper Tribunal from tribunals in Wales}

6.---(1)  An appeal against a decision of a tribunal listed in paragraph~(2) lies to~the Upper Tribunal.

(2) The tribunals referred to~in paragraph~(1) are—
\begin{enumerate}\item[]
($a$) the Mental Health Review Tribunal for~Wales established under section~65 of the Mental Health Act 1983\footnote{1983 c.~20. Section~65 was amended by paragraph~107 of Schedule~1 to~the Health Authorities Act 1995 (c.~17), and~is further amended by section~38 of the Mental Health Act 2007 (c.~12) and~by Schedule~3 to~this Order.}; and

($b$) the Special Educational Needs Tribunal for~Wales established under section~336ZA of the Education Act 1996\footnote{1996 c.~56. Section~336ZA was inserted by paragraph~5 of Schedule~18 to~the Education Act 2002 (c.~32). Schedule~3 to~this Order omits section~336ZA and~amends section~333 so that it refers to~the Special Educational Needs Tribunal to~Wales.}.
\end{enumerate}

\subsection[7. Appeal to~Upper Tribunal from tribunals in Scotland]{Appeal to~Upper Tribunal from tribunals in Scotland}

7.  An appeal against a decision of the Pensions Appeal Tribunal in Scotland~under section~5 of the Pensions Appeal Tribunals Act 1943\footnote{6 \& 7 Geo.~6 c.~39. Section~5 was amended by section~23 of the Chronically Sick and~Disabled Persons Act 1970 (c.~44) and~section~16(3) of the Social Security Act 1980 (c.~30), and~is further amended by Schedule~3 to~this Order.} (assessment decision) lies to~the Upper Tribunal.

\subsection[8. Appeal to~Upper Tribunal from tribunals in Northern Ireland]{Appeal to~Upper Tribunal from tribunals in Northern Ireland}

8.  An appeal against a decision of the Pensions Appeal Tribunal in Northern Ireland~under section~5 of the Pensions Appeal Tribunals Act 1943 (assessment decision) lies to~the Upper Tribunal.

\subsection[9. Minor, consequential and~transitional provisions]{Minor, consequential and~transitional provisions}

9.---(1)  Schedule~3 contains minor, consequential and~supplemental amendments, and~repeals and~revocations as a consequence of those amendments.

(2) Schedule~4 contains transitional provisions. 

\bigskip

%Signed 
%by authority of the 
%Secretary of State for~Work and~Pensions.
%I concur
By authority of the Lord Chancellor

{\raggedleft
\emph{Bridget Prentice}\\*
%Secretary
%Minister
Parliamentary Under-Secretary 
of State\\%*Department 
%for~Work and~Pensions
Ministry of Justice

}

29th October 2008

\small

\part[Schedule~1 --- Functions transferred to~the First-tier Tribunal and~Upper Tribunal]{Schedule~1\\*Functions transferred to~the First-tier Tribunal and~Upper Tribunal}

\renewcommand\parthead{--- Schedule~1}

\section*{\itshape Table~1: Functions transferred to~the First-tier Tribunal}

{\noindent
%\begin{tabulary}{\linewidth}{JJ}
\begin{longtable}{p{153.47604pt}p{212.50781pt}}
\hline
\itshape Tribunal	& \itshape Enactment\\
\hline
\endhead
\hline
\endlastfoot
Adjudicator	&Section~5 of the Criminal Injuries Compensation Act 1995 (c.~53)\\
Appeal tribunal	&Chapter~I of Part~I of the Social Security Act 1998 (c.~14)\\
Asylum Support Adjudicators	&Section~102 of the Immigration and~Asylum Act 1999 (c.~33)\\
Mental Health Review Tribunal for~a region of England	&Section~65(1) and~(1A)($a$)  of the Mental Health Act 1983 (c.~20)\\
Pensions Appeal Tribunal in England~and~Wales	&Section~8(2) of the War Pensions (Administrative Provisions) Act 1919 (9 \& 10 Geo.~5 c.~53) and~paragraph~1(1) of the Schedule~to~the Pensions Appeal Tribunals Act 1943 (6 \& 7 Geo.~6 c.~39)\\
Special Educational Needs and Disability Tribunal	&Section~28H of the Disability Discrimination Act 1995 (c.~50) and~section~333 of the Education Act 1996 (c.~56) and\\
Tribunal, except in respect of its functions under section~4 of the Safeguarding Vulnerable Groups Act 2006 (c.~47)	&Section~9 of the Protection of Children Act 1999 (c.~14)\\
%\hline
%\end{tabulary}
\end{longtable}

}

\section*{\itshape Table~2: Functions transferred to~the Upper Tribunal}

{\noindent
%\begin{tabulary}{\linewidth}{JJ}
\begin{longtable}{p{241.53224pt}p{124.46164pt}}
\hline
\itshape Tribunal	& \itshape Enactment\\
\hline
\endhead
\hline
\endlastfoot
Child Support Commissioner	&Section~22 of the Child Support Act 1991 (c.~48)\\
Social Security Commissioner	&Schedule~4 to~the Social Security Act 1998 (c.~14)\\
Tribunal, in respect of its functions under section~4 of the Safeguarding Vulnerable Groups Act 2006 (c.~47)	&Section~9 of the Protection of Children Act 1999 (c.~14)\\
%\hline
%\end{tabulary}
\end{longtable}

}

\vfill

\part[Schedule~2 --- Persons transferred as judges and~members of the First-tier Tribunal and~Upper Tribunal]{Schedule~2\\*Persons transferred as judges and~members of the First-tier Tribunal and~Upper Tribunal}

\renewcommand\parthead{--- Schedule~2}

\section*{\itshape\sloppy Table~1: Members becoming transferred-in judges of the First-tier Tribunal}

{\noindent\hbadness=1158
%\begin{tabulary}{\linewidth}{JJ}
\begin{longtable}{p{151.4971pt}p{214.49799pt}}
\hline
\itshape Tribunal	Member & \itshape Enactment\\
\hline
\endhead
\hline
\endlastfoot
A legal member of the Criminal Injuries Compensation Appeals Panel	&Section~5 of the Criminal Injuries Compensation Act 1995 (c.~53) and~the Criminal Injuries Compensation Schemes\\
A legally qualified panel member	&Section~6 of the Social Security Act 1998 (c.~14)\\
The Deputy Chief Asylum Support Adjudicator~or~an adjudicator	&Section~102 of and~paragraph~1($a$)  and~($c$)  of Schedule~10 to~the Immigration and~Asylum Act 1999 (c.~33)\\
A legal member	&Paragraph 1($a$)  of Schedule~2 to~the Mental Health Act 1983 (c.~20)\\
The Deputy President of Pensions Appeal Tribunals or a legally qualified member	&Paragraphs 2A(1)($a$)  and~2B(1) of the Schedule~to~the Pensions Appeal Tribunals Act 1943 (6 \& 7 Geo.~6 c.~39)\\
A member of the chairmen’s panel	&Section~333(2)($b$)  of the Education Act 1996 (c.~56)\\
A member of the chairmen’s panel	&Paragraph 1(1)($a$)  of the Schedule~to~the Protection of Children Act 1999 (c.~14)\\
%\hline
%\end{tabulary}
\end{longtable}

}

\section*{\itshape\sloppy Table~2: Members becoming transferred-in other members of the First-tier Tribunal}

{\noindent
%\begin{tabulary}{\linewidth}{JJ}
\begin{longtable}{p{188.27951pt}p{177.71112pt}}
\hline
\itshape Tribunal	Member & \itshape Enactment\\
\hline
\endhead
\hline
\endlastfoot
A member of the Criminal Injuries Compensation Appeals Panel other than the Chairman or~a legal member	&Section~5 of the Criminal Injuries Compensation Act 1995 (c.~53) and the Criminal Injuries Compensation Schemes\\
A financially qualified panel member, a medically qualified panel member or~a panel member with a disability qualification	&Section~6 of the Social Security Act 1998 (c.~14)\\
A medical member or~other member	&Paragraph 1($b$)  or~($c$)  of Schedule~2 to~the Mental Health Act 1983 (c.~20)\\
A medically qualified member, a member with knowledge or~experience of service, or~other member	&Paragraph 2A(1)($b$), ($c$)  or~($d$)  of the Schedule~to~the Pensions Appeal Tribunals Act 1943 (6 \& 7 Geo.~6 c.~39)\\
A member of the lay panel	&Section~333(2)($c$)  of the Education Act 1996 (c.~56)\\
A member of the lay panel, other than a member in Table~5	&Paragraph 1(1)($c$)  of the Schedule to the Protection of Children Act 1999 (c.~14)\\
%\hline
%\end{tabulary}
\end{longtable}

}

\section*{\itshape\sloppy
Table~3: Members becoming transferred-in judges of the First-tier Tribunal and~deputy judges of the Upper Tribunal
}

{\noindent\hbadness=10000
%\begin{tabulary}{\linewidth}{JJ}
\begin{longtable}{p{85.252pt}p{280.7461pt}}
\hline
\itshape Tribunal	Member & \itshape Enactment\\
\hline
\endhead
\hline
\endlastfoot
The Chairman	&Section~5(3)($b$)  of the Criminal Injuries Compensation Act 1995 (c.~53) and~the Criminal Injuries Compensation Schemes\\
The President	&Section~5 of the Social Security Act 1998 (c.~14)\\
The Chief Asylum Support Adjudicator	&Section~102 of and~paragraph~1($b$)  of Schedule~10 to~the Immigration and~Asylum Act 1999 (c.~33)\\
A chairman of a Mental Health Review Tribunal	&Paragraph 3 of Schedule~2 to~the Mental Health Act 1983 (c.~20)\\
A President of Pensions Appeal Tribunals	&Paragraph 2B(1) of the Schedule~to~the Pensions Appeal Tribunals Act 1943 (6 \& 7 Geo.~6 c.~39)\\
A President	&Section~333(2)($a$)  of the Education Act 1996 (c.~56)\\
The President	&Paragraph 1(1)($a$)  of the Schedule~to~the Protection of Children Act 1999 (c.~14)\\
The Deputy President	&Appointed as a member of the chairmen’s panel under paragraph~1(1)($b$)  of the Schedule~to~the Protection of Children Act 1999 (c.~14) and~also appointed as deputy president of the Tribunal\\
A deputy Child Support Commissioner	&Paragraph 4 of Schedule~4 to~the Child Support Act 1991 (c.~48)\\
A deputy Commissioner	&Paragraph 1(2) of Schedule~4 to~the Social Security Act 1998 (c.~14)\\
%\hline
%\end{tabulary}
\end{longtable}

}


\section*{\itshape\sloppy Table~4: Members becoming transferred-in judges of the Upper Tribunal}

{\noindent
%\begin{tabulary}{\linewidth}{JJ}
\begin{longtable}{p{196.03664pt}p{169.96071pt}}
\hline
\itshape Tribunal	Member & \itshape Enactment\\
\hline
\endhead
\hline
\endlastfoot
The Chief Child Support Commissioner or~a Child Support Commissioner	&Section~22 of the Child Support Act 1991 (c.~48)\\
The Chief Social Security Commissioner or~a Social Security Commissioner	&Paragraph 1 of Schedule~4 to~the Social Security Act 1998 (c.~14)\\
%\hline
%\end{tabulary}
\end{longtable}

}

\section*{\itshape\sloppy Table~5: Members becoming transferred-in other members of the Upper Tribunal}

{\noindent
%\begin{tabulary}{\linewidth}{JJ}
\begin{longtable}{p{243.72656pt}p{122.26358pt}}
\hline
\itshape Tribunal	Member & \itshape Enactment\\
\hline
\endhead
\hline
\endlastfoot
A member of the lay panel who was appointed on the ground that the member satisfied the requirements referred to~in regulation 41(1) of the Protection of Children Act Tribunal Regulations 2000 (S.I.~2000/2619) or~regulation 3(1)($a$)  or~($b$)  of the Protection of Children and~Vulnerable Adults and~Care Standards Tribunal Regulations 2002 (S.I.~2002/816)	&Paragraph 1(1)($c$)  of the Schedule~to~the Protection of Children Act 1999 (c.~14)\\
%\hline
%\end{tabulary}
\end{longtable}

}

\vfill

\part[Schedule~3 --- Minor, consequential and~supplemental provisions]{Schedule~3\\*Minor, consequential and~supplemental provisions}

\renewcommand\parthead{--- Schedule~3}

\section*{\itshape\sloppy War Pensions (Administrative Provisions) Act 1919}

1.  The War Pensions (Administrative Provisions) Act 1919\footnote{9 \& 10 Geo.~5 c.~53. Section~8(1) was amended by Part~III of the Statute Laws (Repeals) Act 1986 (c.~12), section~8(2) of the War Pensions Act 1920 (c.~23), and~paragraph~10 of Schedule~26 to~the Civil Partnership Act 2004 (c.~33).} is amended as follows.

\medskip

2.  In section~8 (appeals to~Pensions Appeal Tribunals)—
\begin{enumerate}\item[]
($a$) in the heading, omit “to~Pensions Appeal Tribunals”;

($b$) in subsection~(1) for~the words from “a Pensions Appeal Tribunal” to~the end substitute “the appropriate tribunal, whose decision shall be final (subject, in the case of a decision of the First-Tier Tribunal, to~provision made by or~under Chapter~II of Part~I of the Tribunals, Courts and~Enforcement Act 2007).”;

($c$) after that subsection~insert—
\begin{quotation}
“(1A) For~the purposes of subsection~(1) above “the appropriate tribunal” means—
\begin{enumerate}\item[]
($a$) in relation to~England~and~Wales, the First-tier Tribunal;

($b$) in relation to~Scotland, a Pensions Appeal Tribunal for~Scotland~established under this section; and

($c$) in relation to~Northern Ireland, a Pensions Appeal Tribunal for~Northern Ireland~established under this section.”; and
\end{enumerate}
\end{quotation}

($d$) in subsection~(2)—
\begin{enumerate}\item[]
(i) for~“such parts of the United Kingdom as may be determined” substitute “Scotland~and~Northern Ireland”; and

(ii) for~“of Pensions Appeal Tribunals” substitute “of those tribunals”.
\end{enumerate}
\end{enumerate}

\medskip

3.  In paragraph~1 of the Schedule~(constitution jurisdiction and~procedure of Pensions Appeal Tribunals) for~the words from the beginning to~“Kingdom” substitute “Such number of pensions appeal tribunals shall be constituted for~Scotland~and~Northern Ireland”.

\section*{\itshape Pensions Appeal Tribunals Act 1943}

4.  The Pensions Appeal Tribunals Act 1943\footnote{6 \& 7 Geo.~6 c.~39.  Section~2(2) was amended by section~23(1) and~(2)($c$) of the Chronically Sick and~Disabled Persons Act 1970 (c.~44). Sections~5A and~5B were inserted by sections~57(1) and~59 of the Child Support, Pensions and~Social Security Act 2000 (c.~19). Section~6(2C) was inserted by section~43(1) of the Social Security and~Housing Benefits Act 1982 (c.~24) and~subsections~(2C) and~(3) were amended by paragraphs~1 and~3(1), (3) and~(4) of Schedule~1 to~the Armed Forces (Pensions and~Compensation) Act 2004 (c.~32). Sections~6A, 6B, 6C, 6D and~11A were inserted by section~5 of, and~paragraphs~1 and~4 to~6 of Schedule~1 to, the 2004 Act. Section~6D(9) was inserted by paragraphs~24 and~25 of Schedule~4 to~the Constitutional Reform Act 2005 (c.~4). Section~8 was amended by paragraphs~1 and~5 of Schedule~1 to~the 2004 Act. In section~12 the definitions of “Chief Commissioner” and~“Commissioner” were inserted by paragraphs~1 and~7(1) and~(2)($b$) of Schedule~1 to~the 2004 Act. In the Schedule: paragraph~1 was substituted by section~15(1) of, and~paragraphs~24 and~28(1) and~(2) of Schedule~4 to, the Constitutional Reform Act 2005 (c.~4); paragraph~2 was substituted by section~26 of, and~paragraph~39 of Schedule~6 to, the Judicial Pensions and~Retirement Act 1993 (c.~8), sub-paragraph~(2A) was inserted by section~60(2) of the 2000 Act and~amended by paragraphs~24 and~28(1) and~(3)($b$) of Schedule~4 to~the 2005 Act, sub-paragraph~(3A) was inserted by paragraphs~24 and~28(1) and~(3)($c$) of Schedule~4 to~the 2005 Act and~sub-paragraph~(4) was amended by paragraphs~24 and~28(1) and~(3)($d$) of Schedule~4 to~the 2005 Act; paragraphs~2B,~3A,~3B and~3C were inserted by section~60(3) and~(4) of the 2000 Act; paragraph~5(1A) was inserted by paragraphs~24 and~28(1) and~(6)($c$) of Schedule~4 of the 2005 Act; paragraph~6 was amended by paragraphs~1 and~10(1) and~(3) of Schedule~1 to~the 2004 Act; paragraph~6B was inserted by paragraphs~1 and~10(1) and~(5) of Schedule~1 to~the 2004 Act; and~paragraph~7B was inserted by paragraphs~24 and~28(1) and~(7) of Schedule~4 to~the 2005 Act. } is amended as follows.

\medskip

5.  In section~1 (appeals against rejection of war pension claims)—
\begin{enumerate}\item[]
($a$) in subsection~(1) for~“a Pensions Appeal Tribunal constituted under this Act (hereafter in this Act referred to~as “the Tribunal”)” substitute “the appropriate tribunal”; and

($b$) in subsections~(2), (3), (3A) and~(4) for~“Tribunal” substitute “appropriate tribunal”.
\end{enumerate}

\medskip

6.  In section~2(1) and~(2) (appeals against rejection of war pension claims made in respect of mariners, pilots etc) for~“Tribunal” substitute “appropriate tribunal”.

\medskip

7.  In section~3(1) and~(2) (appeals against rejection of war pension claims made in respect of civil defence volunteers and~other civilians) for~“Tribunal” substitute “appropriate tribunal”.

\medskip

8.  In section~4(1) and~(2) (appeals in cases where award is withheld or~reduced on ground of serious negligence or~misconduct) for~“Tribunal” substitute “appropriate tribunal”.

\medskip

9.  In section~5 (appeals against assessment of extent of disablement)—
\begin{enumerate}\item[]
($a$) in subsection~(1) for~“Tribunal”, in both places, substitute “appropriate tribunal”; and

($b$) in subsection~(2)—
\begin{enumerate}\item[]
(i) for~“Tribunal”, in each place, substitute “appropriate tribunal”; and

(ii) for~“Tribunal’s” substitute “appropriate tribunal’s”.
\end{enumerate}
\end{enumerate}

\medskip

10.  In subsection~5A(1)($b$)  (appeals in other cases) for~“Tribunal” substitute “appropriate tribunal”.

\medskip

11.  In section~5B (matters relevant on appeal) for~“appeal, a Pensions Appeal Tribunal” substitute “appeal under any provision of this Act, the appropriate tribunal”.

\medskip

12.---(1)  Section~6 (constitution, jurisdiction and~procedure of Pensions Appeal Tribunals) is amended as follows.

(2) In the heading, at the end insert “for~Scotland~and~Northern Ireland~etc”.

(3) In subsection~(1), at the end insert “for~Scotland~and~Northern Ireland”.

(4) In subsection~(2C)—
\begin{enumerate}\item[]
($a$) in paragraph~($a$)  for~“Tribunal, or” substitute “Pensions Appeal Tribunal for~Scotland~or~Northern Ireland,”;

($b$) after paragraph~($b$)  insert—
\begin{quotation}
“($c$) the First-tier Tribunal reviews a decision made by it under this Act which it sets aside under section~9(4)($c$)  of the Tribunals, Courts and~Enforcement Act 2007, or

($d$) a case involving a decision made by the First-tier Tribunal under this Act is remitted to~it by the Upper Tribunal under section~12(2)($b$)(i)  of that Act,”; and
\end{quotation}

($c$) for~“or~direction” substitute “, direction, setting aside or~remittal”.
\end{enumerate}

(5) In subsection~(3)—
\begin{enumerate}\item[]
($a$) omit the “and” at the end of paragraph~($a$);

($b$) after paragraph~($b$)  insert—
\begin{quotation}
“, and

($c$) provision made by or~under Chapter~II of Part~I of the Tribunals, Courts and~Enforcement Act 2007,”; and
\end{quotation}

($c$) for~“the Tribunal” substitute “the appropriate tribunal”.
\end{enumerate}

(6) In subsection~(4) for~“Tribunal”, in both places, substitute “appropriate tribunal”.

\medskip

13.---(1)  Section~6A (appeals from Tribunal to~Social Security Commissioner) is amended as follows.

(2) For~the heading substitute “Appeals from Pensions Appeal Tribunal for~Scotland~or~Northern Ireland”.

(3) For~subsection~(1) substitute—
\begin{quotation}
“(1) Subject to~the provisions of this section, an appeal shall lie to~the appropriate body from any decision of a Pensions Appeal Tribunal for~Scotland~or~Northern Ireland~under any of sections~1 to~5A of this Act on the ground that the decision was erroneous in point of law.

(1A) For~the purposes of this section~“the appropriate body” means—
\begin{enumerate}\item[]
($a$) in relation to~a decision of a Pensions Appeal Tribunal for~Scotland, the Upper Tribunal; and

($b$) in relation to~a decision of a Pensions Appeal Tribunal for~Northern Ireland—
\begin{enumerate}\item[]
(i) the Upper Tribunal in the case of a decision under section~5 of this Act; and

(ii) a Northern Ireland~Social Security Commissioner in any other case.”.
\end{enumerate}
\end{enumerate}
\end{quotation}

(4) In subsection~(2) for~“Tribunal” substitute “Pensions Appeal Tribunal for~Scotland~or~Northern Ireland”.

(5) In subsection~(3)—
\begin{enumerate}\item[]
($a$) for~“the appeal” substitute “an appeal under this section~to~a Northern Ireland~Social Security Commissioner”; and

($b$) for~“the Tribunal” substitute “a Pensions Appeal Tribunal for~Northern Ireland”.
\end{enumerate}

(6) In subsection~(4)—
\begin{enumerate}\item[]
($a$) after “Where” insert “an appeal is made to~a Northern Ireland~Social Security Commissioner and”;

($b$) in paragraph~($a$)(i)  for~“the Tribunal” substitute “the Pensions Appeal Tribunal for~Northern Ireland”; and

($c$) in paragraph~($b$)  for~“the Tribunal” substitute “a Pensions Appeal Tribunal for~Northern Ireland”.
\end{enumerate}

(7) After subsection~(4) insert—
\begin{quotation}
“(4A) Section~12 of the Tribunals, Courts and~Enforcement Act 2007 (proceedings on appeal to~Upper Tribunal) applies in relation to~appeals to~the Upper Tribunal under this section~as it applies in relation to~appeals to~it under section~11 of that Act, but as if references to~the First-tier Tribunal were references to~the Pensions Appeal Tribunal for~Scotland~or~Northern Ireland.”.
\end{quotation}

(8) In subsection~(5)—
\begin{enumerate}\item[]
($a$) for~“the Commissioner” substitute “the Northern Ireland~Social Security Commissioner”; and

($b$) for~“Tribunal” substitute “Pensions Appeal Tribunal for~Scotland~or~Northern Ireland”.
\end{enumerate}

(9) After subsection~(5) insert—
\begin{quotation}
“(5A) No appeal lies under this section~to~the Upper Tribunal without the leave of the Pensions Appeal Tribunal for~Scotland~or~Northern Ireland~concerned, or~of the Upper Tribunal, on an application by the party.”.
\end{quotation}

(10) In subsection~(6)—
\begin{enumerate}\item[]
($a$) after “under this section” insert “to~a Northern Ireland~Social Security Commissioner”;

($b$) in paragraph~($a$)  for~“the Tribunal” substitute “the tribunal concerned”;

($c$) in paragraph~($b$)  for~“the part of the United Kingdom for~which the Tribunal was appointed” substitute “Northern Ireland”; and

($d$) in paragraph~($c$)  for~“an appropriate Social Security Commissioner” substitute “a Northern Ireland~Social Security Commissioner”.
\end{enumerate}

(11) In subsection~(7)—
\begin{enumerate}\item[]
($a$) after “appeals” insert “to~a Northern Ireland~Social Security Commissioner”; and

($b$) for~“to~appeal” substitute “to~bring such appeals”.
\end{enumerate}

(12) In subsection~(8) for~“Commissioner” substitute “Northern Ireland~Social Security Commissioner”.

(13) Omit subsection~(9).

(14) In subsection~(10) for~“Tribunal, a Great Britain Social Security Commissioner may direct that an application or~appeal to~him” substitute “appropriate tribunal under section~1, 2, 3, 4 or~5A, the Upper Tribunal may direct that an application or~appeal to~it”.

\medskip

14.  In section~6B (redetermination etc of appeals by Pensions Appeal Tribunal)—
\begin{enumerate}\item[]
($a$) in the title for~“Pensions Appeal Tribunal” substitute “appropriate tribunal”;

($b$) for~subsection~(1) substitute—
\begin{quotation}
“(1) Subsections~(2) and~(3) apply where an application is made to—
\begin{enumerate}\item[]
($a$) a Pensions Appeal Tribunal for~Scotland~or~Northern Ireland~under section~6A(5A) of this Act, or

($b$) a person under section~6A(6)($a$)  of this Act,
\end{enumerate}
for~leave to~appeal from a decision of the tribunal concerned.”;
\end{quotation}

($c$) in subsection~(2)—
\begin{enumerate}\item[]
(i) for~“the person” substitute “the tribunal or~person to~whom the application is made”;

(ii) for~“he” substitute “that tribunal or~person”;

(iii) for~“the Tribunal” substitute “the tribunal concerned”; and

(iv)  for~“constituted Tribunal” substitute “constituted Pensions Appeal Tribunal for~Scotland~or~Northern Ireland”; and
\end{enumerate}

($d$) in subsection~(3)—
\begin{enumerate}\item[]
(i) for~“the person” substitute “the tribunal or~person to~whom the application is made”; and

(ii) for~“constituted Tribunal” substitute “constituted Pensions Appeal Tribunal for~Scotland~or~Northern Ireland”; and
\end{enumerate}

($e$) after that subsection~insert—
\begin{quotation}
“(4) Subsection~(5) applies where an application is made to~the First-tier Tribunal for~permission to~appeal to~the Upper Tribunal from any decision of the First-tier Tribunal under this Act.

(5) If each of those who would be parties to~the appeal if permission were granted expresses the view that the decision was erroneous in point of the law, the First-tier Tribunal shall set aside the decision and~refer the case for~determination by a differently constituted First-tier Tribunal.”.
\end{quotation}
\end{enumerate}

\medskip

15.  In section~6C (appeals from Commissioner)—
\begin{enumerate}\item[]
($a$) in subsections~(1) to~(4) for~“a Commissioner” substitute “a Northern Ireland~Social Security Commissioner”; and

($b$) in subsection~(3)($a$)  for~“the Tribunal” substitute “the tribunal concerned”.
\end{enumerate}

\medskip

16.  In section~6D (procedure in proceedings before Commissioner)—
\begin{enumerate}\item[]
($a$) in subsections~(1), (3), (4) and~(6)($a$)  for~“a Commissioner” substitute “a Northern Ireland~Social Security Commissioner”;

($b$) in subsection~(1)—
\begin{enumerate}\item[]
(i) for~“section~16 of the Social Security Act 1998” substitute “Article~16 of the Social Security (Northern Ireland) Order 1998\footnote{S.I.~1998/1506 (N.I.~10).}”; and

(ii) for~“that Act” substitute “that Order”;
\end{enumerate}

($c$) in subsection~(2)($a$)—
\begin{enumerate}\item[]
(i) omit “or, in Scotland, by the Secretary of State”; and

(ii) for~“Commissioners” substitute “Northern Ireland~Social Security Commissioners”;
\end{enumerate}

($d$) in subsection~(5)—
\begin{enumerate}\item[]
(i) for~“the Chief Commissioner” substitute “the Chief Social Security Commissioner appointed under the Social Security Administration (Northern Ireland) Act 1992\footnote{1992 c.~8.}”; and

(ii) for~“Commissioners”, in each place, substitute “Northern Ireland~Social Security Commissioners”;
\end{enumerate}

($e$) in subsection~(8) omit “England~and~Wales or”; and

($f$) omit subsection~(9).
\end{enumerate}

\medskip

17.  In section~8(1), (3) and~(5) (time limit for~appeals) for~“the Tribunal” substitute “a Pensions Appeal Tribunal for~Scotland~or~Northern Ireland”.

\medskip

18.  In section~9 (notices) for~“Tribunal” substitute “appropriate tribunal”.

\medskip

19.  In section~11A(5) (regulations) omit paragraph~($b$)  (together with the “or” immediately before it).

\medskip

20.  In section~12 (interpretation)—
\begin{enumerate}\item[]
($a$) before the definition of “detention” insert—
\begin{quotation}
““the appropriate tribunal” means the First-tier Tribunal or~a Pensions Appeal Tribunal for~Scotland~or~Northern Ireland~(and~see paragraphs~6 to~6B of the Schedule~for~determining which of those tribunals hears an appeal under this Act);”;
\end{quotation}

($b$) omit the definition of “Chief Commissioner”;

($c$) omit the definition of “Commissioner”;

($d$) omit the definition of “Great Britain Social Security Commissioner”; and

($e$) in the definition of “Northern Ireland~Social Security Commissioner” at the end insert “, and~includes a tribunal of Commissioners constituted under section~6D(5) of this Act”.
\end{enumerate}

\medskip

21.---(1)  The Schedule~(constitution, jurisdiction and~procedure of Pensions Appeal Tribunals) is amended as follows.

(2) In paragraph~1—
\begin{enumerate}\item[]
($a$) omit sub-paragraph~(1);

($b$) after sub-paragraph~(3) insert—
\begin{quotation}
“(3A) In this Schedule~“Tribunal” means a Pensions Appeal Tribunal for~Scotland~or~Northern Ireland~constituted in accordance with the provisions of this Schedule.”; and
\end{quotation}

($c$) omit sub-paragraph~(4).
\end{enumerate}

(3) In paragraph~2—
\begin{enumerate}\item[]
($a$) omit sub-paragraph~(1)($a$);

($b$) in sub-paragraph~(2A) omit “(3A),”;

($c$) omit sub-paragraph~(3A); and

($d$) in sub-paragraph~(4) for~“sub-paragraphs~(3A) and~(3B)” substitute “sub-paragraph~(3B)”.
\end{enumerate}

(4) In paragraph~2B—
\begin{enumerate}\item[]
($a$) in sub-paragraph~(1) for~“each part of the United Kingdom” substitute “Scotland~or~Northern Ireland”;

($b$) omit sub-paragraph~(2)($a$); and

($c$) in sub-paragraphs~(5) and~(6)—
\begin{enumerate}\item[]
(i) for~“any part of the United Kingdom” substitute “Scotland~or~Northern Ireland”; and

(ii) for~“that part of the United Kingdom” substitute “Scotland~or~Northern Ireland”.
\end{enumerate}
\end{enumerate}

(5) In paragraph~3A—
\begin{enumerate}\item[]
($a$) for~“any part of the United Kingdom” substitute “Scotland~or~Northern Ireland”; and

($b$) in paragraph~($a$)  for~“that part of the United Kingdom” substitute “Scotland~or~Northern Ireland”.
\end{enumerate}

(6) In paragraph~3B—
\begin{enumerate}\item[]
($a$) for~“any part of the United Kingdom” substitute “Scotland~or~Northern Ireland”; and

($b$) for~“such Tribunals in that part of the United Kingdom” substitute “a Pensions Appeal Tribunal for~Scotland~or~Northern Ireland”.
\end{enumerate}

(7) In paragraph~3C(2)—
\begin{enumerate}\item[]
($a$) for~“any part of the United Kingdom” substitute “Scotland~or~Northern Ireland”; and

($b$) omit paragraph~($a$).
\end{enumerate}

(8) In paragraph~5—
\begin{enumerate}\item[]
($a$) in sub-paragraph~(1)($a$)  for~“Pensions Appeals Tribunals” substitute “Tribunals”; and

($b$) omit sub-paragraph~(1A)($a$).
\end{enumerate}

(9) In paragraph~6—
\begin{enumerate}\item[]
($a$) for~“the Tribunal” substitute “the appropriate tribunal”;

($b$) for~“such one of the Tribunals appointed for~England~as may be prescribed by or~under rules made for~those Tribunals under this Schedule” substitute “the First-tier Tribunal”;

($c$) for~“a Tribunal” substitute “the appropriate tribunal”; and

($d$) for~“another Tribunal” substitute “another such tribunal”.
\end{enumerate}

(10) In paragraph~6B—
\begin{enumerate}\item[]
($a$) for~“the Tribunal” substitute “the appropriate tribunal”; and

($b$) for~“a Tribunal appointed for~another part of the United Kingdom” substitute “another appropriate tribunal”.
\end{enumerate}

(11) In paragraph~7 for~“such appeal” substitute “appeal to~a Tribunal”.

(12) In paragraph~7B, omit sub-paragraph~(1).

\section*{\itshape Administration of Justice Act 1960}

22.  In section~12(1)($b$)  of the Administration of Justice Act 1960\footnote{8 \& 9 Eliz.~2 c.~65. Section~12(1)($b$) was amended by paragraph~10 of Schedule~6 to the Mental Incapacity Act 2005 (c.~9).} (publication of information relating to~proceedings in private) for~“a Mental Health Review Tribunal or~to” substitute “the First-tier Tribunal, the Mental Health Review Tribunal for~Wales or”.

\section*{\itshape Parliamentary Commissioner Act 1967}

23.  The Parliamentary Commissioner Act 1967\footnote{1967 c.~13. Section~11B and~paragraph~6C of Schedule~3 were inserted by section~10(1) and~(2) of the Criminal Injuries Compensation Act 1995 (c.~53). Schedule~4 was inserted by section~1(3) of the Parliamentary Commissioner Act 1994 (c.~14) and~substituted by article~3 of, and~Schedule~2 to, the Parliamentary Commissioner Order 2007 (S.I.~2007/3470).} is amended as follows.

\medskip

24.  Omit section~11B(2)($b$)  and~(3)($b$)  (the Criminal Injuries Compensation Scheme).

\medskip

25.  Omit paragraph~6C of Schedule~3 (matters not subject to~investigation).

\medskip

26.---(1)  Schedule~4 (relevant tribunals for~the purposes of section~5(7)) is amended as follows.

(2) Omit the entries relating to—
\begin{enumerate}\item[]
($a$) the Care Standards Tribunal constituted under section~9 of the Protection of Children Act 1999; and

($b$) the Special Educational Needs and~Disability Tribunal constituted under section~333 of the Education Act 1996.
\end{enumerate}

(3) In the entry relating to~the Mental Health Review Tribunals, for~“Tribunals” substitute “Tribunal for~Wales”.

\section*{\itshape Local Authority Social Services Act 1970}

27.  In Schedule~1 to~the Local Authority Social Services Act 1970\footnote{1970 c.~42. The entry relating to the Mental Health Act 1983 was amended by paragraph~27 of Schedule~4 to the Mental Health Act 1983 (c.~20) and~section~55(2) of the Children Act 2004 (c.~31).} (social services functions), in the entry relating to~the Mental Health Act 1983, for~“Mental Health Review Tribunals” substitute “the First-tier Tribunal or~the Mental Health Review Tribunal for~Wales”.

\section*{\itshape House of Commons Disqualification Act 1975}

28.---(1)  Schedule~1 to~the House of Commons Disqualification Act 1975\footnote{1975 c.~24. The entry for~an Asylum Support Adjudicator~was inserted by paragraph~71($b$) of Schedule~14 to the Immigration and~Asylum Act 1999 (c.~33).} (disqualifying offices) is amended as follows.

(2) In Part~I (judicial offices) omit the first entry beginning “Chief or~other Child Support Commissioner”.

(3) In Part~III (other offices) omit the entries relating to—
\begin{enumerate}\item[]
($a$) an adjudicator~appointed under section~5 of the Criminal Injuries Compensation Act 1995;

($b$) an Asylum Support Adjudicator; and

($c$) the President of the Special Educational Needs Tribunal, or~a member of a panel of persons appointed to~act as chairman or~other member of that Tribunal.
\end{enumerate}

\section*{\itshape Northern Ireland~Assembly Disqualification Act 1975}

29.---(1)  Schedule~1 to~the Northern Ireland~Assembly Disqualification Act 1975\footnote{1975 c.~25. The entry for~an Asylum Support Adjudicator~was inserted by paragraph~72($b$) of Schedule~14 to the Immigration and~Asylum Act 1999 (c.~33).} (disqualifying offices) is amended as follows.

(2) In Part~I (judicial offices) omit the first entry beginning “Chief or~other Child Support Commissioner”.

(3) In Part~III (other offices) omit the entry relating to~an Asylum Support Adjudicator.

\section*{\itshape Vaccine Damage Payments Act 1979}

30.  The Vaccine Damage Payments Act 1979\footnote{1979 c.~17. Sections~3A and~4 were substituted by sections~45 and~46 of the Social Security Act 1998 (c.~14) respectively. Section~4(1A) was inserted by section~57 of the Welfare Reform Act 2007 (c.~5) and~sections~4(2) and~(3) were amended by paragraph~1(1) to (3) of Schedule~7 to the 2007 Act. Section~7A was inserted by section~47 of the 1998 Act and~subsection~(1)($a$) was amended by paragraph~1(1) and~(4) of Schedule~7 to the 2007 Act. Section~9A is to be inserted from a date to be appointed by paragraph~1(1) and~(8) of Schedule~7 to the 2007 Act.} is amended as follows.

\medskip

31.  In section~3A(1) (decisions reversing earlier decisions) for~“an appeal tribunal” substitute “a tribunal”.

\medskip

32.  In section~4 (appeals to~appeal tribunals) as it has effect before the commencement of section~57(2) of, and~paragraph~1(2) and~(3) of Schedule~7 to, the Welfare Reform Act 2007—
\begin{enumerate}\item[]
($a$) in subsection~(1) for~“an appeal tribunal” substitute “the First-tier Tribunal”;

($b$) omit subsection~(2)($b$)  (together with the “and” immediately before it); and

($c$) in subsection~(4) for~“an appeal tribunal” substitute “the First-tier Tribunal”.
\end{enumerate}

\medskip

33.  In section~4 (appeals to~appeal tribunals) as it has effect after the commencement of section~57(2) of, and~paragraph~1(2) and~(3) of Schedule~7 to, the Welfare Reform Act 2007—
\begin{enumerate}\item[]
($a$) in subsection~(1A)—
\begin{enumerate}\item[]
(i) for~“In subsection~(1) the reference” substitute “In this section~any reference”; and

(ii) in paragraph~($b$)  for~“an appeal tribunal constituted under Chapter~I of Part~I of the Social Security Act 1998” substitute “the First-tier Tribunal”;
\end{enumerate}

($b$) in subsection~(2)—
\begin{enumerate}\item[]
(i) for~“an appeal tribunal constituted under Chapter~I of Part~I of the Social Security Act 1998” substitute “the First-tier Tribunal”; and

(ii) omit paragraph~($b$)  (together with the “and” immediately before it); and
\end{enumerate}

($c$) in subsection~(4) for~“an appeal tribunal” substitute “an appropriate appeal tribunal”.
\end{enumerate}

\medskip

34.  In section~7A(1) (correction of errors and~setting aside of decisions)—
\begin{enumerate}\item[]
($a$) in subsection~(1)—
\begin{enumerate}\item[]
(i) in paragraph~($a$), as it has effect both before and~after the commencement of paragraph~1(4) of Schedule~7 to~the Welfare Reform Act 2007, for~“3, 3A or~4” substitute “3 or~3A”;

(ii) in paragraph~($a$), as it has effect after the commencement of that provision, omit “, other than a decision of an appeal tribunal constituted under Chapter~I of Part~II of the Social Security (Northern Ireland) Order 1998”; and

(iii) omit paragraph~($b$)  (together with the “and” immediately before it), as it has effect both before and~after the commencement of that provision; and
\end{enumerate}

($b$) in subsection~(2) omit “or~set aside decisions”.
\end{enumerate}

\medskip

35.  Omit section~9A (interpretation).

\medskip

36.  In section~12(3) (financial provision)—
\begin{enumerate}\item[]
($a$) omit paragraph~($b$); and

($b$) in paragraph~($c$)  omit “or~tribunal”.
\end{enumerate}

\section*{\itshape Judicial Pensions Act 1981}

37.  In section~13 of the Judicial Pensions Act 1981\footnote{1981 c.~20. Section~13(1A) and~(7) were inserted by paragraphs~109 and~113(1), (2) and~(3) of Part~I of Schedule~4 to the Constitutional Reform Act 2005 (c.~4).} (Social Security Commissioners) omit subsections~(1A)($a$)  and~(7).

\section*{\itshape Forfeiture Act 1982}

38.  In section~4 of the Forfeiture Act 1982\footnote{1982 c.~34. Subsections~(1A) to (1H) were inserted by section~76(2) of the Social Security Act 1986 (c.~50). Subsection~(2)($b$) was amended by paragraph~11(1) of Schedule~7 to the Social Security Act 1998 (c.~14). In subsection~(5) the definition of “Commissioner” was amended by paragraph~11(2) of Schedule~7 to the 1998 Act.} (Commissioner to~decide whether rule applies to~social security benefits)—
\begin{enumerate}\item[]
($a$) in the heading for~“Commissioner” substitute “Upper Tribunal”;

($b$) in subsections~(1), (1A), (1G) and~(1H) for~“a Commissioner”, in each place, substitute “the Upper Tribunal”;

($c$) in subsections~(1A), (1B) and~(1E) for~“the Commissioner”, in each place, substitute “the Upper Tribunal”;

($d$) in subsection~(1B) for~“he” substitute “it”;

($e$) in subsection~(1E) for~“he may direct that his” substitute “the Upper Tribunal may direct that its”;

($f$) in subsection~(2)—
\begin{enumerate}\item[]
(i) for~the words from the beginning to~“expedient” substitute “Tribunal Procedure Rules may make provision”;

(ii) for~“the regulations” substitute “the rules”; and

(iii) omit paragraph~($b$)  (together with the “and” immediately before it);
\end{enumerate}

($g$) omit subsections~(3) and~(4); and

($h$) in subsection~(5) omit the definition of “Commissioner”.
\end{enumerate}

\section*{\itshape Mental Health Act 1983}

39.  The Mental Health Act 1983\footnote{1983 c.~20. Section~21(3) was inserted by section~37 of the Mental Health Act 2007 (c.~12). Section~65(1A) was substituted by paragraph~107 of Schedule~1 to the Health Authorities Act 1995 (c.~17) and~by section~38 of the 2007 Act. Sections~68 and~68A were substituted by section~37 of the 2007 Act. Section~72(1)($c$) was inserted by paragraph~21 of Schedule~3 to the 2007 Act. Section~72(3A) was inserted by paragraph~10 of Schedule~1 to the Mental Health (Patients in the Community) Act 1995 (c.~52) and~substituted by paragraph~21 of Schedule~3 to the 2007 Act. Section~73(1) was substituted by article~4 of the Mental Health Act 1983 (Remedial) Order 2001 (S.I.~2001/3712). Section~74(5A) was inserted by section~295 of the Criminal Justice Act 2003 (c.~44). Section~79(7) was inserted by paragraph~107 of Schedule~1 to the Health Authorities Act 1995 (c.~17) and~substituted by section~38 of the 2007 Act. Section~132A was inserted by paragraph~30 of Schedule~3 to the 2007 Act.} is amended as follows.

\medskip

40.  In section~21(3) (patients absent without leave) for~“a Mental Health Review Tribunal” substitute “the appropriate tribunal”.

\medskip

41.  In section~41(3)($b$)  (power of higher courts to~restrict discharge from hospital) for~“a Mental Health Review Tribunal” substitute “the appropriate tribunal”.

\medskip

42.  In section~50(1) (prisoners under sentence) for~“a Mental Health Review Tribunal” substitute “the appropriate tribunal”.

\medskip

43.  In section~51(3) (detained persons) for~“a Mental Health Review Tribunal” substitute “the appropriate tribunal”.

\medskip

44.  In section~53(2) (civil prisoners and~persons detained under the Immigration Acts) for~“a Mental Health Review Tribunal” substitute “the appropriate tribunal”.

\medskip

45.  In section~65 (Mental Health Review Tribunals)—
\begin{enumerate}\item[]
($a$) in the title for~“Tribunals” substitute “Tribunal for~Wales”;

($b$) in subsection~(1) (as substituted by section~38(2) of the Mental Health Act 2007\footnote{2007 c.~12.}) for~the words from “be” to~the end substitute “be a Mental Health Review Tribunal for~Wales.”;

($c$) in subsection~(1A) (as substituted by section~38(2) of that Act) for~“the Mental Health Review Tribunals” substitute “that tribunal”;

($d$) in subsection~(2) for~“Mental Health Review Tribunals” substitute “the Mental Health Review Tribunal for~Wales”;

($e$) in subsection~(3) for~“a Mental Health Review Tribunal”, in both places, substitute “the Mental Health Review Tribunal for~Wales”; and

($f$) for~subsection~(4) substitute—
\begin{quotation}
“(4) The Welsh Ministers may pay to~the members of the Mental Health Review Tribunal for~Wales such remuneration and~allowances as they may determine, and~defray the expenses of that tribunal to~such amount as they may determine, and~may provide for~that tribunal such officers and~servants, and~such accommodation, as that tribunal may require.”
\end{quotation}
\end{enumerate}

\medskip

46.  In section~66 (applications to~tribunals)—
\begin{enumerate}\item[]
($a$) in subsection~(1) for~“a Mental Health Review Tribunal” substitute “the appropriate tribunal”; and

($b$) after subsection~(3) insert—
\begin{quotation}
“(4) In this Act “the appropriate tribunal” means the First-tier Tribunal or~the Mental Health Review Tribunal for~Wales.

(5) For~provision determining to~which of those tribunals applications by or~in respect of a patient under this Act shall be made, see section~77(3) and~(4) below.”
\end{quotation}
\end{enumerate}

\medskip

47.  In section~67(1) (references to~tribunals by Secretary of State concerning Part~II patients) for~“a Mental Health Review Tribunal” substitute “the appropriate tribunal”.

\medskip

48.  In section~68(2), (6) and~(7) (duty of managers of hospitals to~refer cases to~tribunal) for~“a Mental Health Review Tribunal” substitute “the appropriate tribunal”.

\medskip

49.  In section~68A(5) (power to~reduce periods under section~68) for~“a Mental Health Review Tribunal” substitute “the appropriate tribunal”.

\medskip

50.  In section~69(1) and~(2) (applications to~tribunals concerning patients subject to~hospital and~guardianship orders) for~“a Mental Health Review Tribunal” substitute “the appropriate tribunal”.

\medskip

51.  In section~70 (applications to~tribunals concerning restricted patients) for~“a Mental Health Review Tribunal” substitute “the appropriate tribunal”.

\medskip

52.  In section~71(1) and~(2) (references by Secretary of State concerning restricted patients) for~“a Mental Health Review Tribunal” substitute “the appropriate tribunal”.

\medskip

53.  In section~72 (powers of tribunals)—
\begin{enumerate}\item[]
($a$) in subsections~(1), (4) and~(6) for~“a Mental Health Review Tribunal” substitute “the appropriate tribunal”;

($b$) in subsections~(1)($a$)  to~($c$)  and~(4) for~“they are” substitute “it is”;

($c$) in subsection~(3) for~“do not” substitute “does not”;

($d$) in subsection~(3A) for~“they think” substitute “it thinks”; and

($e$) in subsection~(6) for~“such a tribunal” substitute “the appropriate tribunal”.
\end{enumerate}

\medskip

54.  In section~73 (power to~discharge restricted patients)—
\begin{enumerate}\item[]
($a$) in subsection~(1)—
\begin{enumerate}\item[]
(i) for~“a Mental Health Review Tribunal” substitute “the appropriate tribunal”;

(ii) for~“such a tribunal” substitute “the appropriate tribunal”; and

(iii) for~“the tribunal are”, in both places, substitute “the tribunal is”; and
\end{enumerate}

($b$) in subsection~(7) for~“their satisfaction” substitute “its satisfaction”.
\end{enumerate}

\medskip

55.  In section~74 (restricted patients subject to~restriction directions)—
\begin{enumerate}\item[]
($a$) in subsection~(1)—
\begin{enumerate}\item[]
(i) for~“a Mental Health Review Tribunal” substitute “the appropriate tribunal”;

(ii) for~“such a tribunal” substitute “the appropriate tribunal”;

(iii) for~“their” substitute “its”; and

(iv)  for~“they notify” substitute “the tribunal notifies”;
\end{enumerate}

($b$) in subsections~(2) and~(4) for~“notify” substitute “notifies”; and

($c$) in subsections~(3), (4) and~(5A) for~“the tribunal have” substitute “the tribunal has”.
\end{enumerate}

\medskip

56.  In section~75 (applications and~references concerning conditionally discharged restricted patients)—
\begin{enumerate}\item[]
($a$) in subsections~(1)($a$)  and~(2) for~“a Mental Health Review Tribunal” substitute “the appropriate tribunal”; and

($b$) in subsection~(3) for~“give” substitute “gives”.
\end{enumerate}

\medskip

57.  In section~76(1) (visiting and~examination of patients) for~“a Mental Health Review Tribunal” substitute “the appropriate tribunal”.

\medskip

58.  In section~77 (general provisions concerning tribunal applications)—
\begin{enumerate}\item[]
($a$) in subsection~(1) for~“a Mental Health Review Tribunal by or~in respect of a patient” substitute “the appropriate tribunal by or~in respect of a patient under this Act”;

($b$) in subsection~(2) for~“a Mental Health Review Tribunal” substitute “the appropriate tribunal”;

($c$) in subsection~(3)—
\begin{enumerate}\item[]
(i) for~“a Mental Health Review Tribunal” substitute “a tribunal”;

(ii) in paragraph~($a$), for~“to~the tribunal for~the area in which that hospital is situated” substitute “to~the First-tier Tribunal where that hospital is in England~and~to~the Mental Health Review Tribunal for~Wales where that hospital is in Wales”;

(iii) in paragraph~($b$), for~“to~the tribunal for~the area in which the responsible hospital is situated” substitute “to~the First-tier Tribunal where the responsible hospital is in England~and~to~the Mental Health Review Tribunal for~Wales where that hospital is in Wales”; and

(iv)  in paragraph~($c$), for~“to~the tribunal for~the area in which the patient is residing” substitute “to~the First-tier Tribunal where the patient resides in England~and~to~the Mental Health Review Tribunal for~Wales where the patient resides in Wales”; and
\end{enumerate}

($d$) in subsection~(4) for~“to~the tribunal for~the area in which the patient resides” substitute “to~the First-tier Tribunal where the patient resides in England~and~to~the Mental Health Review Tribunal for~Wales where the patient resides in Wales”.
\end{enumerate}

\medskip

59.  Section~78 (procedure of tribunals) is amended as follows.

(1) In the heading, for~“tribunals” substitute “Mental Health Review Tribunal for~Wales”.

(2) In subsection~(1)—
\begin{enumerate}\item[]
($a$) for~“Mental Health Review Tribunals” substitute “the Mental Health Review Tribunal for~Wales”; and

($b$) for~“such tribunals” substitute “that tribunal”.
\end{enumerate}

(3) In subsection~(2)—
\begin{enumerate}\item[]
($a$) for~“a tribunal”, in each place, substitute “the tribunal”;

($b$) in paragraph~($a$)  for~the words from “by that or” to~the end substitute “under this Act by the tribunal or~the First-tier Tribunal”;

($c$) for~paragraph~($b$)  substitute—
\begin{quotation}
“($b$) for~the transfer of proceedings to~or~from the Mental Health Review Tribunal for~Wales in any case where, after the making of the application, the patient is moved into~or~out of Wales;”; and
\end{quotation}

($d$) in paragraph~($j$)—
\begin{enumerate}\item[]
(i) for~“tribunals” substitute “tribunal”; and

(ii) for~“their” substitute “its”.
\end{enumerate}
\end{enumerate}

(4) In subsection~(3)—
\begin{enumerate}\item[]
($a$) for~“Mental Health Review Tribunals” substitute “the Mental Health Review Tribunal for~Wales”; and

($b$) for~“such tribunals” substitute “that tribunal”.
\end{enumerate}

(5) In subsection~(4)—
\begin{enumerate}\item[]
($a$) for~“a tribunal”, in each place, substitute “the tribunal”; and

($b$) for~paragraph~($b$)  substitute—
\begin{quotation}
“($b$) for~the transfer of proceedings to~or~from the tribunal in any case where, after the making of a reference or~application in accordance with section~71(4) or~77(4) above, the patient begins or~ceases to~reside in Wales.”
\end{quotation}
\end{enumerate}

(6) In subsection~(6) for~“a Mental Health Review Tribunal” substitute “the Mental Health Review Tribunal for~Wales”.

(7) In subsection~(7) for~“A Mental Health Review Tribunal” substitute “The Mental Health Review Tribunal for~Wales”.

(8) Omit subsection~(8).

(9) In subsection~(9) for~“a Mental Health Review Tribunal” substitute “the Mental Health Review Tribunal for~Wales”.

\medskip

60.  After section~78 insert—
\begin{quotation}
\subsection*{“78A. Appeal from the Mental Health Review Tribunal for~Wales to~the Upper Tribunal}

(1)  A party to~any proceedings before the Mental Health Review Tribunal for~Wales may appeal to~the Upper Tribunal on any point of law arising from a decision made by the Mental Health Review Tribunal for~Wales in those proceedings.

(2) An appeal may be brought under subsection~(1) above only if, on an application made by the party concerned, the Mental Health Review Tribunal for~Wales or~the Upper Tribunal has given its permission for~the appeal to~be brought.

(3) Section~12 of the Tribunals, Courts and~Enforcement Act 2007 (proceedings on appeal to~the Upper Tribunal) applies in relation to~appeals to~the Upper Tribunal under this section~as it applies in relation to~appeals to~it under section~11 of that Act, but as if references to~the First-tier Tribunal were references to~the Mental Health Review Tribunal for~Wales.”
\end{quotation}

\medskip

61.  In section~79 (interpretation of Part~V) omit subsection~(7).

\medskip

62.  In section~86(3) (removal of alien patients) for~“a Mental Health Review Tribunal” substitute “the appropriate tribunal”.

\medskip

63.  In section~132(1)($b$)  (duty of managers of hospitals to~give information to~detained patients) for~“Mental Health Review Tribunal” substitute “tribunal”.

\medskip

64.  In section~132A(1)($b$)  (duty of managers of hospitals to~give information to~community patients) for~“Mental Health Review Tribunal” substitute “tribunal”.

\medskip

65.  In section~134(3) (correspondence of patients)—
\begin{enumerate}\item[]
($a$) in paragraph~($d$)  for~“a Mental Health Review Tribunal” substitute “the First-tier Tribunal or~the Mental Health Review Tribunal for~Wales”; and

($b$) after paragraph~($h$)  insert—
\begin{quotation}
“and~for~the purposes of paragraph~($d$)  above the reference to~the First-tier Tribunal is a reference to~that tribunal so far as it is acting for~the purposes of any proceedings under this Act or~paragraph~5(2) of the Schedule~to~the Repatriation of Prisoners Act 1984\footnote{1984 c.~47.}.”
\end{quotation}
\end{enumerate}

\medskip

66.  In section~145(1) (interpretation) after the definition of “application for~admission for~treatment” insert—
\begin{quotation}
““the appropriate tribunal” has the meaning given by section~66(4) above;”.
\end{quotation}

\medskip

67.  In Schedule~2 (Mental Health Review Tribunals)—
\begin{enumerate}\item[]
($a$) in the title for~“Tribunals” substitute “Tribunal for~Wales”;

($b$) in paragraph~1 for~“Each of the Mental Health Review Tribunals” substitute “The Mental Health Review Tribunal for~Wales”;

($c$) in paragraph~2 for~“Mental Health Review Tribunals” substitute “the Mental Health Review Tribunal for~Wales”;

($d$) in paragraph~2A for~“a Mental Heath Review Tribunal” substitute “the Mental Health Review Tribunal for~Wales”;

($e$) in paragraph~3, as it has effect before the commencement of section~38(6) of the Mental Health Act 2007, for~“each Mental Health Review Tribunal” substitute “the Mental Health Review Tribunal for~Wales”;

($f$) in paragraph~3, as it has effect after the commencement of section~38(6) of that Act, omit sub-paragraph~(1);

($g$) in paragraph~4 for~“a Mental Health Review Tribunal” substitute “the Mental Health Review Tribunal for~Wales”; and

($h$) omit paragraph~5.
\end{enumerate}

\medskip

68.  In paragraph~34(4) of Schedule~5 (transitional and~saving provisions) for~“a Mental Health Review Tribunal” substitute “the appropriate tribunal”.

\section*{\itshape Repatriation of Prisoners Act 1984}

69.  In paragraph~5 of the Schedule~to~the Repatriation of Prisoners Act 1984 (operation of mental health legislation in relation to~the prisoner)—
\begin{enumerate}\item[]
($a$) in sub-paragraph~(2) for~“a Mental Health Review Tribunal” substitute “the appropriate tribunal”; and

($b$) after that sub-paragraph~insert—
\begin{quotation}
“(2A) For~the purposes of sub-paragraph~(2) above “the appropriate tribunal” means—
\begin{enumerate}\item[]
($a$) the First-tier Tribunal, in any case where the prisoner is detained in England;

($b$) the Mental Health Review Tribunal for~Wales, in any case where the prisoner is detained in Wales; and

($c$) the Mental Health Review Tribunal for~Northern Ireland, in any case where the prisoner is detained in Northern Ireland.”
\end{enumerate}
\end{quotation}
\end{enumerate}

\section*{\itshape\sloppy\hbadness=2875 Disabled Persons (Services, Consultation and Representation) Act 1986}

70.  In section~7(2)($a$)  of the Disabled Persons (Services, Consultation and~Representation) Act 1986\footnote{1986 c.~33.} (persons discharged from hospital) for~“a Mental Health Review Tribunal” substitute “the First-tier Tribunal or~the Mental Health Review Tribunal for~Wales”.

\section*{\itshape Children Act 1989}

71.  The Children Act 1989\footnote{1989 c.~41. Section~65(3)($b$) was substituted and~section~65A was inserted by paragraph~14 of Schedule~4 to the Care Standards Act 2000 (c.~14). Sections~79B,~79H and~79M were inserted by section~79(1) of that Act.} is amended as follows.

\medskip

72.  In section~65(3)($b$)  (persons disqualified from carrying on, or~being employed in, children’s homes) for~“Tribunal established under section~9 of the Protection of Children Act 1999” substitute “First-tier Tribunal”.

\medskip

73.  In section~65A(1) (appeal against refusal of authority to~give consent under section~65) for~“Tribunal established under section~9 of the Protection of Children Act 1999” substitute “First-tier Tribunal”.

\medskip

74.  Omit section~79B(8) (other definitions, etc).

\medskip

75.  In section~79H(2) (suspension of registration) before “Tribunal” insert “First-tier”.

\medskip

76.  In section~79M(1) and~(2) (appeals) before “Tribunal” insert “First-tier”.

\section*{\itshape Child Support Act 1991}

77.  The Child Support Act 1991\footnote{1991 c.~48.} is amended as follows.

\medskip

78.  In section~16(1A)($c$)  (revision of decisions)\footnote{Section~16 was substituted by section~40 of the Social Security Act 1998 (c.~14). Subsection~(1A) was inserted by section~8 of the Child Support, Pensions and~Social Security Act 2000 (c.~19).} for~“an appeal tribunal” substitute “the First-tier Tribunal”.

\medskip

79.  In section~17(1)(decisions superseding earlier decisions)\footnote{Section~17 was substituted by section~41 of the Social Security Act 1998 (c.~14). Paragraphs ($c$) to ($e$) of subsection~17(1) were substituted for previous paragraph~($c$) by section~9 of, and~Part I of Schedule~9 to, the Child Support, Pensions and~Social Security Act 2000 (c.~19).}—
\begin{enumerate}\item[]
($a$) in paragraphs~($b$)  and~($d$)  for~“an appeal tribunal” substitute “the First-tier Tribunal”; and

($b$) in paragraph~($e$)  for~“a Child Support Commissioner” substitute “the Upper Tribunal”.
\end{enumerate}

\medskip

80.  In section~20 (appeals to~appeal tribunals)\footnote{Section~20 was substituted by section~42 of the Social Security Act 1998 (c.~14).} as it has effect without the substitution made by section~10 of the Child Support, Pensions and~Social Security Act 2000—
\begin{enumerate}\item[]
($a$) in the heading for~“appeal tribunals” substitute “First-tier Tribunal”;

($b$) in subsections~(1) to~(3) and~(7) for~“an appeal tribunal” substitute “the First-tier Tribunal”; and

($c$) in subsection~(5) omit paragraph~($b$)  (together with the “and” immediately before it).
\end{enumerate}

\medskip

81.  In section~20 (appeals to~appeal tribunals) as substituted by section~10 of that Act—
\begin{enumerate}\item[]
($a$) in the heading for~“appeal tribunals” substitute “First-tier Tribunal”;

($b$) in subsections~(1), (7) and~(7A)\footnote{Subsection~(7A) was inserted by paragraph~1(6) of Schedule~7 to the Child Maintenance and~Other Payments Act 2008 (c.~6).} for~“an appeal tribunal” substitute “the First-tier Tribunal”;

($c$) in subsection~(4) omit paragraph~($b$)  (together with the “and” immediately before it); and

($d$) in subsection~(8) for~“appeal tribunal” substitute “First-tier Tribunal”.
\end{enumerate}

\medskip

82.  Omit section~22 (Child Support Commissioners)\footnote{Section~22 was amended by paragraph~22(1) and~(2) of Schedule~10 to the Tribunals, Courts and~Enforcement Act 2007 (c.~15), paragraph~29 of Schedule~7 to the Social Security Act 1998 (c.~14) and~article~2(1) of, and~the Schedule~to, the Transfer of Functions (Lord Advocate and~Secretary of State) Order 1999 (S.I.~1999/678). Functions under subsection~(3) were further transferred to Scottish Ministers under the Scotland~Act 1998 (Transfer of Functions to the Scottish Ministers etc) Order 1999 (S.I.~1999/1750).}.

\medskip

83.  In section~23(3) (Child Support Commissioners for~Northern Ireland) omit “, subject to~the modifications set out in paragraph~8”.

\medskip

84.  In section~23A (redetermination of appeals)\footnote{Section~23A was inserted by section~11 of the Child Support, Pensions and~Social Security Act 2000 (c.~19).}—
\begin{enumerate}\item[]
($a$) in subsection~(1) for~the words from “to~a person” to~the end substitute “to~the First-tier Tribunal for~permission to~appeal to~the Upper Tribunal from any decision of the First-tier Tribunal under section~20”;

($b$) omit subsection~(2); and

($c$) in subsection~(3)—
\begin{enumerate}\item[]
(i) for~“the person” substitute “the First-tier Tribunal”; and

(ii) for~“tribunal” substitute “First-tier Tribunal”.
\end{enumerate}
\end{enumerate}

\medskip

85.  In section~24 (appeal to~Child Support Commissioner)\footnote{Section~24 was amended by paragraph~30 of Schedule~7 to the Social Security Act 1998 (c.~19) and~the Schedule~to the Transfer of Functions (Lord Advocate and~Secretary of State) Order 1999 (S.I.~1999/678). Functions under subsection~(9) were transferred further to Scottish Ministers under the Scotland~Act 1998 (Transfer of Functions to the Scottish Ministers etc) Order 1999 (S.I.~1999/1750).}—
\begin{enumerate}\item[]
($a$) for~the heading substitute “Appeals to~Upper Tribunal”;

($b$) for~subsection~(1), as it has effect before the substitution made by paragraph~16(2) of Schedule~3 to~the Child Maintenance and~Other Payments Act 2008, substitute—
\begin{quotation}
“(1) Each of the following may appeal to~the Upper Tribunal under section~11 of the Tribunals, Courts and~Enforcement Act 2007 from any decision of the First-tier Tribunal under section~20 of this Act—
\begin{enumerate}\item[]
($a$) the Secretary of State, and

($b$) any person who is aggrieved by the decision of the First-tier Tribunal.”;
\end{enumerate}
\end{quotation}

($c$) in subsection~(1), as it has effect after the substitution made by paragraph~16(2) of Schedule~3 to~the Child Maintenance and~Other Payments Act 2008, for~“to~a Child Support Commissioner on a question of law” substitute “to~the Upper Tribunal under section~11 of the Tribunals, Courts and~Enforcement Act 2007 from any decision of the First-tier Tribunal under section~20 of this Act”; and

($d$) for~subsections~(2) to~(9) substitute—
\begin{quotation}
“(2) Where a question which would otherwise fall to~be determined by the Commission or~the Secretary of State under this Act first arises in the course of an appeal to~the Upper Tribunal, that tribunal may, if it thinks fit, determine the question even though it has not been considered by the Commission or~the Secretary of State.”.
\end{quotation}
\end{enumerate}

\medskip

86.  Omit section~25 (appeal from Child Support Commissioner on question of law)\footnote{Section~25 was amended by the Schedule~to the Transfer of Functions (Lord Advocate and~Secretary of State) Order 1999 (S.I.~1999/678). Functions under subsection~(3) were further transferred to Scottish Ministers under the Scotland~Act 1998 (Transfer of Functions to the Scottish Ministers etc) Order 1999 (S.I.~1999/1750).}.

\medskip

87.  In section~28ZA(1)($b$)  (decisions involving issues that arise on appeal in other cases)\footnote{Section~28ZA was inserted by section~43 of the Social Security Act 1998 (c.~14).}, as it has effect both with and~without the amendment made by paragraph~11(11)($b$)  of Schedule~3 to~the Child Support, Pensions and~Social Security Act 2000\footnote{2000 c.~19.}, for~“a Child Support Commissioner” substitute “the Upper Tribunal”.

\medskip

88.---(1)  Section~28ZB (appeals involving issues that arise on appeal in other cases)\footnote{Section~28ZB was inserted by section~43 of the Social Security Act 1998 (c.~14).} is amended as follows.

(2) In subsection~(1)($a$), as it has effect both with and~without the amendment made by paragraph~11(12)($a$)  of Schedule~3 to~the Child Support, Pensions and~Social Security Act 2000, for~“an appeal tribunal, or~from an appeal tribunal to~a Child Support Commissioner” substitute “the First-tier Tribunal, or~from the First-tier Tribunal to~the Upper Tribunal”.

(3) In subsection~(1)($b$)  for~“a Child Support Commissioner” substitute “the Upper Tribunal”.

(4) In subsection~(2) for~“tribunal or~Child Support Commissioner” substitute “First-tier Tribunal or~Upper Tribunal”.

(5) In subsection~(3)($a$)  and~($b$)  for~“tribunal” substitute “First-tier Tribunal”.

(6) In subsection~(4)—
\begin{enumerate}\item[]
($a$) for~“appeal tribunal or~Child Support Commissioner” substitute “First-tier Tribunal or~Upper Tribunal”;

($b$) in paragraph~($b$)  for~“tribunal or~Child Support Commissioner” substitute “First-tier Tribunal or~Upper Tribunal”.
\end{enumerate}

(7) In subsection~(5)—
\begin{enumerate}\item[]
($a$) for~“appeal tribunal or~Child Support Commissioner” substitute “First-tier Tribunal or~Upper Tribunal”; and

($b$) for~“tribunal or~Child Support Commissioner” substitute “First-tier Tribunal or~Upper Tribunal”.
\end{enumerate}

(8) In subsection~(7)($a$)  for~“a Child Support Commissioner”, in both places, substitute “the Upper Tribunal”.

\medskip

89.  In section~28ZC (restrictions on liability in certain cases of error)\footnote{Section~28ZC was inserted by section~44 of the Social Security Act 1998 (c.~14). Subsection~(6) was amended by paragraph~11(1) and~(13)($e$) of Schedule~3 to the Child Support, Pensions and~Social Security Act 2000 (c.~19).}—
\begin{enumerate}\item[]
($a$) in subsection~(1)($a$)  for~“a Child Support Commissioner” substitute “the Upper Tribunal”;

($b$) in subsection~(3) for~“Commissioner” substitute “Upper Tribunal”;

($c$) in subsection~(6) in the definition of “adjudicating authority” for~“an appeal tribunal” substitute “the First-tier Tribunal”; and

($d$) in subsection~(8)($a$)  and~($b$)  for~“a Child Support Commissioner” substitute “the Upper Tribunal”.
\end{enumerate}

\medskip

90.  In section~28ZD(1) (correction of errors and~setting aside of decisions)\footnote{Section~28ZD was inserted by section~44 of the Social Security Act 1998 (c.~14).}—
\begin{enumerate}\item[]
($a$) in subsection~(1)
\begin{enumerate}\item[]
(i) in paragraph~($a$)  after “decision”, in both places, insert “of the Secretary of State”; and

(ii) omit paragraph~($b$)  (together with the “and” immediately before it); and
\end{enumerate}

($b$) in subsection~(2) omit “or~set aside decisions”.
\end{enumerate}

\medskip

91.  In section~28D (determination of applications)\footnote{Section~28D was inserted by section~4 of the Child Support Act 1995 (c.~34). Subsections~(1)($b$) and~(3) were amended by paragraph~36 of Schedule~7 to the Social Security Act 1998 (c.~14).}---
\begin{enumerate}\item[]
($a$) in subsections~(1)($b$)  and~(3), as it has effect both with and~without the amendments made by section~5(3)($a$)  of the Child Support, Pensions and~Social Security Act 2000, for~“an appeal tribunal” substitute “the First-tier Tribunal”.
\end{enumerate}

\medskip

92.  In section~45 (jurisdiction of courts in certain proceedings under Act)\footnote{Subsections~(1) and~(6) of section~45 were amended by paragraph~42 of Schedule~7 to the Social Security Act 1998 (c.~14) and~subsection~(6) was also amended by article~2(1) of, and~the Schedule~to, the Transfer of Functions (Lord Advocate and~Secretary of State) Order 1999 (S.I.~1999/678).}—
\begin{enumerate}\item[]
($a$) in subsection~(1)($a$)  for~“an appeal tribunal” substitute “the First-tier Tribunal”;

($b$) omit subsection~(6); and

($c$) in subsection~(7) omit “or~(6)”.
\end{enumerate}

\medskip

93.  In section~46A(1) (finality of decisions)\footnote{Section~46A was inserted by paragraph~44 of Schedule~7 to the Social Security Act 1998 (c.~14).}—
\begin{enumerate}\item[]
($a$) after “Subject to~the provisions of this Act” insert “and~to~any provision made by or~under Chapter~II of Part~I of the Tribunals, Courts and~Enforcement Act 2007”; and

($b$) for~“an appeal tribunal” substitute “the First-tier Tribunal”.
\end{enumerate}

\medskip

94.  In section~46B(1) (matters arising as respects decisions)\footnote{Section~46B was inserted by paragraph~44 of Schedule~7 to the Social Security Act 1998 (c.~14).}—
\begin{enumerate}\item[]
($a$) in paragraph~($b$)  for~“an appeal tribunal” substitute “the First-tier Tribunal”; and

($b$) in paragraph~($c$)  for~“a Child Support Commissioner under section~24” substitute “the Upper Tribunal in relation to~a decision of the First-tier Tribunal under this Act”.
\end{enumerate}

\medskip

95.  In section~50 (unauthorised disclosure of information)\footnote{Section~50(1A) was inserted by paragraph~1(20) of Schedule~7 to the Child Maintenance and~Other Payments Act 2008 (c.~6).}—
\begin{enumerate}\item[]
($a$) in subsection~(1A) before paragraph~($a$)  insert—
\begin{quotation}
“($za$) any member of staff appointed under section~40(1) of the Tribunals, Courts and~Enforcement Act 2007 in connection with the carrying out of any functions in relation to~appeals from decisions made under this Act;”;
\end{quotation}

($b$) in subsection~(1A)($a$)  after “tribunal” insert “constituted under Chapter~I of Part~I of the Social Security Act 1998”;

($c$) in subsection~(1A)($b$)  for~“an” substitute “any such”;

($d$) in subsection~(5) (as it has effect until the commencement of the repeal of that subsection~made by Schedule~8 to~the Child Maintenance and~Other Payments Act 2008) in paragraph~($c$)  after “an appeal tribunal” insert “constituted under Chapter~I of Part~I of the Social Security Act 1998”; and

($e$) in subsection~(5) (as it so has effect) after paragraph~($d$)  insert—
\begin{quotation}
“($da$) any member of staff appointed under section~40(1) of the Tribunals, Courts and~Enforcement Act 2007 in connection with the carrying out of any functions in relation to~appeals from decisions made under this Act;”.
\end{quotation}
\end{enumerate}

\medskip

96.  In section~54 (interpretation)\footnote{In section~54 the definition of “appeal tribunal” was inserted by paragraph~47($a$) of Schedule~7 to the Social Security Act 1998 (c.~14).} omit the definition of “appeal tribunal”.

\medskip

97.---(1)  Schedule~4 (Child Support Commissioners)\footnote{Paragraphs 1(3A) and~(3B) and~8($ab$) were inserted by paragraphs~218 and~221(1) and~(2) of Schedule~4 to the Constitutional Reform Act 2005 (c.~4). Paragraphs 2A and~8($bb$) were inserted by paragraph~18(1) and~(2) of Schedule~3 to the Child Support Act 1995 (c.~34) and~sub-paragraph~(1) substituted by paragraph~51 of Schedule~7 to the Social Security Act 1998 (c.~14). Paragraph 4(2)($a$) was amended by paragraph~22(1) and~(4) of Schedule~10 to the Tribunals, Courts and~Enforcement Act 2007 (c.~15). The effect of the amendment to paragraph~8 made by paragraph~22(5) of that Schedule~has been retained in the amendments made by paragraph~97(7)($c$) of this Schedule. Paragraph 4A was inserted by section~17(1) of the Child Support Act 1995 (c.~34). Paragraphs 5 and~6 were amended by paragraph~52 of Schedule~7 to the Social Security Act 1998 (c.~14). Paragraph 7 was amended by paragraph~23(4) of Schedule~6 to the Judicial Pensions and~Retirement Act 1993 (c.~8). Paragraph 8($aa$) was inserted by paragraph~47 of Schedule~12 to the Justice (Northern Ireland) Act 2002 (c.~26) and~sub-Paragraphs (ai) and~(ia) of paragraph~8($d$) were inserted by paragraph~22 of Schedule~3 of that Act.} is amended as follows.

(2) In the heading after “Commissioners” insert “for~Northern Ireland”.

(3) In paragraph~1—
\begin{enumerate}\item[]
($a$) in sub-paragraph~(1) after “Child Support Commissioner” insert “for Northern Ireland”; and

($b$) omit sub-paragraphs~(3) to~(3B).
\end{enumerate}

(4) In paragraph~2—
\begin{enumerate}\item[]
($a$) in sub-paragraph~(1)—
\begin{enumerate}\item[]
(i) omit “pensions,”; and

(ii) after “Child Support Commissioners” insert “for~Northern Ireland”; and
\end{enumerate}

($b$) in sub-paragraph~(2) after “Child Support Commissioner” insert “for Northern Ireland”.
\end{enumerate}

(5) Omit paragraph~2A.

(6) For~paragraph~3 substitute—
\begin{quotation}
“3.  A Child Support Commissioner for~Northern Ireland, so long as he holds office as such, shall not practise as a barrister or~act for~any remuneration to~himself as arbitrator~or~referee or~be directly or~indirectly concerned in any matter as a conveyancer, notary public or~solicitor.”.
\end{quotation}

(7) In paragraph~4—
\begin{enumerate}\item[]
($a$) in sub-paragraph~(1)—
\begin{enumerate}\item[]
(i) for~“Lord Chancellor” substitute “First Minister and~deputy First Minister, acting jointly,”; and

(ii) after “Child Support Commissioners”, in each place, insert “for Northern Ireland”;
\end{enumerate}

($b$) in sub-paragraph~(2) after “Child Support Commissioner” insert “for Northern Ireland”;

($c$) in sub-paragraph~(2) for~paragraph~($a$)  (together with the “and” at the end of that paragraph) substitute—
\begin{quotation}
“($a$) from among persons who are barristers or~solicitors of not less than the number of years’ standing specified in section~23(2); and”;
\end{quotation}

($d$) in sub-paragraph~(2)($b$)  for~“Lord Chancellor~thinks” substitute “First Minister and~deputy First Minister think”;

($e$) in sub-paragraph~(2A) after “Child Support Commissioner” insert “for Northern Ireland”; and

($f$) for~sub-paragraph~(3) substitute—
\begin{quotation}
“(3) Paragraph 2 applies to~deputy Child Support Commissioners for~Northern Ireland, but paragraph~3 does not apply to~them.”
\end{quotation}
\end{enumerate}

(8) Omit paragraphs~4A to~8.

\medskip

98.  In Schedule~4A (departure directions)\footnote{Schedule~4A was inserted by section~1 of, and~Schedule~1 to, the Child Support Act 1995 (c.~34) and~was amended by paragraph~53 of Schedule~7 to the Social Security Act 1998 (c.~14).}, as it has effect without the substitution made by Part~I of Schedule~2 to~the Child Support, Pensions and~Social Security Act 2000—
\begin{enumerate}\item[]
($a$) omit paragraphs~2($b$)  and~8(2); and

($b$) omit paragraph~9.
\end{enumerate}

\medskip

99.  In Schedule~4A (applications for~a variation), as substituted by Part~I of Schedule~2 to~the Child Support, Pensions and~Social Security Act 2000, omit paragraphs~2($b$)  and~5(3).

\medskip

100.---(1)  Schedule~4C (decisions and~appeals: departure directions and~reduced benefit directions etc)\footnote{Schedule~4C was repealed by Part I of Schedule~9 to the Child Support, Pensions and~Social Security Act 2000 (c.~19) but remains in for force for certain cases until a day to be appointed.} is amended as follows.

(2) In paragraph~1($c$)  for~“an appeal tribunal” substitute “the First-tier Tribunal”.

(3) In paragraph~2 for~“an appeal tribunal”, in each place, substitute “the First-tier Tribunal”.

(4) In paragraph~3(2) for~“the appeal tribunal” substitute “the First-tier Tribunal”.

(5) In paragraph~4—
\begin{enumerate}\item[]
($a$) in sub-paragraph~(1)($a$)(ii)  for~“an appeal tribunal” substitute “the First-tier Tribunal”;

($b$) in sub-paragraph~(1)($b$)  for~“a Child Support Commissioner” substitute “the Upper Tribunal”; and

($c$) in sub-paragraph~(3)($a$)(i)  and~($b$)(i)  for~“appeal tribunal” substitute “First-tier Tribunal”.
\end{enumerate}

(6) In paragraph~5(1)—
\begin{enumerate}\item[]
($a$) in paragraph~($a$)  for~“an appeal tribunal” substitute “the First-tier Tribunal”; and

($b$) in paragraph~($b$)  for~“a Child Support Commissioner” substitute “the Upper Tribunal”.
\end{enumerate}

(7) In paragraph~6—
\begin{enumerate}\item[]
($a$) in sub-paragraph~(1)($a$)  for~“a Child Support Commissioner” substitute “the Upper Tribunal”; and

($b$) in sub-paragraph~(3) for~“an appeal tribunal” substitute “the First-tier Tribunal”.
\end{enumerate}

\section*{\itshape Social Security Administration Act 1992}

101.  The Social Security Administration Act 1992\footnote{1992 c.~5. Section~2B was inserted by section~57 of the Welfare Reform and~Pensions Act 1999 (c.~30). Section~71(2) was inserted by section~1(2) of the Social Security (Overpayments) Act 1996 (c.~51) and~amended by paragraph~81(1) of Schedule~7 to the Social Security Act 1998 (c.~14). In Part I of Schedule~4 the entry at ($a$) under “adjudicating bodies” was amended by paragraph~113($b$) of Schedule~7 to the Social Security Act 1998 (c.~14).} is amended as follows.

\medskip

102.  In section~2B(6) (supplementary provisions relating to~work-focused interviews) for~“appeal tribunal” substitute “First-tier Tribunal”.

\medskip

103.  In section~71(2) (over-payments—general)—
\begin{enumerate}\item[]
($a$) for~“a tribunal” substitute “the First-tier Tribunal”; and

($b$) for~“a Commissioner” substitute “the Upper Tribunal”.
\end{enumerate}

\medskip

104.---(1)  Schedule~4 (persons employed in social security administration or~adjudication) is amended as follows.

(2) In Part~I in the entry headed “Government departments” for~“Lord Chancellor’s Department” substitute “Ministry of Justice”.

(3) In Part~I in the entry headed “Adjudicating bodies” omit paragraph~($a$).

(4) In Part~I in the entry headed “Former officers” at the end insert—
\begin{quotation}
“The clerk to, or~other officer or~member of the staff of, an appeal tribunal.

The clerk to, or~other officer or~member of the staff of, a Pensions Appeal Tribunal for~England~and~Wales.”
\end{quotation}

(5) In paragraph~3 of Part~II—
\begin{enumerate}\item[]
($a$) for~“Lord Chancellor’s Department” substitute “Ministry of Justice”;

($b$) for~“that Department” substitute “that Ministry”;

($c$) in paragraph~($a$)  after “functions of” insert “the First-tier Tribunal or~Upper Tribunal which relate to~social security or~to~occupational or~personal pension schemes or~to~war pensions or~functions of”; and

($d$) insert at the end—
\begin{quotation}
“The reference in paragraph~($b$)  to~the Administrative Justice and~Tribunals Council and~the Scottish Committee of that Council includes a reference to~the former Council of Tribunals and~the Scottish Committee of that former Council.”.
\end{quotation}
\end{enumerate}

(6) After that paragraph~insert—
\begin{quotation}
“3ZA.  Any reference in Part~I of this Schedule~to~the Ministry of Justice includes a reference to—
\begin{enumerate}\item[]
($a$) the former Lord Chancellor’s Department, and

($b$) the former Department of Constitutional Affairs,
\end{enumerate}
to~the extent that the functions carried out by persons in its employ were, or~were connected with, functions of the Chief, or~any other, Social Security Commissioner (and~paragraph~3 above does not apply for~the purposes of this paragraph).”.
\end{quotation}

\section*{\itshape Tribunals and~Inquiries Act 1992}

105.  The Tribunals and~Inquiries Act 1992\footnote{1992 c.~53. Section~11(1) was amended by paragraph~20 of Schedule~8 to the Special Educational Needs and~Disability Act 2001 (c.~10). In Schedule~1 the entry relating to an Asylum Support Adjudicator was inserted by paragraphs~94 and~95 of Schedule~14 to the Immigration and~Asylum Act 1999 (c.~33). The entry relating to adjudicators appointed under the Criminal Injuries Compensation Act 1995 was substituted by section~5(8) of the Criminal Injuries Compensation Act 1995 (c.~53). The entry relating to the tribunal constituted under section~9 of the Protection of Children Act 1999 was inserted by paragraph~8 of the Schedule~to the Protection of Children Act 1999 (c.~14) and~amended by paragraph~21 of Schedule~4 to the Care Standards Act 2000 (c.~14). The entry relating to the Special Educational Needs and~Disability Tribunal was inserted by paragraphs~19 and~22 of Schedule~8 to the Special Educational Needs and~Disability Act 2001 (c.~10) and~amended by paragraph~15($a$) of Schedule~18 to the Education Act 2002 (c.~32).} is amended as follows.

\medskip

106.  In section~11(1) (appeals from certain tribunals) omit “, 40B”.

\medskip

107.  In Part~I of Schedule~1 (tribunals under direct supervision of the Council on Tribunals)—
\begin{enumerate}\item[]
($a$) omit the entry at paragraph~2A relating to~asylum-seekers support;

($b$) omit the entry at paragraph~12 relating to~criminal injuries compensation;

($c$) in the entry relating to~pensions, in the second column, omit paragraph~35($a$)  and~($b$);

($d$) omit the entry at paragraph~36B relating to~the protection of children and~vulnerable adults, and~care standards; and

($e$) in the entry relating to~special educational needs and~disability discrimination, in the second column, omit paragraph~40B($a$).
\end{enumerate}

\section*{\itshape Judicial Pensions and~Retirement Act 1993}

108.  The Judicial Pensions and~Retirement Act 1993\footnote{1993 c.~8.} is amended as follows.

\medskip

109.  In Part~II of Schedule~1\footnote{In Schedule~1: the entry relating to the chairman of a Mental Health Review Tribunal for England~was inserted by the Judicial Pensions and~Retirement Act 1993 (Addition of Qualifying Judicial Offices) (No.~2) Order 2003 (S.I.~2003/2589) and~amended by article~2(1) of the Judicial Pensions and~Retirement Act 1993 (Addition of Qualifying Judicial Offices) (No.~2) Order 2008 (S.I.~2008/171); the entry relating to an Asylum Support Adjudicator was inserted by articles 2 and~3 of the Judicial Pensions and~Retirement Act 1993 (Addition of Qualifying Judicial Offices) Order 2007 (S.I.~2007/675); and~the entry relating to the President of the tribunal constituted under the Protection of Children Act 1999 and~members of the chairmen’s panel appointed under paragraph~1(1)($b$) of the Schedule~to that Act was inserted by article~2 of the Judicial Pensions and~Retirement Act 1993 (Addition of Qualifying Judicial Offices) (No.~2) Order 2007 (S.I.~2007/2185).} (other appointments – members of tribunals) omit—
\begin{enumerate}\item[]
($a$) the first entry beginning “Chief or~other Child Support Commissioner”;

($b$) the entry relating to~the chairman of a Mental Health Review Tribunal for~England;

($c$) the entry relating to~an Asylum Support Adjudicator; and

($d$) the entry relating to~the President of the tribunal constituted under section~9 of the Protection of Children Act 1999 and~members of the chairmen’s panel appointed under paragraph~1(1)($b$)  of the Schedule~to~that Act.
\end{enumerate}

110.  In Schedule~5\footnote{In Schedule~5: the entry relating to an Asylum Support Adjudicator was inserted by article~4 of the Judicial Pensions and~Retirement Act 1993 (Addition of Qualifying Judicial Offices) Order 2007 (S.I.~2007/675); and~the entry relating to the President of the tribunal constituted under the Protection of Children Act 1999 and~members of the chairmen’s panel appointed under paragraph~1(1)($b$) of the Schedule~to that Act was inserted by article~3 of the Judicial Pensions and~Retirement Act 1993 (Addition of Qualifying Judicial Offices) (No.~2) Order 2007 (S.I.~2007/2185).} (retirement provisions: the relevant offices) omit—
\begin{enumerate}\item[]
($a$) the first entry beginning “Chief or~other Child Support Commissioner”;

($b$) the entry relating to~an Asylum Support Adjudicator; and

($c$) the entry relating to~the President of the tribunal constituted under section~9 of the Protection of Children Act 1999 and~members of the chairmen’s panel appointed under paragraph~1(1)($b$)  of the Schedule~to~that Act.
\end{enumerate}

\section*{\itshape Pensions Schemes Act 1993}

111.  The Pensions Schemes Act 1993\footnote{1993 c.~48. Section~170 was substituted by paragraph~131 of Schedule~7 to the Social Security Act 1998 (c.~14). Section~171A was inserted by section~18 of, and~paragraph~20 of Schedule~7 to, the Social Security Contributions (Transfer of Functions, etc) Act 1999 (c.~2).} is amended as follows.

\medskip

112.  In section~170(6) (decisions and~appeals) for~“appeal tribunal” substitute “First-tier Tribunal”.

\medskip

113.  In section~171A(1) (reports by Inland~Revenue) for~“an appeal tribunal constituted under Chapter~I of Part~I of the Social Security Act 1998” substitute “the First-tier Tribunal”.

\section*{\itshape Disability Discrimination Act 1995}

114.  The Disability Discrimination Act 1995\footnote{1995 c.~50. Section~28H was inserted by section~17 of the Special Educational Needs and~Disability Act 2001 (c.~10); the heading and~subsection~(2) of that section~were substituted by paragraph~8 of Schedule~18 to the Education Act 2002 (c.~32). Section~28I was inserted by section~18 of the 2001 Act; subsection~(5) was inserted by paragraph~9 of Schedule~18 to the 2002 Act. Section~28J was inserted by section~19 of the 2001 Act; subsection~(2A) was inserted and~subsections~(3) and~(5) to (8) were amended by paragraph~10 of Schedule~18 to the 2002 Act; and~subsections~(2A) and~(6) were amended by paragraph~53 of Schedule~1 to the Government of Wales Act 2006 (Consequential Modifications and~Transitional Provisions) Order 2007 (S.I.~2007/1388). Section~28M was inserted by section~22 of the 2001 Act and~subsection~(5) was amended by paragraph~11 of Schedule~18 to the 2002 Act. Section~28N was inserted by section~23 of the 2001 Act and~subsection~(5) was amended by paragraph~24 of Schedule~1 to the Disability Discrimination Act 2005 (c.~13). Part III of Schedule~3 was inserted by paragraph~1 of Schedule~3 to the 2001 Act and~amended by paragraph~12 of Schedule~18 to the 2002 Act.} is amended as follows.

\medskip

115.  In section~28H (tribunals)—
\begin{enumerate}\item[]
($a$) omit subsection~(1);

($b$) in subsection~(2) omit the definition of “the Tribunal” (together with the “and” following that definition); and

($c$) in subsection~(3)—
\begin{enumerate}\item[]
(i) for~“those tribunals” substitute “the Welsh Tribunal”; and

(ii) for~“each of them” substitute “the Welsh Tribunal”.
\end{enumerate}
\end{enumerate}

\medskip

116.  In section~28I(5)($a$)  (jurisdiction and~powers of the Tribunal) for~“Tribunal” substitute “First-tier Tribunal”.

\medskip

117.  In section~28J (procedure)—
\begin{enumerate}\item[]
($a$) in subsection~(1)—
\begin{enumerate}\item[]
(i) after “Regulations may” insert “, with the agreement of the Welsh Ministers,”;

(ii) in paragraph~($a$)  before “Tribunal” insert “Welsh”; and

(iii) in paragraph~($b$)  after “claim” insert “to~the Welsh Tribunal”;
\end{enumerate}

($b$) in subsection~(2)($b$)  and~($k$)  before “Tribunal” insert “Welsh”;

($c$) omit subsection~(2A);

($d$) in subsection~(3) omit “the Tribunal or”;

($e$) in subsection~(5)—
\begin{enumerate}\item[]
(i) after “State may” insert “, with the agreement of the Welsh Ministers,”; and

(ii) omit “the Tribunal or”;
\end{enumerate}

($f$) omit subsection~(6);

($g$) in subsection~(7) omit “the Tribunal or”;

($h$) in subsection~(8)($b$)  for~“Tribunal” substitute “First-tier Tribunal”;

($i$) after subsection~(9) insert—
\begin{quotation}
“(9A) A person who without reasonable excuse fails to~comply with a requirement which—
\begin{enumerate}\item[]
($a$) is imposed by Tribunal Procedure Rules in relation to claims of unlawful discrimination under this Chapter made to the First-tier Tribunal, and

\begin{sloppypar}
($b$) corresponds to~a requirement mentioned in subsection~(9)($a$)  or~($b$),
\end{sloppypar}
\end{enumerate}
is guilty of an offence.”; and
\end{quotation}

($j$) in subsection~(10) after “subsection~(9)” insert “or~(9A)”.
\end{enumerate};

\medskip

118.  After section~28J insert—
\begin{quotation}
\subsection*{“28JA. Appeal from the Welsh Tribunal to~the Upper Tribunal}

(1)  A party to~any proceedings under this Chapter~before the Welsh Tribunal may appeal to~the Upper Tribunal on any point of law arising from a decision made by the Welsh Tribunal in those proceedings.

(2) An appeal may be brought under subsection~(1) only if, on an application made by the party concerned, the Welsh Tribunal or~the Upper Tribunal has given its permission for~the appeal to~be brought.

(3) Section~12 of the Tribunals, Courts and~Enforcement Act 2007 (proceedings on appeal to~Upper Tribunal) applies in relation to~appeals to~the Upper Tribunal under this section~as it applies in relation to~appeals to~it under section~11 of that Act, but as if references to~the First-tier Tribunal were references to~the Welsh Tribunal.”
\end{quotation}

\medskip

119.  In section~28M(5) (roles of the Secretary of State and~the Welsh Ministers) for~“Tribunal” substitute “First-tier Tribunal”.

\medskip

120.  In section~28N(5)($b$)  (civil proceedings: Scotland) for~“Tribunal” substitute “First-tier Tribunal”.

\medskip

121.  In paragraph~10 of Part~III of Schedule~3 (discrimination in schools) in sub-paragraphs~(1), (3) and~(4) omit “Tribunal or~the”.

\section*{\itshape Criminal Injuries Compensation Act 1995}

122.  The Criminal Injuries Compensation Act 1995\footnote{1995 c.~53. The definition of “adjudicator” in section~1(4) and~section~5 were amended, and~subsections~(1A) and~(1B) of section~5 were inserted, by paragraph~2 of Schedule~10 to the Scotland~Act 1998 (Cross-Border Public Authorities) (Adaptation of Functions etc) Order 1999 (S.I.~1999/1747).} is amended as follows.

\medskip

123.  In section~1(4) (the Criminal Injuries Compensation Scheme) omit the definition of “adjudicator”.

\medskip

124.  In section~5 (appeals)—
\begin{enumerate}\item[]
($a$) for~subsections~(1) to~(7) substitute—
\begin{quotation}
“(1) The Scheme shall include provision for~rights of appeal to~the First-tier Tribunal against decisions taken on reviews under provisions of the Scheme made by virtue of section~4.”; and
\end{quotation}

($b$) in subsection~(9) for~“adjudicator~or~adjudicators” substitute “First-tier Tribunal”.
\end{enumerate}

\medskip

125.  In section~9 (financial provisions)—
\begin{enumerate}\item[]
($a$) in subsection~(1) omit “(other than adjudicators)”; and

($b$) omit subsections~(2) and~(3).
\end{enumerate}

\medskip

126.  In section~11(4) (parliamentary control) omit paragraph~($b$)  (together with the “or” immediately before it).

\section*{\itshape Education Act 1996}

127.  The Education Act 1996\footnote{1996 c.~56. Section~313(5) was inserted by paragraph~2 of Schedule~18 to the Education Act 2002 (c.~32). Section~326A was inserted by section~5 of the Special Educational Needs and~Disability Act 2001 (c.~10) and~subsection~(6) of that section~was substituted by paragraph~3 of Schedule~18 to the 2002 Act. In section~333, subsection~(1Z) was inserted by paragraph~4 of Schedule~18 to the 2002 Act and~subsection~(1) was substituted by paragraph~3 of Schedule~8 to the 2001 Act. Section~336 was amended by paragraph~13 of Schedule~8 to the 2001 Act. Section~336ZA was inserted by paragraph~5 of Schedule~18 to the 2002 Act. Section~336A was inserted by section~4 of the 2001 Act and~subsection~(2) of that section~was substituted by paragraph~6 of Schedule~18 to the 2002 Act.} is amended as follows.

\medskip

128.  In section~313(5) (code of practice)—
\begin{enumerate}\item[]
($a$) omit “(except sections~333 to~336)”; and

($b$) in paragraph~($a$)  for~“the Special Educational Needs and~Disability Tribunal” substitute “the First-tier Tribunal”.
\end{enumerate}

\medskip

129.  In section~326A(6) (unopposed appeals)—
\begin{enumerate}\item[]
($a$) in paragraph~($a$)  for~“to~the Special Educational Needs and~Disability Tribunal” substitute “against a decision of a local education authority in England”; and

($b$) in paragraph~($b$)  for~“to~the Special Educational Needs Tribunal for~Wales, by the National Assembly for~Wales” substitute “against a decision of a local education authority in Wales, by the Welsh Ministers”.
\end{enumerate}

\medskip

130.  In section~333 (constitution of tribunal)—
\begin{enumerate}\item[]
($a$) in the title before “Tribunal” insert “Welsh”;

($b$) omit subsection~(1Z);

($c$) after that subsection~insert—

\begin{quotation}
“(1ZA) There continues to~be a tribunal known as {\selectlanguage{welsh}Tribiwnlys Anghenion Addysgol Arbennig Cymru} or~the Special Educational Needs Tribunal for~Wales.

(1ZB) In this section~and~sections~334 to~336ZB “Welsh Tribunal” means {\selectlanguage{welsh}Tribiwnlys Anghenion Addysgol Arbennig Cymru} or~the Special Educational Needs Tribunal for~Wales.”;
\end{quotation}

($d$) in subsections~(1), (2)($a$)  to~($c$)  and~(5)($a$)  and~($b$)  before “Tribunal” insert “Welsh”;

($e$) in subsection~(4) after “appointed by” insert “the Welsh Ministers with the agreement of”;

($f$) in subsection~(5)—
\begin{enumerate}\item[]
(i) for~“Regulations may” substitute “Regulations made by the Welsh Ministers with the agreement of the Secretary of State may”; and

(ii) in paragraph~($b$)  for~“Secretary of State considers” substitute “Welsh Ministers, with the agreement of the Secretary of State, consider”; and
\end{enumerate}

($g$) for~subsection~(6) substitute—
\begin{quotation}
“(6) The Welsh Ministers may provide such staff and~accommodation as the Welsh Tribunal may require.”
\end{quotation}
\end{enumerate}

\medskip

131.  In section~334 (the President and~members of the panels)—
\begin{enumerate}\item[]
($a$) in subsection~(2) after “prescribed” insert “in regulations made by the Welsh Ministers with the agreement of the Secretary of State”; and

($b$) in subsection~(5)($a$)  for~“Secretary of State” substitute “Welsh Ministers”.
\end{enumerate}

\medskip

132.  For~section~335(1) and~(2) (remuneration and~expenses) substitute—
\begin{quotation}
“(1) The Welsh Ministers may pay to~the President, and~to~any other person in respect of his service as a member of the Welsh Tribunal, such remuneration and~allowances as the Welsh Ministers may determine.

(2) The Welsh Ministers may defray the expenses of the Welsh Tribunal to~such amount as they may determine.”
\end{quotation}

\medskip

133.  In section~336 (tribunal procedure)—
\begin{enumerate}\item[]
($a$) in subsection~(1)—
\begin{enumerate}\item[]
(i) after “Regulations” insert “made by the Welsh Ministers”, and

(ii) before “Tribunal” insert “Welsh”;
\end{enumerate}

($b$) in subsection~(2)—
\begin{enumerate}\item[]
(i) in paragraphs~($b$)  and~($o$)  before “Tribunal” insert “Welsh”; and

(ii) in paragraph~($j$)  for~“prescribed circumstances” substitute “circumstances prescribed in the regulations”;
\end{enumerate}

($c$) in subsection~(2A)—
\begin{enumerate}\item[]
(i) before “Tribunal” insert “Welsh”; and

(ii) for~“prescribed circumstances” substitute “circumstances prescribed in the regulations”;
\end{enumerate}

($d$) for~subsection~(3) substitute—
\begin{quotation}
“(3) The Welsh Ministers may pay such allowances for~the purpose of or~in connection with the attendance of persons at the Welsh Tribunal as the Welsh Ministers may determine.”;
\end{quotation}

($e$) in subsection~(4)—
\begin{enumerate}\item[]
(i) before “Tribunal” insert “Welsh”; and

(ii) after “regulations” insert “made by the Welsh Ministers”;
\end{enumerate}

($f$) in subsection~(4A)—
\begin{enumerate}\item[]
(i) for~“The regulations” substitute “Regulations made under subsection~(1)”; and

(ii) for~“prescribed circumstances” substitute “circumstances prescribed in the regulations”;
\end{enumerate}

($g$) after subsection~(5) insert—
\begin{quotation}
“(5A) Any person who without reasonable excuse fails to~comply with any requirement which—
\begin{enumerate}\item[]
($a$) is imposed by Tribunal Procedure Rules in relation to~appeals under this Part~made to~the First-Tier Tribunal, and

\begin{sloppypar}
($b$) corresponds to~a requirement mentioned in subsection~(5)($a$)  or~($b$),
\end{sloppypar}
\end{enumerate}
is guilty of an offence.”; and
\end{quotation}

($h$) in subsection~(6) after “subsection~(5)” insert “or~(5A)”.
\end{enumerate}

\medskip

134.  Omit section~336ZA (Special Educational Needs Tribunal for~Wales).

\medskip

135.  After that section~insert—
\begin{quotation}
\subsection*{\sloppy “336ZB.  Appeals from the Welsh Tribunal to~the Upper Tribunal}

(1)  A party to~any proceedings under this Part~before the Welsh Tribunal may appeal to~the Upper Tribunal on any point of law arising from a decision made by the Welsh Tribunal in those proceedings.

(2) An appeal may be brought under subsection~(1) only if, on an application made by the party concerned, the Welsh Tribunal or~the Upper Tribunal has given its permission for~the appeal to~be brought.

(3) Section~12 of the Tribunals, Courts and~Enforcement Act 2007 (proceedings on appeal to~Upper Tribunal) applies in relation to~appeals to~the Upper Tribunal under this section~as it applies in relation to~appeals to~it under section~11 of that Act, but as if references to~the First-tier Tribunal were references to~the Welsh Tribunal.”
\end{quotation}

\medskip

136.  In section~336A(2) (compliance with orders)—
\begin{enumerate}\item[]
($a$) in paragraph~($a$)  for~“the Special Educational Needs and~Disability Tribunal” substitute “the First-tier Tribunal”; and

($b$) in paragraph~($b$)  for~“the Special Educational Needs Tribunal for~Wales, by the National Assembly for~Wales” substitute “the Welsh Tribunal, by the Welsh Ministers”.
\end{enumerate}

\section*{\itshape Employment Tribunals Act 1996}

137.  In section~16(5)($d$)  of the Employment Tribunals Act 1996 (power to~provide for~recoupment of social security benefits)\footnote{1996 c.~17. Section 16(5)($d$) was substituted by paragraph~147($b$) of Schedule~7 to the Social Security Act 1998 (c.~14).} for~“an appeal tribunal constituted under Chapter~I of Part~I of the Social Security Act 1998” substitute “the First-tier Tribunal”.

\section*{\itshape\sloppy Social Security (Recovery of Benefits) Act 1997}

138.  The Social Security (Recovery of Benefits) Act 1997\footnote{1997 c.~27. Sections 12(1) and~(4) and~13(1) and~(3) were amended by paragraphs~151(1) and~(3) and~152(1) and~(3) respectively of Schedule~7 to the Social Security Act 1998 (c.~14). In section 25 the definitions of “appeal tribunal” and~“Commissioner” were inserted by paragraph~153 of Schedule~7 to the 1998 Act.} is amended as follows.

\medskip

139.  In section~11(5) (appeals against certificates of recoverable benefits) omit paragraph~($b$)  (but not the “and” at the end of that paragraph).

\medskip

140.  In section~12 (reference of questions to~medical appeal tribunal)—
\begin{enumerate}\item[]
($a$) in the heading for~“medical appeal tribunal” substitute “First-tier Tribunal”; and

($b$) in subsections~(1) and~(4) for~“an appeal tribunal” substitute “the First-tier Tribunal”.
\end{enumerate}

\medskip

141.  In section~13 (appeal to~Social Security Commissioner)—
\begin{enumerate}\item[]
($a$) in the heading for~“Social Security Commissioner” substitute “Upper Tribunal”;

($b$) omit subsection~(1);

($c$) in subsection~(2) for~“under this section” substitute “to~the Upper Tribunal under section~11 of the Tribunals, Courts and~Enforcement Act 2007 which arises from any decision of the First-Tier Tribunal made under section~12 of this Act”; and

($d$) omit subsection~(3).
\end{enumerate}

142.  In section~29 (general interpretation) omit the definitions of “appeal tribunal” and~“Commissioner”.

\section*{\itshape Social Security Act 1998}

143.  The Social Security Act 1998\footnote{1998 c.~14.} is amended as follows.

\medskip

144.  Omit section~4 (unified appeal tribunals).

\medskip

145.  Omit section~5 (President of appeal tribunals)\footnote{Subsections~(1) and~(2) of section 5 were amended by paragraph~29(1) and~(2) of Schedule~10 to the Tribunals, Courts and~Enforcement Act 2007 (c.~15) and~subsection (1) was also amended by article~2(1) of, and~the Schedule~to, the Transfer of Functions (Lord Advocate and~Secretary of State) Order 1999 (S.I.~1999/678).}.

\medskip

146.  Omit section~6 (panel for~appointment to~appeal tribunals).

\medskip

147.  Omit section~7 (constitution of appeal tribunals).

\medskip

148.  In section~10(1)($b$)  (decisions superseding earlier decisions) for~“of an appeal tribunal or~a Commissioner” substitute “of the First-tier Tribunal or~any decision of the Upper Tribunal which relates to~any such decision”.

\medskip

149.  In section~12 (appeal to~appeal tribunal)\footnote{Subsection (2) of section 12 was inserted by paragraph~25(3) of Schedule~7 to the Social Security Contributions (Transfer of Functions, etc) Act 1999 (c.~2).}—
\begin{enumerate}\item[]
($a$) in the heading for~“appeal tribunal” substitute “First-tier Tribunal”; and

($b$) in subsections~(2), (4), (5) and~(8) for~“an appeal tribunal” substitute “the First-tier Tribunal”.
\end{enumerate}

\medskip

150.  In section~13 (redetermination etc of appeals by tribunal)\footnote{Subsection (4) of section 13 was inserted by paragraph~26 of Schedule~7 to the Social Security Contributions (Transfer of Functions, etc) Act 1999 (c.~2).}—
\begin{enumerate}\item[]
($a$) in subsection~(1) for~the words from “to~a person” to~the end substitute “to~the First-tier Tribunal for~permission to~appeal to~the Upper Tribunal from any decision of the First-tier Tribunal under section~12 or~this section”;

($b$) omit subsection~(2); and

($c$) in subsection~(3)—
\begin{enumerate}\item[]
(i) for~“the person” substitute “the First-tier Tribunal”; and

(ii) for~“tribunal” substitute “First-tier Tribunal”.
\end{enumerate}
\end{enumerate}

\medskip

151.  In section~14 (appeal from tribunal to~Commissioner)\footnote{Section 14 was amended by paragraph~27($a$) of Schedule~7 to the Social Security Contributions (Transfer of Functions, etc) Act 1999 (c.~2).}—
\begin{enumerate}\item[]
($a$) in the heading for~“tribunal to~Commissioner” substitute “First-tier Tribunal to~Upper Tribunal”;

($b$) omit subsection~(1);

($c$) in subsections~(3) and~(4) for~“lies under this section” substitute “to~the Upper Tribunal under section~11 of the Tribunals, Courts and~Enforcement Act 2007 from any decision of the First-Tier Tribunal under section~12 or~13 above lies”; and

($d$) omit subsections~(7) to~(12).
\end{enumerate}

\medskip

152.  In section~15 (appeal from Commissioner on point of law)—
\begin{enumerate}\item[]
($a$) for~the heading substitute “Applications for~permission to~appeal against a decision of the Upper Tribunal”;

($b$) omit subsections~(1) and~(2);

($c$) in subsection~(3)—
\begin{enumerate}\item[]
(i) for~“An application for~leave under this section~in respect of a Commissioner’s decision” substitute “An application for~permission to~appeal from a decision of the Upper Tribunal in respect of a decision of the First-tier Tribunal under section~12 or~13”;

(ii) in paragraph~($a$)  for~“Commissioner”, in both places, substitute “Upper Tribunal”;

(iii) in that paragraph~for~“Commissioner’s” substitute “Upper Tribunal’s”;

(iv)  in paragraph~($c$)  for~“leave” substitute “permission”; and

($v$)  omit the words from “and~regulations” to~the end; and
\end{enumerate}

($d$) omit subsections~(4) and~(5).
\end{enumerate}

\medskip

153.  After section~15 insert—
\begin{quotation}
\subsection*{\sloppy “15A.  Functions of Senior~President of Tribunals}

(1) The Senior~President of Tribunals shall ensure that appropriate steps are taken by the First-tier Tribunal to~secure the confidentiality, in such circumstances as may be prescribed, of any prescribed material, or~any prescribed classes or~categories of material.

(2) Each year the Senior~President of Tribunals shall make to~the Secretary of State and~the Child Maintenance and~Enforcement Commission a written report, based on the cases coming before the First-tier Tribunal, on the standards achieved by the Secretary of State and~the Child Maintenance and~Enforcement Commission in the making of decisions against which an appeal lies to~the First-tier Tribunal.

(3) The Lord Chancellor~shall publish the report.”.
\end{quotation}

\medskip

154.  In section~16 (procedure) omit subsections~(2), (3)($a$)  (together with the “and” at the end of that paragraph) and~(6) to~(9).

\medskip

155.  In section~17(1) (finality of decisions) after “this Chapter”, in the first place where it occurs, insert “and~to~any provision made by or~under Chapter~II of Part~I of the Tribunals, Courts and~Enforcement Act 2007”.

\medskip

156.  In section~18(1)($a$)  (matters arising as respects decisions) for~“, an appeal tribunal or~a Commissioner” substitute “or~the First-tier Tribunal, or~any decision of the Upper Tribunal which relates to~any decision under this Chapter~of the First-Tier Tribunal,”.

\medskip

157.  In section~20 (medical examination required by appeal tribunal)—
\begin{enumerate}\item[]
($a$) in subsection~(2)—
\begin{enumerate}\item[]
(i) for~“An eligible person may, if prescribed conditions” substitute “The First-tier Tribunal may, if conditions prescribed by Tribunal Procedure Rules”;

(ii) for~“the eligible person” substitute “the First-tier Tribunal”;

(iii) for~“an appeal tribunal” substitute “it”; and

(iv)  omit the second sentence beginning “In this subsection”; and
\end{enumerate}

($b$) in subsection~(3) for~“an appeal tribunal, except in prescribed cases or~circumstances,” substitute “the First-tier Tribunal, except in cases or~circumstances prescribed by Tribunal Procedure Rules,”.
\end{enumerate}

\medskip

158.  After section~20 insert—
\begin{quotation}
\subsection*{“20A.  Travelling and~other allowances}

(1) The Lord Chancellor~may pay to~any person required under this Part~(whether for~the purposes of this Part~or~otherwise) to~attend for~or~to~submit to~medical or~other examination or~treatment such travelling and~other allowances as the Lord Chancellor~may determine.

(2) In subsection~(1) the reference to~travelling and~other allowances includes compensation for~loss of remunerative time but such compensation shall not be paid to~any person in respect of any time during which the person is in receipt of remuneration under section~28 of, or~paragraph~5 of Schedule~2 to, the Tribunals, Courts and~Enforcement Act 2007 (assessors and~judges of First-Tier Tribunal).”.
\end{quotation}

\medskip

159.  In section~21 (suspension in prescribed cases)—
\begin{enumerate}\item[]
($a$) in subsection~(2)—
\begin{enumerate}\item[]
(i) in paragraph~($c$)  for~“an appeal tribunal, a Commissioner” substitute “the First-tier Tribunal, the Upper Tribunal”; and

(ii) in paragraph~($d$)  for~“a Commissioner” substitute “the Upper Tribunal”; and
\end{enumerate}

($b$) in subsection~(3)($b$)  and~($c$)  for~“leave” substitute “permission”.
\end{enumerate}

\medskip

160.  In section~24A (appeals dependent on issues falling to~be decided by Inland~Revenue)\footnote{Section 24A was inserted by paragraph~33 of Schedule~7 to the Social Security Contributions (Transfer of Functions, etc) Act 1999 (c.~2).}—
\begin{enumerate}\item[]
($a$) in subsection~(1) for~“an appeal tribunal or~Commissioner” substitute “the First-tier Tribunal or~Upper Tribunal”; and

($b$) in subsection~(2)($c$)(iii)  for~“appeal tribunal or~Commissioner” substitute “First-tier Tribunal or~Upper Tribunal”.
\end{enumerate}

\medskip

161.  In section~25(1)($b$)  (decisions involving issues that arise on appeal in other cases) for~“a Commissioner” substitute “the Upper Tribunal”.

\medskip

162.---(1)  Section~26 (appeals involving issues that arise on appeal in other cases) is amended as follows.

(2) In subsection~(1)—
\begin{enumerate}\item[]
($a$) in paragraph~($a$)  for~“an appeal tribunal, or~from an appeal tribunal to~a Commissioner” substitute “the First-tier Tribunal, or~from the First-tier Tribunal to~the Upper Tribunal”; and

($b$) in paragraph~($b$)  for~“a Commissioner” substitute “the Upper Tribunal”.
\end{enumerate}

(3) In subsection~(2) for~“tribunal or~Commissioner” substitute “First-tier Tribunal or~Upper Tribunal”.

(4) In subsection~(3)($a$)  and~($b$)  for~“tribunal” substitute “First-tier Tribunal”.

(5) In subsection~(4)—
\begin{enumerate}\item[]
($a$) for~“appeal tribunal or~Commissioner” substitute “First-tier Tribunal or~Upper Tribunal”; and

($b$) in paragraph~($b$)  for~“tribunal or~Commissioner” substitute “First-tier Tribunal or~Upper Tribunal”.
\end{enumerate}

(6) In subsection~(5)—
\begin{enumerate}\item[]
($a$) for~“appeal tribunal or~Commissioner” substitute “First-tier Tribunal or~Upper Tribunal”; and

($b$) for~“tribunal or~Commissioner” substitute “First-tier Tribunal or~Upper Tribunal”.
\end{enumerate}

(7) In subsection~(7)($a$)  for~“a Commissioner”, in both places, substitute “the Upper Tribunal”.

\medskip

163.  In section~27 (restrictions on entitlement to~benefit in certain cases of error)—
\begin{enumerate}\item[]
($a$) in subsections~(1)($a$)  and~(10)($a$)  and~($b$)  for~“a Commissioner” substitute “the Upper Tribunal”; and

($b$) in subsection~(3) for~“the Commissioner” substitute “the Upper Tribunal”.
\end{enumerate}

\medskip

164.  In section~28 (correction of errors and~setting aside of decisions)\footnote{Section 28(1A) was inserted by paragraph~34 of Schedule~7 to the Social Security Contributions (Transfer of Functions, etc) Act 1999 (c.~2).}—
\begin{enumerate}\item[]
($a$) in subsection~(1)—
\begin{enumerate}\item[]
(i) in paragraph~($a$)  after “decision”, in both places, insert “of the Secretary of State”; and

(ii) omit paragraph~($b$)  (together with the “and” immediately before it);
\end{enumerate}

($b$) in subsection~(1A) after “not include” insert “any decision of the First-tier Tribunal or”; and

($c$) in subsection~(2) omit “or~set aside decisions”.
\end{enumerate}

\medskip

165.  In section~29(3) (decision that accident is an industrial accident), for~“,~an appeal tribunal or~a Commissioner” substitute “, the First-tier Tribunal or~the Upper Tribunal”.

\medskip

166.  Before section~39 insert—
\begin{quotation}
\subsection*{“39ZA.  Certificates}

    A document bearing a certificate which—
\begin{enumerate}\item[]
    ($a$) 
    is signed by a person authorised in that behalf by the Secretary of State, and

    ($b$) 
    states that the document, apart from the certificate, is a record of a decision of an officer of the Secretary of State,
\end{enumerate}
    shall be conclusive evidence of the decision; and~a certificate purporting to~be so signed shall be deemed to~be so signed unless the contrary is proved.”. 
\end{quotation}

\medskip

167.  In section~39(1) (interpretation etc of Chapter~II) omit the definitions of “appeal tribunal” and~“Commissioner”.

\medskip

168.  In section~79 (regulations and~orders)—
\begin{enumerate}\item[]
($a$) in subsection~(1) for~the words from the beginning to~“to~this Act,” substitute “Subject to~subsection~(2A) below,”; and

($b$) omit subsections~(2) and~(9).
\end{enumerate}

\medskip

169.  In section~80 (Parliamentary control of regulations)\footnote{Subsection (4) of section 80 was inserted by paragraph~29(5) of Schedule~10 to the Tribunals, Courts and~Enforcement Act 2007 (c.~15).}—
\begin{enumerate}\item[]
($a$) in subsection~(1)($a$)  omit “7, ”;

($b$) in subsection~(1)($b$)—
\begin{enumerate}\item[]
(i) omit “paragraph~12 of Schedule~1,”; and

(ii) omit “or~paragraph~2 of Schedule~5”; and
\end{enumerate}

($c$) omit subsections~(3) and~(4).
\end{enumerate}

\medskip

170.  In section~81(1) (reports by Secretary of State) for~“an appeal tribunal constituted under Chapter~I of Part~I” substitute “the First-tier Tribunal”.

\medskip

171.  Omit Schedule~1 (appeal tribunals: supplementary provisions)\footnote{In Schedule~1: Paragraph 1 was amended by article 2(1) of, and~the Schedule~to, the Transfer of Functions (Lord Advocate and~Secretary of State) Order 1999 (S.I.~1999/678) and~paragraphs~271 and~273 of Schedule~4 to the Constitutional Reform Act 2005 (c.~4); paragraph~8 was amended by paragraph~4 of Schedule~5 to the National Assembly of Wales (Transfer of Functions) Order 2000 (S.I.~2000/253); and~ministerial functions in respect of Scotland~were further transferred by article 2 of, and~Schedule~1 to, the Scotland~Act 1998 (Transfer of Functions to the Scottish Ministers etc) Order 1999 (S.I.~1999/1750).}.

\medskip

172.  Omit Schedule~4 (Social Security Commissioners)\footnote{In Schedule~4: paragraph~1 was amended by paragraph~5 of Schedule~11 to the Constitutional Reform Act 2005 (c.~4) and~paragraph~29(1), (6) and~(7) of Schedule~10 to the Tribunals, Courts and~Enforcement Act 2007 (c.~15); paragraph~3 was amended by section 7(2) of the Armed Forces (Pensions and~Compensation) Act 2004 (c.~32) and~paragraph~22(3) of Schedule~7 to the Child Support, Pensions and~Social Security Act 2000 (c.~19); sub-paragraphs~(1A) and~(1B) of paragraph~5 were inserted by paragraphs~271 and~274 of Schedule~4 to the 2005 Act; paragraph~8 was amended by article 2(1) of, and~the Schedule~to, the Transfer of Functions (Lord Advocate and~Secretary of State) Order 1999 (S.I.~1999/678); and~ministerial functions in respect of Scotland~were further transferred by article 2 of, and~Schedule~1 to, the Scotland~Act 1998 (Transfer of Functions to the Scottish Ministers etc) Order 1999 (S.I.~1999/1750).}.

\medskip

173.  In Schedule~5 (regulations as to~procedure: provision which may be made)—
\begin{enumerate}\item[]
($a$) in paragraph~1($a$)  and~($b$)  omit “, an appeal tribunal or~a Commissioner”; and

($b$) omit paragraphs~2 and~5 to~8.
\end{enumerate}

\section*{\itshape Protection of Children Act 1999}

174.  The Protection of Children Act 1999\footnote{1999 c.~14. Section 9(2) was amended by paragraph~26 of Schedule 4 to the Care Standards Act 2000 (c.~14), paragraph~157 of Schedule 7 to the Criminal Justice and Court Services Act 2000 (c.~43), paragraph~6 of Schedule 14, paragraph~122 of Schedule 21 and Part III of Schedule 23 to the Education Act 2002 (c.~32), paragraph~23 of Schedule 9 and Part I of Schedule 19 to the Education Act 2005 (c.~18), paragraph~38 of Schedule 2 and Part II of Schedule 3 to the Childcare Act 2006 (c.~21), and section 170(3) of the Education and Inspections Act 2006 (c.~40). Subsections~(3A) to (3C) of section 9 were inserted by paragraph~26 of Schedule 4 to the Care Standards Act 2000. The Schedule was amended by paragraph~31 of Schedule 10 to the Tribunals, Courts and Enforcement Act 2007 (c.~15).} is amended as follows.

\medskip

175.  In section~9 (the tribunal)—
\begin{enumerate}\item[]
($a$) omit subsection~(1);

($b$) in subsection~(2) for~the words from the beginning to~“Tribunal” substitute “Tribunal Procedure Rules may make any provision within subsection~(3) in relation to~the proceedings of the First-tier Tribunal (“the Tribunal”)—”;

($c$) for~subsection~(3) substitute—
\begin{quotation}
“(3) The provision within this subsection~is provision—
\begin{enumerate}\item[]
($a$) as to~the circumstances in which applications for~permission may be made; or

($b$) for~obtaining a medical report in a case where the decision appealed against was made on medical grounds.”;
\end{enumerate}
\end{quotation}

($d$) in subsection~(3A) (as that subsection~has effect before the commencement of its repeal by paragraph~8(3)($c$)  of Schedule~9 to~the Safeguarding Vulnerable Groups Act 2006\footnote{2006 c.~47.})—
\begin{enumerate}\item[]
(i) for~“The regulations” substitute “Tribunal Procedure Rules”; and

(ii) omit “; and~the provision that may be made by virtue of subsection~(3)($j$)  and~($k$)  above includes provision in relation to~such investigations”;
\end{enumerate}

($e$) omit subsection~(3B);

($f$) for~subsection~(3C) substitute—
\begin{quotation}
“(3C) Before making in Tribunal Procedure Rules provision within subsection~(3) in relation to~proceedings of the Tribunal on an appeal or~determination within subsection~(2)($c$)  or~($d$), the Tribunal Procedure Committee must consult the Welsh Ministers.”;
\end{quotation}

($g$) omit subsection~(4);

($h$) for~subsection~(5) substitute—
\begin{quotation}
“(5) Any person who without reasonable excuse fails to~comply with any requirement—
\begin{enumerate}\item[]
($a$) which is imposed by Tribunal Procedure Rules in relation to~any of the proceedings of the Tribunal mentioned in subsection~(2) above, and

($b$) which is—
\begin{enumerate}\item[]
(i) a requirement imposing reporting restrictions,

(ii) a requirement in respect of the discovery or~inspection of documents of a kind which could be imposed by a county court, or

(iii) a requirement for~persons to~attend to~give evidence or~produce documents,
\end{enumerate}
\end{enumerate}
is liable on summary conviction to~a fine not exceeding level 3 on the standard scale.”; and
\end{quotation}

($i$) omit subsections~(6) and~(7).
\end{enumerate}

\medskip

176.  In section~12(1) (interpretation) in the definition of “Tribunal” (as that definition has effect before the commencement of its repeal by paragraph~8(4)($a$)  of Schedule~9 to~the Safeguarding Vulnerable Groups Act 2006) for~“tribunal established under section~9 above” substitute “First-tier Tribunal”.

\medskip

177.  Omit the Schedule~(the tribunal).

\section*{\itshape Access to~Justice Act 1999}

178.  In paragraph~2(1) of Schedule~2 to~the Access to~Justice Act 1999\footnote{1999 c.~22.} (Community Legal Service: excluded services) for~paragraph~($g$)  substitute—
\begin{quotation}
“($g$) the First-tier Tribunal under any provision of the Mental Health Act 1983 or~paragraph~5(2) of the Schedule~to~the Repatriation of Prisoners Act 1984, or~the Mental Health Review Tribunal for~Wales,

($ga$) the Upper Tribunal arising out of proceedings within paragraph~($g$),”.
\end{quotation}

\section*{\itshape Immigration and~Asylum Act 1999}

179.  The Immigration and~Asylum Act 1999\footnote{1999 c.~33. Subsection (2A) was inserted into section 103 (as it had effect before the commencement of section 53 of the Nationality, Immigration and Asylum Act 2002) by section 10 of the Asylum and Immigration (Treatment of Claimants, etc) Act 2004 (c.~19). Section 103A is substituted, from a date to be appointed, by section 53 of the 2002 Act.} is amended as follows.

\medskip

180.  In section~94(1) (interpretation of Part~VI) omit the definition of “adjudicator”.

\medskip

181.  Omit section~102 (Asylum Support Adjudicators).

\medskip

182.  In section~103 (appeals) as it has effect before the commencement of section~53 of the Nationality, Immigration and~Asylum Act 2002\footnote{2002 c.~41.}—
\begin{enumerate}\item[]
($a$) in subsections~(1), (2), (2A) and~(7) for~“an adjudicator” substitute “the First-tier Tribunal”;

($b$) in subsections~(3) and~(5) for~“adjudicator” substitute “First-tier Tribunal”;

($c$) in subsection~(3)($b$)  for~“his” substitute “its”; and

($d$) omit subsection~(4).
\end{enumerate}

\medskip

183.  In section~103 (appeals) as it has effect after the commencement of section~53 of the Nationality, Immigration and~Asylum Act 2002—
\begin{enumerate}\item[]
(i) in subsections~(2), (3) and~(5) for~“an adjudicator” substitute “the First-tier Tribunal”;

(ii) in subsection~(5)($b$)  for~“his” substitute “its”; and

(iii) omit subsection~(6).
\end{enumerate}

\medskip

184.  In section~103A(1) (appeals: location of support under section~4 or~95) for~“an adjudicator” substitute “the First-tier Tribunal”.

\medskip

185.  Omit section~104 (Lord Chancellor’s rules).

\medskip

186.  Omit Schedule~10 (Asylum Support Adjudicators).

\section*{\itshape Care Standards Act 2000}

187.  In section~121(1) of the Care Standards Act 2000\footnote{2000 c.~14.} (general interpretation) in the definition of “the Tribunal” for~“tribunal established by section~9 of the 1999 Act” substitute “First-tier Tribunal”.

\section*{\itshape Freedom of Information Act 2000}

188.  In Part~VI of Schedule~1 of the Freedom of Information Act 2000\footnote{2000 c.~36. The entry relating to the Criminal Injuries Compensation Appeals Panel was inserted by article 2 of, and Schedule to, the Freedom of Information (Additional Public Authorities) Order 2002 (S.I.~2002/2623).} (public bodies and~offices) omit the entry relating to~the Criminal Injuries Compensation Appeals Panel.

\section*{\itshape\sloppy\hbadness=1092 Criminal Justice and~Court Services Act 2000}

189.  In section~42(1) of the Criminal Justice and~Court Services Act 2000\footnote{2000 c.~43.} (interpretation of Part~II) in the definition of “the Tribunal” for~“tribunal established by section~9 of the Protection of Children Act 1999” substitute “First-tier Tribunal”.

\section*{\itshape Child Support, Pensions and~Social Security Act 2000}

190.---(1)  Schedule~7 to~the Child Support, Pensions and~Social Security Act 2000\footnote{2000 c.~19.} (housing benefit and~council tax benefit: revisions and~appeals) is amended as follows.

(2) In paragraph~4 (decisions superseding earlier decisions)—
\begin{enumerate}\item[]
($a$) in sub-paragraph~(1)($b$)  for~“of an appeal tribunal or~a Commissioner” substitute “of the First-tier Tribunal or~any decision of the Upper Tribunal which relates to~any such decision”; and

($b$) in sub-paragraph~(2)—
\begin{enumerate}\item[]
(i) for~“tribunal” substitute “First-tier Tribunal”; and

(ii) for~“Commissioner” substitute “Upper Tribunal”.
\end{enumerate}
\end{enumerate}

(3) In paragraph~6 (appeal to~appeal tribunal)—
\begin{enumerate}\item[]
($a$) in the heading for~“appeal tribunal” substitute “First-tier Tribunal”; and

($b$) in sub-paragraphs~(3), (6) and~(9) for~“an appeal tribunal” substitute “the First-tier Tribunal”.
\end{enumerate}

(4) In paragraph~7 (redetermination etc of appeals by tribunal)—
\begin{enumerate}\item[]
($a$) in sub-paragraph~(1) for~the words from “to~a person” to~the end substitute “to~the First-tier Tribunal for~permission to~appeal to~the Upper Tribunal from any decision of the First-tier Tribunal under paragraph~6”;

($b$) omit sub-paragraph~(2); and

($c$) in sub-paragraph~(3)—
\begin{enumerate}\item[]
(i) for~“the person” substitute “the First-tier Tribunal”; and

(ii) for~“tribunal” substitute “First-tier Tribunal”.
\end{enumerate}
\end{enumerate}

(5) In paragraph~8 (appeal from tribunal to~Commissioner)—
\begin{enumerate}\item[]
($a$) in the heading for~“tribunal to~Commissioner” substitute “First-tier Tribunal to~Upper Tribunal”;

($b$) omit sub-paragraph~(1);

($c$) in sub-paragraph~(2)—
\begin{enumerate}\item[]
(i) for~“lies under this paragraph” substitute “to~the Upper Tribunal under section~11 of the Tribunals, Courts and~Enforcement Act 2007 from any decision of the First-tier Tribunal under paragraph~6 or~7 lies”; and

(ii) in paragraph~($c$)  for~“appeal tribunal” substitute “First-tier Tribunal”; and
\end{enumerate}

($d$) omit sub-paragraphs~(3) to~(8).
\end{enumerate}

(6) In paragraph~9 (appeal from Commissioner on point of law)—
\begin{enumerate}\item[]
($a$) for~the heading substitute “Applications for~permission to~appeal against a decision of the Upper Tribunal”;

($b$) omit sub-paragraphs~(1) and~(2); and

($c$) in sub-paragraph~(3)—
\begin{enumerate}\item[]
(i) for~“An application for~leave under this paragraph~in respect of a Commissioner’s decision” substitute “An application for~permission to~appeal from a decision of the Upper Tribunal in respect of a decision of the First-tier Tribunal under paragraph~6 or~7”;

(ii) in paragraph~($a$)  for~“Commissioner”, in both places, substitute “Upper Tribunal”;

(iii) in paragraphs~($a$)  and~($b$)  for~“Commissioner’s” substitute “Upper Tribunal’s”;

(iv)  in paragraph~($c$)  for~“leave” substitute “permission”; and

(v)  omit the words from “and~regulations” to~the end; and
\end{enumerate}

($d$) omit sub-paragraphs~(4) and~(5).
\end{enumerate}

(7) In paragraph~10 (procedure) omit sub-paragraphs~(2) to~(8).

(8) In paragraph~11 (finality of decisions) after “Subject to~the provisions of this Schedule” insert “and~to~any provision made by or~under Chapter~II of Part~I of the Tribunals, Courts and~Enforcement Act 2007”.

(9) In paragraph~12($a$)  (matters arising as respects decisions) for~“, an appeal tribunal or~a Commissioner” substitute “or~the First-tier Tribunal, or~any decision of the Upper Tribunal which relates to~any decision under this Schedule~of the First-Tier Tribunal,”.

(10) In paragraph~13 (suspension in prescribed circumstances)—
\begin{enumerate}\item[]
($a$) in sub-paragraph~(2)—
\begin{enumerate}\item[]
(i) in paragraph~($c$)  for~“an appeal tribunal, a Commissioner” substitute “the First-tier Tribunal, the Upper Tribunal”; and

(ii) in paragraph~($d$)  for~“a Commissioner” substitute “the Upper Tribunal”; and
\end{enumerate}

($b$) in sub-paragraph~(3)($b$)  and~($c$)  for~“leave” substitute “permission”.
\end{enumerate}

(11) In paragraph~16(1)($b$)  (decisions involving issues that arise on appeal in other cases) for~“a Commissioner” substitute “the Upper Tribunal”.

(12) In paragraph~17 (appeals involving issues that arise on appeal in other cases)—
\begin{enumerate}\item[]
($a$) in sub-paragraph~(1)—
\begin{enumerate}\item[]
(i) in paragraph~($a$)  for~“an appeal tribunal, or~from an appeal tribunal to~a Commissioner” substitute “the First-tier Tribunal, or~from the First-tier Tribunal to~the Upper Tribunal”; and

(ii) in paragraph~($b$)  for~“a Commissioner” substitute “the Upper Tribunal”;
\end{enumerate}

($b$) in sub-paragraph~(2) for~“tribunal or~Commissioner” substitute “First-tier Tribunal or~Upper Tribunal”;

($c$) in sub-paragraph~(3)($a$)  and~($b$)  for~“tribunal” substitute “First-tier Tribunal”;

($d$) in sub-paragraph~(4)—
\begin{enumerate}\item[]
(i) for~“appeal tribunal or~Commissioner” substitute “First-tier Tribunal or~Upper Tribunal”; and

(ii) in paragraph~($b$)  for~“tribunal or~Commissioner” substitute “First-tier Tribunal or~Upper Tribunal”;
\end{enumerate}

($e$) in sub-paragraph~(5)—
\begin{enumerate}\item[]
(i) for~“appeal tribunal or~Commissioner” substitute “First-tier Tribunal or~Upper Tribunal”; and

(ii) for~“tribunal or~Commissioner” substitute “First-tier Tribunal or~Upper Tribunal”; and
\end{enumerate}

($f$) in sub-paragraph~(7)($a$)  for~“a Commissioner”, in both places, substitute “the Upper Tribunal”.
\end{enumerate}

(13) In paragraph~18 (restrictions on entitlement to~benefit in certain cases of error)—
\begin{enumerate}\item[]
($a$) in sub-paragraph~(1)($a$)  for~“by virtue of this Schedule~to~a Commissioner” substitute “to~the Upper Tribunal”;

($b$) in sub-paragraph~(3) for~“the Commissioner” substitute “the Upper Tribunal”; and

($c$) in sub-paragraph~(9)($a$)  and~($b$)  for~“a Commissioner” substitute “the Upper Tribunal”.
\end{enumerate}

(14) In paragraph~19 (correction of errors and~setting aside of decisions)—
\begin{enumerate}\item[]
($a$) in sub-paragraph~(1)—
\begin{enumerate}\item[]
(i) in paragraph~($a$)  after “record of a decision made” insert “by the relevant authority”; and

(ii) omit paragraph~($b$)  (together with the “and” immediately before it); and
\end{enumerate}

($b$) in sub-paragraph~(2) omit “or~set aside decisions”.
\end{enumerate}

(15) In paragraph~20 (regulations)—
\begin{enumerate}\item[]
($a$) in paragraph~(1) for~the words from “exercisable—” to~the end substitute “exercisable by the Secretary of State”; and

($b$) omit sub-paragraph~(6).
\end{enumerate}

(16) In paragraph~23 (interpretation) omit the definitions of “appeal tribunal”, “the Chief Commissioner” and~“Commissioner”.

\section*{\itshape Tax Credits Act 2002}

191.---(1)  Section~63 of the Tax Credits Act 2002\footnote{2002 c.~21.} (tax credits appeals etc: temporary modifications) is amended as follows.

(2) In subsections~(2), (3), (4), (6) and~(8) for~“an appeal tribunal” substitute “the appropriate tribunal”.

(3) In subsection~(5)($a$)  for~“appeal tribunal” substitute “appropriate tribunal”.

(4) In subsection~(6) for~“a Social Security Commissioner” substitute “the Upper Tribunal or~a Northern Ireland~Social Security Commissioner”.

(5) In subsection~(7) for~“the Social Security Commissioner” substitute “the Upper Tribunal or~the Northern Ireland~Social Security Commissioner”.

(6) In subsection~(8) for~“a Social Security Commissioner” substitute “the Upper Tribunal or~a Northern Ireland~Social Security Commissioner”.

(7) For~subsection~(10) substitute—
\begin{quotation}
“(10) “Appropriate tribunal” means—
\begin{enumerate}\item[]
($a$) the First-tier Tribunal, or

($b$) an appeal tribunal constituted under Chapter~I of Part~II of the Social Security (Northern Ireland) Order 1998.”.
\end{enumerate}
\end{quotation}

(8) In subsection~(13) for~the words from the beginning to~“in Northern Ireland,” substitute ““Northern Ireland~Social Security Commissioner” means”.

\section*{\itshape Education Act 2002}

192.  The Education Act 2002\footnote{2002 c.~32. Section 167B is inserted, from a date to be appointed, by section 169 of the Education and Inspections Act 2006 (c.~40).} is amended as follows.

\medskip

193.  In section~144(1) (directions under section~142: appeal) for~“Tribunal established under section~9 of the Protection of Children Act 1999 (c.~14)” substitute “First-tier Tribunal”.

\medskip

194.  In section~166(1) (appeals) for~“tribunal established under section~9 of the Protection of Children Act 1999 (c.~14)” substitute “First-tier Tribunal”.

\medskip

195.  In subsection~167(1) (determination of appeals) for~“tribunal established under section~9 of the Protection of Children Act 1999 (c.~14)” substitute “First-tier Tribunal”.

\medskip

196.  In section~167B(1) (directions under section~167A: appeals) for~“Tribunal established under section~9 of the Protection of Children Act 1999” substitute “First-tier Tribunal”.

\section*{\itshape\sloppy Nationality, Immigration and~Asylum Act 2002}

197.  In section~36(6) of the Nationality, Immigration and~Asylum Act 2002\footnote{2002 c.~41.} (education: general) for~“Special Educational Needs Tribunal” substitute “First-tier Tribunal or~the Special Educational Needs Tribunal for~Wales”.

\section*{\itshape Health and~Social Care (Community Health and~Standards) Act 2003}

198.  The Health and~Social Care (Community Health and~Standards) Act 2003\footnote{2003 c.~43.} is amended as follows.

\medskip

199.  In section~157 (appeal against a certificate or~a waiver decision)—
\begin{enumerate}\item[]
($a$) in subsection~(6) for~“sections~158 and~159” substitute “section~158”; and

($b$) in subsection~(7) omit paragraph~($c$)  (but not the “and” at the end of the paragraph).
\end{enumerate}

\medskip

200.  In section~158 (appeal tribunals)—
\begin{enumerate}\item[]
($a$) in subsection~(1) for~“an appeal tribunal constituted under Chapter~I of Part~I of the Social Security Act 1998 (c.~14)” substitute “the First-tier Tribunal”;

($b$) in subsection~(4) for~“the tribunal” substitute “a tribunal”; and

($c$) omit subsection~(7).
\end{enumerate}

\medskip

201.  Omit section~159 (appeal to~Social Security Commissioner).

\section*{\itshape Child Trust Funds Act 2004}

202.---(1)  Section~24 of the Child Trust Funds Act 2004\footnote{2004 c.~6.} (temporary modifications) is amended as follows.

(2) In subsection~(2)—
\begin{enumerate}\item[]
($a$) for~“appeal tribunal”, in both places, substitute “appropriate tribunal”;

($b$) for~“a Social Security Commissioner” substitute “the Upper Tribunal or~a Northern Ireland~Social Security Commissioner”; and

($c$) for~“the Social Security Commissioner” substitute “the Upper Tribunal or~the Northern Ireland~Social Security Commissioner”.
\end{enumerate}

(3) In subsection~(3)—
\begin{enumerate}\item[]
($a$) for~“an appeal tribunal” substitute “the appropriate tribunal”; and

($b$) in paragraph~($b$)  for~“appeal tribunal” substitute “appropriate tribunal”.
\end{enumerate}

(4) In subsection~(4) for~“an appeal tribunal” substitute “the appropriate tribunal”.

(5) In subsection~(5)—
\begin{enumerate}\item[]
($a$) for~“an appeal tribunal” substitute “the appropriate tribunal”; and

($b$) for~“a Social Security Commissioner” substitute “the Upper Tribunal or~a Northern Ireland~Social Security Commissioner”.
\end{enumerate}

(6) For~subsection~(6) substitute—
\begin{quotation}
“(6) “Appropriate tribunal” means---
\begin{enumerate}\item[]
($a$) the First-tier Tribunal, or

($b$) an appeal tribunal constituted under Chapter~I of Part~II of the Social Security (Northern Ireland) Order 1998.”.
\end{enumerate}
\end{quotation}

(7) In subsection~(7) for~the words from the beginning to~“in Northern Ireland,” substitute ““Northern Ireland~Social Security Commissioner” means”.

\section*{\itshape\sloppy\hbadness=10000 Asylum and~Immigration (Treatment of Claimants, etc) Act 2004}

203.  In section~9(4) of the Asylum and~Immigration (Treatment of Claimants, etc) Act 2004\footnote{2004 c.~19.} (failed asylum seekers: withdrawal of support) for~“adjudicator” substitute “First-tier Tribunal”.

\section*{\itshape Domestic Violence, Crime and~Victims Act 2004}

204.  The Domestic Violence, Crime and~Victims Act 2004\footnote{2004 c.~28. Section 37A was inserted by paragraph 5 of Schedule 6 to the Mental Health Act 2007 (c.~12). Section 38A was inserted by paragraph 7 of Schedule 6 to the 2007 Act. Section 43A was inserted by paragraph 13 of Schedule 6 to the 2007 Act. Section 44A was inserted by paragraph 15 of Schedule 6 to the 2007 Act.} is amended as follows.

\medskip

205.  In section~37(5) (representations where restriction order made) for~“A Mental Health Review Tribunal” substitute “The First-tier Tribunal or~the Mental Health Review Tribunal for~Wales”.

\medskip

206.  In section~37A(6) (representations where restriction order not made) for~“A Mental Health Review Tribunal” substitute “The First-tier Tribunal or~the Mental Health Review Tribunal for~Wales”.

\medskip

207.  In section~38(5)($a$)  and~($b$)  (information where restriction order made) for~“a Mental Health Review Tribunal” substitute “the First-tier Tribunal or~the Mental Health Review Tribunal for~Wales”.

\medskip

208.  In section~38A(4)($a$)  to~($c$)  (information where restriction order not made) for~“a Mental Health Review Tribunal” substitute “the First-tier Tribunal or~the Mental Health Review Tribunal for~Wales”.

\medskip

209.  In section~40(5) (representations) for~“A Mental Health Review Tribunal” substitute “The First-tier Tribunal or~the Mental Health Review Tribunal for~Wales”.

\medskip

210.  In section~41(5)($a$)  and~($b$)  (information) for~“a Mental Health Review Tribunal” substitute “the First-tier Tribunal or~the Mental Health Review Tribunal for~Wales”.

\medskip

211.  In section~43(5) (representations where restriction direction made) for~“A Mental Health Review Tribunal” substitute “The First-tier Tribunal or~the Mental Health Review Tribunal for~Wales”.

\medskip

212.  In section~43A(6) (representations where restriction direction not given) for~“A Mental Health Review Tribunal” substitute “The First-tier Tribunal or~the Mental Health Review Tribunal for~Wales”.

\medskip

213.  In section~44(5)($a$)  and~($b$)  (information where restriction direction made) for~“a Mental Health Review Tribunal” substitute “the First-tier Tribunal or~the Mental Health Review Tribunal for~Wales”.

\medskip

214.  In section~44A(4)($a$)  to~($c$)  (information where restriction direction not given) for~“a Mental Health Review Tribunal” substitute “the First-tier Tribunal or~the Mental Health Review Tribunal for~Wales”.

\medskip

215.  In paragraph~17 of Schedule~9 (authorities within the remit of the Commissioner for~Victims and~Witnesses) for~“Criminal Injuries Compensation Appeals Panel” substitute “Persons exercising functions relating to~the carrying on of the business of the First-tier Tribunal in respect of appeals under the Criminal Injuries Compensation Scheme by virtue of section~5(1) of the Criminal Injuries Compensation Act 1995”.

\section*{\itshape Constitutional Reform Act 2005}

216.  The Constitutional Reform Act 2005\footnote{2005 c.~4. Sections 3(7B) and 94B were inserted by sections~1 and 53(1) and (2) of the Tribunals, Courts and Enforcement Act 2007 (c.~15) respectively.} is amended as follows.

\medskip

217.  Omit section~3(7B)($f$)  (guarantee of continued judicial independence).

\medskip

218.  In the table in section~94B(3) (appointments not subject to~section~85: tribunals) omit the entries relating to~a Deputy Child Support Commissioner and~Deputy Social Security Commissioner.

\medskip

219.---(1)  Paragraph 4 of Schedule~7 (protected functions of the Lord Chancellor~under particular enactments) is amended as follows.

(2) In the entry relating to~the Child Support Act 1991—
\begin{enumerate}\item[]
($a$) omit “Section~22”, “Section~24” and~“Section~25”; and

($b$) in the entry relating to~Schedule~4, for~“1(3), 2(1) and~(2), 2A(1), 4(1), 4A(1) and~7” substitute “2(1) and~(2)”.
\end{enumerate}

(3) Omit the entry relating to~the Protection of Children Act 1999.

(4) Omit the entry relating to~the Child Support, Pensions and~Social Security Act 2000.

\medskip

220.---(1)  Schedule~14 (the Judicial Appointments Commission: relevant offices and~enactments) is amended as follows.

(2) In Part~I (appointments by Her Majesty) omit the entries relating to—
\begin{enumerate}\item[]
($a$) a Chief Child Support Commissioner and~Child Support Commissioner; and

($b$) a Chief Social Security Commissioner and~Social Security Commissioner.
\end{enumerate}

(3) In Part~III (appointments by the Lord Chancellor: offices to~which paragraph~2(2)($d$)  of Schedule~12 applies) omit the entries relating to—
\begin{enumerate}\item[]
($a$) a Deputy Child Support Commissioner;

($b$) the President of the Special Educational Needs and~Disability Tribunal and~a member of the chairmen’s panel of that tribunal;

($c$) the President of appeal tribunals appointed under section~5(1) of the Social Security Act 1998;

($d$) a member of the panel of persons appointed under section~6(2) of that Act to~act as members of appeal tribunals;

($e$) a Social Security Commissioner (deputy);

($f$) the President of the Tribunal, and~a member of the chairmen’s panel of the Tribunal, appointed under paragraph~2(1) of the Schedule~to~the Protection of Children Act 1999; and

($g$) a member of the lay panel of the Tribunal appointed under paragraph~2(3) of that Schedule.
\end{enumerate}

\section*{\itshape Childcare Act 2006}

221.  In section~69(11) of the Childcare Act 2006\footnote{2006 c.~21.} (suspension of registration) for~“Tribunal established by section~9 of the Protection of Children Act 1999 (c.~14)” substitute “First-tier Tribunal”.

\section*{\itshape Safeguarding Vulnerable Groups Act 2006}

222.  In section~4 of the Safeguarding Vulnerable Groups Act 2006\footnote{2006 c.~47.} (appeals)—
\begin{enumerate}\item[]
($a$) before “Tribunal”, in each place, insert “Upper”; and

($b$) omit subsections~(8) to~(11).
\end{enumerate}

\section*{\itshape Mental Health Act 2007}

223.  In paragraph~2(2)($b$)  of Schedule~10 to~the Mental Health Act 2007\footnote{2007 c.~12.} (transitional provisions and~savings) for~“a Mental Health Review Tribunal” substitute “the First-tier Tribunal or~the Mental Health Review Tribunal for~Wales”.

\section*{\itshape Child Maintenance and~Other Payments Act 2008}

224.  The Child Maintenance and~Other Payments Act 2008\footnote{2008 c.~6.} is amended as follows.

\medskip

225.  In section~6 (fees)—
\begin{enumerate}\item[]
($a$) in subsection~(5) for~“an appeal tribunal” substitute “the First-tier Tribunal”; and

($b$) in subsection~(6) for~“appeal tribunals” substitute “First-tier Tribunal”.
\end{enumerate}

\medskip

226.  In section~50 (appeal to~appeal tribunal)—
\begin{enumerate}\item[]
($a$) in the heading for~“appeal tribunal” substitute “First-tier Tribunal”;

($b$) in subsection~(2) for~the words from “an appeal tribunal” to~the end substitute “the First-tier Tribunal”; and

($c$) omit subsection~(4)($b$).
\end{enumerate}

\medskip

227.  Omit section~51 (appeal to~Social Security Commissioner).

\section*{\itshape Repeals and~revocations}

228.  In consequence of the amendments made by the above provisions of this Schedule, the following provisions are repealed or~(as the case may be) revoked—
\begin{enumerate}\item[]
($a$) paragraph~63(1), (2) and~(3)($a$)  of Schedule~2 to~the Social Security (Consequential Provisions) Act 1992\footnote{1992 c.~6.};

($b$) paragraph~23(4) of Schedule~6 to~the Judicial Pensions and~Retirement Act 1993\footnote{1993 c.~8.};

($c$) section~17 of, and~paragraph~18 of Schedule~3 to, the Child Support Act 1995\footnote{1995 c.~34.};

($d$) paragraphs~3(1), 11, 29, 30, 36, 42, 47($a$), 51, 52, 113($b$), 152(1) and~(3) and~153 of Schedule~7 to~the Social Security Act 1998\footnote{1998 c.~14.};

($e$) paragraph~8 of the Schedule~to~the Protection of Children Act 1999\footnote{1999 c.~14.};

($f$) paragraphs~71($b$), 72($b$)  and~95 of Schedule~14 to~the Immigration and~Asylum Act 1999\footnote{1999 c.~33.};

($g$) paragraph~21 of Schedule~4 to~the Care Standards Act 2000\footnote{2000 c.~14.};

($h$) paragraph~22(2) and~(3) of Schedule~7 to~the Child Support, Pensions and~Social Security Act 2000\footnote{2000 c.~19.};

($i$) paragraph~20($b$)  of Schedule~8 to~the Special Educational Needs and~Disability Act 2001\footnote{2001 c.~10.};

($j$) paragraph~22 of Schedule~3 and~paragraph~47 of Schedule~12 to~the Justice (Northern Ireland) Act 2002\footnote{2002 c.~26.};

($k$) paragraphs~4, 5 and~10(2) of Schedule~18 to~the Education Act 2002\footnote{2002 c.~32.};

($l$) section~7(2) of the Armed Forces (Pensions and~Compensation) Act 2004\footnote{2004 c.~32.};

($m$) paragraphs~25, 28(3)($d$)  and~221 of Schedule~4 to~the Constitutional Reform Act 2005\footnote{2005 c.~4.};

($n$) paragraphs~273 and~274 of Schedule~4 to~the Constitutional Reform Act 2005;

($o$) paragraph~1(4)($a$), (8) and~(9) of Schedule~7 to~the Welfare Reform Act 2007\footnote{2007 c.~5.};

($p$) section~38(3)($b$)  and~($c$), (4) and~(8) of the Mental Health Act 2007\footnote{2007 c.~12.};

($q$) paragraphs~22(2), (4) and~(5) and~31 of Schedule~10 to~the Tribunals, Courts and~Enforcement Act 2007\footnote{2007 c.~15.};

($r$) paragraph~29 of Schedule~10 to~the Tribunals, Courts and~Enforcement Act 2007;

\eject

($s$) paragraphs~16(3) to~(5), 17 and~54 of Schedule~3 to~the Child Maintenance and~Other Payments Act 2008\footnote{2008 c.~6.};

($t$) the entries in the Schedule~to~the Transfer of Functions (Lord Advocate and~Secretary of State) Order 1999\footnote{S.I. 1999/678.} relating to~sections~22, 24 and~25 of, and~Schedule~4 to, the Child Support Act 1991;

($u$) paragraph~2 of Schedule~10 to~the Scotland~Act 1998 (Cross-Border Public Authorities) (Adaptation of Functions etc) Order 1999\footnote{S.I.~1999/1747.};

($v$)  the entries in Schedule~1 to~the Scotland~Act 1998 (Transfer of Functions to~the Scottish Ministers etc) Order 1999\footnote{S.I.~1999/1750.} relating to~sections~22(3),~24(9) and~25(6) of, and~Schedule~4 to, the Child Support Act 1991;

($w$) the Judicial Pensions and~Retirement Act 1993 (Addition of Qualifying Judicial Offices) (No.~2) Order 2003\footnote{S.I.~2003/2589.};

($x$) paragraph~53 of Schedule~1 to~the Government of Wales Act 2006 (Consequential Modifications and~Transitional Provisions) Order 2007\footnote{S.I.~2007/1388.};

($y$) the Judicial Pensions and~Retirement Act 1993 (Addition of Qualifying Judicial Offices) (No.~2) Order 2007\footnote{S.I.~2007/2185.}.
\end{enumerate}

\part[Schedule 4 --- Transitional provisions]{Schedule 4\\*Transitional provisions}

\renewcommand\parthead{--- Schedule~4}

\section*{\itshape Transitional provisions}

1.  Subject to~article~3(3)($a$)  any proceedings before a tribunal listed in Table~1 of Schedule~1 which are pending immediately before 3rd November 2008 shall continue on and~after 3rd November 2008 as proceedings before the First-tier Tribunal.

\medskip

2.  Subject to~article~3(3)($b$)  any proceedings before a tribunal listed in Table~2 of Schedule~1 which are pending immediately before 3rd November 2008 shall continue on and~after 3rd November 2008 as proceedings before the Upper Tribunal.

\medskip

3.---(1)  The following sub-paragraphs~apply where proceedings are continued in the First-tier Tribunal or~Upper Tribunal by virtue of paragraph~1 or~2.

(2) Where a hearing began before 3rd November 2008 but was not completed by that date, the First-tier Tribunal or~the Upper Tribunal, as the case may be, must be comprised for~the continuation of that hearing of the person or~persons who began it.

(3) The First-tier Tribunal or~Upper Tribunal, as the case may be, may give any direction to~ensure that proceedings are dealt with fairly and, in particular, may—
\begin{enumerate}\item[]
($a$) apply any provision in procedural rules which applied to~the proceedings before 3rd November 2008; or

($b$) disapply provisions of Tribunal Procedure Rules.
\end{enumerate}

(4) In sub-paragraph~(3) “procedural rules” means provision (whether called rules or~not) regulating practice or~procedure before a tribunal.

(5) Any direction or~order given or~made in proceedings which is in force immediately before 3rd November 2008 remains in force on and~after that date as if it were a direction or~order of the First-tier Tribunal or~Upper Tribunal, as the case may be.

(6) A time period which has started to~run before 3rd November 2008 and~which has not expired shall continue to~apply.

(7) An order for~costs may only be made if, and~to~the extent that, an order could have been made before 3rd November 2008.

\medskip

4.  Subject to~article~3(3)($a$)  and~($b$)  where an appeal lies to~a Child Support or~Social Security Commissioner from any decision made before 3rd November 2008 by a tribunal listed in Table~1 of Schedule~1, section~11 of the 2007 Act (right to~appeal to~Upper Tribunal) shall apply as if the decision were a decision made on or~after 3rd November 2008 by the First-tier Tribunal.

\medskip

5.  Subject to~article~3(3)($b$)  where an appeal lies to~a court from any decision made before 3rd November 2008 by a Child Support or~Social Security Commissioner, section~13 of the 2007 Act (right to~appeal to~Court of Appeal etc.) shall apply as if the decision were a decision made on or~after 3rd November 2008 by the Upper Tribunal.

\medskip

6.  Subject to~article~3(3)($a$)  and~($b$)  any case to~be remitted by a court on or~after 3rd November 2008 in relation to~a tribunal listed in Schedule~1 shall be remitted to~the First-tier Tribunal or~Upper Tribunal as the case may be.

\section*{\itshape Savings provisions}

7.---(1)  Section~78(8) of the Mental Health Act 1983\footnote{1983 c.~20.} shall continue to~apply to~any decision given by a Mental Health Review Tribunal before 3rd November 2008 as if the amendments to~it in Schedule~3 had not been made.

(2) Section~11(1) of the Tribunals and~Inquiries Act 1992\footnote{1992 c.~53; the relevant amendment is made by paragraphs 19 and 20($b$) of Schedule 8 to the Special Educational Needs and Disability Act 2001 (c.~10).} shall continue to~apply to~any decision given by the Special Educational Needs and~Disability Tribunal or~the Special Educational Needs Tribunal for~Wales before 3rd November 2008 as if the amendments to~it in Schedule~3 had not been made.

(3) Section~9(6) of the Protection of Children Act 1999\footnote{1999 c.~14.} shall continue to~apply to~any decision given by the tribunal under section~9(1) of that Act before 3rd November 2008 as if the amendments to~it in Schedule~3 had not been made. 

\part{Explanatory Note}

\renewcommand\parthead{— Explanatory Note}

\subsection*{(This note is not part of the Order)}

This Order is made under the Tribunals, Courts and~Enforcement Act 2007 (“the 2007 Act”). Part~I of the 2007 Act creates a new two tier tribunal structure; the First-tier Tribunal and~the Upper Tribunal (“the new tribunals”) are established under section~3 of the 2007 Act. Order making powers are provided under Part~I of the 2007 Act to~enable existing tribunals to~be transferred into~the new structure. This Order has various primary functions to~effect the transfers, and~in addition contains various minor, consequential and~transitional provisions, as explained below.

\section*{Transfer of functions of tribunals}

Article~3 transfers the functions of the tribunals listed in the tables in Schedule~1 to~the new tribunals.

Appeals under the Health and~Social Care (Community Health and~Standards) Act 2003 are made to~the appeal tribunal constituted under Chapter~I of Part~I of the Social Security Act 1998 and~onward appeals from that tribunal are to~the Social Security Commissioners. The functions related to~these appeal rights in respect of Scotland~are not transferred as part of this Order. Therefore, the appeal tribunal and~Social Security Commissioners are retained for~the purposes of these appeals. The extent provisions in article~1(5) retain the relevant legislation for~the purpose of these appeals.

In the case of Pensions Appeal Tribunals the transfer relates only to~tribunals in England~and~Wales; Pensions Appeal Tribunals in Scotland~and~Northern Ireland~retain their functions. In the case of Mental Health Review Tribunals the transfer relates only to~tribunals in England; the Mental Health Review Tribunal for~Wales retains its functions. In the special educational needs regime the functions of the Special Educational Needs and~Disability Tribunal (which relates only to~England) are transferred but the functions of the Special Educational Needs Tribunal for~Wales are not.

\section*{\sloppy Abolition of tribunals following transfer of functions}

Article~4 abolishes the tribunals from which the functions are transferred under article~3 (with exceptions to~provide for~tribunals to~remain in place to~hear the Scottish appeals which are not transferred).

\section*{Transfer of members of tribunals}

Article~5 provides for~members of the tribunals from which the functions are transferred by article~3 to~hold the offices of transferred-in judge or~transferred-in other member of the First-tier Tribunal, or~deputy judge or~transferred-in judge of the Upper Tribunal. Those members becoming deputy judges of the Upper Tribunal also become transferred-in judges of the First-tier Tribunal. The tables in Schedule~2 set out which tribunal members hold which offices in the new tribunals.

\section*{Additions to~Schedule~6 to~the 2007 Act}

Article~2 adds three tribunals to~the table in Part~IV of Schedule~6 to~the 2007 Act, bringing them within the scope of the Lord Chancellor’s power to~transfer tribunal functions to~the First-tier Tribunal or~the Upper Tribunal. None of the three tribunals are transferred to~the new tribunals in this Order.

\section*{Appeals to~the Upper Tribunal from tribunals in Wales, Scotland~and~Northern Ireland}

Article~6 provides for~an onward appeal right to~the Upper Tribunal from decisions of the Mental Health Review Tribunal for~Wales and~the Special Educational Needs Tribunal for~Wales in place of the previous onward appeal right to~the High Court.

The transfer of the Pensions Appeal Tribunal for~England~and~Wales to~the First-tier Tribunal creates an onward appeal right for~decisions under section~5 of the Pensions Appeal Tribunals Act 1943. To ensure parity across the jurisdictions articles 7 and~8 provide for~an onward appeal from the Pensions Appeal Tribunals for~Scotland~and~Northern Ireland~to~the Upper Tribunal for~decisions under section~5 of the 1943 Act.

\section*{Minor~and~consequential provisions}

Article~9 brings Schedule~3 into~effect. Schedule~3 contains minor~and~consequential amendments.

The amendments in Schedule~3 are amendments to~primary legislation resulting from the transfer of tribunal functions and~members, abolition of tribunals and~new appeal rights provided for~in this Order. To a great extent the amendments redirect the existing appeal rights by the replacement of references to~the tribunals being abolished with references to~the tribunals to~which the functions are being transferred. The Order also deletes provisions relating to~the abolished tribunals where the effects of those provisions are or~will be provided for~in the Tribunals, Courts and~Enforcement Act 2007 or~in Tribunal Procedure Rules made under that Act.

Amendments to~the Mental Health Act 1983 ensure that that Act as amended refers to~the First-tier Tribunal so far as proceedings relating to~England~are concerned, but to~the Mental Health Review Tribunal for~Wales (“MHRTfW”) so far as proceedings relating to~Wales are concerned, and~retains provisions relating to~the constitution and~procedure of the MHRTfW. That Act as amended also includes provision for~the new appeal right from the MHRTfW to~the Upper Tribunal.

Amendments to~the Disability Discrimination Act 1995 and~the Education Act 1996 ensure that those Acts as amended refer to~the First-tier Tribunal so far as proceedings relating to~England~are concerned, but to~the Special Educational Needs Tribunal for~Wales (“SENTfW”) so far as proceedings relating to~Wales are concerned, and~retain provisions relating to~the constitution and~procedure of the SENTfW. Those Acts as amended also include provision for~the new appeal right from the SENTfW to~the Upper Tribunal.

Amendments to~the War Pensions (Administrative Provisions) Act 1919 and~the Pensions Appeal Tribunal Act 1943 ensure that those Acts as amended refer to~the First-tier Tribunal so far as proceedings relating to~England~and~Wales are concerned, but to~the Pensions Appeal Tribunals so far as proceedings relating to~Scotland~and~Northern Ireland~are concerned. Onward appeals from decisions of the Pension Appeals Tribunals for~Scotland~are to~the Upper Tribunal. Onward appeals from decisions of the Pensions Appeal Tribunal for~Northern Ireland~remain (apart from the new appeal right under section~5 of the 1943 Act) to~the Social Security Commissioners for~Northern Ireland.

Article~104(5)($d$)  inserts a minor~amendment into~paragraph~3 of Part~II of Schedule~4 to~the Social Security Administration Act 1992. This amendment is in consequence of the creation of the Administrative Justice and~Tribunals Council and~the abolition of the Council on Tribunals under sections~44 and~445 of the 2007 Act.

\section*{Transitional and~saving provisions}

Article~9(2) brings Schedule~4 into~effect. Schedule~4 make transitional and~saving provisions for~the treatment of cases which would previously have been dealt with by the tribunals from which the functions are transferred by article~3, or~onward appeals from those tribunals, following the coming into~force of this Order.

The Schedule~provides for~proceedings which have been started in tribunals from which the functions are transferred by article~3 to~be transferred to~the new tribunals; new proceedings will be started in the new tribunals. In transferred cases the following provisions apply:
\begin{enumerate}\item[]
    a hearing which has already been commenced but not completed will need to~be completed in the new tribunal but comprised of the same members;

    directions and~orders made prior~to~this Order coming into~force will continue in force as if they were directions or~orders of the new tribunals;

    Tribunal Procedure Rules made under the Tribunals, Courts and~Enforcement Act 2007 will apply to~all cases from day one, but the new tribunals will be able to~disapply Tribunal Procedure Rules, apply procedural rules which applied to~the abolished tribunals or~make other directions to~ensure that proceedings are dealt with fairly;

    time limits which begin to~run before this Order comes into~force continue to~apply after the Order comes into~force; and

    the new tribunals will only be able to~make a costs order if and~to~the extent that the tribunal from which the case was transferred could have made such an order. 
\end{enumerate}

Onward appeals against the decisions of tribunals from which the functions are transferred by article~3 are dealt with as follows:
\begin{enumerate}\item[]
    onward appeals against decisions given before 3rd November 2008 by a Mental Health Review Tribunal, the Special Educational Needs and~Disability Tribunal, the Special Educational Needs Tribunal for~Wales or~the “Care Standards Tribunal” will continue to~be made to~the High Court, even if the onward appeal proceedings are not commenced until after that date. Onward appeals against decisions given on or~after 3rd November 2008 must be made under the new regime;

    onward appeals against decisions given before 3rd November 2008 which would before that date have been heard by a Social Security Commissioner or~Child Support Commissioner, but which are made after that date, must be made under the new regime; and

    onward appeals against decisions given by a Social Security Commissioner or~Child Support Commissioner before 3rd November 2008, which are made after that date, must be made under the new regime. 
\end{enumerate}

A Regulatory Impact Assessment was prepared for~the Tribunals, Courts and~Enforcement Act 2007. This can be found at:
\url{http://www.justice.gov.uk/publications/tribunalscourtsandenforcementact.htm}

\end{document}