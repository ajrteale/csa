\documentclass[12pt,a4paper]{article}

\newcommand\regstitle{The Child Support (Meaning of Child and New Calculation Rules) (Consequential and Miscellaneous Amendment) Regulations 2012}

\newcommand\regsnumber{2012/2785}

\title{\regstitle}

\author{S.I.\ 2012 No.\ 2785}

\date{Made
2nd November 2012\\
Laid before Parliament
8th November 2012\\
Coming into force
in accordance with regulation 1
}

%\opt{oldrules}{\newcommand\versionyear{1993}}
%\opt{newrules}{\newcommand\versionyear{2003}}
%\opt{2012rules}{\newcommand\versionyear{2012}}

\usepackage{csa-regs}

\setlength\headheight{42.11603pt}

%\hbadness=10000

\begin{document}

\maketitle

\enlargethispage{\baselineskip}

\noindent
The Secretary of State makes the following Regulations in exercise of the powers conferred by sections 14(1), 29(3) and (3A), 51(1) and (2), 52(4), 54 and 55(1) of the Child Support Act 1991\footnote{1991 c.~48. Section 29(3) was substituted, and section 29(3A) was inserted, by the Welfare Reform Act 2009 (c.~24). Section 32C was inserted by section 22 of the Child Maintenance and Other Payments Act 2008 (c.~6) (“the 2008 Act”). Section 54 is cited because of the meaning given to the word “prescribed”. Section 55 was substituted by section 42 of the 2008 Act.} and sections 55(4) and 57(2) of the Child Maintenance and Other Payments Act 2008\footnote{2008 c.~6.}. 

{\sloppy

\tableofcontents

}

\bigskip

\setcounter{secnumdepth}{-2}

\section[Part I --- General]{Part I\\*General}

\renewcommand\parthead{--- Part I}

\subsection[1. Citation, commencement and interpretation]{Citation, commencement and interpretation}

1.---(1)  These Regulations may be cited as the Child Support (Meaning of Child and New Calculation Rules) (Consequential and Miscellaneous Amendment) Regulations 2012.

(2) This regulation and regulation 11 come into force on 10th December 2012.

(3) Regulations 2 and 3 come into force on the day on which section 42 of the 2008 Act (meaning of “child”) comes into force.

(4) Subject to paragraph (5), regulations 4 to 10 and 12 come into force in relation to a particular case on the day on which paragraph 2 of Schedule~4 to the 2008 Act (calculation by reference to gross weekly income) comes into force in relation to that type of case.

(5) Regulations 4(3) to (6) and 12 come into force in relation to an arrears-only case on 10th December 2012, subject to the saving in regulation~11(1).

(6) In these Regulations—
\begin{enumerate}\item[]
“2008 Act” means the Child Maintenance and Other Payments Act 2008.

%“arrears of child support maintenance” means—
%\begin{enumerate}\item[]
%($a$) 
%any payment of child support maintenance which has become due in relation to a maintenance assessment, or a maintenance calculation made under 2003 scheme rules, and not paid; and
%
%($b$) 
%the Secretary of State is arranging for the collection of that maintenance under section 29 of the 1991 Act;
%\end{enumerate}

% Definition of ``arrears of child support maintenance'' substituted (30.9.13) by SI 2013/1517 reg 9
“arrears of child support maintenance” means any payment of child support maintenance—
\begin{enumerate}\item[]
($a$) 
which has become due in relation to a maintenance assessment, or a maintenance calculation made under 2003 scheme rules, and not paid; and

($b$) 
in respect of which the Secretary of State is arranging collection under section 29 of the 1991 Act;
\end{enumerate}

“arrears-only case” means a case in which—
\begin{enumerate}\item[]
($a$) 
there are arrears of child support maintenance; and

($b$) 
there is—
\begin{enumerate}\item[]
(i) 
no maintenance assessment, or maintenance calculation made under 2003 scheme rules, still in force; and

(ii) 
no application for a maintenance assessment, or a maintenance calculation falling to be made under 2003 scheme rules, still to be determined;
\end{enumerate}
\end{enumerate}

“the Collection and Enforcement Regulations” means the Child Support (Collection and Enforcement) Regulations 1992\footnote{S.I.~1992/1989.}.
\end{enumerate}

(7) For the purposes of this regulation, a maintenance calculation is made (or will fall to be made) under 2003 scheme rules if the amount of the periodical payments required to be paid in accordance with it is (or will be) determined otherwise than in accordance with Part~I of Schedule 1 to the Child Support Act 1991 as amended by Schedule 4 to the Child Maintenance and Other Payments Act 2008.

\amendment{
Definition of ``arrears of child support maintenance'' in reg.~1(6) substituted (30.9.13) by the Child Support (Miscellaneous Amendments) Regulations 2013 reg.~9.
}

\section[Part II --- Meaning of “Child”]{Part II\\*Meaning of “Child”}

\renewcommand\parthead{--- Part II}

\subsection[2. Amendment of the Child Support (Maintenance Assessment Procedure) Regulations 1992]{Amendment of the Child Support (Maintenance Assessment Procedure) Regulations 1992}

2.---(1)  Schedule 1 (meaning of “child” for the purposes of the Act) to the Child Support (Maintenance Assessment Procedure) Regulations 1992\footnote{S.I.~1992/1813.} is amended as follows.

(2) For paragraph 1, substitute—
\begin{quotation}
“1.---(1)   A person satisfies such conditions as may be prescribed for the purposes of section 55(1)($b$) of the Act\footnote{Section 55 was substituted by section 42 of the 2008 Act.} if that person satisfies any of the conditions in sub-paragraphs (2) and (3).

(2) The person is receiving full-time education (which is not advanced education)—
\begin{enumerate}\item[]
($a$) by attendance at a recognised educational establishment; or

($b$) elsewhere, if the education is recognised by the Secretary of State.
\end{enumerate}

(3) The person is a person in respect of whom child benefit is payable.”.
\end{quotation}

(3) Omit paragraph 1A.

(4) In paragraph 2—
\begin{enumerate}\item[]
($a$) for “section 55 of the Act” substitute “this Schedule”; and

($b$) in sub-paragraph ($a$), after “education” insert “, a higher national certificate”.
\end{enumerate}

(5) In paragraph 3, for “section 55 of the Act” substitute “this Schedule”.

(6) In paragraph 4, for “section 55(1)($b$)  of the Act” substitute “paragraph~1(2)”.

(7) For paragraph 6, substitute—
\begin{quotation}
“6.  In this Schedule, “recognised educational establishment” means an establishment recognised by the Secretary of State for the purposes of this Schedule as being, or as comparable to, a university, college or school.”
\end{quotation}

(8) After paragraph 6 insert—
\begin{quotation}
\section*{\itshape “Education otherwise than at a recognised educational establishment}

7.  For the purposes of paragraph 1(2), the Secretary of State may recognise education provided for a person otherwise than at a recognised educational establishment only if satisfied that education was being so provided for that person immediately before that person attained the age of 16.”.
\end{quotation}

\subsection[3. Amendment of the Child Support (Maintenance Calculation Procedure) Regulations 2000]{Amendment of the Child Support (Maintenance Calculation Procedure) Regulations 2000}

3.---(1)  Schedule 1 (meaning of “child” for the purposes of the Act) to the Child Support (Maintenance Calculation Procedure) Regulations 2000\footnote{S.I.~2001/157.} is amended as follows.

(2) For paragraph 1, substitute—
\begin{quotation}
“1.---(1)   A person satisfies such conditions as may be prescribed for the purposes of section 55(1)($b$) of the Act\footnote{Section 55 was substituted by section 42 of the 2008 Act.} if that person satisfies any of the conditions in sub-paragraphs (2) and (3).

(2) The person is receiving full-time education (which is not advanced education)—
\begin{enumerate}\item[]
($a$) by attendance at a recognised educational establishment; or

($b$) elsewhere, if the education is recognised by the Secretary of State.
\end{enumerate}

(3) The person is a person in respect of whom child benefit is payable.”.
\end{quotation}

(3) Omit paragraph 1A.

(4) In paragraph 2—
\begin{enumerate}\item[]
($a$) for “section 55 of the Act” substitute “this Schedule”; and

($b$) in sub-paragraph ($a$), after “education” insert “, a higher national certificate”.
\end{enumerate}

(5) In paragraph 3, for “section 55 of the Act” substitute “this Schedule”.

(6) In paragraph 4, for “section 55(1)($b$)  of the Act” substitute “paragraph~1(2)”.

(7) For paragraph 6, substitute—
\begin{quotation}
“6.  In this Schedule, “recognised educational establishment” means an establishment recognised by the Secretary of State for the purposes of this Schedule as being, or as comparable to, a university, college or school.”
\end{quotation}

(8) After paragraph 6 insert—
\begin{quotation}
\section*{\itshape “Education otherwise than at a recognised educational establishment}

7.  For the purposes of paragraph 1(2), the Secretary of State may recognise education provided for a person otherwise than at a recognised educational establishment only if satisfied that education was being so provided for that person immediately before that person attained the age of 16.”.
\end{quotation}

\section[Part III --- New calculation rules---consequential and miscellaneous amendments]{Part III\\*New calculation rules---consequential and miscellaneous amendments}

\renewcommand\parthead{--- Part III}

\subsection[4. Amendment of the Child Support (Collection and Enforcement) Regulations 1992]{Amendment of the Child Support (Collection and Enforcement) Regulations 1992}

4.---(1)  The Collection and Enforcement Regulations\footnote{S.I.~1992/1989; relevant amending instruments are S.I.~1995/1045 and 2001/162.} are amended as follows.

(2) For regulation 4 (intervals of payment) and its heading, substitute—
\begin{quotation}
\subsection*{“Payments to be scheduled over reference period}

4.---(1)  The Secretary of State may, for the purposes of determining the frequency and amount of the payments of child support maintenance required to be made by a liable person—
\begin{enumerate}\item[]
($a$) determine the total amount payable for the reference period on the assumption that the weekly rate of child support maintenance will not change over that period; and

($b$) require that amount to be paid by equal instalments over that period at intervals determined by the Secretary of State.
\end{enumerate}

(2) The reference period in relation to the maintenance calculation is, subject to paragraph (3), the period of 52 weeks mentioned in section 29(3A) of the Act beginning with—
\begin{enumerate}\item[]
($a$) the initial effective date (where it is the first such period in relation to the maintenance calculation); or

($b$) the review date.
\end{enumerate}

(3) In this regulation “initial effective date” and “review date” have the meanings given by regulations 12 and 19 of the Child Support Maintenance Calculation Regulations 2012\footnote{S.I.~2012/2677.} respectively.”.
\end{quotation}

(3) In regulation 8(1) (interpretation of Part III), in the definition of “normal deduction rate” for “week, month or other period” substitute “month and the equivalent of that sum for a 1, 2 and 4 week period”.

(4) For regulation 10 (normal deduction rate), substitute—
\begin{quotation}
“10.---(1)  The period by reference to which the normal deduction rate is set must be the period by reference to which the liable person is normally paid where that period is a 1, 2 or 4 weekly or monthly period.

(2) The employer must select the normal deduction rate which applies depending on the period by reference to which the liable person’s earnings are normally paid.

(3) Where the liable person is paid by reference to a period other than at a 1, 2 or 4 weekly or monthly period, the Secretary of State must discharge the deduction from earnings order in accordance with regulation 20.”.
\end{quotation}

(5) For regulation 11 (protected earnings proportion and protected earnings rate) and its heading, substitute—
\begin{quotation}
\subsection*{“Protected earnings proportion}

11.---(1)  The period by reference to which the protected earnings proportion is set must be the same as the period by reference to which the normal deduction rate is set in accordance with regulation 10(1).

(2) The protected earnings proportion in respect of any period shall be 60\% of the liable person’s net earnings in respect of that period as calculated at the pay-day of the liable person by the employer.”.
\end{quotation}

(6) In regulation 20\footnote{Regulation 20(1) was substituted by S.I.~1995/1045 and amended by S.I.~2001/162.} (discharge of deduction from earnings orders)—
\begin{enumerate}\item[]
($a$) omit “or” at the end of paragraph (1)($e$);

($b$) at the end of paragraph (1)($f$)  insert—
\begin{quotation}
“; or

($g$) the circumstances in regulation 10(3) apply.”.
\end{quotation}
\end{enumerate}

(7) In regulations 25C(1)($a$)  (maximum deduction rate for regular deduction order) and 25G(2)($d$)  (review of a regular deduction order)\footnote{Regulations 25C and 25G was inserted by S.I.~2009/1815.} for “net” substitute “gross”.

\subsection[5. Amendment of the Child Support (Maintenance Arrangements and Jurisdiction) Regulations 1992]{\sloppy Amendment of the Child Support (Maintenance Arrangements and Jurisdiction) Regulations 1992}

5.---(1)  The Child Support (Maintenance Arrangements and Jurisdiction) Regulations 1992\footnote{S.I.~1992/2645; relevant amending instruments are S.I.~2001/161 and 2005/785.} are amended as follows.

(2) In regulation 1(2) (interpretation) omit the definitions of “Maintenance Calculation Procedure Regulations” and “Maintenance Calculations and Special Cases Regulations”.

(3) In regulation 5(3)($c$)  (notifications by the Secretary of State) for “regulation 8 of the Maintenance Calculations and Special Cases Regulations” substitute “regulation 50 of the Child Support Maintenance Calculation Regulations 2012\footnote{S.I.~2012/2677.}”.

(4) In regulation 8A($d$)\footnote{Regulation 8A was inserted by S.I.~2005/785.} (maintenance calculations and maintenance orders---payments) omit the words from “in accordance with” to the end of that paragraph.

\subsection[6. Amendment of the Social Security and Child Support (Decisions and Appeals) Regulations 1999]{Amendment of the Social Security and Child Support (Decisions and Appeals) Regulations 1999}

6.---(1)  The Social Security and Child Support (Decisions and Appeals) Regulations 1999\footnote{S.I.~1999/991; relevant amending instruments are S.I.~1999/2570, 2000/3185, 2001/158, 2008/2683, 2009/396 and 2011/1464.} are amended as follows.

(2) In regulation 1(3) (interpretation), omit the definitions of “the Arrears, Interest and Adjustment of Maintenance Assessments Regulations”, “the Maintenance Calculation Procedure Regulations”, “the Maintenance Calculations and Special Cases Regulations”, “relevant other child”, “relevant person” and “Variations Regulations”.

(3) Omit regulations 3A, 5A, 6A, 6B, 7B, 7C, 15A, 15B, 15C, 23 and 24.

(4) In regulation 4 (late application for a revision)—
\begin{enumerate}\item[]
($a$) in paragraph (1), omit “or 3A(1)($a$)”;

($b$) in paragraph (2), omit “the relevant person”;

($c$) in sub-paragraph ($c$)  of paragraph (4), omit “or 3A”; and

($d$) in paragraph (5), omit “and regulation 3A(1)($a$)”.
\end{enumerate}

(5) In the heading to regulation 30 (appeal against a decision which has been replaced or revised) omit “replaced or”.

(6) In regulation 30 (appeal against a decision which has been revised)—
\begin{enumerate}\item[]
($a$) for paragraph (1) substitute—
\begin{quotation}
“(1) An appeal against a decision of the Secretary of State or the Board or an officer of the Board shall not lapse where—
\begin{enumerate}\item[]
($a$) the decision is revised under section 9 before the appeal is determined; and

($b$) the decision as revised is not more advantageous to the appellant than the decision before it was revised.”;
\end{enumerate}
\end{quotation}

($b$) for paragraph (3) substitute—
\begin{quotation}
“(3) Where a decision as revised under section 9 is not more advantageous to the appellant than the decision before it was revised, the appeal shall be treated as though it had been brought against the decision as revised.”; and
\end{quotation}

($c$) in paragraphs (4) and (5), omit “replaced or”.
\end{enumerate}

(7) In regulation 33 (notice of appeal), omit paragraph (2)($d$).

(8) Omit Schedule 3D (effective dates for supersession of child support decisions).

\subsection[7. Amendment of the Child Support (Voluntary Payments) Regulations 2000]{\sloppy Amendment of the Child Support (Voluntary Payments) Regulations 2000}

7.---(1)  The Child Support (Voluntary Payments) Regulations 2000\footnote{S.I.~2000/3177.} are amended as follows.

(2) In regulation 1(2) (interpretation)—
\begin{enumerate}\item[]
($a$) omit the definition of “the Maintenance Calculations and Special Cases Regulations”;

($b$) in the definition of “the qualifying child’s home” omit the words from “and “home” has” to the end; and

($c$) in the definition of “relevant person”, in paragraph ($c$), for the words from “regulation 8” to the end substitute “regulation 50 of the Child Support Maintenance Calculation Regulations 2012\footnote{S.I.~2012/2677.}”.
\end{enumerate}

(3) In regulation 2(1)($c$)  (voluntary payment) omit the words from “and for this purpose” to “2000”.

\subsection[8. Amendment of the Child Support Information Regulations 2008]{Amendment of the Child Support Information Regulations 2008}

8.---(1)  The Child Support Information Regulations 2008\footnote{S.I.~2008/2551.} are amended as follows.

(2) In regulation 2 (interpretation)—
\begin{enumerate}\item[]
($a$) in paragraph (1), for the definition of “Maintenance Calculation Procedure Regulations” substitute—
\begin{quotation}
““the Maintenance Calculation Regulations” means the Child Support Maintenance Calculation Regulations 2012\footnote{S.I.~2012/2677.};”; and
\end{quotation}

($b$) omit paragraphs (2) and (3).
\end{enumerate}

(3) In regulation 7 (duty of persons from whom information requested) omit paragraph (3).

(4) After regulation 9 (duty to notify change of address) insert—
\begin{quotation}
\subsection*{“Duty to notify increase in current income}

9A.---(1)  In a case falling within paragraphs (2) or (3), the Secretary of State may notify the non-resident parent that that parent is required to notify the Secretary of State of any relevant change of circumstances in relation to that income.

(2) A case falls within this paragraph if, in relation to a maintenance calculation in force—
\begin{enumerate}\item[]
($a$) gross weekly income is determined by reference to the non-resident parent’s current income as an employee or officeholder (in accordance with regulation 38 of the Maintenance Calculation Regulations); and

($b$) paragraph 5($b$)  of Schedule 1 to the 1991 Act (nil rate) does not apply.
\end{enumerate}

(3) A case falls within this paragraph if, in relation to a maintenance calculation in force—
\begin{enumerate}\item[]
($a$) gross weekly income is determined by reference to the non-resident parent’s current income (in accordance with regulation 37 of the Maintenance Calculation Regulations); and

($b$) paragraph 5($b$)  of Schedule 1 to the 1991 Act applies (nil rate).
\end{enumerate}

(4) A notification by the Secretary of State under paragraph~(1) must be in writing.

(5) Where a relevant change of circumstances occurs after the non-resident parent has been notified of a requirement under paragraph (1), the non-resident parent must notify the Secretary of State of that change—
\begin{enumerate}\item[]
($a$) within fourteen days beginning with the day on which the change occurs; or

($b$) within such other period as the Secretary of State has specified in the notification.
\end{enumerate}

(6) For the purposes of a case falling within paragraph (2), a relevant change of circumstances occurs where—
\begin{enumerate}\item[]
($a$) the non-resident parent—
\begin{enumerate}\item[]
(i) commences a new employment or office; or

(ii) in relation to an existing employment or office, commences a new rate of remuneration or a new working pattern,
\end{enumerate}
and could reasonably be expected to know that would result in an increased liability under the maintenance calculation in force if reported to the Secretary of State; or

($b$) the non-resident parent receives from their employment or office the following number of consecutive payments, each of which (if it were taken as a weekly average) exceeds the gross weekly income taken into account in the maintenance calculation in force by 25\% or more—
\begin{enumerate}\item[]
(i) five payments, in the case of a non-resident parent paid weekly;

(ii) three payments, in the case of a non-resident parent paid fortnightly;

(iii) two payments, in the case of a non-resident parent paid four weekly or monthly.
\end{enumerate}
\end{enumerate}

(7) The payments referred to in paragraph (6)($b$)  are the gross remuneration from the employment or office in question less any pension contributions deducted under net pay arrangements.

(8) In paragraph (7) “net pay arrangements” means arrangements for relief in respect of pension contributions under section~193 of the Finance Act 2004\footnote{2004 c. 12.}.

(9) For the purposes of a case falling within paragraph (3), a relevant change of circumstances occurs where the non-resident parent’s income increases to a gross weekly income of £5 or more.

(10) For the purposes of paragraph (9), gross weekly income is to be calculated in accordance with regulation 45(2) of the Maintenance Calculation Regulations.”.
\end{quotation}

(5) In regulation 13 (disclosure of information to other persons)—
\begin{enumerate}\item[]
($a$) in paragraph (1)($d$), for “regulation 23” to “Procedure Regulations” substitute “regulation 25 of the Maintenance Calculation Regulations (notification of a maintenance calculation)”; and

($b$) in paragraph (2)($c$), for “regulation 34” to “Regulations 1999” substitute “paragraph 4 of the Schedule to the Maintenance Calculation Regulations”.
\end{enumerate}

\amendment{
Reg.~9 revoked (30.9.13) by the Child Support (Miscellaneous Amendments) Regulations 2013 reg.~10.
}

% Reg 9 revoked (30.9.13) by SI 2013/1517 reg 10
%\subsection[9. Amendment of the Tribunal Procedure (First-tier Tribunal) (Social Entitlement Chamber) Rules 2008]{Amendment of the Tribunal Procedure (First-tier Tribunal) (Social Entitlement Chamber) Rules 2008}
%
%9.---(1)  Schedule 1 to the Tribunal Procedure (First-tier Tribunal) (Social Entitlement Chamber) Rules 2008\footnote{S.I.~2008/2685 (L.~13); relevant amending instrument is S.I.~2010/2653.} is amended as follows.
%
%(2) In the second column of the first entry (cases other than those listed below)—
%\begin{enumerate}\item[]
%($a$) for paragraph ($c$)(i), substitute—
%\begin{quotation}
%“(i) regulation 14 of the Child Support Maintenance Calculation Regulations 2012\footnote{S.I.~2012/2677.}”; and
%\end{quotation}
%
%($b$) in paragraph ($c$)(ii)  omit “or 3A(1)”.
%\end{enumerate}

\subsection[10. Revocations]{Revocations}

10.  The following Regulations are revoked—
\begin{enumerate}\item[]
($a$) the Child Support (Maintenance Assessment Procedure) Regulations 1992\footnote{S.I.~1992/1813.};

($b$) the Child Support (Maintenance Assessments and Special Cases) Regulations 1992\footnote{S.I.~1992/1815.};

($c$) the Child Support (Maintenance Assessments and Special Cases) Amendment Regulations 1993\footnote{S.I.~1993/925.};

($d$) the Child Support Departure Direction and Consequential Amendments Regulations 1996\footnote{S.I.~1996/2907.};

($e$) the Child Support (Maintenance Calculations and Special Cases) Regulations 2000\footnote{S.I.\ 2001/155.};

($f$) the Child Support (Variations) Regulations 2000\footnote{S.I.~2001/156.}; and

($g$) the Child Support (Maintenance Calculation Procedure) Regulations 2000\footnote{S.I.~2001/157}.
\end{enumerate}

\section[Part IV --- Savings and transitional provision]{Part IV\\*Savings and transitional provision}

\renewcommand\parthead{--- Part IV}

\subsection[11. Saving where arrears-only case]{Saving where arrears-only case}

11.---(1)  Regulations 8, 10, 11 and 20 of the Collection and Enforcement Regulations continue to apply in relation to an arrears-only case, as they were in force immediately before the amendments made by regulation 4(3) to (6) come into force, until notice is given to the non-resident parent by the Secretary of State that the provisions of the Regulations as amended by regulation 4(3) to (6) apply to that case.

(2) Any notice given under paragraph (1) must be in writing and sent by post to the non-resident parent’s last known or notified address and will be treated as having been given on the second day following the day on which it is posted.

(3) For the purposes of this regulation any reference to a non-resident parent includes reference to an absent parent.

\subsection[12. Transitional provision]{Transitional provision}

12.---(1)  Where, in any case, a deduction from earnings order was made before the date on which the Collection and Enforcement Regulations as amended by regulation 4(3) to (6) apply in relation to that case, this regulation shall apply in respect of that order.

(2) Where the deduction from earnings order still has effect immediately before regulation 4(3) to (6) comes into force in relation to that case—
\begin{enumerate}\item[]
($a$) the order continues to take effect for the purposes of any deductions which are required to be made under the order until it is discharged or lapses;

($b$) the Collection and Enforcement Regulations, as they were in force before the amendments made by regulation 4(3) to (6) came into force, continue to apply in relation to the order until it is discharged or lapses; and

($c$) the order is to be treated as discharged, if it has not otherwise lapsed or been discharged, on the date that the first deduction from earnings order made under the Collection and Enforcement Regulations as amended by regulation 4(3) to (6) takes effect. 
\end{enumerate}

\bigskip

\pagebreak[3]

Signed 
by authority of the 
Secretary of State for~Work and~Pensions.
%I concur
%By authority of the Lord Chancellor

{\raggedleft
\emph{Steve Webb}\\*
%Secretary
Minister
%Parliamentary Under-Secretary 
of State\\*Department 
for~Work and~Pensions

}

2nd November 2012

\small

\part{Explanatory Note}

\renewcommand\parthead{— Explanatory Note}

\subsection*{(This note is not part of the Regulations)}

These Regulations contain provisions consequential on, or connected with, the bringing into force of the changes to the meaning of “child” for the purposes of the Child Support Act 1991, provided for in section 42 of the Child Maintenance and Other Payments Act 2008 (c.~6) (“the 2008 Act”), and the changes to the rules for the calculation of child support maintenance, provided for in Schedule 4 to the 2008 Act and the Child Support Maintenance Calculation Regulations 2012 (S.I.~2012/2677).\looseness=-1

Part~II of the Regulations, dealing with the meaning of “child”, comes into force on the day on which section 42 of the 2008 Act (meaning of “child”) is commenced. Part~III of the Regulations comes into force in relation to a particular case on the day on which paragraph 2 of Schedule 4 to the 2008 Act (calculation by reference to gross weekly income) comes into force in relation to that type of case. Commencement Orders will set out when paragraph 2 of Schedule 4 will come into force for the purposes of particular types of cases. In relation to an arrears-only case, regulation 4(3) to (6), which amends provisions relating to deduction from earnings orders, comes into force on 10th December 2012, subject to the saving in regulation 11. Regulation 1(5) defines an “arrears-only case”.

Regulation 2 amends the Child Support (Maintenance Assessment Procedure) Regulations 1992 (S.I.~1992/1813), prescribing conditions for the purposes of section 55(1)($b$)  of the 1991 Act (meaning of “child”). Regulation 3 does the same in relation to the Child Support (Maintenance Calculation Procedure) Regulations 2000 (S.I.~2000/157).\looseness=-1

Regulation 4 amends the Child Support (Collection and Enforcement) Regulations 1992 (S.I.~1992/1989) (“the Collection and Enforcement Regulations”). Paragraph (2) substitutes a provision allowing payments of child support maintenance to be scheduled as equal instalments payable over an annual period. Regulation 4(3) amends regulation 8 of the Collection and Enforcement Regulations changing the definition of “normal deduction rate” to set out the amount to be deducted per month and the amount for a 1, 2 and 4 week equivalent of that amount. Paragraph (4) substitutes regulation 10 of the Collection and Enforcement Regulations to provide that the normal deduction rate is set by reference to the period by reference to which the liable person is paid where that payment period is monthly or 1, 2 or 4 weekly. The employer must select the normal deduction rate which applies depending on the period by reference to which the liable person is paid. Where the liable person is not paid by reference to a monthly, 1, 2 or 4 weekly period the deduction from earnings order must be discharged in accordance with regulation 20. Regulation 4(6) amends regulation 20 to allow for deduction from earnings orders to be discharged in these circumstances. Regulation 4(5) substitutes regulation 11 of the Collection and Enforcement Regulations to provide that the protected earnings proportion must be 60\% of net earnings at the liable person’s pay day for each deduction made under the order. Regulation 4(7) makes an amendment to the maximum deduction rate for a regular deduction order, so it is calculated by reference to gross, rather than net, income.\looseness=-1

Regulation 6 amends the Social Security and Child Support (Decisions and Appeals) Regulations 1999 (S.I.~1999/991) by removing provisions relating to child support maintenance.

Regulation 8 inserts new provision in the Child Support Information Regulations 2008 (S.I.~2008/2551) allowing the Secretary of State to require a non-resident parent whose income has been calculated by reference to current employment as an employee or officeholder, or whose income has been calculated by reference to current income to whom the nil rate applies, to report an increase in that income. Failure to comply with the new provisions will be an offence under section 14A of the Child Support Act 1991. Regulation 8 also makes consequential amendments.

Regulations 5, 7 and 9 make consequential amendments to, respectively, the Child Support (Maintenance Arrangements and Jurisdiction) Regulations 1992 (S.I.~1992/2645), Child Support (Voluntary Payments) Regulations 2000 (S.I.~2000/3177) and the procedure rules for the First-Tier Tribunal (Social Entitlement Chamber) (S.I.~2008/2685).

Regulation 10 revokes a number of sets of regulations which are replaced by the Child Support Maintenance Calculation Regulations 2012.

Regulation 11 saves regulations 8, 10, 11 and 20 of the Collection and Enforcement Regulations for the purposes of arrears-only cases as they have effect immediately before regulation 4(3) to (6) comes into force until notice is given to the non-resident parent in such cases that the regulations as amended apply in their case.

Regulation 12 makes transitional provision. Where a deduction from earnings order made under the Collection and Enforcement Regulations, prior to the amendments in regulation 4(3) to (6) coming into force, has effect immediately before those provisions come into force in the case, the existing deduction from earnings order will continue to take effect until it lapses or is discharged. The existing order will be discharged, if it is still in effect, on the date on which the first order made under the Collection and Enforcement Regulations as amended by regulation 4(3) to (6) takes effect.

These Regulations reduce costs on the private sector and civil society organisations. An assessment of the impact has been made; a copy of the impact assessment is available in the libraries of both Houses of Parliament and is annexed to the Explanatory Memorandum, which is available alongside the instrument on \url{www.legislation.gov.uk}. Copies of the impact assessment may also be obtained from the Better Regulation Unit of the Department for Work and Pensions, 2D Caxton House, Tothill Street, London \textsc{\lowercase{SW1H 9NA}}, or from the DWP website: \url{http://www.dwp.gov.uk/resourcecentre/ria.asp}.

A full impact assessment of the effect that the Child Support Maintenance Calculation Regulations 2012 will have on the costs of business and the voluntary sector is also available from the same address and is annexed to the Explanatory Memorandum to those Regulations, which is available alongside that instrument on \url{www.legislation.gov.uk}. 

\end{document}