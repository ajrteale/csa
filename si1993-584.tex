\documentclass[12pt,a4paper]{article}

\newcommand\regstitle{The Child Support (Northern Ireland Reciprocal Arrangements) Regulations 1993}

\newcommand\regsnumber{1993/584}

%\opt{newrules}{
\title{\regstitle}
%}

%\opt{2012rules}{
%\title{Child Maintenance and Other Payments Act 2008\\(2012 scheme version)}
%}

\author{S.I. 1993 No. 584}

\date{Made 10th March 1993\\Laid before Parliament 15th March 1993\\Coming into force 5th April 1993}

%\opt{oldrules}{\newcommand\versionyear{1993}}
%\opt{newrules}{\newcommand\versionyear{2003}}
%\opt{2012rules}{\newcommand\versionyear{2012}}

\usepackage{csa-regs}

\setlength\headheight{27.61603pt}

\begin{document}

\maketitle

\noindent
The Secretary of State for Social Security, in excercise of the powers conferred upon him by Section 56(3) and (4) of the Child Support Act 1991\footnote{\frenchspacing 1991 c. 48. Section 56(2) provides for the Secretary of State to make arrangements with the Department of Health and Social Services for Northern Ireland to secure provision for a single child support system within the United Kingdom.} and of all other powers enabling him in that behalf, hereby makes the following Regulations:

{\sloppy

\tableofcontents

}

\setcounter{secnumdepth}{-2}

\subsection[1. Citation and commencement]{Citation and commencement}

1.  These Regulations may be cited as the Child Support (Northern Ireland Reciprocal Arrangements) Regulations 1993 and shall come into force on 5th April 1993.

\subsection[2. Adaptation of the Child Support Act 1991 and regulations in respect of child support]{Adaptation of the Child Support Act 1991 and regulations in respect of child support}

2.—(1) The provisions contained in the Memorandum of Arrangements set out in Schedule 1 
as amended by the 
%Exchange of Letters set out in Schedule 1A  % Words inserted by SI 1995/3261 reg 53
Exchanges of Letters set out in Schedules~%
%1A and 1B  % Words substituted by SI 2002/771 reg 2(2)
%1A,~1B and 1C  % Words substituted by SI 2012/2380 reg 2(2)
1A,~1B, 1C and 1D  % Words substituted by SI 2014/1423 reg 2(2)
to these Regulations shall have effect so far as the same relate to Great Britain.

(2) In particular and without prejudice to paragraph (1) above any act, omission and event which has effect for the purposes of the provision of the Northern Ireland legislation specified in column 2 of Schedule 2 
or column 2 of Schedule 3  % Words inserted (16.4.02) by SI 2002/771 reg 2(3)(a)
to these Regulations shall also have effect as an act, omission and event for the purposes of the provision of the Child Support Act 1991 specified in the corresponding paragraph of column~1 of Schedule 2 
or column~1 of Schedule 3 and for the purposes of the provision of the Family Law Act 1986\footnote{1986 c.\ 55.} specified in the corresponding paragraph of column 1 of Schedule 3  % Words inserted (16.4.02) by SI 2002/771 reg 2(3)(b)
to the said Regulations; and in the provisions specified in column 1 of Schedule 2 to these Regulations the references to—
\begin{enumerate}\item[]
($a$) “the Act” shall be construed as including references to the Child Support (Northern Ireland) Order 1991\footnote{\frenchspacing S.I. 1991/2628 (N.I. 23).};

($b$) “the Secretary of State” shall be construed as including references to the Department 
%of Health and Social Services for Northern Ireland
for Social Development%  % Words substituted (16.4.02) by SI 2002/771 reg 2(3)(c)
;

% Reg 2(2)(c) omitted (16.4.02) by SI 2002/771 reg 2(3)(d)
%($c$) any “child support officer” shall be construed as including references to such an officer appointed by the Department of Health and Social Services for Northern Ireland;

($d$) “child support maintenance” shall be construed as including references to child support maintenance within the meaning of the Child Support (Northern Ireland) Order 1991;
\end{enumerate}
and cognate expressions shall be construed accordingly.

\amendment{
Words inserted in reg. 2(1) (22.1.96) by the Child Support (Miscellaneous Amendments) (No. 2) Regulations 1995 reg. 53.

Words inserted in reg. 2(2), words substituted in reg. 2(1), (2) and reg. 2(2)(c) omitted (16.4.02) by the Child Support (Northern Ireland Reciprocal Arrangements) Amendment Regulations 2002 reg. 2.

Words inserted in reg.~2(1) (29.10.12) by the Child Support (Northern Ireland Reciprocal Arrangements) Amendment Regulations 2012 reg.~2(2).

Words substituted in reg.~2(1) (30.6.14) by the Child Support (Northern Ireland Reciprocal Arrangements) Amendment Regulations 2014 reg.~2(2).
}

\bigskip

Signed by authority of the Secretary of State for Social Security.

{\raggedleft
\emph{Alistair Burt}\\*Parliamentary Under-Secretary of State,\\*Department of Social Security

}

10th March 1993

\vfill

\small

\part[Schedule 1 --- Memorandum of Arrangements relating to the provision made for Child Support Maintenance in the United Kingdom between the Secretary of State for Social Security of the one part and the Department of Health and Social Services for Northern Ireland of the other part]{Schedule 1\\*Memorandum of Arrangements relating to the provision made for Child Support Maintenance in the United Kingdom between the Secretary of State for Social Security of the one part and the Department of Health and Social Services for Northern Ireland of the other part}

\section[Part I --- Interpretation and General Provisions]{Part I\\*Interpretation and General Provisions}

\renewcommand\parthead{--- Schedule 1 Part I}

1.  In this Memorandum, unless the context otherwise requires:
\begin{enumerate}\item[]
“the Act” means the Child Support Act 1991 and “the Order” means the Child Support (Northern Ireland) Order 1991;

% Definition of ``the 2000 Act'' inserted (16.4.02) by Sch 1B
“the 2000 Act” means the Child Support, Pensions and Social Security Act 2000\footnote{2000 c.\ 19.};

“application”, for the purposes of Article 5, includes an application by an absent parent and an application under section 7 of the Act;

% Definition of ``the Department'' inserted (16.4.02) by Sch 1B
“the Department” means the Department for Social Development;

%“determining authority” means, in relation to Great Britain, a child support officer, a child support appeal tribunal, a Child Support Commissioner, or a tribunal consisting of any three of the Child Support Commissioners, and appointed or constituted under the Act, and, in relation to Northern Ireland, a child support officer or a child support appeal tribunal appointed or constituted under the Order, a Child Support Commissioner for Northern Ireland appointed under the Act or a tribunal consisting of any two or three of the Child Support Commissioners for Northern Ireland constituted under the Order;
% Definition of ``determining authority'' substituted (16.4.02) by Sch 1B
“determining authority” means, in relation to Great Britain, the Secretary of State, an appeal tribunal or a Commissioner, and, in relation to Northern Ireland, the Department, an appeal tribunal or a Commissioner;

“parent with care” means a person who, in respect of the same child or children, is both a parent and a person with care;

\looseness=1
“territory” means Great Britain or Northern Ireland, as the case may be.
\end{enumerate}

\amendment{
Definitions of ``the 2000 Act'', ``the Department'' inserted in art. 1 and definition of ``determining authority'' in art. 1 substituted (16.4.02) by Sch. 1B to these Regulations.
}

\medskip

1A.---(1)  In these arrangements—
\begin{enumerate}\item[]
($a$) references to a maintenance assessment shall, in relevant cases, include references to a maintenance calculation;

($b$) references to an absent parent shall, in relevant cases, include references to a non-resident parent; and

($c$) references to cases where an application for a maintenance assessment is made shall, in relevant cases, include references to cases where an application for a maintenance calculation is treated as having been made.
\end{enumerate}

(2) In this Article, “relevant cases” means cases for the purposes of which section 1 of the 2000 Act has come into force or cases for the purposes of which section 1 of the Child Support, Pensions and Social Security Act (Northern Ireland) 2000\footnote{2000 c.\ 4 (N.I.).} has come into operation.

\amendment{
Art. 1A inserted (16.4.02) by Sch. 1B to these Regulations.
}

\medskip

2.—(1) Unless the context otherwise requires, in the application of this Memorandum to a territory, expressions used in this Memorandum shall have the same respective meanings as in the Act, in relation to Great Britain, or in the Order, in relation to Northern Ireland.

(2) The rules for the construction of Acts of Parliament contained in the Interpretation Act 1978 shall apply for the purposes of the interpretation of this Memorandum as they apply for the purposes of the interpretation of an Act of Parliament or statutory instrument.

\medskip

3.—(1) Subject to Articles 5 to 
%12 
12A  % Word substituted (16.4.02) by Sch 1B
of this Memorandum, the provision made for Great Britain and the provision made for Northern Ireland shall operate as a single system within the United Kingdom.

(2) For the purposes of paragraph (1), all acts, omissions and events and in particular any application, declaration, direction, decision% 
, determination  % Word inserted (16.4.02) by Sch 1B
or order having effect for the provision made for Great Britain and having effect in that territory or for the provision made for Northern Ireland and having effect in that territory, shall have a corresponding effect for the purpose of the provision made for child support maintenance made in the other territory.

\amendment{
Word inserted in art. 3(2) and word substituted in art. 3(1) (16.4.02) by Sch. 1B to these Regulations.
}

\medskip

4.  Nothing in this Memorandum shall require the payment of a fee under the provision made for one territory if such a fee is paid or liable to be paid in the same circumstances under the provision made for the other territory.

\section[Part II --- Case Ownership]{Part II\\*Case Ownership}

\renewcommand\parthead{--- Schedule 1 Part II}

5.—(1) Subject to paragraphs (2)% 
%and (4)
,~(4) and (8)%  % Words substituted (29.10.12) by Sch 1C
, where two or more applications for a maintenance assessment are made in relation to the same absent parent or a person treated as such, under the provision made for one territory and under the provision made for the other territory, all the said applications shall be dealt with in, and in accordance with the provision made for, the territory in which the person with care resides in respect of whom the first application was received.

(2) Subject to 
%paragraph (4)
paragraphs~(4) and (8)%  % Words substituted (29.10.12) by Sch 1C
, where the applications specified in paragraph~(1) include an application under section 7 of the Act by a qualifying child (right of child in Scotland to apply for assessment), all the applications shall be dealt with in, and in accordance with the provision made for, the territory in which the person with care of the said qualifying child resides.

(3) Subject to 
%paragraph (4)
paragraphs~(4) and (8)%  % Words substituted (29.10.12) by Sch 1C
, where a person with care whose application is dealt with under the provisions of paragraph (1) makes an application in respect of another absent parent, that further application shall be dealt with in, and in accordance with the provision made for, the territory specified in that paragraph.

(4) Where paragraphs (1), (2) or (3) apply, the determining authority shall, in determining the amount of child support maintenance to be fixed by any maintenance assessment, take into account in calculating that amount, any provisions which would otherwise have been applicable to that calculation had the assessment been made in accordance with the provision made for the other territory.

%Art 5(5)--(7) inserted by Sch 1A
(5) Subject to 
%paragraph (7)
paragraphs~(7) and (8)%  % Words substituted (29.10.12) by Sch 1C
, where an application for a maintenance assessment is made under the provisions for one territory in relation to an absent parent, a person treated as such, or an alleged absent parent who resides in the other territory, that application shall be dealt with in, and in accordance with the provision made for, the territory in which the person with care resides.

(6) Subject to 
%paragraph (7)
paragraphs~(7) and (8)%  % Words substituted (29.10.12) by Sch 1C
, where an application for a maintenance assessment is made under section 7 of the Act by a qualifying child, the application shall be dealt with in, and in accordance with the provision made for, the territory in which the person with care of that child resides.

(7) Where paragraphs (5) and (6) apply, the determining authority shall, in determining the amount of child support maintenance to be fixed by any maintenance assessment, take into account in calculating that amount, any provisions which would otherwise have been applicable to that calculation had the assessment been made in accordance with the provision made for the other territory.

% Art 5(8)--(12) inserted by Sch 1C
%(8) An application for a maintenance calculation which is to be determined in accordance with the new calculation rules shall be dealt with in, and in accordance with the provision made for, the territory in which the person who is, or is treated or alleged to be, the non-resident parent in relation to that application resides.

% Art 5(8) substituted by Sch 1D
(8) An application for a maintenance calculation which is to be determined in
accordance with the new calculation rules shall be dealt with in, and in accordance
with the provision made for, the territory in which the person who makes the
application resides until---
\begin{enumerate}\item[]
($a$) where the applicant resides in Great Britain---
\begin{enumerate}\item[]
(i) the application is taken to have been made for the purposes of
regulation 9(2) (applications under section~4 or 7 of the Act) of the
Child Support Maintenance Calculation Regulations 2012\footnote{S.I.~2012/2677.},

(ii) any application fee payable under regulation 3(1) of the Child Support Fees Regulations 2014\footnote{S.I.~2014/612.} has been paid or waived in accordance with those Regulations,

(iii) the Secretary of State has ascertained and verified the address of the
non-resident parent in relation to the application, and

(iv) where the application is one to which paragraph~(8A) or (8B) applies,
the condition in that paragraph is satisfied; or
\end{enumerate}

($b$) where the applicant resides in Northern Ireland---
\begin{enumerate}\item[]
(i) the application is taken to have been made for the purposes of
regulation 9(2) (applications under Article 7) of the Child Support
Maintenance Calculation Regulations (Northern Ireland) 2012\footnote{S.R.~2012 No.~427.},

(ii) the Department has ascertained and verified the address of the non-%
resident parent in relation to the application, and

(iii) where the application is one to which paragraph~(8A) or (8B) applies,
the condition in that paragraph is satisfied,
\end{enumerate}
\end{enumerate}
from which point the case shall be dealt with in, and in accordance with the
provision made for, the territory in which the non-resident parent in relation to the 
application resides.

% Art 5(8A)--(8D) inserted by Sch 1D
(8A) Where there is an existing case related to the application, in relation to
which the interested parties have been required to choose whether or not to stay in 
the statutory scheme (under Schedule 5 to the 2008 Act\footnote{Schedule 5 was amended by section 136 of the Welfare Reform Act 2012.} or Schedule 2 to the Child
Maintenance Act (Northern Ireland) 2008\footnote{2008 c.~10 (N.I.).}) as a result of that application, the
condition is that any liability end date in relation to that existing case must have
passed.

(8B) Where the applicant has chosen to remain in the statutory scheme, in response
to being requied to choose in an existing case whether or not to stay in the statutory 
scheme (under Schedule~5 to the 2008 Act or Schedule 2 to the Child Maintenance 
Act (Northern Ireland) 2008), the condition is that the liability end date in relation to
that existing case must have passed.

(8C) For the purposes of paragraph (8A), an existing case is related to an
application if---
\begin{enumerate}\item[]
($a$) the non-resident parent in relation to that application is also the non-resident
parent in relation to the existing case and the person with care in relation to
that application is not the person with care in relation to the existing case, or

($b$) the non-resident parent in relation to that application is a partner of a non-%
resident parent in relation to the existing case and either or both are in 
receipt of a benefit prescribed by regulations made under paragraph~4(1)($c$)
(flat rate) of Schedule 1 to the Act\footnote{The substitution of Part~I of Schedule~1 to the Act by section~1(3) of, and Schedule~1 to, the Child Support, Pensions and Social Security Act 2000 (c.~19) was partially commenced for the types of cases specified in article 3 of the Child Support, Pensions and Social Security Act 2000 (Commencement No.~12) Order 2003 (S.I.~2003/192).} or paragraph 4(1)($c$) (flat rate) of
Schedule~1 to the Order\footnote{The substitution of Part~I of Schedule~1 to the Order by section~1(3) of, and Schedule~1 to, the Child Support, Pensions and Social Security Act (Northern Ireland) 2000 (c.~4 (N.I.)) was partially commenced for the types of cases specified in Article 3 of the Child Support, Pensions and Social Security (2000 Act) (Commencement No.~9) Order (Northern Ireland) 2003 (S.R.~2003 No.~53).}.
\end{enumerate}

(8D) For the purposes of paragraphs (8) and (8C), a non-resident parent includes a 
person who is treated as or alleged to be a non-resident parent.

% Art 5(9) omitted by Sch 1D
%(9) Where paragraph (8) applies to an application for a maintenance calculation and there is an existing case in respect of which the same person is, or is treated as or alleged to be, the non-resident parent, that case shall also be dealt with (insofar as it is not already) in, and in accordance with the provision made for, the territory in which that person resides.

(10) For the purpose of 
%paragraphs 
paragraph  % Word substituted by Sch 1D
(8)%
% and (9)  % Words omitted by Sch 1D
, where the person who is, or is treated as or alleged to be, the non-resident parent in relation to the application falls within section 44(2A) of the Act, or Article 41(2A) of the Order, that person shall be treated as if residing in Great Britain.

(11) In this Article---
\begin{enumerate}\item[]
% Definition of ``the 2008 Act'' inserted by Sch 1D
``the 2008 Act'' means the Child Maintenance and Other Payments Act 2008\footnote{2008 c.~6.};

``existing case'' means any case where the maintenance assessment or maintenance calculation is made, or will fall to be made, otherwise than in accordance with the new calculation rules;

% Definition of ``interested parties'' inserted by Sch 1D
``interested parties'' means the non-resident parent, the person with care and, in
the case of an application made by a qualifying child under section 7(1) of the
Act, or a maintenance calculation or assessment made in response to an
application under that section, the child in question;

% Definition of ``liability end date'' inserted by Sch 1D
``liability end date'' means the date determined in accordance with---
\begin{enumerate}\item[]
($a$) regulations made under Schedule 5 (maintenance calculations: transfer
of cases to new rules) to the 2008 Act as the date beyond which no 
further liability accrues in relation to the existing case for the purposes
of paragraph~5(1) and (2) of that Schedule, or

($b$) regulations made under Schedule 2 (maintenance calculations: transfer
of cases to new rules) to the Child Maintenance Act (Northern Ireland)
2008 as the date beyond which no further liability accrues in relation to
the existing case for the purposes of paragraph 5(1) and~(2) of that
Schedule;
\end{enumerate}

``new calculation rules'' means Part~I of Schedule~1 to the Act as amended by Schedule~4 to the Child Maintenance and Other Payments Act 2008, or Part~I of Schedule~1 to the Order as amended by Schedule~1 to the Child Maintenance Act (Northern Ireland) 2008;

% Definition of ``partner'' inserted by Sch 1D
``partner'' means a person falling within the definition of ``partner'' given in paragraph 10C(4) of Schedule 1 (maintenance calculations---reference to various terms) to the Act or paragraph~10C(4) of Schedule~1 (maintenance calculations---reference to various terms) to the Order.
\end{enumerate}

(12) In paragraphs 
%(9) and (10)
(8C), (8D), (10) and~(11)%  % Words substituted by Sch 1D
, where relevant, references to non-resident parent include references to absent parent.

\amendment{
Art. 5(5)--(7) inserted (22.1.96) by Sch. 1A to these Regulations.

Words substituted in Art.~5(1)--(3), (5), (6) and Art.~5(8)--(12) inserted (29.10.12) by Sch.~1C to these Regulations.

Art. 5(8) substituted, Art. 5(8A)--(8D) inserted, Art. 5(9) omitted, words substituted and omitted in Art. 5(10), definitions of ``the 2008 Act'', ``interested parties'', ``liability end date'' and ``partner'' inserted in Art. 5(11) and words substituted in Art. 5(12) (30.6.14) by Sch. 1D to these Regulations.
}

\section[Part III --- Multiple Applications]{Part III\\*Multiple Applications}

\renewcommand\parthead{--- Schedule 1 Part III}

6.  Where—
\begin{enumerate}\item[]
($a$) no maintenance assessment is in force and an application for such an assessment is made in one territory and another such application is made in the other territory in respect of the same qualifying child or children and the same person with care and absent parent or parents or person treated as such; and

($b$) but for the fact that the person with care, and the absent parent or parents or person treated as such reside in different territories the provisions regarding multiple applications made under the provision for Great Britain or the provision made for Northern Ireland would apply,
\end{enumerate}
those provisions shall have effect to determine which application shall be proceeded with.

\section[Part IV --- Disclosure of Information and Inspectors]{Part IV\\*Disclosure of Information and Inspectors}

\renewcommand\parthead{--- Schedule 1 Part IV}

7.—(1) Subject to paragraph (2) where the Secretary of State
%, the Department, or a child support officer appointed under the provision made for Great Britain or for Northern Ireland, 
or the Department  % Words substituted (16.4.02) by SI 2002/771
has in his or its possession any information or evidence held for the purposes of the provision made for his or its territory, that information or evidence may be disclosed to the Secretary of State
%, the Department or the child support officer for the other territory 
or the Department  % Words substituted (16.4.02) by SI 2002/771
for the purposes of the provision made for Great Britain or for Northern Ireland, as the case may be.

(2) Where information is disclosed under the provisions of paragraph (1), the provision made for Northern Ireland or, as the case may be, Great Britain, relating to unauthorised disclosure of information shall apply to that information.

\amendment{
Words substituted in art. 7(1) (16.4.02) by Sch. 1B to these Regulations.
}

\medskip

8.  Where in relation to a particular case, for the purposes of the provision made for one territory (the first provision) it is necessary for an inspector to be appointed, an inspector may be appointed under the provision for the other territory to exercise his powers of inspection for the purposes of the first provision.

\section[Part V --- Appeals]{Part V\\*Appeals}

\renewcommand\parthead{--- Schedule 1 Part V}

9.  Subject to Article 12, any appeal from any decision of a determining authority made under the provision for one territory shall be heard and determined—
\begin{enumerate}\item[]
($a$) in a case which is being dealt with in accordance with the provisions of Article 5 above, or

($b$) in a case where the relevant persons to the appeal are resident in different territories,
\end{enumerate}
in, and in accordance with the provision made for, the territory in which case is being dealt with.

\section[Part VI --- Collection and Enforcement]{Part VI\\*Collection and Enforcement}

\renewcommand\parthead{--- Schedule 1 Part VI}

10.   Where a deduction from earnings order is made under the provision made for one territory and the liable person works for an employer in the other territory, the deduction from earnings order shall have effect in the territory in which the liable person works as if it was made under provision for that territory.

\medskip

11.   Where an application for a liability order is to be made against a liable person under the provision made for one territory and the liable person is resident in the other territory, the application shall be made under the provision for the territory in which the liable person is resident, notwithstanding the fact that the liability arose or the maintenance assessment was made under the provision for the other territory.

\medskip

12.   Where a deduction from earnings order has been applied or a liability order has been obtained in accordance with Articles 10 or 11, any appeal in connection with that deduction from earnings order or liability order or action as a consequence of the deduction from earnings order or liability order shall be made under the provision for the territory in which the liable person is resident.

\section[Part VIA --- Parentage]{Part VIA\\*Parentage}

\renewcommand\parthead{--- Schedule 1 Part VIA}

12A.  Where a person with care resides in one territory and an alleged parent who denies that he is one of the parents of a child in respect of whom an application for a maintenance assessment has been made resides in the other territory:—
\begin{enumerate}\item[]
($a$) The person with care or the Secretary of State may apply for a declaration as to whether or not the alleged parent is one of the child’s parents, under
%Article 28 of the Order
Article 31B of the Matrimonial and Family Proceedings (Northern Ireland) Order 1989\footnote{S.I.\ 1989/677 (N.I.\ 4).}%  % Words substituted (16.4.02) by Sch 1B
;

($b$) The person with care or the Department 
%of Health and Social Services   % Words omitted (16.4.02) by Sch 1B
may apply for such a declaration under 
%section 27 of the Act
section 55A of the Family Law Act 1986\footnote{1986 c.\ 55.}%  Words substituted (16.4.02) by Sch 1B
; and

($c$) The Department 
%of Health and Social Services   % Words omitted (16.4.02) by Sch 1B
may bring an action for declarator of parentage under the provisions of section 28 of the Act.
\end{enumerate}

\amendment{
Pt. VIA inserted (22.1.96) by Sch. 1A to these regulations.

Words substituted in art. 12A(a), (b) and words omitted in art. 12A(b), (c) (16.4.02) by Sch. 1B to these Regulations.
}

\section[Part VII --- Administrative Procedures]{Part VII\\*Administrative Procedures}

\renewcommand\parthead{--- Schedule 1 Part VII}

13.   The Secretary of State and the Department may from time to time determine the administrative procedures appropriate for the purposes of giving effect to this Memorandum.

\section[Part VIII --- Operative Date]{Part VIII\\*Operative Date}

\renewcommand\parthead{--- Schedule 1 Part VIII}

14.   The arrangements in this Memorandum shall come into effect on 5th April 1993 but either Party may terminate them by giving not less than six months notice in writing to the other.

\vfill

%Sch 1A inserted (22.1.96) by SI 1995/3261 reg 54 and Sch
\part[Schedule 1A --- Exchange of letters amending the Memorandum of Arrangements relating to the provision made for child support maintenance in the United Kingdom]{Schedule 1A\\*Exchange of letters amending the Memorandum of Arrangements relating to the provision made for child support maintenance in the United Kingdom}

\renewcommand\parthead{--- Schedule 1A}

\amendment{
Sch. 1A inserted (22.1.96) by the Child Support (Miscellaneous Amendments) (No. 2) Regulations 1995 reg. 54 and Sch.
}

\section*{\sloppy No. 1\\*The Secretary of State for Social Security and the Department of Health and Social Services for Northern Ireland}

7th November 1995

  Sir,

  I have the honour to refer to the Memorandum of Arrangements relating to the provision made for Child Support Maintenance between the Secretary of State for Social Security of the one part and the Department of Health and Social Services for Northern Ireland of the other part which came in to effect on 5 April 1993 (which in this letter is referred to as “the Principal Memorandum”) and to recent discussions between the Department of Social Security and the Department of Health and Social Services for Northern Ireland concerning the need to amend the Principal Memorandum so as to make further provision in relation to child support matters.

  I now have the honour to propose the following amendments to the Principal Memorandum:

  After paragraph (4) of Article 5 there shall be inserted:—
\begin{quotation}
 “(5) Subject to paragraph (7), where an application for a maintenance assessment is made under the provisions for one territory in relation to an absent parent, a person treated as such, or an alleged absent parent who resides in the other territory, that application shall be dealt with in, and in accordance with the provision made for, the territory in which the person with care resides.

(6) Subject to paragraph (7), where an application for a maintenance assessment is made under section 7 of the Act by a qualifying child, the application shall be dealt with in, and in accordance with the provision made for, the territory in which the person with care of that child resides.

\looseness=1
(7) Where paragraphs (5) and (6) apply, the determining authority shall, in determining the amount of child support maintenance to be fixed by any maintenance assessment, take into account in calculating that amount, any provisions which would otherwise have been applicable to that calculation had the assessment been made in accordance with the provision made for the other territory.”.
\end{quotation}

  After Part VI there shall be inserted the following Part:—
\begin{quotation}
 \section*{“Part VIA\\Parentage}

12A.  Where a person with care resides in one territory and an alleged parent who denies that he is one of the parents of a child in respect of whom an application for a maintenance assessment has been made resides in the other territory:—
\begin{enumerate}\item[]
($a$) The person with care or the Secretary of State may apply for a declaration as to whether or not the alleged parent is one of the child’s parents, under Article 28 of the Order;

($b$) The person with care or the Department of Health and Social Services may apply for such a declaration under section 27 of the Act; and

($c$) The Department of Health and Social Services may bring an action for declarator of parentage under the provisions of section~28 of the Act.”.
\end{enumerate}
\end{quotation}

  If the foregoing proposals are acceptable to you, I have the honour to propose that this letter and your reply to that effect shall constitute a Memorandum of Arrangements between us which shall come into effect on 21st January 1996.

  \emph{Andrew Mitchell}

  For the Secretary of State for Social Security

\section*{\sloppy No. 2\\*The Department of Health and Social Services for Northern Ireland to the Secretary of State for Social Security}

8th November 1995

  Sir

  I refer to your letter of 7th November 1995 which reads as follows:

\begin{quotation}
  “I have the honour to refer to the Memorandum of Arrangements relating to the provision made for Child Support Maintenance between the Secretary of State for Social Security of the one part and the Department of Health and Social Services for Northern Ireland of the other part which came in to effect on 5 April 1993 (which in this letter is referred to as “the Principal Memorandum”) and to recent discussions between the Department of Social Security and the Department of Health and Social Services for Northern Ireland concerning the need to amend the Principal Memorandum so as to make further provision in relation to child support matters.

\begin{sloppypar}
  I now have the honour to propose the following amendments to the Principal Memorandum:
\end{sloppypar}

\begin{sloppypar}
  After paragraph (4) of Article 5 there shall be inserted:—
\end{sloppypar}
\begin{quotation}
 “(5) Subject to paragraph (7), where an application for a maintenance assessment is made under the provisions for one territory in relation to an absent parent, a person treated as such, or an alleged absent parent who resides in the other territory, that application shall be dealt with in, and in accordance with the provision made for, the territory in which the person with care resides.

(6) Subject to paragraph (7), where an application for a maintenance assessment is made under section 7 of the Act by a qualifying child, the application shall be dealt with in, and in accordance with the provision made for, the territory in which the person with care of that child resides.

\begin{sloppypar}
(7) Where paragraphs (5) and (6) apply, the determining authority shall, in determining the amount of child support maintenance to be fixed by any maintenance assessment, take into account in calculating that amount, any provisions which would otherwise have been applicable to that calculation had the assessment been made in accordance with the provision made for the other territory.”.
\end{sloppypar}
\end{quotation}

  After Part VI there shall be inserted the following Part:—
\begin{quotation}
 \section*{“Part VIA\\Parentage}

12A.  Where a person with care resides in one territory and an alleged parent who denies that he is one of the parents of a child in respect of whom an application for a maintenance assessment has been made resides in the other territory:—
\begin{enumerate}\item[]
\begin{sloppypar}
($a$) The person with care or the Secretary of State may apply for a declaration as to whether or not the alleged parent is one of the child’s parents, under Article 28 of the Order;
\end{sloppypar}

($b$) The person with care or the Department of Health and Social Services may apply for such a declaration under section 27 of the Act; and

($c$) The Department of Health and Social Services may bring an action for declarator of parentage under the provisions of section 28 of the Act.”.”
\end{enumerate}
\end{quotation}
\end{quotation}

  I have the honour to confirm that the foregoing proposals are acceptable to the Department of Health and Social Services for Northern Ireland and agree that your letter and this reply shall constitute a Memorandum of Arrangements between us which shall come into effect on 21st January 1996.

  Sealed with the Official Seal of the Department of Health and Social Services for Northern Ireland on the 8th day of November 1995.

  \emph{F. A. Elliott}

  Permanent Secretary.

\vfill

\part[Schedule 1B --- Exchange of letters amending the Memorandum of Arrangements relating to the provision made for Child Support Maintenance in the United Kingdom]{Schedule 1B\\*Exchange of letters amending the Memorandum of Arrangements relating to the provision made for Child Support Maintenance in the United Kingdom}

\renewcommand\parthead{--- Schedule 1B}

\amendment{
Sch. 1B inserted (16.4.02) by the Child Support (Northern Ireland Reciprocal Arrangements) Amendment Regulations 2002 reg. 3, Sch. 1.
}

\section*{No.\ 1\\The Parliamentary Under-Secretary of State for Work and Pensions, with the consent of the Treasury, to the Minister for Social Development}

11th March 2002

Sir,

I have the honour to refer to the Memorandum of Arrangements relating to the provision made for Child Support Maintenance between the Secretary of State for Social Security of the one part and the Department of Health and Social Services for Northern Ireland of the other part which came into effect on 5th April 1993, as amended in accordance with the Exchange of Letters from the Secretary of State for Social Security to the Department of Health and Social Services for Northern Ireland of 7th November 1995 and from the Department of Health and Social Services for Northern Ireland to the Secretary of State for Social Security of 8th November 1995 (which Memorandum in its amended form is referred to in this letter as “the Principal Memorandum”). I refer also to recent discussions between the Department for Work and Pensions and the Department for Social Development concerning the need to amend the Principal Memorandum so as to make further provision in relation to child support matters.

I now have the honour, with the consent of the Treasury, to propose the following amendments to the Principal Memorandum:

In Article 1—
\begin{enumerate}\item[]
($a$) after the definition of “the Act” there shall be inserted the following definition—
\begin{quotation}
““the 2000 Act” means the Child Support, Pensions and Social Security Act 2000\footnote{2000 c.\ 19.};”;
\end{quotation}

($b$) after the definition of “application” there shall be inserted the following definition—
\begin{quotation}
““the Department” means the Department for Social Development;”; and
\end{quotation}

($c$) for the definition of “determining authority” there shall be substituted the following definition—
\begin{quotation}
    ““determining authority” means, in relation to Great Britain, the Secretary of State, an appeal tribunal or a Commissioner, and, in relation to Northern Ireland, the Department, an appeal tribunal or a Commissioner;”. 
\end{quotation}
\end{enumerate}

After Article 1 there shall be inserted—
\begin{quotation}
“1A.---(1)  In these arrangements—
\begin{enumerate}\item[]
($a$) references to a maintenance assessment shall, in relevant cases, include references to a maintenance calculation;

($b$) references to an absent parent shall, in relevant cases, include references to a non-resident parent; and

($c$) references to cases where an application for a maintenance assessment is made shall, in relevant cases, include references to cases where an application for a maintenance calculation is treated as having been made.
\end{enumerate}

(2) In this Article, “relevant cases” means cases for the purposes of which section 1 of the 2000 Act has come into force or cases for the purposes of which section 1 of the Child Support, Pensions and Social Security Act (Northern Ireland) 2000\footnote{2000 c.\ 4 (N.I.).} has come into operation.”.
\end{quotation}

Article 3 shall be amended as follows—
\begin{enumerate}\item[]
($a$) in paragraph (1), for the word “12”, there shall be substituted the word “12A”; and

($b$) in paragraph (2), after the word “decision” there shall be inserted the word “, determination”.
\end{enumerate}

Article 7(1) shall be amended as follows—
\begin{enumerate}\item[]
($a$) for the words “, the Department, or a child support officer appointed under the provision made for Great Britain or for Northern Ireland,” there shall be substituted the words “or the Department”; and

($b$) for the words “, the Department or the child support officer for the other territory”, there shall be substituted the words “or the Department”.
\end{enumerate}

Article 12A shall be amended as follows—
\begin{enumerate}\item[]
($a$) in paragraph ($a$), for the words “Article 28 of the Order” there shall be substituted the words “Article 31B of the Matrimonial and Family Proceedings (Northern Ireland) Order 1989\footnote{S.I.\ 1989/677 (N.I.\ 4).}”;

($b$) in paragraph ($b$), for the words “section 27 of the Act” there shall be substituted the words “section 55A of the Family Law Act 1986\footnote{1986 c.\ 55.}”; and

($c$) in paragraphs ($b$)  and ($c$), the words “of Health and Social Services” shall be omitted on each occasion where they occur.
\end{enumerate}

If the foregoing proposals are acceptable to you, I have the honour to propose that this letter and your reply to that effect shall constitute a Memorandum of Arrangements between us which shall come into effect on 16th April 2002.

\section*{No.\ 2\\The Minister for Social Development, with the consent of the Department of Finance and Personnel, to the Parliamentary Under-Secretary of State for Work and Pensions}

14th March 2002

Madam,

I refer to your letter of 11th March 2002 which reads as follows:
\begin{quotation}
I have the honour to refer to the Memorandum of Arrangements relating to the provision made for Child Support Maintenance between the Secretary of State for Social Security of the one part and the Department of Health and Social Services for Northern Ireland of the other part which came into effect on 5th April 1993, as amended in accordance with the Exchange of Letters from the Secretary of State for Social Security to the Department of Health and Social Services for Northern Ireland of 7th November 1995 and from the Department of Health and Social Services for Northern Ireland to the Secretary of State for Social Security of 8th November 1995 (which Memorandum in its amended form is referred to in this letter as “the Principal Memorandum”). I refer also to recent discussions between the Department for Work and Pensions and the Department for Social Development concerning the need to amend the Principal Memorandum so as to make further provision in relation to child support matters.

I now have the honour, with the consent of the Treasury, to propose the following amendments to the Principal Memorandum:

In Article 1—
\begin{enumerate}\item[]
($a$) after the definition of “the Act” there shall be inserted the following definition—
\begin{quotation}
““the 2000 Act” means the Child Support, Pensions and Social Security Act 2000\footnote{2000 c.\ 19.};”;
\end{quotation}

($b$) after the definition of “application” there shall be inserted the following definition—
\begin{quotation}
““the Department” means the Department for Social Development;”; and
\end{quotation}

($c$) for the definition of “determining authority” there shall be substituted the following definition—
\begin{quotation}
    ““determining authority” means, in relation to Great Britain, the Secretary of State, an appeal tribunal or a Commissioner, and, in relation to Northern Ireland, the Department, an appeal tribunal or a Commissioner;”. 
\end{quotation}
\end{enumerate}

After Article 1 there shall be inserted—
\begin{quotation}
“1A.---(1)  In these arrangements—
\begin{enumerate}\item[]
($a$) references to a maintenance assessment shall, in relevant cases, include references to a maintenance calculation;

($b$) references to an absent parent shall, in relevant cases, include references to a non-resident parent; and

($c$) references to cases where an application for a maintenance assessment is made shall, in relevant cases, include references to cases where an application for a maintenance calculation is treated as having been made.
\end{enumerate}

(2) In this Article, “relevant cases” means cases for the purposes of which section 1 of the 2000 Act has come into force or cases for the purposes of which section 1 of the Child Support, Pensions and Social Security Act (Northern Ireland) 2000\footnote{2000 c.\ 4 (N.I.).} has come into operation.”.
\end{quotation}

Article 3 shall be amended as follows—
\begin{enumerate}\item[]
($a$) in paragraph (1), for the word “12”, there shall be substituted the word “12A”; and

($b$) in paragraph (2), after the word “decision” there shall be inserted the word “, determination”.
\end{enumerate}

Article 7(1) shall be amended as follows—
\begin{enumerate}\item[]
($a$) for the words “, the Department, or a child support officer appointed under the provision made for Great Britain or Northern Ireland,” there shall be substituted the words “or the Department”; and

($b$) for the words “, the Department or the child support officer for the other territory”, there shall be substituted the words “or the Department”.
\end{enumerate}

Article 12A shall be amended as follows—
\begin{enumerate}\item[]
($a$) in paragraph ($a$), for the words “Article 28 of the Order” there shall be substituted the words “Article 31B of the Matrimonial and Family Proceedings (Northern Ireland) Order 1989\footnote{S.I.\ 1989/677 (N.I.\ 4).}”;

($b$) in paragraph ($b$), for the words “section 27 of the Act” there shall be substituted the words “section 55A of the Family Law Act 1986\footnote{1986 c.\ 55.}”; and

($c$) in paragraphs ($b$)  and ($c$), the words “of Health and Social Services” shall be omitted on each occasion where they occur.”.
\end{enumerate}
\end{quotation}

\looseness=-1
I have the honour to confirm, with the consent of the Department of Finance and Personnel, that the foregoing proposals are acceptable to the Minister for Social Development, and agree that your letter and this reply shall constitute a Memorandum of Arrangements between us which shall come into effect on 16th April 2002.

% Sch 1C inserted (29.10.12) by SI 2012/2380 reg 2(3) Sch 1
\part[Schedule 1C --- Exchange of letters amending the Memorandum of Arrangements relating to the provision made for child support maintenance in the United Kingdom]{Schedule 1C\\*Exchange of letters amending the Memorandum of Arrangements relating to the provision made for child support maintenance in the United Kingdom}

\renewcommand\parthead{--- Schedule 1C}

\section*{No 1\\*The Parliamentary Under-Secretary of State for Work and Pensions, with the consent of the Treasury, to the Minister for Social Development}

9 August 2012

Sir,

I have the honour to refer to the Memorandum of Arrangements relating to the 
provision made for Child Support Maintenance between the Secretary of State for
Social Security of the one part and the Department of Health and Social Services for
Northern Ireland of the other part which came into effect on 5th April 1993, as
amended in accordance with---
\begin{enumerate}\item[]
($a$) the Exchange of Letters from the Secretary of State for Social Security to the Department of Health and Social Services for Northern Ireland of 7th November 1995 and from the Department of Health and Social Services for Northern Ireland to the Secretary of State for Social Security of 8th November 1995; and

($b$) the Exchange of Letters from the Parliamentary Under-Secretary of State for Work and Pensions to the Minister of Social Development of 11th March 2002 and from the Minister for Social Development to the Parliamentary Under-Secretary of State for Work and Pensions of 14th March 2002,
\end{enumerate}
(which Memorandum in its amended form is referred to in this letter as ``the Principal Memorandum'').

I refer also to recent discussions between the Department for Work and Pensions and the Department for Social Development concerning the need to amend the Principal Memorandum so as to make further provision in relation to child support matters.

I now have the honour, with the consent of the Treasury, to propose the following amendments to the Principal Memorandum:

In Article 5---
\begin{enumerate}\item[]
($a$) in paragraph (1) for ``and (4)'' substitute ``,~(4) and (8)'';

($b$) in paragraph (2) for ``paragraph (4)'' substitute ``paragraphs~(4) and (8)'';

($c$) in paragraph (3) for ``paragraph (4)'' substitute ``paragraphs~(4) and (8)'';

($d$) in paragraph (5) for ``paragraph (7)'' substitute ``paragraphs~(7) and (8)'';

($e$) in paragraph (6) for ``paragraph (7)'' substitute ``paragraphs~(7) and (8)''; and

($f$) after paragraph (7) insert the following paragraphs---
\begin{quotation}
``(8) An application for a maintenance calculation which is to be determined in accordance with the new calculation rules shall be dealt with in, and in accordance with the provision made for, the territory in which the person who is, or is treated or alleged to be, the non-resident parent in relation to that application resides.

(9) Where paragraph (8) applies to an application for a maintenance calculation and there is an existing case in respect of which the same person is, or is treated as or alleged to be, the non-resident parent, that case shall also be dealt with (insofar as it is not already) in, and in accordance with the provision made for, the territory in which that person resides.

(10) For the purpose of paragraphs (8) and (9), where the person who is, or is treated as or alleged to be, the non-resident parent in relation to the application falls within section 44(2A) of the Act, or Article 41(2A) of the Order, that person shall be treated as if residing in Great Britain.

(11) In this Article---
\begin{enumerate}\item[]
``existing case'' means any case where the maintenance assessment or maintenance calculation is made, or will fall to be made, otherwise than in accordance with the new calculation rules;

``new calculation rules'' means Part~I of Schedule~1 to the Act as amended by Schedule~4 to the Child Maintenance and Other Payments Act 2008, or Part~I of Schedule~1 to the Order as amended by Schedule~1 to the Child Maintenance Act (Northern Ireland) 2008.
\end{enumerate}

(12) In paragraphs (9) and (10), where relevant, references to non-resident parent include references to absent parent.''
\end{quotation}
\end{enumerate}

If the foregoing proposals are acceptable to you, I have the honour to propose that this letter and your reply to that effect shall constitute a Memorandum of Arrangements between us which it is proposed shall come into effect on 29th October 2012.

\bigskip

\pagebreak[3]

Signed 
by authority of the 
Secretary of State for~Work and~Pensions.
%I concur
%By authority of the Lord Chancellor

{\raggedleft
\emph{Maria Miller}\\*
%Secretary
%Minister
Parliamentary Under-Secretary 
of State\\*Department 
for~Work and~Pensions

}

9th August 2012

\bigskip

\pagebreak[3]

%Signed 
%by authority of the 
%Secretary of State for~Work and~Pensions.
%I concur
%By authority of the Lord Chancellor
We consent

{\raggedleft
\emph{Jeremy Wright}\\*
\emph{Brooks Newmark}\\*
Two of the Lords Commissioners of Her Majesty's Treasury

}

4th September 2012

\vfill

\section*{No 2\\*The Minister for Social Development, with the consent of the Department of Finance and Personnel, to the Parliamentary Under-Secretary of State for Work and Pensions}

10th September 2012

Madam,

I refer to your letter of 9th August 2012 which reads as follows:
\begin{quotation}
I have the honour to refer to the Memorandum of Arrangements relating to the 
provision made for Child Support Maintenance between the Secretary of State for
Social Security of the one part and the Department of Health and Social Services for
Northern Ireland of the other part which came into effect on 5th April 1993, as
amended in accordance with---
\begin{enumerate}\item[]
($a$) the Exchange of Letters from the Secretary of State for Social Security to the Department of Health and Social Services for Northern Ireland of 7th November 1995 and from the Department of Health and Social Services for Northern Ireland to the Secretary of State for Social Security of 8th November 1995; and

($b$) the Exchange of Letters from the Parliamentary Under-Secretary of State for Work and Pensions to the Minister of Social Development of 11th March 2002 and from the Minister for Social Development to the Parliamentary Under-Secretary of State for Work and Pensions of 14th March 2002,
\end{enumerate}
(which Memorandum in its amended form is referred to in this letter as ``the Principal Memorandum'').

I refer also to recent discussions between the Department for Work and Pensions and the Department for Social Development concerning the need to amend the Principal Memorandum so as to make further provision in relation to child support matters.

I now have the honour, with the consent of the Treasury, to propose the following amendments to the Principal Memorandum:

In Article 5---
\begin{enumerate}\item[]
($a$) in paragraph (1) for ``and (4)'' substitute ``,~(4) and~(8)'';

\begin{sloppypar}
($b$) in paragraph (2) for ``paragraph (4)'' substitute ``paragraphs~(4) and (8)'';
\end{sloppypar}

\begin{sloppypar}
($c$) in paragraph (3) for ``paragraph (4)'' substitute ``paragraphs~(4) and (8)'';
\end{sloppypar}

\begin{sloppypar}
($d$) in paragraph (5) for ``paragraph (7)'' substitute ``paragraphs~(7) and (8)'';
\end{sloppypar}

\begin{sloppypar}
($e$) in paragraph (6) for ``paragraph (7)'' substitute ``paragraphs~(7) and (8)''; and
\end{sloppypar}

($f$) after paragraph (7) insert the following paragraphs---
\begin{quotation}
``(8) An application for a maintenance calculation which is to be determined in accordance with the new calculation rules shall be dealt with in, and in accordance with the provision made for, the territory in which the person who is, or is treated or alleged to be, the non-resident parent in relation to that application resides.

(9) Where paragraph (8) applies to an application for a maintenance calculation and there is an existing case in respect of which the same person is, or is treated as or alleged to be, the non-resident parent, that case shall also be dealt with (insofar as it is not already) in, and in accordance with the provision made for, the territory in which that person resides.

(10) For the purpose of paragraphs (8) and~(9), where the person who is, or is treated as or alleged to be, the non-resident parent in relation to the application falls within section~44(2A) of the Act, or Article 41(2A) of the Order, that person shall be treated as if residing in Great Britain.

(11) In this Article---
\begin{enumerate}\item[]
``existing case'' means any case where the maintenance assessment or maintenance calculation is made, or will fall to be made, otherwise than in accordance with the new calculation rules;

``new calculation rules'' means Part~I of Schedule~1 to the Act as amended by Schedule~4 to the Child Maintenance and Other Payments Act 2008, or Part~I of Schedule~1 to the Order as amended by Schedule~1 to the Child Maintenance Act (Northern Ireland) 2008.\looseness=-1
\end{enumerate}

(12) In paragraphs (9) and (10), where relevant, references to non-resident parent include references to absent parent.''
\end{quotation}
\end{enumerate}
\end{quotation}

I have the honour to confirm, with the consent of the Department of Finance and Personnel, that the foregoing proposals are acceptable to the Minister for Social Development, and agree that your letter and this reply shall constitute a Memorandum of Arrangements between us which it is proposed shall come into effect on 29th October 2012.

\bigskip

\pagebreak[3]

Sealed with the Official Seal of the Department for Social Development on 10th September 2012.

{\raggedleft
\emph{Nelson McCausland}\\*
Minister for Social Development

}

\bigskip

\pagebreak[3]

The Department of Finance and Personnel hereby consents.

Sealed with the Official Seal of the Department of Finance and Personnel on 10th September 2012.

{\raggedleft
\emph{John McKibbin}\\*
Senior Officer of the Department of Finance and Personnel

}

\vfill

\part[Schedule 1D --- Exchange of Letters amending the Memorandum of Arrangements relating to the provision made for child support maintenance in the United Kingdom]{Schedule 1D\\*Exchange of Letters amending the Memorandum of Arrangements relating to the provision made for child support maintenance in the United Kingdom}

\renewcommand\parthead{--- Schedule 1D}

\section*{\sloppy No. 1\\*The Minister of State for Work and Pensions, with the consent of the Treasury, to the Minister for Social Development}

14th May 2014

Sir,

I have the honour to refer to the Memorandum of Arrangements relating to the provision 
made for Child Support Maintenance between the Secretary of State for Social Security of 
the one part and the Department of Health and Social Services for Northern Ireland of the 
other part which came into effect on 5th April 1993, as amended in accordance with---
\begin{enumerate}\item[] 
($a$) the Exchange of Letters from the Secretary of State for Social Security to the 
Department of Health and Social Services for Northern Ireland of 7th November 
1995 and from the Department of Health and Social Services for Northern Ireland 
to the Secretary of State for Social Security of 8th November 1995;

($b$) the Exchange of Letters from the Parliamentary Under-Secretary of State for Work
and Pensions to the Minister for Social Development of 11th March 2002 and from
the Minister for Social Development to the Parliamentary Under-Secretary of State
for Work and Pensions of 14th March 2002; and

($c$) the Exchange of Letters from the Parliamentary Under-Secretary of State for Work
and Pensions to the Minister for Social Development of 9th August 2012 and from
the Minister for Social Development to the Parliamentary Under-Secretary of State
for Work and Pensions of 10th September 2012,
\end{enumerate}
(which Memorandum in its amended form is referred to in this letter as ``the Principal
Memorandum'').

I refer also to recent discussions between the Department for Work and Pensions and the
Department for Social Development concerning the need to amend the Principal
Memorandum so as to make further provision in relation to child support matters.

I now have the honour, with the consent of the Treasury, to propose the following amendments to the Principal Memorandum.

In Article 5---
\begin{enumerate}\item[]
($a$) for paragraph (8) substitute---
\begin{quotation}
``(8) An application for a maintenance calculation which is to be determined in
accordance with the new calculation rules shall be dealt with in, and in accordance
with the provision made for, the territory in which the person who makes the
application resides until---
\begin{enumerate}\item[]
($a$) where the applicant resides in Great Britain---
\begin{enumerate}\item[]
(i) the application is taken to have been made for the purposes of
regulation 9(2) (applications under section~4 or 7 of the Act) of the
Child Support Maintenance Calculation Regulations 2012\footnote{S.I.~2012/2677.},

(ii) any application fee payable under regulation 3(1) of the Child Support Fees Regulations 2014\footnote{S.I.~2014/612.} has been paid or waived in accordance with those Regulations,

(iii) the Secretary of State has ascertained and verified the address of the
non-resident parent in relation to the application, and

\begin{sloppypar}
(iv) where the application is one to which paragraph~(8A) or (8B) applies,
the condition in that paragraph is satisfied; or
\end{sloppypar}
\end{enumerate}

($b$) where the applicant resides in Northern Ireland---
\begin{enumerate}\item[]
(i) the application is taken to have been made for the purposes of
regulation 9(2) (applications under Article 7) of the Child Support
Maintenance Calculation Regulations (Northern Ireland) 2012\footnote{S.R.~2012 No.~427.},

(ii) the Department has ascertained and verified the address of the non-%
resident parent in relation to the application, and

\begin{sloppypar}
(iii) where the application is one to which paragraph~(8A) or (8B) applies,
the condition in that paragraph is satisfied,
\end{sloppypar}
\end{enumerate}
\end{enumerate}
from which point the case shall be dealt with in, and in accordance with the
provision made for, the territory in which the non-resident parent in relation to the 
application resides.'';
\end{quotation}

($b$) after paragraph (8), insert---

\begin{quotation}
``(8A) Where there is an existing case related to the application, in relation to
which the interested parties have been required to choose whether or not to stay in 
the statutory scheme (under Schedule 5 to the 2008 Act\footnote{Schedule 5 was amended by section 136 of the Welfare Reform Act 2012.} or Schedule 2 to the Child
Maintenance Act (Northern Ireland) 2008\footnote{2008 c.~10 (N.I.).}) as a result of that application, the
condition is that any liability end date in relation to that existing case must have
passed.

(8B) Where the applicant has chosen to remain in the statutory scheme, in response
to being requied to choose in an existing case whether or not to stay in the statutory 
scheme (under Schedule~5 to the 2008 Act or Schedule 2 to the Child Maintenance 
Act (Northern Ireland) 2008), the condition is that the liability end date in relation to
that existing case must have passed.

(8C) For the purposes of paragraph (8A), an existing case is related to an
application if---
\begin{enumerate}\item[]
($a$) the non-resident parent in relation to that application is also the non-resident
parent in relation to the existing case and the person with care in relation to
that application is not the person with care in relation to the existing case, or

($b$) the non-resident parent in relation to that application is a partner of a non-%
resident parent in relation to the existing case and either or both are in 
receipt of a benefit prescribed by regulations made under paragraph 4(1)($c$)
(flat rate) of Schedule 1 to the Act\footnote{The substitution of Part~I of Schedule~1 to the Act by section~1(3) of, and Schedule~1 to, the Child Support, Pensions and Social Security Act 2000 (c.~19) was partially commenced for the types of cases specified in article 3 of the Child Support, Pensions and Social Security Act 2000 (Commencement No.~12) Order 2003 (S.I.~2003/192).} or paragraph 4(1)($c$) (flat rate) of
Schedule 1 to the Order\footnote{The substitution of Part~I of Schedule~1 to the Order by section~1(3) of, and Schedule~1 to, the Child Support, Pensions and Social Security Act (Northern Ireland) 2000 (c.~4 (N.I.)) was partially commenced for the types of cases specified in Article 3 of the Child Support, Pensions and Social Security (2000 Act) (Commencement No.~9) Order (Northern Ireland) 2003 (S.R.~2003 No.~53).}.
\end{enumerate}

(8D) For the purposes of paragraphs (8) and (8C), a non-resident parent includes a 
person who is treated as or alleged to be a non-resident parent.'';
\end{quotation}

($c$) paragraph (9) is omitted;

($d$) in paragraph (10)---
\begin{enumerate}\item[]
(i) for ``paragraphs'' substitute ``paragraph'',

(ii) omit ``and (9)'';
\end{enumerate}

($e$) in paragraph (11)---
\begin{enumerate}\item[]
(i) before the definition of ``existing case'' insert---
\begin{quotation}
````the 2008 Act'' means the Child Maintenance and Other Payments Act 2008\footnote{2008 c.~6.};'',
\end{quotation}

(ii) insert the following definitions where they fall alphabetically---
\begin{quotation}
````interested parties'' means the non-resident parent, the person with care and, in
the case of an application made by a qualifying child under section 7(1) of the
Act, or a maintenance calculation or assessment made in response to an
application under that section, the child in question;'',

````liability end date'' means the date determined in accordance with---
\begin{enumerate}\item[]
($a$) regulations made under Schedule 5 (maintenance calculations: transfer
of cases to new rules) to the 2008 Act as the date beyond which no 
further liability accrues in relation to the existing case for the purposes
of paragraph~5(1) and (2) of that Schedule, or

($b$) regulations made under Schedule 2 (maintenance calculations: transfer
of cases to new rules) to the Child Maintenance Act (Northern Ireland)
2008 as the date beyond which no further liability accrues in relation to
the existing case for the purposes of paragraph 5(1) and~(2) of that
Schedule;'',
\end{enumerate}

````partner'' means a person falling within the definition of ``partner'' given in 
paragraph 10C(4) of Schedule 1 (maintenance calculations---reference to various
terms) to the Act or paragraph~10C(4) of Schedule~1 (maintenance calculations---%
reference to various terms) to the Order.'';
\end{quotation}
\end{enumerate}

($f$) in paragraph (12), for ``(9) and (10)'' substitute ``(8C), (8D), (10) and~(11)''.
\end{enumerate}

If the foregoing proposals are acceptable to you, I have the honour to propose that this 
letter and your reply to that effect shall constitute a Memorandum of Arrangements
between us which it is proposed shall come into effect on 30th June 2014.

\bigskip

\pagebreak[3]

Signed 
by authority of the 
Secretary of State for~Work and~Pensions.
%I concur
%By authority of the Lord Chancellor

{\raggedleft
\emph{Steve Webb}\\*
%Secretary
Minister
%Parliamentary Under Secretary 
of State\\*Department 
for~Work and~Pensions

}

14th May 2014

\bigskip

\pagebreak[3]

We consent

{\raggedleft
\emph{Sam Gyimah\\*
Mark Lancaster}\\*Two of the Lords Commissioners of Her Majesty's Treasury

}

12th May 2014

\section*{No. 2\\*
The Minister for Social Development, with the consent of the Department of Finance and Personnel, to the Minister of State for Work and Pensions}

Sir,

I refer to your letter of 14th May 2014 which reads as follows:

I have the honour to refer to the Memorandum of Arrangements relating to the provision 
made for Child Support Maintenance between the Secretary of State for Social Security of 
the one part and the Department of Health and Social Services for Northern Ireland of the 
other part which came into effect on 5th April 1993, as amended in accordance with---
\begin{enumerate}\item[] 
($a$) the Exchange of Letters from the Secretary of State for Social Security to the 
Department of Health and Social Services for Northern Ireland of 7th November 
1995 and from the Department of Health and Social Services for Northern Ireland 
to the Secretary of State for Social Security of 8th November 1995;

($b$) the Exchange of Letters from the Parliamentary Under-Secretary of State for Work
and Pensions to the Minister for Social Development of 11th March 2002 and from
the Minister for Social Development to the Parliamentary Under-Secretary of State
for Work and Pensions of 14th March 2002; and

($c$) the Exchange of Letters from the Parliamentary Under-Secretary of State for Work
and Pensions to the Minister for Social Development of 9th August 2012 and from
the Minister for Social Development to the Parliamentary Under-Secretary of State
for Work and Pensions of 10th September 2012,
\end{enumerate}
(which Memorandum in its amended form is referred to in this letter as ``the Principal
Memorandum'').

I refer also to recent discussions between the Department for Work and Pensions and the
Department for Social Development concerning the need to amend the Principal
Memorandum so as to make further provision in relation to child support matters.

I now have the honour, with the consent of the Treasury, to propose the following amendments to the Principal Memorandum.

In Article 5---
\begin{enumerate}\item[]
($a$) for paragraph (8) substitute---
\begin{quotation}
``(8) An application for a maintenance calculation which is to be determined in
accordance with the new calculation rules shall be dealt with in, and in accordance
with the provision made for, the territory in which the person who makes the
application resides until---
\begin{enumerate}\item[]
($a$) where the applicant resides in Great Britain---
\begin{enumerate}\item[]
(i) the application is taken to have been made for the purposes of
regulation 9(2) (applications under section~4 or 7 of the Act) of the
Child Support Maintenance Calculation Regulations 2012\footnote{S.I.~2012/2677.},

(ii) any application fee payable under regulation 3(1) of the Child Support Fees Regulations 2014\footnote{S.I.~2014/612.} has been paid or waived in accordance with those Regulations,

(iii) the Secretary of State has ascertained and verified the address of the
non-resident parent in relation to the application, and

\begin{sloppypar}
(iv) where the application is one to which paragraph~(8A) or (8B) applies,
the condition in that paragraph is satisfied; or
\end{sloppypar}
\end{enumerate}

($b$) where the applicant resides in Northern Ireland---
\begin{enumerate}\item[]
(i) the application is taken to have been made for the purposes of
regulation 9(2) (applications under Article 7) of the Child Support
Maintenance Calculation Regulations (Northern Ireland) 2012\footnote{S.R.~2012 No.~427.},

(ii) the Department has ascertained and verified the address of the non-%
resident parent in relation to the application, and

\begin{sloppypar}
(iii) where the application is one to which paragraph~(8A) or (8B) applies,
the condition in that paragraph is satisfied,
\end{sloppypar}
\end{enumerate}
\end{enumerate}
from which point the case shall be dealt with in, and in accordance with the
provision made for, the territory in which the non-resident parent in relation to the 
application resides.'';
\end{quotation}

($b$) after paragraph (8), insert---

\begin{quotation}
``(8A) Where there is an existing case related to the application, in relation to
which the interested parties have been required to choose whether or not to stay in 
the statutory scheme (under Schedule 5 to the 2008 Act\footnote{Schedule 5 was amended by section 136 of the Welfare Reform Act 2012.} or Schedule 2 to the Child
Maintenance Act (Northern Ireland) 2008\footnote{2008 c.~10 (N.I.).}) as a result of that application, the
condition is that any liability end date in relation to that existing case must have
passed.

(8B) Where the applicant has chosen to remain in the statutory scheme, in response
to being requied to choose in an existing case whether or not to stay in the statutory 
scheme (under Schedule~5 to the 2008 Act or Schedule 2 to the Child Maintenance 
Act (Northern Ireland) 2008), the condition is that the liability end date in relation to
that existing case must have passed.

(8C) For the purposes of paragraph (8A), an existing case is related to an
application if---
\begin{enumerate}\item[]
($a$) the non-resident parent in relation to that application is also the non-resident
parent in relation to the existing case and the person with care in relation to
that application is not the person with care in relation to the existing case, or

($b$) the non-resident parent in relation to that application is a partner of a non-%
resident parent in relation to the existing case and either or both are in 
receipt of a benefit prescribed by regulations made under paragraph 4(1)($c$)
(flat rate) of Schedule 1 to the Act\footnote{The substitution of Part~I of Schedule~1 to the Act by section~1(3) of, and Schedule~1 to, the Child Support, Pensions and Social Security Act 2000 (c.~19) was partially commenced for the types of cases specified in article 3 of the Child Support, Pensions and Social Security Act 2000 (Commencement No.~12) Order 2003 (S.I.~2003/192).} or paragraph 4(1)($c$) (flat rate) of
Schedule 1 to the Order\footnote{The substitution of Part~I of Schedule~1 to the Order by section~1(3) of, and Schedule~1 to, the Child Support, Pensions and Social Security Act (Northern Ireland) 2000 (c.~4 (N.I.)) was partially commenced for the types of cases specified in Article 3 of the Child Support, Pensions and Social Security (2000 Act) (Commencement No.~9) Order (Northern Ireland) 2003 (S.R.~2003 No.~53).}.
\end{enumerate}

(8D) For the purposes of paragraphs (8) and (8C), a non-resident parent includes a 
person who is treated as or alleged to be a non-resident parent.'';
\end{quotation}

($c$) paragraph (9) is omitted;

($d$) in paragraph (10)---
\begin{enumerate}\item[]
(i) for ``paragraphs'' substitute ``paragraph'',

(ii) omit ``and (9)'';
\end{enumerate}

($e$) in paragraph (11)---
\begin{enumerate}\item[]
(i) before the definition of ``existing case'' insert---
\begin{quotation}
````the 2008 Act'' means the Child Maintenance and Other Payments Act 2008\footnote{2008 c.~6.};'',
\end{quotation}

(ii) insert the following definitions where they fall alphabetically---
\begin{quotation}
````interested parties'' means the non-resident parent, the person with care and, in
the case of an application made by a qualifying child under section 7(1) of the
Act, or a maintenance calculation or assessment made in response to an
application under that section, the child in question;'',

````liability end date'' means the date determined in accordance with---
\begin{enumerate}\item[]
($a$) regulations made under Schedule 5 (maintenance calculations: transfer
of cases to new rules) to the 2008 Act as the date beyond which no 
further liability accrues in relation to the existing case for the purposes
of paragraph~5(1) and (2) of that Schedule, or

($b$) regulations made under Schedule 2 (maintenance calculations: transfer
of cases to new rules) to the Child Maintenance Act (Northern Ireland)
2008 as the date beyond which no further liability accrues in relation to
the existing case for the purposes of paragraph 5(1) and~(2) of that
Schedule;'',
\end{enumerate}

````partner'' means a person falling within the definition of ``partner'' given in 
paragraph 10C(4) of Schedule 1 (maintenance calculations---reference to various
terms) to the Act or paragraph~10C(4) of Schedule~1 (maintenance calculations---%
reference to various terms) to the Order.'';
\end{quotation}
\end{enumerate}

($f$) in paragraph (12), for ``(9) and (10)'' substitute ``(8C), (8D), (10) and~(11)''.
\end{enumerate}

I have the honour to confirm, with the consent of the Department of Finance and
Personnel, that the foregoing proposals are acceptable and agree that your letter and this
reply shall constitute a Memorandum of Arrangements between us which it is proposed
shall come into effect on 30th June 2014.

\bigskip

\pagebreak[3]

Sealed with the Official Seal of the Department for Social Development on 21st May 2014.

{\raggedleft
\emph{Nelson McCausland}\\*
Minister for Social Development

}

\bigskip

\pagebreak[3]

The Department of Finance and Personnel hereby consents.

Sealed with the Official Seal of the Department of Finance and Personnel on 21st May 2014.

{\raggedleft
\emph{John McKibbin}\\*Senior Officer of the Department of Finance and Personnel

}

\vfill

\part[Schedule 2 --- Adaptation of certain provisions of the Child Support Act 1991]{Schedule 2\\*Adaptation of certain provisions of the Child Support Act 1991}

\renewcommand\parthead{--- Schedule 2}

\noindent
%\begin{tabulary}{\linewidth}{JJJ}
\begin{longtable}{p{71.1145pt}p{108.42673pt}p{174.45883pt}}
\hline
Column 1 & Column 2 & Column 3\\
\itshape\raggedright Provisions of the Child Support Act 1991 & \itshape Provisions of the Child Support (Northern Ireland) Order 1991 & \itshape Subject Matter\\
\hline
\endhead
\hline
\endlastfoot
Section 1&Article 5&The duty to maintain\\
Section 2&Article 6&Welfare of children: the general principle\\
Section 8&Article 10&Role of the courts with respect to maintenance for children\\
Section 9&Article 11&Agreements about maintenance\\
Section 10&Article 12&Relationship between maintenance assessments and certain court orders and related matters\\
Section 14A &Article 16A &Information—offences. \\ % Entry inserted (16.4.02) by SI 2002/771 reg 4(2)
Section 15&Article 17&Powers of inspectors\\
%Sections 27 and 28 & Article 28 & Declaration of parentage\\ % Entry inserted (22.1.96) by SI 1995/3261 reg 55, omitted (16.4.02) by SI 2002/771 reg 4(3)
Section 29&Article 29&Collection of child support maintenance\\
Section 30&Article 30&Collection and enforcement of other forms of maintenance\\
%\end{tabulary}
\end{longtable}

\amendment{
Entry relating to ss. 27, 28 inserted (22.1.96) by the Child Support (Miscellaneous Amendments) (No. 2) Regulations 1995 reg. 55.

Entry relating to s. 14A inserted and entry relating to ss. 27, 28 omitted (16.4.02) by the Child Support (Northern Ireland Reciprocal Arrangements) Amendment Regulations 2002 reg. 4.
}

\bigskip

Signed on 8th day of March 1993.

{\raggedleft
\emph{Peter Lilley}\\*Secretary of State for Social Security

}

\bigskip

Sealed with the Official Seal of the Department of Health and Social Services for Northern Ireland on 9th day of March 1993.


{\raggedleft
\emph{F.\ A.\ Elliott}\\*Permanent Secretary

}

\vfill

% Sch 3 inserted (16.4.02) by SI 2002/771 reg 5, Sch 2
\part[Schedule 3 --- Adaptation of the Child Support Act 1991 and the Family Law Act 1986 for the purposes of the Child Support Act 1991]{Schedule 3\\*Adaptation of the Child Support Act 1991 and the Family Law Act 1986\footnote{1986 c.\ 55.} for the purposes of the Child Support Act 1991}

\amendment{
Sch. 3 inserted (16.4.02) by the Child Support (Northern Ireland Reciprocal Arrangements) Amendment Regulations 2002 reg. 5, Sch. 2.
}

\medskip

\noindent
{\footnotesize
\begin{longtable}{p{142.207pt}p{100.4398pt}p{111.34279pt}}
%\begin{tabulary}{\linewidth}{JJJ}
\hline
\itshape Column 1	& \itshape Column 2	& \itshape Column 3\\
\itshape Provision of the Child Support Act 1991 (“the 1991 Act”) or the Family Law Act 1986 (“the 1986 Act”)	& \itshape\raggedright Provision of the Matrimonial Proceedings (Northern Ireland) Order 1989\footnote{S.I.\ 1989/677 (N.I.\ 4).}	& \itshape Subject matter\\
\hline
\endhead
\hline
\endlastfoot
Section 28 of the 1991 Act	&Article 31B	&Application for declaration of parentage for the purposes of the 1991 Act\\
Section 55A of the 1986 Act	&Article 31B	&Application for declaration of parentage for the purposes of the 1991 Act.\\
\end{longtable}
%\end{tabulary}

}


\bigskip

Sealed with the Official Seal of the Department for Social Development on 14th March 2002.

{\raggedleft
\emph{Nigel Dodds}\\
Minister for Social Development

}

\bigskip

The Department of Finance and Personnel hereby consents.

Sealed with the Official Seal of the Department of Finance and Personnel on 19th March 2002.

{\raggedleft
\emph{N.~Taylor}\\
Senior Officer of the\\
Department of Finance and Personnel

}

\part{Explanatory Note}

\renewcommand\parthead{--- Explanatory Note}

\subsection*{(This note is not part of the Regulations)}

 These Regulations give effect in Great Britain to reciprocal arrangements relating to matters for which provision is made in Great Britain by the Child Support Act 1991. The arrangements are contained in the Memorandum set out in Schedule 1 to the Regulations and have been made between the Secretary of State for Social Security and the Department of Health and Social Services for Northern Ireland.

  The Regulations provide that certain matters to which the provisions of the Northern Ireland legislation relate (which are set out in Schedule 2 to the Regulations) have a corresponding effect in respect of the provisions of the Child Support Act 1991.


\end{document}