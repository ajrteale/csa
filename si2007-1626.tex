\documentclass[12pt,a4paper]{article}

\newcommand\regstitle{The Social Security (Miscellaneous Amendments) (No.~2) Regulations 2007}

\newcommand\regsnumber{2007/1626}

%\opt{newrules}{
\title{\regstitle}
%}

%\opt{2012rules}{
%\title{Child Maintenance and~Other Payments Act 2008\\(2012 scheme version)}
%}

\author{S.I.\ 2007 No.\ 1626}

\date{Made
7th June 2007\\
Laid before Parliament
12th June 2007\\
Coming into~force
3rd July 2007
}

%\opt{oldrules}{\newcommand\versionyear{1993}}
%\opt{newrules}{\newcommand\versionyear{2003}}
%\opt{2012rules}{\newcommand\versionyear{2012}}

\usepackage{csa-regs}

\setlength\headheight{27.61603pt}

%\hbadness=10000

\begin{document}

\maketitle

\noindent
The Secretary of State for Work and~Pensions makes the following Regulations in exercise of the powers conferred by sections~171A(3) and~175(1) to (3) and~(5) of the Social Security Contributions and~Benefits Act 1992\footnote{1992 c.~4. Section~171A was inserted by the Social Security (Incapacity for Work) Act 1994 (c.~18), section 5. Section~171A(3) was amended by the Welfare Reform and Pensions Act 1999 (c.~30), Schedule 8, Part II, paragraphs 20, 23(1) and (3). The relevant amendments to section 175(1) and (5) are the Social Security Contributions (Transfer of Functions, etc).\ Act 1999 (c.~2), Schedule 3, paragraph~29(1) and (2) and the Social Security (Incapacity for Work) Act 1994 (c.~18), Schedule 1, paragraph~36.}, sections~9(1) and~189(1), (3), (4) and~(6) of the Social Security Administration Act 1992\footnote{1992 c.~5. Section~189(1) was amended by the Social Security Act 1998 (c.~14), Schedule 7, paragraph~109($a$), Schedule 8, by the Social Security Contributions (Transfer of Functions, etc.)\ Act 1999 (c.~2), Schedule 3, paragraph~57(1) and (2) and by the Tax Credits Act 2002 (c.~21), Schedule 6.} and~sections~11(1), 24($a$)  and~($b$)  and~79(1), (3), (4) and~(7) of the Social Security Act 1998\footnote{1998 c.~14. Section~11 was amended by the State Pension Credit Act 2002 (c.~16), Schedule 3 and Schedule 1 Part II, paragraphs 4 and 7. Section~79(1) was amended by the Tax Credits Act 2002 (c.~21), Schedule 4, paragraphs 12, 13(1) and (2).}.

In accordance with section 173(5)($b$)  of the Social Security Administration Act 1992, this instrument contains only regulations made consequential upon section 62 of the Welfare Reform Act 2007\footnote{2007 c.~5.} and~is made before the end of the period of 6 months beginning with the coming into force of that section. 

{\sloppy

\tableofcontents

}

\bigskip

\setcounter{secnumdepth}{-2}

\subsection[1. Citation and~commencement]{Citation and~commencement}

1.  These Regulations may be cited as the Social Security (Miscellaneous Amendments) (No.~2) Regulations 2007 and~shall come into force on 3rd July 2007.

\subsection[2. Amendment of the Social Security (Claims and~Payments) Regulations 1979]{Amendment of the Social Security (Claims and~Payments) Regulations 1979}

2.---(1)  The Social Security (Claims and~Payments) Regulations 1979\footnote{S.I.~1979/628.} are amended as follows.

(2) In regulation 2(1)\footnote{Regulation 2 was amended by S.I.~1980/1136 and S.I.~1999/1958.} (interpretation), after the definition of “claim for benefit” insert—
\begin{quotation}
““health care professional” means—
\begin{enumerate}\item[]
($a$) 
a registered medical practitioner,

($b$) 
a registered nurse,

($c$) 
an occupational therapist or physiotherapist registered with a regulatory body established by an Order in Council under section 60 of the Health Care Act 1999\footnote{1999 c.~8. Section 60 was amended by the National Health Service Reform and Health Care Professions Act 2002 (c.~17), section 26(9) and by S.I.~2002/254.}, or

($d$) 
a member of such other profession, regulated by a body mentioned in section 25(3) of the National Health Service Reform and~Health Care Professions Act 2002\footnote{2002 c.~17.}, prescribed by the Secretary of State in accordance with powers under section 39(1) of the Social Security Act 1998\footnote{Section 39(1) was amended by section 62(5) of the Welfare Reform Act 2007 (c.~5).}.”
\end{enumerate}
\end{quotation}

(3) In regulation 26(1)($a$)\footnote{Regulation 26 was amended by S.I.~1983/186 and S.I.~1999/1958.} (obligations of claimants for, and~beneficiaries in respect of disablement benefit), for “medical practitioner” substitute “health care professional approved by the Secretary of State”.

\subsection[3. Amendment of the Social Security (Incapacity for Work) (General) Regulations 1995]{\sloppy\hbadness=1142 Amendment of the Social Security (Incapacity for Work) (General) Regulations 1995}

3.---(1)  The Social Security (Incapacity for Work) (General) Regulations 1995\footnote{S.I.~1995/311.} are amended as follows.

(2) In regulation 2(1)\footnote{Regulation 2 was amended by S.I.~1996/3207, S.I.~1999/2422 and S.I.~1999/3109.} (interpretation), after the definition of “doctor” insert—
\begin{quotation}
““health care professional” means—
\begin{enumerate}\item[]
($a$) 
a registered medical practitioner,

($b$) 
a registered nurse,

($c$) 
an occupational therapist or physiotherapist registered with a regulatory body established by an Order in Council under section 60 of the Health Care Act 1999, or

($d$) 
a member of such other profession, regulated by a body mentioned in section 25(3) of the National Health Service Reform and~Health Care Professions Act 2002, prescribed by the Secretary of State in accordance with powers under section 39(1) of the Social Security Act 1998.”
\end{enumerate}
\end{quotation}

(3) In regulation 8(1)\footnote{Regulation 8 was amended by S.I.~1999/3109.} (person may be called for medical examination), for “doctor” substitute “health care professional”.

(4) In regulation 27(2)($b$)\footnote{Regulation 27 was amended by S.I.~1996/3207, S.I.~1999/3109 and S.I.~2000/590.} (exceptional circumstances), for “doctor” substitute “health care professional”.

\subsection[4. Amendment of the Social Security and~Child Support (Decisions and~Appeals) Regulations 1999]{Amendment of the Social Security and~Child Support (Decisions and~Appeals) Regulations 1999}

4.---(1)  The Social Security and~Child Support (Decisions and~Appeals) Regulations 1999\footnote{S.I.~1999/991.} are amended as follows.

(2) In regulation 12\footnote{Regulation 12 was amended by S.I.~2003/916.} (decision of the Secretary of State relating to industrial injuries benefit)—
\begin{enumerate}\item[]
($a$) in paragraph~(2) for “medical practitioner” substitute “health care professional approved by the Secretary of State”;

($b$) in paragraph~(3)($b$) , for the first occurrence of “medical practitioner” substitute “health care professional” and~for the second occurrence of “medical practitioner” substitute “ health care professional approved by the Secretary of State”.
\end{enumerate}

(3) In regulation 19(1)\footnote{Regulation 19 was amended by S.I.~1999/2570 and S.I.~2003/916.} (suspension and~termination for failure to submit to medical examination), for “medical practitioner” substitute “health care professional approved by the Secretary of State”. 

\bigskip

Signed 
by authority of the 
Secretary of State for~Work and~Pensions.
%I concur
%By authority of the Lord Chancellor

{\raggedleft
\emph{James Plaskitt}\\*
%Secretary
%Minister
Parliamentary Under-Secretary 
of State,\\*Department 
for~Work and~Pensions

}

7th June 2007

\small

\part{Explanatory Note}

\renewcommand\parthead{— Explanatory Note}

\subsection*{(This note is not part of the Regulations)}

These Regulations amend the Social Security (Claims and~Payments) Regulations 1979 (the 1979 Regulations), the Social Security (Incapacity for Work) (General) Regulations 1995 (the 1995 Regulations) and~the Social Security and~Child Support (Decisions and~Appeals) Regulations 1999 (the 1999 Regulations) in consequence upon section 62 of the Welfare Reform Act 2007.

The latter provision amends sections~19 and~20 of the Social Security Act 1998 to allow the Secretary of State and~the eligible member of the appeal tribunal to refer a person to a health care professional for medical examination and~report, rather than only allowing a referral to a medical practitioner.

Regulation 2 amends the 1979 Regulations. Paragraph (2) inserts the definition of “health care professional” into the interpretation provisions of the 1979 Regulations. Paragraph (3) amends regulation 26 of the 1979 Regulations by substituting the words “health care professional approved by the Secretary of State” for “medical practitioner”.

Regulation 3 amends the 1995 Regulations. Paragraph (2) inserts the definition of “health care professional” into the interpretation provision of the 1995 Regulations. Paragraph (3) amends regulation 8 of the 1995 Regulations by substituting the words “health care professional” for “doctor”.

Regulation 4 amends the 1999 Regulations. Paragraphs (2) and~(3) amend regulations 12 and~19 of the 1999 Regulations by substituting the words “health care professional” and~“health care professional approved by the Secretary of State” for “medical practitioner”.

A full regulatory impact assessment has not been carried out in respect of these Regulations as they do not impose a cost on business, charities or the voluntary sector. 

\end{document}