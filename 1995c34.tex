\documentclass[a4paper]{article}

\usepackage[welsh,english]{babel}

\usepackage[utf8]{inputenc}
\usepackage[T1]{fontenc}

\usepackage{textcomp}

\usepackage[oldrules]{optional}

\usepackage{lmodern}
\usepackage[osf]{mathpazo}

\usepackage{perpage} %the perpage package
\MakePerPage{footnote} %the perpage package command
\renewcommand{\thefootnote}{\fnsymbol{footnote}}

\usepackage[perpage,para,symbol]{footmisc}

\opt{oldrules}{
\title{Child Support Act 1995\\(1993 scheme version)}
}

\opt{newrules}{
\title{Child Support Act 1995\\(2003 scheme version)}
}

\opt{2012rules}{
\title{Child Support Act 1995\\(2012 scheme version)}
}

\author{1995 Chapter 34}

\date{Royal Assent 19th July 1995}

\opt{oldrules}{\newcommand\versionyear{1993}}
\opt{newrules}{\newcommand\versionyear{2003}}
\opt{2012rules}{\newcommand\versionyear{2012}}

\usepackage{fancyhdr}
\pagestyle{fancy}
\fancyhead[L]{}
\fancyhead[C]{\itshape Child Support Act 1995 (c.~34) \parthead\phantom{...} (\versionyear{} scheme version)
}
\fancyhead[R]{}
\fancyfoot[C]{\thepage}
\newcommand{\parthead}{}

\usepackage{array}
\usepackage{multirow}
\usepackage[debugshow]{tabulary}
\usepackage{longtable}
\usepackage{multicol}
\usepackage{lettrine}

\usepackage[colorlinks=true]{hyperref}
\usepackage{microtype}

\hyphenation{Aw-dur-dod}
\hyphenation{bank-rupt-cy}
\hyphenation{Ec-cles-ton}
\hyphenation{Eux-ton}
\hyphenation{Hogh-ton}
\hyphenation{Pres-ton}
\hyphenation{Pru-den-tial}
\hyphenation{Riv-ing-ton}

\newcolumntype{x}[1]
	{>{\raggedright}p{#1}}
\newcommand{\tn}{\tabularnewline}
\setlength\tymin{50pt}

\newcommand\amendment[1]{\subsubsection*{Notes}{\itshape\frenchspacing\footnotesize #1 \par}}

\begin{document}

\maketitle

\noindent
{\large An Act to make provision with respect to child support maintenance and other maintenance; and to provide for a child maintenance bonus.}

\bigskip

\lettrine{B}{e it enacted} by the Queen’s most Excellent Majesty, by and with the advice and consent of the Lords Spiritual and Temporal, and Commons, in this present Parliament assembled, and by the authority of the same, as follows:—

{\sloppy

\tableofcontents

}

\setcounter{secnumdepth}{-2}

\section{Application for a departure direction}

\opt{oldrules}{

\subsection{1. Applications for departure directions}

(1) In the 1991 Act, insert after section 28—
\begin{quotation}
\section*{“Departure from usual rules for determining maintenance assessments}

\subsection*{28A. Application for a departure direction}

(1) Where a maintenance assessment (“the current assessment”) is in force—
\begin{enumerate}\item[]
($a$) the person with care, or absent parent, with respect to whom it was made, or

($b$) where the application for the current assessment was made under section 7, either of those persons or the child concerned,
\end{enumerate}
may apply to the Secretary of State for a direction under section 28F (a “departure direction”).

(2) An application for a departure direction shall state in writing the grounds on which it is made and shall, in particular, state whether it is based on—
\begin{enumerate}\item[]
($a$) the effect of the current assessment; or

($b$) a material change in the circumstances of the case since the current assessment was made.
\end{enumerate}

(3) In other respects, an application for a departure direction shall be made in such manner as may be prescribed.

(4) An application may be made under this section even though—
\begin{enumerate}\item[]
($a$) an application for a review has been made under section 17 or 18 with respect to the current assessment; or

($b$) a child support officer is conducting a review of the current assessment under section 16 or 19.
\end{enumerate}

(5) If the Secretary of State considers it appropriate to do so, he may by regulations provide for the question whether a change of circumstances is material to be determined in accordance with the regulations.

(6) Schedule 4A has effect in relation to departure directions.”
\end{quotation}

(2) Schedule 1 inserts in the 1991 Act a new Schedule 4A which makes supplemental provision with respect to procedural and other matters.

\subsection{2. Preliminary consideration}

In the 1991 Act, insert after section 28A—
\begin{quotation}
\subsection*{“28B. Preliminary consideration of applications}

(1) Where an application for a departure direction has been duly made to the Secretary of State, he may give the application a preliminary consideration.

(2) Where the Secretary of State does so he may, on completing the preliminary consideration, reject the application if it appears to him—
\begin{enumerate}\item[]
($a$) that there are no grounds on which a departure direction could be given in response to the application; or

($b$) that the difference between the current amount and the revised amount is less than an amount to be calculated in accordance with regulations made by the Secretary of State for the purposes of this subsection and section 28F(4).
\end{enumerate}

(3) In subsection (2)—
\begin{enumerate}\item[]
    “the current amount” means the amount of the child support maintenance fixed by the current assessment; and

    “the revised amount” means the amount of child support maintenance which, but for subsection (2)($b$), would be fixed if a fresh maintenance assessment were to be made as a result of a departure direction allowing the departure applied for. 
\end{enumerate}

(4) Before completing any preliminary consideration, the Secretary of State may refer the current assessment to a child support officer for it to be reviewed as if an application for a review had been made under section 17 or 18.

(5) A review initiated by a reference under subsection (4) shall be conducted as if subsection (4) of section 17, or (as the case may be) subsection (8) of section 18, were omitted.''
%
%(6) Where, as a result of a review of the current assessment under section 16, 17, 18 or 19 (including a review initiated by a reference under subsection (4)), a fresh maintenance assessment is made, the Secretary of State—
%\begin{enumerate}\item[]
%($a$) shall notify the applicant and such other persons as may be prescribed that the fresh maintenance assessment has been made; and
%
%($b$) may direct that the application is to lapse unless, before the end of such period as may be prescribed, the applicant notifies the Secretary of State that he wishes it to stand.”
%\end{enumerate}
\end{quotation}

\amendment{
S. 2 is only in force for the purposes of inserting s. 28B(1)--(5) into the Child Support Act 1991.
}

\subsection{3. Imposition of a regular payments condition}

In the 1991 Act, insert after section 28B—
\begin{quotation}
\subsection*{“28C. Imposition of a regular payments condition}

(1) Where an application for a departure direction is made by an absent parent, the Secretary of State may impose on him one of the conditions mentioned in subsection (2) (“a regular payments condition”).

(2) The conditions are that—
\begin{enumerate}\item[]
($a$) the applicant must make the payments of child support maintenance fixed by the current assessment;

($b$) the applicant must make such reduced payments of child support maintenance as may be determined in accordance with regulations made by the Secretary of State.
\end{enumerate}

(3) Where the Secretary of State imposes a regular payments condition, he shall give written notice to the absent parent and person with care concerned of the imposition of the condition and of the effect of failure to comply with it.

(4) A regular payments condition shall cease to have effect on the failure or determination of the application.

(5) For the purposes of subsection (4), an application for a departure direction fails if—
\begin{enumerate}\item[]
($a$) it lapses or is withdrawn; or

($b$) the Secretary of State rejects it on completing a preliminary consideration under section 28B.
\end{enumerate}

(6) Where an absent parent has failed to comply with a regular payments condition—
\begin{enumerate}\item[]
($a$) the Secretary of State may refuse to consider the application; and

($b$) in prescribed circumstances the application shall lapse.
\end{enumerate}

(7) The question whether an absent parent has failed to comply with a regular payments condition shall be determined by the Secretary of State.

(8) Where the Secretary of State determines that an absent parent has failed to comply with a regular payments condition he shall give that parent, and the person with care concerned, written notice of his decision.”
\end{quotation}

}

\opt{newrules, 2012rules}{
\amendment{
Ss. 1--3 repealed (3.3.03) for the purpose of certain cases only (see S.I. 2003/192) by the Child Support, Pensions and Social Security Act 2000 (c. 19) Sch. 9 Pt. I.

}

}

\subsection{4. Determination of applications}

In the 1991 Act, insert after section 28C—
\begin{quotation}
\subsection*{“28D. Determination of applications}

(1) Where an application for a departure direction has not failed, the Secretary of State shall—
\begin{enumerate}\item[]
($a$) determine the application in accordance with the relevant provisions of, or made under, this Act; or

($b$) refer the application to a child support appeal tribunal for the tribunal to determine it in accordance with those provisions.
\end{enumerate}

(2) For the purposes of subsection (1), an application for a departure direction has failed if—
\begin{enumerate}\item[]
($a$) it has lapsed or been withdrawn; or

($b$) the Secretary of State has rejected it on completing a preliminary consideration under section 28B.
\end{enumerate}

(3) In dealing with an application for a departure direction which has been referred to it under subsection (1)($b$), a child support appeal tribunal shall have the same powers, and be subject to the same duties, as would the Secretary of State if he were dealing with the application.” 
\end{quotation}

\subsection{5. Matters to be taken into account}

In the 1991 Act, insert after section 28D—
\begin{quotation}
\subsection*{“28E. Matters to be taken into account}

(1) In determining any application for a departure direction, the Secretary of State shall have regard both to the general principles set out in subsection (2) and to such other considerations as may be prescribed.

(2) The general principles are that—
\begin{enumerate}\item[]
($a$) parents should be responsible for maintaining their children whenever they can afford to do so;

($b$) where a parent has more than one child, his obligation to maintain any one of them should be no less of an obligation than his obligation to maintain any other of them.
\end{enumerate}

(3) In determining any application for a departure direction, the Secretary of State shall take into account any representations made to him—
\begin{enumerate}\item[]
($a$) by the person with care or absent parent concerned; or

($b$) where the application for the current assessment was made under section 7, by either of them or the child concerned.
\end{enumerate}

(4) In determining any application for a departure direction, no account shall be taken of the fact that—
\begin{enumerate}\item[]
($a$) any part of the income of the person with care concerned is, or would be if a departure direction were made, derived from any benefit; or

($b$) some or all of any child support maintenance might be taken into account in any manner in relation to any entitlement to benefit.
\end{enumerate}

(5) In this section “benefit” has such meaning as may be prescribed.” 
\end{quotation}

\opt{oldrules}{

\subsection{6. Departure directions}

(1) In the 1991 Act, insert after section 28E—
\begin{quotation}
\subsection*{“28F. Departure directions}

(1) The Secretary of State may give a departure direction if—
\begin{enumerate}\item[]
($a$) he is satisfied that the case is one which falls within one or more of the cases set out in Part I of Schedule 4B or in regulations made under that Part; and

($b$) it is his opinion that, in all the circumstances of the case, it would be just and equitable to give a departure direction.
\end{enumerate}

(2) In considering whether it would be just and equitable in any case to give a departure direction, the Secretary of State shall have regard, in particular, to—
\begin{enumerate}\item[]
($a$) the financial circumstances of the absent parent concerned,

($b$) the financial circumstances of the person with care concerned, and

($c$) the welfare of any child likely to be affected by the direction.
\end{enumerate}

(3) The Secretary of State may by regulations make provision—
\begin{enumerate}\item[]
($a$) for factors which are to be taken into account in determining whether it would be just and equitable to give a departure direction in any case;

($b$) for factors which are not to be taken into account in determining such a question.
\end{enumerate}

(4) The Secretary of State shall not give a departure direction if he is satisfied that the difference between the current amount and the revised amount is less than an amount to be calculated in accordance with regulations made by the Secretary of State for the purposes of this subsection and section 28B(2).

(5) In subsection (4)—
\begin{enumerate}\item[]
    “the current amount” means the amount of the child support maintenance fixed by the current assessment, and

    “the revised amount” means the amount of child support maintenance which would be fixed if a fresh maintenance assessment were to be made as a result of the departure direction which the Secretary of State would give in response to the application but for subsection (4). 
\end{enumerate}

(6) A departure direction shall—
\begin{enumerate}\item[]
($a$) require a child support officer to make one or more fresh maintenance assessments; and

($b$) specify the basis on which the amount of child support maintenance is to be fixed by any assessment made in consequence of the direction.
\end{enumerate}

(7) In giving a departure direction, the Secretary of State shall comply with the provisions of regulations made under Part II of Schedule 4B.

(8) Before the end of such period as may be prescribed, the Secretary of State shall notify the applicant for a departure direction, and such other persons as may be prescribed—
\begin{enumerate}\item[]
($a$) of his decision in relation to the application, and

($b$) of the reasons for his decision.”
\end{enumerate}
\end{quotation}

(2) Schedule 2 inserts in the 1991 Act the new Schedule 4B which is referred to in subsections (1)($a$) and (7) of the new section 28F inserted by this section.

\subsection{7. Effect and duration}

In the 1991 Act, insert after section 28F—
\begin{quotation}
\subsection*{“28G. Effect and duration of departure directions}

(1) Where a departure direction is given, it shall be the duty of the child support officer to whom the case is referred to comply with the direction as soon as is reasonably practicable.

(2) A departure direction may be given so as to have effect—
\begin{enumerate}\item[]
($a$) for a specified period; or

($b$) until the occurrence of a specified event.
\end{enumerate}

(3) The Secretary of State may by regulations make provision for the cancellation of a departure direction in prescribed circumstances.

(4) The Secretary of State may by regulations make provision as to when a departure direction is to take effect.

(5) Regulations under subsection (4) may provide for a departure direction to have effect from a date earlier than that on which the direction is given.”
\end{quotation}

\subsection{8. Appeals}

In the 1991 Act, insert after section 28G—
\begin{quotation}
\subsection*{“28H. Appeals in relation to applications for departure directions}

(1) Any qualifying person who is aggrieved by any decision of the Secretary of State on an application for a departure direction may appeal to a child support appeal tribunal against that decision.

(2) In subsection (1), “qualifying person” means—
\begin{enumerate}\item[]
($a$) the person with care, or absent parent, with respect to whom the current assessment was made, or

($b$) where the application for the current assessment was made under section 7, either of those persons or the child concerned.
\end{enumerate}

(3) Except with leave of the chairman of a child support appeal tribunal, no appeal under this section shall be brought after the end of the period of 28 days beginning with the date on which notification was given of the decision in question.

(4) On an appeal under this section, the tribunal shall—
\begin{enumerate}\item[]
($a$) consider the matter—
\begin{enumerate}\item[]
(i) as if it were exercising the powers of the Secretary of State in relation to the application in question; and

(ii) as if it were subject to the duties imposed on him in relation to that application;
\end{enumerate}

($b$) have regard to any representations made to it by the Secretary of State; and

($c$) confirm the decision or replace it with such decision as the tribunal considers appropriate.”
\end{enumerate}
\end{quotation}

\subsection{9. Transitional provisions}

In the 1991 Act, insert after section 28H—
\begin{quotation}
\subsection*{“28I. Transitional provisions}
%
%(1) In the case of an application for a departure direction relating to a maintenance assessment which was made before the coming into force of section 28A, the period within which the application must be made shall be such period as may be prescribed.
%
%(2) The Secretary of State may by regulations make provision for applications for departure directions to be dealt with according to an order determined in accordance with the regulations.
%
%(3) The regulations may, for example, provide for—
%\begin{enumerate}\item[]
%($a$) applications relating to prescribed descriptions of maintenance assessment, or
%
%($b$) prescribed descriptions of application,
%\end{enumerate}
%to be dealt with before applications relating to other prescribed descriptions of assessment or (as the case may be) other prescribed descriptions of application.

(4) The Secretary of State may by regulations make provision—
\begin{enumerate}\item[]
($a$) enabling applications for departure directions made before the coming into force of section 28A to be considered even though that section is not in force;

($b$) for the determination of any such application as if section 28A and the other provisions of this Act relating to departure directions were in force; and

($c$) as to the effect of any departure direction given before the coming into force of section 28A.
\end{enumerate}

(5) Regulations under section 28G(4) may not provide for a departure direction to have effect from a date earlier than that on which that section came into force.”
\end{quotation}

\amendment{

S. 9 is only in force in respect of the insertion of section 28I(4), (5) into the Child Support Act 1991.

\medskip

S. 10 repealed (27.10.08) by the Child Support, Pensions and Social Security Act 2000 s. 23.
}

\section{Reviews of maintenance assessments etc.}

\subsection{11. Reviews: interim maintenance assessments}

In section 12 of the 1991 Act (interim maintenance assessments), for subsection (1) substitute—
\begin{quotation}
“(1) This section applies where a child support officer—
\begin{enumerate}\item[]
($a$) is required to make a maintenance assessment;

($b$) is proposing to conduct a review under section 16, 17, 18 or 19; or

($c$) is conducting such a review.
\end{enumerate}

(1A) If it appears to the child support officer that he does not have sufficient information to enable him—
\begin{enumerate}\item[]
($a$) in a case falling within subsection (1)($a$), to make the assessment,

($b$) in a case falling within subsection (1)($b$), to conduct the proposed review, or

($c$) in a case falling within subsection (1)($c$), to complete the review,
\end{enumerate}
he may make an interim maintenance assessment.”
\end{quotation}

\amendment{
Ss. 12, 13 repealed (1.6.99) by the Social Security Act 1998 (c. 14) Sch. 8.
}

\subsection{14. Cancellation of maintenance assessments on review}

(2) In paragraph 16 of Schedule 1 to the 1991 Act (termination of maintenance assessments), insert after sub-paragraph (4)—
\begin{quotation}
“(4A) A maintenance assessment may be cancelled by a child support officer if he is conducting a review under section 16, 17, 18 or 19 and it appears to him—
\begin{enumerate}\item[]
($a$) that the person with care with respect to whom the maintenance assessment in question was made has failed to provide him with sufficient information to enable him to complete the review; and

($b$) where the maintenance assessment in question was made in response to an application under section 6, that the person with care with respect to whom the assessment was made has ceased to fall within subsection (1) of that section.”
\end{enumerate}
\end{quotation}

(3) In sub-paragraph (7) of paragraph 16 of Schedule 1 to the 1991 Act, after “sub-paragraph” insert “(4A),”.

\amendment{
S. 14(1) repealed (1.6.99) by the Social Security Act 1998 (c. 14) Sch. 8.

\medskip

Ss. 15, 16 repealed (1.6.99) by the Social Security Act 1998 (c. 14) Sch. 8.

\medskip

S. 17 repealed (3.11.08) by the Transfer of Tribunal Functions Order 2008 Sch. 3 para. 228($c$).

}
}

\opt{newrules, 2012rules}{
\amendment{
Ss. 6--11 repealed (3.3.03) for the purposes of certain cases only (see S.I. 2003/192) by the Child Support, Pensions and Social Security Act 2000 (c. 19) Sch. 9 Pt. I.

\medskip

Ss. 12, 13 repealed (1.6.99) by the Social Security Act 1998 (c. 14) Sch. 8.

\medskip

S. 14(1) repealed (1.6.99) by the Social Security Act 1998 (c. 14) Sch. 8.

S. 14(2), (3) repealed (3.3.03) for the purpose of certain cases only (see S.I. 2003/192) by the Child Support, Pensions and Social Security Act 2000 (c. 19) Sch. 9 Pt. I.

\medskip

Ss. 15, 16 repealed (1.6.99) by the Social Security Act 1998 (c. 14) Sch. 8.

\medskip

S. 17 repealed (3.11.08) by the Transfer of Tribunal Functions Order 2008 Sch. 3 para. 228($c$).

}
}

\section{Miscellaneous}

\subsection{18. Deferral of right to apply for maintenance assessment}

(1) In section 4 of the 1991 Act (right of person with care or absent parent to apply for maintenance assessment), insert at the end—
\begin{quotation}
“(10) No application may be made at any time under this section with respect to a qualifying child or any qualifying children if—
\begin{enumerate}\item[]
($a$) there is in force a written maintenance agreement made before 5th April 1993, or a maintenance order, in respect of that child or those children and the person who is, at that time, the absent parent; or

($b$) benefit is being paid to, or in respect of, a parent with care of that child or 
those children.
\end{enumerate}

(11) In subsection (10) “benefit” means any benefit which is mentioned in, or prescribed by regulations under, section 6(1).”
\end{quotation}

(2) In section 7 of the 1991 Act (right of child in Scotland to apply for maintenance assessment), insert at the end—
\begin{quotation}
“(10) No application may be made at any time under this section by a qualifying child if there is in force a written maintenance agreement made before 5th April 1993, or a maintenance order, in respect of that child and the person who is, at that time, the absent parent.”
\end{quotation}

\opt{oldrules}{
(3) In section 8 of the 1991 Act (role of the courts with respect to maintenance for children), after subsection (3) insert—
\begin{quotation}
“(3A) In any case in which section 4(10) or 7(10) prevents the making of an application for a maintenance assessment, and—
\begin{enumerate}\item[]
($a$) no application has been made for a maintenance assessment under section 6, or

($b$) such an application has been made but no maintenance assessment has been made in response to it,
\end{enumerate}
\begin{sloppypar}
\noindent
subsection (3) shall have effect with the omission of the word “vary”.”
\end{sloppypar}
\end{quotation}
}

(4) In section 9 of the 1991 Act (maintenance agreements), at the beginning of subsection (3) insert “Subject to section 4(10)($a$) and section 7(10),” and after subsection (5) insert—
\begin{quotation}
“(6) In any case in which section 4(10) or 7(10) prevents the making of an application for a maintenance assessment, and—
\begin{enumerate}\item[]
($a$) no application has been made for a maintenance assessment under section 6, or

($b$) such an application has been made but no maintenance assessment has been made in response to it,
\end{enumerate}
subsection (5) shall have effect with the omission of paragraph ($b$).”
\end{quotation}

\opt{oldrules}{(5) The Secretary of State may by order repeal any of the provisions of this section.}

(6) Neither section 4(10) nor section 7(10) of the 1991 Act shall apply in relation to a maintenance order made in the circumstances mentioned in subsection (7) or (8) of section 8 of the 1991 Act.

(7) The Secretary of State may by regulations make provision for section 4(10), or section 7(10), of the 1991 Act not to apply in relation to such other cases as may be prescribed.

(8) Part I of the Schedule to the Child Support Act 1991 (Commencement No.\ 3 and Transitional Provisions) Order 1992 (phased take-on of certain cases) is hereby revoked.

(9) At any time before 7th April 1997, neither section 8(3), nor section 9(5)($b$), of the 1991 Act shall apply in relation to any case which fell within paragraph 5(2) of the Schedule to the 1992 order (pending cases during the transitional period set by that order).

\opt{newrules,2012rules}{
\amendment{
S. 18(3), (5) repealed (3.3.03) for the purposes of new-rules and 2012 scheme cases only (see SI 2003/192) by the Child Support, Pensions and Social Security Act 2000 (c. 19) Sch. 3 para. 13(2) and Sch. 9 Pt. I.

\medskip

S. 19 repealed (3.3.03) for the purposes of new-rules and 2012 scheme cases only (see SI 2003/192) by the Child Support, Pensions and Social Security Act 2000 (c. 19) Sch. 3 para. 13(2) and Sch. 9 Pt. I.
}
}

\opt{oldrules}{
\subsection{19. Non-referral of applications for maintenance assessments}

In section 11 of the 1991 Act, after subsection (1) (referral of application for maintenance assessment to child support officer) insert—
\begin{quotation}
“(1A) Where—
\begin{enumerate}\item[]
($a$) an application for a maintenance assessment is made under section 6, but

($b$) the Secretary of State becomes aware, before referring the application to a child support officer, that the claim mentioned in subsection (1) of that section has been disallowed or withdrawn,
\end{enumerate}
he shall, subject to subsection (1B), treat the application as if it had not been made.

(1B) If it appears to the Secretary of State that subsection (10) of section 4 would not have prevented the parent with care concerned from making an application for a maintenance assessment under that section he shall—
\begin{enumerate}\item[]
($a$) notify her of the effect of this subsection, and

($b$) if, before the end of the period of 28 days beginning with the day on which notice was sent to her, she asks him to do so, treat the application as having been made not under section 6 but under section 4.
\end{enumerate}

(1C) Where the application is not preserved under subsection (1B) (and so is treated as not having been made) the Secretary of State shall notify—
\begin{enumerate}\item[]
($a$) the parent with care concerned; and

($b$) the absent parent (or alleged absent parent), where it appears to him that that person is aware of the application.”
\end{enumerate}
\end{quotation}
}

\subsection{20. Disputed parentage}

(5) Section 28 of the 1991 Act (power of Secretary of State to initiate or defend actions of declarator) is amended as set out in subsections (6) and (7).

(6) For subsection (1) substitute—
\begin{quotation}
“(1) Subsection (1A) applies in any case where—
\begin{enumerate}\item[]
($a$) an application for a maintenance assessment has been made, or a maintenance assessment is in force, with respect to a person (“the alleged parent”) who denies that he is a parent of a child with respect to whom the application or assessment was made; and

($b$) a child support officer to whom the case is referred is not satisfied that the case falls within one of those set out in section 26(2).
\end{enumerate}

(1A) In any case where this subsection applies, the Secretary of State may bring an action for declarator of parentage under section 7 of the Law Reform (Parent and Child) (Scotland) Act 1986.”
\end{quotation}

(7) In subsection (2), at the end insert “or in a maintenance assessment which is in force”.

\amendment{
S. 20(1)--(4) repealed (1.4.01) by the Child Support, Pensions and Social Security Act 2000 (c. 19) Sch. 9 Pt. IX.
}

\subsection{21. Fees for scientific tests}

After section 27 of the 1991 Act insert—
\begin{quotation}
\subsection*{“27A. Recovery of fees for scientific tests}

(1) This section applies in any case where—
\begin{enumerate}\item[]
($a$) an application for a maintenance assessment has been made or a maintenance assessment is in force;

($b$) scientific tests have been carried out (otherwise than under a direction or in response to a request) in relation to bodily samples obtained from a person who is alleged to be a parent of a child with respect to whom the application or assessment is made;

($c$) the results of the tests do not exclude the alleged parent from being one of the child’s parents; and

($d$) one of the conditions set out in subsection (2) is satisfied.
\end{enumerate}

(2) The conditions are that—
\begin{enumerate}\item[]
($a$) the alleged parent does not deny that he is one of the child’s parents;

($b$) in proceedings under section 27, a court has made a declaration that the alleged parent is a parent of the child in question; or

($c$) in an action under section 7 of the Law Reform (Parent and Child) (Scotland) Act 1986, brought by the Secretary of State by virtue of section 28, a court has granted a decree of declarator of parentage to the effect that the alleged parent is a parent of the child in question.
\end{enumerate}

(3) In any case to which this section applies, any fee paid by the Secretary of State in connection with scientific tests may be recovered by him from the alleged parent as a debt due to the Crown.

(4) In this section—
\begin{enumerate}\item[]
“bodily sample” means a sample of bodily fluid or bodily tissue taken for the purpose of scientific tests;

“direction” means a direction given by a court under section 20 of the Family Law Reform Act 1969 (tests to determine paternity);

“request” means a request made by a court under section 70 of the Law Reform (Miscellaneous Provisions) (Scotland) Act 1990 (blood and other samples in civil proceedings); and

“scientific tests” means scientific tests made with the object of ascertaining the inheritable characteristics of bodily fluids or bodily tissue.
\end{enumerate}

(5) Any sum recovered by the Secretary of State under this section shall be paid by him into the Consolidated Fund.”
\end{quotation}

\amendment{
S. 22 is not yet in force.
}


\subsection{23. Repayment of overpaid child support maintenance}

In the 1991 Act, insert after section 41A—
\begin{quotation}
\subsection*{\sloppy “41B. Repayment of overpaid child support maintenance}

(1) This section applies where it appears to the Secretary of State that an absent parent has made a payment by way of child support maintenance which amounts to an overpayment by him of that maintenance and that—
\begin{enumerate}\item[]
($a$) it would not be possible for the absent parent to recover the amount of the overpayment by way of an adjustment of the amount payable under a maintenance assessment; or

($b$) it would be inappropriate to rely on an adjustment of the amount payable under a maintenance assessment as the means of enabling the absent parent to recover the amount of the overpayment.
\end{enumerate}

(2) The Secretary of State may make such payment to the absent parent by way of reimbursement, or partial reimbursement, of the overpayment as the Secretary of State considers appropriate.

(3) Where the Secretary of State has made a payment under this section he may, in such circumstances as may be prescribed, require the relevant person to pay to him the whole, or a specified proportion, of the amount of that payment.

(4) Any such requirement shall be imposed by giving the relevant person a written demand for the amount which the Secretary of State wishes to recover from him.

(5) Any sum which a person is required to pay to the Secretary of State under this section shall be recoverable from him by the Secretary of State as a debt due to the Crown.

(6) The Secretary of State may by regulations make provision in relation to any case in which—
\begin{enumerate}\item[]
($a$) one or more overpayments of child support maintenance are being reimbursed to the Secretary of State by the relevant person; and

($b$) child support maintenance has continued to be payable by the absent parent concerned to the person with care concerned, or again becomes so payable.
\end{enumerate}

(7) For the purposes of this section any payments made by a person under a maintenance assessment which was not validly made shall be treated as overpayments of child support maintenance made by an absent parent.%

(8) In this section “relevant person”, in relation to an overpayment, means the person with care to whom the overpayment was made.

(9) Any sum recovered by the Secretary of State under this section shall be paid by him into the Consolidated Fund.%
”
\end{quotation}

\amendment{
S. 24 repealed (2.4.01) by the Child Support, Pensions and Social Security Act 2000 (c. 19) Sch. 9 Pt. I.
}

\subsection{25. Payment of benefit where maintenance payments collected by Secretary of State}

In the Social Security Administration Act 1992, insert after section 74—
\begin{quotation}
\subsection*{\sloppy \textls[25]{“74A. Payment of benefit where maintenance pay\-}ments collected by Secretary of State}

(1) This section applies where—
\begin{enumerate}\item[]
($a$) a person (“the claimant”) is entitled to a benefit to which this section applies;

($b$) the Secretary of State is collecting periodical payments of child or spousal maintenance made in respect of the claimant or a member of the claimant’s family; and

\begin{sloppypar}
($c$) the inclusion of any such periodical payment in the claimant’s relevant income would, apart from this section, have the effect of reducing the amount of the benefit to which the claimant is entitled.
\end{sloppypar}
\end{enumerate}

(2) The Secretary of State may, to such extent as he considers appropriate, treat any such periodical payment as not being relevant income for the purposes of calculating the amount of benefit to which the claimant is entitled.

(3) The Secretary of State may, to the extent that any periodical payment collected by him is treated as not being relevant income for those purposes, retain the whole or any part of that payment.

(4) Any sum retained by the Secretary of State under subsection (3) shall be paid by him into the Consolidated Fund.

(5) In this section—
\begin{enumerate}\item[]
“child” means a person under the age of 16;

“child maintenance”, “spousal maintenance” and “relevant income” have such meaning as may be prescribed;

“family” means—
\begin{enumerate}\item[]
($a$) a married or unmarried couple;

($b$) a married or unmarried couple and a member of the same household for whom one of them is, or both are, responsible and who is a child or a person of a prescribed description;

($c$) except in prescribed circumstances, a person who is not a member of a married or unmarried couple and a member of the same household for whom that person is responsible and who is a child or a person of a prescribed description;
\end{enumerate}

“married couple” means a man and woman who are married to each other and are members of the same household; and

“unmarried couple” means a man and woman who are not married to each other but are living together as husband and wife otherwise than in prescribed circumstances.
\end{enumerate}

(6) For the purposes of this section, the Secretary of State may by regulations make provision as to the circumstances in which—
\begin{enumerate}\item[]
($a$) persons are to be treated as being or not being members of the same household;

($b$) one person is to be treated as responsible or not responsible for another.
\end{enumerate}

(7) The benefits to which this section applies are income support, an income-based jobseeker’s allowance and such other benefits (if any) as may be prescribed.”
\end{quotation}

\section{Supplemental}

\subsection{26. Regulations and orders}

(1) Any power under this Act to make regulations or orders shall be exercisable by statutory instrument.

(2) Any such power may be exercised to make different provision for different cases, including different provision for different areas.

(3) Any such power includes power—
\begin{enumerate}\item[]
($a$) to make such incidental, supplemental, consequential or transitional provision as appears to the Secretary of State to be expedient; and

($b$) to provide for a person to exercise a discretion in dealing with any matter.
\end{enumerate}

(4) Subsection (5) applies to—
\begin{enumerate}\item[]
($a$) the first regulations made under section 10;

($b$) any order made under section 18(5)%;
%
\opt{oldrules}{;

($c$) the first regulations made under section 24}%
.
\end{enumerate}

(5) No regulations or order to which this subsection applies shall be made unless a draft of the statutory instrument containing the regulations or order has been laid before Parliament and approved by a resolution of each House.

(6) Any other statutory instrument made under this Act, other than one made under section 30(4), shall be subject to annulment in pursuance of a resolution of either House of Parliament.

\opt{newrules,2012rules}{
\amendment{
S. 26(4)($c$) repealed (3.3.03) for 2003 scheme and 2012 scheme cases only (see SI 2003/192) by the Child Support, Pensions and Social Security Act 2000 (c. 19) Sch. 9 Pt. I.
}

}

\subsection{27. Interpretation}

(1) In this Act “the 1991 Act” means the Child Support Act 1991.

(2) Expressions in this Act which are used in the 1991 Act have the same meaning in this Act as they have in that Act.

\subsection{28. Financial provisions}

There shall be paid out of money provided by Parliament—
\begin{enumerate}\item[]
($a$) any expenditure incurred by the Secretary of State under or by virtue of this Act;

($b$) any increase attributable to this Act in the sums payable out of money so provided under or by virtue of any other enactment.
\end{enumerate}

\subsection{29. Provision for Northern Ireland}

(1) An Order in Council under paragraph 1(1)($b$) of Schedule 1 to the Northern Ireland Act 1974 (legislation for Northern Ireland in the interim period) which states that it is made only for purposes corresponding to those of this Act—
\begin{enumerate}\item[]
($a$) shall not be subject to paragraph 1(4) and (5) of that Schedule (affirmative resolution of both Houses of Parliament); but

($b$) shall be subject to annulment in pursuance of a resolution of either House of Parliament.
\end{enumerate}

\amendment{
S. 29(2)--(4) repealed (2.12.99) by the Northern Ireland Act 1998 (c. 47) Sch. 15.
}

\subsection{30. Short title, commencement, extent etc.}

(1) This Act may be cited as the Child Support Act 1995.

(2) This Act and the 1991 Act may be cited together as the Child Support Acts 1991 and 1995.

(3) Section 29 and this section (apart from subsection (5)) come into force on the passing of this Act.

(4) The other provisions of this Act come into force on such day as the Secretary of State may by order appoint and different days may be appointed for different purposes.

(5) Schedule 3 makes minor and consequential amendments.

(6) This Act, except for—
\begin{enumerate}\item[]
($a$) sections 17, 27 and 29,

($b$) this section, and

($c$) paragraphs 1, 18, 19 and 20 of Schedule 3,
\end{enumerate}
does not extend to Northern Ireland.

\clearpage

\part*{S C H E D U L E S}

\opt{oldrules}{
\part[Schedule 1 --- Departure directions]{Schedule 1\\*Departure directions}

\renewcommand\parthead{--- Schedule 1}

The following Schedule is inserted in the 1991 Act, after Schedule 4—
\begin{quotation}
\part*{``Schedule 4A\\*Departure directions}

\subsection*{Interpretation}

1. In this Schedule—
\begin{enumerate}\item[]
    “departure application” means an application for a departure direction;

    “regulations” means regulations made by the Secretary of State;

    “review” means a review under section 16, 17, 18 or 19. 
\end{enumerate}

\subsection*{Applications for departure directions}

2. Regulations may make provision—
\begin{enumerate}\item[]
($a$) as to the procedure to be followed in considering a departure application;

($b$) as to the procedure to be followed when a departure application is referred to a child support appeal tribunal under section 28D(1)($b$);

($c$) for the giving of a direction by the Secretary of State as to the order in which, in a particular case, a departure application and a review are to be dealt with;

($d$) for the reconsideration of a departure application in a case where further information becomes available to the Secretary of State after the application has been determined.
\end{enumerate}

\subsection*{Completion of preliminary consideration}

3. Regulations may provide for determining when the preliminary consideration of a departure application is to be taken to have been completed.

\subsection*{Information}

4.---(1) Regulations may make provision for the use for any purpose of this Act of—
\begin{enumerate}\item[]
($a$) information acquired by the Secretary of State in connection with an application for, or the making of, a departure direction;

($b$) information acquired by a child support officer or the Secretary of State in connection with an application for, or the making of, a maintenance assessment.
\end{enumerate}

(2) If any information which is required (by regulations under this Act) to be furnished to the Secretary of State in connection with a departure application has not been furnished within such period as may be prescribed, the Secretary of State may nevertheless proceed to determine the application.

\subsection*{Anticipation of change of circumstances}

5.---(1) A departure direction may be given so as to provide that if the circumstances of the case change in such manner as may be specified in the direction a fresh maintenance assessment is to be made.

(2) Where any such provision is made, the departure direction may provide for the basis on which the amount of child support maintenance is to be fixed by the fresh maintenance assessment to differ from the basis on which the amount of child support maintenance was fixed by any earlier maintenance assessment made as a result of the direction.

\subsection*{Reviews and departure directions}

6. Regulations may make provision—
\begin{enumerate}\item[]
($a$) with respect to cases in which a child support officer is conducting a review of a maintenance assessment which was made as a result of a departure direction;

($b$) with respect to cases in which a departure direction is made at a time when a child support officer is conducting a review.
\end{enumerate}

\subsection*{Subsequent departure directions}

7.---(1) Regulations may make provision with respect to any departure application made with respect to a maintenance assessment which was made as a result of a departure direction.

(2) The regulations may, in particular, provide for the application to be considered by reference to the maintenance assessment which would have been made had the departure direction not been given.

\subsection*{Joint consideration of departure applications and appeals}

8.---(1) Regulations may provide for two or more departure applications with respect to the same current assessment to be considered together.

(2) A child support appeal tribunal considering—
\begin{enumerate}\item[]
($a$) a departure application referred to it under section 28D(1)($b$), or

($b$) an appeal under section 28H,
\end{enumerate}
may consider it at the same time as hearing an appeal under section 20 in respect of the current assessment, if it considers that to be appropriate.

\subsection*{Child support appeal tribunals}

9.---(1) Regulations may provide that, in prescribed circumstances, where—
\begin{enumerate}\item[]
($a$) a departure application is referred to a child support appeal tribunal under section 28D(1)($b$), or

($b$) an appeal is brought under section 28H,
\end{enumerate}
the application or appeal may be dealt with by a tribunal constituted by the chairman sitting alone.

(2) Sub-paragraph (1) does not apply in relation to any appeal which is being heard together with an appeal under section 20.

\subsection*{Current assessments which are replaced by fresh assessments}

10. Regulations may make provision as to the circumstances in which prescribed references in this Act to a current assessment are to have effect as if they were references to any later maintenance assessment made with respect to the same persons as the current assessment.”
\end{quotation}

\part[Schedule 2 --- Departure directions: the cases and controls]{Schedule 2\\*Departure directions: the cases and controls}

\renewcommand\parthead{--- Schedule 2}

The following Schedule is inserted in the 1991 Act, after Schedule 4A—
\begin{quotation}
\part*{“Schedule 4B\\*Departure directions: the cases and controls}

\section*{Part I\\*The cases}

\subsection*{General}

1.---(1) The cases in which a departure direction may be given are those set out in this Part of this Schedule or in regulations made under this Part.

(2) In this Schedule “applicant” means the person whose application for a departure direction is being considered.

\subsection*{Special expenses}

2.---(1) A departure direction may be given with respect to special expenses of the applicant which were not, and could not have been, taken into account in determining the current assessment in accordance with the provisions of, or made under, Part I of Schedule 1.

(2) In this paragraph “special expenses” means the whole, or any prescribed part, of expenses which fall within a prescribed description of expenses.

(3) In prescribing descriptions of expenses for the purposes of this paragraph, the Secretary of State may, in particular, make provision with respect to—
\begin{enumerate}\item[]
($a$) costs incurred in travelling to work;

($b$) costs incurred by an absent parent in maintaining contact with the child, or with any of the children, with respect to whom he is liable to pay child support maintenance under the current assessment;

($c$) costs attributable to a long-term illness or disability of the applicant or of a dependant of the applicant;

($d$) debts incurred, before the absent parent became an absent parent in relation to a child with respect to whom the current assessment was made—
\begin{enumerate}\item[]
(i) for the joint benefit of both parents;

(ii) for the benefit of any child with respect to whom the current assessment was made; or

(iii) for the benefit of any other child falling within a prescribed category;
\end{enumerate}

($e$) pre-1993 financial commitments from which it is impossible for the parent concerned to withdraw or from which it would be unreasonable to expect that parent to have to withdraw;

($f$) costs incurred by a parent in supporting a child who is not his child but who is part of his family.
\end{enumerate}

(4) For the purposes of sub-paragraph (3)($c$)—
\begin{enumerate}\item[]
($a$) the question whether one person is a dependant of another shall be determined in accordance with regulations made by the Secretary of State;

($b$) “disability” and “illness” have such meaning as may be prescribed; and

($c$) the question whether an illness or disability is long-term shall be determined in accordance with regulations made by the Secretary of State.
\end{enumerate}

(5) For the purposes of sub-paragraph (3)($e$), “pre-1993 financial commitments” means financial commitments of a prescribed kind entered into before 5th April 1993 in any case where—
\begin{enumerate}\item[]
($a$) a court order of a prescribed kind was in force with respect to the absent parent and the person with care concerned at the time when they were entered into; or

($b$) an agreement between them of a prescribed kind was in force at that time.
\end{enumerate}

(6) For the purposes of sub-paragraph (3)($f$), a child who is not the child of a particular person is a part of that person’s family in such circumstances as may be prescribed.

\subsection*{Property or capital transfers}

3.---(1) A departure direction may be given if—
\begin{enumerate}\item[]
($a$) before 5th April 1993—
\begin{enumerate}\item[]
(i) a court order of a prescribed kind was in force with respect to the absent parent and either the person with care with respect to whom the current assessment was made or the child, or any of the children, with respect to whom that assessment was made, or

(ii) an agreement of a prescribed kind between the absent parent and any of those persons was in force;
\end{enumerate}

($b$) in consequence of one or more transfers of property of a prescribed kind—
\begin{enumerate}\item[]
(i) the amount payable by the absent parent by way of maintenance was less than would have been the case had that transfer or those transfers not been made; or

(ii) no amount was payable by the absent parent by way of maintenance; and
\end{enumerate}

($c$) the effect of that transfer, or those transfers, is not properly reflected in the current assessment.
\end{enumerate}

(2) For the purposes of sub-paragraph (1)($b$), “maintenance” means periodical payments of maintenance made (otherwise than under this Act) with respect to the child, or any of the children, with respect to whom the current assessment was made.

(3) For the purposes of sub-paragraph (1)($c$), the question whether the effect of one or more transfers of property is properly reflected in the current assessment shall be determined in accordance with regulations made by the Secretary of State.

\medskip

4.---(1) A departure direction may be given if—
\begin{enumerate}\item[]
($a$) before 5th April 1993—

(i) a court order of a prescribed kind was in force with respect to the absent parent and either the person with care with respect to whom the current assessment was made or the child, or any of the children, with respect to whom that assessment was made, or

(ii) an agreement of a prescribed kind between the absent parent and any of those persons was in force;

($b$) in pursuance of the court order or agreement, the absent parent has made one or more transfers of property of a prescribed kind;

($c$) the amount payable by the absent parent by way of maintenance was not reduced as a result of that transfer or those transfers;

($d$) the amount payable by the absent parent by way of child support maintenance under the current assessment has been reduced as a result of that transfer or those transfers, in accordance with provisions of or made under this Act; and

($e$) it is nevertheless inappropriate, having regard to the purposes for which the transfer or transfers was or were made, for that reduction to have been made.
\end{enumerate}

(2) For the purposes of sub-paragraph (1)($c$), “maintenance” means periodical payments of maintenance made (otherwise than under this Act) with respect to the child, or any of the children, with respect to whom the current assessment was made.

\subsection*{Additional cases}

5.---(1) The Secretary of State may by regulations prescribe other cases in which a departure direction may be given.

(2) Regulations under this paragraph may, for example, make provision with respect to cases where—
\begin{enumerate}\item[]
($a$) assets which do not produce income are capable of producing income;

($b$) a person’s life-style is inconsistent with the level of his income;

($c$) housing costs are unreasonably high;

($d$) housing costs are in part attributable to housing persons whose circumstances are such as to justify disregarding a part of those costs;

($e$) travel costs are unreasonably high; or

($f$) travel costs should be disregarded.
\end{enumerate}

\section*{Part II\\*Regulatory Controls}

6.---(1) The Secretary of State may by regulations make provision with respect to the directions which may be given in a departure direction.

(2) No directions may be given other than those which are permitted by the regulations.

(3) Regulations under this paragraph may, in particular, make provision for a departure direction to require—
\begin{enumerate}\item[]
($a$) the substitution, for any formula set out in Part I of Schedule 1, of such other formula as may be prescribed;

($b$) any prescribed amount by reference to which any calculation is to be made in fixing the amount of child support maintenance to be increased or reduced in accordance with the regulations;

($c$) the substitution, for any provision in accordance with which any such calculation is to be made, of such other provision as may be prescribed.
\end{enumerate}

(4) Regulations may limit the extent to which the amount of the child support maintenance fixed by a maintenance assessment made as a result of a departure direction may differ from the amount of the child support maintenance which would be fixed by a maintenance assessment made otherwise than as a result of the direction.

(5) Regulations may provide for the amount of any special expenses to be taken into account in a case falling within paragraph 2, for the purposes of a departure direction, not to exceed such amount as may be prescribed or as may be determined in accordance with the regulations.

(6) No departure direction may be given so as to have the effect of denying to an absent parent the protection of paragraph 6 of Schedule 1.

(7) Sub-paragraph (6) does not prevent the modification of the provisions of, or made under, paragraph 6 of Schedule 1 to the extent permitted by regulations under this paragraph.

(8) Any regulations under this paragraph may make different provision with respect to different levels of income.”
\end{quotation}

}

\opt{newrules,2012rules}{
\amendment{
Schs. 1, 2 repealed (3.3.03) for 2003 scheme and 2012 scheme cases only (see SI 2003/192) by the Child Support, Pensions and Social Security Act 2000 (c. 19) Sch. 9 Pt. I.

}
}

\part[Schedule 3 --- Minor and consequential amendments]{Schedule 3\\*Minor and consequential amendments}

\renewcommand\parthead{--- Schedule 3}

\amendment{
Para. 1 repealed (6.4.03) by the Income Tax (Earnings and Pensions) Act 2003 (c. 1) Sch. 8.
}

\subsection*{Child Support Act 1991}

2. The 1991 Act is amended as follows.

\medskip

3.---(1) In section 14 (information required by Secretary of State), after subsection (1) insert—
\begin{quotation}
“(1A) Regulations under subsection (1) may make provision for notifying any person who is required to furnish any information or evidence under the regulations of the possible consequences of failing to do so.”
\end{quotation}

\amendment{
Para. 3(2) repealed (8.9.98) by the Social Security Act 1998 (c. 14) Sch. 8.

\medskip

Para. 4--6 repealed (1.6.99) by the Social Security Act 1998 (c. 14) Sch. 8.
}

\medskip

7.---(1) Section 24 (appeal to Child Support Commissioner) is amended as follows.

(3) In subsection (3), for paragraph ($c$) substitute—
\begin{quotation}
“($c$) on an appeal by the Secretary of State, refer the case to a child support appeal tribunal with directions for its determination; or

($d$) on any other appeal, refer the case to a child support officer or, if he considers it appropriate, to a child support appeal tribunal with directions for its determination.”
\end{quotation}

\amendment{
Para. 7(2) repealed (1.6.99) by the Social Security Act 1998 (c. 14) Sch. 8.
}

\medskip

8.---(1) In section 25 (appeal from Child Support Commissioner on question of law), insert after subsection (3)—
\begin{quotation}
“(3A) The Child Support Commissioner to whom an application for leave to appeal under this section is made shall specify as the appropriate court either the Court of Appeal or the Court of Session.

(3B) In determining the appropriate court, the Child Support Commissioner shall have regard to the circumstances of the case, and in particular the convenience of the persons who may be parties to the appeal.”
\end{quotation}

(2) In the definition of “appropriate court” in subsection (4) of that section, for the words from “means” to “Session” substitute “,~except in subsections (3A) and (3B), means the court specified in accordance with those subsections”.

\amendment{
Para. 9 repealed (1.6.09) by the Child Maintenance and Other Payments Act 2008 (c. 6) Sch. 8.
}

\medskip

10. In section 33 (liability orders), at the end add—
\begin{quotation}
“(5) If the Secretary of State designates a liability order for the purposes of this subsection it shall be treated as a judgment entered in a county court for the purposes of section 73 of the County Courts Act 1984 (register of judgments and orders).”
\end{quotation}

\medskip

11. In section 41 (retention by Secretary of State of arrears recovered by him in benefit cases) for subsection (2) substitute—
\begin{quotation}
“(2) Where the Secretary of State recovers any such arrears he may, in such circumstances as may be prescribed and to such extent as may be prescribed, retain them if he is satisfied that the amount of any benefit paid to or in respect of the person with care of the child or children in question would have been less had the absent parent made the payment or payments of child support maintenance in question.

(2A) In determining for the purposes of subsection (2) whether the amount of any benefit paid would have been less at any time than the amount which was paid at that time, in a case where the maintenance assessment had effect from a date earlier than that on which it was made, the assessment shall be taken to have been in force at that time.”
\end{quotation}

\opt{oldrules}{
\medskip

12. In section 46(5) (circumstances in which child support officer may give a reduced benefit direction), after “may” insert “,~except in prescribed circumstances,”.
}

\amendment{
\opt{newrules,2012rules}{
Para. 12 repealed (3.3.03) for the purposes of 2003 scheme and 2012 scheme cases only (see SI 2003/192) by the Child Support, Pensions and Social Security Act 2000 (c. 19) Sch. 9 Pt. I.
}

\medskip

Para. 13 is not yet in force,
}

\medskip

14. In section 48(1) (power of Secretary of State to confer right of audience), for “person authorised” substitute “officer of the Secretary of State who is authorised”.

\opt{oldrules}{
\medskip

15. In section 52(2) (statutory instruments subject to affirmative resolution control)—
\begin{enumerate}\item[]
($a$) after “12(2),” insert “28C(2)($b$), 28F(3), 30(5A)”;

($b$) after “or (4)” insert “41A, 41B(6)”; and

($c$) after “Schedule 1” insert “or under Schedule 4B”.
\end{enumerate}
}

\opt{newrules,2012rules}{
\amendment{
Para. 15 repealed (3.3.03) for the purposes of 2003 scheme and 2012 scheme cases only (see SI 2003/192) by the Child Support, Pensions and Social Security Act 2000 (c. 19) Sch. 9 Pt. I.
}
}

\medskip

16. In section 54 (interpretation), insert the following definitions in the appropriate places—
\begin{quotation}
““application for a departure direction” means an application under section 28A;

“current assessment”, in relation to an application for a departure direction, means (subject to any regulations made under paragraph 10 of Schedule 4A) the maintenance assessment with respect to which the application is made;

“departure direction” has the meaning given in section 28A; and

“parent with care” means a person who is, in relation to a child, both a parent and a person with care.”
\end{quotation}

\amendment{
Para. 17 repealed (1.6.99) by the Social Security Act 1998 (c. 14) Sch. 8.

\medskip

Para. 18 repealed (3.11.08) by the Transfer of Tribunal Functions Order 2008 (S.I. 2008/2833) Sch. 3 para. 228($c$).
}

\medskip

19.---(1) In paragraph 3(2) of Schedule 5 (amendment of the House of Commons Disqualification Act 1975), after “Part I” insert “of Schedule 1”.

(3) In paragraph 4(1) of Schedule 5 (amendment of the Northern Ireland Assembly Disqualification Act 1975), after “Part I of” insert “Schedule 1 to”.

\amendment{
Para. 19(2) repealed (1.6.99) by the Social Security Act 1998 (c. 14) Sch. 8.
}

\subsection*{Social Security Administration Act 1992}

20. In section 170(5) of the Social Security Administration Act 1992 (the Social Security Advisory Committee)—
\begin{enumerate}\item[]
\opt{oldrules}{
($a$) in the definition of “the relevant enactments”, after paragraph ($aa$) insert—
\begin{quotation}
“($ab$) section 10 of the Child Support Act 1995;”; and
\end{quotation}
}

($b$) in the definition of “the relevant Northern Ireland enactments”, after paragraph ($aa$) insert—
\begin{quotation}
“($ab$) any enactment corresponding to section 10 of the Child Support Act 1995 having effect with respect to Northern Ireland; and”.
\end{quotation}
\end{enumerate}

\opt{newrules,2012rules}{
\amendment{
Para. 20($a$) repealed (3.3.03) for the purposes of 2003 scheme and 2012 scheme cases only (see SI 2003/192) by the Child Support, Pensions and Social Security Act 2000 (c. 19) Sch. 9 Pt. I.
}

}

\end{document}